Sections

SEARCH

\protect\hyperlink{site-content}{Skip to
content}\protect\hyperlink{site-index}{Skip to site index}

\href{https://myaccount.nytimes3xbfgragh.onion/auth/login?response_type=cookie\&client_id=vi}{}

\href{https://www.nytimes3xbfgragh.onion/section/todayspaper}{Today's
Paper}

FOOD; Regarding Guerard

\url{https://nyti.ms/29aWscV}

\begin{itemize}
\item
\item
\item
\item
\item
\end{itemize}

Advertisement

\protect\hyperlink{after-top}{Continue reading the main story}

Supported by

\protect\hyperlink{after-sponsor}{Continue reading the main story}

FOOD

\hypertarget{food-regarding-guerard}{%
\section{FOOD; Regarding Guerard}\label{food-regarding-guerard}}

By Molly O'Neill

\begin{itemize}
\item
  Feb. 9, 1992
\item
  \begin{itemize}
  \item
  \item
  \item
  \item
  \item
  \end{itemize}
\end{itemize}

\includegraphics{https://s1.graylady3jvrrxbe.onion/timesmachine/pages/1/1992/02/09/049792_360W.png?quality=75\&auto=webp\&disable=upscale}

See the article in its original context from\\
February 9, 1992, Section 6, Page
59\href{https://store.nytimes3xbfgragh.onion/collections/new-york-times-page-reprints?utm_source=nytimes\&utm_medium=article-page\&utm_campaign=reprints}{Buy
Reprints}

\href{http://timesmachine.nytimes3xbfgragh.onion/timesmachine/1992/02/09/049792.html}{View
on timesmachine}

TimesMachine is an exclusive benefit for home delivery and digital
subscribers.

About the Archive

This is a digitized version of an article from The Times's print
archive, before the start of online publication in 1996. To preserve
these articles as they originally appeared, The Times does not alter,
edit or update them.

Occasionally the digitization process introduces transcription errors or
other problems; we are continuing to work to improve these archived
versions.

The name "Michel Guerard" is emblazoned in blue on his starched white
chef jacket, his checkered trousers are well creased. The three-star
French chef who invented cuisine minceur preens under his pleated toque.
He looks ready for more television cameras, or to pose for publicity
purposes once again in front of Eugenie-les-Bains, the luxurious country
inn and spa that he and his wife, Christine, run in the southwest of
France. For nearly two decades, he's been the guru of the international
slimming set -\/- by now his is practically a brand name. "Michel
Guerard" resonates with the French as immediately as "Lean Cuisine" does
with the Americans.

But as the chef called Guerard wraps his apron around his waist this
winter morning, he sighs. A gentle puff of belly presses against his
apron strings. It looks like a "magret," the plump breast of the moulard
duck that is raised and revered in his corner of France, straining
against butcher's twine. "Cooking is a dangerous business," Guerard says
as he walks into his kitchen.

After all these years, his repertoire is constantly expanding. But his
battle with girth remains the same.

Born in Vetheuil, a small town north of Paris, Guerard was 15 years old,
a "skinny, skinny, knob-kneed pastry apprentice," when he first
encountered the hazards of his trade. He moved deeper into the realm of
buttercreams and chocolate ganache when he became pastry chef at the
Hotel Crillon in Paris. Then he bounded into the rich world of savory
food, learning to saute and sauce and becoming, eventually, a chef's
chef, with his self-styled cuisine gourmande. The small bistro that he
opened in a Paris suburb in 1965 dished up the richest of the rich: a
heart-stopping pot-au-feu, a buttery broth of sea urchins and mussels, a
rabbit terrine studded with foie gras, pastries and more pastries.

Ah, he sighs, with the wistfulness of one recalling a lost love, "such
days, such days." The more Michelin stars he gathered, he says, the more
he began looking like the roly-poly Michelin-tire man. When his
wife-to-be mentioned the symmetry, the chef cast himself from the Eden
of indulgence into "a long, punishing march through fields of grated
carrots." Two decades later, the memory of the traditional diet still
sends a shadow of pain over his impish face.

"The ritual of boiled meat and boiled beans," he moans weakly. The
sentence was too much. In "Michel Guerard's Cuisine Minceur" he wrote:
"No longer was I allowed to know the marvelous, profound resonance that
is produced when the eye, the nose and the palate join together in a
symphonic wave to savor a finely crafted dish."

Even now, "I still can't bear the thought," says the chef, who is 58
years old.

In 1972, Guerard closed his bistro in the north of France and migrated
to the southwest, where he opened a more luxurious dining room and inn.
There, in the era that bridged haute and nouvelle cuisines, he mounted
"an evasive action" against traditional diet regimes, creating a style
of cooking that courts every sense except that of deprivation.

The man is a shameless master of deception. He is not above using an
imitation-butter flavoring to send a memory of the real thing through
dishes that he de-creams. He doesn't feel that he's violating the
national culinary trust when he substitutes a mousse of wild mushrooms
for foie gras in his rabbit pate. These sleights of hand are now
instinctive, he says.

The steam is fogging his kitchen windows, and a surly wind whips wet,
brown leaves through the air outside. Michel Guerard sighs -\/- again.
He wants a rib-sticking, festive dinner as much as the next guy.
Speculatively, he weighs two bulbs of fennel in his hands. He whiffs the
duck broth that is barely simmering on the stove.

Voila! A soup, a fennel soup. He can replace heavy cream with chicken
broth and low-fat milk for flavor, use more vegetables for a fuller body
and garnish the soup with a vibrant red-pepper mousse for a contrasting
flavor and visual distraction.

And duck. It's possible to eat duck and stay slim, he says. The breast
meat must be carefully trimmed. The broth must be simmered, chilled and
de-fatted, then simmered again to thicken it for a sauce. A melange of
vegetables cut like one-carat gems will be diverting in the sauce. And
-\/- this is the best part, he giggles -\/- he will carve apples to look
like potatoes to accompany the dish. A pomme de terre pun: People
expecting a potato, getting an apple and forgiving it because the pomme
de terre look-alike has been poached in such an intriguing blend of
spices. "Parfait!" He rubs his hands and checks his watch, as if to see
how long he has to wait before sitting down to this meal with his wife
and four friends.

"A grand simplicity, a lot of duplicity, a little joke, a perfect meal,"
he says. Let other spa chefs limit their imaginations as much as they
limit their fat. "I am still the best."

FENNEL SOUP (ADAPTED FROM MICHEL GUERARD) 1 onion, peeled, finely minced
2 small leeks, white parts only, carefully cleaned and finely minced 1
sprig fresh thyme 1/2 teaspoon salt 1/2 teaspoon freshly ground pepper 6
cups chicken broth, either fresh or low-sodium canned 3 bulbs fennel,
trimmed and finely chopped, tops reserved for garnish 2 tablespoons
low-fat milk 4 sprigs flat parsley leaves, rinsed and finely chopped
Red-pepper mousse (optional, see recipe).

1. Heat a nonstick pot over low heat. Add the onion, leeks, thyme, salt,
pepper and 1/4 cup of the broth, cover the pan and cook over low heat
until the vegetables are soft, about 10 minutes. Add the fennel, cover,
and continue to cook until the fennel is tender, about 15 minutes. Add
the remaining chicken stock and bring it to a boil. Cover, lower the
heat and simmer gently for 25 minutes.

2. Puree the soup in a blender until smooth. Strain it through a
fine-mesh strainer. Return any of the puree that does not pass through
the sieve to the blender, puree again and re-strain, repeating until all
the soup has been used. The soup can be refrigerated at this point for
up to 3 days.

3. Just before serving, slowly bring the soup to a simmer, remove from
the heat and whisk in the milk. Ladle into bowls. The soup can be
garnished simply with the reserved fennel tops and chopped parsley. For
a more festive -\/- and flavorful -\/- dish, add a dollop of red-pepper
mousse.

Yield: Four servings. RED-PEPPER MOUSSE (ADAPTED FROM MICHEL GUERARD) 2
large, sweet red bell peppers 1/4 teaspoon salt 1/4 teaspoon freshly
ground pepper 1 tablespoon low-fat, plain yogurt 3 tablespoons skim-milk
ricotta cheese 2 1/2 teaspoons unflavored gelatin.

1. Char the peppers over an open flame or under the broiler until the
skin blisters and turns black on all sides, about 5 minutes. Transfer
the peppers to a paper bag, close and set aside until the peppers are
tender, 15 to 20 minutes. Split each pepper lengthwise, lay it flat and
use a sharp knife to lift off the skin and remove the veins and seeds.
Do not rinse the peppers.

2. Put the peppers, salt, pepper, yogurt and ricotta into a food
processor and process until smooth.

3. In a small saucepan, heat 1/2 cup of the puree over low heat until it
is hot, but not boiling, and remove from the heat. Soften the gelatin in
1 tablespoon of tepid water and stir into the warmed portion of the
puree. Continue stirring for 5 minutes, until the gelatin is completely
dissolved and the puree has cooled slightly, then stir in the remaining
unheated puree. Place in the food processor and process briefly to
combine well.

4. Put the mousse into a clean bowl, cover and refrigerate for 2 to 3
hours until firm.

Yield: One and one-half cups. PAN-GRILLED DUCK WITH SPINACH AND POACHED
APPLES (APAPTED FROM MICHEL GUERARD) The duck and broth: 1 9-to-10-pound
moulard duck or 2 4 1/2-pound Long Island ducks (see note) 1/2 teaspoon
kosher salt 1/2 teaspoon freshly ground pepper 1 onion, peeled and
quartered 1 leek, white part only, carefully cleaned and cut into large
pieces 2 small carrots, peeled and chopped 1 teaspoon whole black
peppercorns 1 bouquet garni (1 bay leaf, 1 sprig fresh thyme, 2 sprigs
parsley, 1 small stalk celery with leaves -\/- all tied together with a
string) 6 mushrooms, trimmed, stems discarded The sauce: 1 quart duck
broth (see recipe) 1 small leek, white part only, carefully cleaned, or
1 small white onion, finely minced 1 small carrot, peeled, finely diced
1/4 cup finely diced sweet red pepper, seeded 1/4 cup finely diced
yellow pepper, seeded The garnish: 1 to 1 1/4 pounds spinach, leaves
only, rinsed well Spicy poached apples (see recipe).

1. Preheat the oven to 450 degrees. Use a sharp knife to remove any
excess fat from the duck. Remove the neck, cut in half and place in a
shallow roasting pan. Remove the wings, being careful not to cut into
the breast. Place the wings in the roasting pan. Cut along one side of
the ridge of the breastbone, then scrape the knife against the side of
the breastbone, gently freeing one half of the breast from the cavity.
Repeat on the other side. Remove as much fat as possible from the meat
and discard the fat. Remove the loose, sinewy flap of meat from the
underside of both breast pieces. Set aside. Blend the salt and pepper
and rub the mixture into the skin side of the breasts. Cover and
refrigerate. Remove the duck legs and use the back of the knife to break
up the carcass. Add the carcass to the roasting pan and roast in the
oven along with the neck and wings for 20 to 25 minutes, turning
frequently until evenly browned.

2. Put the browned duck parts, onion, leek, carrots, peppercorns and
bouquet garni -\/- and enough water to cover, about 6 quarts -\/- in a
stockpot over low heat. Bring to a boil and reduce the heat to a simmer.
Add the legs, the flaps of meat from the breast and the mushrooms, cover
partially and simmer for 4 hours to get about 2 1/2 to 3 quarts. Skim
any foam as it rises to the top.

3. Strain the broth and discard the solids. Strain again through a fine
sieve or a large strainer lined with cheesecloth. Cool and then
refrigerate overnight to chill completely. Skim and discard the fat.

4. Prepare the spicy poached apples. They can be kept at room
temperature for up to 2 hours.

5. An hour before serving, begin making the sauce. Put the duck broth in
a saucepan over medium heat and cook until it has reduced to 1 1/4 cups,
about 35 minutes. (Freeze the remaining broth to make soup.) Add the
leek or onion and simmer for 5 minutes. Add the carrot, remove from heat
and set aside for 15 minutes.

6. Heat a heavy skillet over medium-high heat until it is hot, but not
smoking. Lay the breasts skin side down, and cook until brown, about 4
to 5 minutes. Turn them over and cook until done, about 3 to 5 minutes,
depending on the thickness of the breasts. Pour out the fat as it
accumulates in the pan. Remove the breasts to a tray, cover loosely with
foil and allow to rest for at least 5 minutes and up to 15 minutes
before slicing.

7. Five minutes before you are ready to serve the duck, return the sauce
to the heat and bring it to a simmer. Add the peppers and set aside for
3 minutes.

8. Steam the spinach until just wilted and drain well.

9. Use a sharp knife to slice the duck breasts thinly. Lay out 4 warmed
plates. Place a mound of steamed spinach in the center of each plate and
fan the duck slices around the spinach. Spoon the sauce over the meat,
garnish with the spiced apples and serve.

Yield: Four servings.

Note: This recipe is best prepared with "magret," the breast of the
moulard duck, which, in the United States, is a cross between the
Muscovy and the Pekin, or Long Island, duck. About twice the size of
other duck breasts, magret is available through specialty butchers or by
mail order for about \$3.50 a pound from D'Artagnan, Jersey City, N.J.,
(800) 327-8246. SPICY POACHED APPLES (ADAPTED FROM MICHEL GUERARD) 1/8
teaspoon each of ground cinnamon, ground cloves, ground nutmeg and
freshly ground pepper 1 quart apple cider 1/2 cup white wine 2 large
Granny Smith apples, peeled and cored, each cut into 8 pieces
lengthwise, each piece carved into a half moon.

1. Mix the spices together.

2. In a saucepan, combine the spices, apple cider and wine and bring to
a boil. Simmer gently for 5 minutes. Add the apples, take the pan off
the heat and set aside at room temperature for up to 2 hours before
serving with the duck.

Yield: Garnish for four servings.

Advertisement

\protect\hyperlink{after-bottom}{Continue reading the main story}

\hypertarget{site-index}{%
\subsection{Site Index}\label{site-index}}

\hypertarget{site-information-navigation}{%
\subsection{Site Information
Navigation}\label{site-information-navigation}}

\begin{itemize}
\tightlist
\item
  \href{https://help.nytimes3xbfgragh.onion/hc/en-us/articles/115014792127-Copyright-notice}{©~2020~The
  New York Times Company}
\end{itemize}

\begin{itemize}
\tightlist
\item
  \href{https://www.nytco.com/}{NYTCo}
\item
  \href{https://help.nytimes3xbfgragh.onion/hc/en-us/articles/115015385887-Contact-Us}{Contact
  Us}
\item
  \href{https://www.nytco.com/careers/}{Work with us}
\item
  \href{https://nytmediakit.com/}{Advertise}
\item
  \href{http://www.tbrandstudio.com/}{T Brand Studio}
\item
  \href{https://www.nytimes3xbfgragh.onion/privacy/cookie-policy\#how-do-i-manage-trackers}{Your
  Ad Choices}
\item
  \href{https://www.nytimes3xbfgragh.onion/privacy}{Privacy}
\item
  \href{https://help.nytimes3xbfgragh.onion/hc/en-us/articles/115014893428-Terms-of-service}{Terms
  of Service}
\item
  \href{https://help.nytimes3xbfgragh.onion/hc/en-us/articles/115014893968-Terms-of-sale}{Terms
  of Sale}
\item
  \href{https://spiderbites.nytimes3xbfgragh.onion}{Site Map}
\item
  \href{https://help.nytimes3xbfgragh.onion/hc/en-us}{Help}
\item
  \href{https://www.nytimes3xbfgragh.onion/subscription?campaignId=37WXW}{Subscriptions}
\end{itemize}
