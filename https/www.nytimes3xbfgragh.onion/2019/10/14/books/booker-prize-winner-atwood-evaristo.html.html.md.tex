Sections

SEARCH

\protect\hyperlink{site-content}{Skip to
content}\protect\hyperlink{site-index}{Skip to site index}

\href{https://www.nytimes3xbfgragh.onion/section/books}{Books}

\href{https://myaccount.nytimes3xbfgragh.onion/auth/login?response_type=cookie\&client_id=vi}{}

\href{https://www.nytimes3xbfgragh.onion/section/todayspaper}{Today's
Paper}

\href{/section/books}{Books}\textbar{}Margaret Atwood and Bernardine
Evaristo Share Booker Prize

\url{https://nyti.ms/2nKGAFf}

\begin{itemize}
\item
\item
\item
\item
\item
\end{itemize}

Advertisement

\protect\hyperlink{after-top}{Continue reading the main story}

Supported by

\protect\hyperlink{after-sponsor}{Continue reading the main story}

\hypertarget{margaret-atwood-and-bernardine-evaristo-share-booker-prize}{%
\section{Margaret Atwood and Bernardine Evaristo Share Booker
Prize}\label{margaret-atwood-and-bernardine-evaristo-share-booker-prize}}

The judges rebelled against the literary prize's rules and awarded it to
``The Testaments'' and ``Girl, Woman, Other.''

\includegraphics{https://static01.graylady3jvrrxbe.onion/images/2019/10/15/books/14Booker1/14Booker1-articleLarge.jpg?quality=75\&auto=webp\&disable=upscale}

By \href{https://www.nytimes3xbfgragh.onion/by/alex-marshall}{Alex
Marshall} and
\href{https://www.nytimes3xbfgragh.onion/by/alexandra-alter}{Alexandra
Alter}

\begin{itemize}
\item
  Oct. 14, 2019
\item
  \begin{itemize}
  \item
  \item
  \item
  \item
  \item
  \end{itemize}
\end{itemize}

LONDON --- Margaret Atwood and Bernardine Evaristo have both won this
year's Booker Prize, it was announced at a ceremony on Monday, after the
judges for the literary award rebelled against its rules.

``We were told quite firmly that the rules state you can only have one
winner,'' Peter Florence, the chairman of the Booker judges, said at a
news conference. But the ``consensus was to flout the rules and divide
this year's prize to celebrate two winners.''

Evaristo, who won for her novel ``Girl, Woman, Other,'' is the first
black woman to win the Booker Prize. ``I hope that honor doesn't last
too long,'' she said in her acceptance speech. Atwood, who won in 2000
for ``The Blind Assassin,'' was considered a front-runner this year for
``The Testaments,'' the sequel to her 1985 dystopian classic, ``The
Handmaid's Tale.''

It is not the first time the award has been shared.
\href{https://www.nytimes3xbfgragh.onion/1992/10/14/books/book-notes-a-lost-work-by-joyce-fuels-scholarly-debate.html}{In
1992}, Michael Ondaatje's ``The English Patient'' shared it with Barry
Unsworth's ``Sacred Hunger,'' but the prize's organizers then changed
the rules to only allow one winner to avoid undermining either book.

Several judging panels had tried to split the prize since, said Gaby
Wood, the Booker Prize Foundation's literary director, but settled on
single winners after being told they had to.

The decision to rebel and award the prize to two writers this year was
not taken lightly. The judges, who included the author Xiaolu Guo and
editor Liz Calder, spent over three hours trying to pick a winner before
asking if they could choose both. They were told they couldn't. The
judges then --- to the Booker organizers' ``horror,'' Florence said ---
spent another ``hour-and-a-half agonizing how to resolve the issue,''
before deciding it was the only result they wanted.

They were again told it was unacceptable. It was only at a third
attempt, 30 minutes later, that the Booker Prize's trustees accepted the
decision.

Wood dodged the question when asked if she supported the final result.
``I support the mechanism by which the judges made their decision,'' she
said. She then joked the judges would not be paid for their involvement,
which included reading 151 submitted books.

For Atwood, the prize comes at a moment of renewed cultural relevance
for ``The Handmaid's Tale,'' which has sold more than 8 million copies
worldwide in English. The novel was adapted into a hit television series
on Hulu, and the story has taken on fresh political resonance, as
\href{https://www.nytimes3xbfgragh.onion/video/us/100000005479098/handmaids-protest-nationwide.html?module=inline}{women
dressed as handmaids} have flooded Congress and state capitols to
protest new restrictions on reproductive rights. In
\href{https://www.nytimes3xbfgragh.onion/2019/09/03/books/review/testaments-margaret-atwood-handmaids-tale.html?module=inline}{``The
Testaments,''} Atwood weaves together the stories of three female
narrators in Gilead, a religious autocracy in what was formerly the
United States.

Evaristo, an experimental writer who is well established in Britain but
not widely known internationally, is a more surprising choice. In her
eight works of fiction, Evaristo, who was born in London in 1959 to a
white English mother and a Nigerian father, often explores the lives of
members of the African diaspora. ``Girl, Woman, Other'' features a dozen
characters, most of them black British women. It's written in a blend of
poetry and prose, a hybrid that Evaristo calls ``fusion fiction.''

In an interview with The Times on Monday night, Evaristo said that the
novel grew out of her frustration over the lack of representation in
British literature.

``When I started the book six years ago, I was so fed up with black
British women being absent from British literature,'' she said. ``So I
wanted to see how many characters I could put into a novel and pull it
off.''

The other novels on the shortlist included
\href{https://www.nytimes3xbfgragh.onion/2019/09/03/books/review-ducks-newburyport-lucy-ellmann.html}{Lucy
Ellmann's ``Ducks, Newburyport,''} a 1,000 page novel about a
middle-aged woman in Ohio reflecting on her life while baking, which
unfolds almost entirely in a single sentence; Chigozie Obioma's
\href{https://www.nytimes3xbfgragh.onion/2019/01/21/books/review/chigozie-obioma-orchestra-minorities.html?module=inline}{``An
Orchestra of Minorities,''} about a Nigerian poultry farmer called
Chinonso who stops a woman from jumping to her death and falls in love
with her; Salman
Rushdie's\href{https://www.nytimes3xbfgragh.onion/2019/09/03/books/review/quichotte-salman-rushdie.html}{``Quichotte,''}
a retelling of ``Don Quixote'' that features a traveling salesman on a
quest to win over a beautiful television host; and Elif Shafak's
\href{https://www.theguardian.com/books/2019/jun/16/10-minutes-38-seconds-in-this-strange-world-by-elif-shafak-book-review}{``10
Minutes 38 Seconds in This Strange World,''} a story about a sex worker
in Istanbul who is murdered and left in the garbage on the outskirts of
the city.

Compared to previous years, in which Americans were heavily represented,
writers from the United States were scarce this year. The sole American
on the shortlist is Ellmann, a native of Illinois who now lives in
Scotland.

Evaristo and Atwood will split the prize money of 50,000 pounds, around
\$63,000, although the Booker, first awarded in 1969, normally delivers
a sales boost.
\href{https://www.nytimes3xbfgragh.onion/2018/11/29/books/anna-burns-interview-booker-prize-milkman-no-bones.html?searchResultPosition=5}{Anna
Burns's ``Milkman,''} an experimental novel about a woman during
Northern Ireland's civil conflict, has sold over 500,000 copies since
winning the prize last year.

The Booker is one of the literary world's most prestigious prizes. Past
winners include Rushdie, who was
\href{https://www.nytimes3xbfgragh.onion/2019/09/03/books/booker-prize-shortlist.html}{shortlisted
for this year's prize}, as well as such literary heavyweights as Hilary
Mantel and J.M. Coetzee. Atwood now joins Mantel, Coetzee and Peter
Carey in the small club of authors to have won twice.

\emph{Follow New York Times Books on}
\href{https://www.facebookcorewwwi.onion/nytbooks/}{\emph{Facebook}}\emph{,}
\href{https://twitter.com/nytimesbooks}{\emph{Twitter}} \emph{and}
\href{https://www.instagram.com/nytbooks/}{\emph{Instagram}}\emph{, sign
up for}
\href{https://www.nytimes3xbfgragh.onion/newsletters/books-review}{\emph{our
newsletter}} \emph{or}
\href{https://www.nytimes3xbfgragh.onion/interactive/2017/books/books-calendar.html}{\emph{our
literary calendar}}\emph{. And listen to us on the}
\href{https://www.nytimes3xbfgragh.onion/column/book-review-podcast}{\emph{Book
Review podcast}}\emph{.}

Advertisement

\protect\hyperlink{after-bottom}{Continue reading the main story}

\hypertarget{site-index}{%
\subsection{Site Index}\label{site-index}}

\hypertarget{site-information-navigation}{%
\subsection{Site Information
Navigation}\label{site-information-navigation}}

\begin{itemize}
\tightlist
\item
  \href{https://help.nytimes3xbfgragh.onion/hc/en-us/articles/115014792127-Copyright-notice}{©~2020~The
  New York Times Company}
\end{itemize}

\begin{itemize}
\tightlist
\item
  \href{https://www.nytco.com/}{NYTCo}
\item
  \href{https://help.nytimes3xbfgragh.onion/hc/en-us/articles/115015385887-Contact-Us}{Contact
  Us}
\item
  \href{https://www.nytco.com/careers/}{Work with us}
\item
  \href{https://nytmediakit.com/}{Advertise}
\item
  \href{http://www.tbrandstudio.com/}{T Brand Studio}
\item
  \href{https://www.nytimes3xbfgragh.onion/privacy/cookie-policy\#how-do-i-manage-trackers}{Your
  Ad Choices}
\item
  \href{https://www.nytimes3xbfgragh.onion/privacy}{Privacy}
\item
  \href{https://help.nytimes3xbfgragh.onion/hc/en-us/articles/115014893428-Terms-of-service}{Terms
  of Service}
\item
  \href{https://help.nytimes3xbfgragh.onion/hc/en-us/articles/115014893968-Terms-of-sale}{Terms
  of Sale}
\item
  \href{https://spiderbites.nytimes3xbfgragh.onion}{Site Map}
\item
  \href{https://help.nytimes3xbfgragh.onion/hc/en-us}{Help}
\item
  \href{https://www.nytimes3xbfgragh.onion/subscription?campaignId=37WXW}{Subscriptions}
\end{itemize}
