Sections

SEARCH

\protect\hyperlink{site-content}{Skip to
content}\protect\hyperlink{site-index}{Skip to site index}

\href{https://www.nytimes3xbfgragh.onion/section/world/asia}{Asia
Pacific}

\href{https://myaccount.nytimes3xbfgragh.onion/auth/login?response_type=cookie\&client_id=vi}{}

\href{https://www.nytimes3xbfgragh.onion/section/todayspaper}{Today's
Paper}

\href{/section/world/asia}{Asia Pacific}\textbar{}Kim Jong-un Orders
`Shabby' South Korean Hotels in Resort Town Destroyed

\url{https://nyti.ms/2qEHdkX}

\begin{itemize}
\item
\item
\item
\item
\item
\end{itemize}

Advertisement

\protect\hyperlink{after-top}{Continue reading the main story}

Supported by

\protect\hyperlink{after-sponsor}{Continue reading the main story}

\hypertarget{kim-jong-un-orders-shabby-south-korean-hotels-in-resort-town-destroyed}{%
\section{Kim Jong-un Orders `Shabby' South Korean Hotels in Resort Town
Destroyed}\label{kim-jong-un-orders-shabby-south-korean-hotels-in-resort-town-destroyed}}

The North Korean leader has been pushing the South to restart a joint
tourism project that once made money for his regime.

\includegraphics{https://static01.graylady3jvrrxbe.onion/images/2019/10/23/world/23nkorea-1sub/merlin_163163346_2def652c-d01a-40ca-86d7-1a164358eb0f-articleLarge.jpg?quality=75\&auto=webp\&disable=upscale}

\href{https://www.nytimes3xbfgragh.onion/by/choe-sang-hun}{\includegraphics{https://static01.graylady3jvrrxbe.onion/images/2018/07/18/multimedia/author-choe-sang-hun/author-choe-sang-hun-thumbLarge.png}}

By \href{https://www.nytimes3xbfgragh.onion/by/choe-sang-hun}{Choe
Sang-Hun}

\begin{itemize}
\item
  Oct. 23, 2019
\item
  \begin{itemize}
  \item
  \item
  \item
  \item
  \item
  \end{itemize}
\end{itemize}

SEOUL, South Korea --- North Korea said on Wednesday that its leader,
Kim Jong-un, had ordered the demolition of South Korean hotels and other
buildings in a resort complex that the two countries once operated
together.

The resort town at
\href{https://www.nytimes3xbfgragh.onion/2006/10/30/world/asia/30iht-mount.3329914.html}{Diamond
Mountain}, or Kumgang, just north of the inter-Korean border, opened in
1998, at a time of reduced tensions between the Koreas. Until it was
closed during a dispute in 2008, it served as a major source of foreign
currency for the cash-starved North, frequently hosting South Korean
tour groups.

Mr. Kim said during a recent visit that the South Korean facilities were
``shabby'' and lacked ``national character,'' comparing them to
``makeshift tents in a disaster-stricken area,'' the North's official
Korean Central News Agency reported on Wednesday.

Mr. Kim has pressed South Korea to reopen the complex since last year,
when he first met with the South's president, Moon Jae-in. But the South
said it could only consider doing so as part of a broader agreement
between the United States and North Korea to end the North's nuclear
weapons program.

Mr. Kim called for ``building new modern service facilities our own way
that go well with the natural scenery of Mount Kumgang.'' He also
criticized what he called ``the mistaken policy of the predecessors''
--- a reference to his father, Kim Jong-il, the North's previous
dictator --- for the decision to ``rely on others'' for the resort
project, meaning the South.

\includegraphics{https://static01.graylady3jvrrxbe.onion/images/2019/10/23/world/23nkorea-2/merlin_146498475_c37c7899-3c92-4dfe-9145-6cded483f123-articleLarge.jpg?quality=75\&auto=webp\&disable=upscale}

Such criticism of the ruling Kim family's policies would be a capital
crime for most North Koreans. But since he took power in 2011, Kim
Jong-un ​has often broken that tradition, notably by
\href{https://www.nytimes3xbfgragh.onion/2018/08/20/world/asia/kim-jong-un-north-korea-economy-nuclear-talks.html}{criticizing
state-run factories and construction projects​} as unproductive, as he
has tried to rebuild his country's economy.

A key element of Mr. Kim's plan for building a
``\href{https://www.nytimes3xbfgragh.onion/2019/04/18/world/asia/north-korea-economy-sanctions.html}{self-reliant''
economy}, in the face of international sanctions over his nuclear
program, is developing tourism along its scenic east coast and near
Mount Baekdu along the Chinese border. Tourism is excluded from the
sanctions that the United Nations has imposed on the North.

The Diamond Mountain resort was Kim Jong-il's major effort in the
tourism sector. He gave the South Korean conglomerate Hyundai the right
to build and run a resort town there, in a joint venture with his
totalitarian government. Starting in 1998, Hyundai built or renovated
hotels, port facilities, restaurants, spas, ​a concert hall and a golf
course​ at the scenic spot​.

Nearly two million South Korean tourists visited before it was closed,
which helped North Korea earn ​hundreds of millions of dollars at a time
when was struggling to recover from a devastating famine.

Image

Villas at the resort complex in 2011.Credit...Ng Han Guan/Associated
Press

Diamond Mountain was one of the most visible symbols of an era of
inter-Korean cooperation that ended in 2008, when a new, conservative
​government​ took power​ in Seoul​. That government, led by President
Lee Myung-bak, suspected that the tourist revenue was going to the
North's nuclear weapons development. It
\href{https://www.nytimes3xbfgragh.onion/2008/08/11/world/asia/11iht-11korea.15158606.html}{pulled
Hyundai out of the project} after a North Korean security guard shot and
killed a South Korean tourist who apparently had wandered into a
restricted area.

Since then, North Korea has occasionally threatened to confiscate and
liquidate the shuttered South Korean properties at Diamond Mountain,
whose value has been estimated at about \$400 million​. The resort has
occasionally been used to host reunions of families separated during the
Korean War.

North Korea has recently invited tourists from China and other countries
to the mountain for hiking trips, according to news reports.

On Wednesday, Lee Sang-min, a spokesman for the South's Unification
Ministry, said South Korea hoped to hold discussions with North Korea to
defend its property rights at the resort. Hyundai said it was closely
watching for further developments.​

But ​Mr. Kim indicated that North Korea was no longer interested in
letting South Koreans run the facilities​ again, though they would be
welcome to visit​.

``He said that we will always welcome our compatriots from the South if
they want to come to Mount Kumgang, after it is wonderfully built as the
world-level tourist destination,'' the North Korean news agency said.

Image

A meeting room at Kumgangsan Hotel, part of the resort
complex.Credit...Dita Alangkara/Associated Press

Advertisement

\protect\hyperlink{after-bottom}{Continue reading the main story}

\hypertarget{site-index}{%
\subsection{Site Index}\label{site-index}}

\hypertarget{site-information-navigation}{%
\subsection{Site Information
Navigation}\label{site-information-navigation}}

\begin{itemize}
\tightlist
\item
  \href{https://help.nytimes3xbfgragh.onion/hc/en-us/articles/115014792127-Copyright-notice}{©~2020~The
  New York Times Company}
\end{itemize}

\begin{itemize}
\tightlist
\item
  \href{https://www.nytco.com/}{NYTCo}
\item
  \href{https://help.nytimes3xbfgragh.onion/hc/en-us/articles/115015385887-Contact-Us}{Contact
  Us}
\item
  \href{https://www.nytco.com/careers/}{Work with us}
\item
  \href{https://nytmediakit.com/}{Advertise}
\item
  \href{http://www.tbrandstudio.com/}{T Brand Studio}
\item
  \href{https://www.nytimes3xbfgragh.onion/privacy/cookie-policy\#how-do-i-manage-trackers}{Your
  Ad Choices}
\item
  \href{https://www.nytimes3xbfgragh.onion/privacy}{Privacy}
\item
  \href{https://help.nytimes3xbfgragh.onion/hc/en-us/articles/115014893428-Terms-of-service}{Terms
  of Service}
\item
  \href{https://help.nytimes3xbfgragh.onion/hc/en-us/articles/115014893968-Terms-of-sale}{Terms
  of Sale}
\item
  \href{https://spiderbites.nytimes3xbfgragh.onion}{Site Map}
\item
  \href{https://help.nytimes3xbfgragh.onion/hc/en-us}{Help}
\item
  \href{https://www.nytimes3xbfgragh.onion/subscription?campaignId=37WXW}{Subscriptions}
\end{itemize}
