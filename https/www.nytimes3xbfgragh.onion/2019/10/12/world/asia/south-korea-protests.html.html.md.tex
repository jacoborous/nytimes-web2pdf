Sections

SEARCH

\protect\hyperlink{site-content}{Skip to
content}\protect\hyperlink{site-index}{Skip to site index}

\href{https://www.nytimes3xbfgragh.onion/section/world/asia}{Asia
Pacific}

\href{https://myaccount.nytimes3xbfgragh.onion/auth/login?response_type=cookie\&client_id=vi}{}

\href{https://www.nytimes3xbfgragh.onion/section/todayspaper}{Today's
Paper}

\href{/section/world/asia}{Asia Pacific}\textbar{}In Seoul, Crowds
Denounce a Divisive Politician. Days Later, Others Defend Him.

\url{https://nyti.ms/2MzYqTP}

\begin{itemize}
\item
\item
\item
\item
\item
\end{itemize}

Advertisement

\protect\hyperlink{after-top}{Continue reading the main story}

Supported by

\protect\hyperlink{after-sponsor}{Continue reading the main story}

\hypertarget{in-seoul-crowds-denounce-a-divisive-politician-days-later-others-defend-him}{%
\section{In Seoul, Crowds Denounce a Divisive Politician. Days Later,
Others Defend
Him.}\label{in-seoul-crowds-denounce-a-divisive-politician-days-later-others-defend-him}}

Vast rallies have called for the ouster of South Korea's justice
minister. But huge counterprotests have assailed the prosecutors who are
investigating him.

\href{https://www.nytimes3xbfgragh.onion/by/choe-sang-hun}{\includegraphics{https://static01.graylady3jvrrxbe.onion/images/2018/07/18/multimedia/author-choe-sang-hun/author-choe-sang-hun-thumbLarge.png}}

By \href{https://www.nytimes3xbfgragh.onion/by/choe-sang-hun}{Choe
Sang-Hun}

\begin{itemize}
\item
  Oct. 12, 2019
\item
  \begin{itemize}
  \item
  \item
  \item
  \item
  \item
  \end{itemize}
\end{itemize}

\includegraphics{https://static01.graylady3jvrrxbe.onion/images/2019/10/12/world/12korea-protests3-sub/merlin_162567027_bd58afc1-5185-4677-8821-60939b729b2c-articleLarge.jpg?quality=75\&auto=webp\&disable=upscale}

SEOUL, South Korea --- For weeks, huge crowds have been gathering in
Seoul to denounce a man named Cho Kuk --- or to defend him.

Mr. Cho, South Korea's justice minister, and his family are being
investigated over a number of allegations, ranging from financial
malfeasance to pulling strings to get a daughter into medical school.
Demonstrators, most of them conservatives who oppose President Moon
Jae-in, have rallied in large numbers to demand Mr. Cho's arrest and Mr.
Moon's impeachment.

``Lord, please save this country by dragging Moon Jae-in out of office
as soon as possible,'' the Rev. Jeon Kwang-hoon, ​head of the Christian
Council of Korea​, said at an enormous rally this month.

But others see the issue very differently. Mr. Cho had been tasked by
Mr. Moon with overhauling the national prosecutors' office --- the same
agency now investigating him. Progressive supporters of the president
have held increasingly large counterprotests, accusing the prosecutors
of targeting Mr. Cho to preserve their own power.

``Cho Kuk means prosecutors' reform. He is our flag, he is our
general,'' Kim Min-woong, a political philosopher who teaches at Kyung
Hee University, said
\href{https://www.youtube.com/watch?v=d-bbuTfZuo4\&t=362s}{during a
rally} this month. ``We can win when we defend our flag and our
general.''

Image

Mr. Cho in Seoul this month. He and members of his family have been
accused of a variety of misdeeds, including financial
malfeasance.~Credit...Yonhap, via EPA and Shutterstock

The protests recall the
\href{https://www.nytimes3xbfgragh.onion/2016/11/13/world/asia/korea-park-geun-hye-protests.html}{huge
weekly rallies} in 2016 that preceded the impeachment and ouster of Mr.
Moon's predecessor, Park Geun-hye. Mr. Moon has not been accused of
wrongdoing, and his removal seems a distant prospect at best.

But the demonstrations show how polarized South Korean politics have
become, and they have cast a new light on the prosecutors' office ---
one of the country's most powerful and disliked institutions.

Prosecutors in South Korea have powers well beyond those of their
counterparts in most countries. They decide exclusively who is indicted
and who is not (South Korea has no grand jury system). They also have
authority over the police, and they reserve investigations of
politically sensitive cases for themselves, leaving the police to handle
more mundane matters.

Surveys have found that they are deeply mistrusted by the public, often
seen as doing the political bidding of whoever is in power. For decades,
every president --- including Mr. Moon --- has vowed to take politics
out of the agency, only to be later accused of using it to harass their
political opponents or divert attention from domestic crises.

``South Korea is a veritable republic of prosecutors,'' said Prof. Ha
Tae-hoon of Korea University School of Law, who called the agency a
``beast'' that had defied democratic progress. Another law professor,
Han Sang-hoon of Yonsei University, said the prosecutors' rigid
``command-and-compliance culture'' made internal checks and balances all
but impossible.

Few have condemned the prosecutors' shortcomings more vigorously than
Mr. Cho.

Articulate, good-looking and charismatic, he became a social media star
while teaching at Seoul National University School of Law, calling for
more social equality and high ethical standards for politicians. In
\href{https://www.youtube.com/watch?v=_kPLYE2ueOg}{a clip from a 2011
forum} that recently went viral, he said South Korea needed leadership
that ``refuses to join hands'' with prosecutors, and that a justice
minister should reform their office.

``But I warn you that if the minister tries to do that, prosecutors can
go after him, digging up dirt against him,'' Mr. Cho said. ``They are
the kind of group fully capable of shaking and toppling the minister by
spreading rumors against him.''

After Mr. Cho joined Mr. Moon's staff in 2017 as his chief legal
counsel, many saw him as a possible successor.

But his image soured drastically soon after Mr. Moon named him justice
minister in August. News outlets began reporting on allegations of
misdeeds by Mr. Cho or members of his family, including embezzlement and
trying to destroy evidence.

Most of the accusations remain unsubstantiated. But his wife has been
indicted on a charge of forging a certificate to help their daughter get
into medical school. Mr. Cho **** has denied any lawbreaking by family
members, but he acknowledged that his daughter had benefited from
advantages denied to other students --- a sensitive matter in a country
where anger over economic inequality runs high. College students began
holding rallies against him and calling him a hypocrite.

As public anger over the various accusations grew, the chief prosecutor,
Yoon Seok-yeol --- another Moon appointee --- assigned about 20
prosecutors to investigate Mr. Cho and his family.

His apartment was raided and his children questioned. His wife, already
under indictment, has been repeatedly interviewed by prosecutors seeking
possible links to a relative who has been arrested on suspicion of
embezzlement.

Image

A demonstration against Mr. Cho in Seoul on Wednesday.~The rival
protests show how polarized South Korean politics have
become.Credit...Jeon Heon-Kyun/EPA, via Shutterstock

As Mr. Cho's supporters see it, the prosecutors are punishing him for
finally enacting some of the reforms to their office that presidents
have been promising for years.

His ministry, for example, ended the humiliating practice of forcing
people to stand before a bank of news cameras before entering a
prosecutor's office for questioning. A bill that would create an agency
to investigate corruption among prosecutors, as well as other senior
officials, is pending in Parliament. The bill is perhaps the most
important part of the efforts by Mr. Moon and Mr. Cho to check
prosecutors' power.

But critics accuse Mr. Cho of trying to change the subject, noting that
most of those actions were taken only after the allegations against him
began getting attention.

Before Mr. Cho was appointed, prosecutors' relationship with the
presidency had seemed as cordial as ever, despite Mr. Moon's promises of
change. They pursued an anticorruption campaign initiated by the
president, which led to the imprisonment of two of his conservative
predecessors ---
\href{https://www.nytimes3xbfgragh.onion/2018/08/24/world/asia/park-geun-hye-sentenced-south-korea.html}{Ms.
Park}and
\href{http://nytimes3xbfgragh.onion/2018/10/05/world/asia/lee-myung-bak-south-korea-convicted.html}{Lee
Myung-bak} --- as well as
\href{https://www.nytimes3xbfgragh.onion/2019/01/23/world/asia/south-korea-chief-justice-japan.html}{a
former chief justice}of the Supreme Court.

Conservatives who had vilified Mr. Yoon, the top prosecutor, as Mr.
Moon's henchman are now hailing him as a hero for taking on Mr. Cho. And
liberals who had cheered the convictions of the conservative
ex-presidents are calling prosecutors ``wolves'' who are attempting ``a
coup d'état'' against Mr. Moon.

The rallies have drawn hundreds of thousands of people, at least;
organizers on both sides say their crowds have topped two million. One
side chants ``Let's defend Cho ​Kuk!'' (His name, as it happens, sounds
the same as the Korean word for ``fatherland.'') The other shouts,
``Arrest Cho Kuk!''

The anti-Cho rallies have been led by ​evangelical Christian activists,
joined by other conservatives --- mostly older people --- who have long
opposed Mr. Moon's economic policies and his conciliatory stance toward
North Korea. A recent protest
\href{https://www.youtube.com/watch?v=-Ym1zeAV5UA}{led by Mr. Jeon}
resembled a Christian revival meeting, with invocations of God's
blessing, choruses of ``Hallelujah!'' and staff members weaving through
the crowd with \href{https://www.youtube.com/watch?v=GGO4vYV0OQ4}{cash
donation boxes}.

\href{https://www.youtube.com/watch?v=7KxAqTnlBJU}{The pro-Cho crowds}
are more diverse, including many young, urban white-collar workers. Rock
bands have performed at many of the rallies, with people in the crowd
singing along and waving signs that read ``We are Cho Kuk!''

``Prosecutors think that if they can force Cho Kuk out, they can stop
the reform efforts and return to business as usual,'' said Hwang Gyo-ik,
a newspaper columnist, during a rally in support of the justice minister
on Saturday.

Hong Yoon-gi, a professor at Dongguk University, said that for many of
those now supporting Mr. Cho, anger over his alleged misdeeds had been
trumped by their loathing for the prosecutors.

``Prosecutors have always been an object of awe, hatred and fear,''
Professor Hong said. ``When people watched the way prosecutors conducted
their investigation of Cho Kuk's family, their hatred of his hypocrisy
was overtaken by their fear of the prosecutors' power.''

Advertisement

\protect\hyperlink{after-bottom}{Continue reading the main story}

\hypertarget{site-index}{%
\subsection{Site Index}\label{site-index}}

\hypertarget{site-information-navigation}{%
\subsection{Site Information
Navigation}\label{site-information-navigation}}

\begin{itemize}
\tightlist
\item
  \href{https://help.nytimes3xbfgragh.onion/hc/en-us/articles/115014792127-Copyright-notice}{©~2020~The
  New York Times Company}
\end{itemize}

\begin{itemize}
\tightlist
\item
  \href{https://www.nytco.com/}{NYTCo}
\item
  \href{https://help.nytimes3xbfgragh.onion/hc/en-us/articles/115015385887-Contact-Us}{Contact
  Us}
\item
  \href{https://www.nytco.com/careers/}{Work with us}
\item
  \href{https://nytmediakit.com/}{Advertise}
\item
  \href{http://www.tbrandstudio.com/}{T Brand Studio}
\item
  \href{https://www.nytimes3xbfgragh.onion/privacy/cookie-policy\#how-do-i-manage-trackers}{Your
  Ad Choices}
\item
  \href{https://www.nytimes3xbfgragh.onion/privacy}{Privacy}
\item
  \href{https://help.nytimes3xbfgragh.onion/hc/en-us/articles/115014893428-Terms-of-service}{Terms
  of Service}
\item
  \href{https://help.nytimes3xbfgragh.onion/hc/en-us/articles/115014893968-Terms-of-sale}{Terms
  of Sale}
\item
  \href{https://spiderbites.nytimes3xbfgragh.onion}{Site Map}
\item
  \href{https://help.nytimes3xbfgragh.onion/hc/en-us}{Help}
\item
  \href{https://www.nytimes3xbfgragh.onion/subscription?campaignId=37WXW}{Subscriptions}
\end{itemize}
