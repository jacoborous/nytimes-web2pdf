Sections

SEARCH

\protect\hyperlink{site-content}{Skip to
content}\protect\hyperlink{site-index}{Skip to site index}

\href{https://www.nytimes3xbfgragh.onion/section/politics}{Politics}

\href{https://myaccount.nytimes3xbfgragh.onion/auth/login?response_type=cookie\&client_id=vi}{}

\href{https://www.nytimes3xbfgragh.onion/section/todayspaper}{Today's
Paper}

\href{/section/politics}{Politics}\textbar{}California Requires Trump
Tax Returns Under New Election Law

\url{https://nyti.ms/2Yyq1IU}

\begin{itemize}
\item
\item
\item
\item
\item
\item
\end{itemize}

\begin{itemize}
\item
  \href{https://www.nytimes3xbfgragh.onion/live/2020/08/20/us/dnc-convention-election?action=click\&pgtype=Article\&state=default\&region=TOP_BANNER\&context=storylines_menu}{D.N.C.
  Updates}
\item
  \href{https://www.nytimes3xbfgragh.onion/2020/08/20/us/politics/biden-presidential-nomination-dnc.html?action=click\&pgtype=Article\&state=default\&region=TOP_BANNER\&context=storylines_menu}{Biden's
  Speech}
\item
  \href{https://www.nytimes3xbfgragh.onion/interactive/2019/us/elections/2020-presidential-election-calendar.html?action=click\&pgtype=Article\&state=default\&region=TOP_BANNER\&context=storylines_menu}{Election
  Calendar}
\item
  \href{https://www.nytimes3xbfgragh.onion/interactive/2020/08/11/us/politics/vote-by-mail-us-states.html?action=click\&pgtype=Article\&state=default\&region=TOP_BANNER\&context=storylines_menu}{Voting
  by Mail}
\item
  \href{https://www.nytimes3xbfgragh.onion/newsletters/politics?action=click\&pgtype=Article\&state=default\&region=TOP_BANNER\&context=storylines_menu}{Politics
  Newsletter}
\end{itemize}

Advertisement

\protect\hyperlink{after-top}{Continue reading the main story}

Supported by

\protect\hyperlink{after-sponsor}{Continue reading the main story}

\hypertarget{california-requires-trump-tax-returns-under-new-election-law}{%
\section{California Requires Trump Tax Returns Under New Election
Law}\label{california-requires-trump-tax-returns-under-new-election-law}}

\includegraphics{https://static01.graylady3jvrrxbe.onion/images/2019/07/30/us/politics/30california-returns/merlin_154999209_7e2b6f6b-23f8-4775-9162-085a575cf93e-articleLarge.jpg?quality=75\&auto=webp\&disable=upscale}

By \href{https://www.nytimes3xbfgragh.onion/by/jennifer-medina}{Jennifer
Medina} and
\href{https://www.nytimes3xbfgragh.onion/by/annie-karni}{Annie Karni}

\begin{itemize}
\item
  July 30, 2019
\item
  \begin{itemize}
  \item
  \item
  \item
  \item
  \item
  \item
  \end{itemize}
\end{itemize}

LOS ANGELES --- President Trump will not be eligible for California's
primary ballot unless he releases his tax returns, under a new law
signed by Gov. Gavin Newsom on Tuesday.

The law requires that all presidential candidates release their tax
returns in order to be placed on the ballot for the state's primary next
year, in a move that will almost certainly lead to legal challenges. Mr.
Newsom's decision to sign the legislation seemed designed to escalate a
running feud between the White House and California.

The state is currently involved in more than 40 lawsuits with the Trump
administration on issues ranging from environmental regulation to
immigration.

The California State Legislature approved a similar measure in 2017, but
Gov. Jerry Brown vetoed it, questioning whether it was constitutional.
Mr. Brown, who left office in January, also said it would create a
precedent for requiring other information --- including medical records
or certified birth certificates --- from candidates.

Mr. Newsom sent mixed messages on whether he would sign the law, but
finally did so on the final day before the bill would become law without
his signature. The legislation does not explicitly cite Mr. Trump, but
lawmakers made no secret that he was the target when they passed the
bill along party lines.

\emph{{[}Follow our coverage of the}
\href{https://www.nytimes3xbfgragh.onion/2019/07/29/us/politics/democratic-debate-schedule-lineup.html?action=click\&module=Intentional\&pgtype=Article}{\emph{Democratic
primary debates}}\emph{.{]}}

The law, which goes into effect immediately, requires candidates for
president or governor to submit copies of their tax returns from the
last five years with the California secretary of state at least three
months ahead of the state's primary. That means Mr. Trump would have to
provide his tax returns by the end of this year.

``These are extraordinary times and states have a legal and moral duty
to do everything in their power to ensure leaders seeking the highest
offices meet minimal standards, and to restore public confidence,'' Mr.
Newsom said in a statement as he signed the legislation. ``The
disclosure required by this bill will shed light on conflicts of
interest, self-dealing, or influence from domestic and foreign business
interest.''

\hypertarget{latest-updates-2020-election}{%
\section{\texorpdfstring{\href{https://www.nytimes3xbfgragh.onion/live/2020/08/19/us/dnc-convention-election?action=click\&pgtype=Article\&state=default\&region=MAIN_CONTENT_1\&context=storylines_live_updates}{Latest
Updates: 2020
Election}}{Latest Updates: 2020 Election}}\label{latest-updates-2020-election}}

\href{https://www.nytimes3xbfgragh.onion/live/2020/08/19/us/dnc-convention-election?action=click\&pgtype=Article\&state=default\&region=MAIN_CONTENT_1\&context=storylines_live_updates\#night-3-featured-more-policy-a-focus-on-women-and-a-full-throated-rejection-of-trump-by-his-predecessor}{7h
ago}

\href{https://www.nytimes3xbfgragh.onion/live/2020/08/19/us/dnc-convention-election?action=click\&pgtype=Article\&state=default\&region=MAIN_CONTENT_1\&context=storylines_live_updates\#night-3-featured-more-policy-a-focus-on-women-and-a-full-throated-rejection-of-trump-by-his-predecessor}{Night
3 featured more policy, a focus on women and a full-throated rejection
of Trump by his predecessor.}

\href{https://www.nytimes3xbfgragh.onion/live/2020/08/19/us/dnc-convention-election?action=click\&pgtype=Article\&state=default\&region=MAIN_CONTENT_1\&context=storylines_live_updates\#trump-live-tweeted-obamas-speech-tonight-hell-appear-on-fox-news-right-before-bidens-tomorrow}{9h
ago}

\href{https://www.nytimes3xbfgragh.onion/live/2020/08/19/us/dnc-convention-election?action=click\&pgtype=Article\&state=default\&region=MAIN_CONTENT_1\&context=storylines_live_updates\#trump-live-tweeted-obamas-speech-tonight-hell-appear-on-fox-news-right-before-bidens-tomorrow}{Trump
live-tweeted Obama's speech tonight. He'll appear on Fox News right
before Biden's tomorrow.}

\href{https://www.nytimes3xbfgragh.onion/live/2020/08/19/us/dnc-convention-election?action=click\&pgtype=Article\&state=default\&region=MAIN_CONTENT_1\&context=storylines_live_updates\#advocates-for-domestic-violence-survivors-praised-biden-in-a-video}{9h
ago}

\href{https://www.nytimes3xbfgragh.onion/live/2020/08/19/us/dnc-convention-election?action=click\&pgtype=Article\&state=default\&region=MAIN_CONTENT_1\&context=storylines_live_updates\#advocates-for-domestic-violence-survivors-praised-biden-in-a-video}{Advocates
for domestic violence survivors praised Biden in a video.}

\href{https://www.nytimes3xbfgragh.onion/live/2020/08/19/us/dnc-convention-election?action=click\&pgtype=Article\&state=default\&region=MAIN_CONTENT_1\&context=storylines_live_updates}{See
more updates}

The governor cited several legal scholars who signaled support for such
a requirement, but it will probably be left to the courts to decide.

Tim Murtaugh, a spokesman for the Trump campaign, declined to comment on
potential lawsuits, but called the legislation unconstitutional.

``The Constitution is clear on the qualifications for someone to serve
as president and states cannot add additional requirements on their
own,'' Mr. Murtaugh said. ``The bill also violates the First Amendment
right of association since California can't tell political parties which
candidates their members can or cannot vote for in a primary election.''

The Trump campaign, which has been closely tracking ballot access issues
for months and coordinating with the White House Counsel's Office, is
likely to respond with a lawsuit, according to an official with the
campaign. That suit could potentially include a number of plaintiffs,
including the Republican National Committee, the California Republican
Party and the Trump campaign, but the official warned that nothing about
a suit had been finalized.

Nearly a dozen similar bills are active in other states, including New
York, New Jersey, Washington and Pennsylvania, according to the National
Conference of State Legislatures.

The vast majority of presidential nominees in the last several decades
have released their tax returns, with the exception of Gerald Ford in
1976. Mr. Brown and his Republican opponents declined to release their
returns during the governor's races in 2010 and 2014.

Mr. Brown warned in his veto message that such legislation would be a
slippery slope.

``A qualified candidate's ability to appear on the ballot is fundamental
to our democratic system,'' he wrote. ``For that reason, I hesitate to
start down a road that well might lead to an ever escalating set of
differing state requirements for presidential candidates.''

After Democratic state senators introduced the same legislation this
year, aides to Mr. Newsom asked the sponsors to add the requirements for
candidates for governor as well.

``This is a huge step forward for financial transparency from people who
are trying to become the most powerful person in the world,'' said Scott
Wiener, the state senator from San Francisco who sponsored the bill.
``It is absolutely not just written for Donald Trump. This is for every
Democratic and Republican and presidential candidate until the end of
time.''

Mr. Wiener said that the requirement would apply only to the primary, a
decision sponsors made ``because it strikes the right balance.''

``It creates a strong incentive for a candidate to disclose their tax
returns,'' he said. ``Losing California's large number of primary
delegates is significant, while ensuring that someone who is a party's
nominee isn't kept off the ballot in the general election.

While Mr. Trump remains deeply unpopular in California, Mr. Newsom could
face a backlash for escalating the state's longstanding feud with the
president in a way that even some Democrats believe is a distraction.
Blair Ellis, a spokesman for the Republican National Committee, accused
California officials of ``trying to deny voting rights to the millions
of Californians who support President Trump and wish to vote for him in
the primary.''

``Instead of trying to beat President Trump at the ballot box next
November, he said, ``Democrats are resorting to gimmicky tactics that
are unconstitutional, undemocratic and just plain dumb.''

Mr. Wiener introduced the legislation in 2017, after a conversation with
Brad Hoylman, a state senator in New York who had been a classmate at
Harvard Law School. Mr. Hoylman's legislation stalled in Albany, but his
office has tracked similar bills in 30 states in the last two years.

``We now can point to California as a model for the substance and the
politics of passing this innovative concept into law,'' Mr. Hoylman
said. ``That will give us an enormous boost of credibility with my
colleagues.''

Erwin Chemerinsky, the dean of the law school at the University of
California, Berkeley, and a First Amendment expert, said he was
confident the state was on firm legal ground. He said that states also
have the right to make a similar requirement for a general election
ballot, and that he hoped other states would do so.

``The Supreme Court has said that states have broad latitude over who is
going to be on the ballot so long as they aren't discriminating based on
wealth and ideology,'' he said. ``I think the state has an important
interest in that the tax returns can provide vital information to
voter.''

But by asserting themselves in national elections, states find
themselves in uncertain territory, said Richard H. Pildes, a professor
of constitutional law at New York University.

``There's no question there are serious constitutional issues that are
posed by this, particularly because it is a national election and it has
implications beyond the state of California,'' Mr. Pildes said. ``What
other kinds of regulations can one imagine that states might impose on
presidential candidates to get onto the ballot?''

\hypertarget{our-2020-election-guide}{%
\section{Our 2020 Election Guide}\label{our-2020-election-guide}}

Updated Aug. 20, 2020

\begin{itemize}
\item
  \begin{center}\rule{0.5\linewidth}{\linethickness}\end{center}

  \hypertarget{convention-recap}{%
  \subsection{Convention Recap}\label{convention-recap}}

  \begin{itemize}
  \tightlist
  \item
    Joe Biden accepted the Democratic nomination, urging Americans to
    have faith that they could
    \href{https://www.nytimes3xbfgragh.onion/2020/08/20/us/politics/Joe-Biden-accepts-democratic-nomination.html?action=click\&pgtype=Article\&state=default\&region=BELOW_MAIN_CONTENT\&context=storylines_guide}{``overcome
    this season of darkness.''}
  \end{itemize}
\item
  \begin{center}\rule{0.5\linewidth}{\linethickness}\end{center}

  \hypertarget{news-analysis}{%
  \subsection{News Analysis}\label{news-analysis}}

  \begin{itemize}
  \tightlist
  \item
    Looming over Mr. Biden's nomination was the ever-present shadow of
    another man who's poised to dominate the campaign:
    \href{https://www.nytimes3xbfgragh.onion/2020/08/20/us/politics/biden-dnc-speech-trump.html?action=click\&pgtype=Article\&state=default\&region=BELOW_MAIN_CONTENT\&context=storylines_guide}{Donald
    J. Trump}.
  \end{itemize}
\item
  \begin{center}\rule{0.5\linewidth}{\linethickness}\end{center}

  \hypertarget{keep-up-with-our-coverage}{%
  \subsection{Keep Up With Our
  Coverage}\label{keep-up-with-our-coverage}}

  \begin{itemize}
  \tightlist
  \item
    Get an
    \href{https://www.nytimes3xbfgragh.onion/newsletters/politics?action=click\&pgtype=Article\&state=default\&region=BELOW_MAIN_CONTENT\&context=storylines_guide}{email}
    recapping the day's news
  \end{itemize}

  \begin{itemize}
  \tightlist
  \item
    Download our mobile app on
    \href{https://apps.apple.com/us/app/nytimes/id284862083?ls=1\&mat_click_id=5c79ae7455014fd1bd66b5610c05b8f2-20191112-16948\&referrer=mat_click_id\%3D5c79ae7455014fd1bd66b5610c05b8f2-20191112-16948\%26link_click_id\%3D722930677036718082}{iOS}
    and
    \href{http://a.localytics.com/android?id=com.nytimes.android\&referrer=utm_source\%3Dother_nyt_mobile_web\%26utm_medium\%3DWeb\%2520page\%26utm_term\%3DGeneral\%2520Mobile\%2520Page\%26utm_campaign\%3DNYT\%2520Mobile\%2520General\%2520Page}{Android}
    and turn on Breaking News and Politics alerts
  \end{itemize}
\end{itemize}

Advertisement

\protect\hyperlink{after-bottom}{Continue reading the main story}

\hypertarget{site-index}{%
\subsection{Site Index}\label{site-index}}

\hypertarget{site-information-navigation}{%
\subsection{Site Information
Navigation}\label{site-information-navigation}}

\begin{itemize}
\tightlist
\item
  \href{https://help.nytimes3xbfgragh.onion/hc/en-us/articles/115014792127-Copyright-notice}{©~2020~The
  New York Times Company}
\end{itemize}

\begin{itemize}
\tightlist
\item
  \href{https://www.nytco.com/}{NYTCo}
\item
  \href{https://help.nytimes3xbfgragh.onion/hc/en-us/articles/115015385887-Contact-Us}{Contact
  Us}
\item
  \href{https://www.nytco.com/careers/}{Work with us}
\item
  \href{https://nytmediakit.com/}{Advertise}
\item
  \href{http://www.tbrandstudio.com/}{T Brand Studio}
\item
  \href{https://www.nytimes3xbfgragh.onion/privacy/cookie-policy\#how-do-i-manage-trackers}{Your
  Ad Choices}
\item
  \href{https://www.nytimes3xbfgragh.onion/privacy}{Privacy}
\item
  \href{https://help.nytimes3xbfgragh.onion/hc/en-us/articles/115014893428-Terms-of-service}{Terms
  of Service}
\item
  \href{https://help.nytimes3xbfgragh.onion/hc/en-us/articles/115014893968-Terms-of-sale}{Terms
  of Sale}
\item
  \href{https://spiderbites.nytimes3xbfgragh.onion}{Site Map}
\item
  \href{https://help.nytimes3xbfgragh.onion/hc/en-us}{Help}
\item
  \href{https://www.nytimes3xbfgragh.onion/subscription?campaignId=37WXW}{Subscriptions}
\end{itemize}
