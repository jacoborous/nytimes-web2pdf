Sections

SEARCH

\protect\hyperlink{site-content}{Skip to
content}\protect\hyperlink{site-index}{Skip to site index}

\href{https://www.nytimes3xbfgragh.onion/section/politics}{Politics}

\href{https://myaccount.nytimes3xbfgragh.onion/auth/login?response_type=cookie\&client_id=vi}{}

\href{https://www.nytimes3xbfgragh.onion/section/todayspaper}{Today's
Paper}

\href{/section/politics}{Politics}\textbar{}Iran Breaches Critical Limit
on Nuclear Fuel Set by 2015 Deal

\url{https://nyti.ms/2KRoPxU}

\begin{itemize}
\item
\item
\item
\item
\item
\item
\end{itemize}

Advertisement

\protect\hyperlink{after-top}{Continue reading the main story}

Supported by

\protect\hyperlink{after-sponsor}{Continue reading the main story}

\hypertarget{iran-breaches-critical-limit-on-nuclear-fuel-set-by-2015-deal}{%
\section{Iran Breaches Critical Limit on Nuclear Fuel Set by 2015
Deal}\label{iran-breaches-critical-limit-on-nuclear-fuel-set-by-2015-deal}}

\includegraphics{https://static01.graylady3jvrrxbe.onion/images/2019/06/26/us/00dc-iranHFO-1/merlin_157023003_794f0493-87e3-43c4-861d-09be0703c417-articleLarge.jpg?quality=75\&auto=webp\&disable=upscale}

By \href{https://www.nytimes3xbfgragh.onion/by/david-e-sanger}{David E.
Sanger}

\begin{itemize}
\item
  July 1, 2019
\item
  \begin{itemize}
  \item
  \item
  \item
  \item
  \item
  \item
  \end{itemize}
\end{itemize}

WASHINGTON --- Iran on Monday violated a key provision of the 2015
international accord to restrict its nuclear program and signaled that
it would soon breach another as it seeks more leverage in its escalating
confrontation with the United States.

International inspectors confirmed that Iran had exceeded a critical
limit on how much nuclear fuel it can possess under the agreement, which
\href{https://www.nytimes3xbfgragh.onion/2018/05/08/world/middleeast/trump-iran-nuclear-deal.html}{President
Trump abandoned} more than a year ago. By itself, the move does not give
Iran enough material to produce a single nuclear weapon, though it
inches it in that direction.

Hours later, Iran's foreign minister said his nation now intended to
begin enriching its nuclear fuel to a purer level, a provocative action
that, depending on how far Tehran goes with it, could move the country
closer to possessing fuel that with further processing could be used in
a weapon.

The moves completed a sharp shift in strategy for Iran, which for the
past 14 months had continued to respect the terms of the complex deal it
struck with the Obama administration, even after Mr. Trump reimposed
sanctions in an effort to strangle Iran's economy by driving its oil
revenues to zero. President Hassan Rouhani of Iran signaled the change
in approach in May, but Tehran did not actually breach a central element
of the agreement until Monday.

But while the moves appear to return Iran to its two-decade pursuit of
the technology necessary to develop a nuclear arsenal, the real goal may
have been to gain a diplomatic advantage for any future negotiations.
Iranian leaders are betting they can force European countries, who were
deeply critical of Mr. Trump's scrapping of the nuclear deal, to deliver
on promises to help compensate Tehran for the effects of American
sanctions.

Mr. Trump, who has vowed that Iran will never get a nuclear weapon, told
reporters that Iran was ``playing with fire,'' and in a statement the
State Department criticized Iran's moves as an effort ``to extort the
international community and threaten regional security.''

The administration has insisted that Iran continue to abide by the 2015
deal's terms, even though Mr. Trump was the first to repudiate it,
imposing escalating sanctions that are spurring high inflation and deep
budget cuts in Iran.

But the administration made no overt threats of military action. Iran's
bit-by-bit violations of the accord are all reversible, and it is not
clear how much either side wants to further escalate given that tensions
have already been running high after the downing of an American
surveillance drone by Iran last month
\href{https://www.nytimes3xbfgragh.onion/2019/06/20/world/middleeast/iran-us-drone.html}{nearly
resulted in military strikes}.

\includegraphics{https://static01.graylady3jvrrxbe.onion/images/2019/07/01/world/01dc-iran3/merlin_154544277_9ffef66e-15c0-42f2-bbfa-5a115bb2569b-articleLarge.jpg?quality=75\&auto=webp\&disable=upscale}

Iran's moves nonetheless brought expressions of concern from American
allies, some of whom fear Washington and Tehran are on a collision
course.

``Deeply worried by Iran's announcement that it has broken existing
nuclear deal obligations,'' Jeremy Hunt, the British foreign minister
and a contender for prime minister, said in a tweet. He said that
Britain ``remains committed to making deal work \& using all diplomatic
tools.''

Israel's prime minister, Benjamin Netanyahu, who lobbied Congress hard
to defeat the deal four years ago, put the move in far more dire terms.

``Iran is taking a significant step toward producing nuclear weapons,''
he said at a ceremony honoring reserve units of the Israel Defense
Forces. ``Israel will not allow Iran to develop nuclear weapons.''

He urged Europe to impose ``snapback'' sanctions against Iran, under
provisions that were written into the arrangement to deal with
violations.

But European officials have long argued that Mr. Trump essentially
pushed the Iranians into the violations, and they are likely to be
divided on the question of whether to pursue sanctions that would most
likely terminate the arrangement entirely. The Iranians argue that they
are under no obligations to adhere to the deal's terms since Mr. Trump
abandoned the pact.

``The E.U. remains fully committed to the agreement as long as Iran
continues to fully implement its nuclear commitments,'' said Maja
Kocijancic, a spokeswoman for the European Union, adding that Iran had
complied with the deal for 14 months after the United States'
withdrawal. ``We urge Iran to reverse this step and to refrain from
further measures that undermine the nuclear deal,'' she said.

The 2015 agreement with Iran was negotiated by the United States under
President Barack Obama along with Britain, France, Germany, Russia and
China. The European powers have been trying to keep Iran in the deal
even after the withdrawal of the United States, but negotiations on a
possible agreement for Europe to help Iran financially by coming up with
a workaround to some American sanctions ended inconclusively last week.

The breach of the limit on how much nuclear fuel Iran can possess
restricted its stockpile of low-enriched uranium to about 660 pounds.
The decision was the strongest warning yet that Iran may be willing to
rebuild the far larger stockpile that it agreed to send abroad under the
deal.

Shortly after Iranian news agencies and the International Atomic Energy
Agency confirmed that Iran had exceeded the stockpile limit, Mohammad
Javad Zarif, the country's foreign minister and the man who negotiated
the agreement with the Obama administration, said Iran would now turn to
enriching the nuclear fuel.

Image

President Trump, who has vowed that Iran will never get a nuclear
weapon, told reporters that Tehran was ``playing with
fire.''Credit...Gabriella Demczuk for The New York Times

``Our next step will be enriching uranium beyond the 3.67 percent
allowed under the deal,'' he said, according to a state-run Iranian
broadcaster. He blamed the Europeans, who he said ``have failed to
fulfill their promises of protecting Iran's interests'' by compensating
for billions of dollars in losses to the Iranian economy caused by the
American sanctions.

The enrichment level limit in the 2015 deal was set to assure that
Iran's small amount of fuel could be useful only in producing nuclear
energy, not a bomb. Higher enrichment levels take Iran closer to making
the kind of material needed for a bomb --- which requires something
closer to 90 percent purity.

Iran has consistently denied that it has any intention of making a
nuclear weapon, but a trove of nuclear-related documents, spirited out
of a Tehran warehouse by Israeli agents last year, showed extensive work
before 2003 to design a nuclear warhead.

Mr. Trump said last month that any effort by Iran to race to build a
bomb might prompt him to take military action. But the move signaled by
Iran on Monday fell far short of that threshold, and could have been
intended to impress on the Europeans the importance of returning to
negotiations over giving Tehran some relief from the sanctions.

Secretary of State Mike Pompeo said in May that the United States would
never allow Iran to get within one year of possessing enough fuel to
produce a nuclear weapon.

His special envoy for Iran, Brian H. Hook, has often said that under a
new deal, the United States would insist on
``\href{https://www.nytimes3xbfgragh.onion/2019/05/17/us/politics/trump-iran-nuclear-deal.html?module=inline}{zero
enrichment for Iran}.'' Mr. Hook has estimated that the sanctions have
cost Iran \$50 billion in lost oil sales, far more than the system the
Europeans are putting in place would generate.

Iran has so far rejected beginning any negotiation with Washington,
saying that the United States must first return to the 2015 agreement
and comply with all of its terms.

In fact, there is an argument to be made that Mr. Trump pushed Iran into
exceeding the stockpile limit. Among the recently imposed sanctions was
one that threatened action against any country that bought low-enriched
uranium from Tehran. To comply with the stockpile limits, Iran shipped
low-enriched uranium to Russia in return for natural uranium. With that
exchange now barred, it was only a matter of time before Iran exceeded
the limits.

Even before the announcement, the Pentagon and the United States'
intelligence agencies --- led by the C.I.A. and the National Security
Agency --- were beginning to review what steps to take if the president
determined that Iran was getting too close to producing a bomb.

But any operation against Iran's nuclear infrastructure, with either
conventional arms or cyberweapons, would be highly risky. And some
administration officials warn that acting now would be premature. Even
if Iran possesses 800 or 900 kilograms of uranium, it would be
insufficient for a single bomb. That threshold is not likely to be
crossed until later this summer.

Advertisement

\protect\hyperlink{after-bottom}{Continue reading the main story}

\hypertarget{site-index}{%
\subsection{Site Index}\label{site-index}}

\hypertarget{site-information-navigation}{%
\subsection{Site Information
Navigation}\label{site-information-navigation}}

\begin{itemize}
\tightlist
\item
  \href{https://help.nytimes3xbfgragh.onion/hc/en-us/articles/115014792127-Copyright-notice}{©~2020~The
  New York Times Company}
\end{itemize}

\begin{itemize}
\tightlist
\item
  \href{https://www.nytco.com/}{NYTCo}
\item
  \href{https://help.nytimes3xbfgragh.onion/hc/en-us/articles/115015385887-Contact-Us}{Contact
  Us}
\item
  \href{https://www.nytco.com/careers/}{Work with us}
\item
  \href{https://nytmediakit.com/}{Advertise}
\item
  \href{http://www.tbrandstudio.com/}{T Brand Studio}
\item
  \href{https://www.nytimes3xbfgragh.onion/privacy/cookie-policy\#how-do-i-manage-trackers}{Your
  Ad Choices}
\item
  \href{https://www.nytimes3xbfgragh.onion/privacy}{Privacy}
\item
  \href{https://help.nytimes3xbfgragh.onion/hc/en-us/articles/115014893428-Terms-of-service}{Terms
  of Service}
\item
  \href{https://help.nytimes3xbfgragh.onion/hc/en-us/articles/115014893968-Terms-of-sale}{Terms
  of Sale}
\item
  \href{https://spiderbites.nytimes3xbfgragh.onion}{Site Map}
\item
  \href{https://help.nytimes3xbfgragh.onion/hc/en-us}{Help}
\item
  \href{https://www.nytimes3xbfgragh.onion/subscription?campaignId=37WXW}{Subscriptions}
\end{itemize}
