Sections

SEARCH

\protect\hyperlink{site-content}{Skip to
content}\protect\hyperlink{site-index}{Skip to site index}

\href{https://www.nytimes3xbfgragh.onion/section/politics}{Politics}

\href{https://myaccount.nytimes3xbfgragh.onion/auth/login?response_type=cookie\&client_id=vi}{}

\href{https://www.nytimes3xbfgragh.onion/section/todayspaper}{Today's
Paper}

\href{/section/politics}{Politics}\textbar{}Texas Latina Emerges as
House's Voice of Passion and Reason on the Border

\url{https://nyti.ms/2Ler1Qg}

\begin{itemize}
\item
\item
\item
\item
\item
\item
\end{itemize}

Advertisement

\protect\hyperlink{after-top}{Continue reading the main story}

Supported by

\protect\hyperlink{after-sponsor}{Continue reading the main story}

\hypertarget{texas-latina-emerges-as-houses-voice-of-passion-and-reason-on-the-border}{%
\section{Texas Latina Emerges as House's Voice of Passion and Reason on
the
Border}\label{texas-latina-emerges-as-houses-voice-of-passion-and-reason-on-the-border}}

\includegraphics{https://static01.graylady3jvrrxbe.onion/images/2019/07/11/us/11dc-escobar-2/merlin_157517913_85a78eb9-3a9f-4f24-a7d5-0bace0eb3115-articleLarge.jpg?quality=75\&auto=webp\&disable=upscale}

By \href{https://www.nytimes3xbfgragh.onion/by/emily-cochrane}{Emily
Cochrane}

\begin{itemize}
\item
  July 11, 2019
\item
  \begin{itemize}
  \item
  \item
  \item
  \item
  \item
  \item
  \end{itemize}
\end{itemize}

\href{https://www.nytimes3xbfgragh.onion/es/2019/07/12/veronica-escobar-migrantes-eeuu}{Leer
en español}

CIUDAD JUÁREZ, Mexico --- In the blazing Mexican heat last week,
Representative Veronica Escobar consoled a young woman named Fatima, who
described in a flood of Spanish how she had been raped in her native
Nicaragua, separated by American authorities from her 5-year-old
daughter in May, and sent back to Mexico to wait alone on her asylum
claim. Her only wish now, Fatima said, was to be with her daughter
again.

``How lucky people like me are to have been born on the other side of
that skinny river,'' Ms. Escobar, a freshman Democrat from El Paso, said
later, after crossing to the American side of the Rio Grande.

Ms. Escobar, one of the first of two Latina women to represent Texas in
the House, has become a leader in the bruising, emotional border debate
on Capitol Hill. Elevated as a voice of authority by Speaker Pelosi, Ms.
Escobar has been passionate but also less confrontational than some of
the other freshmen when highlighting President Trump's hard-line
immigration policies.

On Friday, she will take another televised star turn as one of the main
witnesses at a House Oversight and Reform Committee hearing on migrant
child separation. Her proposal, aimed at toughening oversight provisions
for the Department of Homeland Security, is one of the pieces of
immigration legislation that Ms. Pelosi plans to move forward.

``She's a person who listens to other people; other people listen to
her,'' Ms. Pelosi said on Thursday. ``She's quite a spectacular member
of Congress.''

Ms. Escobar's prominence and activism have opened her up to attacks. On
Monday, she said she was receiving death threats because of reports that
her office was coaching asylum-seekers in Mexico to help them return to
the United States, reports she dismissed as ``fueled by xenophobia and
misinformation.''

But for her, the spotlight has been a chance to represent her heavily
Latino border district and to explain the realities of a vibrant,
binational region that is home to migrants and immigration agents alike.

Hours after her encounter with Fatima, Ms. Escobar and her husband, a
federal immigration judge, were at the minor league El Paso Chihuahuas
baseball game, an annual ritual to celebrate their anniversary, when
tears began to roll down the congresswoman's cheeks as triumphant
fireworks burst overhead. Her mind had returned to Fatima and all of the
migrants she had seen in Mexico, she said, and the detained women who
had wept in the arms of lawmakers two days earlier at El Paso Border
Patrol Station No. 1 in Texas.

\includegraphics{https://static01.graylady3jvrrxbe.onion/images/2019/07/11/us/11dc-escobar-3/merlin_157517871_fe7774f0-c270-47c9-9aa3-86adbf328e57-articleLarge.jpg?quality=75\&auto=webp\&disable=upscale}

``People are locked out of what we did absolutely nothing to earn at
birth,'' Ms. Escobar, a third-generation El Pasoan, later said, wiping
her eyes as she described the migrants she has met over the last six
months. ``I've never cried so much in my life. I just feel it so deeply,
and so profoundly.''

The border, as she tells it, has always been a magical place where a
mere line separates the El Paso dairy farm where she grew up and the
Juárez streets where she would shop with her mother, and where her
brothers would get terrible haircuts for \$2.

The realization that the border carried a more ominous connotation
outside of El Paso didn't register to her, she said, until the early
1990s, when she returned from New York University, where she earned a
master's degree in English literature in part by writing about the two
worlds she easily crossed.

She volunteered with a local immigration group and political campaigns
before running for office herself to be part of the El Paso County
governing body. In 2017, when Beto O'Rourke gave up his House seat for
an unsuccessful run against Senator Ted Cruz, she
\href{https://www.youtube.com/watch?v=vv8L7EeEAj8}{decided to run} for
Congress.

Now, when she strides across the bridge from Mexico in high heels in
order to avoid the traffic jam at the port of entry, she flashes her
identification at Border Patrol agents and shakes her head at the added
security --- the fences, the increase in manpower and the closed-off
traffic lanes.

She
\href{https://www.nytimes3xbfgragh.onion/2013/02/11/opinion/gridlock-on-the-rio-grande.html?searchResultPosition=24}{voiced
displeasure} with the Obama administration's approach to immigration
reform, but argues that the Trump administration's policies have
worsened the strain on El Paso's resources and ability to support the
influx of migrants.

``We are literally on the front line of an administration that
intentionally uses cruelty in communities like mine,'' Ms. Escobar said.
``There's always a sense of urgency.''

In June 2017, Jeff Sessions, then the attorney general,
\href{https://www.justice.gov/eoir/pr/executive-office-immigration-review-swears-11-immigration-judges}{appointed}
her husband, Michael Pleters, to the El Paso Service Processing Center
immigration court, which had one of the
\href{https://www.justice.gov/eoir/page/file/1107056/download}{lowest
rates} for approving asylum cases in fiscal year 2017. Ms. Escobar notes
that he had been offered the position during the final months of the
Obama administration and was formally approved after the change in
leadership.

Primary opponents argued that her husband's position tainted her
immigration record because he was tasked with upholding Trump
administration policies. She argued that he was doing his job, and she
was the one running to change those laws.

The accusations trailed her to Washington, where she said another member
made a comment early on about her husband's career.

Image

Ms. Escobar and Ms. Byrd spoke with a policy research analyst from the
Hope Border Institute when they visited a migrant shelter in Ciudad
Juárez in early July.Credit...Adria Malcolm for The New York Times

``I just remember thinking, `Aw jeez, are you serious? The campaign's
over,''' she said. ``People who are critics of an immigration judge
don't understand the significance of moving people through a judicial
process.''

But the controversy surrounding her husband reflects the tortured debate
within Congress, where lawmakers are torn over how to fund enforcement
efforts at the southwestern border. While a quartet of liberal
congresswomen, led by Representative Alexandria Ocasio-Cortez of New
York, voted against sending any humanitarian aid to the border, Ms.
Escobar quietly pushed behind the scenes for changes to the bill that
would add conditions to the funds. Then, when those restrictions were
stripped out, she voted against the final bill.

Her work has earned her the trust of House leadership. On Thursday
afternoon, she joined committee leaders and other top lawmakers for a
meeting about the next steps on immigration reform.

But the bruising debate and its rancorous aftermath have made her job
much more difficult. She must counter what she sees as ever-more
distorted misconceptions about the border from the right while beating
back angrier calls from the left to strip all funding from immigration
enforcement agencies.

``It's also on my side of the aisle, the danger of using blanket
stereotypes and generalizations for all people inside agencies when
Trump is the president,'' she said. ``I can't tell you how many times
I've heard people say all Border Patrol agents are bad, and if there's
any good ones left, they should quit.''

``It kills me,'' she added, ``because I talk to the good ones.''

Ms. Escobar's passion has earned her some bipartisan respect,
particularly late last month, after she led an emotional moment of
silence on the House floor for the tens of thousands of migrants who
have died trying to seek asylum in the United States. (Republicans are
quick to point out, however, that they disagree with her liberal
policies.)

``In Spanish, the term is called confianza,'' said Representative Raul
Ruiz, Democrat of California, who presented his legislation about
improving migrant health care to Ms. Pelosi alongside Ms. Escobar.
``It's like she's in the family --- somebody trustworthy, somebody you
can feel comfortable around.''

Members also rallied around her last week after a visit to the Clint
detention facility, where she sparred with Homeland Security officials
over derogatory posts about her in a Facebook group for current and
former Border Patrol agents, then faced a handful of angry conservative
protesters during a news conference.

As antagonists heckled ``One-Term Veronica'' and taunted Ms. Escobar for
appearing to support migrant children more than El Paso's children,
Representative Rashida Tlaib, Democrat of Michigan, yelled across the
podium that ``Veronica Escobar is the best goddamn congresswoman'' to
emphatic nods from her colleagues.

As the group of lawmakers drove away, the screams of the protesters
fading behind them, Ms. Escobar could be seen through a van window,
pressing her hands together in prayer and thanking the agents escorting
them.

Advertisement

\protect\hyperlink{after-bottom}{Continue reading the main story}

\hypertarget{site-index}{%
\subsection{Site Index}\label{site-index}}

\hypertarget{site-information-navigation}{%
\subsection{Site Information
Navigation}\label{site-information-navigation}}

\begin{itemize}
\tightlist
\item
  \href{https://help.nytimes3xbfgragh.onion/hc/en-us/articles/115014792127-Copyright-notice}{©~2020~The
  New York Times Company}
\end{itemize}

\begin{itemize}
\tightlist
\item
  \href{https://www.nytco.com/}{NYTCo}
\item
  \href{https://help.nytimes3xbfgragh.onion/hc/en-us/articles/115015385887-Contact-Us}{Contact
  Us}
\item
  \href{https://www.nytco.com/careers/}{Work with us}
\item
  \href{https://nytmediakit.com/}{Advertise}
\item
  \href{http://www.tbrandstudio.com/}{T Brand Studio}
\item
  \href{https://www.nytimes3xbfgragh.onion/privacy/cookie-policy\#how-do-i-manage-trackers}{Your
  Ad Choices}
\item
  \href{https://www.nytimes3xbfgragh.onion/privacy}{Privacy}
\item
  \href{https://help.nytimes3xbfgragh.onion/hc/en-us/articles/115014893428-Terms-of-service}{Terms
  of Service}
\item
  \href{https://help.nytimes3xbfgragh.onion/hc/en-us/articles/115014893968-Terms-of-sale}{Terms
  of Sale}
\item
  \href{https://spiderbites.nytimes3xbfgragh.onion}{Site Map}
\item
  \href{https://help.nytimes3xbfgragh.onion/hc/en-us}{Help}
\item
  \href{https://www.nytimes3xbfgragh.onion/subscription?campaignId=37WXW}{Subscriptions}
\end{itemize}
