Sections

SEARCH

\protect\hyperlink{site-content}{Skip to
content}\protect\hyperlink{site-index}{Skip to site index}

\href{https://myaccount.nytimes3xbfgragh.onion/auth/login?response_type=cookie\&client_id=vi}{}

\href{https://www.nytimes3xbfgragh.onion/section/todayspaper}{Today's
Paper}

\href{/section/opinion}{Opinion}\textbar{}A Major Police Body Cam
Company Just Banned Facial Recognition

\url{https://nyti.ms/2X0iOAI}

\begin{itemize}
\item
\item
\item
\item
\item
\item
\end{itemize}

Advertisement

\protect\hyperlink{after-top}{Continue reading the main story}

\href{/section/opinion}{Opinion}

Supported by

\protect\hyperlink{after-sponsor}{Continue reading the main story}

\hypertarget{a-major-police-body-cam-company-just-banned-facial-recognition}{%
\section{A Major Police Body Cam Company Just Banned Facial
Recognition}\label{a-major-police-body-cam-company-just-banned-facial-recognition}}

Its ethics board says the technology is not reliable enough to justify
using.

\href{https://www.nytimes3xbfgragh.onion/by/charlie-warzel}{\includegraphics{https://static01.graylady3jvrrxbe.onion/images/2019/03/15/opinion/charlie-warzel/charlie-warzel-thumbLarge-v3.png}}

By \href{https://www.nytimes3xbfgragh.onion/by/charlie-warzel}{Charlie
Warzel}

Mr. Warzel is an Opinion writer at large.

\begin{itemize}
\item
  June 27, 2019
\item
  \begin{itemize}
  \item
  \item
  \item
  \item
  \item
  \item
  \end{itemize}
\end{itemize}

\includegraphics{https://static01.graylady3jvrrxbe.onion/images/2019/06/27/opinion/sunday/27Warzel/58268f35d5d84f90a5fc3be1d69c0130-articleLarge.jpg?quality=75\&auto=webp\&disable=upscale}

Axon, the company that supplies 47 out of the 69 largest police agencies
in the United States with body cameras and software, announced Thursday
that it would ban the use of facial recognition systems on its devices.

``Face recognition technology is not currently reliable enough to
ethically justify its use,'' the company's independent ethics board
concluded.

Even as facial recognition systems are rolled out by privacy companies
--- from airlines to smartphone makers --- institutions nationwide are
balking at government use of algorithmically powered surveillance tools.

In May, San Francisco's Board of Supervisors
\href{https://www.nytimes3xbfgragh.onion/2019/05/14/us/facial-recognition-ban-san-francisco.html?module=inline}{voted
to ban} use of facial recognition technology by the city's police and
other agencies. Other cities, including
\href{https://www.govtech.com/public-safety/Oakland-Berkeley-Might-Follow-SFs-Facial-Recognition-Ban.html}{Berkeley}
and
\href{https://twitter.com/matt_cagle/status/1143725978293698565?s=12}{Oakland},
Calif., and Somerville, Mass., are also mulling or close on bans.
Earlier this month,
\href{https://www.latimes.com/business/technology/la-fi-tn-face-recognition-ban-california-police-body-camera-20190607-story.html}{California
lawmakers announced} they're considering a statewide ban on facial
recognition in police body cams.

In a 28-page report, Axon's ethics board, which was handpicked by
members of the Policing Project at New York University School of Law,
argued that the technology ``does not perform as well on people of color
compared to whites, on women compared to men, or young people compared
to older people.''

The report also cautioned that facial recognition is especially prone to
inaccuracy when used with police body cameras, which frequently operate
in low-light conditions and produce shaky footage.

``The tech is just not accurate enough,'' Barry Friedman, founding
director of N.Y.U.'s Policing Project and a member of the ethics board,
told me. ``Until that's fixed we don't need to say another word. And
that could be years.''

Axon's move is a rare departure from the ``move fast and break things''
style of innovation traditionally associated with new technologies. And
it may very well indicate that, when it comes to facial tracking and
privacy, policing may be where we draw the line.

It's a crucial moment for facial recognition. Though most police
departments have yet to deploy it, some uses by law enforcement have
been troubling.

This year researchers found that Detroit had
\href{https://www.nytimes3xbfgragh.onion/2019/05/16/opinion/columnists/facial-recognition-ban-privacy.html?rref=collection\%2Fbyline\%2Ffarhad-manjoo\&action=click\&contentCollection=undefined\&region=stream\&module=stream_unit\&version=latest\&contentPlacement=5\&pgtype=collection}{signed
a \$1 million deal} with a vendor to continuously screen hundreds of
public cameras throughout the city without citizen approval. In May,
Clare Garvie, a facial recognition researcher at Georgetown Law,
\href{https://www.nytimes3xbfgragh.onion/2019/05/16/opinion/columnists/facial-recognition-ban-privacy.html?rref=collection\%2Fbyline\%2Ffarhad-manjoo\&action=click\&contentCollection=undefined\&region=stream\&module=stream_unit\&version=latest\&contentPlacement=5\&pgtype=collection}{revealed
sketchy tactics} used by the New York Police Department to match
security camera footage with potential suspects who looked like
celebrities.

\emph{{[}If you're online --- and, well, you are --- chances are someone
is using your information. We'll tell you what you can do about it.}
\href{https://www.nytimes3xbfgragh.onion/newsletters/privacy-project?action=click\&module=Intentional\&pgtype=Article}{\emph{Sign
up for our limited-run newsletter}}\emph{.{]}}

The technical limitations and biases of facial recognition technology
are not well understood even by the companies that market the systems,
which makes oversight of its use in the real world particularly
problematic. Critics, meanwhile, worry that widespread deployment of the
technology risks laying the foundation of a comprehensive surveillance
state (just
\href{https://www.nytimes3xbfgragh.onion/interactive/2019/04/04/world/asia/xinjiang-china-surveillance-prison.html?module=inline}{look
at China}).

``There's a race to the bottom right now with this technology, and the
challenge is to stop that elevator before it goes through the ground
floor,'' Mr. Friedman said. It's something Axon's ethics board report
fought to change. According to the ethics board report, in early
conversations about facial recognition, Axon initially argued that it
``could not dictate to customers how products were used, nor its
customers' policies, and that it could not feasibly patrol misuse of its
product.'' That's Big Tech's version of ``guns don't kill people, people
kill people.'' And it's a view that's very widely held across the
industry.

Mr. Friedman hopes that Axon's pledge will force other vendors to think
about where the new technology might be headed and how it could impact
the most vulnerable. ``We want them to remember that just because you
can build it, doesn't mean you should.''

The ultimate goal of the ethics board goes a step further: forcing the
company to see that the customer for Axon products is not law
enforcement but ``the community that those law enforcement and public
safety organizations serve.''

Axon's ban isn't necessarily foolproof. An Axon representative confirmed
that law enforcement officials could potentially download Axon body cam
footage and then transfer it to a third-party service, like Amazon's
Rekognition. However, Eric Piza, an associate professor at the John Jay
College of Criminal Justice, said that the process is time-consuming and
requires spending the money for yet another tech service. ``If six
officers respond to a scene, that's hours of manpower and extra expense,
which might reduce the likelihood they use the technology,'' he said.

Still, Mr. Piza sees Axon's moratorium as an important step.
``Everyone's concerned about big data policing and that they put privacy
above short-term financials is not something that we see enough.''

Axon's decision won't completely stop law enforcement from using the
technology --- police departments could still use it on surveillance
videos, for instance. And true progress will have to come from
regulation at the city, state or federal level.

But the move demonstrates the potential for independent ethics boards to
help guide technology companies whose products could drastically alter
public life. If the stewards of our biggest technology companies don't
operate with an internal conscience, the least they could do is
outsource one.

\emph{Like other media companies, The Times collects data on its
visitors when they read stories like this one. For more detail please
see}
\href{https://help.nytimes3xbfgragh.onion/hc/en-us/articles/115014892108-Privacy-policy?module=inline}{\emph{our
privacy policy}} \emph{and}
\href{https://www.nytimes3xbfgragh.onion/2019/04/10/opinion/sulzberger-new-york-times-privacy.html?rref=collection\%2Fspotlightcollection\%2Fprivacy-project-does-privacy-matter\&action=click\&contentCollection=opinion\&region=stream\&module=stream_unit\&version=latest\&contentPlacement=8\&pgtype=collection}{\emph{our
publisher's description}} \emph{of The Times's practices and continued
steps to increase transparency and protections.}

\emph{Follow}
\href{https://twitter.com/privacyproject}{\emph{@privacyproject}}
\emph{on Twitter and The New York Times Opinion Section on}
\href{https://www.facebookcorewwwi.onion/nytopinion}{\emph{Facebook}}
\emph{and}\href{https://www.instagram.com/nytopinion/}{\emph{Instagram}}\emph{.}

\hypertarget{glossary-replacer}{%
\subsection{glossary replacer}\label{glossary-replacer}}

Advertisement

\protect\hyperlink{after-bottom}{Continue reading the main story}

\hypertarget{site-index}{%
\subsection{Site Index}\label{site-index}}

\hypertarget{site-information-navigation}{%
\subsection{Site Information
Navigation}\label{site-information-navigation}}

\begin{itemize}
\tightlist
\item
  \href{https://help.nytimes3xbfgragh.onion/hc/en-us/articles/115014792127-Copyright-notice}{©~2020~The
  New York Times Company}
\end{itemize}

\begin{itemize}
\tightlist
\item
  \href{https://www.nytco.com/}{NYTCo}
\item
  \href{https://help.nytimes3xbfgragh.onion/hc/en-us/articles/115015385887-Contact-Us}{Contact
  Us}
\item
  \href{https://www.nytco.com/careers/}{Work with us}
\item
  \href{https://nytmediakit.com/}{Advertise}
\item
  \href{http://www.tbrandstudio.com/}{T Brand Studio}
\item
  \href{https://www.nytimes3xbfgragh.onion/privacy/cookie-policy\#how-do-i-manage-trackers}{Your
  Ad Choices}
\item
  \href{https://www.nytimes3xbfgragh.onion/privacy}{Privacy}
\item
  \href{https://help.nytimes3xbfgragh.onion/hc/en-us/articles/115014893428-Terms-of-service}{Terms
  of Service}
\item
  \href{https://help.nytimes3xbfgragh.onion/hc/en-us/articles/115014893968-Terms-of-sale}{Terms
  of Sale}
\item
  \href{https://spiderbites.nytimes3xbfgragh.onion}{Site Map}
\item
  \href{https://help.nytimes3xbfgragh.onion/hc/en-us}{Help}
\item
  \href{https://www.nytimes3xbfgragh.onion/subscription?campaignId=37WXW}{Subscriptions}
\end{itemize}
