Sections

SEARCH

\protect\hyperlink{site-content}{Skip to
content}\protect\hyperlink{site-index}{Skip to site index}

\href{https://www.nytimes3xbfgragh.onion/section/world/asia}{Asia
Pacific}

\href{https://myaccount.nytimes3xbfgragh.onion/auth/login?response_type=cookie\&client_id=vi}{}

\href{https://www.nytimes3xbfgragh.onion/section/todayspaper}{Today's
Paper}

\href{/section/world/asia}{Asia Pacific}\textbar{}U.S. Versus China: A
New Era of Great Power Competition, but Without Boundaries

\url{https://nyti.ms/2YfeNK5}

\begin{itemize}
\item
\item
\item
\item
\item
\item
\end{itemize}

Advertisement

\protect\hyperlink{after-top}{Continue reading the main story}

Supported by

\protect\hyperlink{after-sponsor}{Continue reading the main story}

News Analysis

\hypertarget{us-versus-china-a-new-era-of-great-power-competition-but-without-boundaries}{%
\section{U.S. Versus China: A New Era of Great Power Competition, but
Without
Boundaries}\label{us-versus-china-a-new-era-of-great-power-competition-but-without-boundaries}}

\includegraphics{https://static01.graylady3jvrrxbe.onion/images/2019/06/26/world/26us-china-1/merlin_147580629_22118e15-ca37-4046-add4-7ea62b66786b-articleLarge.jpg?quality=75\&auto=webp\&disable=upscale}

By \href{https://www.nytimes3xbfgragh.onion/by/edward-wong}{Edward Wong}

\begin{itemize}
\item
  June 26, 2019
\item
  \begin{itemize}
  \item
  \item
  \item
  \item
  \item
  \item
  \end{itemize}
\end{itemize}

\href{https://cn.nytimes3xbfgragh.onion/china/20190627/united-states-china-conflict/}{阅读简体中文版}\href{https://cn.nytimes3xbfgragh.onion/china/20190627/united-states-china-conflict/zh-hant/}{閱讀繁體中文版}

WASHINGTON --- When President Trump
\href{https://www.nytimes3xbfgragh.onion/2019/06/18/us/politics/trump-china-meeting-trade.html}{meets}
President Xi Jinping of China this week to discuss contentious trade
issues, they will face each other in another nation that was once the
United States' main commercial rival, seen as a threat to American
dominance.

But the competition between the United States and Japan, which
\href{https://www.nytimes3xbfgragh.onion/2019/06/25/us/politics/trump-g-20-japan.html?smid=nytcore-ios-share}{hosts}
the
\href{https://www.nytimes3xbfgragh.onion/2019/06/07/business/economy/g20-trade-war-trump-tariffs.html}{Group
of 20 summit} this week for the first time, settled into a normal
struggle among businesses after waves of American anxiety in the 1980s.
Japan hit a decade of stagnation, and in 2010, China overtook it as the
world's second-largest economy.

There is no sign, though, that the rivalry between the United States and
China will reach the same kind of equilibrium. For one thing, Japan is a
democracy that has a military alliance with the United States, while
China is an authoritarian nation that most likely seeks to
\href{https://www.nytimes3xbfgragh.onion/2018/11/14/world/asia/usa-china-trade-pacific.html}{displace
American military dominance of the western Pacific}. In China's
competition with the United States, a rancorous trade war has persisted
for a year, and issues of national security are
\href{https://www.nytimes3xbfgragh.onion/2019/06/08/business/trump-economy-national-security.html}{bleeding
by the week} into economic ones. Some senior American officials are
pushing for ``decoupling'' the two economies.

\emph{\href{https://www.nytimes3xbfgragh.onion/2019/06/27/world/asia/what-is-the-g20.html}{{[}What
is the G-20?{]}}}

The main elements in relations --- economic and commercial ties --- have
become unmoored, and few agree on the future contours of the
relationship or the magnitude of the conflicts.

For American officials, the stakes seem much higher now than in the race
with Japan. Most economists estimate China will overtake the United
States as the largest economy in 10 to 15 years. And some senior
officials in Washington now view China as a steely ideological rival,
where the Communist Party aims not only to subjugate citizens but to
\href{https://www.nytimes3xbfgragh.onion/2019/04/24/technology/ecuador-surveillance-cameras-police-government.html}{spread
tools of authoritarian control globally} --- particularly surveillance,
communications and artificial intelligence technology --- and
\href{https://www.nytimes3xbfgragh.onion/2018/09/20/world/asia/south-china-sea-navy.html}{establish
military footholds} across
\href{https://www.nytimes3xbfgragh.onion/2017/02/25/world/africa/us-djibouti-chinese-naval-base.html}{oceans}
and
\href{https://www.washingtonpost.com/world/asia_pacific/in-central-asias-forbidding-highlands-a-quiet-newcomer-chinese-troops/2019/02/18/78d4a8d0-1e62-11e9-a759-2b8541bbbe20_story.html?utm_term=.0322993a30c0}{mountains}.

Though Mr. Trump incessantly praises Mr. Xi --- he said they
\href{https://twitter.com/realdonaldtrump/status/982954355509907457?lang=en}{``will
always be friends''} --- the idea of China as a dangerous juggernaut,
more formidable than the Soviet Union, has become increasingly
widespread in the administration. It was articulated by Secretary of
State Mike Pompeo during a visit to the Netherlands, part of a weeklong
trip across Europe this month in which he talked about China at each
stop.

\includegraphics{https://static01.graylady3jvrrxbe.onion/images/2019/06/26/world/26us-china-2/merlin_156982923_95d119c1-85a0-463e-ac4b-4d23bc462f4e-articleLarge.jpg?quality=75\&auto=webp\&disable=upscale}

``China has inroads too on this continent that demand our attention,''
he told a news conference in The Hague. ``China wants to be the dominant
economic and military power of the world, spreading its authoritarian
vision for society and its corrupt practices worldwide.''

The
\href{https://www.whitehouse.gov/wp-content/uploads/2017/12/NSS-Final-12-18-2017-0905.pdf}{National
Security Strategy} issued by the White House in December 2017 sounded
the alarm: The United States was re-entering an era of great power
competition, in which China and Russia ``want to shape a world
antithetical to U.S. values and interests.'' But since then, Mr. Trump
and cabinet officials,
\href{https://www.nytimes3xbfgragh.onion/2019/06/20/world/middleeast/iran-us-drone.html}{distracted
by Iran} and other foreign policy matters, have failed to outline a
coherent strategy.

That has left administration officials struggling to piece together an
approach to China that has elements of competition, containment and
constructive engagement, none of them sharply focused.

Mr. Trump's closest advisers on China are split on strategies. His top
foreign policy officials,
\href{https://www.nytimes3xbfgragh.onion/2019/05/28/us/politics/trump-john-bolton-north-korea-iran.html}{John
R. Bolton} and
\href{https://www.nytimes3xbfgragh.onion/2019/06/22/world/middleeast/trump-pompeo-iran.html}{Mr}.\href{https://www.nytimes3xbfgragh.onion/2019/06/22/world/middleeast/trump-pompeo-iran.html}{Pompeo},
have pushed for tough policies, as has Peter Navarro, the trade adviser
and creator of a polemical book and documentary film,
\href{http://deathbychina.com/}{``Death by China.''} In the opposite
camp are
\href{https://www.nytimes3xbfgragh.onion/2018/10/24/us/politics/trump-phone-security.html}{tycoons}
--- among them Treasury Secretary Steven Mnuchin, Stephen A. Schwarzman
and Steve Wynn.

Midlevel bureaucrats are formulating their own ideas. The view of a
drawn-out ideological conflict was laid out in stark terms by Kiron
Skinner, the head of policy planning at the State Department,
\href{https://www.washingtonexaminer.com/policy/defense-national-security/state-department-preparing-for-clash-of-civilizations-with-china}{in
a talk in Washington on April 29}.

``This is a fight with a really different civilization and a different
ideology, and the United States hasn't had that before,'' she said.
``The Soviet Union and that competition, in a way, it was a fight within
the Western family.''

Image

China's first aircraft carrier, the Liaoning, at sea last year. China
likely seeks to displace American military dominance of the western
Pacific.Credit...Agence France-Presse --- Getty Images

Now, she said, ``it's the first time that we will have a great power
competitor that is not Caucasian.''

Many analysts have tried to discern whether the striking remarks point
to a new policy direction. Officials say privately that is not the case.

While there has been bipartisan praise in Washington for the
administration's tougher line --- with measures ranging from
\href{https://www.nytimes3xbfgragh.onion/2019/05/09/us/politics/china-trade-tariffs.html}{tariffs}
to
\href{https://www.nytimes3xbfgragh.onion/2019/05/15/business/huawei-ban-trump.html}{sanctions
of Chinese technology companies} --- critics say they see strategic
ambiguity without the strategy.

``The economic and security and technological and even scientific
components of the U.S.-China relationship are now being conflated,''
said \href{https://government.cornell.edu/jessica-chen-weiss}{Jessica
Chen Weiss}, a professor of government at Cornell University who studies
\href{https://www.foreignaffairs.com/articles/china/2019-06-11/world-safe-autocracy}{Chinese
politics} and nationalism. ``What's worrying to many is not being able
to decipher different levels of risk and how far and how quickly the
efforts to indiscriminately decouple the United States and China will
go.''

That idea of decoupling rests on the premise that two economies so
intertwined poses a significant security risk to the United States. The
linking accelerated when China
\href{https://www.foreignaffairs.com/articles/china/2018-04-02/was-letting-china-wto-mistake}{entered
the World Trade Organization} in 2001 and in recent years had seemed
irreversible. But Mr. Trump's hard-line trade advisers want the two
nations to unwind their supply chains, which means some American
businesses exit China, and others stop selling components to Chinese
companies.

Mr. Trump is narrowly focused on
\href{https://www.nytimes3xbfgragh.onion/2019/03/06/us/politics/us-trade-deficit.html}{cutting
the trade}deficit with China, which many economists say is not
meaningful. But his imposition of tariffs and the general uncertainty
around the economic relationship are forcing some American companies to
rethink keeping operations in China. And putting Chinese companies,
notably Huawei, the giant maker of communications technology, on what
officials call an entity list to cut off the supply of American
components is
\href{https://www.nytimes3xbfgragh.onion/2019/05/20/technology/google-android-huawei.html}{having
an effect}.

Image

``China wants to be the dominant economic and military power of the
world, spreading its authoritarian vision for society and its corrupt
practices worldwide,'' Secretary of State Mike Pompeo said during a news
conference in The Hague this month.Credit...Piroschka Van De
Wouw/Reuters

``After a long period of globalization and squeezing out economic
efficiencies, you do see national security rising to the forefront,''
said \href{https://www.cnas.org/people/daniel-kliman}{Daniel M. Kliman},
director of the Asia-Pacific Security Program at the
\href{https://www.cnas.org/}{Center for a New American Security}.

That has not gone unnoticed in China. This spring, with trade tensions
rising, Chinese state-run television began showing
\href{https://www.economist.com/china/2019/05/23/amid-trade-tensions-with-america-china-is-showing-old-war-films}{old
Korean War films} depicting American aggression. Newspapers ran
editorials on the war.

\href{http://en.rdcy.org/Index/news_cont/id/14791}{Wang Wen}, executive
dean of the Chongyang Institute for Financial Studies at the Renmin
University of China, said in an interview that the new model for United
States-China relations was ``fight but not break.''

The fallout from the struggle is widening. Chinese security officers
have
\href{https://www.nytimes3xbfgragh.onion/2019/05/16/world/asia/china-canadian-arrested.html}{arrested
two Canadian men on charges of spying} in apparent retaliation for the
\href{https://www.nytimes3xbfgragh.onion/2018/12/06/us/politics/huawei-meng-china-iran.html}{arrest
in Canada} of Meng Wanzhou, a top Huawei executive, on an extradition
request by the United States. The F.B.I. has been
\href{https://www.nytimes3xbfgragh.onion/2019/04/14/world/asia/china-academics-fbi-visa-bans.html}{canceling
visas} of Chinese scholars suspected of intelligence ties.

Some observers say they fear a new Red Scare.

``Rather than proclaiming a `whole of society' threat from a hostile
`civilization,' U.S. officials would be wise to emphasize the value that
immigrants from China and other countries have brought, while
establishing policies to safeguard against theft of intellectual
property,'' Ms. Weiss said.

The case of Huawei is at the nexus of concerns in Washington over both
Chinese economic dominance and security threats. The Trump
administration has been pushing countries to bar Huawei from developing
\href{https://www.nytimes3xbfgragh.onion/2018/12/31/technology/personaltech/5g-what-you-need-to-know.html}{next-generation
5G communications networks}, arguing it
\href{https://www.nytimes3xbfgragh.onion/2019/01/26/us/politics/huawei-china-us-5g-technology.html?module=inline}{poses
a national security risk}. Huawei, a private company, denies the charge.

Image

A screen showing images of Mr. Xi in Kashgar, in Xinjiang. China's
extraordinary human rights abuses in the region are one major reason
many American officials have given up on any notion of a future turn
toward liberalism within the Communist Party.Credit...Greg Baker/Agence
France-Presse --- Getty Images

But the
\href{https://www.nytimes3xbfgragh.onion/2019/03/17/us/politics/huawei-ban.html}{reluctance
of even close allies} to adopt a ban,
\href{https://www.nytimes3xbfgragh.onion/2018/08/23/technology/huawei-banned-australia-5g.html}{except
for Australia}, shows how nations are unwilling to jeopardize their
economic relations with China. That includes Japan, where the government
has not issued a ban and is trying to
\href{https://www.cnbc.com/2019/06/25/japan-abe-and-china-xi-look-to-strengthen-ties-at-g-20-as-trump-looms.html}{strengthen
ties} with China in other ways --- at the G20 in Osaka, Prime Minister
Shinzo Abe plans to host a dinner for Mr. Xi.

The Trump administration has also been
\href{https://www.nytimes3xbfgragh.onion/2019/01/13/world/africa/china-loans-africa-usa.html}{pushing
countries to reject} China's Belt-and-Road infrastructure projects and
what American officials call
\href{https://www.nytimes3xbfgragh.onion/2018/06/25/world/asia/china-sri-lanka-port.html}{``debt
diplomacy,''} with
\href{https://www.nytimes3xbfgragh.onion/2019/03/23/world/europe/italy-china-xi-silk-road.html}{mixed
results}.

Some American companies are trying to bypass the limits set by the Trump
administration on their dealings with China. Semiconductor companies,
for example, have found a legal basis for
\href{https://www.nytimes3xbfgragh.onion/2019/06/25/technology/huawei-trump-ban-technology.html}{sidestepping}
the Commerce Department prohibition on selling components to Huawei.

But the administration itself sometimes pulls punches on China in the
name of economic relations --- a sign that the traditional foundation of
the relationship still stands to a degree.

Since last year, the administration has debated imposing sanctions on
Chinese officials for their role in interning one million or more
Muslims in the Xinjiang region. Though Mr. Pompeo and other officials
have pushed for the sanctions, the Treasury Department, led by Mr.
Mnuchin, has
\href{https://www.nytimes3xbfgragh.onion/2019/05/04/world/asia/trump-china-uighurs-trade-deal.html}{opposed
them} for fear of derailing the trade talks. So the administration has
taken no action.

China's extraordinary
\href{https://www.nytimes3xbfgragh.onion/2018/09/08/world/asia/china-uighur-muslim-detention-camp.html}{human
rights abuses in Xinjiang} are one major reason many American officials
have abandoned any notion of a future turn toward liberalism within the
Communist Party.

For their part, Chinese officials have seized on the Trump
administration's actions to argue that the United States is trying to
stop China's rise. On Tuesday, People's Daily, the official Communist
Party newspaper,
\href{Http://opinion.people.com.cn/n1/2019/0625/c1003-31177840.html}{ran
a commentary} urging citizens to fight for the nation's dignity.

``The Chinese people deeply understand that the American government's
suppression and containment of China is an external challenge that China
must bear in its development and growth,'' the paper said, ``and it is a
hurdle that we must overcome in the great rejuvenation of the Chinese
nation.''

Advertisement

\protect\hyperlink{after-bottom}{Continue reading the main story}

\hypertarget{site-index}{%
\subsection{Site Index}\label{site-index}}

\hypertarget{site-information-navigation}{%
\subsection{Site Information
Navigation}\label{site-information-navigation}}

\begin{itemize}
\tightlist
\item
  \href{https://help.nytimes3xbfgragh.onion/hc/en-us/articles/115014792127-Copyright-notice}{©~2020~The
  New York Times Company}
\end{itemize}

\begin{itemize}
\tightlist
\item
  \href{https://www.nytco.com/}{NYTCo}
\item
  \href{https://help.nytimes3xbfgragh.onion/hc/en-us/articles/115015385887-Contact-Us}{Contact
  Us}
\item
  \href{https://www.nytco.com/careers/}{Work with us}
\item
  \href{https://nytmediakit.com/}{Advertise}
\item
  \href{http://www.tbrandstudio.com/}{T Brand Studio}
\item
  \href{https://www.nytimes3xbfgragh.onion/privacy/cookie-policy\#how-do-i-manage-trackers}{Your
  Ad Choices}
\item
  \href{https://www.nytimes3xbfgragh.onion/privacy}{Privacy}
\item
  \href{https://help.nytimes3xbfgragh.onion/hc/en-us/articles/115014893428-Terms-of-service}{Terms
  of Service}
\item
  \href{https://help.nytimes3xbfgragh.onion/hc/en-us/articles/115014893968-Terms-of-sale}{Terms
  of Sale}
\item
  \href{https://spiderbites.nytimes3xbfgragh.onion}{Site Map}
\item
  \href{https://help.nytimes3xbfgragh.onion/hc/en-us}{Help}
\item
  \href{https://www.nytimes3xbfgragh.onion/subscription?campaignId=37WXW}{Subscriptions}
\end{itemize}
