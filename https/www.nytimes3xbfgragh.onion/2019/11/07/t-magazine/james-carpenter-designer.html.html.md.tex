The Designer Bringing Sunlight Back to Manhattan

\url{https://nyti.ms/2NJk4FO}

\begin{itemize}
\item
\item
\item
\item
\item
\end{itemize}

\includegraphics{https://static01.graylady3jvrrxbe.onion/images/2019/11/17/t-magazine/17tmag-nordstrom-slide-VY8J/17tmag-nordstrom-slide-VY8J-articleLarge.jpg?quality=75\&auto=webp\&disable=upscale}

Sections

\protect\hyperlink{site-content}{Skip to
content}\protect\hyperlink{site-index}{Skip to site index}

In Studio

\hypertarget{the-designer-bringing-sunlight-back-to-manhattan}{%
\section{The Designer Bringing Sunlight Back to
Manhattan}\label{the-designer-bringing-sunlight-back-to-manhattan}}

James Carpenter created the facade for the new Nordstrom store,
fashioned from 17-foot-tall sections of glass.

Carpenter pictured next to a maquette of Nordstrom's Manhattan
store.Credit...Linda Xiao

Supported by

\protect\hyperlink{after-sponsor}{Continue reading the main story}

By \href{https://www.nytimes3xbfgragh.onion/by/nancy-hass}{Nancy Hass}

\begin{itemize}
\item
  Published Nov. 7, 2019Updated Nov. 8, 2019
\item
  \begin{itemize}
  \item
  \item
  \item
  \item
  \item
  \end{itemize}
\end{itemize}

THE STEEL AND glass towers that have come to define TriBeCa might seem
like an ideal landscape for the designer
\href{https://www.nytimes3xbfgragh.onion/2010/08/01/arts/design/01carpenter.html}{James
Carpenter}'s studio. After all, he is responsible for some of our era's
most riveting expanses of architectural glass, from the cable-net wall
in the atrium of the Time Warner Center to the undulating facade of
\href{https://www.nytimes3xbfgragh.onion/2019/10/23/style/nordstrom-family-department-stores.html}{Nordstrom}'s
colossal new store on West 57th Street.

But for all his technological ingenuity, in some ways Carpenter remains
doggedly old-fashioned: For more than 20 years, he has kept his work
space in a former printing plant on Hudson Street that stands as a
staunch reminder of the neighborhood's industrial origins. Though
several upper floors of the 16-story building, constructed in 1929, have
been refurbished into sumptuously minimal condominiums (Carpenter
designed the famed glass cube penthouse addition on the top), his own
5,000-square-foot fourth-floor space retains a craggy analog feel.
Carpenter, 70, is the longest remaining commercial tenant in the
building, and his studio's rows of white workstations are punctuated by
exposed concrete columns and vast awning windows. On drafting tables sit
maquettes of projects in development, like crystal Jenga towers. Leaned
up against the walls are tall slices of dichroic glass that change color
depending on the angle from which the light hits them, shading from
green-indigo to gold and magenta. The overhead lights aren't turned on
until the sun sets. ``It's a remarkable space because you can get both
history and the brightness. Usually you have to make a choice between
the two,'' he says.

\emph{{[}}\href{https://www.nytimes3xbfgragh.onion/newsletters/t-list?module=inline}{\emph{Sign
up here}} \emph{for the T List newsletter, a weekly roundup of what T
Magazine editors are noticing and coveting now.{]}}

Light, of course, is everything to him, a quixotic obsession in a city
that is losing its connection to the sky because of giant edifices that
block the sun; the super rich move ever upward, to unobscured views and
sunshine, while on the ground, life can be shadowy. But Carpenter's
specialty is maximizing light, amplifying its effect by bouncing it off
innovative materials. His often monumental installations have the
quality of grand-scale statuary. At the 52-story
\href{https://www.nytimes3xbfgragh.onion/2004/05/05/nyregion/rising-above-ground-zero-tower-slowly-takes-shape.html}{7
World Trade Center}, he installed blue stainless-steel reflectors that
cast an icy glow he calls ``volumetric'' through clear panes suspended
inches from the building's surface; for a Washington, D.C., law firm in
an office building with little sunlight, he mounted a heliostat on the
roof and created a 120-foot glass cone, a sort of snorkel, to bring
shafts of light into the space, splashing the walls with changing
intensity as the day progresses.

Image

The view from Nordstrom's new store in New York City, designed by James
Carpenter.Credit...Carter Love

Image

A model of the new building, shot inside the Nordstrom store in New York
City.Credit...Linda Xiao

IT'S NOT SURPRISING that Carpenter brings the high-concept tactility of
a sculptor to what many others see as mere structural cladding. The
designer got his degree in illustration at the Rhode Island School of
Design and, after he began blowing glass on the side, collaborated with
\href{https://www.nytimes3xbfgragh.onion/2017/04/26/arts/design/are-there-glass-snakes-in-dale-chihulys-fragile-eden.html}{Dale
Chihuly}, the Seattle-based glass artist, on his early conceptual work.
In 1971, the two men created a seminal exhibition at New York City's
Museum of Contemporary Crafts (now the
\href{https://www.nytimes3xbfgragh.onion/topic/organization/museum-of-arts-and-design}{Museum
of Arts and Design}), a 500-square-foot blown-glass environment of white
opaque tubes with argon and neon gas against a background of black
vinyl.

While Chihuly went on to create crowd-pleasing blown-glass fantasias,
Carpenter became a consultant for the Corning glass company, where he
grew passionate about bringing light into the public sphere --- while
also considering local history. As such, the facade he created for the
320,000-square-foot Nordstrom, which he calls ``waveform,'' was inspired
by the artists' bay-windowed studios that lined 57th Street at the
beginning of the 20th century. Carpenter's homage to that era is
fashioned from 17-foot-tall sections of glass engineered in Italy, made
in Germany and molded into shape in Spain (only a handful of ovens in
the world are big enough to hold the parabolic double S-curves that span
the height of each of the seven floors). Each wave, lined on the inside
with a moving metal mesh curtain, forms an occupiable space within the
store. Shoppers can navigate around merchandise to peer out down 57th
Street, almost to the East River. ``It's important for the store to
maintain a relationship to the outside city, which is completely
different from the way department stores are built, as a sealed-off
refuge,'' he says.

But Carpenter still feels a particular connection to the way his
creations are viewed from street level. Conscious that big buildings rob
the public of an expanse of sky, he makes sure they return something
valuable. The Nordstrom facade asserts itself in the cityscape in daring
fashion; arguably as far as possible in spirit from the orangy-brick
industrial bulwark where the designer works, it makes no less powerful a
statement about cosmopolitan life. From the outside, the sculptured
glass seems to change radically as you approach because of the folds,
which reflect the sky and the skyscrapers against it. Walking down 57th
Street toward the wavy confection is like descending a grand staircase:
With each step, the whole comes into focus, the image shifting from
abstraction to realistic portrait. Standing at the base in daylight, you
look up to see a Cubist mirror rendering a city in flux; in the evening,
lights glow soft inside, turning the structure transparent, the ballet
of shoppers silhouetted in motion. ``Yes, it is a store, of course, you
are always cognizant of that,'' he says, ``but that is only part of the
story. Ultimately, it's the story of the city unfolding.''

Advertisement

\protect\hyperlink{after-bottom}{Continue reading the main story}

\hypertarget{site-index}{%
\subsection{Site Index}\label{site-index}}

\hypertarget{site-information-navigation}{%
\subsection{Site Information
Navigation}\label{site-information-navigation}}

\begin{itemize}
\tightlist
\item
  \href{https://help.nytimes3xbfgragh.onion/hc/en-us/articles/115014792127-Copyright-notice}{©~2020~The
  New York Times Company}
\end{itemize}

\begin{itemize}
\tightlist
\item
  \href{https://www.nytco.com/}{NYTCo}
\item
  \href{https://help.nytimes3xbfgragh.onion/hc/en-us/articles/115015385887-Contact-Us}{Contact
  Us}
\item
  \href{https://www.nytco.com/careers/}{Work with us}
\item
  \href{https://nytmediakit.com/}{Advertise}
\item
  \href{http://www.tbrandstudio.com/}{T Brand Studio}
\item
  \href{https://www.nytimes3xbfgragh.onion/privacy/cookie-policy\#how-do-i-manage-trackers}{Your
  Ad Choices}
\item
  \href{https://www.nytimes3xbfgragh.onion/privacy}{Privacy}
\item
  \href{https://help.nytimes3xbfgragh.onion/hc/en-us/articles/115014893428-Terms-of-service}{Terms
  of Service}
\item
  \href{https://help.nytimes3xbfgragh.onion/hc/en-us/articles/115014893968-Terms-of-sale}{Terms
  of Sale}
\item
  \href{https://spiderbites.nytimes3xbfgragh.onion}{Site Map}
\item
  \href{https://help.nytimes3xbfgragh.onion/hc/en-us}{Help}
\item
  \href{https://www.nytimes3xbfgragh.onion/subscription?campaignId=37WXW}{Subscriptions}
\end{itemize}
