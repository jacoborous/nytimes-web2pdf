Sections

SEARCH

\protect\hyperlink{site-content}{Skip to
content}\protect\hyperlink{site-index}{Skip to site index}

\href{https://www.nytimes3xbfgragh.onion/section/politics}{Politics}

\href{https://myaccount.nytimes3xbfgragh.onion/auth/login?response_type=cookie\&client_id=vi}{}

\href{https://www.nytimes3xbfgragh.onion/section/todayspaper}{Today's
Paper}

\href{/section/politics}{Politics}\textbar{}Fiona Hill Viewed Serving
Trump as Risky. Now She's an Impeachment Witness.

\url{https://nyti.ms/2KGe8wH}

\begin{itemize}
\item
\item
\item
\item
\item
\end{itemize}

Advertisement

\protect\hyperlink{after-top}{Continue reading the main story}

Supported by

\protect\hyperlink{after-sponsor}{Continue reading the main story}

\hypertarget{fiona-hill-viewed-serving-trump-as-risky-now-shes-an-impeachment-witness}{%
\section{Fiona Hill Viewed Serving Trump as Risky. Now She's an
Impeachment
Witness.}\label{fiona-hill-viewed-serving-trump-as-risky-now-shes-an-impeachment-witness}}

Dr. Hill's decision to be the president's top adviser on Russia and
Europe strained friendships, made her a target of conspiracy theories
--- and landed her in the center of the tumult over Ukraine.

\includegraphics{https://static01.graylady3jvrrxbe.onion/images/2019/11/20/nyregion/20DC-HILL/merlin_163857780_70bc56af-696d-4177-a076-11b1b454186a-articleLarge.jpg?quality=75\&auto=webp\&disable=upscale}

\href{https://www.nytimes3xbfgragh.onion/by/sheryl-gay-stolberg}{\includegraphics{https://static01.graylady3jvrrxbe.onion/images/2018/11/26/multimedia/author-sheryl-gay-stolberg/author-sheryl-gay-stolberg-thumbLarge.png}}

By
\href{https://www.nytimes3xbfgragh.onion/by/sheryl-gay-stolberg}{Sheryl
Gay Stolberg}

\begin{itemize}
\item
  Nov. 21, 2019
\item
  \begin{itemize}
  \item
  \item
  \item
  \item
  \item
  \end{itemize}
\end{itemize}

\href{https://www.nytimes3xbfgragh.onion/2019/11/21/us/politics/impeachment-hearing.html}{\emph{Follow
our live coverage of David Holmes and Fiona Hill testifying in the
impeachment hearings}}.

WASHINGTON --- Fiona Hill knew she was taking a risk in going to work
for President Trump.

A British-born coal-miner's daughter with a Ph.D. from Harvard, Dr. Hill
is a respected Russia expert, former intelligence analyst and co-author
of a 500-page book analyzing the psyche of its president, Vladimir V.
Putin. So the prospect of working for a president who speaks admiringly
of Mr. Putin and has expressed doubts that Russia interfered in the 2016
election gave her pause.

Her decision to join the National Security Council in April 2017 --- and
to stay for more than two years after Mr. Trump cozied up to Mr. Putin
and publicly disparaged the nation's intelligence agencies --- strained
friendships and made her a target of right-wing conspiracy theorists who
spread rumors that she was a Democratic mole.

Now, it has landed her near the center of the impeachment inquiry into
whether Mr. Trump abused his power to enlist a foreign leader to help
him in the 2020 presidential election. Her planned
\href{https://www.nytimes3xbfgragh.onion/2019/11/21/us/politics/impeachment-hearing.html}{appearance
before the House Intelligence Committee} on Thursday represents the
fulfillment of Dr. Hill's worst fears about what could happen if she
swallowed her reservations and went to work for Mr. Trump.

``The risk was what we see playing out in front of us --- that something
wrong would happen, that she would do the right thing and other people
wouldn't, and there would be a reckoning,'' said Tom Wright, a former
colleague and friend of Dr. Hill's. ``And afterward there could be
hearings --- with, at worst case, the fate of the presidency riding on
it.''

On Thursday, Dr. Hill will take her turn as the latest in a series of
witnesses to testify publicly before Congress. Many have been
nonpartisan diplomats and national security experts who went to work for
the president thinking they might be the proverbial ``adults in the
room'' checking Mr. Trump's impulses, only to find themselves caught up
in a mess of his making, and in danger of being attacked.

\emph{{[}Here's}
\href{https://www.nytimes3xbfgragh.onion/2019/11/21/us/politics/david-holmes-impeachment.html}{\emph{background
on David Holmes}}\emph{, another witness testifying on Thursday.{]}}

In closed-door testimony, Dr. Hill called her gripping account ``my
worst nightmare.'' In it, she revealed how she and her boss at the time,
John R. Bolton, the former White House national security adviser, were
alarmed at a rogue effort by allies of Mr. Trump, led by his personal
lawyer, Rudolph W. Giuliani, to deliver on the president's desire for
Ukraine to announce investigations into his political rivals.

In
\href{https://www.nytimes3xbfgragh.onion/2019/11/20/us/politics/sondland-defiant-says-he-followed-trumps-orders-to-pressure-ukraine.html}{testimony
on Wednesday}, one of those allies --- Gordon D. Sondland, a Trump
megadonor turned ambassador to the European Union --- turned on the
president and top administration officials. He told lawmakers that he
was only doing Mr. Trump's bidding in pressing Ukraine for the
investigations, and that Vice President Mike Pence, Secretary of State
Mike Pompeo and Mick Mulvaney, the acting chief of staff, were among
those well aware of it.

\includegraphics{https://static01.graylady3jvrrxbe.onion/images/2019/11/20/us/politics/20dc-hill4/merlin_164741931_002ef550-076f-47ee-95a5-51b38499f2b7-articleLarge.jpg?quality=75\&auto=webp\&disable=upscale}

In Mr. Sondland's telling during a private interview with impeachment
investigators last month, Dr. Hill was furious to the point of shaking
when he stopped by her office to say goodbye to her before she left the
White House, about a week before the now-infamous July 25 telephone call
in which Mr. Trump pressed President Volodymyr Zelensky of Ukraine to
investigate former Vice President Joseph R. Biden Jr. and his son
Hunter. (Dr. Hill had argued against the call, saying she did not
understand its purpose.)

``She was pretty upset about her role in the administration, about her
superiors, about the president,'' Mr. Sondland recalled in a
\href{https://docs.house.gov/meetings/IG/IG00/CPRT-116-IG00-D006.pdf}{closed-door
deposition}. ``She was sort of shaking. She was pretty mad.''

A lawyer for Dr. Hill, Lee Wolosky, has disputed that characterization,
\href{https://twitter.com/LeeWolosky/status/1192066926563999744}{writing
on Twitter} that Mr. Sondland ``fabricated communications with Dr.
Hill.''

Dr. Hill is neither pro-Trump nor a ``Never Trumper,'' and she was
always circumspect in talking about Mr. Trump, friends said. She refused
speaking invitations of the sort that would be routine for top advisers
in past administrations --- even at the Brookings Institution, where she
was on leave as director of the Center on the United States and Europe.

\includegraphics{https://static01.graylady3jvrrxbe.onion/images/2019/11/16/video/xx-impeachment-explainer-vid/xx-impeachment-explainer-vid-videoSixteenByNineJumbo1600.jpg}

But her own closed-door testimony reveals how fraught her time in the
administration was.

In it, she described a tense White House meeting with Mr. Sondland, Mr.
Bolton, Energy Secretary Rick Perry and Ukrainian officials in which it
became apparent that Mr. Mulvaney was working with Mr. Sondland and Mr.
Giuliani to execute the president's plan.

Dr. Hill described her horror that the Ukrainians --- foreign nationals
--- were hanging around the West Wing, outside the Situation Room, one
of the most secure and sensitive spots in the White House. When Mr.
Sondland moved the meeting down to a room in the White House basement,
Mr. Bolton instructed her to follow them to find out what was going on.

She did so, and confronted Mr. Sondland, cutting him off when he dangled
the prospect of a White House meeting between Mr. Trump and Mr.
Zelensky.

``It has to go through proper procedure,'' Dr. Hill insisted. Then she
reported back to Mr. Bolton, who told her to report it to the National
Security Council's top lawyer, John A. Eisenberg.

Image

John R. Bolton, the former national security adviser, was Dr. Hill's
boss at the White House.Credit...Doug Mills/The New York Times

``You go and tell Eisenberg that I am not part of whatever drug deal
Sondland and Mulvaney are cooking up on this,'' she recalled Mr. Bolton
saying.

Friends said that sounded like the Dr. Hill they know: straight, to the
point, unafraid to push back.

``Fiona has served impeccably in the executive branch,'' said Strobe
Talbott, the former president of the Brookings Institution, ``and, now,
she's helping Congress understand the disaster Trump has visited on the
country and the world.''

Republicans view her as suspect because she worked with Christopher
Steele, who later wrote an infamous dossier on Mr. Trump's ties to
Russia, when she was an intelligence officer and he was her British
counterpart. And her time as an unpaid adviser to the Central Eurasia
Project of the Open Society Foundation, founded by the Democratic
philanthropist George Soros, fueled rumors spread by the conspiracy
theorist Alex Jones.

``My entire first year of my tenure at the National Security Council was
filled with hateful calls, conspiracy theories, which has started
again,'' she told House investigators, saying her attackers accused her
``of being a Soros mole in the White House, of colluding with all kinds
of enemies of the president.''

Dr. Hill, 54, had an unusual path to academia. The daughter of a coal
miner and a midwife, she had a hardscrabble childhood in northeast
England --- a childhood that bred toughness, her friends say. Once, when
she was 11, a boy in her class set one of her pigtails on fire while she
was taking a test. She put the fire out with her hands, and finished the
test.

She learned to speak Russian and eventually made her way across the
Atlantic to Harvard for a fellowship, where she studied under the
scholar Richard Pipes, known for his hard-line views about what was then
the Soviet Union.

Dr. Hill's own views are more nuanced, friends and colleagues say; she
is not so much a Russia hawk as a cleareyed realist. She was also very
clear about the threat Russia posed to Ukraine.

``She comes from this realist tradition where you start with the
proposition that this other actor is capable of killing me,'' said
Graham Allison, a Harvard political scientist who worked with Dr. Hill
on an initiative to teach foreign governments about democracy. ``I can't
figure out how to kill them without committing suicide, so now I have to
find a way to live with them.''

In 2006, Dr. Hill joined the National Intelligence Council as national
intelligence officer for Russia and Eurasia, a job that required her to
assess the Russian threat. In 2009, she rejoined Brookings, where she
had previously been a fellow. In 2013, she and Clifford Gaddy published
``Mr. Putin: Operative in the Kremlin.''

``She confirmed what I thought, which is what I've said very publicly
for a long time: He's the most dangerous guy on Earth,'' said Adm. Mike
Mullen, the former chairman of the Joint Chiefs of Staff, who got to
know Dr. Hill when she was an intelligence analyst.

Yet for all of her scholarly work, it was an appearance on television
that landed Dr. Hill her White House job. After Mr. Trump was elected,
K.T. McFarland, a Fox News commentator, recommended her to Gen. Michael
T. Flynn, Mr. Trump's first national security adviser.

General Flynn, whose tenure ended in scandal after 24 days, offered her
the job as the National Security Council's senior director for Europe
and Russia, though she came on after he left. Some friends warned her
against it. Among them was Marvin Kalb, a senior fellow at Brookings,
who thought Dr. Hill might have trouble in part because she was an
immigrant.

``I was concerned that she might run into problems that others might not
run into, and I thought that her judgment of Putin might not sit well
with the president,'' he said, adding: ``My recommendation to her was to
stay away. But she believed very strongly in the opportunity to serve.''

She got off to an uncertain start; Mr. Trump once mistook her for a
low-level member of support staff. And if there was any doubt that the
president had little interest in national security protocol and would
rely on no one but himself, it was erased when he took notes away from
his interpreter during a private meeting with Mr. Putin in Hamburg,
Germany, in 2017.

Then came the disastrous Helsinki, Finland, summit in 2018, where Mr.
Trump accepted the Russian president's denial that his country had
interfered in the 2016 race. In a stunning break with protocol, he also
told Mr. Putin that he might let Russia interrogate a former American
ambassador, Michael A. McFaul, a staunch critic of Russia's record on
human rights.

Mr. McFaul visited her at the White House to complain.

``I thought they were going to clean it up when they got back to
Washington, and they didn't,'' Mr. McFaul said. ``They just doubled
down.''

Some colleagues of Dr. Hill's wondered why she did not quit then.
Others, like Angela Stent, a Russia expert at Georgetown University and
mentor to Dr. Hill, said she contemplated leaving at times, but stayed
because she wanted ``to minimize the damage of some things that were
happening with Russia.''

When she left the White House in July, it was as planned; she wanted to
spend more time with her husband and 12-year-old daughter and her
mother, who is ill. If she had been frustrated there, Mr. Wright said,
she kept it to herself.

``This exit was not what she had planned,'' Mr. Wright said. ``I don't
think she was thinking, `I'm going to go out in a blaze of glory, take a
moral stand and testify.' That was definitely not her intention. She
just wanted to do her job with no fuss or drama.''

Advertisement

\protect\hyperlink{after-bottom}{Continue reading the main story}

\hypertarget{site-index}{%
\subsection{Site Index}\label{site-index}}

\hypertarget{site-information-navigation}{%
\subsection{Site Information
Navigation}\label{site-information-navigation}}

\begin{itemize}
\tightlist
\item
  \href{https://help.nytimes3xbfgragh.onion/hc/en-us/articles/115014792127-Copyright-notice}{©~2020~The
  New York Times Company}
\end{itemize}

\begin{itemize}
\tightlist
\item
  \href{https://www.nytco.com/}{NYTCo}
\item
  \href{https://help.nytimes3xbfgragh.onion/hc/en-us/articles/115015385887-Contact-Us}{Contact
  Us}
\item
  \href{https://www.nytco.com/careers/}{Work with us}
\item
  \href{https://nytmediakit.com/}{Advertise}
\item
  \href{http://www.tbrandstudio.com/}{T Brand Studio}
\item
  \href{https://www.nytimes3xbfgragh.onion/privacy/cookie-policy\#how-do-i-manage-trackers}{Your
  Ad Choices}
\item
  \href{https://www.nytimes3xbfgragh.onion/privacy}{Privacy}
\item
  \href{https://help.nytimes3xbfgragh.onion/hc/en-us/articles/115014893428-Terms-of-service}{Terms
  of Service}
\item
  \href{https://help.nytimes3xbfgragh.onion/hc/en-us/articles/115014893968-Terms-of-sale}{Terms
  of Sale}
\item
  \href{https://spiderbites.nytimes3xbfgragh.onion}{Site Map}
\item
  \href{https://help.nytimes3xbfgragh.onion/hc/en-us}{Help}
\item
  \href{https://www.nytimes3xbfgragh.onion/subscription?campaignId=37WXW}{Subscriptions}
\end{itemize}
