Sections

SEARCH

\protect\hyperlink{site-content}{Skip to
content}\protect\hyperlink{site-index}{Skip to site index}

\href{https://www.nytimes3xbfgragh.onion/section/food}{Food}

\href{https://myaccount.nytimes3xbfgragh.onion/auth/login?response_type=cookie\&client_id=vi}{}

\href{https://www.nytimes3xbfgragh.onion/section/todayspaper}{Today's
Paper}

\href{/section/food}{Food}\textbar{}Changing Chefs, Gotham Bar and Grill
Starts a New Era

\url{https://nyti.ms/2KCVF45}

\begin{itemize}
\item
\item
\item
\item
\item
\item
\end{itemize}

Advertisement

\protect\hyperlink{after-top}{Continue reading the main story}

Supported by

\protect\hyperlink{after-sponsor}{Continue reading the main story}

\href{/column/restaurant-review}{Restaurant Review}

\hypertarget{changing-chefs-gotham-bar-and-grill-starts-a-new-era}{%
\section{Changing Chefs, Gotham Bar and Grill Starts a New
Era}\label{changing-chefs-gotham-bar-and-grill-starts-a-new-era}}

\href{https://www.nytimes3xbfgragh.onion/slideshow/2019/11/19/dining/gotham-bar-and-grill-nyc.html}{}

\hypertarget{a-new-york-classic-cleans-the-slate}{%
\subsection{A New York Classic Cleans the
Slate}\label{a-new-york-classic-cleans-the-slate}}

9 Photos

View Slide Show ›

\includegraphics{https://static01.graylady3jvrrxbe.onion/images/2019/11/19/dining/19REST-GOTH-slide-FORF/19REST-GOTH-slide-FORF-articleLarge.jpg?quality=75\&auto=webp\&disable=upscale}

Daniel Krieger for The New York Times

\begin{itemize}
\tightlist
\item
  Gotham Bar and Grill\\
  **NYT Critic's Pick ★★★ American \$\$\$\$ 12 East 12th Street
  212-620-4020
\end{itemize}

\href{https://resy.com/cities/ny/gotham-bar-and-grill?utm_source=nyt\&utm_medium=restoprofile\&utm_campaign=affiliates\&aff_id=c1fe784}{Reserve
a Table}

When you make a reservation at an independently reviewed restaurant
through our site, we earn an affiliate commission.

By \href{https://www.nytimes3xbfgragh.onion/by/pete-wells}{Pete Wells}

\begin{itemize}
\item
  Nov. 19, 2019
\item
  \begin{itemize}
  \item
  \item
  \item
  \item
  \item
  \item
  \end{itemize}
\end{itemize}

Until this year, \href{https://www.gothambarandgrill.com/}{Gotham Bar
and Grill} without Alfred Portale was as hard to imagine as Cuba without
at least one Castro. Mr. Portale was not the first chef at Gotham, but
he took the job so soon after the restaurant opened in 1984, and kept
doing it for so long, that when he finally
\href{https://www.nytimes3xbfgragh.onion/2019/09/03/dining/nyc-restaurants-marc-forgione-alfred-portale.html}{left
in May} his name and its name were almost inseparable.

It was Mr. Portale who lured Tom Valenti, Bill Telepan, Tom Colicchio
and Wylie Dufresne to work at Gotham early in their careers. Gotham's
\href{https://www.nytimes3xbfgragh.onion/1985/10/04/arts/resaturants.html}{first
three-star review in The New York Times}, from Bryan Miller in 1985? Mr.
Portale was in charge. Its most recent review,
\href{https://www.nytimes3xbfgragh.onion/2011/05/18/dining/reviews/gotham-bar-and-grill-nyc-restaurant-review.html}{three
stars from Sam Sifton} in 2011? Mr. Portale was in charge. The three
\href{https://www.nytimes3xbfgragh.onion/1989/09/29/arts/restaurants-260589.html}{other}
\href{https://www.nytimes3xbfgragh.onion/1993/08/27/arts/restaurants-011693.html}{Times}
\href{https://www.nytimes3xbfgragh.onion/1996/02/23/arts/restaurants-063177.html}{reviews}
in between? Always Mr. Portale. Always three stars, too.

Anyone following a long-running success like that has two choices. She
can try not to startle the horses by speaking softly and moving slowly.
Or she can get it over with and shoot the horses.
\href{https://www.victoriablamey.com/}{Victoria Blamey}, who was hired
to
\href{https://www.nytimes3xbfgragh.onion/2019/07/23/dining/gotham-bar-and-grill-chef-victoria-blamey.html}{take
over for Mr. Portale} at Gotham Bar and Grill, shot the horses. In
\href{https://www.nytimes3xbfgragh.onion/2019/09/03/dining/gotham-bar-and-grill-victoria-blamey.html}{an
interview published} in September, speaking of the architectural plating
style Mr. Portale was once known for, she said, ``no one wants to see
that right now.''

The next day she let loose with a menu that buried every one of her
predecessor's dishes (here's your hat, tuna tartare; what's your hurry,
seafood salad?) under a heap of her own ideas. No question, Gotham Bar
and Grill is in the Blamey era now. Loyal customers are going to have to
get out or strap in.

Ms. Blamey's new menu is not self-consciously avant-garde, the way
\href{https://www.nytimes3xbfgragh.onion/2012/07/18/dining/reviews/atera-in-tribeca.html}{Atera's
tasting menus} were when she was the sous-chef there, but it may be more
original. It's certainly not like anything else around town. Her plating
style is stark --- not aggressive, but a little severe and occasionally
challenging. The flavors aren't like that, though. They're deep and
enveloping; they tend to open and change in your mouth, and linger after
you swallow, like wine.

\includegraphics{https://static01.graylady3jvrrxbe.onion/images/2019/11/19/dining/19rest-goth/merlin_164215971_73ed9ce1-01ec-4e59-8c1f-bdcf082bf02b-articleLarge.jpg?quality=75\&auto=webp\&disable=upscale}

Take her ceviche. Normally, you know where you stand with a ceviche. You
know how the marinade will change the seafood, how the seafood's juices
will knock the edge off the citrus and how the chile fits in.

The sea scallop ceviche at Gotham is a trickier character. The
splayed-open scallops sit in a pale yellow juice that tastes of corn,
first sweetly, like a fruit, and then not so sweetly. Then other things
enter the picture, smoke and citrus and chiles. The scallops taste like
scallops, so of course they're wonderful, but they've also been rolled
in chile salt and each one has a pink dot of aged umeboshi paste in the
center. There is more to the story, including some charred baby corn
whose kernels are the size of poppy seeds, but it should be clear
already that you will need to hold on tight if you are going to follow
every twist and turn at Gotham.

As anyone knows who has eaten
\href{https://www.nytimes3xbfgragh.onion/2017/01/31/dining/chumleys-review-bar-west-village.html}{the
drippily sophisticated burger} incorporating bone marrow and American
cheese that she invented when she was the chef at Chumley's, Ms. Blamey
has a way of using fat to underline her points. Most of the time, this
comes off as smart rather than obvious. (With the burger, now available
at Gotham's long, much-loved, pink-granite bar, it's both.)

Try the Baywater Sweet oysters farmed in the Hood Canal in Washington
State; they're creamy enough to bring on a swoon, but instead of the
mignonette that might keep the fat in check, Ms. Blamey spoons a
cauliflower and white chocolate purée into each silvery shell. Half
vegetable and half candy, it even upstages the black pearls of caviar on
top.

Triangles of foie gras torchon under a translucent layer of
black-truffle jelly are really quite fluffy and rich, even by the
standards of foie gras torchons. So of course Ms. Blamey serves it with
butter --- kombu butter, which goes so well with the warm kombu brioche.
If the city's
\href{https://www.nytimes3xbfgragh.onion/2019/10/30/nyregion/foie-gras-ban-nyc.html}{new}
\href{https://www.nytimes3xbfgragh.onion/2019/10/30/nyregion/foie-gras-ban-nyc.html}{ban
on foie gras} ever goes into effect, I already know one place where I'll
stop to say goodbye.

While the torchon will not win Ms. Blamey many fans among the vegans,
the cabbage should. The warm, satiny leaves --- she uses the caraflex
variety, tender and virtually ribless --- are dressed with sugary brown
juice pressed from charred onions; golden pearls of fregola give the
dish the heft of a main course.

Although Ms. Blamey is quite clearly a cheerful carnivore, she turns
vegetables into compelling events. She amplifies the smokiness of creamy
charred Japanese eggplant with a lapsang souchong broth; her dal is made
with coconut milk, mustard seeds, fried curry leaves and the care a good
Indian cook would put into it. The legumes, though, are the dense,
creamy red peas long cultivated by Gullah rice farmers on the Sea
Islands of South Carolina and Georgia.

Image

The interior, a classic of 1980s design, was lightened and
brightened.Credit...Daniel Krieger for The New York Times

The animals she does cook, and there are quite a few, are ones she
believes were sustainably raised or caught from healthy populations.
Tilefish, not popular enough yet to be overfished, is on the menu, under
a thick coral pelt of caramelized sea urchin from Maine. Oregon
Dungeness crab stewed with tomatoes and ají dulce peppers, a twist on an
abalone dish Ms. Blamey ate in Chile as a girl, is stuffed between
layers of puff pastry to make a cross between an empanada and pithivier.

Rabbit leg is braised in olive oil and an allium stock flavored with
charred fig leaves and seasoned with fish sauce. I learned that later
on, in a phone call with Ms. Blamey; at the time, I just knew that I'd
rarely had rabbit as rich or as sneakily packed with flavor. The rabbit
is served under black lentils. The lentils are augmented with chicken
fat.

The owners and Ms. Blamey have said they want to entice younger diners
to give Gotham a try. Josh Lit, the wine director, does his part by
adding five pages of fashionably noninterventionist, sulfur-phobic
winemakers like Christian Tschida and
\href{https://www.nytimes3xbfgragh.onion/2016/08/31/dining/organic-wine-frank-cornelissen-sicily.html}{Frank
Cornelissen} to a list that is
already\href{https://static1.squarespace.com/static/5ca4ce5ae8ba4423cb55c088/t/5db72721f3cffa778791a022/1572284194223/winelist-186-393+\%281\%29.pdf}{as
long as a novella}. There's a whole page for
\href{https://www.swickwines.com/}{Joe Swick}, a young Oregonian who
calls his chardonnay ``Wyd? U up?'' and his field blend ``Only Zuul,'' a
``Ghostbusters'' reference that the list handily decodes with the
hashtag \#thereisnodana.

The interior may be a sticking point in the youth campaign; it hasn't
kept up with the pace of change in the kitchen. The walls were freshly
painted and the lighting fine-tuned this summer before the new menu was
unwrapped, but Gotham is still a time capsule of postmodernist
restaurant design motifs from the 1980s, when vast loftlike spaces were
automatically exciting. There is still some drama left in the
theatrical, multilevel dining room; the parachute chandeliers that show
off how far away the ceilings are; and the plants from the garden that
press their leaves up against the back door as if they wanted to come
inside. But it's not the kind of drama that translates well to
Instagram, and so far the new crowd looks a lot like the old crowd.

At first you might wonder why a pastry chef with more modern leanings
hasn't been recruited. Then you eat
\href{https://www.starchefs.com/cook/chefs/bio/ron-paprocki}{Ron
Paprocki}'s desserts and you stop wondering. Who would wish for
vegetable ice creams, shattered cakes and fermented fruit when we could
be eating sticks of pain perdu under a crunchy caramel shell, or
near-levitating soufflés made from the red flesh of Ruby Prince peaches?

Mr. Paprocki also makes what must be the greatest tarte Tatin in the
city, a ring of apples in a state of collapse on a Frisbee of puff
pastry stained with what always seems to be the right amount of burned
sugar. If somebody tries to tell you it needs to be updated, just laugh
and walk away.

\emph{Follow} \emph{\href{https://twitter.com/nytfood}{NYT Food on
Twitter}} \emph{and}
\emph{\href{https://www.instagram.com/nytcooking/}{NYT Cooking on
Instagram},}
\emph{\href{https://www.facebookcorewwwi.onion/nytcooking/}{Facebook},}
\emph{\href{https://www.youtube.com/nytcooking}{YouTube}} \emph{and}
\emph{\href{https://www.pinterest.com/nytcooking/}{Pinterest}.}
\emph{\href{https://www.nytimes3xbfgragh.onion/newsletters/cooking}{Get
regular updates from NYT Cooking, with recipe suggestions, cooking tips
and shopping advice}.}

Advertisement

\protect\hyperlink{after-bottom}{Continue reading the main story}

\hypertarget{site-index}{%
\subsection{Site Index}\label{site-index}}

\hypertarget{site-information-navigation}{%
\subsection{Site Information
Navigation}\label{site-information-navigation}}

\begin{itemize}
\tightlist
\item
  \href{https://help.nytimes3xbfgragh.onion/hc/en-us/articles/115014792127-Copyright-notice}{©~2020~The
  New York Times Company}
\end{itemize}

\begin{itemize}
\tightlist
\item
  \href{https://www.nytco.com/}{NYTCo}
\item
  \href{https://help.nytimes3xbfgragh.onion/hc/en-us/articles/115015385887-Contact-Us}{Contact
  Us}
\item
  \href{https://www.nytco.com/careers/}{Work with us}
\item
  \href{https://nytmediakit.com/}{Advertise}
\item
  \href{http://www.tbrandstudio.com/}{T Brand Studio}
\item
  \href{https://www.nytimes3xbfgragh.onion/privacy/cookie-policy\#how-do-i-manage-trackers}{Your
  Ad Choices}
\item
  \href{https://www.nytimes3xbfgragh.onion/privacy}{Privacy}
\item
  \href{https://help.nytimes3xbfgragh.onion/hc/en-us/articles/115014893428-Terms-of-service}{Terms
  of Service}
\item
  \href{https://help.nytimes3xbfgragh.onion/hc/en-us/articles/115014893968-Terms-of-sale}{Terms
  of Sale}
\item
  \href{https://spiderbites.nytimes3xbfgragh.onion}{Site Map}
\item
  \href{https://help.nytimes3xbfgragh.onion/hc/en-us}{Help}
\item
  \href{https://www.nytimes3xbfgragh.onion/subscription?campaignId=37WXW}{Subscriptions}
\end{itemize}
