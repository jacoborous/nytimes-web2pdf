Sections

SEARCH

\protect\hyperlink{site-content}{Skip to
content}\protect\hyperlink{site-index}{Skip to site index}

\href{https://www.nytimes3xbfgragh.onion/section/technology/personaltech}{Personal
Tech}

\href{https://myaccount.nytimes3xbfgragh.onion/auth/login?response_type=cookie\&client_id=vi}{}

\href{https://www.nytimes3xbfgragh.onion/section/todayspaper}{Today's
Paper}

\href{/section/technology/personaltech}{Personal Tech}\textbar{}In Data
Journalism, Tech Matters Less Than the People

\url{https://nyti.ms/2CJgDJW}

\begin{itemize}
\item
\item
\item
\item
\item
\end{itemize}

Advertisement

\protect\hyperlink{after-top}{Continue reading the main story}

Supported by

\protect\hyperlink{after-sponsor}{Continue reading the main story}

Tech We're Using

\hypertarget{in-data-journalism-tech-matters-less-than-the-people}{%
\section{In Data Journalism, Tech Matters Less Than the
People}\label{in-data-journalism-tech-matters-less-than-the-people}}

Ben Casselman, an economics reporter, uses a programming language called
R and works with vast data sets. But he says interviews still make for
the best stories.

\includegraphics{https://static01.graylady3jvrrxbe.onion/images/2019/11/12/business/12techusing1/merlin_163866909_50471ecf-057e-44c0-af90-026ab858ada0-articleLarge.jpg?quality=75\&auto=webp\&disable=upscale}

\href{https://www.nytimes3xbfgragh.onion/by/ben-casselman}{\includegraphics{https://static01.graylady3jvrrxbe.onion/images/2018/11/09/multimedia/author-ben-casselman/author-ben-casselman-thumbLarge.png}}

Featuring \href{https://www.nytimes3xbfgragh.onion/by/ben-casselman}{Ben
Casselman}

\begin{itemize}
\item
  Nov. 13, 2019
\item
  \begin{itemize}
  \item
  \item
  \item
  \item
  \item
  \end{itemize}
\end{itemize}

\emph{How do New York Times journalists use technology in their jobs and
in their personal lives? Ben Casselman, an economics and business
reporter, discussed the tech he's using.}

\textbf{You compile complicated data sets and distill them into stories
for our readers. What are the tech tools and methods behind this
madness?}

``Madness'' is tough but fair. If you walk by my desk on any given day
you'll find my computer monitor littered with charts, spreadsheets and
way more Chrome tabs than any sane person would consider reasonable. And
my physical desktop is just as cluttered with half-used notebooks,
printed-out economics papers and my trusty TI-86 calculator (a holdover
from college calculus that I still find inexplicably useful ---
apparently a
\href{https://www.nytimes3xbfgragh.onion/2018/08/08/technology/personaltech/fred-economics-writers.html}{theme
among economics reporters} at The Times).

Really, though, the most important piece of technology on my desk is my
landline telephone. I think some people have the idea that ``data
journalism'' means staring at spreadsheets until a story magically
appears, but in the real world that almost never happens. The best
stories almost always emerge from talking to people, whether they are
experts or just ordinary people affected by the issues we write about.
They're the ones who pose the questions that data can help answer, or
who help explain the trends that the data reveals, or who can provide
the wrinkles and nuances that the data glosses over.

For example, one of my favorite stories from the past year is one that I
wrote with my colleague Conor Dougherty about
\href{https://www.nytimes3xbfgragh.onion/interactive/2019/06/20/business/economy/starter-homes-investors.html}{investors
buying up single-family homes}. We had access to a huge data set ---
millions and millions of real estate transactions --- that showed how
investors had come to dominate the market for starter homes in many
cities. But what really made the story come to life was when we zoomed
in on one house that had changed hands several times and got to talk to
all the people who had touched it --- the investor who flipped it, the
family that bought it, the would-be buyer who kept losing out to
investors.

At the end of the day, data isn't the story; people are the story.

\textbf{Fair enough. But I assume you aren't doing all that data
analysis on your phone. Or on your TI calculator, for that matter.}

Yes, that's true. I do most of my data analysis in a statistical
programming language called R. It lets me work with data sets that have
hundreds of thousands or even millions of rows, much too large for a
spreadsheet program like Excel. It also makes it easy to automate tasks
that I perform regularly --- so when the Labor Department's
\href{https://www.nytimes3xbfgragh.onion/2018/02/01/insider/insider-jobs-report.html}{monthly
jobs report} comes out, for example, I can download the new data and
update my analysis with just a few keystrokes.

R also has a great suite of tools for making charts, which is a critical
part of my work. The Times has the best graphics team in the business,
and I can't come close to doing what it does in terms of making
beautiful, readable charts and interactive graphics. But I'm not trying
to make charts like that. I just need something that lets me spot a
trend or relationship, or that helps identify examples that are worth
exploring further. For that, quick-and-dirty is just fine.

In terms of hardware, I wish I could tell you I was running some hotshot
rig with multicore processors and a boatload of RAM. But these days I do
pretty much all my work on a standard-issue laptop. I store some data in
the cloud, but that's mostly because I don't want to lose it all if my
computer melts down.

That's mostly a reflection of how quickly technology has improved. A few
years back, when I was at The Wall Street Journal, I had to go begging
for extra processing power, and even then my co-workers used to complain
that my computer sounded like a jet airplane struggling to gain
altitude. These days, my computer can manage much larger data sets
without breaking a sweat.

\includegraphics{https://static01.graylady3jvrrxbe.onion/images/2019/11/13/business/13techusing2/merlin_163866855_a610d968-ff7f-422b-80ec-594c7e445baf-articleLarge.jpg?quality=75\&auto=webp\&disable=upscale}

\textbf{There has been a running debate in journalism circles about
whether reporters should learn to code. I take it you're firmly in the
``yes'' camp?}

Not at all! Coding has been a valuable tool for me, and I'm glad that
more journalists are learning to code (and that more coders are getting
interested in journalism). But I was a reporter long before I learned
how to code, and most of the reporters I admire most have never written
a line of code in their life.

I do think that pretty much all reporters need to have a basic comfort
with numbers and statistics. Not that everyone needs to be able to run
regressions or calculate p-values, or even define what a p-value is
(which
\href{https://fivethirtyeight.com/features/not-even-scientists-can-easily-explain-p-values/}{isn't
easy}, by the way). But they should be fluent enough in the language of
statistics to understand when an argument makes sense, and when it's
suspect enough that they should start probing more deeply. And, yes, I
think you should be able to do a percent change calculation without
turning to Google for help.

In the last couple of years, The Times has developed a course to teach
basic data skills to reporters and editors --- things like how to vet
data to make sure it's legitimate, how to evaluate statistical claims
and how to use a spreadsheet program to explore a data set. One of the
core messages is that you can do a lot of this stuff \emph{without}
learning to code. I do a session where I walk through how to replicate a
\href{https://www.nytimes3xbfgragh.onion/2017/10/24/business/economy/future-jobs.html}{story
of mine} using nothing but Google Sheets and a few basic functions.

\textbf{Outside of work, what tech product are you personally obsessed
with? What do you do with it?}

Despite my love of data, I'm not much of a techie in my nonwork life.

I don't have a smart watch or a smart thermostat or anything like that,
and I still had an old cathode-ray TV until my in-laws finally broke
down and sent us a flat screen. (I did give my wife a HomePod, Apple's
smart speaker, for her birthday a couple of years ago, but it mostly
just terrifies me by having Siri pipe up seemingly at random.)

Probably my most prized digital possession is my kitchen scale. I've
gotten really into baking sourdough bread, without any commercial yeast,
which means it's about the lowest-tech form of baking imaginable.

Advertisement

\protect\hyperlink{after-bottom}{Continue reading the main story}

\hypertarget{site-index}{%
\subsection{Site Index}\label{site-index}}

\hypertarget{site-information-navigation}{%
\subsection{Site Information
Navigation}\label{site-information-navigation}}

\begin{itemize}
\tightlist
\item
  \href{https://help.nytimes3xbfgragh.onion/hc/en-us/articles/115014792127-Copyright-notice}{©~2020~The
  New York Times Company}
\end{itemize}

\begin{itemize}
\tightlist
\item
  \href{https://www.nytco.com/}{NYTCo}
\item
  \href{https://help.nytimes3xbfgragh.onion/hc/en-us/articles/115015385887-Contact-Us}{Contact
  Us}
\item
  \href{https://www.nytco.com/careers/}{Work with us}
\item
  \href{https://nytmediakit.com/}{Advertise}
\item
  \href{http://www.tbrandstudio.com/}{T Brand Studio}
\item
  \href{https://www.nytimes3xbfgragh.onion/privacy/cookie-policy\#how-do-i-manage-trackers}{Your
  Ad Choices}
\item
  \href{https://www.nytimes3xbfgragh.onion/privacy}{Privacy}
\item
  \href{https://help.nytimes3xbfgragh.onion/hc/en-us/articles/115014893428-Terms-of-service}{Terms
  of Service}
\item
  \href{https://help.nytimes3xbfgragh.onion/hc/en-us/articles/115014893968-Terms-of-sale}{Terms
  of Sale}
\item
  \href{https://spiderbites.nytimes3xbfgragh.onion}{Site Map}
\item
  \href{https://help.nytimes3xbfgragh.onion/hc/en-us}{Help}
\item
  \href{https://www.nytimes3xbfgragh.onion/subscription?campaignId=37WXW}{Subscriptions}
\end{itemize}
