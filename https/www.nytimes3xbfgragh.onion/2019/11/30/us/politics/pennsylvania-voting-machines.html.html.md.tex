Sections

SEARCH

\protect\hyperlink{site-content}{Skip to
content}\protect\hyperlink{site-index}{Skip to site index}

\href{https://www.nytimes3xbfgragh.onion/section/politics}{Politics}

\href{https://myaccount.nytimes3xbfgragh.onion/auth/login?response_type=cookie\&client_id=vi}{}

\href{https://www.nytimes3xbfgragh.onion/section/todayspaper}{Today's
Paper}

\href{/section/politics}{Politics}\textbar{}A Pennsylvania County's
Election Day Nightmare Underscores Voting Machine Concerns

\url{https://nyti.ms/2rGZTkk}

\begin{itemize}
\item
\item
\item
\item
\item
\end{itemize}

Advertisement

\protect\hyperlink{after-top}{Continue reading the main story}

Supported by

\protect\hyperlink{after-sponsor}{Continue reading the main story}

\hypertarget{a-pennsylvania-countys-election-day-nightmare-underscores-voting-machine-concerns}{%
\section{A Pennsylvania County's Election Day Nightmare Underscores
Voting Machine
Concerns}\label{a-pennsylvania-countys-election-day-nightmare-underscores-voting-machine-concerns}}

How ``everything went wrong'' in Northampton County.

\includegraphics{https://static01.graylady3jvrrxbe.onion/images/2019/11/29/us/politics/29votingmachines-01/merlin_165091683_5b6e365e-0cdf-444c-b458-63152bdbad76-articleLarge.jpg?quality=75\&auto=webp\&disable=upscale}

\href{https://www.nytimes3xbfgragh.onion/by/nick-corasaniti}{\includegraphics{https://static01.graylady3jvrrxbe.onion/images/2018/06/13/multimedia/author-nick-corasaniti/author-nick-corasaniti-thumbLarge-v2.png}}

By \href{https://www.nytimes3xbfgragh.onion/by/nick-corasaniti}{Nick
Corasaniti}

\begin{itemize}
\item
  Nov. 30, 2019
\item
  \begin{itemize}
  \item
  \item
  \item
  \item
  \item
  \end{itemize}
\end{itemize}

EASTON, Pa. --- It was a few minutes after the polls closed here on
Election Day when panic began to spread through the county election
offices.

Vote totals in a Northampton County judge's race showed one candidate,
Abe Kassis, a Democrat, had just 164 votes out of 55,000 ballots across
more than 100 precincts. Some machines reported zero votes for him. In a
county with the ability to vote for a straight-party ticket, one
candidate's zero votes was a near statistical impossibility. Something
had gone quite wrong.

Lee Snover, the chairwoman of the county Republicans, said her anxiety
began to pick up at 9:30 p.m. on Nov. 5. She had trouble getting someone
from the election office on the phone. When she eventually got through,
she said: ``I'm coming down there and you better let me in.''

With clearly faulty results in at least the judge's election, officials
began counting the paper backup ballots generated by the same machines.
The paper ballots showed Mr. Kassis winning narrowly, 26,142 to 25,137,
over his opponent, the Republican Victor Scomillio.

``People were questioning, and even I questioned, that if some of the
numbers are wrong, how do we know that there aren't mistakes with
anything else?'' said Matthew Munsey, the chairman of the Northampton
County Democrats, who, along with Ms. Snover, was among the observers as
county officials worked through the night to feed the paper ballots by
hand through scanning machines.

The snafu in Northampton County did not just expose flaws in both the
election machine testing and procurement process. It also highlighted
the fears, frustrations and mistrust over election security that many
voters are feeling ahead of the 2020 presidential contest, given how
faith in American elections has never been more fragile. The problematic
machines were also used in Philadelphia and its surrounding suburbs ---
areas of Pennsylvania that could prove decisive next year in
\href{https://www.nytimes3xbfgragh.onion/2019/11/26/upshot/democratic-trump-voters-2020.html}{one
of the most critical presidential swing states} in the country.

\includegraphics{https://static01.graylady3jvrrxbe.onion/images/2019/11/29/us/politics/29votingmachines-02/merlin_165091650_ebe53acb-1380-4dfa-8586-757caa8f51f8-articleLarge.jpg?quality=75\&auto=webp\&disable=upscale}

In an era where some candidates and incumbents try to challenge or
discredit a close loss by questioning the system, either with unfounded
allegations of voter fraud or claims of a ``rigged'' election, the
proper functioning and security of election machines have never been
more crucial.

``There are concerns for 2020,'' Ms. Snover said, questioning whether
the paper ballots generated by the same machine that had a digital error
could be trusted. ``Nothing went right on Election Day. Everything went
wrong. That's a problem.''

Election Day here had been marred by complaints of long lines,
glitch-prone touch screens and frustrated poll workers. Voters across
the county said the experience further eroded their already shaken
confidence in the election process.

``It made me sad because with everything that's going on, you kind of
worry about: Was something tampered with, or was it just a mistake,''
said Michelle Broadhecker, 48, of Easton, who said her anxiety about
elections began after 2016. ``There's just too much going on that you
worry about those things. And you don't want the wrong people in the
wrong places.''

Though there has been no conclusive study as to what caused the machines
to malfunction, as the machines are locked away for 20 days after an
election according to state law, the prevailing theory is that the touch
screens were plagued by a bug in the software. A senior intelligence
official who focuses on election security said there were no visible
signs of outside meddling by any foreign actors.

County officials who led the purchase of the machines have argued that
the system actually functioned as it should: The paper ballot backup
process worked. The touch screens failed, but the backups had the
correct vote, so while it was inconvenient, it proved the necessity of a
paper backup.

``We also need to focus on the outcome, which is that voter-verified
paper ballots provided fair, accurate and legal election results, as
indicated by the county's official results reporting and successful
postelection risk-limiting audit,'' said Katina Granger, a spokeswoman
for Election Systems \& Software, the manufacturer of the machines.
``The election was legal and fair.''

Image

Michelle Broadhecker at the Quadrant Book Mart and Coffee House in
Easton, Pa.Credit...Mark Makela for The New York Times

But for others, it underscored the fractured system for selecting voting
systems. Major decisions for testing, purchasing and operating complex
machines are often left to county and city officials. Federal testing
standards for election machines haven't been updated since 2005, when a
large percentage of the machines were not digital.

``Not only is that a decade before the current cybersecurity threats to
our elections, it is two years before the first iPhone,'' said Kevin
Skoglund, a senior technical adviser for the National Election Defense
Coalition, a nonpartisan group that focuses on election security issues.
``There is a newer 2015 standard, but the Election Assistance Commission
lets voting system vendors choose which one to use.''

The machines that broke in Northampton County are called the
ExpressVoteXL and are made by Election Systems \& Software, a major
manufacturer of election machines used across the country. The
ExpressVoteXL is among their newest and most high-end machines, a luxury
``one-stop'' voting system that combines a 32-inch touch screen and a
paper ballot printer.

To initiate a vote, a voter places a blank ballot-shaped piece of paper
in the machine, makes their selections on the screen, and then presses
the word ``vote.'' The machine prints a ballot that is protected under a
plate of glass for the voter to review. The voter then clicks ``cast''
on the screen, the digital votes are recorded on a USB and the backup
ballot is transmitted to a sealed canister in the back of the machine.

The machines began arriving in the county in August, having gone through
a federal and state certification process. The only remaining testing to
be done was what officials called a ``logic and accuracy test,'' which
is a quick dry run of roughly 20 dummy ballots. But the ExpressVoteXL
has an auto-test function in which the machines can simulate a full
digital test, a feature that election security experts say is
ill-advised.

``It doesn't test if the touch screen or the scanner work. It doesn't
even cast votes for everyone on the ballot,'' Mr. Skoglund said. ``It is
especially concerning that it can send made-up votes to the vote
counting software without needing a real ballot. Fake ballots are a
feature no voting machine should have.''

The automatic tests in Northampton proved problematic, and did not even
cast a test vote for every candidate, according to test receipts shown
to The New York Times. But the machines were still rolled out on
Election Day.

Image

Angela Anderson with her dog, Frankie. Ms. Anderson's vote on the new
machine did not register correctly at first; she was able to reset to
vote for her party's candidates.Credit...Mark Makela for The New York
Times

And instantly, there were problems.

``I walked into my booth, and I knew that I was going to vote straight
Democratic and I'm voting that way until we get some balance back into
the government, but when I hit straight Democratic, straight Republican
is what registered,'' said Angela Anderson, 55, of Forks Township, who
said that many of her neighbors shared similar stories. ``I kind of
panicked for a second. But thankfully it easily reset, and I reset my
system, and that time it registered Democratic.''

Deb Hunter, a member of the county election commission, said they were
actually lucky that the county judge election went so poorly because
that made the problem obvious.

``What would have happened if there was a glitch there that got at a 10
percent or 20 percent undercount?'' she said. ``That worries me. That
worries me going forward.''

Ms. Granger noted that there are nearly 6,300 ExpressVoteXL voting
machines in use across the country, and none had experienced similar
counting problems to those in Northampton County.

It was the way the machines were selected by Philadelphia elected
officials that drew the most scrutiny over the last year. Since 2013,
E.S.\&S. had been courting the two city commissioners who were
responsible for choosing the next voting machine, according to
\href{https://3og1cv1uvq3u3skase2jhb69-wpengine.netdna-ssl.com/wp-content/uploads/2019/09/VOTING-TECHNOLOGY-PROCUREMENT-INVESTIGATION-PUBLIC.pdf}{a
report from the city comptroller}.

The lobbying firm for E.S.\&S. had donated \$1,000 in 2013 to the
campaign of Al Schmidt, one of the city commissioners, and again to a
group supporting his re-election effort in 2018. They also spent more
than \$27,000 in direct lobbying of Mr. Schmidt.

Mr. Schmidt made a visit to only one company's headquarters: E.S.\&S.

In total, E.S.\&S. spent more than \$425,000 in lobbying expenses
related to the City of Philadelphia.

Emails obtained by the city comptroller also found that E.S.\&S. had
influenced the writing of the city commissioners' \$22 million budget
request for new election machines, tilting the process in favor of its
machine, the ExpressVoteXL. The city eventually purchased the machines
for \$29 million in February.

``It showed a very, very flawed process,'' said Rebecca Rhynhart, the
\href{https://controller.phila.gov/}{city controller} in Philadelphia.
``I want to make sure, and the country should want to make sure, that
our voting machines are the best they can be.''

As for Northampton,
\href{https://www.wfmz.com/features/think-tank/northampton-county-voting-machine-accuracy-and-reliability/article_87dfa09a-0fa8-11ea-8acb-87da637467b7.html}{some
on the county council} have a new goal: new, simpler paper-ballot
machines ahead of the presidential election, as well as some money back.

\emph{Matthew Rosenberg contributed reporting from Washington.}

Advertisement

\protect\hyperlink{after-bottom}{Continue reading the main story}

\hypertarget{site-index}{%
\subsection{Site Index}\label{site-index}}

\hypertarget{site-information-navigation}{%
\subsection{Site Information
Navigation}\label{site-information-navigation}}

\begin{itemize}
\tightlist
\item
  \href{https://help.nytimes3xbfgragh.onion/hc/en-us/articles/115014792127-Copyright-notice}{©~2020~The
  New York Times Company}
\end{itemize}

\begin{itemize}
\tightlist
\item
  \href{https://www.nytco.com/}{NYTCo}
\item
  \href{https://help.nytimes3xbfgragh.onion/hc/en-us/articles/115015385887-Contact-Us}{Contact
  Us}
\item
  \href{https://www.nytco.com/careers/}{Work with us}
\item
  \href{https://nytmediakit.com/}{Advertise}
\item
  \href{http://www.tbrandstudio.com/}{T Brand Studio}
\item
  \href{https://www.nytimes3xbfgragh.onion/privacy/cookie-policy\#how-do-i-manage-trackers}{Your
  Ad Choices}
\item
  \href{https://www.nytimes3xbfgragh.onion/privacy}{Privacy}
\item
  \href{https://help.nytimes3xbfgragh.onion/hc/en-us/articles/115014893428-Terms-of-service}{Terms
  of Service}
\item
  \href{https://help.nytimes3xbfgragh.onion/hc/en-us/articles/115014893968-Terms-of-sale}{Terms
  of Sale}
\item
  \href{https://spiderbites.nytimes3xbfgragh.onion}{Site Map}
\item
  \href{https://help.nytimes3xbfgragh.onion/hc/en-us}{Help}
\item
  \href{https://www.nytimes3xbfgragh.onion/subscription?campaignId=37WXW}{Subscriptions}
\end{itemize}
