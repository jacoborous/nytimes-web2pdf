Sections

SEARCH

\protect\hyperlink{site-content}{Skip to
content}\protect\hyperlink{site-index}{Skip to site index}

\href{https://myaccount.nytimes3xbfgragh.onion/auth/login?response_type=cookie\&client_id=vi}{}

\href{https://www.nytimes3xbfgragh.onion/section/todayspaper}{Today's
Paper}

Louis Vuitton to Go, a New Kyoto Hotel and More

\url{https://nyti.ms/2Nuc8td}

\begin{itemize}
\item
\item
\item
\item
\item
\end{itemize}

Advertisement

\protect\hyperlink{after-top}{Continue reading the main story}

Supported by

\protect\hyperlink{after-sponsor}{Continue reading the main story}

Notes on the Culture

\hypertarget{louis-vuitton-to-go-a-new-kyoto-hotel-and-more}{%
\section{Louis Vuitton to Go, a New Kyoto Hotel and
More}\label{louis-vuitton-to-go-a-new-kyoto-hotel-and-more}}

T's roundup of people, places and things to know now.

Published Nov. 8, 2019Updated Nov. 19, 2019

\begin{itemize}
\item
\item
\item
\item
\item
\end{itemize}

\includegraphics{https://static01.graylady3jvrrxbe.onion/images/2019/11/08/t-magazine/08tmag-notc-slide-G0AB/08tmag-notc-slide-G0AB-articleLarge.jpg?quality=75\&auto=webp\&disable=upscale}

\hypertarget{louis-vuittons-performance-clothes--for-tomorrow}{%
\subsubsection{Louis Vuitton's Performance Clothes --- for
Tomorrow}\label{louis-vuittons-performance-clothes--for-tomorrow}}

\href{https://www.nytimes3xbfgragh.onion/2018/03/02/t-magazine/virgil-abloh-off-white-paris-fashion-week.html}{Virgil
Abloh}, the artistic director of
\href{https://www.louisvuitton.com/}{Louis Vuitton}'s men's wear, is
nothing if not intellectually nimble. He trained as an architect and has
worked as a D.J., artist, musician and Kanye West whisperer. When it
comes to fashion, his references are equally wide-ranging, and he's
proved as likely to send trench coats and pleated pants down the runway
as mesh tees and floral harnesses. With his latest capsule collection,
he's baked versatility into individual pieces. Included in the 14-item
line, called Louis Vuitton 2054 (the year the brand will turn 200), is a
shirt that turns into a pillow, a weekend bag that morphs into a
sleeping bag and a coat that doubles as a backpack. It was, according to
Abloh, an exercise in rethinking the nature of apparel and what the
future of fashion will be. He arrived at his answer --- technical and
transformable --- after looking at collapsible camping equipment. ``I
was very much inspired by the materials and folding ingenuity that
exists in that world of products,'' Abloh said, in comments emailed by
the brand. Indeed, folding is an integral component in experiencing
these multifunctional items: The shirt, papery nylon with removable arms
and plexiglass zipper pulls, can be tucked into its own back pocket, and
the sleeping bag rolls out of the side compartment of an oversize
lambskin duffel.

\emph{{[}}\href{https://www.nytimes3xbfgragh.onion/newsletters/t-list?module=inline}{\emph{Sign
up here}} \emph{for the T List newsletter, a weekly roundup of what T
Magazine editors are noticing and coveting now.{]}}

These days, of course,
\href{https://www.nytimes3xbfgragh.onion/interactive/2019/04/10/t-magazine/humberto-leon-carol-lim.html}{imagining
the future} can be a grim undertaking, and technical ware feels imbued
with a sharper survivalist edge. But Abloh seems optimistic. What first
appears as a somber, no-nonsense palette reveals splashes of
rainbow-colored camouflage --- a glossy nylon puffer scarf, for
instance, is black on one side and tie-dye-like on the other. The
designer also retains faith in the relationship between man and nature.
The muse for the collection was someone who ``actively engages with the
outdoors,'' he said. That's not the sort of client apt to shut
themselves away, and why should it be, given Louis Vuitton's origins as
a luggage company? Though here Abloh has foregone stacks of trunks to
send a different message: For maximum agility, and for the sake of our
planet, it's best to travel light. ---
\href{https://www.nytimes3xbfgragh.onion/by/kate-guadagnino}{KATE
GUADAGNINO}

\begin{center}\rule{0.5\linewidth}{\linethickness}\end{center}

Image

A pair of black hyedua zigzag pulls.Credit...Image by Nicolas
Kern/Styled by Ashley Helvey

Image

A kitchen corner, featuring Green River Project cabinets and hardware
--- and cats Scotty and Ozzie.Credit...Andrew Jacobs

\hypertarget{where-form-meets-function}{%
\subsubsection{Where Form Meets
Function}\label{where-form-meets-function}}

The pair behind the conceptual furniture collective
\href{https://www.nytimes3xbfgragh.onion/2019/08/07/t-magazine/green-river-project-aaron-aujla-ben-bloomstein.html}{Green
River Project} met two years ago while working in New York's art world:
Aaron Aujla was an artist's assistant and painter, and Benjamin
Bloomstein was an art handler and craftsman. One of their first clients
was their friend
\href{https://www.nytimes3xbfgragh.onion/2017/04/11/t-magazine/design/michael-bargo-apartment.html}{Michael
Bargo}, an interior designer whose 2,000-square-foot Chinatown apartment
doubles as a design gallery. In the main room, there are displays of
arresting and rare midcentury pieces --- like a 1950s-era Alexandre Noll
wooden cross sculpture and a Pierre Chareau steel-and-brass side table
from the 1920s --- that are his only until he sells them.

The loft's one uninspiring space was, Bargo felt, the generic
contemporary kitchen, so he called Aujla and Bloomstein for help. For
inspiration, the designers looked to the modular color-block kitchen
that
\href{https://www.nytimes3xbfgragh.onion/2013/11/14/t-magazine/charlotte-perriand-design-le-corbusier.html}{Charlotte
Perriand} and
\href{https://www.nytimes3xbfgragh.onion/2018/08/08/t-magazine/le-corbusier-japan-modernism.html}{Le
Corbusier} conceived for Unité d'Habitation --- a 1940s social-housing
project in Marseille, France --- finding themselves particularly drawn
to a pair of chunky wooden handles on an upper cabinet. They hand-carved
a 15-piece series of sculptural handles and pulls from African mahogany,
cherry, ebony and black hyedua --- one is fluted like a trumpet, while
several others echo the undulating totems of
\href{https://www.nytimes3xbfgragh.onion/2016/09/22/t-magazine/art/impasse-ronsin-artists-montparnasse-constantin-brancusi.html}{Constantin
Brancusi} --- and affixed a 10-inch set to a fridge that they paneled
with black- and Bauhaus-red-colored Formica. It stands to the side of an
aluminum-edged Douglas-fir island and a pair of wooden bar stools with
crescent-shaped seats, also carved by hand. The kitchen, as well as
free-standing versions of the hardware and two Green River Project
perforated aluminum sconces, will soon go on view at Bargo's
apartment-cum-gallery. ``We want people to experience high-level
craftsmanship on a daily basis in their home,'' says Aujla. --- AHNNA
LEE

\begin{center}\rule{0.5\linewidth}{\linethickness}\end{center}

Image

Stone steps leading to one of Aman Kyoto's guest pavilions.Credit...Aman

\hypertarget{a-new-kyoto-hotel-complete-with-its-own-forest}{%
\subsubsection{A New Kyoto Hotel, Complete With Its Own
Forest}\label{a-new-kyoto-hotel-complete-with-its-own-forest}}

The Aman hotel group has long upheld the Japanese architectural
principle that buildings should be in harmony with their natural
environments, but that thinking is especially evident at
\href{https://www.aman.com/resorts/aman-kyoto}{its latest location} ---
a secluded eight-acre garden-within-a-forest at the base of
\href{https://www.nytimes3xbfgragh.onion/2014/03/02/travel/36-hours-in-kyoto-japan.html}{Kyoto}'s
Mount Hidari Daimonji. ``It's all about the grounds,'' says the
designer, Justin Hill, which are accessed via an ancient copper gate and
include a large main lawn, naturally occurring streams and moss-covered
stone walkways surrounded by dense plots of Japanese maple and cedar
trees. The land likely inspired members of the Rinpa school, an
Edo-period art movement that encouraged a resurgence of indigenous
techniques and motifs, and once belonged to a textile collector and
amateur landscape designer. Now, it encompasses 11 slatted stained-cedar
pavilions housing 24 rooms and two two-bedroom suites between them,
though on a recent visit, Hill was pleased to find that the pavilions
are difficult to photograph: ``They almost disappear into the
landscape,'' he says.

Image

A communal garden terrace.Credit...Aman

Inside, the structures allude to a classic ryokan, with tatami mats,
orb-shaped lanterns, hinoki tubs and tokonoma, or wall niches, here used
to display local pottery. Instead of paper screens (``We wanted to pay
tribute to Japanese design, not mimic it,'' says Hill), there are blond
wood wall panels and sliding doors, as well as floor-to-ceiling windows
that offer verdant views. The hotel restaurant serves \emph{kaiseki},
seasonal multicourse meals, and guests can experience other local
customs by soaking in the open-air onsen or venturing past the garden's
edge for meditative hikes known as \emph{shinrin-yoku}, or forest
bathing. ---
\href{https://www.nytimes3xbfgragh.onion/by/amelia-lester}{AMELIA
LESTER}

\begin{center}\rule{0.5\linewidth}{\linethickness}\end{center}

Image

Credit...Weichia Huang (7)

\hypertarget{mini-market-childlike-jewelry}{%
\subsubsection{Mini Market: Childlike
Jewelry}\label{mini-market-childlike-jewelry}}

Candy-colored pieces reminiscent of the craft table and princess games.

Clockwise from top left: \textbf{Irene Neuwirth} necklace, price on
request, \href{https://ireneneuwirth.com/}{ireneneuwirth.com}.
\textbf{Carolina Bucci} bracelet, \$630,
\href{https://carolinabucci.com/}{carolinabucci.com}. \textbf{Taffin}
necklace, price on request, (212) 421-6222. \textbf{Marie-Hélène de
Taillac} ring, \$4,900, (212) 249-0371. \textbf{Brent Neale} necklace,
\$35,000, (205) 871-6747. \textbf{Bea Bongiasca} ring, \$1,500,
\href{https://www.modaoperandi.com/women}{modaoperandi.com}.
\textbf{Paul Morelli} cuff, \$22,000,
\href{http://paulmorelli.com/}{paulmorelli.com}.

\begin{center}\rule{0.5\linewidth}{\linethickness}\end{center}

Image

Credit...Photos by Mari Maeda and Yuji Oboshi.~Photo assistant: Fumi
Sugino

\hypertarget{easy-to-pack-wrap-sandals}{%
\subsubsection{Easy-to-Pack Wrap
Sandals}\label{easy-to-pack-wrap-sandals}}

Twisting delicately up the calf, these strappy staples are a playful
update on the Greek classic.

Top row from left: \textbf{Celine by Hedi Slimane} sandals, \$940,
(212) 226-8001. \textbf{Ancient Greek Sandals}, \$305,
\href{https://www.shopbop.com/}{shopbop.com}. \textbf{Ulla Johnson}
sandals, \$395, \href{https://ullajohnson.com/}{ullajohnson.com}.
\textbf{Bottega Veneta} sandals, \$1,190,
\href{https://www.bottegaveneta.com/us}{bottegaveneta.com}.
\textbf{Missoni} sandals, \$915,
\href{https://www.missoni.com/us}{missoni.com}.

Bottom row from left: \textbf{Saint Laurent by Anthony Vaccarello}
sandals, \$595, \href{https://www.ysl.com/us/}{ysl.com}. \textbf{Dondoks
Paris} sandals, \$285, \href{https://dondoks.com/}{dondoks.com}.
\textbf{Etro} sandals, \$440, (212) 317-9096. \textbf{Manolo Blahnik for
Carolina Herrera} sandals, price on request, (212) 249-6552. \textbf{Jil
Sander} sandals, \$550, \href{https://www.barneys.com/}{barneys.com}.

\begin{center}\rule{0.5\linewidth}{\linethickness}\end{center}

Image

Credit...Photo by Patricia Heal. Styled by Chloe Daley. Retouching:
Anonymous Retouch. Photo assistant: Russell Underwood. Prop stylist's
assistant: Camille Coutherut.

\hypertarget{art-deco-blooms-anew}{%
\subsubsection{Art Deco Blooms Anew}\label{art-deco-blooms-anew}}

The Art Deco style --- a geometric update of late 19th century Art
Nouveau flourishes, recast through the machine-made polish of European
Futurism --- emerged soon after the end of World War I. But it wasn't
until a few years later, in 1925, that the seminal movement coalesced
with the legendary Paris Exposition Internationale des Arts Décoratifs
et Industriels Modernes.
\href{https://www.nytimes3xbfgragh.onion/2018/11/19/t-magazine/giorgio-armani-home-broni-italy.html}{Giorgio
Armani}, the endlessly entrepreneurial 85-year-old fashion designer
whose Asian-tinged minimalism has made its own mark on the culture over
nearly five decades, references the period often, especially in his
Privé collection of couture-level evening clothes. Thus, it makes
elegant sense that his first \emph{haute joaillerie} collection would
nod to the spare yet bold simplicity and primary colors of Art Deco.
These cascading earrings are made of white gold (common in the 1920s as
an alternative to platinum), coral, diamonds, sapphires and rubies --- a
glamorous yet unfussy fusion of the then and now. \emph{Price on
request, (212) 988-9191.} ---
\href{https://www.nytimes3xbfgragh.onion/by/nancy-hass}{NANCY HASS}

\begin{center}\rule{0.5\linewidth}{\linethickness}\end{center}

Image

Credit...Photo by Lauren Coleman. Styled by Todd Knopke. Photo
assistant: Gareth Smit. Stylist's assistant: Genevieve Ward

\hypertarget{the-return-of-a-futurist-classic}{%
\subsubsection{The Return of a Futurist
Classic}\label{the-return-of-a-futurist-classic}}

The Kentucky-born architect and designer
\href{https://www.nytimes3xbfgragh.onion/1997/08/09/arts/paul-rudolph-is-dead-at-78-modernist-architect-of-the-60-s.html}{Paul
Rudolph}, widely considered a spiritual father of American
\href{https://www.nytimes3xbfgragh.onion/2016/10/06/t-magazine/design/brutalist-architecture-revival.html}{Brutalism}
--- the mid-20th-century movement of raw concrete and obstinate
minimalism --- never put a premium on comfort. Instead, Rudolph, whose
best-known projects are Halston's much-photographed white-on-white Upper
East Side townhouse in Manhattan and Yale's School of Architecture
building (he was chairman of the department when it was built in 1963),
championed machine-made materials and modularity. Rarely able to find
appropriate furniture, he designed custom pieces. His Rolling chair,
created in 1968 for his own Beekman Place apartment, is now for sale by
Modulightor, the retail lighting firm that Rudolph, who died in 1997,
co-founded in the mid-1970s. Made of steel and plexiglass with casters,
its parts are interchangeable with several of his other seats and tables
that the company manufactures. Echoing the uncompromising angularity of
earlier European purists, among them the architects Gerrit Rietveld and
\href{https://www.nytimes3xbfgragh.onion/topic/person/le-corbusier}{Le
Corbusier}, the armchair eschews quotidian cushiness for transparent
provocation. \emph{\$3,450,}
\href{http://modulightor.com/}{\emph{modulightor.com}}\emph{.} --- NANCY
HASS

\begin{center}\rule{0.5\linewidth}{\linethickness}\end{center}

\href{https://www.nytimes3xbfgragh.onion/slideshow/2019/11/08/t-magazine/of-a-kind-joseph-altuzarras-owls.html}{}

\hypertarget{of-a-kind-joseph-altuzarras-owls}{%
\subsection{Of a Kind: Joseph Altuzarra's
Owls}\label{of-a-kind-joseph-altuzarras-owls}}

6 Photos

View Slide Show ›

Illustration by Aurora de la Morinerie

\hypertarget{of-a-kind-joseph-altuzarras-owls-1}{%
\subsubsection{Of a Kind: Joseph Altuzarra's
Owls}\label{of-a-kind-joseph-altuzarras-owls-1}}

Four years ago, the Paris-born, New York-based fashion designer
\href{https://www.nytimes3xbfgragh.onion/2017/10/02/t-magazine/joseph-altuzarra-paris-fashion-week.html}{Joseph
Altuzarra} received his first good-luck charm from his mother, Karen.
Just before his spring 2016 show, she gave him a small brass owl she
found at an arcade in Florence, Italy. Every show since, she has
presented Altuzarra with another owl. ``I used to love the bird as a
child and would draw them constantly,'' he says. ``My mom says owls
remind her of me --- they are always observing the world.''

His collection now consists of 16 miniatures, which Altuzarra's mother
has found on her travels across Asia, California and Europe, from toy
stores in Paris to a cast-iron merchant in Kyoto, Japan. The 36-year-old
designer keeps them in a small black box on his desk or on a nearby
bookshelf so that ``they're always within view.'' In early 2020,
Altuzarra will be featured as a judge alongside
\href{https://www.nytimes3xbfgragh.onion/2016/09/07/t-magazine/fashion/naomi-campbell-vfiles-interview.html}{Naomi
Campbell}, Nicole Richie and his muse
\href{https://www.nytimes3xbfgragh.onion/2019/02/22/style/carine-roitfeld-perfume.html}{Carine
Roitfeld} in Amazon Prime Video's new fashion competition series
``Making the Cut.'' ---
\href{https://www.nytimes3xbfgragh.onion/by/john-wogan}{JOHN WOGAN}

Advertisement

\protect\hyperlink{after-bottom}{Continue reading the main story}

\hypertarget{site-index}{%
\subsection{Site Index}\label{site-index}}

\hypertarget{site-information-navigation}{%
\subsection{Site Information
Navigation}\label{site-information-navigation}}

\begin{itemize}
\tightlist
\item
  \href{https://help.nytimes3xbfgragh.onion/hc/en-us/articles/115014792127-Copyright-notice}{©~2020~The
  New York Times Company}
\end{itemize}

\begin{itemize}
\tightlist
\item
  \href{https://www.nytco.com/}{NYTCo}
\item
  \href{https://help.nytimes3xbfgragh.onion/hc/en-us/articles/115015385887-Contact-Us}{Contact
  Us}
\item
  \href{https://www.nytco.com/careers/}{Work with us}
\item
  \href{https://nytmediakit.com/}{Advertise}
\item
  \href{http://www.tbrandstudio.com/}{T Brand Studio}
\item
  \href{https://www.nytimes3xbfgragh.onion/privacy/cookie-policy\#how-do-i-manage-trackers}{Your
  Ad Choices}
\item
  \href{https://www.nytimes3xbfgragh.onion/privacy}{Privacy}
\item
  \href{https://help.nytimes3xbfgragh.onion/hc/en-us/articles/115014893428-Terms-of-service}{Terms
  of Service}
\item
  \href{https://help.nytimes3xbfgragh.onion/hc/en-us/articles/115014893968-Terms-of-sale}{Terms
  of Sale}
\item
  \href{https://spiderbites.nytimes3xbfgragh.onion}{Site Map}
\item
  \href{https://help.nytimes3xbfgragh.onion/hc/en-us}{Help}
\item
  \href{https://www.nytimes3xbfgragh.onion/subscription?campaignId=37WXW}{Subscriptions}
\end{itemize}
