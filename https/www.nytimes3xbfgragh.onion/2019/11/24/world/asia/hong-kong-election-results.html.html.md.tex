Sections

SEARCH

\protect\hyperlink{site-content}{Skip to
content}\protect\hyperlink{site-index}{Skip to site index}

\href{https://www.nytimes3xbfgragh.onion/section/world/asia}{Asia
Pacific}

\href{https://myaccount.nytimes3xbfgragh.onion/auth/login?response_type=cookie\&client_id=vi}{}

\href{https://www.nytimes3xbfgragh.onion/section/todayspaper}{Today's
Paper}

\href{/section/world/asia}{Asia Pacific}\textbar{}Hong Kong Election
Results Give Democracy Backers Big Win

\url{https://nyti.ms/2ribbeX}

\begin{itemize}
\item
\item
\item
\item
\item
\item
\end{itemize}

Advertisement

\protect\hyperlink{after-top}{Continue reading the main story}

Supported by

\protect\hyperlink{after-sponsor}{Continue reading the main story}

\hypertarget{hong-kong-election-results-give-democracy-backers-big-win}{%
\section{Hong Kong Election Results Give Democracy Backers Big
Win}\label{hong-kong-election-results-give-democracy-backers-big-win}}

A surge in voting, especially by young people, allowed democracy
advocates to win many more seats on local councils.

\includegraphics{https://static01.graylady3jvrrxbe.onion/images/2019/11/24/world/24hk-elect-sub/24hk-elect-sub-articleLarge.jpg?quality=75\&auto=webp\&disable=upscale}

\href{https://www.nytimes3xbfgragh.onion/by/keith-bradsher}{\includegraphics{https://static01.graylady3jvrrxbe.onion/images/2018/10/08/multimedia/author-keith-bradsher/author-keith-bradsher-thumbLarge.png}}\href{https://www.nytimes3xbfgragh.onion/by/austin-ramzy}{\includegraphics{https://static01.graylady3jvrrxbe.onion/images/2018/10/15/multimedia/author-austin-ramzy/author-austin-ramzy-thumbLarge.png}}\href{https://www.nytimes3xbfgragh.onion/by/tiffany-may}{\includegraphics{https://static01.graylady3jvrrxbe.onion/images/2019/12/04/reader-center/author-tiffany-may/author-tiffany-may-thumbLarge.png}}

By \href{https://www.nytimes3xbfgragh.onion/by/keith-bradsher}{Keith
Bradsher},
\href{https://www.nytimes3xbfgragh.onion/by/austin-ramzy}{Austin Ramzy}
and \href{https://www.nytimes3xbfgragh.onion/by/tiffany-may}{Tiffany
May}

\begin{itemize}
\item
  Published Nov. 24, 2019Updated Nov. 25, 2019
\item
  \begin{itemize}
  \item
  \item
  \item
  \item
  \item
  \item
  \end{itemize}
\end{itemize}

\href{https://cn.nytimes3xbfgragh.onion/china/20191125/hong-kong-election/}{阅读简体中文版}\href{https://cn.nytimes3xbfgragh.onion/china/20191125/hong-kong-election/zh-hant/}{閱讀繁體中文版}

HONG KONG --- Pro-democracy candidates buoyed by months of street
\href{https://www.nytimes3xbfgragh.onion/2019/11/25/world/asia/hong-kong-election-protests.html}{protests}
in
\href{https://www.nytimes3xbfgragh.onion/2019/11/25/world/asia/hong-kong-election-protests.html}{Hong
Kong} won
\href{https://www.nytimes3xbfgragh.onion/interactive/2019/11/24/world/asia/hong-kong-election-results.html}{a
stunning victory in local elections} on Sunday, as record numbers voted
in a vivid expression of the city's aspirations and its anger with the
Chinese government.

It was a pointed rebuke of
\href{https://www.nytimes3xbfgragh.onion/2019/11/25/world/asia/hong-kong-election-protests.html}{Beijing}
and its allies in
\href{https://www.nytimes3xbfgragh.onion/2019/11/25/world/asia/hong-kong-election-protests.html}{Hong
Kong}, and the turnout --- seven in 10 eligible voters --- suggested
that the public continues to back the democracy movement, even as the
protests grow increasingly violent. Young Hong Kongers, a major force
behind the demonstrations of the past six months, played a leading role
in the voting surge.

With three million voters casting ballots, pro-democracy candidates
captured 389 of 452 elected seats, up from only 124 and far more than
they have ever won. The government's allies held just 58 seats, a
remarkable collapse from 300.

\href{https://www.nytimes3xbfgragh.onion/interactive/2019/11/24/world/asia/hong-kong-election-results.html}{}

\includegraphics{https://static01.graylady3jvrrxbe.onion/images/2019/11/24/us/hong-kong-election-map-promo-1574645538143/hong-kong-election-map-promo-1574645538143-articleLarge-v5.png}

\hypertarget{hong-kong-election-results-mapped}{%
\subsection{Hong Kong Election Results
Mapped}\label{hong-kong-election-results-mapped}}

More than half of the 452 seats in Sunday's local district council
elections flipped from pro-Beijing to pro-democracy candidates.

To many democracy advocates, Sunday was a turning point.

``There has been a very deep awakening of the
\href{https://www.nytimes3xbfgragh.onion/2019/11/25/world/asia/hong-kong-election-winners.html}{Hong
Kong} people,'' said Alan Leong, chairman of the Civic Party, one of the
largest pro-democracy parties.

The elections were for district councils, one of the lowest elected
offices in
\href{https://www.nytimes3xbfgragh.onion/2019/11/25/world/asia/hong-kong-election-winners.html}{Hong
Kong}, and they are typically a subdued affair focused on community
issues. The job mostly entails pushing for neighborhood needs like bus
stops and traffic lights.

But this
\href{https://www.nytimes3xbfgragh.onion/2019/11/25/world/asia/hong-kong-election-winners.html}{election}
took on outsize significance, and was
\href{https://www.nytimes3xbfgragh.onion/2019/11/23/world/asia/hong-kong-election-protests-district-council.html}{viewed
as a referendum on the unrest} that has created the city's worst
political crisis in decades. In a semiautonomous part of
\href{https://www.nytimes3xbfgragh.onion/2019/11/25/world/asia/hong-kong-election-winners.html}{China}
where greater democracy is one of the protesters' biggest demands, it
gave residents a rare chance to vote.

The gains at the ballot box are likely to embolden a democracy movement
that has struggled with how to balance peaceful and violent protests to
achieve its goals.

They are also likely to
\href{https://www.nytimes3xbfgragh.onion/2019/11/25/world/asia/hong-kong-election-protests.html?action=click\&module=Top\%20Stories\&pgtype=Homepage}{deepen
the challenges
for}\href{https://www.nytimes3xbfgragh.onion/2019/11/25/world/asia/hong-kong-election-protests.html?action=click\&module=Top\%20Stories\&pgtype=Homepage}{China's
central government}, which wants to curb the unrest in Hong Kong. And
they might exacerbate Beijing's fears about giving the city's residents
even greater say in choosing their government.

\includegraphics{https://static01.graylady3jvrrxbe.onion/images/2019/11/24/world/24hk-pro-demoncracy-hfo/merlin_164927778_32f56a05-30db-4123-adf2-76965c9877b5-articleLarge.jpg?quality=75\&auto=webp\&disable=upscale}

The district councils are among the most democratic bodies in
\href{https://www.nytimes3xbfgragh.onion/2019/11/25/world/asia/hong-kong-election-winners.html}{Hong
Kong}. Almost all the seats are directly elected, unlike the
legislature, where the proportion is just over half. The territory's
chief executive is also not chosen directly by voters, but is instead
selected by a committee stacked in favor of Beijing.

The election results will give democracy forces considerably more
influence on that committee, which is scheduled to choose a new chief
executive in 2022.

The district councils name about a tenth of the group's 1,200 members,
and now all of these will flip from pro-Beijing to pro-democracy seats.
Democracy advocates already control about a quarter of the seats, while
other previously pro-Beijing sectors of the committee are now starting
to lean toward democracy, most notably accountants and real estate
lawyers.

Mr. Leong, the Civic Party chairman, called on the Chinese Communist
Party to change its policies in Hong Kong.

``Unless the C.C.P. is doing something concrete to address the concerns
of the Hong Kong people,'' he said, ``I think this movement cannot
end.''

Regina Ip, a cabinet member and the leader of a pro-Beijing political
party, said she was surprised to see so many young voters, many of whom
tried to confront her with the protesters' demands.

``Normally,'' she said, ``the young people do not come out to vote. But
this time, the opposition managed to turn them out.''

Ahead of the election, the city's leadership was concerned that the vote
would be marred by the chaos of recent months. Some of the most violent
clashes yet between protesters and the police took place last week,
\href{https://www.nytimes3xbfgragh.onion/interactive/2019/11/18/world/asia/hong-kong-protest-universities.html}{turning
two university campuses into battlegrounds}.

But the city remained relatively calm on Sunday as voters turned out in
droves. Long lines formed at polling centers in the morning, snaking
around skyscrapers and past small shops. Riot police officers were
deployed near polling stations on Sunday.

Image

Campaign volunteers lined a street in Lok Fu on Sunday.Credit...Lam Yik
Fei for The New York Times

David Lee, a retired printer approaching his 90th birthday, was among
the earliest voters on Hong Kong Island and said he had come because he
wanted democracy.

``This is important,'' he said.

Some analysts had predicted that pro-democracy candidates would have
difficulty making big gains. **** Pro-Beijing candidates are much better
financed, and the district races have traditionally been won on purely
local issues, not big questions like democracy, said Joseph Cheng, a
retired professor at City University of Hong Kong.

But voter turnout soared to 71 percent, far surpassing expectations.
Typically in district council elections, it is little more than 40
percent. Four years ago, after the 2014 Umbrella Movement increased
public interest in politics, turnout climbed to 47 percent. This year,
the number of registered voters hit a record.

On Sunday, several prominent pro-Beijing politicians lost their races,
among them Michael Tien, a longtime establishment lawmaker. After his
defeat, he said the increase in young voters signaled that they were
becoming more politically engaged, adding that the government should
listen to them.

In the district of Tuen Mun, about a hundred people celebrated with
cheers and champagne the defeat of
\href{https://www.nytimes3xbfgragh.onion/2019/11/05/world/asia/junius-ho-stabbed-hong-kong.html}{Junius
Ho, a controversial lawmaker} many protesters accused of supporting mob
attacks against them.

The victory on Sunday eclipsed the pro-democracy camp's last big win in
these elections, when they won 198 seats, still short of a majority,
following huge protests in 2003. Those demonstrations led the government
to scrap a national security bill requested by Beijing that critics said
would have endangered civil liberties in Hong Kong.

The government's allies dominated the elections that followed, though.
Beijing began investing heavily in grass-roots mobilization efforts,
including busing large numbers of older Hong Kong citizens from
retirement homes in mainland China to polling places in Hong Kong.

Image

Polling officials counting votes on Sunday.Credit...Lam Yik Fei for The
New York Times

Instead of just focusing on local issues, many pro-democracy candidates
ran on the broad themes of the protest movement, especially anger at
police brutality, and the intensity of the demonstrations sometimes
spilled into the race. Candidates on both sides were attacked while
campaigning.

Mandy Lee, 53, a homemaker who voted at the Kowloon Bay neighborhood,
showed up to vote for the pro-Beijing establishment and criticized the
protests.

``It's not that I have no sympathy toward young people, but I strongly
believe that their efforts are futile,'' she said. ``We are a tiny
island; it's only a matter of time before China takes us over and
integrates us.''

The outcome of the election could further complicate the position of
\href{https://www.nytimes3xbfgragh.onion/2019/09/03/world/asia/hong-kong-protests-carrie-lam.html}{Carrie
Lam, Hong Kong's embattled chief executive}. Critics say that she has
failed to engage with the community over the protests and many have
demanded she step down.

On Monday, Mrs. Lam said in a statement that the government respected
the results of the election. ``Many have pointed out that the results
reflect the public's dissatisfaction with the social situation and
deep-seated problems,'' she said, adding that the government would
``listen to the views of the public with an open mind and seriously
reflect on them.''

Image

Carrie Lam, Hong Kong's chief executive, casting her vote on
Sunday.Credit...Lam Yik Fei for The New York Times

In June, Mrs. Lam set off enormous protests by pushing ahead with a bill
that would have allowed the extradition of Hong Kong residents to the
opaque judicial system in mainland China. The issue played to deeper
worries about Beijing's encroachment on Hong Kong, which has maintained
its own political and judicial system since the former British colony
was reclaimed by China in 1997.

Mrs. Lam
\href{https://www.nytimes3xbfgragh.onion/2019/09/04/world/asia/carrie-lam-hong-kong-protests.html}{withdrew
her proposal} after months of protests, but many said she acted too
late. The protesters are now demanding additional concessions, including
the introduction of universal suffrage and an independent inquiry into
police conduct.

The election results on Sunday will allow them to argue that the public
supports them. About 57 percent of voters cast ballots for pro-democracy
candidates, while nearly 40 percent voted for Beijing's allies. The
remaining 3 percent voted for independents, who won five seats.

Many pro-Beijing political parties receive large donations from the Hong
Kong subsidiaries of state-owned enterprises in mainland China, which
they use to organize picnics and other campaign events. But the results
on Sunday showed the limits of these efforts.

Reporting was contributed by K.K. Rebecca Lai in New York and Jin Wu and
Katherine Li in Hong Kong.

Advertisement

\protect\hyperlink{after-bottom}{Continue reading the main story}

\hypertarget{site-index}{%
\subsection{Site Index}\label{site-index}}

\hypertarget{site-information-navigation}{%
\subsection{Site Information
Navigation}\label{site-information-navigation}}

\begin{itemize}
\tightlist
\item
  \href{https://help.nytimes3xbfgragh.onion/hc/en-us/articles/115014792127-Copyright-notice}{©~2020~The
  New York Times Company}
\end{itemize}

\begin{itemize}
\tightlist
\item
  \href{https://www.nytco.com/}{NYTCo}
\item
  \href{https://help.nytimes3xbfgragh.onion/hc/en-us/articles/115015385887-Contact-Us}{Contact
  Us}
\item
  \href{https://www.nytco.com/careers/}{Work with us}
\item
  \href{https://nytmediakit.com/}{Advertise}
\item
  \href{http://www.tbrandstudio.com/}{T Brand Studio}
\item
  \href{https://www.nytimes3xbfgragh.onion/privacy/cookie-policy\#how-do-i-manage-trackers}{Your
  Ad Choices}
\item
  \href{https://www.nytimes3xbfgragh.onion/privacy}{Privacy}
\item
  \href{https://help.nytimes3xbfgragh.onion/hc/en-us/articles/115014893428-Terms-of-service}{Terms
  of Service}
\item
  \href{https://help.nytimes3xbfgragh.onion/hc/en-us/articles/115014893968-Terms-of-sale}{Terms
  of Sale}
\item
  \href{https://spiderbites.nytimes3xbfgragh.onion}{Site Map}
\item
  \href{https://help.nytimes3xbfgragh.onion/hc/en-us}{Help}
\item
  \href{https://www.nytimes3xbfgragh.onion/subscription?campaignId=37WXW}{Subscriptions}
\end{itemize}
