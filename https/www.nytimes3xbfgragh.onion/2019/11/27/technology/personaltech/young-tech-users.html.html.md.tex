Sections

SEARCH

\protect\hyperlink{site-content}{Skip to
content}\protect\hyperlink{site-index}{Skip to site index}

\href{https://www.nytimes3xbfgragh.onion/section/technology/personaltech}{Personal
Tech}

\href{https://myaccount.nytimes3xbfgragh.onion/auth/login?response_type=cookie\&client_id=vi}{}

\href{https://www.nytimes3xbfgragh.onion/section/todayspaper}{Today's
Paper}

\href{/section/technology/personaltech}{Personal Tech}\textbar{}A Big
Screen to Sift Through Recruits

\url{https://nyti.ms/35BNzAG}

\begin{itemize}
\item
\item
\item
\item
\item
\end{itemize}

Advertisement

\protect\hyperlink{after-top}{Continue reading the main story}

Supported by

\protect\hyperlink{after-sponsor}{Continue reading the main story}

Tech We're Using

\hypertarget{a-big-screen-to-sift-through-recruits}{%
\section{A Big Screen to Sift Through
Recruits}\label{a-big-screen-to-sift-through-recruits}}

Many people joining the newsroom are digitally savvy and helping media
with a digital transition, says Theodore Kim, who runs fellowships and
internships.

\includegraphics{https://static01.graylady3jvrrxbe.onion/images/2019/11/27/business/27techusing1/27techusing1-articleLarge.jpg?quality=75\&auto=webp\&disable=upscale}

Featuring Theodore Kim

\begin{itemize}
\item
  Nov. 27, 2019
\item
  \begin{itemize}
  \item
  \item
  \item
  \item
  \item
  \end{itemize}
\end{itemize}

\emph{How do New York Times journalists use technology in their jobs and
in their personal lives? Theodore Kim, director of newsroom fellowships
and internships, discussed the tech he's using.}

\textbf{What are your most important tech tools for doing your work and
finding people to recruit?}

There's lots of interest in working at The Times, and there are periods
when I have to spend hours sifting through applications. Last year, our
newsroom fellowship program drew some 5,000 applicants.

So I power through them with a company-owned MacBook Pro at work and my
personal MacBook Pro at home. Also a necessity: a 27-inch, 4K LG monitor
on my work desk, which I use to spread out multiple job applications
onscreen.

The Google suite of apps is critical. I collaborate with some 70 editors
over the course of our yearlong fellowship program, and we deploy just
about every kind of shared document there is.

In addition, I'm a regular user of headphones during the workday. I pop
in Apple AirPods for mobile calls or to listen to podcasts. I also have
a pair of fancier Beats Studio 3 noise-canceling headphones that I put
on when I require absolute silence or if I want to lose myself in the
sounds of Miles Davis.

\textbf{What surprises you about how younger interns use tech? Anything
new you've never heard of?}

On the whole, I think young people use much of the same tech we all do.
But I often see them use it so much faster and smarter, certainly, than
I do.

I'm talking about simple things like mastering keyboard shortcuts, being
able to type with thumbs really fast or having the ability to sprint
through on-screen menus and dialogue boxes with no hesitation. Only
people born with digital devices in their hands have that kind of tech
familiarity.

\includegraphics{https://static01.graylady3jvrrxbe.onion/images/2019/11/27/business/27techusing2/merlin_164818095_cdcc602f-4af7-4db8-87ff-d2b342a2151a-articleLarge.jpg?quality=75\&auto=webp\&disable=upscale}

\textbf{You've had several roles at The Times, including helping with
our transition to digital. What stood out to you as the biggest
differences between how we used to publish and how we do things today?}

The biggest evolution is how stories move through the reporting and
editing process. That cycle used to be: Reporter reports and writes.
Editors edit. And voilà! It's published in print and online.

But now baked into that process is a layer of sophisticated digital
thinking: What story form should we use? What headline might best
capture the essence of my piece? Is my topic searchable? Do we even need
to do a written story? Let's do
\href{https://www.nytimes3xbfgragh.onion/column/the-daily}{a podcast}
instead.

There is now so much information in the world that all media companies,
even the biggest ones, are re-evaluating what they bring to the table.
It's about asking ourselves what our value proposition is. And for The
Times, our value comes from doing unique, high-impact journalism that is
told in ways that sync up with how people are actually consuming
content.

\textbf{You're a bit of a gear head. What are your favorite gadgets, and
what do you do with them?}

Oh, gosh. Where do I start?

I definitely fall into the category of ``early adopter.'' Our house is
an Apple Store in miniature. At home, we shuffle between two MacBooks
and an iPad Pro. I'm on my second Apple Watch, fifth iPad, seventh
iPhone and, I think, 10th Mac. Every drawer in our house has some kind
of Apple dongle in it. With all the money we've given to Apple, I'm
pretty sure we've paid for at least part of Tim Cook's private jet.
Surely one of the winglets.

Beyond the Apple stuff, the gadget I use most often is the
\href{https://www.nytimes3xbfgragh.onion/2019/08/01/technology/personaltech/amazon-kindle-oasis-review-e-reader.html}{new
Kindle Oasis, Amazon's 10th-generation e-reader}. For my money, it
matches printed paper in clarity and experience. I think I've owned the
last seven generations of Kindles. They are like a fine wine in reverse:
The new ones just get better and better.

We also have a smattering of Google products around our home: a Nest
thermostat and a Nest security system, a Google
\href{https://www.nytimes3xbfgragh.onion/2017/04/26/technology/personaltech/mesh-network-vs-router.html}{mesh
Wi-Fi network} and a Google Home speaker. Our daughter, 8, and son, 3,
have grown quite adept at asking the Google Home funny questions. Our
son is barely potty-trained but is already astute enough to declare with
the proper volume and inflection, ``Hey, Google, play `Kids Bop'!''

It's both adorable and, candidly, a little unnerving.

But our pride and joy from a home gadget perspective is our basement
home theater. My wife and I love movies, but we don't get out to the
theater as often as we used to because of the kids and all of our other
commitments. So we invested in a 65-inch Sony OLED television. It's
mesmerizing, better than all but the best movie theaters we've been to.

In addition, we installed a Sony audio system that projects
theater-quality surround sound, including above the listener, as well as
a 4K disc player and an Apple TV 4K. While streaming video is obviously
the future, there's still nothing like the jaw-dropping, sonic quality
of physical 4K discs. All of our A/V tech, meantime, is connected
together with AudioQuest Cinnamon cables. (Yes, good cables really do
make a difference in high-end video and audio, and I will die on this
hill!)

We all look forward to family movie nights. We're probably the only
family on the block that insists on watching ``Hotel Transylvania 3'' in
Dolby Cinema-quality picture and sound. Or maybe just Dad insists on it.

Image

Mr. Ted with Amelia Nierenberg, center, a reporter in the New York Times
fellowship program, and Jahaan Singh, a project manager for the
program.Credit...Haruka Sakaguchi for The New York Times

\textbf{You're also into digital photography. With the progress that
smartphones have made in camera tech over the last several years, do you
use a normal camera much these days?}

I have
\href{https://www.nytimes3xbfgragh.onion/2019/09/17/technology/personaltech/iphone-11-review.html}{the
new iPhone 11 Pro} (of course). To my eyes, it has the best camera of
any phone. That's useful to me as I take many, many photos. I attended a
family wedding recently and took only my iPhone, leaving my DSLR at
home. The iPhone yielded pretty good results, especially in capturing 4K
video.

Yet as advanced as it is, the new iPhone still doesn't beat even a
low-end DSLR camera, particularly with photos involving motion. I have a
full-frame Nikon DSLR camera that I often tote with me on trips or
special occasions. I've used Nikons for so long that they feel like an
extension of my arm.

I sometimes get looks when I take out my Nikon zoom lens, which is the
size of a small Chihuahua. But we're still a few years away from phone
cameras achieving the magazine-quality images of the best cameras. When
that day comes, and if my purchase history is any guide, I'll be first
in line to get one.

Advertisement

\protect\hyperlink{after-bottom}{Continue reading the main story}

\hypertarget{site-index}{%
\subsection{Site Index}\label{site-index}}

\hypertarget{site-information-navigation}{%
\subsection{Site Information
Navigation}\label{site-information-navigation}}

\begin{itemize}
\tightlist
\item
  \href{https://help.nytimes3xbfgragh.onion/hc/en-us/articles/115014792127-Copyright-notice}{©~2020~The
  New York Times Company}
\end{itemize}

\begin{itemize}
\tightlist
\item
  \href{https://www.nytco.com/}{NYTCo}
\item
  \href{https://help.nytimes3xbfgragh.onion/hc/en-us/articles/115015385887-Contact-Us}{Contact
  Us}
\item
  \href{https://www.nytco.com/careers/}{Work with us}
\item
  \href{https://nytmediakit.com/}{Advertise}
\item
  \href{http://www.tbrandstudio.com/}{T Brand Studio}
\item
  \href{https://www.nytimes3xbfgragh.onion/privacy/cookie-policy\#how-do-i-manage-trackers}{Your
  Ad Choices}
\item
  \href{https://www.nytimes3xbfgragh.onion/privacy}{Privacy}
\item
  \href{https://help.nytimes3xbfgragh.onion/hc/en-us/articles/115014893428-Terms-of-service}{Terms
  of Service}
\item
  \href{https://help.nytimes3xbfgragh.onion/hc/en-us/articles/115014893968-Terms-of-sale}{Terms
  of Sale}
\item
  \href{https://spiderbites.nytimes3xbfgragh.onion}{Site Map}
\item
  \href{https://help.nytimes3xbfgragh.onion/hc/en-us}{Help}
\item
  \href{https://www.nytimes3xbfgragh.onion/subscription?campaignId=37WXW}{Subscriptions}
\end{itemize}
