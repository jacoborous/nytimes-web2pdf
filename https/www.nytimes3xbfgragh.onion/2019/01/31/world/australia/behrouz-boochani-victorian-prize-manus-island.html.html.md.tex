Sections

SEARCH

\protect\hyperlink{site-content}{Skip to
content}\protect\hyperlink{site-index}{Skip to site index}

\href{https://www.nytimes3xbfgragh.onion/section/world/australia}{Australia}

\href{https://myaccount.nytimes3xbfgragh.onion/auth/login?response_type=cookie\&client_id=vi}{}

\href{https://www.nytimes3xbfgragh.onion/section/todayspaper}{Today's
Paper}

\href{/section/world/australia}{Australia}\textbar{}Book Written by
Detainee via WhatsApp Gets a Top Prize

\url{https://nyti.ms/2UzRrNb}

\begin{itemize}
\item
\item
\item
\item
\item
\item
\end{itemize}

Advertisement

\protect\hyperlink{after-top}{Continue reading the main story}

Supported by

\protect\hyperlink{after-sponsor}{Continue reading the main story}

\hypertarget{book-written-by-detainee-via-whatsapp-gets-a-top-prize}{%
\section{Book Written by Detainee via WhatsApp Gets a Top
Prize}\label{book-written-by-detainee-via-whatsapp-gets-a-top-prize}}

\includegraphics{https://static01.graylady3jvrrxbe.onion/images/2019/02/02/world/02oz-detainee-1/merlin_114968525_7b8614f2-1a12-4311-abe6-3774cf25377c-articleLarge.jpg?quality=75\&auto=webp\&disable=upscale}

By \href{https://www.nytimes3xbfgragh.onion/by/isabella-kwai}{Isabella
Kwai} and
\href{https://www.nytimes3xbfgragh.onion/by/livia-albeck-ripka}{Livia
Albeck-Ripka}

\begin{itemize}
\item
  Jan. 31, 2019
\item
  \begin{itemize}
  \item
  \item
  \item
  \item
  \item
  \item
  \end{itemize}
\end{itemize}

SYDNEY, Australia --- A stateless Kurdish-Iranian asylum-seeker detained
by the Australian government won the country's highest-paying literary
prize on Thursday. But he could not attend the festivities to accept the
award.

Behrouz Boochani, a writer, journalist and filmmaker who has been held
in offshore detention on Manus Island in Papua New Guinea for more than
five years, won the
\href{https://www.wheelercentre.com/news/behrouz-boochani-wins-the-2019-victorian-prize-for-literature}{2019
Victorian Prize for Literature} for his book, ``No Friend but the
Mountains.''

The prestigious award, selected from a short list of winners in other
categories, grants the winning author 125,000 Australian dollars (about
\$90,000) and counts the country's most prominent writers among its
recipients.

Mr. Boochani
\href{https://www.nytimes3xbfgragh.onion/2017/02/13/insider/manus-island-refugee-australia.html}{fled
Iran} after the police there arrested several of his journalist
colleagues and raided his office. After the Australian Navy intercepted
his boat as he was trying to reach the country, he was sent to Manus
Island in 2013.

Since then, he has written articles for numerous
\href{https://www.thesaturdaypaper.com.au/contributor/behrouz-boochani}{local}
and
\href{https://www.theguardian.com/profile/behrouz-boochani}{international
outlets}. His book, which recounts his experiences in detention, was
written over five years through WhatsApp texts in Farsi to his
translator, Omid Tofighian, who accepted the award in his stead on
Thursday night in Melbourne.

Reached by telephone on Friday, Mr. Boochani said the award felt
paradoxical.

``I am happy because it is a great achievement for me and all of the
refugees, and it is a victory against this system,'' he said. But the
suffering he has witnessed on Manus Island deeply saddened him.

``Hundreds of people have been separated from their families for
years,'' he said, adding, ``We are living in a place worse than a
prison.''

He wrote the book through WhatsApp because at the time, detention center
guards would search detainees' rooms and seize their phones. ``I was
worried that if they attacked my room they would take my property,'' he
said.

Under Australia's migration policy, asylum seekers who try to enter by
sea are barred from entering the country. Since 2013, more than 3,000
refugees and asylum seekers have been sent to Australia's offshore
detention centers on the Republic of Nauru and Manus Island.

The government has defended the policy as an effective deterrent against
smugglers, but global human rights activists have strongly condemned it.

In 2017, the Australian government
\href{https://www.nytimes3xbfgragh.onion/2017/11/02/world/australia/manus-island-refugees.html}{closed
the Manus Island center} and moved asylum seekers there to alternative
accommodations. Some refugees have since been
\href{https://www.nytimes3xbfgragh.onion/2018/01/23/world/australia/manus-refugees-trump.html}{resettled
in the United States}, but hundreds of people like Mr. Boochani remain
in limbo.

\emph{{[}\emph{\emph{\textbf{Read more:}} \emph{The New York Times
went}\href{https://www.nytimes3xbfgragh.onion/interactive/2017/11/18/world/australia/manus-island-australia-detainees.html}{\emph{inside
the detention camp on Manus Island}}}.{]}}

``It's ironic that a book writing about the harrowing reality of our
government's punitive offshore detention policy is celebrated by winning
such a prestigious award,'' said Jana Favero, the director of advocacy
and campaigns at the Asylum Seeker Resource Center.

While Mr. Boochani's work deserved the award, she said, ``it has come at
the cost of his freedom.''

Typically, only Australian citizens or permanent residents are eligible
for the award. But an exception was made in Mr. Boochani`s case because
judges considered his story an Australia story, said Michael Williams,
the director of the Wheeler Center, a literary institution that
administers the award on behalf of the state government.

``We canvassed the critical and broader literary reception of the book,
and we made our decision on that basis,'' Mr. Williams said. ``This is
an extraordinary literary work that is an indelible contribution to
Australian publishing and storytelling.''

Though Mr. Boochani could not accept his award in person at the ceremony
on Thursday night, the writer had a prerecorded message for the
attendees.

``I believe that literature has the potential to make change and
challenge structures of power,'' he said in the message. ``Literature
has the power to give us freedom.''

Advertisement

\protect\hyperlink{after-bottom}{Continue reading the main story}

\hypertarget{site-index}{%
\subsection{Site Index}\label{site-index}}

\hypertarget{site-information-navigation}{%
\subsection{Site Information
Navigation}\label{site-information-navigation}}

\begin{itemize}
\tightlist
\item
  \href{https://help.nytimes3xbfgragh.onion/hc/en-us/articles/115014792127-Copyright-notice}{©~2020~The
  New York Times Company}
\end{itemize}

\begin{itemize}
\tightlist
\item
  \href{https://www.nytco.com/}{NYTCo}
\item
  \href{https://help.nytimes3xbfgragh.onion/hc/en-us/articles/115015385887-Contact-Us}{Contact
  Us}
\item
  \href{https://www.nytco.com/careers/}{Work with us}
\item
  \href{https://nytmediakit.com/}{Advertise}
\item
  \href{http://www.tbrandstudio.com/}{T Brand Studio}
\item
  \href{https://www.nytimes3xbfgragh.onion/privacy/cookie-policy\#how-do-i-manage-trackers}{Your
  Ad Choices}
\item
  \href{https://www.nytimes3xbfgragh.onion/privacy}{Privacy}
\item
  \href{https://help.nytimes3xbfgragh.onion/hc/en-us/articles/115014893428-Terms-of-service}{Terms
  of Service}
\item
  \href{https://help.nytimes3xbfgragh.onion/hc/en-us/articles/115014893968-Terms-of-sale}{Terms
  of Sale}
\item
  \href{https://spiderbites.nytimes3xbfgragh.onion}{Site Map}
\item
  \href{https://help.nytimes3xbfgragh.onion/hc/en-us}{Help}
\item
  \href{https://www.nytimes3xbfgragh.onion/subscription?campaignId=37WXW}{Subscriptions}
\end{itemize}
