Sections

SEARCH

\protect\hyperlink{site-content}{Skip to
content}\protect\hyperlink{site-index}{Skip to site index}

\href{https://myaccount.nytimes3xbfgragh.onion/auth/login?response_type=cookie\&client_id=vi}{}

\href{https://www.nytimes3xbfgragh.onion/section/todayspaper}{Today's
Paper}

This Pioneering Artist Is on the Brink of Her First Big Retrospective,
at 98

\url{https://nyti.ms/2SbKDb4}

\begin{itemize}
\item
\item
\item
\item
\item
\end{itemize}

Advertisement

\protect\hyperlink{after-top}{Continue reading the main story}

Supported by

\protect\hyperlink{after-sponsor}{Continue reading the main story}

\hypertarget{this-pioneering-artist-is-on-the-brink-of-her-first-big-retrospective-at-98}{%
\section{This Pioneering Artist Is on the Brink of Her First Big
Retrospective, at
98}\label{this-pioneering-artist-is-on-the-brink-of-her-first-big-retrospective-at-98}}

A string of solo exhibitions will shine new light on the work of the
Venezuelan-born artist Luchita Hurtado.

\includegraphics{https://static01.graylady3jvrrxbe.onion/images/2019/01/29/t-magazine/29tmag-luchita-slide-EJNV/29tmag-luchita-slide-EJNV-articleLarge.jpg?quality=75\&auto=webp\&disable=upscale}

By \href{https://www.nytimes3xbfgragh.onion/by/anna-furman}{Anna Furman}

\begin{itemize}
\item
  Jan. 29, 2019
\item
  \begin{itemize}
  \item
  \item
  \item
  \item
  \item
  \end{itemize}
\end{itemize}

On a cloudless afternoon in October, I meet the artist Luchita Hurtado,
98, in her Santa Monica home studio --- a sand-colored three-story
building a 20-minute walk from the Pacific Ocean. Inside, her riotously
colorful paintings --- in which genderless figures transform into trees
--- animate the walls of her compact 145-square-foot studio,
interspersed with dried leaves and a framed rare butterfly. **** She
offers me a bowl of wrinkled red jujubes, then settles into a padded
**** arm chair in the middle of her tchotchke-filled **** living room
and regales me with stories. She recounts **** searching for Olmec
colossal heads from a two-seater plane above San Lorenzo Tenochtitlán;
camping at the Lascaux Cave in southern France before the site closed
permanently to the public in 1963; posing for Man Ray, and forging
friendships with Frida Kahlo, Isamu Noguchi and Leonora Carrington.

Last summer, Hurtado's lush paintings, rich with cosmic motifs and
geometric abstractions, captivated visitors of the Hammer Museum's
``\href{https://hammer.ucla.edu/exhibitions/2018/made-in-la-2018/}{Made
in L.A. 2018}'' show. Exhibited sporadically over the course of her
life, and almost exclusively in group shows, Hurtado has recently
experienced a rise to fame that has been thrilling to witness --- albeit
maddening in its lateness. Later this month,
\href{https://www.hauserwirth.com/hauser-wirth-exhibitions/23119-luchita-hurtado-dark-years}{Hauser
\& Wirth} will dedicate three floors of its gallery on New York's Upper
East Side to her **** charged figurative drawings from the '40s and
'50s; in May, the \href{https://www.serpentinegalleries.org/}{Serpentine
Sackler Gallery} in London will mount a **** solo exhibition that spans
seven decades of her work; and in 2020, the year she turns 100,
Hurtado's first international retrospective will debut at the
\href{http://museotamayo.org/}{Museo Tamayo} in Mexico City and then
travel to a series of art institutions in the United States.

\href{https://www.nytimes3xbfgragh.onion/slideshow/2019/01/29/t-magazine/in-the-studio-with-luchita-hurtado.html}{}

\hypertarget{in-the-studio-with-luchita-hurtado}{%
\subsection{In the Studio With Luchita
Hurtado}\label{in-the-studio-with-luchita-hurtado}}

6 Photos

View Slide Show ›

\includegraphics{https://static01.graylady3jvrrxbe.onion/images/2019/01/29/t-magazine/29tmag-luchita-slide-WZHT/29tmag-luchita-slide-WZHT-articleLarge.jpg?quality=75\&auto=webp\&disable=upscale}

Laure Joliet

\emph{{[}}\href{https://www.nytimes3xbfgragh.onion/newsletters/t-list?module=inline}{\emph{Sign
up here}} \emph{for the T List newsletter, a weekly roundup of what T
Magazine editors are noticing and coveting now.{]}}

In her **** expansive oil paintings, ink-based drawings, fabric collages
and patterned garments, Hurtado explores what she sees as the
interconnectedness of all beings. Her paintings from the '70s ---
sinuous bodies that morph into mountains, bare nipples that juxtapose
spiky leaves, bulbous fruits that echo curving belly shapes ---
represent women as sacred beings, powerful subjects of their own lives.
Hurtado also incorporated womb imagery into her work before the feminist
art movement made popular the same subject matter in the late '70s.

Born in the seaside town of Maiquetía, Venezuela, in 1920, Hurtado
migrated to New York at age 8. At the then-all-girls high school
Washington Irving, she studied fine art and developed a keen interest in
anti-fascist political movements. After graduating, she volunteered at
the Spanish-language newspaper La Prensa and met her first husband, a
Chilean journalist twice her age. When he abandoned her and their infant
son, she supported herself by creating imaginative installations for
Lord \& Taylor and fashion illustrations for Vogue --- at night, she
created totemic figure drawings with watercolor and crayon*.* **** (In
1946, at age 26, she met and married the Austrian-Mexican painter
Wolfgang Paalen, moving with her two sons to join him in Mexico City.)

``Luchita has always had this very fluid identity, which makes her art
so 21st century,'' says the curator Hans Ulrich Obrist, who is
organizing her retrospective in London. ``We have to contextualize her
clearly with the historic avant-garde, because she is a contemporary of
Frida Kahlo, she knew Diego Rivera and was married to Wolfgang Paalen, a
key figure of surrealism --- and she is a key figure of spiritual
surrealism, with a connection to pre-Columbian art, but we cannot lock
her in that.'' Hurtado's work blurs the lines between micro- and
macroscopic worlds; she was at the forefront of not just spiritual
surrealism, but also the environmental and feminist art movements. As
Obrist puts it, ``she navigated a century of different contexts and
played an important role in all of those.''

\includegraphics{https://static01.graylady3jvrrxbe.onion/images/2019/01/29/t-magazine/29tmag-luchita-slide-PNHJ/29tmag-luchita-slide-PNHJ-articleLarge.jpg?quality=75\&auto=webp\&disable=upscale}

Hurtado possesses the grace of someone who has not spent her **** life
promoting her art, but quietly and diligently producing it --- at her
kitchen table, in backyards and closets and, at one point, in a
stand-alone studio in the Santa Monica Canyon (by 1951, she had
relocated to Los Angeles). ``I never stopped drawing, looking, living,''
she tells me. ``It's all the same thing, all solving your own life.''
Yet in the '70s --- when she was producing pioneering **** fabric
collages punctuated with words including ``you'' and ``womb'' --- she
wrote to Noguchi to request professional favors for her third husband,
the artist Lee Mullican, but rarely, if ever, did she tell him about her
own practice. When asked why she didn't openly share her paintings with
artist friends, she says, ``I always felt shy of it. I didn't feel
comfortable with people looking at my work.'' She adds, too, that
``there was a time when women really didn't show their work.''

Later in the afternoon, we zip across town to a nondescript brick
warehouse in Los Angeles's West Adams neighborhood --- the same building
where, nearly four years ago, her studio director, Ryan Good, ****
stumbled on nearly 1,200 works **** that were undated, many signed with
the initials ``LH.'' While family and close friends were aware that
Hurtado painted, her cross-disciplinary practice, distinct visual
vernacular and prolific output remained largely unknown. ``We didn't
know the extent of it,'' says Good, while leafing through a photo album.
``I knew that Luchita had made some paintings, but it was a different
thing to look at her entire career.'' He continues, ``we know Isamu
Noguchi and Sam Francis have jewelry she made, that Agnes Martin has
clothes Luchita made and Gordon Onslow-Ford has some things, but it
doesn't seem like she gave other work away to the prominent artists she
knew.''

Image

In the early '60s, Hurtado drew enigmatic, muscular figures in vibrant
ink washes. This untitled work from 1960 will be exhibited in May at the
Serpentine alongside other anatomical drawings.Credit...Jeff McLane.
Image courtesy of the artist and Hauser \& Wirth.

In preparation for her upcoming shows, Hurtado's studio registrar, Cole
Root, has been sifting through photographs --- self-portraits, family
travel shots, abstract shadow studies --- looking for clues about
paintings that might have been sold or given to friends. ``When I
started there was no archive whatsoever,'' says Root. ``We've gone from
a casual pace to moving at the speed of light.''

Part of the challenge in organizing Hurtado's archive is that she moved
frequently throughout her life --- early works from New York City and
Mexico City, for instance, are mostly lost --- and she doesn't remember
many of the pieces herself. ``Some things survived, some things
didn't,'' says Hurtado, ``I've gotten use to loss.'' Also, few of her
works have ever been publicly exhibited. ``Women artists have not had
the visibility they should have and we need to protest, systematically,
against forgetting --- through books and exhibitions,'' Obrist says. The
exhibition at the Serpentine will be **** animated by what he calls
``decisive moments or epiphanies'' throughout Hurtado's life.

``I remember my childhood more and more,'' Hurtado tells me, tucking a
tortoiseshell comb into her hair, which she had cut short herself the
day before. She shares memories from Venezuela --- hiding under
fan-shaped leaves, watching crabs scuttle across the beach, devouring
mangoes in a cool stream. Lately, when she wakes, she sees a vision of a
pink ceiling floating above her. I imagine the series of paintings she
created in 1975 in which bright-white squares are framed by mesmerizing
planes of blue, goldenrod and fiery red --- intended to draw moths to an
illusory light, they give off a sense of ascension and expansion. ``I've
concluded that I'm going somewhere,'' she tells me. ``It's not death;
it's a border that we cross. I don't think I'll be able to come back and
tell you, but if I can, I'll find a way. If you suddenly see a pink
ceiling, that's me.''

``Luchita Hurtado. Dark Years'' is on view from Jan. 31 -- April 6,
2019, at Hauser \& Wirth, 32 East 69th Street, New York,
\href{https://www.hauserwirth.com/}{hauserwirth.com}.

Advertisement

\protect\hyperlink{after-bottom}{Continue reading the main story}

\hypertarget{site-index}{%
\subsection{Site Index}\label{site-index}}

\hypertarget{site-information-navigation}{%
\subsection{Site Information
Navigation}\label{site-information-navigation}}

\begin{itemize}
\tightlist
\item
  \href{https://help.nytimes3xbfgragh.onion/hc/en-us/articles/115014792127-Copyright-notice}{©~2020~The
  New York Times Company}
\end{itemize}

\begin{itemize}
\tightlist
\item
  \href{https://www.nytco.com/}{NYTCo}
\item
  \href{https://help.nytimes3xbfgragh.onion/hc/en-us/articles/115015385887-Contact-Us}{Contact
  Us}
\item
  \href{https://www.nytco.com/careers/}{Work with us}
\item
  \href{https://nytmediakit.com/}{Advertise}
\item
  \href{http://www.tbrandstudio.com/}{T Brand Studio}
\item
  \href{https://www.nytimes3xbfgragh.onion/privacy/cookie-policy\#how-do-i-manage-trackers}{Your
  Ad Choices}
\item
  \href{https://www.nytimes3xbfgragh.onion/privacy}{Privacy}
\item
  \href{https://help.nytimes3xbfgragh.onion/hc/en-us/articles/115014893428-Terms-of-service}{Terms
  of Service}
\item
  \href{https://help.nytimes3xbfgragh.onion/hc/en-us/articles/115014893968-Terms-of-sale}{Terms
  of Sale}
\item
  \href{https://spiderbites.nytimes3xbfgragh.onion}{Site Map}
\item
  \href{https://help.nytimes3xbfgragh.onion/hc/en-us}{Help}
\item
  \href{https://www.nytimes3xbfgragh.onion/subscription?campaignId=37WXW}{Subscriptions}
\end{itemize}
