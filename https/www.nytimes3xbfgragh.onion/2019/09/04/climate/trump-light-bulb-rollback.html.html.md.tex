Sections

SEARCH

\protect\hyperlink{site-content}{Skip to
content}\protect\hyperlink{site-index}{Skip to site index}

\href{https://www.nytimes3xbfgragh.onion/section/climate}{Climate}

\href{https://myaccount.nytimes3xbfgragh.onion/auth/login?response_type=cookie\&client_id=vi}{}

\href{https://www.nytimes3xbfgragh.onion/section/todayspaper}{Today's
Paper}

\href{/section/climate}{Climate}\textbar{}White House to Relax Energy
Efficiency Rules for Light Bulbs

\url{https://nyti.ms/2NVpIqd}

\begin{itemize}
\item
\item
\item
\item
\item
\item
\end{itemize}

\hypertarget{climate-and-environment}{%
\subsubsection{\texorpdfstring{\href{https://www.nytimes3xbfgragh.onion/section/climate?name=styln-climate\&region=TOP_BANNER\&variant=undefined\&block=storyline_menu_recirc\&action=click\&pgtype=Article\&impression_id=15513250-e3b5-11ea-8546-498a940a9789}{Climate
and
Environment}}{Climate and Environment}}\label{climate-and-environment}}

\begin{itemize}
\tightlist
\item
  \href{https://www.nytimes3xbfgragh.onion/2020/08/17/climate/alaska-oil-drilling-anwr.html?name=styln-climate\&region=TOP_BANNER\&variant=undefined\&block=storyline_menu_recirc\&action=click\&pgtype=Article\&impression_id=15513251-e3b5-11ea-8546-498a940a9789}{Arctic
  Refuge}
\item
  \href{https://www.nytimes3xbfgragh.onion/interactive/2020/climate/trump-environment-rollbacks.html?name=styln-climate\&region=TOP_BANNER\&variant=undefined\&block=storyline_menu_recirc\&action=click\&pgtype=Article\&impression_id=15513252-e3b5-11ea-8546-498a940a9789}{Trump's
  Changes}
\item
  \href{https://www.nytimes3xbfgragh.onion/interactive/2020/04/19/climate/climate-crash-course-1.html?name=styln-climate\&region=TOP_BANNER\&variant=undefined\&block=storyline_menu_recirc\&action=click\&pgtype=Article\&impression_id=15515960-e3b5-11ea-8546-498a940a9789}{Climate
  101}
\item
  \href{https://www.nytimes3xbfgragh.onion/interactive/2018/08/30/climate/how-much-hotter-is-your-hometown.html?name=styln-climate\&region=TOP_BANNER\&variant=undefined\&block=storyline_menu_recirc\&action=click\&pgtype=Article\&impression_id=15515961-e3b5-11ea-8546-498a940a9789}{Is
  Your Hometown Hotter?}
\end{itemize}

Advertisement

\protect\hyperlink{after-top}{Continue reading the main story}

Supported by

\protect\hyperlink{after-sponsor}{Continue reading the main story}

\hypertarget{white-house-to-relax-energy-efficiency-rules-for-light-bulbs}{%
\section{White House to Relax Energy Efficiency Rules for Light
Bulbs}\label{white-house-to-relax-energy-efficiency-rules-for-light-bulbs}}

\includegraphics{https://static01.graylady3jvrrxbe.onion/images/2019/09/03/climate/00CLI-LIGHTBULBS1b/merlin_160134876_ee9f8912-e321-4c94-9042-4fe4c642bca5-articleLarge.jpg?quality=75\&auto=webp\&disable=upscale}

\href{https://www.nytimes3xbfgragh.onion/by/john-schwartz}{\includegraphics{https://static01.graylady3jvrrxbe.onion/images/2018/02/16/multimedia/author-john-schwartz/author-john-schwartz-thumbLarge.jpg}}

By \href{https://www.nytimes3xbfgragh.onion/by/john-schwartz}{John
Schwartz}

\begin{itemize}
\item
  Published Sept. 4, 2019Updated Sept. 6, 2019
\item
  \begin{itemize}
  \item
  \item
  \item
  \item
  \item
  \item
  \end{itemize}
\end{itemize}

The Trump administration plans to significantly weaken federal rules
that would have forced Americans to use much more energy-efficient light
bulbs, a move that could contribute to greenhouse gas emissions that
cause global warming.

The proposed changes would eliminate requirements that effectively meant
that most light bulbs sold in the United States --- not only the
familiar, pear-shaped ones, but several other styles as well --- must be
either LEDs or fluorescent to meet new efficiency standards.

The rules being weakened, which dated from 2007 and the administration
of President George W. Bush and slated to start in the new year, would
have all but ended the era of the incandescent bulb invented more than a
century ago. Eliminating inefficient bulbs nationwide would save
electricity equivalent to the output of at least 25 large power plants,
enough to power all homes in New Jersey and Pennsylvania, according to
an estimate by the Natural Resources Defense Council.

The Trump administration said the changes would benefit consumers by
keeping prices low and eliminating government regulation.

``The Energy Department flat out got it wrong today,'' said Jason
Hartke, president of the Alliance to Save Energy, a nonprofit coalition
of business and environmental groups. Calling the move an ``unforced
error,'' he said, ``Wasting energy with inefficient light bulbs isn't
just costly for homes and businesses, it's terrible for our climate.''

\href{https://www.nytimes3xbfgragh.onion/section/climate?action=click\&pgtype=Article\&state=default\&region=MAIN_CONTENT_1\&context=storylines_keepup}{}

\hypertarget{climate-and-environment-}{%
\subsubsection{Climate and Environment
›}\label{climate-and-environment-}}

\hypertarget{keep-up-on-the-latest-climate-news}{%
\paragraph{Keep Up on the Latest Climate
News}\label{keep-up-on-the-latest-climate-news}}

Updated Aug. 18, 2020

Here's what you need to know this week:

\begin{itemize}
\item
  \begin{itemize}
  \tightlist
  \item
    Five automakers
    \href{https://www.nytimes3xbfgragh.onion/2020/08/17/climate/california-automakers-pollution.html?action=click\&pgtype=Article\&state=default\&region=MAIN_CONTENT_1\&context=storylines_keepup}{sealed
    a binding agreement} with California to follow the state's stricter
    tailpipe emissions rules.
  \item
    The Trump
    administration\href{https://www.nytimes3xbfgragh.onion/2020/08/13/climate/trump-methane.html?action=click\&pgtype=Article\&state=default\&region=MAIN_CONTENT_1\&context=storylines_keepup}{eliminated
    a major methane rule}, even as leaks are worsening, in a decision
    that researchers warned ignored science.
  \item
    Climate change leaders said
    \href{https://www.nytimes3xbfgragh.onion/2020/08/12/climate/kamala-harris-environmental-justice.html?action=click\&pgtype=Article\&state=default\&region=MAIN_CONTENT_1\&context=storylines_keepup}{the
    vice-presidential choice of Kamala Harris} signaled that Democrats
    will have a focus on environmental justice.
  \end{itemize}
\end{itemize}

The actions are the latest by the Trump administration to
\href{https://www.nytimes3xbfgragh.onion/2019/08/29/climate/climate-rule-trump-reversing.html}{weaken
a broad array of rules designed to fight climate change}. Last week it
announced a far-reaching plan to
\href{https://www.nytimes3xbfgragh.onion/2019/08/29/climate/epa-methane-greenhouse-gas.html}{cut
back on the regulation of emissions of methane}, a powerful greenhouse
gas. Earlier this year it
\href{https://www.nytimes3xbfgragh.onion/2018/08/02/climate/trump-auto-emissions-california.html?module=inline}{proposed
freezing antipollution and fuel-efficiency standards for cars}, and
\href{https://www.nytimes3xbfgragh.onion/2019/06/19/climate/epa-coal-emissions.html?module=inline}{tried
to replace the Clean Power Plan}, a signature emissions-reduction
measure of the Obama administration.

\href{https://www.nytimes3xbfgragh.onion/interactive/2019/climate/winter-cold-weather.html}{President
Trump has repeatedly dismissed the scientific consensus} that climate
change is caused by human activity and requires urgent action to avoid
its most dire effects, even as
\href{https://www.nytimes3xbfgragh.onion/2018/11/23/climate/us-climate-report.html}{government
scientists have warned about the damage}that global warming is already
causing the United States' economy.

Shaylyn Hynes, a spokeswoman for the Department of Energy, said the 2007
law requires the department to issue standards ``only when doing so
would be economically justified. These standards are not.'' She added
that the administration's action ``will ensure that the choice of how to
light homes and businesses is left to the American people, not the
federal government.''

The trade association for companies that make light bulbs applauded the
Energy Department's decision. In a statement, the National Electrical
Manufacturers Association said Americans are already buying the more
efficient bulbs and the final rule ``will not impact the market's
continuing, rapid adoption of energy-saving lighting.''

The group estimates that by the end of 2019, as much as 84 percent of
``general purpose'' light sockets will be filled by LED and compact
fluorescent bulbs.

Rapid technological change in the lowly light bulb has been
\href{https://www.nytimes3xbfgragh.onion/interactive/2019/03/08/climate/light-bulb-efficiency.html}{one
of the largely unsung success stories} in the fight to reduce energy use
and greenhouse gas emissions.

\href{https://energyathaas.wordpress.com/2017/05/08/evidence-of-a-decline-in-electricity-use-by-u-s-households/}{Energy
consumption in American homes had been on the rise for
decades}\href{https://energyathaas.wordpress.com/2017/05/08/evidence-of-a-decline-in-electricity-use-by-u-s-households/}{.
But that has reversed significantly in recent years}, thanks in part to
the growing acceptance of technologies like LED bulbs and compact
fluorescents. Since 2010, energy consumption in American homes has
dropped by 6 percent, according to Lucas Davis, an energy economist at
the Haas School of Business, which is part of the University of
California, Berkeley.

In 2007, Congress passed legislation to phase out inefficient
incandescent and halogen bulbs. As part of that process, the oldest
incandescent technology had already disappeared from standard
pear-shaped bulbs by 2014 in favor of ``halogen incandescents,'' which
look the same but use less power.

Around that time, some conservative lawmakers and commentators
\href{https://www.foxnews.com/politics/house-to-consider-bill-nixing-light-bulb-restrictions}{turned
the transition into a partisan dispute during the Obama administration},
warning that the Democratic administration would force people to buy
inferior bulbs. More recently, though, that notion of a partisan divide
has faded, Professor Davis said. ``LEDs are being sold in large volumes
in all 50 states,'' he said, not just blue states.

\emph{Want climate news in your inbox?}
\href{https://www.nytimes3xbfgragh.onion/newsletters/climate-change}{\emph{Sign
up here
for}}\textbf{\href{https://www.nytimes3xbfgragh.onion/newsletters/climate-change}{\emph{Climate
Fwd:}}}\emph{, our email newsletter.}

LED bulbs show how seemingly modest shifts in technology can have a
profound effect on people's lives and wallets.

Because of their long life and energy
efficiency,\href{https://www.energy.gov/energysaver/save-electricity-and-fuel/lighting-choices-save-you-money/how-energy-efficient-light}{an
LED bulb can save consumers}
\href{https://www.nrdc.org/experts/noah-horowitz/annual-cost-light-bulb-standards-rollback-12-billion}{an
estimated \$50 to \$100 over its several-year lifetime}, while reducing
the number of times a year they need to climb a stepladder or kitchen
table to replace burnt-out bulbs. LED bulbs, once many times more
expensive than incandescent bulbs, have plunged in price and can often
be found for
\href{https://www.walmart.com/ip/Great-Value-LED-Light-Bulb-8-5W-60W-Equivalent-A19-Lamp-E26-Medium-Base-Non-Dimmable-Soft-White-4-Pack/51497369}{less
than \$2}each.

Two rollbacks were unveiled on Wednesday.

One would eliminate new energy efficiency requirements for pear-shaped
bulbs that were supposed to take effect Jan. 1, 2020. The department
\href{https://s3.amazonaws.com/public-inspection.federalregister.gov/2019-18941.pdf?utm_source=federalregister.gov\&utm_medium=email\&utm_campaign=pi+subscription+mailing+list}{is
proposing a new rule} that would end that requirement, subject to a
60-day comment period.

A second rollback targets rules that, next year, would have required
adding several additional kinds of incandescent and halogen light bulbs
to the energy-efficient group: three-way bulbs; the candle-shaped bulbs
used in chandeliers; the globe-shaped bulbs found in bathroom lighting;
and reflector bulbs used in recessed fixtures and track lighting. Under
the Energy Department's proposed plan, those requirements will be
eliminated and sales of traditional incandescent bulbs for those
purposes can continue.

The changes are likely to be challenged in court.

California's attorney general, Xavier Becerra, said he would fight the
administration's action in court, calling the shift ``another dim-witted
move that will waste energy at the expense of our planet.''

Noah Horowitz, director of the Center for Energy Efficiency Standards at
the Natural Resources Defense Council, said, ``We will explore all
options, including litigation, to stop this completely misguided and
unlawful action.'' He said regulation remains necessary.
``Energy-wasting incandescents and halogens still make up more than a
third of new bulb sales,'' he said.

For more news on climate and the environment,
\href{https://twitter.com/nytclimate}{follow @NYTClimate on Twitter}.

Advertisement

\protect\hyperlink{after-bottom}{Continue reading the main story}

\hypertarget{site-index}{%
\subsection{Site Index}\label{site-index}}

\hypertarget{site-information-navigation}{%
\subsection{Site Information
Navigation}\label{site-information-navigation}}

\begin{itemize}
\tightlist
\item
  \href{https://help.nytimes3xbfgragh.onion/hc/en-us/articles/115014792127-Copyright-notice}{©~2020~The
  New York Times Company}
\end{itemize}

\begin{itemize}
\tightlist
\item
  \href{https://www.nytco.com/}{NYTCo}
\item
  \href{https://help.nytimes3xbfgragh.onion/hc/en-us/articles/115015385887-Contact-Us}{Contact
  Us}
\item
  \href{https://www.nytco.com/careers/}{Work with us}
\item
  \href{https://nytmediakit.com/}{Advertise}
\item
  \href{http://www.tbrandstudio.com/}{T Brand Studio}
\item
  \href{https://www.nytimes3xbfgragh.onion/privacy/cookie-policy\#how-do-i-manage-trackers}{Your
  Ad Choices}
\item
  \href{https://www.nytimes3xbfgragh.onion/privacy}{Privacy}
\item
  \href{https://help.nytimes3xbfgragh.onion/hc/en-us/articles/115014893428-Terms-of-service}{Terms
  of Service}
\item
  \href{https://help.nytimes3xbfgragh.onion/hc/en-us/articles/115014893968-Terms-of-sale}{Terms
  of Sale}
\item
  \href{https://spiderbites.nytimes3xbfgragh.onion}{Site Map}
\item
  \href{https://help.nytimes3xbfgragh.onion/hc/en-us}{Help}
\item
  \href{https://www.nytimes3xbfgragh.onion/subscription?campaignId=37WXW}{Subscriptions}
\end{itemize}
