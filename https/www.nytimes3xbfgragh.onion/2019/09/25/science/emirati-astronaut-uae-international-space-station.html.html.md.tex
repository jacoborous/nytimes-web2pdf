Sections

SEARCH

\protect\hyperlink{site-content}{Skip to
content}\protect\hyperlink{site-index}{Skip to site index}

\href{https://www.nytimes3xbfgragh.onion/section/science}{Science}

\href{https://myaccount.nytimes3xbfgragh.onion/auth/login?response_type=cookie\&client_id=vi}{}

\href{https://www.nytimes3xbfgragh.onion/section/todayspaper}{Today's
Paper}

\href{/section/science}{Science}\textbar{}Hazzaa al-Mansoori, First
U.A.E. Astronaut, Launches to Space Station

\url{https://nyti.ms/2ljpPQt}

\begin{itemize}
\item
\item
\item
\item
\item
\item
\end{itemize}

Advertisement

\protect\hyperlink{after-top}{Continue reading the main story}

Supported by

\protect\hyperlink{after-sponsor}{Continue reading the main story}

\hypertarget{hazzaa-al-mansoori-first-uae-astronaut-launches-to-space-station}{%
\section{Hazzaa al-Mansoori, First U.A.E. Astronaut, Launches to Space
Station}\label{hazzaa-al-mansoori-first-uae-astronaut-launches-to-space-station}}

The Persian Gulf country has an ambitious, budding space program.

\includegraphics{https://static01.graylady3jvrrxbe.onion/images/2019/09/25/science/25ASTRONAUT/25ASTRONAUT-articleLarge.jpg?quality=75\&auto=webp\&disable=upscale}

\href{https://www.nytimes3xbfgragh.onion/by/kenneth-chang}{\includegraphics{https://static01.graylady3jvrrxbe.onion/images/2018/02/16/multimedia/author-kenneth-chang/author-kenneth-chang-thumbLarge.jpg}}

By \href{https://www.nytimes3xbfgragh.onion/by/kenneth-chang}{Kenneth
Chang}

\begin{itemize}
\item
  Published Sept. 25, 2019Updated July 19, 2020
\item
  \begin{itemize}
  \item
  \item
  \item
  \item
  \item
  \item
  \end{itemize}
\end{itemize}

The
\href{https://www.nytimes3xbfgragh.onion/2020/07/14/science/mars-united-arab-emirates.html}{United
Arab Emirates} has sent its first astronaut to space. That is a step in
a budding, ambitious space program for an oil-rich country the size of
Maine along the southern side of the Persian Gulf. Next year, it plans
to send a robotic spacecraft to Mars, and its leaders talk of
\href{https://government.ae/en/more/uae-future/2030-2117}{colonizing the
red planet a century from now.}

Emirati officials hope that space will inspire and train a generation of
engineers and scientists who can help prepare the country for a post-oil
future.

Hazzaa al-Mansoori, a former Emirati F-16 pilot, launched for the
International Space Station in a Soyuz space capsule from a Russian
spaceport in Kazakhstan on Wednesday. Also aboard were Jessica Meir of
NASA and Oleg Skripochka of Russia. After a quick, six-hour trip, the
spacecraft docked with the station at 3:42 p.m. Eastern time.

\begin{quote}
The
\href{https://twitter.com/hashtag/Soyuz?src=hash\&ref_src=twsrc\%5Etfw}{\#Soyuz}
MS-15 spacecraft, carrying the
\href{https://twitter.com/hashtag/FirstEmiratiAstronaut?src=hash\&ref_src=twsrc\%5Etfw}{\#FirstEmiratiAstronaut},
\href{https://twitter.com/hashtag/HazzaaAlMansoori?src=hash\&ref_src=twsrc\%5Etfw}{\#HazzaaAlMansoori},
has successfully docked with the
\href{https://twitter.com/hashtag/InternationalSpaceStation?src=hash\&ref_src=twsrc\%5Etfw}{\#InternationalSpaceStation}.
After all the necessary checks are complete, the astronauts will enter
the
ISS\href{https://twitter.com/astro_hazzaa?ref_src=twsrc\%5Etfw}{@astro\_hazzaa}
\href{https://t.co/k6UVgiCJGS}{pic.twitter.com/k6UVgiCJGS}

--- MBR Space Centre (@MBRSpaceCentre)
\href{https://twitter.com/MBRSpaceCentre/status/1176946873531285504?ref_src=twsrc\%5Etfw}{September
25, 2019}
\end{quote}

``I will try to remember each second of the launch itself,'' Mr.
al-Mansoori said during a news conference this month. ``Because it will
be really very important for me to share it with everyone and my
country, the entire world and the Arab region.''

Hours before launching, Mr. al-Mansoori also tweeted about his journey.

The station will be crowded for the next eight days with nine occupants
before three of them, including Mr. al-Mansoori, head back to Earth on
Oct. 3.

\hypertarget{why-has-the-uae-sent-an-astronaut-to-space}{%
\subsection{Why has the U.A.E. sent an astronaut to
space?}\label{why-has-the-uae-sent-an-astronaut-to-space}}

During his time in orbit, Mr. al-Mansoori is to help perform a series of
experiments and conduct a tour of the space station in Arabic.

But his trip will also highlight new opportunities for countries looking
to enter the space race. The Emirates is not part of the consortium of
countries that participate in the International Space Station. Two years
ago, the nation did not have any astronauts, either.

In December 2017, Sheikh Mohammed bin Rashid al-Maktoum, the ruler of
Dubai, which is one of the seven sheikhdoms that make up the U.A.E.,
posted on Twitter the nation's plans to start a human spaceflight
program.

Without rockets or a spacecraft of its own, the Mohammed bin Rashid
Space Center in Dubai purchased a seat on the Soyuz from the Russian
space agency in the same way that wealthy space tourists have also
bought trips to the space station. That is why NASA refers to Mr.
al-Mansoori as a
\href{https://www.nasa.gov/image-feature/spaceflight-participant-hazzaa-ali-almansoori}{``spaceflight
participant''} and not as a professional astronaut.

The price has not been publicly revealed.

From more than 4,000 applicants who wanted to fill the Soyuz seat, the
space center selected two: Mr. al-Mansoori and his backup, Sultan
al-Neyadi.

Mr. al-Mansoori, 35, is a father of four.

The two headed to Russia for training, including outdoor survival skills
in case the return Soyuz capsule landed far off course. Mr. al-Mansoori
has posted on Twitter about his astronaut experiences, mostly in Arabic,
occasionally in English:

Some of the experiments that Mr. al-Mansoori will conduct are already
waiting for him on the space station. NanoRacks, a Houston-based
company, collaborated with the Mohammed bin Rashid Space Center on a
competition that selected 32 experiments from Emirati students studying
the effect of weightlessness on materials like sand, steel, corn oil,
cement and egg whites.

Additional Emirati experiments include one studying oil emulsification
in a weightless environment, as well as a second to germinate a palm
date seed native to the country.

\hypertarget{what-other-plans-does-the-uae-have-for-space}{%
\subsection{What other plans does the U.A.E. have for
space?}\label{what-other-plans-does-the-uae-have-for-space}}

NanoRacks announced last week that it will be opening an office in Abu
Dhabi, the largest emirate.

``They are serious about becoming a space-faring nation,'' Jeffrey
Manber, chief executive of NanoRacks, said. ``I also like the fact, to
be candid, that they comfortably work with Russia, they comfortably work
with China and they comfortably work with the United States and the
European Space Agency. I think that is a model for the future.''

Euroconsult, an international consulting firm specializing on space
markets, reported that the Emirates spent \$383 million on space last
year. That is much less than the nearly \$41 billion spent by the United
States or even the \$1.5 billion by India, but is more than Canada
spent.

Virgin Galactic signed a memorandum of understanding with the United
Arab Emirates space agency in March that aims to set up a spaceport in
the country.

Next year, the Emirates intends to launch its Mars mission, a spacecraft
called Hope. The probe, on top of a Japanese rocket, is to carry five
instruments that are to study the loss of hydrogen and oxygen gases from
the upper parts of the Martian atmosphere.

For Hope, the Emirates is working with three American universities: the
University of Colorado, Arizona State University and the University of
California, Berkeley.

\hypertarget{who-were-earlier-astronauts-from-the-arab-world}{%
\subsection{Who were earlier astronauts from the Arab
world?}\label{who-were-earlier-astronauts-from-the-arab-world}}

Sultan bin Salman Al Saud, a member of the royal family of Saudi Arabia,
was the first Arab and Muslim to go into space as a member of a NASA
space shuttle mission in 1985. He now leads the Saudi Space Agency.

Muhammed Ahmed Faris, a Syrian military pilot, flew to the Russian Mir
space station in 1987.

Advertisement

\protect\hyperlink{after-bottom}{Continue reading the main story}

\hypertarget{site-index}{%
\subsection{Site Index}\label{site-index}}

\hypertarget{site-information-navigation}{%
\subsection{Site Information
Navigation}\label{site-information-navigation}}

\begin{itemize}
\tightlist
\item
  \href{https://help.nytimes3xbfgragh.onion/hc/en-us/articles/115014792127-Copyright-notice}{©~2020~The
  New York Times Company}
\end{itemize}

\begin{itemize}
\tightlist
\item
  \href{https://www.nytco.com/}{NYTCo}
\item
  \href{https://help.nytimes3xbfgragh.onion/hc/en-us/articles/115015385887-Contact-Us}{Contact
  Us}
\item
  \href{https://www.nytco.com/careers/}{Work with us}
\item
  \href{https://nytmediakit.com/}{Advertise}
\item
  \href{http://www.tbrandstudio.com/}{T Brand Studio}
\item
  \href{https://www.nytimes3xbfgragh.onion/privacy/cookie-policy\#how-do-i-manage-trackers}{Your
  Ad Choices}
\item
  \href{https://www.nytimes3xbfgragh.onion/privacy}{Privacy}
\item
  \href{https://help.nytimes3xbfgragh.onion/hc/en-us/articles/115014893428-Terms-of-service}{Terms
  of Service}
\item
  \href{https://help.nytimes3xbfgragh.onion/hc/en-us/articles/115014893968-Terms-of-sale}{Terms
  of Sale}
\item
  \href{https://spiderbites.nytimes3xbfgragh.onion}{Site Map}
\item
  \href{https://help.nytimes3xbfgragh.onion/hc/en-us}{Help}
\item
  \href{https://www.nytimes3xbfgragh.onion/subscription?campaignId=37WXW}{Subscriptions}
\end{itemize}
