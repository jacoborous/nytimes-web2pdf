Sections

SEARCH

\protect\hyperlink{site-content}{Skip to
content}\protect\hyperlink{site-index}{Skip to site index}

\href{https://www.nytimes3xbfgragh.onion/section/food}{Food}

\href{https://myaccount.nytimes3xbfgragh.onion/auth/login?response_type=cookie\&client_id=vi}{}

\href{https://www.nytimes3xbfgragh.onion/section/todayspaper}{Today's
Paper}

\href{/section/food}{Food}\textbar{}A Brazilian Chef Plans Her
4,800-Mile Commute to New York

\url{https://nyti.ms/2MTbUfS}

\begin{itemize}
\item
\item
\item
\item
\item
\end{itemize}

Advertisement

\protect\hyperlink{after-top}{Continue reading the main story}

Supported by

\protect\hyperlink{after-sponsor}{Continue reading the main story}

The Restaurant Preview

\hypertarget{a-brazilian-chef-plans-her-4800-mile-commute-to-new-york}{%
\section{A Brazilian Chef Plans Her 4,800-Mile Commute to New
York}\label{a-brazilian-chef-plans-her-4800-mile-commute-to-new-york}}

Manoella Buffara, who won raves at Manu, will challenge stereotypes
about that country's cuisine at Ella.

\includegraphics{https://static01.graylady3jvrrxbe.onion/images/2019/09/04/dining/03Preview-Brazil1/03Preview-Brazil1-articleLarge.jpg?quality=75\&auto=webp\&disable=upscale}

\href{https://www.nytimes3xbfgragh.onion/by/florence-fabricant}{\includegraphics{https://static01.graylady3jvrrxbe.onion/images/2018/07/16/multimedia/author-florence-fabricant/author-florence-fabricant-thumbLarge.png}}

By
\href{https://www.nytimes3xbfgragh.onion/by/florence-fabricant}{Florence
Fabricant}

\begin{itemize}
\item
  Sept. 3, 2019
\item
  \begin{itemize}
  \item
  \item
  \item
  \item
  \item
  \end{itemize}
\end{itemize}

At first, Manoella Buffara thought the idea impossible and even ``a
little bit crazy'': commuting nearly 5,000 miles from southern Brazil to
open a restaurant in New York.

But after years of trying to persuade the
\href{https://www.youtube.com/watch?v=BC3T3IOYBhQ}{critically acclaimed
chef} to take the proposal seriously, the American entrepreneurs Michael
Satsky and Brian Gefter finally succeeded. Her new restaurant, Ella, is
on track to open in late November in Chelsea.

Ms. Buffara and the partners will not reproduce
\href{https://www.theworlds50best.com/discovery/Manu.html}{Manu}, her
20-seat tasting menu restaurant in Curitiba, southwest of São Paulo.
(The new place is much larger, seating 98.) But she plans to maintain
her Brazilian flair, showcasing vegetables and fruits alongside seafood
in colorful presentations. The hyperlocal and sustainable ingredients to
which she is devoted may not be the same (in Brazil, her sources are the
rain forest, nearby farms and the sea), but what's local to New York
will inform many of her ingredient choices.

``It's important to show that there's more to Brazilian food than
feijoada and churrasco,'' she said during a brief visit to New York last
month. She said she was encouraged in that cliché-defying direction by
Mauro Colagreco, the Argentine-born chef of
\href{https://www.mirazur.fr/}{Mirazur}, a restaurant on the French
Riviera with three Michelin stars.

\emph{{[}}\href{https://www.nytimes3xbfgragh.onion/2019/09/03/dining/fall-restaurant-preview-nyc.html}{\emph{Click
here to read more from our restaurant preview.}}\emph{{]}}

Among the items likely to show up on her à la carte menu are sea
urchins; oysters; grilled and smoked leeks with mussels; raw fish with
plantains; roasted cauliflower; fermented cassava; and oils she makes
made from various nuts.

\includegraphics{https://static01.graylady3jvrrxbe.onion/images/2019/09/04/dining/03Preview-Brazil2/merlin_159759087_5dc163a3-1271-4b31-9a49-7934987b1428-articleLarge.jpg?quality=75\&auto=webp\&disable=upscale}

Image

Leeks in a mussel sauce, with lardo.Credit...Gabriela Portilho for The
New York Times

Mr. Satsky said her cooking would bring a new vision to New York. ``No
one is doing food like hers, Brazilian casual fine-dining,'' he said.

Ms. Buffara, 34, has had a patchwork career. At 16, she was sent by her
parents to learn English at a school outside Seattle, where she found
odd jobs in restaurants. She later worked on a fishing boat and in a
cannery in Alaska, backpacked in Europe and finally went home to get an
undergraduate degree in journalism while attending hotel school.

Next came cooking school in Italy, kitchen jobs in top restaurants there
and two months at
\href{https://www.nytimes3xbfgragh.onion/2018/04/24/dining/noma-restaurant-copenhagen.html}{Noma},
in Copenhagen. She returned to Brazil to be a cook, then opened Manu in
2012.

Ella will be decorated by the Brazilian designer Marcio Kogan, with
``lots of wood,'' Ms. Buffara said, an open kitchen and a wood-fired
grill. She'll offer natural wines from South America and elsewhere, and
a playlist of bossa nova, samba and jazz.

A manager from Manu and the sous-chef, Lucas Correia, will relocate
here. And though Ms. Buffara has a husband, two young daughters, several
community projects involving farming and beekeeping, and a restaurant in
Brazil, she said she expected to spend 10 days each month in New York.

\textbf{Ella} 436 West 15th Street, November.

\emph{Follow} \href{https://twitter.com/nytfood}{\emph{NYT Food on
Twitter}} \emph{and}
\href{https://www.instagram.com/nytcooking/}{\emph{NYT Cooking on
Instagram}}\emph{,}
\href{https://www.facebookcorewwwi.onion/nytcooking/}{\emph{Facebook}}\emph{,}
\href{https://www.youtube.com/nytcooking}{\emph{YouTube}} \emph{and}
\href{https://www.pinterest.com/nytcooking/}{\emph{Pinterest}}\emph{.}
\href{https://www.nytimes3xbfgragh.onion/newsletters/cooking}{\emph{Get
regular updates from NYT Cooking, with recipe suggestions, cooking tips
and shopping advice}}\emph{.}

Advertisement

\protect\hyperlink{after-bottom}{Continue reading the main story}

\hypertarget{site-index}{%
\subsection{Site Index}\label{site-index}}

\hypertarget{site-information-navigation}{%
\subsection{Site Information
Navigation}\label{site-information-navigation}}

\begin{itemize}
\tightlist
\item
  \href{https://help.nytimes3xbfgragh.onion/hc/en-us/articles/115014792127-Copyright-notice}{©~2020~The
  New York Times Company}
\end{itemize}

\begin{itemize}
\tightlist
\item
  \href{https://www.nytco.com/}{NYTCo}
\item
  \href{https://help.nytimes3xbfgragh.onion/hc/en-us/articles/115015385887-Contact-Us}{Contact
  Us}
\item
  \href{https://www.nytco.com/careers/}{Work with us}
\item
  \href{https://nytmediakit.com/}{Advertise}
\item
  \href{http://www.tbrandstudio.com/}{T Brand Studio}
\item
  \href{https://www.nytimes3xbfgragh.onion/privacy/cookie-policy\#how-do-i-manage-trackers}{Your
  Ad Choices}
\item
  \href{https://www.nytimes3xbfgragh.onion/privacy}{Privacy}
\item
  \href{https://help.nytimes3xbfgragh.onion/hc/en-us/articles/115014893428-Terms-of-service}{Terms
  of Service}
\item
  \href{https://help.nytimes3xbfgragh.onion/hc/en-us/articles/115014893968-Terms-of-sale}{Terms
  of Sale}
\item
  \href{https://spiderbites.nytimes3xbfgragh.onion}{Site Map}
\item
  \href{https://help.nytimes3xbfgragh.onion/hc/en-us}{Help}
\item
  \href{https://www.nytimes3xbfgragh.onion/subscription?campaignId=37WXW}{Subscriptions}
\end{itemize}
