Sections

SEARCH

\protect\hyperlink{site-content}{Skip to
content}\protect\hyperlink{site-index}{Skip to site index}

\href{https://www.nytimes3xbfgragh.onion/section/sports/rugby}{Rugby}

\href{https://myaccount.nytimes3xbfgragh.onion/auth/login?response_type=cookie\&client_id=vi}{}

\href{https://www.nytimes3xbfgragh.onion/section/todayspaper}{Today's
Paper}

\href{/section/sports/rugby}{Rugby}\textbar{}Chester Williams, Rugby
Champion Who Signaled Apartheid's Defeat, Dies at 49

\url{https://nyti.ms/2I4CPBq}

\begin{itemize}
\item
\item
\item
\item
\item
\end{itemize}

Advertisement

\protect\hyperlink{after-top}{Continue reading the main story}

Supported by

\protect\hyperlink{after-sponsor}{Continue reading the main story}

\hypertarget{chester-williams-rugby-champion-who-signaled-apartheids-defeat-dies-at-49}{%
\section{Chester Williams, Rugby Champion Who Signaled Apartheid's
Defeat, Dies at
49}\label{chester-williams-rugby-champion-who-signaled-apartheids-defeat-dies-at-49}}

Williams was the only nonwhite on the Springboks when it won the 1995
World Cup and was embraced by Nelson Mandela in a message of
reconciliation.

\includegraphics{https://static01.graylady3jvrrxbe.onion/images/2019/09/13/obituaries/10Williams1/merlin_160336356_8e0bdcbb-8432-439c-a79c-7dd85ad11c76-articleLarge.jpg?quality=75\&auto=webp\&disable=upscale}

By Huw Richards

\begin{itemize}
\item
  Sept. 12, 2019
\item
  \begin{itemize}
  \item
  \item
  \item
  \item
  \item
  \end{itemize}
\end{itemize}

Chester Williams, the only nonwhite player on the South Africa rugby
team that won the Rugby Union World Cup in 1995, a victory on home soil
that marked a historic moment of reconciliation just a year after
apartheid fell, died on Sept. 6 in Cape Town. He was 49.

\href{https://www.sarugby.co.za/}{South African Rugby}, the sport's
governing body in that country, said the cause was a heart attack.

Williams, who was classified as colored in South Africa's system of
racial divisions, was an important presence in the 1995 tournament
because his team, the Springboks, had historically been white, playing
the sport most cherished by the Afrikaners, the dominant political group
under apartheid.

More significantly,
\href{https://www.nytimes3xbfgragh.onion/2013/12/06/world/africa/nelson-mandela_obit.html}{Nelson
Mandela,} who had taken office as South Africa's first democratically
elected president a year earlier, had embraced the Springboks --- long a
symbol of repression to most nonwhites --- signaling that there was a
place for white South Africans in the new order.

Mandela attended the final match wearing a Springbok jersey with the No.
6, the number worn by the team's captain, Francois Pienaar.

Photographs of Mandela presenting the trophy to Pienaar after a 15-12
extra-time victory over the favored New Zealand, and of Williams smiling
among his white teammates, flew around the world as images of a
rehabilitated nation. The moment inspired the
\href{https://www.nytimes3xbfgragh.onion/2009/12/06/movies/06invictus.html}{Clint
Eastwood film ``Invictus''} (2009), on which Williams worked as an
adviser.

Williams, a powerful winger, was initially kept out of the tournament
because of injury, leaving South Africa with an all-white squad. But the
suspension of another wing, Pieter Hendricks, after an epic brawl during
a game against Canada, paved the way for a healed Williams to return.

Williams rose to the occasion. In the quarterfinal against Samoa, he
became the first South African to score four tries (rugby's touchdown)
in an international match. He also played a key defensive role in the
team's winning exhaustingly tight contests against France and New
Zealand.

In most any other society, Williams would have been recognized as a
member of sports royalty. He was born on Aug. 8, 1970, in Paarl, about
40 miles east of Cape Town, one of five children of Wilfred and Julene
Williams. Wilfred Williams played for the national team of the South
African Rugby Federation, a nonwhite apartheid-era league, from 1969 to
1981. Julene Williams's brother Adam Dombas was captain of the
federation team in 1971. And Wilfred's younger brother Avril became in
1984 the second nonwhite player to join the Springboks.

Chester Williams was selected for a nonsegregated
\href{http://www.wpschoolrugby.co.za/}{Western Province Schools Rugby}
team and played alongside Avril for the powerful Defence club while
working for the South African navy and then the army --- jobs that
involved a two-hour commute by train each way from his home in a
township. He would usually have to stand in segregated cars.

Thickset at 5-foot-9 and 180 pounds, he had the speed, skills and smarts
needed by a winger, a position whose best practitioners are both regular
scorers and strong defenders. After joining the Western Province rugby
union team in 1991, he became so popular that one administrator jokingly
suggested renaming its venerable Newlands Stadium in Cape Town
``Chesterfield.''

He made the Springboks squad in 1993, scoring a try in his first
international match, against Argentina in Buenos Aires, and became a
regular in 1994, scoring in four consecutive matches.

Williams had been quiet, undemonstrative and uncomplaining in his
playing days, so the rugby world was all the more surprised when, in a
2002 autobiography, he told of how he had experienced ugly moments. In
one instance a teammate used a racist slur and told him: ``Why do you
want to play our game? You know you can't play it.''

Williams's personal circumstances improved when Rugby Union was
professionalized after the 1995 World Cup, allowing him to be paid. Two
tries in a match against England that November took his career total to
13, putting him second on the Springboks all-time list.

But the rest of his career was clouded by persistent knee injuries and
disillusionment. When he was left off the 1999 World Cup team, he said,
he was told it was because Springboks had fulfilled its racial quotas,
demonstrating to him that a belief in white superiority remained
ingrained.

He retired from the South Africa team in 2000 with totals of 27 matches
and 14 tries in international matches.

His survivors include his wife, Maria; their twins, Matthew and Chloe;
and a stepson, Ryan.

\includegraphics{https://static01.graylady3jvrrxbe.onion/images/2019/09/10/obituaries/10Williams2/merlin_160345326_87b14ff3-7de2-4efa-bc44-ed98c546f118-articleLarge.jpg?quality=75\&auto=webp\&disable=upscale}

Williams later moved into coaching, most recently taking charge of the
Ugandan and Tunisian national teams. Since 2017 he had been director of
rugby at the University of the Western Cape, a historically nonwhite
institution near Cape Town.

In 2014 he established the Chester Williams Foundation, a children's
charity, and recently introduced Chester's IPA and Chester's Lager, the
only South African beers to be on sale at this year's World Cup, which
begins on Sept. 20 in Japan.

Williams is the fourth member of the storied 1995 team
\href{https://www.nytimes3xbfgragh.onion/2017/02/07/sports/rugby/joost-van-der-westhuizen-dead-south-africa-rugby.html}{to
die in recent years}. ``Not you too, Chester?'' President Cyril
Ramaphosa of South Africa wrote on Twitter. ``Your impact both on and
off the field in forging unity amongst all South Africans remains your
greatest legacy.''

Advertisement

\protect\hyperlink{after-bottom}{Continue reading the main story}

\hypertarget{site-index}{%
\subsection{Site Index}\label{site-index}}

\hypertarget{site-information-navigation}{%
\subsection{Site Information
Navigation}\label{site-information-navigation}}

\begin{itemize}
\tightlist
\item
  \href{https://help.nytimes3xbfgragh.onion/hc/en-us/articles/115014792127-Copyright-notice}{©~2020~The
  New York Times Company}
\end{itemize}

\begin{itemize}
\tightlist
\item
  \href{https://www.nytco.com/}{NYTCo}
\item
  \href{https://help.nytimes3xbfgragh.onion/hc/en-us/articles/115015385887-Contact-Us}{Contact
  Us}
\item
  \href{https://www.nytco.com/careers/}{Work with us}
\item
  \href{https://nytmediakit.com/}{Advertise}
\item
  \href{http://www.tbrandstudio.com/}{T Brand Studio}
\item
  \href{https://www.nytimes3xbfgragh.onion/privacy/cookie-policy\#how-do-i-manage-trackers}{Your
  Ad Choices}
\item
  \href{https://www.nytimes3xbfgragh.onion/privacy}{Privacy}
\item
  \href{https://help.nytimes3xbfgragh.onion/hc/en-us/articles/115014893428-Terms-of-service}{Terms
  of Service}
\item
  \href{https://help.nytimes3xbfgragh.onion/hc/en-us/articles/115014893968-Terms-of-sale}{Terms
  of Sale}
\item
  \href{https://spiderbites.nytimes3xbfgragh.onion}{Site Map}
\item
  \href{https://help.nytimes3xbfgragh.onion/hc/en-us}{Help}
\item
  \href{https://www.nytimes3xbfgragh.onion/subscription?campaignId=37WXW}{Subscriptions}
\end{itemize}
