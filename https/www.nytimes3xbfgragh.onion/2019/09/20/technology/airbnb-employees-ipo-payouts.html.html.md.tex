\href{/section/technology}{Technology}\textbar{}Inside Airbnb, Employees
Eager for Big Payouts Pushed It to Go Public

\url{https://nyti.ms/2M8R2i7}

\begin{itemize}
\item
\item
\item
\item
\item
\end{itemize}

\includegraphics{https://static01.graylady3jvrrxbe.onion/images/2019/09/20/business/20airbnb1/merlin_111284672_792a8a10-0c79-436f-ad3c-05323adeb805-articleLarge.jpg?quality=75\&auto=webp\&disable=upscale}

Sections

\protect\hyperlink{site-content}{Skip to
content}\protect\hyperlink{site-index}{Skip to site index}

\hypertarget{inside-airbnb-employees-eager-for-big-payouts-pushed-it-to-go-public}{%
\section{Inside Airbnb, Employees Eager for Big Payouts Pushed It to Go
Public}\label{inside-airbnb-employees-eager-for-big-payouts-pushed-it-to-go-public}}

Tension has grown among a 6,000-person work force as it waits to sell
company shares, people with knowledge of the situation said.

Airbnb headquarters in San Francisco, where employees have become
frustrated about not being able to cash in the stock they received as
part of their compensation.Credit...Jason Henry for The New York Times

Supported by

\protect\hyperlink{after-sponsor}{Continue reading the main story}

By \href{https://www.nytimes3xbfgragh.onion/by/erin-griffith}{Erin
Griffith}

\begin{itemize}
\item
  Sept. 20, 2019
\item
  \begin{itemize}
  \item
  \item
  \item
  \item
  \item
  \end{itemize}
\end{itemize}

SAN FRANCISCO --- Last summer, several Airbnb employees wrote a letter
to the online room-rental start-up's founders.

On behalf of more than a dozen employees, they pleaded to be able to
sell their Airbnb stock options. Because Airbnb is privately held, its
shares cannot be easily traded or cashed in. So the employees also asked
that the company go public, a move that would let them freely sell their
shares, said five people who saw or were briefed on the document and
were not authorized to speak publicly.

The letter was a sign of the tension that has built up among Airbnb's
workers.

According to interviews with more than a dozen current and former
employees and investors, most of whom declined to be identified for fear
of retaliation, Airbnb's 6,000-person work force has become increasingly
frustrated by not being able to cash in the company stock that was
received in compensation packages. Waiting for the start-up to go public
has become a growing source of stress, many said, preventing some from
making career changes, starting a family or moving on with their lives.

Questions about going public have risen to the top of an internal
message board where employees vote for topics for executives to address
every few months, the people said. The discontent has been exacerbated
because Airbnb, which has been valued at \$31 billion, doled out two
tranches of employee equity that are set to start expiring in November
2020 and in mid-2021; those shares will become worthless if the company
is not trading publicly by then, they said.

To try to keep employees happy, Brian Chesky, Airbnb's chief executive,
and other top executives have made some adjustments, the people said.
They began offering sabbaticals to longtime employees, extended Airbnb's
parental leave policy and increased the retirement matching program.
They also created a program to provide low-interest general-purpose
loans of hundreds of thousands of dollars to employees. In performance
reviews this spring, the start-up issued higher bonuses and raises, one
of the people said.

On Thursday, Airbnb took the biggest step of all: It released
\href{https://press.airbnb.com/airbnb-announces-intention-to-become-a-publicly-traded-company-during-2020/}{a
one-sentence announcement} saying it planned to go public next year.

\includegraphics{https://static01.graylady3jvrrxbe.onion/images/2019/09/20/business/20airbnb2/merlin_161076924_18a77aec-d268-441f-801c-2538e71394d2-articleLarge.jpg?quality=75\&auto=webp\&disable=upscale}

``We are deeply committed to our employees, and our focus on the
long-term has helped build a company that is highly successful and true
to our mission and values,'' Chris Lehane, Airbnb's senior vice
president for policy and communications, said in a statement. He added
that Airbnb was consistently ranked as a great place to work ``because
of the spirit, energy, values and morale of our employees.'' He declined
to comment on the employee letter.

Vivek Wagle, a marketing executive who left Airbnb in 2014, said
Thursday's announcement ``was definitely welcome news for a lot of us
early employees, who may have been wondering whether we'd be rewarded
for our part in the company's success.''

Airbnb's situation illustrates a paradox of the start-up dream. Many
tech workers join fast-growing privately held companies with the hope of
gaining stock in the firms and converting those shares to riches when
the start-ups go public. But employees are dependent on the company's
founders and board before that can become a reality.

Mr. Chesky, who co-founded Airbnb in 2008, has been vocal about not
rushing to take it public. In January 2018, he
\href{https://press.airbnb.com/brian-cheskys-open-letter-to-the-airbnb-community-about-building-a-21st-century-company/}{published
a letter saying} the company will have an ``infinite time horizon.'' He
is now exploring a nontraditional initial public offering by potentially
listing the shares directly, or on the Long-Term Stock Exchange, which
is backed by venture capital but not yet operational, three people with
knowledge of the situation said.

Doug Leone, a venture capitalist at Sequoia Capital, one of Airbnb's
backers, said that while start-ups had ``an implied social contract'' to
go public at some point, there was no rush for them to do so. ``The
I.P.O. is just a moment in time,'' he said.

Yet Mr. Chesky's go-slow stance has become problematic as other
high-profile start-ups of the same generation as Airbnb have started
listing their shares on the stock market. This year, the ride-hailing
companies Uber and
\href{https://www.nytimes3xbfgragh.onion/2019/03/29/technology/lyft-stock-price.html}{Lyft},
the online pinboard company Pinterest and the
\href{https://www.nytimes3xbfgragh.onion/2019/06/02/technology/slack-stewart-butterfield.html}{business
software maker Slack} are among those that have gone public. That has
allowed their employees to cash in their shares.

Employee tension is unusual for Airbnb, known for its cheery mission of
``belong anywhere'' and for fostering a kumbaya culture among its staff.
The company has grown rapidly, with more than seven million listings in
100,000 cities. In the second quarter, its revenue exceeded \$1 billion.
Many employees work out of an airy building in San Francisco, which
features rooms that replicate its famous listings. Several former
employees said they were grateful for the windfall they would eventually
receive from their shares.

Image

Dave Stephenson, Airbnb's chief financial officer, was injured in a ski
accident early this year.Credit...via Airbnb

But any reward from owning Airbnb stock has been held back. Starting in
2011, when the young company topped a \$1 billion valuation, Airbnb
prohibited workers from selling shares, while allowing its three
founders --- Mr. Chesky, Nathan Blecharczyk and Joe Gebbia --- to
\href{http://allthingsd.com/20111001/vcs-unite-chamath-palihapitiya-decries-airbnbs-recent-112m-funding-for-excessive-founder-control-and-cashout-in-email/}{cash
out a total of \$21 million}.

In its early days, Airbnb paid employees partly in grants of stock
options, which allow them to eventually buy --- or ``exercise'' ---
shares in the company at a low price. Airbnb later began offering a
different form of equity compensation, called restricted stock units,
which do not need to be bought.

Gabriel Cole, who worked in Airbnb's food department, said he had spent
his life savings to buy his stock after he left the company in 2015.
That incurred a \$180,000 tax bill, which he couldn't afford, he said.

``I was returning bottles to buy groceries,'' he said. He said he had
asked Airbnb's founders for help and had been told that nothing could be
done.

Over the years, Airbnb has extended rules around exercising stock
options to make the ``golden handcuffs'' less onerous. In 2016, it
allowed longtime employees who were still at the company to
\href{https://www.nytimes3xbfgragh.onion/2016/08/12/technology/airbnb-and-others-set-terms-for-employees-to-cash-out.html}{sell
portions of their stock}. Those who sold had to agree to stricter
restrictions on offering any remaining stock.

But those changes did not benefit all of Airbnb's stockholders. Some
current and former Airbnb employees have tried to circumvent the
prohibitions by selling their stock on a shadow, or secondary, market
for private share sales. In recent weeks, those Airbnb shares have
traded as high as \$166, which implies a fully diluted company valuation
of \$52 billion, three people familiar with the secondary market said.

Investment firms have also sprung up to offer loans to former Airbnb
employees, using their stock as collateral, in what is known as a
``prepaid variable forward contract.'' The firms aggressively court
former employees, often inundating them with offers for their stock the
minute they change their employment status on LinkedIn.

Image

Employee tension is unusual for Airbnb, which is known for fostering a
kumbaya culture among its staff.Credit...Jason Henry for The New York
Times

The transactions typically involve a cash loan in exchange for a pledge
of shares to a buyer at an agreed-upon price when the company goes
public, according to investment offers viewed by The New York Times. The
firms charge as much as a 15 percent fee and more for insurance. Former
employees who did these deals said they existed in a legal gray area ---
not authorized by Airbnb, but not explicitly banned.

``Opportunistic brokers and firms push them due to their fat fees,''
said Barrett Cohn, chief executive at Scenic Advisement, which works
with companies on secondary share sales. ``They're dangerous to buyers
and sellers, and expensive.''

When the small group of Airbnb employees sent their letter to Mr. Chesky
and other top executives last year, they received no response, two of
the people who viewed or were briefed on the letter said.

At the same time, Airbnb took several steps that appeared to signal it
was preparing for a public offering. It added independent board members
and hired Dave Stephenson, a seasoned finance executive, from Amazon to
become its chief financial officer. Current and former employees said
they had taken the moves as signs that the company was finally set to
reach the stock market.

In February, Mr. Stephenson was injured in a skiing accident. That
slowed Airbnb's I.P.O. timeline, two people with knowledge of the
situation said. An Airbnb spokesman said the accident did not have any
impact.

On Thursday evening, at an informal gathering of Airbnb alumni in San
Francisco, the company's announcement that it would go public next year
was the topic du jour. But the attendees reserved their excitement for
when the company officially files to do so, two people who were present
said.

Until then, one attendee said in a text message, the general sentiment
is ``🤷.''

Advertisement

\protect\hyperlink{after-bottom}{Continue reading the main story}

\hypertarget{site-index}{%
\subsection{Site Index}\label{site-index}}

\hypertarget{site-information-navigation}{%
\subsection{Site Information
Navigation}\label{site-information-navigation}}

\begin{itemize}
\tightlist
\item
  \href{https://help.nytimes3xbfgragh.onion/hc/en-us/articles/115014792127-Copyright-notice}{©~2020~The
  New York Times Company}
\end{itemize}

\begin{itemize}
\tightlist
\item
  \href{https://www.nytco.com/}{NYTCo}
\item
  \href{https://help.nytimes3xbfgragh.onion/hc/en-us/articles/115015385887-Contact-Us}{Contact
  Us}
\item
  \href{https://www.nytco.com/careers/}{Work with us}
\item
  \href{https://nytmediakit.com/}{Advertise}
\item
  \href{http://www.tbrandstudio.com/}{T Brand Studio}
\item
  \href{https://www.nytimes3xbfgragh.onion/privacy/cookie-policy\#how-do-i-manage-trackers}{Your
  Ad Choices}
\item
  \href{https://www.nytimes3xbfgragh.onion/privacy}{Privacy}
\item
  \href{https://help.nytimes3xbfgragh.onion/hc/en-us/articles/115014893428-Terms-of-service}{Terms
  of Service}
\item
  \href{https://help.nytimes3xbfgragh.onion/hc/en-us/articles/115014893968-Terms-of-sale}{Terms
  of Sale}
\item
  \href{https://spiderbites.nytimes3xbfgragh.onion}{Site Map}
\item
  \href{https://help.nytimes3xbfgragh.onion/hc/en-us}{Help}
\item
  \href{https://www.nytimes3xbfgragh.onion/subscription?campaignId=37WXW}{Subscriptions}
\end{itemize}
