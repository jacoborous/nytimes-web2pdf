Sections

SEARCH

\protect\hyperlink{site-content}{Skip to
content}\protect\hyperlink{site-index}{Skip to site index}

\href{https://www.nytimes3xbfgragh.onion/section/style}{Style}

\href{https://myaccount.nytimes3xbfgragh.onion/auth/login?response_type=cookie\&client_id=vi}{}

\href{https://www.nytimes3xbfgragh.onion/section/todayspaper}{Today's
Paper}

\href{/section/style}{Style}\textbar{}At Hudson Yards, One Mall for the
Rich, and One for Everyone Else

\url{https://nyti.ms/2YV0viu}

\begin{itemize}
\item
\item
\item
\item
\item
\end{itemize}

Advertisement

\protect\hyperlink{after-top}{Continue reading the main story}

Supported by

\protect\hyperlink{after-sponsor}{Continue reading the main story}

Critical Shopper

\hypertarget{at-hudson-yards-one-mall-for-the-rich-and-one-for-everyone-else}{%
\section{At Hudson Yards, One Mall for the Rich, and One for Everyone
Else}\label{at-hudson-yards-one-mall-for-the-rich-and-one-for-everyone-else}}

The new development has high-end retail, art you can play with and not
enough bathrooms.

By \href{https://www.nytimes3xbfgragh.onion/by/jon-caramanica}{Jon
Caramanica}

\begin{itemize}
\item
  April 10, 2019
\item
  \begin{itemize}
  \item
  \item
  \item
  \item
  \item
  \end{itemize}
\end{itemize}

\includegraphics{https://static01.graylady3jvrrxbe.onion/images/2019/04/11/fashion/11CRITIC-neiman-2/11CRITIC-neiman-2-articleLarge.jpg?quality=75\&auto=webp\&disable=upscale}

One recent afternoon, I took a cab down to Hudson Yards from the Upper
West Side. With little prompting, the driver lamented how the
development had taken away two things he'd held sacred: the relative
calm of the neighborhood before development started and the view
eastward of the Midtown skyline, now obscured.

After a rueful few seconds, he mumbled, ``I feel choked.''

Certainly, there is a falsity to the Hudson Yards complex, which sits
atop an active train yard and shoots up to the sky impressively but
generally thoughtlessly. But to be fair, the whole of 11th Avenue is a
zoo of gauche intrusions, from the residential developments in the 60s
to the car dealerships in the 50s all the way down to the renovated
industrial buildings in the 20s. There's never been a true neighborhood
along it. Why start now?

Agita of that sort is why, so often in recent weeks, you've heard that
Hudson Yards is New York City's Dubai.

That's a grave insult to Dubai, where there are indoor surfing waves and
artificial islands shaped like all the countries of the globe.

More accurately, Hudson Yards is Midtown's Battery Park City, both less
insidious and more mundane than how it has been advertised. Up top, it's
gleaming, but at ground level, it's deeply unromantic.

Image

The shopping complex is home to various art installations. Here,
visitors interact with "I Was Here," by Lara Schnitger.Credit...Karsten
Moran for The New York Times

When you come up the escalator from the 7 train station, what you see
first aren't artisanal food trucks, but regular coffee and hot dog
carts. From the front, the building that houses the Shops \& Restaurants
at Hudson Yards is shiny but unremarkable; look up and you're greeted by
the posteriors of Lululemon mannequins.

The Hudson Yards shopping complex is two malls in one, really. The first
is at the top and bottom: The fifth floor (and above) has the first New
York outposts of two Dallas multibrand retail forces --- Neiman Marcus,
the 110-plus-year and 40-plus-store luxury specialist, and Forty Five
Ten, a forward-looking emporium just beginning to develop a national
presence. The ground level features stand-alone storefronts for Fendi,
Coach, Tory Burch, Dunhill and others, as well as Rolex, Cartier and
Piaget.

And then there's everything in between, largely of spick-and-span
versions of chain stores available at dozens of other places in the
city.

The result has the feel of one of New York's 80/20 buildings, where a
certain percentage of apartments in new construction are given over to
affordable housing as a make-good for a whole heap of market-rate
chicanery. I would not be surprised to learn that there is a secret
elevator that goes straight from a private car entrance to the fifth
floor.

Up on 5, the diversity of rich people on three recent visits was
staggering: people with \$300 sneakers, and also with \$1,200 sneakers;
art-gallery-owner chic, and Upper East Side matriarch chic; people who
D.J. for bottle service clubs, and people who buy bottle service; young
people with designer fanny packs who looked like Jerrika Karlae, or
Jimin, or Davido, or Anuel AA and Karol G.

By contrast, on the third floor, a smiling kid wearing a hoodie from
Anuel AA and Karol G's tour was going up and down the escalators, having
a great time.

Image

The boutique Forty Five Ten, also newly arrived from
Dallas.Credit...Karsten Moran for The New York Times

NEIMAN MARCUS AND FORTY FIVE TEN together make for a worthy addition to
the city's retail landscape. Forty Five Ten is broken into four distinct
storefronts, making the fifth floor feel like a theme park. There is a
well-stocked women's designer section, a cluster of emerging designers,
some vintage and a men's section.

The Forty Five Ten in Dallas is so effective because it has the most
forward-looking brand mix in the city. Here, though, in the men's
department, the thick scrums of Rick Owens, Thom Browne and Stone Island
feel predictable, a selection salvaged by a handful of stunning pieces:
an oversize yellow print parka with aviator hood from Takahiromiyashita
the Soloist (\$2,500) and a ruggedly cut military-style jacket with
detachable liner by Jil Sander (\$2,750).

The women's store is far more effective, with wilder styles from a wider
range of companies: a short-sleeved cropped floral sweater from Molly
Goddard (\$350) just a couple of feet from a stunning princess cape
covered in glittery embroidery by Rodarte (\$10,388); a Monse patchwork
wrap skirt and shorts set made of what looked like rep-tie-patterned
silk (\$1,890) and also a sharp black top with architectural pouf
sleeves from Dice Kayek (\$1,195).

Image

Here, the Lucchese boot rack at Neiman Marcus.Credit...Karsten Moran for
The New York Times

The emerging designer space had less ornate forms of cleverness:
diffusion prairie dresses from Batsheva, stern folk-art slides from
Nicole Saldana, nü-basics from Sandy Liang and Eckhaus Latta. And in the
vintage shop, there are framed copies of Avant Garde magazine, from the
company president's personal collection. (Not for sale, sadly.)

On a recent Saturday, Neiman Marcus, which begins on the fifth floor and
rises two more, felt like a true amusement park. In the men's section,
there was a foosball table, an arcade game and a Skee-Ball game all
getting loads of use. But at any given time, only a handful of clerks
were working, eyeing the huge floor like a center fielder at Fenway.

The selection was wide but not particularly deep. Some fine-enough
Balenciaga and a Burberry collection that could plausibly pass for
bootleg. There were generous helpings of Kiton and Brunello Cucinelli.

But the real showcase was the shoe section. Now me, I'm old enough to
remember when you had to hunt for \$1,000 sneakers, but here they all
are, indifferent to who might buy them. The only refreshing disruption
was a thorough selection of Lucchese boots, the biggest in the city. I
tried on a \$3,000 pair of alligator boots, and immediately my posture
improved. They made me feel beautiful (and also poor).

Image

Shoppers amble through the shopping complex.Credit...Karsten Moran for
The New York Times

Even though this Neiman Marcus is vast, it's less a space for serious
clothes shopping and more for superluxe trinkets, which is why the whole
of the fifth floor is given over to bags, shoes, scarves and other
low-hanging fruit of wealth signification, whether it's a saddle bag
from the Row (\$1,980) only a few will be able to identify or a
graffitied kitten heel from Balenciaga that will lose half its value the
moment it leaves the store.

A woman walked up to the display of exquisitely brittle-looking René
Caovilla sandals and whispered to her partner, ``Cinderella shoesssss.''
You will never miss Phoebe Philo more than when looking at the spiceless
wall of Celine handbags here.

ONCE YOU GO BELOW THE FIFTH FLOOR, however, things change radically.
Given the size of this building, there is surprisingly little shopping
to be done. I've never been somewhere where the ratio of people walking
through the halls to those actually in a store was as high as it was on
the middle floors here. Those spaces feel more like tourist attractions,
especially given the large-scale art installations scattered throughout.

Image

Artwork by Serban Ionescu.Credit...Karsten Moran for The New York Times

Much of the art is interactive, meant for lingering: a wall of scrapable
sequins by Lara Schnitger, which never didn't have a few dozen teenagers
fiddling with it; pastel industrial-collapse animal statue benches from
Serban Ionescu, which provided succor for glum-looking shoppers; a
psychedelic mural with embedded QR codes from Jeanette Hayes. (Whoever
selected the artists follows the same people on Instagram that I do.
Please contact me for your future curatorial needs.)

Even though there are plenty of attractions and distractions, the act of
actually accommodating people still feels like a challenge. The mall is
curiously under-bathroomed, and trash and recycling bins aren't
prominent. The music can be comically loud, like the hi-NRG
goth-SoulCycle assault that defied Shazaming.

That tension, between being a place for people to shop and a place for
people to merely linger and marvel, is the clearest on the ground floor.
Cartier, Piaget, Rolex --- inside, they're beatific, and there are
guards at the doors to ensure they remain that way. Out in the
corridors, greeters in black suits and blue ties enthusiastically guide
lost tourists to the escalators, perhaps hoping they'll be sated by some
H\&M or Athleta, or some ice cream from Van Leeuwen (which always had a
line).

The grimmest space in the whole building is the art store near the exit,
Avant Gallery, which sells overpriced, absurd post-graffiti canvases
that would be gauche even in the middle of a third-tier suburban mall.

For a development whose idea of an art installation is the
staircase-to-nowhere Vessel, as conceptually rigorous as the sprinkle
pit at the ice cream museum, this perhaps isn't much of a surprise.

But as a barometer of the level of sophistication the owners expect of
their customers, it's telling. They're counting on the fact that money
doesn't know where it came from, and it doesn't care where it goes ---
all the way to the sky, or buried deep in the ground.

Advertisement

\protect\hyperlink{after-bottom}{Continue reading the main story}

\hypertarget{site-index}{%
\subsection{Site Index}\label{site-index}}

\hypertarget{site-information-navigation}{%
\subsection{Site Information
Navigation}\label{site-information-navigation}}

\begin{itemize}
\tightlist
\item
  \href{https://help.nytimes3xbfgragh.onion/hc/en-us/articles/115014792127-Copyright-notice}{©~2020~The
  New York Times Company}
\end{itemize}

\begin{itemize}
\tightlist
\item
  \href{https://www.nytco.com/}{NYTCo}
\item
  \href{https://help.nytimes3xbfgragh.onion/hc/en-us/articles/115015385887-Contact-Us}{Contact
  Us}
\item
  \href{https://www.nytco.com/careers/}{Work with us}
\item
  \href{https://nytmediakit.com/}{Advertise}
\item
  \href{http://www.tbrandstudio.com/}{T Brand Studio}
\item
  \href{https://www.nytimes3xbfgragh.onion/privacy/cookie-policy\#how-do-i-manage-trackers}{Your
  Ad Choices}
\item
  \href{https://www.nytimes3xbfgragh.onion/privacy}{Privacy}
\item
  \href{https://help.nytimes3xbfgragh.onion/hc/en-us/articles/115014893428-Terms-of-service}{Terms
  of Service}
\item
  \href{https://help.nytimes3xbfgragh.onion/hc/en-us/articles/115014893968-Terms-of-sale}{Terms
  of Sale}
\item
  \href{https://spiderbites.nytimes3xbfgragh.onion}{Site Map}
\item
  \href{https://help.nytimes3xbfgragh.onion/hc/en-us}{Help}
\item
  \href{https://www.nytimes3xbfgragh.onion/subscription?campaignId=37WXW}{Subscriptions}
\end{itemize}
