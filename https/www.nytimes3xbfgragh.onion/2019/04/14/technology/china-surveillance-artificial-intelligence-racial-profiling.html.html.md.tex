Sections

SEARCH

\protect\hyperlink{site-content}{Skip to
content}\protect\hyperlink{site-index}{Skip to site index}

\href{https://www.nytimes3xbfgragh.onion/section/technology}{Technology}

\href{https://myaccount.nytimes3xbfgragh.onion/auth/login?response_type=cookie\&client_id=vi}{}

\href{https://www.nytimes3xbfgragh.onion/section/todayspaper}{Today's
Paper}

\href{/section/technology}{Technology}\textbar{}One Month, 500,000 Face
Scans: How China Is Using A.I. to Profile a Minority

\url{https://nyti.ms/2UjDUJ0}

\begin{itemize}
\item
\item
\item
\item
\item
\end{itemize}

Advertisement

\protect\hyperlink{after-top}{Continue reading the main story}

Supported by

\protect\hyperlink{after-sponsor}{Continue reading the main story}

\hypertarget{one-month-500000-face-scans-how-china-is-using-ai-to-profile-a-minority}{%
\section{One Month, 500,000 Face Scans: How China Is Using A.I. to
Profile a
Minority}\label{one-month-500000-face-scans-how-china-is-using-ai-to-profile-a-minority}}

In a major ethical leap for the tech world, Chinese start-ups have built
algorithms that the government uses to track members of a largely Muslim
minority group.

\includegraphics{https://static01.graylady3jvrrxbe.onion/images/2019/04/11/business/00chinaprofiling-1/merlin_137008323_51c7d2a2-2429-4cba-9bcc-e379ace76016-articleLarge.jpg?quality=75\&auto=webp\&disable=upscale}

\href{https://www.nytimes3xbfgragh.onion/by/paul-mozur}{\includegraphics{https://static01.graylady3jvrrxbe.onion/images/2018/10/15/multimedia/author-paul-mozur/author-paul-mozur-thumbLarge.png}}

By \href{https://www.nytimes3xbfgragh.onion/by/paul-mozur}{Paul Mozur}

\begin{itemize}
\item
  April 14, 2019
\item
  \begin{itemize}
  \item
  \item
  \item
  \item
  \item
  \end{itemize}
\end{itemize}

\href{https://cn.nytimes3xbfgragh.onion/technology/20190415/china-surveillance-artificial-intelligence-racial-profiling/}{阅读简体中文版}\href{https://cn.nytimes3xbfgragh.onion/technology/20190415/china-surveillance-artificial-intelligence-racial-profiling/zh-hant/}{閱讀繁體中文版}

The Chinese government has drawn wide
\href{https://www.nytimes3xbfgragh.onion/2019/04/08/world/asia/china-muslims-camps.html?rref=collection\%2Ftimestopic\%2FUighurs\%20(Chinese\%20Ethnic\%20Group)\&action=click\&contentCollection=timestopics\&region=stream\&module=stream_unit\&version=latest\&contentPlacement=2\&pgtype=collection}{international
condemnation} for its harsh crackdown on ethnic Muslims in its western
region, including holding as many as a million of them in detention
camps.

Now, documents and interviews show that the authorities are also using a
vast, secret system of advanced facial recognition technology to track
and control the Uighurs, a
\href{https://www.nytimes3xbfgragh.onion/2018/09/08/world/asia/china-uighur-muslim-detention-camp.html}{largely
Muslim minority}. It is the first known example of a government
intentionally using artificial intelligence for racial profiling,
experts said.

The facial recognition technology, which is integrated into China's
rapidly expanding networks of surveillance cameras, looks exclusively
for Uighurs based on their appearance and keeps records of their comings
and goings for search and review. The practice makes China a pioneer in
applying next-generation technology to watch its people, potentially
ushering in a new era of automated racism.

The technology and its use to keep tabs on China's 11 million Uighurs
were described by five people with direct knowledge of the systems, who
requested anonymity because they feared retribution. The New York Times
also reviewed databases used by the police, government procurement
documents and advertising materials distributed by the A.I. companies
that make the systems.

\includegraphics{https://static01.graylady3jvrrxbe.onion/images/2017/01/29/podcasts/the-daily-album-art/the-daily-album-art-articleInline-v2.jpg?quality=75\&auto=webp\&disable=upscale}

\hypertarget{listen-to-the-daily-the-chinese-surveillance-state-part-1}{%
\subsubsection{Listen to `The Daily': The Chinese Surveillance State,
Part
1}\label{listen-to-the-daily-the-chinese-surveillance-state-part-1}}

Using facial recognition software for ethnic profiling is only one way
that Beijing is harnessing technology for social control.

transcript

Back to The Daily

bars

0:00/23:22

-23:22

transcript

\hypertarget{listen-to-the-daily-the-chinese-surveillance-state-part-1-1}{%
\subsection{Listen to `The Daily': The Chinese Surveillance State, Part
1}\label{listen-to-the-daily-the-chinese-surveillance-state-part-1-1}}

\hypertarget{hosted-by-michael-barbaro-produced-by-andy-mills-alexandra-leigh-young-jessica-cheung-and-luke-vander-ploeg-and-edited-by-lisa-tobin}{%
\subsubsection{Hosted by Michael Barbaro, produced by Andy Mills,
Alexandra Leigh Young, Jessica Cheung and Luke Vander Ploeg, and edited
by Lisa
Tobin}\label{hosted-by-michael-barbaro-produced-by-andy-mills-alexandra-leigh-young-jessica-cheung-and-luke-vander-ploeg-and-edited-by-lisa-tobin}}

\hypertarget{using-facial-recognition-software-for-ethnic-profiling-is-only-one-way-that-beijing-is-harnessing-technology-for-social-control}{%
\paragraph{Using facial recognition software for ethnic profiling is
only one way that Beijing is harnessing technology for social
control.}\label{using-facial-recognition-software-for-ethnic-profiling-is-only-one-way-that-beijing-is-harnessing-technology-for-social-control}}

\begin{itemize}
\item
  michael barbaro\\
  From The New York Times, I'm Michael Barbaro. This is ``The Daily.''

  Today: Under Xi Jinping, China is pioneering a new form of governance
  by surveillance. In the first of a two-part series, my colleague Paul
  Mozur on how China piloted that system on one minority group in the
  country. It's Monday, May 6.

  Hi.
\item
  paul mozur\\
  Hi.
\item
  michael barbaro\\
  Headphones.

  Paul, we've actually never met. You are in town from China. I'm hoping
  you're going to tell me why.
\item
  paul mozur\\
  Yeah, so I've been reporting in and around China for about 12 years.
  And there's always been a lot of control. I think people are aware of
  that. They're aware there's censorship. They're aware that people can
  be followed, and there is a certain amount of surveillance. But in the
  past five years, things have really changed and taken a much more
  dramatic and darker turn, really, when it comes to, especially,
  surveillance. And that coincides with the rise of Xi Jinping. China's
  president who came into power about five years ago has really doubled
  down on control. And he has been not shy at all about using technology
  to exert that control. And there's a lot of things that are invisible
  in how that works, but one of the very few visible symptoms are the
  cameras. There were always some cameras in China, but recently, past
  couple of years, the cameras have just gone in in this dramatic way.
  Some of them look like these baroque modernist sculptures or
  something. It's like four cameras stretching off of a different pole,
  or you have a camera hanging from a tree. There's these almost hidden
  cameras in the subway cars, these little holes. And if you look
  closely at them, you say, oh, my God, that's actually a lens. I
  counted the cameras on my way to work one day, which is a
  two-subway-stop ride. And I passed, I think, 250 cameras.
\item
  michael barbaro\\
  Wow.
\item
  paul mozur\\
  Yeah.
\item
  michael barbaro\\
  In what kind of places? Where are you seeing them?
\item
  paul mozur\\
  All kinds of places. Every intersection will have dozens of cameras to
  catch people's license plates as they drive by. About every 50 yards,
  you'll have a camera on a pole that's a dome camera that can zoom in
  and grab their faces or follow somebody if you have to. And when you
  walk down stairs, there's these high-powered facial recognition
  cameras aimed at your face, with the idea of trying to figure out who
  you are as you walk by.
\item
  michael barbaro\\
  And who's on the other side of those cameras?
\item
  paul mozur\\
  Yeah, it's what I always wonder. We don't always know. And this is the
  thing about China, is that it is an autocratic system with very little
  transparency. For the most part, what we assume is a newly empowered
  police force is using these to try to learn as much as they can about
  the population and track them.
\item
  michael barbaro\\
  But to the degree that there's a rationale for this, what is it?
\item
  paul mozur\\
  Security. Safety. We want to make sure that if something bad were to
  happen in our neighborhood, we could protect ourselves. But in some
  recent reporting, what we discovered is the true breathtaking ways in
  which the police are already assembling lists of faces of people that
  they're worried about, and even using it to mark people based on
  ethnicity and race, and track them and keep a record. It's as if you
  were just counting only one group of people as they went around a city
  and keeping tabs so that you can go back and see which person that
  was. And in America, this would be horrendously unconstitutional. But
  in China, it had been happening for almost two years without anybody
  even noticing.
\item
  michael barbaro\\
  And why would China want to do that? Why would it track a group of
  people by race through cameras and this classification system?
\item
  paul mozur\\
  Right. So China has had this long issue with a Muslim minority known
  as the Uighurs, who live out in western China, this massive province,
  a fifth the size of China's landmass, called Xinjiang. It's mostly
  desert and really high mountains. It's the old Silk Road. And these
  people have lived there for more than 1,000 years in these tiny little
  oasis cities around the desert. And China has occupied their land for
  several hundred years now. And as China has occupied it, for the most
  part, until maybe the past 50 or 60 years, it's mostly just been a
  far-flung place. But under the Chinese Communist Party, they've really
  solidified power. And they've started to change the demographics. So
  they've created all these passive incentives to move Han Chinese into
  this region.
\item
  michael barbaro\\
  To basically make it less Muslim? More Chinese?
\item
  paul mozur\\
  Exactly. And so, 50 or 60 years ago, there were almost no Chinese, and
  it was all Uighurs. Now it's 50 percent Chinese, 50 percent Uighur.
  And that's created all of these conflicts.

  But everything really changed in 2009. What happens is, there's
  violence in this small factory in southern China. And it turns out
  that there was a rumor that these two Uighur factory workers raped a
  Chinese woman, and then when the ethnic Chinese confront the Uighur
  population at the factory, a big fight breaks out, where, ultimately,
  two Uighurs end up beaten to death.
\item
  archived recording\\
  {[}SHOUTING{]}
\end{itemize}

paul mozur

And there's a video of this on YouTube, and it goes around.

\begin{itemize}
\tightlist
\item
  archived recording\\
  {[}SPEAKING MANDARIN{]}
\end{itemize}

paul mozur

And in a tinderbox like Xinjiang, where you have all of these other
tensions there, it becomes one of the main causes for this massive
outburst of rioting and anger in the capital of Urumqi.

Thousands of Uighurs take to the streets, some with knives, and they
murder about 200 Han Chinese.

michael barbaro

Wow.

paul mozur

So it's a brutal and large-scale race riot.

\begin{itemize}
\tightlist
\item
  archived recording\\
  {[}SHOUTING{]}
\end{itemize}

paul mozur

And following that, the military is mobilized. The internet is cut off
in Xinjiang, so you cannot get online. Even phone calls outside the
country are no longer allowed. It's a new level of suffocating
technological response. And so in the ensuing decade, what they've tried
to do is figure out methods to systematize it. And so they've turned to
the police, and they've turned to technology.

michael barbaro

Have you been back there recently?

paul mozur

Yes. So I went back in October to Kashgar.

It's a transformed place. It's one of the most bizarre places I think
I've ever been. We're, of course, followed by secret police wherever we
go. There are checkpoints every couple hundred yards. And they've
created these things called convenience police centers. So think of a
convenience store, but it's a police station instead. So these small
concrete boxes with constantly flashing lights. And they're every couple
hundred yards, and police are in them. And they'll set up checkpoints
there. But the idea is to blanket the city with this very suffocating
level of police presence and surveillance. This is an old, mud-brick
city, filled with bazaars.

And now what you have is that look with these tremendously powerful
facial recognition cameras hanging from a mud-brick wall, and there are
cameras absolutely everywhere.

And so you have this very bizarre contrast of a place that in some ways
feels like it could be timeless and 1,000 years old, with these
hyper-modern technological solutions attempting to understand and track
the populations.

michael barbaro

So tell me about that tracking. So clearly, China is very anxious that
this Muslim population is going to revolt or just generally disobey the
desires of the Chinese government. So how does that translate into this
surveillance apparatus? What are they going to do with the image of a
Muslim man or woman in this place that's going to stop that?

paul mozur

Well, so they've already thrown about a million people in camps.

michael barbaro

Labor camps?

paul mozur

Well, they call them re-education camps. And we don't have a lot of
understanding of what happens inside. But it seems to be day-long
classes and people being made to sit and hear Chinese Communist Party
theory and propaganda and things like that. They need excuses to put
people in these places. So if you have this massive surveillance system,
you can find people that you think might be dangerous or might be risky.
But the thing is, it's so over the top and so extreme, people get thrown
in because they're an academic, because they're influential, because
they use technology, because they wouldn't shave their beard, because
they read the Quran. There's a million different ways. What they've done
is just tracked everybody all the time in a way that nobody even can go
out their door without feeling the weight of the gaze of the state.

michael barbaro

Right. Everything you just described would be something that you could
capture on camera. You would see someone with a beard. You might see
someone reading the Quran. And that could be the trigger.

paul mozur

Right. And they've hung lots of cameras in mosques. So the Id Kah Mosque
is this beautiful, mustard yellow mosque that sits in the center of old
Kashgar. It's the heart of Uighur Islam. And I think I counted more than
200 cameras inside the mosque, trying to capture worshippers who would
come and go. And there aren't many worshippers anymore, of course,
because who's going to go walk in front of those cameras and show their
faces? And then that very easily can just go into a database, and then
they have a data point. They know that Michael was right outside the Id
Kah Mosque at this time. And then when he leaves the Id Kah Mosque,
he'll have to give his ID again. And then when he goes down to the
marketplace, he has to give his ID again. And that way, you can build a
comprehensive map of where you're going. If you want to go to the bank,
if you want to go to a grocery store, you have to do this. If you want
to enter the old city, you have to do it. And so, it effectively just
makes it impossible to do anything in this society without constantly
giving up your private information to the state and to the police.

michael barbaro

And what you're describing is the definition of dystopia.

paul mozur

Yeah. And it goes even deeper than this. Around 2017, 2016, in Kashgar,
we've heard that many people were called in for compulsory medical
checkups. And they never got the results of the medical checkup. But
what the medical checkup was was they had to give a blood sample, and
their faces were scanned. And they had to give a voice sample, irises
were scanned. And so just a mass collection of a single ethnicity's
biometric information. And we don't really know entirely what they'll do
with all of that. In our reporting, we've seen parts of this. We've seen
some of the dossiers. And so they can map people's family relationships.

michael barbaro

Wow. And what might the Chinese government do with that information?
About family members, all those connections?

paul mozur

They use it to lean on people. And they use it to intimidate people. And
they use it to show that they are so powerful that there would be no
point, in a way, to resist or push back. And you could see it in the
population, the fear.

michael barbaro

I'm trying to understand where this leads to, because this doesn't seem
like an effort to acculturate people or to encourage them toward a
Chinese identity. If anything, the people who are being subjected to
this would most likely resent the Chinese government, right?

paul mozur

Right. I think the thinking goes further than that. The hope is
ultimately to, I think, change the population fundamentally --- to
re-engineer a new way of life for these people that is basically
Chinese. And I think the ultimate goal here is to eradicate Uighur
culture. And the thing is, if they fail, well, then they have a culture
so completely in their control that it's no longer a threat in any way.

michael barbaro

Paul, what's the relationship between what's happening to the Uighurs
and the larger surveillance state in China? If the rest of China is
already Chinese, how does this all connect?

paul mozur

So a lot of people like to call Xinjiang the laboratory for Chinese
surveillance. So if you have any kind of draconian solution to tracking
somebody or figuring out what somebody is doing on their phone, you can
try it out in Xinjiang, and then see what happens. In Xinjiang, they can
get away with a lot more, because you have an ethnic minority that is
already so beset that they can't really push back.

michael barbaro

A group without any power.

paul mozur

Right, exactly. In the rest of China, you see something that's a little
bit more passive, but you see a constant creep.

On the subway, for instance, you start to see more checkpoints. The
police just sit out where people are transferring, and they just stop
people at random, and they scan their ID card, just like what happened
in Xinjiang. And one of the things our reporting showed is that it's not
just Uighurs they're looking for in these cities. They're making lists
of people's faces depending on what kind of group they are. So they are
making lists of the mentally ill. They're making lists of people with a
past history of drug use. They're making lists of people who would
petition the government or complain about the government. But they also
have lists of every single person registered to live in that city. So
the idea isn't just to track these small groups. It's to track everyone,
with the idea that if somebody were to get out of line, then you know
everything about them to begin with.

michael barbaro

So this is about every single person in China?

paul mozur

Yes.

{[}music{]}

michael barbaro

I'm struck that all of this is happening at the same time that China is
becoming a world power whose influence is growing so much overseas,
because those things don't quite seem to be consistent. In fact, they
seem to be very contradictory.

paul mozur

Right. And I think they've basically proved that wrong, that you can
have censorship and you can have a closed society, in some ways, and a
controlled society, but also have a booming tech sector. And this is the
first time in probably 30 years that we've had an autocratic state
alongside the United States at the cutting edge of technology. So if you
think about it, democracies have dominated technological creation since
the fall of the Berlin Wall, effectively. Now China's coming along, and
they're making technologies, but these technologies are suited for their
purposes. And in a lot of cases, those purposes have some authoritarian
component to them, or some point of control to them. Very intentional
control. And in fact, as they've risen, they've used all of this as a
selling point. So think to the Beijing Olympics in 2008. This is China's
coming-out party as a new superpower.

{[}music{]}

paul mozur

They've outfitted the capital with tons of security to make sure it goes
well, to make sure there's no protests, but also to make sure there's no
attacks or anything. And so they load up the city with 300,000 cameras
that the government was controlling.

michael barbaro

Because, of course, this is a moment where you actually do want a lot of
security.

paul mozur

Yeah, exactly. Yeah. So they really pulled out all the stops.

But then what they did when all these international leaders arrived to
see the Olympics is they took them into the back rooms where you could
see all these cameras operating. They show the screens. They show, this
is how our policing system works.

michael barbaro

So who is in there?

paul mozur

So we don't know everybody that visited. But what we do know is that
countries like Ecuador sent delegations --- places that might be
struggling with democracy or even already being led by strongmen, who
have come to check this out. And there's screens up with video footage
from these thousands and thousands of cameras. And they can see how the
Chinese security forces can see everything. They look at it, and they
say, well, this is pretty powerful. I wonder if we could get this. And
that's where it starts. And so now what we're seeing is those
technologies are beginning to flow to the world. And so all of a sudden,
on the streets of Quito, you see the same cameras that you would see in
Shanghai. And that's not just happening there. That's happening in
Venezuela. That's happening in Bolivia. That's happening in Angola.
That's happening in Pakistan. It's happening around the world.

michael barbaro

Paul, what do you make of this global spread of surveillance, starting
in China? What does it tell you about the changes you've seen in China
in recent years and where all this is headed?

paul mozur

It tells me that, I think, the Chinese government believes it has
created a different model and a new model, and they want to propagate
it. They want to spread it. And they want to give other countries the
ability to do what they've done and, in that way, influence the world.
So this is --- governance by data, governance by mass surveillance, is,
in a way, the Chinese model now, and they want to bring it to the world.
And what this encourages is authoritarianism, because it uses technology
unapologetically to consolidate power by understanding what everybody's
doing and where they are at any given moment. And I think it's an
important moment for democracies like the United States, because they
need to recognize this is happening, but also say, well, what does the
United States stand for in all this? Do they stand for data collection,
as well, without telling anybody? Do you stand for something else?
Because the United States at this point is so lost in its own debates
---

michael barbaro

Right. So do we stand in contrast to that?

paul mozur

Right. Exactly. And that's the thing --- as I write about this from
China, it's unclear where the United States stands in all of it.

michael barbaro

In a world where this model that you're describing is spreading around
the world, how exactly does China benefit from that? Because there are
fewer and fewer places where a democratic government without
surveillance challenges it?

paul mozur

I think the idea is if you give the people you're dealing with these
systems, you increase their power. And that means the people you're
dealing with are more likely to keep dealing with you and be the ones in
power. So there's a sort of perpetuation. But I think there's also just
a broader sense of the more countries around the world that do this, the
more it's deemed acceptable by the world, and the more that they have
reliable partners who are following what they're doing and reliant on
them and allow them to push how governance works. And so in a way, they
become the axle, and all of these different places become the spokes in
this wheel, the new version of global governance, a new alternative to
the messy democracies of the past.

michael barbaro

Governance by data.

paul mozur

Governance by data and surveillance.

michael barbaro

Paul, thank you very much.

paul mozur

Thanks.

michael barbaro

On Sunday, The Times reported that the Trump administration has decided
not to confront China over its repressive treatment of the Uighurs, for
fear that it could disrupt the final stages of a major trade deal
between the two countries. The administration had considered imposing
economic sanctions on Chinese officials involved in the repression, but
has since backed away from that plan. In Part 2, we'll hear from one
Uighur man living in the U.S. who is trying to fight for his family in
the camps in China.

We'll be right back.

{[}music{]}

michael barbaro

Here's what else you need to know today.

On Sunday, fighting between Israel and Gaza escalated into the worst
combat since the full-blown war between them in 2014. Four Israeli
civilians were killed by Palestinian rocket and missile attacks,
prompting Israel to take aim at individual militants in Gaza, killing at
least nine of them and as many civilians.

\begin{itemize}
\tightlist
\item
  archived recording (benjamin netanyahu)\\
  {[}SPEAKING HEBREW{]}
\end{itemize}

michael barbaro

During a news conference, Israel's prime minister, Benjamin Netanyahu,
promised massive attacks against the militants in Gaza.

\begin{itemize}
\tightlist
\item
  archived recording (benjamin netanyahu)\\
  {[}SPEAKING HEBREW{]}
\end{itemize}

michael barbaro

The Palestinian rocket attacks mostly struck civilian targets in
southern Israel with no military value, including a building that houses
a kindergarten and the oncology department of a medical center. The
violence is the latest in a long-running series of clashes that have
produced temporary ceasefires that are quickly broken. And President
Trump on Sunday said that Special Counsel Robert Mueller should not
testify before Congress, setting up another confrontation with
congressional Democrats, who have requested Mueller's appearance. In a
tweet, the president said that Mueller's report was conclusive and that
Americans do not need to hear from him again. ``No redos for the Dems,''
he wrote. Because Mueller was appointed by the Department of Justice,
which answers to the president, it appears that Trump has the authority
to prevent Mueller from testifying.

That's it for ``The Daily.'' I'm Michael Barbaro. See you tomorrow.

Chinese authorities already maintain a
\href{https://www.nytimes3xbfgragh.onion/interactive/2019/04/04/world/asia/xinjiang-china-surveillance-prison.html}{vast
surveillance net},
\href{https://www.nytimes3xbfgragh.onion/2019/02/21/business/china-xinjiang-uighur-dna-thermo-fisher.html}{including
tracking people's DNA}, in the western region of Xinjiang, which many
Uighurs call home. But the scope of the new systems, previously
unreported, extends that monitoring into many other corners of the
country.

\includegraphics{https://static01.graylady3jvrrxbe.onion/images/2019/04/11/business/00chinaprofiling-2/00chinaprofiling-2-articleLarge.jpg?quality=75\&auto=webp\&disable=upscale}

The police are now using facial recognition technology to target Uighurs
in wealthy eastern cities like Hangzhou and Wenzhou and across the
coastal province of Fujian, said two of the people. Law enforcement in
the central Chinese city of Sanmenxia, along the Yellow River, ran a
system that over the course of a month this year screened whether
residents were Uighurs 500,000 times.

Police documents show demand for such capabilities is spreading. Almost
two dozen police departments in 16 different provinces and regions
across China sought such technology beginning in 2018, according to
procurement documents. Law enforcement from the central province of
Shaanxi, for example, aimed to acquire a smart camera system last year
that ``should support facial recognition to identify Uighur/non-Uighur
attributes.''

Some police departments and technology companies described the practice
as ``minority identification,'' though three of the people said that
phrase was a euphemism for a tool that sought to identify Uighurs
exclusively. Uighurs often look distinct from China's majority Han
population, more closely resembling people from Central Asia. Such
differences make it easier for software to single them out.

For decades, democracies have had a near monopoly on cutting-edge
technology. Today, a new generation of start-ups catering to Beijing's
authoritarian needs are beginning to set the tone for emerging
technologies like artificial intelligence. Similar tools could automate
biases based on skin color and ethnicity elsewhere.

``Take the most risky application of this technology, and chances are
good someone is going to try it,'' said Clare Garvie, an associate at
the Center on Privacy and Technology at Georgetown Law. ``If you make a
technology that can classify people by an ethnicity, someone will use it
to repress that ethnicity.''

From a technology standpoint, using algorithms to label people based on
race or ethnicity has become relatively easy. Companies like I.B.M.
\href{https://www.ibm.com/support/knowledgecenter/SS88XH_2.0.0/iva/attribute_detectors_ranked_search.html}{advertise
software} that can sort people into broad groups.

But China has broken new ground by identifying one ethnic group for law
enforcement purposes. One Chinese start-up, CloudWalk, outlined a sample
experience in marketing its own surveillance systems. The technology, it
said, could recognize ``sensitive groups of people.''

Image

A screen shot from the CloudWalk website details a possible use for its
facial recognition technology. One of them: recognizing ``sensitive
groups of people.''Credit...CloudWalk

Image

A translation of marketing material for CloudWalk's facial recognition
technology.Credit...The New York Times

``If originally one Uighur lives in a neighborhood, and within 20 days
six Uighurs appear,'' it said on its website, ``it immediately sends
alarms'' to law enforcement.

In practice, the systems are imperfect, two of the people said. Often,
their accuracy depends on environmental factors like lighting and the
positioning of cameras.

In the United States and Europe, the debate in the artificial
intelligence community has focused on the unconscious biases of those
designing the technology. Recent tests showed facial recognition systems
made
\href{https://www.nytimes3xbfgragh.onion/2019/04/03/technology/amazon-facial-recognition-technology.html}{by
companies like I.B.M. and Amazon}
\href{https://www.nytimes3xbfgragh.onion/2019/01/24/technology/amazon-facial-technology-study.html}{were
less accurate} at identifying the features of darker-skinned people.

China's efforts raise starker issues. While facial recognition
technology uses aspects like skin tone and face shapes to sort images in
photos or videos, it must be told by humans to categorize people based
on social definitions of race or ethnicity. Chinese police, with the
help of the start-ups, have done that.

``It's something that seems shocking coming from the U.S., where there
is most likely racism built into our algorithmic decision making, but
not in an overt way like this,'' said Jennifer Lynch, surveillance
litigation director at the Electronic Frontier Foundation. ``There's not
a system designed to identify someone as African-American, for
example.''

The Chinese A.I. companies behind the software include Yitu, Megvii,
SenseTime, and CloudWalk, which are each valued at more than \$1
billion. Another company, Hikvision, that sells cameras and software to
process the images, offered a minority recognition function, but began
phasing it out in 2018, according to one of the people.

The companies' valuations soared in 2018 as China's Ministry of Public
Security, its top police agency, set aside billions of dollars under two
government plans, called Skynet and Sharp Eyes,
\href{https://www.nytimes3xbfgragh.onion/2018/07/08/business/china-surveillance-technology.html}{to
computerize surveillance}, policing and intelligence collection.

In a statement, a SenseTime spokeswoman said she checked with ``relevant
teams,'' who were not aware its technology was being used to profile.
Megvii said in a statement it was focused on ``commercial not political
solutions,'' adding, ``we are concerned about the well-being and safety
of individual citizens, not about monitoring groups.'' CloudWalk and
Yitu did not respond to requests for comment.

China's Ministry of Public Security did not respond to a faxed request
for comment.

Selling products with names like Fire Eye, Sky Eye and Dragonfly Eye,
the start-ups promise to use A.I. to analyze footage from China's
surveillance cameras. The technology is not mature --- in 2017 Yitu
promoted a one-in-three success rate when the police responded to its
alarms at a train station --- and many of China's cameras are not
powerful enough for facial recognition software to work effectively.

Yet they help advance China's architecture for social control. To make
the algorithms work, the police have put together face-image databases
for people with criminal records, mental illnesses, records of drug use,
and those who petitioned the government over grievances, according to
two of the people and procurement documents. A national database of
criminals at large includes about 300,000 faces, while a list of people
with a history of drug use in the city of Wenzhou totals 8,000 faces,
they said.

Image

A security camera in a rebuilt section of the Old City in Kashgar,
Xinjiang.Credit...Thomas Peter/Reuters

Using a process called machine learning, engineers feed data to
artificial intelligence systems to train them to recognize patterns or
traits. In the case of the profiling, they would provide thousands of
labeled images of both Uighurs and non-Uighurs. That would help generate
a function to distinguish the ethnic group.

The A.I. companies have taken money from major investors. Fidelity
International and Qualcomm Ventures were a part of a consortium that
invested \href{https://www.sensetime.com/news/682.html}{\$620 million}
in SenseTime. Sequoia invested in Yitu. Megvii is backed by Sinovation
Ventures, the fund of the well-known Chinese tech investor Kai-Fu Lee.

A Sinovation spokeswoman said the fund had recently sold a part of its
stake in Megvii and relinquished its seat on the board. Fidelity
declined to comment. Sequoia and Qualcomm did not respond to emailed
requests for comment.

Mr. Lee, a booster of Chinese A.I., has argued that China has an
advantage in developing A.I. because its leaders are less fussed by
``legal intricacies'' or ``moral consensus.''

``We are not passive spectators in the story of A.I. --- we are the
authors of it,'' Mr. Lee wrote last year. ``That means the values
underpinning our visions of an A.I. future could well become
self-fulfilling prophecies.'' He declined to comment on his fund's
investment in Megvii or its practices.

Ethnic profiling within China's tech industry isn't a secret, the people
said. It has become so common that one of the people likened it to the
short-range wireless technology Bluetooth. Employees at Megvii were
warned about the sensitivity of discussing ethnic targeting publicly,
another person said.

China has devoted major resources toward tracking Uighurs, citing ethnic
violence in Xinjiang and Uighur terrorist attacks elsewhere. Beijing has
thrown hundreds of thousands of Uighurs and others in Xinjiang
\href{https://www.nytimes3xbfgragh.onion/2018/09/08/world/asia/china-uighur-muslim-detention-camp.html}{into
re-education camps}.

The software extends the state's ability to label Uighurs to the rest of
the country. One national database stores the faces of all Uighurs who
leave Xinjiang, according to two of the people.

Government procurement documents from the past two years also show
demand has spread. In the city of Yongzhou in southern Hunan Province,
law enforcement officials sought software to ``characterize and search
whether or not someone is a Uighur,'' according to one document.

In two counties in Guizhou Province, the police listed a need for Uighur
classification. One asked for the ability to recognize Uighurs based on
identification photos at better than 97 percent accuracy. In the central
megacity of Chongqing and the region of Tibet, the police put out
tenders for similar software. And a procurement document for Hebei
Province described how the police should be notified when multiple
Uighurs booked the same flight on the same day.

A study in 2018 by the authorities described a use for other types of
databases. Co-written by a Shanghai police official, the
\href{https://image.hanspub.org/pdf/JSST20180300000_77144365.pdf}{paper}
said facial recognition systems installed near schools could screen for
people included in databases of the mentally ill or crime suspects.

One database generated by Yitu software and reviewed by The Times showed
how the police in the city of Sanmenxia used software running on cameras
to attempt to identify residents more than 500,000 times over about a
month beginning in mid-February.

Included in the code alongside tags like ``rec\_gender'' and
``rec\_sunglasses'' was ``rec\_uygur,'' which returned a 1 if the
software believed it had found a Uighur. Within the half million
identifications the cameras attempted to record, the software guessed it
saw Uighurs 2,834 times. Images stored alongside the entry would allow
the police to double check.

Yitu and its rivals have ambitions to expand overseas. Such a push could
easily put ethnic profiling software in the hands of other governments,
said Jonathan Frankle, an A.I. researcher at the Massachusetts Institute
of Technology.

``I don't think it's overblown to treat this as an existential threat to
democracy,'' Mr. Frankle said. ``Once a country adopts a model in this
heavy authoritarian mode, it's using data to enforce thought and rules
in a much more deep-seated fashion than might have been achievable 70
years ago in the Soviet Union. To that extent, this is an urgent crisis
we are slowly sleepwalking our way into.''

Image

An undercover police officer in Kashgar.Credit...Paul Mozur

Advertisement

\protect\hyperlink{after-bottom}{Continue reading the main story}

\hypertarget{site-index}{%
\subsection{Site Index}\label{site-index}}

\hypertarget{site-information-navigation}{%
\subsection{Site Information
Navigation}\label{site-information-navigation}}

\begin{itemize}
\tightlist
\item
  \href{https://help.nytimes3xbfgragh.onion/hc/en-us/articles/115014792127-Copyright-notice}{©~2020~The
  New York Times Company}
\end{itemize}

\begin{itemize}
\tightlist
\item
  \href{https://www.nytco.com/}{NYTCo}
\item
  \href{https://help.nytimes3xbfgragh.onion/hc/en-us/articles/115015385887-Contact-Us}{Contact
  Us}
\item
  \href{https://www.nytco.com/careers/}{Work with us}
\item
  \href{https://nytmediakit.com/}{Advertise}
\item
  \href{http://www.tbrandstudio.com/}{T Brand Studio}
\item
  \href{https://www.nytimes3xbfgragh.onion/privacy/cookie-policy\#how-do-i-manage-trackers}{Your
  Ad Choices}
\item
  \href{https://www.nytimes3xbfgragh.onion/privacy}{Privacy}
\item
  \href{https://help.nytimes3xbfgragh.onion/hc/en-us/articles/115014893428-Terms-of-service}{Terms
  of Service}
\item
  \href{https://help.nytimes3xbfgragh.onion/hc/en-us/articles/115014893968-Terms-of-sale}{Terms
  of Sale}
\item
  \href{https://spiderbites.nytimes3xbfgragh.onion}{Site Map}
\item
  \href{https://help.nytimes3xbfgragh.onion/hc/en-us}{Help}
\item
  \href{https://www.nytimes3xbfgragh.onion/subscription?campaignId=37WXW}{Subscriptions}
\end{itemize}
