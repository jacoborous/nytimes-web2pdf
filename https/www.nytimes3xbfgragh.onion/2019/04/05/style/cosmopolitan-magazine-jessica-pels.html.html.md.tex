Sections

SEARCH

\protect\hyperlink{site-content}{Skip to
content}\protect\hyperlink{site-index}{Skip to site index}

\href{https://www.nytimes3xbfgragh.onion/section/style}{Style}

\href{https://myaccount.nytimes3xbfgragh.onion/auth/login?response_type=cookie\&client_id=vi}{}

\href{https://www.nytimes3xbfgragh.onion/section/todayspaper}{Today's
Paper}

\href{/section/style}{Style}\textbar{}At Cosmopolitan Magazine, Data Is
the New Sex

\begin{itemize}
\item
\item
\item
\item
\item
\item
\end{itemize}

Advertisement

\protect\hyperlink{after-top}{Continue reading the main story}

Supported by

\protect\hyperlink{after-sponsor}{Continue reading the main story}

\hypertarget{at-cosmopolitan-magazine-data-is-the-new-sex}{%
\section{At Cosmopolitan Magazine, Data Is the New
Sex}\label{at-cosmopolitan-magazine-data-is-the-new-sex}}

Jessica Pels, the editor, is trying to save the magazine from the jaws
of Instagram.

\href{https://www.nytimes3xbfgragh.onion/by/katherine-rosman}{\includegraphics{https://static01.graylady3jvrrxbe.onion/images/2019/03/22/multimedia/author-katherine-rosman/author-katherine-rosman-thumbLarge-v2.png}}

By
\href{https://www.nytimes3xbfgragh.onion/by/katherine-rosman}{Katherine
Rosman}

\begin{itemize}
\item
  April 5, 2019
\item
  \begin{itemize}
  \item
  \item
  \item
  \item
  \item
  \item
  \end{itemize}
\end{itemize}

\includegraphics{https://static01.graylady3jvrrxbe.onion/images/2019/04/06/fashion/06COSMOEDITOR-1/merlin_150992718_717204d9-868a-49ae-919c-aee4798089b4-articleLarge.jpg?quality=75\&auto=webp\&disable=upscale}

``Bad ideas first!'' is how Jessica Pels began a session brainstorming
cover lines for the May issue of Cosmopolitan. She was named editor in
chief at the Hearst Magazines publication last fall.

Ms. Pels's legs dangled as she perched on a credenza on one end of the
long conference room and revealed to some 40 staff members from various
departments, including video and social media, that
\href{https://www.nytimes3xbfgragh.onion/2018/01/09/arts/television/grown-ish-review-freeform.html}{Yara
Shahidi}, a star of the sitcom ``Black-ish,'' would be appearing on the
cover, alongside \href{https://www.instagram.com/melton/?hl=en}{Charles
Melton}, who plays Reggie on ``Riverdale.'' They are both in the
forthcoming movie ``The Sun Is Also a Star.''

Cosmopolitan, which has the highest circulation of any Hearst magazine,
was taken from a sleepy literary journal to a sensational pro-sex
feminist magazine by its longtime editor,
\href{https://www.nytimes3xbfgragh.onion/2012/08/14/business/media/helen-gurley-brown-who-gave-cosmopolitan-its-purr-is-dead-at-90.html}{Helen
Gurley Brown}, who worked there from 1965 to 1997.
\href{https://www.nytimes3xbfgragh.onion/2015/08/23/fashion/helen-gurley-brown-cosmopolitan-editor-hearst-legacy.html}{She
stepped down} at age 74 and became editor in chief of Cosmo's
international editions until her death in 2012 at 90.

Cover lines, long thought to compel buyers to pluck a magazine off a
crowded newsstand, were always a main ingredient of Ms. Brown's success.
Hers (often written by her husband, the Hollywood producer David Brown)
were especially breathy and enticing:
``\href{https://www.cnn.com/2012/08/17/living/helen-gurley-brown-legacy/index.html}{World's
Greatest Lover} --- what it was like to be wooed by him!''

Now, cover lines are mere adornment to the print product --- something
that may be thought of as a loss leader for a brand aimed at women aged
18 to 34, possibly the most mobile-phone-obsessed demo there is. Ms.
Pels hopes Hearst will come up with a way to easily let readers
subscribe by text and pay with Venmo.

``She opens Instagram 42 times a day,'' said Ms. Pels of the Cosmo
reader. ``Anything she can do on her phone, she will.''

Addressing her staff, Ms. Pels stressed that the cover-line language
shouldn't be too sexy because Ms. Shahidi is just 19 and Mr. Melton is
nearly a decade older. She wanted to avoid doing the film's marketing by
presenting them as a couple with chemistry.

Moreover, in real life Mr. Melton is dating
\href{https://www.instagram.com/camimendes/?hl=en}{Camila Mendes}, who
plays Veronica on ``Riverdale.'' ``We don't want to lean out of that, we
want the Cami-stans to want to pick it up,'' one editor piped in. (For
those over the age of 40: a ``stan'' is
\href{https://www.nytimes3xbfgragh.onion/2011/10/06/fashion/scratching-the-celebrity-itch.html}{a
kind of}superfan.)

The ideas started flying. Many made Ms. Pels laugh.

``Win a night out with Yara and Charles. *JK: they're not dating, you're
not going, it's not happening.''

``Yara Shahidi and Charles Melton: The girl you want to be and the boy
you want to \ldots{}''

The goal, Ms. Pels said, ``is to call them the prom king and queen of
young Hollywood without calling them the prom king and queen. They're a
new paradigm in Hollywood,'' she said.

The cover line she ultimately went with was ``Hollywood just got
hotter.''

\hypertarget{hans-and-stans}{%
\subsection{Hans and Stans}\label{hans-and-stans}}

Ms. Pels herself is representative of a new paradigm. At 32, she is at
least 10 years younger than any of her recent predecessors. And though
she likely draws a smaller paycheck than those who came before her, she
has more digital media experience than most of them combined.

Image

Paging: Ms. Pels looks over the layouts for an issue of the
magazine.Credit...Amy Lombard for The New York Times

She grew up in East Cobb, Ga., attended the Tisch School of the Arts at
New York University and now lives in Brooklyn with her husband. While
she was a student, she had internships at The New Yorker and Vogue,
before graduating in 2008 with a degree in film and video production.
She was hired by Condé Nast, first at Glamour and then Teen Vogue,
before landing at Hearst.

For four years Ms. Pels headed MarieClaire.com, starting with a staff of
two that grew to about 11 by the time she got the same job at Cosmo.

Ms. Pels ran digital platforms there for 10 months before being granted
the entire kit and caboodle: the print magazine, the website, all the
social channels, video platforms and
``\href{https://www.cosmopolitan.com/style-beauty/fashion/a26795655/how-to-use-stitch-fix/}{branded
content}'' (what used to be known as advertorial).

With Nancy Berger, the new publisher of Cosmo, she is working on
licensing, e-commerce, events and partnerships. ``Anytime I can raise my
hand and lean into something new that the brand is doing, I want to do
that,'' Ms. Pels said. (``Leaning in'' and ``out'' is a big thing at
Cosmo.)

She can spout data points as easily as Ms. Brown would sex tips. She has
a big office with a large glass desk that is nearly bare but for a
computer, a notebook and a pair of Apple earbuds. All she really needs
are the computer and her phone, which she checks a lot for
up-to-the-minute Cosmo analytics.

Sitting at her desk in a plaid suit she got at Macy's and a pair of high
heels impractical for walking, she gazed at a graph showing the previous
week's web traffic. She pointed to a spike that showed that one day the
site had 4.28 million unique visitors. ``It's such a pretty sight,'' she
said.

By the end of February, Cosmopolitan.com had logged 41 million unique
visitors, according to comScore. In February 2018, the month after Ms.
Pels
\href{https://www.adweek.com/digital/how-cosmopolitans-digital-director-is-opening-doors-to-new-coverage/}{started
as the top editor}, there were 15 million.

Digital subscriptions from December 2016 to December 2018 have increased
from 85,060 to 242,075, according to the Alliance for Audited Media.

E-commerce on the site has doubled in the last year, Hearst said. Cosmo
readers like to buy small-ticket items from the site, which takes the
tone of a girlfriend making recommendations. Books, streamed movies,
bras and vibrators sell particularly well from the site, Ms. Pels said.
Hearst earns affiliate revenue from items that readers buy via Cosmo's
website.

``Jess is nailing it on digital, and she knows how to connect with the
audience in print too,'' said Ms. Berger, who previously worked as
publisher of Marie Claire. ``What is happening at Cosmo right now is the
new way forward.''

But can Ms. Pels save the print edition of Cosmo from the fate of her
alma maters Glamour and Teen Vogue, whose regular print editions have
been killed? Not to mention Cosmo's little sister CosmoGirl, which
\href{https://wwd.com/business-news/media/cosmogirl-to-close-1832842/}{folded}
in 2008?

Single-copy print sales, a metric that refers mostly to airport and
newsstand sales, dropped from 2016 to 2018 from 274,833 to 123,250, the
Alliance for Audited Media reported. (Paid print subscriptions have
increased slightly, to 2.67 million from 2.64 million.)

Kate Lewis, the chief content officer at Hearst Magazines and Ms. Pels's
biggest booster, said she expects Cosmo to continue as a print-based
publication for at least the next five years. And these days overall
revenue, which includes advertorial and e-commerce, is more important to
the company than newsstand sales.

Image

The well-hydrated staff of Cosmopolitan gathers to consider cover lines
for the May issue.Credit...Amy Lombard for The New York Times

Ms. Pels's ascendence at Cosmo has come amid a changing of the guard at
the magazine company. In July 2018,
\href{https://www.nytimes3xbfgragh.onion/2020/07/22/business/media/hearst-harassment-troy-young.html}{Troy
Young} was named president of Hearst Magazines, after more than five
years as the company's head digital executive. (He replaced David Carey,
who had served as president for eight years.)

Mr. Young has provided for the company an internal data analyzing
product that staffers refer to as Hans (``Hearst answers''). Staffers
can enter a question into a Hans Slack channel, asking for
sliced-and-diced information as they might a salad at the airy company
cafeteria.

Ms. Pels, for example, might message the Hans bot asking what is the
day's best-performing e-commerce item or which print magazine story is
drawing the highest numbers online at a particular time of day.

``And he will answer, which is really cool,'' she said.

For all the meeting time spent on print cover lines, they are now far
less important than trying to create a social-media moment that will
bring readers to the Cosmo website.

The April cover features the four women who will star in the coming
reboot of ``The Hills,'' the MTV hit, posed in a vintage metallic powder
blue car. To ``launch the cover,'' Ms. Pels and her team called upon
Spencer Pratt, a love-to-hate-him figure from the original show, who is
married to Heidi Montag, a star of the original and the reboot.

In a livestreamed video, Mr. Pratt sits in his kitchen trying to
\href{https://www.youtube.com/watch?v=l-dnrweE4sA}{read} the entire
issue of Cosmo, front to back, to his
\href{https://www.youtube.com/watch?v=l-dnrweE4sA}{son}, who eventually
toddles himself out of the situation.

Image

May 2019 cover of Cosmopolitan magazine.Credit...Eric Ray Davidson

For the March cover, the actress Lana Condor was photographed grabbing a
slice from a heart-shaped pizza. As the issue was being released, Cosmo
sent Ms. Condor a heart-shaped pizza and tipped off her boyfriend,
Anthony De La Torre, so he could record the moment and Ms. Condor could
share it. (Cosmo also sent the issue and heart pizzas to well-followed
``Lana-stan'' Instagram accounts.)

\hypertarget{instagram-inspo}{%
\subsection{Instagram Inspo}\label{instagram-inspo}}

Back in the day, Cosmo covers were known for the wind-machined coifs,
contoured cheekbones and plunging cleavage favored by one signature
photographer,
\href{https://www.nytimes3xbfgragh.onion/2004/01/07/arts/francesco-scavullo-fashion-photographer-dies-at-82.html}{Francesco
Scavullo}. The next era brought fewer models and more actresses,
standing against a solid-color backdrop.

Ms. Pels, working with the magazine's new creative director, Andy
Turnbull, wanted the covers to look more like Instagram photos.

So she has data pulled that identifies ``top -performing Instagrams,''
which then inspire Cosmo photo shoots. As it happens, heart-shaped pizza
is popular on Insta. As are vintage metallic blue cars. As are bathtubs,
which can be seen in the background of Ms. Condor's photo spread.
``We're using Instagram to back into our planning print shoots,'' Ms.
Pels said.

From Hans, Ms. Pels and her staff have learned a lot about what sort of
articles website visitors prefer and what drives them to click. They
also collect data through polls, lots of them, couched as simple newsy
questions like
``\href{https://www.cosmopolitan.com/entertainment/a26998326/kylie-jenner-admits-not-self-made-billionaire/}{Do
you think that Kylie is self-made?}'' and
``\href{https://www.cosmopolitan.com/entertainment/celebs/a26930204/yale-rescinds-admission-student-college-bribery-scandal/}{Do
you think the students who were accepted under false pretenses should
have their admissions revoked?}''

The typical reader loves astrology, so there is now a multipage
astrology section in every issue. (A regular column is called ``Ask Your
Astrolobestie.'') She favors true crime stories and celebrity
interviews, but also wants to be in the know about issues important to
young women, hence substantive features like
``\href{https://www.cosmopolitan.com/politics/a26026217/sexual-abstinence-joshua-harris-purity-movement-scam/}{Inside
the Scam of the `Purity Movement}.'''

And, obvz, sex, a Cosmo staple for 50 years, is not going anywhere. But
Ms. Pels's idea of sex is less ``heteronormative,'' she said. In March,
there was a two-page spread on how to give oral sex headlined ``Get in
Formation,'' with explicit drawings of two entwined figures. One is
clearly a woman, the other's gender is not obvious.

Though the magazine's staff is not strikingly diverse, it strives for
fair representation of models and article subjects; one recent story was
``\href{https://www.cosmopolitan.com/sex-love/positions/g26723862/disability-sex-positions/}{5
Disability-Friendly Sex Positions You Need in Your Life.}''

Covering web culture is paramount, and recent stories have included
``What Your Go-To Selfie Face Says About You'' and a granular analysis
of stalking habits. (''I was out at a bar one night going through my
Snap feed when I saw this guy who ghosted me in the background of a
friend's Story.'')

The voice Ms. Pels wants, she said, is ``your texts with your
girlfriends on your funniest day.''

She insisted that she is guided not just by data, but also instinct and
affinity. ``I am her,'' she said. ``I'm smack-dab in our demo. I think
she's bold. I think she's unapologetic about having fun. I think she
wants to have an impact on the world around her, and that she should.''

\hypertarget{the-news-cycle}{%
\subsection{The News Cycle}\label{the-news-cycle}}

Cindi Leive, the former editor of Glamour who was Ms. Pels's boss there,
said that her onetime assistant was from the start an overachiever who
greatly exceeded expectations.

When Ms. Leive asked her to help find authors for a Glamour book of
essays, she delivered Maya Angelou. When Ms. Leive asked her to round up
some motivational quotes for a speech Ms. Leive would be delivering, Ms.
Pels managed to get original material from Jimmy Fallon.

``I would have her edit my editor's letters not as an act of charity or
to make her feel good,'' Ms. Leive said. ``I did it because she made the
pieces better.''

Image

Let's discuss: the Jordyn Woods-Tristan Thompson drama.Credit...Amy
Lombard for The New York Times

At MarieClaire.com, Ms. Lewis, a former Condé Nast human resources
executive who had recruited Ms. Pels to Hearst, was impressed when Ms.
Pels came to her asking for money to produce an ambitious package about
\href{https://www.marieclaire.com/politics/a18016/women-and-guns/}{women
and guns}. ``It became clear at that moment that she understood video in
a way that was special,'' she said.

As Cosmo's digital director, Ms. Pels arranged to get
\href{https://www.cosmopolitan.com/politics/a19183766/video-parkland-marjory-stoneman-douglas-teachers-first-week-back/}{video
footage and interviews} in Parkland, Fla., the week when students
returned to Marjory Stoneman Douglas High School after the mass shooting
in February 2018.

The cover-lines meeting, however, fell on a different kind of news day.
The main item was that Khloé Kardashian, the 34-year-old star of reality
and talk shows like ``Khloé \& Lamar'' and ``Kocktails With Khloé,'' had
broken up with Tristan Thompson, an N.B.A. center for the Cleveland
Cavaliers with whom Khloé had a baby girl last April, named True.

Ms. Kardashian,
\href{https://www.tmz.com/2019/02/19/khloe-kardashian-splits-tristan-thompson-cheating-kylie-jenner-jordyn-woods/}{TMZ
had reported}, had learned that Mr. Thompson had hooked up with Jordyn
Woods, 21, who became famous for being Kardashian-adjacent: As followers
of Kylie Jenner's Instagram know, Ms. Woods is/was BFF with Kylie, whose
half sister is Khloé.

The story quickly became Cosmo's Mueller report. As it happened, the
magazine had in its current issue a multipage feature on Ms. Woods,
along with original photographs of her, and Cosmo staffers
\href{https://www.cosmopolitan.com/uk/search/?q=jordyn+woods}{flooded
the zone} with new articles. One of its
\href{https://www.cosmopolitan.com/entertainment/a26420650/kardashians-unfollow-tristan-thompson-jordyn-woods-instagram/}{best-read
stories} tracked --- with frequent updates and screen-shotted visual
evidence --- which Kardashians had unfollowed Ms. Woods and Mr. Thompson
on Instagram.

By the end of the day, nearly 4.3 million readers had come to the site,
nearly doubling the traffic from the previous best day (Feb. 11, after
the Grammy Awards).

As soon as the meeting ended, Ms. Pels stepped into the hall, checked
the data on her phone and kicked one heel insouciantly behind her, a fun
fearless female.

``It's going to be our biggest day ever,'' she said.

Advertisement

\protect\hyperlink{after-bottom}{Continue reading the main story}

\hypertarget{site-index}{%
\subsection{Site Index}\label{site-index}}

\hypertarget{site-information-navigation}{%
\subsection{Site Information
Navigation}\label{site-information-navigation}}

\begin{itemize}
\tightlist
\item
  \href{https://help.nytimes3xbfgragh.onion/hc/en-us/articles/115014792127-Copyright-notice}{©~2020~The
  New York Times Company}
\end{itemize}

\begin{itemize}
\tightlist
\item
  \href{https://www.nytco.com/}{NYTCo}
\item
  \href{https://help.nytimes3xbfgragh.onion/hc/en-us/articles/115015385887-Contact-Us}{Contact
  Us}
\item
  \href{https://www.nytco.com/careers/}{Work with us}
\item
  \href{https://nytmediakit.com/}{Advertise}
\item
  \href{http://www.tbrandstudio.com/}{T Brand Studio}
\item
  \href{https://www.nytimes3xbfgragh.onion/privacy/cookie-policy\#how-do-i-manage-trackers}{Your
  Ad Choices}
\item
  \href{https://www.nytimes3xbfgragh.onion/privacy}{Privacy}
\item
  \href{https://help.nytimes3xbfgragh.onion/hc/en-us/articles/115014893428-Terms-of-service}{Terms
  of Service}
\item
  \href{https://help.nytimes3xbfgragh.onion/hc/en-us/articles/115014893968-Terms-of-sale}{Terms
  of Sale}
\item
  \href{https://spiderbites.nytimes3xbfgragh.onion}{Site Map}
\item
  \href{https://help.nytimes3xbfgragh.onion/hc/en-us}{Help}
\item
  \href{https://www.nytimes3xbfgragh.onion/subscription?campaignId=37WXW}{Subscriptions}
\end{itemize}
