Sections

SEARCH

\protect\hyperlink{site-content}{Skip to
content}\protect\hyperlink{site-index}{Skip to site index}

\href{https://www.nytimes3xbfgragh.onion/section/your-money}{Your Money}

\href{https://myaccount.nytimes3xbfgragh.onion/auth/login?response_type=cookie\&client_id=vi}{}

\href{https://www.nytimes3xbfgragh.onion/section/todayspaper}{Today's
Paper}

\href{/section/your-money}{Your Money}\textbar{}A Journey of 1,000 Miles
Begins With the Current Step, Not the Next One

\url{https://nyti.ms/2vau1Du}

\begin{itemize}
\item
\item
\item
\item
\item
\end{itemize}

Advertisement

\protect\hyperlink{after-top}{Continue reading the main story}

Supported by

\protect\hyperlink{after-sponsor}{Continue reading the main story}

\href{/column/sketch-guy}{Sketch Guy}

\hypertarget{a-journey-of-1000-miles-begins-with-the-current-step-not-the-next-one}{%
\section{A Journey of 1,000 Miles Begins With the Current Step, Not the
Next
One}\label{a-journey-of-1000-miles-begins-with-the-current-step-not-the-next-one}}

\includegraphics{https://static01.graylady3jvrrxbe.onion/images/2019/04/15/business/041519bucks-carl-sketch/041519bucks-carl-sketch-articleLarge.jpg?quality=75\&auto=webp\&disable=upscale}

By \href{https://www.nytimes3xbfgragh.onion/by/carl-richards}{Carl
Richards}

\begin{itemize}
\item
  April 17, 2019
\item
  \begin{itemize}
  \item
  \item
  \item
  \item
  \item
  \end{itemize}
\end{itemize}

As a culture, we tend to be obsessed over the next step*.* But how often
do we hear people talking about the current one?

My guess is not a lot, even though we have to take the current step
before taking the next one. So let me tell you a story.

I used to ride my bicycle a lot on roads in the American West. I was
obsessed, and I was even pretty decent at it. My friends and I used to
do a long race from Logan, Utah, over three mountain passes to Jackson,
Wyo. It was just over 200 miles, and we used to do it in 10 to 12 hours.

One year, the night before the race, I got wretchedly sick and was
throwing up all night. By morning, it was clear I could not ride the
whole way. But the thing about these races is that they're team events.
One of the ways you can help the team is simply ride in the front of
your pack and bear the brunt of the wind so your teammates can draft off
you and save energy.

There was a spot about 30 miles into the race where I knew I could exit
the course, so I decided to make it just that far and use all my energy
helping my teammates for that stretch.

Even 30 miles sounded like an awful lot to me that morning. But I had a
coach who had been training me leading up to the race, and he was always
trying to get me to focus only on the current breath.

The inhalation going in right now. The exhalation going out right now.

Then, we'd translate that way of thinking to the pedal stroke. He taught
me how to ride one stroke at a time --- how to focus on the current
stroke.

I had no aspirations of finishing the race. I had no focus on the end
goal. Instead, I just did what my coach taught me and rode completely
focused on the current moment. All of a sudden, I looked up and noticed
that I was at the 30-mile mark. I was feeling pretty good, so I decided
not to quit.

``I'll just go to the next break,'' I thought.

As you may have guessed, I didn't get off at the next break, either. Or
the next one. Or the one after that. I just kept going, one stroke at a
time, until the race was over. In the end, somehow, it ended up being my
best-ever performance in that race.

Was it a fluke? A freak accident? Or maybe, just maybe, was there
something to what my coach had taught me?

If I had focused on the next steps --- and crossing the finish line for
my fastest time ever --- I never would have started the race in the
first place. But when I focused on the process at the most basic level,
everything clicked.

Of course, this isn't just a biking story. This could apply to business,
a relationship, a big project. It doesn't matter what we're talking
about, because one fact is always the same.

The current step comes first. You have to focus here before you focus
there.

Advertisement

\protect\hyperlink{after-bottom}{Continue reading the main story}

\hypertarget{site-index}{%
\subsection{Site Index}\label{site-index}}

\hypertarget{site-information-navigation}{%
\subsection{Site Information
Navigation}\label{site-information-navigation}}

\begin{itemize}
\tightlist
\item
  \href{https://help.nytimes3xbfgragh.onion/hc/en-us/articles/115014792127-Copyright-notice}{©~2020~The
  New York Times Company}
\end{itemize}

\begin{itemize}
\tightlist
\item
  \href{https://www.nytco.com/}{NYTCo}
\item
  \href{https://help.nytimes3xbfgragh.onion/hc/en-us/articles/115015385887-Contact-Us}{Contact
  Us}
\item
  \href{https://www.nytco.com/careers/}{Work with us}
\item
  \href{https://nytmediakit.com/}{Advertise}
\item
  \href{http://www.tbrandstudio.com/}{T Brand Studio}
\item
  \href{https://www.nytimes3xbfgragh.onion/privacy/cookie-policy\#how-do-i-manage-trackers}{Your
  Ad Choices}
\item
  \href{https://www.nytimes3xbfgragh.onion/privacy}{Privacy}
\item
  \href{https://help.nytimes3xbfgragh.onion/hc/en-us/articles/115014893428-Terms-of-service}{Terms
  of Service}
\item
  \href{https://help.nytimes3xbfgragh.onion/hc/en-us/articles/115014893968-Terms-of-sale}{Terms
  of Sale}
\item
  \href{https://spiderbites.nytimes3xbfgragh.onion}{Site Map}
\item
  \href{https://help.nytimes3xbfgragh.onion/hc/en-us}{Help}
\item
  \href{https://www.nytimes3xbfgragh.onion/subscription?campaignId=37WXW}{Subscriptions}
\end{itemize}
