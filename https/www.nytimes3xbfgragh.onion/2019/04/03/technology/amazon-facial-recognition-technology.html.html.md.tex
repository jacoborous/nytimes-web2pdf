Sections

SEARCH

\protect\hyperlink{site-content}{Skip to
content}\protect\hyperlink{site-index}{Skip to site index}

\href{https://www.nytimes3xbfgragh.onion/section/technology}{Technology}

\href{https://myaccount.nytimes3xbfgragh.onion/auth/login?response_type=cookie\&client_id=vi}{}

\href{https://www.nytimes3xbfgragh.onion/section/todayspaper}{Today's
Paper}

\href{/section/technology}{Technology}\textbar{}A.I. Experts Question
Amazon's Facial-Recognition Technology

\url{https://nyti.ms/2Uqin5E}

\begin{itemize}
\item
\item
\item
\item
\item
\end{itemize}

Advertisement

\protect\hyperlink{after-top}{Continue reading the main story}

Supported by

\protect\hyperlink{after-sponsor}{Continue reading the main story}

\hypertarget{ai-experts-question-amazons-facial-recognition-technology}{%
\section{A.I. Experts Question Amazon's Facial-Recognition
Technology}\label{ai-experts-question-amazons-facial-recognition-technology}}

\includegraphics{https://static01.graylady3jvrrxbe.onion/images/2019/04/03/business/03amazonletter1/merlin_146148123_26da2645-ae5a-4651-aef5-b8736ba6698b-articleLarge.jpg?quality=75\&auto=webp\&disable=upscale}

By \href{https://www.nytimes3xbfgragh.onion/by/cade-metz}{Cade Metz} and
\href{https://www.nytimes3xbfgragh.onion/by/natasha-singer}{Natasha
Singer}

\begin{itemize}
\item
  April 3, 2019
\item
  \begin{itemize}
  \item
  \item
  \item
  \item
  \item
  \end{itemize}
\end{itemize}

SAN FRANCISCO --- At least 25 prominent artificial-intelligence
researchers, including experts at Google, Facebook, Microsoft and a
recent winner of
\href{https://www.nytimes3xbfgragh.onion/2019/03/27/technology/turing-award-ai.html}{the
prestigious Turing Award}, have signed a letter calling on Amazon to
stop selling its facial-recognition technology to law enforcement
agencies because it is biased against women and people of color.

The letter, which was publicly released Wednesday, reflects growing
concern in academia and the tech industry that bias in
facial-recognition technology is a systemic problem. Some researchers
--- and even some companies --- are arguing the technology cannot be
properly controlled without government regulation.

Amazon sells a product called Rekognition through its cloud-computing
division, Amazon Web Services. The company said last year that early
customers included the Orlando Police Department in Florida and the
Washington County Sheriff's Office in Oregon.

In January, two researchers at the Massachusetts Institute of Technology
\href{https://www.nytimes3xbfgragh.onion/2019/01/24/technology/amazon-facial-technology-study.html}{published
a peer-reviewed study} showing that Amazon Rekognition had more trouble
identifying the gender of female and darker-skinned faces in photos than
similar services from IBM and Microsoft. It mistook women for men 19
percent of the time, the study showed, and misidentified darker-skinned
women for men 31 percent of the time.

Before publishing their findings on Amazon Rekognition, the M.I.T.
researchers released a similar study examining services from IBM,
Microsoft and Megvii, an artificial-intelligence company in China. All
three updated their services to address concerns raised by the
researchers.

In separate blog posts from the Amazon executives
\href{https://aws.amazon.com/blogs/machine-learning/thoughts-on-recent-research-paper-and-associated-article-on-amazon-rekognition/}{Matthew
Wood} and
\href{https://aws.amazon.com/blogs/machine-learning/some-thoughts-on-facial-recognition-legislation/}{Michael
Punke}, the company disputed the study and
\href{https://www.nytimes3xbfgragh.onion/2019/01/24/technology/amazon-facial-technology-study.html}{a
Jan. 24 article} in The New York Times describing it.

``The answer to anxieties over new technology is not to run `tests'
inconsistent with how the service is designed to be used, and to amplify
the test's false and misleading conclusions through the news media,''
Dr. Wood wrote. Amazon did not directly engage with the M.I.T.
researchers.

The letter released on Wednesday was signed by the Google researchers
Margaret Mitchell, Andrea Frome and Timnit Gebru; the Facebook
researcher Georgia Gkioxari; William Isaac, a researcher at DeepMind,
the London lab owned by Google's parent company, Alphabet; and Yoshua
Bengio, one of the world's most important A.I. researchers.

Last week, Dr. Bengio was one of three people to receive the Turing
Award --- often called ``the Nobel Prize of computing'' --- for his work
with
\href{https://www.nytimes3xbfgragh.onion/2018/03/06/technology/google-artificial-intelligence.html}{neural
networks}, the technology that underpins modern facial recognition
services.

``There are no laws or required standards to ensure that Rekognition is
used in a manner that does not infringe on civil liberties,'' the A.I.
researchers wrote. ``We call on Amazon to stop selling Rekognition to
law enforcement.''

The researchers added that Dr. Wood and Mr. Punke had ``misrepresented
the technical details'' of the M.I.T. study and modern
facial-recognition technology. Amazon declined to comment on the letter
on Wednesday.

A day after this article was published, an Amazon spokeswoman responded,
saying that the company had updated its Rekognition service since the
M.I.T. researchers completed their study and that it had found no
differences in error rates by gender and race when running similar
tests.

Microsoft, by contrast,
\href{https://blogs.microsoft.com/ai/gender-skin-tone-facial-recognition-improvement/}{improved
the accuracy} of its facial recognition last year after an earlier
M.I.T. study reported that its system was better at identifying the
gender of lighter-skinned men in a photo database than darker-skinned
women.

During a February talk at the Cornell Tech graduate school in New York,
Brad Smith, Microsoft's president and chief legal officer, said the
company had ``participated in the market for law enforcement in the
United States,'' but had also turned down sales when there was concern
it could unreasonably infringe on people's rights.

In February, Microsoft backed a bill in Washington State that would
require notices to be posted in public places using facial-recognition
tech and ensure that government agencies obtain a court order when
looking for specific people. The bill is still pending. But the company
did not back other legislation that provides much stronger protections.

Mr. Punke wrote in his February blog post that Amazon also supported
regulation of facial-recognition technology and called for law
enforcement agencies to ``be transparent in how they use
facial-recognition technology.'' But Amazon has declined to disclose how
police or intelligence agencies are using its Rekognition system and
whether the company puts any restrictions on its use.

Amazon has said that it has not received any reports of Rekognition
misuse by law enforcement, and that the company's acceptable use policy
prohibits customers from using its services in ways that violate laws.

Advertisement

\protect\hyperlink{after-bottom}{Continue reading the main story}

\hypertarget{site-index}{%
\subsection{Site Index}\label{site-index}}

\hypertarget{site-information-navigation}{%
\subsection{Site Information
Navigation}\label{site-information-navigation}}

\begin{itemize}
\tightlist
\item
  \href{https://help.nytimes3xbfgragh.onion/hc/en-us/articles/115014792127-Copyright-notice}{©~2020~The
  New York Times Company}
\end{itemize}

\begin{itemize}
\tightlist
\item
  \href{https://www.nytco.com/}{NYTCo}
\item
  \href{https://help.nytimes3xbfgragh.onion/hc/en-us/articles/115015385887-Contact-Us}{Contact
  Us}
\item
  \href{https://www.nytco.com/careers/}{Work with us}
\item
  \href{https://nytmediakit.com/}{Advertise}
\item
  \href{http://www.tbrandstudio.com/}{T Brand Studio}
\item
  \href{https://www.nytimes3xbfgragh.onion/privacy/cookie-policy\#how-do-i-manage-trackers}{Your
  Ad Choices}
\item
  \href{https://www.nytimes3xbfgragh.onion/privacy}{Privacy}
\item
  \href{https://help.nytimes3xbfgragh.onion/hc/en-us/articles/115014893428-Terms-of-service}{Terms
  of Service}
\item
  \href{https://help.nytimes3xbfgragh.onion/hc/en-us/articles/115014893968-Terms-of-sale}{Terms
  of Sale}
\item
  \href{https://spiderbites.nytimes3xbfgragh.onion}{Site Map}
\item
  \href{https://help.nytimes3xbfgragh.onion/hc/en-us}{Help}
\item
  \href{https://www.nytimes3xbfgragh.onion/subscription?campaignId=37WXW}{Subscriptions}
\end{itemize}
