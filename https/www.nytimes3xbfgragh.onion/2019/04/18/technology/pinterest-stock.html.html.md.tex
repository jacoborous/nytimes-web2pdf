Sections

SEARCH

\protect\hyperlink{site-content}{Skip to
content}\protect\hyperlink{site-index}{Skip to site index}

\href{https://www.nytimes3xbfgragh.onion/section/technology}{Technology}

\href{https://myaccount.nytimes3xbfgragh.onion/auth/login?response_type=cookie\&client_id=vi}{}

\href{https://www.nytimes3xbfgragh.onion/section/todayspaper}{Today's
Paper}

\href{/section/technology}{Technology}\textbar{}I.P.O. Day for Pinterest
and Zoom Ends With Shares Sharply Higher

\url{https://nyti.ms/2Zk1Xez}

\begin{itemize}
\item
\item
\item
\item
\item
\end{itemize}

Advertisement

\protect\hyperlink{after-top}{Continue reading the main story}

Supported by

\protect\hyperlink{after-sponsor}{Continue reading the main story}

\hypertarget{ipo-day-for-pinterest-and-zoom-ends-with-shares-sharply-higher}{%
\section{I.P.O. Day for Pinterest and Zoom Ends With Shares Sharply
Higher}\label{ipo-day-for-pinterest-and-zoom-ends-with-shares-sharply-higher}}

\includegraphics{https://static01.graylady3jvrrxbe.onion/images/2019/04/17/business/00pinterestlisting-1/merlin_152478903_86e14783-fdd9-410c-a1e2-891cd5e4d3eb-articleLarge.jpg?quality=75\&auto=webp\&disable=upscale}

By \href{https://www.nytimes3xbfgragh.onion/by/erin-griffith}{Erin
Griffith}

\begin{itemize}
\item
  April 18, 2019
\item
  \begin{itemize}
  \item
  \item
  \item
  \item
  \item
  \end{itemize}
\end{itemize}

SAN FRANCISCO --- The rush of so-called unicorn start-ups toward the
public markets had a rocky start. But Thursday indicated that investors
remain eager to get a piece of them.

Shares in Pinterest, the digital pin board, jumped over 28 percent on
its first day of trading as a public company. The company's stock began
trading at \$23.75, above the initial public offering price of \$19, and
finished the day at \$24.40.

The company's fully diluted market capitalization totaled over \$16
billion, making it more valuable than Macy's or Nordstrom, the retail
chains. More important to investors, the price put the company's value
above its last private valuation of \$12 billion, avoiding a
disappointing outcome.

Zoom, a videoconferencing company, also went public to tremendous
investor demand on Thursday. Shares in the company, which was last
valued by private investors at \$1 billion --- the threshold for unicorn
status among private start-ups --- skyrocketed 80 percent in early
trading. The shares ended the day up more than 72 percent, closing at
\$62. The company's fully diluted market capitalization now exceeds \$18
billion.

Eric Yuan, chief executive and founder of Zoom, said the spike created
pressure for his company to deliver on investors' high expectations.

``I looked at the price this morning and I thought, `Wow, I better go
back tonight to get back to work,''' he said.

Ahead of the companies' I.P.O.s, there were many questions about whether
investors were willing to swallow the risk of the latest crop of tech
companies. The ride-hailing company Lyft, which
\href{https://www.nytimes3xbfgragh.onion/2019/03/29/technology/lyft-stock-price.html}{went
public in March}, has had a troubled start.
\href{https://www.nytimes3xbfgragh.onion/2019/04/01/technology/lyft-stock.html}{Lyft
shares surged, then quickly sank} below their initial price.

Pinterest almost became an ``undercorn'' --- a company that goes public
for less than its private market valuation --- but public market
investors warmed to the company during its pitches ahead of its I.P.O.,
leading it to raise its proposed share price before its stock offering.
The company's debut bodes well for Uber, Slack and others, which are
expected to go public this year.

Jeremy Levine, a partner at Bessemer Venture Partners, Pinterest's
biggest shareholder, said a combination of investor optimism and low
volatility was helping tech I.P.O.s.

``The market is moody, but right now it's in a good mood,'' he said.

One investor concern about the tech companies heading for the public
markets is their lack of profits. One of the exceptions is Zoom.

The demand for Zoom stock, and the leap in its share price, showed that
investors are just as eager to back lesser-known enterprise software
companies --- particularly profitable ones --- as they are to support
high-profile apps geared toward consumers. Shares in PagerDuty, a
smaller software start-up,
\href{https://www.nytimes3xbfgragh.onion/2019/04/15/technology/ipo-tech-companies-horde.html}{soared
60 percent} on its first day of trading last week.

Pinterest is closer to turning a profit than Uber and Lyft. It lost \$63
million on revenue of \$756 million last year, a sharp contrast to the
\href{https://www.nytimes3xbfgragh.onion/2019/03/01/technology/lyft-ipo-filing.html}{nearly
\$1 billion} Lyft lost and the
\href{https://www.nytimes3xbfgragh.onion/2019/04/11/technology/uber-ipo-filing.html}{\$1.8
billion} Uber lost in the same period.

Pinterest's chief executive and co-founder, Ben Silbermann, has avoided
the pizazz that has led many of his Silicon Valley peers to become minor
celebrities. But as the leader of a publicly traded company, he will
need to woo Wall Street investors and analysts.

Mr. Silbermann said in an interview that he planned to celebrate by
taking a nap. He said he had told his colleagues that the I.P.O. was a
little like graduating from school.

``You have to throw your hat in the air and take a moment,'' he said.
``But it's not like when you graduate, all your problems are solved.''

Rick Heitzmann, managing director at FirstMark Capital, one of
Pinterest's first investors, said Pinterest's understated culture served
it well in the I.P.O. process. Pushing for a higher share price in its
offering would not have been a good long-term strategy, he said.

``You don't want to overhype anything,'' Mr. Heitzmann said. ``You want
to set reasonable expectations and work really hard to exceed them.''

Pinterest is not a social media app for interacting with celebrities or
broadcasting one's life, the company said in its I.P.O. prospectus. It
is meant to be personal instead. Its 250 million monthly active users,
or pinners, use the site to plan important aspects of their lives,
including home projects, weddings and meals.

The focus on personal growth and planning, rather than on comments and
interactions with others, has helped Pinterest sidestep much of the
bullying, toxic behavior and disinformation that have plagued other
social platforms in recent years.

But Pinterest, which makes money from advertising, faces heavy
competition from those companies, including Facebook and its Instagram
subsidiary. Other rivals include Allrecipes, a recipe website; Houzz, a
home-improvement website; and Tastemade, a cooking content company.

As a private company, Pinterest raised \$1.5 billion from investors,
many of whom will reap outsize paydays. Bessemer Venture Partners,
FirstMark Capital and Andreessen Horowitz, which invested in the
company's early days, will score big. Fidelity and Valiant Capital
Partners also hold significant stakes.

In addition to providing a way for Pinterest's investors and employees
to cash out, Mr. Silbermann said, the money that Pinterest raised in its
I.P.O. will allow the company to look at potential acquisitions and new
lines of business.

``Should there be an opportunity to acquire a business or invest in an
opportunity we see that looks really great, it's a little easier,'' he
said.

He said he didn't identify his company with the pack of unicorns racing
to the public market, including Slack, Uber and Lyft, because the
companies are in different industries like transportation and business
software.

``In a few years, people might remember it as a moment in time, but the
companies will be judged very differently five years down the line,''
Mr. Silbermann said.

Advertisement

\protect\hyperlink{after-bottom}{Continue reading the main story}

\hypertarget{site-index}{%
\subsection{Site Index}\label{site-index}}

\hypertarget{site-information-navigation}{%
\subsection{Site Information
Navigation}\label{site-information-navigation}}

\begin{itemize}
\tightlist
\item
  \href{https://help.nytimes3xbfgragh.onion/hc/en-us/articles/115014792127-Copyright-notice}{©~2020~The
  New York Times Company}
\end{itemize}

\begin{itemize}
\tightlist
\item
  \href{https://www.nytco.com/}{NYTCo}
\item
  \href{https://help.nytimes3xbfgragh.onion/hc/en-us/articles/115015385887-Contact-Us}{Contact
  Us}
\item
  \href{https://www.nytco.com/careers/}{Work with us}
\item
  \href{https://nytmediakit.com/}{Advertise}
\item
  \href{http://www.tbrandstudio.com/}{T Brand Studio}
\item
  \href{https://www.nytimes3xbfgragh.onion/privacy/cookie-policy\#how-do-i-manage-trackers}{Your
  Ad Choices}
\item
  \href{https://www.nytimes3xbfgragh.onion/privacy}{Privacy}
\item
  \href{https://help.nytimes3xbfgragh.onion/hc/en-us/articles/115014893428-Terms-of-service}{Terms
  of Service}
\item
  \href{https://help.nytimes3xbfgragh.onion/hc/en-us/articles/115014893968-Terms-of-sale}{Terms
  of Sale}
\item
  \href{https://spiderbites.nytimes3xbfgragh.onion}{Site Map}
\item
  \href{https://help.nytimes3xbfgragh.onion/hc/en-us}{Help}
\item
  \href{https://www.nytimes3xbfgragh.onion/subscription?campaignId=37WXW}{Subscriptions}
\end{itemize}
