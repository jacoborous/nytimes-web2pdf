Sections

SEARCH

\protect\hyperlink{site-content}{Skip to
content}\protect\hyperlink{site-index}{Skip to site index}

\href{https://www.nytimes3xbfgragh.onion/section/technology}{Technology}

\href{https://myaccount.nytimes3xbfgragh.onion/auth/login?response_type=cookie\&client_id=vi}{}

\href{https://www.nytimes3xbfgragh.onion/section/todayspaper}{Today's
Paper}

\href{/section/technology}{Technology}\textbar{}Amazon Faces Investor
Pressure Over Facial Recognition

\url{https://nyti.ms/2HsPIFJ}

\begin{itemize}
\item
\item
\item
\item
\item
\item
\end{itemize}

Advertisement

\protect\hyperlink{after-top}{Continue reading the main story}

Supported by

\protect\hyperlink{after-sponsor}{Continue reading the main story}

\hypertarget{amazon-faces-investor-pressure-over-facial-recognition}{%
\section{Amazon Faces Investor Pressure Over Facial
Recognition}\label{amazon-faces-investor-pressure-over-facial-recognition}}

\includegraphics{https://static01.graylady3jvrrxbe.onion/images/2019/05/20/business/20amazon4/merlin_146147871_0c96c466-5a89-4bd3-b73a-b45f48b42c23-articleLarge.jpg?quality=75\&auto=webp\&disable=upscale}

By \href{https://www.nytimes3xbfgragh.onion/by/natasha-singer}{Natasha
Singer}

\begin{itemize}
\item
  May 20, 2019
\item
  \begin{itemize}
  \item
  \item
  \item
  \item
  \item
  \item
  \end{itemize}
\end{itemize}

Facial recognition software is coming under increasing scrutiny from
civil liberties groups and lawmakers. Now Amazon, one of the most
visible purveyors of the technology, is facing pressure from another
corner as well: its own shareholders.

As part of Amazon's
\href{https://ir.aboutamazon.com/2019-annual-meeting-shareholders}{annual
meeting in Seattle} on Wednesday, investors are voting on whether the
tech giant's aggressive push to spread the surveillance software
threatens civil rights --- and, as a consequence, the company's
reputation and profits.

Shareholders have introduced two proposals on facial recognition for a
vote. One asks the company to prohibit sales of its facial recognition
system, called Amazon Rekognition, to government agencies, unless its
board concludes that the technology does not facilitate human rights
violations. The other asks the company to commission an independent
report examining the extent to which Rekognition may threaten civil,
human and privacy rights, and the company's finances.

``This piece of equipment that Amazon has fostered and developed and is
really propagating at this point doesn't seem to us to be in the best
interest of the common good,'' said
\href{https://brentwoodcsj.org/news/sister-pat-mahoney-leads-effort-to-end-bank-prison-financing/}{Sister
Pat Mahoney}, a member of the Sisters of St. Joseph, a religious
community in Brentwood, N.Y., that is an Amazon investor and introduced
the proposed sales ban. ``Facial recognition all over the place just
makes everyone live in a police state.''

The proposals are nonbinding, meaning they do not require the company to
take action, even if they receive a majority vote. But they add to the
growing resistance to facial surveillance technology by
\href{https://cbc.house.gov/news/documentsingle.aspx?DocumentID=898}{elected
officials},
\href{https://www.aclu.org/letter-nationwide-coalition-amazon-ceo-jeff-bezos-regarding-rekognition}{civil
liberties groups} and even some
\href{https://medium.com/s/powertrip/im-an-amazon-employee-my-company-shouldn-t-sell-facial-recognition-tech-to-police-36b5fde934ac}{Amazon
employees}.

Last week,
\href{https://www.nytimes3xbfgragh.onion/2019/05/14/us/facial-recognition-ban-san-francisco.html}{San
Francisco banned the use of} facial surveillance technology by the
police and other city agencies.
\href{https://www.sfchronicle.com/bayarea/article/Oakland-considers-banning-facial-recognition-13826426.php}{Oakland,
Calif}., and
\href{https://www.aclum.org/en/news/somerville-becomes-first-east-coast-community-consider-ban-face-recognition}{Somerville,
Mass., near Boston, are considering similar bans}. Earlier this year,
state lawmakers in
\href{https://malegislature.gov/Bills/191/SD671}{Massachusetts} and
\href{https://leginfo.legislature.ca.gov/faces/billTextClient.xhtml?bill_id=201920200AB1215}{California}
introduced bills that would restrict its use by government agencies. On
Wednesday, the House Committee on Oversight and Reform is
\href{https://oversight.house.gov/legislation/hearings/facial-recognition-technology-part-1-its-impact-on-our-civil-rights-and}{holding
a hearing} on the civil rights implications of facial surveillance.

\includegraphics{https://static01.graylady3jvrrxbe.onion/images/2019/05/20/business/20amazon1-sub/merlin_141702396_3a157d9b-4510-448d-83dc-8803ac2ed1e1-articleLarge.jpg?quality=75\&auto=webp\&disable=upscale}

The Amazon shareholder proposals also highlight the rise of activism
among investors in the country's top tech companies.

Last year,
\href{https://www.nytimes3xbfgragh.onion/2018/01/08/business/apple-investors-children.html}{investors
successfully pressured} Apple to create
\href{https://techcrunch.com/2018/06/04/apple-unveils-new-screen-time-controls-for-children/}{stronger
parental controls} for iPhones, warning that the device could be too
compelling for young children. In the coming weeks, shareholders of
\href{http://d18rn0p25nwr6d.cloudfront.net/CIK-0001326801/ffdb441a-71d1-4bd0-9d7b-c1583143b218.pdf}{Facebook},
\href{https://s22.q4cdn.com/826641620/files/doc_financials/proxy/2019/Proxy2019.pdf}{Twitter}
and
\href{https://www.sec.gov/Archives/edgar/data/1652044/000130817919000205/lgoog2019_def14a.htm}{Alphabet}
will vote on issues related to election interference, hate speech,
disinformation and the creation of censored services for China.

``We're not Luddites, we're not anti-technology,'' said Michael Connor,
the executive director of Open MIC, a nonprofit group that works with
activist investors in the tech sector and helped draft the facial
surveillance proposals with Amazon shareholders. ``But we do think all
these technologies have to be handled and introduced in a responsible
way.''

For Amazon's annual meeting on Wednesday, employees who are stockholders
have also introduced
\href{https://www.nytimes3xbfgragh.onion/2019/04/10/technology/amazon-climate-change-letter.html}{a
proposal on climate change}, pushing the company to make firm
commitments to reduce its carbon footprint.

But Amazon fought particularly hard to prevent the votes on facial
surveillance. In
\href{https://www.sec.gov/divisions/corpfin/cf-noaction/14a-8/2019/johnharringtonetal032819-14a8.pdf}{a
letter to the Securities and Exchange Commission} in January, the
company said that it was not aware of any reported misuse of Rekognition
by law enforcement customers. It also argued that the technology did not
present a financial risk because it was just one of the more than 165
services Amazon offered.

``The proposals raise only conjecture and speculation about possible
risks that might arise'' from clients misusing the technology, lawyers
for Amazon wrote in the letter. The
\href{https://static1.squarespace.com/static/57693891579fb3ab7149f04b/t/5ca5354d9fbb780001c8eca3/1554330983496/Amazon.com\%2C+Inc.+SEC+Response.pdf}{agency
disagreed}, ultimately requiring Amazon to allow the facial surveillance
resolutions to proceed.

In a statement in response to a reporter's questions, Amazon said it
offered
\href{https://docs.aws.amazon.com/rekognition/latest/dg/considerations-public-safety-use-cases.html}{clear
guidelines} on using Rekognition for public safety --- including a
recommendation that law enforcement agencies have humans review any
possible facial matches suggested by its system. The company added that
its customers had used Rekognition for beneficial purposes, including
identifying more than 3,000 victims of human trafficking.

Image

The company says Amazon Rekognition automatically recognizes
celebrities. On its website, Amazon uses Mr. Bezos as an
example.Credit...Amazon

``We have not seen law enforcement agencies use Amazon Rekognition to
infringe on citizens' civil liberties,'' the Amazon statement said.

(The New York Times used Amazon Rekognition last year to help identify
guests at the
\href{https://www.nytimes3xbfgragh.onion/2018/05/19/world/europe/royal-wedding-live.html?module=inline}{royal
wedding of Prince Harry and Meghan Markle}.)

Amazon is becoming a national magnet for mounting opposition to facial
surveillance --- a technology that may be used to identify and track
people at a distance without their knowledge or consent.

Facial recognition uses artificial intelligence to scan a photo of an
unknown person. The software then compares the facial template of the
unknown person with a database of templates of known people and, if the
templates are very similar, may suggest a name or match.

Proponents of the technology argue that such systems help law
enforcement agencies more easily identify crime suspects and missing
children. Civil liberties groups warn that the technology could easily
be misused to disproportionately pursue immigrants, people of color and
protesters, infringing on their rights to free speech and movement.

Other companies have long sold facial surveillance to law enforcement
agencies, but Amazon has differentiated itself by, in part, playing down
warnings about the technology.

Last year,
\href{https://www.blog.google/around-the-globe/google-asia/ai-social-good-asia-pacific/}{Google
said that it would refrain} from offering facial recognition for general
purposes until it had worked through the policy implications. This year,
Bradford L. Smith, the president of Microsoft, said that his company had
decided not to sell the surveillance technology to a police department
seeking to freely use it on the general public.

Image

Jeff Talbot, a Washington County sheriff's deputy, demonstrating how his
agency, whose headquarters are in Hillsboro, Ore., uses facial
recognition software.Credit...Gillian Flaccus/Associated Press

Amazon, in contrast, recently pitched its facial recognition services to
Immigration and Customs Enforcement,
\href{https://www.documentcloud.org/documents/5014186-Amazon-ICE-emails-FOIA.html}{according
to company emails} obtained under open records law by the Project on
Government Oversight, a nonprofit group based in Washington.

Institutional Shareholder Services and Glass Lewis, two prominent firms
that advise many large institutional investors, each recommended this
month that shareholders vote in favor of the resolution calling for an
outside report on Rekognition's risks.

In its analysis, Institutional Shareholder Services wrote that Amazon
``may be lagging its peers'' because it has ``not developed rules for
bidding on government contracts, has not formed an artificial
intelligence ethics committee and has not announced partnerships with
civil liberties organizations.''

Industry analysts said there was little chance that the proposal to ban
Rekognition would gain traction among shareholders.

But at least a few large institutional investors --- including the New
York City Pension Funds, which have about \$1 billion in Amazon holdings
--- plan to vote in favor of the proposal for an independent report on
facial surveillance.

``We want Amazon's board to oversee and disclose how Amazon is
addressing the significant risks posed by the sale of facial recognition
technology,'' said Scott Stringer, the New York City comptroller and the
investment adviser to the funds. He described the software as ``a
product that could lead to violations of human and civil rights around
the world, especially if sold to authoritarian governments.''

Even so, that may not sway Amazon, whose largest investor prefers a
wait-and-see approach to the risks of emerging technologies.

``Technologies always are two-sided. There are ways they can be
misused,'' Jeff Bezos, Amazon's chief executive,
\href{https://www.wired.com/story/amazons-jeff-bezos-says-tech-companies-should-work-with-the-pentagon/}{said
at a Wired tech conference} last fall, adding: ``That's always been the
case, and we will figure it out. The last thing I'd ever want to do is
stop the progress of new technologies, even when they are dual use.''

Advertisement

\protect\hyperlink{after-bottom}{Continue reading the main story}

\hypertarget{site-index}{%
\subsection{Site Index}\label{site-index}}

\hypertarget{site-information-navigation}{%
\subsection{Site Information
Navigation}\label{site-information-navigation}}

\begin{itemize}
\tightlist
\item
  \href{https://help.nytimes3xbfgragh.onion/hc/en-us/articles/115014792127-Copyright-notice}{©~2020~The
  New York Times Company}
\end{itemize}

\begin{itemize}
\tightlist
\item
  \href{https://www.nytco.com/}{NYTCo}
\item
  \href{https://help.nytimes3xbfgragh.onion/hc/en-us/articles/115015385887-Contact-Us}{Contact
  Us}
\item
  \href{https://www.nytco.com/careers/}{Work with us}
\item
  \href{https://nytmediakit.com/}{Advertise}
\item
  \href{http://www.tbrandstudio.com/}{T Brand Studio}
\item
  \href{https://www.nytimes3xbfgragh.onion/privacy/cookie-policy\#how-do-i-manage-trackers}{Your
  Ad Choices}
\item
  \href{https://www.nytimes3xbfgragh.onion/privacy}{Privacy}
\item
  \href{https://help.nytimes3xbfgragh.onion/hc/en-us/articles/115014893428-Terms-of-service}{Terms
  of Service}
\item
  \href{https://help.nytimes3xbfgragh.onion/hc/en-us/articles/115014893968-Terms-of-sale}{Terms
  of Sale}
\item
  \href{https://spiderbites.nytimes3xbfgragh.onion}{Site Map}
\item
  \href{https://help.nytimes3xbfgragh.onion/hc/en-us}{Help}
\item
  \href{https://www.nytimes3xbfgragh.onion/subscription?campaignId=37WXW}{Subscriptions}
\end{itemize}
