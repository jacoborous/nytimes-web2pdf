Sections

SEARCH

\protect\hyperlink{site-content}{Skip to
content}\protect\hyperlink{site-index}{Skip to site index}

\href{https://www.nytimes3xbfgragh.onion/section/technology/personaltech}{Personal
Tech}

\href{https://myaccount.nytimes3xbfgragh.onion/auth/login?response_type=cookie\&client_id=vi}{}

\href{https://www.nytimes3xbfgragh.onion/section/todayspaper}{Today's
Paper}

\href{/section/technology/personaltech}{Personal Tech}\textbar{}Google
Pixel 3A Review: The \$400 Smartphone You've Been Waiting For

\url{https://nyti.ms/2Wu4Wza}

\begin{itemize}
\item
\item
\item
\item
\item
\end{itemize}

Advertisement

\protect\hyperlink{after-top}{Continue reading the main story}

Supported by

\protect\hyperlink{after-sponsor}{Continue reading the main story}

tech Fix

\hypertarget{google-pixel-3a-review-the-400-smartphone-youve-been-waiting-for}{%
\section{Google Pixel 3A Review: The \$400 Smartphone You've Been
Waiting
For}\label{google-pixel-3a-review-the-400-smartphone-youve-been-waiting-for}}

The first midrange Pixel is packed with innovations, without a shocking
price.

\includegraphics{https://static01.graylady3jvrrxbe.onion/images/2019/05/08/business/08techfix/merlin_154476501_96d369ce-9f02-4432-997c-8a87aca541b3-articleLarge.jpg?quality=75\&auto=webp\&disable=upscale}

\href{https://www.nytimes3xbfgragh.onion/by/brian-x-chen}{\includegraphics{https://static01.graylady3jvrrxbe.onion/images/2018/02/16/multimedia/author-brian-x-chen/author-brian-x-chen-thumbLarge.jpg}}

By \href{https://www.nytimes3xbfgragh.onion/by/brian-x-chen}{Brian X.
Chen}

\begin{itemize}
\item
  May 7, 2019
\item
  \begin{itemize}
  \item
  \item
  \item
  \item
  \item
  \end{itemize}
\end{itemize}

Dear readers, I hear you: Plenty of you are making it loud and clear
that you are frustrated with today's smartphone prices, which are
approaching the cost of a decent used car.

I've read your testy emails about skyrocketing prices for devices like
Apple's \$1,100 iPhone XS Max and Samsung's \$2,000
\href{https://www.nytimes3xbfgragh.onion/2019/04/18/technology/samsung-galaxy-screen.html}{Galaxy
Fold}. I've seen your anguish in the comments on our
\href{https://www.nytimes3xbfgragh.onion/2018/10/23/technology/personaltech/apple-iphone-xr-review.html}{smartphone
reviews}.
\href{https://www.idc.com/getdoc.jsp?containerId=prUS45042319}{Sales of
smartphones are slowing down} worldwide, researchers say, partly because
people are turned off by the escalating costs.

So this will probably come as good news. As of Tuesday, Google is
selling the Pixel 3A, a new version of its popular Pixel smartphone, for
about \$400 --- or roughly half the price of its high-end phones. It is
the first time that Google is introducing its Pixel phones for the
midrange and low-end market.

``We're seeing the fatigue with some of the flagship pricing of
smartphones going up and up and up, and people thinking, `You know, five
years ago I could buy the best possible phone for half this price,'''
said Brian Rakowski, a vice president of product management for Google.

The Pixel 3A lacks some frills you may find in premium devices, like
wireless charging and water resistance. But based on my tests, it is a
great value. It's fast and capable with a very good camera and a
nice-looking screen --- and, yes, especially for this price.

I wouldn't hesitate to recommend this phone to those who don't mind
going without some cutting-edge features. In fact, the Pixel 3A is so
satisfying that at this point, I might hesitate to recommend its \$800
counterpart, the
\href{https://www.nytimes3xbfgragh.onion/2018/10/15/technology/personaltech/google-pixel-3-review.html}{Pixel
3}, to people other than gear heads and tech enthusiasts. While I rated
the Pixel 3 an excellent Android phone last fall, it is not
two-times-the-cost better than the Pixel 3A.

Here were my impressions after a week of testing the Pixel 3A.

\hypertarget{smarts-where-it-matters}{%
\subsection{Smarts where it matters}\label{smarts-where-it-matters}}

\includegraphics{https://static01.graylady3jvrrxbe.onion/images/2019/05/07/business/07techfix/96d2773f78084af599f590460396ecaf-articleLarge.jpg?quality=75\&auto=webp\&disable=upscale}

The high-end Pixel 3 was widely lauded for its camera system, which has
software features powered by artificial intelligence and machine
learning. Fortunately, the Pixel's most important camera features are
also baked into the Pixel 3A.

Among the clever camera features is a software mode called Night Sight,
which makes photos taken in low light look as if they had been shot in
normal conditions, without a flash. Google accomplishes this with some
A.I. sorcery that involves taking a burst of photos with short exposures
and reassembling them into an image.

Image

Night Sight makes photos shot in low light look as if they had been
taken in normal lighting.Credit...Brian X. Chen for The New York Times

I was delighted to see that Night Sight worked well with the Pixel 3A.
It was especially useful indoor, like in dimly lit restaurants or rooms.
In one test, I dimmed my bedroom lamp to the lowest setting and took a
photo of my dog as he slept. The image looked nicely lit up without
seeming unnatural. With smartphones that lacked a similar low-light
mode, including Samsung Galaxy phones or iPhones, the photos came out
very dark.

The Pixel 3A can also shoot images with portrait mode, also known as the
bokeh effect, which puts the picture's main subject in sharp focus while
gently blurring the background. Portrait mode was effective at producing
artsy-looking pictures of red flowers in a garden and of my dogs in a
field.

However, in some photos of people, portrait mode made faces look grainy
and unappealing. Google said that in my test shots, more light was
coming from the background than from the person's face. To capture both
the face and the background, the Pixel 3A added noise to their faces,
the company said. Anecdotally, I've had better results with portrait
mode on the pricier Pixel 3 and iPhones.

Otherwise, normal shots in good lighting consistently looked crisp and
clear, with nice shadow detail. Like other Google phones I have tested,
the Pixel 3A left colors looking colder and slightly less natural than
photos taken with an iPhone.

Still, on average, the Pixel 3A has a very good camera that plenty of
people will enjoy. In
\href{https://www.nytimes3xbfgragh.onion/2018/02/28/technology/personaltech/cheaper-smartphone.html}{cheaper
phones in years past}, a low-quality camera was always the biggest
downside, but the Pixel 3A's camera isn't much of a compromise.

\hypertarget{insignificant-trade-offs}{%
\subsection{Insignificant trade-offs}\label{insignificant-trade-offs}}

Other features missing from the Pixel 3A include support for wireless
charging, a wide-angle lens on its front-facing camera and water
resistance. Most of these omissions are negligible.

Wireless charging is a neat innovation, but it's a novelty. The
technology relies on magnetic induction, which uses an electrical
current to generate a magnetic field, creating voltage that powers the
phone.

My problem with wireless charging? Wires are still involved. While you
don't have to plug a cable into the phone, the accessories themselves
--- like charging pads or stands --- have to be hooked up to a power
outlet. There are only a few times when
\href{https://www.nytimes3xbfgragh.onion/2018/10/03/technology/personaltech/wireless-charging-pros-cons.html}{charging
with induction is more practical} than charging with a wire.

With no wide-angle lens for the front-facing camera, the framing won't
be as broad when you take a selfie that includes lots of people. As an
older millennial with no interest in taking selfies, I can live without
that feature.

Image

Credit...Damien Maloney for The New York Times

Image

Credit...Damien Maloney for The New York Times

The biggest downside is the lack of waterproofing. Many people's gadgets
have fallen victim to heavy rain or spilled beverages. Still, this isn't
a deal breaker. Plenty of accessory makers sell inexpensive cases and
pouches that protect phones from water damage. Or you can just be extra
careful around liquids.

The other trade-offs are even less significant. The Pixel 3A is slightly
slower than the Pixel 3, but not noticeably. The cheaper phone's screen
also has marginally less accurate colors than the high-end Pixel's
display, but you would need to hold the devices side by side and look
very closely to notice the difference.

There is one upside to the omissions: The Pixel 3A includes a headphone
jack, which many high-end smartphones eliminated to make room for other
advanced components. So you can still quickly plug in a pair of
headphones without shelling out for
\href{https://www.nytimes3xbfgragh.onion/2019/04/03/technology/personaltech/apple-airpods-review.html}{wireless
earbuds}.

\hypertarget{bottom-line}{%
\subsection{Bottom line}\label{bottom-line}}

There's little that casual technology users would want from a phone that
the Pixel 3A doesn't provide.

The device works well with Google's software and internet services,
which many already rely on. It will be sold through a large number of
retailers where customers can get technical support, including Verizon
Wireless, Sprint and T-Mobile in the United States; Google is also
selling the device in Canada, Taiwan, Ireland, Spain, Japan and India,
among other countries. People can buy a model with a 5.6-inch screen for
\$400 or a model with a six-inch screen for \$480.

In many ways, the Pixel 3A feels like the phone Google should have
delivered in the first place. The internet company built a reputation on
making its products free or cheap and thereby accessible to as broad an
audience as possible. With the Pixel's latest iteration, Google is
making a statement that many will agree with: Communication devices
should be a tool for everyone, not just the elite.

Advertisement

\protect\hyperlink{after-bottom}{Continue reading the main story}

\hypertarget{site-index}{%
\subsection{Site Index}\label{site-index}}

\hypertarget{site-information-navigation}{%
\subsection{Site Information
Navigation}\label{site-information-navigation}}

\begin{itemize}
\tightlist
\item
  \href{https://help.nytimes3xbfgragh.onion/hc/en-us/articles/115014792127-Copyright-notice}{©~2020~The
  New York Times Company}
\end{itemize}

\begin{itemize}
\tightlist
\item
  \href{https://www.nytco.com/}{NYTCo}
\item
  \href{https://help.nytimes3xbfgragh.onion/hc/en-us/articles/115015385887-Contact-Us}{Contact
  Us}
\item
  \href{https://www.nytco.com/careers/}{Work with us}
\item
  \href{https://nytmediakit.com/}{Advertise}
\item
  \href{http://www.tbrandstudio.com/}{T Brand Studio}
\item
  \href{https://www.nytimes3xbfgragh.onion/privacy/cookie-policy\#how-do-i-manage-trackers}{Your
  Ad Choices}
\item
  \href{https://www.nytimes3xbfgragh.onion/privacy}{Privacy}
\item
  \href{https://help.nytimes3xbfgragh.onion/hc/en-us/articles/115014893428-Terms-of-service}{Terms
  of Service}
\item
  \href{https://help.nytimes3xbfgragh.onion/hc/en-us/articles/115014893968-Terms-of-sale}{Terms
  of Sale}
\item
  \href{https://spiderbites.nytimes3xbfgragh.onion}{Site Map}
\item
  \href{https://help.nytimes3xbfgragh.onion/hc/en-us}{Help}
\item
  \href{https://www.nytimes3xbfgragh.onion/subscription?campaignId=37WXW}{Subscriptions}
\end{itemize}
