Sections

SEARCH

\protect\hyperlink{site-content}{Skip to
content}\protect\hyperlink{site-index}{Skip to site index}

\href{/section/world/americas}{Americas}\textbar{}Women Are Fleeing
Death at Home. The U.S. Wants to Keep Them Out.

\url{https://nyti.ms/30adRro}

\begin{itemize}
\item
\item
\item
\item
\item
\item
\end{itemize}

\includegraphics{https://static01.graylady3jvrrxbe.onion/images/2019/08/19/world/19guatemala-a1/Guatemala-articleLarge.jpg?quality=75\&auto=webp\&disable=upscale}

\hypertarget{women-are-fleeing-death-at-home-the-us-wants-to-keep-them-out}{%
\section{Women Are Fleeing Death at Home. The U.S. Wants to Keep Them
Out.}\label{women-are-fleeing-death-at-home-the-us-wants-to-keep-them-out}}

Violence against women is driving an exodus of migrants from Central
America, but the Trump administration is determined to deny them asylum.

Lubia Sasvin Pérez sitting on her mother's grave with her sisters
Marleny, left, and Heidy. Lubia's former boyfriend, Gehovany Ramirez,
murdered the sisters' mother.Credit...Meridith Kohut for The New York
Times

Supported by

\protect\hyperlink{after-sponsor}{Continue reading the main story}

By \href{https://www.nytimes3xbfgragh.onion/by/azam-ahmed}{Azam Ahmed}

Photographs by Meridith Kohut and Daniel Berehulak

\begin{itemize}
\item
  Aug. 18, 2019
\item
  \begin{itemize}
  \item
  \item
  \item
  \item
  \item
  \item
  \end{itemize}
\end{itemize}

\href{https://www.nytimes3xbfgragh.onion/es/2019/08/19/espanol/america-latina/guatemala-migrantes-mujeres-violencia.html}{Leer
en español}

JALAPA, Guatemala --- They climbed the terraced hillside in single file,
their machetes tapping the stones along the darkened footpath.

Gehovany Ramirez, 17, led his brother and another accomplice to his
ex-girlfriend's home. He struck the wooden door with his machete,
sending splinters into the air.

His girlfriend, Lubia Sasvin Pérez, had left him a month earlier,
fleeing his violent temper for her parents' home here in southeast
Guatemala. Five months pregnant, her belly hanging from her tiny
16-year-old frame, she feared losing the child to his rage.

Lubia and her mother slipped outside and begged him to leave, she said.
They could smell the sour tang of alcohol on his breath. Unmoved, he
raised the blade and struck her mother in the head, killing her.

Hearing a stifled scream, her father rushed outside. Lubia recalled
watching in horror as the other men set upon him, splitting his face and
leaving her parents splayed on the concrete floor.

For prosecutors, judges and even defense lawyers in Guatemala, the case
exemplifies the national scourge of domestic violence, motivated by a
deep-seated sense of ownership over women and their place in
relationships.

But instead of facing the harsher penalties meant to stop such crimes in
Guatemala, Gehovany received only four years in prison, a short sentence
even by the country's lenient standard for minors. More than three years
later, now 21, he will be released next spring, perhaps sooner.

And far from being kept from the family he tore apart, under Guatemalan
law Gehovany has the right to visit his son upon release, according to
legal officials in Guatemala.

The prospect of his return shook the family so thoroughly that Lubia's
father, who survived the attack, sold their home and used the money to
pay a smuggler to reach the United States. Now living outside of San
Francisco, he is pinning his hopes on winning asylum to safeguard his
family. They all are.

But that seems more distant than ever. Two extraordinary legal decisions
**** by the Trump administration have struck at the core of asylum
claims rooted in domestic violence or threats against families like
Lubia's --- not only casting doubt on their case, but almost certainly
on thousands of others as well, immigration lawyers say.

``How can this be justice?'' Lubia said before the family fled, sitting
under the portico where her mother was killed. ``All I did was leave him
for beating me and he took my mother from us.''

``What kind of system protects him, and not me?'' she said, gathering
her son in her lap.

\includegraphics{https://static01.graylady3jvrrxbe.onion/images/2019/08/19/world/19guatemala-JP3-SUB/17guatemala98-articleLarge.jpg?quality=75\&auto=webp\&disable=upscale}

Their case offers a glimpse into the staggering number of Central
Americans fleeing violence and dysfunction --- and the dogged fight the
Trump administration is waging to keep them out.

Across Latin America,
\href{https://www.nytimes3xbfgragh.onion/interactive/2019/05/04/world/americas/honduras-gang-violence.html}{a
murder epidemic is underway}. Most years, more than 100,000 people are
killed, largely young men on the periphery of broken societies, where
gangs and cartels sometimes take the place of the state.

The turmoil has forced millions to flee the region and seek refuge in
the United States, where they confront a system strained by record
demand and a bitter fight over whether to accept them.

But violence against women, and domestic violence in particular, is a
powerful and often overlooked factor in the migration crisis. Latin
America and the Caribbean are home to 14 of the 25 deadliest nations in
the world for women, according to available data collected by the Small
Arms Survey, which tracks violence globally.

And Central America, the region where most of those seeking asylum in
the United States are fleeing, is at the heart of the crisis.

Here in Guatemala, the homicide rate for women is more than three times
the global average. In El Salvador, it is nearly six times. In Honduras,
it is one of the highest in the world --- almost 12 times the global
average.

Image

Friends and family mourning during the funeral procession for Cristina
Yulisa Godínez, 18, in Guatemala City. Ms. Godínez was murdered in her
home in May, where she was found hanging by her neck from the ceiling,
her hands bound with blue rope.~Credit...Meridith Kohut for The New York
Times

Image

Ms. Godínez was killed in front of her 3-year-old son and her daughter,
who was a few months old. Her son told the police that a man came in,
tied her up and hanged her from the ceiling.Credit...Meridith Kohut for
The New York Times

In the most violent pockets of Central America, the United Nations says,
the danger is like living in a war zone.

``Despite the risk associated with migration, it is still lower than the
risk of being killed at home,'' said Angela Me, the chief of research
and trend analysis at the United Nations Office on Drugs and Crime.

The issue is so central to migration that former Attorney General Jeff
Sessions, eager to advance the Trump administration's priority of
closing the southern border to migrants,
\href{https://www.nytimes3xbfgragh.onion/2018/06/11/us/politics/sessions-domestic-violence-asylum.html}{issued
a decision last year} to try to halt victims of domestic violence, among
other crimes, from seeking asylum.

To win asylum in the United States, applicants must show specific
grounds for their persecution back home, like their race, religion,
political affiliation or membership in a particular social group.
Lawyers have sometimes pushed successfully for women to qualify as a
social group because of the overwhelming violence they face, citing
\href{https://www.nytimes3xbfgragh.onion/2014/08/30/us/victim-of-domestic-violence-in-guatemala-is-ruled-eligible-for-asylum-in-us.html}{a
2014 case} in which a Guatemalan woman fleeing domestic violence was
found to be eligible to apply for asylum in the United States.

But Mr. Sessions overruled that precedent, questioning whether women ---
in particular, women fleeing domestic violence --- can be members of a
social group. The decision challenged what had become common practice in
asylum courts.

Then, last month, the new attorney general, William P. Barr, went
further. Breaking with decades of precedent, he
\href{https://www.nytimes3xbfgragh.onion/2019/07/29/world/americas/justice-department-asylum-families.html}{issued
a decision}making it harder for families, like Lubia's, to qualify as
social groups also.

Violence against women in the region is so prevalent that 18 countries
have passed laws to protect them, creating a class of homicide known as
femicide, which adds tougher penalties and greater law enforcement
attention to the issue.

And yet, despite that broad effort, the new laws have failed to reduce
the killings of girls and women in the region, the United Nations says.

That reflects how deep the gender gap runs. For the new laws to make a
difference, experts say, they must go far beyond punishment to change
education, political discourse, social norms and basic family dynamics.

Though gangs and cartels in the region play a role in the violence, most
women are killed by lovers, family members, husbands or partners --- men
angered by women acting independently, enraged by jealousy or, like
Gehovany, driven by a deeply ingrained sense of control over women's
lives.

``Men end up thinking they can dispose of women as they wish,'' said
Adriana Quiñones, the United Nations Women's country representative in
Guatemala.

A vast majority of female homicides in the region are never solved. In
Guatemala, only about 6 percent result in convictions, researchers say.
And in the rare occasions when they do, as in Lubia's case, they are not
always prosecuted vigorously.

Even defense attorneys believe Gehovany should have been charged with
femicide, which would have put him in prison a couple of years longer.
The fact that he was not, some Guatemalan officials acknowledge,
underscores the many ways in which the nation's legal system, even when
set up to protect women, continues to fail them.

In the courtroom, Lubia's father, Romeo de Jesus Sasvin Dominguez, spoke
up just once.

It didn't make sense, he told the judge, shaking his head. A long white
scar ran over the bridge of his nose, a relic of the attack. How could
the laws of Guatemala favor the man who killed his wife, who hurt his
daughter?

``We had a life together,'' he told the judge, nearly in tears. ``And he
came and took that away from us just because my daughter didn't want to
be in an abusive relationship.''

``I just don't understand,'' he said.

Image

Though Jalapa has a lower homicide rate than other areas of Guatemala,
the region is still very dangerous for women.Credit...Meridith Kohut for
The New York Times

\hypertarget{its-like-our-daily-bread}{%
\subsection{`It's like our daily
bread'}\label{its-like-our-daily-bread}}

Lubia's son crawled with purpose, clutching a toy truck he had just
relieved of its back wheel.

The family watched in grateful distraction. Years after the murder, they
still lived like prisoners, trapped between mourning and fear. A
rust-colored stain blotted the floor where Lubia's mother died. The
dimpled doorjamb, hacked by the machete, had not been repaired. Lubia's
three younger sisters refused even to set foot in the bedroom where they
hid during the attack.

Santiago Ramirez, Gehovany's brother, never went to prison, spared
because of a mental illness. Neighbors often saw him walking the village
streets.

Soon, Gehovany would be, too. The family worried the men would come
back, to finish what they started.

``There's not much we can do,'' said Mr. Sasvin Dominguez, sending
Lubia's son on his way with the toy truck. ``We don't have the law in
our hands.''

He had no money to move and owned nothing but the house, which the
family clung to but could hardly bear. His two sons lived in the United
States and had families of their own to support. He hadn't seen them in
years.

``I'm raising my daughters on my own now, four of them,'' he said.

He woke each morning at 3 a.m., hiking into the mountains to work as a
farm hand. The girls, whose high cheekbones and raven-colored hair
resembled their mother's, no longer went to school. With the loss of her
income from selling knickknacks on the street, they couldn't afford to
pay for it.

His youngest daughter especially loved classes: the routine, the books,
the chance to escape her circumscribed world. But even she had resigned
herself to voluntary confinement. The stares and whispers of classmates
--- and the teasing of especially cruel ones --- had grown unbearable.
In town, some residents openly blamed Lubia for what happened. Even her
own aunts did.

``There's no justice here,'' said Lubia, who added that she wanted to
share **** her story with the public for that very reason. Her father
did, too.

In her area, Jalapa, a region of rippled hills, rutted roads and a
cowboy culture, men go around on horseback with holstered pistols, their
faces shaded by wide-brimmed hats. Though relatively peaceful for
Guatemala, with a lower homicide rate than most areas, it is very
dangerous for women.

Insulated from Guatemala's larger cities, Jalapa is a concentrated
version of the gender inequality that fuels the femicide crisis, experts
say.

``It's stark,'' said Mynor Carrera, who served as dean of the Jalapa
campus of the nation's largest university for 25 years. ``The woman is
treated often like a child in the home. And violence against them is
accepted.''

Domestic abuse is the most common crime here. Of the several dozen
complaints the Jalapa authorities receive each week, about half involve
violence against women.

``It's like our daily bread,'' said Dora Elizabeth Monson, the
prosecutor for women's issues in Jalapa. ``Women receive it morning,
afternoon and night.''

At the courthouse, Judge Eduardo Alfonso Campos Paz maintains a docket
filled with such cases. The most striking part, he said, is that most
men struggle to understand what they've done wrong.

The problem is not easily erased by legislation or enforcement, he said,
because of a mind-set ingrained in boys early on and reinforced
throughout their lives.

``When I was born, my mom or sister brought me food and drink,'' the
judge said. ``My sister cleaned up after me and washed my clothes. If I
wanted water, she would get up from wherever she was and get it for
me.''

``We are molded to be served, and when that isn't accomplished, the
violence begins,'' he said.

Image

A beauty pageant for girls in the Jalapa region. From an early age,
officials say, girls are expected to be subservient.Credit...Daniel
Berehulak for The New York Times

Image

The police investigating the crime scene of a suspected killing of a
woman in Guatemala City.Credit...Daniel Berehulak for The New York Times

Across Guatemala, complaints of domestic violence have skyrocketed as
more women come forward to report abuse. Every week, it seems, a new,
gruesome case emerges in newspapers, of a woman tortured, mutilated or
dehumanized. It is an echo of the systematic rape and torture women
endured during the nation's 36-year civil war, which left an indelible
mark on Guatemalan society.

But today, the countries with the highest rates of femicide in the
region, like Guatemala, also suffer the highest homicide rates overall
--- often leaving the killing of women overlooked or dismissed as
private domestic matters, with few national implications.

The result is more disparity. While murders in Guatemala have dropped
remarkably over the last decade, there is a notable difference by
gender: Homicides of men have fallen by 57 percent, while killings of
women have declined more slowly, by about 39 percent, according to
government data.

``The policy is to investigate violence that has more political
interest,'' said Jorge Granados, the head of the science and technology
department at Guatemala's National Institute of Forensic Sciences. ``The
public policy is simply not focused on the murder of women.''

The femicide law required every region in the nation to install a
specialized court focused on violence against women. But more than a
decade later, only 13 of 22 are in operation.

``The abuse usually happens in the home, in a private context,'' said
Evelyn Espinoza, the coordinator of the Observatory on Violence at
Diálogos, a Guatemalan research group. ``And the state doesn't involve
itself in the home.''

In Lubia's case, she fell in love with Gehovany in the fast, unstoppable
way that teenagers do. By the time they moved in together, she was
already pregnant.

But Gehovany's drinking, abuse and stultifying expectations quickly
became clear. He wanted her home at all times, even when he was out, she
said. He told her not to visit her family.

She knew Gehovany would consider her leaving a betrayal, especially
being pregnant with his child. She knew society might, too. But she had
to go, for the baby's sake, and was relieved to be free of him.

Until the night of Nov. 1, 2015, at around 9 p.m., when he came to
reclaim her.

The New York Times tried to reach Gehovany, who fled after the killing
and later turned himself in. But because he was a minor at the time of
the murder, officials said, they could not arrange an interview or
comment on the case.

His oldest brother, Robert Ramirez, argued that Gehovany had acted in
self-defense and killed Lubia's mother accidentally.

Still, Mr. Ramirez defended his brother's decision to confront Lubia's
family that night, citing a widely held view of a woman's place in
Jalapa.

``He was right to go back and try to claim her,'' he said. ``She
shouldn't have left him.''

He looked toward his own house, etched into a clay hillside, a thread of
smoke from a small fire curling through the doorway.

``I'd never allow my wife to leave me,'' he said.

\hypertarget{the-smugglers-road-north}{%
\subsection{The smugglers' road north}\label{the-smugglers-road-north}}

Mr. Sasvin Dominguez woke suddenly, startled by an idea.

He rushed to town in the dark, insects thrumming, a dense fog filling
the mountains. In a single day, it was all arranged. He would sell his
home and use the proceeds to flee to the United States.

The \$6,500 was enough to buy passage for him and his youngest daughter,
then 12. Traveling with a young child was cheaper, and often meant
better treatment by American officials. At least, that's what the
smuggler said.

He hoped to reach his sons in California. With luck, he could find work,
support the girls back home --- and get asylum for the entire family.

\hypertarget{the-dominguez-familys-journey}{%
\subsection{The Dominguez Family's
Journey}\label{the-dominguez-familys-journey}}

San

Francisco

UNITED STATES

Weslaco, Texas

Reynosa, Mexico

Querétaro, Mexico

Mexico

Guatemala

Tapachula, Mexico

Ciudad Tecún Umán, Guatemala

San

Francisco

UNITED STATES

Weslaco, Texas

Reynosa, Mexico

Querétaro, Mexico

Mexico

Guatemala

Tapachula, Mexico

Ciudad Tecún Umán, Guatemala

San

Francisco

UNITED STATES

Weslaco, Texas

Reynosa, Mexico

Mexico

Querétaro, Mexico

Guatemala

Tapachula, Mexico

Ciudad Tecún Umán, Guatemala

Satellite imagery from NASA

By The New York Times

A week later, in October of last year, he left with his daughter. A
guide crossed them into Mexico. Soon, they reached the side of a
highway, where a container truck sat idling. Inside, men, women and
children were packed tight, with hardly enough space to move.

A dense heat filled the space, the sun baking the metal box as bodies
brushed against one another. They spent nearly three days in the
container before the first stop, he said.

The days went by in a blur, a log of images snatched from the fog of
exhaustion. An open hangar, grumbling with trucks. Rolling desert,
dotted by cactus. Sunlight glaring off the metal siding of a safe house.

They rode in at least five container trucks, as best they can remember.
Hunger chased them. Some days, they got half an apple. On others, they
got rice and beans. Sometimes they got nothing.

One night, they saw a man beaten unconscious for talking after the
smugglers told him to be quiet.

``I remember that moment,'' said his daughter, whose name is being
withheld because she is still a minor. Her hands twisted at the memory.
``I felt terrified,'' she said.

Days later, starved for food, water and fresh air, she passed out in a
container crammed with more than 200 migrants, her father holding her,
fanning her with whatever documents he had.

In early November, they arrived in the Mexican border town of Reynosa,
and were spirited into a safe house. After weeks on the road, they were
getting close.

That day, the smugglers called one of Mr. Sasvin Dominguez's sons,
demanding an extra \$400 to ferry the two across the river to Texas. If
not, they would be tossed out of the safe house, left to the seething
violence of Reynosa.

Mr. Sasvin Dominguez's son sent the money. Last-minute extortions have
come to be expected. A day later, they boarded a raft and entered the
United States.

They wandered the dense brush before they stumbled upon a border patrol
truck and turned themselves in.

Mr. Sasvin Dominguez said he and his daughter spent four days in Texas,
in a facility with no windows. The fluorescent glare of the overhead
lights continued day and night, troubling their sleep. It was cold. The
migrants called it the icebox.

When they were released in November, Mr. Sasvin Dominguez was fitted
with an ankle bracelet and instructed to check in with the immigration
authorities in San Francisco, where he could begin the long process of
applying for asylum.

His son bought them bus tickets and met them at the station. It was the
first time they had seen each other in seven years.

\hypertarget{california}{%
\subsection{California}\label{california}}

On a sunny day in June, Mr. Sasvin Dominguez shuffled to a park, his
daughter riding in front, hunched over the bars of a pink bicycle meant
for a girl half her age. Behind him, his son and grandson tottered
along, hand in hand.

They traversed a quintessential American landscape --- bungalows perched
on tidy green yards, wide sidewalks shaded by soaring live oaks.

He and his daughter live in the family's modest one-bedroom apartment,
now bursting at the seams. The trappings of suburban life fill the
backyard: toolboxes, wheelbarrows, recycling bins.

But Mr. Sasvin Dominguez remains suspended in the sadness and fear he
left behind in Guatemala. His other daughters are still trapped, and
there is no money to move them.

Besides, he says, the journey north, even if they could afford it, is
far too dangerous for three young women and a toddler to take on their
own. His only hope, he says, is asylum.

That could take years, he is told, if it happens at all. A heavy backlog
of cases is gumming up the courts. He does not even have a date yet for
his first hearing.

Image

Romeo de Jesus Sasvin Dominguez in the Bay Area, where he is seeking
asylum for his family.Credit...Daniel Berehulak for The New York Times

In the meantime, he lives in self-imposed austerity, scared to embrace
his new life, as if doing so might belittle the danger his daughters
still face.

In the park, families cooked out and blasted reggaeton. His daughter
play-fought with her nephew, who never tired, no matter how many
handfuls of grass she stuffed down his shirt, or how many times he
retreated in tears.

She has found a better rhythm in their new life. In June, she finished
sixth grade at the local school, which she loves. Her older brother
keeps the graduation certificate on the small dining table.

She has dyed the tips of her hair purple, a style she's grown fond of.
Her face often falls back into the wide smile of the past, when her
mother enrolled her in local beauty contests.

But she grows stormy and unpredictable at times, refusing to speak. She
misses her mother. Her sisters, too.

Stuck in Guatemala, Lubia and her two other sisters moved into a small
apartment, where they share a single bed. A portrait of their mother
hangs on the wall.

They all work now, making tortillas in town. But they go straight home
after, to avoid being spotted. Not long ago, Lubia ran into Gehovany's
mother.

Life for the sisters is measured in micro-improvements, pockets of air
in the stifling fear. They are scarcely more than children themselves,
raising children alone. Lubia's 18-year-old sister now has an infant of
her own.

They sometimes visit their mother's grave, a green concrete box
surrounded by paddle-shaped cactus.

``We are left here with nothing,'' Lubia said.

She still bears the stigma of what happened. Neighbors, men and women
alike, continue to blame her for her mother's death. It doesn't surprise
her anymore. Now 20, she says she understands that women almost always
bear the blame for problems at home.

She worries about the world her son will grow up in, what she can teach
him and what he will ultimately come to believe. One day, she will tell
him about his father, she says, but not now, or anytime soon.

By then, she hopes to be in the United States, free of the poverty,
violence and suffocating confines for women in Guatemala.

``Here in Guatemala,'' she said, ``justice only exists in the law. Not
in reality.''

\emph{Meridith Kohut in Jalapa, Guatemala and Paulina Villegas in Mexico
City contributed reporting.}

Advertisement

\protect\hyperlink{after-bottom}{Continue reading the main story}

\hypertarget{site-index}{%
\subsection{Site Index}\label{site-index}}

\hypertarget{site-information-navigation}{%
\subsection{Site Information
Navigation}\label{site-information-navigation}}

\begin{itemize}
\tightlist
\item
  \href{https://help.nytimes3xbfgragh.onion/hc/en-us/articles/115014792127-Copyright-notice}{©~2020~The
  New York Times Company}
\end{itemize}

\begin{itemize}
\tightlist
\item
  \href{https://www.nytco.com/}{NYTCo}
\item
  \href{https://help.nytimes3xbfgragh.onion/hc/en-us/articles/115015385887-Contact-Us}{Contact
  Us}
\item
  \href{https://www.nytco.com/careers/}{Work with us}
\item
  \href{https://nytmediakit.com/}{Advertise}
\item
  \href{http://www.tbrandstudio.com/}{T Brand Studio}
\item
  \href{https://www.nytimes3xbfgragh.onion/privacy/cookie-policy\#how-do-i-manage-trackers}{Your
  Ad Choices}
\item
  \href{https://www.nytimes3xbfgragh.onion/privacy}{Privacy}
\item
  \href{https://help.nytimes3xbfgragh.onion/hc/en-us/articles/115014893428-Terms-of-service}{Terms
  of Service}
\item
  \href{https://help.nytimes3xbfgragh.onion/hc/en-us/articles/115014893968-Terms-of-sale}{Terms
  of Sale}
\item
  \href{https://spiderbites.nytimes3xbfgragh.onion}{Site Map}
\item
  \href{https://help.nytimes3xbfgragh.onion/hc/en-us}{Help}
\item
  \href{https://www.nytimes3xbfgragh.onion/subscription?campaignId=37WXW}{Subscriptions}
\end{itemize}
