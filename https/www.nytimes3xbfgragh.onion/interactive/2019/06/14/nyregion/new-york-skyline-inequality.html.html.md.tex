 **NYTimes.com no longer supports Internet Explorer 9 or earlier. Please
upgrade your browser.
\href{http://www.nytimes3xbfgragh.onion/content/help/site/ie9-support.html}{LEARN
MORE »}

**Sections

**Home

**Search

\hypertarget{the-new-york-times}{%
\subsection{\texorpdfstring{\href{http://www.nytimes3xbfgragh.onion/}{The
New York Times}}{The New York Times}}\label{the-new-york-times}}

\hypertarget{-new-york-}{%
\subsubsection{\texorpdfstring{ \href{/section/nyregion}{New York}
}{ New York }}\label{-new-york-}}

 \href{/section/nyregion}{New York} \textbar{}How New York's Skyline Is
Changing to Give the Wealthy a Better View

**Close search

\hypertarget{site-search-navigation}{%
\subsection{Site Search Navigation}\label{site-search-navigation}}

Search NYTimes.com

**Clear this text input

Go

\url{https://nyti.ms/2XQ4coM}

\hypertarget{site-navigation}{%
\subsection{Site Navigation}\label{site-navigation}}

\hypertarget{site-mobile-navigation}{%
\subsection{Site Mobile Navigation}\label{site-mobile-navigation}}

\hypertarget{how-new-yorks-skyline-is-changing-to-give-the-wealthy-a-better-view}{%
\section{How New York's Skyline Is Changing to Give the Wealthy a Better
View}\label{how-new-yorks-skyline-is-changing-to-give-the-wealthy-a-better-view}}

1,500

feet

1,000

500

One World

Trade Center

Empire

State

1,500

feet

1,000

500

One World

Trade Center

Empire

State

1,500

feet

1,000

500

One World

Trade Center

Empire

State

1,500

feet

1,000

500

1,500

1,000

500

1,500

feet

1,000

500

1,500

1,000

500

1,500

feet

1,000

500

1,500

1,000

500

ALL or partly residential

ALL or partly residential

ALL or partly residential

111 West 57 St.

\$30 million

64th floor

111 West 57 St.

\$30 million

64th floor

\$18 million

32ND floor

111 West 57 St.

\$30 million

64th floor

\$18 million

32ND floor

111 West 57 St.

\$30 million

64th floor

\$18 million

32ND floor

220 Central

Park South

~

The most

expensive

apartment in

the country is

in this building:

\$238 million.

220 Central

Park South

~

The most

expensive

apartment in

the country is

in this building:

\$238 million.

220 Central

Park South

~

The most

expensive

apartment in

the country is in

this building:

\$238 million.

Central

Park

Tower

Central

Park

Tower

Central

Park

Tower

Central Park

Tower site

West 58th St.

7th Ave.

Broadway

West 57th St.

Central Park

Tower site

West 58th St.

7th Ave.

Broadway

West 57th St.

1,500

feet

1,000

Mechanical

floors

500

central

park

tower

Mechanical

floors

1,500

feet

1,000

Mechanical

floors

500

central park

tower

1,500

FEET

1,000

500

One World

Trade Center

Empire

State

central park

tower

1,500

feet

1,000

500

One World

Trade Center

Empire

State

central park

tower

1,500

feet

1,000

500

One World

Trade Center

Empire

State

central park

tower

About

15 bonus

floors

About

15 bonus

floors

About

15 bonus

floors

West 70th St.

Building site

AMSTERDAM AVE.

WEST END AVE.

Zoning

lot for

the project

West 66th St.

West 70th St.

Building site

AMSTERDAM AVE.

WEST END AVE.

Zoning

lot for

the project

West 66th St.

West 70th St.

Building site

AMSTERDAM AVE.

WEST END AVE.

Zoning

lot for

the project

West 70th St.

Building site

AMSTERDAM AVE.

WEST END AVE.

Zoning

lot for

the project

West 66th St.

\hypertarget{how-new-yorks-skyline-is-changing-to-give-the-wealthy-a-better-view-1}{%
\subsection{How New York's Skyline Is Changing to Give the Wealthy a
Better
View}\label{how-new-yorks-skyline-is-changing-to-give-the-wealthy-a-better-view-1}}

By \textbf{SERGIO PEÇANHA} Photographs by \textbf{KARSTEN MORAN}\\
Produced by \textbf{JOSH WILLIAMS} and \textbf{JEFFREY FURTICELLA}
\textbar{}

In 2014, the Empire State Building lost its title as the tallest tower
in New York City to One World Trade Center.

By 2024, the Empire State Building could fall to 11th on the list.

Twenty buildings that are \textbf{proposed or under construction} would
figure among the city's 30 tallest.

Many of the new buildings are luxury residential high-rises, part of a
trend in major cities around the world fueled by increased land and
construction costs and by more demand from high-end buyers.

Taller buildings yield more return on the investment for developers.
Higher floors also come with more privacy, better views and much higher
sales prices.

``People with some common sense should look at this and realize that
there's no downside of it,'' said Gary Barnett, founder of Extell
Development, a company behind a number of high-end towers in the city.

Others see the changing skyline as a reflection of a changing New York.

When it opened in 1931, the Empire State Building was celebrated as a
feat of engineering. President Hoover pushed a button at the White
House, and on went the building's lights in New York.

But the new towers, with their luxury apartments designed for foreign
investors and the superwealthy, can feel like a feat of inequality.

``These towers are the physical manifestation of the 1 percent,'' said
Richard Florida, an urbanist and distinguished fellow at the N.Y.U.
Schack Institute of Real Estate.

Continue reading

As in most American cities, New York's zoning rules show little concern
for the aesthetics of a building or how it might obstruct its neighbors'
views.

This straightforward approach explains how a nearly 1,000-foot tower
recently popped up next to the iconic Manhattan Bridge.

Zoning and lot size determine how tall a developer can build. Here are
some examples of possible building configurations on the same
hypothetical lot.

But to go really tall, developers often need to combine several methods.
One example is Central Park Tower, a luxury building under construction
in Midtown.

First, the developers merged several adjacent lots into one. This meant
they could add up the lots' air rights --- or the right to build up.

They also bought air rights from buildings nearby. Almost half the
building's usable space was transferred from its neighbors.

And about 300 feet of mechanical floors --- which contain equipment like
heaters and elevator machine rooms --- push up the building's top
floors.

As a result, the new tower is set to become the tallest residential
building in the country.

Developers willing to create affordable housing, privately owned public
spaces or to make improvements to subway stations can get further height
incentives.

A project at 45 Broad Street in the Financial District received a
71,000-square-foot bonus in exchange for adding elevators to a nearby
subway station.

The elevators will cost \$21 million. The building's extra square
footage is estimated to be worth more than twice as much.

According to Robert Gladstone, a developer of 45 Broad Street, the
planned elevators would not be built otherwise. ``There's a social good,
a civic good that comes out of it,'' he said.

Some New Yorkers remain unconvinced of the benefits of the changing
skyline.

Suszannah Warner, a personal trainer who has been living in New York for
30 years, squinted at the skyline on a recent day in Central Park. ``If
I was a visitor who had never been here before, then I'm sure I would
find some sort of beauty and fascination in it. But because I'm living
here, I don't like it,'' she said.

Adrian Mars, a horse carriage driver who works nearby, passes the towers
known as Billionaires Row every day.

``They are encroaching on the majesty of the Empire State Building,'' he
said.

The race to the skies is starting to see some stronger pushback.

In March, the State Supreme Court overruled a building permit at 200
Amsterdam Avenue. Developers had merged pieces of lots to add up air
rights for a tower.

The court said the plan was inconsistent with a plain reading of zoning
regulations. But construction at the site remains underway as litigation
continues.

Then in May, the City Council approved a zoning amendment to limit
mechanical floors. If they are taller than 25 feet, they will count
toward the building's square footage allowance.

But the fact is that much of New York City's skyline remains up for
grabs.

According to the Municipal Art Society of New York, a preservation
advocacy group, 3.7 billion square feet of unused development rights are
still available.

That's enough to build more than 1,300 Empire State Buildings.

Sources: Department of City Planning; Department of Buildings; George M.
Janes \& Associates, a planning firm; Josh Vogel, an urban planner who
runs the blog NYC Urbanism; NeighborhoodX, a real estate data and
analytics company; David Wolf, president of ON Collaborative. The
building heights ranking comes from the Skyscraper Center of the Council
on Tall Buildings and Urban Habitat.

\hypertarget{more-on-nytimescom}{%
\subsection{More on NYTimes.com}\label{more-on-nytimescom}}

Advertisement

\hypertarget{site-information-navigation}{%
\subsection{Site Information
Navigation}\label{site-information-navigation}}

\begin{itemize}
\tightlist
\item
  \href{https://help.nytimes3xbfgragh.onion/hc/en-us/articles/115014792127-Copyright-notice}{©
  2020 The New York Times Company}
\item
  \href{https://www.nytimes3xbfgragh.onion}{Home}
\item
  \href{https://www.nytimes3xbfgragh.onion/search/}{Search}
\item
  Accessibility concerns? Email us at
  \href{mailto:accessibility@NYTimes.com}{\nolinkurl{accessibility@NYTimes.com}}.
  We would love to hear from you.
\item
  \href{https://help.nytimes3xbfgragh.onion/hc/en-us/articles/115015385887-Contact-Us}{Contact
  Us}
\item
  \href{https://www.nytco.com/careers/}{Work with us}
\item
  \href{https://nytmediakit.com/}{Advertise}
\item
  \href{https://help.nytimes3xbfgragh.onion/hc/en-us/articles/115014892108-Privacy-policy\#pp}{Your
  Ad Choices}
\item
  \href{https://help.nytimes3xbfgragh.onion/hc/en-us/articles/115014892108-Privacy-policy}{Privacy}
\item
  \href{https://help.nytimes3xbfgragh.onion/hc/en-us/articles/115014893428-Terms-of-service}{Terms
  of Service}
\item
  \href{https://help.nytimes3xbfgragh.onion/hc/en-us/articles/115014893968-Terms-of-sale}{Terms
  of Sale}
\end{itemize}

\hypertarget{site-information-navigation-1}{%
\subsection{Site Information
Navigation}\label{site-information-navigation-1}}

\begin{itemize}
\tightlist
\item
  \href{https://spiderbites.nytimes3xbfgragh.onion}{Site Map}
\item
  \href{https://help.nytimes3xbfgragh.onion/hc/en-us}{Help}
\item
  \href{https://help.nytimes3xbfgragh.onion/hc/en-us/articles/115015385887-Contact-Us?redir=myacc}{Site
  Feedback}
\item
  \href{https://www.nytimes3xbfgragh.onion/subscription?campaignId=37WXW}{Subscriptions}
\end{itemize}
