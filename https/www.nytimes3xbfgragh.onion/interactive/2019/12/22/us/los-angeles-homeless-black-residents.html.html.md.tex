Sections

SEARCH

\protect\hyperlink{site-content}{Skip to
content}\protect\hyperlink{site-index}{Skip to site index}

\hypertarget{comments}{%
\subsection{\texorpdfstring{\protect\hyperlink{commentsContainer}{Comments}}{Comments}}\label{comments}}

\href{}{Black, Homeless and Burdened by L.A.'s Legacy of
Racism}\href{}{Skip to Comments}

The comments section is closed. To submit a letter to the editor for
publication, write to
\href{mailto:letters@NYTimes.com}{\nolinkurl{letters@NYTimes.com}}.

\hypertarget{black-homeless-and-burdened-by-las-legacy-of-racism}{%
\section{Black, Homeless and Burdened by L.A.'s Legacy of
Racism}\label{black-homeless-and-burdened-by-las-legacy-of-racism}}

By \href{https://www.nytimes3xbfgragh.onion/by/jugal-k-patel}{Jugal K.
Patel}, \href{https://www.nytimes3xbfgragh.onion/by/tim-arango}{Tim
Arango},
\href{https://www.nytimes3xbfgragh.onion/by/anjali-singhvi}{Anjali
Singhvi} and \href{https://www.nytimes3xbfgragh.onion/by/jon-huang}{Jon
Huang}Dec. 23, 2019

\begin{itemize}
\item
\item
\item
\item
\item
  \emph{275}
\end{itemize}

\includegraphics{https://static01.graylady3jvrrxbe.onion/newsgraphics/2019/12/05/homeless/assets/images/homeless_053-2000.jpg}

Bethany Mollenkof for The New York Times

\includegraphics{https://static01.graylady3jvrrxbe.onion/newsgraphics/2019/12/05/homeless/assets/sketch1_dark.png}

\includegraphics{https://static01.graylady3jvrrxbe.onion/newsgraphics/2019/12/05/homeless/assets/images/timwlegs.png}
\includegraphics{https://static01.graylady3jvrrxbe.onion/newsgraphics/2019/12/05/homeless/assets/images/lil8.png}
\includegraphics{https://static01.graylady3jvrrxbe.onion/newsgraphics/2019/12/05/homeless/assets/images/lil15.png}
\includegraphics{https://static01.graylady3jvrrxbe.onion/newsgraphics/2019/12/05/homeless/assets/images/lil6.png}
\includegraphics{https://static01.graylady3jvrrxbe.onion/newsgraphics/2019/12/05/homeless/assets/images/lil1.png}
\includegraphics{https://static01.graylady3jvrrxbe.onion/newsgraphics/2019/12/05/homeless/assets/images/lil9.png}
\includegraphics{https://static01.graylady3jvrrxbe.onion/newsgraphics/2019/12/05/homeless/assets/images/lil4.png}
\includegraphics{https://static01.graylady3jvrrxbe.onion/newsgraphics/2019/12/05/homeless/assets/images/lil13.png}
\includegraphics{https://static01.graylady3jvrrxbe.onion/newsgraphics/2019/12/05/homeless/assets/images/lil10.png}
\includegraphics{https://static01.graylady3jvrrxbe.onion/newsgraphics/2019/12/05/homeless/assets/images/lil17.png}
\includegraphics{https://static01.graylady3jvrrxbe.onion/newsgraphics/2019/12/05/homeless/assets/images/lil12.png}
\includegraphics{https://static01.graylady3jvrrxbe.onion/newsgraphics/2019/12/05/homeless/assets/images/lil14.png}
\includegraphics{https://static01.graylady3jvrrxbe.onion/newsgraphics/2019/12/05/homeless/assets/images/lil18.png}
\includegraphics{https://static01.graylady3jvrrxbe.onion/newsgraphics/2019/12/05/homeless/assets/images/lil11.png}
\includegraphics{https://static01.graylady3jvrrxbe.onion/newsgraphics/2019/12/05/homeless/assets/images/lil2.png}
\includegraphics{https://static01.graylady3jvrrxbe.onion/newsgraphics/2019/12/05/homeless/assets/images/filled10_1.png}
\includegraphics{https://static01.graylady3jvrrxbe.onion/newsgraphics/2019/12/05/homeless/assets/images/filled10_1.png}
\includegraphics{https://static01.graylady3jvrrxbe.onion/newsgraphics/2019/12/05/homeless/assets/images/filled10_2.png}
\includegraphics{https://static01.graylady3jvrrxbe.onion/newsgraphics/2019/12/05/homeless/assets/images/filled10_2.png}
\includegraphics{https://static01.graylady3jvrrxbe.onion/newsgraphics/2019/12/05/homeless/assets/images/filled10_3.png}
\includegraphics{https://static01.graylady3jvrrxbe.onion/newsgraphics/2019/12/05/homeless/assets/images/filled10_2.png}
\includegraphics{https://static01.graylady3jvrrxbe.onion/newsgraphics/2019/12/05/homeless/assets/images/filled10_1.png}
\includegraphics{https://static01.graylady3jvrrxbe.onion/newsgraphics/2019/12/05/homeless/assets/images/filled10_3.png}
\includegraphics{https://static01.graylady3jvrrxbe.onion/newsgraphics/2019/12/05/homeless/assets/images/filled10_2.png}
\includegraphics{https://static01.graylady3jvrrxbe.onion/newsgraphics/2019/12/05/homeless/assets/images/filled10_2.png}
\includegraphics{https://static01.graylady3jvrrxbe.onion/newsgraphics/2019/12/05/homeless/assets/images/filled10_3.png}
\includegraphics{https://static01.graylady3jvrrxbe.onion/newsgraphics/2019/12/05/homeless/assets/images/filled10_3.png}
\includegraphics{https://static01.graylady3jvrrxbe.onion/newsgraphics/2019/12/05/homeless/assets/images/filled10_1.png}
\includegraphics{https://static01.graylady3jvrrxbe.onion/newsgraphics/2019/12/05/homeless/assets/images/filled10_2.png}
\includegraphics{https://static01.graylady3jvrrxbe.onion/newsgraphics/2019/12/05/homeless/assets/images/filled10_2.png}
\includegraphics{https://static01.graylady3jvrrxbe.onion/newsgraphics/2019/12/05/homeless/assets/images/filled10_3.png}
\includegraphics{https://static01.graylady3jvrrxbe.onion/newsgraphics/2019/12/05/homeless/assets/images/filled10_3.png}
\includegraphics{https://static01.graylady3jvrrxbe.onion/newsgraphics/2019/12/05/homeless/assets/images/filled10_3.png}
\includegraphics{https://static01.graylady3jvrrxbe.onion/newsgraphics/2019/12/05/homeless/assets/images/filled10_2.png}

\includegraphics{https://static01.graylady3jvrrxbe.onion/newsgraphics/2019/12/05/homeless/assets/images/lil16.png}
\includegraphics{https://static01.graylady3jvrrxbe.onion/newsgraphics/2019/12/05/homeless/assets/images/lil21.png}
\includegraphics{https://static01.graylady3jvrrxbe.onion/newsgraphics/2019/12/05/homeless/assets/images/filled10_1.png}
\includegraphics{https://static01.graylady3jvrrxbe.onion/newsgraphics/2019/12/05/homeless/assets/images/filled10_3.png}
\includegraphics{https://static01.graylady3jvrrxbe.onion/newsgraphics/2019/12/05/homeless/assets/images/filled10_3.png}
\includegraphics{https://static01.graylady3jvrrxbe.onion/newsgraphics/2019/12/05/homeless/assets/images/filled10_2.png}
\includegraphics{https://static01.graylady3jvrrxbe.onion/newsgraphics/2019/12/05/homeless/assets/images/filled10_2.png}
\includegraphics{https://static01.graylady3jvrrxbe.onion/newsgraphics/2019/12/05/homeless/assets/images/filled10_2.png}
\includegraphics{https://static01.graylady3jvrrxbe.onion/newsgraphics/2019/12/05/homeless/assets/images/filled10_1.png}
\includegraphics{https://static01.graylady3jvrrxbe.onion/newsgraphics/2019/12/05/homeless/assets/images/filled10_3.png}
\includegraphics{https://static01.graylady3jvrrxbe.onion/newsgraphics/2019/12/05/homeless/assets/images/filled10_2.png}
\includegraphics{https://static01.graylady3jvrrxbe.onion/newsgraphics/2019/12/05/homeless/assets/images/filled10_3.png}
\includegraphics{https://static01.graylady3jvrrxbe.onion/newsgraphics/2019/12/05/homeless/assets/images/filled10_2.png}
\includegraphics{https://static01.graylady3jvrrxbe.onion/newsgraphics/2019/12/05/homeless/assets/images/filled10_3.png}
\includegraphics{https://static01.graylady3jvrrxbe.onion/newsgraphics/2019/12/05/homeless/assets/images/filled10_3.png}
\includegraphics{https://static01.graylady3jvrrxbe.onion/newsgraphics/2019/12/05/homeless/assets/images/filled10_1.png}
\includegraphics{https://static01.graylady3jvrrxbe.onion/newsgraphics/2019/12/05/homeless/assets/images/filled10_2.png}
\includegraphics{https://static01.graylady3jvrrxbe.onion/newsgraphics/2019/12/05/homeless/assets/images/filled10_1.png}
\includegraphics{https://static01.graylady3jvrrxbe.onion/newsgraphics/2019/12/05/homeless/assets/images/filled10_2.png}
\includegraphics{https://static01.graylady3jvrrxbe.onion/newsgraphics/2019/12/05/homeless/assets/images/filled10_1.png}
\includegraphics{https://static01.graylady3jvrrxbe.onion/newsgraphics/2019/12/05/homeless/assets/images/filled10_2.png}
\includegraphics{https://static01.graylady3jvrrxbe.onion/newsgraphics/2019/12/05/homeless/assets/images/filled10_3.png}

In a crowd of 200 black Angelenos,\\
at least \textbf{15} were homeless at some point this year. In a crowd
of 100 black Angelenos, \textbf{8} were homeless at some point this
year.

Among all other Angelenos,\\
the rate was \textbf{2} in 200. Among all other Angelenos, the rate was
\textbf{1} in 100.

\begin{itemize}
\item
\item
\item
\item
\item
\item
\item
\end{itemize}

 hidden text

 hidden text

 As a child, Timothy Wynn lived in a single-family home in South Los
Angeles. He went to college. He worked in retail and banking. He never
imagined he would become homeless.

 As a child, Timothy Wynn lived in a single-family home in South Los
Angeles. He went to college. He worked in retail and banking. He never
imagined he would become homeless.

 When he landed on the streets in his late 50s, the experience left him
shattered. ``I grew up here,'' he said. ``I never thought I'd end up on
Skid Row. Never thought that in a hundred years.''

 When he landed on the streets in his late 50s, the experience left him
shattered. ``I grew up here,'' he said. ``I never thought I'd end up on
Skid Row. Never thought that in a hundred years.''

 Mr. Wynn's struggles with mental illness and a criminal conviction for
drug trafficking were part of the problem. But so was the rampant
discrimination that restricted the housing options for his family and so
many other black Angelenos.

 Mr. Wynn's struggles with mental illness and a criminal conviction for
drug trafficking were part of the problem. But so was the rampant
discrimination that restricted the housing options for his family and so
many other black Angelenos.

 There is a straight line from the history of redlining to today's
homelessness crisis. Mr. Wynn's experiences show why black residents are
dramatically overrepresented among those living on L.A.'s streets.

 There is a straight line from the history of redlining to today's
homelessness crisis. Mr. Wynn's experiences show why black residents are
dramatically overrepresented among those living on L.A.'s streets.

 African-Americans make up 8 percent of Los Angeles County's population.
But they are 42 percent of the homeless population. More than 60,000
black Angelenos experienced homelessness this year, county records show.

 African-Americans make up 8 percent of Los Angeles County's population.
But they are 42 percent of the homeless population. More than 60,000
black Angelenos experienced homelessness this year, county records show.

 hidden text

 hidden text

Homelessness is Los Angeles's defining crisis. Income inequality, a
shortage of housing, failing mental health services and drug addiction
all contribute to growing scenes of squalor across America's
second-largest city. The federal government recently estimated that a
nearly 3 percent rise in homelessness nationwide this year was
\href{https://www.nytimes3xbfgragh.onion/2019/12/20/us/politics/homelessness-trump-california.html}{driven
mostly} by California.

Yet it does not affect everyone equally. The historic displacement and
fracturing of black communities in South Los Angeles have pushed black
Angelenos like Mr. Wynn onto the streets at more than eight times the
rate of other groups. In interviews with more than a dozen black men who
are homeless in Los Angeles, the bitter inheritance of racism came up
again and again.

\includegraphics{https://static01.graylady3jvrrxbe.onion/packages/flash/multimedia/ICONS/transparent.png}

\includegraphics{https://static01.graylady3jvrrxbe.onion/newsgraphics/2019/12/05/homeless/assets/images/homeless_009-2000.jpg}

L.A.'s Skid Row has the largest population of homeless people in the
country.Bethany Mollenkof for The New York Times

Peter Lynn, the longtime head of the Los Angeles Homeless Services
Authority, said discrimination played a major role in the origins of the
crisis. ``There is a staggering overrepresentation of black people in
homelessness, and that is not based on poverty,'' he said. ``That is
based on structural and institutional racism.''

Marqueece Harris-Dawson, a City Council member who represents
communities in South Los Angeles, said, ``The homelessness crisis we are
living in now is the result of a housing crisis that has been in the
making for decades.''

\hypertarget{decades-of-displacement}{%
\subsection{Decades of Displacement}\label{decades-of-displacement}}

Like many black families during the Second Great Migration, Mr. Wynn's
parents came to Los Angeles in the 1950s, seeking an escape from
segregation and a path to a middle class life.

``Better life, job opportunities,'' Mr Wynn said. ``At that time
California was booming. It was, `Go West.'''

But his family arrived in Los Angeles at a time of discrimination in
housing and mortgage lending, which largely restricted black families to
certain neighborhoods in South Los Angeles.

Through a practice known as
\href{https://clkrep.lacity.org/onlinedocs/2019/19-0600_misc_5-6-19.pdf}{redlining},
real estate agents and lenders marked these neighborhoods as
\textbf{areas undesirable for investment}, preventing residents there
from obtaining home loans.

Downtown

10

Crenshaw

South

Park

Undesirable for

investment

South L.A.

Mr. Wynn's

childhood home

Inglewood

110

Westmont

Watts

405

Gardena

Compton

Downtown

10

Crenshaw

South

Park

South

L.A.

Mr. Wynn's

childhood home

Inglewood

110

Westmont

Watts

405

Undesirable for

investment

Compton

Source: Home Owners' Loan Corporation

By 1970, three-quarters of Los Angeles County's black population lived
in just two dozen neighborhoods in South L.A. That concentration made
the area a center of black culture and the site of a burgeoning black
middle class.

When
\href{https://www.labor.ucla.edu/wp-content/uploads/2017/03/UCLA_BWC_report_5-3_27-1.pdf}{manufacturing
jobs declined} in the 1980s, black unemployment nearly doubled. Drugs
and gangs ravaged the neighborhoods, kicking off a period of black
flight. Harsh policing and high incarceration hollowed out the
community.

In the 1990s, Latinos started moving to the area and the cost of living
rose. Black residents moved to Inglewood and the Crenshaw corridor, or
out of the county entirely.

These maps show the loss of \textbf{majority-black neighborhoods} in Los
Angeles County over the last 50 years.

Black share of population

50

60

70

80\%

Downtown

10

Crenshaw

South

Park

110

Inglewood

South L.A.

Westmont

Watts

Gardena

405

Compton

2.5 miles

Downtown

10

Crenshaw

Inglewood

110

South

L.A.

Watts

Compton

405

5 miles

1970

10

110

405

10

110

405

1980

10

110

405

10

110

405

1990

10

110

405

10

110

405

2000

Skid

Row

10

110

405

Skid

Row

10

110

405

2012

10

110

405

10

110

405

2017

By The New York Times·Source:
\href{https://www.socialexplorer.com/}{Social Explorer} analysis of
census data.

By 2000, South L.A. had a new racial makeup. Once predominantly black
spaces were now majority Latino, and the black residents who remained
were among the city's poorest. The Great Recession hit them the hardest
while the recovery offered them the least. About a third of South L.A.'s
black residents now live in poverty.

``It has been a vicious barrage of public and private policies and
actions that have placed and will continue to place black individuals
and families into a downward spiral into poverty,'' said Chancela
Al-Mansour, the director of a local housing advocacy group.

And the poor are least able to cope with L.A.'s skyrocketing housing
costs. As rents rise across the county, they are rising faster in South
L.A. Black households there have a particularly hard time affording
rent.

Across the county, a third of black households experience severe rent
burdens, with their housing costs equaling half or more of their income.

In South L.A., the \textbf{share of black households experiencing severe
rent burdens} is about 50 percent.

33\%

23\%

Downtown

50\%

10

33\%

48\%

47\%

30\%

Crenshaw

57\%

South

L.A.

South Park

46\%

47\%

21\%

Inglewood

27\%

110

31\%

54\%

Westmont

Watts

50\%

33\%

405

Gardena

39\%

Compton

11\%

37\%

42\%

33\%

Downtown

10

48\%

33\%

47\%

Crenshaw

57\%

South

L.A.

South Park

46\%

47\%

Inglewood

110

31\%

54\%

Westmont

Watts

33\%

405

Gardena

Compton

37\%

42\%

By The New York Times·Source:
\href{https://www.socialexplorer.com/}{Social Explorer} analysis of
census data from IPUMS.

South L.A. --- once the heart of the city's black life --- is now the
\href{http://publichealth.lacounty.gov/ha/reports/LAHealthBrief2011/HousingHealth/SD_Housing_Fs.pdf}{epicenter
of housing instability} for black Angelenos. It is where black residents
are most at risk of falling into homelessness.

Latinos in the area do not experience homelessness at nearly the same
rate as African-Americans. Experts cite a
\href{https://dornsife.usc.edu/assets/sites/731/docs/RootsRaices_Full_Report_CSII_USC_Final2016_Web_Small.pdf}{variety
of reasons}. Rates of homelessness among white Angelenos are similar to
those of Latinos, at about one in 100 residents. Asians and Pacific
Islanders in Los Angeles experience homelessness at even lower rates.

\hypertarget{from-south-central-to-skid-row}{%
\subsection{From South Central to Skid
Row}\label{from-south-central-to-skid-row}}

Mr. Wynn, now 60, had a troubled home life from the start. His parents,
he said, were married ``only on paper,'' and he never lived with his
father.

He shuttled between the homes of other family members --- in Compton,
East L.A., South L.A. --- while his mother worked long hours at a county
hospital.

At the time, the city around him was in turmoil. He recalled being
terrified during the 1965 Watts riots, watching rioters loot a corner
liquor store. ``They drove up and down the street saying, `Burn baby
burn,''' he said.

\includegraphics{https://static01.graylady3jvrrxbe.onion/packages/flash/multimedia/ICONS/transparent.png}

\includegraphics{https://static01.graylady3jvrrxbe.onion/newsgraphics/2019/12/05/homeless/assets/images/homeless_020-2000.jpg}

Mr. Wynn lived in a rented home in South Central when he was growing
up.Bethany Mollenkof for The New York Times

Mr. Wynn found a measure of stability when he moved with his mother to
Inglewood, part of the first wave of black families moving to what was
then a white neighborhood. The old lines that separated black areas from
white ones were shifting.

``Blacks moved in, and I watched the white flight out,'' he said.

He eventually came to like his new, mostly white school, but it was
tough at first.

``I never really felt true prejudice until I moved to Inglewood,'' he
said, recalling an episode of being accosted by the police at an ice
cream parlor.

Mr. Wynn had a lot of dreams over the years. He fell in love with
fashion. He saw a black man present the news on television, and thought
he too could do that.

But he struggled with depression. And when he came out as gay, his
father didn't accept him. ``I did everything in my power to be
straight,'' he said.

He made it to college, studying biology at the historically black Clark
Atlanta College. He thought of being a doctor.

But he got caught up in drug running and was arrested with 250 pounds of
marijuana. He avoided prison time, but has a felony on his record, which
has haunted him ever since.

``I was so limited to what I could do,'' he said.

\includegraphics{https://static01.graylady3jvrrxbe.onion/packages/flash/multimedia/ICONS/transparent.png}

\includegraphics{https://static01.graylady3jvrrxbe.onion/newsgraphics/2019/12/05/homeless/assets/images/homeless_003-2000.jpg}

The car where Mr. Wynn lived with his dog, Smokey, when he was homeless
now sits in a parking lot.Bethany Mollenkof for The New York Times

By the time he returned to Los Angeles, his mother was preparing to
retire and move to Riverside, in the Inland Empire. Black families had
for years been flowing out of Los Angeles, in search of cheaper housing
and an escape from crime. He followed to care for her and held several
jobs, including working as a shoe salesman at Nordstrom and as a
mortgage broker.

After his mother died of cancer, Mr. Wynn couldn't get back to Los
Angeles fast enough.

``I was devastated losing my mom,'' he said. ``She was all I had.''

He left Riverside with his dog, a chihuahua and Jack Russell mix, and
about \$1,000 in cash. For a time, he stayed in motels, but the money
soon ran out and he was living in his car, a 12-year-old beat-up black
Toyota Matrix.

``It was kind of weird because every evening your mind said, `I'm going
to go home,''' he said. ``And finally reality started kicking in, and I
said, `Wow, there's no home, there's no home, there's no home to go
to.'''

\hypertarget{record-numbers-are-homeless}{%
\subsection{Record Numbers Are
Homeless}\label{record-numbers-are-homeless}}

Los Angeles voters have in recent years approved more than a billion
dollars to fund more housing for the homeless, and the city and county
are opening up new shelters all the time. But the problem continues to
worsen.

The number of people living on Los Angeles streets is at an all-time
high and growing. The annual street count from January 2019 showed
44,214 \textbf{people living without shelter} across Los Angeles County.

Hollywood

El

Monte

Downtown

10

Santa

Monica

Crenshaw

Pico

Rivera

South L.A.

Inglewood

110

Watts

Norwalk

Gardena

Compton

Number of people

without shelter

405

1,000

Torrance

600

200

Long

Beach

5 miles

Hollywood

Downtown

10

Santa

Monica

Crenshaw

South L.A.

Inglewood

110

Watts

Compton

405

Number of people

without shelter

Torrance

1,000

600

Long

Beach

200

5 miles

By The New York Times·Note: Not all areas of Los Angeles County are
shown.·Source: 2019 Los Angeles Homeless Services Authority Street Count

There are unprecedented plans to build temporary shelters with 4,000
beds and 15,000 permanent housing units for the homeless by 2026.

The estimated \textbf{number of people housed} by those efforts ---
about 27,000 combined --- would leave thousands still living on the
streets.

Hollywood

El

Monte

Downtown

10

Santa

Monica

Crenshaw

Pico

Rivera

South L.A.

Inglewood

110

Watts

Capacity of

new housing

Norwalk

Gardena

Compton

1,000 people

405

600

200

Torrance

Permanent housing

Temporary shelters

Long

Beach

5 miles

Hollywood

Downtown

10

Santa

Monica

Crenshaw

South L.A.

Inglewood

110

Watts

Capacity of

new housing

Compton

405

1,000 people

Torrance

600

200

Long

Beach

Permanent housing

Temporary shelters

5 miles

By The New York Times·Note: Not all areas of Los Angeles County are
shown.·Source: Los Angeles County Chief Executive Office; Los Angeles
Housing and Community Investment Department

And none of this construction is likely to help solve the specific
challenge of black homelessness, which experts say requires efforts that
go beyond building more housing units or opening more shelters. Those
efforts must address bias in everything from the rental markets to
employment to criminal justice.

A
\href{https://www.lahsa.org/documents?id=2823-report-and-recommendations-of-the-ad-hoc-committee-on-black-people-experiencing-homelessness}{report
on black homelessness} published a year ago by the Los Angeles Homeless
Services Authority found racism to be the root cause, saying that black
Angelenos continue to face discrimination in many areas. Over the past
50 years, for example, black homeownership in L.A. County has declined
to 36 percent from 44 percent.

Mr. Lynn, the homelessness agency director, pointed to the criminal
justice system, saying, ``There is probably no more single significant
factor than incarceration in terms of elevating somebody's prospects of
homelessness.''

The black overrepresentation in homelessness roughly tracks the same
dynamic in California's prisons: Black people make up about 6 percent of
the state's population but about 30 percent of those in prison.

\includegraphics{https://static01.graylady3jvrrxbe.onion/packages/flash/multimedia/ICONS/transparent.png}

\includegraphics{https://static01.graylady3jvrrxbe.onion/newsgraphics/2019/12/05/homeless/assets/images/homeless_010-2000.jpg}

Tent camps are a common sight across Los Angeles.Bethany Mollenkof for
The New York Times

Inmates return to the world with a criminal record that gets in the way
of getting a job or securing a mortgage or lease on an apartment.

Advocates for the homeless have been pushing state and local officials
to spend more on re-entry programs to stop the revolving door between
prison cells and homeless camps.

``If you are a felon, you can't get a good job,'' said Marlon Jackson,
44, who is homeless in South L.A.

Mr. Jackson does not have a felony record, but many of his friends do.
``Most black people I have been around all my life, all my friends, have
felonies,'' he said. ``Everybody I grew up with is either dead or in
jail or they are out of jail with felonies.''

\hypertarget{from-the-streets-to-supportive-housing}{%
\subsection{From the Streets to Supportive
Housing}\label{from-the-streets-to-supportive-housing}}

At Mr. Wynn's lowest point he was living in his car, dealing with
hypertension and heart problems, in addition to his persistent
depression. His weight dropped from 190 pounds to 130 pounds on his
six-foot-three-inch frame.

``I didn't want to commit suicide, but I wanted to die,'' he said.

He was eventually accepted into a residential program at the Weingart
Center on Skid Row, which served as a bridge between the streets and a
new apartment. He took part in group therapy, and started to feel less
alone.

``There's a lot of depressed black men in Los Angeles,'' he said.

For a few months now, after being on the streets for almost three years,
Mr. Wynn has lived in a small one-bedroom apartment provided by the
county. He has a new dream: to learn sign language so that he can find a
job working with the deaf.

But he hasn't slept in his bed, which still has the plastic cover on it.

He calls the bedroom his ``depression room'' and sleeps instead on a
small black couch. This is not uncommon: some formerly homeless people
set up their tents inside their apartments.

Of the many losses he has experienced, one continues to torment him.

\includegraphics{https://static01.graylady3jvrrxbe.onion/newsgraphics/2019/12/05/homeless/assets/homeless_056-2000_x2.jpg}

Bethany Mollenkof for The New York Times

\includegraphics{https://static01.graylady3jvrrxbe.onion/newsgraphics/2019/12/05/homeless/assets/timmy_end_small.png}

timmy\_end

\begin{itemize}
\item
\item
\item
\end{itemize}

 When he left Riverside, he put his mother's belongings --- furniture,
family photos --- in storage. When he failed to make the payments, the
company auctioned everything off.

 When he left Riverside, he put his mother's belongings --- furniture,
family photos --- in storage. When he failed to make the payments, the
company auctioned everything off.

 For Mr. Wynn, whose possessions could fit in one trash bag, it was one
more sign that the promise of Los Angeles had eluded him --- was denied
to him, he would say --- and so many others.

 For Mr. Wynn, whose possessions could fit in one trash bag, it was one
more sign that the promise of Los Angeles had eluded him --- was denied
to him, he would say --- and so many others.

Sources: Where not otherwise noted, annual homelessness figures are
based on administrative records from the Los Angeles County Chief
Executive Office for the fiscal year ending June 30, 2019, and an
analysis of census data by \href{https://www.socialexplorer.com/}{Social
Explorer}. South Los Angeles neighborhood boundaries are from the
University of Southern California.

Read 275 Comments

\begin{itemize}
\item
\item
\item
\item
\end{itemize}

Advertisement

\protect\hyperlink{after-bottom}{Continue reading the main story}

\hypertarget{site-index}{%
\subsection{Site Index}\label{site-index}}

\hypertarget{site-information-navigation}{%
\subsection{Site Information
Navigation}\label{site-information-navigation}}

\begin{itemize}
\tightlist
\item
  \href{https://help.nytimes3xbfgragh.onion/hc/en-us/articles/115014792127-Copyright-notice}{©~2020~The
  New York Times Company}
\end{itemize}

\begin{itemize}
\tightlist
\item
  \href{https://www.nytco.com/}{NYTCo}
\item
  \href{https://help.nytimes3xbfgragh.onion/hc/en-us/articles/115015385887-Contact-Us}{Contact
  Us}
\item
  \href{https://www.nytco.com/careers/}{Work with us}
\item
  \href{https://nytmediakit.com/}{Advertise}
\item
  \href{http://www.tbrandstudio.com/}{T Brand Studio}
\item
  \href{https://www.nytimes3xbfgragh.onion/privacy/cookie-policy\#how-do-i-manage-trackers}{Your
  Ad Choices}
\item
  \href{https://www.nytimes3xbfgragh.onion/privacy}{Privacy}
\item
  \href{https://help.nytimes3xbfgragh.onion/hc/en-us/articles/115014893428-Terms-of-service}{Terms
  of Service}
\item
  \href{https://help.nytimes3xbfgragh.onion/hc/en-us/articles/115014893968-Terms-of-sale}{Terms
  of Sale}
\item
  \href{https://spiderbites.nytimes3xbfgragh.onion}{Site Map}
\item
  \href{https://help.nytimes3xbfgragh.onion/hc/en-us}{Help}
\item
  \href{https://www.nytimes3xbfgragh.onion/subscription?campaignId=37WXW}{Subscriptions}
\end{itemize}
