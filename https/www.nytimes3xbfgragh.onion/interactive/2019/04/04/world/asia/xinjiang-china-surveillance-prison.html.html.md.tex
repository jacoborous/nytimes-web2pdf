 **NYTimes.com no longer supports Internet Explorer 9 or earlier. Please
upgrade your browser.
\href{http://www.nytimes3xbfgragh.onion/content/help/site/ie9-support.html}{LEARN
MORE »}

**Sections

**Home

**Search

\hypertarget{the-new-york-times}{%
\subsection{\texorpdfstring{\href{http://www.nytimes3xbfgragh.onion/}{The
New York Times}}{The New York Times}}\label{the-new-york-times}}

\hypertarget{-world-}{%
\subsubsection{\texorpdfstring{ \href{/section/world}{World}
}{ World }}\label{-world-}}

 \href{/section/world/asia}{Asia Pacific} \textbar{}How China Turned a
City Into a Prison

**Close search

\hypertarget{site-search-navigation}{%
\subsection{Site Search Navigation}\label{site-search-navigation}}

Search NYTimes.com

**Clear this text input

Go

\url{https://nyti.ms/2HW7P8j}

\hypertarget{site-navigation}{%
\subsection{Site Navigation}\label{site-navigation}}

\hypertarget{site-mobile-navigation}{%
\subsection{Site Mobile Navigation}\label{site-mobile-navigation}}

\hypertarget{how-china-turned-a-city-into-a-prison}{%
\section{How China Turned a City Into a
Prison}\label{how-china-turned-a-city-into-a-prison}}

A surveillance state reaches new heights.

Xinjiang Dispatch

\hypertarget{how-china-turned-a-city-into-a-prison-1}{%
\section{How China Turned a City Into a
Prison}\label{how-china-turned-a-city-into-a-prison-1}}

\subsection{}

A surveillance state reaches new heights.

By \textbf{Chris Buckley}, \textbf{Paul Mozur} and \textbf{Austin Ramzy}

April 4, 2019

Start

\includegraphics{//static01.graylady3jvrrxbe.onion/packages/flash/multimedia/ICONS/transparent.png}

We visited Kashgar several times to see what life is like. We couldn't
interview residents --- that would have been too risky for them, because
we were constantly followed by the police. But the restrictions were
everywhere.

\includegraphics{//static01.graylady3jvrrxbe.onion/packages/flash/multimedia/ICONS/transparent.png}

Every 100 yards or so, the police stand at checkpoints with guns,
shields and clubs. Many are Uighurs. The surveillance couldn't work
without them.

\includegraphics{//static01.graylady3jvrrxbe.onion/packages/flash/multimedia/ICONS/transparent.png}

Muslim minorities line up, stone-faced, to swipe their official identity
cards. At big checkpoints, they lift their chins while a machine takes
their photos, and wait to be notified if they can go on.

\includegraphics{//static01.graylady3jvrrxbe.onion/packages/flash/multimedia/ICONS/transparent.png}

The police sometimes take Uighurs' phones and check to make sure they
have installed compulsory software that monitors calls and messages.

\includegraphics{//static01.graylady3jvrrxbe.onion/packages/flash/multimedia/ICONS/transparent.png}

\includegraphics{//static01.graylady3jvrrxbe.onion/packages/flash/multimedia/ICONS/transparent.png}

Xinjiang is in China's far west, but it feels more part of central Asia.
Ethnic minorities --- including Uighurs, Kazakhs and Tajiks ---
outnumber the Han Chinese majority here. They are mostly Sunni Muslims
with their own cultures and languages.

\includegraphics{//static01.graylady3jvrrxbe.onion/packages/flash/multimedia/ICONS/transparent.png}

Sometimes their choices made no sense. One erased this picture of a
camel, though I was able to restore it. ``In China there are no whys,''
he said.

\includegraphics{//static01.graylady3jvrrxbe.onion/packages/flash/multimedia/ICONS/transparent.png}

For Uighurs, the surveillance is even more pervasive. Neighborhood
monitors are assigned to watch over groups of families, as in this
photo. An army of millions of police and official monitors can question
Uighurs and search their homes. They grade residents for reliability. A
low grade brings more visits, maybe detention.

\includegraphics{//static01.graylady3jvrrxbe.onion/packages/flash/multimedia/ICONS/transparent.png}

This is Dilnur. She fled Kashgar to Turkey three years ago and has lost
touch with her family in Xinjiang. But she remembers the searches:
``They don't care if it's morning or night, they would come in every
time they want.''

\includegraphics{//static01.graylady3jvrrxbe.onion/packages/flash/multimedia/ICONS/transparent.png}

Orphanages have been taking away the children of detainees. We don't
know how many, but the government says that orphanages like this one
held 7,000 children across Kashgar alone last year.

Surveillance cameras are everywhere. In streets, doorways, shops,
mosques. Look at this stretch of street. We counted 20 cameras.

Chinese companies are earning a fortune selling this surveillance
technology. They make it sound like a sci-fi miracle allowing the police
to track people with laser precision.

Demo at the World Internet Conference, 2017

\includegraphics{//static01.graylady3jvrrxbe.onion/packages/flash/multimedia/ICONS/transparent.png}

But spend time in Xinjiang and you see that the surveillance state acts
more like a sledgehammer --- sweeping, indiscriminate; as much about
intimidation as monitoring.

Demo at a trade show, 2017

At the mosque, worshipers register and go through a security check.

Inside, they pray under surveillance cameras that the police can
monitor.

\includegraphics{//static01.graylady3jvrrxbe.onion/packages/flash/multimedia/ICONS/transparent.png}

Children are interrogated. ``In the kindergarten, they would ask little
children, `Do your parents read the Quran?''' Dilnur told us. ``My
daughter had a classmate who said, `My mom teaches me the Quran.' The
next day, they are gone.''

The very architecture of Kashgar has been altered to make the city
easier to control.

\includegraphics{//static01.graylady3jvrrxbe.onion/packages/flash/multimedia/ICONS/transparent.png}

The Old City, a maze-like area of mudbrick homes, has mostly been
demolished. The government said it was for safety and sanitation. But
the rebuilding has also created wider streets that are easier to monitor
and patrol.

\includegraphics{//static01.graylady3jvrrxbe.onion/packages/flash/multimedia/ICONS/transparent.png}

Some areas are still undergoing demolition and reconstruction.

\includegraphics{//static01.graylady3jvrrxbe.onion/packages/flash/multimedia/ICONS/transparent.png}

The new brick homes seem more comfortable, but Uighurs mourn their old
neighborhoods. Tourists wander the refurbished alleys, often unaware of
the ancient lanes they replaced. But visitors are kept far from the
indoctrination camps on the edge of town.

\includegraphics{//static01.graylady3jvrrxbe.onion/packages/flash/multimedia/ICONS/transparent.png}

This piece of land in southern Kashgar was empty in August 2016.

\includegraphics{//static01.graylady3jvrrxbe.onion/packages/flash/multimedia/ICONS/transparent.png}

Now this is a re-education camp with a capacity of roughly 20,000
people. The government says it is a vocational training center. A recent
satellite image shows the camp occupies more than 195,000 square meters.

\includegraphics{//static01.graylady3jvrrxbe.onion/packages/flash/multimedia/ICONS/transparent.png}

This camp is not the only one growing. These 13 camps in Kashgar have
all jumped in size, reaching 1 million square meters last year.

\textbf{Christopher Buckley}, \textbf{Paul Mozur} and \textbf{Austin
Ramzy} are foreign correspondents for The New York Times. Chris and Paul
last visited Kashgar in October.

Produced by \textbf{Josh Williams}, \textbf{Sergio Pecanha} and
\textbf{Joe Ward}.\\
Additional work by \textbf{Malachy Browne} and \textbf{Meg Felling}.

\hypertarget{more-stories-like-this}{%
\paragraph{More stories like this}\label{more-stories-like-this}}

\begin{itemize}
\tightlist
\item
  \href{https://www.nytimes3xbfgragh.onion/2018/09/08/world/asia/china-uighur-muslim-detention-camp.html}{\includegraphics{https://static01.graylady3jvrrxbe.onion/images/2018/09/04/world/00xinjiang-6/00xinjiang-6-thumbStandard.jpg}China
  Is Detaining Muslims in Vast Numbers. The Goal: `Transformation.'
  Sept. 8, 2018}
\item
  \href{https://www.nytimes3xbfgragh.onion/interactive/2018/11/25/world/asia/china-freedoms-control.html}{\includegraphics{https://static01.graylady3jvrrxbe.onion/images/2018/11/21/world/china-culture-staticpromo/china-culture-staticpromo-thumbStandard.png}How
  China's Rulers Control Society: Opportunity, Nationalism, Fear Nov.
  25, 2018}
\item
  \href{https://www.nytimes3xbfgragh.onion/2019/01/05/world/asia/china-xinjiang-uighur-intellectuals.html}{\includegraphics{https://static01.graylady3jvrrxbe.onion/images/2019/01/06/world/00xinjiang-intellectuals-2/00xinjiang-intellectuals-2-thumbStandard.jpg}China
  Targets Prominent Uighur Intellectuals to Erase an Ethnic Identity
  Jan. 5., 2019}
\end{itemize}

Map sources: Australian Strategic Policy Institute; Digital Globe

Next

How China Turned a City Into a Prison

Map sources: Australian Strategic Policy Institute; Digital Globe

Produced by \textbf{Josh Williams}, \textbf{Sergio Pecanha} and
\textbf{Joe Ward}.\\
Additional work by \textbf{Malachy Browne} and \textbf{Meg Felling}.

\hypertarget{more-on-nytimescom}{%
\subsection{More on NYTimes.com}\label{more-on-nytimescom}}

Advertisement

\hypertarget{site-information-navigation}{%
\subsection{Site Information
Navigation}\label{site-information-navigation}}

\begin{itemize}
\tightlist
\item
  \href{https://help.nytimes3xbfgragh.onion/hc/en-us/articles/115014792127-Copyright-notice}{©
  2020 The New York Times Company}
\item
  \href{https://www.nytimes3xbfgragh.onion}{Home}
\item
  \href{https://www.nytimes3xbfgragh.onion/search/}{Search}
\item
  Accessibility concerns? Email us at
  \href{mailto:accessibility@NYTimes.com}{\nolinkurl{accessibility@NYTimes.com}}.
  We would love to hear from you.
\item
  \href{https://help.nytimes3xbfgragh.onion/hc/en-us/articles/115015385887-Contact-Us}{Contact
  Us}
\item
  \href{https://www.nytco.com/careers/}{Work with us}
\item
  \href{https://nytmediakit.com/}{Advertise}
\item
  \href{https://help.nytimes3xbfgragh.onion/hc/en-us/articles/115014892108-Privacy-policy\#pp}{Your
  Ad Choices}
\item
  \href{https://help.nytimes3xbfgragh.onion/hc/en-us/articles/115014892108-Privacy-policy}{Privacy}
\item
  \href{https://help.nytimes3xbfgragh.onion/hc/en-us/articles/115014893428-Terms-of-service}{Terms
  of Service}
\item
  \href{https://help.nytimes3xbfgragh.onion/hc/en-us/articles/115014893968-Terms-of-sale}{Terms
  of Sale}
\end{itemize}

\hypertarget{site-information-navigation-1}{%
\subsection{Site Information
Navigation}\label{site-information-navigation-1}}

\begin{itemize}
\tightlist
\item
  \href{https://spiderbites.nytimes3xbfgragh.onion}{Site Map}
\item
  \href{https://help.nytimes3xbfgragh.onion/hc/en-us}{Help}
\item
  \href{https://help.nytimes3xbfgragh.onion/hc/en-us/articles/115015385887-Contact-Us?redir=myacc}{Site
  Feedback}
\item
  \href{https://www.nytimes3xbfgragh.onion/subscription?campaignId=37WXW}{Subscriptions}
\end{itemize}
