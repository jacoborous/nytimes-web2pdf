 **NYTimes.com no longer supports Internet Explorer 9 or earlier. Please
upgrade your browser.
\href{http://www.nytimes3xbfgragh.onion/content/help/site/ie9-support.html}{LEARN
MORE »}

**Sections

**Home

**Search

\hypertarget{the-new-york-times}{%
\subsection{\texorpdfstring{\href{http://www.nytimes3xbfgragh.onion/}{The
New York Times}}{The New York Times}}\label{the-new-york-times}}

\hypertarget{-world-}{%
\subsubsection{\texorpdfstring{ \href{/section/world}{World}
}{ World }}\label{-world-}}

 \href{/section/world/europe}{Europe} \textbar{}Genoa Bridge Collapse:
The Road to Tragedy

**Close search

\hypertarget{site-search-navigation}{%
\subsection{Site Search Navigation}\label{site-search-navigation}}

Search NYTimes.com

**Clear this text input

Go

\url{https://nyti.ms/2MUvBo3}

\hypertarget{site-navigation}{%
\subsection{Site Navigation}\label{site-navigation}}

\hypertarget{site-mobile-navigation}{%
\subsection{Site Mobile Navigation}\label{site-mobile-navigation}}

\hypertarget{genoa-bridge-collapse-the-road-to-tragedy}{%
\section{Genoa Bridge Collapse: The Road to
Tragedy}\label{genoa-bridge-collapse-the-road-to-tragedy}}

The New York Times has reconstructed how the disaster happened, from
beginning to end.

\href{https://www.nytimes3xbfgragh.onion/interactive/2018/09/08/world/europe/genoa-italy-bridge-italian.html}{Leggere
in italiano}

GENOA, Italy --- An off-duty firefighter, Davide Capello, had just
driven out of a tunnel and onto the main bridge over Genoa in a heavy
summer rain, when he heard a low, dull rumble all around his car. It was
not thunder.

Mr. Capello, 33, glanced upward and saw a huge cloud of white dust
rising up in the fog and rain. A white sedan 20 or 30 yards ahead of him
seemed to disappear into a void. He hit the brakes. But the emptiness
advanced toward him as the road fell away, section by section, like a
staircase to oblivion.

In a split second, his car was plummeting, nose down, the windshield
darkened by dust and concrete blocks flying past him. ``I am dead! I am
dead!'' he cried.

He was in free fall.

The bridge he was driving across, a viaduct designed by Riccardo
Morandi,
\href{https://www.nytimes3xbfgragh.onion/2018/08/14/world/europe/italy-genoa-bridge-collapse.html}{collapsed
that day}, Aug. 14,
\href{https://www.nytimes3xbfgragh.onion/2018/08/18/world/europe/italy-genoa-morandi-bridge-funeral.html}{leaving
43 people dead} as dozens of cars fell some 150 feet onto the riverbed,
railroad tracks and gritty streets below.

The collapse of the bridge --- a signature of the port city, a source of
deep civic pride, and an
\href{https://www.nytimes3xbfgragh.onion/2018/08/16/world/europe/genoa-morandi-bridge-political-paralysis.html}{indispensable
daily transportation link} for thousands --- has scarred Genoa and set
off a bitter debate in Italy about who bears responsibility for the
disaster and precisely what caused it.

Those questions remain under investigation by the chief magistrate of
the region, Francesco Cozzi, and a team of engineers, security and
government officials.

The New York Times has recreated what happened by using investigators'
descriptions of a central piece of evidence --- video footage captured
by a security camera.

About 5 miles

to central Genoa

About 30 miles

to Savona

About 5 miles

to central Genoa

About 30 miles

to Savona

About 5 miles

to central Genoa

About 30 miles

to Savona

About 30 miles

to Savona

About 5 miles

to central Genoa

About 30

miles to

Savona

About 5

miles to

central

Genoa

Southern

stays

Southern

stays

Southern

stays

Southern

stays

Southern

stays

Northern

stays

Connector

Connector

Northern

stays

Connector

Connector

Northern

stays

Connector

Connector

Northern

stays

Connector

Connector

Northern

stays

Connector

Connector

Tower

Tower

Tower

Tower

Tower

\begin{itemize}
\item
\item
\item
\item
\item
\item
\end{itemize}

The roadway connected to this tower ran nearly 150 feet above a dry
riverbed and a railway.

The cables in the bridge's southern stays broke, causing the stays
suddenly to go slack. Sections of the road started to tilt southward.

As sections of the roadway began to break apart, the weight of the road
rested fully on the northern stays \ldots{}

\ldots{}and the remaining cable and concrete stays broke apart.

The severed ends of the stays dangled from the top of the tower as
pieces of the roadway crashed to the ground, some landing upside down.

Finally the nearly 300-foot tower collapsed into the center of the pile
of rubble.

The southern supports that initially went slack are the same in which a
structural engineering professor at the Politecnico di Milano, Carmelo
Gentile, found troubling signals of corrosion or other possible damage
in tests performed last October.

He warned the company that manages the bridge, Autostrade per l'Italia,
or Highways for Italy, but he said that it never followed up on his
recommendation to perform a fuller computer study and to outfit the
bridge with permanent sensors.

``Probably they underestimated the importance of the information,''
Professor Gentile said in an interview.

Autostrade never denied Professor Gentile's findings, but said that no
one had perceived any urgency. **** In a statement this week, Autostrade
said that Professor Gentile's suggestions were included in a proposal to
retrofit the viaduct that was approved in June, but blamed the Ministry
of Infrastructure for months of delays in authorizing the work.

The footage from the security camera has not been previously reported in
detail, and has not been made public. But it was described to The New
York Times by two senior members of the investigation, including a
member of the Genoa financial police, which is commanded by Col. Filippo
Ivan Bixio.

Interviews with dozens of rescue workers, investigators and expert
engineers, as well as an examination of drone and helicopter footage and
the rubble itself, allow for a sketch of the collapse from beginning to
end.

\includegraphics{https://static01.graylady3jvrrxbe.onion/packages/flash/multimedia/ICONS/transparent.png}

A view of the east side of the bridge from a nearby apartment. Nadia
Shira Cohen for The New York Times

Because the investigation is in its early stages, conclusions could
still change. For example, an unseen failure in the roadway structure or
a shift in the tower foundations could have triggered the failure of the
stays. Further work remains to eliminate such possibilities.

Still, the rumble Mr. Capello heard, according to Professor Gentile, was
most probably the initial snapping of the steel within the supports he
had warned about.

Barring the emergence of further evidence, said Vijay K. Saraf,
principal engineer at Exponent, an infrastructure and construction
consulting firm in Menlo Park, Calif., ``everything you have is
consistent with the failure of the south stays.''

From that point, with dynamic, thousand-ton loads weighing on the
remaining structure, the bridge had no chance of remaining upright,
Professor Gentile said.

The video shows that it took no more than three or four seconds for the
other elements of the bridge, burdened with the additional weight, to
crumble, too.

``It was not possible to save the bridge, but maybe it was possible to
save the persons who died in the collapse,'' he said.

\hypertarget{a-simple-design-invited-dangers}{%
\subsection{A Simple Design Invited
Dangers}\label{a-simple-design-invited-dangers}}

When it was built in the 1960s, Genoa's viaduct was more than just a
bridge. It was a 3,600-foot journey of artistry and innovation that had
garnered its designer, Riccardo Morandi, fame in architectural and
engineering circles around the world.

Its figure was so light and airy it seemed to have leapt from an elegant
line drawing on an engineer's gridded pad to where it soared over
Genoa's deep, rolling valleys.

The signature elements were three narrow, A-frame towers that rose
nearly 300 feet, paired with just 12 stays --- supports that extended
from the towers and fastened to the sides of the roadway, holding it up.

Even in a country with countless historical structures, it became ``one
of Italy's most important bridges,'' said Marzia Marandola, an adjunct
professor of architectural history at Sapienza University in Rome who is
a specialist in Mr. Morandi's work.

The structure ``gave an identity to the site and to the entire area,''
she said. ``It managed to become a part of the landscape.''

\includegraphics{https://static01.graylady3jvrrxbe.onion/packages/flash/multimedia/ICONS/transparent.png}

A view of the viaduct during the construction in 1967. Mario De
Biasi/Mondadori Portfolio, via Getty Images

\includegraphics{https://static01.graylady3jvrrxbe.onion/packages/flash/multimedia/ICONS/transparent.png}

A man at work on one of the bridge's stays. Mario De Biasi/Mondadori
Portfolio, via Getty Images

Its beauty was its simplicity. But engineers gradually recognized that
the structure had so few crucial supports that if even one of them
failed, an entire section could collapse.

``There is no robustness, or the possibility of redistribution of the
forces,'' said Massimo Majowiecki, an architect and engineer in Bologna,
northern Italy.

That lack of redundancy, as it is now often called, ``is not necessarily
inconsistent with how bridges were designed in the 1960s,'' said Donald
Dusenberry, a structural engineer with Simpson Gumpertz \& Heger in
Boston.

For that reason, the design ``is hard to criticize,'' Mr. Dusenberry
said, although he said it placed a premium on inspection and
maintenance.

Andrew Herrmann, a structural engineer who is a former president of the
American Society of Civil Engineers, described the peril in the clearest
possible terms.

``If you lose one stay,'' Mr. Herrmann said, ``the whole thing comes
down.''

Those concerns were far from theoretical, not least because of another
of Mr. Morandi's innovations: He suspended the roadway from the stays,
essentially cables bundled together and encased in what is known as a
prestressed concrete shell.

Inside the Concrete-Encased Cable Stays

Prestressed concrete

Secondary

steel cables

Main steel cables

Inside the Concrete-Encased Cable Stays

Prestressed concrete

Secondary

steel cables

Main steel cables

Inside the Concrete-

Encased Cable Stays

Prestressed concrete

Secondary

steel cables

Main steel cables

He believed using the system would reduce the sway of the bridge.
Structural engineers seemed to agree.

But Mr. Morandi also believed that the concrete coating would protect
the steel cables inside from the wear and tear of the elements.

``Concrete structures seemed to be eternal,'' Mr. Majowiecki said.
``This was the mentality.''

In that hope, he added, Mr. Morandi was greatly mistaken.

The concrete of the day turned out to be highly vulnerable to
degradation, worsened perhaps by salty air from the Mediterranean Sea
and the harsh fumes from nearby factories.

Cracks in the concrete shell let water in, and
\href{https://www.nytimes3xbfgragh.onion/2018/08/15/world/europe/italy-genoa-bridge-collapse.html}{the
steel began corroding} almost as soon as the bridge was opened for
traffic in 1967. But unlike with bare cables, any corrosion was hidden
deep inside, making it hard to detect.

By the late 1970s, concrete on the bridge had already begun visibly
deteriorating, forcing Mr. Morandi, who died in 1989, to defend his
creation.

In 1979 and in 1981, Mr. Morandi himself conducted surveys of the bridge
and concluded that the roadway and elements of the towers were already
degraded.

The findings raised alarms about similar Morandi structures around the
world.

How the Morandi Bridge Compares With Modern Cable-Stayed Bridges

Concrete-encased

cable stays

If a single stay fails, the bridge is more likely to collapse.

Tension

forces

Compression

forces

Pier supports

Load-bearing tower

MORANDI BRIDGE

Modern cable-stayed bridges use many more stays, so the failure of a
stay is less likely to affect the structure of the bridge.

Tension

forces

Steel cable stays

Compression

forces

Load-bearing tower

MODERN CABLE-STAYED BRIDGE

Concrete-encased

cable stays

If a single stay fails, the bridge is more likely to collapse.

Tension

forces

Compression

forces

Pier supports

Load-bearing tower

MORANDI BRIDGE

Modern cable-stayed bridges use many more stays, so the failure of a
stay is less likely to affect the structure of the bridge.

Tension

forces

Steel cable

stays

Compression

forces

Load-bearing tower

MODERN CABLE-STAYED BRIDGE

If a single stay fails, the bridge is more likely to collapse.

Concrete-encased

cable stays

Tension

forces

Compression

forces

Pier supports

Load-bearing tower

MORANDI BRIDGE

Modern cable-stayed bridges use many more stays, so the failure of a
stay is less likely to affect the structure of the bridge.

Tension

forces

Steel cable

stays

Compression

forces

Load-bearing tower

MODERN CABLE-STAYED BRIDGE

All the supports on a similar Morandi bridge in Venezuela were replaced
by cables with a protective sheath, but no concrete, said David
Goodyear, chief bridge engineer at T.Y. Lin International, an
engineering services firm in San Francisco.

In the late 1990s, corrosion and other problems with the easternmost of
the three towers on the Genoa bridge became serious enough that their
supports were given a similar refurbishment.

For reasons it has not fully explained, Autostrade, which took over
management of the bridge in 1999, did not carry out the same operation
on the supports of the other two towers --- including the tower that
collapsed.

But Autostrade was concerned enough by the bridge's apparent
deterioration that, last October, it asked Professor Gentile to test for
damage that might be hidden deep inside the concrete.

\hypertarget{warnings-of-potential-weakness}{%
\subsection{Warnings of Potential
Weakness}\label{warnings-of-potential-weakness}}

When it comes to bridges, Professor Gentile is the closest thing to a
musician that exists among structural engineers. He listens to the
sounds that bridges make.

From those sounds, he determines whether the bridges are safe. Placing
small devices at various points, he has performed his tests on some 300
bridges around the world.

Each part vibrates much like a guitar string: higher loads on the
element, like a tighter string, produce higher frequencies; bigger
elements, like thicker strings, produce lower notes.

\includegraphics{https://static01.graylady3jvrrxbe.onion/packages/flash/multimedia/ICONS/transparent.png}

The stays on the east side of the Morandi Bridge were reinforced with
new cables in the 1990s. No such work was done on the stays that
collapsed. Nadia Shira Cohen for The New York Times

\includegraphics{https://static01.graylady3jvrrxbe.onion/packages/flash/multimedia/ICONS/transparent.png}

Clearing debris near the tower that collapsed. Nadia Shira Cohen for The
New York Times

Beyond the frequencies, Professor Gentile also tests to see if the
vibrations have the same smooth, predictable wave --- like those that
would make lovely notes on a violin. Consistency indicates integrity.

When the sound is discordant, the work must be accompanied by further
studies, usually involving computer models, to determine exactly what is
wrong.

Over four nights last October, Professor Gentile recorded the
frequencies of the Morandi bridge. Most of the stays on the two towers
he tested sounded fine, Professor Gentile said. But the two southern
stays on what was known as Tower 9 sounded wrong, he found. They were
like damaged strings.

He suspected corrosion of the cables hidden inside the concrete, some
problem with the points where the supports connected to the tower or the
roadway, or problems elsewhere in the structure, he said.

To be sure, he recommended that the bridge be permanently outfitted with
sensors and that he undertake additional studies to pin down the reason
for what he called ``anomalous'' findings.

The Autostrade subcontractor in charge of the work never contacted him
after his contract expired on Oct. 31, Professor Gentile said.

Autostrade, which handles media queries on behalf of the subcontractor,
Spea Engineering, declined to comment.

\hypertarget{questions-of-management}{%
\subsection{Questions of Management}\label{questions-of-management}}

Autostrade never raised any specific alarm regarding Genoa's viaduct
with the Ministry of Infrastructure, according to Italian news reports.
Managers at the company, however, exchanged messages noting
``criticalities'' in the bridge in the weeks before the collapse, police
investigators told The New York Times. But the officials said it was not
yet clear to what the company managers were referring specifically, or
how aware Autostrade was of the real condition of the bridge.

The company also said that it ``promptly fulfilled the concession
obligations'' it committed to, including upkeep of the bridge.

But maintenance costs around Genoa, with its aging infrastructure, were
on average twice as high as elsewhere in Italy, and four times as high
for bridges, viaducts and overpasses, according to company documents
published on their website after the bridge collapse.

\includegraphics{https://static01.graylady3jvrrxbe.onion/packages/flash/multimedia/ICONS/transparent.png}

The collapsed Morandi bridge seen from the Coronata Hill, on the west
side. Nadia Shira Cohen for The New York Times

There is now fierce debate in Italy over whether the company did in fact
do enough --- along with calls to roll back the privatization of much of
Italy's roadways.

Autostrade won the concession to run nearly half of Italy's highways
from a cash-strapped Italian government, starting in 1999. After that,
there were no major renovations of the Morandi bridge.

In late 2017 or early 2018, tests indicated that the bridge had weakened
by an ``average'' of 10 to 20 percent, Roberto Ferrazza, superintendent
at the local ministry of infrastructure and transportation, told The
Times in an interview.

Experts in bridge construction say it is notoriously difficult to
measure the exact amount of degradation of steel elements buried in
concrete, in the way that they are on the Morandi bridge.

``There's nothing more imprecise than trying to evaluate the condition
of internal cables,'' said Gary J. Klein, a member of the National
Academy of Engineering in the United States who studies structural
failures and is executive vice president at Wiss, Janney, Elstner, an
engineering and architecture firm in Northbrook, Ill. ``It's a very
imperfect science.''

Because the weakest point could be anywhere on the structure, Mr. Klein
added, ``you have to be in the right place at the right time, and so I'm
very skeptical of the accuracy of any such estimate.''

A decision was made, by Autostrade in accordance with the Ministry of
Infrastructure, to refurbish and repair the stays on two towers,
including the one that would collapse. But the work was not yet
scheduled.

\includegraphics{https://static01.graylady3jvrrxbe.onion/packages/flash/multimedia/ICONS/transparent.png}

A piece of roadway narrowly missed residential buildings under the east
side of the bridge. Nadia Shira Cohen for The New York Times

There were so many constant small repairs that two years ago, an
engineer at the University of Genoa, Antonio Brencich, wrote a paper
recommending that the entire bridge be replaced.

In a brief interview near the bridge's rubble, Dr. Brencich, who was
part of the team investigating the collapse, said it had been like
owning a car that needed constant fixing --- at some point, it would
make more sense to buy a new one.

But that would have been costly.

Immediately after the collapse of the Morandi bridge, Italy's new
populist governing coalition blamed Autostrade for the catastrophe and
accused the company of poor maintenance.

The anti-establishment Five Star Movement in particular has threatened
to revoke the company's contract and impose fines of hundreds of
millions of euros.

But their far-right partner in the coalition, the League, led by the
deputy prime minister, Matteo Salvini, has been more cautious.

The party, when it was still named the Northern League, received a
150,000-euro donation from Autostrade, which has given to other
political parties as well.

In 2008, Mr. Salvini and his Northern League voted in Parliament in
favor of renewing the license to Autostrade.

\hypertarget{it-was-a-war-scene}{%
\subsection{`It Was a War Scene'}\label{it-was-a-war-scene}}

Once the roadway dropped out from under him, Mr. Capello had no idea how
long he fell. Or how far.

He thought he was surely dead, he told The Times in an interview.

The fall shattered his rear window. He touched his head and his neck,
feeling for blood. He checked his hands, too. His seatbelt, still
fastened, was brushing against his neck. He was all right, but needed
help.

Outside, the rain was incessant, creating a slick mess. One truck from
the bridge had landed on the road underneath, blocking traffic. The
hundreds of water bottles it had been carrying were scattered
everywhere.

\includegraphics{https://static01.graylady3jvrrxbe.onion/packages/flash/multimedia/ICONS/transparent.png}

Davide Capello, a firefighter, who survived the collapse. ``As I walked,
I looked back and saw that bridge was gone,'' he said. ``Only then I
realized the magnitude of the disaster.'' Nadia Shira Cohen for The New
York Times

``Trying to reach the site was like walking on bars of soap,'' said
Sergio Olcese, one of the first firefighters to arrive.

Some cars still lay on the bridge roadway, but were so flattened by the
heavy concrete debris that had rained down, it was hard even to make out
their model. Others were dangling from steel cables.

A few trucks rested on their sides in the fields, after a 50-yard flight
from the bridge. The smell of gas filled the air. But the heavy rain
dampened the dust of the pulverized concrete.

Some remember the eerie silence, others the firefighters shouting
directions.

Mr. Olcese could hear a mother calling out to her daughter, who was
buried alive under the rubble. ``Camilla! Camilla!,'' she screamed.

``Her voice begging us to save her daughter first is the only voice I
remember from the bridge,'' he recalled.

His team dug and moved chunks of concrete for more than an hour and a
half to free the girl and her mother, who was lying on top of the pile
that had buried her daughter, while holding her hand. When the
firefighters managed to extract the women alive from the rubble, they
were exhausted, he said.

``It was a war scene,'' said Maurizio Volpara, an experienced fireman
and coordinator of a team that rescued a man from a van dangling from a
steel cable, 25 yards in the air.

The cabin of the van was facing down, toward the ground, he said, and
the back had been ``bombarded'' by concrete blocks that had plummeted
from the bridge's deck.

The young man trapped inside, his face pushed against the dashboard, was
screaming, ``Please come and get me! Get me out of here!''

Mr. Volpara recalled people screaming everywhere --- a handful of
survivors fallen from the bridge, rescuers, personnel from the city's
garbage company whose warehouse was right next to the bridge, and
finally residents.

``It was raining so strongly and had rained all morning,'' Giuseppe
Crosetti, another firefighter, said. ``Our boots sank in the grass,
under the weight of the equipment.''

\includegraphics{https://static01.graylady3jvrrxbe.onion/packages/flash/multimedia/ICONS/transparent.png}

A memorial to the victims of the disaster in front of a chapel on the
Cornigliano bridge overlooking the fallen Morandi viaduct. Nadia Shira
Cohen for The New York Times

Residents helped the rescuers to carry the heavy equipment needed to cut
open cars and free possible survivors.

Inside Mr. Capello's car, the sports radio he had on was still playing.
He stretched out a hand to reach the touch screen on the dashboard. He
then dialed the Italian equivalent of 911.

The line was busy, so he turned to one of his most frequent calls, his
own firehouse in Savona, a city 30 miles west of Genoa.

``The bridge collapsed and I am in here, hanging in the balance,'' he
told his colleague. ``I am between Genoa East and Genoa West highway
exits, and it was full of cars all around me.''

Once he knew that the firefighters were on their way, he felt relieved.
He dialed two more numbers.

``I am all right,'' he told his girlfriend, whom he had been with just
an hour earlier, before filling the tank and setting off to Genoa. ``The
bridge collapsed, but I am alive, don't worry. I don't have a scratch.''

Mr. Capello's fiance was in disbelief, he said. ``What bridge?'' she
asked. He had to cut her short. He wanted to reassure his parents, too,
and he knew he had to get out of the car as soon as possible.

``I feared I'd never hear their voices again,'' Mr. Capello said.

When he phoned, his father, a retired fireman with 30 years of service,
told him: ``Get out of the car immediately!''

Then, he heard the voice of a young man screaming outside. ``Is anyone
in there?''

``Help me! I am here,'' Mr. Capello replied, unfastening his seatbelt,
and climbing over the seats, through the smashed rear window.

``Get out of there,'' a police officer shouted.

He was on top of a tall pile of rubble. The young man living **** in the
area, who had called for him was a local resident and helped him climb
down.

``As I walked, I looked back and saw that the bridge was gone,'' he said
with a shaking voice. ``Only then did I realize the magnitude of the
disaster.''

Bystanders looked at him incredulously, he said, as he walked away in
his blue rain jacket with Italy's national flag sewn on it, gray shorts
and white sneakers, soaking wet.

``They looked at me as if I was a phantom,'' he said.

Sources: Genoa financial police; Italian security officials; Vijay K.
Saraf, principal engineer, Exponent; Italian National Fire Department

Produced by Andrew Rossback. Additional work by Anjali Singhvi.

Sources: Genoa financial police; Italian security officials; Vijay K.
Saraf, principal engineer, Exponent; Italian National Fire Department

Produced by Andrew Rossback. Additional work by Anjali Singhvi.

\hypertarget{more-on-nytimescom}{%
\subsection{More on NYTimes.com}\label{more-on-nytimescom}}

Advertisement

\hypertarget{site-information-navigation}{%
\subsection{Site Information
Navigation}\label{site-information-navigation}}

\begin{itemize}
\tightlist
\item
  \href{https://help.nytimes3xbfgragh.onion/hc/en-us/articles/115014792127-Copyright-notice}{©
  2020 The New York Times Company}
\item
  \href{https://www.nytimes3xbfgragh.onion}{Home}
\item
  \href{https://www.nytimes3xbfgragh.onion/search/}{Search}
\item
  Accessibility concerns? Email us at
  \href{mailto:accessibility@NYTimes.com}{\nolinkurl{accessibility@NYTimes.com}}.
  We would love to hear from you.
\item
  \href{https://help.nytimes3xbfgragh.onion/hc/en-us/articles/115015385887-Contact-Us}{Contact
  Us}
\item
  \href{https://www.nytco.com/careers/}{Work with us}
\item
  \href{https://nytmediakit.com/}{Advertise}
\item
  \href{https://help.nytimes3xbfgragh.onion/hc/en-us/articles/115014892108-Privacy-policy\#pp}{Your
  Ad Choices}
\item
  \href{https://help.nytimes3xbfgragh.onion/hc/en-us/articles/115014892108-Privacy-policy}{Privacy}
\item
  \href{https://help.nytimes3xbfgragh.onion/hc/en-us/articles/115014893428-Terms-of-service}{Terms
  of Service}
\item
  \href{https://help.nytimes3xbfgragh.onion/hc/en-us/articles/115014893968-Terms-of-sale}{Terms
  of Sale}
\end{itemize}

\hypertarget{site-information-navigation-1}{%
\subsection{Site Information
Navigation}\label{site-information-navigation-1}}

\begin{itemize}
\tightlist
\item
  \href{https://spiderbites.nytimes3xbfgragh.onion}{Site Map}
\item
  \href{https://help.nytimes3xbfgragh.onion/hc/en-us}{Help}
\item
  \href{https://help.nytimes3xbfgragh.onion/hc/en-us/articles/115015385887-Contact-Us?redir=myacc}{Site
  Feedback}
\item
  \href{https://www.nytimes3xbfgragh.onion/subscription?campaignId=37WXW}{Subscriptions}
\end{itemize}
