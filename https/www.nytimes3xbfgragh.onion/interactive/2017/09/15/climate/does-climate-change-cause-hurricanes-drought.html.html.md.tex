 **NYTimes.com no longer supports Internet Explorer 9 or earlier. Please
upgrade your browser.
\href{http://www.nytimes3xbfgragh.onion/content/help/site/ie9-support.html}{LEARN
MORE »}

**Sections

**Home

**Search

\hypertarget{the-new-york-times}{%
\subsection{\texorpdfstring{\href{http://www.nytimes3xbfgragh.onion/}{The
New York Times}}{The New York Times}}\label{the-new-york-times}}

 \href{/section/climate}{Climate} \textbar{}From Heat Waves to
Hurricanes: What We Know About Extreme Weather and Climate Change

Log In

**0

**Settings

**Close search

\hypertarget{site-search-navigation}{%
\subsection{Site Search Navigation}\label{site-search-navigation}}

Search NYTimes.com

**Clear this text input

Go

\url{https://nyti.ms/2x2viyM}

\begin{enumerate}
\def\labelenumi{\arabic{enumi}.}
\item
  Loading...
\end{enumerate}

See next articles

See previous articles

\hypertarget{site-navigation}{%
\subsection{Site Navigation}\label{site-navigation}}

\hypertarget{site-mobile-navigation}{%
\subsection{Site Mobile Navigation}\label{site-mobile-navigation}}

Advertisement

\hypertarget{-climate-}{%
\subsection{\texorpdfstring{ \href{/section/climate}{Climate}
}{ Climate }}\label{-climate-}}

\hypertarget{from-heat-waves-to-hurricanes-what-we-know-about-extreme-weather-and-climate-change}{%
\section{From Heat Waves to Hurricanes: What We Know About Extreme
Weather and Climate
Change}\label{from-heat-waves-to-hurricanes-what-we-know-about-extreme-weather-and-climate-change}}

\includegraphics{https://static01.graylady3jvrrxbe.onion/newsgraphics/2017/09/12/climate-extreme-events/f6ff710e2a37c1f760f9e6cb451ed76a053fed24/heat.png}
\includegraphics{https://static01.graylady3jvrrxbe.onion/newsgraphics/2017/09/12/climate-extreme-events/f6ff710e2a37c1f760f9e6cb451ed76a053fed24/hurricane.png}
\includegraphics{https://static01.graylady3jvrrxbe.onion/newsgraphics/2017/09/12/climate-extreme-events/f6ff710e2a37c1f760f9e6cb451ed76a053fed24/rainfall.png}
\includegraphics{https://static01.graylady3jvrrxbe.onion/newsgraphics/2017/09/12/climate-extreme-events/f6ff710e2a37c1f760f9e6cb451ed76a053fed24/drought.png}

It's been a hectic end to summer, meteorologically speaking.

\href{https://www.nytimes3xbfgragh.onion/2017/09/06/climate/storms-harvey-irma-jose.html}{Back-to-back
hurricanes} raked Texas, Florida and the Caribbean. A Labor Day heat
wave
\href{https://weather.com/forecast/regional/news/west-heat-wave-all-time-record-heat-early-september-2017}{broke
temperature records in San Francisco} and
\href{http://www.latimes.com/local/lanow/la-me-ln-heat-warnings-socal-20170829-story.html}{strained
California's electricity grid}. Wildfires continue to rage
\href{https://www.nytimes3xbfgragh.onion/interactive/2017/09/16/us/wildfires-smoke-pacific-northwest.html}{in
the Pacific Northwest}.

This string of extreme events has brought new focus to a familiar
question: Is climate change to blame?

\includegraphics{https://static01.graylady3jvrrxbe.onion/packages/flash/multimedia/ICONS/transparent.png}

\hypertarget{extreme-heat}{%
\subsection{Extreme Heat}\label{extreme-heat}}

\includegraphics{https://static01.graylady3jvrrxbe.onion/packages/flash/multimedia/ICONS/transparent.png}

A heat wave in Los Angeles in late August. Frederic J. Brown/Agence
France-Presse --- Getty Images

\includegraphics{https://static01.graylady3jvrrxbe.onion/packages/flash/multimedia/ICONS/transparent.png}

A heat wave in Los Angeles in late August. Frederic J. Brown/Agence
France-Presse --- Getty Images

Extreme heat is the hallmark of global warming.

Average temperatures are shifting upward. In the process, very hot days
become more frequent and the hottest days are now even hotter. (See
\href{https://www.nytimes3xbfgragh.onion/interactive/2017/07/28/climate/more-frequent-extreme-summer-heat.html}{our
animation of how this has played out} over the past few decades.)

This matters because extreme heat is associated with increased
mortality, lower crop yields and other consequences.

Over the past decade, scientists have become more confident in linking
heat waves to climate change. As with other extreme events, scientists
can't say that climate change caused a specific heat wave, but they can
say it made the heat wave more likely to occur.

Researchers from World Weather Attribution, an international group that
analyzes extreme weather events for links to climate change, found that
a deadly June heat wave that blanketed much of Western Europe was
\href{https://wwa.climatecentral.org/analyses/europe-heat-june-2017/}{up
to 10 times more likely} in our current warming climate than in a
simulated world in which there were no greenhouse gas emissions from
human activities. Similar studies have identified the fingerprints of
global warming on many other extreme heat events.

\includegraphics{https://static01.graylady3jvrrxbe.onion/packages/flash/multimedia/ICONS/transparent.png}

\hypertarget{heavy-rainfall}{%
\subsection{Heavy Rainfall}\label{heavy-rainfall}}

\includegraphics{https://static01.graylady3jvrrxbe.onion/packages/flash/multimedia/ICONS/transparent.png}

Heavy rains in Houston in the aftermath of Hurricane Harvey. Brendan
Smialowski/Agence France-Presse --- Getty Images

\includegraphics{https://static01.graylady3jvrrxbe.onion/packages/flash/multimedia/ICONS/transparent.png}

Heavy rains in Houston in the aftermath of Hurricane Harvey. Brendan
Smialowski/Agence France-Presse --- Getty Images

Warmer air can hold more water vapor, which leads to heavier downpours.
``Really powerful storms are really good at squeezing that water vapor
out and turning it into rain,'' said
\href{http://eesc.columbia.edu/faculty/prof-adam-h-sobel}{Adam Sobel},
an atmospheric scientist at Columbia University.

Across the United States, heavy precipitation events have increased in
intensity and frequency since the beginning of the 20th century,
according to
\href{https://www.nytimes3xbfgragh.onion/interactive/2017/08/07/climate/document-Draft-of-the-Climate-Science-Special-Report.html}{a
climate report} from 13 federal agencies awaiting approval by the Trump
administration. But there are significant regional and seasonal
differences to this trend. The largest increases in heavy downpours were
seen in the Northeast and Midwest. Fall had the greatest increase in
precipitation, with little change observed during winter.

Several
\href{http://onlinelibrary.wiley.com/doi/10.1002/grl.51010/full}{studies}
have found that greenhouse gas emissions have contributed to the
increase in heavy precipitation, but the link to climate change is less
solid than with extreme heat.

\includegraphics{https://static01.graylady3jvrrxbe.onion/packages/flash/multimedia/ICONS/transparent.png}

\hypertarget{droughts}{%
\subsection{Droughts}\label{droughts}}

\includegraphics{https://static01.graylady3jvrrxbe.onion/packages/flash/multimedia/ICONS/transparent.png}

The La Tuna Fire near Burbank, Calif., this month. David McNew/Getty
Images

\includegraphics{https://static01.graylady3jvrrxbe.onion/packages/flash/multimedia/ICONS/transparent.png}

The La Tuna Fire near Burbank, Calif., this month. David McNew/Getty
Images

While it may seem counterintuitive, global warming is expected to lead
to both more intense precipitation and more intense drought --- just not
in the same place at the same time.

It's well understood that warmer temperatures lead to drier soils. And
while scientists do not expect less precipitation overall, shifting
weather patterns mean that some areas will get fewer downpours even as
others see more.

Scientists have found a climate change link to droughts in the Middle
East and the Mediterranean. The federal report describes the influence
of climate change on recent major droughts in the United States as
``complicated,'' with stronger evidence of human influence on loss of
soil moisture and mixed results about precipitation.

Changes in drought and precipitation patterns also influence forest
fires, which have become more frequent in the Western United States and
Alaska since the 1980s. Other natural and human variables, including
forest management and fire suppression programs, play a role. But
studies have found that global warming has already
\href{https://www.nytimes3xbfgragh.onion/2016/10/11/science/climate-change-forest-fires.html?mcubz=0}{made
Western forests drier}, so conditions are more conducive to fires.

\includegraphics{https://static01.graylady3jvrrxbe.onion/packages/flash/multimedia/ICONS/transparent.png}

\hypertarget{hurricanes}{%
\subsection{Hurricanes}\label{hurricanes}}

\includegraphics{https://static01.graylady3jvrrxbe.onion/packages/flash/multimedia/ICONS/transparent.png}

Trey Holladay herded animals in a flooded neighborhood west of Houston.
Andrew Burton for The New York Times

\includegraphics{https://static01.graylady3jvrrxbe.onion/packages/flash/multimedia/ICONS/transparent.png}

Trey Holladay herded animals in a flooded neighborhood west of Houston.
Andrew Burton for The New York Times

Climate change is expected to lead to stronger, wetter hurricanes and is
likely to increase the frequency of ``very intense'' storms, according
to the federal report. But the effects of global warming on individual
storms have been difficult to determine so far.

``With hurricanes, there is broad agreement now on what we can expect in
the future and with how much confidence, but there's a lot less
agreement on what we can already see in the data,'' Dr. Sobel said.

Hurricanes pose ``a few inherent challenges'' for climate scientists,
said
\href{https://woods.stanford.edu/about/woods-faculty/noah-diffenbaugh}{Noah
Diffenbaugh}, a professor of earth system science at Stanford
University. They are rare events, with only about a dozen observed even
in an active season, so there is a sparse historical record for study.
And hurricanes are complex weather systems, so computer modeling
requires huge amounts of processing power.

So far, a strong climate change signal has not emerged from the natural
variability in observational data. But that doesn't mean climate change
isn't affecting storms today.

``It's like trying to hear someone talk very quietly in a noisy room,''
Dr. Sobel said. ``It doesn't mean they're not talking, but you can't
hear it.''

Additional reporting by Henry Fountain; additional design by Claire
O'Neill.

\hypertarget{more-on-nytimescom}{%
\subsection{More on NYTimes.com}\label{more-on-nytimescom}}

Advertisement

\hypertarget{site-information-navigation}{%
\subsection{Site Information
Navigation}\label{site-information-navigation}}

\begin{itemize}
\tightlist
\item
  \href{https://help.nytimes3xbfgragh.onion/hc/en-us/articles/115014792127-Copyright-notice}{©
  2020 The New York Times Company}
\item
  \href{https://www.nytimes3xbfgragh.onion}{Home}
\item
  \href{https://www.nytimes3xbfgragh.onion/search/}{Search}
\item
  Accessibility concerns? Email us at
  \href{mailto:accessibility@NYTimes.com}{\nolinkurl{accessibility@NYTimes.com}}.
  We would love to hear from you.
\item
  \href{https://help.nytimes3xbfgragh.onion/hc/en-us/articles/115015385887-Contact-Us}{Contact
  Us}
\item
  \href{https://www.nytco.com/careers/}{Work with us}
\item
  \href{https://nytmediakit.com/}{Advertise}
\item
  \href{https://help.nytimes3xbfgragh.onion/hc/en-us/articles/115014892108-Privacy-policy\#pp}{Your
  Ad Choices}
\item
  \href{https://help.nytimes3xbfgragh.onion/hc/en-us/articles/115014892108-Privacy-policy}{Privacy}
\item
  \href{https://help.nytimes3xbfgragh.onion/hc/en-us/articles/115014893428-Terms-of-service}{Terms
  of Service}
\item
  \href{https://help.nytimes3xbfgragh.onion/hc/en-us/articles/115014893968-Terms-of-sale}{Terms
  of Sale}
\end{itemize}

\hypertarget{site-information-navigation-1}{%
\subsection{Site Information
Navigation}\label{site-information-navigation-1}}

\begin{itemize}
\tightlist
\item
  \href{https://spiderbites.nytimes3xbfgragh.onion}{Site Map}
\item
  \href{https://help.nytimes3xbfgragh.onion/hc/en-us}{Help}
\item
  \href{https://help.nytimes3xbfgragh.onion/hc/en-us/articles/115015385887-Contact-Us?redir=myacc}{Site
  Feedback}
\item
  \href{https://www.nytimes3xbfgragh.onion/subscription?campaignId=37WXW}{Subscriptions}
\end{itemize}
