 **NYTimes.com no longer supports Internet Explorer 9 or earlier. Please
upgrade your browser.
\href{http://www.nytimes3xbfgragh.onion/content/help/site/ie9-support.html}{LEARN
MORE »}

**Sections

**Home

**Search

\hypertarget{the-new-york-times}{%
\subsection{\texorpdfstring{\href{http://www.nytimes3xbfgragh.onion/}{The
New York Times}}{The New York Times}}\label{the-new-york-times}}

\hypertarget{-climate-}{%
\subsubsection{\texorpdfstring{ \href{/section/climate}{Climate}
}{ Climate }}\label{-climate-}}

 \href{/section/climate}{Climate} \textbar{}Climate Change Is Complex.
We've Got Answers to Your Questions.

**Close search

\hypertarget{site-search-navigation}{%
\subsection{Site Search Navigation}\label{site-search-navigation}}

Search NYTimes.com

**Clear this text input

Go

\url{https://nyti.ms/2jFxUNv}

\hypertarget{site-navigation}{%
\subsection{Site Navigation}\label{site-navigation}}

\hypertarget{site-mobile-navigation}{%
\subsection{Site Mobile Navigation}\label{site-mobile-navigation}}

\hypertarget{climate-change-is-complex-weve-got-answers-to-your-questions}{%
\section{Climate Change Is Complex. We've Got Answers to Your
Questions.}\label{climate-change-is-complex-weve-got-answers-to-your-questions}}

We know. Global warming is daunting. So here's a place to start: 17
often-asked questions with some straightforward answers.

\includegraphics{https://static01.graylady3jvrrxbe.onion/newsgraphics/2017/07/12/climate-qa/assets/images/02-vert-640.jpg}

\includegraphics{https://static01.graylady3jvrrxbe.onion/newsgraphics/2017/07/12/climate-qa/assets/images/01-vert-640.jpg}

\includegraphics{https://static01.graylady3jvrrxbe.onion/newsgraphics/2017/07/12/climate-qa/assets/images/00-vert-640.jpg}

\includegraphics{https://static01.graylady3jvrrxbe.onion/newsgraphics/2017/07/12/climate-qa/assets/images/05-vert-640.jpg}

\includegraphics{https://static01.graylady3jvrrxbe.onion/newsgraphics/2017/07/12/climate-qa/assets/images/04-vert-640.jpg}

\includegraphics{https://static01.graylady3jvrrxbe.onion/newsgraphics/2017/07/12/climate-qa/assets/images/03-vert-640.jpg}

\includegraphics{https://static01.graylady3jvrrxbe.onion/newsgraphics/2017/07/12/climate-qa/assets/images/08-vert-640.jpg}

\includegraphics{https://static01.graylady3jvrrxbe.onion/newsgraphics/2017/07/12/climate-qa/assets/images/07-vert-640.jpg}

\includegraphics{https://static01.graylady3jvrrxbe.onion/newsgraphics/2017/07/12/climate-qa/assets/images/06-vert-640.jpg}

\includegraphics{https://static01.graylady3jvrrxbe.onion/newsgraphics/2017/07/12/climate-qa/assets/images/11-vert-640.jpg}

\includegraphics{https://static01.graylady3jvrrxbe.onion/newsgraphics/2017/07/12/climate-qa/assets/images/10-vert-640.jpg}

\includegraphics{https://static01.graylady3jvrrxbe.onion/newsgraphics/2017/07/12/climate-qa/assets/images/09-vert-640.jpg}

Part 1

\hypertarget{what-is-happening}{%
\section{What is happening?}\label{what-is-happening}}

\protect\hyperlink{chapterone}{What is happening?}

\protect\hyperlink{chaptertwo}{What could happen?}

\protect\hyperlink{chapterthree}{What can we do?}

\hypertarget{1-climate-change-global-warming-what-do-we-call-it}{%
\subsection{1. Climate change? Global warming? What do we call
it?}\label{1-climate-change-global-warming-what-do-we-call-it}}

Both are accurate, but they mean different things.

You can think of global warming as one type of climate change. The
broader term covers changes beyond warmer temperatures, such as shifting
rainfall patterns.

President Trump has claimed that scientists stopped referring to global
warming and started calling it climate change because ``the weather has
been so cold'' in winter. But the claim is false. Scientists have used
both terms for decades.

\hypertarget{2-how-much-is-the-earth-heating-up}{%
\subsection{2. How much is the Earth heating
up?}\label{2-how-much-is-the-earth-heating-up}}

Two degrees is more significant than it sounds.

As of early 2017, the Earth had warmed by roughly 2 degrees Fahrenheit
(more than 1 degree Celsius) since 1880, when records began at a global
scale. The number may sound low, but as an average over the surface of
an entire planet, it is actually high, which explains why much of the
world's land ice is starting to melt and the oceans are rising at an
accelerating pace. If greenhouse gas emissions continue unchecked,
scientists say, the global warming could ultimately exceed 8 degrees
Fahrenheit, which would undermine the planet's capacity to support a
large human population.

\hypertarget{3-what-is-the-greenhouse-effect-and-how-does-it-cause-global-warming}{%
\subsection{3. What is the greenhouse effect, and how does it cause
global
warming?}\label{3-what-is-the-greenhouse-effect-and-how-does-it-cause-global-warming}}

We've known about it for more than a century. Really.

In the 19th century, scientists discovered that certain gases in the air
trap and slow down heat that would otherwise escape to space. Carbon
dioxide is a major player; without any of it in the air, the Earth would
be a frozen wasteland. The first
\href{http://www.rsc.org/images/Arrhenius1896_tcm18-173546.pdf}{prediction}
that the planet would warm as humans released more of the gas was made
in 1896. The gas has increased 43 percent above the pre-industrial level
so far, and the Earth has warmed by roughly the amount that scientists
predicted it would.

\hypertarget{interested-in-keeping-up-with-climate-change}{%
\subsubsection{Interested in keeping up with climate
change?}\label{interested-in-keeping-up-with-climate-change}}

Sign up to receive our in-depth journalism about climate change around
the world.

\hypertarget{4-how-do-we-know-humans-are-responsible-for-the-increase-in-carbon-dioxide}{%
\subsection{4. How do we know humans are responsible for the increase in
carbon
dioxide?}\label{4-how-do-we-know-humans-are-responsible-for-the-increase-in-carbon-dioxide}}

This one is nailed down.

Hard evidence, including studies that use radioactivity to distinguish
industrial emissions from natural emissions, shows that the extra gas is
coming from human activity. Carbon dioxide levels rose and fell
naturally in the long-ago past, but those changes took thousands of
years. Geologists say that humans are now pumping the gas into the air
much faster than nature has ever done.

\hypertarget{5-could-natural-factors-be-the-cause-of-the-warming}{%
\subsection{5. Could natural factors be the cause of the
warming?}\label{5-could-natural-factors-be-the-cause-of-the-warming}}

Nope.

In theory, they could be. If the sun were to start putting out more
radiation, for instance, that would definitely warm the Earth. But
scientists have looked carefully at the natural factors known to
influence planetary temperature and found that they are not changing
nearly enough. The warming is extremely rapid on the geologic time
scale, and no other factor can explain it as well as human emissions of
greenhouse gases.

\hypertarget{6-why-do-people-deny-the-science-of-climate-change}{%
\subsection{6. Why do people deny the science of climate
change?}\label{6-why-do-people-deny-the-science-of-climate-change}}

Mostly because of ideology.

Instead of negotiating over climate change policies and trying to make
them more market-oriented, some political conservatives have taken the
approach of blocking them by trying to undermine the science.

President Trump has sometimes claimed that scientists are engaged in a
\href{https://www.amazon.com/The-Greatest-Hoax-Conspiracy-Threatens/dp/1936488493}{worldwide
hoax} to fool the public, or that global warming was
\href{https://www.nytimes3xbfgragh.onion/2016/11/19/world/asia/china-trump-climate-change.html}{invented
by China} to disable American industry. The climate denialists'
arguments have become so strained that even oil and coal companies have
distanced themselves publicly, though some still help to finance the
campaigns of politicians who espouse such views.

Part 2

\hypertarget{what-could-happen}{%
\section{What could happen?}\label{what-could-happen}}

\hypertarget{1-how-much-trouble-are-we-in}{%
\subsection{1. How much trouble are we
in?}\label{1-how-much-trouble-are-we-in}}

Big trouble.

Over the coming 25 or 30 years, scientists say, the climate is likely to
gradually warm, with more extreme weather. Coral reefs and other
sensitive habitats are already
\href{https://www.nytimes3xbfgragh.onion/2017/03/15/science/great-barrier-reef-coral-climate-change-dieoff.html}{starting
to die}. Longer term, if emissions rise unchecked, scientists fear
climate effects so severe that they might destabilize governments,
produce
\href{https://www.nytimes3xbfgragh.onion/2017/04/19/magazine/how-a-warming-planet-drives-human-migration.html}{waves
of refugees}, precipitate the
\href{https://www.nytimes3xbfgragh.onion/2017/07/11/climate/mass-extinction-animal-species.html}{sixth
mass extinction} of plants and animals in the Earth's history, and melt
the polar ice caps, causing the seas to rise high enough to flood most
of the world's coastal cities. The emissions that create those risks are
happening now, raising deep
\href{https://www.nytimes3xbfgragh.onion/2017/08/10/climate/climate-change-lawsuits-courts.html}{moral
questions} for our generation.

\hypertarget{2-how-much-should-i-worry-about-climate-change-affecting-me-directly}{%
\subsection{2. How much should I worry about climate change affecting me
directly?}\label{2-how-much-should-i-worry-about-climate-change-affecting-me-directly}}

Are you rich enough to shield your descendants?

The simple reality is that people are already feeling the effects,
whether they know it or not. Because of sea level rise, for instance,
some 83,000 more residents of New York and New Jersey were flooded
during Hurricane Sandy than would have been the case in a stable
climate, scientists have calculated. Tens of thousands of people are
already dying in heat waves made worse by global warming. The refugee
flows that have destabilized politics around the world have been traced
in part to climate change. Of course, as with almost all other social
problems, poor people will be hit first and hardest.

\hypertarget{3-how-much-will-the-seas-rise}{%
\subsection{3. How much will the seas
rise?}\label{3-how-much-will-the-seas-rise}}

The real question is how fast.

The ocean has accelerated and is now rising at a rate of about a foot
per century, forcing governments and property owners to spend tens of
billions of dollars fighting coastal erosion. But if that rate
continued, it would probably be manageable, experts say.

The risk is that the rate will increase still more. Scientists who study
the Earth's history say waters could rise by a foot per decade in a
worst-case scenario, though that looks unlikely. Many experts believe
that even if emissions stopped tomorrow, 15 or 20 feet of sea level rise
is already inevitable, enough to
\href{https://www.nytimes3xbfgragh.onion/2016/03/15/science/rising-sea-levels-global-warming-climate-change.html}{flood
many cities} unless trillions of dollars are spent protecting them. How
long it will take is unclear. But if emissions continue apace, the
ultimate rise could be 80 or 100 feet.

\hypertarget{4-is-recent-crazy-weather-tied-to-climate-change}{%
\subsection{4. Is recent crazy weather tied to climate
change?}\label{4-is-recent-crazy-weather-tied-to-climate-change}}

Some of it is.

Scientists have published strong evidence that the warming climate is
making heat waves
\href{https://www.nytimes3xbfgragh.onion/interactive/2017/09/15/climate/does-climate-change-cause-hurricanes-drought.html}{more
frequent and intense}. It is also causing
\href{https://www.nytimes3xbfgragh.onion/2014/05/13/science/looks-like-rain-again-and-again.html?_r=0}{heavier
rainstorms}, and coastal flooding is
\href{https://www.nytimes3xbfgragh.onion/2016/09/04/science/flooding-of-coast-caused-by-global-warming-has-already-begun.html}{getting
worse}as the oceans rise because of human emissions. Global warming has
intensified droughts in regions like the Middle East, and it may have
\href{https://www.nytimes3xbfgragh.onion/2015/08/21/science/climate-change-intensifies-california-drought-scientists-say.html}{strengthened}
a recent drought in California.

In many other cases, though ---
\href{https://www.nytimes3xbfgragh.onion/2017/08/25/us/hurricane-harvey-climate-change-texas.html}{hurricanes,
for example} --- the linkage to global warming for particular trends is
uncertain or disputed. Scientists are gradually improving their
understanding as computer analyses of the climate grow more powerful.

Part 3

\hypertarget{what-can-we-do}{%
\section{What can we do?}\label{what-can-we-do}}

\hypertarget{1-are-there-any-realistic-solutions-to-the-problem}{%
\subsection{1. Are there any realistic solutions to the
problem?}\label{1-are-there-any-realistic-solutions-to-the-problem}}

Yes, but change is happening too slowly.

Society has put off action for so long that the risks are now severe,
scientists say. But as long as there are still unburned fossil fuels in
the ground, it is not too late to act. The warming will slow to a
potentially manageable pace only when human emissions are reduced to
zero. The good news is that they are now falling in many countries as a
result of programs like fuel-economy standards for cars, stricter
building codes and emissions limits for power plants. But experts say
the energy transition needs to speed up drastically to head off the
worst effects of climate change.

\hypertarget{2-what-is-the-paris-agreement}{%
\subsection{2. What is the Paris
Agreement?}\label{2-what-is-the-paris-agreement}}

Virtually every country agreed to limit future emissions.

The
\href{https://www.nytimes3xbfgragh.onion/2015/12/13/world/europe/climate-change-accord-paris.html}{landmark
deal} was reached outside Paris in December 2015. The reductions are
voluntary and the pledges do not do enough to head off severe effects.
But the agreement is supposed to be reviewed every few years so that
countries ramp up their commitments. President Trump announced in 2017
that he would pull the United States
\href{https://www.nytimes3xbfgragh.onion/2017/06/01/climate/paris-climate-change-guide.html}{out
of the deal}, though that will take years, and other countries have said
they would go forward regardless of American intentions.

\hypertarget{3-does-clean-energy-help-or-hurt-the-economy}{%
\subsection{3. Does clean energy help or hurt the
economy?}\label{3-does-clean-energy-help-or-hurt-the-economy}}

Job growth in renewable energy is strong.

The energy sources with the lowest emissions include wind turbines,
solar panels, hydroelectric dams and nuclear power stations. Power
plants burning natural gas also produce fewer emissions than those
burning coal. Converting to these cleaner sources may be somewhat
costlier in the short term, but they could ultimately pay for themselves
by heading off climate damages and reducing health problems associated
with dirty air. And expansion of the market is driving down the costs of
renewable energy so fast that it may ultimately beat dirty energy on
price alone --- it
\href{https://www.nytimes3xbfgragh.onion/2017/06/06/climate/renewable-energy-push-is-strongest-in-the-reddest-states.html}{already
does} in some areas.

The transition to cleaner energy certainly produces losers, like coal
companies, but it also creates jobs. The solar industry in the United
States now employs more than twice as many people as coal mining.

\hypertarget{4-what-about-fracking-or-clean-coal}{%
\subsection{4. What about fracking or `clean
coal'?}\label{4-what-about-fracking-or-clean-coal}}

Both could help clean up the energy system.

Hydraulic fracturing, or ``fracking,'' is one of a set of drilling
technologies that has helped produce a new abundance of natural gas in
the United States and some other countries. Burning gas instead of coal
in power plants reduces emissions in the short run, though gas is still
a fossil fuel and will have to be phased out in the long run. The
fracking itself can also create local pollution.

\href{https://www.nytimes3xbfgragh.onion/2017/08/23/climate/what-clean-coal-is-and-isnt.html}{``Clean
coal''} is an approach in which the emissions from coal-burning power
plants would be captured and pumped underground. It has yet to be proven
to work economically, but some experts think it could eventually play a
major role.

\hypertarget{5-whats-the-latest-with-electric-cars}{%
\subsection{5. What's the latest with electric
cars?}\label{5-whats-the-latest-with-electric-cars}}

Sales are still small overall, but they are rising fast.

The cars draw power at night from the electric grid and give off no
pollution during the day as they move around town. They are inherently
more efficient than gasoline cars and would represent an advance even if
the power were generated by burning coal, but they will be far more
important as the electric grid itself becomes greener through renewable
power. The cars are improving so fast that some countries are already
talking about
\href{https://www.nytimes3xbfgragh.onion/2017/07/06/business/energy-environment/france-cars-ban-gas-diesel.html}{banning
the sale of gasoline cars} after 2030.

\hypertarget{6-what-are-carbon-taxes-carbon-trading-and-carbon-offsets}{%
\subsection{6. What are carbon taxes, carbon trading and carbon
offsets?}\label{6-what-are-carbon-taxes-carbon-trading-and-carbon-offsets}}

It's just jargon for putting a price on pollution.

The greenhouse gases being released by human activity are often called
``carbon emissions'' for short. That is because two of the most
important gases, carbon dioxide and methane, contain carbon. (Some other
pollutants are lumped into the same category, even if they do not
actually contain carbon.) When you hear about carbon taxes, carbon
trading and so on, these are just shorthand descriptions of methods to
put a price on emissions, which economists say is one of the most
important steps society could take to limit them.

\hypertarget{7-climate-change-seems-so-overwhelming-what-can-i-personally-do-about-it}{%
\subsection{7. Climate change seems so overwhelming. What can I
personally do about
it?}\label{7-climate-change-seems-so-overwhelming-what-can-i-personally-do-about-it}}

Start by sharing this with 50 of your friends.

Experts say the problem can only be
\href{https://www.nytimes3xbfgragh.onion/interactive/2017/06/09/climate/drawdown-climate-solutions-quiz.html}{solved}
by large-scale, collective action. Entire states and nations have to
decide to clean up their energy systems, using every tool available and
moving as quickly as they can. So the most important thing you can do is
to exercise your rights as a citizen, speaking up and demanding change.

You can also take direct personal action to reduce your carbon footprint
in simple ways that will save you money. You can plug leaks in your home
insulation to save power, install a smart thermostat, switch to more
efficient light bulbs, turn off unused lights, drive fewer miles by
consolidating trips or taking public transit, waste less food, and eat
less meat.

Taking one or two
\href{https://www.nytimes3xbfgragh.onion/2017/07/27/climate/airplane-pollution-global-warming.html}{fewer
plane rides} per year can save as much in emissions as all the other
actions combined. If you want to be at the cutting edge, you can look at
buying an electric or hybrid car or putting solar panels on your roof.
If your state has a competitive electricity market, you may be able to
buy 100 percent green power.

Leading corporations, including large manufacturers like carmakers, are
starting to demand clean energy for their operations. You can pay
attention to company policies, support the companies taking the lead,
and let the others know you expect them to do better.

These personal steps may be small in the scheme of things, but they can
raise your own consciousness about the problem --- and the awareness of
the people around you. In fact, discussing this issue with your friends
and family is one of the most meaningful things you can do.

Produced by Gray Beltran and Rumsey Taylor.

\hypertarget{more-on-nytimescom}{%
\subsection{More on NYTimes.com}\label{more-on-nytimescom}}

Advertisement

\hypertarget{site-information-navigation}{%
\subsection{Site Information
Navigation}\label{site-information-navigation}}

\begin{itemize}
\tightlist
\item
  \href{https://help.nytimes3xbfgragh.onion/hc/en-us/articles/115014792127-Copyright-notice}{©
  2020 The New York Times Company}
\item
  \href{https://www.nytimes3xbfgragh.onion}{Home}
\item
  \href{https://www.nytimes3xbfgragh.onion/search/}{Search}
\item
  Accessibility concerns? Email us at
  \href{mailto:accessibility@NYTimes.com}{\nolinkurl{accessibility@NYTimes.com}}.
  We would love to hear from you.
\item
  \href{https://help.nytimes3xbfgragh.onion/hc/en-us/articles/115015385887-Contact-Us}{Contact
  Us}
\item
  \href{https://www.nytco.com/careers/}{Work with us}
\item
  \href{https://nytmediakit.com/}{Advertise}
\item
  \href{https://help.nytimes3xbfgragh.onion/hc/en-us/articles/115014892108-Privacy-policy\#pp}{Your
  Ad Choices}
\item
  \href{https://help.nytimes3xbfgragh.onion/hc/en-us/articles/115014892108-Privacy-policy}{Privacy}
\item
  \href{https://help.nytimes3xbfgragh.onion/hc/en-us/articles/115014893428-Terms-of-service}{Terms
  of Service}
\item
  \href{https://help.nytimes3xbfgragh.onion/hc/en-us/articles/115014893968-Terms-of-sale}{Terms
  of Sale}
\end{itemize}

\hypertarget{site-information-navigation-1}{%
\subsection{Site Information
Navigation}\label{site-information-navigation-1}}

\begin{itemize}
\tightlist
\item
  \href{https://spiderbites.nytimes3xbfgragh.onion}{Site Map}
\item
  \href{https://help.nytimes3xbfgragh.onion/hc/en-us}{Help}
\item
  \href{https://help.nytimes3xbfgragh.onion/hc/en-us/articles/115015385887-Contact-Us?redir=myacc}{Site
  Feedback}
\item
  \href{https://www.nytimes3xbfgragh.onion/subscription?campaignId=37WXW}{Subscriptions}
\end{itemize}
