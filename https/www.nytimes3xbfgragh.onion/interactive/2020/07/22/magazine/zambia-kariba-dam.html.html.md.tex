Sections

SEARCH

\protect\hyperlink{site-content}{Skip to
content}\protect\hyperlink{site-index}{Skip to site index}

\hypertarget{comments}{%
\subsection{\texorpdfstring{\protect\hyperlink{commentsContainer}{Comments}}{Comments}}\label{comments}}

\href{}{Learning From the Kariba Dam}\href{}{Skip to Comments}

The comments section is closed. To submit a letter to the editor for
publication, write to
\href{mailto:letters@NYTimes.com}{\nolinkurl{letters@NYTimes.com}}.

\hypertarget{learning-from-the-kariba-dam}{%
\section{Learning From the Kariba
Dam}\label{learning-from-the-kariba-dam}}

By Namwali SerpellJuly 22, 2020

\begin{itemize}
\item
\item
\item
\item
\item
  \emph{91}
\end{itemize}

Climate change and neglect have brought the mammoth structure at the
border of Zambia and Zimbabwe to the brink of calamity --- a crisis
prefigured in the dam's troubling colonial history.

\hypertarget{learning-from-the-kariba-dam-1}{%
\section{Learning From the Kariba
Dam}\label{learning-from-the-kariba-dam-1}}

Climate change and neglect have brought the mammoth structure at the
border of Zambia and Zimbabwe to the brink of calamity --- a crisis
prefigured in the dam's troubling colonial history.

By Namwali Serpell

July 22, 2020

SHARE

\hypertarget{listen-to-this-article}{%
\subsection{Listen to This Article}\label{listen-to-this-article}}

\emph{To hear more audio stories from publishers like The New York
Times,}
\emph{\href{https://www.audm.com/?utm_source=nytmag\&utm_medium=embed\&utm_campaign=river_of_time}{download
Audm for iPhone or Android}.}

The Kariba Dam is failing. Since the late 1950s, it has sat on the
Zambezi River, on the border between Zambia and Zimbabwe, in one of the
zigzagging gorges that ripple the land there. It provides 1,830
megawatts of hydroelectric power to both countries and holds back the
world's largest reservoir. For the last decade, scientists and reporters
have issued warnings about the dam's potential to cause ecological
disasters --- of opposite kinds. On one hand, low rainfall has yielded
water levels that barely reach the minimum necessary to generate
electricity. On the other hand, heavy rainfall has threatened to flood
the surrounding areas. When the floodgates were opened in 2010, 6,000
people had to be evacuated.

Climate change catastrophizes the weather --- and when it comes to such
extremes, dams are, well, inflexible. They cannot be narrowed enough to
eke more force from less water during droughts, and far worse, they
cannot be expanded enough to accommodate floods. The only other ways to
handle floods are to let the water flow over the top of the dam or to
open up a spillway for controlled release. Neither of these measures is
foolproof at the Kariba Dam because of how the passage of time has worn
it down. The dam was built on gneiss and quartzite and is made of
concrete --- 80 feet at its thickest point. But over six decades of the
waters' rushing through it, tumbling over it and crashing down on its
other side have carved a pit at its base and erosion threatens its
foundations. Its plunge pool is now a 266-foot-deep crater.

As the stony facade continues to crumble, the likelihood rises that the
Kariba Dam will not just fail but \emph{fall}. If the dam collapses, the
BBC reported in 2014, a tsunami would tear through the Zambezi River
Valley, a torrent so powerful that it would knock down another dam a
hundred miles away, the Cahora Bassa in Mozambique --- twin disasters
that would take out 40 percent of the hydroelectric capacity in all of
southern Africa. At the same time, longer hot seasons have drained the
reservoir to record lows, and drought-induced power cuts have become a
daily reality for homes and businesses. The World Bank is supporting
efforts to secure the Kariba Dam, but any attempts to fix or expand it
risk weakening it further, which would be disastrous in the event of a
flood.

Whether the water is too high or too low, the lives of millions of
people are at stake, to say nothing of the natural ecosystem. It's a
familiar, seemingly inevitable tale of human folly: One of our most
ambitious efforts to harness the power of nature has left us exposed to
nature's vagaries.

Is this just a failure of our power of prophecy? When we talk about
climate change, we talk about our inability to predict and control
what's coming, to step into the same river twice. We're out of time, in
more than one sense: We've fallen out of rhythm with the circulatory
relations between sun and rain and earth. We've damned ourselves,
foreclosed some of the future's forking paths --- this is the aspect of
time we call the subjunctive, the grammatical mood for what is imagined
or wished. A river's branches suggest to us what \emph{could},
\emph{would}, \emph{should} be. But the subjunctive mood --- when it
comes to rivers, when it comes to time --- doesn't move in only one
direction. If we look back, it's clear: It didn't have to be this way.

\textbf{The history of} the Kariba Dam is the story of a war over the
past and the future of a river. That war was fought in the 1950s between
European colonial powers and the local people in a place then called the
Central African Federation or the Federation of Rhodesia and Nyasaland.
The federation was a short-lived colonial experiment --- or fiasco,
depending on your perspective --- that merged three adjacent territories
with historically disparate relationships to the British Empire.
Southern Rhodesia (now Zimbabwe) was a self-governing colony founded by
the British South Africa Company; Northern Rhodesia (now Zambia) and
Nyasaland (now Malawi) had been demarcated as British protectorates. The
decision to conglomerate the three territories into one came from the
colonialists, whose motivations were exploitatively economic and crudely
economical.

Colonial officers had brought some of the tribal chiefs in line by
appointing them to largely nominal positions in the native authorities.
But the younger, educated, radical Africans --- some of whom fought for
the British in World War II --- wanted more say in their fate. They
resisted federation fiercely. They spoke up from their positions on
local councils. They staged protests and boycotts: ``Down with
federation! To hell with federation!'' They were worried by the fact
that federation would move the center of power to Southern Rhodesia,
whose more deeply entrenched system of segregation, the Jim Crow-like
``color bar'' --- Africans couldn't go to bars, hotels or movie theaters
at the same time as Europeans --- seemed destined to seep into the
neighboring territories if they were merged.

The choice of where on the Zambezi River to build a dam was dictated by
the same gravitational shift. The river's source was in the northwest of
the nascent federation, near the border with Angola and what was then
the Belgian Congo. It curled down through Northern Rhodesia before
heading east, following --- in fact constituting --- its border with
Southern Rhodesia, then slanting across Mozambique to its mouth in the
Indian Ocean. The largest tributary of the Zambezi was the Kafue, which
flowed into it from the north at the center of the segment of the river
between the two Rhodesias. Just south of that confluence of currents was
a gorge known as Kariba.

From the mid-1940s on, there was debate about whether to build a dam on
the Kafue or at Kariba. Northern Rhodesia had decided to begin
construction on the Kafue, which was closer to the Copperbelt, a
valuable mining hub and urban center. The Kafue runs through natural
floodplains. A dam there --- which was eventually completed in the 1970s
--- would be smaller and more complicated to build but cause far less
trouble for the people and the environment. After the federation was
formed in 1953, however, Southern Rhodesia fought for the Kariba Dam to
be built first. At that crucial juncture, why did the federation's
government follow the Kariba fork?

It was a question of power. A French engineer, André Coyne, advocated
the Kariba site because it would supply more power, at greater value for
the cost. The Southern Rhodesians also wanted the dam to be closer to
the new seat of political power in the federation's capital, Salisbury.
The larger Kariba Dam would be a technological triumph and a grand
imperial project, raising the reputation of the backwater colonies.
Newsweek later described it as a monument to ``the know-how of Western
capital'': ``When the Zambezi River was harnessed, the queen mother
cheered.''

Coyne's French company designed the double curvature dam; an Italian
company, Impresit, was hired to build it; the World Bank granted a loan
to pay for it. The Kariba Lake Development Company --- largely made up
of British personnel --- was established in 1957 to conduct research and
piece together some ad hoc environmental and social regulations. There
was barely any assessment of the potential ecological impact of the dam,
much less the human costs.

So it was only in the middle of construction that the federation's
government began to take seriously the question of what to do with the
57,000 people who lived in the Gwembe Valley that was to be flooded to
build the dam --- a place where, for centuries, they'd fished in the
Zambezi and farmed on soil made rich by seasonal floods, a place they
called home.

\textbf{The word} \emph{\textbf{kariba}} was a corruption of
\emph{kariva} or \emph{kaliba}, a local term meaning ``trap.'' It
already named a place on the river, a massive stone slab that jutted out
of the water at the opening of the gorge. One legend among the local
Tonga people claimed that this rock was one of three that had once
formed a kind of bridge across the river --- a lintel that resembled the
animal traps they used --- until a flood washed the other two away. It
was the sole remnant of a geological event --- and from another point of
view, a warning. Other legends said that this was the home of a river
god named Nyaminyami, with the head of a fish and the twisting
whirlpool-like body of a snake. The British took one look at that big
rock and decided it was the best place to build a dam, and the best word
--- mispronounced because they couldn't wrap their lips around the soft
``b'' and ``l'' common in Bantu languages --- to explain to the Tonga
exactly what a dam was.

Trap a river? The notion was so outlandish that the Tonga began to
ignore the district commissioners, who despaired of convincing the
villagers --- only a few of whom had ever even witnessed electricity ---
that the dam was really going to be built, that their ancestral homes
would soon be underwater. As David Howarth puts it in his blinkered but
engaging 1961 history of the Kariba, ``The Shadow of the Dam,'' ``the
whole idea of stopping the river was absurd'' for the Tonga: ``Most of
them admitted that the Europeans would probably try, but the Europeans
did not know the river as the Tonga knew it; and the old men argued that
if anyone thought he could stop the river by building a wall across it,
it only showed he had no idea how strong the river was. Let them try
\ldots{} the river will push the wall over, or run round the ends of
it.''

This is exactly what happened. Seasonal rains can swell the Zambezi up
to 20 times its dry-season size. In late 1956, news came from upriver
that an ``exceptional flood'' --- so exceptional it would come to be
called the Hundred Years' Flood --- was on its way. The water rose 66
feet and drowned the cofferdam that was in place for construction. When
the waters finally subsided, only a crane had been lost, but the
engineers were shaken by the unexpected and awesome sight of the
torrential deluge.

They built a second cofferdam higher --- but not high enough. The very
next rainy season, the tributaries joined forces once more. This time
the chances were deemed one in a thousand. The Thousand Years' Flood of
1958 swept away a suspension bridge, which ``writhed like a snake when
the water touched it.'' The river rose 116 feet to the top of the second
cofferdam and poured over it, creating a waterfall 28 feet high. The
Tonga had been roundly mocked for superstitious predictions that the
``huge serpent'' living in the Zambezi would ``be angry with the white
man's wall and knock it down.'' Now, the journalist Frank Clements
declared: ``Nyaminyami had made good his threat. He had recaptured the
gorge.''

The dam seemed cursed. Late in the construction, some scaffolding gave
way. Seventeen workers fell into a hole and were buried in wet concrete.
Some say their remains were picked out, others that they remain entombed
in the dam. When the floods receded, the engineers rushed to make sure
the dam was complete before the following rainy season.

This meant that the wildlife now urgently needed to be rescued before
the Gwembe Valley became the largest man-made lake in the world.
``Operation Noah,'' as it was messianically named by white
conservationists, managed to capture and remove 6,000 animals, though
thousands more died in the floods. (This focus on the wildlife as the
principal victims has persisted as the central story of Kariba; a recent
BBC article about the dam revolves around a lone baboon ``marooned'' on
an island in the Zambezi.)

The people proved to be more intransigent than the animals when it came
to forced resettlement. The government determined that the Tonga were to
move to Lusitu, an area to the north, and began resettling 193 villages
one at a time, carting the people and their property there in trucks.
These new lands had poor, stony soil. There was an almost immediate
outbreak of dysentery. The Tonga way of farming, which relied on
seasonal floods and leaving land fallow, wasn't possible here. The ratio
of population to land was radically unbalanced. Traditional laws
regarding the distribution of property were upended.

Those who had not yet left the Gwembe Valley, already concerned about
the disruption of ancestral shrines and the lack of adequate
compensation for the loss of their homeland, now had even less reason to
leave. Some had been radicalized by the African National Congress --- a
nascent, nonviolent political party whose members agitated for the
breakup of the federation and later led the movements that decolonized
its three nations. The congress encouraged civil disobedience in the
face of the relocation.

As is often the colonial way, over time the federation's persuasion
campaign gave way to insistence, then violence. The laws of Northern
Rhodesia in fact prohibited forced removal, so the Tonga Native
Authority was persuaded to approve a legal order, which was translated
and broadcast to the people: ``The Government is quite satisfied that
the Lusitu plan is in your best interests and now intends to carry out
this move without delay. Those who resist will be moved by force, using
the police you see here today\ldots{}. Anybody who obstructs the move
will be prosecuted. When people have moved from a village, the huts will
be destroyed.''

The people rebelled. The villagers of Chisamu, who were governed by a
chief named Chipepo, made a series of charges at the police, shouting
and gesturing with their spears, playing drums and singing war songs.
The standoff lasted for days, the police conducting drills, Chipepo's
people imitating them. ``They marched and countermarched in single
file,'' Howarth writes, ``carrying their spears like rifles on their
shoulders, and instructors marched at the sides of the columns like
sergeants or platoon commanders. Sometimes it looked like a parody, but
perhaps they did it to convince themselves.'' The governor of Northern
Rhodesia was brought in for an \emph{indaba} with the leaders, but to no
avail. When the constables moved in on the villagers, violence broke
out. Eight Tonga were shot and killed. The people relented.

The dam was completed. The valley was flooded. Nowadays, fishing boats
and ``sunset cruises'' slip up and down the dwindling lake above the
dam. The eeriest, most beautiful thing about Lake Kariba --- its main
attraction for tourists --- is that the submerged trees of the Gwembe
Valley still stand. You can see them reaching up from the depths,
branching up out of the water, forking against the sky

\textbf{``The whole might} of modern technology was nearly caught by the
primeval, savage forces of Africa,'' Clements wrote of the Kariba in
1959. With this Manichaean hyperbole, he tidily conflates the power of
nature, the myth of Nyaminyami and the resistance of the Tonga, even as
he diminishes all three. In the end, the might of modern technology won,
escaped the trap --- or perhaps became one. Many historians cast the
story of the Kariba Dam as a paternalistic tale about how a zealous
belief in ``progress'' overwhelmed a hapless tribe of what David
Livingstone once called a ``degraded'' people. Another way to see it is
that the building of the Kariba Dam redirected enormous wealth to
colonial parties at the expense of the rightful dwellers of the Gwembe
Valley, who are now considered ``development refugees'' and lack
adequate access to water and electricity. As late as 2000, three of the
nearby districts where the Tonga now live were still not connected to
the national grid lines.

This dam business now directs wealth to neocolonial parties. The China
National Complete Engineering Corporation is building another \$449
million megadam on a tributary of the Zambezi. Within its own borders,
the Chinese government is turning away from hydroelectricity and toward
solar and wind energy. They know that, in the midst of a global
climate-change crisis, finding alternatives to dams is better than
trying to fix them.

Africans know it, too. In 2014, Partson Mbiriri, then the chairman of
the Zambezi River Authority,
\href{https://www.bbc.com/news/business-30547267}{told the BBC,} ``It's
equally important to think about solar --- on the assumption, of course,
that we'll continue to have sunshine.'' While various figures of
authority --- colonial, governmental, environmentalist, journalistic;
then and now, well-meaning and mercenary --- have all been deeply
concerned to explain to Africans what will happen to us if we do not
move out of the path of progress, they have never really bothered to
listen to us.

The Africans of the federation did in fact articulate a set of prescient
questions and demands --- subjunctive possibilities. In 1955, the
Northern Rhodesian African National Congress leader, Harry Nkumbula,
wrote to the queen of England, asking her to appoint a commission
including Africans ``to determine whether it is just that the people
should be dispossessed of their land''; whether the power generated by
the dam ``could not be better generated by nuclear energy''; whether the
compensation the people received was sufficient and whether ``the lands
to which the people are being moved are equal in value'' and fertility
to those that would be flooded. Perhaps human folly is culturally
relative.

When they were first informed about the dam, the Gwembe Native Authority
made a set of 24 demands respecting their rights --- to land, property,
reparations, protection, information. The 11th was: ``That in moving
people, their choices shall be seriously considered before they shall be
ignored.'' And when Chipepo's people staged their ultimately futile
uprising, they wrote messages in English, which they sent to the
district officers and the native authorities or nailed to trees on the
battlefield: ``We shall die in our land\ldots{}. We don't want to be
removed to Lusitu or to any place. We will not go home until you dismiss
your army of policemen. We will not fight with weapons but with words.''
What would paying attention and respect to their words have made
possible?

The Tonga knew the Zambezi. They knew that a river keeps time, not like
a clock but like a chronicle. They knew its sediments and grooves, the
patterns of the beings dwelling within it and nearby, its might and its
tendencies. Kariva rock itself was testament to a river that had knocked
away its stony triplets, a river so powerful that it seemed that a god
must live inside it.

A river can channel water into an immense power. A river can also flood,
spread into the spaces open to it. A river is both a singular, driving
force and a distributive, branching one. The Tonga had long lived
peacefully on both sides of the Zambezi, crossing back and forth to
court brides, borrow food, visit relatives. They knew that you don't
stop a river; you move over, through and with it. You follow its paths.
You may step into it as often as you wish, but you do not stay.

\hypertarget{the-great-climate-migrationthe-teenagers-at-the-end-of-the-worlddestroying-a-way-of-life-to-save-louisianathe-fearsome-thunderstorms-of-cuxf3rdoba-provincelearning-from-the-kariba-dam}{%
\paragraph{\texorpdfstring{\href{https://www.nytimes3xbfgragh.onion/interactive/2020/07/23/magazine/climate-migration.html}{The
Great Climate
Migration}\href{https://www.nytimes3xbfgragh.onion/interactive/2020/07/21/magazine/teenage-activist-climate-change.html}{The
Teenagers at the End of the
World}\href{https://www.nytimes3xbfgragh.onion/interactive/2020/07/21/magazine/louisiana-coast-engineering.html}{Destroying
a Way of Life to Save
Louisiana}\href{https://www.nytimes3xbfgragh.onion/interactive/2020/07/22/magazine/worst-storms-argentina.html}{The
Fearsome Thunderstorms of Córdoba
Province}\href{https://www.nytimes3xbfgragh.onion/interactive/2020/07/22/magazine/zambia-kariba-dam.html}{Learning
From the Kariba
Dam}}{The Great Climate MigrationThe Teenagers at the End of the WorldDestroying a Way of Life to Save LouisianaThe Fearsome Thunderstorms of Córdoba ProvinceLearning From the Kariba Dam}}\label{the-great-climate-migrationthe-teenagers-at-the-end-of-the-worlddestroying-a-way-of-life-to-save-louisianathe-fearsome-thunderstorms-of-cuxf3rdoba-provincelearning-from-the-kariba-dam}}

\begin{center}\rule{0.5\linewidth}{\linethickness}\end{center}

\begin{center}\rule{0.5\linewidth}{\linethickness}\end{center}

Kariba Dam source photo: Dmitriy Kandinskiy/Shutterstock

\textbf{Namwali Serpell} is a Zambian writer who teaches at the
University of California, Berkeley. Her first novel, ``The Old Drift,''
was published in 2019.
\href{https://www.nytimes3xbfgragh.onion/2019/06/06/magazine/old-town-road-meme-game-of-thrones.html}{She
last wrote a Screenland column about ``Game of Thrones'' memes.}

The Climate Issue

\begin{itemize}
\tightlist
\item
  \href{https://www.nytimes3xbfgragh.onion/interactive/2020/07/23/magazine/climate-migration.html}{The
  Great Climate Migration}
\item
  The Teenagers at the End of the World
\item
  \href{https://www.nytimes3xbfgragh.onion/interactive/2020/07/21/magazine/louisiana-coast-engineering.html}{Destroying
  a Way of Life to Save Louisiana}
\item
  \href{https://www.nytimes3xbfgragh.onion/interactive/2020/07/22/magazine/zambia-kariba-dam.html}{Learning
  From the Kariba Dam}
\item
  \href{https://www.nytimes3xbfgragh.onion/interactive/2020/07/22/magazine/worst-storms-argentina.html}{What's
  Going on Inside the Fearsome Thunderstorms of Córdoba Province?}
\end{itemize}

\protect\hyperlink{}{} \protect\hyperlink{}{}

\includegraphics{https://static01.graylady3jvrrxbe.onion/newsgraphics/2020/07/26/climate/fbd0a5f2a975dc16f0d1a24d64f70f4d843e50a3/caret.svg}

\textbf{Correction:}~July 28, 2020

An earlier~version of this article misstated an aspect of the Kariba
Dam. Erosion threatens the foundation; it is not the case that the
foundation is eroded. The article also misstated the location of a
future dam. It will~be on a tributary of the Zambezi River, not on the
Zambezi River. And the trees in Lake Kariba are dead trees; they do not
continue to grow.

Read 91 Comments

\begin{itemize}
\item
\item
\item
\item
\end{itemize}

Advertisement

\protect\hyperlink{after-bottom}{Continue reading the main story}

\hypertarget{site-index}{%
\subsection{Site Index}\label{site-index}}

\hypertarget{site-information-navigation}{%
\subsection{Site Information
Navigation}\label{site-information-navigation}}

\begin{itemize}
\tightlist
\item
  \href{https://help.nytimes3xbfgragh.onion/hc/en-us/articles/115014792127-Copyright-notice}{©~2020~The
  New York Times Company}
\end{itemize}

\begin{itemize}
\tightlist
\item
  \href{https://www.nytco.com/}{NYTCo}
\item
  \href{https://help.nytimes3xbfgragh.onion/hc/en-us/articles/115015385887-Contact-Us}{Contact
  Us}
\item
  \href{https://www.nytco.com/careers/}{Work with us}
\item
  \href{https://nytmediakit.com/}{Advertise}
\item
  \href{http://www.tbrandstudio.com/}{T Brand Studio}
\item
  \href{https://www.nytimes3xbfgragh.onion/privacy/cookie-policy\#how-do-i-manage-trackers}{Your
  Ad Choices}
\item
  \href{https://www.nytimes3xbfgragh.onion/privacy}{Privacy}
\item
  \href{https://help.nytimes3xbfgragh.onion/hc/en-us/articles/115014893428-Terms-of-service}{Terms
  of Service}
\item
  \href{https://help.nytimes3xbfgragh.onion/hc/en-us/articles/115014893968-Terms-of-sale}{Terms
  of Sale}
\item
  \href{https://spiderbites.nytimes3xbfgragh.onion}{Site Map}
\item
  \href{https://help.nytimes3xbfgragh.onion/hc/en-us}{Help}
\item
  \href{https://www.nytimes3xbfgragh.onion/subscription?campaignId=37WXW}{Subscriptions}
\end{itemize}
