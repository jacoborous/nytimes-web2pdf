Sections

SEARCH

\protect\hyperlink{site-content}{Skip to
content}\protect\hyperlink{site-index}{Skip to site index}

\href{https://www.nytimes3xbfgragh.onion/section/us}{U.S.}

\href{https://myaccount.nytimes3xbfgragh.onion/auth/login?response_type=cookie\&client_id=vi}{}

\href{https://www.nytimes3xbfgragh.onion/section/todayspaper}{Today's
Paper}

\href{/section/us}{U.S.}\textbar{}Minneapolis Police Use Force Against
Black People at 7 Times the Rate of Whites

\url{https://nyti.ms/2XrIWb0}

\begin{itemize}
\item
\item
\item
\item
\item
\item
\end{itemize}

\hypertarget{race-and-america}{%
\subsubsection{\texorpdfstring{\href{https://www.nytimes3xbfgragh.onion/news-event/george-floyd-protests-minneapolis-new-york-los-angeles?name=styln-george-floyd\&region=TOP_BANNER\&variant=undefined\&block=storyline_menu_recirc\&action=click\&pgtype=Interactive\&impression_id=48631ff0-e3ab-11ea-adff-4dc7383067ec}{Race
and America}}{Race and America}}\label{race-and-america}}

\begin{itemize}
\tightlist
\item
  \href{https://www.nytimes3xbfgragh.onion/interactive/2020/07/03/us/george-floyd-protests-crowd-size.html?name=styln-george-floyd\&region=TOP_BANNER\&variant=undefined\&block=storyline_menu_recirc\&action=click\&pgtype=Interactive\&impression_id=48634700-e3ab-11ea-adff-4dc7383067ec}{Black
  Lives Matter Movement}
\item
  \href{https://www.nytimes3xbfgragh.onion/interactive/2020/06/28/us/i-cant-breathe-police-arrest.html?name=styln-george-floyd\&region=TOP_BANNER\&variant=undefined\&block=storyline_menu_recirc\&action=click\&pgtype=Interactive\&impression_id=48634701-e3ab-11ea-adff-4dc7383067ec}{History
  of `I Can't Breathe'}
\item
  \href{https://www.nytimes3xbfgragh.onion/interactive/2020/06/10/upshot/black-lives-matter-attitudes.html?name=styln-george-floyd\&region=TOP_BANNER\&variant=undefined\&block=storyline_menu_recirc\&action=click\&pgtype=Interactive\&impression_id=48634702-e3ab-11ea-adff-4dc7383067ec}{How
  Public Opinion Shifted}
\item
  \href{https://www.nytimes3xbfgragh.onion/interactive/2020/07/16/us/black-lives-matter-protests-louisville-breonna-taylor.html?name=styln-george-floyd\&region=TOP_BANNER\&variant=undefined\&block=storyline_menu_recirc\&action=click\&pgtype=Interactive\&impression_id=48636e10-e3ab-11ea-adff-4dc7383067ec}{45
  Days in Louisville}
\end{itemize}

Advertisement

\protect\hyperlink{after-top}{Continue reading the main story}

\hypertarget{comments}{%
\subsection{\texorpdfstring{\protect\hyperlink{commentsContainer}{Comments}}{Comments}}\label{comments}}

\href{}{Minneapolis Police Use Force Against Black People at 7 Times the
Rate of Whites}\href{}{Skip to Comments}

The comments section is closed. To submit a letter to the editor for
publication, write to
\href{mailto:letters@NYTimes.com}{\nolinkurl{letters@NYTimes.com}}.

\hypertarget{minneapolis-police-use-force-against-black-people-at-7-times-the-rate-of-whites}{%
\section{Minneapolis Police Use Force Against Black People at 7 Times
the Rate of
Whites}\label{minneapolis-police-use-force-against-black-people-at-7-times-the-rate-of-whites}}

By
\href{https://www.nytimes3xbfgragh.onion/by/richard-a-oppel-jr}{Richard
A. Oppel Jr.} and
\href{https://www.nytimes3xbfgragh.onion/by/lazaro-gamio}{Lazaro
Gamio}June 3, 2020

\begin{itemize}
\item
\item
\item
\item
\item
  \emph{411}
\end{itemize}

\hypertarget{black-people-in-minneapolis-as-a-share-of-}{%
\subsubsection{Black people in Minneapolis as a share of
...}\label{black-people-in-minneapolis-as-a-share-of-}}

Population

19\%

Police officers

9\%

Subjects of police

use of force

58\%

Population

19\%

Police officers

9\%

Subjects of police

use of force

58\%

Video of George Floyd's last conscious moments horrified the nation,
spurring protests that have led to curfews and National Guard
interventions in many large cities.

But for the black community in Minneapolis --- where Mr. Floyd died
after an officer pressed a knee into his neck for 8 minutes 46 seconds
--- seeing the police use some measure of force is disturbingly common.

About
\href{https://www.census.gov/quickfacts/minneapoliscityminnesota}{20
percent} of Minneapolis's population of 430,000 is black. But when the
police get physical --- with kicks, neck holds, punches, shoves,
takedowns, Mace, Tasers or other forms of muscle --- nearly 60 percent
of the time the person subject to that force is black. And that is
according to
\href{http://opendata.minneapolismn.gov/datasets/police-use-of-force}{the
city's own figures}.

\hypertarget{police-shootings-and-use-of-force-against-black-people-in-minneapolis-since-2015}{%
\subsubsection{Police shootings and use of force against black people in
Minneapolis since
2015}\label{police-shootings-and-use-of-force-against-black-people-in-minneapolis-since-2015}}

Number of times police

used force against black

people per block

10

50

100

200

Thurman Blevins

June 2018

CAMDEN

NORTHEAST

Mario Benjamin

August 2019

NEAR NORTH

Jamar Clark

November 2015

UNIVERSITY

CENTRAL

Mississippi River

CALHOUN-ISLES

PHILLIPS

LONGFELLOW

Bde

Maka

Ska

POWDERHORN

Where officers

pinned George Floyd

Lake

Harriet

SOUTHWEST

Lake

Nokomis

NOKOMIS

Police shootings of black people

Share of population that is black

Fatal

Nonfatal

20\%

40\%

60\%

Number of times police used force

against black people per block

10

50

100

200

Police shootings of black people

Fatal

Nonfatal

Thomas Blevins

June 2018

Mario Benjamin

August 2019

Jamar Clark

November 2015

Where officers

pinned George Floyd

Share of population that is black

20\%

40\%

60\%

Number of times police used force

against black people per block

Police shootings of black people

Fatal

Nonfatal

10

50

100

200

Share of population that is black

Thurman Blevins

June 2018

20\%

40\%

60\%

More than one-fourth of all uses of force were in the northwestern parts
of the city.

CAMDEN

NORTHEAST

Mario Benjamin

August 2019

NEAR NORTH

Jamar Clark

November 2015

UNIVERSITY

The downtown area accounts for an additional one-third of uses of force.

CENTRAL

Mississippi River

CALHOUN-ISLES

PHILLIPS

LONGFELLOW

Bde Maka Ska

POWDERHORN

Where officers

pinned George Floyd

Lake Harriet

SOUTHWEST

Lake

Nokomis

NOKOMIS

Number of times police used force

against black people per block

Police shootings of black people

Fatal

Nonfatal

Share of population that is black

10

50

100

200

Thurman Blevins

June 2018

20\%

40\%

60\%

More than one-fourth of all uses of force were in the northwestern parts
of the city.

CAMDEN

NORTHEAST

Mario Benjamin

August 2019

NEAR NORTH

Jamar Clark

November 2015

UNIVERSITY

The downtown area accounts for an additional one-third of uses of force.

CENTRAL

Mississippi River

CALHOUN-ISLES

PHILLIPS

LONGFELLOW

Bde Maka Ska

POWDERHORN

Where officers

pinned George Floyd

Lake Harriet

SOUTHWEST

Lake

Nokomis

NOKOMIS

Note: Cases for which location was not listed or that occurred outside
city limits are not shown.

Community leaders say the frequency with which the police use force
against black residents helps explain a fury in the city that goes
beyond Mr. Floyd's death, which the
\href{https://www.nytimes3xbfgragh.onion/article/george-floyd-autopsy-michael-baden.html}{medical
examiner} ruled a homicide.

Since 2015, the Minneapolis police have documented using force about
11,500 times. For at least 6,650 acts of force, the subject of that
force was black.

By comparison, the police have used force about 2,750 times against
white people, who make up
\href{https://www.census.gov/quickfacts/minneapoliscityminnesota}{about
60 percent} of the population.

All of that means that the police in Minneapolis used force against
black people at a rate at least seven times that of white people during
the past five years.

Those figures reflect the total number of acts of force used by the
Minneapolis police since 2015. So if an officer slapped, punched and
body-pinned one person during the same scuffle, that may be counted as
three separate acts of force. There have been about 5,000 total episodes
since 2015 in which the police used at least one act of force on
someone.

The disparities in the use of force in Minneapolis parallel large racial
gaps in vital measures in the city, like income, education and
unemployment, said David Schultz, a professor at Hamline University in
St. Paul who has studied local police tactics for two decades.

``It just mirrors the disparities of so many other things in which
Minneapolis comes in very badly,'' Mr. Schultz said.

When he taught a course years ago on potential liability officers face
in the line of duty, Mr. Schultz said, he would describe Minneapolis as
``a living laboratory on everything you shouldn't do when it comes to
police use of force.''

\hypertarget{police-reported-uses-of-force-in-minneapolis-by-year}{%
\subsubsection{Police-reported uses of force in Minneapolis by
year}\label{police-reported-uses-of-force-in-minneapolis-by-year}}

3,000

Uses of

force in

2019

2,000

41\%

All others

1,000

59\%

Black

people

0

'10

'15

'19

3,000

Uses of

force in

2019

2,000

41\%

All others

1,000

59\%

Black

people

'10

'15

'19

Mr. Schultz credits the current police chief, Medaria Arradondo, for
seeking improvements but said that in a lot of respects the department
still operates like it did decades ago.

``We have a pattern that goes back at least a generation,'' Mr. Schultz
said.

The protests in Minneapolis have also been fueled by memories of several
black men killed by police officers who either never faced charges or
were acquitted. They include Jamar Clark, 24,
\href{https://www.nytimes3xbfgragh.onion/2016/03/31/us/jamar-clark-shooting-minneapolis.html}{shot
in Minneapolis in 2015} after, prosecutors said, he tried to grab an
officer's gun; Thurman Blevins, 31,
\href{https://www.nytimes3xbfgragh.onion/2018/07/30/us/minneapolis-police-thurman-blevins.html}{shot
in Minneapolis in 2018} as he yelled, ``Please don't shoot me,'' while
he ran through an alley; and Philando Castile, 32, whose girlfriend
live-streamed the aftermath of his
\href{https://www.nytimes3xbfgragh.onion/2017/06/16/us/police-shooting-trial-philando-castile.html}{2016
shooting in a Minneapolis suburb}.

The officer seen in the video pressing a knee into Mr. Floyd's neck,
Derek Chauvin, was fired from the force and charged with manslaughter
and third-degree murder. Minneapolis police officials did not respond to
questions about the type of force he used.

The city's use-of-force policy covers chokeholds, which apply direct
pressure to the front of the neck, but those are considered deadly force
to be used only in the most extreme circumstances. Neck restraints are
also part of the policy, but those are explicitly defined only as
putting direct pressure on the side of the neck --- and not the trachea.

``Unconscious neck restraints,'' in which an officer is trying to render
someone unconscious, have been used 44 times in the past five years ---
27 of those on black people.

For years, experts say,
\href{https://www.nytimes3xbfgragh.onion/2020/05/29/us/knee-neck-george-floyd-death.html}{many
police departments} around the country have sought to move away from
neck restraints and chokeholds that might constrict the airway as being
just too risky.

\hypertarget{types-of-force-used-by-minneapolis-police}{%
\subsubsection{Types of force used by Minneapolis
police}\label{types-of-force-used-by-minneapolis-police}}

TYPE OF FORCE

SHARE USED ON BLACK PEOPLE

TOTAL

Gunpoint display

68\%

171

Chemical irritants

66\%

1,748

Neck restraints

66\%

258

Improvised weapon

64\%

115

Dogs

61\%

77

Body-weight pin

60\%

3,630

Taser

60\%

785

Takedowns, joint locks

59\%

1,820

Restraint techniques

59\%

127

Hitting

58\%

2,159

Other methods

56\%

110

SHARE USED ON

BLACK PEOPLE

TYPE OF FORCE

TOTAL

Gunpoint display

68\%

171

Chemical irritants

66\%

1,748

Neck restraints

66\%

258

Improvised weapon

64\%

115

Dogs

61\%

77

Body-weight pin

60\%

3,630

Taser

60\%

785

Takedowns, joint locks

59\%

1,820

Restraint techniques

59\%

127

Hitting

58\%

2,159

Other methods

56\%

110

Note: Cases for which force type or race were not listed are not shown.

Dave Bicking, a former member of the Minneapolis civilian police review
authority, said the tactic used on Mr. Floyd was not a neck restraint
under city policy because it resulted in pressure to the front of Mr.
Floyd's neck.

If anything, he said, it was an unlawful type of body-weight pin, a
category that is the most frequently deployed type of force in the city:
Since 2015, body-weight pinning has been used about 2,200 times against
black people, more than twice the number of times it was used against
whites.

Mr. Bicking, a board member of Communities United Against Police
Brutality, a Minnesota-based group, said that since 2012 more than 2,600
civilian complaints have been filed against Minneapolis police officers.

Other investigations have led to some officers' being terminated or
disciplined --- like Mohamed Noor, the officer who killed an Australian
woman in 2017 and was later fired and
\href{https://www.nytimes3xbfgragh.onion/2019/04/30/us/minneapolis-police-noor-verdict.html}{convicted
of third-degree murder}.

But, Mr. Bicking said, in only a dozen cases involving 15 officers has
any discipline resulted from a civilian complaint alleging misconduct.
The worst punishment, he said, was 40 hours of unpaid suspension.

``That's a week's unpaid vacation,'' said Mr. Bicking, who contends that
the city has abjectly failed to discipline wayward officers, which he
said contributed to last week's tragedy. He noted that the former
officer now charged with Mr. Floyd's murder had faced at least 17
complaints.

``If discipline had been consistent and appropriate, Derek Chauvin would
have either been a much better officer, or would have been off the
force,'' he said. ``If discipline had been done the way it should be
done, there is virtually no chance George Floyd would be dead now.''

The city's use-of-force numbers almost certainly understate the true
number of times force is used on the streets, Mr. Bicking said. But he
added that even the official reported data go a long way to explain the
anger in Minneapolis.

``This has been years and years in the making,'' he said. ``George Floyd
was just the spark.''

Fears that the Minneapolis police may have an uncontrollable problem
appeared to prod state officials into action Tuesday. The governor, Tim
Walz, a Democrat, said the State Department of Human Rights launched an
investigation into whether the police department ``engaged in systemic
discriminatory practices towards people of color'' over the past decade.
One possible outcome: a court-enforced decree requiring major changes in
how the force operates.

Announcing the inquiry, Governor Walz pledged to ``use every tool at our
disposal to deconstruct generations of systemic racism in our state.''

While some activists believe the Minneapolis department is one of the
worst-behaving urban forces in the country, comparative national numbers
on use of force are hard to come by.

According to Philip M. Stinson, a criminologist at Bowling Green State
University, some of the most thorough U.S. data comes from a study by
the Justice
Department\href{https://www.bjs.gov/content/pub/pdf/punf0211.pdf}{}\href{https://www.bjs.gov/content/pub/pdf/punf0211.pdf}{published
in November 2015}: The study found that 3.5 percent of black people said
they had been subject to nonfatal force --- or the threat of such force
--- during their most recent contact with the police, compared with 1.4
percent of white people.

Minneapolis police officials did not respond to questions about their
data and use-of-force rates. In other places,
\href{https://www.nytimes3xbfgragh.onion/2014/11/21/us/activists-wield-search-data-to-challenge-and-change-police-policy.html}{studies
have shown} disparate treatment of black people, such as in searches
during traffic stops. Some law enforcement officials have reasoned that
since high-crime areas are often disproportionately populated by black
residents, it is no surprise that black residents would be subject to
more police encounters. (The same studies have also shown that black
drivers, when searched, possessed contraband no more often than white
drivers.)

The Minneapolis data shows that most use of force happens in areas where
more black people live. Although crime rates are higher in those areas,
black people are also subject to police force more often than white
people in some mostly white and wealthy neighborhoods, though the total
number of episodes in those areas is small.

Mr. Stinson, who is also a former police officer, said he believes that
at some point during the arrest of Mr. Floyd, the restraint applied to
him became ``intentional premeditated murder.''

``In my experience, applying pressure to somebody's neck in that fashion
is always understood to be the application of deadly force,'' Mr.
Stinson said.

But equally revealing in the video, he said, was that other officers
failed to intercede, despite knowing they were being filmed. He said
that suggests the same thing that the use-of-force data also suggest:
That police in the city ``routinely beat the hell out of black men.''

``Whatever that officer was doing was condoned by his colleagues,'' Mr.
Stinson said. ``They didn't seem surprised by it at all. It was business
as usual.''

Note: Police use-of-force data was retrieved on May 29, 2020, and shows
cases up to May 26, 2020. Data on officer-involved shootings is recorded
separately and shows cases through 2019; these episodes are shown on the
map but not included in the analysis or charts of use of force.
Instances of use of force for which race information was not available
are not shown in the charts or map.

Sources: U.S. Census Bureau; Bureau of Justice Statistics; City of
Minneapolis.

Read 411 Comments

\begin{itemize}
\item
\item
\item
\item
\end{itemize}

Advertisement

\protect\hyperlink{after-bottom}{Continue reading the main story}

\hypertarget{site-index}{%
\subsection{Site Index}\label{site-index}}

\hypertarget{site-information-navigation}{%
\subsection{Site Information
Navigation}\label{site-information-navigation}}

\begin{itemize}
\tightlist
\item
  \href{https://help.nytimes3xbfgragh.onion/hc/en-us/articles/115014792127-Copyright-notice}{©~2020~The
  New York Times Company}
\end{itemize}

\begin{itemize}
\tightlist
\item
  \href{https://www.nytco.com/}{NYTCo}
\item
  \href{https://help.nytimes3xbfgragh.onion/hc/en-us/articles/115015385887-Contact-Us}{Contact
  Us}
\item
  \href{https://www.nytco.com/careers/}{Work with us}
\item
  \href{https://nytmediakit.com/}{Advertise}
\item
  \href{http://www.tbrandstudio.com/}{T Brand Studio}
\item
  \href{https://www.nytimes3xbfgragh.onion/privacy/cookie-policy\#how-do-i-manage-trackers}{Your
  Ad Choices}
\item
  \href{https://www.nytimes3xbfgragh.onion/privacy}{Privacy}
\item
  \href{https://help.nytimes3xbfgragh.onion/hc/en-us/articles/115014893428-Terms-of-service}{Terms
  of Service}
\item
  \href{https://help.nytimes3xbfgragh.onion/hc/en-us/articles/115014893968-Terms-of-sale}{Terms
  of Sale}
\item
  \href{https://spiderbites.nytimes3xbfgragh.onion}{Site Map}
\item
  \href{https://help.nytimes3xbfgragh.onion/hc/en-us}{Help}
\item
  \href{https://www.nytimes3xbfgragh.onion/subscription?campaignId=37WXW}{Subscriptions}
\end{itemize}
