Sections

SEARCH

\protect\hyperlink{site-content}{Skip to
content}\protect\hyperlink{site-index}{Skip to site index}

\hypertarget{comments}{%
\subsection{\texorpdfstring{\protect\hyperlink{commentsContainer}{Comments}}{Comments}}\label{comments}}

\href{}{How Architecture Could Help Us Adapt to the
Pandemic}\href{}{Skip to Comments}

The comments section is closed. To submit a letter to the editor for
publication, write to
\href{mailto:letters@NYTimes.com}{\nolinkurl{letters@NYTimes.com}}.

\hypertarget{how-architecture-could-help-us-adapt-to-the-pandemic}{%
\section{How Architecture Could Help Us Adapt to the
Pandemic}\label{how-architecture-could-help-us-adapt-to-the-pandemic}}

By Kim TingleyJune 10, 2020

\begin{itemize}
\item
\item
\item
\item
\item
  \emph{134}
\end{itemize}

The virus isn't simply a health crisis; it is also a design problem.

\hypertarget{how-architecture-could-help-us-adapt-to-the-pandemic-1}{%
\section{How Architecture Could Help Us Adapt to the
Pandemic}\label{how-architecture-could-help-us-adapt-to-the-pandemic-1}}

The virus isn't simply a health crisis; it is also a design problem.

By Kim Tingley

June 9, 2020

SHARE

The last class Joel Sanders taught in person at the Yale School of
Architecture, on Feb. 17, took place in the modern wing of the Yale
University Art Gallery, a structure of brick, concrete, glass and steel
that was designed by Louis Kahn. It is widely hailed as a masterpiece.
One long wall, facing Chapel Street, is windowless; around the corner, a
short wall is all windows. The contradiction between opacity and
transparency illustrates a fundamental tension museums face, which
happened to be the topic of Sanders's lecture that day: How can a
building safeguard precious objects and also display them? How do you
move masses of people through finite spaces so that nothing --- and no
one --- is harmed?

All semester, Sanders, who is a professor at Yale and also runs Joel
Sanders Architect, a studio located in Manhattan, had been asking his
students to consider a 21st-century goal for museums: to make facilities
that were often built decades, if not centuries, ago more inclusive.
They had conducted workshops with the gallery's employees to learn how
the iconic building could better meet the needs of what Sanders calls
``noncompliant bodies.'' By this he means people whose age, gender,
race, religion or physical or cognitive abilities often put them at odds
with the built environment, which is typically designed for people who
embody dominant cultural norms. In Western architecture, Sanders points
out, ``normal'' has been explicitly defined --- by the ancient Roman
architect Vitruvius, for instance, whose concepts inspired Leonardo da
Vinci's ``Vitruvian Man,'' and, in Kahn's time, by Le Corbusier's
``Modulor Man'' --- as a youngish, tallish white male.

When the coronavirus crisis prompted Yale to move classes online,
Sanders's first thought was: ``How do you make the content of your class
seem relevant during a global pandemic? Why should we be talking about
museums when we have more urgent issues to fry?'' Off campus, built
environments and the ways people moved in them began to change
immediately in desperate, ad hoc ways. Grocery stores erected plexiglass
shields in front of registers and put stickers or taped lines on the
floor to create six-foot spacing between customers; as a result, fewer
shoppers fit safely inside, and lines snaked out the door. People became
hyperaware of themselves in relation to others and the surfaces they
might have to touch. Suddenly, Sanders realized, everyone had become a
``noncompliant body.'' And places deemed essential were wrestling with
how near to let them get to one another. The virus wasn't simply a
health crisis; it was also a design problem.

The tensions created by particular persons interacting with particular
spaces has long been an interest of Sanders's. ``I love beautiful
things, but I'm not interested in form for its own sake,'' he says.
``What counts is human experience and human interaction, and how form
facilitates that.''

The beginning of his career coincided with the AIDS crisis in New York.
That time, when as a gay man he felt unwelcome or threatened in public
spaces, informed his design ethos. His portfolio includes residences
with open, flexible floor plans that allow people to assume different
roles --- a sitting area could be used for work or leisure, say --- and
adopt nontraditional family arrangements. About five years ago, as the
fight over whether transgender people should have the right to use
public bathrooms corresponding with their gender identity became
national news, Sanders was struck by the fact that ``nobody talked about
it from a design perspective,'' he says. ``And everyone took for granted
and accepted sex-segregated bathrooms.'' How, he wondered, had we ended
up with men's and women's rooms in the first place?

While working on an article with Susan Stryker, a professor of gender
and women's studies then at the University of Arizona, he learned that
public bathing had been a coed activity at various points in history; so
was defecating, which, when it didn't happen in the street or involve a
chamber pot, sometimes took place in a separate communal facility. Only
with the advent of indoor plumbing and municipal sanitation systems in
the 19th century did bathing and eliminating begin to come together.
According to the legal scholar Terry Kogan, the first indoor bathrooms
that were sex-specific and open to the public appeared in the U.S. in
the mid-1800s, where they were extensions of separate parlor spaces for
men and women.

Segregating toilets by sex clearly wasn't a biological imperative. It
expressed men's and women's social roles in Victorian times. What if,
Sanders and Stryker asked, you instead organized that space around the
activity being performed and how much privacy it required? The entire
``bathroom'' could be an area with no walls or doors except on private
stalls near the back. Activities requiring less privacy, like
hand-washing, could be located in a middle, openly visible zone. ``You
could make the toilet a space that isn't a sense of heightened danger
because there's a closed door and someone who isn't supposed to be there
is there,'' Stryker, who is transgender, says.

Greater visibility, they hoped, would make bathrooms safer for
transgender women, who are at increased risk of violence there. Sanders
had also begun to encounter others for whom these spaces meant constant
anxiety for a number of reasons: wheelchair users, those assisting
elderly parents or small children, Muslims performing ablutions, women
breastfeeding. It dawned on him how limited his own perspective was, as
well as that of the clients he typically consulted on their projects.
``You need to get the lived experience of the end user,'' he told me.
``That's what architects like me were never trained to do, and we're not
good at it.''

In 2018, Sanders, Stryker and Kogan published their research and
prototypes for multiuser, multigender restrooms on a website as part of
an initiative they named ``Stalled!'' Around the same time, Sanders
formed a new branch of his firm called MIXdesign to function as a think
tank and consultancy. The goal was to identify those whose needs have
rarely been considered in architecture --- who might even be avoiding
public spaces --- and to collaborate with them on recommendations that
designers could use to make buildings more welcoming for as many people
as possible.

The chaos that Covid-19 has brought to once-familiar places lent an
urgency to this mission: Could MIX use the approach it was developing to
imagine spaces not just for a wider variety of individuals, but for an
entirely new reality?

\textbf{Architecture has to} mediate between the perceived needs of the
moment versus the unknowable needs of the future; between the immediate
needs of our bodies and the desire to create something that will outlast
generations. As public venues begin to reopen, authorities are
scrambling to put out advice on how to adapt them for a pandemic. On May
6, the American Institute of Architects first released guidance aiming
to ``provide a range of general mitigation measures to consider,'' such
as moving activities outside and reconfiguring furniture to keep people
farther apart indoors. It's far too soon to say how architects will
rethink more permanent aspects of projects in progress. ``I think
there's way too much prognostication going on,'' says Vishaan
Chakrabarti, the founder of the architecture firm PAU and the incoming
dean of the University of California, Berkeley, College of Environmental
Design. Chakrabarti was the planning director for Manhattan under Mayor
Bloomberg after Sept. 11. ``A lot of the fortunetelling that went on
then has not aged well,'' he told me. ``People said there will never be
skyscrapers again and cities are dead.'' Instead, what changed was
increased surveillance and security.

Sanders and MIX have a number of active commissions they are just
beginning to revisit with an eye to making them Covid-compliant: A
renovation of the SoCal Club, an outreach initiative by the Men's Health
Foundation in L.A. that seeks to engage young gay men and transgender
men and women of color in medical care, is in progress, undertaken with
a local firm; a potential remaking of the Queens Museum entryway is in
the preliminary stages.

Rather than respond with temporary barriers or signs, Sanders is trying
to use MIX's research process to arrive at designs that minimize the
spread of the coronavirus and appeal to diverse users. This, he hopes,
will result in buildings that endure, whether or not a vaccine becomes
available. ``MIX is really leading the way on this particular set of
issues,'' Rosalie Genevro, executive director of the Architectural
League of New York, told me. ``There are a lot of people quickly trying
to think about spatial life in the Covid era. MIX has the most explicit
commitment that I've seen so far to making sure that thinking is as
inclusive as possible.''

Soon after founding MIX, Sanders approached Eron Friedlaender, a
pediatric emergency-medicine physician at the Children's Hospital of
Philadelphia. From the Queens Museum, Sanders had learned that people
with autism found the main atrium --- a wide open, reverberant space ---
especially upsetting. Friedlaender has a teenage son with autism, and
she had been looking for ways to make health care facilities more
accessible to others on the spectrum, who often find them overwhelming.
As a result, they seek medical services less frequently than their peers
do and are sicker when they do show up. When the MIX group first started
talking about the pandemic, on a video call, the overlap between the
anxiety everyone was feeling in public spaces and the anxiety people
with autism already feel in those same environments was striking. And
the consequences were similar, too. Friedlaender noted that hospitals
across the country, including her E.R., had seen a stark drop in their
overall number of patients, who, they believe, are still experiencing
the same health problems but are too afraid to come in.

The isolation people were suffering while sheltering at home was also
familiar to her, she said in an early MIX meeting. People with autism
frequently experience loneliness, in part because closeness to others
tends to make them uncomfortable, which often keeps them from crowded
places. From their perspective, ``you can be physically distant'' --- by
maintaining space between bodies, she told me --- ``and more socially
engaged.''

That seeming paradox resonated with Hansel Bauman, another MIX member,
for a different reason, he told the group. As the former campus
architect at Gallaudet University, an institution for students who are
deaf and hard of hearing, he needed to double any amount of space
typically allocated for hearing people --- to give students more room
between one another to sign. At Gallaudet, Bauman worked with students
and faculty members to come up with DeafSpace, a set of design
principles that took into account their needs; they did this by filming
hallways and cafeterias, for example, and watching hundreds of hours of
interactions there. ``Corners in the hearing world,'' he said, are not
designed ``to visually anticipate the movement of others.'' Sound
communicates to hearing people when someone is coming --- and in the
past it didn't matter as much to them if they missed the signals and
brushed against one another. ``In the Covid world, you bump into
somebody coming around the corner and they're not wearing a mask,''
Bauman went on, ``all of a sudden, now there's a potential for
infection.'' DeafSpace recommendations would most likely help:
``Strategic sight lines; the use of color and light as means of
way-finding.'' Promoting more efficient, less reactive movement was, he
said, the kind of thing ``we've been wrestling with in DeafSpace for the
last 15 years.''

Designing to promote social distancing, it seemed, could actually make
spaces more universally hospitable. But it was harder to guess what the
overall effect of other Covid accommodations might be. ``One thing that
has been interesting, as more and more articles are being written about
Covid --- they don't want the high-powered dryers,'' Seb Choe, MIX's
associate director, noted during a design meeting in late May. ``Because
dryers blow germs around the room.'' The group had added big windows to
one of its prototypes to disinfect surfaces with sunlight, but Bauman
pointed out that glare would make it harder for people to see one
another, making it especially difficult for deaf users to communicate
and causing everyone to potentially draw closer together. He suggested
adding, among other things, an overhang outside for shade.

Choe pointed out a news story that day that re-emphasized the C.D.C.'s
guidance that the virus is not transmitted as easily through surface
contact as it is through the air. Maybe sunshine wasn't as much of a
priority anymore? Indeed, the following week, in a Washington Post
op-ed, Joseph Allen, the director of the Healthy Buildings program at
the Harvard T.H. Chan School of Public Health, called for open windows
and improved ventilation and suggested 10 feet between people would be
better than six.

``This is the conundrum,'' Sanders said. ``How do you design with this
as a moving target? You don't want to lock in dimensions.'' And suppose
the way coronavirus is transmitted could be perfectly understood and
avoided --- would that change the hesitation people feel about riding
elevators together or using touch screens? Designers might have to
reconcile settled science with people's lingering uneasiness.

\textbf{Helping clients} \textbf{articulate} how a design makes them
feel, and why, is notoriously challenging. ``The way architects get
people to tell us what they think about a space is to walk them through
the space and say, `What do you think?' Or we show them pictures,''
Sanders told me. He wanted to engage people with autism in his design
process, in part to learn other ways of posing those questions.

In January, along with Bauman and Friedlaender, Sanders convened a group
of experts, including Magda Mostafa, a Cairo-based architect and the
author of ``Autism ASPECTSS,'' a set of design guidelines, to discuss
ways to understand how people with autism feel about their surroundings.
In May, they met again, along with researchers from the Center for
Autism and Neurodiversity at Jefferson University Hospital in
Philadelphia, to continue that discussion, while considering how the
coronavirus might impact their work. ``My concern,'' Friedlaender said,
``is people with autism don't necessarily know how to articulate what
they're thinking. I don't think we can just depend on their words.''

The group began to brainstorm various ways of engaging people with
autism in the design process. Perhaps participants could experience
spaces using virtual reality while researchers monitored their physical
reactions. Sanders wondered aloud whether this might also be a useful
way to work with other focus groups on design responses to the pandemic.
The Queens Museum had been planning to host a dance for people from a
senior center to get their reactions to the space; now large gatherings
are dangerous, and the museum is being transformed into a
food-distribution center.

``When I think of a space that is Covid-friendly, I think of one that
can be quickly closed off,'' Joseph McCleery, an autism researcher at
St. Joseph's University, told the group. ``You have stuff that's
available that's maybe in the basement but can be quickly brought out.''

``Flexibility and agility of space, but also compartmentalization of
space,'' Mostafa said. Her designs include breakout pods off
high-traffic areas that can serve as an escape for those who feel
overstimulated. ``But,'' she noted, ``they also happen to create spaces
with different air circulation, occupied by fewer people.''

Listening to them describe various approaches to being together while
remaining apart, it was easy to see how people with autism, and other
groups that have faced difficulties in the built environment, are in a
special position to identify creative solutions to the spatial
challenges the virus poses --- and to suggest improvements to pervasive
design flaws no one else has identified yet. Perhaps Covid would inspire
broader collaborations.

But fear also has the potential to trigger reactionary responses.
Sanders emphasized this concern every time we spoke. He worries that
funding earmarked for expanding inclusivity will be diverted toward
making existing facilities safer for those they already privilege.
Throughout history, he observed, the built environment has reflected and
reinforced inequality by physically separating one group from another,
often in the presumed interests of health or safety. Women-only
bathrooms, so designated by men, supposedly preserved their innocence
and chastity; white-only bathrooms separated their users from supposedly
less ``clean'' black people. It's no coincidence that Covid-19 has
disproportionately sickened and killed members of demographic groups ---
people who are black, Indigenous and Latino; who are homeless; who are
immigrants --- that have been targets of systemic segregation that
increased their vulnerability. It's also not hard to imagine the
pandemic, and a person's relative risk of infection, being used to
justify new versions of these discriminatory practices. ``Who will be
demonized?'' Sanders said. ``We must not'' --- he smacked what sounded
like a glass-topped table for emphasis --- ``repeat the mistakes of the
past.''

Mabel O. Wilson, a professor of architecture and African-American and
African Diaspora Studies at Columbia University, thinks that Covid
``could be leveraged to remind people that many people don't feel
comfortable in public.'' But that doesn't mean it will be. ``My sense is
what's going to happen is, having clean rooms, having greater
circulation of air, is going to be the purview of the wealthy who can
afford it in their homes,'' she says. ``It will be determined by the
marketplace and not necessarily be a public amenity.''

\textbf{A future in} which we commingle again is hard to envision right
now. At the most basic level, what must happen for society to resume is
this: You approach the door of a building, open and pass through it and
navigate your way to a destination within. Architects call this critical
series of steps an entry sequence, a journey throughout which a person
is deciding whether to leave or stay. Toward the end of May, Marco Li, a
senior associate at MIX, created plans and 3-D renderings of an entry
sequence to a hypothetical campus building that incorporated some of the
group's ideas for pandemic adaptations. He showed them to Sanders,
Bauman and Choe over teleconference. They had invited a frequent
collaborator, Quemuel Arroyo, who is a former chief accessibility
specialist at the New York City Department of Transportation and a
wheelchair user, to critique them over a video call. The prototypes were
intended to spark discussion about how they might rethink entry
sequences for universities as well as museums and health care
facilities. ``What architects do well,'' Choe told me, ``is providing
imagination in terms of designing something that doesn't exist. Once
people see it, they can talk about it.''

Through the front door, in a vestibule, one-way entry and exit routes
were mediated by a planter. Each side had a hand-sanitizing station
along the wall. A second, interior door separated this transition zone
from the rest of the building. Once inside, a visitor encountered a wide
lobby. Across it, directly ahead, an information desk was positioned
back-to-back with a bank of lockers. Behind that partition were
multigender restroom stalls; rooms, with showers, that could be used by
caregivers, nursing mothers and even bike commuters; and prayer rooms
and foot-washing stations for religious practices. Motion-activated
sinks abutted the walkway. The space is more of a ``wellness hub'' now
than a ``bathroom,'' Sanders said --- so they decided to put it front
and center rather than hide it.

All along the lobby were ``calm zones'' delineated by flooring of a
different color and texture, with flexible seating options. ``Becoming
particularly important with Covid is differentiating bodies at rest from
bodies in motion,'' Sanders said, so that people don't crash into one
another. ``Defining those areas by color intensity allows people to
locate where they need to be in space.'' Someone who is avoiding an
obstacle, or who is confused or lost, causes a ripple of unpredictable
movements in others. ``Social distancing isn't people standing still in
space at a dotted line at the grocery store,'' Bauman had observed
previously. ``It's a dynamic situation.''

Arroyo asked about the textural demarcation between areas where people
walk and where they sit. Sanders explained that blind users could feel
them with a cane. ``Are these detectable edges beveled?'' Arroyo asked.
``Most people in wheelchairs hate that. You want to make sure that's
detectable but not a trip hazard.'' He also noted that none of the
bathroom sinks were low enough for a seated person. ``In a world of
Covid and germs being shared, my biggest pet peeve is flat surfaces,
because the water pools,'' he said. When he reached for the tap,
standing water dripped on his lap and wet his sleeves.

I felt a flash of recognition. Taking my 5-year-old to a public bathroom
almost always results in his shirt getting soaked. I'd imagined other,
better parents avoided this somehow. The relief I felt at learning that
this was a problem for someone else --- that it might be the sink's
fault, not mine --- was instructive in thinking about Sanders's work,
which on paper doesn't always register as so starkly different from the
places we inhabit now.

``What Joel's mission is for MIXdesign is to make these goals of
inclusivity in the built environment so inevitable that they're not
visible,'' says Deborah Berke, the dean of the Yale School of
Architecture and founder of an eponymous design firm in Manhattan. ``I
would put the visible at where you tack a ramp on the outside of a
building and say, `Great, we're done. We met A.D.A.,''' she told me,
referring to the Americans with Disabilities Act. ``This is about
sending such a fully inclusive message that you don't see it as that.
It's just a building that works for everybody.''

When we don't notice the built environment, it's silently affirming our
right to be there, our value to society. When we do, too often it is
because it's telling us we don't belong. Those messages can be so subtle
that we don't recognize them for what they are. ``We sleepwalk our way
through the world,'' Sanders told me. ``Unless a building interior is
strikingly different or lavish or unusual, we are unaware of it.''
Covid, he added, ``is forcing all of us to be aware of how the design of
the built environment dictates how we experience the world and each
other.''

\hypertarget{developing-a-covid-19-vaccinewhat-if-working-from-home-goes-on--foreverthe-pandemic-and-architectureinformation-can-be-the-best-medicine}{%
\paragraph{\texorpdfstring{\href{https://www.nytimes3xbfgragh.onion/interactive/2020/06/09/magazine/covid-vaccine.html}{Developing
a Covid-19
Vaccine}\href{https://www.nytimes3xbfgragh.onion/interactive/2020/06/09/magazine/remote-work-covid.html}{What
If Working From Home Goes on \ldots{}
Forever}\href{https://www.nytimes3xbfgragh.onion/interactive/2020/06/09/magazine/architecture-covid.html}{The
Pandemic and
Architecture}\href{https://www.nytimes3xbfgragh.onion/interactive/2020/06/10/magazine/covid-data.html}{Information
Can Be the Best
Medicine}}{Developing a Covid-19 VaccineWhat If Working From Home Goes on \ldots{} ForeverThe Pandemic and ArchitectureInformation Can Be the Best Medicine}}\label{developing-a-covid-19-vaccinewhat-if-working-from-home-goes-on--foreverthe-pandemic-and-architectureinformation-can-be-the-best-medicine}}

\begin{center}\rule{0.5\linewidth}{\linethickness}\end{center}

Kim Tingley is a contributing writer for the magazine and the Studies
Show columnist; topics have included the potential health impacts of
\href{https://www.nytimes3xbfgragh.onion/2020/01/22/magazine/can-mindfulness-evolve-from-wellness-pursuit-to-medical-treatment.html}{mindfulness,}
\href{https://www.nytimes3xbfgragh.onion/2019/07/23/magazine/when-you-wear-sunscreen-youre-taking-part-in-a-safety-study.html}{sunscreen}
and
\href{https://www.nytimes3xbfgragh.onion/2019/10/24/magazine/why-isnt-there-a-diet-that-works-for-everyone.html}{diets.}

Source photographs for photo illustrations: Getty Images.

The Tech \& Design Issue

\begin{itemize}
\tightlist
\item
  Developing a Covid-19 Vaccine
\item
  \href{https://www.nytimes3xbfgragh.onion/interactive/2020/06/09/magazine/remote-work-covid.html}{What
  If Working From Home Goes on \ldots{} Forever}
\item
  \href{https://www.nytimes3xbfgragh.onion/interactive/2020/06/09/magazine/architecture-covid.html}{The
  Pandemic and Architecture}
\item
  \href{https://www.nytimes3xbfgragh.onion/interactive/2020/06/10/magazine/covid-data.html}{Information
  Can Be the Best Medicine}
\end{itemize}

\protect\hyperlink{}{} \protect\hyperlink{}{}

\includegraphics{https://static01.graylady3jvrrxbe.onion/newsgraphics/2020/06/04/2020-magtechdesign/a4fdae938d97ed501d80d0405d6760d098697e64/caret.svg}

Read 134 Comments

\begin{itemize}
\item
\item
\item
\item
\end{itemize}

Advertisement

\protect\hyperlink{after-bottom}{Continue reading the main story}

\hypertarget{site-index}{%
\subsection{Site Index}\label{site-index}}

\hypertarget{site-information-navigation}{%
\subsection{Site Information
Navigation}\label{site-information-navigation}}

\begin{itemize}
\tightlist
\item
  \href{https://help.nytimes3xbfgragh.onion/hc/en-us/articles/115014792127-Copyright-notice}{©~2020~The
  New York Times Company}
\end{itemize}

\begin{itemize}
\tightlist
\item
  \href{https://www.nytco.com/}{NYTCo}
\item
  \href{https://help.nytimes3xbfgragh.onion/hc/en-us/articles/115015385887-Contact-Us}{Contact
  Us}
\item
  \href{https://www.nytco.com/careers/}{Work with us}
\item
  \href{https://nytmediakit.com/}{Advertise}
\item
  \href{http://www.tbrandstudio.com/}{T Brand Studio}
\item
  \href{https://www.nytimes3xbfgragh.onion/privacy/cookie-policy\#how-do-i-manage-trackers}{Your
  Ad Choices}
\item
  \href{https://www.nytimes3xbfgragh.onion/privacy}{Privacy}
\item
  \href{https://help.nytimes3xbfgragh.onion/hc/en-us/articles/115014893428-Terms-of-service}{Terms
  of Service}
\item
  \href{https://help.nytimes3xbfgragh.onion/hc/en-us/articles/115014893968-Terms-of-sale}{Terms
  of Sale}
\item
  \href{https://spiderbites.nytimes3xbfgragh.onion}{Site Map}
\item
  \href{https://help.nytimes3xbfgragh.onion/hc/en-us}{Help}
\item
  \href{https://www.nytimes3xbfgragh.onion/subscription?campaignId=37WXW}{Subscriptions}
\end{itemize}
