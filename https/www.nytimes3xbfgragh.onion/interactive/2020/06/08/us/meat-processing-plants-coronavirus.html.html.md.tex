\hypertarget{comments}{%
\subsection{\texorpdfstring{\protect\hyperlink{commentsContainer}{Comments}}{Comments}}\label{comments}}

\href{}{Take a Look at How Covid-19 Is Changing Meatpacking
Plants}\href{}{Skip to Comments}

The comments section is closed. To submit a letter to the editor for
publication, write to
\href{mailto:letters@NYTimes.com}{\nolinkurl{letters@NYTimes.com}}.

\includegraphics{https://static01.graylady3jvrrxbe.onion/newsgraphics/2020/05/14/meat-processing-covid/1d5517d56c515053314f633f4fbd3a81341c66be/line-mo-01.png}

\includegraphics{https://static01.graylady3jvrrxbe.onion/newsgraphics/2020/05/14/meat-processing-covid/1d5517d56c515053314f633f4fbd3a81341c66be/next_arrow.svg}

\hypertarget{take-a-look-at-how-covid-19-is-changing-meatpacking-plants}{%
\section{Take a Look at How Covid-19 Is Changing Meatpacking
Plants}\label{take-a-look-at-how-covid-19-is-changing-meatpacking-plants}}

\hypertarget{see-how-workers-stand-shoulder-to-shoulder-to-provide-americans-with-affordable-meat-as-plants-reopen-the-coronavirus-remains-a-threat}{%
\subsection{See how workers stand shoulder to shoulder to provide
Americans with affordable meat. As plants reopen, the coronavirus
remains a
threat.}\label{see-how-workers-stand-shoulder-to-shoulder-to-provide-americans-with-affordable-meat-as-plants-reopen-the-coronavirus-remains-a-threat}}

By
\href{https://www.nytimes3xbfgragh.onion/by/yuliya-parshina-kottas}{Yuliya
Parshina-Kottas},
\href{https://www.nytimes3xbfgragh.onion/by/larry-buchanan}{Larry
Buchanan},
\href{https://www.nytimes3xbfgragh.onion/by/aliza-aufrichtig}{Aliza
Aufrichtig} and
\href{https://www.nytimes3xbfgragh.onion/by/michael-corkery}{Michael
Corkery} June 8, 2020

\includegraphics{https://static01.graylady3jvrrxbe.onion/newsgraphics/2020/05/14/meat-processing-covid/1d5517d56c515053314f633f4fbd3a81341c66be/line-mo-02.png}

\includegraphics{https://static01.graylady3jvrrxbe.onion/newsgraphics/2020/05/14/meat-processing-covid/1d5517d56c515053314f633f4fbd3a81341c66be/next_arrow.svg}

Dozens of meatpacking plants were temporarily closed in April after the
\href{https://www.nytimes3xbfgragh.onion/2020/04/15/us/coronavirus-south-dakota-meat-plant-refugees.html?searchResultPosition=3}{coronavirus}
\href{https://thefern.org/2020/04/mapping-covid-19-in-meat-and-food-processing-plants/}{infected}
about 25,000 workers and killed more than 90 across the United States.
To prevent more outbreaks, the Centers for Disease Control and
Prevention has
\href{https://www.cdc.gov/coronavirus/2019-ncov/community/organizations/meat-poultry-processing-workers-employers.html}{recommended}
that reopened facilities use masks and glass partitions, and also
\textbf{keep workers at least six feet apart}. But because
shoulder-to-shoulder assembly lines may be necessary if plants want to
ramp up their production speeds again, keeping these guidelines in place
may be difficult over the long run.

\includegraphics{https://static01.graylady3jvrrxbe.onion/newsgraphics/2020/05/14/meat-processing-covid/1d5517d56c515053314f633f4fbd3a81341c66be/next_arrow.svg}

\hypertarget{designed-for-speed-and-efficiency}{%
\subsubsection{Designed for speed and
efficiency}\label{designed-for-speed-and-efficiency}}

As the simplified layout of a typical pork plant in the next slide
shows, human labor is vital to the process. Some areas allow workers to
have more space. But in the fabrication area, where meat is prepared for
retail or wholesale, workers are much closer together. ``The striking
thing if you walk around a meatpacking plant is how they are designed
for cheap labor,'' said James MacDonald, professor of Agricultural
Economics at the University of Maryland.

Holding

pens

Animals

enter

Kill floor

Workers

Carcass

chilling

Fabrication floor

Packing and

refrigeration

Trucks

\includegraphics{https://static01.graylady3jvrrxbe.onion/newsgraphics/2020/05/14/meat-processing-covid/1d5517d56c515053314f633f4fbd3a81341c66be/next_arrow.svg}

Workers take animals to be inspected and stunned.

\includegraphics{https://static01.graylady3jvrrxbe.onion/newsgraphics/2020/05/14/meat-processing-covid/1d5517d56c515053314f633f4fbd3a81341c66be/next_arrow.svg}

\hypertarget{where-the-line-starts}{%
\subsubsection{Where the line starts}\label{where-the-line-starts}}

When pigs reach the plant, they average 280 pounds. Most large plants
typically process about 1,000 pigs per hour, or about 17 per minute.
Across the United States, 0 pigs have been processed since you began
reading.

\includegraphics{https://static01.graylady3jvrrxbe.onion/newsgraphics/2020/05/14/meat-processing-covid/1d5517d56c515053314f633f4fbd3a81341c66be/line-mo-06.png}

\includegraphics{https://static01.graylady3jvrrxbe.onion/newsgraphics/2020/05/14/meat-processing-covid/1d5517d56c515053314f633f4fbd3a81341c66be/next_arrow.svg}

Plant closures and slowing production lines have led to pigs growing too
heavy for workers to handle safely. Thousands of pigs have been
\href{https://www.nytimes3xbfgragh.onion/2020/05/14/business/coronavirus-farmers-killing-pigs.html}{euthanized}
in recent weeks.

\includegraphics{https://static01.graylady3jvrrxbe.onion/newsgraphics/2020/05/14/meat-processing-covid/1d5517d56c515053314f633f4fbd3a81341c66be/next_arrow.svg}

\hypertarget{fewer-workers-more-space}{%
\subsubsection{Fewer workers, more
space}\label{fewer-workers-more-space}}

Workers in the harvesting area stun and slaughter the animals, remove
their innards and heads, inspect the carcasses and send them to the
refrigerated section to chill overnight.

Stunned animals are put on overhead conveyor and bled.

Carcasses are scalded, dehaired and singed.

\includegraphics{https://static01.graylady3jvrrxbe.onion/newsgraphics/2020/05/14/meat-processing-covid/1d5517d56c515053314f633f4fbd3a81341c66be/next_arrow.svg}

Workers remove organs and heads and split carcasses.

\includegraphics{https://static01.graylady3jvrrxbe.onion/newsgraphics/2020/05/14/meat-processing-covid/1d5517d56c515053314f633f4fbd3a81341c66be/next_arrow.svg}

Workers have seconds to perform these wet, messy and physically grueling
tasks on each carcass, which can make wearing a mask challenging.

\includegraphics{https://static01.graylady3jvrrxbe.onion/newsgraphics/2020/05/14/meat-processing-covid/1d5517d56c515053314f633f4fbd3a81341c66be/next_arrow.svg}

\includegraphics{https://static01.graylady3jvrrxbe.onion/newsgraphics/2020/05/14/meat-processing-covid/1d5517d56c515053314f633f4fbd3a81341c66be/line-mo-11.png}

\includegraphics{https://static01.graylady3jvrrxbe.onion/newsgraphics/2020/05/14/meat-processing-covid/1d5517d56c515053314f633f4fbd3a81341c66be/next_arrow.svg}

Use of sharp, heavy machinery keeps workers farther apart, but this
doesn't protect against prolonged exposure to potentially contagious
droplets and aerosols that can accumulate in the air.

Carcasses are inspected and chilled.

\includegraphics{https://static01.graylady3jvrrxbe.onion/newsgraphics/2020/05/14/meat-processing-covid/1d5517d56c515053314f633f4fbd3a81341c66be/next_arrow.svg}

Cold temperatures and powerful ventilation systems designed to protect
meat from contamination can contribute to the virus surviving longer and
traveling farther throughout the plant, said Maria King, an assistant
professor in the department of biological and agricultural engineering
at Texas A\&M University.

Carcasses are divided into primal cuts of shoulder, loin, leg and belly.

\includegraphics{https://static01.graylady3jvrrxbe.onion/newsgraphics/2020/05/14/meat-processing-covid/1d5517d56c515053314f633f4fbd3a81341c66be/next_arrow.svg}

\hypertarget{tighter-quarters-as-line-progresses}{%
\subsubsection{Tighter quarters as line
progresses}\label{tighter-quarters-as-line-progresses}}

About 65 percent of laborers at a plant work on the fabrication floor,
where large carcasses are cut up into smaller parts, deboned, trimmed
and prepared for shipping to restaurants, supermarkets and
further-processing facilities.

Large cuts are deboned, trimmed and cut into smaller portions.

\includegraphics{https://static01.graylady3jvrrxbe.onion/newsgraphics/2020/05/14/meat-processing-covid/1d5517d56c515053314f633f4fbd3a81341c66be/next_arrow.svg}

As a chunk of meat travels down the production line, it becomes smaller
and smaller, and so does the space between workers.

\includegraphics{https://static01.graylady3jvrrxbe.onion/newsgraphics/2020/05/14/meat-processing-covid/1d5517d56c515053314f633f4fbd3a81341c66be/next_arrow.svg}

Before the pandemic, some workers had as little as three feet of space
at the cutting table. Plants try to use every square inch of the
fabrication floor in order to maximize the number of workers and
increase production.

6 ft

\includegraphics{https://static01.graylady3jvrrxbe.onion/newsgraphics/2020/05/14/meat-processing-covid/1d5517d56c515053314f633f4fbd3a81341c66be/next_arrow.svg}

\hypertarget{changes-during-the-pandemic}{%
\subsubsection{Changes during the
pandemic}\label{changes-during-the-pandemic}}

To precisely meet the C.D.C.'s guidelines of spacing workers \textbf{six
feet apart}, about two out of every three workers would need to be
removed in the densest section of the fabrication floor.

\includegraphics{https://static01.graylady3jvrrxbe.onion/newsgraphics/2020/05/14/meat-processing-covid/1d5517d56c515053314f633f4fbd3a81341c66be/line-mo-17.png}

\includegraphics{https://static01.graylady3jvrrxbe.onion/newsgraphics/2020/05/14/meat-processing-covid/1d5517d56c515053314f633f4fbd3a81341c66be/next_arrow.svg}

The C.D.C. also advised plants to adjust airflow to minimize potential
exposure to the coronavirus, but this could mean a big financial
investment for older facilities with less modern ventilation systems.

\includegraphics{https://static01.graylady3jvrrxbe.onion/newsgraphics/2020/05/14/meat-processing-covid/1d5517d56c515053314f633f4fbd3a81341c66be/line-mo-18.png}

\includegraphics{https://static01.graylady3jvrrxbe.onion/newsgraphics/2020/05/14/meat-processing-covid/1d5517d56c515053314f633f4fbd3a81341c66be/next_arrow.svg}

Most plants are taking some steps, to varying degrees. Some are
positioning workers so that they don't directly face each other and
installing plexiglass dividers between them.

\includegraphics{https://static01.graylady3jvrrxbe.onion/newsgraphics/2020/05/14/meat-processing-covid/1d5517d56c515053314f633f4fbd3a81341c66be/line-mo-19.png}

\includegraphics{https://static01.graylady3jvrrxbe.onion/newsgraphics/2020/05/14/meat-processing-covid/1d5517d56c515053314f633f4fbd3a81341c66be/next_arrow.svg}

They are also providing personal protective equipment like masks and
making more sinks and hand sanitizers available. Other measures include
staggering start times and adding shifts to minimize crowding at
entrances, locker rooms and other shared spaces.

Workers package cuts of meat for distribution.

\includegraphics{https://static01.graylady3jvrrxbe.onion/newsgraphics/2020/05/14/meat-processing-covid/1d5517d56c515053314f633f4fbd3a81341c66be/next_arrow.svg}

To reduce labor, some cuts of meat are being trimmed and packaged
differently and more crudely than before.

\includegraphics{https://static01.graylady3jvrrxbe.onion/newsgraphics/2020/05/14/meat-processing-covid/1d5517d56c515053314f633f4fbd3a81341c66be/line-mo-21.png}

\includegraphics{https://static01.graylady3jvrrxbe.onion/newsgraphics/2020/05/14/meat-processing-covid/1d5517d56c515053314f633f4fbd3a81341c66be/next_arrow.svg}

This crisis underscores how the meat industry has failed to automate,
which can be costly and time-consuming. ``I think you'll see a
significant investment in redesigning facilities,'' said Dr. Keith Belk,
Department Head of Animal Science in Colorado State University. ``There
may not be fewer employees, but what they do will be different.''

\includegraphics{https://static01.graylady3jvrrxbe.onion/newsgraphics/2020/05/14/meat-processing-covid/1d5517d56c515053314f633f4fbd3a81341c66be/line-mo-22.png}

\includegraphics{https://static01.graylady3jvrrxbe.onion/newsgraphics/2020/05/14/meat-processing-covid/1d5517d56c515053314f633f4fbd3a81341c66be/next_arrow.svg}

In the United States, about half a million pigs are killed for meat
every day. Because a very small number of plants control a large share
of production in the country, any changes to the assembly line could
have a big impact on prices and availability of meat. Pork production
was down 6 percent for the week ending May 30 from a year ago because of
slowed production.

\includegraphics{https://static01.graylady3jvrrxbe.onion/newsgraphics/2020/05/14/meat-processing-covid/1d5517d56c515053314f633f4fbd3a81341c66be/next_arrow.svg}

Advertisement

Sources: North American Meat Institute; Centers for Disease Control and
Prevention; Pork.org; United Food and Commercial Workers Union; Cattle
Buyers Weekly, Maria King, Texas A\&M University; Keith Belk, Department
Head of Animal Science, Colorado State University; James MacDonald,
professor of Agricultural Economics at the University of Maryland; Food
\& Environment Reporting Network

Read more

\href{https://www.nytimes3xbfgragh.onion/interactive/2020/05/27/magazine/coronavirus-nebraska-unemployment-jobs.html}{}

As Meatpacking Plants Look to Reopen, Some Families are Wary

May 27

\includegraphics{https://static01.graylady3jvrrxbe.onion/images/2020/05/31/magazine/31nebraska-05/31nebraska-05-articleLarge.jpg}

\href{https://www.nytimes3xbfgragh.onion/2020/05/25/business/coronavirus-meatpacking-plants-cases.html}{}

As Meatpacking Plants Reopen, Data About Worker Illness Remains Elusive

May 25

\includegraphics{https://static01.graylady3jvrrxbe.onion/images/2020/05/23/business/23virus-meatdata-1/23virus-meatdata-1-articleLarge.jpg}

\href{https://www.nytimes3xbfgragh.onion/2020/05/14/business/coronavirus-farmers-killing-pigs.html}{}

Meat Plant Closures Mean Pigs Are Gassed or Shot Instead

May 14

\includegraphics{https://static01.graylady3jvrrxbe.onion/images/2020/05/12/business/12virus-hogeuthanize-1/merlin_172417743_6c712e92-ed2b-4a8d-b099-c468a3ba4af5-articleLarge.jpg}
