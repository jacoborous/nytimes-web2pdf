Sections

SEARCH

\protect\hyperlink{site-content}{Skip to
content}\protect\hyperlink{site-index}{Skip to site index}

\href{https://www.nytimes3xbfgragh.onion/section/well}{Well}

\href{https://myaccount.nytimes3xbfgragh.onion/auth/login?response_type=cookie\&client_id=vi}{}

\href{https://www.nytimes3xbfgragh.onion/section/todayspaper}{Today's
Paper}

\href{/section/well}{Well}\textbar{}Could My Symptoms Be Covid-19?

\url{https://nyti.ms/2DCJHGM}

\begin{itemize}
\item
\item
\item
\item
\item
\end{itemize}

\hypertarget{could-my-symptoms-be-covid-19}{%
\section{Could My Symptoms Be
Covid-19?}\label{could-my-symptoms-be-covid-19}}

By \href{https://www.nytimes3xbfgragh.onion/by/tara-parker-pope}{Tara
Parker-Pope} and
\href{https://www.nytimes3xbfgragh.onion/by/mika-grondahl}{Mika
Gröndahl}Aug. 5, 2020

\href{https://www.nytimes3xbfgragh.onion/es/interactive/2020/08/06/espanol/ciencia-y-tecnologia/tengo-covid-19-sintomas.html}{Leer
en español}

\begin{itemize}
\item
\item
\item
\item
\end{itemize}

Congestion or runny nose

Wet cough

Tightness in chest

Feeling out of breath with activity

Nausea and vomiting

Diarrhea and abdominal pain

Chills and body aches

Congestion or runny nose

Wet cough

Tightness in chest

Feeling out of breath with activity

Nausea and vomiting

Chills and body aches

Diarrhea and abdominal pain

Congestion or runny nose

Wet cough

Tightness in chest

Out of breath with activity

Nausea and vomiting

Diarrhea and abdominal pain

Chills and body aches

Headache

Dizziness or impaired thinking

Fever

Eye discomfort

Loss of taste or smell

Sore throat

Congestion or runny nose

Severe shortness of breath at rest

Dry cough

Palpitations, chest pain

Wet cough

Blood clots

Tightness in chest

Severe muscle and joint pain

Feeling out of breath with activity

Rashes

Nausea and vomiting

Blisters on toes and fingers

Chills and body aches

Diarrhea and abdominal pain

Fatigue

Headache

Dizziness or impaired thinking

Eye discomfort

Fever

Congestion or runny nose

Loss of taste or smell

Sore throat

Dry cough

Wet cough

Severe shortness of breath at rest

Tightness in chest

Feeling out of breath with activity

Palpitations, chest pain

Nausea and vomiting

Rashes

Diarrhea and abdominal pain

Blisters on toes and fingers

Severe muscle and joint pain

Chills and body aches

Fatigue

Blood clots

Headache

Dizziness or impaired thinking

Eye discomfort

Fever

Congestion or runny nose

Loss of taste or smell

Sore throat

Dry cough

Wet cough

Shortness of breath at rest

Tightness in chest

Out of breath with activity

Palpitations, chest pain

Nausea and vomiting

Rashes

Diarrhea and abdominal pain

Severe muscle and joint pain

Blisters on toes and fingers

Chills and body aches

Fatigue

Blood clots

Headache

Dizziness or impaired thinking

Eye discomfort

Fever

Congestion or runny nose

Loss of taste and smell

Sore throat

Dry cough

Wet cough

Shortness of breath at rest

Tightness in chest

Out of breath with activity

Palpitations, chest pain

Nausea and vomiting

Rashes

Diarrhea and abdominal pain

Severe muscle and joint pain

Blisters on toes and fingers

Chills and body aches

Fatigue

Blood clots

Headache

Dizziness or impaired thinking

Eye discomfort

Fever

Congestion or runny nose

Loss of taste and smell

Sore throat

Dry cough

Wet cough

Severe shortness of breath at rest

Tightness in chest

Feeling out of breath with activity

Palpitations, chest pain

Nausea and vomiting

Rashes

Diarrhea and abdominal pain

Blisters on toes and fingers

Severe muscle and joint pain

Chills and body aches

Fatigue

Blood clots

Headache

Dizziness or impaired thinking

Fever

Eye discomfort

Loss of taste and smell

Sore throat

Congestion or runny nose

Severe shortness of breath at rest

Dry cough

Palpitations, chest pain

Wet cough

Blood clots

Tightness in chest

Severe muscle and joint pain

Feeling out of breath with activity

Rashes

Nausea and vomiting

Blisters on toes and fingers

Chills and body aches

Diarrhea and abdominal pain

Fatigue

Headache

Dizziness or impaired thinking

Fever

Eye discomfort

Loss of taste and smell

Sore throat

Congestion or runny nose

Severe shortness of breath at rest

Dry cough

Palpitations, chest pain

Wet cough

Blood clots

Tightness in chest

Severe muscle and joint pain

Feeling out of breath with activity

Rashes

Nausea and vomiting

Blisters on toes and fingers

Chills and body aches

Diarrhea and abdominal pain

Fatigue

Headache

Dizziness or impaired thinking

Eye discomfort

Fever

Congestion or runny nose

Loss of taste and smell

Sore throat

Dry cough

Wet cough

Severe shortness of breath at rest

Tightness in chest

Feeling out of breath with activity

Palpitations, chest pain

Nausea and vomiting

Rashes

Diarrhea and abdominal pain

Blisters on toes and fingers

Severe muscle and joint pain

Chills and body aches

Fatigue

Blood clots

Headache

Dizziness or impaired thinking

Eye discomfort

Fever

Congestion or runny nose

Loss of taste and smell

Sore throat

Dry cough

Wet cough

Shortness of breath at rest

Tightness in chest

Out of breath with activity

Palpitations, chest pain

Nausea and vomiting

Rashes

Diarrhea and abdominal pain

Blisters on toes and fingers

Severe muscle and joint pain

Chills and body aches

Fatigue

Blood clots

Anosmia, the \textbf{loss of sense of smell,} is a defining symptom. In
one study, 87 percent of patients lost their sense of smell and 56
percent reported loss of taste.

At first, a stuffy nose and post-nasal drip were not considered
symptoms, but now the C.D.C. says \textbf{congestion} and \textbf{runny
nose} are common signs.

A persistent dry cough is often an early sign of Covid-19, but some
patients have a \textbf{wet cough} or \textbf{sore throat}.

Anosmia, the \textbf{loss of sense of smell,} is a defining symptom. In
one study, 87 percent of patients lost their sense of smell and 56
percent reported loss of taste.

At first, a stuffy nose and post-nasal drip were not considered
symptoms, but now the C.D.C. says \textbf{congestion} and \textbf{runny
nose} are common signs.

A persistent dry cough is often an early sign of Covid-19, but some
patients have a \textbf{wet cough} or \textbf{sore throat}.

Anosmia, the \textbf{loss of sense of smell,} is a defining symptom. In
one study, 87 percent of patients lost their sense of smell and 56
percent reported loss of taste.

At first, a stuffy nose and post-nasal drip were not considered
symptoms, but now the C.D.C. says \textbf{congestion} and \textbf{runny
nose} are common signs.

A persistent dry cough is often an early sign of Covid-19, but some
patients have a \textbf{wet cough} or \textbf{sore throat}.

\textbf{Dizziness} and impaired thinking are more common in older
patients.

Although \textbf{fever} and \textbf{painful headache} are common, many
patients --- up to 60 percent in one study --- never get a fever.

Eye symptoms, including \textbf{eye pain}, \textbf{itching},
\textbf{tearing} and \textbf{redness}, may occur in up to 25 percent of
patients.

Although \textbf{fever} and \textbf{painful headache} are common, many
patients --- up to 60 percent in one study --- never get a fever.

\textbf{Dizziness} and impaired thinking are more common in older
patients.

Eye symptoms, including \textbf{eye pain}, \textbf{itching},
\textbf{tearing} and \textbf{redness}, may occur in up to 25 percent of
patients.

\textbf{Dizziness} and impaired thinking are more common in older
patients.

Although \textbf{fever} and \textbf{painful headache} are common, many
patients --- up to 60 percent in one study --- never get a fever.

Eye symptoms, including \textbf{eye pain}, \textbf{itching},
\textbf{tearing} and \textbf{redness}, may occur in up to 25 percent of
patients.

Feeling \textbf{out of breath} and \textbf{tightness in the chest}
\emph{with activity} is common. If these symptoms persist or progress to
severe shortness of breath, see a doctor.

Chest pain, \textbf{palpitations} and \textbf{heart problems} are less
common, and may be caused by inflammation, low blood pressure or
declining oxygen.

Hospitalized patients may have kidney problems which can cause
\textbf{leg swelling} and \textbf{bloody urine}.

\textbf{Diarrhea}, \textbf{abdominal pain}, \textbf{nausea} and
\textbf{vomiting} are common. In some patients, gastrointestinal
problems are the only symptom.

Feeling \textbf{out of breath} and \textbf{tightness in the chest}
\emph{with activity} is common. If these symptoms persist or progress to
severe shortness of breath, see a doctor.

Chest pain, \textbf{palpitations} and \textbf{heart problems} are less
common, and may be caused by inflammation, low blood pressure or
declining oxygen.

Hospitalized patients may have kidney problems which can cause
\textbf{leg swelling} and \textbf{bloody urine}.

\textbf{Diarrhea}, \textbf{abdominal pain}, \textbf{nausea} and
\textbf{vomiting} are common. In some patients, gastrointestinal
problems are the only symptom.

Feeling \textbf{out of breath} and \textbf{tightness in the chest}
\emph{with activity} is common. If these symptoms persist or progress to
severe shortness of breath, see a doctor.

Chest pain, \textbf{palpitations} and \textbf{heart problems} are less
common, and may be caused by inflammation, low blood pressure or
declining oxygen.

\textbf{Diarrhea}, \textbf{abdominal pain}, \textbf{nausea} and
\textbf{vomiting} are common. In some patients, gastrointestinal
problems are the only symptom.

Hospitalized patients may have kidney problems which can cause
\textbf{leg swelling} and \textbf{bloody urine}.

An immune response may cause \textbf{rashes} on the body. Blood clots
may cause painful red lesions on fingers and toes.

Many patients report \textbf{muscle aches}, \textbf{chills} and
\textbf{fatigue}.

An immune response may cause \textbf{rashes} on the body. Blood clots
may cause painful red lesions on fingers and toes.

Many patients report \textbf{muscle aches}, \textbf{chills} and
\textbf{fatigue}.

An immune response may cause \textbf{rashes} on the body. Blood clots
may cause painful red lesions on fingers and toes.

Many patients report \textbf{muscle aches}, \textbf{chills} and
\textbf{fatigue}.

\textbf{Blood clots} and \textbf{strokes} are rare but life-threatening.

\textbf{Blood clots} and \textbf{strokes} are rare but life-threatening.

\textbf{Blood clots} and \textbf{strokes} are rare but life-threatening.

Prolonged fever (5+ days)

Lethargy, irritability or confusion

Too sick to eat or drink

Pale, patchy and/or blue skin for children with light skin

Trouble breathing or rapid breaths

Racing heart or chest pain

Severe abdominal pain, diarrhea or vomiting

Decreased urination

Prolonged fever (5+ days)

Lethargy, irritability or confusion

Too sick to eat or drink

Pale, patchy and/or blue skin for children with light skin

Trouble breathing or rapid breaths

Racing heart or chest pain

Severe abdominal pain, diarrhea or vomiting

Decreased urination

Prolonged fever (5+ days)

Lethargy, irritability or confusion

Too sick to eat or drink

Pale, patchy and/or blue skin for children with light skin

Trouble breathing or rapid breaths

Racing heart or chest pain

Severe abdominal pain, diarrhea or vomiting

Decreased urination

\begin{itemize}
\item
\item
\item
\item
\item
\item
\item
\item
\item
\item
\item
\end{itemize}

These days, every cough, sneeze or headache makes you wonder: Could it
be Covid-19? Medical experts are viewing Covid-19 as a
\textbf{multi-organ disease} that can affect the body from head to toe
and everywhere in between. Here's a guide to help you understand the
symptoms.

These four symptoms are very common among Covid patients. Unlike flu
symptoms, which typically come on fast, Covid-19 symptoms may emerge
over several days.

Many patients commonly report one or more of these symptoms too. Some
patients have only mild illness, but others begin to feel terrible, with
worsening symptoms and a sense of constant discomfort.

Covid is not just a respiratory illness, and can show up in a number of
unusual ways. These symptoms are less common or rare, but they can also
be signs of Covid.

The \textbf{nose} is
\href{https://www.nature.com/articles/s41591-020-0868-6}{ground zero}
for Covid-19. It's rich in a receptor
\href{https://www.nytimes3xbfgragh.onion/2020/06/01/health/coronavirus-mysteries.html}{called
ACE2}, which the virus uses to get into our cells. As the virus
replicates inside the nose and spreads down the respiratory tract and
into the lungs, patients may develop various respiratory symptoms.

Painful headache is common, but more serious neurological problems are
less common or rare. Mild symptoms include dizziness or feeling
lightheaded. Symptoms needing urgent care include confusion, an
inability to wake, uncoordinated movement or signs of stroke like facial
drooping, numbness or garbled speech.

Some patients develop Covid pneumonia as the virus attacks the lungs.
Sometimes oxygen levels can drop so slowly that the patient doesn't
notice. Short, rapid breathing or severe shortness of breath,
particularly at rest, are signs that require urgent medical attention.

The virus can show up in unusual ways across the body. Strange rashes
--- bumpy, smooth, itchy or innocuous --- have been reported. In rare
cases, the virus inflames joints or damages muscles in the thighs,
shoulders or back, causing severe pain.

The virus also appears to attach to the insides of blood vessels, and in
rare cases causes life-threatening blood clots that travel to the lungs,
heart or brain. In very rare cases, clots can cut off blood flow in the
limbs, requiring amputation. Patients sick enough to visit the hospital
may be given blood thinning medications to prevent or treat blood clots.

Covid-19 typically is mild in children. In very rare cases, it can cause
a severe inflammatory response. Seek emergency care if a child shows any
of these warning signs or symptoms that cause concern.

If you have a symptom that might be Covid-19, doctors say you should
isolate until you can be tested. Most patients will recover on their own
within a few weeks.

It's a good idea
\href{https://www.nytimes3xbfgragh.onion/2020/04/24/well/live/coronavirus-pulse-oximeter-oxygen.html}{to
monitor oxygen levels at home} with a pulse oximeter. Pay close
attention to symptoms during
\href{https://www.nytimes3xbfgragh.onion/2020/04/30/well/live/coronavirus-days-5-through-10.html}{days
five to 10 of the illness}, when oxygen levels may drop to dangerously
low levels.

Seek medical care at any time if you experience trouble breathing, any
concerning symptom or take a turn for the worse.

From a sniffle or cough that feels like allergies to severe body aches
and crippling fatigue, the symptoms of coronavirus can be unpredictable
from head to toe.
\href{https://www.nytimes3xbfgragh.onion/2020/08/05/well/live/coronavirus-covid-symptoms.html}{Read
more} about the many symptoms of Covid-19 and join the conversation.

Notes and sources: This graphic is based on unpublished work by Dr. Mark
A. Perazella, Yale University School of Medicine; Dr. Kenar D. Jhaveri,
Zucker School of Medicine at Hofstra/Northwell; and Dr. Hassan Izzedine,
Peupliers Private Hospital. It is also based on studies published in
\href{https://onlinelibrary.wiley.com/doi/full/10.1111/acem.14009}{Academic
Emergency Medicine} (May 2020);
\href{https://www.acpjournals.org/doi/pdf/10.7326/M20-2428}{Annals of
Internal Medicine} (May 2020);
\href{https://pubmed.ncbi.nlm.nih.gov/32492712/}{Blood} (July 2020);
\href{https://onlinelibrary.wiley.com/doi/10.1111/bjd.19327}{British
Journal of Dermatology} (June 2020);
\href{https://www.ncbi.nlm.nih.gov/pmc/articles/PMC7194555/}{Gastroenterology}
(April 2020);
\href{https://www.hopkinsguides.com/hopkins/view/Johns_Hopkins_ABX_Guide/540747/all/Coronavirus_COVID_19__SARS_CoV_2_}{Hopkins
Guides} (July 2020);
\href{https://jamanetwork.com/journals/jama/fullarticle/2761044}{JAMA}
(Feb 2020);
\href{https://jamanetwork.com/journals/jamacardiology/fullarticle/2763845}{JAMA
Cardiology} (March 2020);
\href{https://jamanetwork.com/journals/jamadermatology/fullarticle/2767775}{JAMA
Dermatology} (June 2020); JAMA Neurology
(\href{https://jamanetwork.com/journals/jamaneurology/fullarticle/2764549}{April
2020} and
\href{https://jamanetwork.com/journals/jamaneurology/fullarticle/2768098}{July
2020});
\href{https://www.jaad.org/article/S0190-9622(20)32126-5/fulltext}{Journal
of the American Academy of Dermatology} (July 2020);
\href{https://www.ncbi.nlm.nih.gov/pmc/articles/PMC7088708/}{Journal of
General Internal Medicine} (May 2020); The Lancet
(\href{https://www.thelancet.com/journals/lancet/article/PIIS0140-6736(20)30937-5/fulltext}{April}
and
\href{https://www.thelancet.com/journals/laneur/article/PIIS1474-4422(20)30221-0/fulltext}{July}
2020); \href{https://www.nature.com/articles/s41591-020-0868-6}{Nature
Medicine} (April 2020);
\href{https://www.nejm.org/doi/pdf/10.1056/NEJMoa2002032}{New England
Journal of Medicine} (February 2020);
\href{https://www.nejm.org/doi/full/10.1056/NEJMc2008597}{NEJM} (June
2020);
\href{https://www.sciencedirect.com/science/article/pii/S016164202030405X?via\%3Dihub}{Ophthalmology}
(July 2020);
\href{https://www.reviewofoptometry.com/news/article/ocular-covid19-symptoms-more-common-than-thought}{Review
of Optometry} (May 2020);
\href{https://www.ahajournals.org/doi/pdf/10.1161/STROKEAHA.120.030335}{Stroke}
(July 2020); and reports and guidance from the
\href{https://www.cdc.gov/coronavirus/2019-ncov/symptoms-testing/symptoms.html}{Centers
for Disease Control and Prevention}, the
\href{https://coronavirus.health.ny.gov/childhood-inflammatory-disease-related-covid-19?gclid=Cj0KCQjwyJn5BRDrARIsADZ9ykHf-lzC09eZFv1lbtLJ0n8TgUM3xN0jt_cBakctwMU7lV__mJyMC3waAmcxEALw_wcB}{New
York Health Department} and the
\href{https://www.who.int/docs/default-source/coronaviruse/who-china-joint-mission-on-covid-19-final-report.pdf}{World
Health Organization}.

Additional work by Josh Williams and Lalena Fisher.

\begin{itemize}
\item
\item
\item
\item
\end{itemize}

Advertisement

\protect\hyperlink{after-bottom}{Continue reading the main story}

\hypertarget{site-index}{%
\subsection{Site Index}\label{site-index}}

\hypertarget{site-information-navigation}{%
\subsection{Site Information
Navigation}\label{site-information-navigation}}

\begin{itemize}
\tightlist
\item
  \href{https://help.nytimes3xbfgragh.onion/hc/en-us/articles/115014792127-Copyright-notice}{©~2020~The
  New York Times Company}
\end{itemize}

\begin{itemize}
\tightlist
\item
  \href{https://www.nytco.com/}{NYTCo}
\item
  \href{https://help.nytimes3xbfgragh.onion/hc/en-us/articles/115015385887-Contact-Us}{Contact
  Us}
\item
  \href{https://www.nytco.com/careers/}{Work with us}
\item
  \href{https://nytmediakit.com/}{Advertise}
\item
  \href{http://www.tbrandstudio.com/}{T Brand Studio}
\item
  \href{https://www.nytimes3xbfgragh.onion/privacy/cookie-policy\#how-do-i-manage-trackers}{Your
  Ad Choices}
\item
  \href{https://www.nytimes3xbfgragh.onion/privacy}{Privacy}
\item
  \href{https://help.nytimes3xbfgragh.onion/hc/en-us/articles/115014893428-Terms-of-service}{Terms
  of Service}
\item
  \href{https://help.nytimes3xbfgragh.onion/hc/en-us/articles/115014893968-Terms-of-sale}{Terms
  of Sale}
\item
  \href{https://spiderbites.nytimes3xbfgragh.onion}{Site Map}
\item
  \href{https://help.nytimes3xbfgragh.onion/hc/en-us}{Help}
\item
  \href{https://www.nytimes3xbfgragh.onion/subscription?campaignId=37WXW}{Subscriptions}
\end{itemize}
