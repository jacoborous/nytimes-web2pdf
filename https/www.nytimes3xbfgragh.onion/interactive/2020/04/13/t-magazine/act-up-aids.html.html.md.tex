\hypertarget{how-act-up-remade-political-organizing-in-america}{%
\section{How ACT UP Remade Political Organizing in
America}\label{how-act-up-remade-political-organizing-in-america}}

April 13, 2020

\begin{itemize}
\item
\item
\item
\item
\end{itemize}

The coalition that fought against AIDS stigma and worked to slow the
plague changed patients' rights and contemporary protest movements.

\href{https://www.nytimes3xbfgragh.onion/interactive/2020/04/13/t-magazine/culture-issue-2020.html}{We
Are Family}

\hypertarget{chapter-3-legends-pioneers-and-survivors}{%
\subparagraph{Chapter 3: Legends Pioneers and
Survivors}\label{chapter-3-legends-pioneers-and-survivors}}

\hypertarget{previous}{%
\subparagraph{Previous}\label{previous}}

\hypertarget{next}{%
\subparagraph{Next}\label{next}}

\hypertarget{how-act-up-remade-political-organizing-in-america-1}{%
\section{How ACT UP Remade Political Organizing in
America}\label{how-act-up-remade-political-organizing-in-america-1}}

\hypertarget{the-activists}{%
\subsection{The Activists}\label{the-activists}}

How ACT UP --- the coalition that fought against AIDS stigma and won
medications that slowed the plague --- forever changed patients' rights,
protests and American political organizing as it's practiced today.

By David France

April 13, 2020

SHARE

SUBMICROSCOPIC INFECTIOUS agents have a way of revealing the worst in
us, and the best. That is the
\href{https://www.nytimes3xbfgragh.onion/2018/04/27/t-magazine/times-journalists-aids-gay-history.html}{story
of the AIDS epidemic} generally, and in particular of
\href{https://actupny.com/}{ACT UP} --- the 33-year-old radical
direct-action group formally and loftily called the AIDS Coalition to
Unleash Power. For nearly a decade in the 1980s and 1990s, ACT UP was a
ubiquitous and unnerving presence, not only in America but in 19
countries worldwide. At its peak, it claimed 148 chapters, and though
its ranks remained relatively small --- numbering perhaps no more than
10,000 --- it terrified and angered much of the population, whether by
halting rush-hour traffic and taking over public spaces with ``die-ins''
and ``kiss-ins,'' at which members laid on the ground or made out with
one another, or by disrupting scientific conferences and political
affairs with foghorns, fake blood and smoke bombs (even, in one
instance, overturning banquet tables).

Generally, the news media didn't think much of their work, branding the
group both vulgar and counterproductive. ``Far from inspiring
sympathy,'' The New York Times
\href{https://www.nytimes3xbfgragh.onion/1989/12/12/opinion/the-storming-of-st-pat-s.html}{said}
of ACT UP in 1989, their methods were ``another reason to reject both
the offensive protesters and their ideas.'' Coverage wasn't much warmer
in some gay newspapers, which were owned by older and more conservative
types who saw them as churlish mobsters, spoiled and rude. ACT UP didn't
entirely disagree. They were, as their motto goes, ``united in anger.''

The T List \textbar{}

Sign up here

And there was much to be angry about. It's hard to remember now how
criminally inept the establishment's reaction was to a disease that
disproportionately affected gay men. It took four years after the
virus's discovery for President Ronald Reagan even to mention AIDS in
public, in 1985, and almost 600 American deaths for The Times to give it
a front-page headline. Congress, barely acknowledging the problem,
attached anti-queer provisions to public health budgets while some
states considered proposals to quarantine those who were H.I.V.
positive. It's also hard to remember that on the night ACT UP was
founded, in 1987 --- six years into the epidemic and 15,000 American
deaths later --- there was still not a single pill on the market to
prescribe. It sure seemed likely that every gay man would perish without
a tear from the rest of the world.

At least that was the nightmare scenario of the playwright
\href{https://www.nytimes3xbfgragh.onion/2020/03/28/nyregion/coronavirus-larry-kramer-aids.html}{Larry
Kramer}. In New York in those days, Kramer had a certain reputation for
screaming about AIDS when everyone else was crying about it. While so
many of us were caring for the dying and mourning the dead, he ranted at
the White House and City Hall and lashed out at his beloved gay
community, whose members he denounced as useless ``sissies'' incapable
of fighting for our own survival. His fury did little to galvanize
people behind him. Instead, it got him removed from the board of Gay
Men's Health Crisis, an AIDS services agency he'd co-founded in 1982,
and frozen out of gay society. But it wasn't long before more people
were ready to listen to him: By the time he was asked to give a talk at
the Lesbian and Gay Community Services Center in 1987 --- he was a
substitute that night;
\href{https://www.nytimes3xbfgragh.onion/topic/person/nora-ephron}{Nora
Ephron} had canceled --- the virus had found a second generation. These
younger queers, many of them in their 20s, had a certain feeling of
entitlement that eluded people who'd come of age before Stonewall, and
they were furious --- shocked, even --- to realize how the health
establishment had forsaken them.

In New York City, their weekly meetings quickly became the epicenter for
all information about the disease, the World Wide Web for rumors and
facts about new drug compounds and warnings about political perils, the
text thread for strategizing our collective survival. Doctors and
researchers from neighborhood hospitals made sojourns to ACT UP's
run-down meeting hall at the center on West 13th Street with news from
the front. Elected officials and community organizers market-tested new
policies there. And ordinary people --- mostly white gay men at first,
but always alongside a strong contingent of lesbians and people of color
--- came looking for ways to respond. They soon diversified as their
numbers grew, from 50 to 350 to 800 and more.

They had little in common beyond what political scientists call a linked
fate: Everyone in those meetings knew someone who was dying or had died,
or else they were marked for death themselves. This brought a ferocious
urgency into the room. With no formal leadership (unlike many civil
rights movements that came before, but much like most of today's protest
groups), ACT UP was the kind of chaotic public square that Hollywood
screenwriters might dream up, an unruly Athenian democracy where ideas
were aired and debated and where people thought --- and screamed, and
cried --- out loud. It was where \emph{anger} was converted to
\emph{action}. Protests were planned for nearly every week, against
targets ranging from City Hall to Wall Street, hospitals to homeless
shelters. Brandishing instantly iconic banners and posters --- like the
arresting one-sheet that warned that ``Silence = Death'' --- the group's
members took over evening news broadcasts and shut down Grand Central
Terminal and St. Patrick's Cathedral in highly photographed
interventions that proved the effectiveness of their methods and their
design-driven messaging. Not even Cosmopolitan magazine was spared, not
even the Mets (the former for telling young women they weren't at risk,
and the latter more opportunistically: showing up at the stadium got
their point --- ``Men, Use Condoms or Beat It!'' --- on national
television). ACT UP was a place to find a sense of empowerment, if not
always power itself.

DESPITE THESE MUTINOUS mobilizations, little changed in those first
years. Just one pill materialized --- called AZT and released in 1987,
it was the most expensive medication ever marketed --- but it did
nothing to extend life. Congressional spending was woefully inadequate.
Pharmaceutical companies lacked the necessary urgency. The nation's
leading Catholic cardinal still inveighed against condom use and gay
rights. Televangelists still welcomed our deaths.

In the face of frustration, ACT UP pivoted brilliantly. Instead of
demanding action from others, they took on the work themselves. Breaking
down the myriad problems inhibiting the response to AIDS, the group spun
off into committees to address them one by one: a women's committee,
because women were excluded from drug trials and disease statistics; a
needle-exchange committee, because no one else was trying to prevent the
spread of disease among IV-drug users; a committee concentrated on
minorities, because cases were growing in those communities; a housing
committee, because so many lost their homes after lengthy
hospitalizations; even a science committee, because the labyrinthine
research institutions lacked a cogent agenda. Each group functioned
autonomously, but all were bound together by ACT UP's belief that,
through deep self-study, its members could bring the pandemic under
control.

And that is indeed what they did. Members went on to write (and help
push through) legislation redirecting federal funding, change the ways
insurance companies function and build housing for the homeless.
Eventually, ACT UP and its spinoffs proved to be full partners in
bringing effective antiretroviral drugs to market in 1996. Along the
way, they revolutionized how pharmaceutical sciences are practiced and
health care is provided. Today, patients of most diagnoses are involved
in research through formal advisory boards, a legacy of ACT UP's
citizen-science activism. Roughly 23 million people are alive today
thanks to the drugs that members helped pioneer; few human beings can
claim such a massive humanitarian impact. They did this not by being
nice --- oh, they were never nice --- but by being right. And by helping
their adversaries find their way as well. Many people the organization
targeted because of their inaction, from Dr. Anthony Fauci at the
National Institutes of Health (now
\href{https://www.nytimes3xbfgragh.onion/2020/03/28/technology/coronavirus-fauci-trump-conspiracy-target.html}{back
in the news}, with Covid-19) to politicians like
\href{https://www.nytimes3xbfgragh.onion/topic/person/jesse-helms}{Jesse
Helms} and drug companies like Merck \& Co., had an eventual change of
heart. Even President Reagan came around, though tepidly, and not until
he was long out of power and more than 90,000 Americans had already
died.

In the meantime, ACT UP had also turbocharged the L.G.B.T.Q. movement in
ways that no one dreamed possible, fueling one of the fastest social
transformations in human history. When the group began, homosexuality
was illegal in half of U.S. states and much of the world. Today,
marriage equality is legal in 28 countries. ``We saw there was a degree
of possibility in life that we didn't expect as a community of queers,''
says the New York artist
\href{http://www.actuporalhistory.org/interviews/video/vazquez.html}{Robert
Vazquez-Pacheco}, who attended his first ACT UP meeting in 1988. ``We
saw that things could change.''

WHAT LESSONS CAN these alumni, most now in their senior years, pass
along to people craving change today? For starters, please remain calm.
(Coincidentally, ``Please Remain Calm'' is the title of early member
\href{https://www.nytimes3xbfgragh.onion/2019/03/21/podcasts/the-daily/hiv-aids-cure.html}{Peter
Staley}'s upcoming memoir.) In a piece of good timing, new books are
impending from several ACT UP veterans, including
\href{https://twitter.com/thegarance}{Garance Franke-Ruta},
\href{https://www.treatmentactiongroup.org/about-us/staff/mark-harrington/}{Mark
Harrington},
\href{https://www.nytimes3xbfgragh.onion/2018/04/16/t-magazine/1980s-protest-movements.html}{Sarah
Schulman} and \href{https://c4aa.org/2016/04/ron-goldberg}{Ron
Goldberg}, who was responsible for ACT UP's most memorable protest
chants. Inside the chaos of this mass-death experience, people found
that being a part of a group steadied their minds. ``ACT UP didn't just
save lives of people with H.I.V., to the extent that it did that,'' says
Franke-Ruta, who joined ACT UP in 1988 and is now a political
journalist. ``It also saved a lot of other people who otherwise would
have been overwhelmed by the times.''

It wasn't always easy to keep levelheaded in ACT UP meetings, which
could devolve into fractious infighting and bitter turf wars. But for
eight years, the group's members managed to come together again and
again, keeping a steady focus on the plague. It's worth noting that ACT
UP never did stop meeting. Though only a handful of chapters remain
active, the core New York group still gathers every Monday night in the
same West Village space, now glamorously renovated. And many original
members continue to fight AIDS in numerous related ways --- as leaders
of powerful agencies like \href{https://www.amfar.org/}{amfAR},
\href{https://www.housingworks.org/}{Housing Works} and
\href{https://www.treatmentactiongroup.org/}{Treatment Action Group} ---
and even unrelated ones: \href{http://www.seanstrub.com/}{Sean Strub},
who went on to found POZ magazine in 1994 and publish his own book,
``Body Counts: A Memoir of Politics, Sex, AIDS and Survival,'' in 2014,
now battles AIDS stigma as the mayor of Milford, Pa. Their work is not
over.

Which leads to a second lesson: Be patient. And to a third: Don't be
intimidated by experts; anybody can become an expert. These principles
have certainly influenced the
\href{https://www.nytimes3xbfgragh.onion/2018/03/28/insider/black-lives-matter-stress.html}{Black
Lives Matter} movement, which has continued to evolve since 2013. Like
ACT UP, it brings together people tackling a waterfront of disparate
issues: everything from voting rights to gender justice, health care,
decarceration and immigration, says the Columbia law professor
\href{https://www.law.columbia.edu/faculty/kendall-thomas}{Kendall
Thomas}, whose ACT UP bona fides date to 1987. ``The movement for black
lives would look very different if its thought leaders --- many of whom
are self-identified black queer people --- hadn't been able to draw on
the example of ACT UP,'' he says. ``Black activists and their allies now
understand that the struggle for black freedom has to make connections
across many different constituencies and concerns that used to be seen
as different and disconnected.''

You can find ACT UP's DNA in other contemporary movements as well. The
ill-fated
\href{https://www.nytimes3xbfgragh.onion/topic/organization/occupy-movement-occupy-wall-street}{Occupy
Wall Street} protests that began in 2011 have regrown into an array of
splinter groups battling income inequality, student debt, the gun lobby
and climate change. How much does the ``Greta Thunberg effect'' owe to
ACT UP in its success in launching a global student movement to fight
carbon-dioxide emissions? Like their forebears, Thunberg and her
followers --- mostly teenagers --- know the scientific literature
thoroughly, leaving detractors little to attack them with besides ad
hominems.

Add to ACT UP's offspring the gun-control activists whose numbers
\href{https://www.nytimes3xbfgragh.onion/2019/02/13/us/parkland-shooting.html}{grew
following the Parkland massacre} in 2018, the growing global movement to
end transgender murders, the
\href{https://www.nytimes3xbfgragh.onion/2017/01/21/us/womens-march.html}{Women's
Marchers} and even those working independently across nations to develop
safety strategies during the
\href{https://www.nytimes3xbfgragh.onion/news-event/coronavirus}{current
Covid-19 pandemic}. Progressive activism owes a debt to these survivors
of the AIDS crisis, these stalwarts from ACT UP, for their legacy has
left us all better equipped to ``challenge and reconfigure'' power
imbalances, as Thomas sees it. ``We have this archive of a political
practice that is available to everybody.'' ACT UP's veterans, aging as
they may be, aren't yet finished unleashing power.

David France is the author of ``How to Survive a Plague'' and the
director of the forthcoming documentary ``Welcome to Chechnya.'' Rosie
Marks is a documentary photographer. Her first book, ``08.14-10.19,''
will be published this year. Photo assistants: Evie Shandilya and Tucker
Wyden.

\href{https://www.nytimes3xbfgragh.onion/2020/04/13/t-magazine/act-up-members.html}{}

\hypertarget{12-people-on-joining-act-up-i-went-to-that-first-meeting-and-never-leftapril-13-2020}{%
\paragraph{12 People on Joining ACT UP: `I Went to That First Meeting
and Never Left'April 13,
2020}\label{12-people-on-joining-act-up-i-went-to-that-first-meeting-and-never-leftapril-13-2020}}

\includegraphics{https://static01.graylady3jvrrxbe.onion/images/2020/04/13/t-magazine/art/13tmag-actup-slide-NNCT/13tmag-actup-slide-NNCT-mediumThreeByTwo210.jpg}
\href{https://www.nytimes3xbfgragh.onion/2018/04/27/t-magazine/times-journalists-aids-gay-history.html}{}

\hypertarget{six-times-journalists-on-the-papers-history-of-covering-aids-and-gay-issuesapril-27-2018}{%
\paragraph{Six Times Journalists on the Paper's History of Covering AIDS
and Gay IssuesApril 27,
2018}\label{six-times-journalists-on-the-papers-history-of-covering-aids-and-gay-issuesapril-27-2018}}

\includegraphics{https://static01.graylady3jvrrxbe.onion/images/2018/04/27/t-magazine/mini-essays-slide-5TZT/mini-essays-slide-5TZT-mediumThreeByTwo210-v2.jpg}
\href{https://www.nytimes3xbfgragh.onion/2018/04/13/t-magazine/nyt-writers-80s-coverage.html}{}

\hypertarget{5-new-york-times-writers-on-what-they-got-right-and-wrong-in-the-early-80sapril-13-2018}{%
\paragraph{5 New York Times Writers on What They Got Right and Wrong in
the Early '80sApril 13,
2018}\label{5-new-york-times-writers-on-what-they-got-right-and-wrong-in-the-early-80sapril-13-2018}}

\includegraphics{https://static01.graylady3jvrrxbe.onion/images/2018/04/04/t-magazine/04tmag-reconsiderations-slide-T3RJ-copy/04tmag-reconsiderations-slide-T3RJ-copy-mediumThreeByTwo210-v2.jpg}
\href{https://www.nytimes3xbfgragh.onion/interactive/2018/04/17/t-magazine/aids-epidemic-deaths-new-york.html}{}

\hypertarget{those-we-lost-to-the-aids-epidemicapril-17-2018}{%
\paragraph{Those We Lost to the AIDS EpidemicApril 17,
2018}\label{those-we-lost-to-the-aids-epidemicapril-17-2018}}

\includegraphics{https://static01.graylady3jvrrxbe.onion/images/2018/04/17/t-magazine/17tmag-lostpromo/17tmag-lostpromo-mediumThreeByTwo210.jpg}

\hypertarget{we-are-family-1}{%
\subsubsection{We Are Family}\label{we-are-family-1}}

\hypertarget{chapter-1-heirs-and-alumni}{%
\paragraph{Chapter 1: Heirs and
Alumni}\label{chapter-1-heirs-and-alumni}}

\href{/interactive/2020/04/13/t-magazine/black-art-galleries.html}{}

\hypertarget{the-artists}{%
\subparagraph{The Artists}\label{the-artists}}

\href{/interactive/2020/04/13/t-magazine/italian-fashion-design-houses.html}{}

\hypertarget{the-dynasties}{%
\subparagraph{The Dynasties}\label{the-dynasties}}

\href{/interactive/2020/04/13/t-magazine/gordon-parks.html}{}

\hypertarget{the-directors}{%
\subparagraph{The Directors}\label{the-directors}}

\href{/interactive/2020/04/13/t-magazine/enrique-olvera-chef.html}{}

\hypertarget{the-disciples}{%
\subparagraph{The Disciples}\label{the-disciples}}

\href{/interactive/2020/04/13/t-magazine/royal-academy-antwerp.html}{}

\hypertarget{the-graduates}{%
\subparagraph{The Graduates}\label{the-graduates}}

\hypertarget{chapter-2-reunions-and-reconsiderations}{%
\paragraph{Chapter 2: Reunions and
Reconsiderations}\label{chapter-2-reunions-and-reconsiderations}}

\href{/interactive/2020/04/13/t-magazine/ninth-street-greenwich-village-neighbors.html}{}

\hypertarget{the-neighbors}{%
\subparagraph{The Neighbors}\label{the-neighbors}}

\href{/interactive/2020/04/13/t-magazine/omen-restaurant-nyc.html}{}

\hypertarget{the-regulars}{%
\subparagraph{The Regulars}\label{the-regulars}}

\href{/interactive/2020/04/13/t-magazine/hair-musical-broadway.html}{}

\hypertarget{hair-1967}{%
\subparagraph{Hair (1967)}\label{hair-1967}}

\href{/interactive/2020/04/13/t-magazine/sweeney-todd-revival.html}{}

\hypertarget{sweeney-todd-2005-revival}{%
\subparagraph{Sweeney Todd (2005
Revival)}\label{sweeney-todd-2005-revival}}

\href{/interactive/2020/04/13/t-magazine/daughters-of-the-dust.html}{}

\hypertarget{daughters-of-the-dust-1991}{%
\subparagraph{Daughters of the Dust
(1991)}\label{daughters-of-the-dust-1991}}

\hypertarget{chapter-3-legends-pioneers-and-survivors-1}{%
\paragraph{Chapter 3: Legends Pioneers and
Survivors}\label{chapter-3-legends-pioneers-and-survivors-1}}

\href{/interactive/2020/04/13/t-magazine/butch-stud-lesbian.html}{}

\hypertarget{the-renegades}{%
\subparagraph{The Renegades}\label{the-renegades}}

\href{/interactive/2020/04/13/t-magazine/act-up-aids.html}{}

\hypertarget{the-activists-1}{%
\subparagraph{The Activists}\label{the-activists-1}}

\href{/interactive/2020/04/13/t-magazine/artist-recluse.html}{}

\hypertarget{the-shadows}{%
\subparagraph{The Shadows}\label{the-shadows}}

\href{/interactive/2020/04/13/t-magazine/black-actresses-bassett-berry-blige-henson-whitfield-elise.html}{}

\hypertarget{the-veterans}{%
\subparagraph{The Veterans}\label{the-veterans}}

\hypertarget{chapter-4-the-new-guard}{%
\paragraph{Chapter 4: The New Guard}\label{chapter-4-the-new-guard}}

\href{/interactive/2020/04/13/t-magazine/asian-american-fashion-designers.html}{}

\hypertarget{the-designers}{%
\subparagraph{The Designers}\label{the-designers}}

\href{13tmag-beauties.html}{}

\hypertarget{the-beauties}{%
\subparagraph{The Beauties}\label{the-beauties}}

\href{/interactive/2020/04/13/t-magazine/nyc-downtown-nightlife-party-scene.html}{}

\hypertarget{the-scenemakers}{%
\subparagraph{The Scenemakers}\label{the-scenemakers}}

\href{/interactive/2020/04/13/t-magazine/maria-cornejo-olivier-rousteing-telfar-clemens-alessandro-michele.html\#olivier-rousteing-and-co}{}

\hypertarget{olivier-rousteing-and-co}{%
\subparagraph{Olivier Rousteing and
Co.}\label{olivier-rousteing-and-co}}

\href{/interactive/2020/04/13/t-magazine/maria-cornejo-olivier-rousteing-telfar-clemens-alessandro-michele.html\#maria-cornejo-and-co}{}

\hypertarget{maria-cornejo-and-co}{%
\subparagraph{Maria Cornejo and Co.}\label{maria-cornejo-and-co}}

\href{/interactive/2020/04/13/t-magazine/maria-cornejo-olivier-rousteing-telfar-clemens-alessandro-michele.html\#telfar-clemens-and-co}{}

\hypertarget{telfar-clemens-and-co}{%
\subparagraph{Telfar Clemens and Co.}\label{telfar-clemens-and-co}}

\href{/interactive/2020/04/13/t-magazine/maria-cornejo-olivier-rousteing-telfar-clemens-alessandro-michele.html\#alessandro-michele-and-co}{}

\hypertarget{alessandro-michele-and-co}{%
\subparagraph{Alessandro Michele and
Co.}\label{alessandro-michele-and-co}}

\href{/interactive/2020/04/13/t-magazine/foreign-correspondents.html}{}

\hypertarget{the-journalists}{%
\subparagraph{The Journalists}\label{the-journalists}}

\begin{itemize}
\item
\item
\item
\item
\end{itemize}

Advertisement

\protect\hyperlink{after-bottom}{Continue reading the main story}

\hypertarget{site-index}{%
\subsection{Site Index}\label{site-index}}

\hypertarget{site-information-navigation}{%
\subsection{Site Information
Navigation}\label{site-information-navigation}}

\begin{itemize}
\tightlist
\item
  \href{https://help.nytimes3xbfgragh.onion/hc/en-us/articles/115014792127-Copyright-notice}{©~2020~The
  New York Times Company}
\end{itemize}

\begin{itemize}
\tightlist
\item
  \href{https://www.nytco.com/}{NYTCo}
\item
  \href{https://help.nytimes3xbfgragh.onion/hc/en-us/articles/115015385887-Contact-Us}{Contact
  Us}
\item
  \href{https://www.nytco.com/careers/}{Work with us}
\item
  \href{https://nytmediakit.com/}{Advertise}
\item
  \href{http://www.tbrandstudio.com/}{T Brand Studio}
\item
  \href{https://www.nytimes3xbfgragh.onion/privacy/cookie-policy\#how-do-i-manage-trackers}{Your
  Ad Choices}
\item
  \href{https://www.nytimes3xbfgragh.onion/privacy}{Privacy}
\item
  \href{https://help.nytimes3xbfgragh.onion/hc/en-us/articles/115014893428-Terms-of-service}{Terms
  of Service}
\item
  \href{https://help.nytimes3xbfgragh.onion/hc/en-us/articles/115014893968-Terms-of-sale}{Terms
  of Sale}
\item
  \href{https://spiderbites.nytimes3xbfgragh.onion}{Site Map}
\item
  \href{https://help.nytimes3xbfgragh.onion/hc/en-us}{Help}
\item
  \href{https://www.nytimes3xbfgragh.onion/subscription?campaignId=37WXW}{Subscriptions}
\end{itemize}
