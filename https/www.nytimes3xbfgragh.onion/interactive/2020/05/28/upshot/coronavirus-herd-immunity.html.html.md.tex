Sections

SEARCH

\protect\hyperlink{site-content}{Skip to
content}\protect\hyperlink{site-index}{Skip to site index}

\href{https://www.nytimes3xbfgragh.onion/section/upshot}{The Upshot}

\href{https://myaccount.nytimes3xbfgragh.onion/auth/login?response_type=cookie\&client_id=vi}{}

\href{https://www.nytimes3xbfgragh.onion/section/todayspaper}{Today's
Paper}

\href{/section/upshot}{The Upshot}\textbar{}The World Is Still Far From
Herd Immunity for Coronavirus

\url{https://nyti.ms/3ejjryq}

\begin{itemize}
\item
\item
\item
\item
\item
\item
\end{itemize}

\hypertarget{the-coronavirus-outbreak}{%
\subsubsection{\texorpdfstring{\href{https://www.nytimes3xbfgragh.onion/news-event/coronavirus?name=styln-coronavirus-national\&region=TOP_BANNER\&variant=undefined\&block=storyline_menu_recirc\&action=click\&pgtype=Interactive\&impression_id=82050880-e39d-11ea-8096-716e66d3d197}{The
Coronavirus
Outbreak}}{The Coronavirus Outbreak}}\label{the-coronavirus-outbreak}}

\begin{itemize}
\tightlist
\item
  live\href{https://www.nytimes3xbfgragh.onion/2020/08/21/world/covid-19-coronavirus.html?name=styln-coronavirus-national\&region=TOP_BANNER\&variant=undefined\&block=storyline_menu_recirc\&action=click\&pgtype=Interactive\&impression_id=82050881-e39d-11ea-8096-716e66d3d197}{Latest
  Updates}
\item
  \href{https://www.nytimes3xbfgragh.onion/interactive/2020/us/coronavirus-us-cases.html?name=styln-coronavirus-national\&region=TOP_BANNER\&variant=undefined\&block=storyline_menu_recirc\&action=click\&pgtype=Interactive\&impression_id=82052f90-e39d-11ea-8096-716e66d3d197}{Maps
  and Cases}
\item
  \href{https://www.nytimes3xbfgragh.onion/interactive/2020/science/coronavirus-vaccine-tracker.html?name=styln-coronavirus-national\&region=TOP_BANNER\&variant=undefined\&block=storyline_menu_recirc\&action=click\&pgtype=Interactive\&impression_id=82052f91-e39d-11ea-8096-716e66d3d197}{Vaccine
  Tracker}
\item
  \href{https://www.nytimes3xbfgragh.onion/2020/08/19/us/colleges-closing-covid.html?name=styln-coronavirus-national\&region=TOP_BANNER\&variant=undefined\&block=storyline_menu_recirc\&action=click\&pgtype=Interactive\&impression_id=82052f92-e39d-11ea-8096-716e66d3d197}{Colleges
  Closing}
\item
  \href{https://www.nytimes3xbfgragh.onion/live/2020/08/20/business/stock-market-today-coronavirus?name=styln-coronavirus-national\&region=TOP_BANNER\&variant=undefined\&block=storyline_menu_recirc\&action=click\&pgtype=Interactive\&impression_id=82052f93-e39d-11ea-8096-716e66d3d197}{Economy}
\end{itemize}

Advertisement

\protect\hyperlink{after-top}{Continue reading the main story}

\hypertarget{comments}{%
\subsection{\texorpdfstring{\protect\hyperlink{commentsContainer}{Comments}}{Comments}}\label{comments}}

\href{}{The World Is Still Far From Herd Immunity for
Coronavirus}\href{}{Skip to Comments}

The comments section is closed. To submit a letter to the editor for
publication, write to
\href{mailto:letters@NYTimes.com}{\nolinkurl{letters@NYTimes.com}}.

Upshot

\hypertarget{the-world-is-still-far-from-herd-immunity-for-coronavirus}{%
\section{The World Is Still Far From Herd Immunity for
Coronavirus}\label{the-world-is-still-far-from-herd-immunity-for-coronavirus}}

By \href{https://www.nytimes3xbfgragh.onion/by/nadja-popovich}{Nadja
Popovich} and
\href{https://www.nytimes3xbfgragh.onion/by/margot-sanger-katz}{Margot
Sanger-Katz}May 28, 2020

\begin{itemize}
\item
\item
\item
\item
\item
  \emph{876}
\end{itemize}

The coronavirus still has a long way to go. That's the message from a
crop of new studies across the world that are trying to quantify how
many people have been infected.

Official case counts often substantially underestimate the number of
coronavirus infections. But in new studies that test the population more
broadly, the percentage of people who have been infected so far is still
in the single digits. The numbers are a fraction of the threshold known
as herd immunity, at which the virus can no longer spread widely. The
precise herd immunity threshold for the novel coronavirus is not yet
clear; but several experts said they believed it would be higher than 60
percent.

\hypertarget{herd-immunity-estimate}{%
\subsection{Herd immunity estimate}\label{herd-immunity-estimate}}

At least 60\% of population

\hypertarget{new-york-citymay-2}{%
\subsection{New York CityMay 2}\label{new-york-citymay-2}}

19.9\% have antibodies May 2

\hypertarget{londonmay-21}{%
\subsection{LondonMay 21}\label{londonmay-21}}

17.5\% have antibodies May 21

\hypertarget{madridmay-13}{%
\subsection{MadridMay 13}\label{madridmay-13}}

11.3\% have antibodies May 13

\hypertarget{wuhan-returning-workersapril-20}{%
\subsection{Wuhan (returning workers)April
20}\label{wuhan-returning-workersapril-20}}

10\% have antibodies April 20

\hypertarget{bostonmay-15}{%
\subsection{BostonMay 15}\label{bostonmay-15}}

9.9\% have antibodies May 15

\hypertarget{stockholm-regionmay-20}{%
\subsection{Stockholm regionMay 20}\label{stockholm-regionmay-20}}

7.3\% have antibodies May 20

\hypertarget{barcelonamay-13}{%
\subsection{BarcelonaMay 13}\label{barcelonamay-13}}

7.1\% have antibodies May 13

Note: Studies represent best current estimates, but are inexact and may
overestimate immunity where coronavirus infections are low. Reported
dates reflect when study results were publicly released. The study from
Wuhan, China, evaluated immunity only among people returning to work,
not in the general population. Broader estimates from the city are
unavailable. Sources:
\href{https://www.governor.ny.gov/news/amid-ongoing-covid-19-pandemic-governor-cuomo-announces-results-completed-antibody-testing}{New
York State};
\href{https://www.gov.uk/government/publications/national-covid-19-surveillance-reports/sero-surveillance-of-covid-19}{Public
Health England};
\href{https://www.lamoncloa.gob.es/serviciosdeprensa/notasprensa/sanidad14/Documents/2020/130520-ENE-COVID_Informe1.pdf}{Carlos
III Health Institute};
\href{https://onlinelibrary.wiley.com/doi/10.1002/jmv.25904}{Wu et al.,
Journal of Medical Virology};
\href{https://www.boston.gov/news/results-released-antibody-and-covid-19-testing-boston-residents}{City
of Boston};
\href{https://www.folkhalsomyndigheten.se/nyheter-och-press/nyhetsarkiv/2020/maj/forsta-resultaten-fran-pagaende-undersokning-av-antikroppar-for-covid-19-virus/}{The
Public Health Agency of Sweden}

Even in some of the hardest-hit cities in the world, the studies
suggest, the vast majority of people still remain vulnerable to the
virus.

Some countries ---
\href{https://www.nature.com/articles/d41586-020-01098-x}{notably
Sweden}, and
\href{https://www.nytimes3xbfgragh.onion/2020/03/13/world/europe/coronavirus-britain-boris-johnson.html}{briefly
Britain} --- have experimented with limited lockdowns in an effort to
build up immunity in their populations. But even in these places, recent
studies indicate that no more than 7 to 17 percent of people have been
infected so far. In New York City, which has had the largest coronavirus
outbreak in the United States, around 20 percent of the city's residents
have been infected by the virus as of early May,
\href{https://www.governor.ny.gov/news/amid-ongoing-covid-19-pandemic-governor-cuomo-announces-results-completed-antibody-testing}{according
to
a}\href{https://www.governor.ny.gov/news/amid-ongoing-covid-19-pandemic-governor-cuomo-announces-results-completed-antibody-testing}{}\href{https://www.governor.ny.gov/news/amid-ongoing-covid-19-pandemic-governor-cuomo-announces-results-completed-antibody-testing}{survey}
of people in grocery stores and community centers released by the
governor's office.

Similar surveys
\href{https://www.bloomberg.com/news/articles/2020-04-18/china-tests-thousands-to-calculate-true-spread-of-coronavirus}{are
underway} in China, where the coronavirus first emerged, but results
have not yet been reported. A study from
\href{https://onlinelibrary.wiley.com/doi/epdf/10.1002/jmv.25904}{a
single hospital in the city of Wuhan} found that about 10 percent of
people seeking to go back to work had been infected with the virus.

Viewed together, the studies show herd immunity protection is unlikely
to be reached ``any time soon,'' said Michael Mina, an epidemiologist at
the Harvard T.H. Chan School of Public Health.

The herd immunity threshold for this new disease is still uncertain, but
many epidemiologists believe it will be reached when between 60 percent
and 80 percent of the population has been infected and develops
resistance. A lower level of immunity in the population can slow the
spread of a disease somewhat, but the herd immunity number represents
the point where infections are substantially less likely to turn into
large outbreaks.

``We don't have a good way to safely build it up, to be honest, not in
the short term,'' Dr. Mina said. ``Unless we're going to let the virus
run rampant again --- but I think society has decided that is not an
approach available to us.''

The new studies look for antibodies in people's blood, proteins produced
by the immune system that indicate a past infection. An advantage of
this test is that it can capture people who may have been asymptomatic
and didn't know they were sick. A disadvantage is that the tests are
sometimes wrong --- and several studies, including
\href{https://www.sfgate.com/news/editorspicks/article/Santa-Clara-antibody-study-revised-Stanford-death-15263047.php}{a
notable one in California}, have been criticized for not accounting for
the possibility of inaccurate results or for not representing the whole
population.

Studies that use these tests to examine a cross section of a population,
often called serology surveys, are being undertaken around the country
and the world.

These studies are far from perfect, said Carl Bergstrom, a professor of
biology at the University of Washington. But in aggregate, he said, they
give a better sense of how far the coronavirus has truly spread --- and
its potential for spreading further.

The herd immunity threshold may differ from place to place, depending on
factors like density and social interaction, he said. But, on average,
experts say it will require at least 60 percent immunity in the
population. If the disease spreads more easily than is currently
believed, the number could be higher. If there is a lot of variation in
people's likelihood of becoming infected when they are exposed, that
could push the number down.

All estimates of herd immunity assume that a past infection will protect
people from becoming sick a second time. There is suggestive evidence
that people do achieve immunity to the coronavirus, but it is not yet
certain whether that is true in all cases; how robust the immunity may
be; or how long it will last.

Dr. Mina of Harvard suggested thinking about population immunity as a
firebreak, slowing the spread of the disease.

If you are infected with the virus and walk into a room where everyone
is susceptible to it, he said, you might infect two or three other
people on average.

``On the other hand, if you go in and three out of four people are
already immune, then on average you will infect one person or fewer in
that room,'' he said. That person in turn would be able to infect fewer
new people, too. And that makes it much less likely that a large
outbreak can bloom.

Even with herd immunity, some people will still get sick. ``Your own
risk, if exposed, is the same,'' said Gypsyamber D'Souza, a professor of
epidemiology at Johns Hopkins University. ``You just become much less
likely to be exposed.''

Diseases like measles and chickenpox, once very common among children,
are now extremely rare in the United States because vaccines have helped
build enough herd immunity to contain outbreaks.

We don't have a vaccine for the coronavirus, so getting to herd immunity
without a new and more effective treatment could mean many more
infections and many more deaths.

If you assume that herd protection could be achieved when 60 percent of
the population becomes resistant to the virus, that means New York City
is only one-third of the way there. And, so far,
\href{https://www.nytimes3xbfgragh.onion/interactive/2020/us/coronavirus-us-cases.html}{nearly
250}\href{https://www.nytimes3xbfgragh.onion/interactive/2020/us/coronavirus-us-cases.html}{of
every 100,000 city residents has died}. New York City still has millions
of residents vulnerable to catching and spreading this disease, and tens
of thousands more who are at risk of dying.

``Would someone advise that people go through something like what New
York went through?'' said Natalie Dean, an assistant professor of
biostatistics at the University of Florida. ``There's a lot of people
who talk about this managed infection of young people, but it just feels
like hubris to think you can manage this virus. It's very hard to
manage.''

Infections have not been evenly distributed throughout the population,
with low-income and minority communities in the United States bearing a
greater burden. On Thursday, Gov. Andrew Cuomo announced that antibody
testing showed that some neighborhoods in the Bronx and Brooklyn had
double the infection rate of New York City in general. Those areas are
already approaching the herd immunity threshold, when new outbreaks
become less likely. But because they are not isolated from the city at
large, where immunity rates are much lower, residents are still at risk.

In other cities, serology surveys are showing much smaller shares of
people with antibodies. The quality of these studies is somewhat varied,
either because the samples weren't random or because the tests were not
accurate enough. But the range of studies shows that most places would
have to see 10 or more times as many illnesses --- and possibly, deaths
--- to reach the point where an outbreak would not be able to take off.

The serology studies can also help scientists determine how deadly the
virus really is. Currently, estimates for what's called the infection
fatality rate are rough.
\href{https://www.nytimes3xbfgragh.onion/interactive/2020/03/07/upshot/how-deadly-is-coronavirus-what-we-know.html}{To
calculate them precisely}, it's important to know how many people in a
place died from the virus versus how many were infected. Official case
rates, which rely on testing, undercount the true extent of infections
in the population. Serology helps us see the true footprint of the
outbreak.

In New York City, where 20 percent of people were infected with the
virus by May 2, according to antibody testing, and where more than
18,000 had died by then, the infection fatality rate appears to be
around 1 percent.

For comparison, the infection fatality rate for influenza is estimated
at 0.1 percent to 0.2 percent. But the way the government estimates flu
cases every year is less precise than using serology tests and tends to
undercount the number of infections, skewing the fatality number higher.

But even if the fatality rates were identical, Covid-19 would be a much
more dangerous disease than influenza. It has to do with the number of
people who are at risk of getting sick and dying as the disease spreads.

With the flu, only about half the population is at risk of getting sick
in a given flu season. Many people have some immunity already, either
because they have been sick with a similar strain of flu, or because
they got a flu shot that was a good match for the version of the virus
they encountered that year.

That number isn't high enough to
\href{https://www.sciencedirect.com/science/article/pii/S0264410X18306571}{fully
reach herd immunity} --- and the flu still circulates every year. But
there are benefits to partial immunity in the population: Only a
fraction of adults are at risk of catching the flu in a normal year, and
they can spread it less quickly, too. That means that the number of
people at risk of dying is also much lower.

Covid-19, unlike influenza, is a brand-new disease. Before this year, no
one in the world had any immunity to it at all. And that means that,
even if infection fatality rates were similar, it has the potential to
kill many more people. One percent of a large number is bigger than 1
percent of a smaller number.

``There aren't 328 million Americans who are susceptible to the flu
every fall at the beginning of the flu season,'' said Andrew Noymer, an
associate professor of public health at the University of California,
Irvine. ``But there are 328 million Americans who were susceptible to
this when this started.''

Additional research by Anna Joyce.

Read 876 Comments

\begin{itemize}
\item
\item
\item
\item
\end{itemize}

Advertisement

\protect\hyperlink{after-bottom}{Continue reading the main story}

\hypertarget{site-index}{%
\subsection{Site Index}\label{site-index}}

\hypertarget{site-information-navigation}{%
\subsection{Site Information
Navigation}\label{site-information-navigation}}

\begin{itemize}
\tightlist
\item
  \href{https://help.nytimes3xbfgragh.onion/hc/en-us/articles/115014792127-Copyright-notice}{©~2020~The
  New York Times Company}
\end{itemize}

\begin{itemize}
\tightlist
\item
  \href{https://www.nytco.com/}{NYTCo}
\item
  \href{https://help.nytimes3xbfgragh.onion/hc/en-us/articles/115015385887-Contact-Us}{Contact
  Us}
\item
  \href{https://www.nytco.com/careers/}{Work with us}
\item
  \href{https://nytmediakit.com/}{Advertise}
\item
  \href{http://www.tbrandstudio.com/}{T Brand Studio}
\item
  \href{https://www.nytimes3xbfgragh.onion/privacy/cookie-policy\#how-do-i-manage-trackers}{Your
  Ad Choices}
\item
  \href{https://www.nytimes3xbfgragh.onion/privacy}{Privacy}
\item
  \href{https://help.nytimes3xbfgragh.onion/hc/en-us/articles/115014893428-Terms-of-service}{Terms
  of Service}
\item
  \href{https://help.nytimes3xbfgragh.onion/hc/en-us/articles/115014893968-Terms-of-sale}{Terms
  of Sale}
\item
  \href{https://spiderbites.nytimes3xbfgragh.onion}{Site Map}
\item
  \href{https://help.nytimes3xbfgragh.onion/hc/en-us}{Help}
\item
  \href{https://www.nytimes3xbfgragh.onion/subscription?campaignId=37WXW}{Subscriptions}
\end{itemize}
