Sections

SEARCH

\protect\hyperlink{site-content}{Skip to
content}\protect\hyperlink{site-index}{Skip to site index}

\href{https://www.nytimes3xbfgragh.onion/section/opinion/sunday}{Sunday
Review}

\href{https://myaccount.nytimes3xbfgragh.onion/auth/login?response_type=cookie\&client_id=vi}{}

\href{https://www.nytimes3xbfgragh.onion/section/todayspaper}{Today's
Paper}

\href{/section/opinion/sunday}{Sunday Review}\textbar{}Is It Safer to
Visit a Coffee Shop or a Gym?

\url{https://nyti.ms/2WtKAY4}

\begin{itemize}
\item
\item
\item
\item
\item
\item
\end{itemize}

Advertisement

\protect\hyperlink{after-top}{Continue reading the main story}

\hypertarget{comments}{%
\subsection{\texorpdfstring{\protect\hyperlink{commentsContainer}{Comments}}{Comments}}\label{comments}}

\href{}{Is It Safer to Visit a Coffee Shop or a Gym?}\href{}{Skip to
Comments}

The comments section is closed. To submit a letter to the editor for
publication, write to
\href{mailto:letters@NYTimes.com}{\nolinkurl{letters@NYTimes.com}}.

\href{/section/opinion}{Opinion}

\hypertarget{is-it-safer-to-visit-a-coffee-shop-or-a-gym}{%
\section{Is It Safer to Visit a Coffee Shop or a
Gym?}\label{is-it-safer-to-visit-a-coffee-shop-or-a-gym}}

By Katherine Baicker, Oeindrila Dube, Sendhil Mullainathan, Devin Pope
and Gus WezerekMay 6, 2020

\begin{itemize}
\item
\item
\item
\item
\item
  \emph{531}
\end{itemize}

As states begin to reopen, Americans are looking at any trip outside
through the lens of contagion. Is it safe to go back to Starbucks? What
about the gym? Nail salons are out of the question, right?

The country faces an ugly trade-off. Keep the economy closed and prolong
the economic misery. Or open up the economy and risk a resurgence of
Covid-19, undoing the gains earned through weeks of social isolation.

We believe there's another option.

Using anonymized cellphone location data from April 2019, we can measure
how crowded businesses get. Each bubble on the chart represents a
business, like Walmart or Waffle House. Larger bubbles have more
locations and customers.

\textbf{Fast-food restaurants} tend to be small and busy, with a high
number of weekly visits per square foot.

\textbf{Gyms} have fewer visitors per square foot, but visitors linger,
increasing the chance they'll interact and spread the virus.

\textbf{Sit-down restaurants} and bars should take extra care when
reopening. These venues draw large crowds with long average stays.

Some businesses, like some people, are ``super-spreaders.'' Through the
lens of contagion, a yoga class, a busy corner store or a crowded
neighborhood bar may look a lot like a wet market in China.

Cellphone data can't tell us everything. For example, businesses in
low-income neighborhoods with fewer smartphones may appear to have less
foot traffic. We looked into this, and to date, we have not found any
appreciable bias in the measures we are using.

The anonymized location pings also don't give us any insight into how
customers interacted or how many surfaces they touched. And it's tricky
to determine whether people were inside a building or moving around
outdoors, where air can move freely, and infection risk may be lower.

To overcome some of these limitations, we asked people to rate, on a
scale of 1 to 10, how often they interacted with people or touched
shared surfaces at various businesses, as well as how much activity in
different sectors occurs indoors.

These numbers help us flag risky industries, like beauty and nail
salons, that our other metrics didn't. These businesses should be
particularly attentive to maintaining social-distancing measures.

The variation in risk between different types of businesses was
surprising. People spend twice as much time at electronics stores as
they do at lawn and garden stores. A display of new phones and gadgets
is an invitation to mill around; you don't linger over fertilizer.
Similarly, we found that people spend nearly three times as much time
searching through the racks at a Salvation Army as they do scanning the
shelves at a Dollar General.

Another reason for differences is how concentrated people are: The same
number of customers spaced out evenly over the day poses less risk than
if they all arrive in a few short windows of time.

Even within a sector, there is tremendous variation. Consider two
similar restaurants: Denny's and the Original Pancake House. Both serve
a similar number of customers every week, who stay for a similar length
of time. But customers at the Original Pancake House are far more
concentrated (at breakfast, of course), producing a far higher risk of
customers getting crowded into the same space at the same time.

The existence of super-spreader businesses might seem like bad news. In
fact, it means that most of the disease-spreading risk generated by the
economy is concentrated in a small portion of it -- which means that we
can resume a lot of economic activity with minimal risk.

Many governors are considering contagion risk as a factor in determining
which businesses to reopen first. Gov. Gavin Newsom of California, for
example, has called
for\href{https://www.nytimes3xbfgragh.onion/reuters/2020/05/04/us/04reuters-health-coronavirus-california.html}{reopening
``low risk'' stores} such as those selling toys, books, sporting goods
and flowers. Indeed, in our data, florists are among the lowest risk.
But toy stores, bookstores and sporting goods stores are in the top
quartile of risk. Curbside pickup, as Governor Newsom suggested, could
mitigate these risks, but that would be true for many other sectors,
too.

But these data alone cannot tell us which businesses to open first, and
we can't simplify all these different metrics into a ``yes'' or ``no''
decision on any single business. Common sense and local knowledge are
just as important. And we should ensure that policies based on these
data do not have a disparate impact on people who are already more
heavily hit by Covid-19.

Researchers \href{https://www.nber.org/papers/w27042}{have already}
\href{http://ide.mit.edu/sites/default/files/publications/Full\%20paper\%204-26.pdf}{begun}
using these data. But to make policy, we must work with models from
epidemiology. The ultimate health consequences of any contagion-risk
measure depend on health system capacity, available treatments and
disease prevalence -- all of which will change over time and across
areas.

Second, we must account for economic factors. Reopening certain
businesses will create or diminish demand for others, and post-Covid
consumer behavior with a partially open economy may look quite different
from before.

Finally, this data comes from standard operations, but American
companies are already modifying ``business as usual.'' They can continue
to limit the number of people in stores, modify how employees work and
change how customers shop.

Our research provides a baseline to spur further ingenuity and
adaptation. With the right mix of numbers and on-the-ground knowledge,
we can develop policies to minimize both the spread of the virus and the
economic hardship of the pandemic.

Notes: The first chart includes businesses with at least 200
establishments in the data, and the second chart includes sectors with
at least 300 establishments in the data. Bubbles are scaled relative to
the product of a business or sector's average weekly visits and its
total number of establishments.

Sources: Cellphone location data and geofence files used to calculate
establishment sizes were provided by
\href{https://www.safegraph.com/}{Safegraph} and
\href{https://www.veraset.com/}{Veraset}. Business classifications come
from the North American Industry Classification System.

Katherine Baicker, Oeindrila Dube, Sendhil Mullainathan and Devin Pope
are professors at the University of Chicago. Gus Wezerek is a graphics
editor in the Opinion section.

Read 531 Comments

\begin{itemize}
\item
\item
\item
\item
\end{itemize}

Advertisement

\protect\hyperlink{after-bottom}{Continue reading the main story}

\hypertarget{site-index}{%
\subsection{Site Index}\label{site-index}}

\hypertarget{site-information-navigation}{%
\subsection{Site Information
Navigation}\label{site-information-navigation}}

\begin{itemize}
\tightlist
\item
  \href{https://help.nytimes3xbfgragh.onion/hc/en-us/articles/115014792127-Copyright-notice}{©~2020~The
  New York Times Company}
\end{itemize}

\begin{itemize}
\tightlist
\item
  \href{https://www.nytco.com/}{NYTCo}
\item
  \href{https://help.nytimes3xbfgragh.onion/hc/en-us/articles/115015385887-Contact-Us}{Contact
  Us}
\item
  \href{https://www.nytco.com/careers/}{Work with us}
\item
  \href{https://nytmediakit.com/}{Advertise}
\item
  \href{http://www.tbrandstudio.com/}{T Brand Studio}
\item
  \href{https://www.nytimes3xbfgragh.onion/privacy/cookie-policy\#how-do-i-manage-trackers}{Your
  Ad Choices}
\item
  \href{https://www.nytimes3xbfgragh.onion/privacy}{Privacy}
\item
  \href{https://help.nytimes3xbfgragh.onion/hc/en-us/articles/115014893428-Terms-of-service}{Terms
  of Service}
\item
  \href{https://help.nytimes3xbfgragh.onion/hc/en-us/articles/115014893968-Terms-of-sale}{Terms
  of Sale}
\item
  \href{https://spiderbites.nytimes3xbfgragh.onion}{Site Map}
\item
  \href{https://help.nytimes3xbfgragh.onion/hc/en-us}{Help}
\item
  \href{https://www.nytimes3xbfgragh.onion/subscription?campaignId=37WXW}{Subscriptions}
\end{itemize}
