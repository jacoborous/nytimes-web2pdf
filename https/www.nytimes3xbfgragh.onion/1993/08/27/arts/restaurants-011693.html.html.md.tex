Sections

SEARCH

\protect\hyperlink{site-content}{Skip to
content}\protect\hyperlink{site-index}{Skip to site index}

\href{https://www.nytimes3xbfgragh.onion/section/arts}{Arts}

\href{https://myaccount.nytimes3xbfgragh.onion/auth/login?response_type=cookie\&client_id=vi}{}

\href{https://www.nytimes3xbfgragh.onion/section/todayspaper}{Today's
Paper}

\href{/section/arts}{Arts}\textbar{}Restaurants

\url{https://nyti.ms/29dXYKD}

\begin{itemize}
\item
\item
\item
\item
\item
\end{itemize}

Advertisement

\protect\hyperlink{after-top}{Continue reading the main story}

Supported by

\protect\hyperlink{after-sponsor}{Continue reading the main story}

\hypertarget{restaurants}{%
\section{Restaurants}\label{restaurants}}

By Molly O'Neill

\begin{itemize}
\item
  Aug. 27, 1993
\item
  \begin{itemize}
  \item
  \item
  \item
  \item
  \item
  \end{itemize}
\end{itemize}

\includegraphics{https://s1.graylady3jvrrxbe.onion/timesmachine/pages/1/1993/08/27/011693_360W.png?quality=75\&auto=webp\&disable=upscale}

See the article in its original context from\\
August 27, 1993, Section C, Page
24\href{https://store.nytimes3xbfgragh.onion/collections/new-york-times-page-reprints?utm_source=nytimes\&utm_medium=article-page\&utm_campaign=reprints}{Buy
Reprints}

\href{http://timesmachine.nytimes3xbfgragh.onion/timesmachine/1993/08/27/011693.html}{View
on timesmachine}

TimesMachine is an exclusive benefit for home delivery and digital
subscribers.

About the Archive

This is a digitized version of an article from The Times's print
archive, before the start of online publication in 1996. To preserve
these articles as they originally appeared, The Times does not alter,
edit or update them.

Occasionally the digitization process introduces transcription errors or
other problems; we are continuing to work to improve these archived
versions.

When Gotham Bar and Grill opened in the early 1980's, it epitomized the
high-living, brash-spending decade. The warehouse space on East 12th
Street was an eclectic meld of new-wave Americana and Italian moderne.
There were the stenciled checkerboard wood floor, the Arts and Crafts
painted chairs, the gauzy, billowing parachute lamps hanging from the
ceiling, the stippled walls and the faux capitals on the soaring columns
that punctuate the rambling room.

The music boomed. The crowd hollered to converse and windmilled in vain
for absentee waiters. The kitchen stumbled. In its first month, I had
two of the worst meals of my life at Gotham Bar and Grill. Still, there
was something compelling about the place. Maybe it was the restaurant's
overblown ambition and its dogged commitment to live up to its own grand
vision. Ten years later, Gotham Bar and Grill has done that, and more.

Under the tutelage of its chef, Alfred Portale, the kitchen has
sustained its imagination but grown steadier, more reliable and
workmanlike. There are still glitches. Five years ago, I was served an
ungutted pan-fried flounder; last week I faced a quail the age of
Methuselah, perched on risotto with savory and Swiss chard. Also
recently, there was a squab whose age was not masked by the sweet
roasted garlic and pancetta that came with it, and a dish of pasta
tossed with smoked salmon and a cream sauce that had been reduced to a
rich, salty glue. When the kitchen goofs, it does so in grand style. But
it does so rarely.

Usually, the groceries are squeaky-fresh and handled solicitously. Skate
wings sauteed in brown butter and served as a first course with an
artichoke-and-white-bean salad could have been eaten as sushi: that
fresh, that briny, and a brilliant counterpoint to the artichoke and
buttery beans. Likewise, slender fillets of mackerel were delicious,
quickly seared with slow-grilled tomatoes and a sprightly salad of
arugula and chickpeas. The house seafood salad, a tower of scallops,
squid, octopus and lobster buttressed by avocado, is the kind of thing
one dreams possible but only in a seaside restaurant.

Ravioli of goat cheese snuggled up to ravioli stuffed with a husky
caponata were tender and deliciously complemented by the sweet fennel
broth and bits of roasted peppers and tart capers. Salads -\/- whether
the tumble of mixed greens in sherry vinaigrette, the mountain of
watercress and endive with a scaffolding of stacked beets, walnuts and
strong Roquefort, or the marinated goat cheese with leeks and sweet peas
-\/- do not get better than Gotham's. None are lean cuisine. The kitchen
uses fine oils with abandon. Diet somewhere else. As sprightly and fresh
as Gotham's cuisine appears, it is a worthy indulgence.

The dishes soar in height as well as flavor. Gotham Bar and Grill is the
home of tall food. The salads look like mountain ranges. The delectable
sauteed soft-shell crabs are perched high on pillars of grilled yellow
and green squash that flank a pyramid of couscous and tasty baby bok
choy. The roast chicken is excellent, but it is served with such a
towering haystack of fried matchstick potatoes that the diner's initial
response is to call for a pitchfork and start making bales. Inevitably,
the soaring constructions collapse and the diner becomes the clean-up
crew. This is not food made for first dates.

In fact, the sprawling restaurant itself isn't for cozy romance. While
there is now ample room between the white-clothed tables and the music
has been subdued, Gotham Bar and Grill remains more a brasserie than a
well-padded luxury dining room. But in that way, the place is very New
York. Diners create an invisible bubble of intimacy around their tables
that insulates but still lets them see and be seen.

Since it is one of the few spots where tattered and paint-speckled
T-shirts mingle with Armani suits, the crowd is alluring. At Gotham Bar
and Grill, downtown meets uptown; the avant-garde meets the
establishment, the young and pretty chow down with the older and rich.
One wonders who is paying the check for the tables of struggling
artists, students and lawyers too young to be partners; the menu and
intelligent wine list are pricey. Then again, the restaurant's airy,
subdued style and the superb cooking are worth saving for.

Even if I weren't on an expense account, I'd order the sauteed halibut
flanked with sweet grilled fennel and artichokes in a
dill-and-green-pepper vinaigrette or the monkfish roasted with Manila
clams and spicy Portuguese sausage. Likewise, the bouillabaisse of
scallops and giant prawns, mussels, squid and lobster is wonderfully
fresh and soulful in a saffron broth.

Grilled salmon is nicely complemented by blanched fava beans, wilted
chard, chanterelles and a roasted corn custard. The saddle of rabbit is
stupendous, grilled and perfumed with wild fennel and served with
spinach and white beans. And the lamb, with its sweet garlic custard,
the veal chop with mustard greens and the New York steak with a flan
made of marrow and mustard are all excellent eating.

Desserts, too, are well made, though a little too sweet and busy for my
taste. Companions swooned for the fragile constructions of passion fruit
Bavarian, pignoli tartlette and perfect raspberries, as well as the
towering napoleon built of mascarpone and raspberry sorbet, raspberries
and burnt-orange ice cream. I was happiest with the simple espresso
creme brulee and the peach tart with ginger ice cream and almond
brittle.

After the Herculean portions, one might want only the tiny petits fours
and something from the extensive selection of dessert wines offered.
Besides, although the service is enthusiastic, warm and quite
knowledgeable, it is still slow. Parties wait 20 minutes for cocktails
and the wine list; the first courses take forever. One lingers,
therefore, before eating, and a stroll afterward can seem more inviting
than a sweet.

But it's almost churlish to cavil against Gotham Bar and Grill. Hundreds
of restaurants that opened for the hip and well-heeled in the 80's, but
this place grunted along yeoman-style, getting better and better,
building a steady and loyal clientele, a solid kitchen and a smart wine
cellar and improving its service though not yet perfecting it

Unlike its contemporaries, the restaurant wasn't simply a spasm of
financial investment, a flurry of chic and hype. It plugged along, up a
slow, traditional trajectory to become the real thing. Gotham Bar and
Grill

***

12 East 12th Street, Greenwich Village, (212) 620-4020.

Atmosphere: Sprawling, stylish warehouse space.

Service: Amiable though sometimes slow.

Recommended dishes: Skate wings with artichoke-and-white-bean salad;
goat cheese and caponata ravioli in fennel broth; all green salads;
marinated seafood salad; halibut with artichoke; monkfish roasted with
Manila clams and sausage; sauteed soft-shell crabs with grilled squash,
couscous and bok choy; shellfish bouillabaisse; salmon with fava beans
and chanterelles; rabbit with fennel; roast chicken; rack of lamb; veal
chop with mustard greens; espresso creme brulee; warm peach tart with
ginger ice cream.

Wine: A large and intelligent selection of international wines, with
very few bargains.

Price range: Lunch: appetizers \$11 to \$13.50, main courses \$12.50 to
\$19; prix fixe \$19.93; dinner: appetizers \$11 to \$18, main courses
\$26.50 to \$32.

Credit cards: All major cards.

Hours: Lunch: noon to 2 P.M. Mondays through Fridays; dinner, 5:30 to 10
P.M. Sundays through Thursdays, until 11 P.M. Fridays and Saturdays.

Reservations: Recommended.

Wheelchair accessibility: The restaurant is divided into several levels
by short flights of steps; restrooms are a full flight below the dining
room.

Advertisement

\protect\hyperlink{after-bottom}{Continue reading the main story}

\hypertarget{site-index}{%
\subsection{Site Index}\label{site-index}}

\hypertarget{site-information-navigation}{%
\subsection{Site Information
Navigation}\label{site-information-navigation}}

\begin{itemize}
\tightlist
\item
  \href{https://help.nytimes3xbfgragh.onion/hc/en-us/articles/115014792127-Copyright-notice}{©~2020~The
  New York Times Company}
\end{itemize}

\begin{itemize}
\tightlist
\item
  \href{https://www.nytco.com/}{NYTCo}
\item
  \href{https://help.nytimes3xbfgragh.onion/hc/en-us/articles/115015385887-Contact-Us}{Contact
  Us}
\item
  \href{https://www.nytco.com/careers/}{Work with us}
\item
  \href{https://nytmediakit.com/}{Advertise}
\item
  \href{http://www.tbrandstudio.com/}{T Brand Studio}
\item
  \href{https://www.nytimes3xbfgragh.onion/privacy/cookie-policy\#how-do-i-manage-trackers}{Your
  Ad Choices}
\item
  \href{https://www.nytimes3xbfgragh.onion/privacy}{Privacy}
\item
  \href{https://help.nytimes3xbfgragh.onion/hc/en-us/articles/115014893428-Terms-of-service}{Terms
  of Service}
\item
  \href{https://help.nytimes3xbfgragh.onion/hc/en-us/articles/115014893968-Terms-of-sale}{Terms
  of Sale}
\item
  \href{https://spiderbites.nytimes3xbfgragh.onion}{Site Map}
\item
  \href{https://help.nytimes3xbfgragh.onion/hc/en-us}{Help}
\item
  \href{https://www.nytimes3xbfgragh.onion/subscription?campaignId=37WXW}{Subscriptions}
\end{itemize}
