Sections

SEARCH

\protect\hyperlink{site-content}{Skip to
content}\protect\hyperlink{site-index}{Skip to site index}

\href{https://www.nytimes3xbfgragh.onion/section/arts}{Arts}

\href{https://myaccount.nytimes3xbfgragh.onion/auth/login?response_type=cookie\&client_id=vi}{}

\href{https://www.nytimes3xbfgragh.onion/section/todayspaper}{Today's
Paper}

\href{/section/arts}{Arts}\textbar{}RESTAURANTS

\url{https://nyti.ms/29vKAid}

\begin{itemize}
\item
\item
\item
\item
\item
\end{itemize}

Advertisement

\protect\hyperlink{after-top}{Continue reading the main story}

Supported by

\protect\hyperlink{after-sponsor}{Continue reading the main story}

\hypertarget{restaurants}{%
\section{RESTAURANTS}\label{restaurants}}

By Bryan Miller

\begin{itemize}
\item
  March 28, 1986
\item
  \begin{itemize}
  \item
  \item
  \item
  \item
  \item
  \end{itemize}
\end{itemize}

\includegraphics{https://s1.graylady3jvrrxbe.onion/timesmachine/pages/1/1986/03/28/023486_360W.png?quality=75\&auto=webp\&disable=upscale}

See the article in its original context from\\
March 28, 1986, Section C, Page
24\href{https://store.nytimes3xbfgragh.onion/collections/new-york-times-page-reprints?utm_source=nytimes\&utm_medium=article-page\&utm_campaign=reprints}{Buy
Reprints}

\href{http://timesmachine.nytimes3xbfgragh.onion/timesmachine/1986/03/28/023486.html}{View
on timesmachine}

TimesMachine is an exclusive benefit for home delivery and digital
subscribers.

About the Archive

This is a digitized version of an article from The Times's print
archive, before the start of online publication in 1996. To preserve
these articles as they originally appeared, The Times does not alter,
edit or update them.

Occasionally the digitization process introduces transcription errors or
other problems; we are continuing to work to improve these archived
versions.

SEAFOOD restaurants are forever banging the gong of freshness on their
menus, using such dubious terms as ''From Local Waters'' and ''Catch of
the Day,'' even when the locale is Greenland and the catch is iced
onboard for a week at sea. It is so rare these days to find really fresh
seafood, and rarer yet to get it prepared with passion and skill. At Le
Bernardin, a stunning three-month-old restaurant in the new Equitable
Assurance Tower on Seventh Avenue at 51st Street, the fish is so fresh
and lovingly prepared as to be an epiphany of sorts.

Le Bernardin is the creation of Gilbert and Maguy Le Coze, the dynamic
siblings who own the acclaimed Parisian restaurant by the same name. The
all-seafood menu at the Manhattan establishment is virtually identical
to that of the original, except that American fish are used here. Add to
the superlative food a luxurious setting and intelligent, low-key
service and you have a four-star dining experience that even unreformed
carnivores would be foolish to miss.

The first impression that strikes you upon entering the already
perpetually booked restaurant is what is missing: clatter. The dining
room, designed by Miss Le Coze with the architect Philip George (and
bankrolled by Equitable, which has a lease arrangement with the owners),
exudes a lavishly clubby and corporate feeling: a soaring teak ceiling,
gray-blue walls, generously spaced tables and larger-than-life paintings
of fishermen and their catch.

Mr. Le Coze, who grew up on the Brittany coast where his family has a
hotel and restaurant, is fanatical when it comes to freshness. Sea
scallops, for instance, which are prepared in a host of ways, are kept
alive in the kitchen and shucked to order. Once opened, they are stroked
with a knife; if they don't swat back, they are discarded.

Such dedication can be seen in a glistening array of oysters (belons,
Cotuits, bluepoints) and addictive little necks on the half shell, and
in the lagniappe of periwinkles that you pick out of their shells with
pins. The flavor is akin to being gently washed by an ocean wave.

Under the category of raw appetizers, don't miss the pearly sheets of
black bass flecked with coriander and basil and lacquered with extra
virgin olive oil, a sensational combination, or the sparkling salad of
marinated fish. The pristine quality of a tuna carpaccio, though, was
obscured by an oversalted ginger sauce. Three little mounds of fish
tartare -salmon, red snapper and tuna - were invigoratingly seasoned one
evening, rather flat another time.

The rest of the starters are terrific; and like all dishes here, are
minimally cooked to allow the freshness of the sea to shine through.
Among my favorites are sea scallops in various guises. In one
preparation, attributed to the French chef Georges Blanc, three giant
scallops on the half shell are served in an exquisite sauce combining
the scallops' brine, some butter and a dash of saffron. They are
garnished with asparagus, thin strips of fresh tomato and fennel sprigs

\begin{itemize}
\tightlist
\item
  a heavenly combination. Equally memorable are the same scallops in a
  salad with a gossamer cream sauce perfumed with truffle juice.
\end{itemize}

If you have never tried a sea urchin, those saline little porcupines of
the sea, let Mr. Le Coze make the introduction. He scoops out the orange
roe and blends it with butter, then returns it to the shell, where it is
mixed with the urchin's warm briny nectar - an ineffably delicious
creation. It is difficult to single out other winning starters, for all
linger fondly in taste memory: vivid fricassee of shellfish, faintly
poached oysters in truffle cream sauce, and slivers of black bass warmed
in a spirited coriander infusion. At lunch, many of these dishes are
available as part of a \$35 prix fixe (\$55 at dinner).

The tone of the crack service team is set by the beguiling hostess,
Maguy Le Coze, a wisp and a smile that twirls around the room like a
benign cyclone.

If you have any doubts about wine choices, let Miss Le Coze guide you
through the extensive list, for there are some first-rate finds in the
\$20 to \$25 range, especially the St. Aubin, a crisp and fragrant white
Burgundy (by contrast, the more prestigious wines are priced on the high
side).

Now that the initial shakedown period in the kitchen is over, Mr. Le
Coze's team turns out an entree list that is extraordinarily consistent.
The same sea scallops that were so alluring as appetizers rise to the
occasion again when combined with a delicate curry sauce and asparagus,
or in a tomato-tinted sorrel sauce. Thin strips of blush-pink salmon are
presented on a red-hot platter in a bubbling sorrel and white wine
sauce. Mr. Le Coze times the presentation so perfectly that the fish
actually finishes cooking at the table on the hot plate. A thick slab of
salmon, just this side of sushi in the center, comes in a small
casserole atop an ethereal tomato-cream sauce, and garnished with fresh
mint. There was some quibbling at my table about whether the poached
halibut in a warm herb vinaigrette was a tad too acidic, but all agreed
they had never tasted halibut so moist and fresh. The only disappointing
entree was pasta in a stodgy lobster cream sauce.

The list of superlatives goes on: roasted red snapper with fennel;
sliced grouper fillet over melted leeks; roasted monkfish atop sauteed
shredded cabbage and cubes of salt pork; pompano with bright Italian
parsley.

When it's time for dessert, you can either remain afloat with a palette
of intense, fresh fruit sorbets or sink to the ocean floor with the
meltingly rich chocolate cake, a buttery thin apple tart or a dazzling
sampler of caramel sweets: caramel ice cream, flan, oeuf a la neige and
caramel mousse. The millefeuille with apples and raisins, though, had a
slight burned-butter flavor. My favorite is the trio of pears, which
combines a cinnamon-tinged poached pear, a warm tiny pear tart and pear
sorbet.

Without question, the Le Cozes have now proved their mettle on this side
of the Atlantic. The only question that remains is whether they can
maintain such a world-class pace on two continents simultaneously. As a
purely selfish motive, I would like to see their passports confiscated
and their movement restricted to this fish-starved island.

Le Bernardin

*

*

*

* 155 West 51st Street, 489-1515.

Atmosphere: Luxurious and clubby room with well-spaced tables. Service:
Knowledgeable and efficient. Recommended dishes: Raw oysters and clams,
sea scallops Georges Blanc, scallops with truffle sauce, sea urchins,
fricassee of shellfish, poached oysters in truffle cream sauce, black
bass and coriander, scallops with asparagus and curry, scallops with
sorrel and tomatoes, salmon in sorrel sauce, salmon with tomato-cream
sauce and mint, roasted snapper with fennel, grouper with leeks,
monkfish with cabbage, pompano with Italian parsley, fruit sorbets,
chocolate cake, apple tart, caramel assortment, trio of pears.

Price range: Lunch prix fixe \$35; dinner \$55 (some supplements from
\$4 to \$8). Credit cards: All major cards. Hours: Lunch, Monday through
Saturday noon to 2 P.M.; dinner, Monday through Saturday 6 to 11.

Reservations: Required. -\/-\/-\/- What the stars mean: (None)Poor to
fair

* Good

*

* Very good

*

*

* Excellent

*

*

*

* Extraordinary These ratings reflect the reviewer's reaction primarily
to food, with ambiance and service taken into consideration. Prices and
menus are subject to change.

Advertisement

\protect\hyperlink{after-bottom}{Continue reading the main story}

\hypertarget{site-index}{%
\subsection{Site Index}\label{site-index}}

\hypertarget{site-information-navigation}{%
\subsection{Site Information
Navigation}\label{site-information-navigation}}

\begin{itemize}
\tightlist
\item
  \href{https://help.nytimes3xbfgragh.onion/hc/en-us/articles/115014792127-Copyright-notice}{©~2020~The
  New York Times Company}
\end{itemize}

\begin{itemize}
\tightlist
\item
  \href{https://www.nytco.com/}{NYTCo}
\item
  \href{https://help.nytimes3xbfgragh.onion/hc/en-us/articles/115015385887-Contact-Us}{Contact
  Us}
\item
  \href{https://www.nytco.com/careers/}{Work with us}
\item
  \href{https://nytmediakit.com/}{Advertise}
\item
  \href{http://www.tbrandstudio.com/}{T Brand Studio}
\item
  \href{https://www.nytimes3xbfgragh.onion/privacy/cookie-policy\#how-do-i-manage-trackers}{Your
  Ad Choices}
\item
  \href{https://www.nytimes3xbfgragh.onion/privacy}{Privacy}
\item
  \href{https://help.nytimes3xbfgragh.onion/hc/en-us/articles/115014893428-Terms-of-service}{Terms
  of Service}
\item
  \href{https://help.nytimes3xbfgragh.onion/hc/en-us/articles/115014893968-Terms-of-sale}{Terms
  of Sale}
\item
  \href{https://spiderbites.nytimes3xbfgragh.onion}{Site Map}
\item
  \href{https://help.nytimes3xbfgragh.onion/hc/en-us}{Help}
\item
  \href{https://www.nytimes3xbfgragh.onion/subscription?campaignId=37WXW}{Subscriptions}
\end{itemize}
