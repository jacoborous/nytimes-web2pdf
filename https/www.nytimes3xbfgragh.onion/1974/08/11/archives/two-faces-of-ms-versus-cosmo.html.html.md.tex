Sections

SEARCH

\protect\hyperlink{site-content}{Skip to
content}\protect\hyperlink{site-index}{Skip to site index}

\href{https://myaccount.nytimes3xbfgragh.onion/auth/login?response_type=cookie\&client_id=vi}{}

\href{https://www.nytimes3xbfgragh.onion/section/todayspaper}{Today's
Paper}

Archives\textbar{}Ms. versus Cosmo

\url{https://nyti.ms/1MOfBxO}

\begin{itemize}
\item
\item
\item
\item
\item
\end{itemize}

Advertisement

\protect\hyperlink{after-top}{Continue reading the main story}

Supported by

\protect\hyperlink{after-sponsor}{Continue reading the main story}

\hypertarget{ms-versus-cosmo}{%
\section{Ms. versus Cosmo}\label{ms-versus-cosmo}}

By Stephanie Harrington

\begin{itemize}
\item
  Aug. 11, 1974
\item
  \begin{itemize}
  \item
  \item
  \item
  \item
  \item
  \end{itemize}
\end{itemize}

\includegraphics{https://s1.graylady3jvrrxbe.onion/timesmachine/pages/1/1974/08/11/99178554_360W.png?quality=75\&auto=webp\&disable=upscale}

See the article in its original context from\\
August 11, 1974, Page
10\href{https://store.nytimes3xbfgragh.onion/collections/new-york-times-page-reprints?utm_source=nytimes\&utm_medium=article-page\&utm_campaign=reprints}{Buy
Reprints}

\href{http://timesmachine.nytimes3xbfgragh.onion/timesmachine/1974/08/11/99178554.html}{View
on timesmachine}

TimesMachine is an exclusive benefit for home delivery and digital
subscribers.

About the Archive

This is a digitized version of an article from The Times's print
archive, before the start of online publication in 1996. To preserve
these articles as they originally appeared, The Times does not alter,
edit or update them.

Occasionally the digitization process introduces transcription errors or
other problems; we are continuing to work to improve these archived
versions.

\emph{To the Editor: I am a survivor. (What woman isn't?) Of a
suffocating marriage, two destructive affairs, even thoughts of suicide.
(I suppose that sounds melodramatic‐‐arsenic after black lace.) I was
brought up to believe that a woman could live only through a man. And
social and economic realities mane it hard to do anything else. But your
magazine let me know that I wasn't alone, that I am not crazy, that
there are women all across the country who are determined to start
considering their own needs and to accomplish something for themselves
by themselves. The support I find in your magazine has given me the
courage to finally reorder my priorities.}

Right on!

EMMA BOVARY

Yonville Parish

But . . . a letter to the editor of what? Of Ms., the political
self‐help magazine whose publisher says, ``I think of us as a kind of
connective tissue for women all across this country who felt isolated
until we came along and let them know they were not alone, that they
certainly weren't crazy and that they shouldn't feel guilty''? Or of
Cosmopolitan, the psychosexual self‐help magazine, whose articles editor
says, ``There are a lot of women out there who need help and they are
\$100 and 50 miles away from a psychiatrist, and what we are trying to
do is absolve them of guilt by letting them know they're not alone'?

Ms. or Cosmopolitan? But isn't the idea that the same woman could feel
solidarity with both Ms. and Cosmopolitan a little like saying liberty
or death is an echo, not a choice? After all, there is Ms., the feminist
monthly, grabbing women by the consciousness in every issue and.asking
them, ``Can you learn to love yourselves and change the system?'' And
Cosmopolitan, in the 10th year of its fleshly and highly profitable
incarnation as the masscult update of the ``Kama Sutra'' as interpreted
by Baby Snooks. Cosmopolitan, the magazine that goes on and on asking
women in italicized Cosmospeak: ``Don't you just love loving men, and
don't you feel just miserable when you don't have a man to love, and
wouldn't you love to learn how to love them better, and without fear or
guilt and---beat of all---to get the right one to love you?''

But if the idea of Cosmopolitan as an alternative to Ms. seems only as
serious as the idea that Helen Gurley Brown, the prototypical Cosmo
Girl, will one clay emerge from a phone booth and stand revealed as Rosa
Luxemburg, the fact is that both magazines rode in on a tide of
``revolution''---cultural revolutions set in motion by a technology that
has reduced the amount of time women have to spend on household chores
and by a rising living standard that has sent more women out to work to
supplement family incomes.

Cosmopolitan has been the working ``girl's'' chronicle of the sexual
revolution that took off in the nineteen‐sixties; Ms. is the
journalistic clearinghouse for the current phase of the feminist revolt,
which had its political seeds in the civil‐rights and student movements
of the nineteen‐sixties. Both magazines are trying to be supportive of
women who, to varying degrees, feel alienated from some or all
traditional feminine roles and are attempting, some radically, some
moderately, to live outside of the old assumptions. But one revolution
does not always wait on another. So the reformists of Cosmo
Consciousness and the relatively radical cadres of Ms. Consciousness,
who insist that the sexual revolution freed women ``only from the right
to say no to sexual intercourse,'' are working out their life‐styles
side by side, occasionally coming together on issues like economic
equality and emancipation from housework.

\textbf{the same Eve}

So there is the Cosmo Girl, ready for an article like ``How To Make A
Man's Pay,'' but not beyond day‐dreaming over ``The Poor Girl's Guide to
America's Rich Young Men'; there she is, considering the questions of
``Women, Men and Kinky Sex'' and whether ``One Man {[}Is{]} Truly
Enough,'' knowing that, out there in Peoria, she is on the front lines,
that the keypunch operator sitting next to her isn't even ready for
Cosmopolitan. And that very same moment, the magazine that is the Cosmo
Girl's guide to revolution, is, to feminists, a journal of reaction.

Helen Gurley Brown, however, did have her day on the barricades. Even if
she did not start the sexual revolution, as she insists in her own
defense against those who greeted Cosmopolitan as the Gospel according
to Jezebel, she was its one‐woman Committee of Correspondence. In the
process of, as she says, ``simply reporting what is, and trying to be
helpful if someone is troubled by what is,'' she tried to separate guilt
from sex, and that made her as suspect as fluoride or the New Deal to
fundamentalist believers in clean thoughts and the American Way. She was
denounced by Jerry Lewis and shouted down on television talk shows.
``Cosmopolitan,'' Ms. editor Gloria Steinem observes, ``was a step
forward from the formula of the traditional magazines that if you got
divorced or had affairs before marriage, you would come to a bad end . .
. If one magazine says women are sexual, it's an improvement. In the
land of the blind the one‐eyed is king.'' (Sic: queen?)

And Steinem ought to know, having been, between 1963 and 1970, a
contributing editor to Glamour, The Ladies' Home Journal, McCall's and
Seventeen. She also wrote for the now defunct Show magazine (making her
first big splash by working as a Playboy Bunny and writing an exposé for
Show called ``A Bunny's Tale'), Esquire, and The New York Times
Magazine. But, as she has complained, and at least one male editor has
conceded, until she became a contributing editor to New York magazine,
she could not get the political assignments she wanted and male writers
usually got. The general rule that women either wrote for women's
publications or were assigned to ``women's stories'' was important in
the founding of Ms.

And, if Steinem and Brown are not like Bogey and Claude Rains in the
final fade‐out of ``Casablanca,'' walking arm in arm into the mists of
the antisexist resistance, Brown has had still another revolutionary
thrust---her assumption that the American Dream is not an exclusively
male fantasy, that women, too, could take up the challenge of Horatio
Alger. It is an assumption drawn from her own life---the saga of a
scared little girl from Green Forest, Ark., who overcame a fatherless,
financially insecure Depression childhood, helped care for a sister
crippled by polio, worked as a secretary for 18 years, pushed her
natural resources to their limits (she had her acne scars sanded off,
her nose fixed, her hairline adjusted, she dieted, exercised and gulped
vitamins), and was, at long last, made an advertising copywriter because
her boss was impressed by her ``entertaining letters.'' And, at the
desperate age of 37, this tiny, whispery woman whose apparent fragility
belies a survivor's determination, made the big M---marriage to movie
executive David Brown (who produced ``The Sting''). At his urging she
wrote ``Sex and the Single Girl,'' the how‐to book on the pursuit of, uh
. . . relationthat her into the wonderful world of celebrity.

``My own credo,'' Brown has said, ``has been to marshal everything I
had'' and keep ``on struggling and working and . . . one thing led to
another. . . .'' Helen Gurley Brown''s credo. America's credo! Hard
work, selfreliance, courage, stick‐to‐itiveness. And Brown really
believes that we can all climb, climb up Sunshine Mountain. So every
month in Cosmopolitan, her faith in the possibilities inherent in the
sensual union of the Greatest Happiness Principle and the Protestant
Ethic are packaged in the inspirational rhetoric of Coue, Norman Vincent
Peale, Dale Carnegie, Dr. David Reuben, Adam Smith, and Betty and
Veronica. If feminist protests have led her to wonder if it's a
disservice to promote the bosomy, artificial Cosmo cover girl as a
physical ideal, to push ``young women to be something they may not be
able to become,'' she has concluded that just trying ``makes you feel
good.'' And every month she exhorts her girls to apply themselves to
their sexual activity with persistence and optimism, to follow a regimen
of physical fitness and good grooming, to employ Yankee ingenuity and
the ethics of an account executive. While Ms., adorned with coverpersons
like George McGovern, Bella Abzug, Helen Gahagan Douglas and Wonder
Woman, discusses the need to redefine sex roles and to overhaul our
economic system to ensure equal opportunity for all women, Brown
instructs her readers on how each of them can try to carve for herself a
bigger, juicier piece of the existing social and

Yet, between the cleavage on the cover and the ads for Frederick's of
Hollywood underwear in the back, Cosmopolitan has run articles on Bella
Abzug, Margaret Meade and Bernadette Devlin. And this does not
contradict Cosmopolitan's editorial thrust because these women, whatever
their politics (a subject that does not engage Brown), have been
successful. Brown wants her girls to achieve. Isn't it progress if a
magazine that encouraged upward mobility with ``The Complete
Husband‐Hunting Wardrobe'' and ``The Case for A LessThan‐Red‐Hot
Marriage Versus Not Marrying At All'' is now also insisting that a woman
can be ``a sex object and president of General Motors''?

This is not to say that mantrapping is not still the primary
preoccuption of the Cosmo Girl---just that the approach is not quite so
crass. ``Today,'' Brown explains, ``our major articles have to do with
women understanding their own psyches, and men's, so men and women can
live together.'' And with articles advising women on how to deal with
shyness, on anger, frigidity, masochism or the fear of commitment,
articles telling women they are not alone in their anxieties,
Cosmopolitan reflected very early the growing national sense of
togetherness in neurosis. It was a media pioneer in the pop therapy of
discovering, through the public sharing of private hangups, that the
pioblems we had clung to as our own were pretty common after all. In her
efforts, then, to enhance the male‐female connection, Brown anticipated
consciousness‐raising --- the psychopolitical tool with which feminists
hope to loosen that

\textbf{S}o, Cosmopolitan contains as much feminism as it can
incorporate into its own message---just enough to keep up with events
without losing readers. The rock of Helen Gurley Brown's faith, the
revealed truth for everyone at Cosmopolitan, from the publisher to the
elevator man, is that Helen Gurley Brown knows whom she is trying to
reach---herself 20 years ago. Which is to say, demographically, a
single, man‐hunting career woman between the ages of 18 and 34 who lives
in a metropolitan area outside New York. And 60 per cent of
Cosmopolitan's readers are 18 to 34 (70 per cent of Ms. readers are), 78
per cent of them do live in metropolitan areas (66.6 per cent of Ms.
readers do) and 61.3 per cent do work (74.6 per cent of Ms. readers
work). But only 37.8 per cent of Cosmopolitan's nearly two million
readers are single, while 49.8 per cent of Ms.'s 400,000 readers are.
More than twice as many Ms. ers attended college, and more than a third
of Ms. readers hold advanced degrees. Ms. readers, nearly half of whom
personally earn \$10,000 or more, are more affluent than those of
Cosmopolitan. And only 5.5 per cent of Ms. readers also read
Cosmopolitan.

After all, a woman who is a college graduate and earning \$14,000 a year
is not apt to take kindly to being called ``little love'' or ``little
Cosmo girl,'' as Brown addresses her readers. Nor is she likely to
respond to the rococo girlishness of Cosmopolitan's style, which is
passed on to writers in a 16‐page, mimeographed pamphlet, ``Editing (and
Writing) Rules for Cosmopolitan,'' which offers observations like ``the
theme {[}of an article{]} will probably have something to do with the
title,'' and suggestions that ``profound statements must be attributed
to somebody appropriate (even if the writer has to invent the
authority),'' and a warning to ``avoid attacking advertisers . . . and
where convenient mention advertised brands rather than nonadvertised
competition.''

And a woman with a graduate degree, whose taste in magazines runs to
Time, Newsweek, Psychology Today, Saturday Review World and Intellectual
Digest (the top five choices of Ms. readers---not one of their top 10 is
another woman's magazine), is not likely to buy a magazine whose editor
insists that ideas be made ``baby simple'' and is, as she emphasizes,
``dedicated to not doing merely critical reviews {[}of books, movies,
etc.{]} because . . . the best thing we can do for girls and the books
is . . . to recommend books that would bring a girl pleasure.''

Ms., on the other hand, does not provide the promise of sweet delights
the Cosmo girl craves. After all, a woman who likes her celebrity
interviews to star Elizabeth Taylor or Robert Redford, or Robert Redford
or Elizabeth Taylor, is not going to feel any urgency about ``Barbara
Mikulski and The Blue‐Collar Worker'' or thrill to Ms.'s wall‐to‐wall
reports on the progress of the Equal Rights Amendment.

In its ``Gazette News,'' in which readers share experiences and exchange
information on mutual‐help projects; in articles on women in higher
education, in law and medicine, in the space program, in offices and in
factories, and in articles on the almost total exclusion of women from
finance, Ms. has provided valuable information on the status of women
and ways to improve it. Some of it is digestible, but much of the prose
is as riveting as the telephone directory---the gray, not the yellow
pages.

There are, however, enough women on both levels of consciousness to
support both magazines in profitable coexistence. Cosmopolitan's
circulation has climbed steadily from 782,701, when Brown took over in
1965, to 1,810,362 by the end of 1973. Its advertising pages brought in
a revenue of \$16,276.771 last year. Cosmopolitan ran more ad pages in
1973 --- 1,532 ---than any other woman's or woman's fashion magazine
except Glamour. There are 12 foreign‐language editions of Cosmopolitan.

One practical reason for Cosmopolitan's good financial health is that
the businessmen at Hearst anticipated what rising production costs and
postal rates have established as today's prevailing wisdom in the
magazine business: Cut down on money‐losing cut‐rate subscriptions and
shift the marketing emphasis to newsstand sales. Hearst applied that
life‐saving technique as far back as the early nineteen‐fifties, pulling
Cosmopolitan out of a near‐fatal nose dive and steadily building a
newsstand audience that currently accounts for 93 per cent of its sales.
Cosmopolitan not only declines to offer its subscribers cut rates, but
has made the flamboyant gesture of raising its subscription price to
\$15, or 25 cents more per issue than the newsstand price of \$1.

Sixty per cent of Ms. readers are subscribers, but its newsstand price
is also \$1, and its subscription rates represent only modest savings
for subscribers. Less than a year after the spring of 1972, when its
Preview Issue sold out, at \$1.50 a copy, in eight days, Ms. began
paying for itself on an issue‐by‐issue basis. It is probably the first
slick, four‐color, illustrated, masscirculation monthly dedicated to a
particular political point of view to not only pay for itself but also
be in a position to anticipate turning over 10 per cent of its profits
to itscause.

Ms.'s 130 advertising pages and \$476,458 in ad revenue for the first
six months of 1974, although an increase over the same period last year,
seem modest indeed compared with Cosmopolitan. But Ms. does not want
advertising to go much higher than an average: of 35 per cent of total
pages per issue. They have, in fact, tried, says Steinem, ``to structure
ourselves with as little advertising as possible,'' leaving the magazine
free to refuse advertising that is demeaning to women or ads for
products that are potentially dangerous. Ms; will not yet advertise
feminine hygiene deodorants and has gone so far as to sacrifice nearly
\$80,000 worth of advertising that aroused significant complaints from
readers. But, in its campaign to counter the assumption that women spend
half their money on food and half on cosmetics, Ms. is, in Steinem's
words, ``making a major effort to get people products . . . cars, books,
airline tickets, cameras, records, stereos, sporting equipment, gasoline
insurance. . . .''

Finding funding for a national feminist magazine was hard enough, but
when the women who founded Ms. told potential backers that the staff
must retain the controlling interest in the magazine (that, in other
words, the major investor have a minority voice), that advertising would
be scrutinized for demeaning inferences about women, and that 10 per
cent of the magazine's profits would be donated to the women's movement,
they were laughed right out of the board rooms. Had not Washington Post
publisher Katharine Graham sustained them with \$20,000 (later redeemed
in stock), the Ms. idea might not have survived long enough for the
women to receive an offer they couldn't refuse from New York magazine
publisher Clay Felker, journalism's fast‐food caterer who markets words
and pictures like prepackaged hors d'oeuvres.

Under the editorial direction of the Ms. group, New York magazine's
production staff turned out a 300,000‐copy Preview Issue of Ms.,
selections from which appeared first as an insert in New York's 1971
year‐end double issue. On the basis of 36,000 subscription orders
brought in by the Preview Issue, Warner Communications agreed to invest
\$1‐million in Ms. for only 25 per cent of the stock.

\textbf{M}s. has grown from a gamble to what Newsweek dulped a
``miniconglomerate.'' Its spinoffs include a paperback ``Ms. Reader'; a
hit children's record, ``Free to Be You and Me,'' which was parlayer
into a book and an Emmy‐winning television special; ``Woman Alive!,'' a
television pilot for a possible magazine‐format series; Ms. Marketing,
Inc., a research and consulting operation that will stress women's
influence on purchases of ``nonfemale'' products; and the Ms. Foundation
for Women, the vehicle through which profits will be channeled into the
movement.

But it is not all joy in Ms.vile. A number of writers (despite some
tokenism, most Ms. contributors are women; Cosmopolitan is sexually
integrated) have been sharply critical of Ms.'s policy of editing by
consensus, which, according to a report in Ms.'s second issue, just
evolved naturally: ``The work got done, and the decisions got made. They
happened communally. We never had time to sit down and discuss our
intellectual aversion to the hierarchy of most offices . . . We just
chose not to do anything with which one of us strongly disagreed . . .
On the masthead, we listed ourselves alphabetically. . . .''

Ms. contributors have charged, however, that decisions did not get made,
or got made by one editor only to be altered after discussion with
others. They complain of having had to satisfy ``editorial committees'';
of manuscripts completely rewritten without consultation with the
writer; humor deleted because, they say, Ms. is reluctant to treat
women's problems lightly; of ``feminist humor'' edited into pieces. They
blame Ms. for being less interesting than it could because it rejects
individual perceptions of writers who are critical of women and because
it is too anxious to make all articles all things to all women. What has
been a communal experience for Ms., has, to these writers, seemed a
``tyranny of structurelessness.'' And a few insist that although Steinem
is listed next to last on the editorial masthead, she, and not any
editorial commune, has final say on the political tone of articles.

The staff replies that individual editors are consulted on articles
touching their specialties, and Stemem's specialty is politics. As for
daily functioning, it appears to the uninitiated that not all decisions
are communal and that publisher Pat Carbine and managing editor Suzanne
Levine are there to dispel hesitations. The staff does concede that, in
the beginning, a certain lack of organization and efficiency did result
from a combination of their own inexperience in working with writers,
the number of writers they were dealing with, and an overwhelming amount
of unsolivited material.

Carbine sees the criticism as the product of mutual misunderstanding.
``The intensity of commitment to launching the magazine was so extreme
and unrelieved that we expected the writers to share in that feeling and
to understand. It was unrealistic of us to expect them to understand,
but it was unrealistic of them to expect us to act like a going concern
and treat them better than anyone ever had.''

A Ms. editorial meeting can be as unstructured as its relations with
writers, as unstructured as its work space ---a duplex of offices at
41st and Lexington, as efficiently laid out as a railroad flat caught in
an illegal U‐turn. In the larger of two communal editorial offices about
20 women and one male mailperson (anyone who works at Ms. can attend
editorial meetings) sit on chairs, desks, the floor, or line the walls.
First, the talk centers on planned urban spaces for families. Then there
is a report on the struggle against sexism on Italy's extraparliamentary
left. ``I want,'' says one editor, ``to raise a question a lot of my
friends have been asking: Is romance dead?'' There is laughter and
shouts of ``What is romance? Is it a magazine?'' Steinem, who has been
sitting quietly at her desk by the door, answers in a feminist context,
remarking that ``women's obsession with romance is a displacement of
their

At Cosmopolitan there is certainly no doubt that romance is alive and
well. It is central to the caricatures of ideas that are bounced around
at its editorial meetings---ideas like ``You Can't Buy Sexual Chemistry
at Bloomingdale's!'' ``Nureyev In Love!'' ``Discover Yourself Through
Your Daydreams!'' The verbal concoctions seem to have nothing to do with
the pleasant, perfectly sensible people reciting them . . . except, of
course, it's a living. They form a crowded circle --- a dozen women
ranging from their late 20's to early 50's, and two men --- squeezed
into Helen Gurley Brown's small Louis‐Something‐Or‐Other office four
floors above, and slightly west of, the Chevrolet showroom at 57th and
Broadway.

``Have we done anxiety lately?'' ``That,'' quips Brown, ``is like asking
if you've eaten in the last week.'' ``We have depression in the works,''
someone else notes; ``this should be separate.'' ``This one is totally
ridiculous---`Are Lesbians Ecological?''' The ideas go shooting up the
flagpole. The Editor salutes them all---well, almost all---with a slow,
encouraging smile, a quiet ``Very nice, thank you.'' One thing she will
have none of is hurting people's feelings.

In this environment is sustained the vision of the Cosmo Girl, who, in
Brown's words, expects ``to make a contribution through her work'' but
``loves men and doesn't feel complete or even alive unless she has
one.'' It is a vision that reached an apotheosis of silliness in a
full‐page ad for Cosmopolitan that will have the editors of Mad Comics
beating their breasts for months for not having thought of it first. It
was, of course, the ad suggesting that Nancy Maginnes Kissinger is
really a Cosmo girl: ``She's as smart as he is . . . developed her own
fabulous career and never lived through him. There was lots of
competition for the man but she waited it out and got more and more and
more imporCosmo!''

But the question, Ms. might point out, is not how she got him, but why,
if she's so smart, isn't she Secretary of State? The answer is obvious:
If she's a good Cosmo girl, she's too busy perfuming light bulbs in her
``man trap apartment,'' and keeping her ``affair going long distance''
by mailing her absent lover her birth control pills. She's having a face
lift, a breast implant, and worrying about how many affairs are good for
her. So how can she possibly---despite Brown's insistence that you can
be a sex object and president of General Motors---muster the time,
energy and concentration for any other work?

``I don't know what to choose,'' says Helen Gurley Brown. ``Having a man
is not more important than great work, it's as important.'' But she also
says that in Cosmopolitan, ``sex pieces, get‐man pieces are stronger,
more frequent . . . because I'm more caught up in the life of the
emotions. . . . My jobs were just always coming up along the way. I know
intellectually that jobs are as important. But they're not as good for
Cosmo in terms of sales. We have had major articles on careers, on
nursing and library work. But they don't have nearly as much clout as an
article on `Find Your Second Husband Before You Divorce Your First One.'
``Back in the sixties, the count on career pieces in Cosmopolitan was
roughly one in six issues; more recently the target has been one major
job piece every third month. Many of those that do run, however, are
about fringe jobs (stunt women, a ``lady hardhat,'' a ``lady
bartender'') or scarce glamour jobs treporters, television newswomen,
corporation presidents) --- jobs that make good copy but are not
realistic options for most

Ms.'s articles on factory work, office work, law and medicine trace the
history of women in these jobs and examine current conditions of work,
pay, benefits, and advancement, and ways to improve them --- grievance
procedures, organizing, implementing legislation or supporting new
legislation. ``Cosmopolitan,'' says Ms.'s Suzanne Levine, ``is talking
to women one by one . . . we're talking about making all women's lives
work.''

Ms.'s efforts to embrace all women, though it vexes its critics, does
separate it from Cosmopolitan, which tries to aid and comfort some women
at the inevitable expense of others. In Cosmopolitan's moral framework,
married men are fair game for single women because ``there simply isn't
enough product to go around. . . . I feel solidarity with singles,''
says Brown. ``I never was much for wives. All those years, I was the
underdog and they were the haves.'' No smiles now, her voice starting to
race in an agitated staccato. ``. . . Single women have to have more
energy, more guts. I'm hostile to wives who don't work, don't do
anything except be housewives

I know I've been able to do what I've done because I have no children .
. . but women don't have to have more children. They could get it
together, but they don't. . . .''

``Professional'' wives and mothers appear in Cosmopolitan as jealous,
competitive, smothering shrews. This is far from the kind of support
wives and mothers try to offer each other in Ms. articles, trying to
help each other get out from under their burden of frustration and guilt
by sharing symptoms and cures and analyzing the system they hold
responsible for emotionally mutilating women. In these Cosmopolitan
pieces, there does not seem to be any understanding that today's Cosmo
Girl, if she is successful, will be tomorrow's wife and mother---a
victim to be preyed on by the Cosmo Girl who rises to take her place.
But, that's free enterprise, and it's the only system Helen Gurley Brown
knows.

``There is,'' says Steinem, ``still the assumption that a woman is not a
complete human being by herself. . . . We have to consider the ways in
which we are man junkies.''

Brown has always been financially independent, so she has no problem
incorporating that half of the feminist message into her
magazine---although it comes through garbled, making it seem, sometimes,
that a job is worth having just to fall back on when you don't have a
man. And Brown has, of course, always been a man junkie. In fact, she
has become even more financially independent by marketing her habit. She
is, in her way, a pusher. But nearly two million women are supporting
her, and most of them are not yet ready to ask the kind of questions Ms.
is exploring. Their magazine preference indicates that they still need
to hear what Helen Gurley has to

``If there's one thing I know anything about, it's being anxiety‐ridden,
it's being fearful,'' says Brown, who had no feminist movement to
bolster her when she was single and scared. ``If there's one thing I
know anything about, it's getting through the night.'' So, she does not
push radical social and economic changes, she just tries to help all
those other Helen Gurley Browns she knows are out there make it through
to morning. It is, for her, a profitable mission. But she is selling
half a feminist message, garbled though it is, to women who might
otherwise buy none. ■

Advertisement

\protect\hyperlink{after-bottom}{Continue reading the main story}

\hypertarget{site-index}{%
\subsection{Site Index}\label{site-index}}

\hypertarget{site-information-navigation}{%
\subsection{Site Information
Navigation}\label{site-information-navigation}}

\begin{itemize}
\tightlist
\item
  \href{https://help.nytimes3xbfgragh.onion/hc/en-us/articles/115014792127-Copyright-notice}{©~2020~The
  New York Times Company}
\end{itemize}

\begin{itemize}
\tightlist
\item
  \href{https://www.nytco.com/}{NYTCo}
\item
  \href{https://help.nytimes3xbfgragh.onion/hc/en-us/articles/115015385887-Contact-Us}{Contact
  Us}
\item
  \href{https://www.nytco.com/careers/}{Work with us}
\item
  \href{https://nytmediakit.com/}{Advertise}
\item
  \href{http://www.tbrandstudio.com/}{T Brand Studio}
\item
  \href{https://www.nytimes3xbfgragh.onion/privacy/cookie-policy\#how-do-i-manage-trackers}{Your
  Ad Choices}
\item
  \href{https://www.nytimes3xbfgragh.onion/privacy}{Privacy}
\item
  \href{https://help.nytimes3xbfgragh.onion/hc/en-us/articles/115014893428-Terms-of-service}{Terms
  of Service}
\item
  \href{https://help.nytimes3xbfgragh.onion/hc/en-us/articles/115014893968-Terms-of-sale}{Terms
  of Sale}
\item
  \href{https://spiderbites.nytimes3xbfgragh.onion}{Site Map}
\item
  \href{https://help.nytimes3xbfgragh.onion/hc/en-us}{Help}
\item
  \href{https://www.nytimes3xbfgragh.onion/subscription?campaignId=37WXW}{Subscriptions}
\end{itemize}
