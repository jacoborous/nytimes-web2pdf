Sections

SEARCH

\protect\hyperlink{site-content}{Skip to
content}\protect\hyperlink{site-index}{Skip to site index}

\href{https://www.nytimes3xbfgragh.onion/section/technology/personaltech}{Personal
Tech}

\href{https://myaccount.nytimes3xbfgragh.onion/auth/login?response_type=cookie\&client_id=vi}{}

\href{https://www.nytimes3xbfgragh.onion/section/todayspaper}{Today's
Paper}

\href{/section/technology/personaltech}{Personal Tech}\textbar{}Sweeping
Away a Search History

\url{https://nyti.ms/1dNmjDu}

\begin{itemize}
\item
\item
\item
\item
\item
\end{itemize}

Advertisement

\protect\hyperlink{after-top}{Continue reading the main story}

Supported by

\protect\hyperlink{after-sponsor}{Continue reading the main story}

\href{/column/machine-learning}{Machine Learning}

\hypertarget{sweeping-away-a-search-history}{%
\section{Sweeping Away a Search
History}\label{sweeping-away-a-search-history}}

\includegraphics{https://static01.graylady3jvrrxbe.onion/images/2014/04/02/multimedia/tech-wood-search/tech-wood-search-videoSixteenByNine1050.jpg}

By \href{http://www.nytimes3xbfgragh.onion/by/molly-wood}{Molly Wood}

\begin{itemize}
\item
  April 2, 2014
\item
  \begin{itemize}
  \item
  \item
  \item
  \item
  \item
  \end{itemize}
\end{itemize}

YOUR search history contains some of the most personal information you
will ever reveal online: your health, mental state, interests, travel
locations, fears and shopping habits.

And that is information most people would want to keep private.
Unfortunately, your web searches are carefully tracked and saved in
databases, where the information can be used for almost anything,
including highly targeted advertising and
\href{http://www.forbes.com/sites/modeledbehavior/2013/09/01/will-big-data-bring-more-price-discrimination/}{price
discrimination} based on your data profile.

``Nobody understands the long-term impact of this data collection,''
said Casey Oppenheim, co-founder of Disconnect, a company that helps
keep people anonymous online. ``Imagine that someone has 40 years of
your search history. There's no telling what happens to that data.''

Fortunately, Google, Microsoft's Bing and smaller companies provide ways
to delete a search history or avoid leaving one, even if hiding from
those ads can be more difficult.

Google makes it easy to find your personal web history, manage it and
even delete it. Just go to \url{http://google.com/history} and log in to
your Google account. There, you will see your entire history and can
browse it by category. For example, in the last month, I've done image
searches for Gal Gadot (who will play the new Wonder Woman), ``pointy
nail trend'' and ``Wayne Rooney hair transplant,'' plus a few more
intelligent things, I'm sure.

If you would like this history to go away, click the gear icon in the
upper right of the page and choose Settings. Here, you can turn off
search history, so Google won't save future searches. You can delete
your history from Google's database or just remove specific items from
your recent history.

This does not opt you out of ad tracking, however. It just gets rid of a
potentially embarrassing or damaging historical record. Google also lets
you opt out of targeted and search ads on the web and in Gmail, at
\url{http://google.com/settings/ads}.

You can turn off and erase your search history on Microsoft Bing at
\href{https://www.bing.com/profile/history}{https://www.bing.com/profile/history.}
Yahoo lets you turn off future search histories but doesn't have a way
to delete the old one. Visit \url{http://search.yahoo.com/preferences/}
to turn off your history.

Even with your history turned off, though, you are still sending a lot
of personal data when you surf or search from all three, especially if
you are logged in to your Google, Microsoft or Yahoo account when you
search.

Gabriel Weinberg, chief executive at the alternative search engine
DuckDuckGo, says there is a different way, and it can still involve
making money from search-related ads.

DuckDuckGo collects no personally identifying information (like your
I.P. address) as you search and doesn't save any search history that can
be tied to you. But DuckDuckGo still makes money on ads.

``It's a myth that the search engines need to track to you to make most
of their money in web search,'' Mr. Weinberg said. ``Most of the search
ads are based on the queries you type in and have nothing to do with
your search history.''

Image

Credit...James Best Jr./The New York Times

DuckDuckGo said its searches more than doubled from 2012 to 2013 to over
a billion queries a year. That is tiny compared with Google (100 billion
searches a \emph{month}) or even Bing or Yahoo, but the growth
demonstrates a real interest in private searching. Other options include
PrivateLee, Qrobe.it and IxQuick, which is based in the Netherlands.

Using DuckDuckGo or another private engine takes a little getting used
to. DuckDuckGo doesn't autocomplete search terms, for example, but
PrivateLee does. They obviously don't filter results on the basis of
your past searches, either. The results may seem a little strange as a
result.

If you are partial to Google, Bing or Yahoo as a search engine but want
it to be anonymous, try Disconnect Search.

The \href{https://search.disconnect.me/}{web version} lets you specify
Google, Bing, Yahoo, DuckDuckGo or Blekko as your engine, but it
searches them without sharing your Internet address or saving a search
history.

You can also install Disconnect Search as a plug-in for the Chrome or
Firefox browsers, so you don't have to remember to go to the site. There
is an Android app available, but none for Apple's iOS. Disconnect also
offers other privacy tools that block ad tracking in browsers and on
iOS.

Disconnect Search isn't perfect. For one thing, it forces all search
searches into whatever search engine you have set as your default. So if
you are clicking search links on the Yahoo home page, you won't end up
in Yahoo's search, you will end up in your default.

It also can't handle Google Maps links. If you click a link to a Google
Maps location from a website, for example, you will be taken to a search
result for that address, rather than the map.

So why do all of this? If you have been wondering why eerily specific
ads keep showing up on every site you visit, in your email, on Facebook
or anywhere else you go online, it's because those advertisers \emph{do}
know you that well.

Search companies like Google feed your queries to advertisers, who use
them to show you ads related to your interests --- and that is just on
Google's site.

When you click search result links, the sites you visit can access your
search terms and your I.P. address, which can determine the location of
the computer you are using. That means those third-party sites also know
what you searched for and who you are or at least where your computer
lives.

In addition, your search history can create something called a
\href{http://dontbubble.us/}{filter bubble}. As you build up a history
of clicks and queries, Google will start delivering search results
tailored to what it thinks you want to see. As a result, your results
start to reinforce your worldview or even start to be less accurate, as
you see only sites like those you have clicked on before.

For me, the right combination of privacy and search convenience came
from making DuckDuckGo the default search engine in my browser. I like
its instant search results, which appear above the rest of the results,
and it's fast and accurate. Ads are clearly marked and often relevant.

While Google does give users some control over their web and search
activities and ad tracking, it will always be in that company's best
interest to share your information to serve you better ads and to
collect as much as they can. That is not necessarily in your best
interest.

Privacy matters for many reasons, both tangible and not, and it's wise
to exercise control when you can.

Advertisement

\protect\hyperlink{after-bottom}{Continue reading the main story}

\hypertarget{site-index}{%
\subsection{Site Index}\label{site-index}}

\hypertarget{site-information-navigation}{%
\subsection{Site Information
Navigation}\label{site-information-navigation}}

\begin{itemize}
\tightlist
\item
  \href{https://help.nytimes3xbfgragh.onion/hc/en-us/articles/115014792127-Copyright-notice}{©~2020~The
  New York Times Company}
\end{itemize}

\begin{itemize}
\tightlist
\item
  \href{https://www.nytco.com/}{NYTCo}
\item
  \href{https://help.nytimes3xbfgragh.onion/hc/en-us/articles/115015385887-Contact-Us}{Contact
  Us}
\item
  \href{https://www.nytco.com/careers/}{Work with us}
\item
  \href{https://nytmediakit.com/}{Advertise}
\item
  \href{http://www.tbrandstudio.com/}{T Brand Studio}
\item
  \href{https://www.nytimes3xbfgragh.onion/privacy/cookie-policy\#how-do-i-manage-trackers}{Your
  Ad Choices}
\item
  \href{https://www.nytimes3xbfgragh.onion/privacy}{Privacy}
\item
  \href{https://help.nytimes3xbfgragh.onion/hc/en-us/articles/115014893428-Terms-of-service}{Terms
  of Service}
\item
  \href{https://help.nytimes3xbfgragh.onion/hc/en-us/articles/115014893968-Terms-of-sale}{Terms
  of Sale}
\item
  \href{https://spiderbites.nytimes3xbfgragh.onion}{Site Map}
\item
  \href{https://help.nytimes3xbfgragh.onion/hc/en-us}{Help}
\item
  \href{https://www.nytimes3xbfgragh.onion/subscription?campaignId=37WXW}{Subscriptions}
\end{itemize}
