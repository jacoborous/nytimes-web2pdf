Sections

SEARCH

\protect\hyperlink{site-content}{Skip to
content}\protect\hyperlink{site-index}{Skip to site index}

\href{https://www.nytimes3xbfgragh.onion/section/world/asia}{Asia
Pacific}

\href{https://myaccount.nytimes3xbfgragh.onion/auth/login?response_type=cookie\&client_id=vi}{}

\href{https://www.nytimes3xbfgragh.onion/section/todayspaper}{Today's
Paper}

\href{/section/world/asia}{Asia Pacific}\textbar{}Obama Vows a Response
to Cyberattack on Sony

\url{https://nyti.ms/1v0KOi9}

\begin{itemize}
\item
\item
\item
\item
\item
\item
\end{itemize}

Advertisement

\protect\hyperlink{after-top}{Continue reading the main story}

Supported by

\protect\hyperlink{after-sponsor}{Continue reading the main story}

\hypertarget{obama-vows-a-response-to-cyberattack-on-sony}{%
\section{Obama Vows a Response to Cyberattack on
Sony}\label{obama-vows-a-response-to-cyberattack-on-sony}}

\includegraphics{https://static01.graylady3jvrrxbe.onion/images/2014/12/19/us/obama-hacking-presser/obama-hacking-presser-videoSixteenByNine3000.jpg}

By \href{http://www.nytimes3xbfgragh.onion/by/david-e-sanger}{David E.
Sanger},
\href{http://www.nytimes3xbfgragh.onion/by/michael-s-schmidt}{Michael S.
Schmidt} and
\href{http://www.nytimes3xbfgragh.onion/by/nicole-perlroth}{Nicole
Perlroth}

\begin{itemize}
\item
  Dec. 19, 2014
\item
  \begin{itemize}
  \item
  \item
  \item
  \item
  \item
  \item
  \end{itemize}
\end{itemize}

WASHINGTON --- President Obama said on Friday that the United States
``will respond proportionally'' against North Korea for its destructive
cyberattacks on Sony Pictures, but he criticized the Hollywood studio
for giving in to intimidation when it withdrew ``The Interview,'' the
satirical movie that provoked the attacks, before it opened.

Deliberately avoiding specific discussion of what kind of steps he was
planning against the reclusive nuclear-armed state, Mr. Obama said that
the response would come ``in a place and time and manner that we
choose.'' Speaking at a White House news conference before leaving for
Hawaii for a two-week vacation, he said American officials ``have been
working up a range of options'' that he said have not yet been presented
to him.

A senior official said Mr. Obama would likely be briefed in Hawaii on
those options. Mr. Obama's threat came just hours after the F.B.I. said
it had assembled extensive evidence that the North Korean government
organized the cyberattack that debilitated the Sony computers.

If he makes good on it, it would be the first time the United States has
been known to retaliate for a destructive cyberattack on American soil
or to have explicitly accused the leaders of a foreign nation of
deliberately damaging American targets, rather than just stealing
intellectual property. Until now, the most aggressive response was the
largely symbolic
\href{http://www.nytimes3xbfgragh.onion/2014/05/20/us/us-to-charge-chinese-workers-with-cyberspying.html}{indictment
of members of a Chinese Army unit} this year for stealing intellectual
property.

The president's determination to act was a remarkable turn in what first
seemed a story about Hollywood backbiting and gossip as revealed by the
release of emails from studio executives and other movie industry
figures describing Angelina Jolie as a ``spoiled brat'' and making
racially tinged lists of what they thought would be Mr. Obama's favorite
movies.

But it quickly escalated, and the combination of the destructive nature
of the attacks --- which wiped out Sony computers --- and a new threat
this week against theatergoers if the ``The Interview,'' whose plot
revolves an attempt to assassinate the North Korean leader, Kim Jong-un,
opened on Christmas Day turned it into a national security issue.
``First it was a game-changer,'' one official said. ``Then it became a
question of what happens if we don't respond? And the president
concluded that's not an option.''

But as striking as his determination to make North Korea pay a price for
its action was his critique of Sony Pictures for its decision to cancel
``The Interview.'' Mr. Obama argued that the precedent that withdrawing
the movie set could be damaging --- and that the United States could not
give in to intimidation.

``I wish they had spoken to me first,'' Mr. Obama said of Sony's
leadership. ``I would have told them, `Do not get into a pattern in
which you're intimidated by these kinds of criminal attacks.'~''

In a clear reference to Mr. Kim, he said, ``We cannot have a society in
which some dictator someplace can start imposing censorship here in the
United States.'' That would encourage others to do the same ``when they
see a documentary that they don't like or news reports that they don't
like.''

The chief executive of Sony Pictures, Michael Lynton, immediately
defended his decision and said Mr. Obama misunderstood the facts. He
argued that when roughly 80 percent of the country's theaters refused to
book the film after the latest threat, ``we had no alternative but to
not proceed with the theatrical release,'' Mr. Lynton told CNN. ``We
have not caved, we have not given in, we have not backed down.''

In a follow-up statement, Sony said that it ``immediately began actively
surveying alternatives'' to theatrical distribution after theater owners
balked. But so far no mainstream cable, satellite or online film
distributor was willing to adopt the movie.

\includegraphics{https://static01.graylady3jvrrxbe.onion/images/2014/12/20/business/20cyber-pic/20cyber-pic-articleLarge.jpg?quality=75\&auto=webp\&disable=upscale}

Mr. Obama did not pass up the opportunity to take a jab at the insecure
North Korean government for worrying about a Hollywood comedy, even a
crude one.

``I think it says something about North Korea that they decided to have
the state mount an all-out assault on a movie studio because of a
satirical movie,'' he said, smiling briefly at the ridiculousness of an
international confrontation set off by a Hollywood comedy.

The case against North Korea was described by the F.B.I. in somewhat
generic terms. It said there were significant ``similarities in specific
lines of code, encryption algorithms, data deletion methods and
compromised networks'' to previous attacks conducted by the North
Koreans.

``The F.B.I. also observed significant overlap between the
infrastructure used in this attack and other malicious cyberactivity the
U.S. government has previously linked directly to North Korea,'' the
bureau said. ``For example, the F.B.I. discovered that several Internet
protocol addresses associated with known North Korean infrastructure
communicated with I.P. addresses that were hard-coded into the data
deletion malware used in this attack.'' An Internet protocol address is
the closest thing to an identifier of where an attack emanated.

Some of the methods employed in the Sony attack were similar to ones
that were used by the North Koreans against South Korean banks and news
media outlets in 2013. That was a destructive attack, as was an attack
several years ago against Saudi Aramco, later attributed to Iran. While
there were common cybertools to the Saudi attack as well, Mr. Obama told
reporters on Friday he had seen no evidence that any other nation was
involved.

The F.B.I.'s announcement was carefully coordinated with the White House
and reflected the intensity of the investigation; just a week ago, a
senior F.B.I. official said he could not say whether North Korea was
responsible. Administration officials noted that the White House had now
described the action against Sony as an ``attack,'' as opposed to mere
theft of intellectual property, and that suggested that Mr. Obama was
now looking for a government response, rather than a corporate one.

The F.B.I.'s statements ``are based on intelligence sources and other
conclusive evidence,'' said James A. Lewis, a cybersecurity expert at
the Center for Strategic and International Studies in Washington. ``Now
the U.S. has to figure out the best way to respond and how much risk
they want to take. It's important that whatever they say publicly
signals to anyone considering something similar that they will be
handled much more roughly.''

While American officials were circumspect about how they had collected
evidence, some has likely been developed from ``implants'' placed by the
National Security Agency. North Korea has proved to be a particularly
hard target because it has relatively low Internet connectivity to the
rest of the world, and its best computer minds do not move out of the
country often, where their machines and USB drives could be accessible
targets.

Private security researchers who specialize in tracing attacks said that
the government's conclusions matched their own findings. George Kurtz, a
founder of CrowdStrike, a California-based security firm, said that his
company had been studying public samples of the Sony malware and had
linked them to hackers inside North Korea --- the firm internally refers
to them as Silent Chollima --- who have been conducting attacks since
2006.

In 2009, a similar campaign of coordinated cyberattacks over the Fourth
of July holiday hit 27 American and South Korean websites, including
those of South Korea's presidential palace, called the Blue House, and
its Defense Ministry, and sites belonging to the United States Treasury
Department, the Secret Service and the Federal Trade Commission. North
Korea was suspected, but a clear link was never established.

But those were all ``distributed denial of service'' attacks, in which
attackers flood the sites with traffic until they fall offline. The Sony
attack was far more sophisticated: It wiped data off Sony's computer
systems, rendering them inoperable.

``The cyberattack against Sony Pictures Entertainment was not just an
attack against a company and its employees,'' Jeh C. Johnson, the
secretary of the Department of Homeland Security, said in a statement.
``It was also an attack on our freedom of expression and way of life.''

Advertisement

\protect\hyperlink{after-bottom}{Continue reading the main story}

\hypertarget{site-index}{%
\subsection{Site Index}\label{site-index}}

\hypertarget{site-information-navigation}{%
\subsection{Site Information
Navigation}\label{site-information-navigation}}

\begin{itemize}
\tightlist
\item
  \href{https://help.nytimes3xbfgragh.onion/hc/en-us/articles/115014792127-Copyright-notice}{©~2020~The
  New York Times Company}
\end{itemize}

\begin{itemize}
\tightlist
\item
  \href{https://www.nytco.com/}{NYTCo}
\item
  \href{https://help.nytimes3xbfgragh.onion/hc/en-us/articles/115015385887-Contact-Us}{Contact
  Us}
\item
  \href{https://www.nytco.com/careers/}{Work with us}
\item
  \href{https://nytmediakit.com/}{Advertise}
\item
  \href{http://www.tbrandstudio.com/}{T Brand Studio}
\item
  \href{https://www.nytimes3xbfgragh.onion/privacy/cookie-policy\#how-do-i-manage-trackers}{Your
  Ad Choices}
\item
  \href{https://www.nytimes3xbfgragh.onion/privacy}{Privacy}
\item
  \href{https://help.nytimes3xbfgragh.onion/hc/en-us/articles/115014893428-Terms-of-service}{Terms
  of Service}
\item
  \href{https://help.nytimes3xbfgragh.onion/hc/en-us/articles/115014893968-Terms-of-sale}{Terms
  of Sale}
\item
  \href{https://spiderbites.nytimes3xbfgragh.onion}{Site Map}
\item
  \href{https://help.nytimes3xbfgragh.onion/hc/en-us}{Help}
\item
  \href{https://www.nytimes3xbfgragh.onion/subscription?campaignId=37WXW}{Subscriptions}
\end{itemize}
