Sections

SEARCH

\protect\hyperlink{site-content}{Skip to
content}\protect\hyperlink{site-index}{Skip to site index}

\href{https://www.nytimes3xbfgragh.onion/section/books/review}{Book
Review}

\href{https://myaccount.nytimes3xbfgragh.onion/auth/login?response_type=cookie\&client_id=vi}{}

\href{https://www.nytimes3xbfgragh.onion/section/todayspaper}{Today's
Paper}

\href{/section/books/review}{Book Review}\textbar{}July's Book Club
Pick: Claudia Rankine's `Citizen'

\url{https://nyti.ms/1x5czve}

\begin{itemize}
\item
\item
\item
\item
\item
\end{itemize}

Advertisement

\protect\hyperlink{after-top}{Continue reading the main story}

Supported by

\protect\hyperlink{after-sponsor}{Continue reading the main story}

\hypertarget{julys-book-club-pick-claudia-rankines-citizen}{%
\section{July's Book Club Pick: Claudia Rankine's
`Citizen'}\label{julys-book-club-pick-claudia-rankines-citizen}}

\includegraphics{https://static01.graylady3jvrrxbe.onion/images/2014/12/16/books/review/28BASS/28BASS-articleLarge-v2.jpg?quality=75\&auto=webp\&disable=upscale}

Buy Book ▾

\begin{itemize}
\tightlist
\item
  \href{https://www.amazon.com/gp/search?index=books\&tag=NYTBSREV-20\&field-keywords=Citizen\%3A+An+American+Lyric+Claudia+Rankine}{Amazon}
\item
  \href{https://du-gae-books-dot-nyt-du-prd.appspot.com/buy?title=Citizen\%3A+An+American+Lyric\&author=Claudia+Rankine}{Apple
  Books}
\item
  \href{https://www.anrdoezrs.net/click-7990613-11819508?url=https\%3A\%2F\%2Fwww.barnesandnoble.com\%2Fw\%2F\%3Fean\%3D9781555976903}{Barnes
  and Noble}
\item
  \href{https://www.anrdoezrs.net/click-7990613-35140?url=https\%3A\%2F\%2Fwww.booksamillion.com\%2Fp\%2FCitizen\%253A\%2BAn\%2BAmerican\%2BLyric\%2FClaudia\%2BRankine\%2F9781555976903}{Books-A-Million}
\item
  \href{https://bookshop.org/a/3546/9781555976903}{Bookshop}
\item
  \href{https://www.indiebound.org/book/9781555976903?aff=NYT}{Indiebound}
\end{itemize}

When you purchase an independently reviewed book through our site, we
earn an affiliate commission.

By Holly Bass

\begin{itemize}
\item
  Dec. 24, 2014
\item
  \begin{itemize}
  \item
  \item
  \item
  \item
  \item
  \end{itemize}
\end{itemize}

In light of the national demonstrations over the Michael Brown and Eric
Garner cases, it's tempting to describe ``Citizen,'' Claudia Rankine's
latest volume of poetry, as ``timely.'' Even the cover image of a
floating hoodie, its sleeves and torso cut away, seems timely. Any
American viewing it would immediately recall a certain black teenager
who was shot and killed by a neighborhood watch volunteer in February
2012. But this work, by the artist David Hammons, was created in 1993
--- well before Trayvon Martin was even born.

And this seems to be part of Rankine's conceit: What passes as news for
some (white) readers is simply quotidian lived experience for (black)
others.

The challenge of making racism relevant, or even evident, to those who
do not bear the brunt of its ill effects is tricky. Rankine brilliantly
pushes poetry's forms to disarm readers and circumvent our carefully
constructed defense mechanisms against the hint of possibly being racist
ourselves.

To wit, in many of her pieces, it's easy to presume the ``you'' is
always black and the ``she'' or ``he'' is always white, but within a few
pages Rankine begins muddying the personas and pronouns in a way that
forces us to work a little harder. This technique reaches its high point
in a breathless, unpunctuated conclusion to her lament on the Jena Six,
the group of black teenagers charged in Louisiana after the 2006 beating
of a white student: ``Boys will be boys being boys feeling their
capacity heaving butting heads righting their wrongs in the violence of
aggravated adolescence . . . for the other boy for the other boys the
fists the feet criminalized already are weapons already exploding the
landscape and then the litigious hitting back is life imprisoned.''

As she did in her 2004 collection ``Don't Let Me Be Lonely,'' Rankine
again works with a form she calls ``an American Lyric.'' The writing
zigs and zags effortlessly between prose poems, images and essays. This
is the poet as conceptual artist, in full mastery of her craft. And
while the themes of this book could have been mined from any point in
America's history, Rankine sets the whole collection resolutely in the
present. Contemporary content and contemporary form mirror each other.

Rankine has published four previous volumes of poetry in addition to
writing plays, creating videos and editing several anthologies. Her
multidisciplinary ethos colors every page of ``Citizen.''

The book is divided into seven sections with no index or table of
contents. Without titles to separate and ground them, her poetic texts
and images function as fragments of memory, coming into sharp focus,
then blurring. Cumulatively, it's like viewing an experimental film or
live performance. One is left with a mix of emotions that linger and
wend themselves into the subconscious.

There is a lengthy essay on Serena Williams that beautifully unpacks the
``angry black woman'' motif in a way that could also be seen as timely.
And there is a series of ``scripts,'' some created in collaboration with
the photographer John Lucas, that blend text and image to create a kind
of revisionist remix of major media coverage of racialized incidents.

In addition to touching on topics like Trayvon Martin and the Jena Six,
Rankine also cites the British citizen Mark Duggan, whose 2011 death at
the hands of the police prompted riots across London, and broader events
like Hurricane Katrina and the 2006 World Cup. Most are dated, with some
as recent as February and August 2014. The section closes with a tribute
to Jordan Davis, the Florida teenager shot and killed in 2012 by a man
who objected to his rap music, along with a companion piece captioned
``February 15, 2014 / The Justice System.'' Both pages are left blank,
aptly expressing what Davis's parents and so many (black) Americans felt
after the initial ``loud music trial'' verdict --- a stunned silence.
(Jurors couldn't agree on murder charges, leading to a mistrial, but the
shooter was subsequently retried and convicted this fall.)

There are also several evocative internal monologues that offer a kind
of breathing space between the texts grounded in fact. ``You sit down,
you sigh. You stand up, you sigh. The sighing is a worrying exhale of an
ache. You wouldn't call it an illness; still it is not the iteration of
a free being. What else to liken yourself to but an animal, the ruminant
kind?'' At best these monologues capture the liminal quality of being
black and American --- what Du Bois called double consciousness ---
though on occasion they lapse into slack tautology: ``Do feelings lose
their feeling if they speak to a lack of feeling?''

Rankine has for the most part abandoned line breaks; she is like a
painter abandoning representation in order to focus on canvas, color and
light. In her world, enjambment, that poetic technique of allowing a
sentence to run into the next line of poetry, often to create layered
meanings, takes place between poems rather than between lines. An
incident in a drugstore in which a man inadvertently cuts the line for
the cashier because he did not see the narrator flows into a haunting
meditation on Hurricane Katrina that ends with the following lines of
dialogue:

\emph{Call out to them.}

\emph{I don't see them.}

\emph{Call out anyway.}

\emph{Did you see their faces?}

The bulk of the book consists of lyric prose poems, in present tense and
second person. A reasoned, measured tone marks these personal accounts
of everyday racism, events enacted by what the writer Ta-Nehisi Coates
calls ``good people.'' Rankine creates an intentionally disorienting
experience, one that mirrors the experience of racial micro-aggressions
her subjects encounter. Race is both referenced and purposely effaced
within the text.

``The real estate woman, who didn't fathom she could have made an
appointment to show her house to you, spends much of the walk-through
telling your friend, repeatedly, how comfortable she feels around her.
Neither you nor your friend bothers to ask who is making her feel
uncomfortable.''

A chancellor of the Academy of American Poets and a professor at Pomona
College, Rankine seems eager to indicate that her narrators belong to
the cultured, well-­educated set. These episodes take place at private
school, on the way to therapy, in the cabin of a plane. These are the
accomplished black professionals and academics whose lives are often
spent in white circles, and often presumed to be free of the strictures
of race. But Rankine wants us to know that no American citizen is ever
really free of race and racism. The potential to say a racist thing or
think a racist thought resides in all of us like an unearthed mine from
a forgotten war.

``The world is wrong. You can't put the past behind you. It's buried in
you; it's turned your flesh into its own cupboard. Not everything
remembered is useful but it all comes from the world to be stored in
you. . . . Did I hear what I think I heard? Did that just come out of my
mouth, his mouth, your mouth?''

As Rankine pointedly writes, ``Just getting along shouldn't be an
ambition.'' ``Citizen'' throws a Molotov cocktail at the notion that a
reduction of injustice is the same as freedom.

Advertisement

\protect\hyperlink{after-bottom}{Continue reading the main story}

\hypertarget{site-index}{%
\subsection{Site Index}\label{site-index}}

\hypertarget{site-information-navigation}{%
\subsection{Site Information
Navigation}\label{site-information-navigation}}

\begin{itemize}
\tightlist
\item
  \href{https://help.nytimes3xbfgragh.onion/hc/en-us/articles/115014792127-Copyright-notice}{©~2020~The
  New York Times Company}
\end{itemize}

\begin{itemize}
\tightlist
\item
  \href{https://www.nytco.com/}{NYTCo}
\item
  \href{https://help.nytimes3xbfgragh.onion/hc/en-us/articles/115015385887-Contact-Us}{Contact
  Us}
\item
  \href{https://www.nytco.com/careers/}{Work with us}
\item
  \href{https://nytmediakit.com/}{Advertise}
\item
  \href{http://www.tbrandstudio.com/}{T Brand Studio}
\item
  \href{https://www.nytimes3xbfgragh.onion/privacy/cookie-policy\#how-do-i-manage-trackers}{Your
  Ad Choices}
\item
  \href{https://www.nytimes3xbfgragh.onion/privacy}{Privacy}
\item
  \href{https://help.nytimes3xbfgragh.onion/hc/en-us/articles/115014893428-Terms-of-service}{Terms
  of Service}
\item
  \href{https://help.nytimes3xbfgragh.onion/hc/en-us/articles/115014893968-Terms-of-sale}{Terms
  of Sale}
\item
  \href{https://spiderbites.nytimes3xbfgragh.onion}{Site Map}
\item
  \href{https://help.nytimes3xbfgragh.onion/hc/en-us}{Help}
\item
  \href{https://www.nytimes3xbfgragh.onion/subscription?campaignId=37WXW}{Subscriptions}
\end{itemize}
