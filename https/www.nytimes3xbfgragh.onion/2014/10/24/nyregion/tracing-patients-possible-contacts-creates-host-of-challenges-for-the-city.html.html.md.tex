Sections

SEARCH

\protect\hyperlink{site-content}{Skip to
content}\protect\hyperlink{site-index}{Skip to site index}

\href{https://www.nytimes3xbfgragh.onion/section/nyregion}{New York}

\href{https://myaccount.nytimes3xbfgragh.onion/auth/login?response_type=cookie\&client_id=vi}{}

\href{https://www.nytimes3xbfgragh.onion/section/todayspaper}{Today's
Paper}

\href{/section/nyregion}{New York}\textbar{}Tracking One Man's Contacts
in a City of 8 Million

\url{https://nyti.ms/1thCXkU}

\begin{itemize}
\item
\item
\item
\item
\item
\end{itemize}

Advertisement

\protect\hyperlink{after-top}{Continue reading the main story}

Supported by

\protect\hyperlink{after-sponsor}{Continue reading the main story}

\hypertarget{tracking-one-mans-contacts-in-a-city-of-8-million}{%
\section{Tracking One Man's Contacts in a City of 8
Million}\label{tracking-one-mans-contacts-in-a-city-of-8-million}}

\includegraphics{https://static01.graylady3jvrrxbe.onion/images/2014/10/24/nyregion/24NYPATIENT6web/24NYPATIENT6web-articleLarge.jpg?quality=75\&auto=webp\&disable=upscale}

By
\href{http://www.nytimes3xbfgragh.onion/by/anemona-hartocollis}{Anemona
Hartocollis}

\begin{itemize}
\item
  Oct. 23, 2014
\item
  \begin{itemize}
  \item
  \item
  \item
  \item
  \item
  \end{itemize}
\end{itemize}

New York City's first confirmed case of Ebola has raised complicated
logistical issues of how to trace the possible contacts of an infected
patient in a city of more than 8 million people with a sprawling mass
transit system and a large population of workers who commute every day
from surrounding suburbs and states.

By the time the patient, Dr. Craig Spencer, an emergency doctor who had
recently returned from Guinea, arrived at Bellevue Hospital Center in
Manhattan by ambulance on Thursday, he was seriously ill, officials
said.

Dr. Spencer complicated the tracing process when he told health
officials that just the night before, he had gone bowling in Brooklyn,
making the long trip there from his home in Upper Manhattan by subway
and then returning in a car hired via the taxi service Uber.

\includegraphics{https://static01.graylady3jvrrxbe.onion/images/2014/10/23/multimedia/cuomo-ebola/cuomo-ebola-videoSixteenByNine3000.jpg}

City health officials were suddenly faced with the challenge of finding
the right balance between trying to find everyone who might have been
exposed and responding to a disease that is transmitted only through
direct exposure to bodily fluids.

It was soon clear that health authorities had other worries, as word
emerged that they were isolating not just Dr. Spencer's fiancée but also
two friends who had been with him in the two days before he arrived at
the hospital. Dr. Spencer said he had started feeling sluggish on
Tuesday.

City officials were making plans to provide case managers for every
family or person who might need to be quarantined. Those managers would
help with the chores of daily life, such as providing school materials
for children or food for people confined to their homes.

New York has some advantage in that it may be able to learn from what
happened in Dallas, where two nurses became infected with Ebola after
treating Thomas E. Duncan, the patient with the first case of Ebola to
be diagnosed in the United States who died on Oct. 8.

\href{https://www.nytimes3xbfgragh.onion/interactive/2014/10/23/nyregion/new-york-city-ebola-patient-timeline-map.html}{}

\includegraphics{https://static01.graylady3jvrrxbe.onion/images/2014/10/23/nyregion/new-york-city-ebola-patient-map-1414137847417/new-york-city-ebola-patient-map-1414137847417-videoLarge.png}

\hypertarget{what-the-new-york-city-ebola-patient-was-doing-before-he-was-hospitalized}{%
\subsection{What the New York City Ebola Patient Was Doing Before He Was
Hospitalized}\label{what-the-new-york-city-ebola-patient-was-doing-before-he-was-hospitalized}}

Locations visited by Craig Spencer, a Manhattan doctor who has tested
positive for Ebola.

Israel Miranda, president of the union of uniformed emergency medical
technicians and paramedics, said on Thursday that he was satisfied with
the way Dr. Spencer's transportation to the hospital had been handled.

Two ambulances responded, and two paramedics fully encased in protective
suits brought Dr. Spencer out of his apartment on West 147th Street; two
others who were not in suits drove.

When the paramedics left the hospital, their suits were sprayed with
disinfectant and cut off from behind by a special unit, Mr. Miranda
said. The ambulance was also decontaminated.

``The suit was peeled off them like an onion,'' he said. ``So everything
went by the book.''

He said the paramedics would have their temperatures taken twice a day
for 21 days to ensure they had not been infected, but in the meantime
they would be free to continue responding to calls because ``there was
no breach.''

Soothing the fears of those who may have been at the Gutter, the
Brooklyn bowling alley Dr. Spencer visited, or who might have ridden in
a subway car with him could well be more challenging.

In the Dallas case, the Frontier Airlines plane on which an infected
nurse flew was taken out of service and decontaminated in a process that
included replacing seat covers and the carpet area near the nurse's
seat. The airline said it had gone to great lengths to ease customer's
fears.

Yet cruise ship officials were widely ridiculed for overreacting when
they decontaminated a ship on which a lab technician traveled after
coming into contact with specimens taken from Mr. Duncan before his
death.

Dr. Spencer has been isolated in a seventh-floor ward at Bellevue, the
city's main public hospital, that was specially designed to treat highly
infectious tuberculosis patients. The unit is locked and guarded, with
rooms where health care workers can be decontaminated and cameras can
monitor patients remotely.

Advertisement

\protect\hyperlink{after-bottom}{Continue reading the main story}

\hypertarget{site-index}{%
\subsection{Site Index}\label{site-index}}

\hypertarget{site-information-navigation}{%
\subsection{Site Information
Navigation}\label{site-information-navigation}}

\begin{itemize}
\tightlist
\item
  \href{https://help.nytimes3xbfgragh.onion/hc/en-us/articles/115014792127-Copyright-notice}{©~2020~The
  New York Times Company}
\end{itemize}

\begin{itemize}
\tightlist
\item
  \href{https://www.nytco.com/}{NYTCo}
\item
  \href{https://help.nytimes3xbfgragh.onion/hc/en-us/articles/115015385887-Contact-Us}{Contact
  Us}
\item
  \href{https://www.nytco.com/careers/}{Work with us}
\item
  \href{https://nytmediakit.com/}{Advertise}
\item
  \href{http://www.tbrandstudio.com/}{T Brand Studio}
\item
  \href{https://www.nytimes3xbfgragh.onion/privacy/cookie-policy\#how-do-i-manage-trackers}{Your
  Ad Choices}
\item
  \href{https://www.nytimes3xbfgragh.onion/privacy}{Privacy}
\item
  \href{https://help.nytimes3xbfgragh.onion/hc/en-us/articles/115014893428-Terms-of-service}{Terms
  of Service}
\item
  \href{https://help.nytimes3xbfgragh.onion/hc/en-us/articles/115014893968-Terms-of-sale}{Terms
  of Sale}
\item
  \href{https://spiderbites.nytimes3xbfgragh.onion}{Site Map}
\item
  \href{https://help.nytimes3xbfgragh.onion/hc/en-us}{Help}
\item
  \href{https://www.nytimes3xbfgragh.onion/subscription?campaignId=37WXW}{Subscriptions}
\end{itemize}
