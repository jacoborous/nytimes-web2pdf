Sections

SEARCH

\protect\hyperlink{site-content}{Skip to
content}\protect\hyperlink{site-index}{Skip to site index}

\href{https://www.nytimes3xbfgragh.onion/section/business}{International
Business}

\href{https://myaccount.nytimes3xbfgragh.onion/auth/login?response_type=cookie\&client_id=vi}{}

\href{https://www.nytimes3xbfgragh.onion/section/todayspaper}{Today's
Paper}

\href{/section/business}{International Business}\textbar{}Samsung's
Uneven Handling of Galaxy Note 7 Fires Angers Chinese

\url{https://nyti.ms/2eglLsI}

\begin{itemize}
\item
\item
\item
\item
\item
\end{itemize}

Advertisement

\protect\hyperlink{after-top}{Continue reading the main story}

Supported by

\protect\hyperlink{after-sponsor}{Continue reading the main story}

\hypertarget{samsungs-uneven-handling-of-galaxy-note-7-fires-angers-chinese}{%
\section{Samsung's Uneven Handling of Galaxy Note 7 Fires Angers
Chinese}\label{samsungs-uneven-handling-of-galaxy-note-7-fires-angers-chinese}}

\includegraphics{https://static01.graylady3jvrrxbe.onion/images/2016/10/18/multimedia/19samsungchina-fire/19samsungchina-fire-videoSixteenByNineJumbo1600-v2.jpg}

By \href{https://www.nytimes3xbfgragh.onion/by/sui-lee-wee}{Sui-Lee Wee}

\begin{itemize}
\item
  Oct. 18, 2016
\item
  \begin{itemize}
  \item
  \item
  \item
  \item
  \item
  \end{itemize}
\end{itemize}

TIANJIN, China --- Zhang Sitong was saving a friend's phone number on
his Samsung Galaxy Note 7 smartphone when it started to vibrate and
smoke. He threw it on the ground and told his friend to start filming.

Two employees from Samsung Electronics showed up at his house later that
day, he said, offering a new Note 7 and about \$900 in compensation on
the condition that he keep the video private. Mr. Zhang angrily refused.
Only weeks before, even as Samsung recalled more than two million Note
7s in the United States and elsewhere, the company had reassured him and
other Chinese customers that the phone was safe.

``They said there was no problem with the phones in China. That's why I
bought a Samsung,'' said Mr. Zhang, a 23-year-old former firefighter.
``This is an issue of deception. They are cheating Chinese consumers.''

Samsung, already reeling from its embarrassing and expensive decision
last week to
\href{http://www.nytimes3xbfgragh.onion/2016/10/12/business/international/samsung-galaxy-note7-terminated.html}{kill
the Note 7}, has a particularly vexing problem in China. On Tuesday,
China's powerful state-run broadcaster, China Central Television, or
CCTV, criticized the way Samsung tested its phones and asked whether its
claims that the phones were safe and reliable were ``fabricated
falsehoods.''

``If Samsung continues to violate the legitimate rights and interests of
Chinese consumers and continues to refuse to make public the samples
used in its testing process as well as the process itself, who would be
able to help Chinese consumers find the truth?'' the CCTV report said.

The South Korean company is paying a price for treating China
differently. Samsung initially said the Chinese version of the Note 7
\href{http://www.nytimes3xbfgragh.onion/2016/09/03/business/samsung-galaxy-note-battery.html}{had
a different battery} and was safe. But last week, after reports in China
of phones catching fire, it finally recalled the Note 7 there before it
scrapped the phone globally.

``The brand has been damaged already,'' said Di Jin, research manager in
China for IDC, a technology research firm. ``It will be really hard for
Samsung to regain its market share in the near future.''

In a statement, Samsung said it ``would like to apologize for any
misunderstandings this may have caused the Chinese consumers due to an
unclear communication in the process.'' It said its quality control was
the same in all countries. ``To Samsung, China is one of the most
important markets and a crucial destination for foreign investment,'' it
said. ``Samsung never holds a double standard against them.''

Samsung was once the top phone maker in China, which is now the world's
largest smartphone market. It enjoyed a reputation for quality in a
country weary of cheap products, as well as the benefit of a Korean name
in a country that
\href{http://www.nytimes3xbfgragh.onion/2015/07/21/arts/television/chinas-love-affair-with-irresistible-korean-tv.html}{adores}
Korean pop music and culture.

\includegraphics{https://static01.graylady3jvrrxbe.onion/images/2016/10/19/business/19SAMSUNGCHINA/19SAMSUNGCHINA-articleLarge.jpg?quality=75\&auto=webp\&disable=upscale}

But Samsung has become the latest global brand to get hit by rising
Chinese rivals that compete on quality as well as price. Samsung has
lost market share in China to
\href{http://bits.blogs.nytimes3xbfgragh.onion/2014/05/07/huawei-unveils-new-phone-to-compete-with-apple-and-samsung/}{Huawei
Technologies},
\href{http://www.nytimes3xbfgragh.onion/2014/12/15/technology/the-rise-of-a-new-smartphone-giant-chinas-xiaomi.html}{Xiaomi},
Oppo Electronics and other local companies making competitive gadgets
with advanced features. The South Korean company's market share dropped
to less than 7 percent in the second quarter of this year from nearly 19
percent in 2013, according to IDC.

The treatment of Chinese consumers by foreign firms is an issue that
resonates in a country where nationalistic sentiment runs deep. The
state news media has vilified foreign brands such as McDonald's, KFC,
Apple and Starbucks for what they perceive as unequal treatment of
Chinese customers.

``Foreign companies who appear to employ any less favorable policy for
the China market can quickly find themselves waist-deep in a P.R.
quagmire,'' said Mark Natkin, the managing director of Marbridge
Consulting, an advisory firm based in Beijing.

``Those who have navigated the Chinese market most successfully are the
companies that have understood they can't win every battle,'' he said,
``and that sometimes, to maintain a happy relationship, it's better just
to say: `I'm sorry. I love you.' ''

The official news media in China has become increasingly critical of
both foreign and domestic companies as China's consumer culture has
grown. The push culminates each year in March, when CCTV takes on
companies for their practices as part of
\href{http://www.consumersinternational.org/our-work/wcrd/about-wcrd/}{Consumer
Rights Day} in
\href{http://english.cntv.cn/2015/03/16/VIDE1426482244132567.shtml}{a
prime-time special} complete with song-and-dance routines.

Foreign brands have a mixed track record dealing with attacks from the
official Chinese news media.
\href{http://www.nytimes3xbfgragh.onion/2013/04/02/technology/apples-chief-tim-cook-apologizes-to-china-over-warranty-policy.html}{Apple
apologized} after CCTV criticized its warranty policies but went on to
enjoy
\href{http://www.nytimes3xbfgragh.onion/2015/08/25/technology/tim-cook-of-apple-seeks-to-quell-china-fears-in-email-to-jim-cramer.html}{strong
sales} in the following years. Starbucks continues to move lattes there
\href{http://sinosphere.blogs.nytimes3xbfgragh.onion/2013/10/21/state-media-call-starbucks-too-pricey/}{at
a brisk pace} despite CCTV's criticism of its pricing three years ago.
But KFC's parent, Yum Brands,
\href{http://www.nytimes3xbfgragh.onion/2013/02/06/business/global/kfc-parent-suffers-after-china-scandal.html}{took
a hit} after CCTV scrutinized its suppliers.

Mr. Zhang, a salesman in the city of Fushun, in northeastern China, was
a Samsung loyalist. He has owned four smartphones made by the company,
in part because of the pen-type stylus that comes with its Note models.

After he rejected the offer from Samsung, Mr. Zhang quit his job and hit
the road. He joined up with Hui Renjie, another man who said his Note 7
had also blown up, to visit laboratories to figure out the problem with
their phones. The trip and the testing were paid for by CCTV, which
featured the two in Tuesday's report.

In the report, CCTV said an independent lab could not determine the
cause of the fire that consumed Mr. Zhang's phone. It said an external
heat source was not responsible for the fire that destroyed Mr. Hui's
phone.

Mr. Hui said Samsung had declined his repeated requests to conduct an
investigation into the cause of his phone catching on fire in his
presence and had ignored his calls.

According to Mr. Hui, a representative from the e-commerce site JD.com,
where he had bought his phone, told him that Samsung had offered to
compensate him for his burned phone and laptop, which he had ruined
after throwing his phone on it. Mr. Hui, in response, said, ``I told JD
to pass this message to Samsung: `Go to hell.' ''

Advertisement

\protect\hyperlink{after-bottom}{Continue reading the main story}

\hypertarget{site-index}{%
\subsection{Site Index}\label{site-index}}

\hypertarget{site-information-navigation}{%
\subsection{Site Information
Navigation}\label{site-information-navigation}}

\begin{itemize}
\tightlist
\item
  \href{https://help.nytimes3xbfgragh.onion/hc/en-us/articles/115014792127-Copyright-notice}{©~2020~The
  New York Times Company}
\end{itemize}

\begin{itemize}
\tightlist
\item
  \href{https://www.nytco.com/}{NYTCo}
\item
  \href{https://help.nytimes3xbfgragh.onion/hc/en-us/articles/115015385887-Contact-Us}{Contact
  Us}
\item
  \href{https://www.nytco.com/careers/}{Work with us}
\item
  \href{https://nytmediakit.com/}{Advertise}
\item
  \href{http://www.tbrandstudio.com/}{T Brand Studio}
\item
  \href{https://www.nytimes3xbfgragh.onion/privacy/cookie-policy\#how-do-i-manage-trackers}{Your
  Ad Choices}
\item
  \href{https://www.nytimes3xbfgragh.onion/privacy}{Privacy}
\item
  \href{https://help.nytimes3xbfgragh.onion/hc/en-us/articles/115014893428-Terms-of-service}{Terms
  of Service}
\item
  \href{https://help.nytimes3xbfgragh.onion/hc/en-us/articles/115014893968-Terms-of-sale}{Terms
  of Sale}
\item
  \href{https://spiderbites.nytimes3xbfgragh.onion}{Site Map}
\item
  \href{https://help.nytimes3xbfgragh.onion/hc/en-us}{Help}
\item
  \href{https://www.nytimes3xbfgragh.onion/subscription?campaignId=37WXW}{Subscriptions}
\end{itemize}
