Sections

SEARCH

\protect\hyperlink{site-content}{Skip to
content}\protect\hyperlink{site-index}{Skip to site index}

\href{https://www.nytimes3xbfgragh.onion/section/politics}{Politics}

\href{https://myaccount.nytimes3xbfgragh.onion/auth/login?response_type=cookie\&client_id=vi}{}

\href{https://www.nytimes3xbfgragh.onion/section/todayspaper}{Today's
Paper}

\href{/section/politics}{Politics}\textbar{}Donald Trump Tax Records
Show He Could Have Avoided Taxes for Nearly Two Decades, The Times Found

\url{https://nyti.ms/2dupGDz}

\begin{itemize}
\item
\item
\item
\item
\item
\item
\end{itemize}

Advertisement

\protect\hyperlink{after-top}{Continue reading the main story}

Supported by

\protect\hyperlink{after-sponsor}{Continue reading the main story}

\hypertarget{donald-trump-tax-records-show-he-could-have-avoided-taxes-for-nearly-two-decades-the-times-found}{%
\section{Donald Trump Tax Records Show He Could Have Avoided Taxes for
Nearly Two Decades, The Times
Found}\label{donald-trump-tax-records-show-he-could-have-avoided-taxes-for-nearly-two-decades-the-times-found}}

\includegraphics{https://static01.graylady3jvrrxbe.onion/images/2016/10/02/us/01trumptaxes1/01trumptaxes1-videoSixteenByNine3000.jpg}

By \href{https://www.nytimes3xbfgragh.onion/by/david-barstow}{David
Barstow},
\href{https://www.nytimes3xbfgragh.onion/by/susanne-craig}{Susanne
Craig}, \href{https://www.nytimes3xbfgragh.onion/by/russ-buettner}{Russ
Buettner} and
\href{https://www.nytimes3xbfgragh.onion/by/megan-twohey}{Megan Twohey}

\begin{itemize}
\item
  Oct. 1, 2016
\item
  \begin{itemize}
  \item
  \item
  \item
  \item
  \item
  \item
  \end{itemize}
\end{itemize}

Donald J.
\href{https://www.nytimes3xbfgragh.onion/2020/07/09/us/politics/trump-taxes.html}{Trump}
declared a \$916 million loss on his 1995 income
\href{https://www.nytimes3xbfgragh.onion/2020/07/15/nyregion/donald-trump-taxes-cyrus-vance.html}{tax
returns}, a tax deduction so substantial it could have allowed him to
legally avoid paying any federal income taxes for up to 18 years,
records obtained by The New York Times show.

The
\href{http://www.nytimes3xbfgragh.onion/interactive/2016/10/01/us/politics/donald-trump-taxes.html}{1995
tax records}, never before disclosed, reveal the extraordinary tax
benefits that Mr. Trump, the Republican presidential nominee, derived
from the financial wreckage he left behind in the early 1990s through
mismanagement of three Atlantic City casinos, his ill-fated foray into
the airline business and his ill-timed purchase of the Plaza Hotel in
Manhattan.

Tax experts hired by The Times to analyze Mr. Trump's 1995 records said
that tax rules especially advantageous to wealthy filers would have
allowed Mr. Trump to use his \$916 million loss to cancel out an
equivalent amount of taxable income over an 18-year period.

\includegraphics{https://static01.graylady3jvrrxbe.onion/images/2016/10/01/us/01TRUMPTAXES-DOC-RIP-SUB/01TRUMPTAXES-DOC-RIP-SUB-articleInline.jpg?quality=75\&auto=webp\&disable=upscale}

Although Mr. Trump's taxable income in subsequent years is as yet
unknown, a \$916 million loss in 1995 would have been large enough to
wipe out more than \$50 million a year in taxable income over 18 years.

The \$916 million loss certainly could have eliminated any federal
income taxes Mr. Trump otherwise would have owed on the
\href{http://www.nytimes3xbfgragh.onion/2016/07/17/business/media/donald-trump-apprentice.html?_r=1}{\$50,000
to \$100,000 he was paid for each episode of ``The Apprentice,''} or the
roughly \$45 million he was paid between 1995 and 2009 when he was
chairman or chief executive of the publicly traded company he created to
assume ownership of his troubled Atlantic City casinos. Ordinary
investors in the new company, meanwhile, saw the value of their shares
plunge to 17 cents from \$35.50, while scores of contractors went unpaid
for work on Mr. Trump's casinos and casino bondholders received pennies
on the dollar.

``He has a vast benefit from his destruction'' in the early 1990s, said
one of the experts, Joel Rosenfeld, an assistant professor at New York
University's Schack Institute of Real Estate. Mr. Rosenfeld offered this
description of what he would advise a client who came to him with a tax
return like Mr. Trump's: ``Do you realize you can create \$916 million
in income without paying a nickel in taxes?''

Mr. Trump declined to comment on the documents. Instead, the campaign
released a
\href{http://www.nytimes3xbfgragh.onion/interactive/2016/10/01/us/politics/donald-trump-letter.html}{statement}
that neither challenged nor confirmed the \$916 million loss.

``Mr. Trump is a highly-skilled businessman who has a fiduciary
responsibility to his business, his family and his employees to pay no
more tax than legally required,'' the statement said. ``That being said,
Mr. Trump has paid hundreds of millions of dollars in property taxes,
sales and excise taxes, real estate taxes, city taxes, state taxes,
employee taxes and federal taxes.''

The statement continued, ``Mr. Trump knows the tax code far better than
anyone who has ever run for President and he is the only one that knows
how to fix it.''

\href{https://www.nytimes3xbfgragh.onion/interactive/2016/10/01/us/politics/donald-trump-letter.html}{}

\includegraphics{https://static01.graylady3jvrrxbe.onion/images/2016/10/01/us/politics/donald-trump-letter/donald-trump-letter-largeHorizontalJumbo-v2.gif}

\hypertarget{donald-trumps-letter}{%
\subsection{Donald Trump's Letter}\label{donald-trumps-letter}}

In response to a Times article revealing pages of his 1995 income tax
records, Mr. Trump said, ``The only news here is that the more than 20
year-old alleged tax document was illegally obtained.''

Separately, a lawyer for Mr. Trump, Marc E. Kasowitz, emailed a letter
to The Times arguing that publication of the records is illegal because
Mr. Trump has not authorized the disclosure of any of his tax returns.
Mr. Kasowitz threatened ``prompt initiation of appropriate legal
action.''

Mr. Trump's refusal to make his tax returns public --- breaking with
decades of tradition in presidential contests --- has emerged as a
central issue in the campaign, with a majority of voters saying he
should release them. Mr. Trump has declined to do so, and has said he is
being audited by the Internal Revenue Service.

At last Monday's presidential debate, when Hillary Clinton suggested Mr.
Trump was refusing to release his tax returns so voters would not know
``he's paid nothing in federal taxes,'' and when she also pointed out
that Mr. Trump had once revealed to casino regulators that he paid no
federal income taxes in the late 1970s, Mr. Trump retorted, ``That makes
me smart.''

\includegraphics{https://static01.graylady3jvrrxbe.onion/images/2016/10/02/us/01trumptax-video2/01trumptax-video2-videoSixteenByNineJumbo1600.jpg}

The tax experts consulted by The Times said nothing in the 1995
documents suggested any wrongdoing by Mr. Trump, even if the
extraordinary size of the loss he declared would have probably attracted
extra scrutiny from I.R.S. examiners. ``The I.R.S., when they see a
negative \$916 million, that has to pop out,'' Mr. Rosenfeld said.

The documents examined by The Times represent a small fraction of the
voluminous tax returns Mr. Trump would have filed in 1995.

The documents consisted of three pages from what appeared to be Mr.
Trump's 1995 tax returns. The pages were mailed last month to Susanne
Craig, a reporter at The Times who has written about Mr. Trump's
finances. The documents were the first page of a New York State resident
income tax return, the first page of a New Jersey nonresident tax return
and the first page of a Connecticut nonresident tax return. Each page
bore the names and Social Security numbers of Mr. Trump and Marla
Maples, his wife at the time. Only the New Jersey form had what appeared
to be their signatures.

The three documents arrived by mail at The Times with a postmark
indicating they had been sent from New York City. The return address
claimed the envelope had been sent from Trump Tower.

\href{https://www.nytimes3xbfgragh.onion/interactive/2016/10/01/us/politics/donald-trump-taxes.html}{}

\includegraphics{https://static01.graylady3jvrrxbe.onion/images/2016/10/01/us/politics/donald-trump-taxes/donald-trump-taxes-articleLarge-v2.gif}

\hypertarget{pages-from-donald-trumps-1995-income-tax-records}{%
\subsection{Pages From Donald Trump's 1995 Income Tax
Records}\label{pages-from-donald-trumps-1995-income-tax-records}}

The tax records obtained by The Times show that Donald J. Trump claimed
a \$916 million loss that could have allowed him to legally avoid paying
federal income taxes for up to 18 years.

On Wednesday, The Times presented the tax documents to Jack Mitnick, a
lawyer and certified public accountant who handled Mr. Trump's tax
matters for more than 30 years, until 1996. Mr. Mitnick was listed as
the preparer on the New Jersey tax form.

Mr. Mitnick, 80, now semiretired and living in Florida, said that while
he no longer had access to Mr. Trump's original returns, the documents
appeared to be authentic copies of portions of Mr. Trump's 1995 tax
returns. Mr. Mitnick said the signature on the tax preparer line of the
New Jersey tax form was his, and he readily explained an obvious anomaly
in the way especially large numbers appeared on the New York tax
document.

A flaw in the tax software program he used at the time prevented him
from being able to print a nine-figure loss on Mr. Trump's New York
return, he said. So, for example, the loss of ``-915,729,293'' on Line
18 of the return printed out as ``5,729,293.'' As a result, Mr. Mitnick
recalled, he had to use his typewriter to manually add the ``-91,'' thus
explaining why the first two digits appeared to be in a different font
and were slightly misaligned from the following seven digits.

``This is legit,'' he said, stabbing a finger into the document.

\href{https://www.nytimes3xbfgragh.onion/interactive/2016/10/07/us/elections/donald-trump-tax-advantages-deductions-losses.html}{}

\includegraphics{https://static01.graylady3jvrrxbe.onion/images/2016/10/05/us/elections/donald-trump-tax-advantages-deductions-losses-1475678438005/donald-trump-tax-advantages-deductions-losses-1475678438005-square640.png}

\hypertarget{how-donald-trump-uses-the-tax-code-in-ways-you-cant}{%
\subsection{How Donald Trump Uses the Tax Code in Ways You
Can't}\label{how-donald-trump-uses-the-tax-code-in-ways-you-cant}}

Real estate developers like Donald J. Trump can combine a number of
breaks in the tax code to generate large deductions, which could have
allowed him to avoid paying any federal income tax for decades.

Because the documents sent to The Times did not include any pages from
Mr. Trump's 1995 federal tax return, it is impossible to determine how
much he may have donated to charity that year. The state documents do
show, though, that Mr. Trump declined the opportunity to contribute to
the New Jersey Vietnam Veterans' Memorial Fund, the New Jersey Wildlife
Conservation Fund or the Children's Trust Fund. He also declined to
contribute \$1 toward public financing of New Jersey's elections for
governor.

The tax documents also do not shed any light on Mr. Trump's claimed net
worth of about \$2 billion at that time. This is because the complex
calculations of business deductions that produced a tax loss of \$916
million are a separate matter from how Mr. Trump valued his assets, the
tax experts said.

Nor does the \$916 million loss suggest that Mr. Trump was insolvent or
effectively bankrupt in 1995. The cash flow generated by his various
businesses that year was more than enough to service his various debts.

But fragmentary as they are, the documents nonetheless provide new
insight into Mr. Trump's finances, a subject of intense scrutiny given
Mr. Trump's emphasis on his business record during the presidential
campaign.

The documents show, for example, that while Mr. Trump reported \$7.4
million in interest income in 1995, he made only \$6,108 in wages,
salaries and tips. They also suggest Mr. Trump took full advantage of
generous tax loopholes specifically available to commercial real estate
developers to claim a \$15.8 million loss in 1995 on his real estate
holdings and partnerships.

But the most important revelation from the 1995 tax documents is just
how much Mr. Trump may have benefited from a tax provision that is
particularly prized by America's dynastic families, which, like the
Trumps, hold their wealth inside byzantine networks of partnerships,
limited liability companies and S corporations.

The provision, known as net operating loss, or N.O.L., allows a dizzying
array of deductions, business expenses, real estate depreciation, losses
from the sale of business assets and even operating losses to flow from
the balance sheets of those partnerships, limited liability companies
and S corporations onto the personal tax returns of men like Mr. Trump.
In turn, those losses can be used to cancel out an equivalent amount of
taxable income from, say, book royalties or branding deals.

\href{https://www.nytimes3xbfgragh.onion/interactive/2016/us/politics/donald-trump-taxes-explained.html}{}

\includegraphics{https://static01.graylady3jvrrxbe.onion/images/2016/10/01/us/02TRUMPTAXES-listy/02TRUMPTAXES-listy-largeHorizontalJumbo.jpg}

\hypertarget{donald-trumps-taxes-what-we-know-and-dont-know}{%
\subsection{Donald Trump's Taxes: What We Know and Don't
Know}\label{donald-trumps-taxes-what-we-know-and-dont-know}}

In the absence of any disclosures from Mr. Trump, The New York Times and
other news outlets have attempted to fill in the gaps.

Better still, if the losses are big enough, they can cancel out taxable
income earned in other years. Under I.R.S. rules in 1995, net operating
losses could be used to wipe out taxable income earned in the three
years before and the 15 years after the loss. (The effect of net
operating losses on state income taxes varies, depending on each state's
tax regime.)

The tax experts consulted by The Times said the \$916 million net
operating loss declared by Mr. Trump in 1995 almost certainly included
large net operating losses carried forward from the early 1990s, when
most of Mr. Trump's key holdings were hemorrhaging money. Indeed, by
1990, his entire business empire was on the verge of collapse. In a few
short years, he had amassed \$3.4 billion in debt --- personally
guaranteeing \$832 million of it --- to assemble a portfolio that
included three casinos and a hotel in Atlantic City, the Plaza Hotel in
Manhattan, an airline and a huge yacht.

Reports that year by New Jersey casino regulators gave glimpses of the
balance sheet carnage. The Trump Taj Mahal casino reported a \$25.5
million net loss during its first six months of 1990; the Trump's Castle
casino lost \$43.5 million for the year. His airline, Trump Shuttle,
lost \$34.5 million during just the first six months of that year.

``Simply put, the organization is in dire financial straits,'' the
casino regulators concluded.

Image

Reports published by New Jersey regulators in 1993, top, and 1995,
above, highlighted the effects of Mr. Trump's net operating losses.

Reports by New Jersey's casino regulators strongly suggested that Mr.
Trump had claimed large net operating losses on his taxes in the early
1990s. Their reports, for example, revealed that Mr. Trump had carried
forward net operating losses in both 1991 and 1993. What's more, the
reports said the losses he claimed were large enough to virtually cancel
out any taxes he might owe on the millions of dollars of debt that was
being forgiven by his creditors. (The I.R.S. considers forgiven debt to
be taxable income.)

But crucially, the casino regulators redacted the precise size of the
net operating losses in the public versions of their reports. Two former
New Jersey officials, who were privy to the unredacted documents, could
not recall the precise size of the numbers, but said they were
substantial.

\href{http://www.politico.com/story/2016/06/donald-trump-no-taxes-224498}{Politico,
which previously reported} that Mr. Trump most likely paid no income
taxes in 1991 and 1993 based on the casino commission's description of
his net operating losses, asked Mr. Trump to comment. ``Welcome to the
real estate business,'' he replied in an email.

Now, thanks to Mr. Trump's 1995 tax records, the degree to which he spun
all those years of red ink into tax write-off gold may finally be
apparent.

Mr. Mitnick, the lawyer and accountant, was the person Mr. Trump leaned
on most to do the spinning. Mr. Mitnick worked for a small Long Island
accounting firm that specialized in handling tax issues for wealthy New
York real estate families. He had long handled tax matters for Mr.
Trump's father, Fred C. Trump, and he said he began doing Donald Trump's
taxes after Mr. Trump turned 18.

In an interview on Wednesday, Mr. Mitnick said he could not divulge
details of Mr. Trump's finances without Mr. Trump's consent. But he did
talk about Mr. Trump's approaches to taxes, and he contrasted Fred
Trump's attention to detail with what he described as Mr. Trump's brash
and undisciplined style. He recalled, for example, that when Donald and
Ivana Trump came in each year to sign their tax forms, it was almost
always Ivana who asked more questions.

Image

The Trump Plaza Hotel and Casino, one of the failed casino properties in
Atlantic City that had been owned by Mr. Trump.Credit...Douglas
Graham/Congressional Quarterly, via Getty Images

But if Mr. Trump lacked a sophisticated understanding of the tax code,
and if he rarely showed any interest in the details behind various tax
strategies, Mr. Mitnick said he clearly grasped the critical role taxes
would play in helping him build wealth. ``He knew we could use the tax
code to protect him,'' Mr. Mitnick said.

According to Mr. Mitnick, Mr. Trump's use of net operating losses was no
different from that of his other wealthy clients. ``This may have had a
couple extra digits compared to someone else's operation, but they all
benefited in the same way,'' he said, pointing to the \$916 million loss
on Mr. Trump's tax returns.

In ``The Art of the Deal,'' his 1987 best-selling book, Mr. Trump
referred to Mr. Mitnick as ``my accountant'' --- although he misspelled
his name. Mr. Trump described consulting with Mr. Mitnick on the tax
implications of deals he was contemplating and seeking his advice on how
new federal tax regulations might affect real estate write-offs.

Mr. Mitnick, though, said there were times when even he, for all his
years helping wealthy New Yorkers navigate the tax code, found it
difficult to face the incongruity of his work for Mr. Trump. He felt
keenly aware that Mr. Trump was living a life of unimaginable luxury
thanks in part to Mr. Mitnick's ability to relieve him of the burden of
paying taxes like everyone else.

``Here the guy was building incredible net worth and not paying tax on
it,'' he said.

Advertisement

\protect\hyperlink{after-bottom}{Continue reading the main story}

\hypertarget{site-index}{%
\subsection{Site Index}\label{site-index}}

\hypertarget{site-information-navigation}{%
\subsection{Site Information
Navigation}\label{site-information-navigation}}

\begin{itemize}
\tightlist
\item
  \href{https://help.nytimes3xbfgragh.onion/hc/en-us/articles/115014792127-Copyright-notice}{©~2020~The
  New York Times Company}
\end{itemize}

\begin{itemize}
\tightlist
\item
  \href{https://www.nytco.com/}{NYTCo}
\item
  \href{https://help.nytimes3xbfgragh.onion/hc/en-us/articles/115015385887-Contact-Us}{Contact
  Us}
\item
  \href{https://www.nytco.com/careers/}{Work with us}
\item
  \href{https://nytmediakit.com/}{Advertise}
\item
  \href{http://www.tbrandstudio.com/}{T Brand Studio}
\item
  \href{https://www.nytimes3xbfgragh.onion/privacy/cookie-policy\#how-do-i-manage-trackers}{Your
  Ad Choices}
\item
  \href{https://www.nytimes3xbfgragh.onion/privacy}{Privacy}
\item
  \href{https://help.nytimes3xbfgragh.onion/hc/en-us/articles/115014893428-Terms-of-service}{Terms
  of Service}
\item
  \href{https://help.nytimes3xbfgragh.onion/hc/en-us/articles/115014893968-Terms-of-sale}{Terms
  of Sale}
\item
  \href{https://spiderbites.nytimes3xbfgragh.onion}{Site Map}
\item
  \href{https://help.nytimes3xbfgragh.onion/hc/en-us}{Help}
\item
  \href{https://www.nytimes3xbfgragh.onion/subscription?campaignId=37WXW}{Subscriptions}
\end{itemize}
