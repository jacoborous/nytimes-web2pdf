Sections

SEARCH

\protect\hyperlink{site-content}{Skip to
content}\protect\hyperlink{site-index}{Skip to site index}

\href{https://www.nytimes3xbfgragh.onion/section/world/middleeast}{Middle
East}

\href{https://myaccount.nytimes3xbfgragh.onion/auth/login?response_type=cookie\&client_id=vi}{}

\href{https://www.nytimes3xbfgragh.onion/section/todayspaper}{Today's
Paper}

\href{/section/world/middleeast}{Middle East}\textbar{}U.S. Allows
Boeing and Airbus to Sell Planes to Iran

\url{https://nyti.ms/2d0XLLw}

\begin{itemize}
\item
\item
\item
\item
\item
\end{itemize}

Advertisement

\protect\hyperlink{after-top}{Continue reading the main story}

Supported by

\protect\hyperlink{after-sponsor}{Continue reading the main story}

\hypertarget{us-allows-boeing-and-airbus-to-sell-planes-to-iran}{%
\section{U.S. Allows Boeing and Airbus to Sell Planes to
Iran}\label{us-allows-boeing-and-airbus-to-sell-planes-to-iran}}

\includegraphics{https://static01.graylady3jvrrxbe.onion/images/2016/09/22/world/22IRAN/22IRAN-articleInline.jpg?quality=75\&auto=webp\&disable=upscale}

By \href{http://www.nytimes3xbfgragh.onion/by/thomas-erdbrink}{Thomas
Erdbrink} and
\href{http://www.nytimes3xbfgragh.onion/by/nicola-clark}{Nicola Clark}

\begin{itemize}
\item
  Sept. 21, 2016
\item
  \begin{itemize}
  \item
  \item
  \item
  \item
  \item
  \end{itemize}
\end{itemize}

TEHRAN --- The United States on Wednesday removed a final hurdle for
Western aircraft manufacturers to sell planes to Iran, a country
desperately in need of hundreds of new aircraft.

The Treasury Department granted the aviation giants Airbus and Boeing
licenses to deliver planes to Tehran. The decision is a boon not only
for the two companies but also for Iranian politicians who want to
expand Iran's engagement with the world now that sanctions linked to
Iran's
\href{http://topics.nytimes3xbfgragh.onion/top/news/international/countriesandterritories/iran/nuclear_program/index.html?inline=nyt-classifier}{nuclear
program} have been lifted.

A spokesman for Boeing said the license covered the sale of 80 planes to
Iran's national carrier, Iran Air. Airbus confirmed that it received a
license for an initial sale of 17 planes, part of a larger deal that
involves a total of 118 planes.

The green light for aircraft sales allows Iran, a country of 80 million,
to start rebuilding its aging fleet of Boeing and Airbus planes and
other secondhand aircraft purchased clandestinely from other countries.
Over the past four decades, hundreds of Iranians have died in crashes
caused by malfunctioning or poorly maintained aircraft.

``From today, we will have safe planes,'' President Hassan Rouhani of
Iran promised in January when the accord between Iran and six world
powers, including the United States, became
\href{http://www.nytimes3xbfgragh.onion/2016/01/17/world/middleeast/iran-sanctions-lifted-nuclear-deal.html}{fully
operational}.

Under that deal, Iran has given up parts of its nuclear program in
exchange for sanctions relief. Since then the country has managed to
increase its oil sales, but it has not been able to sign major deals
with Western companies because of continuing banking restrictions
related to non-nuclear sanctions.

While the United States has relaxed many of its sanctions against Iran,
Washington still demands that even non-American manufacturers wishing to
sell to Iran obtain an export license if their products include
materials made in the United States. Airbus, based in Europe, buys more
than 40 percent of all its aircraft parts in the United States.

The granting of the licenses is likely to draw protests from some
members of Congress, who have noted that Iranian commercial aircraft
have been used to transport troops and weapons into Syria.
Representative Peter Roskam, Republican of Illinois, said in a statement
that the Obama administration ``has once again made a political decision
to appease Iran at the expense of our national security.'' He said
Congress was committed to making the process of delivering the planes as
difficult and expensive as possible.

Western political analysts who specialize in Iran said the Treasury's
decision reflected an effort by the Obama administration to help Mr.
Rouhani, who staked much of his political reputation on promised
economic dividends from the termination of nuclear sanctions.

``The U.S. is interested in constructive engagement with Iran, despite
continuing turmoil in the bilateral relationship,'' said Cliff Kupchan,
the chairman of the Eurasia Group, a Washington-based political
consultancy. He called the license approvals ``a big win for President
Rouhani, who needs to show Iranians that the nuclear deal is bringing
concrete improvements to their lives.''

Iran had called for a meeting on the sidelines of the United Nations
General Assembly with representatives of the world powers that signed
the nuclear agreement --- Russia, China, Germany, France, Britain and
the United States --- to complain about the lack of progress in carrying
out the terms of the deal.

``This is very good news for President Rouhani,'' said Nader Karimi
Joni, an analyst who supports the government. ``He needed this news back
home. Bringing Airbus will fend off critics and make him popular.''

Some in Iran, however, objected to what they said was political
maneuvering in the timing of the Treasury Department's decision to grant
the licenses.

``This is a bribe by the Americans to increase the chances of President
Rouhani to be re-elected,'' said Hamidreza Taraghi, a conservative
analyst and political figure. ``The American president, in his last
months in the White House, wants to give as much support as possible to
Rouhani's government and his pro-Western faction.''

Mr. Rouhani is scheduled to address the General Assembly on Thursday.

According to the Treasury Department, the licenses contain strict
conditions to ensure that the planes will be used exclusively for
commercial passenger use, and cannot be resold or transferred to another
entity.

A Boeing spokesman said the company was hoping to sell 46 single-aisle
737s; 30 wide-body 777s and four 747s.

Justin Dubon, an Airbus spokesman in Toulouse, France, confirmed that
the plane maker had obtained an initial license from the Treasury to
sell 17 planes to Iran --- part of a landmark, multibillion-dollar
\href{http://www.nytimes3xbfgragh.onion/2016/01/26/business/international/airbus-iran-aircraft-talks.html}{order
announced in January} for 118 Airbus aircraft, ranging from smaller
single-aisle jets to several 555-seat A380 ``superjumbo'' wide-body
aircraft.

It remains unclear how Iran is planning to pay for the aircraft. After
years of sanctions and low oil prices, the state's coffers are empty.
International credit continues to be virtually unavailable, with large
banks shying away from dealing with Iran because of complicated
regulations and continuing unilateral American sanctions against the
country, designated a ``state sponsor of terrorism'' by the United
States.

The news that the final barrier to the aircraft sales had fallen was
applauded by an Iranian pilot revered for the kind of heroic action for
which
\href{http://www.nytimes3xbfgragh.onion/2009/01/17/nyregion/17pilot.html?hp}{Capt.
Chesley B. Sullenberger III became famous}.

``This is a moment for happiness,'' said the pilot, Houshang Shahbazi.
``This is good for people. This is wonderful. Basically it means safer
air travel for Iranian passengers.''

In 2011, Captain Shahbazi
\href{http://www.nytimes3xbfgragh.onion/2012/07/14/world/middleeast/irans-airliners-falter-under-sanctions.html}{saved
120 passengers on a flight from Moscow to Tehran} when the landing gear
of the 40-year-old Boeing 727 he was piloting jammed. Captain Shahbazi
deftly manipulated the brakes to balance and slow the plane, then
tipping its nose down for a miraculous controlled crash landing.

The landing, captured on video, became a symbol of the consequences of
nearly four decades of American sanctions against Iran's airline
industry. As a result of the sanctions, Iran was left with a ragtag
fleet of old planes, bought during the era of the pro-Western government
in Iran, and secondhand workhorses purchased from countries like
Ukraine.

The 17 new Airbus planes are only a first step, Captain Shahbazi
emphasized. ``In total, we need around 500 planes, nationwide new
airport infrastructure and updates,'' he said. ``But I'm overjoyed with
this news.''

Advertisement

\protect\hyperlink{after-bottom}{Continue reading the main story}

\hypertarget{site-index}{%
\subsection{Site Index}\label{site-index}}

\hypertarget{site-information-navigation}{%
\subsection{Site Information
Navigation}\label{site-information-navigation}}

\begin{itemize}
\tightlist
\item
  \href{https://help.nytimes3xbfgragh.onion/hc/en-us/articles/115014792127-Copyright-notice}{©~2020~The
  New York Times Company}
\end{itemize}

\begin{itemize}
\tightlist
\item
  \href{https://www.nytco.com/}{NYTCo}
\item
  \href{https://help.nytimes3xbfgragh.onion/hc/en-us/articles/115015385887-Contact-Us}{Contact
  Us}
\item
  \href{https://www.nytco.com/careers/}{Work with us}
\item
  \href{https://nytmediakit.com/}{Advertise}
\item
  \href{http://www.tbrandstudio.com/}{T Brand Studio}
\item
  \href{https://www.nytimes3xbfgragh.onion/privacy/cookie-policy\#how-do-i-manage-trackers}{Your
  Ad Choices}
\item
  \href{https://www.nytimes3xbfgragh.onion/privacy}{Privacy}
\item
  \href{https://help.nytimes3xbfgragh.onion/hc/en-us/articles/115014893428-Terms-of-service}{Terms
  of Service}
\item
  \href{https://help.nytimes3xbfgragh.onion/hc/en-us/articles/115014893968-Terms-of-sale}{Terms
  of Sale}
\item
  \href{https://spiderbites.nytimes3xbfgragh.onion}{Site Map}
\item
  \href{https://help.nytimes3xbfgragh.onion/hc/en-us}{Help}
\item
  \href{https://www.nytimes3xbfgragh.onion/subscription?campaignId=37WXW}{Subscriptions}
\end{itemize}
