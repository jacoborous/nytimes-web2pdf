Sections

SEARCH

\protect\hyperlink{site-content}{Skip to
content}\protect\hyperlink{site-index}{Skip to site index}

\href{https://www.nytimes3xbfgragh.onion/section/world/middleeast}{Middle
East}

\href{https://myaccount.nytimes3xbfgragh.onion/auth/login?response_type=cookie\&client_id=vi}{}

\href{https://www.nytimes3xbfgragh.onion/section/todayspaper}{Today's
Paper}

\href{/section/world/middleeast}{Middle East}\textbar{}Battle Over
Aleppo Is Over, Russia Says, as Evacuation Deal Reached

\url{https://nyti.ms/2hIzf0u}

\begin{itemize}
\item
\item
\item
\item
\item
\end{itemize}

Advertisement

\protect\hyperlink{after-top}{Continue reading the main story}

Supported by

\protect\hyperlink{after-sponsor}{Continue reading the main story}

\hypertarget{battle-over-aleppo-is-over-russia-says-as-evacuation-deal-reached}{%
\section{Battle Over Aleppo Is Over, Russia Says, as Evacuation Deal
Reached}\label{battle-over-aleppo-is-over-russia-says-as-evacuation-deal-reached}}

By \href{http://www.nytimes3xbfgragh.onion/by/anne-barnard}{Anne
Barnard}

\begin{itemize}
\item
  Dec. 13, 2016
\item
  \begin{itemize}
  \item
  \item
  \item
  \item
  \item
  \end{itemize}
\end{itemize}

BEIRUT, Lebanon --- Russia declared on Tuesday that the four-year battle
over
\href{http://www.nytimes3xbfgragh.onion/2016/12/14/world/middleeast/aleppo-syria-evacuation-deal.html}{Aleppo},
Syria's largest city, was over, as the last remaining rebel fighters
agreed to turn over their territory to the Syrian government. While
pro-government forces were moving in, United Nations officials said they
were receiving multiple reports of execution-style killings.

The deal was announced just as civilians inside the rebel enclave said
\href{https://www.nytimes3xbfgragh.onion/2016/12/10/world/middleeast/we-are-dead-either-way-agonizing-choices-for-syrians-in-aleppo.html?action=click\&contentCollection=Middle\%20East\&module=RelatedCoverage\&region=EndOfArticle\&pgtype=article}{they
had lost hope}. They had spent days huddled in abandoned apartments
under heavy shelling, as those with a record of opposing the government
said they were bracing for arrest, conscription or death.

Under the deal, evacuations were set to begin at 5 a.m. Wednesday,
although the departures were delayed and there were reports of renewed
shelling. Earlier on Tuesday, fears had mounted as the United Nations
said it had received reports that Syrian troops or allied Iraqi militias
were gunning down families in apartments and on the streets, with the
toll reaching 82 civilians.

\includegraphics{https://static01.graylady3jvrrxbe.onion/images/2016/12/13/world/middleeast/14ALEPPO-WEB-slide-NXWH/14ALEPPO-WEB-slide-NXWH-articleLarge.jpg?quality=75\&auto=webp\&disable=upscale}

Several residents said they had lost contact with relatives in those
same areas, and a monitoring group, the
\href{http://www.syriahr.com/en/}{Syrian Observatory for Human Rights},
said the number of men forced to join the army upon fleeing to
government areas had reached 6,000. And with no way to treat the
wounded, bodies were piling up on the streets of the shrinking rebel
territory.

But then came the deal, and the shelling quieted down. Russia, Turkey
and Syrian rebel groups announced that they had agreed to evacuate all
of the remaining fighters to rebel-held territory, with civilians free
to join them or move to government-held areas, leaving the whole city of
Aleppo in government hands.

If the deal is carried out, and all rebel fighters leave as agreed, it
would mark
\href{https://www.nytimes3xbfgragh.onion/2016/12/09/world/middleeast/syria-aleppo-united-nations.html?action=click\&contentCollection=Middle\%20East\&module=RelatedCoverage\&region=EndOfArticle\&pgtype=article}{a
major turning point} in Syria's nearly six-year war. It would put all of
the major cities along the country's more populous western spine back
under government control, though Kurdish militias and the so-called
caliphate of the Islamic State continue to hold large areas to the east.

Image

Residents who had fled the violence in the Bustan al-Qasr neighborhood
reached Aleppo's Fardos area on Tuesday.Credit...Agence France-Presse
--- Getty Images

A full evacuation from Aleppo would be the largest success of the
government's starve-or-surrender strategy, bombing and besieging areas
out of its control until fighters and residents agree to surrender. It
would leave the armed opposition to the Syrian president, Bashar
al-Assad, in control of just one provincial capital, Idlib --- where
rebel fighters from Aleppo will be bused --- as well as stretches of
rural territory in northern and southern Syria and several other
isolated patches.

But the victory leaves Mr. Assad more dependent than ever on Iran and
Russia, and so despised by his opponents, including many of Syria's
majority Sunnis, that they might never again accept him as a legitimate
ruler. The territory he has reconquered has been won at a devastating
cost, and much of the country remains in the hands of his enemies.

The battle over Aleppo has been a particularly painful chapter of the
war, dividing and largely destroying one of the world's oldest and most
beautiful cities, a World Heritage site, amid mounting human suffering.
The eastern, rebel-held half of Aleppo had become unlivable, with rebels
unable to stop the government's indiscriminate bombing, which destroyed
entire neighborhoods, let alone deliver a better life. The
government-held districts were far less damaged, and daily life there
was more normal, but residents there suffered too, from indiscriminate
rebel shelling.

Image

The journey from eastern Aleppo has been perilous for civilians, some of
them older people in wheelchairs.Credit...Agence France-Presse --- Getty
Images

Tens of thousands of residents --- Russia says at least 100,000, the
United Nations puts the number at 37,000 --- left the rebel-held
districts as pro-government forces moved in. Thousands more fled to
rebel- or Kurdish-held areas. The government said rebels were keeping
most people inside by force; some residents said in interviews that
fighters had stopped them from leaving, others said they had guided them
out.

When the agreement was announced on Tuesday, the remaining residents ---
who just hours before had been sending what they thought were their last
farewells --- suddenly had to reckon with their mixed feelings about a
bittersweet and uncertain escape.

Now, it seemed, they would survive, and avoid arrest in a country where
dissent can be punished by torture. Yet now they would have to leave
their city, perhaps forever.

Image

About 37,000 people have fled eastern Aleppo for western areas of the
city or the countryside, according to the United Nations.Credit...Agence
France-Presse --- Getty Images

``I don't know whether to laugh or cry,'' Bassem Ayoub, a longtime
antigovernment activist, posted on Facebook. ``My soul is leaving my
body. Aleppo, my life, my life.''

Hours earlier, he and his family made an excruciating decision. His wife
and children, he said, went in search of a route to government-held
territory, hoping to fade into crowds of the displaced and not be
discovered as the relatives of a man the government considers a
terrorist. He stayed behind, unsure when he would see them again, but
certain that if he tried to flee, he would be arrested.

``Some are crying from happiness, others are sad they will no longer be
able to kneel to pray in Aleppo,'' said Malek, an activist who hopes to
join his pregnant wife in northern Aleppo Province. ``I'm sad, as well
--- I paid blood for Aleppo, but I can never again set foot here.
Tyranny has won.''

Image

Government troops have seized areas of the old city around Aleppo's
citadel, used as an army base.Credit...Omar Sanadiki/Reuters..

Others expressed their bitterness about a revolt that started with
protests for political reform but curdled as the government bombarded
rebel-held neighborhoods and Islamist extremists rose to power within
the insurgency. ``The Islamists and love of power screwed up the
revolution,'' Zaher al-Zaher, another activist, wrote in a text message.
``I lost my house, and my family are away from me. My heart is
burning.''

As preparations began for the departures from eastern Aleppo, government
supporters celebrated in the streets.

Rebels, activists and aid workers in eastern Aleppo said they had been
told that civilians and fighters could all leave and travel to
rebel-held areas. That satisfied a key demand that both groups have an
option to avoid going to government-controlled areas, where dissidents
and medical and humanitarian workers working in rebel-held areas have
been punished as terrorists.

Image

A government soldier carried an injured woman on
Tuesday.Credit...Sana/European Pressphoto Agency

But there were still doubts and fears about whether the way out would be
smooth or safe. In previous agreements, like one in Homs in 2014,
pro-government militias, angry that fighters they saw as terrorists were
allowed to leave alive, have fired on evacuees.

Signs of friction emerged early Wednesday in Aleppo. About 1 a.m., four
hours before the 5 a.m. departures set out under the agreement, a convoy
of vans carrying 70 wounded people, mostly fighters and their families,
were filmed pulling out of the enclave. But a short time afterward,
residents reported the convoy had been turned back by pro-government
Shiite militiamen and told it could depart after 6 a.m.

There were concerns that cracks were already emerging in the deal,
perhaps over tensions between Mr. Assad's two main allies, Russia and
Iran, which trains and backs pro-government Shiite militias from Iraq
and elsewhere.

Image

Government fighters celebrated as they seized eastern Aleppo. The United
Nations has accused pro-government forces of killing women and children
as they took the city.Credit...Agence France-Presse --- Getty Images

The deal was struck after widespread concern about the fate of
civilians, with the United Nations warning of ``a complete meltdown of
humanity'' and protests breaking out at United Nations headquarters in
New York and Russian embassies in several cities, including London,
Stockholm and Istanbul.

Hundreds of civilians have died in the offensive; the full toll is not
known because the humanitarian groups that rescued and treated civilians
and tracked casualties largely collapsed under fire in recent days.

The United Nations said the 82 summary killings were reported in four
neighborhoods --- Bustan al-Qasr, al-Fardous, al-Kallaseh and
al-Saleheen --- and included at least 11 women and 13 children, some
shot in the streets as they tried to escape, said Rupert Colville, a
spokesman for the United Nations High Commissioner for Human Rights. Mr.
Colville cited reports that the world body had received from reliable
contacts inside and outside the city.

Later on Tuesday, the United Nations Syria envoy, Staffan de Mistura,
revealed new details of what officials knew of the reported killings:
The dead had been shot with handguns, but it was not clear who had
killed them.

Russia categorically denied what the United Nations secretary general,
Ban Ki-moon, said were ``credible reports'' of atrocities, including
executions. Mr. Ban said the United Nations had been unable to verify
the reports because the Syrian government had repeatedly denied
permission to United Nations staff to monitor the evacuations and aid
civilians stuck on the battlefield.

Asked if there was concern that Idlib Province, where surrendering
rebels and civilians from other cities have been taken, could be ``the
next Aleppo,'' Mr. de Mistura acknowledged that grim possibility,
adding, ``We are working on that.''

Advertisement

\protect\hyperlink{after-bottom}{Continue reading the main story}

\hypertarget{site-index}{%
\subsection{Site Index}\label{site-index}}

\hypertarget{site-information-navigation}{%
\subsection{Site Information
Navigation}\label{site-information-navigation}}

\begin{itemize}
\tightlist
\item
  \href{https://help.nytimes3xbfgragh.onion/hc/en-us/articles/115014792127-Copyright-notice}{©~2020~The
  New York Times Company}
\end{itemize}

\begin{itemize}
\tightlist
\item
  \href{https://www.nytco.com/}{NYTCo}
\item
  \href{https://help.nytimes3xbfgragh.onion/hc/en-us/articles/115015385887-Contact-Us}{Contact
  Us}
\item
  \href{https://www.nytco.com/careers/}{Work with us}
\item
  \href{https://nytmediakit.com/}{Advertise}
\item
  \href{http://www.tbrandstudio.com/}{T Brand Studio}
\item
  \href{https://www.nytimes3xbfgragh.onion/privacy/cookie-policy\#how-do-i-manage-trackers}{Your
  Ad Choices}
\item
  \href{https://www.nytimes3xbfgragh.onion/privacy}{Privacy}
\item
  \href{https://help.nytimes3xbfgragh.onion/hc/en-us/articles/115014893428-Terms-of-service}{Terms
  of Service}
\item
  \href{https://help.nytimes3xbfgragh.onion/hc/en-us/articles/115014893968-Terms-of-sale}{Terms
  of Sale}
\item
  \href{https://spiderbites.nytimes3xbfgragh.onion}{Site Map}
\item
  \href{https://help.nytimes3xbfgragh.onion/hc/en-us}{Help}
\item
  \href{https://www.nytimes3xbfgragh.onion/subscription?campaignId=37WXW}{Subscriptions}
\end{itemize}
