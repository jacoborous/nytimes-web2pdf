Sections

SEARCH

\protect\hyperlink{site-content}{Skip to
content}\protect\hyperlink{site-index}{Skip to site index}

\href{https://www.nytimes3xbfgragh.onion/section/world/asia}{Asia
Pacific}

\href{https://myaccount.nytimes3xbfgragh.onion/auth/login?response_type=cookie\&client_id=vi}{}

\href{https://www.nytimes3xbfgragh.onion/section/todayspaper}{Today's
Paper}

\href{/section/world/asia}{Asia Pacific}\textbar{}South Korea Enters
Period of Uncertainty With President's Impeachment

\url{https://nyti.ms/2hrV0kM}

\begin{itemize}
\item
\item
\item
\item
\item
\item
\end{itemize}

Advertisement

\protect\hyperlink{after-top}{Continue reading the main story}

Supported by

\protect\hyperlink{after-sponsor}{Continue reading the main story}

\hypertarget{south-korea-enters-period-of-uncertainty-with-presidents-impeachment}{%
\section{South Korea Enters Period of Uncertainty With President's
Impeachment}\label{south-korea-enters-period-of-uncertainty-with-presidents-impeachment}}

\includegraphics{https://static01.graylady3jvrrxbe.onion/images/2016/12/10/world/video-park/video-park-videoSixteenByNine3000.jpg}

By \href{http://www.nytimes3xbfgragh.onion/by/choe-sang-hun}{Choe
Sang-Hun}

\begin{itemize}
\item
  Dec. 9, 2016
\item
  \begin{itemize}
  \item
  \item
  \item
  \item
  \item
  \item
  \end{itemize}
\end{itemize}

SEOUL, South Korea --- For her nearly four years in office, President
Park Geun-hye of South Korea cooperated closely with the United States,
particularly when it came to dealing with her volatile neighbor, North
Korea.

A vote on Friday to impeach her now throws both her country and American
policy in the region into deep uncertainty, as the
\href{https://www.nytimes3xbfgragh.onion/topic/subject/north-koreas-nuclear-program}{North's
nuclear program} advances and the incoming administration of
\href{http://www.nytimes3xbfgragh.onion/topic/person/donald-trump}{Donald
J. Trump} deliberates over whether to adjust Washington's stance on how
to best contain North Korean aggression.

Ms. Park, a conservative, had adopted a tough approach toward the North,
focusing on stronger sanctions. Her administration had also agreed to
deploy an American
\href{https://www.nytimes3xbfgragh.onion/2016/07/08/world/asia/south-korea-and-us-agree-to-deploy-missile-defense-system.html}{advanced
missile defense system} that infuriated the Chinese.

Yet her deep unpopularity --- the result of a scandal over
influence-peddling that led members of her own party to want to oust her
--- increases the odds that the next election will be won by an advocate
of friendlier relations with China.

Ms. Park's powers are suspended while the Constitutional Court considers
whether to remove her permanently. If it votes to do so, South Korea
will hold an election for a new president in 60 days. Prime Minister
Hwang Kyo-ahn
\href{https://www.nytimes3xbfgragh.onion/2016/12/09/world/asia/south-korea-who-could-replace-park.html}{will
serve as acting president}.

\includegraphics{https://static01.graylady3jvrrxbe.onion/images/2016/12/10/world/10IMPEACH-7/10IMPEACH-7-videoSixteenByNineJumbo1600.jpg}

Domestically, her undoing provides the latest example of how corruption
remains entrenched at the top echelons of political and corporate life
in South Korea, at a moment when the economy is slowing.

Parliament's impeachment motion accused Ms. Park, the nation's first
female leader, of
``\href{https://www.nytimes3xbfgragh.onion/2016/12/08/world/asia/south-korea-park-geun-hye-accusations-impeachment.html}{extensive
and serious violations} of the Constitution and the law.'' It followed
weeks of damaging disclosures that all but paralyzed the government and
produced
\href{https://www.nytimes3xbfgragh.onion/2016/11/26/world/asia/korea-park-geun-hye-protests.html}{the
largest street protests} in the nation's history.

Ms. Park suggested that she intended to fight her impeachment, telling
cabinet members hours later that she would ``calmly'' prepare for the
court deliberations and giving no hint that she would resign.

``I am gravely accepting the voices of the people and the National
Assembly, and I sincerely hope that the confusion will come to a
satisfactory end,'' she said in remarks broadcast on national
television.

Ms. Park has been accused of allowing a
\href{http://www.nytimes3xbfgragh.onion/2016/11/06/world/asia/south-koreans-ashamed-over-les-secretive-adviser.html}{shadowy
confidante}, the daughter of a religious sect leader, to
\href{http://www.nytimes3xbfgragh.onion/2016/10/28/world/asia/south-korea-choi-soon-sil.html}{exercise
remarkable influence} on matters ranging from choosing top government
officials to her wardrobe, and of helping her extort tens of millions of
dollars from South Korean companies.

\includegraphics{https://static01.graylady3jvrrxbe.onion/images/2016/12/10/world/10IMPEACH-2/10IMPEACH-2-articleInline.jpg?quality=75\&auto=webp\&disable=upscale}

Thousands of people who had gathered outside the Parliament building in
the frigid cold on Friday cheered when the news was announced.

``My heart is beating so fast,'' said Han Joo-young, 47, who had come
from Paju, north of the capital. ``I am so touched that people who are
usually powerless can have so much power when they come together.''

A total of 234 lawmakers voted for impeachment, well over the required
two-thirds threshold in the 300-seat National Assembly, the lone house
of Parliament in South Korea. The vote was by secret ballot, but the
results indicated that nearly half of the 128 lawmakers in Ms. Park's
party, Saenuri, had joined the opposition in moving to oust her.

Ms. Park, 64,
\href{http://www.nytimes3xbfgragh.onion/2013/02/26/world/asia/south-koreas-park-geun-hye-warns-north-against-nuclear-pursuits.html}{came
to power in early 2013}, backed mostly by older Koreans who had hoped
she would be a contemporary version of her father, the military dictator
Park Chung-hee, often viewed as the modernizer of South Korea.

Instead, she became the least popular leader since the country began
democratizing in the late 1980s, according to recent polls. Critics said
she was authoritarian and used state power to muzzle critics while
shielded by a coterie of advisers.

Image

Business leaders, including Jay Y. Lee, the vice chairman of Samsung,
and Chung Mong-koo, the Hyundai chairman, were questioned at a
parliamentary hearing on Tuesday about millions of dollars they gave to
two foundations controlled by Ms. Choi.Credit...Pool photo by Jeon
Heon-Kyun

The last time South Koreans took to the streets to kick out an unpopular
leader, in 1960, they had to fight bloody battles with police officers
armed with rifles.

That uprising forced Syngman Rhee, the country's founding and
authoritarian president, to resign and flee into exile in Hawaii. Vice
President Lee Ki-poong, a Rhee confidant who was at the center of a
corruption scandal, and his family ended their lives in a group suicide
as mobs approached their home in Seoul.

In subsequent decades, when South Koreans demanded more democracy, their
military dictators, including Ms. Park's father, brutally suppressed
them through martial law, torturing and even executing their leaders.

In 1987, violence erupted again as people took to the streets to demand
free presidential elections, forcing the military government to back
down.

This time, in a sign of how far South Korea's democracy has matured,
peaceful crowds achieved their goal without a single arrest.
Increasingly large numbers of protesters gathered in the capital,
including 1.7 million people on Saturday --- the largest protest in
South Korean history.

Image

Ms. Choi, center, at a prosecutor's office in Seoul in October. She has
been indicted on charges of leveraging her influence with Ms. Park to
extort millions of dollars from businesses.Credit...Ed Jones/Agence
France-Presse --- Getty Images

Ms. Park became the first South Korean president to lose an impeachment
vote since 2004, when the National Assembly moved to
\href{http://www.nytimes3xbfgragh.onion/2004/03/13/world/president-s-impeachment-stirs-angry-protests-in-south-korea.html}{impeach
Roh Moo-hyun} for violating election laws. Two months later, the
Constitutional Court ruled that Mr. Roh's offense was too minor to
justify impeachment and
\href{http://www.nytimes3xbfgragh.onion/2004/05/14/world/constitutional-court-reinstates-south-korea-s-impeached-president.html}{restored
him to office}. But Ms. Park faces much more serious accusations.

Still, it is difficult to predict when and how the Constitutional Court
will rule on Ms. Park's fate.

Removing her would require the votes of at least six of the nine
Constitutional Court judges. Among the current judges, six were
appointed by Ms. Park or her conservative predecessor, or are otherwise
seen as being close to her party.

The process, which may include hearings, will buy time for Ms. Park's
embattled party to recover from the scandal and prepare for the next
presidential election if the court decides to formally unseat her.

Ms. Park joins the ranks of South Korean leaders who have been disgraced
near the end of their terms, with their relatives or aides implicated in
corruption scandals. An exception was Ms. Park's father, who was
\href{http://www.nytimes3xbfgragh.onion/1979/10/27/archives/president-park-is-slain-in-korea-by-intelligence-chief-seoul-says.html}{assassinated
in 1979} at the height of his dictatorial power and before anyone dared
to bring corruption charges against him.

Image

Portraits of Ms. Park's parents, former President Park Chung-hee and his
wife, Yuk Young-soo, at a temple in Seoul. Ms. Park rose to power on
strong support from those who revered her father.Credit...Woohae
Cho/Getty Images

His and subsequent governments had favored a handful of family-owned
conglomerates with tax benefits, lucrative business licenses and
buy-Korean and anti-labor policies. The businesses were accused of
returning the favors with bribes and suspicious donations.

Through the years, top corporations have been rocked by recurring
corruption scandals, including the one that implicated Ms. Park and her
confidante, Choi Soon-sil.

In 1988, business tycoons were hauled into a parliamentary hearing to be
questioned about millions of dollars they gave to a foundation
controlled by the military dictator
\href{http://topics.nytimes3xbfgragh.onion/top/reference/timestopics/people/c/chun_doo_hwan/index.html}{Chun
Doo-hwan}.

The scene was repeated this week, when nine business leaders, including
Jay Y. Lee, the vice chairman of Samsung, and Chung Mong-koo, the
Hyundai chairman, appeared at another parliamentary hearing to be
questioned about millions of dollars they gave to two foundations
controlled by Ms. Choi.

Ms. Choi has been indicted on charges of leveraging her influence with
Ms. Park to extort the money from the businesses. Prosecutors have also
\href{http://www.nytimes3xbfgragh.onion/2016/11/20/world/asia/park-geun-hye-south-korea-extortion-accomplice-prosecutors.html}{identified
Ms. Park as a criminal suspect}, a first for a president, though she
cannot be indicted while in office.

The businessmen acknowledged giving the money, confirming that the
requests had come directly from Ms. Park or her aides.

Huh Chang-soo, the chairman of GS Group and the head of the Federation
of Korean Industries, the pro-business lobby group that coordinated the
donations, put the situation this way: ``It is difficult for businesses
to say no to a request from the government. That's the reality in South
Korea.''

Some analysts saw the vote and the huge protests as a repudiation of the
entire system.

``This impeachment is not only an impeachment against Park Geun-hye,''
said Kim Dong-choon, a professor of sociology at Sungkonghoe University
in Seoul, ``but a judgment against the conservative party and the
post-Cold War order that has maintained power in South Korea for so many
years.''

Advertisement

\protect\hyperlink{after-bottom}{Continue reading the main story}

\hypertarget{site-index}{%
\subsection{Site Index}\label{site-index}}

\hypertarget{site-information-navigation}{%
\subsection{Site Information
Navigation}\label{site-information-navigation}}

\begin{itemize}
\tightlist
\item
  \href{https://help.nytimes3xbfgragh.onion/hc/en-us/articles/115014792127-Copyright-notice}{©~2020~The
  New York Times Company}
\end{itemize}

\begin{itemize}
\tightlist
\item
  \href{https://www.nytco.com/}{NYTCo}
\item
  \href{https://help.nytimes3xbfgragh.onion/hc/en-us/articles/115015385887-Contact-Us}{Contact
  Us}
\item
  \href{https://www.nytco.com/careers/}{Work with us}
\item
  \href{https://nytmediakit.com/}{Advertise}
\item
  \href{http://www.tbrandstudio.com/}{T Brand Studio}
\item
  \href{https://www.nytimes3xbfgragh.onion/privacy/cookie-policy\#how-do-i-manage-trackers}{Your
  Ad Choices}
\item
  \href{https://www.nytimes3xbfgragh.onion/privacy}{Privacy}
\item
  \href{https://help.nytimes3xbfgragh.onion/hc/en-us/articles/115014893428-Terms-of-service}{Terms
  of Service}
\item
  \href{https://help.nytimes3xbfgragh.onion/hc/en-us/articles/115014893968-Terms-of-sale}{Terms
  of Sale}
\item
  \href{https://spiderbites.nytimes3xbfgragh.onion}{Site Map}
\item
  \href{https://help.nytimes3xbfgragh.onion/hc/en-us}{Help}
\item
  \href{https://www.nytimes3xbfgragh.onion/subscription?campaignId=37WXW}{Subscriptions}
\end{itemize}
