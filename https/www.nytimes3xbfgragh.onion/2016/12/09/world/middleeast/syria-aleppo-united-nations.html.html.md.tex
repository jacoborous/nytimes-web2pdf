Sections

SEARCH

\protect\hyperlink{site-content}{Skip to
content}\protect\hyperlink{site-index}{Skip to site index}

\href{https://www.nytimes3xbfgragh.onion/section/world/middleeast}{Middle
East}

\href{https://myaccount.nytimes3xbfgragh.onion/auth/login?response_type=cookie\&client_id=vi}{}

\href{https://www.nytimes3xbfgragh.onion/section/todayspaper}{Today's
Paper}

\href{/section/world/middleeast}{Middle East}\textbar{}Chaos and
Desperation as Thousands Flee Aleppo Amid Government Advance

\url{https://nyti.ms/2ht0VGa}

\begin{itemize}
\item
\item
\item
\item
\item
\end{itemize}

Advertisement

\protect\hyperlink{after-top}{Continue reading the main story}

Supported by

\protect\hyperlink{after-sponsor}{Continue reading the main story}

\hypertarget{chaos-and-desperation-as-thousands-flee-aleppo-amid-government-advance}{%
\section{Chaos and Desperation as Thousands Flee Aleppo Amid Government
Advance}\label{chaos-and-desperation-as-thousands-flee-aleppo-amid-government-advance}}

\includegraphics{https://static01.graylady3jvrrxbe.onion/images/2016/12/10/world/10Syria/10Syria-articleInline.jpg?quality=75\&auto=webp\&disable=upscale}

By \href{http://www.nytimes3xbfgragh.onion/by/anne-barnard}{Anne
Barnard} and Nick Cumming-Bruce

\begin{itemize}
\item
  Dec. 9, 2016
\item
  \begin{itemize}
  \item
  \item
  \item
  \item
  \item
  \end{itemize}
\end{itemize}

BEIRUT, Lebanon --- Hundreds of Syrian men who escaped
\href{http://www.nytimes3xbfgragh.onion/2016/12/08/world/middleeast/syria-aleppo-rebels-russia-lavrov-assad.html?ref=world}{rebel-held
areas of eastern Aleppo} to reach government-controlled parts of the
city are missing, United Nations officials said on Friday, adding that
they had received reports of government reprisals, including numerous
arrests and several cases of summary killings of suspected supporters of
the opposition.

At the same time, the officials said, some rebel groups have prevented
civilians from leaving and even killed or kidnapped those who demanded
that insurgents leave their neighborhoods.

The United Nations reports largely reflect what residents of East Aleppo
have told The New York Times in recent days as Syrian government forces
retook control of most of the city.
\href{http://www.nytimes3xbfgragh.onion/2016/12/07/world/middleeast/syria-aleppo.html}{Several
have said family members were detained, arrested or conscripted} after
crossing into government-held areas, and one resident recounted how
rebels in the Bustan al-Qasr neighborhood stopped people from leaving.

Other residents, however, said rebels had helped them cross the front
lines or warned them not to go at certain times or in certain places
because of the danger. Rebel groups inside east Aleppo are fragmented
and do not always act in concert.

As government forces continued to advance on Friday, panic was growing
among those still trapped inside, who either could not make their way
out or were afraid to enter government territory.

\includegraphics{https://static01.graylady3jvrrxbe.onion/images/2016/12/09/world/middleeast/aleppo-slide-O606/aleppo-slide-O606-articleLarge.jpg?quality=75\&auto=webp\&disable=upscale}

More than 10,500 people have left rebel-held areas in the last 24 hours,
Russian officials said on Friday, nearly half of them children, while
others have moved deeper into the rebel-held district under intense
bombardment.

At the same time, government forces have escorted hundreds of the
displaced back to recently retaken districts, such as the Bab al-Hadid
district below.

Image

Syrians of eastern Aleppo head home in the Bab al-Hadid
district.Credit...George Ourfalian/Agence France-Presse --- Getty Images

But those areas are heavily damaged, and officials have told families to
move into any available abandoned apartments.

In areas the government has recently retaken, like this one south of
Aleppo, it has placed posters of President Bashar al-Assad amid rubble
and handed out Syrian government flags.

Image

A picture of Syria's President Bashar al-Assad in Ramouseh.Credit...Omar
Sanadiki/Reuters

Many people who escaped eastern Aleppo have been filmed thanking
government troops and their ally, Russia, and chanting praise for Mr.
Assad.

In the newly retaken Bab al-Hadid neighborhood, a man greeted a
pro-government fighter.

Image

A street in the Bab al-Hadid neighborhood.Credit...George
Ourfalian/Agence France-Presse --- Getty Images

But for those viewed as opposing the government and men wanted for army
service, there is a riskier side to entering government territory.

Rupert Colville, a spokesman for the United Nations high commissioner
for human rights, said family members had reported that they had lost
contact with men aged 30 to 50, echoing several interviews The Times
conducted in which Aleppo residents said their males relatives were
arrested or forced to join the army.

``Given the terrible record of arbitrary detention, torture and enforced
disappearances by the Syrian government, we are of course deeply
concerned about the fate of these individuals,'' Mr. Colville said.

He also said that two members of the pro-government forces that took
over Aleppo's al-Halk neighborhood were reported to have summarily shot
four men in front of their families on Sunday because they were
suspected of working with the opposition.

The last two weeks have seen the fiercest bombardments yet of rebel-held
districts. Those still inside describe chaos and intense crowding in
some areas as people scramble for shelter. They said wounded people and
bodies were left in the streets with no one to help them.

Here, what is believed to be the charred body of a rebel fighter can be
seen in retaken Bab al-Hadid.

Image

On a street in Bab al-Hadid.Credit...George Ourfalian/Agence
France-Presse --- Getty Images

Mr. Colville said the agency had heard reports that two armed opposition
groups had abducted and killed an unknown number of civilians who asked
them to leave their neighborhoods. He added that residents trying to
leave the Bustan al-Qasr neighborhood may have come under fire from
armed opposition groups, something that could amount to the war crime of
hostage-taking.

``Civilians are caught between warring parties that appear to be
operating in flagrant violation of international humanitarian law,'' Mr.
Colville said. ``All sides are deeply culpable.''

The government has retaken most of the medieval Old City, a Unesco World
Heritage site severely damaged by years of fighting. At its center is
the citadel, which the government has held and used as a fortress
throughout the conflict.

Image

Aleppo's historic citadel.Credit...George Ourfalian/Agence France-Presse
--- Getty Images

The government's offensive over the last week has left rebels in east
Aleppo facing near-certain defeat. On Friday, Russian officials said
that 93 percent of Aleppo had been captured by the Syrian Army.

There has been some talk of at least a temporary cease-fire to allow for
evacuation of the city, but on Friday, the Russian foreign minister,
Sergey V. Lavrov, said that military operations ``will go on until
bandits leave eastern Aleppo.''

Image

Syrian government forces in the newly retaken area of Sahat al-Melh and
Qasr al-Adly.

Credit...George Ourfalian/Agence France-Presse --- Getty Images

Mr. Lavrov said he still had hope of reaching a final resolution for the
city in talks with the United States, but Jan Egeland, the United
Nations' humanitarian chief, on Thursday described the two countries as
being ``poles apart.''

At the United Nations on Friday, Russia suffered a sharp diplomatic blow
when a large majority of members voted for a mildly worded, legally
nonbinding General Assembly resolution calling for a pause in fighting
and access to humanitarian aid in Syria.

In practice, the resolution means little, with no specifics and no force
of law. So Syria and Russia, which lobbied vigorously against the
resolution, can continue their military operations, including those on
the rebel-held parts of Aleppo.

The resolution, advanced by Canada, received 122 votes in favor and 13
against. Thirty-six members abstained.

The no votes included Syria, its chief allies Iran and Russia, along
with China, Cuba and Venezuela.

It comes days after Russia, along with China, defeated a resolution in
the Security Council that would have imposed a seven-day cessation of
hostilities to allow aid into besieged parts of Aleppo and allow
residents to get out.

Advertisement

\protect\hyperlink{after-bottom}{Continue reading the main story}

\hypertarget{site-index}{%
\subsection{Site Index}\label{site-index}}

\hypertarget{site-information-navigation}{%
\subsection{Site Information
Navigation}\label{site-information-navigation}}

\begin{itemize}
\tightlist
\item
  \href{https://help.nytimes3xbfgragh.onion/hc/en-us/articles/115014792127-Copyright-notice}{©~2020~The
  New York Times Company}
\end{itemize}

\begin{itemize}
\tightlist
\item
  \href{https://www.nytco.com/}{NYTCo}
\item
  \href{https://help.nytimes3xbfgragh.onion/hc/en-us/articles/115015385887-Contact-Us}{Contact
  Us}
\item
  \href{https://www.nytco.com/careers/}{Work with us}
\item
  \href{https://nytmediakit.com/}{Advertise}
\item
  \href{http://www.tbrandstudio.com/}{T Brand Studio}
\item
  \href{https://www.nytimes3xbfgragh.onion/privacy/cookie-policy\#how-do-i-manage-trackers}{Your
  Ad Choices}
\item
  \href{https://www.nytimes3xbfgragh.onion/privacy}{Privacy}
\item
  \href{https://help.nytimes3xbfgragh.onion/hc/en-us/articles/115014893428-Terms-of-service}{Terms
  of Service}
\item
  \href{https://help.nytimes3xbfgragh.onion/hc/en-us/articles/115014893968-Terms-of-sale}{Terms
  of Sale}
\item
  \href{https://spiderbites.nytimes3xbfgragh.onion}{Site Map}
\item
  \href{https://help.nytimes3xbfgragh.onion/hc/en-us}{Help}
\item
  \href{https://www.nytimes3xbfgragh.onion/subscription?campaignId=37WXW}{Subscriptions}
\end{itemize}
