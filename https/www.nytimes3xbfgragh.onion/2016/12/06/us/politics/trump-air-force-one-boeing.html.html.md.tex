Sections

SEARCH

\protect\hyperlink{site-content}{Skip to
content}\protect\hyperlink{site-index}{Skip to site index}

\href{https://www.nytimes3xbfgragh.onion/section/politics}{Politics}

\href{https://myaccount.nytimes3xbfgragh.onion/auth/login?response_type=cookie\&client_id=vi}{}

\href{https://www.nytimes3xbfgragh.onion/section/todayspaper}{Today's
Paper}

\href{/section/politics}{Politics}\textbar{}`Cancel Order!' Donald Trump
Attacks Plans for Upgraded Air Force One

\href{https://nyti.ms/2gMP09d}{https://nyti.ms/2gMP09d}

\begin{itemize}
\item
\item
\item
\item
\item
\end{itemize}

Advertisement

\protect\hyperlink{after-top}{Continue reading the main story}

Supported by

\protect\hyperlink{after-sponsor}{Continue reading the main story}

\hypertarget{cancel-order-donald-trump-attacks-plans-for-upgraded-air-force-one}{%
\section{`Cancel Order!' Donald Trump Attacks Plans for Upgraded Air
Force
One}\label{cancel-order-donald-trump-attacks-plans-for-upgraded-air-force-one}}

\includegraphics{https://static01.graylady3jvrrxbe.onion/images/2016/12/07/us/07plane-1/07plane-1-videoSixteenByNine3000.jpg}

By \href{http://www.nytimes3xbfgragh.onion/by/michael-d-shear}{Michael
D. Shear} and
\href{http://www.nytimes3xbfgragh.onion/by/christopher-drew}{Christopher
Drew}

\begin{itemize}
\item
  Dec. 6, 2016
\item
  \begin{itemize}
  \item
  \item
  \item
  \item
  \item
  \end{itemize}
\end{itemize}

WASHINGTON --- President-elect Donald J. Trump took a shot on Tuesday at
one of the nation's largest manufacturers, Boeing, sharply criticizing a
pending order for a new Air Force One and suggesting that the company
was ``doing a little bit of a number'' with the cost of the next
generation of presidential aircraft.

``Boeing is building a brand new 747 Air Force One for future
presidents, but costs are out of control, more than \$4 billion,'' Mr.
Trump
\href{https://twitter.com/realDonaldTrump/status/806134244384899072}{wrote
on Twitter}. ``Cancel order!''

Although his post attracted attention because it was about the most
famous airplane in the world, the significance may be broader: For
perhaps the first time since President John F. Kennedy took on the steel
industry in the early 1960s, the heads of big American companies are
being confronted by a leader willing to call them out directly and
publicly for his policy and political aims.

Although President Obama forcefully criticized Wall Street and the
financial industry after Lehman Brothers collapsed in 2008, he tended
not to single out individual companies. But Mr. Trump is now targeting
Boeing a week after
\href{http://www.nytimes3xbfgragh.onion/2016/11/29/business/trump-to-announce-carrier-plant-will-keep-jobs-in-us.html}{he
pushed Carrier} and its parent company, United Technologies, to keep
about 1,000 manufacturing jobs in Indiana, and three weeks after he
singled out
\href{http://www.nytimes3xbfgragh.onion/2016/11/18/us/politics/donald-trump-takes-credit-for-helping-to-save-a-ford-plant-that-wasnt-closing.html}{a
Ford plant in Kentucky}.

Executives who give him what he wants may also be rewarded. On Tuesday
afternoon, the president-elect
\href{http://www.nytimes3xbfgragh.onion/2016/12/04/business/dealbook/masayoshi-son-softbank-mobile.html}{escorted}
the billionaire Japanese businessman Masayoshi Son to the lobby of Trump
Tower to announce that the technology conglomerate SoftBank Group would
be investing \$50 billion in the United States. He called Mr. Son one of
``the great men of industry.''

Mr. Son promised the investment, which will come from a previously
announced \$100 billion fund, as he is pressing
\href{http://dealbook.nytimes3xbfgragh.onion/2014/08/05/sprint-and-softbank-said-to-abandon-bid-for-t-mobile-us/}{to
merge the wireless company Sprint}, which his firm owns a controlling
interest in, with T-Mobile: a merger that Mr. Obama's regulators have
blocked.

What is motivating Mr. Trump is not always clear. His transition team is
receiving information about major federal programs, and Mr. Trump
received a briefing on Monday that included the cost of the Air Force
One project, according to a person familiar with the discussion. But he
also made his post about the Air Force One upgrade just minutes after
The Chicago Tribune
\href{http://www.chicagotribune.com/business/columnists/ct-boeing-china-trump-robert-reed-1206-biz-20161205-column.html}{published
comments} from Boeing's chief executive, Dennis Muilenburg, suggesting
that the president-elect's trade policies could hurt the company, which
does substantial business in China.

But Mr. Trump did not focus on Boeing broadly. Instead, he focused on
the Air Force One upgrade, telling reporters at Trump Tower, ``The plane
is totally out of control.''

``It's going to be over \$4 billion for the Air Force One program, and I
think it's ridiculous,'' he said. ``I think Boeing is doing a little bit
of a number. We want Boeing to make a lot of money, but not that much
money.''

In a statement after Mr. Trump's Twitter post, Boeing said it had a
\$170 million contract to study the equipment that a redesigned Air
Force One might need. That project has just gotten underway, so billions
of dollars in cost overruns at this point appear to be impossible.

``Some of the statistics that have been, uh, cited, shall we say, don't
appear to reflect the nature of the financial agreement between Boeing
and the Department of Defense,'' the White House press secretary, Josh
Earnest, said.

Air Force officials said they were proposing to spend \$2.7 billion over
the next five years to research, develop and test communications
technologies and other advanced systems. The Air Force would then buy
two 747-8 aircraft, which normally cost airlines \$350 million to \$400
million apiece, and refit them to include all the new systems and handle
extra weight.

The planes would not be ready to fly until 2024, so Mr. Trump's \$4
billion estimate may ultimately be about correct. However, since nothing
but the basic study contract has been awarded yet, his administration
could cut back or reshape the Air Force proposal in any way it or
Congress wanted.

``We look forward to working with the U.S. Air Force on subsequent
phases of the program, allowing us to deliver the best planes for the
president at the best value for the American taxpayer,'' Boeing said.

Aviation analysts were more blunt.

``This is getting ridiculous fast, when an important policy and
acquisition decision is being made by Twitter,'' said Richard L.
Aboulafia, an aviation consultant with the Teal Group in Fairfax, Va.

Mr. Trump's willingness to intervene at the individual corporate level
is a stark departure from Republican orthodoxy, which has long objected
to the government's picking winners and losers. Greg Hayes, the chief
executive of United Technologies,
\href{http://www.cnbc.com/2016/12/05/cnbc-transcript-united-technologies-chairman-ceo-greg-hayes-on-cnbcs-mad-money-w-jim-cramer-today.html}{seemed
to imply} on CNBC on Monday that he felt pressured.

``There was a cost as we thought about keeping the Indiana plant open,''
he said. ``At the same time,'' he added, ``I was born at night, but not
last night. I also know that about 10 percent of our revenue comes from
the U.S. government.''

Some of the jobs saved from Mexico will probably fall to automation.
Carrier will invest \$16 million in the Indianapolis plant to automate
its operations and ``drive the cost down so that we can continue to be
competitive,'' Mr. Hayes said. ``What that ultimately means is there
will be fewer jobs.''

Mr. Trump's Air Force One post came out of the blue: He had not focused
in the campaign on the cost of Boeing's plans for a next-generation
presidential plane.

Last week, Mr. Muilenburg, Boeing's chief executive, said that
one-fourth of all the commercial airplanes it sold were for use in
China, where Boeing is in a tense competition with Europe's Airbus, its
main rival. Like other major exporters, it is concerned that if Mr.
Trump offends Chinese leaders or imposes tariffs against imports, China
could retaliate by buying more planes from Airbus, which would reduce
jobs at Boeing.

Mr. Trump certainly understands that as president, he will no longer be
flying his own, well-appointed Boeing 757. The Secret Service and the
Defense Department would object.

Beyond convenience, Air Force One carries an array of top-secret
communications gear for conducting everyday business and for managing a
global crisis --- even wartime operations, if required --- while aloft.
It is also equipped with a number of never-discussed security features.

The communications systems on the planes now in use were designed in the
1980s. The new ones would incorporate the latest advances, as well as
anti-hacking defenses. The planes would also need other highly
classified systems to protect the president that the Air Force will not
discuss. But among the proposals considered several years ago for a new
presidential helicopter were technologies to help prevent terrorist
attacks and to resist the electromagnetic effects of a nuclear blast.

Mr. Aboulafia said Air Force One needed to have antimissile defenses
like jamming and electronic countermeasures to keep the president safe.

Mr. Trump could eliminate some of these features to cut costs. But
``talk about the ultimate in penny wise and pound foolishness,'' Mr.
Aboulafia said. ``We're talking about Pentagon weapons accounts that are
going to \$200 billion a year, and you're going to nickel and dime the
survivability of the president's jet. That is about as dysfunctional as
it gets.''

Mr. Trump could make good on his threat and cut the project from his
budget request for the fiscal year that begins in October 2017, the
first budget year of his presidency. But ultimately, Congress controls
the federal purse strings, and lawmakers with parochial interests are
already weighing in.

``Replacing the 26-year-old Air Force One aircraft will support
good-paying jobs throughout northwest Washington and is important to
ensuring the safety and security of future presidents,'' Senator Patty
Murray, Senator Maria Cantwell and Representative Rick Larsen, all
Democrats of Washington, said in a joint statement; Boeing's largest
factories are in the Seattle area. ``The president-elect's tweet does
nothing to change those basic facts.''

Advertisement

\protect\hyperlink{after-bottom}{Continue reading the main story}

\hypertarget{site-index}{%
\subsection{Site Index}\label{site-index}}

\hypertarget{site-information-navigation}{%
\subsection{Site Information
Navigation}\label{site-information-navigation}}

\begin{itemize}
\tightlist
\item
  \href{https://help.nytimes3xbfgragh.onion/hc/en-us/articles/115014792127-Copyright-notice}{©~2020~The
  New York Times Company}
\end{itemize}

\begin{itemize}
\tightlist
\item
  \href{https://www.nytco.com/}{NYTCo}
\item
  \href{https://help.nytimes3xbfgragh.onion/hc/en-us/articles/115015385887-Contact-Us}{Contact
  Us}
\item
  \href{https://www.nytco.com/careers/}{Work with us}
\item
  \href{https://nytmediakit.com/}{Advertise}
\item
  \href{http://www.tbrandstudio.com/}{T Brand Studio}
\item
  \href{https://www.nytimes3xbfgragh.onion/privacy/cookie-policy\#how-do-i-manage-trackers}{Your
  Ad Choices}
\item
  \href{https://www.nytimes3xbfgragh.onion/privacy}{Privacy}
\item
  \href{https://help.nytimes3xbfgragh.onion/hc/en-us/articles/115014893428-Terms-of-service}{Terms
  of Service}
\item
  \href{https://help.nytimes3xbfgragh.onion/hc/en-us/articles/115014893968-Terms-of-sale}{Terms
  of Sale}
\item
  \href{https://spiderbites.nytimes3xbfgragh.onion}{Site Map}
\item
  \href{https://help.nytimes3xbfgragh.onion/hc/en-us}{Help}
\item
  \href{https://www.nytimes3xbfgragh.onion/subscription?campaignId=37WXW}{Subscriptions}
\end{itemize}
