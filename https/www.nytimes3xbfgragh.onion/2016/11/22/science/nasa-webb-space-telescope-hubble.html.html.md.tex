Sections

SEARCH

\protect\hyperlink{site-content}{Skip to
content}\protect\hyperlink{site-index}{Skip to site index}

\href{https://www.nytimes3xbfgragh.onion/section/science}{Science}

\href{https://myaccount.nytimes3xbfgragh.onion/auth/login?response_type=cookie\&client_id=vi}{}

\href{https://www.nytimes3xbfgragh.onion/section/todayspaper}{Today's
Paper}

\href{/section/science}{Science}\textbar{}Telescope That `Ate Astronomy'
Is on Track to Surpass Hubble

\url{https://nyti.ms/2fhIEuf}

\begin{itemize}
\item
\item
\item
\item
\item
\end{itemize}

Advertisement

\protect\hyperlink{after-top}{Continue reading the main story}

Supported by

\protect\hyperlink{after-sponsor}{Continue reading the main story}

\href{/column/out-there}{Out There}

\hypertarget{telescope-that-ate-astronomy-is-on-track-to-surpass-hubble}{%
\section{Telescope That `Ate Astronomy' Is on Track to Surpass
Hubble}\label{telescope-that-ate-astronomy-is-on-track-to-surpass-hubble}}

\includegraphics{https://static01.graylady3jvrrxbe.onion/images/2016/11/17/science/22OUTTHERE1/22OUTTHERE1-articleLarge.jpg?quality=75\&auto=webp\&disable=upscale}

By \href{http://www.nytimes3xbfgragh.onion/by/dennis-overbye}{Dennis
Overbye}

\begin{itemize}
\item
  Nov. 21, 2016
\item
  \begin{itemize}
  \item
  \item
  \item
  \item
  \item
  \end{itemize}
\end{itemize}

GREENBELT, Md. --- The next great space telescope spread its golden
wings this month.

Like the petals of a 20-foot sunflower seeking the light, the 18
hexagonal mirrors that make up the heart of NASA's
\href{http://www.jwst.nasa.gov/}{James Webb Space Telescope} were faced
toward a glassed-in balcony overlooking a cavernous clean room at the
\href{https://www.nasa.gov/goddard}{Goddard Space Flight Center} here.

Inside the room, reporters and a gaggle of space agency officials,
including the ebullient administrator
\href{http://blogs.nasa.gov/bolden/}{Charles Bolden}, were getting their
pictures taken in front of the giant mirror.

Now, after 20 years with a budget of \$8.7 billion, the Webb telescope
is on track and on budget to be launched in October 2018 and sent a
million miles from Earth, NASA says.

\includegraphics{https://static01.graylady3jvrrxbe.onion/images/2016/11/21/multimedia/220verbye-video/220verbye-video-videoSixteenByNineJumbo1600.jpg}

The telescope, named after NASA Administrator
\href{http://www.nytimes3xbfgragh.onion/1992/03/29/us/james-webb-who-led-moon-program-dies-at-85.html}{James
Webb}, who led the space agency in the 1960s, is the long-awaited
successor of the
\href{http://www.nasa.gov/mission_pages/hubble/story/index.html}{Hubble
Space Telescope}.

Seven times larger than the Hubble in light-gathering ability, the Webb
was designed to see farther out in space and deeper into the past of the
universe. It may solve mysteries about how and when the first stars and
galaxies emerged some 13 billion years ago in the smoky aftermath of the
\href{https://science.nasa.gov/astrophysics/focus-areas/what-powered-the-big-bang}{Big
Bang}.

Equipped with the sort of infrared goggles that give troops and police
officers night vision, the Webb would peer into the dust clouds and gas
storms of the Milky Way in which stars and planets are presently being
birthed. It would be able to study planets around other stars.

\includegraphics{https://static01.graylady3jvrrxbe.onion/images/2016/11/22/science/22OVERBYE10/22OVERBYE10-articleInline.jpg?quality=75\&auto=webp\&disable=upscale}

That has been NASA's dream since 1996 when the idea for the telescope
was conceived with a projected price tag then of \$500 million But as
recently as six years ago, the James Webb Space Telescope was, in the
words of Nature magazine,
``\href{http://www.nature.com/news/2010/101027/full/4671028a.html}{the
telescope that ate astronomy},'' mismanaged, over budget and behind
schedule so that it had crushed everything else out of NASA's science
budget.

A House subcommittee once voted to cancel it. Instead, the program was
rebooted with a strict spending cap.

The scientific capabilities of the telescope emerged unscathed from that
period, astronomers on the project say. The major change, said Jonathan
P. Gardner, the deputy senior project scientist, was to simplify the
testing of the telescope.

\href{https://www.nytimes3xbfgragh.onion/interactive/2015/04/23/science/space/unforgettable-hubble-space-telescope-photos.html}{}

\includegraphics{https://static01.graylady3jvrrxbe.onion/images/2015/04/23/science/hubble_25/hubble_25-videoLarge.jpg}

\hypertarget{unforgettable-hubble-space-telescope-photos}{%
\subsection{Unforgettable Hubble Space Telescope
Photos}\label{unforgettable-hubble-space-telescope-photos}}

On the eve of the 25th anniversary of the launch of the Hubble Space
Telescope, we asked astronomers and others involved in the telescope's
groundbreaking story to tell us about their favorite images.

Most of the pain was dealt to other NASA projects like a proposed
\href{http://www.nytimes3xbfgragh.onion/2011/01/04/science/space/04telescope.html}{space
telescope to study dark energy}, which the
\href{http://nationalacademyofsciences.org/?referrer=https://www.google.com/?referrer=http://nationalacademyofsciences.org/}{National
Academy of Sciences} had hoped to put on a fast track to be launched
this decade. It's now delayed until 2025 or so.

Typically for NASA, the Webb telescope was a technologically ambitious
project, requiring 10 new technologies to make it work. Bill Ochs, a
veteran Goddard engineer who became project manager in 2010 during what
he calls the ``replan,'' said the key to its success so far, was having
enough money in the budget to provide a cushion for nasty surprises.

The telescope smiling up at us like a giant Tiffany shaving mirror is
6.5 meters in diameter, or just over 21 feet, compared with 2.4 meters
for the Hubble. The aim is to explore a realm of cosmic history about
150 million to one billion years after time began --- known as the
\href{http://www.haystack.mit.edu/ast/science/epoch/}{reionization
epoch}, when bright and violent new stars and the searing radiation from
quasars were burning away a gloomy fog of hydrogen gas that prevailed at
the end of the Big Bang.

In fact, astronomers don't know how the spectacle that greets our eyes
every night when the sun goes down or the lights go out wrenched itself
into luminous existence. They theorize that an initial generation of
stars made purely of hydrogen and helium --- the elements created during
the Big Bang --- burned ferociously and exploded apocalyptically,
jump-starting the seeding of the cosmos with progressively more diverse
materials. But nobody has ever seen any so-called
\href{http://astronomy.swin.edu.au/cosmos/P/Population+III}{Population
3} stars, as those first stars are known. They don't exist in the modern
universe. Astronomers have to hunt them in the dim past.

That ambition requires the Webb to be tuned to a different kind of light
than our eyes or the Hubble can see. Because the expansion of the cosmos
is rushing those earliest stars and galaxies away from us so fast, their
light is ``red-shifted'' to longer wavelengths the way the siren from an
ambulance shifts to a lower register as it passes by.

So blue light from an infant galaxy bursting with bright spanking new
stars way back then has been stretched to invisible infrared
wavelengths, or heat radiation, by the time it reaches us 13 billion
years later.

\includegraphics{https://static01.graylady3jvrrxbe.onion/images/2015/04/23/multimedia/out-there-hubble25/out-there-hubble25-videoSixteenByNine1050.jpg}

As a result, the Webb telescope will produce cosmic postcards in colors
no eye has ever seen. It also turns out that infrared emanations are the
best way to study exoplanets, the worlds beyond our own solar system
that have been discovered in the thousands since the Webb telescope was
first conceived.

In order to see those infrared colors, however, the telescope has to be
very cold --- less than 45 degrees Fahrenheit above
\href{https://www.sciencedaily.com/terms/absolute_zero.htm}{absolute
zero} --- so that its own heat does not swamp the heat from outer space.
Once in space, the telescope will unfold a giant umbrella the size of a
tennis court to keep the sun off it. The telescope, marooned in
permanent shade a million miles beyond the moon, will experience an
infinite cold soak.

The sunshield consists of five thin, kite-shaped layers of a material
called
\href{http://www.dupont.com/products-and-services/membranes-films/polyimide-films/brands/kapton-polyimide-film.html}{Kapton}.
Way too big to fit into a rocket, the shield, as well as the telescope
mirror, will have be launched folded up. It will then be unfolded in
space in a series of some 180 maneuvers that look in computer animations
like a cross between a parachute opening and a swimming pool cover going
into place.

Image

The Integrated Science Instrument Module planned for the James Webb
Space Telescope.Credit...Chris Gunn/NASA

Or at least that is the \$8 billion plan.

Engineers have done it on the ground, and it worked. The same people who
refolded the shield after each test will fold it again, in a process Mr.
Ochs compares to packing up your parachute before a jump. The test will
come in space, where no one will be able to help if things go wrong.

That whole process will amount to what Mr. Ochs called ``six months of
high anxiety.''

``For the most part, it all has to work,'' Mr. Ochs said.

The last time NASA did something this big astronomically, in 1990,
things didn't quite work. Once in orbit, the Hubble
\href{http://www.nytimes3xbfgragh.onion/1990/10/27/us/nasa-weighs-mission-to-correct-space-telescope-s-blurred-vision.html}{couldn't
be focused}; it had a misshapen mirror that had never been properly
tested. Astronauts eventually fitted it with corrective lenses, and it
went on to become the crown jewel of astronomy.

Image

Two Exelis Inc. engineers practicing ``snow cleaning'' on a test mirror
for the James Webb Space Telescope.Credit...Chris Gunn/NASA

Making sure that doesn't happen this time is the agenda for the next two
years. ``Our telescope is finished,'' John C. Mather, the senior project
scientist, said. ``Now we are about to prove it works.''

In the coming weeks, the mirror and the box of scientific instruments on
its back will be put on a rig and shaken to simulate the vibrations of a
launch, and then sealed in an acoustic chamber and bombarded with the
noise of a launch.

If the parts survive unscathed, the telescope assembly will be shipped
to a giant vacuum chamber at the Johnson Space Center in Houston. There
it will be chilled to the deep-space temperatures at which it will have
to work, and engineers will actually focus the telescope, twiddling the
controls for seven actuators on each of the 18 mirror segments. No
Hubble surprises here.

Then the telescope will go to Los Angeles to be mounted on its gigantic
sunshield. That whole contraption, now too big for even the giant
\href{http://www.af.mil/AboutUs/FactSheets/Display/tabid/224/Article/104492/c-5-abc-galaxy-c-5m-super-galaxy.aspx}{C-5A}
military transport plane, will travel by ship through the Panama Canal
to French Guiana.

It will be launched on an
\href{http://www.esa.int/Our_Activities/Launchers/Launch_vehicles/Ariane_5}{Ariane
5 rocket} supplied by the \href{http://www.esa.int/ESA}{European Space
Agency} as part of Europe's contribution to the observatory, and go into
orbit around the sun at \href{http://jwst.nasa.gov/orbit.html}{a point
called L2} about a million miles from Earth. Canada, NASA's other
partner, supplied some of the instruments.

Then come the six months of anxiety. Sometime in the spring of 2019, if
all goes well, the telescope will record its first real image --- of
what, the assembled astronomers were not ready to guess. In a bonus
undreamed of when the Webb telescope was first conceived, it looks as if
the Hubble will still be going strong when the Webb is launched. They
will share the sky and the potential for joint observing projects. A
million miles apart, they can view objects in the solar system from
different angles, providing a kind of stereoscopic perspective.

Besides the expected baby galaxies and the exoplanets, there are, as
astronomers like to remind us, always new surprises (like
\href{http://www.nytimes3xbfgragh.onion/2016/02/12/science/ligo-gravitational-waves-black-holes-einstein.html}{colliding
black holes} when the \href{http://ligo.org/}{LIGO observatory} was
turned on last year) when humanity devises a new way to look at the sky.

Asked what the telescope's greatest discovery would be, Dr. Mather said,
``If I knew, I would tell you.''

Nor would the project members talk about contingency plans to rescue the
telescope if anything goes wrong a million miles from Earth. There are
no plans to fix it or bring it back. They know how to attach a probe or
robot to the telescope, Dr. Mather said, but ``we are planning to not
need it, thank you.''

Advertisement

\protect\hyperlink{after-bottom}{Continue reading the main story}

\hypertarget{site-index}{%
\subsection{Site Index}\label{site-index}}

\hypertarget{site-information-navigation}{%
\subsection{Site Information
Navigation}\label{site-information-navigation}}

\begin{itemize}
\tightlist
\item
  \href{https://help.nytimes3xbfgragh.onion/hc/en-us/articles/115014792127-Copyright-notice}{©~2020~The
  New York Times Company}
\end{itemize}

\begin{itemize}
\tightlist
\item
  \href{https://www.nytco.com/}{NYTCo}
\item
  \href{https://help.nytimes3xbfgragh.onion/hc/en-us/articles/115015385887-Contact-Us}{Contact
  Us}
\item
  \href{https://www.nytco.com/careers/}{Work with us}
\item
  \href{https://nytmediakit.com/}{Advertise}
\item
  \href{http://www.tbrandstudio.com/}{T Brand Studio}
\item
  \href{https://www.nytimes3xbfgragh.onion/privacy/cookie-policy\#how-do-i-manage-trackers}{Your
  Ad Choices}
\item
  \href{https://www.nytimes3xbfgragh.onion/privacy}{Privacy}
\item
  \href{https://help.nytimes3xbfgragh.onion/hc/en-us/articles/115014893428-Terms-of-service}{Terms
  of Service}
\item
  \href{https://help.nytimes3xbfgragh.onion/hc/en-us/articles/115014893968-Terms-of-sale}{Terms
  of Sale}
\item
  \href{https://spiderbites.nytimes3xbfgragh.onion}{Site Map}
\item
  \href{https://help.nytimes3xbfgragh.onion/hc/en-us}{Help}
\item
  \href{https://www.nytimes3xbfgragh.onion/subscription?campaignId=37WXW}{Subscriptions}
\end{itemize}
