Sections

SEARCH

\protect\hyperlink{site-content}{Skip to
content}\protect\hyperlink{site-index}{Skip to site index}

\href{https://www.nytimes3xbfgragh.onion/section/world/middleeast}{Middle
East}

\href{https://myaccount.nytimes3xbfgragh.onion/auth/login?response_type=cookie\&client_id=vi}{}

\href{https://www.nytimes3xbfgragh.onion/section/todayspaper}{Today's
Paper}

\href{/section/world/middleeast}{Middle East}\textbar{}An Embassy in
Jerusalem? Trump Promises, but So Did Predecessors

\url{https://nyti.ms/2eOqyEK}

\begin{itemize}
\item
\item
\item
\item
\item
\end{itemize}

Advertisement

\protect\hyperlink{after-top}{Continue reading the main story}

Supported by

\protect\hyperlink{after-sponsor}{Continue reading the main story}

Memo from Jerusalem

\hypertarget{an-embassy-in-jerusalem-trump-promises-but-so-did-predecessors}{%
\section{An Embassy in Jerusalem? Trump Promises, but So Did
Predecessors}\label{an-embassy-in-jerusalem-trump-promises-but-so-did-predecessors}}

\includegraphics{https://static01.graylady3jvrrxbe.onion/images/2016/11/19/world/19Jerusalem1/19Jerusalem1-articleLarge.jpg?quality=75\&auto=webp\&disable=upscale}

By \href{http://www.nytimes3xbfgragh.onion/by/peter-baker}{Peter Baker}

\begin{itemize}
\item
  Nov. 18, 2016
\item
  \begin{itemize}
  \item
  \item
  \item
  \item
  \item
  \end{itemize}
\end{itemize}

JERUSALEM --- America's top diplomat in Jerusalem lives in an elegant
three-story stone house first built by a German Lutheran missionary in
1868, a short walk from the historic Old City. But he is not an
ambassador and the mission is a consulate, not an embassy.

For decades, those distinctions have rankled many Israeli Jews. The
United States, along with the rest of the world, has kept its primary
diplomatic footprint not in Israel's self-declared capital, Jerusalem,
but in the commercial and cultural hub of Tel Aviv to avoid seeming to
take sides in the fraught and never-ending argument over who really has
the right to control this ancient city.

Until now. Maybe.

President-elect
\href{http://www.nytimes3xbfgragh.onion/topic/person/donald-trump}{Donald
J. Trump} vowed during his campaign that he would relocate the mission
``fairly quickly'' after taking office. That in itself is nothing new:
For years, candidates running for president have promised to move the
embassy to Jerusalem, and for years, candidates who actually became
president have opted against doing so.

But just as Mr. Trump broke all the rules of campaigning, some of his
supporters say no amount of hand-wringing by the State Department will
change his mind. Jason Greenblatt, an Orthodox lawyer who is advising
Mr. Trump on Israel, told
\href{http://www.timesofisrael.com/trump-adviser-he-doesnt-see-settlements-as-peace-obstacle/}{Army
Radio} after the election that the president-elect was ``going to do
it'' because he was ``a man who keeps his word.''

Already, many Israelis and Palestinians are buzzing about the prospect.
Where would the embassy go? Would it straddle the line between West
Jerusalem, which is predominantly Jewish, and East Jerusalem, which is
predominantly Arab? Would it touch off street protests in Palestinian
cities or a backlash among Arab allies like
\href{http://www.nytimes3xbfgragh.onion/topic/destination/egypt}{Egypt}
and
\href{http://www.nytimes3xbfgragh.onion/topic/destination/saudi-arabia}{Saudi
Arabia}?

``Jerusalem is a symbolic, emotional and real issue,'' said Itamar
Rabinovich, a former Israeli ambassador to the United States and
president of the Israel Institute. ``It matters to many Israeli Jews
because it would indicate that the United States actually recognizes
Jerusalem as
\href{http://www.nytimes3xbfgragh.onion/topic/destination/israel?inline=nyt-geo}{Israel}'s
capital, which now it effectively does not.''

Which is why Arabs object so strenuously to such a move. ``This is a
sign that he's going to side with Israel,'' said Mustafa Alani, a
scholar at the Gulf Research Center, a research organization with
offices in Saudi Arabia and elsewhere. ``If he does it, it's going to be
a wrong start for his relationship with the Arab world.''

The status of Jerusalem has always been one of the thorniest issues
dividing Jews and Arabs. In 1947, the United Nations recommended that
the city be declared a ``corpus separatum,'' meaning an international
city, rather than incorporated into either the Arab or the Jewish states
then being contemplated on the land between the Jordan River and the
Mediterranean Sea. But in the war that followed its
\href{http://learning.blogs.nytimes3xbfgragh.onion/2012/05/14/may-14-1948-israel-declares-independence/}{declaration
of statehood in 1948}, Israel captured the western portion of the city
while Jordan seized the east.

Israel took control of East Jerusalem in its 1967 war with its Arab
neighbors and annexed it, declaring that the city would remain whole and
unified as its eternal capital (and later building many settlements
there that most of the world considers illegal). The United States and
most other countries refused to recognize the annexation and kept their
embassies in or near Tel Aviv. The last two countries with embassies in
Jerusalem,
\href{http://www.nytimes3xbfgragh.onion/topic/destination/costa-rica}{Costa
Rica} and
\href{http://www.nytimes3xbfgragh.onion/topic/destination/el-salvador}{El
Salvador}, moved out a decade ago.

\href{http://www.nytimes3xbfgragh.onion/topic/person/bill-clinton}{Bill
Clinton} and
\href{http://www.nytimes3xbfgragh.onion/topic/person/george-w-bush}{George
W. Bush} both promised during their presidential campaigns to move the
embassy to Jerusalem. Both later backed away from those promises,
convinced by Middle East experts that doing so would prejudge
negotiations for a final settlement between Israelis and Palestinians.

In 1995,
\href{http://www.nytimes3xbfgragh.onion/1995/10/25/world/congress-backs-israel-embassy-switch-but-gives-clinton-an-out.html}{Congress
passed a law} declaring Jerusalem to be Israel's capital and requiring
the embassy be moved there by 1999 --- or else the State Department
building budget would be cut in half. But the law included a provision
allowing presidents to waive its requirement for six months if they
determined it was in the national interest. So every six months, Mr.
Clinton, Mr. Bush and eventually
\href{http://www.nytimes3xbfgragh.onion/topic/person/barack-obama?inline=nyt-per}{President
Obama} signed such waivers, fearing a violent response in the Arab world
if the embassy moved.

\includegraphics{https://static01.graylady3jvrrxbe.onion/images/2016/11/19/world/19Jerusalem2/19Jerusalem2-articleLarge.jpg?quality=75\&auto=webp\&disable=upscale}

``Every president who reversed his campaign promise did so because he
decided not to take the risk,'' said Dennis B. Ross, a longtime Middle
East envoy who advised multiple presidents, including Mr. Obama.
``Jerusalem has historically been an issue that provoked great passions
--- often as a result of false claims --- that did trigger violence.''

Whether such advice might sway Mr. Trump is unclear. Despite Mr.
Greenblatt's declaration, another Trump adviser on the Middle East,
Walid Phares,
\href{https://soundcloud.com/user-735086019/walid-phares}{told the BBC}
that Mr. Trump would move the embassy ``under consensus.'' He later
clarified that he meant a ``consensus at home,'' since no one could
imagine a consensus including Arabs at this point.

Elliott Abrams, a former Middle East adviser to Mr. Bush, said Mr. Trump
should follow through because even if East Jerusalem is eventually ceded
to the Palestinians as the capital of their own state, no plausible
settlement would deny West Jerusalem to Israel. ``There is simply no
reason not to put a U.S. embassy there,'' he said.

The issue remains so delicate that the Obama administration
\href{http://www.nytimes3xbfgragh.onion/2015/06/09/us/politics/supreme-court-backs-white-house-on-jerusalem-passport-dispute.html?_r=0}{went
all the way to the Supreme Court} to block a law passed by Congress
allowing American parents of children born in Jerusalem to list Israel
as their birthplace on their passports.

When
\href{http://www.nytimes3xbfgragh.onion/2016/10/01/world/middleeast/shimon-peres-funeral.html}{Mr.
Obama came to Jerusalem} in September for the funeral of
\href{http://www.nytimes3xbfgragh.onion/topic/person/shimon-peres?inline=nyt-per}{Shimon
Peres}, the former Israeli president and prime minister, the White House
initially released a transcript of his eulogy that listed ``Jerusalem,
Israel'' as the location of his remarks. A few hours later, it issued a
``corrected'' transcript that literally crossed out the word ``Israel.''

The consulate currently in Jerusalem, run by the consul general, Donald
Blome, a career diplomat, deals mainly with the Palestinians while the
embassy in Tel Aviv, run by Ambassador Daniel B. Shapiro, an Obama
appointee, handles relations with Israel. Mr. Trump could simply declare
the consulate to be an embassy and move the ambassador's home as a
stopgap, but there are other logistical challenges.

The embassy's 800-person staff could not fit in the consular offices
near the Old City, nor in the large, fortresslike building that
processes visa requests and is surrounded by stone walls and tall metal
fences along the line that divides Jerusalem between Jewish and
Palestinian residents.

Israeli Jews cite a long history in Jerusalem dating back thousands of
years, and even many on the left who support a Palestinian state think
the embassy should be housed there. Gilead Sher, who worked as a peace
negotiator for Labor Party leaders, said, ``It seems abnormal that the
city, which is home to all of Israel's governmental, legislative,
judicial and national institutions, does not host foreign embassies.''

But Oded Eran, a retired Israeli diplomat now at the Institute for
National Security Studies in Tel Aviv, noted that Israel has not
invested ``much political capital'' in the matter because of ``a sober
assessment that few, if any, will move their embassy to Jerusalem.''

Indeed, with other perhaps more urgent priorities, Prime Minister
\href{http://www.nytimes3xbfgragh.onion/topic/person/benjamin-netanyahu?inline=nyt-per}{Benjamin
Netanyahu} and his government have made little comment on the
possibility since Mr. Trump's election. ``That has been a constant
commitment by many administrations, and one would expect it will be
acted on at the right time,'' said Dore Gold, a longtime adviser to Mr.
Netanyahu who just stepped down as director general of the Foreign
Ministry.

Palestinian officials presume Mr. Trump ultimately will follow the
course that his predecessors did and leave the issue to final-status
negotiations.

``I don't think he'll move the embassy, and I don't think he'll legalize
settlements,'' said Saeb Erekat, secretary general of the Palestine
Liberation Organization. ``I'm confident we'll work with President-elect
Trump and his administration to achieve peace and to achieve the
two-state solution.''

Advertisement

\protect\hyperlink{after-bottom}{Continue reading the main story}

\hypertarget{site-index}{%
\subsection{Site Index}\label{site-index}}

\hypertarget{site-information-navigation}{%
\subsection{Site Information
Navigation}\label{site-information-navigation}}

\begin{itemize}
\tightlist
\item
  \href{https://help.nytimes3xbfgragh.onion/hc/en-us/articles/115014792127-Copyright-notice}{©~2020~The
  New York Times Company}
\end{itemize}

\begin{itemize}
\tightlist
\item
  \href{https://www.nytco.com/}{NYTCo}
\item
  \href{https://help.nytimes3xbfgragh.onion/hc/en-us/articles/115015385887-Contact-Us}{Contact
  Us}
\item
  \href{https://www.nytco.com/careers/}{Work with us}
\item
  \href{https://nytmediakit.com/}{Advertise}
\item
  \href{http://www.tbrandstudio.com/}{T Brand Studio}
\item
  \href{https://www.nytimes3xbfgragh.onion/privacy/cookie-policy\#how-do-i-manage-trackers}{Your
  Ad Choices}
\item
  \href{https://www.nytimes3xbfgragh.onion/privacy}{Privacy}
\item
  \href{https://help.nytimes3xbfgragh.onion/hc/en-us/articles/115014893428-Terms-of-service}{Terms
  of Service}
\item
  \href{https://help.nytimes3xbfgragh.onion/hc/en-us/articles/115014893968-Terms-of-sale}{Terms
  of Sale}
\item
  \href{https://spiderbites.nytimes3xbfgragh.onion}{Site Map}
\item
  \href{https://help.nytimes3xbfgragh.onion/hc/en-us}{Help}
\item
  \href{https://www.nytimes3xbfgragh.onion/subscription?campaignId=37WXW}{Subscriptions}
\end{itemize}
