Sections

SEARCH

\protect\hyperlink{site-content}{Skip to
content}\protect\hyperlink{site-index}{Skip to site index}

\href{https://www.nytimes3xbfgragh.onion/section/world/asia}{Asia
Pacific}

\href{https://myaccount.nytimes3xbfgragh.onion/auth/login?response_type=cookie\&client_id=vi}{}

\href{https://www.nytimes3xbfgragh.onion/section/todayspaper}{Today's
Paper}

\href{/section/world/asia}{Asia Pacific}\textbar{}U.N. Stiffens
Sanctions on North Korea, Trying to Slow Its Nuclear March

\url{https://nyti.ms/2gkiM4S}

\begin{itemize}
\item
\item
\item
\item
\item
\end{itemize}

Advertisement

\protect\hyperlink{after-top}{Continue reading the main story}

Supported by

\protect\hyperlink{after-sponsor}{Continue reading the main story}

\hypertarget{un-stiffens-sanctions-on-north-korea-trying-to-slow-its-nuclear-march}{%
\section{U.N. Stiffens Sanctions on North Korea, Trying to Slow Its
Nuclear
March}\label{un-stiffens-sanctions-on-north-korea-trying-to-slow-its-nuclear-march}}

\includegraphics{https://static01.graylady3jvrrxbe.onion/images/2016/12/01/world/01NKOREA-1/01NKOREA-1-articleInline.jpg?quality=75\&auto=webp\&disable=upscale}

By \href{http://www.nytimes3xbfgragh.onion/by/somini-sengupta}{Somini
Sengupta} and
\href{http://www.nytimes3xbfgragh.onion/by/jane-perlez}{Jane Perlez}

\begin{itemize}
\item
  Nov. 30, 2016
\item
  \begin{itemize}
  \item
  \item
  \item
  \item
  \item
  \end{itemize}
\end{itemize}

UNITED NATIONS --- In an effort to tighten sanctions that largely failed
to throttle
\href{http://www.nytimes3xbfgragh.onion/topic/destination/north-korea?8qa}{North
Korea}'s nuclear program, the
\href{http://www.nytimes3xbfgragh.onion/topic/organization/security-council?8qa}{United
Nations Security Council} on Wednesday imposed a cap on coal exports,
the country's chief source of hard currency.

The new penalties --- adopted unanimously by the Council, including
China --- came as North Korea advances toward its goal of building a
functional nuclear warhead. That presents a stark national security
challenge to the incoming administration of President-elect Donald J.
Trump, who called North Korea's leader,
\href{http://www.nytimes3xbfgragh.onion/topic/person/kim-jongun?8qa}{Kim
Jong-un}, ``a maniac'' during the campaign, but has said nothing about
how to contain Mr. Kim's nuclear ambitions.

As with the original set of sanctions, which the Security Council
\href{http://www.nytimes3xbfgragh.onion/2016/03/03/world/asia/north-korea-un-sanctions.html}{adopted
in March}, the key to enforcing these new penalties remains in the hands
of China, North Korea's principal patron and coal customer.

The new restrictions on North Korean coal were relatively easy for
Beijing to approve. They serve the purpose of expressing China's
displeasure with Mr. Kim's agenda, yet they fall short of inflicting
crippling pain on North Korea.

China seems to have judged that reducing the coal imports will not upend
the North Korean economy and cause social unrest and flows of refugees
into China, an outcome it most fears.

On Wednesday, China's permanent representative to the United Nations,
Liu Jieyi, called on North Korea to halt its nuclear tests, saying they
undermine regional stability and Beijing's ``strategic interests.'' He
said the resolution demonstrated ``the uniform stance of the
international community.''

Samantha Power, the American ambassador to the United Nations, said
Wednesday that ``the United States recognizes China in working closely
with us.'' Negotiations lasted for three months, since Pyongyang's fifth
and latest nuclear test in September.

The original sanctions --- which the United States at the time hailed as
``comprehensive'' --- had sought to limit coal exports, unless it was
for what the measure called ``livelihood'' reasons. In spite of the
sanctions, exports to China increased.

American officials conceded this week that there was some ``vagueness''
in the original measure. The new resolution aims to shave \$700 million
from North Korea's coal revenues. The resolution says North Korea can
sell no more than 7.5 million metric tons of coal a year, or bring in no
more than \$400 million in sales, whichever comes first.

It also requires countries to tell the United Nations how much North
Korean coal they are buying and expands the list of banned items for
import by North Korea, including luxury goods like bone china worth more
than \$100 as well as equipment with dual-use purposes. ****

The measure also urges countries to allow North Korean diplomatic
missions around the world to have only one bank account. That, the
United States says, is intended to limit the country's penchant for
using its envoys and embassies to further its nuclear program.

How successful the new measures will be, of course, depends on the
willingness of countries to abide by them.

The United Nations secretary general, Ban Ki-moon, called the measures
the ``toughest'' sanctions imposed by the Council. But he warned that
the passage alone would be insufficient. ``It is incumbent on all member
states of the United Nations to make every effort to ensure that these
sanctions are fully implemented,'' he said.

The bigger uncertainty is what posture Mr. Trump will take toward North
Korea and Mr. Kim --- and, by extension, toward Beijing.

``If you look at North Korea, this guy, I mean, he's like a maniac,
O.K.?'' Mr. Trump said at a campaign rally in January. He went on to say
that ``he really does have missiles, and he really does have nukes.''
Since then, Mr. Trump and his transition team are likely to have been
briefed on the nature and scope of North Korea's nuclear arsenal.

American officials have warned for months that the
\href{http://www.nytimes3xbfgragh.onion/2016/05/07/world/asia/north-korea-nuclear-us-strategy.html}{North's
nuclear capabilities have increased sharply}. Its missile and nuclear
tests,
\href{http://www.nytimes3xbfgragh.onion/2016/09/09/world/asia/north-korea-nuclear-test.html?action=click\&contentCollection=Asia\%20Pacific\&module=RelatedCoverage\&region=Marginalia\&pgtype=article}{the
most recent in September}, have accelerated, despite the imposition of
sanctions in March.

The Security Council measures adopted on Wednesday are a response to the
Kim government's fifth and largest nuclear test. The revised sanctions
are aimed at cutting into North Korea's ability to profit from coal
exports and to tighten the noose around individuals and companies
involved in its nuclear program. The measure expands the list of people
subject to asset freezes and travel bans; they include some envoys to
countries like Egypt and Sudan.

American officials said the new sanctions would also clarify that the
``livelihood'' exemption applies to North Korean citizens, and cannot be
used to protect the livelihoods of Chinese importers.

Publicly, China has defended its coal imports since the sanctions
earlier this year. Even if statistics showed increased imports, the
trade was legal under the livelihood exemption, the Foreign Ministry
said.

Chinese analysts point to conditions flourishing in the North Korean
capital, Pyongyang, and say rudimentary market trading in rural areas is
keeping much of the population afloat.

And even though some Chinese steel makers in northern China value North
Korea's high-grade coal, the new curbs would have relatively little
impact on China's industrial output, said Deng Shun, a coal market
analyst with Success Futures, a trading company in Guangzhou.

Advertisement

\protect\hyperlink{after-bottom}{Continue reading the main story}

\hypertarget{site-index}{%
\subsection{Site Index}\label{site-index}}

\hypertarget{site-information-navigation}{%
\subsection{Site Information
Navigation}\label{site-information-navigation}}

\begin{itemize}
\tightlist
\item
  \href{https://help.nytimes3xbfgragh.onion/hc/en-us/articles/115014792127-Copyright-notice}{©~2020~The
  New York Times Company}
\end{itemize}

\begin{itemize}
\tightlist
\item
  \href{https://www.nytco.com/}{NYTCo}
\item
  \href{https://help.nytimes3xbfgragh.onion/hc/en-us/articles/115015385887-Contact-Us}{Contact
  Us}
\item
  \href{https://www.nytco.com/careers/}{Work with us}
\item
  \href{https://nytmediakit.com/}{Advertise}
\item
  \href{http://www.tbrandstudio.com/}{T Brand Studio}
\item
  \href{https://www.nytimes3xbfgragh.onion/privacy/cookie-policy\#how-do-i-manage-trackers}{Your
  Ad Choices}
\item
  \href{https://www.nytimes3xbfgragh.onion/privacy}{Privacy}
\item
  \href{https://help.nytimes3xbfgragh.onion/hc/en-us/articles/115014893428-Terms-of-service}{Terms
  of Service}
\item
  \href{https://help.nytimes3xbfgragh.onion/hc/en-us/articles/115014893968-Terms-of-sale}{Terms
  of Sale}
\item
  \href{https://spiderbites.nytimes3xbfgragh.onion}{Site Map}
\item
  \href{https://help.nytimes3xbfgragh.onion/hc/en-us}{Help}
\item
  \href{https://www.nytimes3xbfgragh.onion/subscription?campaignId=37WXW}{Subscriptions}
\end{itemize}
