Sections

SEARCH

\protect\hyperlink{site-content}{Skip to
content}\protect\hyperlink{site-index}{Skip to site index}

\href{https://www.nytimes3xbfgragh.onion/section/world/asia}{Asia
Pacific}

\href{https://myaccount.nytimes3xbfgragh.onion/auth/login?response_type=cookie\&client_id=vi}{}

\href{https://www.nytimes3xbfgragh.onion/section/todayspaper}{Today's
Paper}

\href{/section/world/asia}{Asia Pacific}\textbar{}Hong Kong Elected 2
Separatists. China Took Drastic Action.

\url{https://nyti.ms/2eec7K1}

\begin{itemize}
\item
\item
\item
\item
\item
\item
\end{itemize}

Advertisement

\protect\hyperlink{after-top}{Continue reading the main story}

Supported by

\protect\hyperlink{after-sponsor}{Continue reading the main story}

\hypertarget{hong-kong-elected-2-separatists-china-took-drastic-action}{%
\section{Hong Kong Elected 2 Separatists. China Took Drastic
Action.}\label{hong-kong-elected-2-separatists-china-took-drastic-action}}

\includegraphics{https://static01.graylady3jvrrxbe.onion/images/2016/11/08/world/08HONGKONG-1/08HONGKONG-1-articleLarge.jpg?quality=75\&auto=webp\&disable=upscale}

By \href{http://www.nytimes3xbfgragh.onion/by/michael-forsythe}{Michael
Forsythe}

\begin{itemize}
\item
  Nov. 6, 2016
\item
  \begin{itemize}
  \item
  \item
  \item
  \item
  \item
  \item
  \end{itemize}
\end{itemize}

\href{http://cn.nytimes3xbfgragh.onion/china/20161107/cc07npc-hk/}{阅读简体中文版}

HONG KONG --- In the nearly two decades since Hong Kong returned to
Chinese rule, the Communist government in Beijing has tolerated all
manner of activity in the city that it generally finds intolerable on
the mainland:
\href{http://www.nytimes3xbfgragh.onion/2016/06/05/world/asia/hong-kong-mark-tiananmen-square-anniversary.html?_r=0}{annual
vigils} for those killed in the Tiananmen Square massacre, newspapers
publishing scurrilous gossip about China's leaders, huge demonstrations
for free elections.

But by deciding to intervene in a local court case and essentially
blocking
\href{http://www.nytimes3xbfgragh.onion/2016/11/05/world/asia/china-hong-kong-leung-yau-dispute.html}{two
politicians from taking seats in Hong Kong's legislature}, China
signaled more clearly than ever on Monday that there was a limit to its
tolerance in this former British colony, which was promised a ``high
degree of autonomy'' in an international treaty.

The two young activists who are testing that limit are advocates of
independence for Hong Kong. While being sworn in, they made a statement
of defiance against Chinese rule, using a
\href{http://www.nytimes3xbfgragh.onion/2016/10/13/world/asia/hong-kong-legislative-council.html}{crude
obscenity} and a term that many consider a slur against Chinese people.

In acting against them, the government of President Xi Jinping has
asserted new authority to set policy in Hong Kong, opening what could be
a more chaotic era here, in which elected officials are held to a vague
standard of political loyalty and blacklisted if they fall short.

The Communist Party's intervention in Hong Kong's independent legal
system could also damage the territory's reputation as an international
trade and finance hub in Asia. Many multinational corporations, banks
and law firms are based here because of the dependability and fairness
of the city's courts.

Businesses also find Hong Kong appealing because of its political
stability, but thousands of people demonstrated and clashed with the
police on Sunday night in anticipation of Beijing's action, which could
incite more street protests.

China's move came in the form of a rare interpretation of the Basic Law,
the charter that governs Hong Kong and that was negotiated with Britain
before the territory's
\href{http://www.nytimes3xbfgragh.onion/learning/general/onthisday/big/0630.html}{return
to Chinese rule} in 1997. The charter gives China's Parliament the right
to interpret the Basic Law, and the Communist leadership has done so
four other times since the handover.

Monday's ruling breaks new ground because it is the first time that
Beijing has acted in a pending court case without a request by the Hong
Kong government or judiciary, and because it appears to establish a
mechanism for the authorities to block critics of Communist rule from
taking elected office or even getting their names on ballots.

Some scholars said the
\href{http://news.xinhuanet.com/english/2016-11/07/c_135811504.htm}{decision}
went beyond interpreting the charter and amounted to rewriting the local
statute governing how officials are to be sworn in. It requires
lawmakers to read their oaths ``completely and solemnly,'' exactly as
written, and orders those who administer oaths to disqualify lawmakers
who alter or deliver the words in an ``insincere or undignified
manner,'' barring them from office without another chance to be sworn
in.

The decision also says lawmakers will be held liable if they violate
their oaths, but it provides no guidance on who has the power to
determine whether a lawmaker is in breach or what the punishment should
be. The fear is that this will inject a degree of arbitrariness into a
system that is based on rules underpinned by centuries of precedent
under British common law.

``Whether it would affect my seat is secondary,'' Nathan Law, 23, a new
member of the Legislative Council who advocates greater
self-determination for Hong Kong, said of the ruling by Beijing.
``What's most important is that the interpretation is vesting so much
power in a person to decide whether someone is sincere and allegiant
enough to take office, and there is no checks-and-balance against that
person.''

The Basic Law says little about oaths, only that officials must swear
allegiance ``in accordance with law'' to the ``Hong Kong Special
Administrative Region of the People's Republic of China.''

Two politicians
\href{http://www.nytimes3xbfgragh.onion/2016/09/05/world/asia/hong-kong-election.html}{elected
to the legislature} in September,
\href{http://www.nytimes3xbfgragh.onion/2016/09/02/world/asia/hong-kong-elections-legco.html}{Sixtus
Leung}, 30, known as Baggio, and
\href{http://www.nytimes3xbfgragh.onion/2016/11/05/world/asia/hong-kong-yau-wai-ching.html}{Yau
Wai-ching}, 25, set off the legal case by changing the wording of their
oaths, replacing the word China with ``Chee-na,'' a term that many find
offensive and that was used by Japan during World War II, when it
occupied much of China, including Hong Kong. Ms. Yau also inserted an
obscenity.

The Chinese government condemned the pair and labeled them threats to
national security for their advocacy of independence, and officials in
Beijing left little doubt in announcing Monday's decision that it was
intended to keep them out of office.

Li Fei, the chairman of China's parliamentary committee on the Basic
Law, compared Mr. Leung and Ms. Yau, and their supporters, to traitors
espousing a ``fascist'' line. ``There is a great patriotic tradition in
the Chinese nation,'' he said. ``All traitors and those who sell out
their countries will come to no good end.''

He added that the government's stance ``will not be ambiguous or
lenient.''

The United States, on the eve of its own presidential election, urged
China not to undermine the ``one country, two systems'' formula that has
protected basic civil liberties in Hong Kong.

The United States believes ``an open society with the highest possible
degree of autonomy and governed by the rule of law is essential for Hong
Kong's continued stability and prosperity as a special administrative
region of the People's Republic of China,'' the State Department
spokesman, Mark C. Toner, said on Monday.

Hong Kong's politicians reacted along predictable lines, with the
pro-Beijing establishment endorsing the decision and the pro-democracy
opposition criticizing it as an infringement on self-rule in the
territory. The unpopular chief executive, Leung Chun-ying, vowed to
fully enforce the decision, saying, ``This is about the country's unity
and sovereignty.''

Michael Tien, a pro-Beijing lawmaker in Hong Kong who endorsed the
ruling, said it could be used to screen future candidates running for
office and to challenge those who have already taken seats in the
70-member Legislative Council, including several opposition lawmakers
who have made statements in favor of self-determination or independence
for Hong Kong.

``This thing is expansive,'' Mr. Tien said in a telephone interview,
adding that Beijing's allies in Hong Kong could ask courts to rule on
whether these lawmakers were sincere in swearing allegiance or had
violated their oaths.

Such a process is likely to be contentious, said Simon Young, a legal
scholar at the University of Hong Kong, because the language of
Beijing's decision is short on specifics. It says nothing about whether
advocating self-determination or independence violates an officeholder's
oath, for example, or how to handle candidates who change their position
on the issue.

As a result, politicians such as Mr. Leung and Ms. Yau could seek to win
back their seats by renouncing support for independence and going back
to court. And that could lead Beijing to intervene further and issue new
decisions to stamp out people it wants to disqualify.

``This will possibly be the first of a possible series of
interpretations,'' Mr. Young said. In the meantime, he added, ``the
courts in Hong Kong will have to interpret the interpretation.''

The Chinese government's overarching goal is to crush a small but
growing independence movement in Hong Kong, which gained momentum after
Beijing rejected calls for free elections in the territory during the
enormous
\href{http://www.nytimes3xbfgragh.onion/2014/12/15/world/asia/three-months-of-protests-end-quietly-in-hong-kong.html}{pro-democracy
demonstrations of 2014}.

But by intervening in the legal case over Mr. Leung and Ms. Yau, Mr. Xi
has drawn more attention to their cause, and he risks provoking a
backlash that could strengthen it. In a scene that resembled the 2014
demonstrations, the police used pepper spray early Monday to battle
crowds of protesters who had gathered around the Chinese government's
liaison office in Hong Kong, some of whom were shouting, ``Hong Kong
independence.''

Advertisement

\protect\hyperlink{after-bottom}{Continue reading the main story}

\hypertarget{site-index}{%
\subsection{Site Index}\label{site-index}}

\hypertarget{site-information-navigation}{%
\subsection{Site Information
Navigation}\label{site-information-navigation}}

\begin{itemize}
\tightlist
\item
  \href{https://help.nytimes3xbfgragh.onion/hc/en-us/articles/115014792127-Copyright-notice}{©~2020~The
  New York Times Company}
\end{itemize}

\begin{itemize}
\tightlist
\item
  \href{https://www.nytco.com/}{NYTCo}
\item
  \href{https://help.nytimes3xbfgragh.onion/hc/en-us/articles/115015385887-Contact-Us}{Contact
  Us}
\item
  \href{https://www.nytco.com/careers/}{Work with us}
\item
  \href{https://nytmediakit.com/}{Advertise}
\item
  \href{http://www.tbrandstudio.com/}{T Brand Studio}
\item
  \href{https://www.nytimes3xbfgragh.onion/privacy/cookie-policy\#how-do-i-manage-trackers}{Your
  Ad Choices}
\item
  \href{https://www.nytimes3xbfgragh.onion/privacy}{Privacy}
\item
  \href{https://help.nytimes3xbfgragh.onion/hc/en-us/articles/115014893428-Terms-of-service}{Terms
  of Service}
\item
  \href{https://help.nytimes3xbfgragh.onion/hc/en-us/articles/115014893968-Terms-of-sale}{Terms
  of Sale}
\item
  \href{https://spiderbites.nytimes3xbfgragh.onion}{Site Map}
\item
  \href{https://help.nytimes3xbfgragh.onion/hc/en-us}{Help}
\item
  \href{https://www.nytimes3xbfgragh.onion/subscription?campaignId=37WXW}{Subscriptions}
\end{itemize}
