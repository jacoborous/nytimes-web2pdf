Sections

SEARCH

\protect\hyperlink{site-content}{Skip to
content}\protect\hyperlink{site-index}{Skip to site index}

\href{https://www.nytimes3xbfgragh.onion/section/world/middleeast}{Middle
East}

\href{https://myaccount.nytimes3xbfgragh.onion/auth/login?response_type=cookie\&client_id=vi}{}

\href{https://www.nytimes3xbfgragh.onion/section/todayspaper}{Today's
Paper}

\href{/section/world/middleeast}{Middle East}\textbar{}Plight of Houthi
Rebels Is Clear in Visit to Yemen's Capital

\url{https://nyti.ms/2g3RIaf}

\begin{itemize}
\item
\item
\item
\item
\item
\end{itemize}

Advertisement

\protect\hyperlink{after-top}{Continue reading the main story}

Supported by

\protect\hyperlink{after-sponsor}{Continue reading the main story}

\hypertarget{plight-of-houthi-rebels-is-clear-in-visit-to-yemens-capital}{%
\section{Plight of Houthi Rebels Is Clear in Visit to Yemen's
Capital}\label{plight-of-houthi-rebels-is-clear-in-visit-to-yemens-capital}}

\includegraphics{https://static01.graylady3jvrrxbe.onion/images/2016/11/24/world/26HOUTHIS-2/26HOUTHIS-2-articleInline.jpg?quality=75\&auto=webp\&disable=upscale}

By \href{http://www.nytimes3xbfgragh.onion/by/ben-hubbard}{Ben Hubbard}

\begin{itemize}
\item
  Nov. 26, 2016
\item
  \begin{itemize}
  \item
  \item
  \item
  \item
  \item
  \end{itemize}
\end{itemize}

SANA, Yemen --- Before the war, the Officers Club in downtown Sana was a
prime recreation destination, known for its pool and garden cafe.

Now, like much of Sana, the Yemeni capital, its bombed-out remnants are
controlled by gun-wielding rebels from the group known as the Houthis.
Dressed in ragtag uniforms and brimming with Islamist fervor, they
pointed out holes from airstrikes and the rubble that had once been the
Police Academy.

Still, they insisted that their seizure of the capital had been good for
\href{http://www.nytimes3xbfgragh.onion/topic/destination/yemen}{Yemen}.

``There was too much corruption and looting before,'' said Masoud Saad,
19, who had dropped out of middle school to become a fighter. ``We
wanted to present the true religion of God in a correct way.''

Once a provincial militant movement in the mountains of northern Yemen,
\href{http://www.nytimes3xbfgragh.onion/2015/01/21/world/middleeast/who-are-the-houthis-of-yemen.html}{the
Houthis} surged to prominence after they seized control of the country's
northwest in 2014. Since then, they have pushed the national government
into exile and set off a new Middle Eastern war in which they are in the
cross hairs of an intensive bombardment campaign by Saudi Arabia and a
coalition of Arab countries.

Now they are struggling to govern in the middle of a war that has ground
to a destructive stalemate.

In an interview in his car, because the Defense Ministry headquarters
had been bombed, Brig. Gen. Sharaf Luqman, a spokesman for Houthi-allied
military units, acknowledged that the front lines had scarcely moved in
the past year.

\includegraphics{https://static01.graylady3jvrrxbe.onion/images/2016/11/24/world/26HOUTHIS-1/26HOUTHIS-1-articleInline.jpg?quality=75\&auto=webp\&disable=upscale}

``We have lost everything, our infrastructure, and we have nothing left
to lose,'' he said. ``Now it is a long war of attrition.''

The Houthis' control of such key territory has made them essential to
international efforts to end the conflict, leaving policy makers and
negotiators struggling to figure out what they want. And the group's
anti-American stance rankles Washington, which used to count on the
Yemeni government as an ally against Al Qaeda and has aided Saudi Arabia
in its military campaign against the rebels.

The rebels' slogan is spray-painted on walls and checkpoints throughout
their territory: ``God is great. Death to America. Death to Israel.
Curse on the Jews. Victory for Islam.''

The Houthi movement began as a religious revival in the 1990s among
Zaydi Muslims, an Arab religious minority in northern Yemen who sought
to push back against efforts by Saudi Arabia to spread its
fundamentalist version of Sunni Islam.

The group takes its name from its founder, Hussein Badr Eddin al-Houthi,
who was killed by Yemeni forces in 2004. His followers launched an
insurgency against the government, and developed as a guerrilla force in
a series of civil wars.

That background of insurgency rooted in backwater parts of the Arab
world's poorest state forged the group into a strong fighting force but
gave it few skilled politicians, intellectuals or technocrats --- a
weakness glaringly apparent during a recent visit by New York Times
journalists in Sana.

Much of the Houthis' administration relies on civil servants who chafe
under their control and on followers of a former president,
\href{http://www.nytimes3xbfgragh.onion/2014/02/01/world/middleeast/even-out-of-office-a-wielder-of-great-power-in-yemen.html}{Ali
Abdullah Saleh}, who has allied with them.

Further impeding their efforts at governance is the Saudi bombardment,
which has
\href{http://www.nytimes3xbfgragh.onion/2016/11/14/world/middleeast/yemen-saudi-bombing-houthis-hunger.html}{gravely
damaged an already weak economy} and infrastructure.

In interviews during a recent trip to Yemen, Houthi leaders and fighters
described themselves as ``revolutionaries'' in the mold of Hezbollah in
Lebanon or Hamas in Gaza, saying their aim was to cleanse the country of
corrupt leaders they considered beholden to foreign powers. In
describing their goals, they spouted beliefs that often clashed with
their behavior.

``I saw that they stood with justice and the oppressed,'' said Majid
Ali, who dropped out of a Sana university to join the Houthis when they
seized the capital. ``The goal was not to take control, but to help the
oppressed and the weak.''

But their enemies in Yemen and Saudi Arabia insist that the Houthis are
a dangerous proxy force being used by Iran to expand its influence and
challenge Saudi influence.

Analysts and diplomats who follow Yemen say the reality is somewhere in
between. While commonly considered Shiite, the Houthis' Zaydi sect
differs significantly from Iran's official Shiite creed, and
historically ties between the Houthis and Iran were not strong.

Image

A camp of displaced families in Yemen in October.Credit...Tyler
Hicks/The New York Times

But their shared hatred for Saudi Arabia has brought them together in
the current conflict, and Iran has given the Houthis weapons and
technical help to attack Saudi forces along the border.

April Longley Alley, a Yemen analyst with the International Crisis
Group, said the Houthis' surge out of the north to seize the capital had
been opportunistic. Their objectives included gaining a decisive stake
in national decision-making and in Yemen's military and security
apparatus.

What remains unclear, she said, is how the war has changed those goals.

``Now that they are in the capital, the question is how much of a stake
do they think they can hold on to after this experience with
governance,'' she said.

During our 10-day trip to Sana and nearby provinces, it was clear that
the Houthis were in charge. Their authorities issued our visas,
determined what sites we could visit and assigned us a minder to make
sure we stuck to the program.

Houthi checkpoints dotted the roads, sometimes less than a mile apart,
and some of the scrappy young fighters who manned them struggled to read
our Houthi-issued permits before allowing us to pass. While this slowed
traffic, Houthi security measures have put a stop to the suicide
bombings and assassinations that used to be frequent in the capital,
perhaps their greatest achievement in governing.

When the internationally recognized Yemeni government of President Abdu
Rabbu Mansour Hadi fled Sana, much of the state bureaucracy remained.
Since then, the Houthis have worked with followers of Mr. Saleh, the
former president, who was
\href{http://www.nytimes3xbfgragh.onion/2012/02/22/world/middleeast/yemen-votes-to-remove-ali-abdullah-saleh.html}{forced
from power in 2012}, to extend their control over the weakened organs of
the state.

Image

Men praying at the Majara camp for displaced Yemenis, in the town of
Hajjah.Credit...Tyler Hicks/The New York Times

Many ministry buildings have been bombed by the Saudi-led coalition, and
those still intact are nearly empty, their employees staying home for
fear of airstrikes and because they are not being paid.

In many cases, the Houthis have installed their loyalists as overseers,
giving militants with few qualifications authority over civil servants
with significant experience.

One Sana-based businessman recalled going to a police station to file a
complaint and finding a dozen officers unable to take action without
orders from their new Houthi boss.

To visit Sana's main pediatric and maternity hospital, we had to get
permission from its new ``director,'' a Houthi in a robe and plastic
sandals who allowed that his sole qualification was a diploma in
nursing.

A nurse who had worked there for 16 years said he had not been paid in
two months and complained that the Houthis had no way to fund their
administration.

``And if you go out and protest to ask for your rights, they could take
you to prison,'' he said, speaking on the condition of anonymity for
fear of arrest. ``It's all armed militias.''

Image

A taxi driver in Sana. Some Yemenis who have been living under Houthi
rule said the Saudi intervention had unified diverse forces against a
common enemy.Credit...Tyler Hicks/The New York Times

Officially overseeing the Houthis' attempt to govern is the High
Political Council, formed this year.

In an interview in the Republican Palace, Yemen's equivalent of the
White House, the council's head, Saleh al-Sammad, called the body ``the
highest authority in the country,'' but acknowledged that the state's
main sources of income were out of its hands.

The council took another hit in September when President Hadi moved
Yemen's Central Bank, which paid the salaries of 1.2 million civil
servants, to the southern port city of Aden, where his rival
administration has a presence.

Mr. Sammad blamed Saudi Arabia, the United States and the United Nations
for Yemen's growing humanitarian crisis, but said it would only
intensify people's will to fight.

``Most of the Yemeni people are armed, and they consider Saudi Arabia
responsible for their humiliation,'' he said. ``The worse the economic
situation gets, the more people will be pushed toward confrontation and
the fronts.''

The Houthis' fighting mettle and alliance with some Yemeni military
units has enabled them to stage painful attacks on Saudi Arabia and fire
ballistic missiles over the border, killing Saudi soldiers and
civilians.

The group's endgame remains unclear, however. It has participated in
peace talks and agreed to a recent cease-fire, but the truce expired on
Monday, amid accusations by both sides of violations and with no sign of
when talks might resume.

Many Yemenis in the country's south and east oppose what they see as a
Houthi coup, and measuring the depth of their support in areas
controlled by the Houthis is difficult.

Human rights organizations have accused the Houthis of indiscriminate
attacks on civilian areas, arbitrary arrests of political opponents and
torture. But some Yemenis who have been living under Houthi rule said
the Saudi intervention had unified diverse forces against a common
enemy.

``What brought the army together with Ansar Allah?'' asked Tariq
Mohammed, a policeman in the town of Hajjah, using another name for the
Houthis. ``The aggression against the country. That is what caused us to
come together as one hand.''

Advertisement

\protect\hyperlink{after-bottom}{Continue reading the main story}

\hypertarget{site-index}{%
\subsection{Site Index}\label{site-index}}

\hypertarget{site-information-navigation}{%
\subsection{Site Information
Navigation}\label{site-information-navigation}}

\begin{itemize}
\tightlist
\item
  \href{https://help.nytimes3xbfgragh.onion/hc/en-us/articles/115014792127-Copyright-notice}{©~2020~The
  New York Times Company}
\end{itemize}

\begin{itemize}
\tightlist
\item
  \href{https://www.nytco.com/}{NYTCo}
\item
  \href{https://help.nytimes3xbfgragh.onion/hc/en-us/articles/115015385887-Contact-Us}{Contact
  Us}
\item
  \href{https://www.nytco.com/careers/}{Work with us}
\item
  \href{https://nytmediakit.com/}{Advertise}
\item
  \href{http://www.tbrandstudio.com/}{T Brand Studio}
\item
  \href{https://www.nytimes3xbfgragh.onion/privacy/cookie-policy\#how-do-i-manage-trackers}{Your
  Ad Choices}
\item
  \href{https://www.nytimes3xbfgragh.onion/privacy}{Privacy}
\item
  \href{https://help.nytimes3xbfgragh.onion/hc/en-us/articles/115014893428-Terms-of-service}{Terms
  of Service}
\item
  \href{https://help.nytimes3xbfgragh.onion/hc/en-us/articles/115014893968-Terms-of-sale}{Terms
  of Sale}
\item
  \href{https://spiderbites.nytimes3xbfgragh.onion}{Site Map}
\item
  \href{https://help.nytimes3xbfgragh.onion/hc/en-us}{Help}
\item
  \href{https://www.nytimes3xbfgragh.onion/subscription?campaignId=37WXW}{Subscriptions}
\end{itemize}
