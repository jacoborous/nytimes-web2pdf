Sections

SEARCH

\protect\hyperlink{site-content}{Skip to
content}\protect\hyperlink{site-index}{Skip to site index}

\href{https://www.nytimes3xbfgragh.onion/section/politics}{Politics}

\href{https://myaccount.nytimes3xbfgragh.onion/auth/login?response_type=cookie\&client_id=vi}{}

\href{https://www.nytimes3xbfgragh.onion/section/todayspaper}{Today's
Paper}

\href{/section/politics}{Politics}\textbar{}Donald Trump, Laying Out
Foreign Policy, Promises Coherence

\url{https://nyti.ms/1Tes5w3}

\begin{itemize}
\item
\item
\item
\item
\item
\end{itemize}

Advertisement

\protect\hyperlink{after-top}{Continue reading the main story}

Supported by

\protect\hyperlink{after-sponsor}{Continue reading the main story}

\hypertarget{donald-trump-laying-out-foreign-policy-promises-coherence}{%
\section{Donald Trump, Laying Out Foreign Policy, Promises
Coherence}\label{donald-trump-laying-out-foreign-policy-promises-coherence}}

\includegraphics{https://static01.graylady3jvrrxbe.onion/images/2016/04/27/us/27live-trump2/27live-trump2-videoSixteenByNineJumbo1600.jpg}

By \href{http://www.nytimes3xbfgragh.onion/by/mark-landler}{Mark
Landler} and
\href{http://www.nytimes3xbfgragh.onion/by/ashley-parker}{Ashley Parker}

\begin{itemize}
\item
  April 27, 2016
\item
  \begin{itemize}
  \item
  \item
  \item
  \item
  \item
  \end{itemize}
\end{itemize}

WASHINGTON --- Donald J. Trump, exuding confidence after his resounding
primary victories in the East,
\href{http://www.nytimes3xbfgragh.onion/2016/04/28/us/politics/transcript-trump-foreign-policy.html}{promised
a foreign policy} on Wednesday that he said would put ``America first.''
He castigated President Obama and Hillary Clinton, a former secretary of
state and a possible opponent in the general election, for what he
described as a string of missteps that have disillusioned the nation's
allies and emboldened its rivals.

Mr. Trump, the front-runner for the Republican presidential nomination,
pledged a major buildup of the military, the swift destruction of the
Islamic State and the rejection of trade deals that he said tied the
nation's hands. But he also pointedly rejected the nation-building of
the George W. Bush administration, reminding his audience that he had
opposed the Iraq war.

``America is going to be strong again; America is going to be great
again; it's going to be a friend again,'' Mr. Trump said. ``We're going
to finally have a coherent foreign policy, based on American interests
and the shared interests of our allies.''

``The world must know that we do not go abroad in search of enemies,
that we are always happy when old enemies become friends and when old
friends become allies,'' he added. ``That's what we want: We want to
bring peace to the world.''

For Mr. Trump, whose campaign appearances are often a gleeful exercise
in showmanship and off-the-cuff wisecracks, the speech had all the
trappings of a serious address. Standing beneath a twinkling chandelier
in a Washington hotel ballroom, backed by American flags and facing a
sedate, largely gray-haired audience, a measured Mr. Trump read his
remarks from a teleprompter, staying almost completely on script.

But if Mr. Trump adopted establishment trappings, his speech still had
an insurgent tone. He criticized allies in Europe and Asia for not
bearing the burden of their own defense, he said he would try to mend
fences with Russia, and he assailed his opponents for being overly
aggressive in foreign affairs. Mr. Trump said he had no plans to take
advice from the foreign policy elite, and his agenda reflected that ---
a mélange of ideas that defied Republican and Democratic orthodoxy.

There were paradoxes throughout Mr. Trump's speech. He called for a
return to the coherence of America's foreign policy during the Cold War.
Yet he was openly suspicious of the institutions that undergirded that
era. He promised to eradicate the Islamic State, but said the campaign
against extremism --- or as he called it, ``radical Islam'' --- was as
much a philosophical struggle as a military one.

``Our friends and enemies must know that if I draw a line in the sand, I
will enforce that line in the sand --- believe me,'' Mr. Trump said.
``However, unlike other candidates for the presidency, foreign
aggression will not be my first instinct.'' He did not mention anyone by
name, though his strongest Republican opponent, Senator Ted Cruz of
Texas, has threatened to carpet-bomb the Islamic State until the desert
sand glows.

Mr. Trump's speech drew negative reaction across the political spectrum.
Senator Lindsey Graham, Republican of South Carolina, posted on Twitter
that ``Ronald Reagan must be rolling over in his grave.'' Lanhee Chen, a
fellow at the Hoover Institution who advised Mitt Romney, the 2012
Republican nominee, said ``There was clearly an isolationist strain to
the speech, but that runs into the reality of the world that we live
in.''

\includegraphics{https://static01.graylady3jvrrxbe.onion/images/2016/04/28/us/28trump-china-web/28trump-china-web-videoSixteenByNine3000.jpg}

R. Nicholas Burns, a former senior State Department official under Mr.
Bush who now advises Mrs. Clinton, said, ``He's casting these
thunderbolts and threats at our allies, and yet there was almost a
kid-glove treatment of Russia and China.''

Even Mr. Trump's embrace of the slogan ``America first'' raised
eyebrows, with critics noting that it was popularized in the 1930s by
the aviator Charles A. Lindbergh and other isolationists who opposed the
United States' entering World War II. ``To fly the banner of America
First shows that he has historical amnesia or just doesn't understand
history,'' Mr. Burns said.

In one of his few concrete proposals, Mr. Trump said he would convene
summit meetings in Europe and Asia to overhaul NATO and rebalance
nuclear security arrangements with Japan and South Korea. He did not
repeat a
\href{http://www.nytimes3xbfgragh.onion/2016/03/27/us/politics/donald-trump-foreign-policy.html}{statement
he made to The New York Times} that those countries should consider
acquiring their own nuclear weapons.

Mr. Trump was scathing about the Obama administration's intervention in
Libya, lashing Mrs. Clinton to the policy, which he said had left a
security vacuum filled by the Islamic State. He also faulted Mr. Obama
for his failure to enforce the red line he laid down in Syria. Yet Mr.
Trump made clear he would use military force only as a last resort.

``We're getting out of the nation-building business and instead focusing
on creating stability on the world,'' Mr. Trump said.

\href{https://www.nytimes3xbfgragh.onion/interactive/2016/04/27/upshot/donald-trumps-path-to-the-nomination-contest-by-contest.html}{}

\includegraphics{https://static01.graylady3jvrrxbe.onion/images/2016/04/27/upshot/donald-trumps-path-to-the-nomination-contest-by-contest-1461774507837/donald-trumps-path-to-the-nomination-contest-by-contest-1461774507837-articleLarge-v5.png}

\hypertarget{donald-trumps-path-to-the-republican-nomination-contest-by-contest}{%
\subsection{Donald Trump's Path to the Republican Nomination, Contest by
Contest}\label{donald-trumps-path-to-the-republican-nomination-contest-by-contest}}

A guide to following the remaining Republican primary contests like a
political pro.

On pressing issues like counterterrorism, Mr. Trump broke little new
ground. He declined, for example, to give any details on how he planned
to destroy the Islamic State to avoid tipping the military's hand,
beyond vowing that ``they will be gone quickly.''

Mr. Trump repeated his desire to seek improved relations with Russia and
its president, Vladimir V. Putin --- a strategy that carried echoes of
Mr. Obama's attempt to ``reset'' relations with Russia after its
invasion of Georgia in 2008. But he said his skills as a deal maker
would make him more successful at it.

``I see improved relations with Russia, from a position of strength, as
possible,'' Mr. Trump said. ``Some say the Russians won't be reasonable;
I intend to find out.''

In another echo of Mr. Obama, Mr. Trump said he would seek advice from
outside the foreign policy establishment. He said he would choose ``the
best minds'' with practical solutions, rather than people with ``perfect
résumés'' and records of failure around the world. He did not mention
any names.

Mr. Trump promised to make the United States more dependable in the eye
of its friends and allies, and more respected by its enemies. Yet just
moments earlier, he also advocated increased unpredictability. ``We have
to be unpredictable,'' he said. ``And we have to be unpredictable
starting now.''

\includegraphics{https://static01.graylady3jvrrxbe.onion/images/2016/04/27/us/27video-trumpjobs/27video-trumpjobs-videoSixteenByNineJumbo1600.jpg}

The relatively small, invitation-only crowd consisted of journalists
(seated in the back), and seven rows of guests --- a largely
inside-the-Beltway crowd that included Senator Jeff Sessions of Alabama,
a senior policy adviser to the campaign; a handful of House members; and
Grover Norquist, the president of Americans for Tax Reform.

Mr. Trump recently overhauled his campaign team, bringing in new
advisers who have begun to impose discipline and organization to what
was often a chaotic campaign. The candidate himself has taken steps
toward ``more presidential'' behavior, as he calls it.

Mr. Trump, who delights in mocking scripted candidates who use
teleprompters, delivered his speech with the help of two teleprompters,
and aides said he had practiced with them over the weekend. The New York
billionaire worked on his foreign policy speech for more than a week,
according to an aide, with the help of some advisers his campaign would
not identify.

``The speech is his words and his thoughts,'' said Paul Manafort, Mr.
Trump's newly installed campaign chief.

Mr. Trump was introduced by Zalmay Khalilzad, an Afghan-born diplomat
who was Mr. Bush's ambassador to Afghanistan and Iraq, and is closely
identified with the American wars in those countries. Mr. Khalilzad said
afterward that he was not advising Mr. Trump formally or informally, and
that the two men met for the first time in a holding room adjacent to
where he delivered his speech.

Advertisement

\protect\hyperlink{after-bottom}{Continue reading the main story}

\hypertarget{site-index}{%
\subsection{Site Index}\label{site-index}}

\hypertarget{site-information-navigation}{%
\subsection{Site Information
Navigation}\label{site-information-navigation}}

\begin{itemize}
\tightlist
\item
  \href{https://help.nytimes3xbfgragh.onion/hc/en-us/articles/115014792127-Copyright-notice}{©~2020~The
  New York Times Company}
\end{itemize}

\begin{itemize}
\tightlist
\item
  \href{https://www.nytco.com/}{NYTCo}
\item
  \href{https://help.nytimes3xbfgragh.onion/hc/en-us/articles/115015385887-Contact-Us}{Contact
  Us}
\item
  \href{https://www.nytco.com/careers/}{Work with us}
\item
  \href{https://nytmediakit.com/}{Advertise}
\item
  \href{http://www.tbrandstudio.com/}{T Brand Studio}
\item
  \href{https://www.nytimes3xbfgragh.onion/privacy/cookie-policy\#how-do-i-manage-trackers}{Your
  Ad Choices}
\item
  \href{https://www.nytimes3xbfgragh.onion/privacy}{Privacy}
\item
  \href{https://help.nytimes3xbfgragh.onion/hc/en-us/articles/115014893428-Terms-of-service}{Terms
  of Service}
\item
  \href{https://help.nytimes3xbfgragh.onion/hc/en-us/articles/115014893968-Terms-of-sale}{Terms
  of Sale}
\item
  \href{https://spiderbites.nytimes3xbfgragh.onion}{Site Map}
\item
  \href{https://help.nytimes3xbfgragh.onion/hc/en-us}{Help}
\item
  \href{https://www.nytimes3xbfgragh.onion/subscription?campaignId=37WXW}{Subscriptions}
\end{itemize}
