Sections

SEARCH

\protect\hyperlink{site-content}{Skip to
content}\protect\hyperlink{site-index}{Skip to site index}

\href{https://www.nytimes3xbfgragh.onion/section/food}{Food}

\href{https://myaccount.nytimes3xbfgragh.onion/auth/login?response_type=cookie\&client_id=vi}{}

\href{https://www.nytimes3xbfgragh.onion/section/todayspaper}{Today's
Paper}

\href{/section/food}{Food}\textbar{}Llama Inn Lights the Way to Peru by
Way of Williamsburg

\url{https://nyti.ms/1Rma16y}

\begin{itemize}
\item
\item
\item
\item
\item
\item
\end{itemize}

Advertisement

\protect\hyperlink{after-top}{Continue reading the main story}

Supported by

\protect\hyperlink{after-sponsor}{Continue reading the main story}

\href{/column/restaurant-review}{Restaurant Review}

\hypertarget{llama-inn-lights-the-way-to-peru-by-way-of-williamsburg}{%
\section{Llama Inn Lights the Way to Peru by Way of
Williamsburg}\label{llama-inn-lights-the-way-to-peru-by-way-of-williamsburg}}

\href{https://www.nytimes3xbfgragh.onion/slideshow/2016/02/10/dining/llama-inn.html}{}

\hypertarget{llama-inn}{%
\subsection{Llama Inn}\label{llama-inn}}

13 Photos

View Slide Show ›

\includegraphics{https://static01.graylady3jvrrxbe.onion/images/2016/02/10/dining/10REST-slide-JIX6/10REST-slide-JIX6-articleLarge.jpg?quality=75\&auto=webp\&disable=upscale}

Cassandra Giraldo for The New York Times

\begin{itemize}
\tightlist
\item
  Llama Inn\\
  ★★ Latin American \$\$ 50 Withers Street 718-387-3434
\end{itemize}

\href{https://resy.com/cities/ny/llama-inn?utm_source=nyt\&utm_medium=restoprofile\&utm_campaign=affiliates\&aff_id=c1fe784}{Reserve
a Table}

When you make a reservation at an independently reviewed restaurant
through our site, we earn an affiliate commission.

By \href{http://www.nytimes3xbfgragh.onion/by/pete-wells}{Pete Wells}

\begin{itemize}
\item
  Feb. 9, 2016
\item
  \begin{itemize}
  \item
  \item
  \item
  \item
  \item
  \item
  \end{itemize}
\end{itemize}

A wedge of light burns in the night on one of Williamsburg's less
picturesque intersections, just below the dark rumble of the
Brooklyn-Queens Expressway. Down the block is an old-school
Brooklyn-Italian dining room serving linguine, veal parm and clams
casino. All around, there's hipster bait: fried chicken, pedigreed
pizza, cans of Narragansett, tacos from a truck in a bar's backyard.

Behind the windows on this triangular lot, the
\href{http://www.llamainnnyc.com/}{Llama Inn} is something different.

There are skewered beef hearts under a red mash of salsa that rattles
with the heat of rocoto peppers.

Bites of goat neck, thickly seared and braised until tender, are dark
under a glaze that gets its mouthwatering tang from chicha, a beer
brewed from Andean corn. Together with a fresh cilantro sauce, it makes
the goat so compulsively good that we were all clamoring for the last
forkful.

Chilly and firm pieces of fluke ceviche, starting to go opaque in the
acid of a smoky dashi, are wonderful to eat with soft bits of fried
sweet plantain and crisp chips of green plantain.

Ripe persimmon flesh and strips of raw sea bream under frizzled yuzu
threads sit in an instant marinade. The liquid is pale orange, tart, a
little fruity, spicy at the edges and generally delicious. You pick up a
spoon reflexively to slurp the sauce, even if you don't know this is the
way Peruvians eat tiradito.

What's been happening inside the Llama Inn since it opened in November
is not just different but very welcome. A careful kitchen is
interpreting the cuisine of Peru, whose vast catalog of ingredients and
wild juxtapositions of influences have not been especially well
displayed in New York, beyond a few ceviches and a lot of roast chicken.
Llama Inn does both, with excellent raw materials, a free hand and a
conviction that all of this should be fun. The restaurant's logo, a
debonair llama dressed in black tie, sets the tone.

The heart of the dining room, which is uncluttered and geometric in a
loosely midcentury way, is occupied by a bar island. Inside it,
bartenders mix drinks that are worth starting with and often sticking
with.

A kind of sangria made with pisco and kept on tap made me smile three
times: when I saw three frozen grapes on a toothpick serving as a
garnish, when I tasted the cloves and cinnamon in the red wine, and when
I asked for the drink by name: the Llama Del Rey. The cocktail called El
Chapo combines tequila and vermouth in case you want to know why they're
not combined more often. But the tall, icy, honey- and ginger-flavored
Matador Swizzle has to be the nicest thing that's happened to
amontillado in this part of Brooklyn.

It and a good number of other sherries are served by the glass, well
chilled. ``Our Peruvian clientele likes it that way,'' my server said
when he caught me warming the glass in my palms to help the flavors
uncoil.

Past some low-lying booths and banquettes is a long counter behind which
the 16 items on the menu are prepared. The chef, Erik Ramirez, worked at
\href{http://www.nytimes3xbfgragh.onion/2015/03/18/dining/restaurant-review-eleven-madison-park-in-midtown-south.html}{Eleven
Madison Park} before running the kitchen at
\href{http://www.rayminyc.com/}{Raymi} in Manhattan. His parents took
him on trips from New Jersey, where he grew up, to Lima, where they had.
But as an adult, he fell out of touch with Peruvian food. He had been
cooking professionally for several years when an employer suggested it
was time to pull up his memories of ceviche.

Sometimes Mr. Ramirez borrows an ingredient or two from Peru. For his
surprisingly good beet and goat cheese salad, he takes a minty herb
called muña and tart yellow gooseberries, native to the Andes.

The most unusual thing about the fluffy drift of quinoa tossed with
avocado, bacon, cashews and caramelized bananas is that Mr. Ramirez has
come up with a quinoa salad that isn't depressing. It's actually kind of
lovable.

At other times, he lifts and twists entire recipes. The roast whole
chicken, with smoke in its meat and blackened chile rub on its skin,
isn't so much an alteration of the Peruvian standard as an
intensification. The bird comes with three sauces, each spicy in a
different way. One, the creamy yellow salsa huancaína, is also squiggled
over a daunting heap of fried potato wedges that comes with the chicken.

I would enjoy meeting regularly with this chicken, but it probably won't
happen until Mr. Ramirez starts selling it in smaller portions.
Available only as a whole bird for \$40, Llama Inn's chicken is another
symptom of the sickness plaguing restaurant menus under the heading
``large format.'' Sized and priced for more than one person,
large-format dishes attract eyeballs online and in the flesh.

They can look cynical, though, when they could be scaled down to
smaller, cheaper portions as easily as Llama Inn's chicken or its take
on lomo saltado. This jumble of onions, tomatoes, cilantro, French fries
and thick, soy-marinated beef tenderloin slices makes a voluptuously
good taco when pinched inside a thin, crisp-edge scallion crepe. But
it's sold in a regal \$48 platter that takes two or three people to
handle.

The simplest dessert is hard to argue with, three hot bracelets of fried
dough called picarones soaked in a raw-sugar syrup. My server lobbied
for a different one with lucuma and chocolate made from Peruvian beans
specifically for the restaurant, but it never arrived after I ordered
it.

I wouldn't mention it except that an order went missing every time I ate
at Llama Inn. The first night this happened, we were presented with
dessert menus and we pointed out that the chicken we had asked for still
hadn't been served. When I later told one of the people who had been
with me that it had happened again, he said, ``What do they have against
their own magnificent chicken?''

With any luck, they'll get their orders straight before summer, when the
door at the top of a staircase in the dining room will open to the roof
deck. It would be nice to sit there with a chicken, or maybe a
half-chicken, and raise a Llama del Rey to the trucks on the expressway.

Advertisement

\protect\hyperlink{after-bottom}{Continue reading the main story}

\hypertarget{site-index}{%
\subsection{Site Index}\label{site-index}}

\hypertarget{site-information-navigation}{%
\subsection{Site Information
Navigation}\label{site-information-navigation}}

\begin{itemize}
\tightlist
\item
  \href{https://help.nytimes3xbfgragh.onion/hc/en-us/articles/115014792127-Copyright-notice}{©~2020~The
  New York Times Company}
\end{itemize}

\begin{itemize}
\tightlist
\item
  \href{https://www.nytco.com/}{NYTCo}
\item
  \href{https://help.nytimes3xbfgragh.onion/hc/en-us/articles/115015385887-Contact-Us}{Contact
  Us}
\item
  \href{https://www.nytco.com/careers/}{Work with us}
\item
  \href{https://nytmediakit.com/}{Advertise}
\item
  \href{http://www.tbrandstudio.com/}{T Brand Studio}
\item
  \href{https://www.nytimes3xbfgragh.onion/privacy/cookie-policy\#how-do-i-manage-trackers}{Your
  Ad Choices}
\item
  \href{https://www.nytimes3xbfgragh.onion/privacy}{Privacy}
\item
  \href{https://help.nytimes3xbfgragh.onion/hc/en-us/articles/115014893428-Terms-of-service}{Terms
  of Service}
\item
  \href{https://help.nytimes3xbfgragh.onion/hc/en-us/articles/115014893968-Terms-of-sale}{Terms
  of Sale}
\item
  \href{https://spiderbites.nytimes3xbfgragh.onion}{Site Map}
\item
  \href{https://help.nytimes3xbfgragh.onion/hc/en-us}{Help}
\item
  \href{https://www.nytimes3xbfgragh.onion/subscription?campaignId=37WXW}{Subscriptions}
\end{itemize}
