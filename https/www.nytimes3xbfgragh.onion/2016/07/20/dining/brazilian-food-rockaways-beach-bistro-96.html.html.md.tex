Sections

SEARCH

\protect\hyperlink{site-content}{Skip to
content}\protect\hyperlink{site-index}{Skip to site index}

\href{https://www.nytimes3xbfgragh.onion/section/food}{Food}

\href{https://myaccount.nytimes3xbfgragh.onion/auth/login?response_type=cookie\&client_id=vi}{}

\href{https://www.nytimes3xbfgragh.onion/section/todayspaper}{Today's
Paper}

\href{/section/food}{Food}\textbar{}A Bit of Brazil Washes Up at Beach
Bistro 96 in Queens

\url{https://nyti.ms/29RP3MJ}

\begin{itemize}
\item
\item
\item
\item
\item
\item
\end{itemize}

Advertisement

\protect\hyperlink{after-top}{Continue reading the main story}

Supported by

\protect\hyperlink{after-sponsor}{Continue reading the main story}

\href{/column/hungry-city}{Hungry City}

\hypertarget{a-bit-of-brazil-washes-up-at-beach-bistro-96-in-queens}{%
\section{A Bit of Brazil Washes Up at Beach Bistro 96 in
Queens}\label{a-bit-of-brazil-washes-up-at-beach-bistro-96-in-queens}}

\href{https://www.nytimes3xbfgragh.onion/slideshow/2016/07/20/dining/beach-bistro-96-review.html}{}

\hypertarget{beach-bistro-96}{%
\subsection{Beach Bistro 96}\label{beach-bistro-96}}

10 Photos

View Slide Show ›

\includegraphics{https://static01.graylady3jvrrxbe.onion/images/2016/07/20/dining/20HUNGRY-BEACH-BISTRO-slide-HMG3/20HUNGRY-BEACH-BISTRO-slide-HMG3-articleLarge.jpg?quality=75\&auto=webp\&disable=upscale}

An Rong Xu for The New York Times

\begin{itemize}
\tightlist
\item
  Beach Bistro 96\\
  Brazilian \$\$ 95-19 Rockaway Beach Boulevard 718-474-6000
\end{itemize}

By Ligaya Mishan

\begin{itemize}
\item
  July 14, 2016
\item
  \begin{itemize}
  \item
  \item
  \item
  \item
  \item
  \item
  \end{itemize}
\end{itemize}

The crepe is moon-white. Folded, it looks like a beaded clutch, studded
with pearls from the world's tiniest oysters.

The pearls are hydrated tapioca flour, sifted into a dry hot pan until
they stick together. The result looks brittle, but under the teeth it
sinks and clings, as chewy inside as Japanese mochi.

At Beach Bistro 96, a low-slung Brazilian sea shack that opened in May a
block from Rockaway Beach in Queens, the crepe may come engorged with
guacamole, whole spinach leaves and relaxed, if not quite melted,
mozzarella; or, for dessert, strawberries striped with condensed milk,
under an anarchy of coconut flakes like a parade's morning after.

On the menu, it is called simply tapioca, which may confuse those for
whom the name conjures up milky pudding or the little eyeballs in bubble
tea. The word belonged first to Tupi, the language of Brazil's
indigenous coastal tribes, whose other gifts to English include
``jaguar'' and ``piranha.''

Tapioca flour likewise lends stretch to pão de queijo, Brazilian cousins
to French gougères, rich with eggs and Parmesan beaten into the dough.
The ones here are among the best I've had, barely there puffs like held
breaths, with stray volcanic fissures and an escalation of texture, from
thin crispy shell to implosion.

There is an elasticity to the mood here, too. The chef, Carlos Varella,
42, grew up in the Brazilian port town Santos, where his parents ran a
restaurant. He learned to surf at age 7 and turned pro in his late
teens, traveling from break to break while modeling on the side.

In 2006 he settled in Manhattan with Andressa Junqueira, a fellow model
and Brazilian, from Belo Horizonte; they married in Central Park and
made pilgrimages to the Rockaways during hurricane swells.

Only recently did he turn to the kitchen, working the line at
\href{http://www.pancany.com/\#happy-hour}{Panca}, a Peruvian restaurant
in the West Village, and studying at the
\href{http://www.internationalculinarycenter.com/}{International
Culinary Center}.

His menu thus far is modest and mostly snacks, like coxinha, a giant
teardrop of a croquette (the shape is meant to evoke a drumstick) creamy
with chicken and Catupiry, a brand of ricottalike Brazilian cheese.

Pasteis are empanadas by another name, flaking dough crimped around
ground beef set off by mustard, green olives and paprika, or cosseting
guava jam and mozzarella, the swoony pair that the Brazilians call Romeu
e Julieta. Quinotto turns quinoa into risotto, the Andean grains clotted
with heavy cream and Parmesan, yet somehow still fluffy.

Among the few entrees, picanha, a plain-spoken slab of beef sirloin cap
rubbed with salt, arrives glowering from the grill: respectable, if
unsurprising. Twice a week, the menu promises feijoada, Brazil's
national stew of disintegrating black beans and deep cuts of pork, but,
sadly, there was none when I visited. Instead, I was offered chicken
stroganoff, a Russian classic mysteriously beloved by Brazilians, here
tinged with soy sauce and, Mr. Varella explained, ``a lot of ketchup.''
It may be a dish only a nostalgist could love.

Beach Bistro 96 is more address than name, marking where Rockaway Beach
Boulevard meets 96th Street. The one outward hint of culinary allegiance
is a sign on the burnt-orange storefront that says Terra Brasilis.
Inside, banana trees crowd the wallpaper. A surfboard leans in a corner,
although not one of Mr. Varella's: Those belong to the waves.

He and Ms. Junqueira, 33, moved to the Rockaways three months ago with
their two young children. She is a serene, capable presence in the
ramshackle dining room, unfolding slat tables while balancing her
7-month-old son, Kalani, on her hip, and dispensing glasses of caju
juice, a nutty extract from the false fruit of the cashew tree, and
excellent near-caipirinhas, with seltzer a chaste stand-in for cachaça.

I couldn't help worrying how the restaurant will last the winter, its
front window just a hole with wooden shutters. (Mr. Varella said he
planned to install glass.) How will they survive, this couple who have
stepped away from the city's maelstrom and staked their future on a
tenuous little shanty, by a beach that, come dusk, is an aftermath of
gravelly sand tagged with chip bags and candy-bar foil?

Still, it's beautiful. Maybe they know something about how to live that
I don't.

Advertisement

\protect\hyperlink{after-bottom}{Continue reading the main story}

\hypertarget{site-index}{%
\subsection{Site Index}\label{site-index}}

\hypertarget{site-information-navigation}{%
\subsection{Site Information
Navigation}\label{site-information-navigation}}

\begin{itemize}
\tightlist
\item
  \href{https://help.nytimes3xbfgragh.onion/hc/en-us/articles/115014792127-Copyright-notice}{©~2020~The
  New York Times Company}
\end{itemize}

\begin{itemize}
\tightlist
\item
  \href{https://www.nytco.com/}{NYTCo}
\item
  \href{https://help.nytimes3xbfgragh.onion/hc/en-us/articles/115015385887-Contact-Us}{Contact
  Us}
\item
  \href{https://www.nytco.com/careers/}{Work with us}
\item
  \href{https://nytmediakit.com/}{Advertise}
\item
  \href{http://www.tbrandstudio.com/}{T Brand Studio}
\item
  \href{https://www.nytimes3xbfgragh.onion/privacy/cookie-policy\#how-do-i-manage-trackers}{Your
  Ad Choices}
\item
  \href{https://www.nytimes3xbfgragh.onion/privacy}{Privacy}
\item
  \href{https://help.nytimes3xbfgragh.onion/hc/en-us/articles/115014893428-Terms-of-service}{Terms
  of Service}
\item
  \href{https://help.nytimes3xbfgragh.onion/hc/en-us/articles/115014893968-Terms-of-sale}{Terms
  of Sale}
\item
  \href{https://spiderbites.nytimes3xbfgragh.onion}{Site Map}
\item
  \href{https://help.nytimes3xbfgragh.onion/hc/en-us}{Help}
\item
  \href{https://www.nytimes3xbfgragh.onion/subscription?campaignId=37WXW}{Subscriptions}
\end{itemize}
