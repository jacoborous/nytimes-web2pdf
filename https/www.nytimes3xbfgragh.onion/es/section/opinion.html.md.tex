Sections

SEARCH

\protect\hyperlink{site-content}{Skip to
content}\protect\hyperlink{site-index}{Skip to site index}

\href{https://www.nytimes3xbfgragh.onion/es/section/opinion}{Opinión}

\href{https://myaccount.nytimes3xbfgragh.onion/auth/login?response_type=cookie\&client_id=vi}{}

\href{https://www.nytimes3xbfgragh.onion/section/todayspaper}{Today's
Paper}

Advertisement

\protect\hyperlink{after-top}{Continue reading the main story}

Supported by

\protect\hyperlink{after-sponsor}{Continue reading the main story}

\hypertarget{en-espauxf1ol}{%
\subsubsection{\texorpdfstring{\href{/es/}{en
Español}}{en Español}}\label{en-espauxf1ol}}

\hypertarget{opiniuxf3n}{%
\section{Opinión}\label{opiniuxf3n}}

\hypertarget{highlights}{%
\subsection{Highlights}\label{highlights}}

\begin{enumerate}
\def\labelenumi{\arabic{enumi}.}
\item
  \includegraphics{https://static01.graylady3jvrrxbe.onion/images/2020/08/03/opinion/03Keyssar-ES-1/03keyssarWeb-jumbo.jpg}

  \hypertarget{comentario}{%
  \subsubsection{Comentario}\label{comentario}}

  \hypertarget{el-colegio-electoral-de-estados-unidos-la-poco-conocida-historia-que-explica-su-vigencia}{%
  \subsection{\texorpdfstring{\href{/es/2020/08/03/espanol/opinion/colegio-electoral-estados-unidos.html}{El
  Colegio Electoral de Estados Unidos: la poco conocida historia que
  explica su
  vigencia}}{El Colegio Electoral de Estados Unidos: la poco conocida historia que explica su vigencia}}\label{el-colegio-electoral-de-estados-unidos-la-poco-conocida-historia-que-explica-su-vigencia}}

  Aunque ha habido intentos recurrentes de reformar el complejo método
  de elegir presidentes en ese país, las políticas raciales han tenido
  un lugar protagónico en impedir cambios.

  Por Alexander Keyssar
\item
  \includegraphics{https://static01.graylady3jvrrxbe.onion/images/2020/08/01/multimedia/01Martinez-ES/merlin_175051002_8b6c73d6-78b1-44b6-9806-b42f2b7f3923-videoLarge.jpg}

  \hypertarget{comentario-1}{%
  \subsubsection{Comentario}\label{comentario-1}}

  \hypertarget{en-el-salvador-todos-han-negociado-con-las-pandillas}{%
  \subsection{\texorpdfstring{\href{/es/2020/08/02/espanol/opinion/pandillas-el-salvador.html}{En
  El Salvador todos han negociado con las
  pandillas}}{En El Salvador todos han negociado con las pandillas}}\label{en-el-salvador-todos-han-negociado-con-las-pandillas}}

  Dialogar con las pandillas ha sido una realidad en el país: políticos
  de todos los colores lo han hecho por casi una década, pero solo unos
  cuantos han sido perseguidos por hacerlo. ¿Es una utopía transparentar
  esos pactos?

  Por Óscar Martínez
\item
  \includegraphics{https://static01.graylady3jvrrxbe.onion/images/2020/08/02/multimedia/02Carrion-ES/merlin_171683256_836a51a0-dc07-4047-b169-4bbbb85b62b4-videoLarge.jpg}

  \hypertarget{comentario-2}{%
  \subsubsection{Comentario}\label{comentario-2}}

  \hypertarget{los-algoritmos-son-los-nuevos-editores}{%
  \subsection{\texorpdfstring{\href{/es/2020/08/02/espanol/opinion/facebook-amazon-instagram.html}{Los
  algoritmos son los nuevos
  editores}}{Los algoritmos son los nuevos editores}}\label{los-algoritmos-son-los-nuevos-editores}}

  ¿Qué tienen en común Instagram, YouTube, Facebook, Amazon, Weibo y
  Twitter? No son solo redes sociales o plataformas: son los grandes
  editores de nuestra realidad.

  Por Jorge Carrión
\item
  \includegraphics{https://static01.graylady3jvrrxbe.onion/images/2020/08/01/multimedia/01Ramos-ES/merlin_175175088_ad2b68a7-8076-4175-a205-2f0a1352507f-videoLarge.jpg}

  \hypertarget{comentario-3}{%
  \subsubsection{Comentario}\label{comentario-3}}

  \hypertarget{tentaciones-autoritarias-cuxf3mo-amuxe9rica-latina-nos-preparuxf3-para-trump}{%
  \subsection{\texorpdfstring{\href{/es/2020/08/01/espanol/opinion/trump-autoritarismo.html}{Tentaciones
  autoritarias: cómo América Latina nos preparó para
  Trump}}{Tentaciones autoritarias: cómo América Latina nos preparó para Trump}}\label{tentaciones-autoritarias-cuxf3mo-amuxe9rica-latina-nos-preparuxf3-para-trump}}

  La democracia en Estados Unidos está a prueba. Quienes hemos vivido o
  trabajado en la región, conocemos bien de mandatarios que juegan con
  los límites de su poder. Adiós al ``excepcionalismo estadounidense''.

  Por Jorge Ramos
\end{enumerate}

Advertisement

\protect\hyperlink{after-mid1}{Continue reading the main story}

\begin{itemize}
\tightlist
\item
  \protect\hyperlink{stream-panel}{Lo más reciente}
\item
  Buscar
\end{itemize}

\begin{enumerate}
\def\labelenumi{\arabic{enumi}.}
\item
  \href{/es/2020/08/04/espanol/opinion/bolsonaro-oficina-odio-brasil.html}{}

  \includegraphics{https://static01.graylady3jvrrxbe.onion/images/2020/08/05/opinion/05campos/04campos-thumbWide.jpg?quality=75\&auto=webp\&disable=upscale}

  \hypertarget{comentario-4}{%
  \subsubsection{Comentario}\label{comentario-4}}

  \hypertarget{por-quuxe9-los-brasileuxf1os-deberuxedan-temer-a-la-oficina-de-odio}{%
  \subsection{Por qué los brasileños deberían temer a la oficina de
  odio}\label{por-quuxe9-los-brasileuxf1os-deberuxedan-temer-a-la-oficina-de-odio}}

  El presidente Jair Bolsonaro, sus hijos y aliados han sembrado el odio
  en línea contra las instituciones que defienden la democracia. Ahora
  la indignación se está desbordando en la calle.

  Por Patrícia Campos Mello

  \href{https://www.nytimes3xbfgragh.onion/2020/08/04/opinion/bolsonaro-office-of-hate-brazil.html}{Read
  in
  English}\href{https://www.nytimes3xbfgragh.onion/pt/2020/08/04/opinion/international-world/bolsonaro-gabinete-do-odio.html}{Ler
  em português}
\item
  \href{/es/2020/08/03/espanol/opinion/servicio-salud-mexico.html}{}

  \includegraphics{https://static01.graylady3jvrrxbe.onion/images/2020/08/03/multimedia/03Rios-ES/03Rios-ES-thumbWide.jpg?quality=75\&auto=webp\&disable=upscale}

  \hypertarget{comentario-5}{%
  \subsubsection{Comentario}\label{comentario-5}}

  \hypertarget{la-salud-puxfablica-en-muxe9xico-es-eso-que-nadie-quiere-usar}{%
  \subsection{La salud pública en México es
  eso-que-nadie-quiere-usar}\label{la-salud-puxfablica-en-muxe9xico-es-eso-que-nadie-quiere-usar}}

  Las clases medias mexicanas claudicaron de intentar atenderse en
  hospitales públicos y los más pobres los usan pensando que ahí solo
  van a morirse. Esto debe cambiar.

  Por Viri Ríos
\item
  \href{/es/2020/08/01/espanol/opinion/coronavirus-aire.html}{}

  \includegraphics{https://static01.graylady3jvrrxbe.onion/images/2020/07/30/opinion/01Marr-ES-1/30Marr-thumbWide.jpg?quality=75\&auto=webp\&disable=upscale}

  \hypertarget{comentario-6}{%
  \subsubsection{Comentario}\label{comentario-6}}

  \hypertarget{suxed-el-coronavirus-estuxe1-en-el-aire}{%
  \subsection{Sí, el coronavirus está en el
  aire}\label{suxed-el-coronavirus-estuxe1-en-el-aire}}

  La transmisión por aerosoles es importante, y quizá sea mucho más
  relevante de lo que hemos podido comprobar hasta ahora.

  Por Linsey C. Marr

  \href{https://www.nytimes3xbfgragh.onion/2020/07/30/opinion/coronavirus-aerosols.html}{Read
  in English}
\item
  \href{/es/2020/07/30/espanol/opinion/usar-cubrebocas-politica.html}{}

  \includegraphics{https://static01.graylady3jvrrxbe.onion/images/2020/07/28/opinion/28friedmanWeb/28friedmanWeb-thumbWide.jpg?quality=75\&auto=webp\&disable=upscale}

  \hypertarget{comentario-7}{%
  \subsubsection{Comentario}\label{comentario-7}}

  \hypertarget{si-nuestros-cubrebocas-pudieran-hablar}{%
  \subsection{Si nuestros cubrebocas pudieran
  hablar}\label{si-nuestros-cubrebocas-pudieran-hablar}}

  ¿Cómo nos volvimos tan ineficaces para combatir al coronavirus? Los
  arqueólogos del futuro que vinieran a excavar al país más rico del
  mundo, encontrarían la clave en un artefacto sencillo: la mascarilla.

  Por Thomas L. Friedman

  \href{https://www.nytimes3xbfgragh.onion/2020/07/28/opinion/coronavirus-masks.html}{Read
  in English}
\item
  \href{/es/2020/07/30/espanol/opinion/aztecas-violencia-narco-amlo.html}{}

  \includegraphics{https://static01.graylady3jvrrxbe.onion/images/2020/07/29/opinion/29villoro-sub/29villoro-sub-thumbWide.jpg?quality=75\&auto=webp\&disable=upscale}

  \hypertarget{comentario-8}{%
  \subsubsection{Comentario}\label{comentario-8}}

  \hypertarget{la-tierra-en-pruxe9stamo-una-gramuxe1tica-de-la-violencia-en-muxe9xico}{%
  \subsection{La tierra en préstamo: una gramática de la violencia en
  México}\label{la-tierra-en-pruxe9stamo-una-gramuxe1tica-de-la-violencia-en-muxe9xico}}

  El hallazgo de un inmenso altar fúnebre azteca permite reflexionar
  sobre las urgencias actuales sin fantasías atávicas pero con un nítido
  sentido de la historia y los desafíos del presente.

  Por Juan Villoro
\item
  \href{/es/2020/07/30/espanol/opinion/john-lewis-derechos-civiles.html}{}

  \includegraphics{https://static01.graylady3jvrrxbe.onion/images/2020/07/29/opinion/00lewis/00lewis-thumbWide.jpg?quality=75\&auto=webp\&disable=upscale}

  \hypertarget{john-lewis-juntos-ustedes-pueden-recuperar-el-alma-de-estados-unidos}{%
  \subsection{John Lewis: Juntos, ustedes pueden recuperar el alma de
  Estados
  Unidos}\label{john-lewis-juntos-ustedes-pueden-recuperar-el-alma-de-estados-unidos}}

  Aunque me haya ido, los animo a responder al llamado más elevado de su
  corazón y a defender lo que realmente creen.

  Por John Lewis

  \href{https://www.nytimes3xbfgragh.onion/2020/07/30/opinion/john-lewis-civil-rights-america.html}{Read
  in English}
\item
  \href{/es/2020/07/29/espanol/opinion/espana-felipe-gonzalez-jose-maria-aznar.html}{}

  \includegraphics{https://static01.graylady3jvrrxbe.onion/images/2020/07/29/multimedia/29Jimenez-ES/29Jimenez-ES-thumbWide.jpg?quality=75\&auto=webp\&disable=upscale}

  \hypertarget{comentario-9}{%
  \subsubsection{Comentario}\label{comentario-9}}

  \hypertarget{manual-de-comportamiento-para-expresidentes-insoportables}{%
  \subsection{Manual de comportamiento para expresidentes
  insoportables}\label{manual-de-comportamiento-para-expresidentes-insoportables}}

  La incapacidad de algunos exmandatarios de España de aceptar su
  jubilación viene en parte de una falta de cultura democrática. Algunos
  países de Latinoamérica tienen el mismo problema.

  Por David Jiménez
\item
  \href{/es/2020/07/28/espanol/opinion/argentina-estallido-2001-coronavirus.html}{}

  \includegraphics{https://static01.graylady3jvrrxbe.onion/images/2020/07/28/multimedia/28garcia-timerman-ES-3/28garcia-timerman-ES-3-thumbWide.jpg?quality=75\&auto=webp\&disable=upscale}

  \hypertarget{comentario-10}{%
  \subsubsection{Comentario}\label{comentario-10}}

  \hypertarget{por-quuxe9-no-explota-argentina}{%
  \subsection{¿Por qué no explota
  Argentina?}\label{por-quuxe9-no-explota-argentina}}

  Lecciones de 2001, una fuerte política social asistencialista y una
  grieta política potente han alejado de momento otro estallido social,
  pero solo una nueva política cooperativa podrá terminar de disipar ese
  fantasma.

  Por Marcelo J. García y Jordana Timerman
\item
  \href{/es/2020/07/27/espanol/opinion/clases-universidad-coronavirus.html}{}

  \includegraphics{https://static01.graylady3jvrrxbe.onion/images/2020/07/24/opinion/00herrcher/00herrcher-thumbWide.jpg?quality=75\&auto=webp\&disable=upscale}

  \hypertarget{comentario-11}{%
  \subsubsection{Comentario}\label{comentario-11}}

  \hypertarget{las-enseuxf1anzas-de-educar-durante-la-pandemia}{%
  \subsection{Las enseñanzas de educar durante la
  pandemia}\label{las-enseuxf1anzas-de-educar-durante-la-pandemia}}

  He sido profesor universitario durante más de dos décadas, y estos
  meses de enseñar durante la pandemia me han hecho entender que la
  educación ha cambiado para siempre.

  Por Roberto Herrscher
\item
  \href{/es/2020/07/27/espanol/opinion/reabrir-escuelas-riesgo-covid.html}{}

  \includegraphics{https://static01.graylady3jvrrxbe.onion/images/2020/07/20/opinion/27reopen-ES/20jogee-thumbWide.jpg?quality=75\&auto=webp\&disable=upscale}

  \hypertarget{comentario-12}{%
  \subsubsection{Comentario}\label{comentario-12}}

  \hypertarget{cuxf3mo-reabrir-la-economuxeda-sin-causar-la-muerte-de-padres-y-maestros}{%
  \subsection{¿Cómo reabrir la economía sin causar la muerte de padres y
  maestros?}\label{cuxf3mo-reabrir-la-economuxeda-sin-causar-la-muerte-de-padres-y-maestros}}

  Todas las clases deberían ser en línea, pero los edificios todavía
  podrían cumplir un propósito importante para los niños que más lo
  necesitan.

  Por Shardha Jogee

  \href{https://www.nytimes3xbfgragh.onion/2020/07/20/opinion/coronavirus-reopen-schools-economy.html}{Read
  in English}
\end{enumerate}

Ver más

Advertisement

\protect\hyperlink{after-mid2}{Continue reading the main story}

Advertisement

\protect\hyperlink{after-mktg}{Continue reading the main story}

\hypertarget{site-index}{%
\subsection{Site Index}\label{site-index}}

\hypertarget{site-information-navigation}{%
\subsection{Site Information
Navigation}\label{site-information-navigation}}

\begin{itemize}
\tightlist
\item
  \href{https://help.nytimes3xbfgragh.onion/hc/en-us/articles/115014792127-Copyright-notice}{©~2020~The
  New York Times Company}
\end{itemize}

\begin{itemize}
\tightlist
\item
  \href{https://www.nytco.com/}{NYTCo}
\item
  \href{https://help.nytimes3xbfgragh.onion/hc/en-us/articles/115015385887-Contact-Us}{Contact
  Us}
\item
  \href{https://www.nytco.com/careers/}{Work with us}
\item
  \href{https://nytmediakit.com/}{Advertise}
\item
  \href{http://www.tbrandstudio.com/}{T Brand Studio}
\item
  \href{https://www.nytimes3xbfgragh.onion/privacy/cookie-policy\#how-do-i-manage-trackers}{Your
  Ad Choices}
\item
  \href{https://www.nytimes3xbfgragh.onion/privacy}{Privacy}
\item
  \href{https://help.nytimes3xbfgragh.onion/hc/en-us/articles/115014893428-Terms-of-service}{Terms
  of Service}
\item
  \href{https://help.nytimes3xbfgragh.onion/hc/en-us/articles/115014893968-Terms-of-sale}{Terms
  of Sale}
\item
  \href{https://spiderbites.nytimes3xbfgragh.onion}{Site Map}
\item
  \href{https://help.nytimes3xbfgragh.onion/hc/en-us}{Help}
\item
  \href{https://www.nytimes3xbfgragh.onion/subscription?campaignId=37WXW}{Subscriptions}
\end{itemize}
