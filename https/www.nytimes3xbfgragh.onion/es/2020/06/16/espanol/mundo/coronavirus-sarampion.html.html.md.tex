Sections

SEARCH

\protect\hyperlink{site-content}{Skip to
content}\protect\hyperlink{site-index}{Skip to site index}

\href{https://www.nytimes3xbfgragh.onion/es/section/mundo}{Mundo}

\href{https://myaccount.nytimes3xbfgragh.onion/auth/login?response_type=cookie\&client_id=vi}{}

\href{https://www.nytimes3xbfgragh.onion/section/todayspaper}{Today's
Paper}

\href{/es/section/mundo}{Mundo}\textbar{}Mientras avanza el coronavirus,
otras enfermedades resurgen

\url{https://nyti.ms/30PnDlO}

\begin{itemize}
\item
\item
\item
\item
\item
\end{itemize}

\hypertarget{el-brote-de-coronavirus}{%
\subsubsection{\texorpdfstring{\href{https://www.nytimes3xbfgragh.onion/es/spotlight/coronavirus?name=styln-coronavirus-es\&region=TOP_BANNER\&variant=undefined\&block=storyline_menu_recirc\&action=click\&pgtype=Article\&impression_id=ebbe6bf0-e39c-11ea-8ef3-3565b30fff31}{El
brote de
coronavirus}}{El brote de coronavirus}}\label{el-brote-de-coronavirus}}

\begin{itemize}
\tightlist
\item
  \href{https://www.nytimes3xbfgragh.onion/es/interactive/2020/espanol/america-latina/coronavirus-en-mexico.html?name=styln-coronavirus-es\&region=TOP_BANNER\&variant=undefined\&block=storyline_menu_recirc\&action=click\&pgtype=Article\&impression_id=ebbe9300-e39c-11ea-8ef3-3565b30fff31}{Mapa
  y casos en México}
\item
  \href{https://www.nytimes3xbfgragh.onion/es/interactive/2020/08/06/espanol/ciencia-y-tecnologia/tengo-covid-19-sintomas.html?name=styln-coronavirus-es\&region=TOP_BANNER\&variant=undefined\&block=storyline_menu_recirc\&action=click\&pgtype=Article\&impression_id=ebbe9301-e39c-11ea-8ef3-3565b30fff31}{Identifica
  los síntomas}
\item
  \href{https://www.nytimes3xbfgragh.onion/es/interactive/2020/science/coronavirus-tratamientos-curas.html?name=styln-coronavirus-es\&region=TOP_BANNER\&variant=undefined\&block=storyline_menu_recirc\&action=click\&pgtype=Article\&impression_id=ebbe9302-e39c-11ea-8ef3-3565b30fff31}{Fármacos
  y tratamientos}
\item
  \href{https://www.nytimes3xbfgragh.onion/es/2020/04/29/espanol/estilos-de-vida/oximetro-para-que-sirve.html?name=styln-coronavirus-es\&region=TOP_BANNER\&variant=undefined\&block=storyline_menu_recirc\&action=click\&pgtype=Article\&impression_id=ebbe9303-e39c-11ea-8ef3-3565b30fff31}{¿Necesitas
  un oxímetro?}
\item
  \href{https://www.nytimes3xbfgragh.onion/es/2020/07/02/espanol/ciencia-y-tecnologia/sobrevivientes-coronavirus-recuperacion.html?name=styln-coronavirus-es\&region=TOP_BANNER\&variant=undefined\&block=storyline_menu_recirc\&action=click\&pgtype=Article\&impression_id=ebbe9304-e39c-11ea-8ef3-3565b30fff31}{Las
  secuelas del virus}
\end{itemize}

Advertisement

\protect\hyperlink{after-top}{Continue reading the main story}

Supported by

\protect\hyperlink{after-sponsor}{Continue reading the main story}

\hypertarget{mientras-avanza-el-coronavirus-otras-enfermedades-resurgen}{%
\section{Mientras avanza el coronavirus, otras enfermedades
resurgen}\label{mientras-avanza-el-coronavirus-otras-enfermedades-resurgen}}

Numerosos esfuerzos de vacunación se detuvieron en todo el mundo para
evitar la propagación de la COVID-19. Las consecuencias han sido
alarmantes: han aumentado los casos de difteria, cólera, poliomielitis y
sarampión.

\includegraphics{https://static01.graylady3jvrrxbe.onion/images/2020/06/11/science/16Virus-other-diseases-ES-1/00VIRUS-VAX3-articleLarge.jpg?quality=75\&auto=webp\&disable=upscale}

Por \href{https://www.nytimes3xbfgragh.onion/by/jan-hoffman}{Jan
Hoffman} y
\href{https://www.nytimes3xbfgragh.onion/by/ruth-maclean}{Ruth Maclean}

\begin{itemize}
\item
  16 de junio de 2020
\item
  \begin{itemize}
  \item
  \item
  \item
  \item
  \item
  \end{itemize}
\end{itemize}

\href{https://www.nytimes3xbfgragh.onion/2020/06/14/health/coronavirus-vaccines-measles.html}{Read
in English}

\href{https://www.nytimes3xbfgragh.onion/newsletters/el-times}{Regístrate
para recibir nuestro boletín} con lo mejor de The New York Times.

\begin{center}\rule{0.5\linewidth}{\linethickness}\end{center}

Mientras los países pobres de todo el mundo buscan detener el
coronavirus, también están contribuyendo de manera involuntaria a que
surjan nuevos brotes de enfermedades y fallecimientos a causa de otros
padecimientos que las vacunas previenen con facilidad.

Esta primavera, después de que la Organización Mundial de la Salud (OMS)
y la \href{https://www.unicef.org/immunization}{UNICEF} advirtieron que
la pandemia podría propagarse con rapidez cuando los niños se reunieran
para recibir vacunas, muchos países suspendieron sus programas de
vacunación. Incluso en los países que intentaron que siguieran vigentes,
los vuelos que transportaban el suministro de vacunas fueron detenidos
por la pandemia y los trabajadores de la salud se dedicaron a
combatirla.

Ahora está resurgiendo la difteria en Pakistán, Bangladés y Nepal.

Hay cólera en Sudán del Sur, Camerún, Mozambique, Yemen y Bangladés.

Se ha informado de la aparición de una cepa mutada del
\href{http://polioeradication.org/}{poliovirus} en más de 30 países.

Además, el sarampión está aumentando por todo el mundo, incluyendo
países como Bangladés, Brasil, Camboya, la República Centroafricana,
Irak, Kazajistán, Nepal, Nigeria y Uzbekistán.

De 29 países que han suspendido las campañas de vacunación contra el
sarampión debido a la pandemia, 18 han reportado brotes. Otros 13 países
están considerando posponerlas. De acuerdo con la
\href{https://measlesrubellainitiative.org/}{Iniciativa contra el
Sarampión y la Rubéola}, 178 millones de personas están en riesgo de no
ser vacunadas contra el sarampión este año.

Ahora se presenta el riesgo de que ``dentro de algunos meses haya una
epidemia que provoque la muerte de más niños que la COVID-19'', señaló
Chibuzo Okonta, presidente de Médicos sin Fronteras en África Central y
África Occidental.

Debido a que la pandemia continúa, la OMS y otros organismos
internacionales de salud están exhortando a los países a reiniciar con
cautela la vacunación, al mismo tiempo que combaten el coronavirus.

\includegraphics{https://static01.graylady3jvrrxbe.onion/images/2020/06/11/science/16Virus-other-diseases-ES-2/00VIRUS-VAX2-articleLarge.jpg?quality=75\&auto=webp\&disable=upscale}

Según un
\href{https://www.vaccineimpact.org/resources/VIMC_impact_estimates-03Sep19.html}{estudio
de 2019 realizado por Vaccine Impact Modeling Consortium}, un grupo de
investigadores de salud pública, está en juego el futuro de una férrea
colaboración de 20 años que ha evitado 35 millones de
\href{https://www.gavi.org/programmes-impact/types-support/vaccine-support}{decesos
por enfermedades prevenibles mediante vacunas} en 98 países y ha
reducido un 44 por ciento la mortalidad de los niños por estas
enfermedades.

``La inmunización es una de las herramientas más potentes y primordiales
en la historia de la salud pública para la prevención de enfermedades'',
afirmó en un comunicado Tedros Adhanom Ghebreyesus,
\href{https://www.who.int/dg}{director general de la OMS}. ``La
interrupción de los programas de inmunización por la pandemia de la
COVID-19 amenaza con revertir décadas de avance contra las enfermedades
prevenibles mediante vacunas, como es el caso del sarampión''.

No obstante, hay muchos obstáculos para reanudar los programas de
vacunación. Es difícil que lleguen los suministros de vacunas. Cada vez
más especialistas sanitarios se enfocan en trabajar de tiempo completo
en el combate de la COVID-19, la enfermedad causada por el coronavirus.
Además, una nueva ola de dudas acerca de las vacunas hace que los padres
no vayan a las clínicas.

Hay muchos países donde la pandemia todavía no azota con toda su fuerza.
Cuando lo haga, su capacidad para manejar los brotes de otras
enfermedades se verá aún más disminuida.

``Habrá naciones tratando de recuperarse de la COVID-19 al mismo tiempo
que se enfrenten el sarampión. Eso comprometería aún más sus sistemas de
salud y tendrá serias consecuencias económicas y humanitarias'', señaló
Robin Nandy, director de inmunización de la UNICEF, organismo que
proporciona vacunas a cien países, por lo cual llega al 45 por ciento de
los niños menores de cinco años.

La interrupción en el reparto de vacunas también tiene implicaciones
considerables para la protección contra el coronavirus.

En una cumbre mundial a principios de este mes,
\href{https://www.gavi.org/our-alliance}{Gavi, la Alianza Mundial para
Vacunas e Inmunización}, una asociación de salud creada por la Fundación
Bill y Melinda Gates, anunció que países y organismos se habían
comprometido a donar
\href{https://www.gavi.org/news/media-room/world-leaders-make-historic-commitments-provide-equal-access-vaccines-all}{8800
millones de dólares} en total para las vacunas básicas de niños en
países pobres y de ingresos medios, y que estaba iniciando una
\href{https://www.gavi.org/news/media-room/gavi-launches-innovative-financing-mechanism-access-covid-19-vaccines}{campaña
para distribuir vacunas contra la COVID-19} cuando estén disponibles.

Sin embargo, los mismos servicios que están colapsando por la pandemia,
``son los que se necesitarán para distribuir las vacunas contra la
COVID-19'', advirtió Katherine O'Brien, directora de inmunización,
vacunas y agentes biológicos de la OMS, durante un reciente seminario
por internet sobre los desafíos de la inmunización.

\hypertarget{luchar-contra-el-sarampiuxf3n-en-el-congo}{%
\subsection{Luchar contra el sarampión en el
Congo}\label{luchar-contra-el-sarampiuxf3n-en-el-congo}}

Image

Niños esperan registrarse para la vacuna contra el sarampión en
Mbata-Siala, en la parte occidental de la República Democrática del
Congo, en marzo.Credit...Junior Kannah/Agence France-Presse --- Getty
Images

Tres trabajadores de la salud con refrigeradores llenos de vacunas y un
equipo de apoyo de pregoneros y tomadores de notas subieron
recientemente a una canoa de madera motorizada para descender por el
ancho río Tshopo en la República Democrática del Congo.

Aunque el sarampión se estaba propagando en todas las 26 provincias del
país, semanas antes la pandemia había cerrado muchos programas de
inoculación.

La tripulación de la canoa necesitaba encontrar un equilibrio entre
evitar la transmisión de un nuevo virus que recién está comenzando a
golpear a África y detener a un viejo pero conocido asesino. Sin
embargo, cuando la larga y estrecha canoa llegó a las comunidades
ribereñas, el mayor desafío de la tripulación no resultó ser la mecánica
de vacunar a los niños mientras seguían las nuevas restricciones de
seguridad de la pandemia. En cambio, la tripulación se encontró
trabajando duro solo para persuadir a los aldeanos de permitir que sus
hijos fueran vacunados.

Muchos padres estaban convencidos de que el equipo estaba mintiendo
sobre la vacuna, que no era para el sarampión sino que, en secreto, era
una vacuna experimental contra el coronavirus, para la cual serían
necesarios conejillos de indias involuntarios.

En abril, la África francófona se había indignado por una
\href{https://www.bbc.com/news/world-europe-52151722}{entrevista en la
televisión francesa en la cual dos investigadores} dijeron que las
vacunas contra el coronavirus debían ser probadas en África, un
comentario que reavivó los recuerdos de una larga historia de abusos
similares. Y en el Congo, el virólogo a cargo de la respuesta al
coronavirus dijo que el país había aceptado participar en ensayos
clínicas de vacunas este verano. Más tarde, aclaró que cualquier vacuna
no se probaría en el Congo antes de ser probada en otro lugar. Pero los
rumores perniciosos ya se habían expandido.

El equipo convenció a los padres lo mejor que pudieron. Aunque los
vacunadores en todo Tshopo finalmente inmunizaron a 16.000 niños, otros
2000 los eludieron.

Este iba a ser el año en que la República Democrática del Congo, el
segundo país más grande de África, lanzaría un programa nacional de
inmunización. El apremio no podría haber sido mayor. La epidemia de
sarampión en el país, que comenzó en 2018, se ha seguido extendiendo:
desde enero, ha habido más de 60.000 casos y 800 fallecimientos. Ahora
ha vuelto a aparecer el ébola, además de la tuberculosis y el cólera,
los cuales afectan al país con frecuencia.

Pese a que no siempre están disponibles, existen vacunas para todas
estas enfermedades. A finales de 2018, el país comenzó un programa de
inmunización en nueve provincias. Fue una hazaña de coordinación y
esfuerzo y, en 2019, el primer año completo, el porcentaje de niños
inmunizados pasó de 42 a 62 por ciento en Kinsasa, la capital.

Esta primavera, cuando el programa estaba preparándose para su
lanzamiento a nivel nacional, embistió el coronavirus. Era casi seguro
que las campañas de vacunación masiva, que por lo general implican
reunir a cientos de niños en los patios de las escuelas y los mercados,
propagarían el coronavirus. Incluso se volvió inviable en muchas
regiones la inmunización de rutina, misma que casi siempre tiene lugar
en las clínicas.

Las autoridades sanitarias del país decidieron que las vacunas siguieran
aplicándose en las regiones con brotes de sarampión, pero donde no
hubiera casos de coronavirus. Sin embargo, la pandemia detuvo los vuelos
internacionales que abastecerían los insumos médicos, y en varias
provincias se empezaron a terminar las vacunas contra la polio, el
sarampión y la tuberculosis.

Cuando finalmente llegaron los suministros a Kinsasa, no pudieron
llevarlos a todo el país. Los vuelos nacionales estaban suspendidos.
Tampoco era posible el transporte terrestre debido a la mala condición
de los caminos. Al final, una misión de paz de las Naciones Unidas llevó
los suministros en sus aeroplanos.

Aún así, los trabajadores de la salud, que no tenían mascarillas,
guantes o gel desinfectante, estaban preocupados por infectarse; muchos
dejaron de trabajar. Otros fueron delegados para ser entrenados para la
COVID-19.

El impacto acumulativo ha sido particularmente grave para la
erradicación de la polio: alrededor de 85.000 niños congoleños no han
recibido esta vacuna.

Pero la enfermedad que más preocupa a los funcionarios de salud pública
es el sarampión.

\hypertarget{muxe1s-contagioso-que-la-covid-19}{%
\subsection{Más contagioso que la
COVID-19}\label{muxe1s-contagioso-que-la-covid-19}}

Image

Trabajadores de la salud inmunizan contra el sarampión en Manila, el mes
pasado.Credit...Aaron Favila/Associated Press

Según los expertos de los Centros para el Control y la Prevención de
Enfermedades (CDC, por su sigla en inglés), el virus del sarampión se
propaga con mucha facilidad por el aire ---por partículas diminutas o
gotículas suspendidas en el aire--- y
\href{https://www.cdc.gov/measles/transmission.html}{es mucho más
contagioso que el coronavirus}.

``Si entran personas a una habitación donde dos horas antes ha estado
alguien con sarampión y ninguna de ellas ha sido inmunizada, el cien por
ciento de esas personas se contagiará'', dijo
\href{https://biox.stanford.edu/people/yvonne-maldonado}{Yvonne
Maldonado}, experta en enfermedades infecciosas pediátricas en la
Universidad de Stanford.

En los países más pobres, la tasa de mortalidad por sarampión en niños
menores de cinco años varía entre el tres y el seis por ciento; algunas
condiciones como la desnutrición o los campamentos abarrotados de
refugiados pueden incrementar la tasa de mortalidad. Los niños pueden
sucumbir a complicaciones como neumonía, encefalitis y diarrea grave.

Se calcula que, en 2018, el año más reciente del que se han recabado
datos a nivel mundial, hubo casi diez millones de casos de sarampión y
142.300 fallecimientos asociados a él, aun cuando los programas de
inmunización a nivel global eran más sólidos en ese entonces.

Antes de la pandemia del coronavirus en Etiopía, el 91 por ciento de los
niños de la capital, Adís Abeba, recibieron su primera vacuna contra el
sarampión durante visitas de rutina, mientras que en las regiones
rurales la recibieron el 29 por ciento de los niños. (Para evitar un
brote de enfermedades muy infecciosas como el sarampión, la cobertura
óptima es de 95 por ciento o más, con dos dosis de la vacuna). Cuando
azotó la pandemia, el país suspendió su campaña de vacunación de abril,
pero el gobierno sigue reportando muchos casos nuevos.

``Los patógenos de los brotes no reconocen fronteras'', señaló O'Brien,
de la OMS. ``En especial el sarampión: si hay sarampión en algún lugar,
hay sarampión en todas partes''.

Las tasas de inmunización
\href{https://www.nytimes3xbfgragh.onion/2020/05/20/nyregion/coronavirus-schools-vaccinations.html}{en
los países más ricos} también
\href{https://www.nytimes3xbfgragh.onion/2020/04/23/health/coronavirus-measles-vaccines.html}{se
han desplomado} durante la pandemia. En Estados Unidos, algunos estados
reportan caídas de hasta un 70 por ciento por debajo del mismo periodo
del año anterior, para el sarampión y otras enfermedades.

Una vez que las personas comiencen a viajar de nuevo, el riesgo de
infección aumentará. ``Esto no me deja dormir'', dijo
\href{https://www.cdc.gov/media/spokesperson/sme-bio/cochi.html}{Stephen
Cochi}, asesor principal de la división de inmunización global en los
CDC. ``Estas enfermedades prevenibles por vacunación están a un solo
viaje de avión''.

\hypertarget{empezar-de-nuevo}{%
\subsection{Empezar de nuevo}\label{empezar-de-nuevo}}

Image

Hawa Hamadou, una trabajadora sanitaria del centro de salud de Gamkale,
en Niamey, Nigeria, ha visto una disminución en las visitas de madres,
que tienen miedo de llevar a sus hijos a vacunarse.Credit...Juan
Haro/UNICEF

Después de que la OMS y sus socios de vacunación publicaron los
resultados de una encuesta el mes pasado que muestra que
\href{https://www.nytimes3xbfgragh.onion/es/2020/05/25/espanol/ciencia-y-tecnologia/vacuna-polio-sarampion-coronavirus.html}{80
millones de bebés menores de un año} corrían el riesgo de perder las
vacunas de rutina, algunos países, incluidos Etiopía, la República
Centroafricana y Nepal, comenzaron el intento de reiniciar sus
programas.

Uganda ahora suministra motocicletas a los trabajadores de la salud. En
Brasil, algunas farmacias ofrecen servicios de inmunización al
automóvil. En el estado indio de Bihar, una trabajadora de la salud de
50 años aprendió a andar en bicicleta en tres días para poder llevar las
vacunas a familias alejadas. La UNICEF alquiló un vuelo para entregar
vacunas a siete países africanos.

Cochi, de las CDC, que brinda apoyo técnico y programático a más de 40
países, dijo que si tales campañas pueden llevarse a cabo durante la
pandemia es una pregunta abierta. ``Estará lleno de limitaciones.
Hablamos de países de bajos ingresos donde el distanciamiento social no
es una realidad, no es posible'', dijo, citando las favelas brasileñas y
las caravanas de migrantes.

Espera que las campañas contra la
\href{https://www.cdc.gov/polio/}{poliomielitis} se reanuden
rápidamente, pues teme que la pandemia pueda retrasar un esfuerzo global
de décadas para erradicar la enfermedad.

Jan Hoffman reportó desde Nueva York y Ruth Maclean, desde Dakar,
Senegal.

Jan Hoffman escribe sobre salud conductual y legislación en torno a la
salud. Sus temas de gran alcance incluyen opioides, vapeo, tribus y
adolescentes. \href{https://twitter.com/JanHoffmanNYT}{@JanHoffmanNYT}

Ruth Maclean es la jefa de la corresponsalía de África Occidental para
The New York Times, con sede en Senegal. Se unió al Times en 2019
después de tres años y medio cubriendo África Occidental para The
Guardian. \href{https://twitter.com/ruthmaclean}{@ruthmaclean}

Advertisement

\protect\hyperlink{after-bottom}{Continue reading the main story}

\hypertarget{site-index}{%
\subsection{Site Index}\label{site-index}}

\hypertarget{site-information-navigation}{%
\subsection{Site Information
Navigation}\label{site-information-navigation}}

\begin{itemize}
\tightlist
\item
  \href{https://help.nytimes3xbfgragh.onion/hc/en-us/articles/115014792127-Copyright-notice}{©~2020~The
  New York Times Company}
\end{itemize}

\begin{itemize}
\tightlist
\item
  \href{https://www.nytco.com/}{NYTCo}
\item
  \href{https://help.nytimes3xbfgragh.onion/hc/en-us/articles/115015385887-Contact-Us}{Contact
  Us}
\item
  \href{https://www.nytco.com/careers/}{Work with us}
\item
  \href{https://nytmediakit.com/}{Advertise}
\item
  \href{http://www.tbrandstudio.com/}{T Brand Studio}
\item
  \href{https://www.nytimes3xbfgragh.onion/privacy/cookie-policy\#how-do-i-manage-trackers}{Your
  Ad Choices}
\item
  \href{https://www.nytimes3xbfgragh.onion/privacy}{Privacy}
\item
  \href{https://help.nytimes3xbfgragh.onion/hc/en-us/articles/115014893428-Terms-of-service}{Terms
  of Service}
\item
  \href{https://help.nytimes3xbfgragh.onion/hc/en-us/articles/115014893968-Terms-of-sale}{Terms
  of Sale}
\item
  \href{https://spiderbites.nytimes3xbfgragh.onion}{Site Map}
\item
  \href{https://help.nytimes3xbfgragh.onion/hc/en-us}{Help}
\item
  \href{https://www.nytimes3xbfgragh.onion/subscription?campaignId=37WXW}{Subscriptions}
\end{itemize}
