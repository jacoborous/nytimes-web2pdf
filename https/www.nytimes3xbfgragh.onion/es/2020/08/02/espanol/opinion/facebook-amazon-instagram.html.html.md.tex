Sections

SEARCH

\protect\hyperlink{site-content}{Skip to
content}\protect\hyperlink{site-index}{Skip to site index}

\href{https://www.nytimes3xbfgragh.onion/es/section/opinion}{Opinión}

\href{https://myaccount.nytimes3xbfgragh.onion/auth/login?response_type=cookie\&client_id=vi}{}

\href{https://www.nytimes3xbfgragh.onion/section/todayspaper}{Today's
Paper}

\href{/es/section/opinion}{Opinión}\textbar{}Los algoritmos son los
nuevos editores

\url{https://nyti.ms/2BU1k4A}

\begin{itemize}
\item
\item
\item
\item
\item
\end{itemize}

Advertisement

\protect\hyperlink{after-top}{Continue reading the main story}

\href{/es/section/opinion}{Opinión}

Supported by

\protect\hyperlink{after-sponsor}{Continue reading the main story}

Comentario

\hypertarget{los-algoritmos-son-los-nuevos-editores}{%
\section{Los algoritmos son los nuevos
editores}\label{los-algoritmos-son-los-nuevos-editores}}

¿Qué tienen en común Instagram, YouTube, Facebook, Amazon, Weibo y
Twitter? No son solo redes sociales o plataformas: son los grandes
editores de nuestra realidad.

\includegraphics{https://static01.graylady3jvrrxbe.onion/images/2020/08/02/multimedia/02Carrion-ES/merlin_171683256_836a51a0-dc07-4047-b169-4bbbb85b62b4-articleLarge.jpg?quality=75\&auto=webp\&disable=upscale}

Por \href{https://www.nytimes3xbfgragh.onion/by/jorge-carrion}{Jorge
Carrión}

Es escritor y crítico cultural.

\begin{itemize}
\item
  2 de agosto de 2020
\item
  \begin{itemize}
  \item
  \item
  \item
  \item
  \item
  \end{itemize}
\end{itemize}

\href{https://www.nytimes3xbfgragh.onion/newsletters/el-times}{Regístrate
para recibir nuestro boletín} con lo mejor de The New York Times.

\begin{center}\rule{0.5\linewidth}{\linethickness}\end{center}

BARCELONA --- Aunque parezca mentira, la poeta indiocanadiense
\href{https://www.instagram.com/rupikaur_/?hl=es}{Rupi Kaur}, el
youtuber mexicano
\href{https://www.youtube.com/user/LuisitoComunicaa}{Luisito Comunica},
el empresario norteamericano Mark Zuckerberg, el escritor español
\href{https://elpais.com/elpais/2020/03/04/icon/1583314253_152494.html}{Javier
Castillo}, la escritora china Fang Fang y el presidente de Estados
Unidos, Donald Trump tienen algo importante en común. El principal canal
de comunicación de los seis es una red social o plataforma.
Respectivamente: Instagram, YouTube, Facebook, Amazon, Weibo y Twitter.

Cuando tienen algo que decir, se dirigen directamente a sus enormes
audiencias, sin edición, sin anestesia. La idea de que el editor no es
necesario se asocia con las redes sociales de mayor impacto y muchas
plataformas tecnológicas. Nada debe interponerse entre el productor de
discurso y su público. Nadie debe corregir, matizar, maquetar o
verificar. Hay que derribar las viejas jerarquías, para que el talento
brille en el nuevo panorama horizontal y democrático.

Pero eso es en realidad un espejismo, porque está claro que sí existe
una intermediación. El intermediario es algorítmico. El editor, en este
caso, es una fórmula matemática, una serie de protocolos automatizados
que no solo se apropia de los procesos de edición: los algoritmos están
editando la mismísima realidad. Ya va siendo hora de que las grandes
plataformas asuman que son, entre otras cosas, las editoriales más
poderosas de hoy. Editan en parte, incluso, a los medios y a las
editoriales tradicionales.
\href{https://www.merriam-webster.com/dictionary/quis\%20custodiet\%20ipsos\%20custodes\%3F}{Versionando
libremente a Juvenal}, me pregunto: ¿y quién edita a las editoriales de
las editoriales?

Los miles de millones de dólares que están ganando Facebook, Instagram,
Twitter o Amazon con nuestro esfuerzo, nuestra artesanía y nuestro ego
deberían provocar que esas gigantescas corporaciones ---y su extenso
parentesco--- asuman su auténtica naturaleza. Les guste o no, son
editores de contenidos y de realidades. Y, como hacen los editores de
libros, películas, series o noticias, deberían pagar a quienes los
producen para ellos.

El trabajo de los medios de comunicación y de las empresas de edición ha
sido, desde siempre, una mezcla de lectura, artesanía y curaduría. De
todo aquello que se crea y produce, los editores han decidido
tradicionalmente lo que merece ser leído. Y han liderado un proceso que
incluye la corrección, el arte, la impresión, la distribución o la
mercadotecnia. De ese modo, mejoran, domestican o embellecen el texto y
las imágenes de la pieza periodística o del libro. Y las hacen visibles.

Entre la producción y la recepción de textos, fotos o vídeos publicados
directamente en línea siguen existiendo mecanismos de selección, filtro
y publicidad. Pero son parcialmente inhumanos. Todo lo que nos llega a
través de Google, YouTube o Tik Tok ha sido decidido por sus respectivos
algoritmos, actualizaciones pixeladas de los tradicionales agentes de la
visibilidad. Si la apuesta por un libro de una editorial tradicional se
traduce en anuncios en prensa o en redes o en compra de espacio en
librerías, la de Amazon ---que depende de cálculos que se producen en
algún lugar entre el Big Data y el
\href{https://www.bbva.com/es/machine-learning-que-es-y-como-funciona/}{\emph{machine
learning}}--- también consiste, finalmente, en destacar ese título, en
hacerlo brillar en la selva oscura de internet.

Esos nuevos mecanismos de prescripción no buscan la mejora, la belleza o
la verificación de los contenidos, sino su viralidad. En eso las grandes
plataformas coinciden con las pequeñas fábricas de desinformación.
Quienes creen ---absurdamente--- que los virus pandémicos han sido
creados en laboratorios biológicos son víctimas de memes y noticias
falsas que, muchas veces, sí han sido diseñados en laboratorios de
desinformación. Mensajes que se benefician tanto de un diseño que apela
a nuestros instintos más primarios como de la tendencia de internet a
difundir lo que ya cuenta con gran difusión.

Se aprovechan ---como dice Marta Peirano en su imprescindible
\href{https://www.megustaleer.com/libros/el-enemigo-conoce-el-sistema/MES-106841}{\emph{El
enemigo conoce el sistema}}--- de que ``Facebook puede publicar noticias
falsas como si fueran reales sin temer una demanda, cosa que un
periódico no puede hacer'', lo que contribuye a la existencia de un
``ecosistema mediático fraudulento''. Entre las respuestas posibles a
ese gravísimo problema, están la artificial y la personal. Son buenas
noticias que la red social haya cambiado el algoritmo para privilegiar
las noticias que estén basadas en reportería,
\href{https://fundaciongabo.org/es/etica-periodistica/noticias/facebook-cambia-su-algoritmo-para-favorecer-al-periodismo-con-reporteria}{como
hizo a finales de junio}; y que potencie la figura del moderador
---entre el editor y el censor---, al tiempo que crea
\href{https://www.nytimes3xbfgragh.onion/es/2020/05/06/espanol/opinion/facebook-junta-supervision.html}{un
comité de asesores en cuestiones éticas}. La intermediación algorítmica
debe convivir con la humana.
\href{https://www.nytimes3xbfgragh.onion/es/2018/12/28/espanol/facebook-moderadores.html}{Con
esa precaria alianza Facebook} modera a sus casi 2500 millones de
usuarios. O lo intenta.

También en su principal dimensión editorial, la que implica la
publicación constante de millones de contenidos informativos y
culturales, las plataformas deberían asumir su papel de intermediadoras.
Económicamente. Si el sistema que hemos heredado del siglo XX paga a los
creadores un porcentaje muy bajo de los derechos de autor que genera su
trabajo, el que ha emergido en el siglo XXI por lo general no paga nada.
Se basa en la consigna de que tú te explotas a ti mismo. Después, con
suerte, consigues articular una comunidad de fans que se autoexplotan en
tu beneficio. Y de todo ese trabajo gratuito tú puedes llegar a extraer
ingresos secundarios, pero quienes más se benefician son las
corporaciones (y sus accionistas). Y la gran mayoría de la humanidad
(conectada) sale perdiendo.

De modo que las mayores redes sociales y plataformas no solo deberían
controlar los contenidos violentos y los mensajes de odio, que sabemos
que están decidiendo
\href{https://www.nytimes3xbfgragh.onion/es/2018/10/18/espanol/facebook-violencia-rohinya-birmania.html}{limpiezas
étnicas} y elecciones democráticas. También tendrían que diseñar
políticas económicas para asumir que están cocreando millones de
contenidos narrativos, artísticos, pedagógicos, humorísticos y
mediáticos; y actuar como editores éticos de sus trabajadores
voluntarios.

El modelo de YouTube ha demostrado ser moderadamente exitoso. Los
youtubers cobran
\href{https://www.nytimes3xbfgragh.onion/2008/12/11/business/media/11youtube.html}{según
la repercusión de sus vídeos}. Y, aunque parezca mentira, los artistas
están ganando de media, por las reproducciones de sus canciones en
Spotify o Apple Music,
\href{https://www.businessinsider.es/cuantas-reproducciones-spotify-necesitan-ganar-1-euro-491653}{más
del 10 por ciento} de derechos de autor que ingresaban (o ingresan) por
sus discos. También deberían cobrar sus honorarios, derechos, comisiones
o anticipos los autores de hilos de Twitter, de historias de Instagram o
de libros casi autoeditados en Amazon. Al fin y al cabo, los creadores
digitales están trabajando tanto para su propia marca personal como para
las plataformas. Y estas se están lucrando con los datos y con la
publicidad gracias a la atención, el prestigio o el tráfico que generan
sus usuarios más constantes y creativos.

Jorge Carrión, colaborador regular de The New York Times, es escritor y
director del máster en Creación Literaria y del posgrado en Creación de
Contenidos y Nuevas Narrativas Digitales de la UPF-BSM. Su nuevo libro
se titula \emph{Lo viral}.

Advertisement

\protect\hyperlink{after-bottom}{Continue reading the main story}

\hypertarget{site-index}{%
\subsection{Site Index}\label{site-index}}

\hypertarget{site-information-navigation}{%
\subsection{Site Information
Navigation}\label{site-information-navigation}}

\begin{itemize}
\tightlist
\item
  \href{https://help.nytimes3xbfgragh.onion/hc/en-us/articles/115014792127-Copyright-notice}{©~2020~The
  New York Times Company}
\end{itemize}

\begin{itemize}
\tightlist
\item
  \href{https://www.nytco.com/}{NYTCo}
\item
  \href{https://help.nytimes3xbfgragh.onion/hc/en-us/articles/115015385887-Contact-Us}{Contact
  Us}
\item
  \href{https://www.nytco.com/careers/}{Work with us}
\item
  \href{https://nytmediakit.com/}{Advertise}
\item
  \href{http://www.tbrandstudio.com/}{T Brand Studio}
\item
  \href{https://www.nytimes3xbfgragh.onion/privacy/cookie-policy\#how-do-i-manage-trackers}{Your
  Ad Choices}
\item
  \href{https://www.nytimes3xbfgragh.onion/privacy}{Privacy}
\item
  \href{https://help.nytimes3xbfgragh.onion/hc/en-us/articles/115014893428-Terms-of-service}{Terms
  of Service}
\item
  \href{https://help.nytimes3xbfgragh.onion/hc/en-us/articles/115014893968-Terms-of-sale}{Terms
  of Sale}
\item
  \href{https://spiderbites.nytimes3xbfgragh.onion}{Site Map}
\item
  \href{https://help.nytimes3xbfgragh.onion/hc/en-us}{Help}
\item
  \href{https://www.nytimes3xbfgragh.onion/subscription?campaignId=37WXW}{Subscriptions}
\end{itemize}
