Sections

SEARCH

\protect\hyperlink{site-content}{Skip to
content}\protect\hyperlink{site-index}{Skip to site index}

\href{https://www.nytimes3xbfgragh.onion/es/}{en Español}

\href{https://myaccount.nytimes3xbfgragh.onion/auth/login?response_type=cookie\&client_id=vi}{}

\href{https://www.nytimes3xbfgragh.onion/section/todayspaper}{Today's
Paper}

\href{/es/}{en Español}\textbar{}Cómo prepararse para el aprendizaje a
distancia

\url{https://nyti.ms/2FanjW4}

\begin{itemize}
\item
\item
\item
\item
\item
\end{itemize}

\hypertarget{schools-reopening}{%
\subsubsection{\texorpdfstring{\href{https://www.nytimes3xbfgragh.onion/spotlight/schools-reopening?name=styln-coronavirus-schools-reopening\&region=TOP_BANNER\&variant=undefined\&block=storyline_menu_recirc\&action=click\&pgtype=Article\&impression_id=cb35c740-e38b-11ea-8aa7-fb86b0867258}{Schools
Reopening}}{Schools Reopening}}\label{schools-reopening}}

\begin{itemize}
\tightlist
\item
  \href{https://www.nytimes3xbfgragh.onion/2020/08/19/us/colleges-closing-covid.html?name=styln-coronavirus-schools-reopening\&region=TOP_BANNER\&variant=undefined\&block=storyline_menu_recirc\&action=click\&pgtype=Article\&impression_id=cb35c741-e38b-11ea-8aa7-fb86b0867258}{Colleges
  Closing}
\item
  \href{https://www.nytimes3xbfgragh.onion/2020/08/20/us/schools-reopening-nurses-covid.html?name=styln-coronavirus-schools-reopening\&region=TOP_BANNER\&variant=undefined\&block=storyline_menu_recirc\&action=click\&pgtype=Article\&impression_id=cb35c742-e38b-11ea-8aa7-fb86b0867258}{Missing
  School Nurses}
\item
  \href{https://www.nytimes3xbfgragh.onion/2020/08/18/parenting/homeschool-families.html?name=styln-coronavirus-schools-reopening\&region=TOP_BANNER\&variant=undefined\&block=storyline_menu_recirc\&action=click\&pgtype=Article\&impression_id=cb35ee50-e38b-11ea-8aa7-fb86b0867258}{Home-Schooling
  Families}
\item
  \href{https://www.nytimes3xbfgragh.onion/2020/08/05/parenting/parents-distance-learning.html?name=styln-coronavirus-schools-reopening\&region=TOP_BANNER\&variant=undefined\&block=storyline_menu_recirc\&action=click\&pgtype=Article\&impression_id=cb35ee51-e38b-11ea-8aa7-fb86b0867258}{Prepare
  for Distance Learning}
\end{itemize}

Advertisement

\protect\hyperlink{after-top}{Continue reading the main story}

Supported by

\protect\hyperlink{after-sponsor}{Continue reading the main story}

\hypertarget{cuxf3mo-prepararse-para-el-aprendizaje-a-distancia}{%
\section{Cómo prepararse para el aprendizaje a
distancia}\label{cuxf3mo-prepararse-para-el-aprendizaje-a-distancia}}

Este nuevo curso escolar no tiene por qué ser como el anterior. Aquí hay
unos consejos para no estar abrumados mientras guiamos a nuestros hijos
durante la educación en línea.

Por \href{https://www.nytimes3xbfgragh.onion/by/jenny-anderson}{Jenny
Anderson}

\begin{itemize}
\item
  12 de agosto de 2020
\item
  \begin{itemize}
  \item
  \item
  \item
  \item
  \item
  \end{itemize}
\end{itemize}

\href{https://www.nytimes3xbfgragh.onion/2020/08/05/parenting/parents-distance-learning.html}{Read
in English}

\includegraphics{https://static01.graylady3jvrrxbe.onion/images/2020/08/03/multimedia/5parenting-education-planning/5parenting-education-planning-articleLarge.jpg?quality=75\&auto=webp\&disable=upscale}

\href{https://www.nytimes3xbfgragh.onion/newsletters/el-times}{Regístrate
para recibir nuestro boletín} con lo mejor de The New York Times.

\begin{center}\rule{0.5\linewidth}{\linethickness}\end{center}

Normalmente confiamos en los maestros y los consejeros o los
entrenadores y personas en nuestras comunidades para ayudarnos a
detectar problemas y luego identificar soluciones. Intentamos construir
una aldea para no tener que hacer todo solos.

Pero ahora la aldea está en cuarentena, y cada vez está más claro que
aún depende de nosotros tratar de tomar las mejores decisiones para
nuestras familias, aunque parezca que hay 200.000 nuevas elecciones que
hacer cada día.

``Estoy abrumada'', dijo Lynn Cooper, agente de bienes raíces y madre de
dos en Reston, Virginia. Sus hijos, de 11 y 13 años, están en el sistema
de escuelas públicas del condado de Fairfax y había planeado enviarlos
dos días a la semana
\href{https://www.washingtonpost.com/local/education/superintendent-of-loudoun-schools-calls-for-all-virtual-start-to-classes/2020/07/21/9f7c6ae4-cb64-11ea-bc6a-6841b28d9093_story.html}{hasta
que el distrito optó}por una vuelta a clases totalmente en línea.
``¿Cómo puedo prepararlos para que aprendan con éxito? Estoy volando a
ciegas, estoy insegura y tengo mucha ansiedad'', dijo.

Hay una diferencia clave entre la escolarización en la primavera y este
otoño: debemos confiar más en los maestros y los consejeros. Eso no
quiere decir que los padres no tendrán un papel importante que
desempeñar como traductores y mensajeros para los maestros, quienes no
podrán desarrollar una relación tan profunda con nuestros hijos a través
de una pantalla como lo haría en un salón de clases.

``Deja que el maestro sea el instructor, pero el padre puede ser el
observador y el facilitador'', dijo Bibb Hubbard, fundadora y directora
ejecutiva de \href{https://bealearninghero.org/}{Learning Heroes}, una
organización que recopila datos y crea recursos para mejorar la relación
entre padres y maestros.

Aquí te mostramos cómo involucrarte más sin pasar el día en un monitoreo
continuo del trabajo de clase, sin contratar tutores costosos o sin
dormir por el tormento de la culpa de que le estamos fallando a nuestros
hijos.

\hypertarget{descubre-cuxf3mo-aprende-tu-hijo}{%
\subsection{Descubre cómo aprende tu
hijo}\label{descubre-cuxf3mo-aprende-tu-hijo}}

Empieza por tener una conversación con tu hijo que sea menos ``busquemos
una manera en la que no tenga que fastidiarte todo el día todos los días
para que cumplas con tus tareas'' y más ``planificamos cómo sería el
aprendizaje exitoso para ti en este extraño mundo en línea''.

``Los padres tienen esta ventana única sobre qué tipo de alumno es su
hijo'', dijo Phyllis Fagell, autora de \emph{Middle School Matters} y
consejera escolar en Washington, D.C. Ahora pueden usar esa información
para ayudar a determinar ``qué tipo de apoyo necesitan para tener
éxito'', señaló.

Aquí hay algunas preguntas para explorar con tu hijo: ¿Está entusiasmado
con un tema determinado? ¿Se cierra con una prueba pero cobra vida al
crear un portafolio en línea? ¿Necesita mucha responsabilidad? ¿Los
videojuegos lo distraen? ¿Ayudaría hacer las matemáticas en grupo a que
fuera menos aburridas?

Una vez que descubras cómo aprende tu hijo, debes comunicar a los
maestros que deseas tener expectativas y metas más claras para tus
hijos.

Bandele McQueen, un padre con un hijo de 7 años y otro de 10 que van a
escuelas públicas en Washington, D.C., dice que quiere más orientación
de los maestros y los administradores sobre cómo apoyar tanto el
aprendizaje de sus hijos como su salud mental durante un cambio tan
dramático de escenario.

``Les enseñaste a los niños cuáles son las expectativas: levantarse, ir
a la escuela, prestar atención, seguir las reglas'', dice. ``Obtienen
calificaciones y las usan para establecer metas''. Pero cuando los niños
se quedaron en casa, las calificaciones se suspendieron y el juego
cambió. Nadie explicó el nuevo paradigma, ni a los padres ni a los
estudiantes.

``Tenemos que mantener a los niños seguros y abrir la economía, pero lo
que no se puede omitir es cómo brindar una educación de calidad que los
niños acepten'', dijo McQueen.

\hypertarget{solicita-muxe1s-comentarios-de-los-profesores}{%
\subsection{Solicita más comentarios de los
profesores}\label{solicita-muxe1s-comentarios-de-los-profesores}}

No pedimos ser profesores asistentes, pero ahora lo somos. Quanshie
Maxwell, una madre soltera de cuatro ---de 12, ocho, cinco y dos años---
en Seattle, dijo que la perspectiva de otro año de aprendizaje en casa
la aterroriza. ``Me siento tan sola'', aseguró. ``Tengo que ser la
enfermera y la profesora de educación física y la señora del almuerzo y
la maestra, y luego tengo que ser simplemente mamá''.

\href{https://www.nytimes3xbfgragh.onion/spotlight/schools-reopening?action=click\&pgtype=Article\&state=default\&region=MAIN_CONTENT_3\&context=storylines_keepup}{}

\hypertarget{schools-reopening-}{%
\subsubsection{Schools Reopening ›}\label{schools-reopening-}}

\hypertarget{back-to-school}{%
\paragraph{Back to School}\label{back-to-school}}

Updated Aug. 20, 2020

The latest on how schools are reopening amid the pandemic.

\begin{itemize}
\item
  \begin{itemize}
  \tightlist
  \item
    Much more is
    \href{https://www.nytimes3xbfgragh.onion/2020/08/20/us/schools-reopening-nurses-covid.html?action=click\&pgtype=Article\&state=default\&region=MAIN_CONTENT_3\&context=storylines_keepup}{expected
    of America's school nurses} during the pandemic, but many schools
    don't have one.
  \item
    A vast majority of parents have resigned themselves to
    \href{https://www.nytimes3xbfgragh.onion/2020/08/19/us/colleges-closing-covid.html?action=click\&pgtype=Article\&state=default\&region=MAIN_CONTENT_3\&context=storylines_keepup}{going
    it alone in the pandemic school year}, according to a new survey for
    The New York Times.
  \item
    Alabama is betting that a
    \href{https://www.nytimes3xbfgragh.onion/2020/08/19/business/alabama-uab-coronavirus-tests.html?action=click\&pgtype=Article\&state=default\&region=MAIN_CONTENT_3\&context=storylines_keepup}{robust
    student testing and technology program} will be enough to hinder
    outbreaks on college campuses.
  \item
    We want to hear from teachers making difficult choices. How are you
    thinking about the start of the school year?
    \href{https://www.nytimes3xbfgragh.onion/2020/08/19/us/teachers-school-reopenings.html?action=click\&pgtype=Article\&state=default\&region=MAIN_CONTENT_3\&context=storylines_keepup}{Tell
    us here}.
  \end{itemize}
\end{itemize}

Debido a que los padres están tan sumergidos, necesitan pedir más ayuda
de la escuela y los profesores. Está claro que en la primavera para
muchas familias el aprendizaje en línea no funcionó. Learning Heroes
\href{https://r50gh2ss1ic2mww8s3uvjvq1-wpengine.netdna-ssl.com/wp-content/uploads/2020/05/LH_2020-Parent-Survey-Partner-1.pdf}{encuestó
a más de 3000 padres de estudiantes de escuelas públicas} en todo
Estados Unidos entre el 14 de abril y el 6 de mayo de este año, y
descubrió que solo el 33 por ciento de los estudiantes tenía contacto
regular con los profesores, el 15 por ciento de los padres recibió
orientación personal sobre cómo apoyar a sus hijos y solo el 13 por
ciento de los estudiantes tuvo tiempo a solas con sus maestros.

Las familias necesitan más que eso. Al principio, un distrito escolar en
el sur de California proporcionó una hoja de ruta sobre cómo hacer que
el aprendizaje en línea fuera efectivo para los maestros y las familias
por igual. Durante la primavera, David Miyashiro, el superintendente del
Distrito Escolar Unificado de Cajon Valley, celebró reuniones semanales
con directores de asociaciones de padres y maestros y personal escolar
para comprobar cómo se sentían ---respecto al aprendizaje a distancia,
pero también sobre la vida--- y diseñar juntos un plan de reapertura
para el otoño.

``Era casi como un grupo de terapia para que los padres se desahogaran y
que los escuchara alguien a quien le importaban'', dijo Miyashiro.
También fue una recopilación de datos útiles.

En julio, cuando el gobernador Gavin Newsom anunció que la
\href{https://www.nytimes3xbfgragh.onion/2020/07/17/us/california-schools-reopening-newsom.html}{mayoría
de las escuelas de California} serían solo remotas, Cajon Valley, que
inicialmente había planeado ofrecer a los padres cuatro opciones,
incluido el aprendizaje en persona, estaba preparado para todas las
posibilidades porque había estado en consulta con su comunidad.

En una carta a los padres, el distrito anunció que se centraría en los
problemas que ---en aquellas reuniones semanales de Zoom--- los padres
habían dicho que eran los más importantes, incluida la instrucción
diaria en vivo dirigida por maestros, las lecciones personalizadas, el
énfasis en los estándares estatales usando guías de ritmo del distrito,
las lecciones de educación física, las tareas calificadas, el
seguimiento de la asistencia diaria, y, por último, los comentarios de
los maestros sobre el progreso del estudiante.

Estas son solicitudes que todo padre puede hacer: personalización,
responsabilidad del alumno, y más comentarios sobre el progreso del
estudiante. Pero asegúrate de ser amable y empático con tus profesores
mientras les pides más. Según un
\href{https://r50gh2ss1ic2mww8s3uvjvq1-wpengine.netdna-ssl.com/wp-content/uploads/2018/12/2018_Research_Report-final_WEB.pdf}{estudio
que Learning Heroes realizó en 2018}, el 71 por ciento de los profesores
informan que tienen miedo de hablar con los padres sobre el aprendizaje
de sus hijos por temor a que se les culpe de las malas noticias, y el 51
por ciento también temen que los padres no les crean.

Ya que todos experimentamos lo difícil que es enseñar y motivar a los
niños, recordemos que los maestros están haciendo esto para hasta 30
niños, mientras que muchos también tienen a sus propios hijos en casa.
También están abrumados. Pero con un poco de empatía y un compromiso con
nuestras comunidades, juntos podemos superar este otoño y más allá.

Jenny Anderson es una periodista galardonada que se centra en la
intersección de la educación, la tecnología y la crianza de los hijos.

Advertisement

\protect\hyperlink{after-bottom}{Continue reading the main story}

\hypertarget{site-index}{%
\subsection{Site Index}\label{site-index}}

\hypertarget{site-information-navigation}{%
\subsection{Site Information
Navigation}\label{site-information-navigation}}

\begin{itemize}
\tightlist
\item
  \href{https://help.nytimes3xbfgragh.onion/hc/en-us/articles/115014792127-Copyright-notice}{©~2020~The
  New York Times Company}
\end{itemize}

\begin{itemize}
\tightlist
\item
  \href{https://www.nytco.com/}{NYTCo}
\item
  \href{https://help.nytimes3xbfgragh.onion/hc/en-us/articles/115015385887-Contact-Us}{Contact
  Us}
\item
  \href{https://www.nytco.com/careers/}{Work with us}
\item
  \href{https://nytmediakit.com/}{Advertise}
\item
  \href{http://www.tbrandstudio.com/}{T Brand Studio}
\item
  \href{https://www.nytimes3xbfgragh.onion/privacy/cookie-policy\#how-do-i-manage-trackers}{Your
  Ad Choices}
\item
  \href{https://www.nytimes3xbfgragh.onion/privacy}{Privacy}
\item
  \href{https://help.nytimes3xbfgragh.onion/hc/en-us/articles/115014893428-Terms-of-service}{Terms
  of Service}
\item
  \href{https://help.nytimes3xbfgragh.onion/hc/en-us/articles/115014893968-Terms-of-sale}{Terms
  of Sale}
\item
  \href{https://spiderbites.nytimes3xbfgragh.onion}{Site Map}
\item
  \href{https://help.nytimes3xbfgragh.onion/hc/en-us}{Help}
\item
  \href{https://www.nytimes3xbfgragh.onion/subscription?campaignId=37WXW}{Subscriptions}
\end{itemize}
