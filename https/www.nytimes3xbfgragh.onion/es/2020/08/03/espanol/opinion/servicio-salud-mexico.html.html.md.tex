Sections

SEARCH

\protect\hyperlink{site-content}{Skip to
content}\protect\hyperlink{site-index}{Skip to site index}

\href{https://www.nytimes3xbfgragh.onion/es/section/opinion}{Opinión}

\href{https://myaccount.nytimes3xbfgragh.onion/auth/login?response_type=cookie\&client_id=vi}{}

\href{https://www.nytimes3xbfgragh.onion/section/todayspaper}{Today's
Paper}

\href{/es/section/opinion}{Opinión}\textbar{}La salud pública en México
es eso-que-nadie-quiere-usar

\url{https://nyti.ms/2XlqmRD}

\begin{itemize}
\item
\item
\item
\item
\item
\end{itemize}

\href{https://www.nytimes3xbfgragh.onion/es/spotlight/coronavirus?action=click\&pgtype=Article\&state=default\&region=TOP_BANNER\&context=storylines_menu}{El
brote de coronavirus}

\begin{itemize}
\tightlist
\item
  \href{https://www.nytimes3xbfgragh.onion/es/interactive/2020/espanol/america-latina/coronavirus-en-mexico.html?action=click\&pgtype=Article\&state=default\&region=TOP_BANNER\&context=storylines_menu}{Mapa
  y casos en México}
\item
  \href{https://www.nytimes3xbfgragh.onion/es/2020/07/31/espanol/ciencia-y-tecnologia/ninos-contagio-coronavirus.html?action=click\&pgtype=Article\&state=default\&region=TOP_BANNER\&context=storylines_menu}{Los
  niños y el virus}
\item
  \href{https://www.nytimes3xbfgragh.onion/es/interactive/2020/science/coronavirus-tratamientos-curas.html?action=click\&pgtype=Article\&state=default\&region=TOP_BANNER\&context=storylines_menu}{Fármacos
  y tratamientos}
\item
  \href{https://www.nytimes3xbfgragh.onion/es/2020/07/06/espanol/ciencia-y-tecnologia/coronavirus-transmision-aire.html?action=click\&pgtype=Article\&state=default\&region=TOP_BANNER\&context=storylines_menu}{Cómo
  se transmite el coronavirus}
\item
  \href{https://www.nytimes3xbfgragh.onion/es/2020/07/14/espanol/estilos-de-vida/botiquin-medicina-coronavirus.html?action=click\&pgtype=Article\&state=default\&region=TOP_BANNER\&context=storylines_menu}{Prepara
  tu botiquín}
\end{itemize}

Advertisement

\protect\hyperlink{after-top}{Continue reading the main story}

\href{/es/section/opinion}{Opinión}

Supported by

\protect\hyperlink{after-sponsor}{Continue reading the main story}

Comentario

\hypertarget{la-salud-puxfablica-en-muxe9xico-es-eso-que-nadie-quiere-usar}{%
\section{La salud pública en México es
eso-que-nadie-quiere-usar}\label{la-salud-puxfablica-en-muxe9xico-es-eso-que-nadie-quiere-usar}}

Las clases medias mexicanas claudicaron de intentar atenderse en
hospitales públicos y los más pobres los usan pensando que ahí solo van
a morirse. Esto debe cambiar.

\includegraphics{https://static01.graylady3jvrrxbe.onion/images/2020/08/03/multimedia/03Rios-ES/merlin_172941285_c80cf40d-d9c9-41fd-9512-ed590e430634-articleLarge.jpg?quality=75\&auto=webp\&disable=upscale}

Por Viri Ríos

La autora es analista política mexicana y doctora en Gobierno por la
Universidad de Harvard.

\begin{itemize}
\item
  3 de agosto de 2020
\item
  \begin{itemize}
  \item
  \item
  \item
  \item
  \item
  \end{itemize}
\end{itemize}

\href{https://www.nytimes3xbfgragh.onion/newsletters/el-times}{Regístrate
para recibir nuestro boletín} con lo mejor de The New York Times.

\begin{center}\rule{0.5\linewidth}{\linethickness}\end{center}

CIUDAD DE MÉXICO --- El sistema de salud pública en México se ha
convertido en la salud a la que recurren las personas de bajos recursos
y no la sociedad en su conjunto. Y ese es un gran problema. Las personas
de mayores ingresos, que más podrían contribuir con crear un sistema de
salud público eficaz y bien financiado, no quieren invertir en algo que
desconocen, rechazan y estigmatizan.

Las clases altas han desistido de usar servicios públicos. Cuatro de
cada diez personas de alto estrato de ingreso que tienen cobertura en el
Instituto Mexicano del Seguro Social (IMSS), Instituto de Seguridad y
Servicios Sociales de los Trabajadores del Estado (ISSSTE) o en alguna
otra institución pública, se atienden con médicos privados cuando se
enferman. El servicio público es ese lugar en el que solo se atienden
cuando ya no les alcanza, la sala de emergencia de enfermedades caras.

Cabilderos del sector empresarial centran sus esfuerzos en solicitar que
los gastos de salud privada sean
\href{https://www.sat.gob.mx/consulta/23972/conoce-las-deducciones-personales}{aún
más deducibles de impuestos}. No están dispuestos a invertir en un
servicio médico que a priori han decidido no usar.

No hay un esfuerzo de construcción de salud pública por parte de los
ciudadanos con mayores ingresos sino de desmantelamiento. Esfuerzo que
beneficia a quien tiene dinero pero perjudica a todo el resto.

Una forma común de dejar de ser clase media es enfermarse de algo caro.
Familias completas de clase media son arrastradas a la pobreza por un
cáncer, una diabetes o un accidente de tránsito grave.

Sin embargo, tal parece que la clase media no lo ve así y ha normalizado
pensar en la salud pública como si fueran ricos, como una dádiva a los
pobres más que como un servicio al que podrían acceder ellos mismos. Se
piensa que el servicio es malo y por ello no se quiere pagar por él. En
realidad, el servicio es malo porque no se paga lo suficiente por él.

Las personas de menores ingresos también han claudicado, pero de
atenderse. Es habitual que los pacientes solo acudan al centro de salud
cuando ya tienen una enfermedad grave. La medicina preventiva es algo
desconocido para un sistema cuya característica principal son las largas
filas.

Llega a tomar semanas lograr ser atendido. El tiempo promedio de espera
entre el primer contacto con el médico general y recibir atención
quirúrgica de manera electiva era de más de 7 meses hasta 2011. A partir
de 2016, el IMSS comenzó a realizar cirugías en fin de semana, lo que se
dijo podría aumentar la disponibilidad de quirófanos hasta en
\href{https://www.gob.mx/imss/acciones-y-programas/cirugias-en-fines-de-semana-optimizacion-de-los-quirofanos-del-imss}{40
por ciento}. Aún si así fuera, y si todo el sector salud siguiera la
misma práctica, estaríamos hablando de 4 meses de espera para ser
atendido. Lo peor es que, aún cuando son atendidos, muchas veces los
pacientes tienen que pagar de su bolsa por curaciones y materiales. No
hay de otra, les dicen.

Hay un mito persistente en México: que todos sus habitantes tienen
cobertura médica pública. La realidad es que tenemos un sistema público
que solo atiende o enfermedades caras o enfermos graves. Los mexicanos
pagan por sí mismos el
\href{https://data.worldbank.org/indicator/SH.XPD.OOPC.CH.ZS?locations=MX-ZJ}{41
por ciento} de sus gastos en salud, una cifra mucho mayor que el
promedio de Latinoamérica, en donde solo se paga el 28 por ciento.

En México nuestro médico de cabecera
\href{https://www.sciencedirect.com/science/article/pii/S2212109917300687\#bib1}{se
ha vuelto la farmacia}. De acuerdo con la Comisión Federal para la
Protección contra Riesgos Sanitarios (Cofepris), hasta 2012 existían
20.000 médicos de farmacia que otorgan 35 millones de consultas al año.
Y no las otorgan gratis. El costo escondido es que los médicos de
farmacia recetan la compra de más medicinas. Mientras que solo alrededor
del 44 por ciento de los pacientes atendidos por la Secretaría de Salud
son recetados con tres o más medicamentos en una consulta, con los
doctores de farmacia la cifra sube a 67 por ciento.

Es momento de construir un sistema de salud que no sea el último recurso
de las personas sino el primero. Ello requiere un aumento sin precedente
en el gasto en salud.

El gasto en salud en México es de
\href{https://data.worldbank.org/indicator/SH.XPD.CHEX.GD.ZS?locations=MX-HN-SV-ZW}{5,5
puntos del PIB}, menor al gasto de Honduras, El Salvador y Zimbabue. El
gasto no solo es extremadamente bajo, sino que de acuerdo a
\href{https://ciep.mx/la-contraccion-del-gasto-per-capita-en-salud-2010-2020/}{los
estudios del Centro de Investigación Económica y Presupuestaria (CIEP}),
una organización civil dedicada al análisis de finanzas públicas, ha
caído significativamente en los últimos diez años. De acuerdo con la
Organización Panamericana de la Salud, un país como México debería al
menos gastar el doble de lo que gasta actualmente en salud.

El gobierno de Andrés Manuel López Obrador ha hecho esfuerzos
importantes por aumentar la cobertura, sobre todo para los más pobres.
El gasto de IMSS-Bienestar, típicamente usado por individuos con menor
nivel de ingreso, aumentó en
\href{https://ciep.mx/la-contraccion-del-gasto-per-capita-en-salud-2010-2020/}{7
por ciento en 2020}, comprado con 2019. Y, en contraste, el gasto en
salud de PEMEX, la petrolera estatal que emplea a funcionario de
relativamente buen nivel de ingreso, se contrajo.

Los gastos en salud privada no deben ser deducibles pues esto crea
incentivos a la privatización del servicio. Solo mediante un servicio
público pagado por todos podremos aumentar la calidad de la atención en
salud pública. Considero un acierto que el gobierno haya dejado de
\href{https://www.eleconomista.com.mx/politica/Seguros-para-burocratas-costaban-al-gobierno-11000-millones-de-pesos-al-ano-20190901-0029.html}{pagar
seguros privados} a los funcionarios públicos, pues ello no solamente
reforzaba la desigualdad (se pagaban mejores seguros a los empleados que
ganaban más dinero) sino que sembraba la idea de que la salud pública
era un servicio de segunda. Lo que no es acertado es que, a la vez de
que se cancelaron los seguros privados, no se haya aumentado
significativamente el gasto del ISSSTE. En 2019, cuando los seguros
fueron cancelados, el presupuesto del ISSSTE solo aumentó
\href{https://ciep.mx/la-contraccion-del-gasto-per-capita-en-salud-2010-2020/}{2,5
por ciento}.

Así mismo, los médicos privados deben ser fiscalizados con mucho más
rigor. El Servicio de Administración Tributaria (SAT) estima que los
médicos privados evaden alrededor del
\href{http://omawww.sat.gob.mx/gobmxtransparencia/Paginas/documentos/estudio_opiniones/EvasionActividadesProfesionales.pdf}{30
por ciento} de sus pagos de ISR. Algunos médicos usan pagos en efectivo
para evadir impuestos. La profesión médica debe de dejar de ser un
paraíso fiscal, pues esos recursos son necesarios para equipar mejor el
sistema de salud pública.

El sistema de salud público europeo se construyó en buena medida como
reacción a la pandemia de fiebre española. El coronavirus debe ser para
México el mismo motor.

Creamos un país que no esta listo para atender a la población y ha
vivido de simular que sí lo hace. Hay que aumentar la recaudación para
dar mejor servicio de salud pública y cobrar mayores impuestos a los
servicios privados. Esta pandemia es un llamado de atención que no
podemos dejar pasar: el gasto en salud debe duplicarse en México.

Viri Ríos es analista política y colaboradora regular en español de The
New York Times. \href{https://twitter.com/Viri_Rios}{@Viri\_Rios}

Advertisement

\protect\hyperlink{after-bottom}{Continue reading the main story}

\hypertarget{site-index}{%
\subsection{Site Index}\label{site-index}}

\hypertarget{site-information-navigation}{%
\subsection{Site Information
Navigation}\label{site-information-navigation}}

\begin{itemize}
\tightlist
\item
  \href{https://help.nytimes3xbfgragh.onion/hc/en-us/articles/115014792127-Copyright-notice}{©~2020~The
  New York Times Company}
\end{itemize}

\begin{itemize}
\tightlist
\item
  \href{https://www.nytco.com/}{NYTCo}
\item
  \href{https://help.nytimes3xbfgragh.onion/hc/en-us/articles/115015385887-Contact-Us}{Contact
  Us}
\item
  \href{https://www.nytco.com/careers/}{Work with us}
\item
  \href{https://nytmediakit.com/}{Advertise}
\item
  \href{http://www.tbrandstudio.com/}{T Brand Studio}
\item
  \href{https://www.nytimes3xbfgragh.onion/privacy/cookie-policy\#how-do-i-manage-trackers}{Your
  Ad Choices}
\item
  \href{https://www.nytimes3xbfgragh.onion/privacy}{Privacy}
\item
  \href{https://help.nytimes3xbfgragh.onion/hc/en-us/articles/115014893428-Terms-of-service}{Terms
  of Service}
\item
  \href{https://help.nytimes3xbfgragh.onion/hc/en-us/articles/115014893968-Terms-of-sale}{Terms
  of Sale}
\item
  \href{https://spiderbites.nytimes3xbfgragh.onion}{Site Map}
\item
  \href{https://help.nytimes3xbfgragh.onion/hc/en-us}{Help}
\item
  \href{https://www.nytimes3xbfgragh.onion/subscription?campaignId=37WXW}{Subscriptions}
\end{itemize}
