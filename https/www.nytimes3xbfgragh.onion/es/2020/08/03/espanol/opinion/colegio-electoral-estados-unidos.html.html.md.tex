\href{/es/section/opinion}{Opinión}\textbar{}El Colegio Electoral de
Estados Unidos: la poco conocida historia que explica su vigencia

\url{https://nyti.ms/3i3Xp4r}

\begin{itemize}
\item
\item
\item
\item
\item
\end{itemize}

\includegraphics{https://static01.graylady3jvrrxbe.onion/images/2020/08/03/opinion/03Keyssar-ES-1/03keyssarWeb-articleLarge.jpg?quality=75\&auto=webp\&disable=upscale}

Sections

\protect\hyperlink{site-content}{Skip to
content}\protect\hyperlink{site-index}{Skip to site index}

Comentario

\hypertarget{el-colegio-electoral-de-estados-unidos-la-poco-conocida-historia-que-explica-su-vigencia}{%
\section{El Colegio Electoral de Estados Unidos: la poco conocida
historia que explica su
vigencia}\label{el-colegio-electoral-de-estados-unidos-la-poco-conocida-historia-que-explica-su-vigencia}}

Aunque ha habido intentos recurrentes de reformar el complejo método de
elegir presidentes en ese país, las políticas raciales han tenido un
lugar protagónico en impedir cambios.

Votantes en Mississippi en 1946Credit...Bettmann Archive/Getty Images

Supported by

\protect\hyperlink{after-sponsor}{Continue reading the main story}

Por Alexander Keyssar

Es profesor de historia y política social en Harvard y autor de
\href{https://www.hup.harvard.edu/catalog.php?isbn=9780674660151\#:~:text=After\%20tracing\%20the\%20Electoral\%20College's,showing\%20why\%20each\%20has\%20failed.}{\emph{Why
Do We Still Have the Electoral College}}\emph{?}

\begin{itemize}
\item
  3 de agosto de 2020
\item
  \begin{itemize}
  \item
  \item
  \item
  \item
  \item
  \end{itemize}
\end{itemize}

\href{https://www.nytimes3xbfgragh.onion/2020/08/03/opinion/electoral-college-racism-white-supremacy.html}{Read
in English}

\href{https://www.nytimes3xbfgragh.onion/newsletters/el-times}{Regístrate
para recibir nuestro boletín} con lo mejor de The New York Times.

\begin{center}\rule{0.5\linewidth}{\linethickness}\end{center}

Como nuestra resucitada conversación nacional sobre la raza ha dejado en
claro, el legado de la esclavitud y la supremacía blanca es extenso y
profundo en la sociedad y la vida política de Estados Unidos. Un legado
de este tipo ---que es particularmente visible en una temporada de
elecciones presidenciales--- ha sido la supervivencia y la preservación
del Colegio Electoral, una institución criticada por más de 200 años.
Nuestro complicado método de elegir presidentes ha sido blanco de
intentos recurrentes de reforma desde principios del siglo XIX, y las
políticas raciales y de religión han tenido un lugar protagónico en su
derrota.

Por supuesto, no es ningún secreto que la esclavitud desempeñó un papel
en el diseño original de nuestro sistema de elecciones presidenciales,
aunque los
\href{https://www.nytimes3xbfgragh.onion/2019/04/04/opinion/the-electoral-college-slavery-myth.html?action=click\&module=RelatedLinks\&pgtype=Article}{historiadores}
\href{https://www.nytimes3xbfgragh.onion/2019/04/06/opinion/electoral-college-slavery.html}{no
están de acuerdo} sobre la centralidad de ese papel. La conocida fórmula
que daba representación a los estados en el Congreso por las tres
quintas partes de sus esclavos se transfirió a la asignación de votos
electorales; el número de votos electorales otorgados a cada estado era
(y sigue siendo) equivalente a la representación de ese estado en la
Cámara y el Senado. Este diseño constitucional dio a los sureños blancos
una influencia desproporcionada en la elección de los presidentes, una
ventaja que podría afectar el resultado de las elecciones.

No es sorprendente que los estados esclavistas se opusieran
enérgicamente a cualquier cambio en el sistema que disminuyera su
ventaja. En 1816, cuando se introdujo por primera vez en el Congreso una
resolución que pedía el voto popular nacional, las protestas de los
senadores del sur la torpedearon. Los estados esclavistas ``perderían el
privilegio que la Constitución ahora les permite, de votos sobre tres
quintos de su población que no sean hombres libres'', objetó William
Wyatt Bibb, de Georgia, en el pleno del Senado. ``Sería profundamente
perjudicial para ellos''.

Lo que es mucho menos conocido, o reconocido, es que mucho después de la
abolición de la esclavitud, los líderes políticos del sur siguieron
resistiendo cualquier intento de reemplazar el Colegio Electoral con el
voto popular nacional. (Algunas veces apoyaron otras reformas, como la
división proporcional de los votos electorales de cada estado, pero esos
son hilos argumentales diferentes de un cuento multifacético). El
razonamiento detrás de esta oposición fue directo, aunque inquietante.
Después de la
\href{https://ar.usembassy.gov/wp-content/uploads/sites/26/2016/10/Capitulo-7_La_guerra_civil_y_la_reconstruccion.pdf}{Reconstrucción},
los gobiernos blancos ``redentores'' que llegaron al poder en los
estados del sur se convirtieron en los beneficiarios políticos de lo que
equivalía a una cláusula de ``cinco quintos'': los afroamericanos
contaban plenamente para la representación (y, por lo tanto, los votos
electorales), pero volvieron a ser privados de sus derechos a pesar de
las protecciones formales descritas en el decimoquinta enmienda,
ratificada en 1870, que declararon que no se podía negar el derecho al
voto ``por motivos de raza, color o condición previa de servidumbre''.
Los sureños blancos, en consecuencia, obtuvieron un beneficio aún mayor
del Colegio Electoral que el que tenían antes de la Guerra de Secesión.

Un voto popular nacional habría eliminado ese beneficio. Como
reconocieron los líderes políticos de la región, la aprobación de una
enmienda constitucional que instituyese un voto nacional popular habría
generado fuertes presiones legales y políticas para otorgar derechos a
los afroestadounidenses. Incluso si se pudieran resistir esas presiones,
un folleto de la campaña de Alabama señaló en 1914 que ``con la mitad
negra de nuestra gente sin votar, nuestra voz en las elecciones
nacionales, que ahora se basa en la población total, se apoyaría
únicamente en nuestra población votante y, por lo tanto, se reduciría a
la mitad''. Las consecuencias políticas de un voto nacional popular
simplemente no podían ser toleradas.

Hacia la década de 1940, muchos sureños también llegaron a creer que su
peso desproporcionado en las elecciones presidenciales, gracias al
Colegio Electoral, era un bastión fundamental contra las crecientes
presiones del norte para ampliar los derechos civiles y políticos de los
afroestadounidenses. En 1947, el influyente tratado \emph{Whither Solid
South?} de Charles Collins, sobre los derechos y el segregacionismo de
los estados, imploró a los sureños rechazar ``cualquier intento de
acabar con el Colegio porque solo este puede permitir que los estados
del sur conserven sus derechos dentro de la Unión''. El libro, que se
convirtió en una lectura obligada entre los Dixiecrats ---del
segregacionista Partido Demócrata de los Derechos de los Estados--- que
huyeron del Partido Demócrata en 1948, fue muy elogiado y distribuido de
forma gratuita por (entre otros) el segregacionista de Mississippi James
Eastland, quien sirvió en el Senado de 1943 a 1978.

Impulsados por tales convicciones, los regímenes de supremacía blanca
del sur se mantuvieron como un obstáculo en el camino de un voto popular
nacional desde las últimas décadas del siglo XIX hasta la década de
1960, cuando la Ley de Derechos Electorales y otras medidas obligaron a
la región a otorgar derechos a los afroestadounidenses. Hubo, por
supuesto, resistencia a la idea de un voto nacional en otras partes del
país, pero fue la bien conocida actitud inflexible del sur ---y el hecho
de que solo los estados del sur podrían estar cerca de bloquear una
enmienda constitucional en el Congreso--- lo que mantuvo la idea en los
márgenes del debate público durante décadas. Numerosos líderes políticos
que personalmente favorecieron el voto popular nacional, como el senador
republicano Henry Cabot Lodge, Jr. de Massachusetts en la década de
1940, concluyeron que tal reforma no tenía posibilidades realistas de
éxito, y cambiaron su defensa a medidas menos radicales.

La política de raza y región también tuvo un lugar destacado en la
derrota punzante de una enmienda al voto popular nacional en el Senado
en 1970, lo más cerca que Estados Unidos ha estado de transformar su
sistema de elecciones presidenciales desde 1821. El apoyo popular y de
élite a la idea había proliferado en la década de 1960, lo que llevó a
que en 1969 la Cámara de Representantes votase abrumadoramente a favor
de una enmienda constitucional que habría abolido el Colegio Electoral.
La propuesta se empantanó en el Senado durante un año en que las
tensiones regionales eran altas: dos candidatos del sur a la Corte
Suprema fueron rechazados por el Senado, y la Ley de Derechos
Electorales se renovó por encima de la fuerte oposición de los senadores
del sur. Mientras tanto, la enmienda del voto popular nacional se
estancó en el Comité Judicial, que fue presidido nada menos que por el
senador Eastland.

Cuando la resolución de la enmienda finalmente llegó al pleno del Senado
en septiembre de 1970, gracias a los prodigiosos esfuerzos de un senador
de Indiana, Birch Bayh, fue recibida por las maniobras obstruccionistas
dirigidas por los segregacionistas Sam Ervin y Strom Thurmond (con la
ayuda del republicano de Nebraska Roman Hruska). Aunque las cosas
cambiaban en el sur, sus líderes políticos seguían inmersos en los
valores y las perspectivas que habían fundamentado su hostilidad al
movimiento de los derechos civiles y a la Ley de Derechos Electorales.
``El Colegio Electoral'', escribió el senador James Allen de Alabama en
1969, ``es una de las pocas salvaguardias políticas que quedan en el
sur. Vamos a conservarlo''.

Las maniobras obstruccionistas tuvieron éxito: los intentos de invocar
la clausura ---para terminar el debate y votar sobre la enmienda en sí
misma--- quedaron unos pocos votos por debajo de la mayoría de dos
tercios que entonces se necesitaban para acabar con la obstrucción. Las
alineaciones regionales en los votos cruciales (hubo dos) fueron
claramente visibles. Más del 75 por ciento de los senadores del sur
votaron en contra de la clausura; una proporción similar de senadores
que no pertenecían al sur votaron a favor.

De ese modo, los líderes políticos del sur ---formados por la
segregación y las creencias de la supremacía blanca--- mantuvieron la
idea de un voto popular nacional fuera de discusión durante muchas
décadas y desempeñaron un papel crucial en el bloqueo de su paso por el
Congreso en una coyuntura histórica cuando el cambio realmente parecía
posible. Sin duda, una reforma electoral es casi siempre un proceso
complejo y difícil, con diversos actores que compiten por defender sus
ideas e intereses. Pero si la política de la raza hubiera sido menos
destacada, tanto en el siglo XIX como en el siglo XX, el Colegio
Electoral probablemente habría sido relegado hace mucho tiempo al estado
de curiosidad histórica. Es posible que deseemos tener en cuenta ese
hecho aleccionador mientras miramos hacia una elección cuyo resultado es
cuestionable solo por la forma peculiar en el que elegimos a nuestros
presidentes.

Alexander Keyssar
(\href{https://twitter.com/alexkeyssar?lang=en}{@AlexKeyssar}), profesor
de historia y política social en Harvard, es el autor de
\href{https://www.hup.harvard.edu/catalog.php?isbn=9780674660151\#:~:text=After\%20tracing\%20the\%20Electoral\%20College's,showing\%20why\%20each\%20has\%20failed.}{\emph{Why
Do We Still Have the Electoral College?}} y
\href{https://www.basicbooks.com/titles/alexander-keyssar/the-right-to-vote/9780465005024/}{\emph{The
Right to Vote: The Contested History of Democracy in the United
States}}.

Advertisement

\protect\hyperlink{after-bottom}{Continue reading the main story}

\hypertarget{site-index}{%
\subsection{Site Index}\label{site-index}}

\hypertarget{site-information-navigation}{%
\subsection{Site Information
Navigation}\label{site-information-navigation}}

\begin{itemize}
\tightlist
\item
  \href{https://help.nytimes3xbfgragh.onion/hc/en-us/articles/115014792127-Copyright-notice}{©~2020~The
  New York Times Company}
\end{itemize}

\begin{itemize}
\tightlist
\item
  \href{https://www.nytco.com/}{NYTCo}
\item
  \href{https://help.nytimes3xbfgragh.onion/hc/en-us/articles/115015385887-Contact-Us}{Contact
  Us}
\item
  \href{https://www.nytco.com/careers/}{Work with us}
\item
  \href{https://nytmediakit.com/}{Advertise}
\item
  \href{http://www.tbrandstudio.com/}{T Brand Studio}
\item
  \href{https://www.nytimes3xbfgragh.onion/privacy/cookie-policy\#how-do-i-manage-trackers}{Your
  Ad Choices}
\item
  \href{https://www.nytimes3xbfgragh.onion/privacy}{Privacy}
\item
  \href{https://help.nytimes3xbfgragh.onion/hc/en-us/articles/115014893428-Terms-of-service}{Terms
  of Service}
\item
  \href{https://help.nytimes3xbfgragh.onion/hc/en-us/articles/115014893968-Terms-of-sale}{Terms
  of Sale}
\item
  \href{https://spiderbites.nytimes3xbfgragh.onion}{Site Map}
\item
  \href{https://help.nytimes3xbfgragh.onion/hc/en-us}{Help}
\item
  \href{https://www.nytimes3xbfgragh.onion/subscription?campaignId=37WXW}{Subscriptions}
\end{itemize}
