Sections

SEARCH

\protect\hyperlink{site-content}{Skip to
content}\protect\hyperlink{site-index}{Skip to site index}

\href{https://www.nytimes3xbfgragh.onion/es/section/opinion}{Opinión}

\href{https://myaccount.nytimes3xbfgragh.onion/auth/login?response_type=cookie\&client_id=vi}{}

\href{https://www.nytimes3xbfgragh.onion/section/todayspaper}{Today's
Paper}

\href{/es/section/opinion}{Opinión}\textbar{}Por qué los brasileños
deberían temer a la oficina de odio

\url{https://nyti.ms/3kafIXH}

\begin{itemize}
\item
\item
\item
\item
\item
\end{itemize}

\href{https://www.nytimes3xbfgragh.onion/es/spotlight/coronavirus?action=click\&pgtype=Article\&state=default\&region=TOP_BANNER\&context=storylines_menu}{El
brote de coronavirus}

\begin{itemize}
\tightlist
\item
  \href{https://www.nytimes3xbfgragh.onion/es/interactive/2020/espanol/america-latina/coronavirus-en-mexico.html?action=click\&pgtype=Article\&state=default\&region=TOP_BANNER\&context=storylines_menu}{Mapa
  y casos en México}
\item
  \href{https://www.nytimes3xbfgragh.onion/es/2020/07/31/espanol/ciencia-y-tecnologia/ninos-contagio-coronavirus.html?action=click\&pgtype=Article\&state=default\&region=TOP_BANNER\&context=storylines_menu}{Los
  niños y el virus}
\item
  \href{https://www.nytimes3xbfgragh.onion/es/interactive/2020/science/coronavirus-tratamientos-curas.html?action=click\&pgtype=Article\&state=default\&region=TOP_BANNER\&context=storylines_menu}{Fármacos
  y tratamientos}
\item
  \href{https://www.nytimes3xbfgragh.onion/es/2020/07/06/espanol/ciencia-y-tecnologia/coronavirus-transmision-aire.html?action=click\&pgtype=Article\&state=default\&region=TOP_BANNER\&context=storylines_menu}{Cómo
  se transmite el coronavirus}
\item
  \href{https://www.nytimes3xbfgragh.onion/es/2020/07/14/espanol/estilos-de-vida/botiquin-medicina-coronavirus.html?action=click\&pgtype=Article\&state=default\&region=TOP_BANNER\&context=storylines_menu}{Prepara
  tu botiquín}
\end{itemize}

Advertisement

\protect\hyperlink{after-top}{Continue reading the main story}

\href{/es/section/opinion}{Opinión}

Supported by

\protect\hyperlink{after-sponsor}{Continue reading the main story}

Comentario

\hypertarget{por-quuxe9-los-brasileuxf1os-deberuxedan-temer-a-la-oficina-de-odio}{%
\section{Por qué los brasileños deberían temer a la oficina de
odio}\label{por-quuxe9-los-brasileuxf1os-deberuxedan-temer-a-la-oficina-de-odio}}

El presidente Jair Bolsonaro, sus hijos y aliados han sembrado el odio
en línea contra las instituciones que defienden la democracia. Ahora la
indignación se está desbordando en la calle.

\includegraphics{https://static01.graylady3jvrrxbe.onion/images/2020/08/05/opinion/05campos/04campos-articleLarge.jpg?quality=75\&auto=webp\&disable=upscale}

Por Patrícia Campos Mello

Es periodista brasilera.

\begin{itemize}
\item
  4 de agosto de 2020 a las 07:07 ET
\item
  \begin{itemize}
  \item
  \item
  \item
  \item
  \item
  \end{itemize}
\end{itemize}

\href{https://www.nytimes3xbfgragh.onion/2020/08/04/opinion/bolsonaro-office-of-hate-brazil.html}{Read
in
English}\href{https://www.nytimes3xbfgragh.onion/pt/2020/08/04/opinion/international-world/bolsonaro-gabinete-do-odio.html}{Ler
em português}

SÃO PAULO --- El 13 de junio, miembros de los ``300 de Brasil'', una
milicia bolsonarista de extrema derecha,
\href{https://www1.folha.uol.com.br/poder/2020/05/sara-winter-xinga-moraes-diz-querer-trocar-socos-com-ele-e-promete-inferniza-lo.shtml}{lanzaron
fuegos artificiales} hacia el edificio del Supremo Tribunal Federal en
Brasilia, simulando un bombardeo. ``Prepárese, Supremo de los
bandidos\ldots{} están llevando el país al comunismo'', dijo uno de los
líderes, que
\href{https://www.metropoles.com/brasil/video-bolsonaristas-lancam-fogos-de-artificio-em-predio-do-stf}{transmitió
la protesta en vivo}. ``¡Se acabó, carajo!'', dijo otro manifestante,
haciéndose eco de las palabras que el presidente
\href{https://www.youtube.com/watch?v=I2bZoC8FHJI}{había usado} para
condenar una investigación del tribunal en contra de algunos de sus
partidarios, que participan en campañas de desinformación y amenazan a
los jueces.

¿De dónde vino este odio al máximo tribunal de Brasil?

En los meses previos al incidente de los fuegos artificiales, miles de
cuentas de redes sociales, muchas de ellas cuentas falsas vinculadas a
partidarios de Bolsonaro o blogueros de extrema derecha, publicaron
\href{https://www1.folha.uol.com.br/poder/2020/05/sara-winter-xinga-moraes-diz-querer-trocar-socos-com-ele-e-promete-inferniza-lo.shtml}{amenazas}
contra los jueces del tribunal. Pidieron que se aboliera la corte o que
el país regresara a una dictadura militar. Uno de los seguidores del
presidente incluso habló de
\href{https://g1.globo.com/politica/noticia/2020/06/17/moraes-vota-pela-legalidade-e-continuidade-do-inquerito-das-fake-news.ghtml}{matar
y desmembrar} a los jueces y sus familias. Era cuestión de tiempo para
que la animosidad se desbordara en la calle.

Este ambiente tóxico ha sido fomentado por lo que los brasileños llaman
la ``oficina de odio'', una operación dirigida por asesores de
Bolsonaro, que patrocinan una red de blogs y cuentas de redes sociales
que difunden noticias falsas y atacan a periodistas, políticos, artistas
y a medios de comunicación que critican al presidente. La oficina de
odio no tiene título ni presupuesto oficial, pero su trabajo se subsidia
con dinero de los contribuyentes. No está claro cuántas personas
trabajan para esta oficina ni quiénes son. De hecho, Bolsonaro y sus
aliados niegan que exista. Pero las semillas del odio y la división que
está sembrando amenazan con deshacer nuestra democracia.

El gobierno de Bolsonaro se enfrenta actualmente a tres investigaciones
directamente relacionadas con esta máquina de odio. Una investigación
del Supremo Tribunal está indagando ataques a miembros de la corte
financiados por líderes empresariales y difundidos por la red
bolsonarista. Otra está examinando el financiamiento de manifestaciones
que piden el cierre del Congreso y del poder judicial. Cuatro
investigaciones en el Tribunal Superior Electoral están analizando el
uso de campañas de desinformación y difamación de mensajes a través de
WhatsApp durante la campaña electoral de 2018, supuestamente financiado
por líderes empresariales.

\includegraphics{https://static01.graylady3jvrrxbe.onion/images/2020/08/04/opinion/04campos-ES-2/04Campos2-articleLarge.jpg?quality=75\&auto=webp\&disable=upscale}

El 8 de julio, Facebook eliminó
\href{https://www1.folha.uol.com.br/poder/2020/07/facebook-remove-contas-falsas-ligadas-aos-bolsonaros-e-ao-gabinete-da-presidencia.shtml}{docenas
de cuentas}, algunas utilizadas por empleados de Bolsonaro y sus hijos.
\href{https://elpais.com/internacional/2020-07-10/facebook-rompe-la-oficina-del-odio-una-red-de-88-cuentas-de-apoyo-a-jair-bolsonaro.html}{Tércio
Arnaud Tomaz}, un asesor especial de Bolsonaro,
\href{https://apnews.com/0c58cccec2004bf250c8dab38172cbc9}{administró
algunas de las cuentas}, y se cree que él dirige la oficina de odio.

Lamentablemente, conozco demasiado de cerca a la oficina de odio.
Durante los últimos dos años, he estado cubriendo la desinformación y la
política. Me convertí en blanco en 2018, cuando expuse en el periódico
Folha de São Paulo que unos líderes empresariales habían estado pagando
por la difusión de millones de mensajes falsos a través de WhatsApp para
influir en las elecciones presidenciales de ese año.

Como resultado, me he enfrentado a crudas amenazas y ataques personales.
Los \emph{trolls} e incluso los políticos han compartido memes en los
que mi cara aparece en montajes pornográficos y refiriéndose a mí como
prostituta. La gente me envía mensajes diciendo que debería ser violada.
Estoy demandando a Bolsonaro, a su hijo Eduardo y a un bloguero
bolsonarista por
\href{http://www.fundamedios.us/incidentes/patriciacampos-demanda-jairbolsonaro-ofensas-periodista/}{daños
morales} por declarar o insinuar repetidamente que ofrezco sexo a cambio
de primicias.

Y no estoy sola. Muchas periodistas respetadas en Brasil también han
sufrido ataques misóginos. La prensa, junto con los tribunales y el
Congreso, es una de las últimas barreras que contiene al presidente.
Pero no sé por cuánto tiempo más podremos resistir a Bolsonaro y sus
seguidores. La retórica y las acciones cada vez más agresivas por parte
del presidente, sus hijos y aliados sirven como justificación para que
las milicias bolsonaristas pasen de los insultos a las injurias.

El 25 de mayo, sus partidarios acosaron a los periodistas cerca de la
residencia presidencial en Brasilia. En los
\href{https://twitter.com/folha/status/1264913877399212034}{videos
tomados} ese día se ve cómo llaman extorsionistas y delincuentes a los
periodistas. Una mujer grita: ``¡Escoria! ¡Basura! ¡Ratas! ¡Bolsonaro
hasta 2050! ¡Prensa podrida! ¡Comunistas!''.

Image

Partidarios de Bolsonaro le gritan insultos a los periodistas después de
que el presidente saliera de la residencia oficial en
Brasilia.Credit...Eraldo Peres/Associated Press

Por supuesto, los periodistas no son los únicos que están siendo
atacados. Durante el último año, la oficina de odio ha hecho que los
brasileños se enfrenten y ha derribado la confianza en las mismas
instituciones que fueron diseñadas para protegerlos.

El grupo estuvo detrás de una campaña difamatoria que calificó a Sergio
Moro ---exministro de Justicia estelar y el juez principal de
\href{https://www.nytimes3xbfgragh.onion/2017/09/18/opinion/brazil-corruption-car-wash.html?searchResultPosition=1}{Lava
Jato}, la investigación de corrupción en Brasil--- como ``traidor'' y
``comunista''. Moro
\href{https://www.nytimes3xbfgragh.onion/2020/04/24/world/americas/brazil-bolsonaro-moro.html}{renunció}
en protesta en abril y denunció la intromisión del presidente en una
investigación de la Policía Federal para proteger a sus hijos y aliados
de investigaciones criminales. Poco después de su renuncia, las redes
sociales se inundaron con memes desde cuentas falsas que amenazaban a
Moro.

Con la propagación del coronavirus,
\href{https://www.bbc.com/news/53361876}{las noticias y curas falsas}
comenzaron a proliferar en las redes sociales, a menudo a través de
legisladores federales con cientos de miles de seguidores. Bolsonaro ha
\href{https://www.hrw.org/news/2020/04/10/brazil-bolsonaro-sabotages-anti-covid-19-efforts}{quebrantado}
las pautas de distanciamiento social establecidas por los gobernadores,
y asesores como Arnaud Tomaz afirmaron que la reacción a la COVID-19 fue
\href{https://www.bbc.com/portuguese/brasil-53353594}{exagerada} y que
la hidroxicloroquina, el fármaco antipalúdico
\href{https://www.nytimes3xbfgragh.onion/2020/06/13/world/americas/virus-brazil-bolsonaro-chloroquine.html}{promovido}
por Bolsonaro como una cura del coronavirus, podría matar el virus.

En abril, el gobierno comenzó a rastrear un ``Marcador de la vida'' en
\href{https://www.facebookcorewwwi.onion/minsaude/posts/3549590468392877}{Facebook}
y
\href{https://twitter.com/secomvc/status/1257836970518200323}{Twitter},
que registraba el número de pacientes que se habían recuperado. Luego,
en junio, el Ministerio de Salud eliminó el número total de casos y
muertes confirmados de la COVID-19 desde el inicio de la pandemia, y
colocaron en su lugar un gráfico con los casos y muertes reportados en
las últimas 24 horas. Posteriormente, el Supremo Tribunal
\href{https://www.nytimes3xbfgragh.onion/2020/06/19/world/coronavirus-live-updates.html}{ordenó}
al gobierno no ocultar datos.

Pero el coronavirus no se detiene por las agendas políticas. La
``\href{https://www.cnn.com/2020/05/23/americas/brazil-coronavirus-hospitals-intl/index.html}{pequeña
gripe}'', como Bolsonaro se ha referido al virus que él y su esposa
contrajeron en julio, ya ha matado a más de
\href{https://www.nytimes3xbfgragh.onion/interactive/2020/world/americas/brazil-coronavirus-cases.html}{94.000
brasileños}, la segunda
\href{https://coronavirus.jhu.edu/map.html}{cifra más alta} de muertos
en el mundo. La campaña de noticias falsas del presidente ha resultado
en una muerte prematura para miles de personas.

Más allá de las campañas de difamación y desinformación, el propósito de
la oficina de odio es mucho más nefasto: debilitar las instituciones
democráticas de Brasil. Las investigaciones del fiscal general revelaron
que algunos legisladores están gastando fondos del gabinete en agencias
de mercadotecnia que utilizan las redes sociales para fomentar protestas
contra el tribunal y el Congreso, y a favor de la intervención militar
en la política.

Esta incitación tiene la intención de convencer a los bolsonaristas de
que los jueces del la tribunal son dictadores, y que la prensa y el
Congreso están impidiendo que el presidente gobierne y conspirando un
golpe de Estado. Bolsonaro podría estar preparando el terreno para
justificar una intervención militar. Y en una democracia joven como
Brasil, las instituciones pueden ser más frágiles de lo que parecen.

Aunque Bolsonaro fue elegido democráticamente, ha profesado admiración
por el régimen militar que gobernó Brasil desde 1964 hasta 1985. Mucho
antes de postularse a la presidencia,
\href{https://www.youtube.com/watch?v=qIDyw9QKIvw\&t=577s}{dijo} que una
guerra civil lograría lo que el régimen militar no pudo. También dijo
que cerraría el Congreso si fuera presidente. Durante las elecciones
presidenciales de 2018, sus hijos y seguidores imprimieron camisetas con
la cara del coronel Alberto Brilhante Ustra, el principal torturador de
la dictadura, una figura celebrada por el presidente.

Image

Un manifestante con la bandera de Brasil participó en un marcha en Río
de Janeiro en contra del Supremo Tribunal Federal en
junio.Credit...Bruna Prado/Getty Images

Bolsonaro ha tratado de cumplir con su visión. En un esfuerzo por
sortear el Congreso, ha firmado un número récord de órdenes ejecutivas y
proyectos de ley diseñados para eliminar la independencia de las
universidades públicas, que describe como guaridas del comunismo;
restringir el acceso a la información, y debilitar sindicatos y
periódicos. Ha intentado desobedecer los fallos del poder judicial.

Él ha dicho que quiere armar a la población entera, para que las
personas puedan defenderse contra la ``dictadura'' del tribunal federal
y los gobernadores. ``Quiero que todos tengan armas porque una población
armada nunca será esclavizada'', dijo durante una reunión de gabinete en
mayo. Más tarde firmó una orden ejecutiva que facilita la importación de
armas y aumenta la cantidad de municiones que una persona puede comprar
en un año. En cualquier democracia funcional, todo esto equivaldría a un
llamado a la insurrección.

Al presidente y sus secuaces no les gustaría nada más que silenciar a
todos aquellos que iluminan los recovecos más oscuros de su gobierno.

Durante el último año, el objetivo de la oficina de odio se ha vuelto
cada vez más importante: poner a los brasileños en contra de aquellos
que han servido de controles y contrapesos contra el auge autoritario de
Bolsonaro.

Ataques contra el tribunal o la agresión hacia un fotoperiodista durante
una protesta contra el Congreso y a favor del golpe militar son señales
de que de alguna manera la oficina de odio está teniendo éxito en su
llamado a la insurrección.

La semana pasada, dos hombres en un automóvil equipado con altavoces
\href{https://esportes.yahoo.com/noticias/aliados-jair-bolsonaro-atacam-casa-felipe-neto-010129218.html}{aparecieron
en la entrada de la casa del
popular}\href{https://esportes.yahoo.com/noticias/aliados-jair-bolsonaro-atacam-casa-felipe-neto-010129218.html}{\emph{youtuber}}\href{https://esportes.yahoo.com/noticias/aliados-jair-bolsonaro-atacam-casa-felipe-neto-010129218.html}{Felipe
Neto} con la intención de intimidarlo, y lo acusaron de destruir la
``institución más importante, que es la familia''. Días antes, Neto
denunció el manejo de la pandemia por parte del presidente en un video
que apareció en esta sección. Uno de los hombres que lo amenazó había
participado en el altercado de fuegos artificiales cerca del Supremo
Tribunal Federal en Brasilia llevados a cabo por los ``300 de Brasil''.
El ataque es otro ejemplo de cómo el vitriolo propagado por la oficina
de odio se extiende cada vez más allá de internet.

\includegraphics{https://static01.graylady3jvrrxbe.onion/images/2020/07/16/autossell/15op-brazil-thumb-print/15op-brazil-thumb-videoSixteenByNineJumbo1600.jpg}

Si hay alguna esperanza de que nuestra joven democracia perdure, debemos
permanecer vigilantes y continuar responsabilizando a este gobierno. Lo
que está en juego no es solo la vida de los brasileños, sino las mismas
instituciones ---el Congreso, el poder judicial, la academia y los
medios de comunicación--- que por el momento han logrado impedir el auge
del autoritarismo.

Patrícia Campos Mello es periodista del diario brasileño Folha de São
Paulo y autora de \emph{A máquina do ódio}, de próxima aparición, sobre
las campañas de desinformación y Bolsonaro. Este ensayo fue traducido
del portugués por Erin Goodman.

Advertisement

\protect\hyperlink{after-bottom}{Continue reading the main story}

\hypertarget{site-index}{%
\subsection{Site Index}\label{site-index}}

\hypertarget{site-information-navigation}{%
\subsection{Site Information
Navigation}\label{site-information-navigation}}

\begin{itemize}
\tightlist
\item
  \href{https://help.nytimes3xbfgragh.onion/hc/en-us/articles/115014792127-Copyright-notice}{©~2020~The
  New York Times Company}
\end{itemize}

\begin{itemize}
\tightlist
\item
  \href{https://www.nytco.com/}{NYTCo}
\item
  \href{https://help.nytimes3xbfgragh.onion/hc/en-us/articles/115015385887-Contact-Us}{Contact
  Us}
\item
  \href{https://www.nytco.com/careers/}{Work with us}
\item
  \href{https://nytmediakit.com/}{Advertise}
\item
  \href{http://www.tbrandstudio.com/}{T Brand Studio}
\item
  \href{https://www.nytimes3xbfgragh.onion/privacy/cookie-policy\#how-do-i-manage-trackers}{Your
  Ad Choices}
\item
  \href{https://www.nytimes3xbfgragh.onion/privacy}{Privacy}
\item
  \href{https://help.nytimes3xbfgragh.onion/hc/en-us/articles/115014893428-Terms-of-service}{Terms
  of Service}
\item
  \href{https://help.nytimes3xbfgragh.onion/hc/en-us/articles/115014893968-Terms-of-sale}{Terms
  of Sale}
\item
  \href{https://spiderbites.nytimes3xbfgragh.onion}{Site Map}
\item
  \href{https://help.nytimes3xbfgragh.onion/hc/en-us}{Help}
\item
  \href{https://www.nytimes3xbfgragh.onion/subscription?campaignId=37WXW}{Subscriptions}
\end{itemize}
