Sections

SEARCH

\protect\hyperlink{site-content}{Skip to
content}\protect\hyperlink{site-index}{Skip to site index}

\href{https://www.nytimes3xbfgragh.onion/es/section/ciencia-y-tecnologia}{Ciencia
y Tecnología}

\href{https://myaccount.nytimes3xbfgragh.onion/auth/login?response_type=cookie\&client_id=vi}{}

\href{https://www.nytimes3xbfgragh.onion/section/todayspaper}{Today's
Paper}

\href{/es/section/ciencia-y-tecnologia}{Ciencia y
Tecnología}\textbar{}Hay dos formas de salir de la panza de una rana y
este escarabajo eligió la más inesperada

\url{https://nyti.ms/2DmRj0i}

\begin{itemize}
\item
\item
\item
\item
\item
\item
\end{itemize}

Advertisement

\protect\hyperlink{after-top}{Continue reading the main story}

Supported by

\protect\hyperlink{after-sponsor}{Continue reading the main story}

Mundo animal

\hypertarget{hay-dos-formas-de-salir-de-la-panza-de-una-rana-y-este-escarabajo-eligiuxf3-la-muxe1s-inesperada}{%
\section{Hay dos formas de salir de la panza de una rana y este
escarabajo eligió la más
inesperada}\label{hay-dos-formas-de-salir-de-la-panza-de-una-rana-y-este-escarabajo-eligiuxf3-la-muxe1s-inesperada}}

Un investigador alimentó unas ranas con escarabajos. La historia parecía
la conocida: cuando el depredador se come a la presa es el fin. Pero
este escarabajo encontró una vía alternativa.

\includegraphics{https://static01.graylady3jvrrxbe.onion/images/2020/08/03/science/04Beetle-ES/merlin_175260999_c4908271-f4de-4d3e-97af-bd0255f15125-articleLarge.jpg?quality=75\&auto=webp\&disable=upscale}

Por Katherine Wu

\begin{itemize}
\item
  4 de agosto de 2020
\item
  \begin{itemize}
  \item
  \item
  \item
  \item
  \item
  \item
  \end{itemize}
\end{itemize}

\href{https://www.nytimes3xbfgragh.onion/2020/08/03/science/beetle-frog-poop.html}{Read
in English}

\href{https://www.nytimes3xbfgragh.onion/newsletters/el-times}{Regístrate
para recibir nuestro boletín} con lo mejor de The New York Times.

\begin{center}\rule{0.5\linewidth}{\linethickness}\end{center}

Es una historia conocida: el depredador caza a sus presa. El depredador
atrapa a la presa. El depredador se traga a la presa.

Suele terminar así. Pero el escarabajo carroñero Regimbartia attenuata
dice: ``No será hoy''. Después de ser tragado por una rana, este pequeño
y valiente insecto puede deslizarse por las entrañas del anfibio y
obligarlo a defecar para así surgir, ligeramente sucio, pero bastante
vivo.

El tránsito del bicho por el tracto digestivo puede durar tan poco como
seis minutos, una mínima fracción de los dos días o más que suele tomar
a una rana hacer completa digestión y defecar la cena, de acuerdo con un
estudio publicado el lunes en
\href{http://dx.doi.org/10.1016/j.cub.2020.06.026}{Current Biology}.

``Este es un comportamiento extrañamente maravilloso del que no había
escucchado antes'', dijo Carla Bardua, una bióloga evolucionista en el
Museo de Historia Natural de Londres que no participó en el estudio.
``Ese pequeño escarabajo puede nadar activamente a través de un sistema
digestivo, es peculiar y sorprendente''.

Shinji Sugiura, biólogo de la Universidad Kobe en Japón, ha estado
catalogando el \href{https://peerj.com/articles/5942/}{comportamiento
extraño de los insectos} y sus depredadores durante años. Algunos
bichos, por ejemplo, punzan a los sapos
\href{https://www.nytimes3xbfgragh.onion/2018/02/06/science/bombardier-beetle-toad-vomit.html}{para
que los vomiten} después de haber sido devorados.

``La morfología y comportamiento de los insectos siempre me inspira'',
dijo por correo electrónico el doctor Sugiura, y agregó que le interesan
particularmente las defensas contra los depredadores que parecen
``inimaginables''.

Luego de notar que los escarabajos Regimbartia y las ranas frecuentan
los mismos arrozales en Japón, Sugiura llevó un ejemplar de cada uno a
su laboratorio, esperando que el insecto fuera escupido. Sin embargo,
salió disparado por el otro lado del tracto digestivo, una hazaña fecal
que Sugiura logró grabar en video.

Ansioso por probar los límites de este comportamiento, el doctor Sugiura
repitió el experimento con cinco especies de ranas que comen insectos en
el laboratorio. Un sorprendente 90 por ciento de los escarabajos
engullidos lograron salir vivos por el otro extremo, todos al cabo de
seis horas de haber sido devorados.

Los escarabajos de otras especies no tuvieron tan buen desempeño y
salieron como cadáveres después de varios días dentro de los anfibios.
Los Regimbartia muertos también demoraron días en salir, lo que sugiere
que sus contrapartes vivos estuvieron trabajando activamente en lograr
el gran escape. Al no poder ver la acción dentro de los intestinos de
las ranas, Sugiura no puede asegurar a ciencia cierta cuál es la
estrategia. Pero cuando inmovilizó las piernas de los escarabajos con
cera, murieron una lenta muerte digestiva.

``Esa fue la prueba irrefutable de que están usando las piernas'', dijo
Nora Moskowitz, que estudia digestión de ranas en la Universidad de
Stanford pero no participó en el estudio. El doctor Sugiura piensa que
los escarabajos Regimbartia pueden usar las piernas para sujetarse y
gatear por las entrañas, que pueden extenderse varias pulgadas, un viaje
arduo para un insecto de cuatro o cinco milímetros de largo. Al llegar
al final de dicho túnel, los insectos tal vez tengan que cosquillear el
esfínter cloacal, el anillo de músculo que funciona como un cierre de
cordón en el trasero de la rana para salir expulsados en una inundación
de heces.

Un viaje a través de este pasadizo tal vez no sea trivial, dijo Aurora
Alvarez-Buylla, investigadora de ranas en la Universidad de Stanford que
no participó en el estudio. Debido a que las ranas tragan enteras a sus
presas, sus jugos gástricos deben ser potentes. ``Estás enfrentando un
ambiente químico y ácido creado para desbaratar y desintegrar cosas'',
dijo.

Pero hasta donde Sugiura ha podido ver, los insectos no se inmutaban
durante su tortuoso viaje a través del tracto. Una vez fuera,
simplemente salieron del estiércol y nadaron felizmente hacia adelante.
Meses después, algunos de los insectos seguían dando vueltas como si el
encuentro traumático jamás hubiera sucedido.

Puede que el exoesqueleto, la resistente carcasa exterior de los
insectos, también ayude. Pero varios viajes a través de la garganta de
una rana podrían terminar por desgastarlos, aseguró Sugiura. Se
necesitan más experimentos para comprender cómo sale todo al final,
dijo.

Las ranas también parecían salir ilesas del encuentro. Según Sugiura,
los desechos de anfibios a menudo van salpicados con las partes duras
del cuerpo de la presa.

``Sin embargo ---dijo---, no quisiera comer este escarabajo si fuera una
rana''.

Katherine J. Wu es reportera del Times, donde cubre ciencia y salud.
Tiene un doctorado en microbiología e inmunobiología por la Universidad
de Harvard. \href{https://twitter.com/KatherineJWu}{@KatherineJWu}

Advertisement

\protect\hyperlink{after-bottom}{Continue reading the main story}

\hypertarget{site-index}{%
\subsection{Site Index}\label{site-index}}

\hypertarget{site-information-navigation}{%
\subsection{Site Information
Navigation}\label{site-information-navigation}}

\begin{itemize}
\tightlist
\item
  \href{https://help.nytimes3xbfgragh.onion/hc/en-us/articles/115014792127-Copyright-notice}{©~2020~The
  New York Times Company}
\end{itemize}

\begin{itemize}
\tightlist
\item
  \href{https://www.nytco.com/}{NYTCo}
\item
  \href{https://help.nytimes3xbfgragh.onion/hc/en-us/articles/115015385887-Contact-Us}{Contact
  Us}
\item
  \href{https://www.nytco.com/careers/}{Work with us}
\item
  \href{https://nytmediakit.com/}{Advertise}
\item
  \href{http://www.tbrandstudio.com/}{T Brand Studio}
\item
  \href{https://www.nytimes3xbfgragh.onion/privacy/cookie-policy\#how-do-i-manage-trackers}{Your
  Ad Choices}
\item
  \href{https://www.nytimes3xbfgragh.onion/privacy}{Privacy}
\item
  \href{https://help.nytimes3xbfgragh.onion/hc/en-us/articles/115014893428-Terms-of-service}{Terms
  of Service}
\item
  \href{https://help.nytimes3xbfgragh.onion/hc/en-us/articles/115014893968-Terms-of-sale}{Terms
  of Sale}
\item
  \href{https://spiderbites.nytimes3xbfgragh.onion}{Site Map}
\item
  \href{https://help.nytimes3xbfgragh.onion/hc/en-us}{Help}
\item
  \href{https://www.nytimes3xbfgragh.onion/subscription?campaignId=37WXW}{Subscriptions}
\end{itemize}
