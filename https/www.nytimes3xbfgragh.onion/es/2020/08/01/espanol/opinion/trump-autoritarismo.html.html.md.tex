Sections

SEARCH

\protect\hyperlink{site-content}{Skip to
content}\protect\hyperlink{site-index}{Skip to site index}

\href{https://www.nytimes3xbfgragh.onion/es/section/opinion}{Opinión}

\href{https://myaccount.nytimes3xbfgragh.onion/auth/login?response_type=cookie\&client_id=vi}{}

\href{https://www.nytimes3xbfgragh.onion/section/todayspaper}{Today's
Paper}

\href{/es/section/opinion}{Opinión}\textbar{}Tentaciones autoritarias:
cómo América Latina nos preparó para Trump

\url{https://nyti.ms/2Dspbsh}

\begin{itemize}
\item
\item
\item
\item
\item
\end{itemize}

Advertisement

\protect\hyperlink{after-top}{Continue reading the main story}

\href{/es/section/opinion}{Opinión}

Supported by

\protect\hyperlink{after-sponsor}{Continue reading the main story}

Comentario

\hypertarget{tentaciones-autoritarias-cuxf3mo-amuxe9rica-latina-nos-preparuxf3-para-trump}{%
\section{Tentaciones autoritarias: cómo América Latina nos preparó para
Trump}\label{tentaciones-autoritarias-cuxf3mo-amuxe9rica-latina-nos-preparuxf3-para-trump}}

La democracia en Estados Unidos está a prueba. Quienes hemos vivido o
trabajado en la región, conocemos bien de mandatarios que juegan con los
límites de su poder. Adiós al ``excepcionalismo estadounidense''.

\includegraphics{https://static01.graylady3jvrrxbe.onion/images/2020/08/01/multimedia/01Ramos-ES/merlin_175175088_ad2b68a7-8076-4175-a205-2f0a1352507f-articleLarge.jpg?quality=75\&auto=webp\&disable=upscale}

\includegraphics{https://static01.graylady3jvrrxbe.onion/images/2019/11/08/opinion/jorge-ramos/jorege-ramos-thumbLarge.png}

Por Jorge Ramos

Es periodista y colaborador regular de opinión de The New York Times.

\begin{itemize}
\item
  1 de agosto de 2020
\item
  \begin{itemize}
  \item
  \item
  \item
  \item
  \item
  \end{itemize}
\end{itemize}

\href{https://www.nytimes3xbfgragh.onion/newsletters/el-times}{Regístrate
para recibir nuestro boletín} con lo mejor de The New York Times.

\begin{center}\rule{0.5\linewidth}{\linethickness}\end{center}

MIAMI --- Para los que hemos vivido o trabajado en América Latina, las
tentaciones autoritarias y los desplantes fotográficos de Donald Trump,
de pronto, se ven familiares. De hecho, los periodistas latinoamericanos
estamos bien entrenados para lidiar con alguien como el actual
presidente de Estados Unidos. Nos ha tocado ver una larga lista de
líderes que abusan de su poder y utilizan a los soldados para su propio
beneficio.

La democracia en Estados Unidos está a prueba. El presidente se preguntó
en
\href{https://twitter.com/realDonaldTrump/status/1288818160389558273}{un
tuit} si se deberían retrasar las elecciones presidenciales de noviembre
por un supuesto fraude en la votación por correo. Por principio, no hay
ningún fraude y Trump no puede tomar
\href{https://www.nytimes3xbfgragh.onion/es/2020/07/30/espanol/estados-unidos/trump-retrasar-elecciones.html}{una
responsabilidad que es del Congreso}. Trump va perdiendo en todas las
encuestas y retrasar las elecciones significaría que él se quedaría más
tiempo del estipulado en la presidencia, como muchos líderes
autoritarios han hecho en el pasado en América Latina.

Además de la preocupación de que extienda su permanencia en el poder,
inquieta el envío por parte de su gobierno de agentes federales a
Portland, Oregón, para contrarrestar las protestas de los últimos dos
meses. La mayoría de los
\href{https://www.nytimes3xbfgragh.onion/2020/07/17/us/portland-protests.html}{2000
agentes movilizados} forma parte de un grupo élite de la Patrulla
Fronteriza (CBP, por sus siglas en inglés). Pero líderes locales creen
que su presencia es contraproducente y solo aumenta las tensiones con
los manifestantes que reclaman, precisamente, el abuso policial y la
desigualdad racial. ``Esto es un ataque a nuestra democracia'', dijo el
alcalde de Portland, Ted Wheeler.

La
\href{http://opb-imgserve-production.s3-website-us-west-2.amazonaws.com/original/ag_rosenblum_xxxx_updated_complaint_1595086491349.pdf}{demanda
presentada} por la procuradora general de Oregón contra el Departamento
de Seguridad Interna, el Servicio de Alguaciles, el Servicio de
Protección Federal y la Patrulla Fronteriza describe imágenes que me
recuerdan las prácticas más tenebrosas de los sistemas totalitarios en
América Latina. El documento dice que agentes federales ``han usado
vehículos sin identificar para circular por el centro de Portland, han
detenido a manifestantes y los han puesto en vehículos sin identificar,
sacándolos de lugares públicos sin arrestarlos o establecer una razón
para su detención''.

Este tipo de abuso contra civiles lo había escuchado de agentes de la
seguridad del Estado en
\href{https://www.nytimes3xbfgragh.onion/es/2020/07/26/espanol/opinion/nicmer-evans-venezuela.html}{Venezuela},
\href{https://www.nytimes3xbfgragh.onion/es/2019/02/18/espanol/opinion/nicaragua-prensa-chamorro.html}{Nicaragua}
y
\href{https://www.nytimes3xbfgragh.onion/es/2019/05/31/espanol/opinion/cuba-jovenes-revolucion.html}{Cuba},
pero no de operativos en Estados Unidos.

A menos de cien días de las elecciones presidenciales, Trump ha
amenazado con enviar a agentes federales a otras ciudades, como
\href{https://www.santafenewmexican.com/news/local_news/trump-announces-deployment-of-federal-agents-to-albuquerque/article_e80a12d6-cc34-11ea-9ab0-5b1cd8827f75.html}{Albuquerque}
y
\href{https://www.nytimes3xbfgragh.onion/2020/07/23/us/politics/trump-chicago-federal-agents.html}{Chicago},
que tienen alcaldes del Partido Demócrata y que, de acuerdo al
presidente, enfrentan problemas de criminalidad. No es ningún secreto
que, detrás de su mensaje de ``ley y orden'', está su explícito deseo de
reelegirse. Son votos a través del uso de la fuerza.

Esto no es nuevo. En junio, días después que se reveló que Trump fue
llevado a un
\href{https://www.nytimes3xbfgragh.onion/2020/05/31/us/politics/trump-protests-george-floyd.html}{búnker
de la Casa Blanca}, miembros de la Guardia Nacional y de la policía
dispersaron con balas de goma y gases irritantes a cientos de
manifestantes pacíficos de la plaza Lafayette. Y todo para que el
presidente pudiera cruzar el parque y
\href{https://www.nytimes3xbfgragh.onion/es/2020/06/03/espanol/mundo/trump-foto-iglesia-protestas.html}{tomarse
una fotografía} con la biblia en la mano frente a la iglesia de St.
John.

El general Mark Milley, el militar de más alto rango en el país y jefe
del Estado Mayor Conjunto, reconoció en un inusual discurso que se
equivocó al acompañar al presidente Trump en esa caminata. ``No debí
haber estado ahí'',
\href{https://www.nytimes3xbfgragh.onion/2020/06/11/us/politics/trump-milley-military-protests-lafayette-square.html}{dijo}
en un video, ``mi presencia {[}\ldots{}{]} creó la percepción que los
militares están involucrados en política doméstica''.

Sacar al ejército para que actúe como policía dentro de Estados Unidos
no es común. Hay que remontarse a una ley de 1807, llamada The
Insurrection Act. Y hasta el mismo secretario de Defensa, Mark Esper,
contradiciendo al presidente,
\href{https://www.nytimes3xbfgragh.onion/video/us/politics/100000007172076/esper-trump-protests-troops.html?playlistId=video/latest-video}{dijo}
que esa opción militar solo debe utilizarse ``como último recurso'' y
que ``no estamos en esa situación ahora mismo''.

A pesar de eso, 1600 soldados en activo de Fort Bragg en Carolina del
Norte y Fort Drum de Nueva York fueron enviados a las afueras de
Washington D.C.,
\href{https://www.nytimes3xbfgragh.onion/2020/06/04/us/politics/trump-troops-washington-pentagon.html}{según
reportó The New York Times}. Ellos, finalmente, nunca fueron utilizados
para controlar las manifestaciones. Pero unos
\href{https://www.nytimes3xbfgragh.onion/2020/06/07/us/politics/trump-military-troops-protests.html}{5000
miembros de la Guardia Nacional} sí llegaron de varios estados a
proteger la capital.

Todo esto generó un enorme malestar. ``Tenemos a los militares para
pelear contra nuestros enemigos'',
\href{https://www.nytimes3xbfgragh.onion/2020/06/07/us/politics/trump-military-troops-protests.html}{dijo}
el almirante retirado Mike Mullen en una entrevista, ``no para pelear
con nuestra propia gente''.

Lo que hizo Trump es muy inusual y destruye cualquier vestigio del
``excepcionalismo estadounidense''. En cambio, lo hemos visto antes en
América Latina. Algunos mandatarios de la región han sacado a soldados y
a agentes federales a las calles para imponer su voluntad y atacar a sus
propios ciudadanos u opositores. Y los resultados han sido desastrosos.

El dictador de Venezuela, Nicolás Maduro, utiliza a sus militares para
\href{https://www.nytimes3xbfgragh.onion/es/2019/07/05/espanol/america-latina/venezuela-faes-derechos-humanos.html}{matar,
reprimir y mantenerse en el poder}. Amnistía Internacional
\href{https://www.amnesty.org/es/countries/americas/venezuela/report-venezuela/}{denunció}
que el año pasado ``la policía y el ejército continuaron haciendo uso de
fuerza excesiva y, en algunos casos, intencionadamente letal contra
manifestantes''. Mucho antes, durante la presidencia de Carlos Andrés
Pérez, los militares venezolanos fueron responsables de
\href{https://www.bbc.com/mundo/noticias-america-latina-47379668}{al
menos 276 muertes}, según cifras oficiales, en el llamado Caracazo.

Las dictaduras militares en Argentina y Chile fueron particularmente
violentas y crueles con los civiles opositores en las décadas de los
setenta y ochenta. En México, el ejército asesinó a decenas y quizás
cientos de estudiantes en la
\href{https://www.nytimes3xbfgragh.onion/es/2018/09/30/espanol/opinion/opinion-enrique-krauze-tlatelolco-68.html}{masacre
de Tlatelolco en 1968}. Y, en Guatemala, la Comisión para el
Esclarecimiento Histórico concluyó que los militares fueron responsables
del
\href{http://www.centrodememoriahistorica.gov.co/descargas/guatemala-memoria-silencio/guatemala-memoria-del-silencio.pdf}{85
por ciento} de las violaciones a los derechos humanos y hechos de
violencia entre 1962 y 1996. A pesar de que la gran mayoría de los
países latinoamericanos son hoy democracias funcionales, hay una larga y
triste historia de militares utilizados por razones ideológicas o
partidistas.

La tan criticada decisión de Trump de enviar a agentes federales a otras
ciudades y su tuitera idea de retrasar las elecciones presentan ahora un
serio desafío para la democracia estadounidense. Pero para que la nación
no caiga en esa ``predisposición fundamental'' para ``limitar la
libertad individual'', como lo describe la profesora Karen Stenner en su
libro
\href{https://www.cambridge.org/core/books/authoritarian-dynamic/7620B99124ED2DBFC6394444838F455A}{\emph{The
Authoritarian Dynamic}}, es preciso una prensa vigilante, una mayoría
bien informada y sin prejuicios, un ejército profesional y apartidista y
la absoluta independencia del Congreso y la Corte Suprema de Justicia.

Al final, estoy convencido, Estados Unidos sobrevivirá las tentaciones
autoritarias de Trump. Es, quizás, mi optimismo de inmigrante. Este
todavía es un país mucho más fuerte que cualquier individuo con falsos
sueños de grandeza.

Jorge Ramos es periodista, conductor de los programas \emph{Noticiero
Univisión} y \emph{Al punto,} y autor del libro \emph{Stranger: El
desafío de un inmigrante latino en la era de Trump}.
\href{https://twitter.com/jorgeramosnews}{@jorgeramosnews}

Advertisement

\protect\hyperlink{after-bottom}{Continue reading the main story}

\hypertarget{site-index}{%
\subsection{Site Index}\label{site-index}}

\hypertarget{site-information-navigation}{%
\subsection{Site Information
Navigation}\label{site-information-navigation}}

\begin{itemize}
\tightlist
\item
  \href{https://help.nytimes3xbfgragh.onion/hc/en-us/articles/115014792127-Copyright-notice}{©~2020~The
  New York Times Company}
\end{itemize}

\begin{itemize}
\tightlist
\item
  \href{https://www.nytco.com/}{NYTCo}
\item
  \href{https://help.nytimes3xbfgragh.onion/hc/en-us/articles/115015385887-Contact-Us}{Contact
  Us}
\item
  \href{https://www.nytco.com/careers/}{Work with us}
\item
  \href{https://nytmediakit.com/}{Advertise}
\item
  \href{http://www.tbrandstudio.com/}{T Brand Studio}
\item
  \href{https://www.nytimes3xbfgragh.onion/privacy/cookie-policy\#how-do-i-manage-trackers}{Your
  Ad Choices}
\item
  \href{https://www.nytimes3xbfgragh.onion/privacy}{Privacy}
\item
  \href{https://help.nytimes3xbfgragh.onion/hc/en-us/articles/115014893428-Terms-of-service}{Terms
  of Service}
\item
  \href{https://help.nytimes3xbfgragh.onion/hc/en-us/articles/115014893968-Terms-of-sale}{Terms
  of Sale}
\item
  \href{https://spiderbites.nytimes3xbfgragh.onion}{Site Map}
\item
  \href{https://help.nytimes3xbfgragh.onion/hc/en-us}{Help}
\item
  \href{https://www.nytimes3xbfgragh.onion/subscription?campaignId=37WXW}{Subscriptions}
\end{itemize}
