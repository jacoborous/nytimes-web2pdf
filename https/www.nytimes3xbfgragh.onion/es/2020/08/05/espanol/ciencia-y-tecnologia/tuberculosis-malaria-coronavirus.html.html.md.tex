\href{/es/section/ciencia-y-tecnologia}{Ciencia y
Tecnología}\textbar{}Se está propagando `el monstruo más grande de
todos'. Y no es el coronavirus

\url{https://nyti.ms/3ioZTuH}

\begin{itemize}
\item
\item
\item
\item
\item
\item
\end{itemize}

\hypertarget{el-brote-de-coronavirus}{%
\subsubsection{\texorpdfstring{\href{https://www.nytimes3xbfgragh.onion/es/spotlight/coronavirus?name=styln-coronavirus-es\&region=TOP_BANNER\&variant=undefined\&block=storyline_menu_recirc\&action=click\&pgtype=Article\&impression_id=174ec720-e3a1-11ea-88c0-59664f00d216}{El
brote de
coronavirus}}{El brote de coronavirus}}\label{el-brote-de-coronavirus}}

\begin{itemize}
\tightlist
\item
  \href{https://www.nytimes3xbfgragh.onion/es/interactive/2020/espanol/america-latina/coronavirus-en-mexico.html?name=styln-coronavirus-es\&region=TOP_BANNER\&variant=undefined\&block=storyline_menu_recirc\&action=click\&pgtype=Article\&impression_id=174eee30-e3a1-11ea-88c0-59664f00d216}{Mapa
  y casos en México}
\item
  \href{https://www.nytimes3xbfgragh.onion/es/interactive/2020/08/06/espanol/ciencia-y-tecnologia/tengo-covid-19-sintomas.html?name=styln-coronavirus-es\&region=TOP_BANNER\&variant=undefined\&block=storyline_menu_recirc\&action=click\&pgtype=Article\&impression_id=174eee31-e3a1-11ea-88c0-59664f00d216}{Identifica
  los síntomas}
\item
  \href{https://www.nytimes3xbfgragh.onion/es/interactive/2020/science/coronavirus-tratamientos-curas.html?name=styln-coronavirus-es\&region=TOP_BANNER\&variant=undefined\&block=storyline_menu_recirc\&action=click\&pgtype=Article\&impression_id=174f1540-e3a1-11ea-88c0-59664f00d216}{Fármacos
  y tratamientos}
\item
  \href{https://www.nytimes3xbfgragh.onion/es/2020/04/29/espanol/estilos-de-vida/oximetro-para-que-sirve.html?name=styln-coronavirus-es\&region=TOP_BANNER\&variant=undefined\&block=storyline_menu_recirc\&action=click\&pgtype=Article\&impression_id=174f1541-e3a1-11ea-88c0-59664f00d216}{¿Necesitas
  un oxímetro?}
\item
  \href{https://www.nytimes3xbfgragh.onion/es/2020/07/02/espanol/ciencia-y-tecnologia/sobrevivientes-coronavirus-recuperacion.html?name=styln-coronavirus-es\&region=TOP_BANNER\&variant=undefined\&block=storyline_menu_recirc\&action=click\&pgtype=Article\&impression_id=174f1542-e3a1-11ea-88c0-59664f00d216}{Las
  secuelas del virus}
\end{itemize}

\includegraphics{https://static01.graylady3jvrrxbe.onion/images/2020/08/04/science/04SCI-VIRUS-GLOBAL-ES-00/merlin_174091377_bd123d23-7d76-4af5-b50f-3ea3a28ecc3f-articleLarge.jpg?quality=75\&auto=webp\&disable=upscale}

Sections

\protect\hyperlink{site-content}{Skip to
content}\protect\hyperlink{site-index}{Skip to site index}

Salud Global

\hypertarget{se-estuxe1-propagando-el-monstruo-muxe1s-grande-de-todos-y-no-es-el-coronavirus}{%
\section{Se está propagando `el monstruo más grande de todos'. Y no es
el
coronavirus}\label{se-estuxe1-propagando-el-monstruo-muxe1s-grande-de-todos-y-no-es-el-coronavirus}}

La tuberculosis mata a 1,5 millones de personas cada año. Los
confinamientos y las interrupciones de la cadena de suministro de
medicamentos amenazan el progreso en la batalla contra esta enfermedad,
el VIH y el paludismo.

El doctor Giorgio Franyuti generalmente se encuentra en las selvas
remotas de México diagnosticando tuberculosis. Pero desde la pandemia,
ha trabajado en un hospital improvisado tratando a pacientes con
coronavirus en Ciudad de México.Credit...Meghan Dhaliwal para The New
York Times

Supported by

\protect\hyperlink{after-sponsor}{Continue reading the main story}

Por
\href{https://www.nytimes3xbfgragh.onion/by/apoorva-mandavilli}{Apoorva
Mandavilli}

\begin{itemize}
\item
  Publicado 5 de agosto de 2020Actualizado 10 de agosto de 2020
\item
  \begin{itemize}
  \item
  \item
  \item
  \item
  \item
  \item
  \end{itemize}
\end{itemize}

\href{https://www.nytimes3xbfgragh.onion/2020/08/03/health/coronavirus-tuberculosis-aids-malaria.html}{Read
in English}

\href{https://www.nytimes3xbfgragh.onion/newsletters/el-times}{Regístrate
para recibir nuestro boletín} con lo mejor de The New York Times.

\begin{center}\rule{0.5\linewidth}{\linethickness}\end{center}

Comienza con una fiebre ligera y malestar general; después, una tos
dolorosa y dificultad para respirar. Las multitudes favorecen el
contagio y lo propagan en las personas cercanas. Contener un brote
requiere de rastreo de contactos, así como aislamiento y tratamiento de
la enfermedad durante semanas o meses.

Esta enfermedad traicionera ha llegado a todos los rincones del planeta.
Es la tuberculosis, la enfermedad infecciosa más mortal del mundo, la
cual cobra la vida de 1,5 millones de personas anualmente.

Hasta este año, la tuberculosis y sus aliados mortales, el VIH y el
paludismo, estaban ausentes. La cantidad total de víctimas de cada
enfermedad a lo largo de la década anterior estuvo en su punto más bajo
en 2018, el último año del que se tienen registros disponibles.

Pero ahora, a medida que la pandemia del coronavirus se propaga por el
mundo, consumiendo los recursos mundiales en materia de salud, estos
adversarios continuamente olvidados están de regreso.

``La COVID-19 amenaza con arruinar todos nuestros esfuerzos y
devolvernos al punto en el que estábamos hace 20 años'', dijo Pedro L.
Alonso, director del Programa Mundial sobre Paludismo de la Organización
Mundial de la Salud.

No es solo que el coronavirus ha provocado que los científicos desvíen
su atención de la tuberculosis, el VIH y el paludismo. Los
confinamientos, en especial en partes de África, Asia y América Latina,
han levantado barreras infranqueables para pacientes que deben viajar a
fin de conseguir diagnósticos o medicamentos, de acuerdo con entrevistas
con más de una veintena de funcionarios de salud pública, médicos y
pacientes de todo el mundo.

El temor al coronavirus y el cierre de las clínicas han mantenido
alejados a muchos pacientes que luchan contra el VIH, la tuberculosis y
el paludismo, mientras que las restricciones a los viajes por aire o por
mar han limitado gravemente la entrega de medicamentos en las regiones
más afectadas.

Aproximadamente el 80 por ciento de los programas para atender la
tuberculosis, el VIH y el paludismo en todo el mundo han
\href{https://www.theglobalfund.org/en/covid-19/news/2020-06-17-global-fund-survey-majority-of-hiv-tb-and-malaria-programs-face-disruptions-as-a-result-of-covid-19/}{reportado
interrupciones} en los servicios y una de cada cuatro personas que viven
con VIH ha reportado problemas para acceder a medicamentos, de acuerdo
con ONUSIDA. Las interrupciones o retrasos en el tratamiento podrían
provocar resistencia a los medicamentos, algo que ya representa un gran
problema en muchos países.

\includegraphics{https://static01.graylady3jvrrxbe.onion/images/2020/07/23/science/04SCI-VIRUS-GLOBAL-ES-01/merlin_172023063_d8d1e991-50c4-4988-a828-f4be5fb5b26e-articleLarge.jpg?quality=75\&auto=webp\&disable=upscale}

En India, donde se encuentra aproximadamente el
\href{https://www.who.int/tb/publications/global_report/gtbr2018_main_text_28Feb2019.pdf}{27
por ciento} de los casos de tuberculosis del mundo, los diagnósticos han
caído \href{https://reports.nikshay.in/Reports/TBNotification}{casi un
75 por ciento} desde el inicio de la pandemia. En Rusia, las clínicas de
VIH
\href{https://www.unaids.org/en/resources/presscentre/featurestories/2020/may/20200514_russian-federation-covid19}{han
sido transformadas para hacer pruebas de coronavirus}.

La temporada de paludismo ha comenzado en África, donde ocurre el 90 por
ciento de los fallecimientos en el mundo a causa de esta enfermedad,
pero las estrategias habituales de prevención (distribución de
mosquiteros tratados con insecticida y aplicación de pesticidas en
aerosol)
\href{https://www.who.int/emergencies/diseases/novel-coronavirus-2019/question-and-answers-hub/q-a-detail/malaria-and-the-covid-19-pandemic}{han
sido restringidas} a causa de los cierres.

De acuerdo con un
\href{http://www.stoptb.org/assets/documents/news/Modeling\%20Report_1\%20May\%202020_FINAL.pdf}{cálculo},
un cierre de tres meses en distintas partes del mundo y un retorno
gradual a la normalidad a lo largo de diez meses podría tener como
consecuencia un aumento de 6,3 millones de casos de tuberculosis y 1,4
millones de fallecimientos a causa de esta enfermedad.

Una interrupción de seis meses de la terapia antirretroviral podría
derivar en más de
\href{https://www.who.int/es/news-room/detail/11-05-2020-the-cost-of-inaction-covid-19-related-service-disruptions-could-cause-hundreds-of-thousands-of-extra-deaths-from-hiv}{500.000
fallecimientos adicionales} por enfermedades relacionadas con el VIH, de
acuerdo con la OMS. Otro modelo de la OMS pronosticó que, en el peor de
los casos, los fallecimientos a causa del paludismo
\href{https://www.who.int/publications/i/item/the-potential-impact-of-health-service-disruptions-on-the-burden-of-malaria}{podrían
duplicarse a 770.000} por año.

Varios expertos en salud pública, algunos al borde del llanto,
advirtieron que, de continuar las tendencias actuales, el coronavirus
podría retrasar varios años, o incluso décadas, el esmerado progreso en
contra de la tuberculosis, el VIH y el paludismo.

El Fondo Mundial, una sociedad pública y privada para el combate de
estas enfermedades, calcula que mitigar este daño
\href{https://www.theglobalfund.org/en/news/2020-06-24-global-fund-covid-19-report-deaths-from-hiv-tb-and-malaria-could-almost-double-in-12-months-unless-urgent-action-is-taken/}{requerirá
al menos 28.500 millones} de dólares, una cantidad que es poco probable
que se materialice.

\hypertarget{retrasos-en-el-diagnuxf3stico}{%
\subsection{Retrasos en el
diagnóstico}\label{retrasos-en-el-diagnuxf3stico}}

Image

Con la mayoría de las clínicas privadas cerradas, los pacientes con VIH,
tuberculosis y paludismo tienen pocos lugares a donde ir para el tipo de
atención médica que se ofrece en esta clínica de Médicos sin Fronteras,
en Nairobi.Credit...Brian Inganga/Associated Press

Si analizamos la historia, el impacto del coronavirus en los pobres será
visible mucho tiempo después de que termine de la pandemia. Por ejemplo,
la crisis socioeconómica en Europa del Este a principios de la década de
1990 derivó en los índices más elevados del mundo de un tipo de
tuberculosis que era resistente a muchos medicamentos, una distinción
dudosa que la región sigue teniendo actualmente.

El punto de inicio de esta terrible cadena de sucesos es la falta de
diagnósticos: mientras más prolongado sea el periodo que una persona
vive sin un diagnóstico, y más tarde el inicio del tratamiento, hay
mayores probabilidades de que la enfermedad infecciosa se propague,
enferme a otras personas y les provoque la muerte.

Para el paludismo, una breve demora en el diagnóstico puede volverse
rápidamente fatal, a veces tras solo 36 horas de una fiebre aguda. ``Es
una de esas enfermedades donde no podemos permitirnos esperar'', dijo
Alonso.

Aprensiva por el aumento del paludismo en África occidental, la OMS
ahora considera administrar medicamentos antipalúdicos a poblaciones
enteras, una estrategia de último recurso utilizada durante la epidemia
de ébola en África occidental y la insurgencia de Boko Haram.

En toda la África subsahariana, cada vez menos mujeres acuden a las
clínicas para el diagnóstico del VIH. Una interrupción de seis meses en
el acceso a medicamentos que evitan que las mujeres con VIH positivo que
están embarazadas transmitan la infección a sus bebés en el útero,
podría
\href{https://reliefweb.int/report/world/estimation-potential-impact-covid-19-responses-hiv-epidemic-analysis-using-goals-model}{incrementar
las infecciones de VIH en los niños} hasta en un 139 por ciento en
Uganda y 162 por ciento en Malaui, según ONUSIDA.

La disminución de la capacidad de diagnóstico puede tener el mayor
efecto sobre la tuberculosis, lo que puede conducir a graves
consecuencias para los hogares porque, como el coronavirus, la bacteria
se propaga de manera más eficiente en ambientes cerrados y entre las
personas en contacto cercano.

Cada persona con tuberculosis puede transmitir la enfermedad
\href{https://www.who.int/es/news-room/fact-sheets/detail/tuberculosis}{a
otras 15 personas durante un año}, lo que aumenta drásticamente la
posibilidad de que las personas se infecten en espacios cerrados y lo
propaguen entre sus comunidades una vez que finalicen los
confinamientos. La perspectiva es especialmente preocupante en lugares
densamente poblados y con altas tasas de tuberculosis, como las favelas
de Río de Janeiro o los barrios marginales de Sudáfrica.

``A mayor cantidad de casos sin diagnóstico ni tratamiento, mayores
casos habrá el año siguiente y el posterior'', señaló Lucica Ditiu,
quien dirige la Alianza Stop TB, un consorcio internacional de 1700
grupos que luchan contra la enfermedad.

La infraestructura construida para diagnosticar el VIH y la tuberculosis
ha sido una ayuda para muchos países que están combatiendo el
coronavirus. GeneXpert, la herramienta utilizada para detectar material
genético de las bacterias de la tuberculosis y del VIH, también puede
amplificar el ácido ribonucleico (ARN) para diagnosticar el coronavirus.

No obstante, ahora muchas clínicas están usando los aparatos únicamente
para detectar el coronavirus. Poner el coronavirus como prioridad sobre
la tuberculosis es ``muy tonto desde el punto de vista de la salud
pública'', dijo Ditiu. ``En realidad debes ser listo y detectar ambos''.

En un país tras otro, la pandemia ha dado lugar a un
\href{https://www.medrxiv.org/content/10.1101/2020.04.28.20079582v1}{fuerte
descenso de los diagnósticos} de tuberculosis: una reducción del 70 por
ciento en Indonesia, del 50 por ciento en Mozambique y
\href{https://www.nicd.ac.za/wp-content/uploads/2020/05/Impact-of-Covid-19-interventions-on-TB-testing-in-South-Africa-10-May-2020.pdf}{Sudáfrica},
y el 20 por ciento en China, según la OMS.

Image

Giorgio Franyuti dijo que muchos pacientes con tuberculosis en un
hospital improvisado en Ciudad de México estaban siendo diagnosticados
erróneamente con la COVID-19.Credit...Meghan Dhaliwal para The New York
Times

A finales de mayo en México, mientras las infecciones por coronavirus
aumentaban, los diagnósticos de tuberculosis registrados por el gobierno
\href{https://www.gob.mx/salud/acciones-y-programas/direccion-general-de-epidemiologia-boletin-epidemiologico}{cayeron
a 263 casos} de los 1097 registrados en la misma semana del año pasado.

Giorgio Franyuti, director ejecutivo de Medical Impact, una organización
no gubernamental con sede en México, normalmente trabaja en las selvas
remotas del país, donde diagnostica y trata la tuberculosis en el pueblo
lacandón. Incapaz de viajar allí durante la pandemia, ha trabajado en un
hospital militar improvisado que trata a pacientes de la COVID-19 en
Ciudad de México.

\emph{\emph{\emph{{[}}\href{https://www.nytimes3xbfgragh.onion/es/interactive/2020/espanol/america-latina/coronavirus-en-mexico.html?action=click\&module=RelatedLinks\&pgtype=Article}{\emph{Coronavirus
en México: mapa de casos y fallecimientos}}}{]}}**

Allí, ha visto a nueve pacientes con una tos llena de expectoración
---característica de la tuberculosis--- que comenzó meses antes pero que
se suponía que tenían la COVID-19. Los pacientes contrajeron el
coronavirus en el hospital y se enfermaron gravemente. Al menos cuatro
han muerto.

``Nadie está haciendo pruebas de tuberculosis en ninguna institución'',
dijo. ``La mente de los médicos en México, así como la de quienes toman
las decisiones, está fijada en la COVID-19''.

``La tuberculosis es el monstruo más grande de todos. Si hablamos de
muertes y pandemias'', dijo, la COVID aún no se compara con los ``diez
millones de casos al año'' de la tuberculosis.

India entró en confinamiento el 24 de marzo, y el gobierno ordenó a los
hospitales públicos que se concentrasen en la COVID-19. Muchos
hospitales
\href{https://timesofindia.indiatimes.com/india/how-covid-war-is-hurting-indias-non-covid-patients/articleshow/74949121.cms}{cerraron
los servicios ambulatorios} para otras enfermedades.

El impacto en los diagnósticos de tuberculosis fue inmediato: el
\href{https://reports.nikshay.in/Reports/TBNotification}{número de casos
nuevos} registrados por el gobierno indio entre el 25 de marzo y el 19
de junio fue de 60.486, en comparación con 179.792 durante el mismo
periodo en 2019.

La pandemia también está reduciendo el suministro de pruebas de
diagnóstico para estas enfermedades asesinas, conforme las empresas
recurren a la fabricación de pruebas más costosas para detectar el
coronavirus. Cepheid, el fabricante de pruebas diagnósticas para la
tuberculosis con sede en California, ha pasado a hacer pruebas de
coronavirus. Las empresas que hacen pruebas de diagnóstico para el
paludismo hacen lo mismo, de acuerdo con Catharina Boehme, directora
ejecutiva de la Fundación para Nuevos Diagnósticos Innovadores.

Las pruebas de coronavirus son mucho más lucrativas, pues tienen un
costo de unos 10 dólares, en comparación con los 18 centavos de una
prueba rápida de paludismo.

Estas empresas ``tienen una gran demanda de pruebas para la COVID-19 en
este momento'', afirmó Madhukar Pai, director del Centro Internacional
McGill para la Tuberculosis en Montreal. ``No puedo imaginar que las
enfermedades de la pobreza reciban atención en este espacio''.

\hypertarget{tratamiento-interrumpido}{%
\subsection{Tratamiento interrumpido}\label{tratamiento-interrumpido}}

Image

Durante el confinamiento de Nairobi, Thomas Wuoto tuvo que pedir
prestado los medicamentos para el VIH de su esposa y estuvo sin ninguno
durante 10 días, lo que lo pone en riesgo de desarrollar resistencia a
las medicinas.Credit...Khadija Farah para The New York Times

La pandemia ha obstaculizado la disponibilidad de medicamentos para el
VIH, la tuberculosis y el paludismo en todo el mundo al interrumpir las
cadenas de suministro, desviar la capacidad de fabricación e imponer
barreras físicas para los pacientes que deben viajar a clínicas
distantes para recoger los medicamentos.

Esta escasez obliga a algunos pacientes a racionar sus medicamentos, lo
que pone en peligro su salud. En Indonesia, la política oficial es
proporcionar un mes de suministro de medicamentos a la vez a los
pacientes con VIH, pero últimamente ha sido difícil conseguir la terapia
antirretroviral fuera de la capital, Yakarta.

Incluso en la ciudad, algunas personas están ampliando el suministro de
un mes a dos, dijo ``Davi'' Sepi Maulana Ardiansyah, activista del grupo
Inti Muda.

El propio Ardiansyah lo ha hecho, aunque sabe que ha puesto en riesgo su
bienestar. ``Esta pandemia y esta falta de disponibilidad de
medicamentos está afectando mucho nuestra salud mental y también nuestra
salud'', dijo.

Durante el encierro en Nairobi, Thomas Wuoto, quien tiene VIH, tomó
prestados medicamentos antirretrovirales de su esposa, quien también
está infectada. Como educador voluntario sobre VIH, Wuoto sabía muy bien
que estaba arriesgándose a desarrollar resistencia a los medicamentos al
mezclar u omitir dosis. Cuando finalmente llegó al Hospital del Condado
de Mbagathi, había pasado diez días sin sus medicinas para el VIH, la
primera vez desde 2002 que había perdido su terapia.

Las personas con VIH y tuberculosis que suspenden el tratamiento tienen
más probabilidades de enfermarse a corto plazo. A largo plazo, hay una
consecuencia aún más preocupante: un aumento de las formas de
resistencia a los medicamentos de estas enfermedades. La tuberculosis
que ya es resistente a los medicamentos es una amenaza tan grande que se
vigila a los pacientes muy de cerca durante el tratamiento, una práctica
que en su mayoría ha sido suspendida durante la crisis del coronavirus.

De acuerdo con la OMS, al menos 121 países han notificado una reducción
en la cantidad de pacientes con tuberculosis que acuden a las clínicas
desde que comenzó la pandemia, lo que pone en peligro los logros
alcanzados con tanto esfuerzo.

``Esto es realmente difícil de procesar'', dijo Ditiu. ``Se requirió de
mucho trabajo para llegar a donde estamos. No estábamos en la cima de la
montaña, pero estábamos lejos del pie, entonces vino una avalancha y nos
lanzó de nuevo hasta abajo''.

En muchos lugares, los cierres se impusieron con tal rapidez que las
existencias de medicamentos se agotaron rápidamente. México ya tenía
medicinas expiradas en su suministro, pero el problema se ha
\href{https://www.reuters.com/article/us-health-coronavirus-lgbt-aids/no-medicine-no-food-coronavirus-restrictions-amplify-health-risks-to-lgbt-people-with-hiv-idUSKBN22W28G}{exacerbado
por la pandemia}, según Franyuti.

En Brasil, los medicamentos contra el VIH y la tuberculosis son
comprados y distribuidos por el Ministerio de Salud. Pero el coronavirus
está arrasando el país, y la distribución de estos tratamientos se ha
vuelto cada vez más difícil a medida que los trabajadores de la salud
intentan hacer frente a las consecuencias de la pandemia.

``Es un gran desafío logístico lograr que los municipios tengan mayores
existencias para que puedan abastecer'', dijo Betina Durovni, científica
principal de la Fiocruz, un instituto de investigación en Brasil.

Image

Algunos pacientes con coronavirus en Tabatinga, Brasil, son
transportados por vía aérea a Manaos, a aproximadamente 1600 kilómetros
de distancia, para recibir tratamiento, pero muchos se están
quedando.Credit...Bruno Kelly/Reuters

Incluso si, con un poco de ayuda de los grandes organismos de asistencia
humanitaria, los gobiernos estuvieran preparados para comprar
medicamentos con meses de antelación, el suministro mundial podría
agotarse pronto.

La pandemia ha restringido severamente el transporte internacional, lo
que dificulta la disponibilidad no solo de ingredientes químicos y
materias primas, sino también de suministros de embalaje.

``La interrupción de las cadenas de suministro es algo que realmente me
preocupa en el caso del VIH, la tuberculosis y el paludismo'', dijo
Carlos del Rio, presidente del consejo científico asesor del Plan
Presidencial de Emergencia para el Alivio del SIDA de Estados Unidos.

La exageración acerca de la cloroquina como posible tratamiento para el
coronavirus ha llevado al acaparamiento del medicamento en algunos
países como Birmania y ha agotado sus reservas mundiales.

Más del 80 por ciento del suministro global de medicamentos
antirretrovirales proviene de solo ocho empresas indias. Solo el costo
de estos
\href{https://www.unaids.org/en/resources/presscentre/pressreleaseandstatementarchive/2020/june/20200622_availability-and-cost-of-antiretroviral-medicines}{podría
aumentar en 225 millones de dólares} al año debido a la escasez de
suministro y mano de obra, interrupciones del transporte y fluctuaciones
monetarias, según ONUSIDA.

También existe un riesgo real de que las empresas indias se vuelquen a
medicamentos más rentables o no puedan satisfacer la demanda mundial
porque los trabajadores migrantes han abandonado las ciudades a medida
que se propaga el virus.

El gobierno indio puede incluso
decidir\href{https://theprint.in/health/india-could-ban-export-of-anti-tb-drugs-as-lockdown-hits-production/406119/}{no
exportar medicamentos contra la tuberculosis}, y así guardar suministros
para sus propios ciudadanos.

``Dependemos mucho de unos cuantos desarrolladores o fabricantes clave
para todos los medicamentos del mundo, y eso debe diversificarse'',
señaló Meg Doherty, quien dirige programas de VIH en la OMS. ``Si
hubiera más depósitos de medicamentos desarrollados localmente o
fabricantes farmacéuticos, estarían más cerca de donde se necesitan''.

Las organizaciones de asistencia humanitaria y los gobiernos tratan de
mitigar algunos de los daños mediante la extensión de suministros y el
almacenamiento de medicamentos. En junio, la OMS modificó
\href{https://www.who.int/news-room/detail/15-06-2020-who-urges-countries-to-enable-access-to-fully-oral-drug-resistant-tb-treatment-regimens}{su
recomendación} para el tratamiento de la tuberculosis resistente a los
medicamentos. En lugar de 20 meses de inyecciones, los pacientes ahora
pueden tomar pastillas de nueve a 11 meses. El cambio significa que los
pacientes no tienen que trasladarse a las clínicas, que cada vez están
menos disponibles a causa de los cierres.

Más de la mitad de los 144 países encuestados por la OMS dijeron que
optaron por dar a los pacientes de VIH medicamentos suficientes para al
menos tres meses ---seis meses en el caso de algunos países como Sudán
del Sur--- y así limitar sus viajes a los hospitales. Pero no queda
claro qué tan exitosos han sido esos esfuerzos.

En algunos países, como Filipinas, los organizaciones no gubernamentales
han organizado depósitos para que los pacientes recojan píldoras
antirretrovirales o han hecho arreglos para
\href{https://www.unaids.org/en/resources/presscentre/featurestories/2020/april/20200408_philippines}{dejarlas
en las casas de los pacientes}.

En algunos países, como Sudáfrica, la mayoría de los pacientes ya
recogen los medicamentos en centros comunitarios en lugar de hospitales,
aseguró Salim S. Abdool Karim, experto en salud mundial en Sudáfrica y
presidente de un comité asesor del gobierno sobre la COVID-19. ``Esa ha
sido una ventaja importante en cierto modo''.

\hypertarget{quuxe9-es-lo-que-no-estamos-haciendo-bien}{%
\subsection{`¿Qué es lo que no estamos haciendo
bien?'}\label{quuxe9-es-lo-que-no-estamos-haciendo-bien}}

Image

El Hospital Central Sally Mugabe, en Harare, donde trabaja Tapiwa
Mungofa, ha cerrado su servicio ambulatorio, donde los pacientes con
tuberculosis y VIH recibían sus medicamentos.Credit...Cynthia R.
Matonhodze para The New York Times

La pandemia ha expuesto fisuras profundas en los sistemas de salud de
muchos países.

En Zimbabue, el personal de los hospitales públicos trabajaba en turnos
reducidos incluso antes de la pandemia, porque el gobierno no podía
pagar sus salarios completos. Algunos hospitales como el Hospital
Central Sally Mugabe, en Harare ---que funcionaba a la mitad de su
capacidad debido a la escasez de agua y otros problemas--- desde
entonces ha cerrado sus departamentos ambulatorios, donde los pacientes
de tuberculosis y VIH recibían sus medicamentos.

``Los hospitales funcionan en modo de emergencia'', dijo Tapiwa Mungofa,
médico del Hospital Sally Mugabe.

La situación no es mejor en KwaZulu-Natal, la provincia que tiene la
mayor prevalencia de VIH en Sudáfrica. Zolelwa Sifumba era una
adolescente cuando vio imágenes de pacientes esqueléticos que morían de
sida. En los últimos años, nuevamente está volviendo a ver pacientes con
sida en ese estado en KwaZulu-Natal.

``Vemos a personas que llegan en un estado en el que básicamente se
encuentran a las puertas de la muerte'', dijo. ``¿Qué es lo que no
estamos haciendo bien?''.

El coronavirus está diezmando algunas partes remotas del mundo, pero su
propia lejanía hace que sea imposible medir el impacto de la pandemia en
estos otros grandes asesinos infecciosos.

La ciudad de Tabatinga en Amazonas, el estado más grande de Brasil, está
a más de 1600 kilómetros de la ciudad más cercana con unidad de cuidados
intensivos, Manaos. El gobierno ha usado aviones para transportar
pacientes con coronavirus a Manaos, pero se han quedado muchos casos,
dijo Marcelo Cordeiro-Santos, investigador de la Fundación de Medicina
Tropical en Manaos.

Los hospitales están administrando cloroquina a personas con la
COVID-19, por recomendación del Ministerio de Salud de Brasil, a pesar
de que la evidencia sugiere que no ayuda, e incluso puede ser dañino.

La
\href{https://www.nytimes3xbfgragh.onion/es/2020/07/23/espanol/america-latina/bolivia-cloro-coronavirus-ivermectina.html}{cloroquina}
también es un medicamento crucial para la malaria, y su uso
indiscriminado puede conducir a la resistencia al medicamento, advirtió
Cordeiro-Santos, con posibles consecuencias graves para las personas
infectadas en el futuro. Pero también dijo que es posible que la
distribución generalizada de la cloroquina pueda ayudar a proteger a los
residentes de Amazonas del paludismo.

Otros expertos dijeron que esperan que la pandemia de coronavirus tenga
algún lado bueno.

Las agencias de ayuda han recomendado durante mucho tiempo que los
países compren medicamentos a granel y proporcionen suministro de varios
meses a sus ciudadanos. Algunos gobiernos ahora consideran hacerlo con
el VIH, según Doherty, de la OMS.

Los proveedores de atención médica también están adoptando videollamadas
o llamadas telefónicas para aconsejar y tratar a los pacientes, lo que a
muchas personas les resulta mucho más fácil que viajar a clínicas
distantes.

``A veces los sistemas son difíciles de cambiar'', dijo Del Rio, ``pero
creo que no hay nada mejor que una crisis para cambiar el sistema,
¿cierto?''.

Lynsey Chutel colaboró con este reportaje desde Johannesburgo.

Image

Algunas áreas remotas en Brasil están siendo diezmadas por el
coronavirus, pero su lejanía hace que el impacto de otros grandes
asesinos sea imposible de medir.Credit...Tarso Sarraf/Agence
France-Presse --- Getty Images

Apoorva Mandavilli es reportera del Times y se especializa en la ciencia
y la salud global. En 2019 ganó el premio Victor Cohn a la excelencia en
la elaboración de informes sobre ciencias médicas.
\href{https://twitter.com/apoorva_nyc}{@apoorva\_nyc}

Advertisement

\protect\hyperlink{after-bottom}{Continue reading the main story}

\hypertarget{site-index}{%
\subsection{Site Index}\label{site-index}}

\hypertarget{site-information-navigation}{%
\subsection{Site Information
Navigation}\label{site-information-navigation}}

\begin{itemize}
\tightlist
\item
  \href{https://help.nytimes3xbfgragh.onion/hc/en-us/articles/115014792127-Copyright-notice}{©~2020~The
  New York Times Company}
\end{itemize}

\begin{itemize}
\tightlist
\item
  \href{https://www.nytco.com/}{NYTCo}
\item
  \href{https://help.nytimes3xbfgragh.onion/hc/en-us/articles/115015385887-Contact-Us}{Contact
  Us}
\item
  \href{https://www.nytco.com/careers/}{Work with us}
\item
  \href{https://nytmediakit.com/}{Advertise}
\item
  \href{http://www.tbrandstudio.com/}{T Brand Studio}
\item
  \href{https://www.nytimes3xbfgragh.onion/privacy/cookie-policy\#how-do-i-manage-trackers}{Your
  Ad Choices}
\item
  \href{https://www.nytimes3xbfgragh.onion/privacy}{Privacy}
\item
  \href{https://help.nytimes3xbfgragh.onion/hc/en-us/articles/115014893428-Terms-of-service}{Terms
  of Service}
\item
  \href{https://help.nytimes3xbfgragh.onion/hc/en-us/articles/115014893968-Terms-of-sale}{Terms
  of Sale}
\item
  \href{https://spiderbites.nytimes3xbfgragh.onion}{Site Map}
\item
  \href{https://help.nytimes3xbfgragh.onion/hc/en-us}{Help}
\item
  \href{https://www.nytimes3xbfgragh.onion/subscription?campaignId=37WXW}{Subscriptions}
\end{itemize}
