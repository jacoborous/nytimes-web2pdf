Sections

SEARCH

\protect\hyperlink{site-content}{Skip to
content}\protect\hyperlink{site-index}{Skip to site index}

\href{https://www.nytimes3xbfgragh.onion/es/section/opinion}{Opinión}

\href{https://myaccount.nytimes3xbfgragh.onion/auth/login?response_type=cookie\&client_id=vi}{}

\href{https://www.nytimes3xbfgragh.onion/section/todayspaper}{Today's
Paper}

\href{/es/section/opinion}{Opinión}\textbar{}¿Por qué son populares los
chicos malos?

\url{https://nyti.ms/3iSo30w}

\begin{itemize}
\item
\item
\item
\item
\item
\end{itemize}

Advertisement

\protect\hyperlink{after-top}{Continue reading the main story}

\href{/es/section/opinion}{Opinión}

Supported by

\protect\hyperlink{after-sponsor}{Continue reading the main story}

Comentario

\hypertarget{por-quuxe9-son-populares-los-chicos-malos}{%
\section{¿Por qué son populares los chicos
malos?}\label{por-quuxe9-son-populares-los-chicos-malos}}

Pese a las crisis, errores y tropiezos en sus gobiernos, los presidentes
de México, El Salvador y Brasil mantienen índices de aprobación nada
desdeñables. ¿Cuáles son las razones de sus seguidores?

\includegraphics{https://static01.graylady3jvrrxbe.onion/images/2020/08/20/multimedia/20Fonseca-ES/20Fonseca-ES-articleLarge.jpg?quality=75\&auto=webp\&disable=upscale}

Por Diego Fonseca

Es escritor y periodista.

\begin{itemize}
\item
  20 de agosto de 2020
\item
  \begin{itemize}
  \item
  \item
  \item
  \item
  \item
  \end{itemize}
\end{itemize}

\href{https://www.nytimes3xbfgragh.onion/newsletters/el-times}{Regístrate
para recibir nuestro boletín} con lo mejor de The New York Times.

\begin{center}\rule{0.5\linewidth}{\linethickness}\end{center}

Es extraño. La violencia criminal aumenta, los femicidios continúan y la
respuesta contra el coronavirus ha hecho de México el
\href{https://www.nytimes3xbfgragh.onion/es/interactive/2020/espanol/america-latina/coronavirus-en-mexico.html}{tercer
país con más muertes} en el mundo, pero el presidente Andrés Manuel
López Obrador
\href{https://www.animalpolitico.com/2020/08/arpobacion-amlo-reprobado-uso-cubrebocas/}{mantiene
un firme apoyo} entre la población. Brasil ve a Jair Bolsonaro en pleno
\href{https://www.reuters.com/article/us-brazil-politics/brazils-bolsonaro-gains-more-popular-approval-says-datafolha-poll-idUSKCN25A1JX?emci=a2e2c5ff-f6de-ea11-8b03-00155d0394bb\&emdi=1b6d81c3-f8de-ea11-8b03-00155d0394bb\&ceid=4606001}{pico
de popularidad} mientras se encamina a ser la nación con más víctimas
por el virus. Y a Nayib Bukele, que dirige El Salvador
\href{https://www.nytimes3xbfgragh.onion/es/2020/04/20/espanol/opinion/bukele-el-salvador-virus.html}{desde
Twitter}, avanza contra la institucionalidad y
\href{https://elfaro.net/es/202007/columnas/24691/Bukele-amenaza-al-periodismo.htm}{la
prensa libre} del país,
\href{https://www.laprensagrafica.com/lpgdatos/Bukele-cierra-su-primer-ano-de-trabajo-con-alta-aprobacion-20200523-0072.html}{lo
adoran} en masa.

¿Qué sucede aquí? Si me lo permiten, diré que es infantilización.

Las crisis recurrentes y sistemáticas en que vivimos han plagado nuestra
época de tales incertidumbres que no vemos un futuro claro ni
esperanzador. En esos momentos de debilidad las sociedades claman por
salvadores de convicciones indudables. Ha pasado decenas de veces, nada
más revisen la historia: un presidente-padre capaz de protegernos de
nuestros temores infantiles con sus certezas sin claroscuros.

Los populismos telemáticos nos hablan desde allí. Cercanos, simples,
directos: se dicen pueblo. Y ganan terreno. Sin otra oferta accesible y
distinta a lo viejo conocido, con crisis de representatividad y sin
mucho más a que aferrarse, sus seguidores llenan los vacíos del discurso
de sus mesías para justificar sus propias creencias.

Sería fácil acusar a los seguidores de esos líderes de ignorantes. Yo lo
he pensado en momentos de desasosiego y enojo: ¿Qué está mal con esta
gente? Quiero ensayar una razón, y es posible que sea la misma que
alimenta sus propias elecciones: todos apetecemos repuestas en un mundo
que ya cambiaba demasiado rápido cuando nos cayó encima la pandemia y
desestructuró uno de nuestros últimos relatos, la omnipotencia humana:
con toda nuestra ciencia y técnica, no hemos sido capaces de controlar
el ataque de un organismo minúsculo. ¿Qué queda cuando todo fracasa y no
se ve un futuro claro? Creer en quien ofrece soluciones
tranquilizadoras.

El amor que
\href{https://www.plazapublica.com.gt/content/el-asalto-de-los-outsiders}{millones
profesan} por los líderes populistas no puede ser explicado sin ciertos
macrofondos. El fin de los grandes relatos que explicaban la realidad
nos dejó solo con una globalización que no ha dado respuestas amables a
las necesidades de todos. Las religiones sufren la tensión de una época
donde la racionalidad puso en crisis las respuestas absolutas. El cambio
tecnológico nos ha hecho vulnerables: el mercado laboral es incierto
incluso para los más formados. Nunca se detiene la saturación
informativa, nadamos en el río de incertidumbre de las redes sociales.
Nada es estable.

Y luego, los políticos que trivializan el mundo simplificando problemas
complejos con promesas y mentiras consoladoras. La democracia produce
dialécticas subóptimas y millones de electores han escuchado por
demasiado tiempo soluciones que se cumplieron a medias, o nunca. Trump,
Bolsonaro, Maduro, Bukele, AMLO (y otros) ofrecen un discurso de ideas
digeribles, con enemigos identificables y soluciones mágicas. Millones
los siguen con devoción casi incondicional porque están hartos. Y no
sabemos leer que quien se harta puede hacer volar la democracia por los
aires.

\includegraphics{https://static01.graylady3jvrrxbe.onion/images/2020/08/20/multimedia/20Fonsece-ES-2/merlin_174737223_98c3830d-f25d-46e1-aebb-c796fdfd2206-articleLarge.jpg?quality=75\&auto=webp\&disable=upscale}

Estos presidentes-padres hablan a pulsiones primarias compartidas y
activan miedos atávicos. Fabrican crisis que solo ellos, claro, pueden
resolver.

Pero el problema tiene otra parte: los críticos también sucumbimos al
simplismo. Nos ocupamos de las formas porque creemos que expresan el
fondo de algún modo. En ocasiones tomamos una distancia que esconde
prejuicios. Y vemos en hombres y mujeres que se aferran a la fe una masa
indivisa y homogénea, muy distante de la realidad. Ese error de
diagnóstico hace fracasar la construcción de soluciones.

Una porción de los seguidores de los líderes personalistas son
activistas aferrados al dogma, pero no son la mayoría. Aunque la
ultraderecha europea, por ejemplo, ha crecido, no deja de ser una
\href{https://www.bbc.com/news/world-europe-46422036}{fuerza electoral
pequeña} y la
\href{https://news.gallup.com/poll/15370/party-affiliation.aspx}{base
conservadora representa algo más de un cuarto} de los votantes de
Estados Unidos. Muchas otras personas, en cambio, tienen flexibilidad.
Sus votos suelen responder a la necesidad personal o cierto humor
colectivo.

Pero no conversamos con ellos. En cuanto eligen mal, los ubicamos en la
casilla de los equivocados.

Vamos, no todo el mundo consume las noticias en los medios más
rigurosos; demasiadas personas abrevan en fuentes que les dicen lo que
desean oír. Muchas otras ni se informan: confían en una visión del mundo
alimentada en sus casas, el trabajo, las
\href{https://www.nytimes3xbfgragh.onion/es/2020/08/13/espanol/estados-unidos/trump-cristianos-evangelicos.html}{iglesias}
o los clubes.

Como los seguidores más recalcitrantes de Trump, Bolsonaro o López
Obrador, a menudo nosotros entramos a nuestra vida-burbuja. Lo hicieron
los intelectuales de Estados Unidos en 2016: se convencieron de pensar
lo correcto y jamás osaron dudar de que la sociedad podía pensar lo
contrario. Pero al final Trump llegó a la Casa Blanca. La burbuja tanto
protege como aísla.

Huimos del conflicto, evitamos los roces de la diferencia. No nos
reunimos con quienes piensan distinto. Nuestros grupos se tribalizan en
la vida real y en las redes, donde pasamos cada vez más tiempo.

Hay un encanto maligno en los personalismos cuasi-religiosos y
paternalistas, pero los líderes carismáticos solo tienen capacidad de
ganar adeptos mientras evitan los cuestionamientos. La apostasía llega
cuando muestras que esos dioses encantadores son de barro, carne y
hueso. El error reiterado, al cabo, derrumba todo, incluidas las
mayorías más consistentes. No hay nadie invulnerable a los fallos ni la
paciencia es infinita. La comparación corroe. Acaso por eso Donald
Trump, así mantenga posibilidades de ser reelecto en Estados Unidos,
cada vez parece
\href{https://www.nytimes3xbfgragh.onion/2020/08/17/upshot/polls-2020-election-convention.html}{más
alejado de su contendiente}, Joe Biden. Los errores continuos tienen
límites, así los índices de popularidad de AMLO, Bolsonaro y Bukele
digan lo contrario.

Claro, no es fácil dejar de ser niños con futuros atemorizantes. En
tiempos de subjetividades profundas, quizás precisemos una revelación
pragmática pero ilusionante capaz de reconstruir una ficción orientadora
colectiva. Biden parece querer encarnar eso en Estados Unidos: en vez de
confrontar y ampliar brechas, unificar. México, Brasil, El Salvador y
los países con presidentes-padres necesitarán alternativas similares.
Porque al final, los populismos que infantilizan a los ciudadanos
revientan y dejan un boquete difícil de restaurar para las democracias.

No hay sociedad exitosa si no aprendemos a convivir con los codos que
nos separan. Por eso los discursos de unificación son importantes: nos
sacan de la zona de confort, obligan al diálogo con roces. Pinchar
burbujas es crucial. Papá no es nuestra salvación: somos nosotros.

Diego Fonseca es colaborador regular de The New York Times y director
del Institute for Socratic Dialogue de Barcelona. \emph{Voyeur}, su
nuevo libro de perfiles, se publicará pronto en España.

Advertisement

\protect\hyperlink{after-bottom}{Continue reading the main story}

\hypertarget{site-index}{%
\subsection{Site Index}\label{site-index}}

\hypertarget{site-information-navigation}{%
\subsection{Site Information
Navigation}\label{site-information-navigation}}

\begin{itemize}
\tightlist
\item
  \href{https://help.nytimes3xbfgragh.onion/hc/en-us/articles/115014792127-Copyright-notice}{©~2020~The
  New York Times Company}
\end{itemize}

\begin{itemize}
\tightlist
\item
  \href{https://www.nytco.com/}{NYTCo}
\item
  \href{https://help.nytimes3xbfgragh.onion/hc/en-us/articles/115015385887-Contact-Us}{Contact
  Us}
\item
  \href{https://www.nytco.com/careers/}{Work with us}
\item
  \href{https://nytmediakit.com/}{Advertise}
\item
  \href{http://www.tbrandstudio.com/}{T Brand Studio}
\item
  \href{https://www.nytimes3xbfgragh.onion/privacy/cookie-policy\#how-do-i-manage-trackers}{Your
  Ad Choices}
\item
  \href{https://www.nytimes3xbfgragh.onion/privacy}{Privacy}
\item
  \href{https://help.nytimes3xbfgragh.onion/hc/en-us/articles/115014893428-Terms-of-service}{Terms
  of Service}
\item
  \href{https://help.nytimes3xbfgragh.onion/hc/en-us/articles/115014893968-Terms-of-sale}{Terms
  of Sale}
\item
  \href{https://spiderbites.nytimes3xbfgragh.onion}{Site Map}
\item
  \href{https://help.nytimes3xbfgragh.onion/hc/en-us}{Help}
\item
  \href{https://www.nytimes3xbfgragh.onion/subscription?campaignId=37WXW}{Subscriptions}
\end{itemize}
