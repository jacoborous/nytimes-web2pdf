Sections

SEARCH

\protect\hyperlink{site-content}{Skip to
content}\protect\hyperlink{site-index}{Skip to site index}

\href{https://www.nytimes3xbfgragh.onion/es/section/opinion}{Opinión}

\href{https://myaccount.nytimes3xbfgragh.onion/auth/login?response_type=cookie\&client_id=vi}{}

\href{https://www.nytimes3xbfgragh.onion/section/todayspaper}{Today's
Paper}

\href{/es/section/opinion}{Opinión}\textbar{}Simbiosfera: hacia otro
modo de entender lo humano

\url{https://nyti.ms/3kMQT4l}

\begin{itemize}
\item
\item
\item
\item
\item
\end{itemize}

\hypertarget{el-brote-de-coronavirus}{%
\subsubsection{\texorpdfstring{\href{https://www.nytimes3xbfgragh.onion/es/spotlight/coronavirus?name=styln-coronavirus-es\&region=TOP_BANNER\&variant=undefined\&block=storyline_menu_recirc\&action=click\&pgtype=Article\&impression_id=0b13a210-e3a0-11ea-945d-0babcf13dc84}{El
brote de
coronavirus}}{El brote de coronavirus}}\label{el-brote-de-coronavirus}}

\begin{itemize}
\tightlist
\item
  \href{https://www.nytimes3xbfgragh.onion/es/interactive/2020/espanol/america-latina/coronavirus-en-mexico.html?name=styln-coronavirus-es\&region=TOP_BANNER\&variant=undefined\&block=storyline_menu_recirc\&action=click\&pgtype=Article\&impression_id=0b13a211-e3a0-11ea-945d-0babcf13dc84}{Mapa
  y casos en México}
\item
  \href{https://www.nytimes3xbfgragh.onion/es/interactive/2020/08/06/espanol/ciencia-y-tecnologia/tengo-covid-19-sintomas.html?name=styln-coronavirus-es\&region=TOP_BANNER\&variant=undefined\&block=storyline_menu_recirc\&action=click\&pgtype=Article\&impression_id=0b13a212-e3a0-11ea-945d-0babcf13dc84}{Identifica
  los síntomas}
\item
  \href{https://www.nytimes3xbfgragh.onion/es/interactive/2020/science/coronavirus-tratamientos-curas.html?name=styln-coronavirus-es\&region=TOP_BANNER\&variant=undefined\&block=storyline_menu_recirc\&action=click\&pgtype=Article\&impression_id=0b13a213-e3a0-11ea-945d-0babcf13dc84}{Fármacos
  y tratamientos}
\item
  \href{https://www.nytimes3xbfgragh.onion/es/2020/04/29/espanol/estilos-de-vida/oximetro-para-que-sirve.html?name=styln-coronavirus-es\&region=TOP_BANNER\&variant=undefined\&block=storyline_menu_recirc\&action=click\&pgtype=Article\&impression_id=0b13c920-e3a0-11ea-945d-0babcf13dc84}{¿Necesitas
  un oxímetro?}
\item
  \href{https://www.nytimes3xbfgragh.onion/es/2020/07/02/espanol/ciencia-y-tecnologia/sobrevivientes-coronavirus-recuperacion.html?name=styln-coronavirus-es\&region=TOP_BANNER\&variant=undefined\&block=storyline_menu_recirc\&action=click\&pgtype=Article\&impression_id=0b13c921-e3a0-11ea-945d-0babcf13dc84}{Las
  secuelas del virus}
\end{itemize}

Advertisement

\protect\hyperlink{after-top}{Continue reading the main story}

\href{/es/section/opinion}{Opinión}

Supported by

\protect\hyperlink{after-sponsor}{Continue reading the main story}

Comentario

\hypertarget{simbiosfera-hacia-otro-modo-de-entender-lo-humano}{%
\section{Simbiosfera: hacia otro modo de entender lo
humano}\label{simbiosfera-hacia-otro-modo-de-entender-lo-humano}}

La pandemia nos recordó que la Tierra no existe para ser el hotel de
nuestras vacaciones. No debemos confundir el progreso humano con la
explotación de los recursos naturales y para eso necesitamos más
narradores del Antropoceno.

\includegraphics{https://static01.graylady3jvrrxbe.onion/images/2020/08/16/multimedia/16Carrion-ES/merlin_171237588_d5c4befe-cee9-46b5-894c-457d1046f89d-articleLarge.jpg?quality=75\&auto=webp\&disable=upscale}

Por \href{https://www.nytimes3xbfgragh.onion/by/jorge-carrion}{Jorge
Carrión}

Es escritor y crítico cultural.

\begin{itemize}
\item
  16 de agosto de 2020
\item
  \begin{itemize}
  \item
  \item
  \item
  \item
  \item
  \end{itemize}
\end{itemize}

\href{https://www.nytimes3xbfgragh.onion/newsletters/el-times}{Regístrate
para recibir nuestro boletín} con lo mejor de The New York Times.

\begin{center}\rule{0.5\linewidth}{\linethickness}\end{center}

BARCELONA --- El amor inmortal solo puede encarnarse en células
cancerígenas. En 2014, la artista Marta de Menezes y el científico Luis
Graça introdujeron genes inductores de cáncer en sus células inmunes y
enamoradas. Crearon así dos núcleos esenciales de vida, dos resúmenes de
sí mismos, pero condenados a no poder interactuar, porque se rechazarían
mutuamente. El precio de la inmortalidad es la soledad eterna, afirma la
ficción ---en forma de instalación artística---
\href{https://martademenezes.com/portfolio/immortality-for-two/}{\emph{Inmortality
for two}}.

En la mortal realidad, en cambio, nunca estamos solos. Porque vivimos en
la simbiosfera.

Si la semiosfera es el universo de los signos y símbolos en que todos
nos encontramos sumergidos, la simbiosfera es el de las relaciones
biológicas y tecnológicas del que también es imposible escapar. Un
espacio planetario de relaciones múltiples e incesantes entre organismos
y objetos diversos, donde lo humano no es necesariamente central. Somos
tan solo una de las cerca de nueve millones de especies de seres vivos
que convivimos en la Tierra.

La pandemia, con su difusión masiva de
\href{https://www.nytimes3xbfgragh.onion/es/2020/05/09/espanol/opinion/zoom-coronavirus.html}{imágenes
microscópicas de virus, de infografías de cuerpos humanos en situaciones
de contagio y de cuadrículas de Zoom}, nos ha familiarizado con la
representación de nuestras innumerables y constantes interacciones, con
nuestra condición simbiótica. Hay distintos tipos de simbiosis, desde
las que benefician a todas las especies que se relacionan entre sí hasta
las parasitarias o las destructivas. El SARS-CoV-2 nos ha recordado con
virulencia ese espectro y también que la Tierra no existe para ser
nuestra granja, nuestra cantera o el hotel de nuestras vacaciones.

``La capacidad para el lenguaje, la ciencia y el pensamiento filosófico
nos convirtieron en los administradores de la biosfera. ¿Poseemos la
inteligencia moral para cumplir con esa tarea?'', se pregunta el
escritor y biólogo Edward O. Wilson en
\href{https://www.planetadelibros.com/libro-genesis/311810}{\emph{Génesis.
El origen de las sociedades}}. Hasta ahora la respuesta ha sido no.

Debemos empezar a imaginar futuros que no sigan los patrones de los
últimos siglos ---o de los últimos 12.000 años, desde el Neolítico---,
que no confundan el progreso humano con la explotación de los recursos
naturales y el imperialismo respecto a las plantas y animales. Para ello
el ser humano tiene que entender que forma parte de la simbiosfera. Que
el mundo no existe para su uso y consumo y que él mismo no es solo un
sujeto ni un cuerpo, una unidad estática, sino un fenómeno de alianzas y
relaciones, una mutación elástica.

La crisis ha hecho llegar a los medios de comunicación de masas esa
realidad, que ya había sido explorada por una de las constelaciones más
importantes del arte y las narrativas de este cambio de siglo: la de los
autores y artistas que se han asociado con científicos e ingenieros para
trabajar los intercambios biológicos o las hibridaciones cíborg. Para
representar y comunicar la simbiosfera es necesario realizar previamente
otro tipo de simbiosis: entre las ciencias y las artes, las tecnologías
y las letras.

Eso es lo que hace, precisamente, el filósofo y curador inglés Timothy
Morton, quien pone en conversación la ecología y la teoría de la ciencia
con el cine y las artes visuales para analizar nuestras
interdependencias. En
\href{https://adrianahidalgo.es/tienda/los-sentidos/humanidad/}{\emph{Humanidad.
Solidaridad con los no-humanos}}, escribe: ``En el genoma humano hay un
retrovirus simbionte llamado ERV-23 que codifica las propiedades
inmunodepresoras de la barrera de la placenta. Usted está leyendo esto
porque un virus en el ADN de su madre evitó que su cuerpo lo abortara
espontáneamente''.

Desde ese momento inicial, toda vida humana se desarrolla en simbiosis.
Aunque los individuos ---como la propia palabra indica---, nos
percibamos eminentemente como sujetos distintos y relativamente
aislados, desde el parto sobrevivimos gracias a la alianza con otras
personas, con otras especies y con diversas tecnologías. Nuestro cuerpo
y nuestra identidad no son autónomas, sino tramas de seres y cosas que
dependen los unos de los otros.

La ampliación brutal de la conciencia de que somos partes
interconectadas de un todo, aunque sea hija de la hipótesis Gaia que
James Lovelock propuso en 1969 ---según la cual la biosfera se comporta
como un sistema autorregulado---, ya no se inscribe en el contexto de la
emergencia de la política ecológica o del pensamiento \emph{new age} de
las últimas décadas del siglo XX, sino en la conciencia y la asunción
del Antropoceno, el nuevo orden climático y la digitalización del mundo
en el siglo XXI.

Por eso el histórico acuerdo al que han llegado los países miembros de
la Unión Europea, para la recuperación por el impacto de la COVID-19,
privilegia la ayuda a los programas económicos que estén relacionados,
precisamente,
\href{http://www.fondos.ciencia.gob.es/portal/site/fondos/menuitem.e1bc720d51edb26fc2b33510026041a0/?vgnextoid=de287d7f3858c610VgnVCM1000001d04140aRCRD}{con
lo digital y con la transición ecológica.} Y ---también por eso--- no es
casual que una de las filósofas más leídas y respetadas en estos
momentos sea Donna Haraway, quien en 1983 publicó el \emph{Manifiesto
Cyborg} y en los últimos años ha
desarrollado\href{https://www.consonni.org/es/publicacion/seguir-con-el-problema-generar-parientes-en-el-chthuluceno}{una
teoría del parentesco multiespecies}.

En la instalación biotecnológica
\href{https://www.agapea.si/en/projects/symbiome-the-economy-of-symbiosis}{\emph{Symbiome.
Economy of Symbiosis}}, ** las artistas Saša Spačal y Mirjan Švagelj **
idearon en 2016 un ecosistema armónico donde se ayudan mutuamente una
planta y una bacteria. Tres años antes, junto con Anil Podgornik,
construyeron \href{https://www.agapea.si/en/projects/myconnect}{una
cápsula de ciencia ficción capaz de conectar a seres humanos con
hongos}. Se trata solamente de dos ejemplos, entre muchos más, del tipo
de investigaciones que se están llevando a cabo en el campo del arte
contemporáneo. ``El arte es pensamiento procedente del futuro'', dice
Morton en
\href{https://www.planetadelibros.com/libro-ecologia-oscura/295794}{\emph{Ecología
oscura. Sobre la coexistencia futura}}. La normalización de esos relatos
transhumanos en nuestros museos, libros y pantallas nos prepara para
asumirnos como seres interdependientes, trenzados, conjuntos.

Más de cinco siglos después del
``\href{https://www.bbc.com/mundo/noticias-50095330}{hombre de
Vitruvio}'' de Leonardo da Vinci y cerca del cuarto centenario del
``pienso, luego existo'' de Descartes, ha llegado la hora de cambiar el
círculo individual por la red y la esfera, por el ecosistema y la
comunidad. Para asumir esa realidad simbiótica es preciso que el
conocimiento humano naturalice sus propias simbiosis. Por eso
necesitamos más que nunca a artistas de la hibridación, a narradores del
Antropoceno, a pensadores contrabandistas de saberes distintos, que
realicen síntesis epifánicas, como Wilson, Menezes, Graça, Spačal,
Morton o Haraway.

Solamente las convergencias entre la ecología y la política, las
ciencias y las humanidades, las tecnologías y las artes pueden conducir
hacia nuevas maneras de entendernos como personas y como seres vivos, en
entornos cada vez más y más complejos.

Jorge Carrión, colaborador regular de The New York Times, es escritor y
director del máster en Creación Literaria y del posgrado en Creación de
Contenidos y Nuevas Narrativas Digitales de la UPF-BSM. Su nuevo libro
se titula \emph{Lo viral}.

Advertisement

\protect\hyperlink{after-bottom}{Continue reading the main story}

\hypertarget{site-index}{%
\subsection{Site Index}\label{site-index}}

\hypertarget{site-information-navigation}{%
\subsection{Site Information
Navigation}\label{site-information-navigation}}

\begin{itemize}
\tightlist
\item
  \href{https://help.nytimes3xbfgragh.onion/hc/en-us/articles/115014792127-Copyright-notice}{©~2020~The
  New York Times Company}
\end{itemize}

\begin{itemize}
\tightlist
\item
  \href{https://www.nytco.com/}{NYTCo}
\item
  \href{https://help.nytimes3xbfgragh.onion/hc/en-us/articles/115015385887-Contact-Us}{Contact
  Us}
\item
  \href{https://www.nytco.com/careers/}{Work with us}
\item
  \href{https://nytmediakit.com/}{Advertise}
\item
  \href{http://www.tbrandstudio.com/}{T Brand Studio}
\item
  \href{https://www.nytimes3xbfgragh.onion/privacy/cookie-policy\#how-do-i-manage-trackers}{Your
  Ad Choices}
\item
  \href{https://www.nytimes3xbfgragh.onion/privacy}{Privacy}
\item
  \href{https://help.nytimes3xbfgragh.onion/hc/en-us/articles/115014893428-Terms-of-service}{Terms
  of Service}
\item
  \href{https://help.nytimes3xbfgragh.onion/hc/en-us/articles/115014893968-Terms-of-sale}{Terms
  of Sale}
\item
  \href{https://spiderbites.nytimes3xbfgragh.onion}{Site Map}
\item
  \href{https://help.nytimes3xbfgragh.onion/hc/en-us}{Help}
\item
  \href{https://www.nytimes3xbfgragh.onion/subscription?campaignId=37WXW}{Subscriptions}
\end{itemize}
