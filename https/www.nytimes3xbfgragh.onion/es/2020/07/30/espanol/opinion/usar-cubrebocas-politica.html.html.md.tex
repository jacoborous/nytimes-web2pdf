Sections

SEARCH

\protect\hyperlink{site-content}{Skip to
content}\protect\hyperlink{site-index}{Skip to site index}

\href{https://www.nytimes3xbfgragh.onion/es/section/opinion}{Opinión}

\href{https://myaccount.nytimes3xbfgragh.onion/auth/login?response_type=cookie\&client_id=vi}{}

\href{https://www.nytimes3xbfgragh.onion/section/todayspaper}{Today's
Paper}

\href{/es/section/opinion}{Opinión}\textbar{}Si nuestros cubrebocas
pudieran hablar

\url{https://nyti.ms/3jXxZaE}

\begin{itemize}
\item
\item
\item
\item
\item
\end{itemize}

\href{https://www.nytimes3xbfgragh.onion/es/spotlight/coronavirus?action=click\&pgtype=Article\&state=default\&region=TOP_BANNER\&context=storylines_menu}{El
brote de coronavirus}

\begin{itemize}
\tightlist
\item
  \href{https://www.nytimes3xbfgragh.onion/es/interactive/2020/espanol/america-latina/coronavirus-en-mexico.html?action=click\&pgtype=Article\&state=default\&region=TOP_BANNER\&context=storylines_menu}{Mapa
  y casos en México}
\item
  \href{https://www.nytimes3xbfgragh.onion/es/2020/07/31/espanol/ciencia-y-tecnologia/ninos-contagio-coronavirus.html?action=click\&pgtype=Article\&state=default\&region=TOP_BANNER\&context=storylines_menu}{Los
  niños y el virus}
\item
  \href{https://www.nytimes3xbfgragh.onion/es/interactive/2020/science/coronavirus-tratamientos-curas.html?action=click\&pgtype=Article\&state=default\&region=TOP_BANNER\&context=storylines_menu}{Fármacos
  y tratamientos}
\item
  \href{https://www.nytimes3xbfgragh.onion/es/2020/07/06/espanol/ciencia-y-tecnologia/coronavirus-transmision-aire.html?action=click\&pgtype=Article\&state=default\&region=TOP_BANNER\&context=storylines_menu}{Cómo
  se transmite el coronavirus}
\item
  \href{https://www.nytimes3xbfgragh.onion/es/2020/07/14/espanol/estilos-de-vida/botiquin-medicina-coronavirus.html?action=click\&pgtype=Article\&state=default\&region=TOP_BANNER\&context=storylines_menu}{Prepara
  tu botiquín}
\end{itemize}

Advertisement

\protect\hyperlink{after-top}{Continue reading the main story}

\href{/es/section/opinion}{Opinión}

Supported by

\protect\hyperlink{after-sponsor}{Continue reading the main story}

Comentario

\hypertarget{si-nuestros-cubrebocas-pudieran-hablar}{%
\section{Si nuestros cubrebocas pudieran
hablar}\label{si-nuestros-cubrebocas-pudieran-hablar}}

¿Cómo nos volvimos tan ineficaces para combatir al coronavirus? Los
arqueólogos del futuro que vinieran a excavar al país más rico del
mundo, encontrarían la clave en un artefacto sencillo: la mascarilla.

\includegraphics{https://static01.graylady3jvrrxbe.onion/images/2020/07/28/opinion/28friedmanWeb/28friedmanWeb-articleLarge.jpg?quality=75\&auto=webp\&disable=upscale}

\href{https://www.nytimes3xbfgragh.onion/by/thomas-l-friedman}{\includegraphics{https://static01.graylady3jvrrxbe.onion/images/2018/04/02/opinion/thomas-l-friedman/thomas-l-friedman-thumbLarge.png}}

Por
\href{https://www.nytimes3xbfgragh.onion/by/thomas-l-friedman}{Thomas L.
Friedman}

Es columnista de Opinión de The New York Times.

\begin{itemize}
\item
  30 de julio de 2020
\item
  \begin{itemize}
  \item
  \item
  \item
  \item
  \item
  \end{itemize}
\end{itemize}

\href{https://www.nytimes3xbfgragh.onion/2020/07/28/opinion/coronavirus-masks.html}{Read
in English}

\href{https://www.nytimes3xbfgragh.onion/newsletters/el-times}{Regístrate
para recibir nuestro boletín} con lo mejor de The New York Times.

\begin{center}\rule{0.5\linewidth}{\linethickness}\end{center}

Cuando la gente me pregunta sobre mi estado de ánimo en estos días, les
digo que me siento como si fuera un reportero del Diario de Pompeya en
el año 79 d. C. y estuviera sentado en la ladera del monte Vesubio
cuando alguien se acerca y me pregunta: ``Oye, ¿no sientes que se
mueve?''.

Claro que sí.

El verano de 2020 podría ser recordado como una de esas fechas
verdaderamente importantes en la historia estadounidense. Adonde voltees
verás padres que no saben adónde irán sus hijos a la escuela o si lo
harán este otoño, inquilinos que no saben si los desalojarán,
desempleados que no saben si el Congreso de Estados Unidos los
respaldará con alguna red de seguridad, negocios que no saben si podrán
aguantar otro día\ldots{} y todos nosotros, que no sabemos si podremos
votar en noviembre.

Esa es mucha ansiedad ardiendo y humeando por debajo de la economía, la
sociedad, las escuelas y las calles de la ciudad ---tan solo a la espera
de hacer erupción por todo el país--- porque hemos fracasado de manera
ejemplar en la batalla contra el coronavirus. Tenemos el 25 por ciento
de todas las infecciones registradas en el mundo y solo representamos el
cuatro por ciento de la población mundial. La gran ironía es que
Vietnam, que tiene menos de un tercio de nuestra población, solo ha
reportado 416 casos y ninguna muerte, la gente siente lástima por
nosotros.

¿Cómo nos volvimos tan ineptos?

Si, Dios no lo quiera, Estados Unidos quedara sepultado bajo lava como
ocurrió en Pompeya, y los arqueólogos del futuro vinieran a excavar el
país, no tengo duda de que el artefacto que desempolvarían y sacarían
primero para responder la gran pregunta sería un artículo sencillo que
cuesta centavos fabricar y que es muy fácil de usar: el cubrebocas.

Para ser algo que debe cubrir nuestra boca, dice muchísimo sobre cuán
dementes se han vuelto las personas. En específico, el cubrebocas nos
dice cómo el país más rico y científicamente avanzado generó un grupo de
líderes y ciudadanos que hicieron del usar un artículo para cubrir la
nariz y la boca (con el fin de evitar la propagación de una enfermedad)
en un problema de libertad de expresión y un indicador cultural, algo
que no se hizo en ningún otro país del mundo.

No hay nada más desmoralizante que eso, nada que nos rezague en la
batalla en contra de la COVID-19 con más fuerza y más rápido. Una
sociedad que puede politizar algo tan sencillo como un cubrebocas en una
pandemia puede politizar cualquier cosa, puede hacer de cualquier cosa
un asunto contencioso: la física, la gravedad, la lluvia, lo que sea. Y
una sociedad que lo politiza todo jamás alcanzará todo su potencial en
las buenas épocas ni evitará lo peor en las malas.

Ahí es donde estamos ahora. Cuando se comparan los sacrificios
---incluyendo la muerte--- que la generación más grandiosa de
estadounidenses hizo para defender a sus conciudadanos del flagelo del
nazismo con lo poco que sacrificarán algunos miembros de las
generaciones actuales para defender a otros estadounidenses del flagelo
de la COVID-19 ---tan solo usar un cubrebocas--- uno se queda atónito.

No hay excusas. Resistirse a usar el cubrebocas durante una pandemia no
es más que una mascarada egoísta, libertaria y sin sentido usada como
defensa risible de la libertad: ``No pisotees mis derechos, pero yo sí
puedo exhalar frente a ti''.

Sin embargo, durante meses, nuestro presidente y vicepresidente, así
como la mayoría de los gobernadores republicanos y sus seguidores
equipararon el hecho de resistirse a usar cubrebocas con resistirse a
una vulneración de la libertad personal, en vez de verlo como la manera
más barata y eficaz de limitar la propagación del virus, con el fin de
que nosotros regresemos al trabajo y los niños vuelvan a la escuela.

La resistencia que mostró el presidente Donald Trump ante los cubrebocas
en realidad no tenía nada que ver con la ideología. Solo era su
oposición primitiva a cualquier cosa que enfatizara la verdadera crisis
sanitaria en la que estábamos y que, por lo tanto, podría afectar su
reelección.

Sin embargo, el vicepresidente Mike Pence, siempre feliz de ensalzar los
excesos de Trump, disfrazó su ordinaria resistencia a usar cubrebocas
con un elegante atuendo constitucional. Cuando un reportero le preguntó
en un mitin de Trump en Tulsa hace unas semanas por qué el presidente no
parecía estar preocupado por la ausencia de cubrebocas y la falta de
distanciamiento social en su evento, Pence, de manera solemne,
respondió: ``Quiero recordarles de nuevo que la libertad de expresión y
el derecho a reunirse pacíficamente se encuentran en la Constitución de
Estados Unidos. Incluso durante una crisis sanitaria, el pueblo
estadounidense no pierde sus
\href{https://www.esquire.com/news-politics/politics/a32984272/mike-pence-masks-social-distancing-trump-rallies/}{derechos
constitucionales}''.

Qué fraude.

Como lo señaló John Finn, profesor emérito de Gobierno en la Universidad
Wesleyan
\href{https://theconversation.com/the-constitution-doesnt-have-a-problem-with-mask-mandates-142335}{en
un artículo de TheConversation.com}, ``hay dos motivos por lo que las
órdenes de usar cubrebocas no violan la primera enmienda. Primero, los
cubrebocas no evitan que la gente se exprese. Además, la primera
enmienda, como todas las libertades garantizadas por la Constitución, no
es absoluta. Todos los derechos constitucionales están sujetos a la
autoridad que tiene el gobierno para proteger la salud, la seguridad y
el bienestar de la comunidad''.

Un
\href{https://www.bcg.com/publications/2020/why-its-not-too-late-to-contain-the-virus}{estudio
de Boston Consulting Group} acerca de cuáles países no solo aplanaron la
curva del coronavirus sino que ``acabaron con ella'' halló que la clave
para reabrir la economía mientras también se contiene la transmisión del
virus era ``el distanciamiento físico, el lavado frecuente de manos y el
uso generalizado de cubrebocas'', así como el hecho de que estos
gobiernos desarrollaron lineamientos detallados para esos tres elementos
cuando se trató de establecer entornos seguros para el trabajo, las
escuelas y el transporte público.

Sin embargo, nuestros arqueólogos del futuro harían bien en enfocarse en
los cubrebocas, pues la intensa resistencia temprana a usarlos de los
líderes republicanos que apoyan a Trump y los simpatizantes del
presidente fue la esencia destilada de lo descarrilados que se
encuentran el Partido Republicano y el ecosistema mediático que lo
respalda. En ese sentido, fue otro recordatorio evidente de que no
podemos estar en nuestro mejor momento como país ---como deberíamos
estarlo durante una pandemia--- sin un partido conservador con
principios, que se base en la ciencia, no solo en los marcadores
culturales y en el libertarianismo irracional e impulsivo.

Tenemos mucho camino por recorrer.
\href{https://www.forbes.com/sites/jackbrewster/2020/07/24/19-states-still-dont-mandate-masks-18-are-run-by-republican-governors/\#4b5e331d6243}{Forbes
informó la semana pasada} que ``de los 19 estados que aún no han emitido
órdenes para usar cubrebocas, 18 son dirigidos por gobernadores
republicanos''.

Pero destaquemos a los gobernadores republicanos Larry Hogan de
Maryland, Mike DeWine de Ohio, Eric Holcomb de Indiana y Kay Ivey de
Alabama, quienes tenían o han adoptado posturas a favor de los
cubrebocas. No solo es bueno para la salud física de sus estados, sino
también para la salud política del país.

Usar cubrebocas en esta pandemia es una señal de respeto para los demás
ciudadanos y vecinos, sin importar su raza, creencias o afiliaciones
políticas. Usar una mascarilla es igual a decir: ``No solo me preocupo
por mí. Me preocupo por ti también. Todos somos parte de la misma
comunidad, el mismo país y la misma lucha para estar sanos''.

Un presidente distinto habría animado a todos los estadounidenses a usar
un cubrebocas con los colores patrios desde el inicio de la pandemia.
Habría usado un cubrebocas como ese para matar dos pájaros de un tiro:
acabar con la COVID-19 y unirnos en el largo camino que debe recorrerse
para lograrlo.

Como dije, eso lo habría hecho un presidente distinto.

Thomas L. Friedman es columnista especializado en temas internacionales.
Se unió al periódico en 1981 y ha ganado tres premios Pulitzer. Es autor
de siete libros, entre ellos \emph{From Beirut to Jerusalem}, que ganó
el National Book Award.
\href{https://twitter.com/tomfriedman}{@tomfriedman}

Advertisement

\protect\hyperlink{after-bottom}{Continue reading the main story}

\hypertarget{site-index}{%
\subsection{Site Index}\label{site-index}}

\hypertarget{site-information-navigation}{%
\subsection{Site Information
Navigation}\label{site-information-navigation}}

\begin{itemize}
\tightlist
\item
  \href{https://help.nytimes3xbfgragh.onion/hc/en-us/articles/115014792127-Copyright-notice}{©~2020~The
  New York Times Company}
\end{itemize}

\begin{itemize}
\tightlist
\item
  \href{https://www.nytco.com/}{NYTCo}
\item
  \href{https://help.nytimes3xbfgragh.onion/hc/en-us/articles/115015385887-Contact-Us}{Contact
  Us}
\item
  \href{https://www.nytco.com/careers/}{Work with us}
\item
  \href{https://nytmediakit.com/}{Advertise}
\item
  \href{http://www.tbrandstudio.com/}{T Brand Studio}
\item
  \href{https://www.nytimes3xbfgragh.onion/privacy/cookie-policy\#how-do-i-manage-trackers}{Your
  Ad Choices}
\item
  \href{https://www.nytimes3xbfgragh.onion/privacy}{Privacy}
\item
  \href{https://help.nytimes3xbfgragh.onion/hc/en-us/articles/115014893428-Terms-of-service}{Terms
  of Service}
\item
  \href{https://help.nytimes3xbfgragh.onion/hc/en-us/articles/115014893968-Terms-of-sale}{Terms
  of Sale}
\item
  \href{https://spiderbites.nytimes3xbfgragh.onion}{Site Map}
\item
  \href{https://help.nytimes3xbfgragh.onion/hc/en-us}{Help}
\item
  \href{https://www.nytimes3xbfgragh.onion/subscription?campaignId=37WXW}{Subscriptions}
\end{itemize}
