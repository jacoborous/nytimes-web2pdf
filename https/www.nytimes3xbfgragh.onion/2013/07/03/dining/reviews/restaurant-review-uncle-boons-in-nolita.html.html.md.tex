Sections

SEARCH

\protect\hyperlink{site-content}{Skip to
content}\protect\hyperlink{site-index}{Skip to site index}

\href{https://www.nytimes3xbfgragh.onion/section/food}{Food}

\href{https://myaccount.nytimes3xbfgragh.onion/auth/login?response_type=cookie\&client_id=vi}{}

\href{https://www.nytimes3xbfgragh.onion/section/todayspaper}{Today's
Paper}

\href{/section/food}{Food}\textbar{}A Jolt of Electricity Courses
Through the Kitchen

\url{https://nyti.ms/1635HNx}

\begin{itemize}
\item
\item
\item
\item
\item
\end{itemize}

Advertisement

\protect\hyperlink{after-top}{Continue reading the main story}

Supported by

\protect\hyperlink{after-sponsor}{Continue reading the main story}

\hypertarget{a-jolt-of-electricity-courses-through-the-kitchen}{%
\section{A Jolt of Electricity Courses Through the
Kitchen}\label{a-jolt-of-electricity-courses-through-the-kitchen}}

\includegraphics{https://static01.graylady3jvrrxbe.onion/images/2013/07/03/dining/03REST_SPAN/03REST_SPAN-articleLarge.jpg?quality=75\&auto=webp\&disable=upscale}

\begin{itemize}
\tightlist
\item
  Uncle Boons\\
  **NYT Critic's Pick ★★ Thai \$\$\$ 7 Spring Street 646-370-6650
\end{itemize}

By \href{https://www.nytimes3xbfgragh.onion/by/pete-wells}{Pete Wells}

\begin{itemize}
\item
  July 2, 2013
\item
  \begin{itemize}
  \item
  \item
  \item
  \item
  \item
  \end{itemize}
\end{itemize}

The smallest blackout I've ever lived through was limited to the
interior of a Thai restaurant on Spring Street, just off the Bowery,
named \href{http://www.uncleboons.com/site/}{Uncle Boons}, where early
one night the electricity kept cutting out, then flicking back on a few
minutes later. Almost every foot of wall space is covered with pictures
that might have been picked up at a Bangkok flea market: portraits of
Thailand's kings, thrift-store paintings, prints of jungle beasts, a
monster-movie poster in Thai and semi-English (``Once again dinosaur
shake the erath!''). But staging a Southeast Asian power failure seemed
to be taking the conceit a bit far.

We didn't stop eating, though. I would bring my own generator if that's
what it took to finish the mee krob with fried sweetbreads. In many Thai
restaurants in the United States, the fried noodles in mee krob are
dispatched into dissolved brown sugar with trace amounts of fish sauce
and lime juice. The result is as interesting as spaghetti in pancake
syrup. The sauce at Uncle Boons, which starts with tamarind pulp and
palm sugar, was rounded and rich and deeply engaging.

And a little thing like iffy wiring wouldn't keep me away from the yum
kai hua pli, a salad of warm roasted chicken torn and tossed with
crunchy threads of sliced banana blossoms. The chicken would have been
great by itself, but it was the dressing that helped it shine through
the power failure: warm coconut milk soured with lime juice and spiced
with dried chiles to an intensity that I think of as handkerchief-hot.

Uncle Boons must have fixed its electrical troubles, because they did
not recur. What did return was my excitement at tasting Thai flavors so
fresh and dynamic.

The restaurant has less in common with Queens Thai dynamos like
\href{http://www.nytimes3xbfgragh.onion/2010/09/08/dining/reviews/08under.html}{Ayada},
\href{http://events.nytimes3xbfgragh.onion/2004/11/03/dining/reviews/03REST.html?pagewanted=all}{Sripraphai},
tiny
\href{http://events.nytimes3xbfgragh.onion/2006/08/16/dining/reviews/16unde.html?fta=y}{Chao
Thai} or its roomier spinoff
\href{http://blogs.villagevoice.com/forkintheroad/2012/10/chao_thai_too_f.php}{Chao
Thai Too} than it does with
\href{http://www.nytimes3xbfgragh.onion/2012/06/27/dining/reviews/pok-pok-ny-brooklyn-restaurant-review.html?pagewanted=all}{Pok
Pok Ny}, Andy Ricker's place on the Brooklyn waterfront. A few of its
tricks might have been lifted from Mr. Ricker's backpack. There is the
charcoal grill, visible behind a glass wall at the end of the bar. There
is the tight menu that doesn't give you the choice of chicken, beef,
pork, shrimp or squid with almost every curry, a concession that holds
back some of the best Thai restaurants.

Both places look like dark, woody Thai pubs, right down to their
motorized ice buckets that shake bottles of Singha until they turn into
alcoholic slush that you drink through a straw. (It won't be the best
beer you've ever had, but it may be the coldest.) Pok Pok Ny follows
through with many other wacky and excellent Thai beverages, while the
drinks are Uncle Boons's prime blind spot. Neither of the two cocktails
is a complete triumph, and even the iced tea with sweetened condensed
milk tastes watery.

But Uncle Boons has many charms all its own. The owners and chefs, Ann
Redding and her husband, Matt Danzer, met while working as cooks at
\href{http://www.nytimes3xbfgragh.onion/2011/10/12/dining/reviews/per-se-nyc-restaurant-review.html?pagewanted=all}{Per
Se}, and the marks of Thomas Keller's ballet academy are far more
visible than you'd expect at a place that plays warbly Thai covers of
``Another Brick in the Wall'' and ``Hang On, Sloopy.''

\href{https://www.nytimes3xbfgragh.onion/slideshow/2013/07/03/dining/20130703-REST.html}{}

\hypertarget{uncle-boons}{%
\subsection{Uncle Boons}\label{uncle-boons}}

12 Photos

View Slide Show ›

\includegraphics{https://static01.graylady3jvrrxbe.onion/images/2013/07/03/dining/20130703-REST-slide-ONT6/20130703-REST-slide-ONT6-articleLarge.jpg?quality=75\&auto=webp\&disable=upscale}

Benjamin Petit for The New York Times

The tables are set with folded cotton napkins and traditional brass
forks and spoons from Thailand. Three months after the restaurant
opened, the dining room staff is already quite polished, poised and
welcoming. The third time the lights went out, a server brought every
table an unnecessary but delicious little apology, a plate of mieng kum,
little one-bite snacks.

Mounded on a soft, fresh betel leaf are fresh ginger, coconut, dried
shrimp, peanuts and very hot chiles. You dab on some shrimp-paste sauce,
fold the leaf over and pop it into your mouth, and right away your taste
buds spring to attention. It's a preview of the flavors that will become
major themes of the meal.

Coconut milk is the foundation of the remarkably aromatic curry of
chicken livers and fresh pineapple called dup kai kaeng supalot, and of
an amazing massaman curry with shredded potato and braised beef cheeks
that break up the minute you dip a spoon into the bowl. In place of the
gloppy peanut-butter sweetness found in many American massamans, Uncle
Boons makes a drier sauce with the exotic muskiness of a walk through a
spice shop, punctuated by the crisp pungency of green peppercorns.

Over four meals, Uncle Boons appeared to be working its way toward
consistent excellence. Still, there was one night when the green mango
salad tilted too far into sourness, and another when damp, clumpy fried
rice with crab suggested that the wok had been a few degrees shy of the
smoking fury that leads to great fried rice.

Spice levels could waver, too. But with only one or two exceptions,
Uncle Boons came through with searing heat when it was called for,
especially in that chicken salad or with a laab of minced lamb wok-fried
to a deep mahogany and served warm; fresh mint and cilantro temper the
spice but in no way tame it.

Subtle flavors have a place in Thai cooking, too. There's a hint of a
1970s suburban party dip to the lon pu kem, which offers raw snap peas,
radishes, crisp green mango and Thai eggplant to be dunked into a bowl
of creamy, coconut-rich crab dip. And there is just enough fresh green
chile in a thin nam prik sauce to bring out the sweetness in
charcoal-grilled blowfish tails or baby octopus that would fit in your
palm if you were silly enough to place them there and not between your
teeth, where they belong.

A coconut-milk-based curry yellow with fresh turmeric is one key to the
greatness of Uncle Boons's khao soi. But really, everything in the bowl
plays a part: the chicken leg stewed to irresistible tenderness, the
pickled mustard greens, the big frizz of fried cilantro-flecked egg
noodles on top and the same noodles boiled in the curry. I wouldn't
scream if Ms. Redding and Mr. Danzer threw a few more dried chiles into
the pot, but this is still the best khao soi I've eaten in New York.

Ms. Redding was born in northeastern Thailand, but before opening Uncle
Boons she and Mr. Danzer roamed the country, and their cooking does,
too. The couple may be in the market for a better electrician. But where
it counts, in the kitchen, they have all the power any lover of Thai
food could ask for.

Advertisement

\protect\hyperlink{after-bottom}{Continue reading the main story}

\hypertarget{site-index}{%
\subsection{Site Index}\label{site-index}}

\hypertarget{site-information-navigation}{%
\subsection{Site Information
Navigation}\label{site-information-navigation}}

\begin{itemize}
\tightlist
\item
  \href{https://help.nytimes3xbfgragh.onion/hc/en-us/articles/115014792127-Copyright-notice}{©~2020~The
  New York Times Company}
\end{itemize}

\begin{itemize}
\tightlist
\item
  \href{https://www.nytco.com/}{NYTCo}
\item
  \href{https://help.nytimes3xbfgragh.onion/hc/en-us/articles/115015385887-Contact-Us}{Contact
  Us}
\item
  \href{https://www.nytco.com/careers/}{Work with us}
\item
  \href{https://nytmediakit.com/}{Advertise}
\item
  \href{http://www.tbrandstudio.com/}{T Brand Studio}
\item
  \href{https://www.nytimes3xbfgragh.onion/privacy/cookie-policy\#how-do-i-manage-trackers}{Your
  Ad Choices}
\item
  \href{https://www.nytimes3xbfgragh.onion/privacy}{Privacy}
\item
  \href{https://help.nytimes3xbfgragh.onion/hc/en-us/articles/115014893428-Terms-of-service}{Terms
  of Service}
\item
  \href{https://help.nytimes3xbfgragh.onion/hc/en-us/articles/115014893968-Terms-of-sale}{Terms
  of Sale}
\item
  \href{https://spiderbites.nytimes3xbfgragh.onion}{Site Map}
\item
  \href{https://help.nytimes3xbfgragh.onion/hc/en-us}{Help}
\item
  \href{https://www.nytimes3xbfgragh.onion/subscription?campaignId=37WXW}{Subscriptions}
\end{itemize}
