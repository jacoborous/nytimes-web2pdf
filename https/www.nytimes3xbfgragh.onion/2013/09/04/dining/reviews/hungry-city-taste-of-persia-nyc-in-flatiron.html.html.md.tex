Sections

SEARCH

\protect\hyperlink{site-content}{Skip to
content}\protect\hyperlink{site-index}{Skip to site index}

\href{https://www.nytimes3xbfgragh.onion/section/food}{Food}

\href{https://myaccount.nytimes3xbfgragh.onion/auth/login?response_type=cookie\&client_id=vi}{}

\href{https://www.nytimes3xbfgragh.onion/section/todayspaper}{Today's
Paper}

\href{/section/food}{Food}\textbar{}Finding Happiness in Paradise

\url{https://nyti.ms/1cl1B8Q}

\begin{itemize}
\item
\item
\item
\item
\item
\item
\end{itemize}

Advertisement

\protect\hyperlink{after-top}{Continue reading the main story}

Supported by

\protect\hyperlink{after-sponsor}{Continue reading the main story}

\href{/column/hungry-city}{Hungry City}

\hypertarget{finding-happiness-in-paradise}{%
\section{Finding Happiness in
Paradise}\label{finding-happiness-in-paradise}}

\includegraphics{https://static01.graylady3jvrrxbe.onion/images/2013/09/04/dining/04HUNGRY_SPAN/04HUNGRY_SPAN-articleLarge.jpg?quality=75\&auto=webp\&disable=upscale}

\begin{itemize}
\tightlist
\item
  Taste of Persia NYC\\
  **NYT Critic's Pick Middle Eastern \$ 12 West 18th Street 917-592-3467
\end{itemize}

By Ligaya Mishan

\begin{itemize}
\item
  Aug. 29, 2013
\item
  \begin{itemize}
  \item
  \item
  \item
  \item
  \item
  \item
  \end{itemize}
\end{itemize}

The scent of rose water is imperceptible at Pizza Paradise, obscured by
pepperoni's more flagrant bouquet. Nevertheless, rose water is here, in
a corner of this otherwise utilitarian pizzeria on 18th Street,
alongside fenugreek, pomegranate molasses and black limes, sour and
smoky from having been slaked with boiling water and left in the sun.

These are not experimental pizza toppings but ingredients in the dishes
at \href{https://www.facebookcorewwwi.onion/TasteOfPersiaNYC}{Taste of
Persia NYC}, a takeout counter tucked into one of Pizza Paradise's front
windows. The space is compact: to the left, a row of chafing dishes; at
the back, two giant rice cookers dwarfing a sink scaled for a child's
play kitchen; and in the middle, Saeed Pourkay, beaming in his crimson
chef's coat.

For two decades, Mr. Pourkay, a Tehrani émigré, ran a print shop across
the street from the pizzeria. After cashing out his share in the
business a few years ago (to go ``searching for my happiness,'' he
said), he started selling ash reshteh, a wondrous, wintry, outrageously
thick Persian soup, at the Union Square Holiday Market. Fans clamored.
Happiness was found. This past March, he returned to 18th Street and set
up under his former neighbor's roof.

Here, in an imposing vat, is the justly fabled ash reshteh, a result of
the eight-hour communion of five kinds of beans, a riot of herbs and
onion cooked down to a sweet density. Dark and luxuriant, it has no
broth and only a trace of oil. Broken strands of linguine snake through
it. Fenugreek lurks, faint but insistently bittersweet, underscoring
cinnamon, cardamom and ginger. But it is the garnishes that turn it into
poetry: caramelized, verging-on-burned garlic; dried mint flicked in a
pan; crispy fried onion; and a swirl of kashk, a Persian whey more sour
than yogurt, with a bite like feta.

\href{https://www.nytimes3xbfgragh.onion/slideshow/2013/09/04/dining/20130904-HUNGRY.html}{}

\hypertarget{taste-of-persia-nyc}{%
\subsection{Taste of Persia NYC}\label{taste-of-persia-nyc}}

7 Photos

View Slide Show ›

\includegraphics{https://static01.graylady3jvrrxbe.onion/images/2013/09/04/dining/20130904-HUNGRY-slide-PCWT/20130904-HUNGRY-slide-PCWT-articleLarge.jpg?quality=75\&auto=webp\&disable=upscale}

Robert Caplin for The New York Times

The remaining dishes (mostly rustic stews simmered for hours, sometimes
days) change daily. Hope for gheimeh bademjan: long, curling halves of
Chinese eggplant trimmed with yellow lentils and hunks of Angus beef;
fesenjan, in which evanescent rose water mediates between sour-sweet
pomegranate molasses and the earthbound pull of walnuts ground fine;
khoresh beh, which employs beef as merely a foil for quince; and ghormeh
sabzi, a deep green fluid patchwork of parsley, cilantro and spinach,
with a whiff of fermentation from black limes. (I did wish that Mr.
Pourkay offered a combination platter, because the stews are best
appreciated in conjunction.)

Each stew is presented on top of Persian basmati rice, which is a labor
in itself. Mr. Pourkay washes the rice four times, pours it into a pot
of boiling salted water, drains it, douses it with cold water, then puts
it into the rice cooker with a little water and oil and leaves it to
steam. What emerges is a cloud base of fluffy grains without a hint of
cling, striped with gold from a sluicing of crushed saffron dissolved in
hot water.

It is worth it to wait while Mr. Pourkay meticulously spoons out tasting
samples and totes up bills on a calculator, with the front door swinging
back and forth and people brushing past, occasionally knocking you on
the shoulder. The food invokes the ancient splendors of the spice route,
but it comes in ordinary plastic bins; forks and napkins must be foraged
from the (very accommodating) pizza workers. In the back is a dining
room with glaring fluorescents, should you care to eat in.

Mr. Pourkay must cook around the pizzeria's kitchen schedule, which
means only in early morning and late afternoon. (``I've been here since
1:30 a.m.,'' he told me one day.) This hardly seems a sustainable state
of affairs.

A call to action, then: please, will some investor swoop in and give Mr.
Pourkay a little more real estate to work with? He has dreams of making
faloodeh, the Persian frozen dessert of rice noodles and rose-water
syrup, and of devising innovative ways to serve tahdig, the crusty layer
of rice at the bottom of the pot. I dream, too.

Advertisement

\protect\hyperlink{after-bottom}{Continue reading the main story}

\hypertarget{site-index}{%
\subsection{Site Index}\label{site-index}}

\hypertarget{site-information-navigation}{%
\subsection{Site Information
Navigation}\label{site-information-navigation}}

\begin{itemize}
\tightlist
\item
  \href{https://help.nytimes3xbfgragh.onion/hc/en-us/articles/115014792127-Copyright-notice}{©~2020~The
  New York Times Company}
\end{itemize}

\begin{itemize}
\tightlist
\item
  \href{https://www.nytco.com/}{NYTCo}
\item
  \href{https://help.nytimes3xbfgragh.onion/hc/en-us/articles/115015385887-Contact-Us}{Contact
  Us}
\item
  \href{https://www.nytco.com/careers/}{Work with us}
\item
  \href{https://nytmediakit.com/}{Advertise}
\item
  \href{http://www.tbrandstudio.com/}{T Brand Studio}
\item
  \href{https://www.nytimes3xbfgragh.onion/privacy/cookie-policy\#how-do-i-manage-trackers}{Your
  Ad Choices}
\item
  \href{https://www.nytimes3xbfgragh.onion/privacy}{Privacy}
\item
  \href{https://help.nytimes3xbfgragh.onion/hc/en-us/articles/115014893428-Terms-of-service}{Terms
  of Service}
\item
  \href{https://help.nytimes3xbfgragh.onion/hc/en-us/articles/115014893968-Terms-of-sale}{Terms
  of Sale}
\item
  \href{https://spiderbites.nytimes3xbfgragh.onion}{Site Map}
\item
  \href{https://help.nytimes3xbfgragh.onion/hc/en-us}{Help}
\item
  \href{https://www.nytimes3xbfgragh.onion/subscription?campaignId=37WXW}{Subscriptions}
\end{itemize}
