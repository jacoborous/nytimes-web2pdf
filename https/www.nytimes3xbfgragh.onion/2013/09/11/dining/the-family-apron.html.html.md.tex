Sections

SEARCH

\protect\hyperlink{site-content}{Skip to
content}\protect\hyperlink{site-index}{Skip to site index}

\href{https://www.nytimes3xbfgragh.onion/section/food}{Food}

\href{https://myaccount.nytimes3xbfgragh.onion/auth/login?response_type=cookie\&client_id=vi}{}

\href{https://www.nytimes3xbfgragh.onion/section/todayspaper}{Today's
Paper}

\href{/section/food}{Food}\textbar{}Inheriting the Family Apron

\url{https://nyti.ms/189JBiP}

\begin{itemize}
\item
\item
\item
\item
\item
\item
\end{itemize}

Advertisement

\protect\hyperlink{after-top}{Continue reading the main story}

Supported by

\protect\hyperlink{after-sponsor}{Continue reading the main story}

\hypertarget{inheriting-the-family-apron}{%
\section{Inheriting the Family
Apron}\label{inheriting-the-family-apron}}

\includegraphics{https://static01.graylady3jvrrxbe.onion/images/2013/09/11/dining/11GENERATION_SPAN/11GENERATION_SPAN-articleLarge-v3.jpg?quality=75\&auto=webp\&disable=upscale}

By \href{https://www.nytimes3xbfgragh.onion/by/julia-moskin}{Julia
Moskin}

\begin{itemize}
\item
  Sept. 10, 2013
\item
  \begin{itemize}
  \item
  \item
  \item
  \item
  \item
  \item
  \end{itemize}
\end{itemize}

There are many routes to a cooking career. Not so long ago, military and
prison kitchens were reliable sources of line cooks for American
restaurants. Now the siren song of reality television lures many
recruits to kitchens and culinary schools.

But until recently, very few American chefs were born into the
profession. Even the luminaries who led the American food revolution of
the 1970s and '80s found their own ways to the stove --- through travel,
like Alice Waters, or anthropology, like Rick Bayless. Their parents
were hardly encouraging.

``Cooking was not considered a respectable or profitable profession,''
said Maria Guarnaschelli, an eminent
\href{http://www.nytimes3xbfgragh.onion/2006/11/01/dining/01jside.html}{cookbook
editor} and the mother of Alex Guarnaschelli, who is the chef at
\href{http://events.nytimes3xbfgragh.onion/mem/nycreview.html?res=9A00EFD8163EF935A15755C0A9649C8B63}{Butter},
in Greenwich Village. ``We never thought our daughter would be a chef.''

Now, a generation of chefs and entrepreneurs who grew up in the kitchen
are shaping American food.

The sons of pioneering American chefs like Norman Van Aken, Bradley
Ogden and Larry Forgione have grown into their own chef's whites. Sara
Jenkins, 48, the chef and owner of \href{http://porsena.com/}{Porsena}
and \href{http://www.porchettanyc.com/}{Porchetta} in the East Village,
trained her palate from childhood by globe-trotting with her mother, the
Mediterranean food expert Nancy Harmon Jenkins.

Two of the most influential chefs in the Austin, Tex., area are Bryce
Gilmore, of \href{http://barleyswine.com/}{Barley Swine}, and his
father, Jack Gilmore, of \href{http://www.jackallenskitchen.com/}{Jack
Allen's Kitchen}. Bryce, 30, grew up learning at Jack's elbow --- but
today Bryce is coaching Jack, who quit a job as a corporate chef after
20 years to get back into the kitchen, in modern culinary arts like
curing and cultivating relationships with farmers.

\includegraphics{https://static01.graylady3jvrrxbe.onion/images/2013/09/11/dining/11JPGENERATION1/11JPGENERATION1-articleLarge.jpg?quality=75\&auto=webp\&disable=upscale}

Second-generation entrepreneurs like Nicolas Jammet, whose parents, Rita
and André Jammet, owned New York's elegant La Caravelle, are using their
food knowledge outside the kitchen. Nic, 28, discovered an appetite for
sophisticated, sustainable fast food while still in college; the chain
he founded in 2007 in Washington, D.C.,
\href{http://sweetgreen.com/}{Sweetgreen}, just opened its 20th store in
the trendy NoMad hotel in Manhattan.

And some chefs who grew up in less rarefied settings --- like
\href{http://www.nytimes3xbfgragh.onion/2013/01/24/fashion/eddie-huang-defies-description.html?pagewanted=all}{Eddie
Huang}, 31, whose Taiwanese parents ran steak and seafood restaurants
around Orlando, Fla. --- are using their own professional kitchens to
revisit the true flavors of their childhoods.

In all these ways, building on the work of food-world pioneers, the next
generation is moving the culinary conversation forward.

That is, when their mothers will let them into the kitchen.

Maria Guarnaschelli published authoritative cookbooks by writers like
Julie Sahni and Barbara Tropp when most Americans neither knew nor cared
about authentic cooking. But she didn't teach Alex to cook, because she
couldn't tolerate messes or mistakes in the kitchen.

``I became a chef in spite of her,'' the daughter said, ``but I am a
perfectionist because of her, and I couldn't be a chef without that.''
(Last year, Alex, 44, worked up the courage to write her own cookbook, a
resolutely lighthearted, messy and nonauthoritative book titled
``Old-School Comfort Food.'')

When Dennis Lee, 33, and his brothers, Daniel, 32, and David, 30,
started a business selling hot dogs from a stand in Golden Gate Park,
their Korean-born mother was not enthusiastic. Even though the hot dogs
were organic and garnished with kimchi and gochujang, cooking was the
family business that she wanted her sons to escape.

Image

Jack Gilmore, second from left, and Bryce, his son, both have
restaurants in the Austin, Tex., area.Credit...Phil Kline for The New
York Times

``We were supposed to be doctors, not hot dog vendors,'' said Dennis,
who is now the chef at Namu Gaji in San Francisco, which is owned by all
three brothers.

The whole family had worked long hours at
\href{http://www.dahmee.com/}{Dah-Mee}, the popular pan-Asian restaurant
in Natick, Mass., where their mother commanded a regiment of Korean,
Japanese and Thai chefs. She insisted that staples like miso and soy
sauce be made from scratch. Dennis became the kitchen's key translator
among languages and cuisines. Namu Gaji's izakaya-style small plates,
like napa cabbage, radish and pluots dressed with tangy ponzu and crisp
seaweed, reflect how the brothers ate; the organic farm they've started
reflects how hard they worked.

``I think that secretly or unconsciously, they were training me to stay
in the food business,'' Dennis said of his parents. ``They instilled in
me this crazy work ethic where I always have to be in the kitchen.''

Multigenerational restaurants are not a new idea. In Europe, toques are
routinely passed down from father to son to grandson --- and in a few
recent cases, like Elena Arzak and Anne-Sophie Pic --- from father to
daughter. Restaurant dynasties in the United States include the
\href{http://canlis.com/}{Canlis} family in Seattle,
\href{http://bandbhg.com/our_restaurants.cfm}{the Bastianichs} in New
York and the prolific \href{http://www.pappas.com/home/}{Pappas family}
in the Southwest, who have birthed about 80 restaurants: Pappas Bros.
Steakhouse, Pappadeaux Seafood Kitchen, Pappasito's Cantina and more.
But in those clans, business responsibilities are passed down, not
culinary inspiration.

Chefs who grow up in working kitchens have both advantages and
disadvantages.

``I'd call it a gifted curse, or a cursed gift,'' said Marc Forgione,
34, whose father, Larry Forgione, was one of the first chefs to put
regional American cooking on the fine-dining map, at his Manhattan
restaurant
\href{http://events.nytimes3xbfgragh.onion/mem/nycreview.html?res=9C02E0D61E39F93BA2575BC0A96F958260}{An
American Place}, which opened in 1983.

Marc grew up not in Manhattan but on Long Island; he worked in the
kitchen part time for pocket money, but never planned to become a chef.
``I didn't know my father was a celebrity chef,'' he said. ``That didn't
even exist at the time. I knew I liked food --- my mother is a great
cook --- and I knew my father's job meant that he worked long hours and
wasn't home a lot. It didn't seem like a great professional choice.''

Image

Marc Forgione keeps a photo of his father at his
restaurant.Credit...Brian Harkin for The New York Times

It was only later, he said (after college, driving around while
``Appetite for Destruction,'' the first Guns N' Roses album, played on
the car stereo) that he understood how cooking could be a calling. ``The
same way a musician uses notes to make a great song, a chef can take a
raw piece of meat and make a great dish,'' he said. He plunged in,
working full time for prominent chefs in New York --- often, he said,
with more senior cooks watching and waiting for him to fail. ``Kitchens
are rough places,'' he said. ``Everyone has to prove themselves. Being
the kid of a legend makes it harder, not easier.''

In 2004, partly to escape the phrase ``Larry Forgione's son,'' he went
to live and work with the influential French chef Michel Guérard, at his
restaurant in a remote corner of Gascony. ``It's in the middle of
nowhere,'' Marc said with characteristic bluntness. ``I knew no one. I
spoke not one word of French. I just put my head down and worked like
everybody else, 7 a.m. to 11 p.m., for a year.'' It was the skills he
gained there, anonymously, he said, that provided the confidence to
return to New York and open his own restaurant.

At the other end of the privilege spectrum, the New York chef Ann
Redding grew up outside Bangkok, in a family where cooking was the only
professional option. Her grandmother raised six daughters alone,
supporting them by growing and selling vegetables from a stall. An aunt
was a cook at the royal palace.

Ms. Redding's aunts became skilled cooks at a young age, making snacks
to sell by the roadside; her mother's specialty was miang kham, a savory
parcel of dried shrimp, chiles, peanuts, lime and coconut, wrapped in
fragrant leaves from the betel nut tree. When she can get fresh betel
leaves, Ms. Redding, 38, serves the dish at
\href{http://www.nytimes3xbfgragh.onion/2013/07/03/dining/reviews/restaurant-review-uncle-boons-in-nolita.html?pagewanted=all}{Uncle
Boons}, her restaurant in NoLIta, where she uses her experience in
kitchens like
\href{http://www.nytimes3xbfgragh.onion/2013/07/24/dining/reviews/restaurant-review-daniel-on-the-upper-east-side.html?pagewanted=all}{Daniel}
and
\href{http://www.nytimes3xbfgragh.onion/2011/10/12/dining/reviews/per-se-nyc-restaurant-review.html?pagewanted=all}{Per
Se} to evoke the Thai flavors she grew up with.

``It is a romantic memory, sitting with my grandmother while she crushed
herbs with the mortar and pestle, talking about what soup my aunt had
made for the royal family,'' she said. ``If you're a kid and you're
around that obsession with food, it does stay with you.''

For Tom Schlesinger-Guidelli, growing up in the kitchen of the
\href{http://eastcoastgrill.net/}{East Coast Grill} in Cambridge, Mass.,
gave him the sense that cooking wasn't just a job, but a way to change
the world.

Image

Larry Forgione, the pioneering chef.Credit...Brian Harkin for The New
York Times

``When he started East Coast Grill, farm-to-table wasn't a thing,'' he
said of Chris Schlesinger, his uncle, who opened the restaurant in 1986.
Mr. Schlesinger embraced an earthy, lively, D.I.Y. approach to cooking
that was revolutionary at the time.

``My great-grandmother made her own soap and grew her own vegetables and
cured her hams and used her own fatback, so he had a basic appreciation
of that stuff built into him,'' said Mr. Schlesinger-Guidelli, 30, who
started in the kitchen at the tender age of 5. ``And I learned from him
that you don't let corporations do for you what you can do for
yourself.''

Now, Mr. Schlesinger-Guidelli works at
\href{http://islandcreekoysterbar.com/}{Island Creek Oyster Bar}, a
deceptively simple place in Boston that incorporates many of the big
ideas about food that his uncle helped promote.

The menu is built around local ingredients like lobster, honey, cream,
monkfish, clams and cucumbers, bought directly from the people who
raise, catch or dig them; its owners, Skip Bennett and Shore Gregory,
also own the sustainable
\href{http://www.islandcreekoysters.com/}{Island Creek Oysters} in
nearby Duxbury, Mass., which supplies many top restaurants in the
Northeast; and the company's foundation promotes aquaculture as a form
of safe global food production, financing projects like a shellfish
hatchery in Zanzibar and tilapia farms in Haiti.

Mr. Schlesinger-Guidelli didn't plan on a career in food. But after he
graduated from Kenyon College with a degree in political science, he
found he had little interest in politics. Now, he says, it is both his
upbringing in the kitchen and his education outside it that inform the
work that he wants to do.

``Political science is about motivating larger swaths of people to care
about issues,'' he said, ``like where their oysters come from.''

Advertisement

\protect\hyperlink{after-bottom}{Continue reading the main story}

\hypertarget{site-index}{%
\subsection{Site Index}\label{site-index}}

\hypertarget{site-information-navigation}{%
\subsection{Site Information
Navigation}\label{site-information-navigation}}

\begin{itemize}
\tightlist
\item
  \href{https://help.nytimes3xbfgragh.onion/hc/en-us/articles/115014792127-Copyright-notice}{©~2020~The
  New York Times Company}
\end{itemize}

\begin{itemize}
\tightlist
\item
  \href{https://www.nytco.com/}{NYTCo}
\item
  \href{https://help.nytimes3xbfgragh.onion/hc/en-us/articles/115015385887-Contact-Us}{Contact
  Us}
\item
  \href{https://www.nytco.com/careers/}{Work with us}
\item
  \href{https://nytmediakit.com/}{Advertise}
\item
  \href{http://www.tbrandstudio.com/}{T Brand Studio}
\item
  \href{https://www.nytimes3xbfgragh.onion/privacy/cookie-policy\#how-do-i-manage-trackers}{Your
  Ad Choices}
\item
  \href{https://www.nytimes3xbfgragh.onion/privacy}{Privacy}
\item
  \href{https://help.nytimes3xbfgragh.onion/hc/en-us/articles/115014893428-Terms-of-service}{Terms
  of Service}
\item
  \href{https://help.nytimes3xbfgragh.onion/hc/en-us/articles/115014893968-Terms-of-sale}{Terms
  of Sale}
\item
  \href{https://spiderbites.nytimes3xbfgragh.onion}{Site Map}
\item
  \href{https://help.nytimes3xbfgragh.onion/hc/en-us}{Help}
\item
  \href{https://www.nytimes3xbfgragh.onion/subscription?campaignId=37WXW}{Subscriptions}
\end{itemize}
