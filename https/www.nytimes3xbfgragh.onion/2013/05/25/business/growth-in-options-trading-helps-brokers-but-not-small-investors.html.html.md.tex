Sections

SEARCH

\protect\hyperlink{site-content}{Skip to
content}\protect\hyperlink{site-index}{Skip to site index}

\href{https://www.nytimes3xbfgragh.onion/section/business}{Business}

\href{https://myaccount.nytimes3xbfgragh.onion/auth/login?response_type=cookie\&client_id=vi}{}

\href{https://www.nytimes3xbfgragh.onion/section/todayspaper}{Today's
Paper}

\href{/section/business}{Business}\textbar{}Growth in Options Trading
Helps Brokers but Not Small Investors

\url{https://nyti.ms/150nndZ}

\begin{itemize}
\item
\item
\item
\item
\item
\end{itemize}

Advertisement

\protect\hyperlink{after-top}{Continue reading the main story}

Supported by

\protect\hyperlink{after-sponsor}{Continue reading the main story}

\hypertarget{growth-in-options-trading-helps-brokers-but-not-small-investors}{%
\section{Growth in Options Trading Helps Brokers but Not Small
Investors}\label{growth-in-options-trading-helps-brokers-but-not-small-investors}}

By
\href{https://www.nytimes3xbfgragh.onion/by/nathaniel-popper}{Nathaniel
Popper}

\begin{itemize}
\item
  May 24, 2013
\item
  \begin{itemize}
  \item
  \item
  \item
  \item
  \item
  \end{itemize}
\end{itemize}

Some of the brokerage firms that helped pique American's interest in
stocks are now luring them into something much riskier: stock options.

As the stock market soars to new heights, E*Trade, Ameritrade and
Charles Schwab are advertising the potential rewards of options, which
give buyers the right to buy or sell stocks at predetermined prices in
the future. Options, like their cousins, futures, have traditionally
been the domain of Wall Street traders. But the brokerage firms say
futures and options can be profitable for ordinary investors, too --- a
claim that, while true, does not square with many investors' actual
experience.

``We're looking for newcomers who want to get serious,'' Schwab says on
its Web site.

While relatively little research has been done on the success ordinary
investors have in trading options, analysis done for The New York Times
by SigFig, a company that tracks 200,000 retail investors, showed that
people who traded options last year received only about one-fifth the
returns of people who did not trade options: 1.1 percent compared to 5.1
percent.

The brokerage firms do not release data about customers' trading, and
they are generally hesitant to detail the expansion of this business.
But it has clearly been an area of growth. An analysis of scattered data
from company filings and presentations indicates that derivatives
trading, which includes options, has risen at all the major firms since
the financial crisis of 2008, which left many Americans with big losses
in their investment portfolios.

At Ameritrade, which has been the most aggressive, derivatives trades
accounted for about 40 percent of all customer trades last year --- more
than double what it was just five years ago. A vast majority of those
trades were in options.

The growth has been a big help for the online brokers at a time when
stock trading has fallen. The commission on the average options trade is
more than twice that on the average stock trade, according to TD
Ameritrade's former treasurer, Michael Chochon.

``We're looking to continue to drive penetration in'' options and
futures, Ameritrade's chief financial officer, Bill Gerber, said in a
call with analysts in February.

But the results have been less of a clear victory for customers. Renaud
Piccinini, who monitored customer accounts for Ameritrade before he left
the company last year, said options could be used wisely in some
circumstances. But he said he saw investors taking up options trading
and ``blowing up'' on an almost daily basis. He said Ameritrade
carefully tracked the risks its customers were taking but did not warn
them until they were close to losing it all, if then.

``We knew that they were taking risky bets,'' Mr. Piccinini said. ``We
knew inside the firm, but there was resistance to sharing that with the
customer.''

Mr. Piccinini is now working with Mr. Chochon and two other former TD
Ameritrade employees to create a program for retail investors, known as
Prairie Smarts, that details the risks a prospective trade adds to a
portfolio.

Steve Quirk, who oversees active traders at TD Ameritrade, said the
former employees were criticizing the company to generate interest for
their firm.

Mr. Quirk said TD Ameritrade gave investors a wide array of tools to
gauge their risks, as well as significant education, before and after
they started trading options.

\includegraphics{https://static01.graylady3jvrrxbe.onion/images/2013/05/25/business/Options/Options-articleLarge.jpg?quality=75\&auto=webp\&disable=upscale}

``We hear from many, many clients that the more they understand about
all the products that are available, the better equipped they are to
deal with the market in any scenario,'' Mr. Quirk said.

The companies began their big push into this area after the financial
crisis, with the purchase of smaller brokerage houses that focused on
options. At E*Trade, filings indicate that options trades rose to 24
percent of all trades last year from about 17 percent in 2010, and the
total number of trades also increased.

Customers at all the brokers must take a number of steps before they are
permitted to begin trading, and they must attest that they have read an
official 186-page document laying out the risks of options. But almost
anyone can go through this process. And the brokers have broadened the
pool of potential customers by allowing investors to trade options in
their retirement accounts.

E*Trade's recent marketing material said: ``Every investor should learn
how options trading could benefit them.''

Options generally represent the right to buy or sell 100 shares of a
stock at some point in the future --- allowing for a small initial
investment that can lead to either big gains or big losses.

Investors have bought in for a variety of reasons. Some believe they
will be able to use the leveraged nature of options to supercharge their
returns and make up for losses suffered during the financial crisis.
Others see options as a way to insure their stock portfolios against
future losses.

But academic research suggests that on the whole, options traders do
worse than stock traders, who, in turn, have been shown in many studies
to underperform buy-and-hold investors. The
\href{http://papers.ssrn.com/sol3/papers.cfm?abstract_id=965810}{most
comprehensive study} looked at 68,000 Dutch retail investors. It found
that from 2000 to 2006, retail options traders lost an average of 4.5
percent each month, while people who just traded stocks lost 1.6
percent.

Daniel Dorn, a professor at Drexel University who has studied options
investors, said studies of American options traders had not found them
to be significantly more successful than the Dutch traders.

``You can very clearly say in the aggregate, this doesn't help
individual investor portfolios,'' he said.

The brokers said they guarded against customer losses by allowing only
wealthier and more experienced customers to proceed to more complex
trading strategies. Joe Vietri, the chief executive of Charles Schwab's
options subsidiary, OptionsXpress, said his company was careful about
monitoring clients because Schwab wanted to keep them as customers for
other parts of the business.

TD Ameritrade gives customer some tools that allow them to analyze the
risks of individual trades. But Mr. Chochon, the former Ameritrade
treasurer, said this was not enough to stop many customers from burning
through the money in their accounts. This put a strain on the company,
he said, because it necessitated expensive marketing campaigns to
capture new clients.

``The churn of those clients was really expensive because you had to
bring in a new customer,'' he said.

Mr. Quirk, at Ameritrade, denied that the company had trouble with
clients burning out.

Advertisement

\protect\hyperlink{after-bottom}{Continue reading the main story}

\hypertarget{site-index}{%
\subsection{Site Index}\label{site-index}}

\hypertarget{site-information-navigation}{%
\subsection{Site Information
Navigation}\label{site-information-navigation}}

\begin{itemize}
\tightlist
\item
  \href{https://help.nytimes3xbfgragh.onion/hc/en-us/articles/115014792127-Copyright-notice}{©~2020~The
  New York Times Company}
\end{itemize}

\begin{itemize}
\tightlist
\item
  \href{https://www.nytco.com/}{NYTCo}
\item
  \href{https://help.nytimes3xbfgragh.onion/hc/en-us/articles/115015385887-Contact-Us}{Contact
  Us}
\item
  \href{https://www.nytco.com/careers/}{Work with us}
\item
  \href{https://nytmediakit.com/}{Advertise}
\item
  \href{http://www.tbrandstudio.com/}{T Brand Studio}
\item
  \href{https://www.nytimes3xbfgragh.onion/privacy/cookie-policy\#how-do-i-manage-trackers}{Your
  Ad Choices}
\item
  \href{https://www.nytimes3xbfgragh.onion/privacy}{Privacy}
\item
  \href{https://help.nytimes3xbfgragh.onion/hc/en-us/articles/115014893428-Terms-of-service}{Terms
  of Service}
\item
  \href{https://help.nytimes3xbfgragh.onion/hc/en-us/articles/115014893968-Terms-of-sale}{Terms
  of Sale}
\item
  \href{https://spiderbites.nytimes3xbfgragh.onion}{Site Map}
\item
  \href{https://help.nytimes3xbfgragh.onion/hc/en-us}{Help}
\item
  \href{https://www.nytimes3xbfgragh.onion/subscription?campaignId=37WXW}{Subscriptions}
\end{itemize}
