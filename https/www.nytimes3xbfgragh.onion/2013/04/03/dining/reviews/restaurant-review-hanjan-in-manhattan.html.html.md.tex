Sections

SEARCH

\protect\hyperlink{site-content}{Skip to
content}\protect\hyperlink{site-index}{Skip to site index}

\href{https://www.nytimes3xbfgragh.onion/section/food}{Food}

\href{https://myaccount.nytimes3xbfgragh.onion/auth/login?response_type=cookie\&client_id=vi}{}

\href{https://www.nytimes3xbfgragh.onion/section/todayspaper}{Today's
Paper}

\href{/section/food}{Food}\textbar{}Splitting the Difference

\url{https://nyti.ms/17b9Y56}

\begin{itemize}
\item
\item
\item
\item
\item
\end{itemize}

Advertisement

\protect\hyperlink{after-top}{Continue reading the main story}

Supported by

\protect\hyperlink{after-sponsor}{Continue reading the main story}

\hypertarget{splitting-the-difference}{%
\section{Splitting the Difference}\label{splitting-the-difference}}

\includegraphics{https://static01.graylady3jvrrxbe.onion/images/2013/04/03/dining/03REST_SPAN/03REST_SPAN-articleLarge.jpg?quality=75\&auto=webp\&disable=upscale}

\begin{itemize}
\tightlist
\item
  Hanjan\\
  **NYT Critic's Pick ★★ Korean \$\$\$ 36 West 26th Street 212-206-7226
\end{itemize}

\href{http://www.opentable.com/single.aspx?ref=4201\&rid=169873}{Reserve
a Table}

When you make a reservation at an independently reviewed restaurant
through our site, we earn an affiliate commission.

By \href{https://www.nytimes3xbfgragh.onion/by/pete-wells}{Pete Wells}

\begin{itemize}
\item
  April 2, 2013
\item
  \begin{itemize}
  \item
  \item
  \item
  \item
  \item
  \end{itemize}
\end{itemize}

The pajeon at Hanjan did not look like any Korean pancake I'd ever seen.
It appeared to be a casual pile of squid tempura interlaced with a few
splinters of fried green stuff, but I took a knife to it and quartered
it anyway. Amazingly, the gold loops and strands held together as they
traveled from the dish of spicy dipping sauce to my mouth.

As I ate, I understood that Hanjan's chef, Hooni Kim, had accomplished
the most delicious kind of engineering feat, minimizing the layer of
rice starch that makes up the bulk of a traditional pajeon and
maximizing not just the crisp fried bits but also the flavor of
scallions and wonderfully fresh squid. At the same time, his pajeon held
on to the essential taste of Korean cuisine. This taller, airier pajeon
was nowhere near as filling as the original, but I didn't mind. Planning
ahead, I had ordered two.

\href{http://www.hanjan26.com/}{Hanjan} opened in December on West 26th
Street, its recessed entrance easy to miss alongside rowdy neighbors
like Hill Country and the Hog Pit. Like Mr. Kim's slightly older
restaurant,
\href{http://www.nytimes3xbfgragh.onion/2011/08/17/dining/reviews/danji-manhattan-restaurant-review.html?pagewanted=all\&_r=0}{Danji},
Hanjan has a menu divided between traditional Korean dishes (the pajeon)
and new ideas (slices of raw wild salmon that you wrap around salad
greens in a spicy sesame dressing). And once again, the cover versions
and the original compositions are so much in touch with the spirit of
Korean cooking that it can be hard to tell which is which.

Hanjan does not cast out any obvious chum for food bloggers the way
Danji did with its
\href{http://www.nytimes3xbfgragh.onion/2011/09/18/magazine/eat-the-bulgogi-slider-is-a-delicious-curveball.html}{bulgogi
sliders} and bacon paella. Mr. Kim may be more confident in his cooking
this time around, or more certain that New Yorkers will get it. On a
menu of mostly small dishes, one of the two substantial enough to anchor
a meal is a large bowl of cod roe in a cloudy seafood broth that
contains so much chile paste, it's pink. Both the esoteric main
ingredient and the fearless seasoning in this remarkable stew suggest
that Mr. Kim trusts his audience, and vice versa.

Certainly, the relationship has progressed to the point where he can
name one of Hanjan's dishes fresh killed chicken wings.
\href{http://dinersjournal.blogs.nytimes3xbfgragh.onion/2013/03/20/at-hanjan-its-no-secret-when-the-chicken-died/}{Just
how fresh}, you may wonder, tilting an ear toward the kitchen to listen
for muffled squawks. Not quite as fresh as that; the birds are
slaughtered off site earlier in the day that they are served. The object
is to obtain poultry with a clean and full flavor, which Mr. Kim simply
underlines by marinating the wings in sake and soy and grilling them.

\href{https://www.nytimes3xbfgragh.onion/slideshow/2013/04/03/dining/20130403-REST.html}{}

\hypertarget{hanjan}{%
\subsection{Hanjan}\label{hanjan}}

7 Photos

View Slide Show ›

\includegraphics{https://static01.graylady3jvrrxbe.onion/images/2013/04/03/dining/20130403-REST-slide-HUM3/20130403-REST-slide-HUM3-articleLarge.jpg?quality=75\&auto=webp\&disable=upscale}

Sasha Maslov for The New York Times

Other parts of those recently graduated birds end up skewered and
grilled, yakitori style, with minimal seasoning. You need to love the
flavor of extremely fresh chicken to love most of these skewers,
especially the very chewy gizzards. The crunchy accordion folds of
grilled skin are probably universally appealing, though. Two of my
guests, at least, declared them the best thing on the table.

A minute later, they claimed to have found something better. All my
meals at Hanjan were like that, with new dishes eclipsing the memory of
what had come before. ``That's my favorite of the night,'' somebody
would say, chopsticks pinched around a fried sandwich of shrimp and pork
paste between shiso leaves. A few minutes later, somebody else would
crown the grilled mackerel fillet, its skin gleaming like candy under a
glassy soy glaze, accompanied by a lemon and grated daikon. Next the
torch was passed to meaty slices of pig trotter braised in soy and
smeared with a far-from-tame paste of fermented shrimp and soy beans.

In the rush to declare winners, a few also-rans were soon forgotten. To
be a contender in a town full of good Korean fried chicken, Hanjan's
needs to be crisper than it was the night I had it. A tiny portion of
noodles with spicy squid vanished without impact. Thin slices of fish
cakes bobbed around in a nearly clear seafood broth that was very, very
good, but the dish wasn't substantial enough for a restaurant that is,
essentially, a Korean pub.

Mr. Kim's small-plates approach leads to one significant miscalculation.
Hanjan charges \$5 for kimchi and \$4 for cold appetizers called namul.
Order both and you have a small but high-quality banchan spread, which
most Korean restaurants bring without charge or request. By putting a
price on banchan, Mr. Kim makes it optional, just another tapas plate.
Without it, the whole spirit of the meal changes.

Lines of two-tops and four-tops run along opposite walls; in the center
is a long communal table that might feel like trend-chasing in another
restaurant but here feels like an invitation to put your elbows down and
have another bottle of sake. After 8 p.m. you will need it, because the
din from hungry-eyed people waiting at the bar and the customers at
their tables shouting to one another that everybody needs to try the
pork fat rice cakes (they're right) will be bouncing off the concrete
walls and the steel ceilings.

After a while, you may get curious about the cloudy, oatmeal-colored
drink served in frosty mugs. It is makgeolli, a yeasty rice brew with
the soft sweetness of cantaloupe. By 10 or so, the crowds may thin a bit
and there might be a seat at the bar. This is good because that is when
the ramen broth is finally ready, after bubbling in a pot for half a day
with the bones of fish and pork and chicken. Ramen is listed on the menu
as ramyun, one of the large plates, which implies sharing. I recommend
eating it alone at the bar, where you can have all four slices of sweet
pork to yourself and make as much noise as you want slurping the bouncy,
squiggly noodles out of the steaming and chile-hot soup.

The ramen shows up on the modern side of the menu, even though nothing
about its long-simmered, bone-rich flavor tastes 21st century. But Mr.
Kim is now the city's leading interpreter of Korean cuisine, and if he
says the dish is modern, the rest of us will have to believe him.

Advertisement

\protect\hyperlink{after-bottom}{Continue reading the main story}

\hypertarget{site-index}{%
\subsection{Site Index}\label{site-index}}

\hypertarget{site-information-navigation}{%
\subsection{Site Information
Navigation}\label{site-information-navigation}}

\begin{itemize}
\tightlist
\item
  \href{https://help.nytimes3xbfgragh.onion/hc/en-us/articles/115014792127-Copyright-notice}{©~2020~The
  New York Times Company}
\end{itemize}

\begin{itemize}
\tightlist
\item
  \href{https://www.nytco.com/}{NYTCo}
\item
  \href{https://help.nytimes3xbfgragh.onion/hc/en-us/articles/115015385887-Contact-Us}{Contact
  Us}
\item
  \href{https://www.nytco.com/careers/}{Work with us}
\item
  \href{https://nytmediakit.com/}{Advertise}
\item
  \href{http://www.tbrandstudio.com/}{T Brand Studio}
\item
  \href{https://www.nytimes3xbfgragh.onion/privacy/cookie-policy\#how-do-i-manage-trackers}{Your
  Ad Choices}
\item
  \href{https://www.nytimes3xbfgragh.onion/privacy}{Privacy}
\item
  \href{https://help.nytimes3xbfgragh.onion/hc/en-us/articles/115014893428-Terms-of-service}{Terms
  of Service}
\item
  \href{https://help.nytimes3xbfgragh.onion/hc/en-us/articles/115014893968-Terms-of-sale}{Terms
  of Sale}
\item
  \href{https://spiderbites.nytimes3xbfgragh.onion}{Site Map}
\item
  \href{https://help.nytimes3xbfgragh.onion/hc/en-us}{Help}
\item
  \href{https://www.nytimes3xbfgragh.onion/subscription?campaignId=37WXW}{Subscriptions}
\end{itemize}
