Sections

SEARCH

\protect\hyperlink{site-content}{Skip to
content}\protect\hyperlink{site-index}{Skip to site index}

\href{https://www.nytimes3xbfgragh.onion/section/movies}{Movies}

\href{https://myaccount.nytimes3xbfgragh.onion/auth/login?response_type=cookie\&client_id=vi}{}

\href{https://www.nytimes3xbfgragh.onion/section/todayspaper}{Today's
Paper}

\href{/section/movies}{Movies}\textbar{}Who Gets to Play the Transgender
Part?

\url{https://nyti.ms/1Qb9yRA}

\begin{itemize}
\item
\item
\item
\item
\item
\end{itemize}

Advertisement

\protect\hyperlink{after-top}{Continue reading the main story}

Supported by

\protect\hyperlink{after-sponsor}{Continue reading the main story}

\hypertarget{who-gets-to-play-the-transgender-part}{%
\section{Who Gets to Play the Transgender
Part?}\label{who-gets-to-play-the-transgender-part}}

\includegraphics{https://static01.graylady3jvrrxbe.onion/images/2015/09/04/arts/04TRANSGENDER/04TRANSGENDER-articleLarge.jpg?quality=75\&auto=webp\&disable=upscale}

By \href{http://www.nytimes3xbfgragh.onion/by/brooks-barnes}{Brooks
Barnes}

\begin{itemize}
\item
  Sept. 3, 2015
\item
  \begin{itemize}
  \item
  \item
  \item
  \item
  \item
  \end{itemize}
\end{itemize}

LOS ANGELES --- More than at any time in its history, Hollywood is under
enormous pressure to find performers who match the racial and ethnic
traits of characters.

Ridley Scott was
\href{http://mashable.com/2014/12/11/exodus-movie-racist/}{harshly
criticized} for using non-Egyptian actors to play Egyptians in
``Exodus.'' The director Cameron Crowe faced an online mob for casting
Emma Stone as an Asian-American woman in ``Aloha.'' (He ultimately
\href{http://artsbeat.blogs.nytimes3xbfgragh.onion/2015/06/03/emma-stone-as-asian-american-cameron-crowe-apologizes/?_r=0}{apologized}.)
When Warner Bros. announced that Rooney Mara would play Tiger Lily in
its forthcoming ``Pan,'' the studio was served with
\href{http://www.cinemablend.com/new/Online-Petition-Launched-Protest-Rooney-Mara-Casting-Pan-42175.html}{a
petition} headlined ``Stop Casting White Actors to Play People of
Color!''

But does it remain acceptable --- at this Caitlyn Jenner and Laverne Cox
moment --- for non-transgender actors and actresses to play transgender
characters?

Hollywood is about to find out. Two high-profile new films, each with
Oscar aspirations, star performers who are not transgender in major
transgender roles. On Saturday, ``The Danish Girl,'' with
\href{http://www.nytimes3xbfgragh.onion/2012/12/16/fashion/eddie-redmayne-of-les-miserables-the-talk-of-the-town.html}{Eddie
Redmayne} in the title role, will have its world premiere at the Venice
Film Festival. ``About Ray,'' starring Elle Fanning as a teenager in
early gender transition, arrives next Saturday at the Toronto
International Film Festival.

Both casting decisions reflect what remains the dominant view in movies
and television. A dozen casting directors, producers, network
programmers and studio executives said in interviews that transgender
roles were best filled by finding the best actor or actress, regardless
of gender identity. In other words, acting is acting. Besides, they
asked, don't transgender performers want to be considered for
non-transgender roles?

Peter Saraf, a producer of ``About Ray,'' which is to be released in
theaters on Sept. 18, said emphatically at the start of an interview
that Hollywood needed to work ``a lot harder to create opportunities for
trans actors to play any kind of role.'' That said, he defended the
casting of Ms. Fanning.

``We try to make the strongest creative choices we can,'' he said.
``Elle, who is one of the most exciting and extraordinary actresses
working today, was passionate about the role, and we had the confidence
that she could carry a movie.''

But some advocates believe that it is flatly offensive for a
non-transgender performer to play a transgender part. Jos Truitt,
executive director of development at
\href{http://feministing.com/2014/01/13/the-golden-globes-give-jared-leto-an-award-for-playing-a-trans-woman-because-hollywood-is-terrible/}{Feministing},
an online network, put it this way: When actors like Mr. Redmayne and
Jared Leto (who won an Oscar for his portrayal of Rayon in ``Dallas
Buyers Club'') play these roles, it perpetuates ``the stereotype that
trans women are just men in drag.''

\href{https://www.nytimes3xbfgragh.onion/interactive/2015/opinion/transgender-today-your-stories.html}{}

\includegraphics{https://static01.graylady3jvrrxbe.onion/images/2015/06/03/opinion/transgendertoday/transgendertoday-videoLarge.jpg}

\hypertarget{transgender-lives-your-stories}{%
\subsection{Transgender Lives: Your
Stories}\label{transgender-lives-your-stories}}

As part of a series of editorials about transgender experiences, we are
featuring personal stories that reflect the strength, diversity and
challenges of the community.

At least in some corners of Hollywood, a similar position is gaining
steam.

``At this moment in time, especially, I think this industry has a
responsibility to put trans actors in trans roles,'' said Sean Baker,
who directed
``\href{http://www.nytimes3xbfgragh.onion/2015/07/10/movies/review-tangerine-a-madcap-buddy-picture-about-transgender-prostitutes.html}{Tangerine},''
an independent film that was released in July and starred two
transgender actresses. ``To not do it seems very wrong in my eyes. There
is plenty of trans talent out there.''

Adding complexity to the matter, Glaad, which aggressively monitors
Hollywood's depictions of gay, lesbian, bisexual and transgender
characters, has taken a more nuanced stance.
\href{http://www.nytimes3xbfgragh.onion/2014/09/27/arts/television/transparent-with-jeffrey-tambor-begins-on-amazon.html}{Jeffrey
Tambor} plays a retired professor beginning a transition on Amazon's
``Transparent,'' and in March that series won a Glaad Media Award, a
prize that lists ``fair, accurate and inclusive representations'' among
its criteria.

``There is a consensus that trans actors bring a certain authenticity to
a trans role and that trans actors should also have the opportunity to
play non-trans characters,'' said
\href{http://www.glaad.org/about/staff/nickadams}{Nick Adams}, who leads
Glaad's transgender efforts. Beyond that, Mr. Adams said, there is
little agreement among advocates, with some supporting Ms. Truitt's
hard-line position and others allowing that ``in certain circumstances,
a non-trans person can play a trans character if they do their homework
and learn from trans people, as Jeffrey Tambor did.''

For the most part, the transgender stories that Hollywood is telling
focus on early transition, perhaps because that process can be mined
easily for drama, Mr. Adams noted. That focus also gives studios cover
to cast non-transgender performers; Mr. Redmayne must appear as a man at
the beginning of ``The Danish Girl,'' for instance. (Glaad is pushing
Hollywood to focus less on transition stories.)

Movies like ``About Ray'' and ``The Danish Girl'' also face business
realities.

``I'm embarrassed to say this, because I do strongly believe that we
should be casting transgender performers in these parts --- it matters
--- but often you don't even seriously consider them, because the studio
needs a name for financing or marketing reasons,'' said one leading
casting director, speaking on the condition of anonymity because, she
said, she considered the topic ``radioactive.''

Ms. Fanning and Mr. Redmayne were not available for interviews,
according to their representatives. None of the filmmakers or studio
executives behind ``The Danish Girl'' would discuss Mr. Redmayne's
casting.

``The Danish Girl'' is the more high-profile film, in part because it
comes from an Academy Award-winning director, Tom Hooper (``The King's
Speech''), and stars the reigning best actor; Mr. Redmayne won an Oscar
in February for his portrayal of Stephen Hawking in ``The Theory of
Everything.'' Mr. Hooper approached Mr. Redmayne for ``The Danish Girl''
in 2011, when they were working on
``\href{http://www.nytimes3xbfgragh.onion/2012/12/25/movies/les-miserables-stars-anne-hathaway-and-hugh-jackman.html}{Les
Misérables}.''

Mr. Hooper recently told Screen Daily that he sensed ``a certain gender
fluidity'' in Mr. Redmayne. ``The Danish Girl,'' set for release in
theaters on Nov. 27 by the Universal-owned Focus Features, tells the
true story of a Copenhagen artist who underwent sex reassignment surgery
in 1930. It was one of the first such efforts.

In the time it took for ``The Danish Girl'' to be made, however,
transgender issues have leaped to the cultural forefront.

\includegraphics{https://static01.graylady3jvrrxbe.onion/images/2015/09/04/arts/04TRANSGENDER4/04TRANSGENDERjp4-articleLarge.jpg?quality=75\&auto=webp\&disable=upscale}

Ms. Cox, the transgender actress who plays Sophia Burset on ``Orange Is
the New Black,'' appeared on the cover of Time last year. President
Obama in January
\href{http://www.cnn.com/2015/01/20/politics/obama-transgender-sotu/}{turned
heads} by using the word ``transgender'' in his State of the Union
address. ``Transparent'' won best comedy series in front of 19 million
viewers at the last Golden Globe Awards.

And then came Ms. Jenner, who spoke about her transition in a prime-time
ABC News special, subsequently
\href{http://www.nytimes3xbfgragh.onion/2015/06/02/business/media/jenner-reveals-new-name-in-vanity-fair-article.html}{posing
for the cover} of Vanity Fair.

At the same time, pushback on casting decisions large and small has
become harder for Hollywood to ignore. Just a few years ago, protests of
insensitivity --- over the hiring of Johnny Depp to play Tonto in ``The
Lone Ranger,'' for instance, or giving Jake Gyllenhaal the lead in
``Prince of Persia: The Sands of Time'' --- were barely blips on the
movie industry's radar. But fans, advocacy groups and rank-and-file
critics have grown more sophisticated in their use of social media to
organize and voice disappointment.

In today's Internet culture, the tendency is to shoot first and ask
questions later. Nuance doesn't always matter. Tiger Lily, who exists in
the public imagination (and not in a particularly sensitive way) as a
Native American woman, was rewritten to be of a nonspecific race, Ms.
Mara
\href{http://www.cinemablend.com/new/Why-White-Tiger-Lily-Works-According-Rooney-Mara-70870.html}{ultimately
explained}. ``Pan'' is scheduled for release on Oct. 9.

Perhaps to get ahead of any blowback, Focus recently had Mr. Redmayne
explain to Out magazine how he met with many transgender women to
educate himself. ``Gosh, it's delicate,'' he said in
\href{http://www.out.com/movies/2015/8/11/eddie-redmayne-education}{that
interview}. ``And complicated.''

Part of the frustration with Hollywood among transgender people involves
the lack of transgender characters, even with heightened cultural
attention. Of the 161 mainstream and art house films that Glaad tracked
in its last Studio Responsibility Index, released in April, none had a
transgender character. ``The list of mainstream films that have depicted
transgender people as multifaceted or even recognizable human beings
remains tragically short,'' Glaad wrote in the report.

Television, which moves faster as a business and does not face the same
pressure to cast stars, is
\href{http://www.nytimes3xbfgragh.onion/2015/06/21/movies/broadening-a-transgender-tale-that-has-only-just-begun.html?_r=0}{doing
a better job}. ``Transparent,'' ``Orange Is the New Black'' and
``Sense8'' --- notably all from streaming services --- prominently
feature transgender characters and transgender actors and actresses.
Glaad gives particular credit to ``The Fosters,'' an ABC Family series
that features the transgender actor
\href{http://www.glaad.org/blog/fosters-actor-tom-phelan-talks-glaad-about-playing-one-tvs-new-trans-characters}{Tom
Phelan} as a transgender teenager.

Glaad is to release a report in November that assesses the television
landscape from a transgender perspective. On Thursday, it released its
annual Network Responsibility Index, which focuses on ``the quantity and
quality'' of images of gay, lesbian, bisexual and transgender people on
television, and used the platform to push for more transgender
representation. It told CW, for instance, that it hopes to ``see a
transgender character make an appearance very soon.''

For transgender actors and actresses, that is encouraging --- if
networks seek them out for any resulting roles.

``Because I am a trans woman in 2015, there are opportunities that
wouldn't have existed for me three years ago,'' said Hari Nef, who will
join the cast of ``Transparent'' when it returns in December. ``But
Hollywood still seems very wary. There is not a rush of casting agents
headed our way. Let's hope that changes. I'm right here!''

Advertisement

\protect\hyperlink{after-bottom}{Continue reading the main story}

\hypertarget{site-index}{%
\subsection{Site Index}\label{site-index}}

\hypertarget{site-information-navigation}{%
\subsection{Site Information
Navigation}\label{site-information-navigation}}

\begin{itemize}
\tightlist
\item
  \href{https://help.nytimes3xbfgragh.onion/hc/en-us/articles/115014792127-Copyright-notice}{©~2020~The
  New York Times Company}
\end{itemize}

\begin{itemize}
\tightlist
\item
  \href{https://www.nytco.com/}{NYTCo}
\item
  \href{https://help.nytimes3xbfgragh.onion/hc/en-us/articles/115015385887-Contact-Us}{Contact
  Us}
\item
  \href{https://www.nytco.com/careers/}{Work with us}
\item
  \href{https://nytmediakit.com/}{Advertise}
\item
  \href{http://www.tbrandstudio.com/}{T Brand Studio}
\item
  \href{https://www.nytimes3xbfgragh.onion/privacy/cookie-policy\#how-do-i-manage-trackers}{Your
  Ad Choices}
\item
  \href{https://www.nytimes3xbfgragh.onion/privacy}{Privacy}
\item
  \href{https://help.nytimes3xbfgragh.onion/hc/en-us/articles/115014893428-Terms-of-service}{Terms
  of Service}
\item
  \href{https://help.nytimes3xbfgragh.onion/hc/en-us/articles/115014893968-Terms-of-sale}{Terms
  of Sale}
\item
  \href{https://spiderbites.nytimes3xbfgragh.onion}{Site Map}
\item
  \href{https://help.nytimes3xbfgragh.onion/hc/en-us}{Help}
\item
  \href{https://www.nytimes3xbfgragh.onion/subscription?campaignId=37WXW}{Subscriptions}
\end{itemize}
