Sections

SEARCH

\protect\hyperlink{site-content}{Skip to
content}\protect\hyperlink{site-index}{Skip to site index}

\href{https://www.nytimes3xbfgragh.onion/section/food}{Food}

\href{https://myaccount.nytimes3xbfgragh.onion/auth/login?response_type=cookie\&client_id=vi}{}

\href{https://www.nytimes3xbfgragh.onion/section/todayspaper}{Today's
Paper}

\href{/section/food}{Food}\textbar{}René Redzepi Plans to Close Noma and
Reopen It as an Urban Farm

\url{https://nyti.ms/1ibpGab}

\begin{itemize}
\item
\item
\item
\item
\item
\item
\end{itemize}

Advertisement

\protect\hyperlink{after-top}{Continue reading the main story}

Supported by

\protect\hyperlink{after-sponsor}{Continue reading the main story}

\hypertarget{renuxe9-redzepi-plans-to-close-noma-and-reopen-it-as-an-urban-farm}{%
\section{René Redzepi Plans to Close Noma and Reopen It as an Urban
Farm}\label{renuxe9-redzepi-plans-to-close-noma-and-reopen-it-as-an-urban-farm}}

\includegraphics{https://static01.graylady3jvrrxbe.onion/images/2015/09/16/dining/16NOMA1/16NOMA1-articleLarge.jpg?quality=75\&auto=webp\&disable=upscale}

By \href{https://www.nytimes3xbfgragh.onion/by/jeff-gordinier}{Jeff
Gordinier}

\begin{itemize}
\item
  Sept. 14, 2015
\item
  \begin{itemize}
  \item
  \item
  \item
  \item
  \item
  \item
  \end{itemize}
\end{itemize}

COPENHAGEN --- ``Welcome to the new Noma,'' the chef René Redzepi said
on a bright summer day. ``This is it.''

Mr. Redzepi, 37, the godfather of the New Nordic movement and the chef
at Noma, arguably the world's most influential restaurant at the moment,
was standing outside what looked like an auditorium-size crack den. Used
spray-paint cans lay in heaps amid the weeds of an abandoned lot. Street
art covered the walls of an empty warehouse; inside, teenagers rumbled
around on skateboards.

World-class culinary destination? The site, right outside the ragged
border of this city's freewheeling Christiania neighborhood, seemed more
like the Four Seasons after an apocalypse.

\includegraphics{https://static01.graylady3jvrrxbe.onion/images/2015/09/16/dining/16NOMA2/16NOMA2-articleLarge.jpg?quality=75\&auto=webp\&disable=upscale}

But Mr. Redzepi envisioned something else as he climbed a staircase to a
tar-papered roof and gazed out at a lake on the edge of the property. In
what qualifies as a wildly risky roll of the dice, he plans to close
Noma after a final service on New Year's Eve in 2016. He hopes to reopen
for business in 2017 with a new menu and a new mission.

As a crucial part of that, he wants to transform this decrepit patch of
land into a state-of-the-art urban farm, with Noma at its center.

``It makes sense to do it here,'' he said, despite visual evidence to
the contrary. ``It makes sense to have your own farm, as a restaurant of
this caliber.'' His plans are nothing if not ambitious. He will put a
greenhouse on the roof. He will dig out the dank old asphalt lot and
truck in fresh soil. He wants part of the farm to float.

Image

The site, currently covered in graffiti and skate ramps, will become a
state-of-the-art urban farm.Credit...Laerke Posselt for The New York
Times

``We'll build a raft, and we'll put a huge field on the raft,'' he said.
``We need a full-time farmer with a team.''

If the concept carries a slight echo of that dream of building an opera
house in Werner Herzog's film ``Fitzcarraldo,'' Mr. Redzepi is fully
aware of the gamble. ``It really, really, really, really makes me
nervous,'' he said. ``I'm not afraid. But it does make me nervous.''

Dan Barber, a New York chef who has dug deep into farming, said one big
challenge for Mr. Redzepi could come from the unpredictability of
agriculture. ``Does that carry risks with it? Sure,'' Mr. Barber said.

Image

Mr. Redzepi believes Noma is ready for a dramatic evolution. His plans
for the new site include a rooftop greenhouse and a farm that partially
floats on the lake abutting the property.Credit...Laerke Posselt for The
New York Times

When a restaurant has its own farm, as Mr. Barber does at
\href{https://www.bluehillfarm.com/dine/stone-barns}{Blue Hill at Stone
Barns} in the Lower Hudson Valley, it can signify a chef's desire for
``the ultimate control'' of ingredients, he said, but ``obviously the
best kind of farming is the lack of control,'' and cooks have to learn
to work with whatever the earth produces.

``I continue to be in awe of the guy,'' he said of Mr. Redzepi. ``It
takes a leader in the field to change the culture.''

The changes at Noma are not driven by necessity. There has not been a
rent increase at the original location; business remains brisk. Mr.
Redzepi simply believes that the restaurant, where he has led the
kitchen for 12 years, is ready for a drastic evolution.

Image

Chefs gather in the Noma kitchen for Saturday Night
Projects.Credit...Laerke Posselt for The New York Times

In his mind, Noma hasn't even crawled out of its infancy. ``We're just
finding our way,'' he said. ``Even though it has been successful, even
though it has had media attention and all that.'' Lately, he has been
asking himself broad existential questions about what it means to be a
local restaurant in the Nordic region. ``What are we?'' he said. ``And
how do we progress?''

To illustrate all this as he showed off the property, he grabbed a
pebble and scratched out the number 12 in the dirt. Then he added a
zero, conveying the notion that Noma could last for a century or more.
``We should make decisions that make this evolution last for 912
years,'' he said.

As for the next two years, he's already committed to giving the menu a
radical shake-up. Over time, he has become less and less sure that it
makes sense for customers to pass through the traditional stages of a
tasting menu, from small nibbles to a slab of meat, culminating in
sweets and coffee. ``We've allowed the format of a tasting menu to
dictate what we cook,'' he said.

Image

The young cooks are asked to put together their own dishes for Mr.
Redzepi's inspection and analysis.Credit...Laerke Posselt for The New
York Times

He intends to replace that predictable progression with a more fervent
adherence to seasonality. In the fall, then, Noma's menu will focus only
on wild game (from goose to moose) and foraged autumnal ingredients like
mushrooms and forest berries. In the winter, when ``the waters are
ice-cold and some of the fish have bellies full of roe,'' as he put it,
Noma will mutate into a seafood restaurant.

Spring and summer? ``The world turns green,'' he said. ``And so will the
menu.'' In an expectation-thwarting move, during those months Noma will
become a fully vegetarian restaurant, with much of the bounty ostensibly
coming from the farm he wants to conjure up.

``It's huge,'' he said. ``How are you going to give a bowl of spinach
the same pleasure that a steak gives? A richness of flavor: That is
something that we will deal with.'' (Even the tableware will shift with
each season.)

Mr. Redzepi, who seems to crave change the way most people crave that
steak, said the transplanting is only one of the shifts he has in store
for the months ahead.

From late December to the middle of **** next April, the restaurant
staff will be relocating to Sydney, Australia, to see what happens when
the Noma approach is applied to Australian ingredients. (It undertook a
similar experiment in Japan this year, and the Japanese reverence for
the seasons clearly had a deep influence on Mr. Redzepi's thinking.
``It's as if everything they eat is at the right moment,'' he said.)

For the first time in his career, Mr. Redzepi is forming a partnership
to open a second restaurant, a more casual enterprise in Copenhagen with
the chef Kristian Baumann heading the kitchen, and he has lured the
Irish chef Trevor Moran to leave the Catbird Seat in Nashville and come
back to Noma (where he worked for four years) to help lead the next
wave.

Evidence suggests that Mr. Redzepi's employees are accustomed to hearing
their captain ordering them to steer for the high seas. Every week at
Noma, after the final high-pressure dinner service, they gather in the
kitchen for Saturday Night Projects, a public and semi-gladiatorial
face-off at which young cooks are asked to put together their own dishes
for Mr. Redzepi's inspection and analysis. He even broadcasts it via
Periscope.

``It's like the Google of restaurants,'' said
\href{http://www.nytimes3xbfgragh.onion/2014/11/05/dining/malcolm-livingston-ii-prepares-to-move-to-noma-from-wd-50-and-the-bronx-.html}{Malcolm
Livingston II}, who moved from New York last year to become Noma's
pastry chef. ``That's what keeps it exciting: It stays fresh.''

Over the years, Noma has pioneered approaches to fermentation, foraging,
aging and even cooking with insects. ``You're taking risks every time
you move forward,'' said the head chef, Daniel Giusti, who grew up in
New Jersey. But a big leap into agriculture could be the riskiest move
of all.

``That's where the challenge will come in,'' Mr. Giusti said. ``Are we
growing the right stuff? Is it good? I know René. It's going to have to
be great.''

Technically, Noma's new home lies in the center of the city, even though
it feels like some faraway vegetation-tangled edge. Mr. Redzepi may be
seeking wilder pastures at the right time. A new walkable bridge will
soon connect Nyhavn, one of the most tourist-packed parts of Copenhagen,
to the docklike zone that currently houses Noma and its fermentation
lab. As often happens in New York City, gentrification is surging in.
Change is in the air.

``Of course we could just keep continuing, just stay put and do what we
do there,'' Mr. Redzepi said. ``But I genuinely think that we won't
progress.'' He seemed rapt as he stood among the weeds and broken glass,
as if admiring the new Noma in its finished state. ``I have yet to meet
anyone who thinks this is a stupid thing,'' he said.

He pondered this observation for a second or two. Then he got antsy, as
he often does. ``I can't stand still like this,'' he said.

Advertisement

\protect\hyperlink{after-bottom}{Continue reading the main story}

\hypertarget{site-index}{%
\subsection{Site Index}\label{site-index}}

\hypertarget{site-information-navigation}{%
\subsection{Site Information
Navigation}\label{site-information-navigation}}

\begin{itemize}
\tightlist
\item
  \href{https://help.nytimes3xbfgragh.onion/hc/en-us/articles/115014792127-Copyright-notice}{©~2020~The
  New York Times Company}
\end{itemize}

\begin{itemize}
\tightlist
\item
  \href{https://www.nytco.com/}{NYTCo}
\item
  \href{https://help.nytimes3xbfgragh.onion/hc/en-us/articles/115015385887-Contact-Us}{Contact
  Us}
\item
  \href{https://www.nytco.com/careers/}{Work with us}
\item
  \href{https://nytmediakit.com/}{Advertise}
\item
  \href{http://www.tbrandstudio.com/}{T Brand Studio}
\item
  \href{https://www.nytimes3xbfgragh.onion/privacy/cookie-policy\#how-do-i-manage-trackers}{Your
  Ad Choices}
\item
  \href{https://www.nytimes3xbfgragh.onion/privacy}{Privacy}
\item
  \href{https://help.nytimes3xbfgragh.onion/hc/en-us/articles/115014893428-Terms-of-service}{Terms
  of Service}
\item
  \href{https://help.nytimes3xbfgragh.onion/hc/en-us/articles/115014893968-Terms-of-sale}{Terms
  of Sale}
\item
  \href{https://spiderbites.nytimes3xbfgragh.onion}{Site Map}
\item
  \href{https://help.nytimes3xbfgragh.onion/hc/en-us}{Help}
\item
  \href{https://www.nytimes3xbfgragh.onion/subscription?campaignId=37WXW}{Subscriptions}
\end{itemize}
