Sections

SEARCH

\protect\hyperlink{site-content}{Skip to
content}\protect\hyperlink{site-index}{Skip to site index}

\href{https://myaccount.nytimes3xbfgragh.onion/auth/login?response_type=cookie\&client_id=vi}{}

\href{https://www.nytimes3xbfgragh.onion/section/todayspaper}{Today's
Paper}

\href{/section/upshot}{The Upshot}\textbar{}How Presidential Campaigns
Became Two-Year Marathons

\url{https://nyti.ms/1NRkZ3p}

\begin{itemize}
\item
\item
\item
\item
\item
\item
\end{itemize}

Advertisement

\protect\hyperlink{after-top}{Continue reading the main story}

Supported by

\protect\hyperlink{after-sponsor}{Continue reading the main story}

Upshot

Road to 2016

\hypertarget{how-presidential-campaigns-became-two-year-marathons}{%
\section{How Presidential Campaigns Became Two-Year
Marathons}\label{how-presidential-campaigns-became-two-year-marathons}}

By \href{https://www.nytimes3xbfgragh.onion/by/alicia-parlapiano}{Alicia
Parlapiano}

\begin{itemize}
\item
  April 16, 2015
\item
  \begin{itemize}
  \item
  \item
  \item
  \item
  \item
  \item
  \end{itemize}
\end{itemize}

The comedian John Oliver doesn't think the length of the American
presidential campaign is funny. ``I have no interest whatsoever in the
2016 election, at the start of 2015,'' he recently told reporters.
``There's a time and a place for that, and it's in 2016.''

Yet 19 months before Election Day, more than a
\href{http://www.nytimes3xbfgragh.onion/interactive/2015/us/politics/2016-presidential-candidates.html}{dozen
presidential hopefuls} have made serious moves toward a run. Media
coverage has followed along with equal intensity.

It hasn't always been this way, at least not out in the open. The public
process has slowly lengthened in recent decades --- a product of rule
changes that spurred the adoption of primaries and the competition among
states for influence. The length of the campaigns has had the effect of
increasing the amount of money required to participate.

It's easy to identify the problems associated with a drawn-out campaign
--- boredom and a focus on the trivial, among them --- but there are
several benefits. For one, the process is more democratic than it used
to be, with more people having a say in who becomes the nominee.

In the country's early days, members of Congress chose the presidential
nominees. By the mid-19th century, the process had moved to back-room
wrangling at national party conventions, generally held in June of an
election year.

It wasn't until after World War II that presidential primary elections
began to catch on, and they became a way for dark-horse candidates to
prove their viability.

John F. Kennedy announced his candidacy for president on Jan. 2, 1960,
11 months before the general election. Between March 8 and June 7 of
that year, only 16 states held Democratic primaries, and Mr. Kennedy
entered and won seven of them. His victory in West Virginia, a
predominately Protestant state, signaled to Democrats that a Catholic
could have a chance in November, and delegates
\href{http://timesmachine.nytimes3xbfgragh.onion/timesmachine/1960/07/14/99949548.html?pageNumber=1}{selected
him overwhelmingly} during the first ballot at the party's convention in
July. (Conventions started going later into the summer after World War
II.)

The start of campaign season began to creep even earlier. A turning
point came when Democrats changed their nomination rules after their
\href{http://timesmachine.nytimes3xbfgragh.onion/timesmachine/1968/08/28/76967975.html?pageNumber=1}{tumultuous
1968 convention}, encouraging more states to hold primaries.

In the 1972 cycle, Iowa moved its caucuses to January, in part because
\href{http://www.nytimes3xbfgragh.onion/1988/01/25/us/the-election-process-iowa-s-weighty-caucuses-significance-by-accident.html}{Democratic
officials there needed time} between precinct caucuses and their state
convention in May to print paperwork on an old mimeograph machine. In
1976, the relatively unknown former governor of Georgia, Jimmy Carter,
gained a foothold by spending significant time early in the state.
Future candidates took note.

In the late 1980s, other states began to move their primary dates
earlier to gain influence, in a practice known as frontloading.
Compressing the primary calendar into those early months of the year
required candidates to have more resources at the ready earlier in the
game. By 2008, four-fifths of the states held their primaries or
caucuses in January, February or March.

Jump to the 2016 cycle, and it may even seem as if things are getting
off to a slow start. Most of the major candidates had declared by late
March in 2007, yet Ted Cruz was the only 2016 hopeful to have
\href{http://www.nytimes3xbfgragh.onion/2015/03/23/us/politics/ted-cruz-to-announce-on-monday-he-plans-to-run-for-president.html}{made
an official announcement} by that time of the year.

But as Mr. Oliver lamented, the campaign season starts well before
formal declarations. The so-called invisible primary, the period during
which potential candidates court donors, recruit staff and gauge
viability, began many months ago.

Although this behind-the-scenes, pre-campaign activity is not new to
American politics, the nature of today's news media means that it's a
lot more visible than it used to be, also contributing to why it feels
so long.

Regardless of when you mark the start, the process --- or at least the
public part of it --- in the United States is probably longer than in
any other country. European systems are generally more party-driven,
with party leaders deciding nominees instead of open elections. Many
have set campaign periods, lasting a few weeks to several months, during
which spending and other activities are heavily regulated. Some with
parliamentary systems don't even have a set election date --- the
current head of government has flexibility to call an election at his or
her discretion.

In France, the official presidential campaign period lasts two weeks per
election round, though candidates are out speaking and debating more
than a year in advance. The process has lengthened in recent cycles as
the country's major parties have begun to conduct primary elections. In
Britain, where the prime minister has traditionally chosen when to call
an election, the coalition government effectively fixed the date of the
next election to May 7 this year. Because candidates are subject to
spending limits during the six months before an election, knowing when
that period will occur has encouraged them to spend money earlier. But
despite the extended process, the election doesn't really begin in the
public's mind until the queen dissolves parliament, five weeks before
voting day.

So what are the downsides of the way we do it in the United States?

For one thing, many voters share the
\href{https://www.youtube.com/watch?v=OjrthOPLAKM\&feature=youtu.be}{sense
of fatigue} Mr. Oliver says he feels, especially with the saturation of
negative advertising. Polls conducted by the Pew Research Center during
the last three presidential elections found that at least half of
Americans consistently said that they would describe the presidential
campaign as too long.

This campaign fatigue also extends to individual candidates. Running for
president becomes a full-time job, sometimes requiring potential
candidates to set aside whatever else they were doing to pursue it. In
2005, for example, Mitt Romney
\href{http://www.nytimes3xbfgragh.onion/2005/12/15/national/15romney.html}{decided
not to run for a second term} as governor of Massachusetts, despite
having had a successful first term by many measures. For candidates who
already hold political office, campaigning and fund-raising overlap
significantly with the job of governing. Longer campaigns also go hand
in hand with more expensive ones. Candidates have to start earlier to
raise the funds necessary to compete.

But as demanding as campaigns may be, there are actually a few good
things about the way our system works. Long campaigns allow candidates
who are not already household names (or party insiders) to introduce
themselves to the public. ``No one knew who Bill Clinton was nine months
before he became president,'' said Bruce Buchanan, a government
professor at the University of Texas at Austin. The same was true of Mr.
Carter in 1976. ``He and his family went to Iowa and spent six months
pounding on doors, on their own nickel,'' he said.

For those convinced that the drawbacks of the two-year slog outweigh the
benefits, the national parties have provided a glimmer of hope. The
Democrats and the Republicans put in new guidelines for the 2012 cycle
requiring most states to wait until March to hold their nominating
contests (though several states bucked the rules), and both parties
voted to apply the rules in 2016.

But the first contest could still take place in January, because four
states --- Iowa, New Hampshire, South Carolina and Nevada --- are exempt
from that timeline. On Fox News last month, Reince Priebus, the chairman
of the Republican National Committee, attempted to explain why those
states get special treatment: ``It is years and years and years of
history --- and that's a debate, too,'' he said. ``It is what it is.''

Advertisement

\protect\hyperlink{after-bottom}{Continue reading the main story}

\hypertarget{site-index}{%
\subsection{Site Index}\label{site-index}}

\hypertarget{site-information-navigation}{%
\subsection{Site Information
Navigation}\label{site-information-navigation}}

\begin{itemize}
\tightlist
\item
  \href{https://help.nytimes3xbfgragh.onion/hc/en-us/articles/115014792127-Copyright-notice}{©~2020~The
  New York Times Company}
\end{itemize}

\begin{itemize}
\tightlist
\item
  \href{https://www.nytco.com/}{NYTCo}
\item
  \href{https://help.nytimes3xbfgragh.onion/hc/en-us/articles/115015385887-Contact-Us}{Contact
  Us}
\item
  \href{https://www.nytco.com/careers/}{Work with us}
\item
  \href{https://nytmediakit.com/}{Advertise}
\item
  \href{http://www.tbrandstudio.com/}{T Brand Studio}
\item
  \href{https://www.nytimes3xbfgragh.onion/privacy/cookie-policy\#how-do-i-manage-trackers}{Your
  Ad Choices}
\item
  \href{https://www.nytimes3xbfgragh.onion/privacy}{Privacy}
\item
  \href{https://help.nytimes3xbfgragh.onion/hc/en-us/articles/115014893428-Terms-of-service}{Terms
  of Service}
\item
  \href{https://help.nytimes3xbfgragh.onion/hc/en-us/articles/115014893968-Terms-of-sale}{Terms
  of Sale}
\item
  \href{https://spiderbites.nytimes3xbfgragh.onion}{Site Map}
\item
  \href{https://help.nytimes3xbfgragh.onion/hc/en-us}{Help}
\item
  \href{https://www.nytimes3xbfgragh.onion/subscription?campaignId=37WXW}{Subscriptions}
\end{itemize}
