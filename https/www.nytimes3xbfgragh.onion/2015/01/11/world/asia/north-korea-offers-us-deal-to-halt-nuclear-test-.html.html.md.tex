Sections

SEARCH

\protect\hyperlink{site-content}{Skip to
content}\protect\hyperlink{site-index}{Skip to site index}

\href{https://www.nytimes3xbfgragh.onion/section/world/asia}{Asia
Pacific}

\href{https://myaccount.nytimes3xbfgragh.onion/auth/login?response_type=cookie\&client_id=vi}{}

\href{https://www.nytimes3xbfgragh.onion/section/todayspaper}{Today's
Paper}

\href{/section/world/asia}{Asia Pacific}\textbar{}North Korea Offers
U.S. Deal to Halt Nuclear Test

\url{https://nyti.ms/1C0lrlU}

\begin{itemize}
\item
\item
\item
\item
\item
\end{itemize}

Advertisement

\protect\hyperlink{after-top}{Continue reading the main story}

Supported by

\protect\hyperlink{after-sponsor}{Continue reading the main story}

\hypertarget{north-korea-offers-us-deal-to-halt-nuclear-test}{%
\section{North Korea Offers U.S. Deal to Halt Nuclear
Test}\label{north-korea-offers-us-deal-to-halt-nuclear-test}}

\includegraphics{https://static01.graylady3jvrrxbe.onion/images/2015/01/11/world/PYONGYANG/PYONGYANG-articleLarge.jpg?quality=75\&auto=webp\&disable=upscale}

By \href{http://www.nytimes3xbfgragh.onion/by/choe-sang-hun}{Choe
Sang-Hun}

\begin{itemize}
\item
  Jan. 10, 2015
\item
  \begin{itemize}
  \item
  \item
  \item
  \item
  \item
  \end{itemize}
\end{itemize}

SEOUL, South Korea --- North Korea said Saturday that it had told the
United States that it would impose a temporary moratorium on nuclear
tests if Washington canceled its joint annual military exercises with
South Korea to help promote dialogue on the divided Korean Peninsula.

The North proposed its ``crucial step'' in a message it delivered to the
United States on Friday through an unspecified channel, the North's
official Korean Central News Agency said. In the past, North Korea has
relayed messages to Washington through its United Nations mission in New
York.

Until now, the United States has dismissed North Korea's routine demand
for an end to its joint military exercises with South Korea. The North
has called them a rehearsal for an invasion while the United States and
South Korea have insisted that their annual war games are defensive in
nature.

But the North's latest proposal included a new incentive for Washington,
offering to temporarily suspend nuclear tests in return for a suspension
of the joint military exercises this year.

The North's overture followed the New Year's Day speech of its leader,
Kim Jong-un, in which he said he was ready to meet with President Park
Geun-hye of South Korea if ``the mood was right.'' Mr. Kim said the two
Koreas should mark their 70th anniversary of liberation from Japanese
colonial rule this year with great strides toward inter-Korean
reconciliation. North Korea has since significantly toned down its
habitually harsh language when referring to South Korea.

The message proposed that the United States ``contribute to easing
tension on the Korean Peninsula by temporarily suspending joint military
exercises in South Korea and its vicinity this year, and said that in
this case the D.P.R.K. is ready to take such responsive step as
temporarily suspending the nuclear test over which the U.S. is
concerned,'' the North Korean news agency said Saturday. D.P.R.K. stands
for the Democratic People's Republic of Korea, the North's official
name.

\href{https://www.nytimes3xbfgragh.onion/interactive/2014/11/20/world/asia/northkorea-timeline.html}{}

\includegraphics{https://static01.graylady3jvrrxbe.onion/images/2013/07/27/world/asia/northkoreatimeline/northkoreatimeline-videoLarge-v4.jpg}

\hypertarget{timeline-on-north-koreas-nuclear-program}{%
\subsection{Timeline on North Korea's Nuclear
Program}\label{timeline-on-north-koreas-nuclear-program}}

The country's nuclear weapons program and its development of long-range
rocket systems have angered many in the West, including in the United
States.

The United States on Saturday rebuffed the North Korean proposal. Jen
Psaki, a State Department spokeswoman, said it ``inappropriately''
linked the ``routine'' joint military drills ``to the possibility of a
nuclear test.''

The South Korean government has proposed a dialogue with North Korea to
discuss tension-reducing steps, such as reunions of Korean families
separated during the 1950-53 Korean War, as well as a possible
North-South summit meeting. The North has not responded to the South
Korean proposal yet.

North Korea's overture to Washington came amid tensions between the two
governments over a Sony Pictures movie that involved a fictional C.I.A.
plot to assassinate Mr. Kim. Washington
a\href{http://www.nytimes3xbfgragh.onion/2014/12/18/world/asia/us-links-north-korea-to-sony-hacking.html}{ccused
the North of hacking} the computer network of the Hollywood studio and
imposed new sanctions last month. The North, which
\href{http://www.nytimes3xbfgragh.onion/2014/12/08/business/north-korea-denies-hacking-sony-but-calls-attack-a-righteous-deed.html}{denied
involvement}, has vowed to retaliate.

North Korea has conducted three nuclear tests --- in 2006, 2009 and 2013
--- prompting a series of United Nations sanctions. A recently published
South Korean Defense Ministry analysis said the North had made
significant advances toward making its nuclear weapons small enough to
fit onto a long-range missile capable of reaching the West Coast of the
United States. Experts said the North needed more tests to demonstrate
such capabilities.

The United States keeps 28,500 troops in South Korea under a mutual
defense treaty, a legacy of the Korean War, when it fought on the
South's side.

On Saturday, North Korea said ``there can be neither trust-based
dialogue nor détente and stability on the peninsula in such a gruesome
atmosphere in which war drills are staged against the dialogue
partner.''

The North added that it was ready to discuss its proposal with the
United States.

Advertisement

\protect\hyperlink{after-bottom}{Continue reading the main story}

\hypertarget{site-index}{%
\subsection{Site Index}\label{site-index}}

\hypertarget{site-information-navigation}{%
\subsection{Site Information
Navigation}\label{site-information-navigation}}

\begin{itemize}
\tightlist
\item
  \href{https://help.nytimes3xbfgragh.onion/hc/en-us/articles/115014792127-Copyright-notice}{©~2020~The
  New York Times Company}
\end{itemize}

\begin{itemize}
\tightlist
\item
  \href{https://www.nytco.com/}{NYTCo}
\item
  \href{https://help.nytimes3xbfgragh.onion/hc/en-us/articles/115015385887-Contact-Us}{Contact
  Us}
\item
  \href{https://www.nytco.com/careers/}{Work with us}
\item
  \href{https://nytmediakit.com/}{Advertise}
\item
  \href{http://www.tbrandstudio.com/}{T Brand Studio}
\item
  \href{https://www.nytimes3xbfgragh.onion/privacy/cookie-policy\#how-do-i-manage-trackers}{Your
  Ad Choices}
\item
  \href{https://www.nytimes3xbfgragh.onion/privacy}{Privacy}
\item
  \href{https://help.nytimes3xbfgragh.onion/hc/en-us/articles/115014893428-Terms-of-service}{Terms
  of Service}
\item
  \href{https://help.nytimes3xbfgragh.onion/hc/en-us/articles/115014893968-Terms-of-sale}{Terms
  of Sale}
\item
  \href{https://spiderbites.nytimes3xbfgragh.onion}{Site Map}
\item
  \href{https://help.nytimes3xbfgragh.onion/hc/en-us}{Help}
\item
  \href{https://www.nytimes3xbfgragh.onion/subscription?campaignId=37WXW}{Subscriptions}
\end{itemize}
