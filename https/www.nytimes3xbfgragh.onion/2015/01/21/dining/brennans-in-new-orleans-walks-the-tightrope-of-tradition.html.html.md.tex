Sections

SEARCH

\protect\hyperlink{site-content}{Skip to
content}\protect\hyperlink{site-index}{Skip to site index}

\href{https://www.nytimes3xbfgragh.onion/section/food}{Food}

\href{https://myaccount.nytimes3xbfgragh.onion/auth/login?response_type=cookie\&client_id=vi}{}

\href{https://www.nytimes3xbfgragh.onion/section/todayspaper}{Today's
Paper}

\href{/section/food}{Food}\textbar{}Brennan's in New Orleans Walks the
Tightrope of Tradition

\url{https://nyti.ms/1AInjNU}

\begin{itemize}
\item
\item
\item
\item
\item
\item
\end{itemize}

Advertisement

\protect\hyperlink{after-top}{Continue reading the main story}

Supported by

\protect\hyperlink{after-sponsor}{Continue reading the main story}

Critic on the Road

\hypertarget{brennans-in-new-orleans-walks-the-tightrope-of-tradition}{%
\section{Brennan's in New Orleans Walks the Tightrope of
Tradition}\label{brennans-in-new-orleans-walks-the-tightrope-of-tradition}}

Slide 1 of 12

1/12

Brennan's, founded in 1946, has been the subject of a foreclosure, a
bankruptcy and a change in ownership, all in the last two years. It
reopened in November, retaining many of its charms, including bananas
Foster.

Credit...Cheryl Gerber for The New York Times

\begin{itemize}
\item
  \includegraphics{https://static01.graylady3jvrrxbe.onion/images/2015/01/21/dining/20150121-CRITIC-slide-TDXJ/20150121-CRITIC-slide-TDXJ-superJumbo.jpg}
\item
  \includegraphics{https://static01.graylady3jvrrxbe.onion/images/2015/01/21/dining/20150121-CRITIC-slide-76LT/20150121-CRITIC-slide-76LT-superJumbo.jpg}
\item
  \includegraphics{https://static01.graylady3jvrrxbe.onion/images/2015/01/21/dining/20150121-CRITIC-slide-HIHI/20150121-CRITIC-slide-HIHI-superJumbo.jpg}
\item
  \includegraphics{https://static01.graylady3jvrrxbe.onion/images/2015/01/21/dining/20150121-CRITIC-slide-CAO0/20150121-CRITIC-slide-CAO0-superJumbo.jpg}
\item
  \includegraphics{https://static01.graylady3jvrrxbe.onion/images/2015/01/21/dining/20150121-CRITIC-slide-284J/20150121-CRITIC-slide-284J-superJumbo.jpg}
\item
  \includegraphics{https://static01.graylady3jvrrxbe.onion/images/2015/01/21/dining/20150121-CRITIC-slide-AHUQ/20150121-CRITIC-slide-AHUQ-superJumbo.jpg}
\item
  \includegraphics{https://static01.graylady3jvrrxbe.onion/images/2015/01/21/dining/20150121-CRITIC-slide-MXRR/20150121-CRITIC-slide-MXRR-superJumbo.jpg}
\item
  \includegraphics{https://static01.graylady3jvrrxbe.onion/images/2015/01/21/dining/20150121-CRITIC-slide-YZ29/20150121-CRITIC-slide-YZ29-superJumbo.jpg}
\item
  \includegraphics{https://static01.graylady3jvrrxbe.onion/images/2015/01/21/dining/20150121-CRITIC-slide-EKBL/20150121-CRITIC-slide-EKBL-superJumbo.jpg}
\item
  \includegraphics{https://static01.graylady3jvrrxbe.onion/images/2015/01/21/dining/20150121-CRITIC-slide-QZXF/20150121-CRITIC-slide-QZXF-superJumbo.jpg}
\item
  \includegraphics{https://static01.graylady3jvrrxbe.onion/images/2015/01/21/dining/20150121-CRITIC-slide-21MQ/20150121-CRITIC-slide-21MQ-superJumbo.jpg}
\item
  \includegraphics{https://static01.graylady3jvrrxbe.onion/images/2015/01/21/dining/20150121-CRITIC-slide-WEUC/20150121-CRITIC-slide-WEUC-superJumbo.jpg}
\end{itemize}

By \href{http://www.nytimes3xbfgragh.onion/by/pete-wells}{Pete Wells}

\begin{itemize}
\item
  Jan. 20, 2015
\item
  \begin{itemize}
  \item
  \item
  \item
  \item
  \item
  \item
  \end{itemize}
\end{itemize}

NEW ORLEANS --- The busiest meal of the day is still breakfast, which
for many customers still begins with a cocktail and ends with dessert.
At tableside trolleys inside the many dining rooms, bananas still go up
in flames every few minutes. The stucco facade on Royal Street in the
French Quarter is still pink, although the paint manufacturer prefers to
call it Tomato Cream Sauce. (The trim color is Mayonnaise.)

At a glance, \href{http://www.brennansneworleans.com/}{Brennan's} seems
to be rolling along just as it has since it was founded in 1946. Over
the last two years, though, the restaurant has been the subject of a
foreclosure,
\href{http://www.nola.com/dining/index.ssf/2013/12/brennans_inc_former_owner_of_t.html}{a
bankruptcy} and a visit by the New Orleans Police Department,
\href{http://www.nola.com/dining/index.ssf/2013/05/family_battle_over_control_of.html}{called
in to pacify} a shareholders meeting. Brennans inveighed against
Brennans. Brennans sued Brennans. The restaurant was sold off in pieces:
the building for \$6.85 million, the name and other accouterments
\href{http://neworleanscitybusiness.com/blog/2014/07/10/brennans-to-reopen-after-bankruptcy-sale/}{for
\$3 million}. An estimated \$20 million more was poured into fixing up
the 18th-century structure and building new kitchens and dining rooms.

When the pieces were put back together, Brennan's was in the hands of a
new Brennan from another branch of the family, Ralph Brennan, and the
kitchen was turned over to a new chef, Slade Rushing. Since
\href{http://www.nola.com/dining/index.ssf/2014/11/brennans_restaurant_french_qua.html}{reopening
the restaurant in November}, the two have tried to keep many Brennan's
traditions alive. At the same time, they live in the modern world. Mr.
Rushing spent the last seven years as chef of MiLa here, and Mr. Brennan
runs six other restaurants. They agree that Brennan's won't survive if
they can't give it two things it hasn't had in recent years: a
reputation for good food, and the local following that goes with it.

It had both in the early days, when Owen Brennan started the business,
and his sister Ella helped manage it. To stand out from its more
established competitors, Brennan's began to specialize in breakfast. It
did this by encouraging diners, starting as early as 8 a.m., to pack
more courses, more sugar, more fat and more cocktails into the morning
meal than most Americans consumed in an entire day. One might begin with
a glass of brandy milk punch as an eye-opener and end with bananas
Foster, finished with a huge shot of flaming rum. This was nominally
dessert, although in New Orleans, the line between drinking and
everything else is often a blurred one.

``If you haven't had breakfast at Brennan's, you haven't really been to
New Orleans at all, you know, che?'' the novelist
\href{http://www.nytimes3xbfgragh.onion/2006/04/23/magazine/23food.html}{Peter
S. Feibleman} wrote in 1971, sliding into local dialect for his
contribution to the Time-Life ``Foods of the World'' series, the volume
on Louisiana. ``Well, go there, you hear? Go there.''

By that time, the Brennans, led by Ella, Dick and their siblings, were
opening restaurants in other neighborhoods and cities, starting with
\href{http://www.commanderspalace.com/}{Commander's Palace}. There was
dissent in the ranks, though, and in 1974 the various branches agreed to
a family divorce. Owen's children kept Brennan's. The other Brennans
would run everything else. Neither side would explain, but it was clear
the split was traumatic.

\includegraphics{https://static01.graylady3jvrrxbe.onion/images/2015/01/21/dining/20150121-CRITIC-slide-76LT/20150121-CRITIC-slide-76LT-articleLarge.jpg?quality=75\&auto=webp\&disable=upscale}

``When the family broke up, it was a huge deal,'' said Emeril Lagasse,
the chef of Commander's Palace from 1982 to 1989. ``My experience with
Ella and Dick over eight years was that the love that they grew up with,
they wanted to express that. For Saturday and Sunday brunch at
Commander's Palace, there was a lot of feeling and dishes on the menu
that were really memories of Ella being at Brennan's on Royal Street.''

Outsiders who hope to understand the importance of Brennan's to the
family, and the importance of the family to New Orleans, should bear in
mind that after Brennan's had changed hands and Ella Brennan returned to
the building in November for her first visit in 40 years,
\href{http://www.nola.com/dining/index.ssf/2014/11/ella_brennan_for_the_first_tim.html}{The
Times-Picayune sent a reporter} to cover it.

Until the name began to appear in headlines, the subject of Brennan's
rarely came up for many locals except in connection with the prices.
``At one point, it started getting really outrageously expensive,'' Mr.
Lagasse said. ``I heard that a lot from people at the end of the last
regime: `Wow, it's super-expensive for eggs.' ''

My breakfast there at 8 on a recent Sunday morning seemed fairly priced
for a meal that included, in the fitness-minded Brennan's style, some
grilled shrimp followed by pan-fried veal grillades over grits and then
dessert, interwoven with a cocktail or two. For the menu, Mr. Rushing
had dug into the restaurant's archives: There were crepes with a bright
satsuma sauce, and there were baked apples with sugary lids of pecans
and oats, both suggested as appetizers; eggs Hussarde and eggs Sardou; a
mixed-seafood gumbo thickened with filé; and a generously sherried
snapping-turtle soup that was almost the color of black beans.

Our server had an impressive ability to read the table, talking slowly
and quietly in the beginning and picking up the pace only once she had
gotten two Bloody Marys and a French 75 into us. I've eaten a morning
feast this substantial, served with this level of sympathy and
formality, just once before; that was at Commander's Palace, and
technically it was brunch, not breakfast.

Creole stalwarts also anchored Brennan's dinner menu, but one page was
given to something you won't find at
\href{http://www.galatoires.com/home}{Galatoire's} or
\href{http://www.antoines.com/}{Antoine's}: a five-course chef's
tasting. ``Brennan's is known for breakfast, and you can't really mess
around with breakfast,'' our server said, by way of explanation. ``So
this is where the chef gets to play around a little bit.''

Without strapping myself in for the full tasting, I still managed to eat
most of Mr. Rushing's playing-around dishes. Remaking New Orleans
barbecue shrimp with a very un-New Orleans ingredient, lobster claw,
seemed like a delicious joke. I was impressed, too, by how much flavor
white soy, threads of tart fermented mustard greens and browned coins of
fried Japanese eggplant brought to slabs of raw amberjack without
overshadowing the fish.

Image

Caribbean milk punch.Credit...Cheryl Gerber for The New York Times

One of the best things I ate at Brennan's was a duck-confit-studded
rutabaga cake, Mr. Rushing's version of a Chinese-Vietnamese pan-fried
turnip cake. To his mind, cooking Creole cuisine means being alert to
flavors that new immigrants are bringing to the city's gumbo pot.

In these early innings, Mr. Rushing's menu sometimes seemed to stretch
too far in its effort to straddle the old and the new, without really
doing justice to either. A surprising number of dishes were undone by
one small detail: The turtle soup needed more brightness; the fried
rabbit seemed to have lost its crunch after leaving the oil; the pompano
didn't take on much color or flavor from the griddle; the lemon-butter
sauce was a bit too thin to cling to its crab meat. The ideas and
flavors were sound, though, so they may all come together once Mr.
Rushing has all the cooks he needs for his three kitchens, which can
serve up to 350 people at once. (By contrast, Mr. Rushing first received
notice when he was cooking in a 28-seat Manhattan restaurant called
\href{http://www.nytimes3xbfgragh.onion/restaurants/1065698268587/jacks-luxury-oyster-bar/details.html}{Jack's
Luxury Oyster Bar}.)

No such problems surfaced among the sweets, although now that I've
tasted bananas Foster in its natural environment, I have to agree with
Ella Brennan, who recently called the dessert ``very ordinary'' despite
having a hand in inventing it. Almost everything else containing sugar
was beautiful and refined, from the simple Creole crepes to a somewhat
abstract version of bread pudding that was mostly a rich vanilla custard
made even richer by big shards of praline. A few dots of French bread
seemed to have been scattered around solely for propriety's sake. The
dessert won't fool bread-pudding traditionalists, but it should please
them.

At the very least, it should help bring bread pudding into the 21st
century. The French Quarter has become something of a Jurassic Park for
Creole cuisine, a contained area in which to see shrimp rémoulade,
oysters Rockefeller and other giants of a former age in all their
lumbering glory. At Arnaud's, Antoine's, Galatoire's and
\href{http://www.tujaguesrestaurant.com/}{Tujague's}, evolution stops at
the kitchen door.

``That's what happens to our institutions,'' said Liz Williams, the
president of the new \href{http://southernfood.org/}{Southern Food and
Beverage Museum}, in the Central City neighborhood. ``They are not
allowed to change. You might go to Antoine's once a year, but you don't
want the menu to be any different than it was a year ago.'' That
dynamic, she said, abandons restaurants to the tourist trade, which
leaves them vulnerable.

Mr. Rushing believes that's how the old Brennan's slouched into
bankruptcy. ``They got stuck in a rut and they took things for
granted,'' he said. ``Restaurants die if they're not progressive. You've
got to have some sort of push for innovation.''

\emph{Brennan's, 417 Royal Street, New Orleans; 504-525-9711;}
\href{http://brennansneworleans.com/}{\emph{brennansneworleans.com}} **

Advertisement

\protect\hyperlink{after-bottom}{Continue reading the main story}

\hypertarget{site-index}{%
\subsection{Site Index}\label{site-index}}

\hypertarget{site-information-navigation}{%
\subsection{Site Information
Navigation}\label{site-information-navigation}}

\begin{itemize}
\tightlist
\item
  \href{https://help.nytimes3xbfgragh.onion/hc/en-us/articles/115014792127-Copyright-notice}{©~2020~The
  New York Times Company}
\end{itemize}

\begin{itemize}
\tightlist
\item
  \href{https://www.nytco.com/}{NYTCo}
\item
  \href{https://help.nytimes3xbfgragh.onion/hc/en-us/articles/115015385887-Contact-Us}{Contact
  Us}
\item
  \href{https://www.nytco.com/careers/}{Work with us}
\item
  \href{https://nytmediakit.com/}{Advertise}
\item
  \href{http://www.tbrandstudio.com/}{T Brand Studio}
\item
  \href{https://www.nytimes3xbfgragh.onion/privacy/cookie-policy\#how-do-i-manage-trackers}{Your
  Ad Choices}
\item
  \href{https://www.nytimes3xbfgragh.onion/privacy}{Privacy}
\item
  \href{https://help.nytimes3xbfgragh.onion/hc/en-us/articles/115014893428-Terms-of-service}{Terms
  of Service}
\item
  \href{https://help.nytimes3xbfgragh.onion/hc/en-us/articles/115014893968-Terms-of-sale}{Terms
  of Sale}
\item
  \href{https://spiderbites.nytimes3xbfgragh.onion}{Site Map}
\item
  \href{https://help.nytimes3xbfgragh.onion/hc/en-us}{Help}
\item
  \href{https://www.nytimes3xbfgragh.onion/subscription?campaignId=37WXW}{Subscriptions}
\end{itemize}
