Sections

SEARCH

\protect\hyperlink{site-content}{Skip to
content}\protect\hyperlink{site-index}{Skip to site index}

\href{https://www.nytimes3xbfgragh.onion/section/world/asia}{Asia
Pacific}

\href{https://myaccount.nytimes3xbfgragh.onion/auth/login?response_type=cookie\&client_id=vi}{}

\href{https://www.nytimes3xbfgragh.onion/section/todayspaper}{Today's
Paper}

\href{/section/world/asia}{Asia Pacific}\textbar{}Court Acquits
Journalist Accused of Defaming South Korean President

\url{https://nyti.ms/229PdDy}

\begin{itemize}
\item
\item
\item
\item
\item
\end{itemize}

Advertisement

\protect\hyperlink{after-top}{Continue reading the main story}

Supported by

\protect\hyperlink{after-sponsor}{Continue reading the main story}

\hypertarget{court-acquits-journalist-accused-of-defaming-south-korean-president}{%
\section{Court Acquits Journalist Accused of Defaming South Korean
President}\label{court-acquits-journalist-accused-of-defaming-south-korean-president}}

\includegraphics{https://static01.graylady3jvrrxbe.onion/images/2015/12/18/world/18Southkorea-web/18Southkorea-web-articleLarge.jpg?quality=75\&auto=webp\&disable=upscale}

By \href{http://www.nytimes3xbfgragh.onion/by/choe-sang-hun}{Choe
Sang-Hun}

\begin{itemize}
\item
  Dec. 17, 2015
\item
  \begin{itemize}
  \item
  \item
  \item
  \item
  \item
  \end{itemize}
\end{itemize}

SEOUL, South Korea --- A Seoul court found a Japanese reporter not
guilty on Thursday of defaming President Park Geun-hye of South Korea,
whose government has been accused of using legal channels to try to
silence news reports unfavorable to her administration.

\href{http://www.nytimes3xbfgragh.onion/2014/10/10/world/asia/japan-protests-an-indictment-of-a-journalist-.html}{Tatsuya
Kato}, a former Seoul bureau chief of Japan's right-wing Sankei Shimbun
newspaper, was on trial on the criminal charge of defaming Ms. Park with
an online article published in August 2014 in which he cited what he
called financial industry rumors that Ms. Park may have been having a
romantic encounter with a former aide as a ferry with hundreds of
passengers was sinking off southwestern South Korea.

More than 300 people, most of them teenagers, were killed in the ferry
disaster in April 2014, and the high death toll was partly attributed to
her government's failure to quickly begin an efficient rescue operation.
One question raised in the domestic news media at the time was whether
Ms. Park was absent from her duties for seven hours on the day of the
sinking.

``This is a natural verdict, and I have no other particular feeling,''
Mr. Kato said in a news conference after his acquittal.

He said that he suspected his indictment was politically motivated and
that it was a ``questionable'' practice to ``indict a reporter over his
article about a most public figure like the president just because they
didn't like it.''

Both Ms. Park's office and the former aide, Chung Yoon-hoi, have
vehemently denied the rumors cited in Mr. Kato's report. They called the
report maliciously defamatory because they said the reporter made little
effort to verify the rumors --- an argument shared by prosecutors when
they indicted him in October 2014.

On Thursday, the Seoul Central District Court delivered its acquittal in
a long-awaited ruling. ``The article by the accused contained things
inappropriate, but given that it was written with the public interest in
mind, it falls within the area where the freedom of the press should be
protected in a democratic society,'' Judge Lee Dong-geun said.

Prosecutors have a week to appeal.

Ms. Park's office did not immediately comment on the ruling. The
acquittal came hours after the Foreign Ministry of South Korea revealed
that it had asked the Justice Ministry to consider Japan's appeal for
leniency for Mr. Kato.

Mr. Kato's legal trouble began last year when conservative South Korean
civic groups, including anti-Japanese nationalist activists, sued him.
His subsequent indictment came as a string of criminal investigations
and lawsuits under Ms. Park led rights groups to criticize the way her
government dealt with its detractors and to question how much freedom of
expression was tolerated.

In a report on South Korea in November, the United Nations Human Rights
Committee voiced concern about ``the increasing use of criminal
defamation laws to prosecute persons who criticize government action and
obstruct business interests, and of the harsh sentences, including
lengthy prison sentences, attached to such legal provisions.''

South Korea should ``promote a culture of tolerance regarding criticism,
which is essential for a functioning democracy,'' the report said.

Phil Robertson, a deputy director for Asia at Human Rights Watch, said
that criminal defamation laws like South Korea's ``have a chilling
effect on freedom of expression, and work against the public interest by
gagging critics and whistle-blowers and stifling a free press.''

``We firmly believe that journalists should not be criminalized for just
doing their jobs,'' he said, commenting on Mr. Kato's case.

Throughout the trial, Mr. Kato and his lawyers pleaded innocence, saying
that his article served the public's interest by asking what the
president was doing during one of the country's biggest disasters in
years. Prosecutors had requested an 18-month prison term for Mr. Kato,
contending that the article was both false and defamatory.

South Korea promotes itself as one of Asia's most vibrant democracies, a
far cry from the dictatorship it once was under military-backed
strongmen, including Ms. Park's father, President Park Chung-hee. Its
news outlets reflect a wide spectrum of political views, and its social
media can be critical of Ms. Park.

South Koreans turn vociferous over any sign of repression of the freedom
of speech.

But many South Koreans did not sympathize with Mr. Kato. His newspaper,
the Sankei, is reviled here for carrying articles that residents say
belittle their country and help bolster kenkan, or ``hate Korea,''
sentiment in Japan. The Sankei serves as a popular channel for
conservative politicians in Japan who contend that Korean women
recruited to serve as sex slaves for Japanese soldiers during World War
II were prostitutes.

Japan repeatedly protested Mr. Kato's indictment, as his case became the
latest spat to divide South Korea and Japan, whose relations have long
been strained over historical and territorial disputes arising from
Japan's 35-year colonial rule of Korea until its defeat in World War II.

Advertisement

\protect\hyperlink{after-bottom}{Continue reading the main story}

\hypertarget{site-index}{%
\subsection{Site Index}\label{site-index}}

\hypertarget{site-information-navigation}{%
\subsection{Site Information
Navigation}\label{site-information-navigation}}

\begin{itemize}
\tightlist
\item
  \href{https://help.nytimes3xbfgragh.onion/hc/en-us/articles/115014792127-Copyright-notice}{©~2020~The
  New York Times Company}
\end{itemize}

\begin{itemize}
\tightlist
\item
  \href{https://www.nytco.com/}{NYTCo}
\item
  \href{https://help.nytimes3xbfgragh.onion/hc/en-us/articles/115015385887-Contact-Us}{Contact
  Us}
\item
  \href{https://www.nytco.com/careers/}{Work with us}
\item
  \href{https://nytmediakit.com/}{Advertise}
\item
  \href{http://www.tbrandstudio.com/}{T Brand Studio}
\item
  \href{https://www.nytimes3xbfgragh.onion/privacy/cookie-policy\#how-do-i-manage-trackers}{Your
  Ad Choices}
\item
  \href{https://www.nytimes3xbfgragh.onion/privacy}{Privacy}
\item
  \href{https://help.nytimes3xbfgragh.onion/hc/en-us/articles/115014893428-Terms-of-service}{Terms
  of Service}
\item
  \href{https://help.nytimes3xbfgragh.onion/hc/en-us/articles/115014893968-Terms-of-sale}{Terms
  of Sale}
\item
  \href{https://spiderbites.nytimes3xbfgragh.onion}{Site Map}
\item
  \href{https://help.nytimes3xbfgragh.onion/hc/en-us}{Help}
\item
  \href{https://www.nytimes3xbfgragh.onion/subscription?campaignId=37WXW}{Subscriptions}
\end{itemize}
