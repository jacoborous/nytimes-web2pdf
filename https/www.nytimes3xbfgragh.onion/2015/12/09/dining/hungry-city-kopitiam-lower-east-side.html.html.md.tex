Sections

SEARCH

\protect\hyperlink{site-content}{Skip to
content}\protect\hyperlink{site-index}{Skip to site index}

\href{https://www.nytimes3xbfgragh.onion/section/food}{Food}

\href{https://myaccount.nytimes3xbfgragh.onion/auth/login?response_type=cookie\&client_id=vi}{}

\href{https://www.nytimes3xbfgragh.onion/section/todayspaper}{Today's
Paper}

\href{/section/food}{Food}\textbar{}At Kopitiam, the Art of Malaysian
Snacking on the Lower East Side

\url{https://nyti.ms/1OA0pDL}

\begin{itemize}
\item
\item
\item
\item
\item
\item
\end{itemize}

Advertisement

\protect\hyperlink{after-top}{Continue reading the main story}

Supported by

\protect\hyperlink{after-sponsor}{Continue reading the main story}

\href{/column/hungry-city}{Hungry City}

\hypertarget{at-kopitiam-the-art-of-malaysian-snacking-on-the-lower-east-side}{%
\section{At Kopitiam, the Art of Malaysian Snacking on the Lower East
Side}\label{at-kopitiam-the-art-of-malaysian-snacking-on-the-lower-east-side}}

\href{https://www.nytimes3xbfgragh.onion/slideshow/2015/12/09/dining/kopitiam.html}{}

\hypertarget{kopitiam}{%
\subsection{Kopitiam}\label{kopitiam}}

10 Photos

View Slide Show ›

\includegraphics{https://static01.graylady3jvrrxbe.onion/images/2015/12/09/dining/09HUNGRY-KOPITIAM-slide-LHA2/09HUNGRY-KOPITIAM-slide-LHA2-articleLarge.jpg?quality=75\&auto=webp\&disable=upscale}

Nancy Borowick for The New York Times

\begin{itemize}
\tightlist
\item
  Kopitiam\\
  Malaysian \$ 51B Canal Street 646-894-7081
\end{itemize}

By Ligaya Mishan

\begin{itemize}
\item
  Dec. 3, 2015
\item
  \begin{itemize}
  \item
  \item
  \item
  \item
  \item
  \item
  \end{itemize}
\end{itemize}

Kopitiam's toast looks like squares of sheet cake, implausibly tall and
as porous as coral. This would not be out of place at a Texan barbecue
shack, except for the celadon-hued coconut jam sealing the slabs of
bread together, the Buddhist shrine in the corner and the twin dragons
guarding the front window overlooking Canal Street.

On another plate is a mound of blue rice. It spent the night with a
fistful of morning glories, soaking up deep-sky shades. The flowers give
it color only; the clinginess and muted sweetness come from coconut milk
and the breakdown of starch in the sticky rice. Until you unlatch the
banana leaf folded around it, all that is visible is a thatch of grated
coconut, bronzed from a turn in the pan with palm sugar and pandan, for
a whiff of pine.

Kopitiam is a kopitiam, named after its genre, a marriage of ``kopi,''
Malay for coffee, and ``tiam,'' Hokkien for shop. In Penang, Malaysia, a
kopitiam may be anonymous in décor and diminutive in scale, and still be
the point from which the life of a neighborhood radiates, where
customers concede hours of their day, trading pages of a single shared
newspaper while sipping tea from porcelain cups.

Kyo Pang, the chef, grew up in Penang and has a collection of those
cups. (Her grandfather ran a coffee shop that her father later turned
into a full-fledged restaurant.) She opened Kopitiam in October with her
fiancée, Xinnan Lin, hoping to keep alive the art of snacks passed down
by descendants of Chinese settlers along the Straits of Malacca.

``I learned to cook sweet from my mother and savory from my father,''
Ms. Pang said. On her father's side of the brief menu are mah-jongg
noodles (``for when you're playing and don't want to leave the table,''
Ms. Pang said), a happy conflagration of sesame oil, fish sauce and
chile, overrun by sesame seeds. Half-boiled eggs --- yolks trembly and
leaking gold, teetering atop curls of white --- are saturated with soy
sauce simmered with shiitake mushrooms; the mushrooms have been
discarded, but their earthy fragrance remains.

A dark, briny sambal of belacan (fermented shrimp paste) and crushed
dried shrimp sends a pulse through a stir-fry of ikan bilis (anchovies)
and peanuts over coconut rice in nasi lemak. A lighter sambal, leavened
by turmeric, lemongrass and kalamansi, counters the funk of dried shrimp
tucked into cylinders of sticky rice called pulut panggang.

All are delicious, but Ms. Pang's mother may have the upper hand, with
desserts like muah chee, larval nubs of hot mochi pitched in roasted
ground peanuts, sesame seeds and sugar, and kuih talam, a dense
jellylike layer cake, the bottom green from pandan and pea flour, the
top creamy with coconut milk and spiked with salt.

Ms. Pang spends an hour and a half vigilantly stirring coconut milk,
eggs, palm sugar and fresh pandan leaves in a double boiler to make
kaya, coconut jam that she sells by the jar and slathers on toast. Other
monumental slices of bread are dipped in eggs, Milo (an Australian
counterpart to Ovaltine), condensed milk and butter, then fried. These
arrive stacked in a kind of grand, ramshackle temple, with more
condensed milk waterfalling down the tilting ledges.

Her porcelain cups are reserved for tea, best hand-pulled, poured from
one pot to another several times. This isn't mere theater: The flights
through the air yield more froth and bring the tea, boiled at 200
degrees to extract maximum flavor, to drinking temperature.

For coffee, there are paper cups --- one size only, because Ms. Pang
imports her coffee from Malaysia preground and (apart from kopi-o, black
coffee) premixed with precisely calibrated amounts of powdered milk and
sugar. The revelation is white coffee, made from beans roasted in milk
and olive oil. It comes splashed with condensed milk and tastes almost
like hot chocolate.

The room is tight, with only four stools along a bright blue counter, a
good patch of which is taken up by the dragons. (They are fenced in by
masking tape, which bears a polite warning: ``Please do not touch
them.'') On a recent morning, when smoke started to billow from the
kitchen, Ms. Pang left the front door ajar. It was icy on the street,
but inside it never grew cold.

Advertisement

\protect\hyperlink{after-bottom}{Continue reading the main story}

\hypertarget{site-index}{%
\subsection{Site Index}\label{site-index}}

\hypertarget{site-information-navigation}{%
\subsection{Site Information
Navigation}\label{site-information-navigation}}

\begin{itemize}
\tightlist
\item
  \href{https://help.nytimes3xbfgragh.onion/hc/en-us/articles/115014792127-Copyright-notice}{©~2020~The
  New York Times Company}
\end{itemize}

\begin{itemize}
\tightlist
\item
  \href{https://www.nytco.com/}{NYTCo}
\item
  \href{https://help.nytimes3xbfgragh.onion/hc/en-us/articles/115015385887-Contact-Us}{Contact
  Us}
\item
  \href{https://www.nytco.com/careers/}{Work with us}
\item
  \href{https://nytmediakit.com/}{Advertise}
\item
  \href{http://www.tbrandstudio.com/}{T Brand Studio}
\item
  \href{https://www.nytimes3xbfgragh.onion/privacy/cookie-policy\#how-do-i-manage-trackers}{Your
  Ad Choices}
\item
  \href{https://www.nytimes3xbfgragh.onion/privacy}{Privacy}
\item
  \href{https://help.nytimes3xbfgragh.onion/hc/en-us/articles/115014893428-Terms-of-service}{Terms
  of Service}
\item
  \href{https://help.nytimes3xbfgragh.onion/hc/en-us/articles/115014893968-Terms-of-sale}{Terms
  of Sale}
\item
  \href{https://spiderbites.nytimes3xbfgragh.onion}{Site Map}
\item
  \href{https://help.nytimes3xbfgragh.onion/hc/en-us}{Help}
\item
  \href{https://www.nytimes3xbfgragh.onion/subscription?campaignId=37WXW}{Subscriptions}
\end{itemize}
