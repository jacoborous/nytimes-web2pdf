Sections

SEARCH

\protect\hyperlink{site-content}{Skip to
content}\protect\hyperlink{site-index}{Skip to site index}

\href{https://www.nytimes3xbfgragh.onion/section/food}{Food}

\href{https://myaccount.nytimes3xbfgragh.onion/auth/login?response_type=cookie\&client_id=vi}{}

\href{https://www.nytimes3xbfgragh.onion/section/todayspaper}{Today's
Paper}

\href{/section/food}{Food}\textbar{}Restaurant Review: Wildair on the
Lower East Side

\url{https://nyti.ms/1OiRk18}

\begin{itemize}
\item
\item
\item
\item
\item
\item
\end{itemize}

Advertisement

\protect\hyperlink{after-top}{Continue reading the main story}

Supported by

\protect\hyperlink{after-sponsor}{Continue reading the main story}

\hypertarget{restaurant-review-wildair-on-the-lower-east-side}{%
\section{Restaurant Review: Wildair on the Lower East
Side}\label{restaurant-review-wildair-on-the-lower-east-side}}

\href{https://www.nytimes3xbfgragh.onion/slideshow/2015/10/07/dining/wildair.html}{}

\hypertarget{wildair}{%
\subsection{Wildair}\label{wildair}}

10 Photos

View Slide Show ›

\includegraphics{https://static01.graylady3jvrrxbe.onion/images/2015/10/07/dining/07REST-WILDAIR-slide-I9ZF/07REST-WILDAIR-slide-I9ZF-articleLarge.jpg?quality=75\&auto=webp\&disable=upscale}

Liz Barclay for The New York Times

\begin{itemize}
\tightlist
\item
  Wildair\\
  ★★ American \$\$ 142 Orchard Street 646-964-5624
\end{itemize}

By \href{http://www.nytimes3xbfgragh.onion/by/pete-wells}{Pete Wells}

\begin{itemize}
\item
  Oct. 6, 2015
\item
  \begin{itemize}
  \item
  \item
  \item
  \item
  \item
  \item
  \end{itemize}
\end{itemize}

When serious, reputation-making restaurants run out of room on the walls
to hang all the awards, chefs open a cheaper, more casual restaurant
nearby. Like a dutiful younger brother, the second restaurant is
supposed to work harder, please more people, make fewer demands, be more
pragmatic and smile as the firstborn sucks up all the attention.

There is rarely any confusion about these roles.
\href{http://www.thomaskeller.com/time-warner-new-york/bouchon-bakery-caf\%C3\%A9}{Bouchon
Bakery \& Café} doesn't try to waylay customers toddling toward
\href{http://www.nytimes3xbfgragh.onion/2011/10/12/dining/reviews/per-se-nyc-restaurant-review.html}{Per
Se}'s blue door and tempt them into having a quiche and salad instead.

Sometimes, though, younger siblings who follow the rules can still pull
off some quiet subversion.

This summer, the chefs Jeremiah Stone and Fabian von Hauske opened a
wine bar two doors down Orchard Street from their showcase restaurant,
\href{http://www.nytimes3xbfgragh.onion/2014/03/12/dining/restaurant-review-contra-on-the-lower-east-side.html}{Contra}.
The new place, called \href{http://www.wildair.nyc/}{Wildair}, plays its
part exactly as written. And somehow, in a subtle and charming way, the
kid upstages his big brother.

Contra does laid-back refinement as well as any downtown restaurant, and
the two chefs pack an impressive amount of nuance into its six- or
seven-course set menus. Wildair is the one I want to go back to, though.
I want to camp out at a high table, taste everything new on the menu to
see whether it clicks and pillage the stash of natural wines like
Genghis Khan.

Format matters. Contra's is a concise tasting menu, and tasting menus
are the parchment scrolls on which chefs inscribe their Magna Cartas
with goose-quill flourishes. Wildair borrows its format from
contemporary European wine bars such as
\href{http://manfreds.dk/en/}{Manfreds \& Vin} in Copenhagen and
\href{http://www.restaurantledauphin.net/}{Le Dauphin} in Paris, places
that extend a hand inviting us to try something new, drink a strange
gamay bottled by an ornery old hermit, eat a little of this and a little
more of that while we catch up, unwind and make plans. You visit Contra.
Wildair is a restaurant you can live in.

The food isn't necessarily better, but it's often more inventive. You
wouldn't guess that from scanning the menu, reading down from oysters
and cheese to fried squid and steak. Mr. Stone and Mr. von Hauske are
serving familiar, likable food. But instead of tying them down, this
seems to have given them permission to lift off with extra altitude.

The squid is fried along with sliced lemons and sections of green onion.
Everything is contained in a golden shell that is implausibly light and
crunchy, with no hint of sogginess. Potato flour mixed with masa is the
secret of the batter. A garlic-powered squid-ink aioli gives the dish
some staying power.

The secret of Wildair's white shrimp with celery is probably the shrimp
themselves, netted off the coast of Georgia. But the kitchen augments
their unmistakably wild flavor by patiently stewing them in
garlic-pimentón oil, heads and all.

On my first two outings at Wildair, I passed up the pork rillettes,
suspecting that they were outsourced. They are not. They are beautifully
made on site, smooth and pinkish and creamy in a glass bowl sealed with
a white blanket of fat. I spread the rillettes on thin toast and took a
bite, letting the fat and its warm, holiday flavor of ginger and other
seasonings melt on my tongue.

I almost skipped the lettuce, too, suspecting that it was lettuce. I was
right about that but wrong about skipping it. With squiggles of
butter-and-caramelized-lettuce dressing insinuated into its folds and
some chopped pistachios on top, the sweet Little Gem lettuce makes for
one of those ideal salads that is as complicated as it needs to be and
no more.

The dream date for that lettuce would be the dry-aged Wagyu steak, slick
with melted beef fat and so tender it doesn't seem entirely real. It
might have made me dizzy if I hadn't already been reeling over the
price, \$70 for roughly 25 ounces of meat, sliced from the bone, that
two companions and I ate while luxuriating in the oak-and-steel embrace
of Wildair's stools. (The discomfort is more acute at the narrow
counters on either side of the dining room.)

For the average mortal's budget, the beef tartare is a more accessible
lettuce partner. Chopped into bits that are big enough to chew, the meat
is crunchy with toasted buckwheat, hot with fresh horseradish, and sharp
with gratings of tangy smoked Cheddar.

Though the two chefs seem liberated by
\href{http://www.nytimes3xbfgragh.onion/2015/09/09/dining/a-blurry-line-between-bar-and-restaurant.html}{cooking
for a simple wine bar}, this freedom sometimes leads them off the path.
Wet blobs of below-par sea urchin had no business glopping up an
excellent potato cake I ate in the early summer. The texture of whole
clams and flabby cubes of lardo over griddled bread with spinach wasn't
as appealing as the flavor. The kitchen may also want to cut the squid
and green onions into smaller pieces; you want to eat them with your
fingers, but they're so big you almost need a knife and fork.

Mr. von Hauske is the baker of the two. The dense white sourdough with
the fantastically crisp crust that he makes at Contra is sold at
Wildair, too, this time with a dish of salted olive oil that catches
nicely in your throat. For dessert, he has come up with a very likable
panna cotta; the sweetness comes from soft shavings of watermelon
granite and a crumble of caramelized milk. There's also a
chocolate-peanut butter tart that tastes exactly the way I've always
wished a Reese's Peanut Butter Cup tasted.

One sign of how well Wildair works is that nearly everybody I know who
has dropped in has a story about discovering a new wine. The central
character in most of these stories is Jorge Riera, the wine director
here and at Contra and a soft-spoken but ardent advocate for winemakers
who believe in letting nature have its way. I owe him for introducing me
to a biodynamic fino sherry from Andalusia and an elegantly
honeysuckle-scented grenache blanc from Catalonia, among others.

Mr. Stone and Mr. von Hauske named their wine bar for a prizewinning
racehorse that lived in the neighborhood before the Civil War. Mr. von
Hauske believes it should be pronounced ``willed air,'' but almost
everybody calls the place ``wild air.'' Maybe everybody is wrong, or
maybe this is just a case of a younger sibling claiming its own
identity.

Advertisement

\protect\hyperlink{after-bottom}{Continue reading the main story}

\hypertarget{site-index}{%
\subsection{Site Index}\label{site-index}}

\hypertarget{site-information-navigation}{%
\subsection{Site Information
Navigation}\label{site-information-navigation}}

\begin{itemize}
\tightlist
\item
  \href{https://help.nytimes3xbfgragh.onion/hc/en-us/articles/115014792127-Copyright-notice}{©~2020~The
  New York Times Company}
\end{itemize}

\begin{itemize}
\tightlist
\item
  \href{https://www.nytco.com/}{NYTCo}
\item
  \href{https://help.nytimes3xbfgragh.onion/hc/en-us/articles/115015385887-Contact-Us}{Contact
  Us}
\item
  \href{https://www.nytco.com/careers/}{Work with us}
\item
  \href{https://nytmediakit.com/}{Advertise}
\item
  \href{http://www.tbrandstudio.com/}{T Brand Studio}
\item
  \href{https://www.nytimes3xbfgragh.onion/privacy/cookie-policy\#how-do-i-manage-trackers}{Your
  Ad Choices}
\item
  \href{https://www.nytimes3xbfgragh.onion/privacy}{Privacy}
\item
  \href{https://help.nytimes3xbfgragh.onion/hc/en-us/articles/115014893428-Terms-of-service}{Terms
  of Service}
\item
  \href{https://help.nytimes3xbfgragh.onion/hc/en-us/articles/115014893968-Terms-of-sale}{Terms
  of Sale}
\item
  \href{https://spiderbites.nytimes3xbfgragh.onion}{Site Map}
\item
  \href{https://help.nytimes3xbfgragh.onion/hc/en-us}{Help}
\item
  \href{https://www.nytimes3xbfgragh.onion/subscription?campaignId=37WXW}{Subscriptions}
\end{itemize}
