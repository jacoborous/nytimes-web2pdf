Sections

SEARCH

\protect\hyperlink{site-content}{Skip to
content}\protect\hyperlink{site-index}{Skip to site index}

\href{https://www.nytimes3xbfgragh.onion/section/us}{U.S.}

\href{https://myaccount.nytimes3xbfgragh.onion/auth/login?response_type=cookie\&client_id=vi}{}

\href{https://www.nytimes3xbfgragh.onion/section/todayspaper}{Today's
Paper}

\href{/section/us}{U.S.}\textbar{}How Did George Floyd Die? Here's What
We Know

\url{https://nyti.ms/3dpT0qC}

\begin{itemize}
\item
\item
\item
\item
\item
\end{itemize}

\href{https://www.nytimes3xbfgragh.onion/news-event/george-floyd-protests-minneapolis-new-york-los-angeles?action=click\&pgtype=Article\&state=default\&region=TOP_BANNER\&context=storylines_menu}{Race
and America}

\begin{itemize}
\tightlist
\item
  \href{https://www.nytimes3xbfgragh.onion/2020/07/26/us/protests-portland-seattle-trump.html?action=click\&pgtype=Article\&state=default\&region=TOP_BANNER\&context=storylines_menu}{Protesters
  Return to Other Cities}
\item
  \href{https://www.nytimes3xbfgragh.onion/2020/07/24/us/portland-oregon-protests-white-race.html?action=click\&pgtype=Article\&state=default\&region=TOP_BANNER\&context=storylines_menu}{Portland
  at the Center}
\item
  \href{https://www.nytimes3xbfgragh.onion/2020/07/23/podcasts/the-daily/portland-protests.html?action=click\&pgtype=Article\&state=default\&region=TOP_BANNER\&context=storylines_menu}{Podcast:
  Showdown in Portland}
\item
  \href{https://www.nytimes3xbfgragh.onion/interactive/2020/07/16/us/black-lives-matter-protests-louisville-breonna-taylor.html?action=click\&pgtype=Article\&state=default\&region=TOP_BANNER\&context=storylines_menu}{45
  Days in Louisville}
\end{itemize}

Advertisement

\protect\hyperlink{after-top}{Continue reading the main story}

Supported by

\protect\hyperlink{after-sponsor}{Continue reading the main story}

\hypertarget{how-did-george-floyd-die-heres-what-we-know}{%
\section{How Did George Floyd Die? Here's What We
Know}\label{how-did-george-floyd-die-heres-what-we-know}}

A private autopsy commissioned by the family concluded that his death
was a homicide, brought about by compression of his neck and back by
Minneapolis police officers.

\includegraphics{https://static01.graylady3jvrrxbe.onion/images/2020/06/01/us/01UNREST-AUTOPSY2/merlin_173076843_c88cace0-f071-496c-9f21-67c9339af0b3-articleLarge.jpg?quality=75\&auto=webp\&disable=upscale}

By \href{https://www.nytimes3xbfgragh.onion/by/frances-robles}{Frances
Robles} and
\href{https://www.nytimes3xbfgragh.onion/by/audra-d-s-burch}{Audra D. S.
Burch}

\begin{itemize}
\item
  June 2, 2020
\item
  \begin{itemize}
  \item
  \item
  \item
  \item
  \item
  \end{itemize}
\end{itemize}

George Floyd, 46, died after being handcuffed on the street in the
custody of the Minneapolis police on Memorial Day. An officer responding
to a report that Mr. Floyd had tried to pass a fake \$20 bill helped
hold him down by lodging a knee on his neck for nearly nine minutes.

He died shortly after lapsing into unconsciousness.

But as protests continue to rage across the country demanding justice
for Mr. Floyd, dueling narratives have emerged about the precise cause
of his death. The criminal complaint supporting a murder charge for the
officer, which referred to the Hennepin County medical examiner's
preliminary findings, said the autopsy had discounted traumatic asphyxia
or strangulation as the cause of Mr. Floyd's death.

But on Monday afternoon, the lawyers representing his family presented a
very different version of how Mr. Floyd died. In their telling, three
officers on the scene killed Mr. Floyd and should be held criminally
responsible.

The private autopsy by doctors hired by Mr. Floyd's family determined
that he died not just because of the knee on his neck --- held there by
the officer, Derek Chauvin --- but also because of two other officers
who helped pin him down by applying pressure on his back. All three
officers were fired last week, as was a fourth officer at the scene.

The cause of death, according to the private autopsy, was mechanical
asphyxia and the manner of death was homicide.

Shortly after the family's autopsy findings were announced, the Hennepin
County medical examiner released its own findings, also concluding that
the manner of death was homicide. The county attributed the cause of
death to ``cardiopulmonary arrest complicating law enforcement subdual,
restraint, and neck compression.''

In other words, Mr. Floyd's heart stopped beating and his lungs stopped
taking in air while he was being restrained by law enforcement. The
one-page summary also noted that Mr. Floyd was intoxicated with fentanyl
and had recently used methamphetamines.

The criminal complaint said that the autopsy ``revealed no physical
findings that support a diagnosis of traumatic asphyxia or
strangulation.'' Mr. Floyd, the complaint said, had underlying health
conditions, including coronary artery disease and hypertensive heart
disease.

A 10-minute cellphone video of the encounter showed Mr. Chauvin holding
his knee on the back of Mr. Floyd's neck for
\href{https://www.nytimes3xbfgragh.onion/video/us/100000007159353/george-floyd-arrest-death-video.html}{eight
minutes and 46 seconds}. Mr. Floyd was unresponsive for nearly three
minutes of the encounter after he repeatedly gasped, ``I can't
breathe.''

The findings by the family's private medical examiners directly
contradict the report that there was no asphyxia, said Dr. Allecia M.
Wilson, of the University of Michigan, one of the doctors who examined
his body. The physical evidence showed that the pressure applied led to
his death, she said. In an interview, Dr. Michael Baden, who also
participated in the private autopsy, said there was also some
hemorrhaging around the right carotid area.

Although she has not had access to the full medical examiner's report,
Dr. Wilson said: ``We have seen accounts from the complaint and based on
that, yes our findings do differ. Some of the information I read from
that complaint states that there was no evidence of traumatic asphyxia.
This is the point in which we do disagree. There is evidence in this
case of mechanical or traumatic asphyxia.''

She noted that she did not have access to toxicology results, tissue
samples or some organs. Those items are not likely to change the
results, she said.

The private doctors also said that any underlying conditions Mr. Floyd
had did not kill him or contribute to his death.

``He was in good health,'' Dr. Baden said.

The private autopsy concluded that even without evidence of
``traumatic'' asphyxia, such as broken bones, the compression caused by
the officers still led to Mr. Floyd's death by depriving his brain of
blood and oxygen and his lungs of air.

But Dr. Baden acknowledged that the pressure was not necessarily visible
in the autopsy because by the time any doctors reviewed the body, the
pressure had been released. He added that abrasions on the left side of
Mr. Floyd's face and shoulder showed how hard he was pressed against the
pavement.

``The video is real,'' Dr. Baden said.

The doctors said the determination of cause and the manner of death were
based on reviewing the circumstances, including the video, but also
additional findings determined in their autopsy.

``We do have physical evidence that supports that there was pressure
applied to his neck,'' Dr. Wilson said.w

Dr. Baden added, ``The main thing is that we both say it's a homicide
due to the way he was being subdued.''

Conducting the examination for the family were Dr. Wilson, director of
autopsy and forensic services at the University of Michigan, and Dr.
Baden, who was the chief medical examiner for New York City in the late
1970s. Dr. Baden is best known as host of the HBO television series
``Autopsy.'' He also did the private autopsy of Michael Brown, who was
killed by a police officer in Ferguson, Mo., in 2014.

It is not unusual, he said, for autopsy reports to differ. ``This is not
necessarily science, it's medicine. Medicine is an art and a science.''

Ben Crump, the lawyer representing the Floyd family, said Mr. Floyd's
death was the result of three officers who came into physical contact
with him, compressing his neck and back.

Mr. Crump said Mr. Chauvin should be charged with first-degree murder
rather than the lesser charge of third-degree murder he is facing, and
that the other officers should face the ``fullest extent of the law.''

He said that the family was distrusting of the Minneapolis Police
Department based, in part, on the sanitized way it described how Mr.
Floyd was detained, and that it was ``encouraged'' that Attorney General
Keith Ellison of Minnesota was taking over the prosecution.

The day after Mr. Floyd's death,
\href{https://slack-redir.net/link?url=https\%3A\%2F\%2Fwww.nytimes3xbfgragh.onion\%2F2020\%2F05\%2F28\%2Fus\%2Fgeorge-floyd-minneapolis-protests.html}{the
police statement described him} as a suspected forger who ``appeared to
be under the influence,'' who ``physically resisted officers'' and who
appeared to be ``suffering medical distress.'' It never mentioned any
officer's role in his death.

Hours later, a graphic video emerged that
\href{https://www.nytimes3xbfgragh.onion/2020/05/26/us/george-floyd-minneapolis-police.html}{showed
a significantly different account}.

Nicholas Bogel-Burroughs contributed reporting.

Advertisement

\protect\hyperlink{after-bottom}{Continue reading the main story}

\hypertarget{site-index}{%
\subsection{Site Index}\label{site-index}}

\hypertarget{site-information-navigation}{%
\subsection{Site Information
Navigation}\label{site-information-navigation}}

\begin{itemize}
\tightlist
\item
  \href{https://help.nytimes3xbfgragh.onion/hc/en-us/articles/115014792127-Copyright-notice}{©~2020~The
  New York Times Company}
\end{itemize}

\begin{itemize}
\tightlist
\item
  \href{https://www.nytco.com/}{NYTCo}
\item
  \href{https://help.nytimes3xbfgragh.onion/hc/en-us/articles/115015385887-Contact-Us}{Contact
  Us}
\item
  \href{https://www.nytco.com/careers/}{Work with us}
\item
  \href{https://nytmediakit.com/}{Advertise}
\item
  \href{http://www.tbrandstudio.com/}{T Brand Studio}
\item
  \href{https://www.nytimes3xbfgragh.onion/privacy/cookie-policy\#how-do-i-manage-trackers}{Your
  Ad Choices}
\item
  \href{https://www.nytimes3xbfgragh.onion/privacy}{Privacy}
\item
  \href{https://help.nytimes3xbfgragh.onion/hc/en-us/articles/115014893428-Terms-of-service}{Terms
  of Service}
\item
  \href{https://help.nytimes3xbfgragh.onion/hc/en-us/articles/115014893968-Terms-of-sale}{Terms
  of Sale}
\item
  \href{https://spiderbites.nytimes3xbfgragh.onion}{Site Map}
\item
  \href{https://help.nytimes3xbfgragh.onion/hc/en-us}{Help}
\item
  \href{https://www.nytimes3xbfgragh.onion/subscription?campaignId=37WXW}{Subscriptions}
\end{itemize}
