Sections

SEARCH

\protect\hyperlink{site-content}{Skip to
content}\protect\hyperlink{site-index}{Skip to site index}

\href{https://www.nytimes3xbfgragh.onion/section/health}{Health}

\href{https://myaccount.nytimes3xbfgragh.onion/auth/login?response_type=cookie\&client_id=vi}{}

\href{https://www.nytimes3xbfgragh.onion/section/todayspaper}{Today's
Paper}

\href{/section/health}{Health}\textbar{}Six Months of Coronavirus:
Here's Some of What We've Learned

\url{https://nyti.ms/3dqEI9f}

\begin{itemize}
\item
\item
\item
\item
\item
\item
\end{itemize}

\hypertarget{the-coronavirus-outbreak}{%
\subsubsection{\texorpdfstring{\href{https://www.nytimes3xbfgragh.onion/news-event/coronavirus?name=styln-coronavirus-national\&region=TOP_BANNER\&block=storyline_menu_recirc\&action=click\&pgtype=Article\&impression_id=5a85b8d0-efba-11ea-a4c9-41c298b92ba2\&variant=undefined}{The
Coronavirus
Outbreak}}{The Coronavirus Outbreak}}\label{the-coronavirus-outbreak}}

\begin{itemize}
\tightlist
\item
  live\href{https://www.nytimes3xbfgragh.onion/2020/09/05/world/coronavirus-covid.html?name=styln-coronavirus-national\&region=TOP_BANNER\&block=storyline_menu_recirc\&action=click\&pgtype=Article\&impression_id=5a85dfe0-efba-11ea-a4c9-41c298b92ba2\&variant=undefined}{Latest
  Updates}
\item
  \href{https://www.nytimes3xbfgragh.onion/interactive/2020/us/coronavirus-us-cases.html?name=styln-coronavirus-national\&region=TOP_BANNER\&block=storyline_menu_recirc\&action=click\&pgtype=Article\&impression_id=5a85dfe1-efba-11ea-a4c9-41c298b92ba2\&variant=undefined}{Maps
  and Cases}
\item
  \href{https://www.nytimes3xbfgragh.onion/interactive/2020/science/coronavirus-vaccine-tracker.html?name=styln-coronavirus-national\&region=TOP_BANNER\&block=storyline_menu_recirc\&action=click\&pgtype=Article\&impression_id=5a85dfe2-efba-11ea-a4c9-41c298b92ba2\&variant=undefined}{Vaccine
  Tracker}
\item
  \href{https://www.nytimes3xbfgragh.onion/2020/09/02/your-money/eviction-moratorium-covid.html?name=styln-coronavirus-national\&region=TOP_BANNER\&block=storyline_menu_recirc\&action=click\&pgtype=Article\&impression_id=5a85dfe3-efba-11ea-a4c9-41c298b92ba2\&variant=undefined}{Eviction
  Moratorium}
\item
  \href{https://www.nytimes3xbfgragh.onion/interactive/2020/09/02/magazine/food-insecurity-hunger-us.html?name=styln-coronavirus-national\&region=TOP_BANNER\&block=storyline_menu_recirc\&action=click\&pgtype=Article\&impression_id=5a85dfe4-efba-11ea-a4c9-41c298b92ba2\&variant=undefined}{American
  Hunger}
\end{itemize}

Advertisement

\protect\hyperlink{after-top}{Continue reading the main story}

Supported by

\protect\hyperlink{after-sponsor}{Continue reading the main story}

\hypertarget{six-months-of-coronavirus-heres-some-of-what-weve-learned}{%
\section{Six Months of Coronavirus: Here's Some of What We've
Learned}\label{six-months-of-coronavirus-heres-some-of-what-weve-learned}}

Much remains unknown and mysterious, but these are some of the things
we're pretty sure of after half a year of this pandemic.

\includegraphics{https://static01.graylady3jvrrxbe.onion/images/2020/06/02/science/02SCI-CV-KNOW-promo/02SCI-CV-KNOW-promo-threeByTwoMediumAt2X.jpg}

By The New York Times

\begin{itemize}
\item
  June 18, 2020
\item
  \begin{itemize}
  \item
  \item
  \item
  \item
  \item
  \item
  \end{itemize}
\end{itemize}

\href{https://www.nytimes3xbfgragh.onion/es/2020/06/04/espanol/datos-coronavirus.html}{Leer
en español}

\emph{{[}Read our live updates on}
\href{https://www.nytimes3xbfgragh.onion/2020/06/18/world/coronavirus-cases-usa-world.html}{\emph{Coronavirus
cases and deaths}} \emph{globally.{]}}

We don't really know when the novel coronavirus first began infecting
people. But as we turn a page on our calendars into June, it is fair to
say that
\href{https://www.nytimes3xbfgragh.onion/2020/06/02/health/coronavirus-profile-covid.html}{Sars-Cov-2
has been with us now for a full six months}.

At first, it had no name or true identity. Early in January, news
reports referred to
\href{https://www.nytimes3xbfgragh.onion/2020/01/06/world/asia/china-SARS-pneumonialike.html}{strange
and threatening symptoms that had sickened dozens of people in a large
Chinese city} with which many people in the world were probably not
familiar. After half a year, that large metropolis, Wuhan, is
well-known, as is the coronavirus and the illness it causes, Covid-19.

In that time, many reporters and editors on the health and science desk
at The New York Times have shifted our journalistic focus as we have
sought to tell the story of the coronavirus pandemic. While
\href{https://www.nytimes3xbfgragh.onion/2020/06/01/health/coronavirus-mysteries.html}{much
remains unknown and mysterious} after six months, there are some things
we're pretty sure of. These are some of those insights.

\hypertarget{here-are-some-things-we-think-we-know-about-coronavirus}{%
\subsubsection{Here are some things we think we know about
coronavirus:}\label{here-are-some-things-we-think-we-know-about-coronavirus}}

\begin{itemize}
\tightlist
\item
  \protect\hyperlink{link-2ee4ffe7}{We'll have to live with this for a
  long time.}
\item
  \protect\hyperlink{link-6ec3dc3a}{You should be wearing a mask.}
\item
  \protect\hyperlink{link-5d80e42a}{American public health
  infrastructure needs an update.}
\item
  \protect\hyperlink{link-1437f014}{Responding to the virus is
  extraordinarily expensive.}
\item
  \protect\hyperlink{link-1da03d75}{We have a long way to go to fix
  virus testing.}
\item
  \protect\hyperlink{link-16ac6f7f}{We can't count on herd immunity to
  keep us healthy.}
\item
  \protect\hyperlink{link-6817bab5}{The virus produces more symptoms
  than expected.}
\item
  \protect\hyperlink{link-52bb8dd1}{We can worry a bit less about
  infection from surfaces.}
\item
  \protect\hyperlink{link-10fd5d3e}{We can also worry less about a
  mutating virus.}
\item
  \protect\hyperlink{link-34a1dcc4}{We can't count on warm weather to
  defeat the virus.}
\end{itemize}

\hypertarget{well-have-to-live-with-this-for-a-long-time}{%
\subsection{We'll have to live with this for a long
time.}\label{well-have-to-live-with-this-for-a-long-time}}

By Denise Grady

\includegraphics{https://static01.graylady3jvrrxbe.onion/images/2020/06/02/science/02SCI-CV-KNOW3/merlin_172974780_ccf9c542-41fd-4e2a-9921-475d23f8d5c0-articleLarge.jpg?quality=75\&auto=webp\&disable=upscale}

Summer is almost here, states are reopening and new coronavirus cases
are declining or, at least, holding steady in many parts of the United
States. At
\href{https://www.nytimes3xbfgragh.onion/interactive/2020/05/20/science/coronavirus-vaccine-development.html?searchResultPosition=6}{least
100 scientific teams} around the world are racing to develop a vaccine.

That's about it for the good news.

The virus has shown no sign of going away: We will be in this pandemic
era for the long haul, likely a year or more. The masks, the social
distancing, the fretful hand-washing, the aching withdrawal from friends
and family --- those steps are still the best hope of staying well, and
will be for some time to come.

``This virus just may become another endemic virus in our communities,
and this virus may never go away,'' Dr. Mike Ryan, the executive
director of the World Health Organization's health emergencies program,
warned last month. Some scientists think that the longer we live with
the virus,
\href{https://www.nytimes3xbfgragh.onion/2020/04/18/health/coronavirus-america-future.html}{the
milder its effects will become,} but that remains to be seen.

Predictions that millions of doses of a vaccine may be available by the
end of this year may be too rosy. No vaccine has ever been created that
fast.

The disease would be less frightening if there were a treatment that
could cure it or, at least, prevent severe illness. But there is not.
Remdesivir, the eagerly awaited antiviral drug? ``Modest'' benefit is
the highest mark experts give it.

Which brings us back to masks and social distancing, which have come to
feel quite antisocial. If only we could go back to life the way it used
to be.

We cannot. Not yet. There are just enough wild cards with this disease
--- perfectly healthy adults and children who inexplicably become very,
very sick --- that no one can afford to be cavalier about catching it.
About
\href{https://www.cdc.gov/coronavirus/2019-ncov/hcp/planning-scenarios.html}{35
percent of infected people have no symptoms} at all, so if they are out
and about, they could unknowingly infect other people.

Enormous questions loom. Can workplaces be made safe? What about trains,
subways, airplanes, school buses? How many people can work from home?
When would it be safe to reopen schools? How do you get a 6-year-old
with the attention span of a squirrel to socially distance?

The bottom line: Wear a mask, keep your distance. When the time comes in
the fall, get a flu shot, to protect yourself from one respiratory
disease you can avoid and to help keep emergency rooms and urgent care
from being overwhelmed. Hope for a treatment, a cure, a vaccine. Be
patient. We have to pace ourselves. If there's such a thing as a disease
marathon, this is it.

\hypertarget{you-should-be-wearing-a-mask}{%
\subsection{You should be wearing a
mask.}\label{you-should-be-wearing-a-mask}}

By Knvul Sheikh

Image

Credit...Jens Mortensen for The New York Times

The debate over whether Americans should wear face masks to control
coronavirus transmission has been settled. Although public health
authorities gave confusing and often contradictory advice in the early
months of the pandemic, most experts now agree that if everyone wears a
mask, individuals protect one another.

\hypertarget{latest-updates-the-coronavirus-outbreak}{%
\section{\texorpdfstring{\href{https://www.nytimes3xbfgragh.onion/2020/09/04/world/covid-19-coronavirus.html?action=click\&pgtype=Article\&state=default\&region=MAIN_CONTENT_1\&context=storylines_live_updates}{Latest
Updates: The Coronavirus
Outbreak}}{Latest Updates: The Coronavirus Outbreak}}\label{latest-updates-the-coronavirus-outbreak}}

Updated 2020-09-05T12:05:40.998Z

\begin{itemize}
\tightlist
\item
  \href{https://www.nytimes3xbfgragh.onion/2020/09/04/world/covid-19-coronavirus.html?action=click\&pgtype=Article\&state=default\&region=MAIN_CONTENT_1\&context=storylines_live_updates\#link-1654f6ad}{Research
  connects vaping to a higher chance of catching the virus --- and
  suffering its worst effects.}
\item
  \href{https://www.nytimes3xbfgragh.onion/2020/09/04/world/covid-19-coronavirus.html?action=click\&pgtype=Article\&state=default\&region=MAIN_CONTENT_1\&context=storylines_live_updates\#link-52e4198a}{Another
  college football game won't be played as planned.}
\item
  \href{https://www.nytimes3xbfgragh.onion/2020/09/04/world/covid-19-coronavirus.html?action=click\&pgtype=Article\&state=default\&region=MAIN_CONTENT_1\&context=storylines_live_updates\#link-181cef0}{Pharmaceutical
  companies plan a joint pledge on safety standards as they move
  vaccines to the marketplace.}
\end{itemize}

\href{https://www.nytimes3xbfgragh.onion/2020/09/04/world/covid-19-coronavirus.html?action=click\&pgtype=Article\&state=default\&region=MAIN_CONTENT_1\&context=storylines_live_updates}{See
more updates}

More live coverage:
\href{https://www.nytimes3xbfgragh.onion/live/2020/09/04/business/stock-market-today-coronavirus?action=click\&pgtype=Article\&state=default\&region=MAIN_CONTENT_1\&context=storylines_live_updates}{Markets}

Researchers know that even simple masks can effectively stop droplets
spewing from an infected wearer's nose or mouth. In a
\href{https://www.nature.com/articles/s41591-020-0843-2}{study published
in April in Nature}, scientists showed that when people who are infected
with influenza, rhinovirus or a mild cold-causing coronavirus wore a
mask, it blocked nearly 100 percent of the viral droplets they exhaled,
as well as some tiny aerosol particles.

Still, mask wearing remains uneven in many parts of the United States.
But governments and businesses are beginning to require, or at least
recommend, that masks be worn in many public settings.

There is also growing evidence that some kinds of masks may protect you
from other people's germs. High-grade N95 masks are cleared by federal
public health agencies because they filter out at least 95 percent of
particles that are 0.3 microns in diameter when properly worn. One study
showed that N95s were able to capture
\href{https://www.ajicjournal.org/article/S0196-6553(05)00911-9/fulltext\#articleInformation}{over
90 percent of viral particles}, even if the particles were about
one-fifth the size of a coronavirus. Other studies have shown that flat,
blue surgical masks block between 50 to 80 percent of particles, whereas
\href{https://academic.oup.com/annweh/article/54/7/789/202744}{cloth
masks block 10 to 30 percent} of tiny particles.

``Wearing a mask is better than nothing,'' said Dr. Robert Atmar, an
infectious disease specialist at Baylor College of Medicine. Because the
coronavirus typically infects people by entering their body through the
mouth and nose, covering these areas can act as the first line of
defense against the virus, he said.

Donning a face covering is also likely to prevent you from touching your
face, which is another way the coronavirus can be transmitted from
contaminated surfaces to unsuspecting individuals. And when combined
with hand washing and other protective measures, such as
\href{https://www.nytimes3xbfgragh.onion/2020/04/22/us/politics/social-distancing-coronavirus.html}{social
distancing}, masks help reduce the transmission of disease, Dr. Atmar
said.

\hypertarget{american-public-health-infrastructure-needs-an-update}{%
\subsection{American public health infrastructure needs an
update.}\label{american-public-health-infrastructure-needs-an-update}}

By Donald G. McNeil Jr.

Image

Credit...Jens Mortensen for The New York Times

The United States knows how to fight wars. But, as the past few months
have shown, the American response to pathogens can easily become a
shambles --- even though pathogens kill more Americans than many wars
have.

We have no viral Pentagon. The Centers for Disease Control and
Prevention is more of an F.B.I. for outbreak investigations than a war
machine. For years --- under both the Obama and Trump administrations
--- its leaders have had to seek clearance for almost every utterance.

Dr. Anthony S. Fauci, the most prominent of the doctors advising the
coronavirus task force, is actually the head of a research institute,
the National Institute for Allergy and Infectious Disease, rather than
of the medical equivalent of a combat battalion.

The Surgeon General is essentially an admiral without a crew. He
dispenses health warnings and recommendations, but the Public Health
Services Commissioned Corps, which reports to him, are only about 6,500
strong, and many members have other jobs, often at the C.D.C.

Almost all the front-line troops --- the contact tracers, the laboratory
technicians, the epidemiologists, the staff in state and city hospitals
--- are paid by state and local health departments whose budgets have
shriveled for years. These soldiers are led by 50 commanders, in the
form of governors, and with that many in charge, it is amazing that any
response moves forward.

The rest of the response is in the hands of thousands of private
militias --- hospitals, insurers, doctors, nurses, respiratory
technicians, pharmacists and so on, all of whom have individual
employers. Within limits, they can do what they want. When they cannot
get something they need from overseas, they are largely powerless
without federal logistical help.

As war does to defeated nations, pandemics expose the weaknesses of
their systems. Our patchwork and uncoordinated response has produced
more than 100,000 deaths; surely we can do better.

``The superpowers have their priorities all wrong,'' Dr. Michael Ryan,
the head of the W.H.O.'s emergencies program, said recently.

``They spend billions on missiles and submarines, and on fighting
terrorism, and pennies on viruses. You can start peace talks with your
enemy. You can change your policies to lessen the threat of terrorism.
But you cannot negotiate with a virus, and we know that new threats are
coming along every year.''

\textbf{\emph{{[}}\href{http://on.fb.me/1paTQ1h}{\emph{Like the Science
Times page on Facebook.}}} ****** \emph{\textbar{} Sign up for the}
\textbf{\href{http://nyti.ms/1MbHaRU}{\emph{Science Times
newsletter.}}\emph{{]}}}

\hypertarget{responding-to-the-virus-is-extraordinarily-expensive}{%
\subsection{Responding to the virus is extraordinarily
expensive.}\label{responding-to-the-virus-is-extraordinarily-expensive}}

By Reed Abelson

The federal government has spent hundreds of billions of dollars
and\href{https://www.nytimes3xbfgragh.onion/2020/04/24/business/congress-coronavirus-stimulus-bill.html}{promised
to spend more than \$2 trillion} to address the coronavirus pandemic.

Of that money,
\href{https://www.nytimes3xbfgragh.onion/2020/05/21/health/coronavirus-vaccine-astrazeneca.html}{\$2
billion has gone to helping companies develop new vaccines,}expanding
testing capacity nationwide and shoring up the economic fallout since
the beginning of March. (Even more could be on the way, but
\href{https://www.nytimes3xbfgragh.onion/2020/05/15/us/politics/house-simulus-vote.html}{how
much and when is unclear}.)

The vast majority of this spending has been aimed at blunting the
economic pain of small businesses shutting down and people losing their
jobs or being furloughed. Congress also provided additional money for
Medicaid and other social programs.

Hospitals, community health centers and other providers have been
allocated \$175 billion to cover the cost of caring for patients with
Covid-19 and for the visits, procedures and surgeries that were canceled
because of the pandemic. In the latest bill, \$25 billion was targeted
for coronavirus testing.

Many experts say more funding is needed, but there is ample controversy
over how the money already allocated is being spent and which entities
are getting funds. Various groups like the Committee for a Responsible
Federal
Budget\href{http://www.crfb.org/blogs/covid-money-tracker-policies-enacted-to-date}{are
tracking the spending.} By that organization's calculation, roughly
\$1.6 trillion has already been disbursed or committed. The Federal
Reserve has also provided more than \$2 trillion in emergency lending,
asset purchases and other activities, it said.

\hypertarget{we-have-a-long-way-to-go-to-fix-virus-testing}{%
\subsection{We have a long way to go to fix virus
testing.}\label{we-have-a-long-way-to-go-to-fix-virus-testing}}

By Katie Thomas

The landscape for testing looks far better than it did in the early days
of the outbreak, when
\href{https://www.nytimes3xbfgragh.onion/2020/03/28/us/testing-coronavirus-pandemic.html}{a
botched rollout of coronavirus tests} failed to detect the spread of the
virus in the United States.

Today, \href{https://covidtracking.com/data/us-daily}{hundreds of
thousands of tests a day are being conducted in the United States}, and
in some areas it is so widely available that public health officials
have complained
\href{https://www.washingtonpost.com/health/as-coronavirus-testing-expands-a-new-problem-arises-not-enough-people-to-test/2020/05/17/3f3297de-8bcd-11ea-8ac1-bfb250876b7a_story.html}{they
do not have enough takers}. In Los Angeles, where testing is free to
everyone,
\href{https://www.latimes.com/california/story/2020-05-26/los-angeles-opens-largest-coronavirus-testing-site-at-dodger-stadiu}{a
drive-through site at Dodgers Stadium} can process 6,000 people a day.

The range of tests available is also expanding. Tests that once required
a health care worker to insert a swab through the nose to the back of
the throat can now be done with a swipe inside the nose, or by spitting
into a cup. A handful of companies
\href{https://www.nytimes3xbfgragh.onion/2020/04/21/health/fda-in-home-test-coronavirus.html}{now
sell at-home test kits}, and
\href{https://www.nytimes3xbfgragh.onion/interactive/2020/05/12/us/coronavirus-testing-white-house.html?action=click\&module=RelatedLinks\&pgtype=Article}{a
test from Abbott} can detect the virus in as little as five minutes.

In addition to the tests that detect active infections, Americans can
also get tested for antibodies to the virus, which shows whether they
have ever been infected, and could help give a better picture for how
widely the coronavirus has spread in communities.

But despite this progress, the United States still has a long way to go.
Public health experts say that anywhere from
\href{https://www.npr.org/sections/health-shots/2020/05/07/851610771/u-s-coronavirus-testing-still-falls-short-hows-your-state-doing}{900,000
tests} to
\href{https://www.newyorker.com/news/q-and-a/paul-romer-on-how-to-survive-the-chaos-of-the-coronavirus}{millions
a day} will be needed to screen hospital patients, nursing home
residents and employees returning to work.

And even as testing is abundant in some areas, it is still hard to come
by in others. Shortages of key supplies needed to run the tests --- such
as swabs and chemical reagents --- have persisted. The
\href{https://www.nytimes3xbfgragh.onion/2020/05/25/health/coronavirus-testing-trump.html}{federal
government has effectively delegated oversight} to the states, creating
a patchwork of policies and putting states in competition with one
another. Even tracking the number of tests conducted has proved
difficult,
\href{https://www.nytimes3xbfgragh.onion/2020/05/22/us/politics/coronavirus-tests-cdc.html}{after
the C.D.C.}and several states began lumping tests for the virus as well
as antibodies together, to the bafflement of epidemiologists trying to
track active infections, which the antibody tests do not show.

\hypertarget{we-cant-count-on-herd-immunity-to-keep-us-healthy}{%
\subsection{We can't count on herd immunity to keep us
healthy.}\label{we-cant-count-on-herd-immunity-to-keep-us-healthy}}

By Gina Kolata

The idea is simplicity itself: If enough of the population has
antibodies to the novel coronavirus, the virus will hit too many dead
ends to continue infecting people.
\href{https://www.nytimes3xbfgragh.onion/interactive/2020/05/28/upshot/coronavirus-herd-immunity.html}{That
is herd immunity.}

That is the great hope for a vaccine. But it may not happen, even if a
vaccine becomes available, as experience with flu vaccines shows.

Dr. Paul Offit of Children's Hospital of Philadelphia and the University
of Pennsylvania noted that while vaccines eliminated measles, rubella
and smallpox and almost eliminated polio in the United States, vaccines
against influenza and whooping cough have not stopped outbreaks. (With
some parents declining measles vaccines, the disease is coming back.)

Influenza and whooping cough have spread, even after enough people in a
community have been vaccinated to, in theory, stop the diseases. That's
because the antibodies that protect people against viruses infecting
mucosal surfaces like the lining of the nose tend to be short-lived.

Vaccines against respiratory diseases are, at best, modestly effective,
agreed Dr. Arnold Monto of the University of Michigan,

\href{https://www.nytimes3xbfgragh.onion/news-event/coronavirus?action=click\&pgtype=Article\&state=default\&region=MAIN_CONTENT_3\&context=storylines_faq}{}

\hypertarget{the-coronavirus-outbreak-}{%
\subsubsection{The Coronavirus Outbreak
›}\label{the-coronavirus-outbreak-}}

\hypertarget{frequently-asked-questions}{%
\paragraph{Frequently Asked
Questions}\label{frequently-asked-questions}}

Updated September 4, 2020

\begin{itemize}
\item ~
  \hypertarget{what-are-the-symptoms-of-coronavirus}{%
  \paragraph{What are the symptoms of
  coronavirus?}\label{what-are-the-symptoms-of-coronavirus}}

  \begin{itemize}
  \tightlist
  \item
    In the beginning, the coronavirus
    \href{https://www.nytimes3xbfgragh.onion/article/coronavirus-facts-history.html?action=click\&pgtype=Article\&state=default\&region=MAIN_CONTENT_3\&context=storylines_faq\#link-6817bab5}{seemed
    like it was primarily a respiratory illness}~--- many patients had
    fever and chills, were weak and tired, and coughed a lot, though
    some people don't show many symptoms at all. Those who seemed
    sickest had pneumonia or acute respiratory distress syndrome and
    received supplemental oxygen. By now, doctors have identified many
    more symptoms and syndromes. In April,
    \href{https://www.nytimes3xbfgragh.onion/2020/04/27/health/coronavirus-symptoms-cdc.html?action=click\&pgtype=Article\&state=default\&region=MAIN_CONTENT_3\&context=storylines_faq}{the
    C.D.C. added to the list of early signs}~sore throat, fever, chills
    and muscle aches. Gastrointestinal upset, such as diarrhea and
    nausea, has also been observed. Another telltale sign of infection
    may be a sudden, profound diminution of one's
    \href{https://www.nytimes3xbfgragh.onion/2020/03/22/health/coronavirus-symptoms-smell-taste.html?action=click\&pgtype=Article\&state=default\&region=MAIN_CONTENT_3\&context=storylines_faq}{sense
    of smell and taste.}~Teenagers and young adults in some cases have
    developed painful red and purple lesions on their fingers and toes
    --- nicknamed ``Covid toe'' --- but few other serious symptoms.
  \end{itemize}
\item ~
  \hypertarget{why-is-it-safer-to-spend-time-together-outside}{%
  \paragraph{Why is it safer to spend time together
  outside?}\label{why-is-it-safer-to-spend-time-together-outside}}

  \begin{itemize}
  \tightlist
  \item
    \href{https://www.nytimes3xbfgragh.onion/2020/05/15/us/coronavirus-what-to-do-outside.html?action=click\&pgtype=Article\&state=default\&region=MAIN_CONTENT_3\&context=storylines_faq}{Outdoor
    gatherings}~lower risk because wind disperses viral droplets, and
    sunlight can kill some of the virus. Open spaces prevent the virus
    from building up in concentrated amounts and being inhaled, which
    can happen when infected people exhale in a confined space for long
    stretches of time, said Dr. Julian W. Tang, a virologist at the
    University of Leicester.
  \end{itemize}
\item ~
  \hypertarget{why-does-standing-six-feet-away-from-others-help}{%
  \paragraph{Why does standing six feet away from others
  help?}\label{why-does-standing-six-feet-away-from-others-help}}

  \begin{itemize}
  \tightlist
  \item
    The coronavirus spreads primarily through droplets from your mouth
    and nose, especially when you cough or sneeze. The C.D.C., one of
    the organizations using that measure,
    \href{https://www.nytimes3xbfgragh.onion/2020/04/14/health/coronavirus-six-feet.html?action=click\&pgtype=Article\&state=default\&region=MAIN_CONTENT_3\&context=storylines_faq}{bases
    its recommendation of six feet}~on the idea that most large droplets
    that people expel when they cough or sneeze will fall to the ground
    within six feet. But six feet has never been a magic number that
    guarantees complete protection. Sneezes, for instance, can launch
    droplets a lot farther than six feet,
    \href{https://jamanetwork.com/journals/jama/fullarticle/2763852}{according
    to a recent study}. It's a rule of thumb: You should be safest
    standing six feet apart outside, especially when it's windy. But
    keep a mask on at all times, even when you think you're far enough
    apart.
  \end{itemize}
\item ~
  \hypertarget{i-have-antibodies-am-i-now-immune}{%
  \paragraph{I have antibodies. Am I now
  immune?}\label{i-have-antibodies-am-i-now-immune}}

  \begin{itemize}
  \tightlist
  \item
    As of right
    now,\href{https://www.nytimes3xbfgragh.onion/2020/07/22/health/covid-antibodies-herd-immunity.html?action=click\&pgtype=Article\&state=default\&region=MAIN_CONTENT_3\&context=storylines_faq}{~that
    seems likely, for at least several months.}~There have been
    frightening accounts of people suffering what seems to be a second
    bout of Covid-19. But experts say these patients may have a
    drawn-out course of infection, with the virus taking a slow toll
    weeks to months after initial exposure.~People infected with the
    coronavirus typically
    \href{https://www.nature.com/articles/s41586-020-2456-9}{produce}~immune
    molecules called antibodies, which are
    \href{https://www.nytimes3xbfgragh.onion/2020/05/07/health/coronavirus-antibody-prevalence.html?action=click\&pgtype=Article\&state=default\&region=MAIN_CONTENT_3\&context=storylines_faq}{protective
    proteins made in response to an
    infection}\href{https://www.nytimes3xbfgragh.onion/2020/05/07/health/coronavirus-antibody-prevalence.html?action=click\&pgtype=Article\&state=default\&region=MAIN_CONTENT_3\&context=storylines_faq}{.
    These antibodies may}~last in the body
    \href{https://www.nature.com/articles/s41591-020-0965-6}{only two to
    three months}, which may seem worrisome, but that's~perfectly normal
    after an acute infection subsides, said Dr. Michael Mina, an
    immunologist at Harvard University. It may be possible to get the
    coronavirus again, but it's highly unlikely that it would be
    possible in a short window of time from initial infection or make
    people sicker the second time.
  \end{itemize}
\item ~
  \hypertarget{what-are-my-rights-if-i-am-worried-about-going-back-to-work}{%
  \paragraph{What are my rights if I am worried about going back to
  work?}\label{what-are-my-rights-if-i-am-worried-about-going-back-to-work}}

  \begin{itemize}
  \tightlist
  \item
    Employers have to provide
    \href{https://www.osha.gov/SLTC/covid-19/standards.html}{a safe
    workplace}~with policies that protect everyone equally.
    \href{https://www.nytimes3xbfgragh.onion/article/coronavirus-money-unemployment.html?action=click\&pgtype=Article\&state=default\&region=MAIN_CONTENT_3\&context=storylines_faq}{And
    if one of your co-workers tests positive for the coronavirus, the
    C.D.C.}~has said that
    \href{https://www.cdc.gov/coronavirus/2019-ncov/community/guidance-business-response.html}{employers
    should tell their employees}~-\/- without giving you the sick
    employee's name -\/- that they may have been exposed to the virus.
  \end{itemize}
\end{itemize}

Since the coronavirus usually starts by infecting the respiratory
system, Dr. Monto suspects that a Covid-19 vaccine would have a similar
effect to a flu vaccine --- it will reduce the incidence of the disease
and make it less severe on average, but it will not make Covid-19 go
away.

He would like the virus to disappear, of course, but a vaccine that
reduces the disease's spread and severity is a lot better than nothing.

``As an older person, what I want is not to end up on a respirator,''
Dr. Monto said.

\hypertarget{the-virus-produces-more-symptoms-than-expected}{%
\subsection{The virus produces more symptoms than
expected.}\label{the-virus-produces-more-symptoms-than-expected}}

By Roni Caryn Rabin

Image

Credit...Jens Mortensen for The New York Times

Covid-19 is a viral respiratory illness. Many early descriptions of
symptoms focused on patients being short of breath and eventually being
placed on ventilators. But the virus does not confine its assault to the
lungs, and doctors have identified a number of symptoms and syndromes
associated with it.

In some patients, the virus propels the immune system into overdrive,
causing the lungs to fill with fluid and damaging multiple organs,
including the brain, heart, kidneys and liver.

The first symptoms of an infection are usually a cough and shortness of
breath. But in April
\href{https://www.nytimes3xbfgragh.onion/2020/04/27/health/coronavirus-symptoms-cdc.html}{the
C.D.C. added to the list of early signs} sore throat, fever, chills and
muscle aches. Gastrointestinal upset, such as diarrhea and nausea, has
also been observed.

Another telltale sign of infection may be a sudden, profound diminution
of one's sense of smell and taste. Teenagers and young adults in some
cases have developed painful red and purple lesions on the fingers and
toes, but few other serious symptoms.

Severe disease leads to pneumonia and acute respiratory distress
syndrome. The blood oxygen levels plummet, and patients may get
supplemental oxygen or be placed on a machine, called a ventilator, to
help them breathe.

But even without lung impairment, the disease can cause injury to the
kidneys, heart or liver. Critically ill patients are prone to developing
dangerous blood clots in the legs and the lungs. In rare cases, the
disease triggers ischemic strokes that block the arteries supplying
blood to the brain, or brain impairments, such as altered mental status
or encephalopathy.

Death
\href{https://www.thelancet.com/journals/lancet/article/PIIS0140-6736(20)30558-4/fulltext}{can
result from} heart failure, kidney failure, multiple organ failure,
respiratory distress or shock.

\hypertarget{we-can-worry-a-bit-less-about-infection-from-surfaces}{%
\subsection{We can worry a bit less about infection from
surfaces.}\label{we-can-worry-a-bit-less-about-infection-from-surfaces}}

By Apoorva Mandavilli

Image

Credit...Jens Mortensen for The New York Times

The news, when it was reported, added a frightening twist to the threat
from the coronavirus: A study in March in The New England Journal of
Medicine found that under laboratory conditions, the virus can
\href{https://www.nytimes3xbfgragh.onion/2020/03/17/health/coronavirus-surfaces-aerosols.html}{survive
for up to three days} on some surfaces, such as plastic and steel, and
on cardboard for up to 24 hours.

Other studies reported finding the virus on air vents in hospital rooms
and on
\href{https://wwwnc.cdc.gov/eid/article/26/7/20-0885_article}{computer
mice, sickbed handrails and doorknobs}.

Many people grew worried that by touching a surface that had been
covered in droplets by an infected person, and then touching their own
mouth, nose or eyes, they then would contract the virus.

You should still wear a mask, avoid touching your face in public and
keep washing your hands. But none of these studies tested for live
virus, only for traces of its genetic material. Other scientists
commenting on these studies said virus on these surfaces might degrade
more quickly. The Centers for Disease Control and Prevention has said
since March that contaminated surfaces are ``not thought to be the main
way'' the virus spreads.

The main driver of infection is thought to be directly inhaling droplets
released when an infected person sneezes, coughs, sings or talks. The
C.D.C. recently made changes to its website to make this message even
more explicit.

\hypertarget{we-can-also-worry-less-about-a-mutating-virus}{%
\subsection{We can also worry less about a mutating
virus.}\label{we-can-also-worry-less-about-a-mutating-virus}}

By Carl Zimmer

In February, three experts on viruses published
\href{https://www.nature.com/articles/s41564-020-0690-4}{an editorial in
a journal headlined} ``We Shouldn't Worry When a Virus Mutates During
Outbreaks.''

But worry we did. As the coronavirus pandemic swept the planet,
headlines and tweets poured forth that the new coronavirus was
undergoing dangerous
\href{https://www.nytimes3xbfgragh.onion/2020/07/02/health/coronavirus-korber-mutation.html}{mutations}.

Many of these worries were based on a misunderstanding of what it means
when a virus mutates. When an infected cell produces new viruses, it
sometimes makes mistakes in copying the viral genes. Those mistakes are
mutations, and it turns out that most are bad for the viruses, getting
in the way of their ability to hijack our cells.

The viruses that do manage to spread to new hosts have mutations, too.
But those mutations often don't have any significant effect. The
alterations they bring to a virus's genes don't lead to any change in
how the virus works.

Scientists have identified harmless new mutations in different lineages
of the new coronavirus. These lineages are not dangerous new strains.

Some of these lineages have come to be the most common version of the
coronavirus in some countries. Again, that doesn't mean that they've got
some evolutionary edge. There's a very common phenomenon in nature
called the founder effect: Whatever mutations happen to be common in the
founders of a new population will end up common in their descendants.

It is possible for viruses to gain mutations that do affect the way they
work. The new coronavirus will be no different. But the only way to know
if a new mutation is significant or not is to carry out research. It
will take a lot of evidence to reject the more likely hypothesis: that a
new mutation has no importance at all.

Fortunately, it doesn't look like coronaviruses will be picking up these
new mutations very quickly. Compared with other viruses, scientists have
found, the new coronavirus has a relatively slow rate of new mutations.

That's a big relief for vaccine makers. Influenza viruses mutate so
quickly that people need to get a new flu shot each year to stay
protected. H.I.V. has so much genetic diversity that an effective
vaccine against it has yet to be found. The new coronavirus poses
immense challenges to vaccine makers, but most of them have to do with
manufacturing billions of doses in a matter of months.

We have enough worries when it comes to Covid-19; no need to add
needless ones to the list.

\hypertarget{we-cant-count-on-warm-weather-to-defeat-the-virus}{%
\subsection{We can't count on warm weather to defeat the
virus.}\label{we-cant-count-on-warm-weather-to-defeat-the-virus}}

By James Gorman

Image

Credit...Jens Mortensen for The New York Times

The hot and humid weather of summer
\href{https://www.nytimes3xbfgragh.onion/2020/05/08/health/virus-summer-pandemic.html}{will
not stop the pandemic}. More sunlight and humidity may slow down its
spread, but we probably won't know by how much. Other factors, like
reduced travel, increased personal distance, closed schools, canceled
gatherings and mask-wearing, have effects that would outweigh the
influence of the weather.

A few things are known about conditions that do or do not favor the
virus. The ultraviolet rays in sunlight help destroy the virus on
surfaces and some studies have shown a small effect from humidity. It
seems to last longest on hard surfaces like plastic and metal. It
\href{https://www.nytimes3xbfgragh.onion/2020/05/15/us/coronavirus-what-to-do-outside.html}{won't
survive in pool or lake or}seawater. Wind disperses it.
\href{https://www.nytimes3xbfgragh.onion/2020/05/15/us/coronavirus-what-to-do-outside.html}{Risk
of transmission is lower outdoors than indoors.}

A wooden bench under a bright sun at a breezy beach is a better bet than
a metal and plastic recliner on the shady side of the pool. But if
someone infected sits near you and coughs, or talks a lot or sings, it
doesn't really matter where you're sitting and how nice a day it is.

``The virus doesn't need favorable conditions,'' said Peter Juni, an
epidemiologist at the University of Toronto. It has a world population
with no immunity waiting to be infected. Bring on the sun; the novel
coronavirus will survive.

Air
conditioning\href{https://www.nytimes3xbfgragh.onion/2020/04/20/health/airflow-coronavirus-restaurants.html}{may
blow the virus right to your restaurant table}.

On Memorial Day, many people in the United States gathered in congenial
closeness in lovely weather without masks. If any of them were infected
and breathing, they probably infected someone else. The same will be
true on July 4. Even if the weather is glorious.

Advertisement

\protect\hyperlink{after-bottom}{Continue reading the main story}

\hypertarget{site-index}{%
\subsection{Site Index}\label{site-index}}

\hypertarget{site-information-navigation}{%
\subsection{Site Information
Navigation}\label{site-information-navigation}}

\begin{itemize}
\tightlist
\item
  \href{https://help.nytimes3xbfgragh.onion/hc/en-us/articles/115014792127-Copyright-notice}{©~2020~The
  New York Times Company}
\end{itemize}

\begin{itemize}
\tightlist
\item
  \href{https://www.nytco.com/}{NYTCo}
\item
  \href{https://help.nytimes3xbfgragh.onion/hc/en-us/articles/115015385887-Contact-Us}{Contact
  Us}
\item
  \href{https://www.nytco.com/careers/}{Work with us}
\item
  \href{https://nytmediakit.com/}{Advertise}
\item
  \href{http://www.tbrandstudio.com/}{T Brand Studio}
\item
  \href{https://www.nytimes3xbfgragh.onion/privacy/cookie-policy\#how-do-i-manage-trackers}{Your
  Ad Choices}
\item
  \href{https://www.nytimes3xbfgragh.onion/privacy}{Privacy}
\item
  \href{https://help.nytimes3xbfgragh.onion/hc/en-us/articles/115014893428-Terms-of-service}{Terms
  of Service}
\item
  \href{https://help.nytimes3xbfgragh.onion/hc/en-us/articles/115014893968-Terms-of-sale}{Terms
  of Sale}
\item
  \href{https://spiderbites.nytimes3xbfgragh.onion}{Site Map}
\item
  \href{https://help.nytimes3xbfgragh.onion/hc/en-us}{Help}
\item
  \href{https://www.nytimes3xbfgragh.onion/subscription?campaignId=37WXW}{Subscriptions}
\end{itemize}
