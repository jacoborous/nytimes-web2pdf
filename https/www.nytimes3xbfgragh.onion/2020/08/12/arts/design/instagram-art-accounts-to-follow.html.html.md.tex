Sections

SEARCH

\protect\hyperlink{site-content}{Skip to
content}\protect\hyperlink{site-index}{Skip to site index}

\href{https://www.nytimes3xbfgragh.onion/section/arts/design}{Art \&
Design}

\href{https://myaccount.nytimes3xbfgragh.onion/auth/login?response_type=cookie\&client_id=vi}{}

\href{https://www.nytimes3xbfgragh.onion/section/todayspaper}{Today's
Paper}

\href{/section/arts/design}{Art \& Design}\textbar{}Five Art Accounts to
Follow on Instagram Now

\url{https://nyti.ms/2FjMyp9}

\begin{itemize}
\item
\item
\item
\item
\item
\item
\end{itemize}

\href{https://www.nytimes3xbfgragh.onion/spotlight/at-home?action=click\&pgtype=Article\&state=default\&region=TOP_BANNER\&context=at_home_menu}{At
Home}

\begin{itemize}
\tightlist
\item
  \href{https://www.nytimes3xbfgragh.onion/2020/08/14/dining/lobster-salad-recipe.html?action=click\&pgtype=Article\&state=default\&region=TOP_BANNER\&context=at_home_menu}{Make:
  Lobster Salad}
\item
  \href{https://www.nytimes3xbfgragh.onion/2020/08/15/at-home/coronavirus-at-home-quick-exercises.html?action=click\&pgtype=Article\&state=default\&region=TOP_BANNER\&context=at_home_menu}{Sneak
  In: Exercise}
\item
  \href{https://www.nytimes3xbfgragh.onion/interactive/2020/at-home/even-more-reporters-editors-diaries-lists-recommendations.html?action=click\&pgtype=Article\&state=default\&region=TOP_BANNER\&context=at_home_menu}{See:
  Reporters' Obsessions}
\item
  \href{https://www.nytimes3xbfgragh.onion/2020/08/15/at-home/coronavirus-fall-patio-furniture.html?action=click\&pgtype=Article\&state=default\&region=TOP_BANNER\&context=at_home_menu}{Deck
  Out: Your Porch}
\end{itemize}

Advertisement

\protect\hyperlink{after-top}{Continue reading the main story}

Supported by

\protect\hyperlink{after-sponsor}{Continue reading the main story}

\hypertarget{five-art-accounts-to-follow-on-instagram-now}{%
\section{Five Art Accounts to Follow on Instagram
Now}\label{five-art-accounts-to-follow-on-instagram-now}}

Our critic looks in on a photographer in Los Angeles, a museum in Cape
Town, fierce young critics in London and culture workers who are out
there fighting for beauty and justice.

\includegraphics{https://static01.graylady3jvrrxbe.onion/images/2020/08/14/arts/13instagram-art-main/merlin_175576860_fe2e96fa-9ebc-4429-ac5f-aa8bdf7ef5e3-articleLarge.jpg?quality=75\&auto=webp\&disable=upscale}

By Siddhartha Mitter

\begin{itemize}
\item
  Aug. 12, 2020
\item
  \begin{itemize}
  \item
  \item
  \item
  \item
  \item
  \item
  \end{itemize}
\end{itemize}

It's August; an attempt at a fall culture season beckons, somehow, but a
sense of great fragmentation persists. On Instagram I see artists and
culture workers in Europe behaving more or less normally for the season
--- that is to say, on vacation. Elsewhere, new horrors have taken over
--- as in Beirut, where in the wake of a cataclysmic warehouse
explosion, artists are sifting through the rubble of devastated
gathering spaces and galleries.

And then there's the United States, where symptoms of collapse are all
over the culture, and maybe also, hopefully, some signs that we can
build a society with more mutual care once we emerge. It's hard to avoid
\href{https://www.nytimes3xbfgragh.onion/2020/07/16/technology/coronavirus-doomscrolling.html}{doomscrolling}.
Yet amid the algorithm's torrential spew, beauty still insists on
breaking out --- in images and insights that honor our communities as we
all try to push through, and ones that remind us of other places and
possibilities.

\hypertarget{kwasi-boyd-bouldin}{%
\subsection{Kwasi Boyd-Bouldin}\label{kwasi-boyd-bouldin}}

\href{https://www.instagram.com/_kwasi_b/}{@\_kwasi\_b} and
\href{https://www.instagram.com/thepublicwork/}{@thepublicwork}

\begin{quote}
\end{quote}

The Los Angeles photographer
\href{http://www.kwasiboydbouldin.com/}{Kwasi Boyd-Bouldin} interprets
his city through the broad streetscapes and utilitarian low-rise
architecture in which the working-class and immigrant people who keep
the place functioning proceed through the day. It's a local's look, keen
to the poetry of auto-body shops and money-transfer agencies, to signs
that hang askew and beat-up vehicles and always the sharp, unyielding
sunlight. Before the coronavirus crisis, Mr. Boyd-Bouldin was not
photographing people directly as much as seeking their traces, like an
archaeologist, in his stark cityscapes. But on the second account he has
put up this year, @thepublicwork, you'll see people --- his kind of
Angelenos, those just getting by --- as they navigate their ordinary
chores in this terrain. These ``snapshots from the lost world,'' as he
calls them in one brief essay, are reminders of community. ``Our casual
interactions with one another were a reflection of the human condition
in its purest form,'' he
\href{http://thepublicwork.com/2020/07/recent-nostalgia/}{writes}.
``It's one of the most valuable aspects of daily life taken from us by
this pandemic.''

\hypertarget{community-access-art-collective}{%
\subsection{Community Access Art
Collective}\label{community-access-art-collective}}

\href{https://www.instagram.com/artcollectivenyc/}{@artcollectivenyc}

Some 40 artists in multiple mediums make up the
\href{https://communityaccessart.org/}{Art Collective} at Community
Access, an organization in New York that provides housing and support
services for people living with mental health conditions. Some are
highly trained working artists with decades of material; others have
found in the studio a fresh, vital outlet. The work can be stunning,
like a recent \href{https://www.instagram.com/p/CCZJVdJFdM1/}{collage by
Zeus Hope} incorporating vintage newspaper with a jazz solo's serrated
energy, or the paintings of John Smith themed on the New York City
subway. The pandemic has meant restrictions on studio work for a group
that, in the last year, has been increasingly visible with exhibitions,
both physical and online; fortunately, its Instagram feed continues to
share not only the art (and links to an online gallery for pieces that
are for sale) but also glimpses of this dynamic crew's productive life
and rich individual stories.

\hypertarget{the-white-pube}{%
\subsection{The White Pube}\label{the-white-pube}}

\href{https://www.instagram.com/thewhitepube/}{@thewhitepube}

When Zarina Muhammad and Gabrielle de la Puente started
\href{https://www.thewhitepube.co.uk/}{The White Pube}, their caustic
but dead-serious criticism platform, they were students at Central Saint
Martins, the art school in London, who had come face to face with the
art world's political and institutional biases. Five years later, the
duo, based in Liverpool and London, have grown a big following without
sacrificing their rollicking, text-messagey style, nor their rigor and
curiosity. This is accountability work, often lambasting major British
museums and celebrity artists, but fundamentally constructive, with care
for community arts organizations and underrepresented voices. The pair,
and occasional co-conspirators, have a rich archive of criticism on
their website, but their Instagram feed is a great point of contact.
Britain is their main arena, but their perspective travels nicely.

\hypertarget{diptyk-magazine}{%
\subsection{Diptyk Magazine}\label{diptyk-magazine}}

\href{https://www.instagram.com/diptykmagazine/}{@diptykmagazine}

Based in Casablanca, Morocco, the bimonthly
\href{https://www.diptykmag.com/}{Diptyk} is a rare bird in today's
media landscape: a high-quality art magazine from the global South that
has managed to go the distance since it began in 2009. The perspective
is both Moroccan and cosmopolitan, covering artists and events across
Africa and the Mediterranean basin. What I appreciate about regional
publications like this one is the way they reorient my perspective,
shifting the center away from the usual hubs of global art and finance.
Diptyk is published in French, and you won't find it on American
newsstands, but its Instagram feed is a rich resource for art discovery,
elegantly selected with lots of links to explore.

\hypertarget{zeitz-museum-of-contemporary-art-africa}{%
\subsection{Zeitz Museum of Contemporary Art
Africa}\label{zeitz-museum-of-contemporary-art-africa}}

\href{https://www.instagram.com/zeitzmocaa/}{@zeitzmocaa}

The \href{https://zeitzmocaa.museum/}{Zeitz MOCAA}, in Cape Town, opened
in 2017 in a spectacular converted granary, with the aim to become
Africa's top contemporary-art venue. After wobbly beginnings, a
leadership overhaul brought in the star Cameroonian curator
\href{https://www.instagram.com/madamekoyo/}{Koyo Kouoh} to run the
place, and with her, sharper programming and fresh energy. The
coronavirus has hit South Africa hard, shutting museums indefinitely,
but Zeitz MOCAA has been busy online, offering digital panels,
children's activities and even dance parties. And Ms. Kouoh and her team
are keeping the intellectual flame burning with an excellent series of
Instagram Live interviews with fellow curators from across Africa as
well as with artists like Wangechi Mutu, archived on the museum's
\href{https://www.youtube.com/channel/UCDJltdsHMhBhgl_y6MMEcyQ/videos}{YouTube}
channel.

Advertisement

\protect\hyperlink{after-bottom}{Continue reading the main story}

\hypertarget{site-index}{%
\subsection{Site Index}\label{site-index}}

\hypertarget{site-information-navigation}{%
\subsection{Site Information
Navigation}\label{site-information-navigation}}

\begin{itemize}
\tightlist
\item
  \href{https://help.nytimes3xbfgragh.onion/hc/en-us/articles/115014792127-Copyright-notice}{©~2020~The
  New York Times Company}
\end{itemize}

\begin{itemize}
\tightlist
\item
  \href{https://www.nytco.com/}{NYTCo}
\item
  \href{https://help.nytimes3xbfgragh.onion/hc/en-us/articles/115015385887-Contact-Us}{Contact
  Us}
\item
  \href{https://www.nytco.com/careers/}{Work with us}
\item
  \href{https://nytmediakit.com/}{Advertise}
\item
  \href{http://www.tbrandstudio.com/}{T Brand Studio}
\item
  \href{https://www.nytimes3xbfgragh.onion/privacy/cookie-policy\#how-do-i-manage-trackers}{Your
  Ad Choices}
\item
  \href{https://www.nytimes3xbfgragh.onion/privacy}{Privacy}
\item
  \href{https://help.nytimes3xbfgragh.onion/hc/en-us/articles/115014893428-Terms-of-service}{Terms
  of Service}
\item
  \href{https://help.nytimes3xbfgragh.onion/hc/en-us/articles/115014893968-Terms-of-sale}{Terms
  of Sale}
\item
  \href{https://spiderbites.nytimes3xbfgragh.onion}{Site Map}
\item
  \href{https://help.nytimes3xbfgragh.onion/hc/en-us}{Help}
\item
  \href{https://www.nytimes3xbfgragh.onion/subscription?campaignId=37WXW}{Subscriptions}
\end{itemize}
