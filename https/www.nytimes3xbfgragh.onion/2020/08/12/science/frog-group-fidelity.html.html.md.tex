Sections

SEARCH

\protect\hyperlink{site-content}{Skip to
content}\protect\hyperlink{site-index}{Skip to site index}

\href{https://www.nytimes3xbfgragh.onion/section/science}{Science}

\href{https://myaccount.nytimes3xbfgragh.onion/auth/login?response_type=cookie\&client_id=vi}{}

\href{https://www.nytimes3xbfgragh.onion/section/todayspaper}{Today's
Paper}

\href{/section/science}{Science}\textbar{}He Doesn't Mind Being Shared,
Unless His Mates Try to Eat Each Other's Eggs

\url{https://nyti.ms/2DYNqP4}

\begin{itemize}
\item
\item
\item
\item
\item
\item
\end{itemize}

Advertisement

\protect\hyperlink{after-top}{Continue reading the main story}

Supported by

\protect\hyperlink{after-sponsor}{Continue reading the main story}

Trilobites

\hypertarget{he-doesnt-mind-being-shared-unless-his-mates-try-to-eat-each-others-eggs}{%
\section{He Doesn't Mind Being Shared, Unless His Mates Try to Eat Each
Other's
Eggs}\label{he-doesnt-mind-being-shared-unless-his-mates-try-to-eat-each-others-eggs}}

A Brazilian frog species engages in reproductive behavior never seen in
amphibians before.

\includegraphics{https://static01.graylady3jvrrxbe.onion/images/2020/08/12/science/12TB-FROGMATING1/12TB-FROGMATING1-articleLarge.jpg?quality=75\&auto=webp\&disable=upscale}

\href{https://www.nytimes3xbfgragh.onion/by/katherine-j--wu}{\includegraphics{https://static01.graylady3jvrrxbe.onion/images/2020/08/11/reader-center/author-katherine-j-wu/author-katherine-j-wu-thumbLarge.png}}

By
\href{https://www.nytimes3xbfgragh.onion/by/katherine-j--wu}{Katherine
J. Wu}

\begin{itemize}
\item
  Aug. 12, 2020
\item
  \begin{itemize}
  \item
  \item
  \item
  \item
  \item
  \item
  \end{itemize}
\end{itemize}

\href{https://www.nytimes3xbfgragh.onion/es/2020/08/12/espanol/ciencia-y-tecnologia/ranas-brasil.html}{Leer
en español}

From the standpoint of sex, male frogs tend to segregate into two camps:
the monogamous bunch and the free wheeling philanderers.

This split seems to apply to all amphibians, which don't really
compromise between these two amorous extremes. Researchers have long
found this odd; plenty of other animals practice group fidelity, wherein
one male strikes up a long-term liaison with several females at once,
but won't engage with anyone else.

Now, a team of scientists has found a rule-breaking river frog in
Brazil's Atlantic rainforest called Thoropa taophora, whose sexual
shenanigans involve everything from cannibalism to peacemaking hugs and
some surprisingly well-muscled forearms.

Over the course of a single breeding season, some males will couple up
with exactly two females, who alternately visit their mate like a
sperm-dispensing timeshare.

``There is a bond between the male and the female, but there is more
than one couple,'' said Fábio de Sá, a biologist at the University of
Campinas in Brazil and an author on a paper published Wednesday in
\href{https://advances.sciencemag.org/lookup/doi/10.1126/sciadv.aay1539}{Science
Advances} that describes the new behavior.

Much of Thoropa mating boils down to location. These frogs prize wet,
rocky habitats called seeps, where their eggs can mature. Males will
wage all-out war over seeps, emitting aggressive calls and jabbing at
their rivals with the spine-studded thumbs at the ends of their bulging,
battleworthy forelimbs. Once their real estate is won, the victors spend
their nights on patrol, in the hopes that a female will drop by and
leave behind a cache of fresh eggs.

Where seeps are abundant, both sexes of Thoropa taophora frogs take
multiple partners, Dr. de Sá said. But when seeps become scarce enough
to render some males homeless, the frogs are sometimes forced into a
seismic sexual shift.

``For females, males become the limiting resource,'' Dr. de Sá said.
Under these circumstances, females queue up to mate with the
seep-straddling males. In a series of video recordings, the researchers
found that two, perhaps three, females might share the same beau, with
one usually emerging as a ``dominant'' consort who monopolized most of
the mating.

\includegraphics{https://static01.graylady3jvrrxbe.onion/images/2020/08/12/autossell/Untitled-copy/Untitled-copy-videoSixteenByNine3000.jpg}

But the group courting the males ``had consistent membership'' over a
long period of time, said Kelly Zamudio, a biologist at Cornell
University and an author on the study. A genetic analysis of the eggs
laid in each male's seep also revealed that the tadpoles were all full-
or half-siblings of various ages --- a hint that they had come from two
moms who repeatedly visited the same dad.

Mating systems like these are something ``people have suspected,'' said
Lauren O'Connell, a biologist at Stanford University who wasn't involved
in the study. ``But this was really a test of that idea.'' She called it
``a really heroic effort of explaining the mating system of a species in
the wild.''

The females, however, were not always eager to share. Upon arriving at
the seep for an evening tryst, some tried to eat the eggs already there,
Dr. de Sá said. To curtail this filial cannibalism, the male sometimes
attempted to create a diversion, clasping the female with his burly arms
from behind --- a typical mating position. But the encounter didn't
always end in sex; sometimes the male seemed to simply wrap the female
in a platonic embrace, ``like a distraction hug,'' Dr. O'Connell said.

This whole situation might sound less than ideal, especially for
non-dominant females, Dr. Zamudio said. But when breeding sites are this
limited, ``it's better to be a secondary, or even the third, female in a
group,'' she said. ``At least you have a chance of laying some eggs with
the male, rather than going off to be on your own.''

The findings also help complete the puzzle of group fidelity, which has
now been documented in all classes of four-legged animals, or tetrapods,
Dr. de Sá said.

``In the big scheme of evolution, we can say we are filling in a piece
of information,'' he said. ``That is pretty cool.''

Advertisement

\protect\hyperlink{after-bottom}{Continue reading the main story}

\hypertarget{site-index}{%
\subsection{Site Index}\label{site-index}}

\hypertarget{site-information-navigation}{%
\subsection{Site Information
Navigation}\label{site-information-navigation}}

\begin{itemize}
\tightlist
\item
  \href{https://help.nytimes3xbfgragh.onion/hc/en-us/articles/115014792127-Copyright-notice}{©~2020~The
  New York Times Company}
\end{itemize}

\begin{itemize}
\tightlist
\item
  \href{https://www.nytco.com/}{NYTCo}
\item
  \href{https://help.nytimes3xbfgragh.onion/hc/en-us/articles/115015385887-Contact-Us}{Contact
  Us}
\item
  \href{https://www.nytco.com/careers/}{Work with us}
\item
  \href{https://nytmediakit.com/}{Advertise}
\item
  \href{http://www.tbrandstudio.com/}{T Brand Studio}
\item
  \href{https://www.nytimes3xbfgragh.onion/privacy/cookie-policy\#how-do-i-manage-trackers}{Your
  Ad Choices}
\item
  \href{https://www.nytimes3xbfgragh.onion/privacy}{Privacy}
\item
  \href{https://help.nytimes3xbfgragh.onion/hc/en-us/articles/115014893428-Terms-of-service}{Terms
  of Service}
\item
  \href{https://help.nytimes3xbfgragh.onion/hc/en-us/articles/115014893968-Terms-of-sale}{Terms
  of Sale}
\item
  \href{https://spiderbites.nytimes3xbfgragh.onion}{Site Map}
\item
  \href{https://help.nytimes3xbfgragh.onion/hc/en-us}{Help}
\item
  \href{https://www.nytimes3xbfgragh.onion/subscription?campaignId=37WXW}{Subscriptions}
\end{itemize}
