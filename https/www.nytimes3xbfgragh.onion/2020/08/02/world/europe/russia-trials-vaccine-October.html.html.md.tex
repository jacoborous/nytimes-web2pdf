Sections

SEARCH

\protect\hyperlink{site-content}{Skip to
content}\protect\hyperlink{site-index}{Skip to site index}

\href{https://www.nytimes3xbfgragh.onion/section/world/europe}{Europe}

\href{https://myaccount.nytimes3xbfgragh.onion/auth/login?response_type=cookie\&client_id=vi}{}

\href{https://www.nytimes3xbfgragh.onion/section/todayspaper}{Today's
Paper}

\href{/section/world/europe}{Europe}\textbar{}Russia Sets Mass
Vaccination for October After Shortened Trial

\url{https://nyti.ms/31g6QHc}

\begin{itemize}
\item
\item
\item
\item
\item
\end{itemize}

\href{https://www.nytimes3xbfgragh.onion/news-event/coronavirus?action=click\&pgtype=Article\&state=default\&region=TOP_BANNER\&context=storylines_menu}{The
Coronavirus Outbreak}

\begin{itemize}
\tightlist
\item
  live\href{https://www.nytimes3xbfgragh.onion/2020/08/02/world/coronavirus-updates.html?action=click\&pgtype=Article\&state=default\&region=TOP_BANNER\&context=storylines_menu}{Latest
  Updates}
\item
  \href{https://www.nytimes3xbfgragh.onion/interactive/2020/us/coronavirus-us-cases.html?action=click\&pgtype=Article\&state=default\&region=TOP_BANNER\&context=storylines_menu}{Maps
  and Cases}
\item
  \href{https://www.nytimes3xbfgragh.onion/interactive/2020/science/coronavirus-vaccine-tracker.html?action=click\&pgtype=Article\&state=default\&region=TOP_BANNER\&context=storylines_menu}{Vaccine
  Tracker}
\item
  \href{https://www.nytimes3xbfgragh.onion/interactive/2020/07/29/us/schools-reopening-coronavirus.html?action=click\&pgtype=Article\&state=default\&region=TOP_BANNER\&context=storylines_menu}{What
  School May Look Like}
\item
  \href{https://www.nytimes3xbfgragh.onion/live/2020/07/31/business/stock-market-today-coronavirus?action=click\&pgtype=Article\&state=default\&region=TOP_BANNER\&context=storylines_menu}{Economy}
\end{itemize}

Advertisement

\protect\hyperlink{after-top}{Continue reading the main story}

Supported by

\protect\hyperlink{after-sponsor}{Continue reading the main story}

\hypertarget{russia-sets-mass-vaccination-for-october-after-shortened-trial}{%
\section{Russia Sets Mass Vaccination for October After Shortened
Trial}\label{russia-sets-mass-vaccination-for-october-after-shortened-trial}}

The announcement raised concerns that Russia would begin inoculations
and declare victory in the race for a coronavirus vaccine without fully
testing its product.

\includegraphics{https://static01.graylady3jvrrxbe.onion/images/2020/08/01/world/00russia-vaccine01/00russia-vaccine01-articleLarge.jpg?quality=75\&auto=webp\&disable=upscale}

\href{https://www.nytimes3xbfgragh.onion/by/andrew-e-kramer}{\includegraphics{https://static01.graylady3jvrrxbe.onion/images/2018/10/15/multimedia/author-andrew-e-kramer/author-andrew-e-kramer-thumbLarge.png}}

By \href{https://www.nytimes3xbfgragh.onion/by/andrew-e-kramer}{Andrew
E. Kramer}

\begin{itemize}
\item
  Aug. 2, 2020
\item
  \begin{itemize}
  \item
  \item
  \item
  \item
  \item
  \end{itemize}
\end{itemize}

MOSCOW --- Russia plans to launch a nationwide vaccination campaign in
October with a coronavirus vaccine that has yet to complete clinical
trials, raising international concern about the methods the country is
using to compete in the global race to inoculate the public.

The minister of health, Mikhail Murashko, said Saturday that the plan
was to begin by vaccinating teachers and health care workers. He also
\href{https://ria.ru/20200801/1575248763.html}{told the RIA} state news
agency that amid accelerated testing, the laboratory that developed the
vaccine was already seeking regulatory approval for it.

Russia is one of a number of countries rushing to develop and administer
a vaccine. Not only would such a vaccine help alleviate a worldwide
health crisis that has killed more than 680,000 people and badly wounded
the global economy, it would also become a symbol of national pride. And
Russia has used the race as a propaganda tool, even in the absence of
published scientific evidence to support its claim as a front-runner.

``I do hope that the Chinese and the Russians are actually testing the
vaccine before they are administering the vaccine to anyone,'' Dr.
Anthony Fauci, director of the National Institute of Allergy and
Infectious Diseases in the United States, warned a congressional hearing
on Friday.

State television in Russia has for several months now promoted the idea
of Russia leading the competition. In May, a government report claimed
that the first person in the world to be vaccinated against the virus
was a Russian researcher who had injected himself with a vaccine early
in the development process.

Russia will start Phase III trials of the vaccine in early August, said
Kirill Dmitriev, a senior official with Russia Direct Investment Fund, a
government-controlled investor in the country's vaccination effort. A
Phase III trial is the only way to determine if a vaccine is effective.

The World Health Organization maintains a
\href{https://www.who.int/publications/m/item/draft-landscape-of-covid-19-candidate-vaccines}{comprehensive
list of worldwide vaccine trials}. But there is no Russian Phase III
trial on the list.

Still, a Russian regulatory agency is expected to approve the vaccine
this month, Mr. Dmitriev said. That is
\href{https://www.nytimes3xbfgragh.onion/interactive/2020/science/coronavirus-vaccine-tracker.html}{far
earlier than timelines suggested by Western regulators}, who have often
said a vaccine would become available no sooner than the end of the
year.

``We believe it will be one of the first vaccines with regulatory
approval,'' Mr. Dmitriev said.

But with limited transparency in the Russian program, separating the
science from the politics and propaganda could prove impossible. Critics
have already drawn attention to Russia's tradition of cutting corners in
research on other pharmaceutical products and accusations of
intellectual property theft.

\includegraphics{https://static01.graylady3jvrrxbe.onion/images/2020/08/01/world/00russia-vaccine02/00russia-vaccine02-articleLarge.jpg?quality=75\&auto=webp\&disable=upscale}

The U.S., Canadian and British governments have all accused Russian
state hackers of attempting to
\href{https://www.nytimes3xbfgragh.onion/2020/07/16/us/politics/vaccine-hacking-russia.html}{steal
vaccine research}, casting a shadow over Russia's claim to have achieved
a medical breakthrough. Russian officials have denied the accusation and
say their leading vaccine is based on a design developed by Russian
scientists to counter Ebola years ago.

\hypertarget{latest-updates-global-coronavirus-outbreak}{%
\section{\texorpdfstring{\href{https://www.nytimes3xbfgragh.onion/2020/08/01/world/coronavirus-covid-19.html?action=click\&pgtype=Article\&state=default\&region=MAIN_CONTENT_1\&context=storylines_live_updates}{Latest
Updates: Global Coronavirus
Outbreak}}{Latest Updates: Global Coronavirus Outbreak}}\label{latest-updates-global-coronavirus-outbreak}}

Updated 2020-08-02T17:52:35.962Z

\begin{itemize}
\tightlist
\item
  \href{https://www.nytimes3xbfgragh.onion/2020/08/01/world/coronavirus-covid-19.html?action=click\&pgtype=Article\&state=default\&region=MAIN_CONTENT_1\&context=storylines_live_updates\#link-34047410}{The
  U.S. reels as July cases more than double the total of any other
  month.}
\item
  \href{https://www.nytimes3xbfgragh.onion/2020/08/01/world/coronavirus-covid-19.html?action=click\&pgtype=Article\&state=default\&region=MAIN_CONTENT_1\&context=storylines_live_updates\#link-780ec966}{Top
  U.S. officials work to break an impasse over the federal jobless
  benefit.}
\item
  \href{https://www.nytimes3xbfgragh.onion/2020/08/01/world/coronavirus-covid-19.html?action=click\&pgtype=Article\&state=default\&region=MAIN_CONTENT_1\&context=storylines_live_updates\#link-2bc8948}{Its
  outbreak untamed, Melbourne goes into even greater lockdown.}
\end{itemize}

\href{https://www.nytimes3xbfgragh.onion/2020/08/01/world/coronavirus-covid-19.html?action=click\&pgtype=Article\&state=default\&region=MAIN_CONTENT_1\&context=storylines_live_updates}{See
more updates}

More live coverage:
\href{https://www.nytimes3xbfgragh.onion/live/2020/07/31/business/stock-market-today-coronavirus?action=click\&pgtype=Article\&state=default\&region=MAIN_CONTENT_1\&context=storylines_live_updates}{Markets}

Russia was once at the forefront in virology and vaccinations. In the
Soviet era, its doctors led the world in some areas of research, but
spending has shriveled in recent decades. Medicines are sometimes
approved with limited or no testing.

Russian researchers have continued to advance a range of vaccines since
the beginning of the pandemic. The candidate to be given in October is
similar to a vaccine developed by Oxford University and AstraZeneca.

The Russian vaccine was developed by the Gamaleya Institute in Moscow.
It uses two strains of adenovirus that typically cause mild colds in
humans. Adenovirus vaccines are in trials in various countries. They are
genetically modified to cause infected cells to make proteins from the
spike of the new coronavirus.

The Gamaleya Institute tested its vaccine on soldiers, raising ethical
questions about consent, though the defense ministry said all of the
soldiers had volunteered. The institute's director, Aleksandr Gintsberg,
went
\href{https://www.1tv.ru/shows/zhit-zdorovo/syuzhety/est-vakcina-ot-koronavirusa-zhit-zdorovo-fragment-vypuska-ot-25-05-2020}{on
television in May} to say he tried the vaccine on himself before
announcing the completion of trials in monkeys.

``There is an escalation in the geopolitics of vaccine research,'' said
Cliff Kupchan, chairman of Eurasia Group, a risk consulting firm. But
``what remains of the vast scientific complex of the Soviet period is a
shadow of what it was,'' he said.

Countries with vaccine production capacity --- abundant in Russia and
\href{https://www.nytimes3xbfgragh.onion/2020/08/01/world/asia/coronavirus-vaccine-india.html?action=click\&module=Top\%20Stories\&pgtype=Homepage}{India}
--- could wind up inoculating their populations by copying a successful
vaccine, even if they did not in fact develop it. In April, the Serum
Institute in India announced that it had plans to
\href{https://www.nytimes3xbfgragh.onion/2020/08/01/world/asia/coronavirus-vaccine-india.html}{mass-produce
a vaccine}, with permission from the developer, before clinical trials
had ended.

``In all likelihood, the country producing on their soil will be the
first to get it, even if they don't own it,'' Mr. Kupchan said. ``I
don't know how much international law and patent protection will apply
here. People are pretty desperate.''

Mr. Dmitriev, of the Russia Direct Investment Fund, has attributed
Russia's research success to the Soviet Union's once-formidable
scientific study of viruses.

``We have this very significant legacy of Russia being a leader of
vaccines in the Soviet time and today,'' he said. ``We don't have to
create many things from scratch.''

He contrasted that history with Trump administration's Operation Warp
Speed program, which is financing experimental research by Pfizer and
Moderna for a genetic vaccine.

Image

The institute's director, Aleksandr Gintsberg, said he tried the vaccine
on himself before completing monkey trials.Credit...Stanislav
Krasilnikov\textbackslash{}TASS, via Getty Images

``In the last 20 years, the world took a turn toward molecular
biology,'' said Aydar A. Ishmukhametov, the director of the Chumakov
Institute, a Russian vaccine maker. ``The Russian school has preserved
virology.''

Russia also has an advantage, Mr. Ishmukhametov said, in its vast,
Soviet-era industrial base for growing viruses for vaccines. In the
pandemic, the country has turned to a secretive laboratory in Siberia
with roots in the Soviet Union's biological weapons program, which
included the study of anthrax to target humans and plant pathogens that
would destroy American crops.

The laboratory, Vektor, is now testing whether viruses that cause
influenza, measles or vascular stomatitis --- a livestock disease ---
can be put to use for a coronavirus vaccine.

\href{https://www.nytimes3xbfgragh.onion/news-event/coronavirus?action=click\&pgtype=Article\&state=default\&region=MAIN_CONTENT_3\&context=storylines_faq}{}

\hypertarget{the-coronavirus-outbreak-}{%
\subsubsection{The Coronavirus Outbreak
›}\label{the-coronavirus-outbreak-}}

\hypertarget{frequently-asked-questions}{%
\paragraph{Frequently Asked
Questions}\label{frequently-asked-questions}}

Updated July 27, 2020

\begin{itemize}
\item ~
  \hypertarget{should-i-refinance-my-mortgage}{%
  \paragraph{Should I refinance my
  mortgage?}\label{should-i-refinance-my-mortgage}}

  \begin{itemize}
  \tightlist
  \item
    \href{https://www.nytimes3xbfgragh.onion/article/coronavirus-money-unemployment.html?action=click\&pgtype=Article\&state=default\&region=MAIN_CONTENT_3\&context=storylines_faq}{It
    could be a good idea,} because mortgage rates have
    \href{https://www.nytimes3xbfgragh.onion/2020/07/16/business/mortgage-rates-below-3-percent.html?action=click\&pgtype=Article\&state=default\&region=MAIN_CONTENT_3\&context=storylines_faq}{never
    been lower.} Refinancing requests have pushed mortgage applications
    to some of the highest levels since 2008, so be prepared to get in
    line. But defaults are also up, so if you're thinking about buying a
    home, be aware that some lenders have tightened their standards.
  \end{itemize}
\item ~
  \hypertarget{what-is-school-going-to-look-like-in-september}{%
  \paragraph{What is school going to look like in
  September?}\label{what-is-school-going-to-look-like-in-september}}

  \begin{itemize}
  \tightlist
  \item
    It is unlikely that many schools will return to a normal schedule
    this fall, requiring the grind of
    \href{https://www.nytimes3xbfgragh.onion/2020/06/05/us/coronavirus-education-lost-learning.html?action=click\&pgtype=Article\&state=default\&region=MAIN_CONTENT_3\&context=storylines_faq}{online
    learning},
    \href{https://www.nytimes3xbfgragh.onion/2020/05/29/us/coronavirus-child-care-centers.html?action=click\&pgtype=Article\&state=default\&region=MAIN_CONTENT_3\&context=storylines_faq}{makeshift
    child care} and
    \href{https://www.nytimes3xbfgragh.onion/2020/06/03/business/economy/coronavirus-working-women.html?action=click\&pgtype=Article\&state=default\&region=MAIN_CONTENT_3\&context=storylines_faq}{stunted
    workdays} to continue. California's two largest public school
    districts --- Los Angeles and San Diego --- said on July 13, that
    \href{https://www.nytimes3xbfgragh.onion/2020/07/13/us/lausd-san-diego-school-reopening.html?action=click\&pgtype=Article\&state=default\&region=MAIN_CONTENT_3\&context=storylines_faq}{instruction
    will be remote-only in the fall}, citing concerns that surging
    coronavirus infections in their areas pose too dire a risk for
    students and teachers. Together, the two districts enroll some
    825,000 students. They are the largest in the country so far to
    abandon plans for even a partial physical return to classrooms when
    they reopen in August. For other districts, the solution won't be an
    all-or-nothing approach.
    \href{https://bioethics.jhu.edu/research-and-outreach/projects/eschool-initiative/school-policy-tracker/}{Many
    systems}, including the nation's largest, New York City, are
    devising
    \href{https://www.nytimes3xbfgragh.onion/2020/06/26/us/coronavirus-schools-reopen-fall.html?action=click\&pgtype=Article\&state=default\&region=MAIN_CONTENT_3\&context=storylines_faq}{hybrid
    plans} that involve spending some days in classrooms and other days
    online. There's no national policy on this yet, so check with your
    municipal school system regularly to see what is happening in your
    community.
  \end{itemize}
\item ~
  \hypertarget{is-the-coronavirus-airborne}{%
  \paragraph{Is the coronavirus
  airborne?}\label{is-the-coronavirus-airborne}}

  \begin{itemize}
  \tightlist
  \item
    The coronavirus
    \href{https://www.nytimes3xbfgragh.onion/2020/07/04/health/239-experts-with-one-big-claim-the-coronavirus-is-airborne.html?action=click\&pgtype=Article\&state=default\&region=MAIN_CONTENT_3\&context=storylines_faq}{can
    stay aloft for hours in tiny droplets in stagnant air}, infecting
    people as they inhale, mounting scientific evidence suggests. This
    risk is highest in crowded indoor spaces with poor ventilation, and
    may help explain super-spreading events reported in meatpacking
    plants, churches and restaurants.
    \href{https://www.nytimes3xbfgragh.onion/2020/07/06/health/coronavirus-airborne-aerosols.html?action=click\&pgtype=Article\&state=default\&region=MAIN_CONTENT_3\&context=storylines_faq}{It's
    unclear how often the virus is spread} via these tiny droplets, or
    aerosols, compared with larger droplets that are expelled when a
    sick person coughs or sneezes, or transmitted through contact with
    contaminated surfaces, said Linsey Marr, an aerosol expert at
    Virginia Tech. Aerosols are released even when a person without
    symptoms exhales, talks or sings, according to Dr. Marr and more
    than 200 other experts, who
    \href{https://academic.oup.com/cid/article/doi/10.1093/cid/ciaa939/5867798}{have
    outlined the evidence in an open letter to the World Health
    Organization}.
  \end{itemize}
\item ~
  \hypertarget{what-are-the-symptoms-of-coronavirus}{%
  \paragraph{What are the symptoms of
  coronavirus?}\label{what-are-the-symptoms-of-coronavirus}}

  \begin{itemize}
  \tightlist
  \item
    Common symptoms
    \href{https://www.nytimes3xbfgragh.onion/article/symptoms-coronavirus.html?action=click\&pgtype=Article\&state=default\&region=MAIN_CONTENT_3\&context=storylines_faq}{include
    fever, a dry cough, fatigue and difficulty breathing or shortness of
    breath.} Some of these symptoms overlap with those of the flu,
    making detection difficult, but runny noses and stuffy sinuses are
    less common.
    \href{https://www.nytimes3xbfgragh.onion/2020/04/27/health/coronavirus-symptoms-cdc.html?action=click\&pgtype=Article\&state=default\&region=MAIN_CONTENT_3\&context=storylines_faq}{The
    C.D.C. has also} added chills, muscle pain, sore throat, headache
    and a new loss of the sense of taste or smell as symptoms to look
    out for. Most people fall ill five to seven days after exposure, but
    symptoms may appear in as few as two days or as many as 14 days.
  \end{itemize}
\item ~
  \hypertarget{does-asymptomatic-transmission-of-covid-19-happen}{%
  \paragraph{Does asymptomatic transmission of Covid-19
  happen?}\label{does-asymptomatic-transmission-of-covid-19-happen}}

  \begin{itemize}
  \tightlist
  \item
    So far, the evidence seems to show it does. A widely cited
    \href{https://www.nature.com/articles/s41591-020-0869-5}{paper}
    published in April suggests that people are most infectious about
    two days before the onset of coronavirus symptoms and estimated that
    44 percent of new infections were a result of transmission from
    people who were not yet showing symptoms. Recently, a top expert at
    the World Health Organization stated that transmission of the
    coronavirus by people who did not have symptoms was ``very rare,''
    \href{https://www.nytimes3xbfgragh.onion/2020/06/09/world/coronavirus-updates.html?action=click\&pgtype=Article\&state=default\&region=MAIN_CONTENT_3\&context=storylines_faq\#link-1f302e21}{but
    she later walked back that statement.}
  \end{itemize}
\end{itemize}

The science of mass producing vaccines has deep roots here. Aleksei
Chumakov, a virologist and son of the founder of the Chumakov Institute,
recalled a summer job he held as a teenager chopping up kidneys
harvested from African green monkeys. Even though the monkeys had been
slaughtered, Mr. Chumakov said, their kidney cells lived on for many
months, used to grow the polio virus in large, rotating glass cylinders.

``You kept stirring it and gradually the clumps came apart,'' he said.

As scientists gained proficiency in growing so-called immortal cell
lines --- human or animal cells that are modified to divide indefinitely
--- they replaced cultures from fresh monkey kidneys.

The Chumakov Institute has used an immortal monkey kidney cell line from
1962 to grow coronavirus for a proposed vaccine using whole, inactivated
viruses, which may be used as an alternative if the vaccine targeting
just
\href{https://www.nytimes3xbfgragh.onion/2020/07/28/health/coronavirus-mutation-spike-treatment.html?action=click\&module=Latest\&pgtype=Homepage}{the
spike protein} fails.

The Gamaleya Institute developed its vaccine using a human cell line
first cultured in 1973, known as Hek293 --- the same line used in the
Oxford-AstraZeneca vaccine. Like a number of other cell lines used in
medical research, Hek293 began with cells taken from an aborted fetus,
raising objections from abortion opponents, including
\href{https://www.sciencemag.org/sites/default/files/Canada-COVID-19\%20Vaccine\%20letter\%20May\%2021st\%202020_updated\%20\%28002\%29.pdf}{Roman
Catholic clerics}.

The first human cell line was derived from the cancer that killed
\href{https://www.nytimes3xbfgragh.onion/2013/08/08/science/after-decades-of-research-henrietta-lacks-family-is-asked-for-consent.html}{Henrietta
Lacks}in 1951. HeLa, as it was known, made its way into Soviet
laboratories during the Cold War. Viktor Zuyev, an 91-year-old emeritus
professor of virology at the Gamaleya Institute, recalled using it to
cultivate flu virus.

He was unbothered by the question of ethics.

``Why not?'' he said. ``It is very humane to the next generation'' to
use a dying person's tissue for scientific experimentation. ``If it can
benefit humanity,'' he said, ``of course it is ethical.''

Image

Human cell lines, such as the HeLa line, is the technology in Russia's
leading vaccine candidate today.Credit...National Center for Microscopy
and Imaging Research, via Associated Press

Advertisement

\protect\hyperlink{after-bottom}{Continue reading the main story}

\hypertarget{site-index}{%
\subsection{Site Index}\label{site-index}}

\hypertarget{site-information-navigation}{%
\subsection{Site Information
Navigation}\label{site-information-navigation}}

\begin{itemize}
\tightlist
\item
  \href{https://help.nytimes3xbfgragh.onion/hc/en-us/articles/115014792127-Copyright-notice}{©~2020~The
  New York Times Company}
\end{itemize}

\begin{itemize}
\tightlist
\item
  \href{https://www.nytco.com/}{NYTCo}
\item
  \href{https://help.nytimes3xbfgragh.onion/hc/en-us/articles/115015385887-Contact-Us}{Contact
  Us}
\item
  \href{https://www.nytco.com/careers/}{Work with us}
\item
  \href{https://nytmediakit.com/}{Advertise}
\item
  \href{http://www.tbrandstudio.com/}{T Brand Studio}
\item
  \href{https://www.nytimes3xbfgragh.onion/privacy/cookie-policy\#how-do-i-manage-trackers}{Your
  Ad Choices}
\item
  \href{https://www.nytimes3xbfgragh.onion/privacy}{Privacy}
\item
  \href{https://help.nytimes3xbfgragh.onion/hc/en-us/articles/115014893428-Terms-of-service}{Terms
  of Service}
\item
  \href{https://help.nytimes3xbfgragh.onion/hc/en-us/articles/115014893968-Terms-of-sale}{Terms
  of Sale}
\item
  \href{https://spiderbites.nytimes3xbfgragh.onion}{Site Map}
\item
  \href{https://help.nytimes3xbfgragh.onion/hc/en-us}{Help}
\item
  \href{https://www.nytimes3xbfgragh.onion/subscription?campaignId=37WXW}{Subscriptions}
\end{itemize}
