Sections

SEARCH

\protect\hyperlink{site-content}{Skip to
content}\protect\hyperlink{site-index}{Skip to site index}

\href{https://www.nytimes3xbfgragh.onion/section/business}{Business}

\href{https://myaccount.nytimes3xbfgragh.onion/auth/login?response_type=cookie\&client_id=vi}{}

\href{https://www.nytimes3xbfgragh.onion/section/todayspaper}{Today's
Paper}

\href{/section/business}{Business}\textbar{}Marathon Is Selling Speedway
Gas Stations to 7-Eleven's Parent for \$21 Billion

\url{https://nyti.ms/3goNndM}

\begin{itemize}
\item
\item
\item
\item
\item
\end{itemize}

Advertisement

\protect\hyperlink{after-top}{Continue reading the main story}

Supported by

\protect\hyperlink{after-sponsor}{Continue reading the main story}

\hypertarget{marathon-is-selling-speedway-gas-stations-to-7-elevens-parent-for-21-billion}{%
\section{Marathon Is Selling Speedway Gas Stations to 7-Eleven's Parent
for \$21
Billion}\label{marathon-is-selling-speedway-gas-stations-to-7-elevens-parent-for-21-billion}}

The largest U.S. independent refining company will get cash, and
7-Eleven will add 4,000 convenience stores.

\includegraphics{https://static01.graylady3jvrrxbe.onion/images/2020/08/03/business/02speedway-print/02speedway1-articleLarge.jpg?quality=75\&auto=webp\&disable=upscale}

\href{https://www.nytimes3xbfgragh.onion/by/clifford-krauss}{\includegraphics{https://static01.graylady3jvrrxbe.onion/images/2018/10/17/multimedia/author-clifford-krauss/author-clifford-krauss-thumbLarge.png}}

By \href{https://www.nytimes3xbfgragh.onion/by/clifford-krauss}{Clifford
Krauss}

\begin{itemize}
\item
  Aug. 2, 2020
\item
  \begin{itemize}
  \item
  \item
  \item
  \item
  \item
  \end{itemize}
\end{itemize}

HOUSTON --- Marathon Petroleum, the largest U.S. independent refiner,
announced on Sunday that it had
\href{https://www.prnewswire.com/news-releases/marathon-petroleum-corp-announces-agreement-for-21-billion-sale-of-speedway-301104405.html}{sold
its Speedway gas station chain} to the Japanese retail group Seven \& I
Holdings for \$21 billion in cash.

The sale of Speedway, one of the country's largest convenience store
chains with nearly 4,000 outlets, is the biggest corporate deal in the
oil sector since the coronavirus slashed demand for fuel early this
year.

With the addition of Speedway, Seven \& I's 7-Eleven chain will have
around 14,000 stores.

Marathon Petroleum, based in Ohio, has been struggling financially and
has shuttered operations in two refineries. It had been seeking to spin
off Speedway for months. Negotiations between Marathon and Seven \& I
had stumbled earlier this year over price.

Marathon praised the deal in its announcement as a step forward in its
efforts to shore up its finances and stave off pressure from activist
investors seeking to dismantle the company.

Marathon said it expected the deal to bring in \$16.5 billion in
after-tax cash proceeds, much of which would go to paying down debts and
to support dividend payments. Included in the deal is a 15-year
agreement in which Marathon would provide 7.7 billion gallons of
petroleum a year to the gasoline station chain. Marathon said it
expected the deal to close in early 2021 after review by antitrust
officials.

Last year, Marathon agreed to spin off its Speedway chain under pressure
from investors that included the hedge fund Elliott Management
Corporation. The Canada-based convenience store operator Alimentation
Couche-Tard had also expressed interest in buying the stores.

The spinoff underlines the current turmoil in oil refining. Struggling
with lower sales, Marathon announced last week that it would not restart
two refineries in New Mexico and California that it had idled in April.

Its refinery in Martinez, Calif., was under particular pressure because
of California's tight environmental specifications, and it will be
converted into a terminal facility for storage and loading. The Gallup
plant in New Mexico was hurt by a small local market and high costs.

Both gasoline and diesel sales have recovered some in recent weeks, but
jet fuel sales are severely depressed and there is no sign of a recovery
for at least another year, analysts say.

Advertisement

\protect\hyperlink{after-bottom}{Continue reading the main story}

\hypertarget{site-index}{%
\subsection{Site Index}\label{site-index}}

\hypertarget{site-information-navigation}{%
\subsection{Site Information
Navigation}\label{site-information-navigation}}

\begin{itemize}
\tightlist
\item
  \href{https://help.nytimes3xbfgragh.onion/hc/en-us/articles/115014792127-Copyright-notice}{©~2020~The
  New York Times Company}
\end{itemize}

\begin{itemize}
\tightlist
\item
  \href{https://www.nytco.com/}{NYTCo}
\item
  \href{https://help.nytimes3xbfgragh.onion/hc/en-us/articles/115015385887-Contact-Us}{Contact
  Us}
\item
  \href{https://www.nytco.com/careers/}{Work with us}
\item
  \href{https://nytmediakit.com/}{Advertise}
\item
  \href{http://www.tbrandstudio.com/}{T Brand Studio}
\item
  \href{https://www.nytimes3xbfgragh.onion/privacy/cookie-policy\#how-do-i-manage-trackers}{Your
  Ad Choices}
\item
  \href{https://www.nytimes3xbfgragh.onion/privacy}{Privacy}
\item
  \href{https://help.nytimes3xbfgragh.onion/hc/en-us/articles/115014893428-Terms-of-service}{Terms
  of Service}
\item
  \href{https://help.nytimes3xbfgragh.onion/hc/en-us/articles/115014893968-Terms-of-sale}{Terms
  of Sale}
\item
  \href{https://spiderbites.nytimes3xbfgragh.onion}{Site Map}
\item
  \href{https://help.nytimes3xbfgragh.onion/hc/en-us}{Help}
\item
  \href{https://www.nytimes3xbfgragh.onion/subscription?campaignId=37WXW}{Subscriptions}
\end{itemize}
