Sections

SEARCH

\protect\hyperlink{site-content}{Skip to
content}\protect\hyperlink{site-index}{Skip to site index}

\href{https://www.nytimes3xbfgragh.onion/section/business}{Business}

\href{https://myaccount.nytimes3xbfgragh.onion/auth/login?response_type=cookie\&client_id=vi}{}

\href{https://www.nytimes3xbfgragh.onion/section/todayspaper}{Today's
Paper}

\href{/section/business}{Business}\textbar{}Deutsche Bank Opens Review
Into Personal Banker to Trump and Kushner

\url{https://nyti.ms/3i3rKjU}

\begin{itemize}
\item
\item
\item
\item
\item
\end{itemize}

Advertisement

\protect\hyperlink{after-top}{Continue reading the main story}

Supported by

\protect\hyperlink{after-sponsor}{Continue reading the main story}

\hypertarget{deutsche-bank-opens-review-into-personal-banker-to-trump-and-kushner}{%
\section{Deutsche Bank Opens Review Into Personal Banker to Trump and
Kushner}\label{deutsche-bank-opens-review-into-personal-banker-to-trump-and-kushner}}

The bank will examine a 2013 transaction between the banker, Rosemary
Vrablic, and a company part-owned by Jared Kushner.

\includegraphics{https://static01.graylady3jvrrxbe.onion/images/2020/08/02/business/02deutsche-kushner/merlin_174921309_516ed7d8-a0d4-4b17-851f-226f2e4e12f1-articleLarge.jpg?quality=75\&auto=webp\&disable=upscale}

By \href{https://www.nytimes3xbfgragh.onion/by/jesse-drucker}{Jesse
Drucker} and
\href{https://www.nytimes3xbfgragh.onion/by/david-enrich}{David Enrich}

\begin{itemize}
\item
  Aug. 2, 2020
\item
  \begin{itemize}
  \item
  \item
  \item
  \item
  \item
  \end{itemize}
\end{itemize}

Deutsche Bank has opened an internal investigation into the longtime
personal banker of President Trump and his son-in-law, Jared Kushner,
over a 2013 real estate transaction between the banker and a company
part-owned by Mr. Kushner.

In June 2013, the banker, Rosemary Vrablic, and two of her Deutsche Bank
colleagues purchased a Park Avenue apartment for about \$1.5 million
from a company called Bergel 715 Associates, according to New York
\href{https://a836-acris.nyc.gov/DS/DocumentSearch/DocumentDetail?doc_id=2013062601437001}{property
records}.

Mr. Kushner, a senior adviser to the president, disclosed in an annual
personal financial report late Friday that he and his wife, Ivanka
Trump, had received \$1 million to \$5 million last year from Bergel
715. A person familiar with Mr. Kushner's finances, who wasn't
authorized to speak publicly, said he held an ownership stake in the
entity at the time of the transaction with Ms. Vrablic.

When Ms. Vrablic and her colleagues bought the apartment on Manhattan's
Upper East Side, Mr. Trump and Mr. Kushner were
\href{https://www.nytimes3xbfgragh.onion/2020/02/04/magazine/deutsche-bank-trump.html}{her
clients at Deutsche Bank}. They had received roughly \$190 million in
loans from the bank and would seek hundreds of millions of dollars more.

Typically banks restrict employees from doing personal business with
clients because of the potential for conflicts between the employees'
interests and those of the bank.

Deutsche Bank said it had not been aware that Ms. Vrablic and her
colleagues had done business with a company part-owned by Mr. Kushner
until being contacted by The New York Times.

``The bank will closely examine the information that came to light on
Friday and the fact pattern from 2013,'' said Daniel Hunter, a bank
spokesman.

A lawyer for Ms. Vrablic, a senior private banker and managing director
at Deutsche Bank, declined to comment.

The White House referred questions to the Kushner family's real estate
company. Christopher Smith, the general counsel at Kushner Companies,
said: ``Kushner is not the managing partner of that entity and has no
involvement with the sales of the apartments.''

Ms. Vrablic bought the apartment, in a brick building at 715 Park
Avenue, with Dominic Scalzi and Matthew Pontoriero. They worked for Ms.
Vrablic in Deutsche Bank's private-banking division, which caters to
wealthy clients. Mr. Scalzi and Mr. Pontoriero didn't respond to
requests for comment on Sunday.

The size of Mr. Kushner's stake in Bergel 715 is unclear. The company
has sold dozens of condo units in the Park Avenue building since the
1980s, according to public records. At least one apartment was sold to
the Kushner family's real estate company.

Mr. Kushner and Ms. Trump had not previously disclosed their stake in
Bergel 715. (They did list the entity used to make the investment in
Bergel 715.) The income they reported in 2019 wasn't related to the
transaction with Ms. Vrablic.

Bergel 715's main owners include
\href{http://gellertglobalgroup.com/holdings.php?id=waltersamuels}{George
Gellert}, a close friend of the Kushner family and an investor in
numerous deals with Kushner Companies.

There is no indication that the three Deutsche Bank employees bought the
apartment
---\href{https://www.zillow.com/homedetails/715-Park-Ave-APT-12A-New-York-NY-10021/31535036_zpid/}{described
on Zillow} as a 908-square-foot, one-bedroom, one-bath unit with a
balcony overlooking Park Avenue --- at a below-market price.

In 2014, the deed for the apartment, Unit 12A,
\href{https://a836-acris.nyc.gov/DS/DocumentSearch/DocumentDetail?doc_id=2013122700639001}{was
transferred} to a limited liability company registered to Ms. Vrablic's
home address, according to property records. The next year, the
apartment
\href{https://a836-acris.nyc.gov/DS/DocumentSearch/DocumentDetail?doc_id=2015111100189001}{was
sold} for \$1.85 million --- a not-unheard-of 22 percent increase from
the 2013 purchase price.

Ms. Vrablic has worked in the Deutsche Bank private-banking division
since 2006. She has a reputation as one of New York's leading private
bankers, generating tens of millions of dollars of annual revenue for
the bank.

The Kushner family has been a client of Ms. Vrablic's since before she
joined Deutsche Bank. In 2011, Mr. Kushner brought Ms. Vrablic to meet
his father-in-law. At the time, most mainstream banks
\href{https://www.nytimes3xbfgragh.onion/2019/03/18/business/trump-deutsche-bank.html}{refused
to do business} with Mr. Trump because of his history of defaults and
bankruptcies.

``I introduced him to this woman Rosemary,'' Mr. Kushner said in
closed-door
\href{https://intelligence.house.gov/uploadedfiles/jk25.pdf}{testimony}
to the House Intelligence Committee in 2017. ``She is one of the biggest
private wealth bankers, probably in the world. Amazing banker, amazing
woman. Very smart banker. And she banked my family for a long time.''

Ms. Vrablic and her superiors soon agreed to take Mr. Trump on as a
client, even though he had defaulted on a loan from the bank three years
earlier. In 2012, Deutsche Bank lent Mr. Trump a total of about \$175
million for his newly acquired Doral golf resort outside Miami and for
his Trump International Hotel \& Tower in Chicago.

Mr. Trump soon came back for more. In 2014 he sought a \$1 billion
commitment from Ms. Vrablic to buy the Buffalo Bills football team. (Mr.
Trump's bid was rejected, making the loan unnecessary.) The bank agreed
to lend Mr. Trump's company \$170 million for its transformation of the
Old Post Office building into the Trump International Hotel in
Washington. And Mr. Kushner and his mother received a \$15 million
personal line of credit from Ms. Vrablic's division, the largest credit
line to which Mr. Kushner or his parents had access, according to
\href{https://www.nytimes3xbfgragh.onion/2018/10/13/business/jared-kushner-taxes.html}{financial
records} reviewed by The Times.

Ms. Vrablic was thrust into the spotlight when Mr. Trump
\href{https://www.nytimes3xbfgragh.onion/2016/05/24/business/dealbook/donald-trump-relationship-bankers.html}{boasted
to The Times} in 2016 about his strong relationship with Deutsche Bank
--- and inflated Ms. Vrablic's role at the bank. ``Why don't you call
the head of Deutsche Bank? Her name is Rosemary Vrablic,'' he said in
the interview. ``She is the boss.''

Advertisement

\protect\hyperlink{after-bottom}{Continue reading the main story}

\hypertarget{site-index}{%
\subsection{Site Index}\label{site-index}}

\hypertarget{site-information-navigation}{%
\subsection{Site Information
Navigation}\label{site-information-navigation}}

\begin{itemize}
\tightlist
\item
  \href{https://help.nytimes3xbfgragh.onion/hc/en-us/articles/115014792127-Copyright-notice}{©~2020~The
  New York Times Company}
\end{itemize}

\begin{itemize}
\tightlist
\item
  \href{https://www.nytco.com/}{NYTCo}
\item
  \href{https://help.nytimes3xbfgragh.onion/hc/en-us/articles/115015385887-Contact-Us}{Contact
  Us}
\item
  \href{https://www.nytco.com/careers/}{Work with us}
\item
  \href{https://nytmediakit.com/}{Advertise}
\item
  \href{http://www.tbrandstudio.com/}{T Brand Studio}
\item
  \href{https://www.nytimes3xbfgragh.onion/privacy/cookie-policy\#how-do-i-manage-trackers}{Your
  Ad Choices}
\item
  \href{https://www.nytimes3xbfgragh.onion/privacy}{Privacy}
\item
  \href{https://help.nytimes3xbfgragh.onion/hc/en-us/articles/115014893428-Terms-of-service}{Terms
  of Service}
\item
  \href{https://help.nytimes3xbfgragh.onion/hc/en-us/articles/115014893968-Terms-of-sale}{Terms
  of Sale}
\item
  \href{https://spiderbites.nytimes3xbfgragh.onion}{Site Map}
\item
  \href{https://help.nytimes3xbfgragh.onion/hc/en-us}{Help}
\item
  \href{https://www.nytimes3xbfgragh.onion/subscription?campaignId=37WXW}{Subscriptions}
\end{itemize}
