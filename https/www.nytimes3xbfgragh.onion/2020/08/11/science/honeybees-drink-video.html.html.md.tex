Sections

SEARCH

\protect\hyperlink{site-content}{Skip to
content}\protect\hyperlink{site-index}{Skip to site index}

\href{https://www.nytimes3xbfgragh.onion/section/science}{Science}

\href{https://myaccount.nytimes3xbfgragh.onion/auth/login?response_type=cookie\&client_id=vi}{}

\href{https://www.nytimes3xbfgragh.onion/section/todayspaper}{Today's
Paper}

\href{/section/science}{Science}\textbar{}A Honeybee's Tongue Is More
Swiss Army Knife Than Ladle

\url{https://nyti.ms/31LnbEf}

\begin{itemize}
\item
\item
\item
\item
\item
\end{itemize}

Advertisement

\protect\hyperlink{after-top}{Continue reading the main story}

Supported by

\protect\hyperlink{after-sponsor}{Continue reading the main story}

Trilobites

\hypertarget{a-honeybees-tongue-is-more-swiss-army-knife-than-ladle}{%
\section{A Honeybee's Tongue Is More Swiss Army Knife Than
Ladle}\label{a-honeybees-tongue-is-more-swiss-army-knife-than-ladle}}

Once again, insects prove to be more complicated than scientists thought
they were.

\includegraphics{https://static01.graylady3jvrrxbe.onion/images/2020/08/11/science/11TB-BEES/11TB-BEES-articleLarge.jpg?quality=75\&auto=webp\&disable=upscale}

By \href{https://www.nytimes3xbfgragh.onion/by/james-gorman}{James
Gorman}

\begin{itemize}
\item
  Aug. 11, 2020
\item
  \begin{itemize}
  \item
  \item
  \item
  \item
  \item
  \end{itemize}
\end{itemize}

For a century, scientists have known how honeybees drink nectar. They
lap it up.

They don't lap like cats or dogs,
\href{https://www.nytimes3xbfgragh.onion/video/science/100000003273592/how-dogs-drink.html}{videos
of whose mesmerizing drinking habits}have been one of the great rewards
of high speed video. But they do dip their hairy tongues rapidly in and
out of syrupy nectar to draw it up into their mouth. For the last
century or so, scientists have been convinced that this is the only way
they drink nectar.

\includegraphics{https://static01.graylady3jvrrxbe.onion/images/2020/11/08/science/11tb-bees-lap-image/11tb-bees-lap-image-videoSixteenByNineJumbo1600.png}

Scientists have now discovered bees can also suck nectar, which is more
efficient when the sugar content is lower and the nectar is less
viscous. High-speed video of bees drinking a nectar substitute in a lab
shows that not only do honeybees have this unexpected ability, they can
go back and forth from one drinking mode to another.

Jianing Wu, an engineering and biophysics specialist, at Sun Yat-sen
University in Guangzhou, China, and the senior researcher on the
experiment, said that while honeybees excel at feeding on highly
concentrated nectar, ``we find that they can also flexibly switch the
feeding strategy from lapping to suction.'' He and his colleagues
\href{http://\%0A22\%0A23\%0A24\%0A25\%0A26\%0A27\%0A28\%0A29\%0A30\%0A31\%0A32\%0A33\%0A34\%0A35\%0A36\%20feeding\%20strategy,\%20nectar\%20intake\%2037\%0Ahttp://dx.doi.org/10.1098/rsbl.2020.0449}{reported
the results on Wednesday in the journal Biology Letters.}

\includegraphics{https://static01.graylady3jvrrxbe.onion/images/2020/11/08/science/11tb-bees-suction-image/11tb-bees-suction-image-videoSixteenByNineJumbo1600.png}

\href{https://www.nytimes3xbfgragh.onion/2018/11/05/science/hu-robotics.html}{David
Hu, a professor at the Georgia Institute of Technology}, who supervised
some of Dr. Wu's earlier research but was not involved in this
experiment, said: ``We thought that insects' mouths were like tools in
your kitchen drawer (straw, fork, spoon), e.g., with single uses.''

``Wu showed that honeybee tongues are like a Swiss army knife, able to
efficiently drink many type of nectar,'' Dr. Hu said.

Alejandro Rico-Guevara, who runs the Behavioral Ecophysics Lab at the
University of Washington, Seattle and studies
\href{https://www.nytimes3xbfgragh.onion/2015/09/08/science/the-hummingbirds-tongue-how-it-works.html}{nectar
feeding in birds}, also worked on the project. He said this flexibility
in nectar drinking behavior means that although bees prefer the more
syrupy nectars, they can efficiently feed from flowers whose nectar is
more watery. ``This has implications at many different scales, from
pollination, for our food, all the way to the role they have in natural
ecosystems,'' he said.

What Dr. Rico-Guevara found most interesting was that the bees are so
sensitive to the viscosity of the nectar that ``they switch at the exact
point you would expect, to get the best reward for the energy
invested.''

The honeybee tongue is adapted perfectly to lapping syrupy nectars. Once
the tongue is dipped into thick nectars, Dr. Wu explained,
``approximately 10,000 bristles covering the tongue erect simultaneously
at a certain angle for trapping the nectar.'' The bee then pulls its
tongue back into its proboscis, which is really a part of its mouth, and
a pumping mechanism in the head sucks the nectar off the tongue.

\includegraphics{https://static01.graylady3jvrrxbe.onion/images/2020/11/08/science/11tb-bees-bristles-image/11tb-bees-bristles-image-superJumbo.png}

When the viscosity changes so that the nectar is less thick, the bees
let their tongues stay in the nectar and sucked it up into their mouths,
apparently using the same pumping mechanism.

Dr. Hu said, ``The result makes perfect sense because honeybees are
already known as generalists.'' They are not limited to feeding on only
one type of flower like some other species of bee.

The bees have been flexible all along. It was the scientists who were
stuck on one idea.

Advertisement

\protect\hyperlink{after-bottom}{Continue reading the main story}

\hypertarget{site-index}{%
\subsection{Site Index}\label{site-index}}

\hypertarget{site-information-navigation}{%
\subsection{Site Information
Navigation}\label{site-information-navigation}}

\begin{itemize}
\tightlist
\item
  \href{https://help.nytimes3xbfgragh.onion/hc/en-us/articles/115014792127-Copyright-notice}{©~2020~The
  New York Times Company}
\end{itemize}

\begin{itemize}
\tightlist
\item
  \href{https://www.nytco.com/}{NYTCo}
\item
  \href{https://help.nytimes3xbfgragh.onion/hc/en-us/articles/115015385887-Contact-Us}{Contact
  Us}
\item
  \href{https://www.nytco.com/careers/}{Work with us}
\item
  \href{https://nytmediakit.com/}{Advertise}
\item
  \href{http://www.tbrandstudio.com/}{T Brand Studio}
\item
  \href{https://www.nytimes3xbfgragh.onion/privacy/cookie-policy\#how-do-i-manage-trackers}{Your
  Ad Choices}
\item
  \href{https://www.nytimes3xbfgragh.onion/privacy}{Privacy}
\item
  \href{https://help.nytimes3xbfgragh.onion/hc/en-us/articles/115014893428-Terms-of-service}{Terms
  of Service}
\item
  \href{https://help.nytimes3xbfgragh.onion/hc/en-us/articles/115014893968-Terms-of-sale}{Terms
  of Sale}
\item
  \href{https://spiderbites.nytimes3xbfgragh.onion}{Site Map}
\item
  \href{https://help.nytimes3xbfgragh.onion/hc/en-us}{Help}
\item
  \href{https://www.nytimes3xbfgragh.onion/subscription?campaignId=37WXW}{Subscriptions}
\end{itemize}
