Sections

SEARCH

\protect\hyperlink{site-content}{Skip to
content}\protect\hyperlink{site-index}{Skip to site index}

\href{https://www.nytimes3xbfgragh.onion/section/us}{U.S.}

\href{https://myaccount.nytimes3xbfgragh.onion/auth/login?response_type=cookie\&client_id=vi}{}

\href{https://www.nytimes3xbfgragh.onion/section/todayspaper}{Today's
Paper}

\href{/section/us}{U.S.}\textbar{}The Virus Is Killing Young Floridians.
Race Is a Big Factor.

\begin{itemize}
\item
\item
\item
\item
\item
\end{itemize}

\hypertarget{the-coronavirus-outbreak}{%
\subsubsection{\texorpdfstring{\href{https://www.nytimes3xbfgragh.onion/news-event/coronavirus?name=styln-coronavirus-national\&region=TOP_BANNER\&variant=undefined\&block=storyline_menu_recirc\&action=click\&pgtype=Article\&impression_id=c3e64bb0-e375-11ea-9198-f319f806a34f}{The
Coronavirus
Outbreak}}{The Coronavirus Outbreak}}\label{the-coronavirus-outbreak}}

\begin{itemize}
\tightlist
\item
  live\href{https://www.nytimes3xbfgragh.onion/2020/08/20/world/coronavirus-covid.html?name=styln-coronavirus-national\&region=TOP_BANNER\&variant=undefined\&block=storyline_menu_recirc\&action=click\&pgtype=Article\&impression_id=c3e64bb1-e375-11ea-9198-f319f806a34f}{Latest
  Updates}
\item
  \href{https://www.nytimes3xbfgragh.onion/interactive/2020/us/coronavirus-us-cases.html?name=styln-coronavirus-national\&region=TOP_BANNER\&variant=undefined\&block=storyline_menu_recirc\&action=click\&pgtype=Article\&impression_id=c3e64bb2-e375-11ea-9198-f319f806a34f}{Maps
  and Cases}
\item
  \href{https://www.nytimes3xbfgragh.onion/interactive/2020/science/coronavirus-vaccine-tracker.html?name=styln-coronavirus-national\&region=TOP_BANNER\&variant=undefined\&block=storyline_menu_recirc\&action=click\&pgtype=Article\&impression_id=c3e64bb3-e375-11ea-9198-f319f806a34f}{Vaccine
  Tracker}
\item
  \href{https://www.nytimes3xbfgragh.onion/2020/08/19/us/colleges-closing-covid.html?name=styln-coronavirus-national\&region=TOP_BANNER\&variant=undefined\&block=storyline_menu_recirc\&action=click\&pgtype=Article\&impression_id=c3e64bb4-e375-11ea-9198-f319f806a34f}{Colleges
  Closing}
\item
  \href{https://www.nytimes3xbfgragh.onion/live/2020/08/20/business/stock-market-today-coronavirus?name=styln-coronavirus-national\&region=TOP_BANNER\&variant=undefined\&block=storyline_menu_recirc\&action=click\&pgtype=Article\&impression_id=c3e672c0-e375-11ea-9198-f319f806a34f}{Economy}
\end{itemize}

Advertisement

\protect\hyperlink{after-top}{Continue reading the main story}

Supported by

\protect\hyperlink{after-sponsor}{Continue reading the main story}

\hypertarget{the-virus-is-killing-young-floridians-race-is-a-big-factor}{%
\section{The Virus Is Killing Young Floridians. Race Is a Big
Factor.}\label{the-virus-is-killing-young-floridians-race-is-a-big-factor}}

\includegraphics{https://static01.graylady3jvrrxbe.onion/images/2020/08/05/us/00virus-youngdeaths-castro/merlin_175159566_3199af58-1749-46c3-a077-d218e1096a2a-articleLarge.jpg?quality=75\&auto=webp\&disable=upscale}

By \href{https://www.nytimes3xbfgragh.onion/by/frances-robles}{Frances
Robles},
\href{https://www.nytimes3xbfgragh.onion/by/robert-gebeloff}{Robert
Gebeloff},
\href{https://www.nytimes3xbfgragh.onion/by/danielle-ivory}{Danielle
Ivory} and
\href{https://www.nytimes3xbfgragh.onion/by/kimiko-de-freytas-tamura}{Kimiko
de Freytas-Tamura}

\begin{itemize}
\item
  Aug. 11, 2020
\item
  \begin{itemize}
  \item
  \item
  \item
  \item
  \item
  \end{itemize}
\end{itemize}

The last time Miriam Castro saw her son Herman, he was in the hospital.
A mask covered his tear-stricken face as he sobbed over a FaceTime call.

``He kept saying: `I love you, Ma, I love you, Ma. Take care of
yourselves. This is no game,''' Mrs. Castro recalled. ``He cried and
cried.''

Herman J. Castro, a 32-year-old manager of a McDonald's in Central
Florida, died two days later.

He is among the more than 100 adults age 25 to 44 who died of Covid-19
in Florida last month.

\hypertarget{coronavirus-deaths-rise-for-younger-adults}{%
\subsection{Coronavirus Deaths Rise for Younger
Adults}\label{coronavirus-deaths-rise-for-younger-adults}}

More than 80 percent of coronavirus deaths in Florida are attributed to
residents over the age of 65, but since a surge in deaths in July, the
number of younger people dying has also increased substantially.

Weekly deaths by age group

25 to 44

45--64

200

150

100

50

0

April 4

April 18

May 2

May 16

May 30

June 13

June 27

July 11

July 25

Weekly deaths by age group

25 to 44

45--64

200

150

100

50

0

April 4

May 2

May 30

June 27

July 25

Weekly deaths by age group

25 to 44

45-64

200

150

100

50

0

April 4

May 2

May 30

June 27

July 25

Source: Florida Department of Health

By Jugal K. Patel

Throughout the pandemic, Gov. Ron DeSantis of Florida has stressed that
the state's coronavirus crisis is largely limited to the very old. He
has repeatedly noted that Florida has seen more coronavirus deaths in
people over the age of 90 than in all people under 65. But data reviewed
by The New York Times shows that's changing: Deaths were greater in July
for residents under 65 than for those over 90.

Additionally, more Floridians in the 25-44 age group died in July than
had died in the previous four months of the pandemic combined, a review
of Florida Department of Health data shows. More than 200 have died in
all.

Although they still make up a relatively small number of the more than
8,000 total coronavirus deaths in the state, the number of younger
adults who died of the disease quadrupled last month, underscoring a
bitter mathematical reality: As more and more young people test positive
for the coronavirus, more of them will die.

Tali Elfassy, an assistant professor of epidemiology at the University
of Miami, said that while the case fatality rate is lower for younger
people, there has recently been a huge shift in the age distribution of
those getting infected --- from middle-aged and older individuals, to
younger adults.

``Even if the majority of people who are dying from Covid are older
individuals, the fact that we have this demographic shift toward younger
people becoming infected is going to give us an increase in deaths among
younger people,'' she said.

Nationally, the share of all deaths that occur in younger age groups
remains small --- just 38 coronavirus deaths out of every 1,000 in July
were attributed to younger people, but that is up from 22 per 1,000 in
May.

\hypertarget{latest-updates-the-coronavirus-outbreak}{%
\section{\texorpdfstring{\href{https://www.nytimes3xbfgragh.onion/2020/08/20/world/coronavirus-covid.html?action=click\&pgtype=Article\&state=default\&region=MAIN_CONTENT_1\&context=storylines_live_updates}{Latest
Updates: The Coronavirus
Outbreak}}{Latest Updates: The Coronavirus Outbreak}}\label{latest-updates-the-coronavirus-outbreak}}

Updated 2020-08-21T05:31:08.268Z

\begin{itemize}
\tightlist
\item
  \href{https://www.nytimes3xbfgragh.onion/2020/08/20/world/coronavirus-covid.html?action=click\&pgtype=Article\&state=default\&region=MAIN_CONTENT_1\&context=storylines_live_updates\#link-68774d88}{Shutdowns,
  warnings and scoldings follow alarming incidents on college campuses.}
\item
  \href{https://www.nytimes3xbfgragh.onion/2020/08/20/world/coronavirus-covid.html?action=click\&pgtype=Article\&state=default\&region=MAIN_CONTENT_1\&context=storylines_live_updates\#link-26b58724}{Biden
  knocks Trump's pandemic response, and outlines a national strategy.}
\item
  \href{https://www.nytimes3xbfgragh.onion/2020/08/20/world/coronavirus-covid.html?action=click\&pgtype=Article\&state=default\&region=MAIN_CONTENT_1\&context=storylines_live_updates\#link-4e542da3}{U.S.
  health agencies announce moves to confront the flu season and
  plummeting child vaccination rates.}
\end{itemize}

\href{https://www.nytimes3xbfgragh.onion/2020/08/20/world/coronavirus-covid.html?action=click\&pgtype=Article\&state=default\&region=MAIN_CONTENT_1\&context=storylines_live_updates}{See
more updates}

More live coverage:
\href{https://www.nytimes3xbfgragh.onion/live/2020/08/20/business/stock-market-today-coronavirus?action=click\&pgtype=Article\&state=default\&region=MAIN_CONTENT_1\&context=storylines_live_updates}{Markets}

Through July, about 3,800 people in the United States in the age group
had died from coronavirus, according to figures published by the
\href{https://data.cdc.gov/NCHS/Weekly-counts-of-deaths-by-jurisdiction-and-age-gr/y5bj-9g5w/}{Centers
for Disease Control and Prevention}. That would make the coronavirus one
of the leading causes of death for this age group, roughly comparable to
the number of younger people who were murdered over the same time period
in recent years.

And these figures include only deaths officially attributed to the
virus. The overall
\href{https://www.cdc.gov/nchs/nvss/vsrr/covid19/excess_deaths.htm}{death
count} in the U.S. is
\href{https://www.nytimes3xbfgragh.onion/interactive/2020/05/05/us/coronavirus-death-toll-us.html}{far
higher} than normal this year, suggesting that the virus's toll is even
greater than the official numbers show. More than 96,000 people age
25-44 have died so far this year, compared with an average of 78,000 at
this point over the previous five years.

Health officials have worried that young people have been overly
reckless in resuming social activities at parties and bars, and the
number of infections among younger people has soared. However, the young
people who are dying are not necessarily those who got sick at a party.

Among the young who succumbed to the coronavirus recently in Florida
were a 26-year-old
\href{https://www.palmbeachpost.com/news/20200716/community-remembers-florida-woman-rsquodesirsquo-who-died-of-covid-19-after-going-to-work-with-respiratory-symptoms}{convenience
store clerk} and a 42-year-old restaurant cook. At least three of the
people who died worked in residential facilities caring for the ill and
disabled; one of them was 41 and two were 31. One 35-year-old woman
worked at the front desk of a hospital.

Records also show the people who died from the virus in Florida among
the young were disproportionately Black. Among people age 25 to 44,
African-Americans make up 18 percent of Florida's population but have
accounted for 44 percent of deaths. Black Floridians over 65 are dying
at twice the rate of white residents, but among younger adults, the
death rate is nearly three times as high.

\hypertarget{deaths-among-black-young-adults-are-disproportionately-higher}{%
\subsection{Deaths Among Black Young Adults Are Disproportionately
Higher}\label{deaths-among-black-young-adults-are-disproportionately-higher}}

African-Americans make up 15.5 percent of Florida's population and 19.4
percent of those who have died from the virus. The disparity is most
stark when accounting for age.

Black share of state population

​ Percentage of deaths

Age 25--44

18\%

44\%

45--64

14\%

33\%

65--89

9\%

18\%

90+

7\%

9\%

Black share of state population

​ Percentage of deaths

Age 25--44

18\%

44\%

45--64

14\%

33\%

65--89

9\%

18\%

90+

7\%

9\%

Black share of state population

​ Percentage of deaths

Age 25--44

18\%

44\%

45--64

14\%

33\%

65--89

9\%

18\%

90+

7\%

9\%

Source: Florida Department of Health; Census Bureau

By Jugal K. Patel

More than a thousand Latinos in Florida have also died from the virus,
though their death rate for most age groups is comparable to the death
rate for non-Hispanic white people.

Many of the younger victims had diabetes or were obese, highlighting the
risks people with health problems face no matter their age.

Robert Ruiz, of Sebring, Fla., worked at a center for people with
traumatic brain injuries. He got sick with the virus from one of the
patients.

``He was only 31,'' his sister, Chenique Mills, said. ``That's why this
is so unbelievable. He's not supposed to be gone. This is crazy.''

\includegraphics{https://static01.graylady3jvrrxbe.onion/images/2020/08/05/us/00youngdeaths-ruiz/merlin_175159626_65499c51-9c55-4958-afe7-5227506a0b9f-articleLarge.jpg?quality=75\&auto=webp\&disable=upscale}

Mr. Ruiz was overweight but had no underlying medical conditions, Ms.
Mills said. His Facebook account is filled with selfies taken at the
gym, where he was trying to slim down, she said. A born-again Christian,
he was a rap artist who enjoyed performing at church functions.

He died July 12 in his bedroom, two days after the onset of symptoms.
The coronavirus test administered every two weeks by his workplace came
back positive two days later. He left behind a toddler.

``He was blowing it off. He didn't think it was that serious, either,''
Ms. Mills said. ``This whole `you're young, it's not going to affect
you' is irritating.''

Mr. DeSantis said last month that the median age of new coronavirus
cases had reached a low of 33, down from 50s and 60s in March and April.
Records show that when tens of thousands of young people started testing
positive shortly after the state reopened, the average age began to
sneak back up, as young people infected the older people around them.

\href{https://www.nytimes3xbfgragh.onion/news-event/coronavirus?action=click\&pgtype=Article\&state=default\&region=MAIN_CONTENT_3\&context=storylines_faq}{}

\hypertarget{the-coronavirus-outbreak-}{%
\subsubsection{The Coronavirus Outbreak
›}\label{the-coronavirus-outbreak-}}

\hypertarget{frequently-asked-questions}{%
\paragraph{Frequently Asked
Questions}\label{frequently-asked-questions}}

Updated August 17, 2020

\begin{itemize}
\item ~
  \hypertarget{why-does-standing-six-feet-away-from-others-help}{%
  \paragraph{Why does standing six feet away from others
  help?}\label{why-does-standing-six-feet-away-from-others-help}}

  \begin{itemize}
  \tightlist
  \item
    The coronavirus spreads primarily through droplets from your mouth
    and nose, especially when you cough or sneeze. The C.D.C., one of
    the organizations using that measure,
    \href{https://www.nytimes3xbfgragh.onion/2020/04/14/health/coronavirus-six-feet.html?action=click\&pgtype=Article\&state=default\&region=MAIN_CONTENT_3\&context=storylines_faq}{bases
    its recommendation of six feet} on the idea that most large droplets
    that people expel when they cough or sneeze will fall to the ground
    within six feet. But six feet has never been a magic number that
    guarantees complete protection. Sneezes, for instance, can launch
    droplets a lot farther than six feet,
    \href{https://jamanetwork.com/journals/jama/fullarticle/2763852}{according
    to a recent study}. It's a rule of thumb: You should be safest
    standing six feet apart outside, especially when it's windy. But
    keep a mask on at all times, even when you think you're far enough
    apart.
  \end{itemize}
\item ~
  \hypertarget{i-have-antibodies-am-i-now-immune}{%
  \paragraph{I have antibodies. Am I now
  immune?}\label{i-have-antibodies-am-i-now-immune}}

  \begin{itemize}
  \tightlist
  \item
    As of right
    now,\href{https://www.nytimes3xbfgragh.onion/2020/07/22/health/covid-antibodies-herd-immunity.html?action=click\&pgtype=Article\&state=default\&region=MAIN_CONTENT_3\&context=storylines_faq}{that
    seems likely, for at least several months.} There have been
    frightening accounts of people suffering what seems to be a second
    bout of Covid-19. But experts say these patients may have a
    drawn-out course of infection, with the virus taking a slow toll
    weeks to months after initial exposure. People infected with the
    coronavirus typically
    \href{https://www.nature.com/articles/s41586-020-2456-9}{produce}
    immune molecules called antibodies, which are
    \href{https://www.nytimes3xbfgragh.onion/2020/05/07/health/coronavirus-antibody-prevalence.html?action=click\&pgtype=Article\&state=default\&region=MAIN_CONTENT_3\&context=storylines_faq}{protective
    proteins made in response to an
    infection}\href{https://www.nytimes3xbfgragh.onion/2020/05/07/health/coronavirus-antibody-prevalence.html?action=click\&pgtype=Article\&state=default\&region=MAIN_CONTENT_3\&context=storylines_faq}{.
    These antibodies may} last in the body
    \href{https://www.nature.com/articles/s41591-020-0965-6}{only two to
    three months}, which may seem worrisome, but that's perfectly normal
    after an acute infection subsides, said Dr. Michael Mina, an
    immunologist at Harvard University. It may be possible to get the
    coronavirus again, but it's highly unlikely that it would be
    possible in a short window of time from initial infection or make
    people sicker the second time.
  \end{itemize}
\item ~
  \hypertarget{im-a-small-business-owner-can-i-get-relief}{%
  \paragraph{I'm a small-business owner. Can I get
  relief?}\label{im-a-small-business-owner-can-i-get-relief}}

  \begin{itemize}
  \tightlist
  \item
    The
    \href{https://www.nytimes3xbfgragh.onion/article/small-business-loans-stimulus-grants-freelancers-coronavirus.html?action=click\&pgtype=Article\&state=default\&region=MAIN_CONTENT_3\&context=storylines_faq}{stimulus
    bills enacted in March} offer help for the millions of American
    small businesses. Those eligible for aid are businesses and
    nonprofit organizations with fewer than 500 workers, including sole
    proprietorships, independent contractors and freelancers. Some
    larger companies in some industries are also eligible. The help
    being offered, which is being managed by the Small Business
    Administration, includes the Paycheck Protection Program and the
    Economic Injury Disaster Loan program. But lots of folks have
    \href{https://www.nytimes3xbfgragh.onion/interactive/2020/05/07/business/small-business-loans-coronavirus.html?action=click\&pgtype=Article\&state=default\&region=MAIN_CONTENT_3\&context=storylines_faq}{not
    yet seen payouts.} Even those who have received help are confused:
    The rules are draconian, and some are stuck sitting on
    \href{https://www.nytimes3xbfgragh.onion/2020/05/02/business/economy/loans-coronavirus-small-business.html?action=click\&pgtype=Article\&state=default\&region=MAIN_CONTENT_3\&context=storylines_faq}{money
    they don't know how to use.} Many small-business owners are getting
    less than they expected or
    \href{https://www.nytimes3xbfgragh.onion/2020/06/10/business/Small-business-loans-ppp.html?action=click\&pgtype=Article\&state=default\&region=MAIN_CONTENT_3\&context=storylines_faq}{not
    hearing anything at all.}
  \end{itemize}
\item ~
  \hypertarget{what-are-my-rights-if-i-am-worried-about-going-back-to-work}{%
  \paragraph{What are my rights if I am worried about going back to
  work?}\label{what-are-my-rights-if-i-am-worried-about-going-back-to-work}}

  \begin{itemize}
  \tightlist
  \item
    Employers have to provide
    \href{https://www.osha.gov/SLTC/covid-19/standards.html}{a safe
    workplace} with policies that protect everyone equally.
    \href{https://www.nytimes3xbfgragh.onion/article/coronavirus-money-unemployment.html?action=click\&pgtype=Article\&state=default\&region=MAIN_CONTENT_3\&context=storylines_faq}{And
    if one of your co-workers tests positive for the coronavirus, the
    C.D.C.} has said that
    \href{https://www.cdc.gov/coronavirus/2019-ncov/community/guidance-business-response.html}{employers
    should tell their employees} -\/- without giving you the sick
    employee's name -\/- that they may have been exposed to the virus.
  \end{itemize}
\item ~
  \hypertarget{what-is-school-going-to-look-like-in-september}{%
  \paragraph{What is school going to look like in
  September?}\label{what-is-school-going-to-look-like-in-september}}

  \begin{itemize}
  \tightlist
  \item
    It is unlikely that many schools will return to a normal schedule
    this fall, requiring the grind of
    \href{https://www.nytimes3xbfgragh.onion/2020/06/05/us/coronavirus-education-lost-learning.html?action=click\&pgtype=Article\&state=default\&region=MAIN_CONTENT_3\&context=storylines_faq}{online
    learning},
    \href{https://www.nytimes3xbfgragh.onion/2020/05/29/us/coronavirus-child-care-centers.html?action=click\&pgtype=Article\&state=default\&region=MAIN_CONTENT_3\&context=storylines_faq}{makeshift
    child care} and
    \href{https://www.nytimes3xbfgragh.onion/2020/06/03/business/economy/coronavirus-working-women.html?action=click\&pgtype=Article\&state=default\&region=MAIN_CONTENT_3\&context=storylines_faq}{stunted
    workdays} to continue. California's two largest public school
    districts --- Los Angeles and San Diego --- said on July 13, that
    \href{https://www.nytimes3xbfgragh.onion/2020/07/13/us/lausd-san-diego-school-reopening.html?action=click\&pgtype=Article\&state=default\&region=MAIN_CONTENT_3\&context=storylines_faq}{instruction
    will be remote-only in the fall}, citing concerns that surging
    coronavirus infections in their areas pose too dire a risk for
    students and teachers. Together, the two districts enroll some
    825,000 students. They are the largest in the country so far to
    abandon plans for even a partial physical return to classrooms when
    they reopen in August. For other districts, the solution won't be an
    all-or-nothing approach.
    \href{https://bioethics.jhu.edu/research-and-outreach/projects/eschool-initiative/school-policy-tracker/}{Many
    systems}, including the nation's largest, New York City, are
    devising
    \href{https://www.nytimes3xbfgragh.onion/2020/06/26/us/coronavirus-schools-reopen-fall.html?action=click\&pgtype=Article\&state=default\&region=MAIN_CONTENT_3\&context=storylines_faq}{hybrid
    plans} that involve spending some days in classrooms and other days
    online. There's no national policy on this yet, so check with your
    municipal school system regularly to see what is happening in your
    community.
  \end{itemize}
\end{itemize}

The median age now is 42.

At the peak of the crisis, the United States saw a point at which 400
young people were dying each week. The most detailed federal data on
coronavirus deaths lags behind state and local case reports by several
weeks, and does not fully reflect the current surge in deaths. Still,
the figures show that nationwide, the toll is approaching 4,000.

What's happening in Florida played out earlier in the Northeast. While
much attention was given to tragedies at nursing homes and assisted
living facilities, nearly 700 younger people died from the coronavirus
in New York City alone. In New Jersey, 370 people in the 25 to 44 age
group died.

The difference in Florida is that the increase in deaths among young
people is occurring now.

\hypertarget{coronavirus-deaths-by-month-and-age-group}{%
\subsection{Coronavirus Deaths by Month and Age
Group}\label{coronavirus-deaths-by-month-and-age-group}}

Florida residents over 65 make up the vast majority of coronavirus
deaths, but there has been substantial growth in deaths among younger
age groups. Gov. Ron DeSantis often notes that deaths are more frequent
for residents over 90 than for those under 65, but this was not true for
deaths recorded in July.

Under 65

65 to 89

90+

1,500

1,000

500

April

May

June

July

Under 65

65 to 89

90+

1,500

1,000

500

April

May

June

July

Under 65

65 to 89

90+

1,500

1,000

500

April

May

June

July

Source: Florida Department of Health

By Jugal K. Patel

``We're seeing more deaths among younger adults,'' said Dr. Russell
Vega, the chief medical examiner who oversees the counties of Sarasota,
Manatee and Desoto.

Considering that young people make up more than half of new reported
infections, some believe the overall risk to the young is negligible.

Still, health experts continue to urge younger people to take caution.
``This is not to be taken lightly,'' said Chighaf Bakour, an assistant
professor of epidemiology at the University of South Florida. ``These
are young people --- they are not supposed to be dying at this age.''

Ms. Bakour pointed out that tracking deaths only tells a portion of the
story since coronavirus infections can result in poor outcomes --- like
chronic breathing problems and other long-term health complications ---
without resulting in death.

Cindy Prins, an associate professor of epidemiology at the University of
Florida, said the public should expect more deaths among young
Americans.

``We've had this notion in people's heads that it's OK because young
people don't get sick from this virus and young people certainly don't
die from it,'' Ms. Prins said. ``Well, that isn't true. Young people are
getting sick. Young people are dying.''

Several Florida-based epidemiologists said that coronavirus deaths among
young people in the state may be helping to expose long-existing social
and health inequities between Black and white people. African-Americans,
they said, were more likely to have co-morbidities, like diabetes or
obesity, even at a young age, which could make them more vulnerable to
the virus and put them at a higher risk of death.

Infections can be transmitted among ethnic minorities and economically
disadvantaged people because they don't have the luxury to socially
distance in their homes, which are often tight quarters, they said. Many
of them also work essential jobs that already put them at risk, and have
to use public transportation to get to in-person work.

``You have this double impact of minorities being more likely to
contract Covid and then more likely to die because of the chronic
conditions,'' Ms. Elfassy said.

Imelda Bernardo believed that her son, Alexander G. Bacani Bernardo,
would recover from the coronavirus. After all, he was only 22, and
didn't suffer from underlying illnesses. He weighed about 300 pounds,
but he was tall and muscular, she said.

``Even though he's kind of big, he's fine. He's 22, he's good, he was
fine,'' Mrs. Bernardo, 49, said.

Mother and son, immigrants from the Philippines, both worked in a
nursing home in Jacksonville --- Mrs. Bernardo as a dietary cook and he
as a dietary aide. While he was sick at home with his family, Mr.
Bernardo is thought to have infected his father, Alvin, 49, who was a
smoker with diabetes and hypertension, according to state health
records.

The son died on July 17. His father, two days later.

Advertisement

\protect\hyperlink{after-bottom}{Continue reading the main story}

\hypertarget{site-index}{%
\subsection{Site Index}\label{site-index}}

\hypertarget{site-information-navigation}{%
\subsection{Site Information
Navigation}\label{site-information-navigation}}

\begin{itemize}
\tightlist
\item
  \href{https://help.nytimes3xbfgragh.onion/hc/en-us/articles/115014792127-Copyright-notice}{©~2020~The
  New York Times Company}
\end{itemize}

\begin{itemize}
\tightlist
\item
  \href{https://www.nytco.com/}{NYTCo}
\item
  \href{https://help.nytimes3xbfgragh.onion/hc/en-us/articles/115015385887-Contact-Us}{Contact
  Us}
\item
  \href{https://www.nytco.com/careers/}{Work with us}
\item
  \href{https://nytmediakit.com/}{Advertise}
\item
  \href{http://www.tbrandstudio.com/}{T Brand Studio}
\item
  \href{https://www.nytimes3xbfgragh.onion/privacy/cookie-policy\#how-do-i-manage-trackers}{Your
  Ad Choices}
\item
  \href{https://www.nytimes3xbfgragh.onion/privacy}{Privacy}
\item
  \href{https://help.nytimes3xbfgragh.onion/hc/en-us/articles/115014893428-Terms-of-service}{Terms
  of Service}
\item
  \href{https://help.nytimes3xbfgragh.onion/hc/en-us/articles/115014893968-Terms-of-sale}{Terms
  of Sale}
\item
  \href{https://spiderbites.nytimes3xbfgragh.onion}{Site Map}
\item
  \href{https://help.nytimes3xbfgragh.onion/hc/en-us}{Help}
\item
  \href{https://www.nytimes3xbfgragh.onion/subscription?campaignId=37WXW}{Subscriptions}
\end{itemize}
