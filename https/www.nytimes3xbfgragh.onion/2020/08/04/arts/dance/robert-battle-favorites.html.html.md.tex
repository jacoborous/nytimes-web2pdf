Sections

SEARCH

\protect\hyperlink{site-content}{Skip to
content}\protect\hyperlink{site-index}{Skip to site index}

\href{https://www.nytimes3xbfgragh.onion/section/arts/dance}{Dance}

\href{https://myaccount.nytimes3xbfgragh.onion/auth/login?response_type=cookie\&client_id=vi}{}

\href{https://www.nytimes3xbfgragh.onion/section/todayspaper}{Today's
Paper}

\href{/section/arts/dance}{Dance}\textbar{}Robert Battle Keeps Going
With Sarah Vaughan and Homegrown Tomatoes

\url{https://nyti.ms/33q3HYj}

\begin{itemize}
\item
\item
\item
\item
\item
\end{itemize}

\href{https://www.nytimes3xbfgragh.onion/spotlight/at-home?action=click\&pgtype=Article\&state=default\&region=TOP_BANNER\&context=at_home_menu}{At
Home}

\begin{itemize}
\tightlist
\item
  \href{https://www.nytimes3xbfgragh.onion/2020/08/03/well/family/the-benefits-of-talking-to-strangers.html?action=click\&pgtype=Article\&state=default\&region=TOP_BANNER\&context=at_home_menu}{Talk:
  To Strangers}
\item
  \href{https://www.nytimes3xbfgragh.onion/2020/08/01/at-home/coronavirus-make-pizza-on-a-grill.html?action=click\&pgtype=Article\&state=default\&region=TOP_BANNER\&context=at_home_menu}{Make:
  Grilled Pizza}
\item
  \href{https://www.nytimes3xbfgragh.onion/2020/07/31/arts/television/goldbergs-abc-stream.html?action=click\&pgtype=Article\&state=default\&region=TOP_BANNER\&context=at_home_menu}{Watch:
  'The Goldbergs'}
\item
  \href{https://www.nytimes3xbfgragh.onion/interactive/2020/at-home/even-more-reporters-editors-diaries-lists-recommendations.html?action=click\&pgtype=Article\&state=default\&region=TOP_BANNER\&context=at_home_menu}{Explore:
  Reporters' Google Docs}
\end{itemize}

Advertisement

\protect\hyperlink{after-top}{Continue reading the main story}

Supported by

\protect\hyperlink{after-sponsor}{Continue reading the main story}

My TEN

\hypertarget{robert-battle-keeps-going-with-sarah-vaughan-and-homegrown-tomatoes}{%
\section{Robert Battle Keeps Going With Sarah Vaughan and Homegrown
Tomatoes}\label{robert-battle-keeps-going-with-sarah-vaughan-and-homegrown-tomatoes}}

The Alvin Ailey artistic director is hunkered down in Connecticut,
tending his vegetable garden and keeping his mood aloft with these 10
essentials.

\includegraphics{https://static01.graylady3jvrrxbe.onion/images/2020/08/09/arts/09my10-battle-web/09my10-battle-web-articleLarge.jpg?quality=75\&auto=webp\&disable=upscale}

By \href{https://www.nytimes3xbfgragh.onion/by/kathryn-shattuck}{Kathryn
Shattuck}

\begin{itemize}
\item
  Aug. 4, 2020, 10:00 a.m. ET
\item
  \begin{itemize}
  \item
  \item
  \item
  \item
  \item
  \end{itemize}
\end{itemize}

``I always say that pessimism and anger is a place that I visit, but my
permanent address is optimism and hope,'' said
\href{https://www.alvinailey.org/alvin-ailey-american-dance-theater/robert-battle}{Robert
Battle}, the artistic director of
\href{https://www.alvinailey.org/}{Alvin Ailey American Dance Theater}.
And this year he's done plenty of traveling back and forth.

Since the pandemic lockdown in March, Battle has been consumed with
keeping the company in shape until its dancers can safely return to the
stage. From Aug. 6-12, a collaboration among Battle, his predecessor,
Judith Jamison, and the choreographer Rennie Harris will stream on
\href{https://www.alvinailey.org/performances-tickets/ailey-all-access}{Ailey
All Access}.

Battle has recently been considering the organization's role in the
Black Lives Matter movement. ``I've been thinking a lot about the notion
of, before it was a hashtag or a movement, that the Ailey company was
demonstrating that Black lives matter in all of the work that we do,''
he said. ``But it's almost not enough to live it. You have to say it
expressly, that this is what we do and we are in solidarity. It's not
that we need to reinvent the wheel, but we need to roll it.''

These days Battle is hunkered down in Connecticut, tending his vegetable
garden alongside his dog, North, with the occasional jaunt into the city
to, say, drop off produce for Jamison. ``I might open a stand if things
don't return to normal,'' he said, laughing, before elaborating on the
10 people and pastimes that have kept his mood aloft.

These are edited excerpts from the conversation.

\textbf{1. Sarah Vaughan's ``Sassy Swings the Tivoli'' (1963)}

My mother and a friend of hers, they would listen to Billy Eckstine and
Ella Fitzgerald, but mostly Sarah Vaughan. When I was a kid, they would
sit on the front porch, have a glass of wine, and that was their way of
winding down. And as somebody who was interested in music, I always
wanted to know, who was Sarah Vaughan? They would explain to me how she
was one of the greatest singers in the world. My mom's friend bought me
a cassette tape of her singing, and I played it until it snapped.

That passion for Sarah Vaughan has never left me, and eventually, when I
was a student at Juilliard, I got this double CD from Tower Records. It
was live at the Tivoli and it was recorded on the outside stage. I
listened to it a lot --- to how beautiful her voice was, to that
rhythmic clapping that audiences do if they really love it. Years and
years later, I'm the third artistic director of the company. And there
we are at the Tivoli, and I was staying in the hotel and I could see out
of my window that particular stage where she had sung. And there was
something full circle about that moment that meant so much to me.

Nothing pumps me up like the sound of Sarah Vaughan's voice. I'm
listening to ``Misty'' while I'm running on the treadmill.

\textbf{2. His Piano}

My mother played piano for the church that I grew up going to, and there
was a Kimball piano at home. They discovered that I had an ear for music
and so they got me piano lessons. I studied until I got involved in
dance, at about 11 or 12, and then it kind of disappeared on me. But
I've always needed to have a piano around, even if I just play the same
songs that I already know.

My piano teacher would come to my house, and when I went to Juilliard, I
would see her when I came home. This particular time she was dying of
cancer, but she wanted me to take her to a store like J.C. Penney. She
had me try on five suits, even in her sickness, and she bought all five
suits. And when I went home, my mother called her and said, ``This is so
generous, but why did you buy him all of these suits?'' And she said to
my mom, ``You know, he's going to be meeting kings and queens and
presidents, and he's going to need a suit.'' Well, many years later the
first Black president, President Obama, posthumously gives the
Presidential Medal of Freedom
\href{https://www.youtube.com/watch?v=Hvr0ZYYLmuw\&list=ULi9SliUpD0og\&index=60\&app=desktop}{to
Alvin Ailey}, and they called me to come and
\href{https://www.youtube.com/watch?v=-_EaZCqnbWc}{receive it on his
behalf}. And all I could think of was that piano teacher saying, ``He's
going to be meeting presidents, and he's going to need a suit.''

\textbf{3. Cherished Family Photos}

My grandfather raised me since I was three weeks old, and I think that's
where my sense of strength and duty and perseverance comes from. He only
made it to the third grade because his parents died and he had to raise
his siblings. My mother inspires me because of her artistic
inclinations. She nurtured that performer in me. And although I was
being bullied in my neighborhood, Liberty City {[}in Miami{]}, I had a
whole different message at home --- that being an artist was almost kind
of normal. And of course Alvin Ailey, so that I'm always reminded of the
shoulders on which I stand.

\textbf{4. Tabitha Brown's Videos}

\href{https://www.youtube.com/watch?v=tM0NAZ2ruCA}{Tabitha Brown} I
found because she was going on this journey of becoming a vegan. It
started because she had a meatless version of a B.L.T. that she'd gotten
from Whole Foods, and she sat in her car and was having a
\href{https://www.youtube.com/watch?v=AP1mnFJG0-s}{spiritual moment},
and I thought it was hilarious. Coming from where I come from, I didn't
know a lot of African-American people that were vegetarian or vegan.
Sometimes the way she talks about it, I'm that close to trying. And then
the meat part of me gets the best of me. Because I love ribs and steak
and it's just --- I'm sorry. I can't.

\textbf{5. Maya Angelou}

I hardly get through a speech or an interview without saying some quote
that I've gotten from
\href{https://www.nytimes3xbfgragh.onion/2014/05/29/arts/maya-angelou-lyrical-witness-of-the-jim-crow-south-dies-at-86.html}{Maya
Angelou}. She is in some ways a spiritual walker. Her life, you know
from
\href{https://www.nytimes3xbfgragh.onion/2014/05/29/arts/her-face-was-a-brown-moon-that-shone-on-me.html}{``I
Know Why the Caged Bird Sings,''} it's really a life well lived and it
wasn't perfect. And she wasn't afraid to express those things that were
difficult for her. So I connected with the poetry. She did, for me, act
as a kind of guide without her even knowing it.

I went to a book signing for
\href{https://www.penguinrandomhouse.com/books/3916/even-the-stars-look-lonesome-by-maya-angelou/}{``Even
the Stars Look Lonesome,''} and I waited three hours. I had all of these
plans that when I got there, I was going to say a quote from her
\href{https://www.nytimes3xbfgragh.onion/video/books/100000002906088/maya-angelou-at-the-clinton-inauguration.html}{inaugural
poem} for President Clinton. I had this arsenal of stuff. But by the
time she looked up at my face, I had nothing. I was in such awe. I sort
of sheepishly bowed my head and handed her the book, and she signed it
and said, ``Thank you.'' And then I moved on.

\textbf{6. Trying New Recipes}

Cooking, to me, it's almost like making a dance, except nobody complains
when you say, ``Slam yourself to the floor.'' The notion of starting
with these few ingredients, or sometimes a lot of ingredients, and
slowly developing the flavor --- there's just something about the
practice that really excites me and relaxes me and gives me some sense
of control. I can't change the pandemic, but I can certainly make a mean
fried chicken with almond flour.

\textbf{7. Dancers Connecting}

When this whole thing went down and we came off the road, Miranda Quinn,
who was a new dancer, had the idea of doing a ``Brady Bunch'' version of
the first part of
\href{https://www.youtube.com/watch?v=gEFW5JznwOY}{``Revelations.''} The
dancers in their different homes --- you could see the dogs running past
--- they made it very real. And it caught fire on social media, which
led to us codifying it into something called
\href{https://www.alvinailey.org/performances-tickets/ailey-all-access}{Ailey
All Access}.

They also did
\href{https://www.youtube.com/watch?v=bOio57DuiUM\&list=PLSYS-T-WpOEUSAdXYrZw-zaZam_UMxJ7r}{DancerDiaries},
where dancers would talk about how they were feeling in this moment, and
physicalize it and verbalize it in a way that was really touching and
beautiful. Their need to connect with audiences no matter what was
really inspiring. It solidified my own ability to see that they didn't
just want to dance, but that that was the only way that they could
express their essential selves.

\textbf{8. Home Gym}

Luckily, I did that right before all of this hit. I had a small room
that I used for storage, and I decided, let me make this into a gym. Now
I have this space that has a treadmill and an elliptical and a weight
machine. And it keeps me sane because I no longer dance, but we still
need to get moving and get that energy out as most dancers will testify
to. So it's been a nice little respite, and it's hard to make excuses
when it's literally two steps from my bedroom. But I still find a way to
make excuses.

\textbf{9.}
\textbf{\href{https://twitter.com/RobinRoberts?ref_src=twsrc\%5Egoogle\%7Ctwcamp\%5Eserp\%7Ctwgr\%5Eauthor}{Robin
Roberts}} \textbf{on ``Good Morning America''}

She's such a fan of the company, and I just love her indomitable spirit.
I've watched her for years through some of the tough times in her life.
People like that have so much to teach us about grace under fire and
about courage being \emph{not} the absence of fear but the presence of
it, and the desire to go forward anyway.

\textbf{10. Backyard Time}

If you had asked somebody who knew me years ago and you'd said, ``Oh
yeah, he has a vegetable garden and a dog,'' they would probably have
said, ``You have the wrong person. No way.'' But being out here in
nature sort of changed my feelings around watching something grow.
Again, it's those little things that you can control, that come to life,
watching a tomato plant go from this little nothing and struggle up and
then bear this wonderfully ripe fruit. It's what creativity is all
about.

And dogs I just love because no matter what, they're always happy to see
you. It doesn't hurt if you have a rib in your mouth. Then they're
doubly happy to see you.

I've been thinking about Toni Morrison, and one of the things she said
was that we think sometimes showing love to our children is seeing
everything that's wrong when they walk into a room. Oh, fix this, do
that. And she said, ``Sometimes they want to see, Do your eyes light up
when I walk into a room?'' And so in some ways, even if I'm dealing with
very difficult stuff, when I see the dog, he just wants me to say ``good
boy'' and pet him, you know? And what pulls you out of some of the
things that threaten to keep you locked into sadness and being
discouraged is he's just happy to see you. Can't you be happy to see
him?

Advertisement

\protect\hyperlink{after-bottom}{Continue reading the main story}

\hypertarget{site-index}{%
\subsection{Site Index}\label{site-index}}

\hypertarget{site-information-navigation}{%
\subsection{Site Information
Navigation}\label{site-information-navigation}}

\begin{itemize}
\tightlist
\item
  \href{https://help.nytimes3xbfgragh.onion/hc/en-us/articles/115014792127-Copyright-notice}{©~2020~The
  New York Times Company}
\end{itemize}

\begin{itemize}
\tightlist
\item
  \href{https://www.nytco.com/}{NYTCo}
\item
  \href{https://help.nytimes3xbfgragh.onion/hc/en-us/articles/115015385887-Contact-Us}{Contact
  Us}
\item
  \href{https://www.nytco.com/careers/}{Work with us}
\item
  \href{https://nytmediakit.com/}{Advertise}
\item
  \href{http://www.tbrandstudio.com/}{T Brand Studio}
\item
  \href{https://www.nytimes3xbfgragh.onion/privacy/cookie-policy\#how-do-i-manage-trackers}{Your
  Ad Choices}
\item
  \href{https://www.nytimes3xbfgragh.onion/privacy}{Privacy}
\item
  \href{https://help.nytimes3xbfgragh.onion/hc/en-us/articles/115014893428-Terms-of-service}{Terms
  of Service}
\item
  \href{https://help.nytimes3xbfgragh.onion/hc/en-us/articles/115014893968-Terms-of-sale}{Terms
  of Sale}
\item
  \href{https://spiderbites.nytimes3xbfgragh.onion}{Site Map}
\item
  \href{https://help.nytimes3xbfgragh.onion/hc/en-us}{Help}
\item
  \href{https://www.nytimes3xbfgragh.onion/subscription?campaignId=37WXW}{Subscriptions}
\end{itemize}
