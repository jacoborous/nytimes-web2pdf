Sections

SEARCH

\protect\hyperlink{site-content}{Skip to
content}\protect\hyperlink{site-index}{Skip to site index}

\href{https://www.nytimes3xbfgragh.onion/section/books/review}{Book
Review}

\href{https://myaccount.nytimes3xbfgragh.onion/auth/login?response_type=cookie\&client_id=vi}{}

\href{https://www.nytimes3xbfgragh.onion/section/todayspaper}{Today's
Paper}

\href{/section/books/review}{Book Review}\textbar{}A History of
Hurricanes in America and the Devastation They Have Wrought

\url{https://nyti.ms/3i6gfIl}

\begin{itemize}
\item
\item
\item
\item
\item
\item
\end{itemize}

Advertisement

\protect\hyperlink{after-top}{Continue reading the main story}

Supported by

\protect\hyperlink{after-sponsor}{Continue reading the main story}

Nonfiction

\hypertarget{a-history-of-hurricanes-in-america-and-the-devastation-they-have-wrought}{%
\section{A History of Hurricanes in America and the Devastation They
Have
Wrought}\label{a-history-of-hurricanes-in-america-and-the-devastation-they-have-wrought}}

\includegraphics{https://static01.graylady3jvrrxbe.onion/images/2020/08/16/books/review/Kolbert/Kolbert-articleLarge.jpg?quality=75\&auto=webp\&disable=upscale}

Buy Book ▾

\begin{itemize}
\tightlist
\item
  \href{https://www.amazon.com/gp/search?index=books\&tag=NYTBSREV-20\&field-keywords=A+Furious+Sky+Eric+Jay+Dolan}{Amazon}
\item
  \href{https://du-gae-books-dot-nyt-du-prd.appspot.com/buy?title=A+Furious+Sky\&author=Eric+Jay+Dolan}{Apple
  Books}
\item
  \href{https://www.anrdoezrs.net/click-7990613-11819508?url=https\%3A\%2F\%2Fwww.barnesandnoble.com\%2Fw\%2F\%3Fean\%3D1631495275}{Barnes
  and Noble}
\item
  \href{https://www.anrdoezrs.net/click-7990613-35140?url=https\%3A\%2F\%2Fwww.booksamillion.com\%2Fp\%2FA\%2BFurious\%2BSky\%2FEric\%2BJay\%2BDolan\%2F1631495275}{Books-A-Million}
\item
  \href{https://bookshop.org/a/3546/1631495275}{Bookshop}
\item
  \href{https://www.indiebound.org/book/1631495275?aff=NYT}{Indiebound}
\end{itemize}

When you purchase an independently reviewed book through our site, we
earn an affiliate commission.

By Elizabeth Kolbert

\begin{itemize}
\item
  Aug. 4, 2020, 5:00 a.m. ET
\item
  \begin{itemize}
  \item
  \item
  \item
  \item
  \item
  \item
  \end{itemize}
\end{itemize}

\textbf{A FURIOUS SKY}\\
\textbf{The Five-Hundred-Year History of America's Hurricanes}\\
By Eric Jay Dolin

When the first Europeans arrived in the Americas, they had no word for
``hurricane.'' In the Atlantic, hurricanes form off the coast of Africa
and travel west, so they'd had no need for one. But the want quickly
asserted itself. In 1502, a hurricane off Santo Domingo sank 24 Spanish
ships, killing almost everyone on board, including Francisco de
Bobadilla, who'd been dispatched to replace Christopher Columbus as
governor of Hispaniola. It's believed that ``hurricane'' is derived from
the Arawak word \emph{hurakan,} meaning ``god of the storm.''

In ``A Furious Sky,'' his lively chronicle of five tempestuous
centuries, Eric Jay Dolin describes the 1502 \emph{hurakan} as ``a most
appropriate prologue to European settlement of the New World.'' Hopes of
the French to colonize what's now Florida were dashed by a hurricane in
1565. Another hurricane, in 1609, delayed aid to the British settlers in
Jamestown, who, by that point, had probably resorted to cannibalism, and
a third, in 1635, spared the life of the Puritan minister Richard
Mather, whose son, Increase Mather, later wrote of his father's
experience that ``God turned the wind about.'' The course of the
American Revolution may have been altered by a pair of hurricanes that
slammed into the Caribbean in 1780. (According to this theory, the
storms prompted the French to send ships they had docked in the
Caribbean north, aiding the Revolutionaries.)

\includegraphics{https://static01.graylady3jvrrxbe.onion/images/2020/07/31/books/review/Kolbert-02/Kolbert-02-articleLarge.jpg?quality=75\&auto=webp\&disable=upscale}

Of course, hurricanes didn't let up once the new nation was founded, and
Dolin pursues them as they churn their way through the first part of the
19th century, wreaking havoc from New Orleans to Newport. In September
1821, a powerful storm tore through New England. The following month, a
meteorologically inclined merchant named William Redfield was traveling
by carriage from his home in the town of Cromwell, in central
Connecticut, to his in-laws, in Stockbridge, Mass. Along the way, he
noticed something odd. The downed trees in his neighborhood were lying
with their tops pointing northwest, while near Stockbridge, they faced
southeast. Pondering this strange fact, Redfield concluded that
hurricane winds blow in a great, revolving circle. From this insight was
born what became known as the ``American Storm Controversy.''

James P. Espy, a classics scholar from Philadelphia who also took a keen
interest in meteorology, opposed Redfield's theory. Espy argued that
hurricanes form when warm, moist air rises from the surface of the
ocean, releasing latent heat, and that, rather than rotating, hurricane
winds rush from the edge of the storm toward the center. The two men
fought for decades, each refusing to concede anything to the other, even
though, as Dolin points out, both were partly correct.

Image

Credit...Liveright

As the century wore on, one scientific breakthrough followed another.
But little progress was made in the science of hurricane prediction, and
even with the advent of the telegraph and the creation of the United
States Weather Bureau, forecasting remained largely a matter of luck.

For instance, the Weather Bureau called for ``light rain'' in the New
York area on the evening of Aug. 23, 1893; instead the city was hit by a
Category 1 storm. Four days later, the bureau warned that a hurricane
was about to make landfall near Savannah, but by that point, the
Category 3 hurricane was already bearing down on the city and even those
who received the alert had no time to prepare. Particularly devastated
were the islands to the northeast of Savannah, off the coast of South
Carolina, which were on the ``dirty,'' or right-hand, side of the storm.
(In the Northern Hemisphere, hurricanes rotate in a counterclockwise
direction; this means that whatever direction the storm is moving, the
danger will be greatest to the right of the eye, because there the speed
of the winds is increased by the forward velocity of the tempest.) Some
2,000 people were killed in the so-called Sea Islands Hurricane, most of
them African-American. Relief was slow to reach the islands, in part
because of all the damage and in part because South Carolina's avowedly
white supremacist governor delayed calling for help.

Four decades later, communications technologies --- radio, telephone ---
had vastly improved, but, as Dolin recounts, the bureau's forecasts had
not. In the hours before the Labor Day Hurricane of 1935 made landfall,
the bureau missed the fact that the storm had taken a sharp northern
turn; the result was that hundreds of unemployed World War I veterans
who'd been sent to build a road connecting the Florida Keys were killed.
Ernest Hemingway, who rode out the Labor Day Hurricane on Key West,
visited the vets' camp, on Lower Matecumbe Key, a few days after the
disaster. (He'd been drinking buddies with many of the men.) The dead,
he reported, were ``everywhere and in the sun all of them were beginning
to be too big for their bluejeans and jackets that they could never fill
when they were on the bum and hungry.''

Satellites now allow hurricanes to be monitored from their inception,
and computer models take vast amounts of data and spit out predictions.
But as Dolin notes, forecasting hurricanes remains a ``tricky
endeavor.'' Hurricanes are susceptible to the ``butterfly effect'' ---
small changes in the initial conditions ramify into very large changes
later on. Meteorologists try to deal with this problem by running their
computer models many times over, starting with different initial
conditions, but they can never overcome what's known as the ``limit of
predictability.'' Thus, hurricane forecasts will always come with a
range of uncertainty.

At the start of ``A Furious Sky,'' Dolin, who has written several
previous books on maritime topics, writes that ``hurricanes have left an
indelible mark on American history.'' He suggests that it's particularly
important to attend to this mark now because climate change is only
going to make storms ``more powerful and more destructive in the
future.'' But he never develops a clear argument as to what the societal
impact of hurricanes has been (besides a lot of devastation and death),
or what we can expect it to be going forward (aside from more of the
same).

Where ``A Furious Sky'' is most compelling is in its often harrowing
details. It's filled with haunting personal stories. Consider that of
Joseph Matoes Sr., a dairy farmer in coastal Rhode Island. On the
afternoon of Sept. 21, 1938, Matoes watched a school bus carrying four
of his five children head toward a causeway that faced a normally placid
cove. The deadliest hurricane in modern New England history was bearing
down on Rhode Island, and the water in the cove was roiling and crashing
over the road. Matoes waved to the bus driver to try to get him to stop;
instead he accelerated. Halfway across the causeway, the bus stalled.
The driver helped the kids out and told them all to hold hands. A wave
wiped them off their feet. Matoes could only watch as his four children
drowned. The bus driver survived, but told Matoes he wished that he
hadn't.

``Everything's gone,'' he said.

Advertisement

\protect\hyperlink{after-bottom}{Continue reading the main story}

\hypertarget{site-index}{%
\subsection{Site Index}\label{site-index}}

\hypertarget{site-information-navigation}{%
\subsection{Site Information
Navigation}\label{site-information-navigation}}

\begin{itemize}
\tightlist
\item
  \href{https://help.nytimes3xbfgragh.onion/hc/en-us/articles/115014792127-Copyright-notice}{©~2020~The
  New York Times Company}
\end{itemize}

\begin{itemize}
\tightlist
\item
  \href{https://www.nytco.com/}{NYTCo}
\item
  \href{https://help.nytimes3xbfgragh.onion/hc/en-us/articles/115015385887-Contact-Us}{Contact
  Us}
\item
  \href{https://www.nytco.com/careers/}{Work with us}
\item
  \href{https://nytmediakit.com/}{Advertise}
\item
  \href{http://www.tbrandstudio.com/}{T Brand Studio}
\item
  \href{https://www.nytimes3xbfgragh.onion/privacy/cookie-policy\#how-do-i-manage-trackers}{Your
  Ad Choices}
\item
  \href{https://www.nytimes3xbfgragh.onion/privacy}{Privacy}
\item
  \href{https://help.nytimes3xbfgragh.onion/hc/en-us/articles/115014893428-Terms-of-service}{Terms
  of Service}
\item
  \href{https://help.nytimes3xbfgragh.onion/hc/en-us/articles/115014893968-Terms-of-sale}{Terms
  of Sale}
\item
  \href{https://spiderbites.nytimes3xbfgragh.onion}{Site Map}
\item
  \href{https://help.nytimes3xbfgragh.onion/hc/en-us}{Help}
\item
  \href{https://www.nytimes3xbfgragh.onion/subscription?campaignId=37WXW}{Subscriptions}
\end{itemize}
