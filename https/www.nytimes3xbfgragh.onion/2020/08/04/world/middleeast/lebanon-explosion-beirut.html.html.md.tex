Sections

SEARCH

\protect\hyperlink{site-content}{Skip to
content}\protect\hyperlink{site-index}{Skip to site index}

\href{https://www.nytimes3xbfgragh.onion/section/world/middleeast}{Middle
East}

\href{https://myaccount.nytimes3xbfgragh.onion/auth/login?response_type=cookie\&client_id=vi}{}

\href{https://www.nytimes3xbfgragh.onion/section/todayspaper}{Today's
Paper}

\href{/section/world/middleeast}{Middle East}\textbar{}I Was Bloodied
and Dazed. Beirut Strangers Treated Me Like a Friend.

\url{https://nyti.ms/31i3a81}

\begin{itemize}
\item
\item
\item
\item
\item
\item
\end{itemize}

Advertisement

\protect\hyperlink{after-top}{Continue reading the main story}

Supported by

\protect\hyperlink{after-sponsor}{Continue reading the main story}

\hypertarget{i-was-bloodied-and-dazed-beirut-strangers-treated-me-like-a-friend}{%
\section{I Was Bloodied and Dazed. Beirut Strangers Treated Me Like a
Friend.}\label{i-was-bloodied-and-dazed-beirut-strangers-treated-me-like-a-friend}}

In a land conditioned by calamity, people knew what to do, including
helping wounded people they didn't know.

\href{https://www.nytimes3xbfgragh.onion/by/vivian-yee}{\includegraphics{https://static01.graylady3jvrrxbe.onion/images/2018/02/20/multimedia/author-vivian-yee/author-vivian-yee-thumbLarge-v2.png}}

By \href{https://www.nytimes3xbfgragh.onion/by/vivian-yee}{Vivian Yee}

\begin{itemize}
\item
  Published Aug. 4, 2020Updated Aug. 5, 2020, 3:44 a.m. ET
\item
  \begin{itemize}
  \item
  \item
  \item
  \item
  \item
  \item
  \end{itemize}
\end{itemize}

\href{https://cn.nytimes3xbfgragh.onion/world/20200805/beirut-explosion-first-person/}{阅读简体中文版}\href{https://cn.nytimes3xbfgragh.onion/world/20200805/beirut-explosion-first-person/zh-hant/}{閱讀繁體中文版}

\includegraphics{https://static01.graylady3jvrrxbe.onion/images/2020/08/04/world/04beirut-first-account/merlin_175303311_606f30a4-a476-48ec-9327-c93d6f96b2d5-articleLarge.jpg?quality=75\&auto=webp\&disable=upscale}

BEIRUT --- I was just about to look at a video a friend had sent me on
Tuesday afternoon --- ``the port seems to be burning,'' she said ---
when my whole building shook, as if startled, by the deepest boom I'd
ever heard. Uneasily, naïvely, I ran to the window, then back to my desk
to check for news.

Then came a much bigger boom, and the sound itself seemed to splinter.
There was shattered glass flying everywhere. Not thinking but moving, I
ducked under my desk.

When the world stopped cracking open, I couldn't see at first because of
the blood running down my face. After blinking the blood from my eyes, I
tried to take in the sight of my apartment turned into a demolition
site. My yellow front door had been hurled on top of my dining table. I
couldn't find my passport, or even any sturdy shoes.

Later, someone would tell me that
\href{https://www.nytimes3xbfgragh.onion/2020/08/04/world/middleeast/lebanon-explosion-beirut.html}{Beirutis}
of her generation, who had been raised during Lebanon's 15-year civil
war, instinctively ran into their hallways as soon as they heard the
first blast, to escape the glass they knew would break.

I was not so well-trained, but the Lebanese who would help me in the
hours to come had the heartbreaking steadiness that comes from having
lived through countless previous disasters. Nearly all of them were
strangers, yet they treated me like a friend.

When I got downstairs, dodging the enormous broken window that rested
jaggedly in my stairwell, my neighborhood, with its graceful old-Beirut
architecture and arched windows, looked like a picture from the wars I
had seen from afar --- a mouth missing all its teeth.

Someone passing on a motorbike saw my bloody face and told me to hop on.
When we couldn't get any closer to the hospital, our way blocked by
hillocks of broken glass and stranded cars, I got off and started
walking.

Everyone on the street seemed to be either bleeding from open gashes or
swathed in makeshift bandages --- all except one woman in a chic,
backless top leading a small dog on a leash. Only an hour before, we had
all been walking dogs or checking email or shopping for groceries. Only
an hour before, there had been no blood.

As I neared the hospital, elderly patients sat dazed in wheelchairs in
the streets, still hooked to their IV bags. A woman lay on the ground in
front of the exploded emergency room, her whole body dripping red, not
moving much. It was clear that they weren't taking new patients,
certainly not any as comparatively lucky as I was.

Someone named Youssef saw me, sat me down and started cleaning and
bandaging my face. Once he was satisfied I could walk, he left and I
started wandering, trying to think of another hospital I could try.

I ran into a friend of a friend, someone I had met only a few times
before, and he bandaged the rest of my wounds, disinfecting the
lacerations with splashes of Lebanon's national liquor, an
anise-flavored drink called arak.

His roommate swept up their terrace as I bloodied their towels. ``I
can't think unless it's clean,'' he explained.

Until then, I hadn't had more than the vaguest guesses about what might
have happened. Someone was reporting that fireworks had exploded at the
port. Much later, Lebanese officials acknowledged that a large cache of
explosive material seized by the government years ago was stored where
the explosions occurred.

Survivors walked by, moving faster than the jammed-up traffic. To anyone
who appeared unhurt, people called out, ``alhamdulillah al-salama,'' or,
roughly translated, thank God for your safety.

Before the end of the night, after my co-workers had found me, after a
passing driver named Ralph had offered to take us to one of the few
hospitals still accepting patients, after a doctor had put 11 staples in
my forehead and another sprinkling on my leg and arms, people would be
saying the same thing to me: Thank God for your safety.

``Thank you,'' I said in reply, ``truly thank you,'' and I didn't mean
just for the good wishes.

Advertisement

\protect\hyperlink{after-bottom}{Continue reading the main story}

\hypertarget{site-index}{%
\subsection{Site Index}\label{site-index}}

\hypertarget{site-information-navigation}{%
\subsection{Site Information
Navigation}\label{site-information-navigation}}

\begin{itemize}
\tightlist
\item
  \href{https://help.nytimes3xbfgragh.onion/hc/en-us/articles/115014792127-Copyright-notice}{©~2020~The
  New York Times Company}
\end{itemize}

\begin{itemize}
\tightlist
\item
  \href{https://www.nytco.com/}{NYTCo}
\item
  \href{https://help.nytimes3xbfgragh.onion/hc/en-us/articles/115015385887-Contact-Us}{Contact
  Us}
\item
  \href{https://www.nytco.com/careers/}{Work with us}
\item
  \href{https://nytmediakit.com/}{Advertise}
\item
  \href{http://www.tbrandstudio.com/}{T Brand Studio}
\item
  \href{https://www.nytimes3xbfgragh.onion/privacy/cookie-policy\#how-do-i-manage-trackers}{Your
  Ad Choices}
\item
  \href{https://www.nytimes3xbfgragh.onion/privacy}{Privacy}
\item
  \href{https://help.nytimes3xbfgragh.onion/hc/en-us/articles/115014893428-Terms-of-service}{Terms
  of Service}
\item
  \href{https://help.nytimes3xbfgragh.onion/hc/en-us/articles/115014893968-Terms-of-sale}{Terms
  of Sale}
\item
  \href{https://spiderbites.nytimes3xbfgragh.onion}{Site Map}
\item
  \href{https://help.nytimes3xbfgragh.onion/hc/en-us}{Help}
\item
  \href{https://www.nytimes3xbfgragh.onion/subscription?campaignId=37WXW}{Subscriptions}
\end{itemize}
