Sections

SEARCH

\protect\hyperlink{site-content}{Skip to
content}\protect\hyperlink{site-index}{Skip to site index}

\href{https://www.nytimes3xbfgragh.onion/section/politics}{Politics}

\href{https://myaccount.nytimes3xbfgragh.onion/auth/login?response_type=cookie\&client_id=vi}{}

\href{https://www.nytimes3xbfgragh.onion/section/todayspaper}{Today's
Paper}

\href{/section/politics}{Politics}\textbar{}Trump Signs Landmark Land
Conservation Bill

\url{https://nyti.ms/39WazO5}

\begin{itemize}
\item
\item
\item
\item
\item
\end{itemize}

Advertisement

\protect\hyperlink{after-top}{Continue reading the main story}

Supported by

\protect\hyperlink{after-sponsor}{Continue reading the main story}

\hypertarget{trump-signs-landmark-land-conservation-bill}{%
\section{Trump Signs Landmark Land Conservation
Bill}\label{trump-signs-landmark-land-conservation-bill}}

The bipartisan Great American Outdoors Act guarantees funding for
federal land use efforts. The president claimed credit for Republicans.

\includegraphics{https://static01.graylady3jvrrxbe.onion/images/2020/08/04/us/politics/04dc-conserve/04dc-conserve-articleLarge.jpg?quality=75\&auto=webp\&disable=upscale}

\href{https://www.nytimes3xbfgragh.onion/by/annie-karni}{\includegraphics{https://static01.graylady3jvrrxbe.onion/images/2019/02/05/multimedia/author-annie-karni/author-annie-karni-thumbLarge.png}}

By \href{https://www.nytimes3xbfgragh.onion/by/annie-karni}{Annie Karni}

\begin{itemize}
\item
  Aug. 4, 2020
\item
  \begin{itemize}
  \item
  \item
  \item
  \item
  \item
  \end{itemize}
\end{itemize}

WASHINGTON --- President Trump on Tuesday signed into law the Great
American Outdoors Act, a measure with broad bipartisan support that
guarantees maximum annual funding for a federal program to acquire and
preserve land for public use.

Mr. Trump --- who has
\href{https://www.nytimes3xbfgragh.onion/2019/11/04/climate/trump-paris-agreement-climate.html}{exited}
the Paris Agreement on climate change,
\href{https://www.nytimes3xbfgragh.onion/2020/06/04/climate/trump-environment-coronavirus.html}{loosened
restrictions} on toxic air pollution and removed climate change from a
list of national security threats --- heralded the new law as a
groundbreaking environmental achievement that he deserved credit for.

``From an environmental standpoint and from just the beauty of our
country standpoint, there hasn't been anything like this since Teddy
Roosevelt, I suspect,'' he said at a bill-signing ceremony at the White
House.

``At some point, they'll have to start thinking about the Republican
Party and all of the incredible things we've done on conservation and
many other fronts,'' Mr. Trump said.

The act, which allocates \$900 million a year to the Land and Water
Conservation Fund and provides up to \$9.5 billion over five years to
begin clearing up a maintenance backlog at national parks, was approved
on a 310-to-107 vote in the House. It was introduced last year by
Representative John Lewis, the Georgia Democrat and civil rights leader
who
\href{https://www.nytimes3xbfgragh.onion/2020/07/17/us/john-lewis-dead.html}{passed
away} last month.

But no Democrats were invited to the signing ceremony, which was
attended by six Republican senators and three Republican congressmen, in
addition to senior administration officials. Mr. Trump did not mention
Mr. Lewis or any of his Democratic colleagues in his remarks.

When asked why Democrats were not invited or acknowledged, Kayleigh
McEnany, the White House press secretary, said that ``the only thing
we're recognizing about congressional Democrats right now is how
appalling it is that there are Americans going who are going without
paychecks because they refuse to partner with Martha McSally,
Republicans and the president to make sure those payments go out."

Ms. McEnany was referring to the
\href{https://www.nytimes3xbfgragh.onion/2020/08/04/us/politics/coronavirus-recovery-plan-negotiations.html}{stalemate
on Capitol Hill} between Republicans, including Ms. McSally of Arizona,
and Democrats as they negotiate another round of federal aid to address
the coronavirus pandemic.

Mr. Trump was
\href{https://www.nytimes3xbfgragh.onion/2020/07/22/us/politics/land-water-conservation-fund.html}{persuaded
to support the bill} by two Senate Republicans from Western states ---
Senator Cory Gardner of Colorado and Senator Steve Daines of Montana ---
who are facing tough re-election battles and saw the measure as helpful
to their states and their electoral chances. Mr. Daines and Mr. Gardner
met with the president last year and told him that signing the measure
would give him a significant conservation legacy.

``That was a meeting that took place, and within about a minute, I was
convinced,'' Mr. Trump said on Tuesday. ``And I wasn't at all convinced
before I walked in.''

Even as he tried to bill himself as an environmentalist with a legacy
that would rival Mr. Roosevelt's, the president also demonstrated a lack
of familiarity with one of the country's most famous national parks.

He
\href{https://www.nytimes3xbfgragh.onion/video/us/100000007272140/trump-stumbles-over-yosemite.html}{bungled
the pronunciation} of Yosemite National Park in California, referring to
it as ``yo Semites" as he read from his prepared remarks, creating an
instant viral moment that was mocked online by the Democratic National
Committee.

And on the same day that Mr. Trump signed the measure, his eldest son,
Donald Trump Jr., and Nick Ayers, a former chief of staff to Vice
President Mike Pence,
\href{https://twitter.com/DonaldJTrumpJr/status/1290723762523045888}{publicly
expressed their opposition} to the administration's longtime efforts to
open the Pebble Mine, a large gold and copper mine in Alaska.

Trump officials concluded last month that opening the mine would not
pose serious environmental risks, a reversal of the Obama
administration's position. Mr. Ayers and the president's son wrote on
Twitter that they wanted Mr. Trump to direct the Environmental
Protection Agency to block the Pebble Mine opening.

Advertisement

\protect\hyperlink{after-bottom}{Continue reading the main story}

\hypertarget{site-index}{%
\subsection{Site Index}\label{site-index}}

\hypertarget{site-information-navigation}{%
\subsection{Site Information
Navigation}\label{site-information-navigation}}

\begin{itemize}
\tightlist
\item
  \href{https://help.nytimes3xbfgragh.onion/hc/en-us/articles/115014792127-Copyright-notice}{©~2020~The
  New York Times Company}
\end{itemize}

\begin{itemize}
\tightlist
\item
  \href{https://www.nytco.com/}{NYTCo}
\item
  \href{https://help.nytimes3xbfgragh.onion/hc/en-us/articles/115015385887-Contact-Us}{Contact
  Us}
\item
  \href{https://www.nytco.com/careers/}{Work with us}
\item
  \href{https://nytmediakit.com/}{Advertise}
\item
  \href{http://www.tbrandstudio.com/}{T Brand Studio}
\item
  \href{https://www.nytimes3xbfgragh.onion/privacy/cookie-policy\#how-do-i-manage-trackers}{Your
  Ad Choices}
\item
  \href{https://www.nytimes3xbfgragh.onion/privacy}{Privacy}
\item
  \href{https://help.nytimes3xbfgragh.onion/hc/en-us/articles/115014893428-Terms-of-service}{Terms
  of Service}
\item
  \href{https://help.nytimes3xbfgragh.onion/hc/en-us/articles/115014893968-Terms-of-sale}{Terms
  of Sale}
\item
  \href{https://spiderbites.nytimes3xbfgragh.onion}{Site Map}
\item
  \href{https://help.nytimes3xbfgragh.onion/hc/en-us}{Help}
\item
  \href{https://www.nytimes3xbfgragh.onion/subscription?campaignId=37WXW}{Subscriptions}
\end{itemize}
