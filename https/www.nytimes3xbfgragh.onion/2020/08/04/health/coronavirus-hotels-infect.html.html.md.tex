Sections

SEARCH

\protect\hyperlink{site-content}{Skip to
content}\protect\hyperlink{site-index}{Skip to site index}

\href{https://www.nytimes3xbfgragh.onion/section/health}{Health}

\href{https://myaccount.nytimes3xbfgragh.onion/auth/login?response_type=cookie\&client_id=vi}{}

\href{https://www.nytimes3xbfgragh.onion/section/todayspaper}{Today's
Paper}

\href{/section/health}{Health}\textbar{}Forget Spas and Bars. Hotels
Tout Housekeeping to Lure Back Travelers.

\url{https://nyti.ms/3i6GfTL}

\begin{itemize}
\item
\item
\item
\item
\item
\end{itemize}

\href{https://www.nytimes3xbfgragh.onion/news-event/coronavirus?action=click\&pgtype=Article\&state=default\&region=TOP_BANNER\&context=storylines_menu}{The
Coronavirus Outbreak}

\begin{itemize}
\tightlist
\item
  live\href{https://www.nytimes3xbfgragh.onion/2020/08/04/world/coronavirus-cases.html?action=click\&pgtype=Article\&state=default\&region=TOP_BANNER\&context=storylines_menu}{Latest
  Updates}
\item
  \href{https://www.nytimes3xbfgragh.onion/interactive/2020/us/coronavirus-us-cases.html?action=click\&pgtype=Article\&state=default\&region=TOP_BANNER\&context=storylines_menu}{Maps
  and Cases}
\item
  \href{https://www.nytimes3xbfgragh.onion/interactive/2020/science/coronavirus-vaccine-tracker.html?action=click\&pgtype=Article\&state=default\&region=TOP_BANNER\&context=storylines_menu}{Vaccine
  Tracker}
\item
  \href{https://www.nytimes3xbfgragh.onion/2020/08/02/us/covid-college-reopening.html?action=click\&pgtype=Article\&state=default\&region=TOP_BANNER\&context=storylines_menu}{College
  Reopening}
\item
  \href{https://www.nytimes3xbfgragh.onion/live/2020/08/04/business/stock-market-today-coronavirus?action=click\&pgtype=Article\&state=default\&region=TOP_BANNER\&context=storylines_menu}{Economy}
\end{itemize}

Advertisement

\protect\hyperlink{after-top}{Continue reading the main story}

Supported by

\protect\hyperlink{after-sponsor}{Continue reading the main story}

\hypertarget{forget-spas-and-bars-hotels-tout-housekeeping-to-lure-back-travelers}{%
\section{Forget Spas and Bars. Hotels Tout Housekeeping to Lure Back
Travelers.}\label{forget-spas-and-bars-hotels-tout-housekeeping-to-lure-back-travelers}}

Hilton has partnered with Lysol, Four Seasons with Johns Hopkins
Medicine. But new research shows hotels can be easily contaminated by
the coronavirus.

\includegraphics{https://static01.graylady3jvrrxbe.onion/images/2020/08/05/science/00VIRUS-HOTELS1b/00VIRUS-HOTELS1b-articleLarge.jpg?quality=75\&auto=webp\&disable=upscale}

By \href{https://www.nytimes3xbfgragh.onion/by/matt-richtel}{Matt
Richtel}

\begin{itemize}
\item
  Aug. 4, 2020
\item
  \begin{itemize}
  \item
  \item
  \item
  \item
  \item
  \end{itemize}
\end{itemize}

When Beau Phillips checked into a hotel near Toledo recently, a table in
front of the counter barricaded him from getting too close to the clerk,
who wore a mask and stood behind a plastic window.

``The key is gently tossed at you from three feet away,'' said Mr.
Phillips, a public affairs executive who was staying at a Radisson
Country Inn \& Suites while visiting family.

The hotel's breakfast buffet was gone, the fitness center closed,
elevators limited to two riders. And to reduce the risk of an in-person
visit, after Mr. Phillips left his room each day, no housekeeper came in
to make the bed.

The pandemic has plunged the hotel industry into a historic downturn.
Average hotel occupancy dipped as low as 22 percent in late March, and
had risen to a still
miserable\href{https://str.com/press-release/str-us-hotel-results-week-ending-25-july}{48.1
percent the week ending July
25,}\href{https://str.com/press-release/str-canada-hotel-results-week-ending-4-july}{}according
to STR, a market research firm. So hotels nationwide have embarked on a
transformation of the most basic ways they run their business**,** aimed
at showing would-be travelers they understand where they're at:
terrified.

Some \href{https://wwwnc.cdc.gov/eid/article/26/9/20-1435_article}{new
research suggests travelers might have a point.} A study scheduled for
publication in September in the journal Emerging Infectious Diseases ---
but already made public by the Centers for Disease Control and
Prevention on its website --- found that people infected with the
coronavirus shed it on pillow cases, duvet covers, sheets, light switch
and bathroom door and faucet handles.

\includegraphics{https://static01.graylady3jvrrxbe.onion/images/2020/07/17/science/17VIRUS-HOTELS3/merlin_174678024_4ab0334f-136c-4aff-9755-e32fa35868f5-articleLarge.jpg?quality=75\&auto=webp\&disable=upscale}

Wyndham Hotels \& Resorts, in its new
``\href{https://www.wyndhamhotels.com/about-us/count-on-us}{Count on
Us}'' pandemic marketing campaign, heralds the use of ``hospital grade''
cleaning products. It is putting on overt shows of sanitation:
Housekeepers now linger and clean around the lobby, conspicuously wiping
down public areas, luggage carts, door knobs, and the counter.

\hypertarget{latest-updates-global-coronavirus-outbreak}{%
\section{\texorpdfstring{\href{https://www.nytimes3xbfgragh.onion/2020/08/04/world/coronavirus-cases.html?action=click\&pgtype=Article\&state=default\&region=MAIN_CONTENT_1\&context=storylines_live_updates}{Latest
Updates: Global Coronavirus
Outbreak}}{Latest Updates: Global Coronavirus Outbreak}}\label{latest-updates-global-coronavirus-outbreak}}

Updated 2020-08-05T07:58:24.076Z

\begin{itemize}
\tightlist
\item
  \href{https://www.nytimes3xbfgragh.onion/2020/08/04/world/coronavirus-cases.html?action=click\&pgtype=Article\&state=default\&region=MAIN_CONTENT_1\&context=storylines_live_updates\#link-762df92}{As
  talks drag on, McConnell signals openness to jobless aid extension,
  and negotiators agree on a deadline.}
\item
  \href{https://www.nytimes3xbfgragh.onion/2020/08/04/world/coronavirus-cases.html?action=click\&pgtype=Article\&state=default\&region=MAIN_CONTENT_1\&context=storylines_live_updates\#link-1228a480}{Novavax
  sees encouraging results from two studies of its experimental
  vaccine.}
\item
  \href{https://www.nytimes3xbfgragh.onion/2020/08/04/world/coronavirus-cases.html?action=click\&pgtype=Article\&state=default\&region=MAIN_CONTENT_1\&context=storylines_live_updates\#link-794484ed}{Mississippians
  must now wear masks in public, governor says.}
\end{itemize}

\href{https://www.nytimes3xbfgragh.onion/2020/08/04/world/coronavirus-cases.html?action=click\&pgtype=Article\&state=default\&region=MAIN_CONTENT_1\&context=storylines_live_updates}{See
more updates}

More live coverage:
\href{https://www.nytimes3xbfgragh.onion/live/2020/08/04/business/stock-market-today-coronavirus?action=click\&pgtype=Article\&state=default\&region=MAIN_CONTENT_1\&context=storylines_live_updates}{Markets}

``In the past, we may have cleaned hotels in the overnight because you
didn't necessarily want to see people cleaning,'' said Lisa Checchio,
the chief marketing officer of Wyndham, the franchise parent of Wyndham,
Days Inn, Super 8, La Quinta and more than a dozen other major brands
among its 6,000 domestic hotels.

Hilton new program (marketing name:
``\href{https://www.hilton.com/en/corporate/cleanstay/}{CleanStay}'')
includes a partnership to use Lysol cleaning products that requires
individual hotels to use the company's products and display the Lysol
logo ``prominently.'' Room cleanings include extra time spent on
``high-touch areas'' that included light and climate control switches,
handles and knobs, telephones and clocks. And, or course, the remote
control ``which has one of the highest ick factors or perceived ick
factors,'' said Phil Cordell, Hilton's global head of new brand
development.

He recalled that one guest had wrapped the plastic lining from the ice
bucket around the remote control before using it.

``People are understandably freaked out or hyper aware,'' Mr. Cordell
said.

The new research looked at the virus residue left by two
``pre-symptomatic'' patients there who were quarantined in China in
March --- students who had returned from overseas and were placed in
hotels during a mandatory waiting period.

Their rooms were swabbed for evidence that the virus lingered after the
students had been there 24 hours, but before the rooms were cleaned. The
researchers said the study shows that hotel rooms must be rigorously
cleaned between guest stays and done so with an eye to how the virus
spreads.

``To minimize the possibility of dispersing virus through the air, we
recommend that used linens not be shaken upon removal,'' the study said,
``and that laundered items be thoroughly cleaned and dried to prevent
additional spread.

Image

Placards in the Hilton's lobby remind guests to wear face masks and
maintain social distance.Credit...Alyssa Schukar for The New York Times

To show they are, indeed, rigorous in their cleaning, several chains are
heralding the consulting they are getting from big-name medical
institutions. Four Seasons said it signed a consulting agreement with
Johns Hopkins Medical International as part of an effort ``to inform
health and safety decisions based on the latest scientific knowledge,''
while Hilton retained counsel from the Mayo Clinic to develop ``enhanced
cleaning standards.''

All the attention to sanitation has created other issues. Since the
masks employees are required to wear shroud smiles, Hilton, which has
hotels throughout the world, has been experimenting with hand gestures
to express warmth and welcome. ``One is a very simple wave. In some
cultures, it could be a bow,'' Mr. Cordell said. ``It could be hats off
but with no hat --- but that could look kind of weird --- or a hand over
heart.''

Choice Hotels, a conglomerate that owns brands including Quality Inn and
EconoLodge, found in surveys that travelers wanted prepackaged
breakfasts, not buffets, and that any fruit should be the kind that
peels --- bananas or oranges instead of, say, apples or strawberries. It
also found that would-be guests wanted outdoor space and so it
\href{https://www.choicehotels.com/california/el-segundo/cambria-hotels/cad84}{revamped
websites} of its upscale Cambria brands to highlight photographs of
pools and rooftop decks.

(Some hotels are requiring reservations for the pool to keep density
low.)

Image

Rooms are sealed after cleanings at the Hilton.Credit...Alyssa Schukar
for The New York Times

Given the industry's dire economic crisis, some of the changes it's
adopting cost little, or even save money, said Bjorn Hanson, former dean
of hospitality at New York University who has also spent years working
inside the industry. For instance, he said, hotels can save money on
housekeeping by not cleaning rooms every night, or by promising not to
put guests in adjoining rooms, as some hotels have done (in reality,
there's not enough occupancy to have high density anyway).

\href{https://www.nytimes3xbfgragh.onion/news-event/coronavirus?action=click\&pgtype=Article\&state=default\&region=MAIN_CONTENT_3\&context=storylines_faq}{}

\hypertarget{the-coronavirus-outbreak-}{%
\subsubsection{The Coronavirus Outbreak
›}\label{the-coronavirus-outbreak-}}

\hypertarget{frequently-asked-questions}{%
\paragraph{Frequently Asked
Questions}\label{frequently-asked-questions}}

Updated August 4, 2020

\begin{itemize}
\item ~
  \hypertarget{i-have-antibodies-am-i-now-immune}{%
  \paragraph{I have antibodies. Am I now
  immune?}\label{i-have-antibodies-am-i-now-immune}}

  \begin{itemize}
  \tightlist
  \item
    As of right
    now,\href{https://www.nytimes3xbfgragh.onion/2020/07/22/health/covid-antibodies-herd-immunity.html?action=click\&pgtype=Article\&state=default\&region=MAIN_CONTENT_3\&context=storylines_faq}{that
    seems likely, for at least several months.} There have been
    frightening accounts of people suffering what seems to be a second
    bout of Covid-19. But experts say these patients may have a
    drawn-out course of infection, with the virus taking a slow toll
    weeks to months after initial exposure. People infected with the
    coronavirus typically
    \href{https://www.nature.com/articles/s41586-020-2456-9}{produce}
    immune molecules called antibodies, which are
    \href{https://www.nytimes3xbfgragh.onion/2020/05/07/health/coronavirus-antibody-prevalence.html?action=click\&pgtype=Article\&state=default\&region=MAIN_CONTENT_3\&context=storylines_faq}{protective
    proteins made in response to an
    infection}\href{https://www.nytimes3xbfgragh.onion/2020/05/07/health/coronavirus-antibody-prevalence.html?action=click\&pgtype=Article\&state=default\&region=MAIN_CONTENT_3\&context=storylines_faq}{.
    These antibodies may} last in the body
    \href{https://www.nature.com/articles/s41591-020-0965-6}{only two to
    three months}, which may seem worrisome, but that's perfectly normal
    after an acute infection subsides, said Dr. Michael Mina, an
    immunologist at Harvard University. It may be possible to get the
    coronavirus again, but it's highly unlikely that it would be
    possible in a short window of time from initial infection or make
    people sicker the second time.
  \end{itemize}
\item ~
  \hypertarget{im-a-small-business-owner-can-i-get-relief}{%
  \paragraph{I'm a small-business owner. Can I get
  relief?}\label{im-a-small-business-owner-can-i-get-relief}}

  \begin{itemize}
  \tightlist
  \item
    The
    \href{https://www.nytimes3xbfgragh.onion/article/small-business-loans-stimulus-grants-freelancers-coronavirus.html?action=click\&pgtype=Article\&state=default\&region=MAIN_CONTENT_3\&context=storylines_faq}{stimulus
    bills enacted in March} offer help for the millions of American
    small businesses. Those eligible for aid are businesses and
    nonprofit organizations with fewer than 500 workers, including sole
    proprietorships, independent contractors and freelancers. Some
    larger companies in some industries are also eligible. The help
    being offered, which is being managed by the Small Business
    Administration, includes the Paycheck Protection Program and the
    Economic Injury Disaster Loan program. But lots of folks have
    \href{https://www.nytimes3xbfgragh.onion/interactive/2020/05/07/business/small-business-loans-coronavirus.html?action=click\&pgtype=Article\&state=default\&region=MAIN_CONTENT_3\&context=storylines_faq}{not
    yet seen payouts.} Even those who have received help are confused:
    The rules are draconian, and some are stuck sitting on
    \href{https://www.nytimes3xbfgragh.onion/2020/05/02/business/economy/loans-coronavirus-small-business.html?action=click\&pgtype=Article\&state=default\&region=MAIN_CONTENT_3\&context=storylines_faq}{money
    they don't know how to use.} Many small-business owners are getting
    less than they expected or
    \href{https://www.nytimes3xbfgragh.onion/2020/06/10/business/Small-business-loans-ppp.html?action=click\&pgtype=Article\&state=default\&region=MAIN_CONTENT_3\&context=storylines_faq}{not
    hearing anything at all.}
  \end{itemize}
\item ~
  \hypertarget{what-are-my-rights-if-i-am-worried-about-going-back-to-work}{%
  \paragraph{What are my rights if I am worried about going back to
  work?}\label{what-are-my-rights-if-i-am-worried-about-going-back-to-work}}

  \begin{itemize}
  \tightlist
  \item
    Employers have to provide
    \href{https://www.osha.gov/SLTC/covid-19/standards.html}{a safe
    workplace} with policies that protect everyone equally.
    \href{https://www.nytimes3xbfgragh.onion/article/coronavirus-money-unemployment.html?action=click\&pgtype=Article\&state=default\&region=MAIN_CONTENT_3\&context=storylines_faq}{And
    if one of your co-workers tests positive for the coronavirus, the
    C.D.C.} has said that
    \href{https://www.cdc.gov/coronavirus/2019-ncov/community/guidance-business-response.html}{employers
    should tell their employees} -\/- without giving you the sick
    employee's name -\/- that they may have been exposed to the virus.
  \end{itemize}
\item ~
  \hypertarget{should-i-refinance-my-mortgage}{%
  \paragraph{Should I refinance my
  mortgage?}\label{should-i-refinance-my-mortgage}}

  \begin{itemize}
  \tightlist
  \item
    \href{https://www.nytimes3xbfgragh.onion/article/coronavirus-money-unemployment.html?action=click\&pgtype=Article\&state=default\&region=MAIN_CONTENT_3\&context=storylines_faq}{It
    could be a good idea,} because mortgage rates have
    \href{https://www.nytimes3xbfgragh.onion/2020/07/16/business/mortgage-rates-below-3-percent.html?action=click\&pgtype=Article\&state=default\&region=MAIN_CONTENT_3\&context=storylines_faq}{never
    been lower.} Refinancing requests have pushed mortgage applications
    to some of the highest levels since 2008, so be prepared to get in
    line. But defaults are also up, so if you're thinking about buying a
    home, be aware that some lenders have tightened their standards.
  \end{itemize}
\item ~
  \hypertarget{what-is-school-going-to-look-like-in-september}{%
  \paragraph{What is school going to look like in
  September?}\label{what-is-school-going-to-look-like-in-september}}

  \begin{itemize}
  \tightlist
  \item
    It is unlikely that many schools will return to a normal schedule
    this fall, requiring the grind of
    \href{https://www.nytimes3xbfgragh.onion/2020/06/05/us/coronavirus-education-lost-learning.html?action=click\&pgtype=Article\&state=default\&region=MAIN_CONTENT_3\&context=storylines_faq}{online
    learning},
    \href{https://www.nytimes3xbfgragh.onion/2020/05/29/us/coronavirus-child-care-centers.html?action=click\&pgtype=Article\&state=default\&region=MAIN_CONTENT_3\&context=storylines_faq}{makeshift
    child care} and
    \href{https://www.nytimes3xbfgragh.onion/2020/06/03/business/economy/coronavirus-working-women.html?action=click\&pgtype=Article\&state=default\&region=MAIN_CONTENT_3\&context=storylines_faq}{stunted
    workdays} to continue. California's two largest public school
    districts --- Los Angeles and San Diego --- said on July 13, that
    \href{https://www.nytimes3xbfgragh.onion/2020/07/13/us/lausd-san-diego-school-reopening.html?action=click\&pgtype=Article\&state=default\&region=MAIN_CONTENT_3\&context=storylines_faq}{instruction
    will be remote-only in the fall}, citing concerns that surging
    coronavirus infections in their areas pose too dire a risk for
    students and teachers. Together, the two districts enroll some
    825,000 students. They are the largest in the country so far to
    abandon plans for even a partial physical return to classrooms when
    they reopen in August. For other districts, the solution won't be an
    all-or-nothing approach.
    \href{https://bioethics.jhu.edu/research-and-outreach/projects/eschool-initiative/school-policy-tracker/}{Many
    systems}, including the nation's largest, New York City, are
    devising
    \href{https://www.nytimes3xbfgragh.onion/2020/06/26/us/coronavirus-schools-reopen-fall.html?action=click\&pgtype=Article\&state=default\&region=MAIN_CONTENT_3\&context=storylines_faq}{hybrid
    plans} that involve spending some days in classrooms and other days
    online. There's no national policy on this yet, so check with your
    municipal school system regularly to see what is happening in your
    community.
  \end{itemize}
\end{itemize}

``Safety doesn't necessarily cost money,'' he said. ``It could be an
excuse for saving money,''

Some would-be travelers say they're just not ready to return, no matter
the assurances.

``I've stayed at nice hotels in the past and found something sticky. If
I found something sticky and smudgy now, it would send me to the moon,''
said Kevin Mercuri, chief executive of a New York public relations firm.
He and colleagues recently decided against visiting a client in Georgia
partly to avoid hotels. His concern about hotels, in a nutshell: ``Fear
of infection.''

The C.D.C. has recommended that people who stay at hotels check in
online, choose properties where staff wear masks and that regularly
clean or remove shared-touch items, like pens or phones, and disinfect
doorknobs, ice and vending machines, among other things.

Charles Gerba, a professor at the University of Arizona who studies
hotel cleanliness, said hotels do not pose significant risk of
transmission of Covid-19 so long as they clean with products known to
kill the virus.

His own prior research has shown that housekeepers can carry viruses
with them from room-to-room and guests can carry them from public areas,
like conference rooms. Proper use of cleaning products, the research
showed, sharply cut risk of transmission.

Image

Placards remind guests to maintain distance.Credit...Alyssa Schukar for
The New York Times

``If a product is EPA approved and you're not using it right, it isn't
doing me any good,'' he said, meaning that cleaning must be thorough and
not taken lightly. He said he'd feel comfortable staying at a hotel, but
would decline daily maid service, and bring his own hand sanitizer and
wipes.

Other public health researchers said that the risk of a hotel stay
depended heavily on an the customer's own commitment to wearing a mask
or remaining socially distant.

``You need to an make informed decision to maintain your space,'' said
Eyal Oren, an associate professor in the Division of Epidemiology and
Bioinformatics at the San Diego State University School of Public
Health. He said hotels do offer the prospect of such distancing, ``which
I'd distinguish from an airplane.''

For people who choose to travel, one perk comes at the expense of the
hotels: the price. STR, the market research
firm,\href{https://str.com/press-release/us-hotel-demand-not-expected-fully-recover-until-2023}{projects
the average cost of a nightly}stay in 2020 will wind up at \$103, down
from \$131 a year ago. (In July, the average rate was \$97).

There are other savings too. Mr. Phillips always leaves a tip for the
cleaning crew and did so again during his recent stay at the Country Inn
\& Suites outside of Toledo.

``The first day, I left a \$20 for the housekeeper like I always do,''
he said. ``It was still there when I got back. No one had come in.''

Advertisement

\protect\hyperlink{after-bottom}{Continue reading the main story}

\hypertarget{site-index}{%
\subsection{Site Index}\label{site-index}}

\hypertarget{site-information-navigation}{%
\subsection{Site Information
Navigation}\label{site-information-navigation}}

\begin{itemize}
\tightlist
\item
  \href{https://help.nytimes3xbfgragh.onion/hc/en-us/articles/115014792127-Copyright-notice}{©~2020~The
  New York Times Company}
\end{itemize}

\begin{itemize}
\tightlist
\item
  \href{https://www.nytco.com/}{NYTCo}
\item
  \href{https://help.nytimes3xbfgragh.onion/hc/en-us/articles/115015385887-Contact-Us}{Contact
  Us}
\item
  \href{https://www.nytco.com/careers/}{Work with us}
\item
  \href{https://nytmediakit.com/}{Advertise}
\item
  \href{http://www.tbrandstudio.com/}{T Brand Studio}
\item
  \href{https://www.nytimes3xbfgragh.onion/privacy/cookie-policy\#how-do-i-manage-trackers}{Your
  Ad Choices}
\item
  \href{https://www.nytimes3xbfgragh.onion/privacy}{Privacy}
\item
  \href{https://help.nytimes3xbfgragh.onion/hc/en-us/articles/115014893428-Terms-of-service}{Terms
  of Service}
\item
  \href{https://help.nytimes3xbfgragh.onion/hc/en-us/articles/115014893968-Terms-of-sale}{Terms
  of Sale}
\item
  \href{https://spiderbites.nytimes3xbfgragh.onion}{Site Map}
\item
  \href{https://help.nytimes3xbfgragh.onion/hc/en-us}{Help}
\item
  \href{https://www.nytimes3xbfgragh.onion/subscription?campaignId=37WXW}{Subscriptions}
\end{itemize}
