Sections

SEARCH

\protect\hyperlink{site-content}{Skip to
content}\protect\hyperlink{site-index}{Skip to site index}

\href{https://www.nytimes3xbfgragh.onion/section/nyregion}{New York}

\href{https://myaccount.nytimes3xbfgragh.onion/auth/login?response_type=cookie\&client_id=vi}{}

\href{https://www.nytimes3xbfgragh.onion/section/todayspaper}{Today's
Paper}

\href{/section/nyregion}{New York}\textbar{}N.Y.C. Health Commissioner
Resigns After Clashes With Mayor Over Virus

\url{https://nyti.ms/33mvVmJ}

\begin{itemize}
\item
\item
\item
\item
\item
\end{itemize}

\href{https://www.nytimes3xbfgragh.onion/news-event/coronavirus?action=click\&pgtype=Article\&state=default\&region=TOP_BANNER\&context=storylines_menu}{The
Coronavirus Outbreak}

\begin{itemize}
\tightlist
\item
  live\href{https://www.nytimes3xbfgragh.onion/2020/08/04/world/coronavirus-cases.html?action=click\&pgtype=Article\&state=default\&region=TOP_BANNER\&context=storylines_menu}{Latest
  Updates}
\item
  \href{https://www.nytimes3xbfgragh.onion/interactive/2020/us/coronavirus-us-cases.html?action=click\&pgtype=Article\&state=default\&region=TOP_BANNER\&context=storylines_menu}{Maps
  and Cases}
\item
  \href{https://www.nytimes3xbfgragh.onion/interactive/2020/science/coronavirus-vaccine-tracker.html?action=click\&pgtype=Article\&state=default\&region=TOP_BANNER\&context=storylines_menu}{Vaccine
  Tracker}
\item
  \href{https://www.nytimes3xbfgragh.onion/2020/08/02/us/covid-college-reopening.html?action=click\&pgtype=Article\&state=default\&region=TOP_BANNER\&context=storylines_menu}{College
  Reopening}
\item
  \href{https://www.nytimes3xbfgragh.onion/live/2020/08/04/business/stock-market-today-coronavirus?action=click\&pgtype=Article\&state=default\&region=TOP_BANNER\&context=storylines_menu}{Economy}
\end{itemize}

Advertisement

\protect\hyperlink{after-top}{Continue reading the main story}

Supported by

\protect\hyperlink{after-sponsor}{Continue reading the main story}

\hypertarget{nyc-health-commissioner-resigns-after-clashes-with-mayor-over-virus}{%
\section{N.Y.C. Health Commissioner Resigns After Clashes With Mayor
Over
Virus}\label{nyc-health-commissioner-resigns-after-clashes-with-mayor-over-virus}}

The resignation of Dr. Oxiris Barbot, the commissioner since 2018, came
after Mayor Bill de Blasio stripped her agency of a key virus-tracing
program.

\includegraphics{https://static01.graylady3jvrrxbe.onion/images/2020/05/18/nyregion/00nyvirus-barbot-HFO/00nyvirus-barbot-HFO-articleLarge.jpg?quality=75\&auto=webp\&disable=upscale}

\href{https://www.nytimes3xbfgragh.onion/by/j-david-goodman}{\includegraphics{https://static01.graylady3jvrrxbe.onion/images/2018/07/18/nyregion/author-j-david-goodman/author-j-david-goodman-thumbLarge.png}}

By \href{https://www.nytimes3xbfgragh.onion/by/j-david-goodman}{J. David
Goodman}

\begin{itemize}
\item
  Aug. 4, 2020
\item
  \begin{itemize}
  \item
  \item
  \item
  \item
  \item
  \end{itemize}
\end{itemize}

New York City's health commissioner, Dr. Oxiris Barbot, resigned on
Tuesday and voiced her ``deep disappointment'' with Mayor Bill de
Blasio's handling of the pandemic, renewing scrutiny of his leadership
during the crisis just as the city faces pressing decisions about how
quickly to reopen schools and businesses.

Dr. Barbot's departure came after
\href{https://www.nytimes3xbfgragh.onion/2020/05/14/nyregion/coronavirus-de-blasio-mitchell-katz.html}{escalating
tensions} between City Hall and top city health department officials,
which had begun at the start of the coronavirus outbreak in March, burst
into public view and raised concerns that the feuding was undermining
crucial public health policies.

``I leave my post today with deep disappointment that during the most
critical public health crisis in our lifetime, that the health
department's incomparable disease control expertise was not used to the
degree it could have been,'' Dr. Barbot said in her resignation email
sent to Mr. de Blasio, a copy of which was shared with The New York
Times.

``Our experts are world renowned for their epidemiology, surveillance
and response work. The city would be well served by having them at the
strategic center of the response not in the background.''

The mayor on Tuesday morning immediately announced a replacement for Dr.
Barbot, and later pushed back against any suggestion that she had
resigned in protest.

After a day of news coverage highlighting her departure, the mayor's
office abruptly released a statement at 8:30 p.m., saying that Dr.
Barbot had been told over the weekend that ``the administration was
moving in a different direction.''

At a hastily called news conference after her resignation earlier in the
day, Mr. de Blasio had defended his handling of the outbreak, saying
that the city had made ''extraordinary progress.''

The virus took a devastating toll in the spring, killing more than
20,000 residents, but it has largely ebbed in recent weeks. On Monday,
for example, only 316 people in the city tested positive out of more
than 30,000 tested.

Still, the turnover in the Department of Health and Mental Hygiene comes
at a pivotal moment: Public schools are scheduled to partially open next
month --- which could be crucial for the city's recovery --- and fears
are growing that the outbreak could surge again when the weather cools.

``It had been clear in recent days that it was time for a change,'' Mr.
de Blasio said about Dr. Barbot. ``We need an atmosphere of unity. We
need an atmosphere of common purpose.''

\hypertarget{latest-updates-global-coronavirus-outbreak}{%
\section{\texorpdfstring{\href{https://www.nytimes3xbfgragh.onion/2020/08/04/world/coronavirus-cases.html?action=click\&pgtype=Article\&state=default\&region=MAIN_CONTENT_1\&context=storylines_live_updates}{Latest
Updates: Global Coronavirus
Outbreak}}{Latest Updates: Global Coronavirus Outbreak}}\label{latest-updates-global-coronavirus-outbreak}}

Updated 2020-08-05T07:58:24.076Z

\begin{itemize}
\tightlist
\item
  \href{https://www.nytimes3xbfgragh.onion/2020/08/04/world/coronavirus-cases.html?action=click\&pgtype=Article\&state=default\&region=MAIN_CONTENT_1\&context=storylines_live_updates\#link-762df92}{As
  talks drag on, McConnell signals openness to jobless aid extension,
  and negotiators agree on a deadline.}
\item
  \href{https://www.nytimes3xbfgragh.onion/2020/08/04/world/coronavirus-cases.html?action=click\&pgtype=Article\&state=default\&region=MAIN_CONTENT_1\&context=storylines_live_updates\#link-1228a480}{Novavax
  sees encouraging results from two studies of its experimental
  vaccine.}
\item
  \href{https://www.nytimes3xbfgragh.onion/2020/08/04/world/coronavirus-cases.html?action=click\&pgtype=Article\&state=default\&region=MAIN_CONTENT_1\&context=storylines_live_updates\#link-794484ed}{Mississippians
  must now wear masks in public, governor says.}
\end{itemize}

\href{https://www.nytimes3xbfgragh.onion/2020/08/04/world/coronavirus-cases.html?action=click\&pgtype=Article\&state=default\&region=MAIN_CONTENT_1\&context=storylines_live_updates}{See
more updates}

More live coverage:
\href{https://www.nytimes3xbfgragh.onion/live/2020/08/04/business/stock-market-today-coronavirus?action=click\&pgtype=Article\&state=default\&region=MAIN_CONTENT_1\&context=storylines_live_updates}{Markets}

The mayor's new health commissioner is Dr. Dave A. Chokshi, a former
senior leader at Health + Hospitals, the city's public hospital system.

At no point in the statement about Dr. Chokshi's appointment did the
mayor thank --- or even mention --- Dr. Barbot, who had served in his
administration since the start of his first term in 2014. Mr. de Blasio
did acknowledge her service during the news conference.

The speed of the appointment and the robustness of the announcement ---
Mr. de Blasio had lined up
\href{https://wayback.archive-it.org/4765/20170106172109/https:/www.hhs.gov/about/leadership/vadm-vivek-murthy/index.html}{a
former surgeon general} to speak highly of Dr. Chokshi --- indicated
that Dr. Barbot's resignation had not occurred in a vacuum.

Dr. Chokshi, who has also worked for health department in Louisiana and
as a health adviser to the United States secretary of Veterans Affairs,
received praise from the former surgeon general, Dr. Vivek Murthy, who
called him an ``extraordinary public health leader who both sees the
forest and the trees.''

Still, former city health officials said the mayor should have done more
to listen to and support Dr. Barbot.

``It's a bad day for the city. She's a very qualified commissioner of
health,'' said Lilliam Barrios-Paoli, a former deputy mayor of health
under Mr. de Blasio who worked with Dr. Barbot. ``There's another woman
of color that goes down. I think it's a really regrettable thing.

``This is not a position you can put anybody just because. It's the
premier public health agency in the country,'' Ms. Barrios-Paoli added.
``It's just a shame that she did not feel that she was supported by the
mayor.''

Current and former health officials said the departure of Dr. Barbot
reflected Mr. de Blasio's
\href{https://www.nytimes3xbfgragh.onion/2020/03/16/nyregion/coronavirus-bill-de-blasio.html}{history
of distrust} in his health department. From early in the coronavirus
outbreak, he has clashed with the department on testing, public
messaging and how quickly to shutter schools.

Mr. de Blasio has been faulted for resisting calls to close down schools
and businesses, which some epidemiologists believe worsened the
outbreak.

Once he decided on closures, Mr. de Blasio pushed for the state to
authorize further restrictions, a move ultimately made by Gov. Andrew M.
Cuomo. And in the intervening months, the mayor has moved cautiously in
reopening the city, guarding the progress that has been made in
controlling the virus.

But even as the outbreak began to ebb in late May, tensions with Dr.
Barbot mounted.

Some public health officials had bristled at the mayor's decision to
place the city's contact-tracing program inside Health + Hospitals. The
health department has performed such tracing for decades; the public
hospitals have not.

Dr. Barbot disagreed with the move, but kept her disapproval private.

Yet the behind-the-scenes tensions flared into the public eye in May,
when an article appeared in The New York Post about a conflict months
earlier between Dr. Barbot and a police commander who wanted personal
protective gear that had been set aside for health workers to be given
instead to the police.
\href{https://nypost.com/2020/05/13/nyc-health-commissioner-wouldnt-supply-nypd-with-masks/}{The
Post quoted} Dr. Barbot as saying at one point, ``I don't give two rats'
asses about your cops.'' Police unions and some elected officials called
for her ouster.

At that point, Dr. Barbot began to make fewer public appearances. And
Mr. de Blasio seemed to be looking elsewhere for public health guidance,
turning to a new senior adviser, Dr. Jay Varma, and to Dr. Mitchell
Katz, the public hospitals chief.

On Tuesday, Mr. de Blasio made clear that he did not believe that Dr.
Barbot was a team player.

``It's never about one agency,'' he said at one point. He used the words
``teamwork'' or ``team'' 15 times in a 38-minute news conference.

New York City's health department is regarded as one of the best
municipal health agencies in the world. But during the epidemic, the
mayor has repeatedly ignored the advice of its top disease-control
experts and sidelined the department.

\href{https://www.nytimes3xbfgragh.onion/news-event/coronavirus?action=click\&pgtype=Article\&state=default\&region=MAIN_CONTENT_3\&context=storylines_faq}{}

\hypertarget{the-coronavirus-outbreak-}{%
\subsubsection{The Coronavirus Outbreak
›}\label{the-coronavirus-outbreak-}}

\hypertarget{frequently-asked-questions}{%
\paragraph{Frequently Asked
Questions}\label{frequently-asked-questions}}

Updated August 4, 2020

\begin{itemize}
\item ~
  \hypertarget{i-have-antibodies-am-i-now-immune}{%
  \paragraph{I have antibodies. Am I now
  immune?}\label{i-have-antibodies-am-i-now-immune}}

  \begin{itemize}
  \tightlist
  \item
    As of right
    now,\href{https://www.nytimes3xbfgragh.onion/2020/07/22/health/covid-antibodies-herd-immunity.html?action=click\&pgtype=Article\&state=default\&region=MAIN_CONTENT_3\&context=storylines_faq}{that
    seems likely, for at least several months.} There have been
    frightening accounts of people suffering what seems to be a second
    bout of Covid-19. But experts say these patients may have a
    drawn-out course of infection, with the virus taking a slow toll
    weeks to months after initial exposure. People infected with the
    coronavirus typically
    \href{https://www.nature.com/articles/s41586-020-2456-9}{produce}
    immune molecules called antibodies, which are
    \href{https://www.nytimes3xbfgragh.onion/2020/05/07/health/coronavirus-antibody-prevalence.html?action=click\&pgtype=Article\&state=default\&region=MAIN_CONTENT_3\&context=storylines_faq}{protective
    proteins made in response to an
    infection}\href{https://www.nytimes3xbfgragh.onion/2020/05/07/health/coronavirus-antibody-prevalence.html?action=click\&pgtype=Article\&state=default\&region=MAIN_CONTENT_3\&context=storylines_faq}{.
    These antibodies may} last in the body
    \href{https://www.nature.com/articles/s41591-020-0965-6}{only two to
    three months}, which may seem worrisome, but that's perfectly normal
    after an acute infection subsides, said Dr. Michael Mina, an
    immunologist at Harvard University. It may be possible to get the
    coronavirus again, but it's highly unlikely that it would be
    possible in a short window of time from initial infection or make
    people sicker the second time.
  \end{itemize}
\item ~
  \hypertarget{im-a-small-business-owner-can-i-get-relief}{%
  \paragraph{I'm a small-business owner. Can I get
  relief?}\label{im-a-small-business-owner-can-i-get-relief}}

  \begin{itemize}
  \tightlist
  \item
    The
    \href{https://www.nytimes3xbfgragh.onion/article/small-business-loans-stimulus-grants-freelancers-coronavirus.html?action=click\&pgtype=Article\&state=default\&region=MAIN_CONTENT_3\&context=storylines_faq}{stimulus
    bills enacted in March} offer help for the millions of American
    small businesses. Those eligible for aid are businesses and
    nonprofit organizations with fewer than 500 workers, including sole
    proprietorships, independent contractors and freelancers. Some
    larger companies in some industries are also eligible. The help
    being offered, which is being managed by the Small Business
    Administration, includes the Paycheck Protection Program and the
    Economic Injury Disaster Loan program. But lots of folks have
    \href{https://www.nytimes3xbfgragh.onion/interactive/2020/05/07/business/small-business-loans-coronavirus.html?action=click\&pgtype=Article\&state=default\&region=MAIN_CONTENT_3\&context=storylines_faq}{not
    yet seen payouts.} Even those who have received help are confused:
    The rules are draconian, and some are stuck sitting on
    \href{https://www.nytimes3xbfgragh.onion/2020/05/02/business/economy/loans-coronavirus-small-business.html?action=click\&pgtype=Article\&state=default\&region=MAIN_CONTENT_3\&context=storylines_faq}{money
    they don't know how to use.} Many small-business owners are getting
    less than they expected or
    \href{https://www.nytimes3xbfgragh.onion/2020/06/10/business/Small-business-loans-ppp.html?action=click\&pgtype=Article\&state=default\&region=MAIN_CONTENT_3\&context=storylines_faq}{not
    hearing anything at all.}
  \end{itemize}
\item ~
  \hypertarget{what-are-my-rights-if-i-am-worried-about-going-back-to-work}{%
  \paragraph{What are my rights if I am worried about going back to
  work?}\label{what-are-my-rights-if-i-am-worried-about-going-back-to-work}}

  \begin{itemize}
  \tightlist
  \item
    Employers have to provide
    \href{https://www.osha.gov/SLTC/covid-19/standards.html}{a safe
    workplace} with policies that protect everyone equally.
    \href{https://www.nytimes3xbfgragh.onion/article/coronavirus-money-unemployment.html?action=click\&pgtype=Article\&state=default\&region=MAIN_CONTENT_3\&context=storylines_faq}{And
    if one of your co-workers tests positive for the coronavirus, the
    C.D.C.} has said that
    \href{https://www.cdc.gov/coronavirus/2019-ncov/community/guidance-business-response.html}{employers
    should tell their employees} -\/- without giving you the sick
    employee's name -\/- that they may have been exposed to the virus.
  \end{itemize}
\item ~
  \hypertarget{should-i-refinance-my-mortgage}{%
  \paragraph{Should I refinance my
  mortgage?}\label{should-i-refinance-my-mortgage}}

  \begin{itemize}
  \tightlist
  \item
    \href{https://www.nytimes3xbfgragh.onion/article/coronavirus-money-unemployment.html?action=click\&pgtype=Article\&state=default\&region=MAIN_CONTENT_3\&context=storylines_faq}{It
    could be a good idea,} because mortgage rates have
    \href{https://www.nytimes3xbfgragh.onion/2020/07/16/business/mortgage-rates-below-3-percent.html?action=click\&pgtype=Article\&state=default\&region=MAIN_CONTENT_3\&context=storylines_faq}{never
    been lower.} Refinancing requests have pushed mortgage applications
    to some of the highest levels since 2008, so be prepared to get in
    line. But defaults are also up, so if you're thinking about buying a
    home, be aware that some lenders have tightened their standards.
  \end{itemize}
\item ~
  \hypertarget{what-is-school-going-to-look-like-in-september}{%
  \paragraph{What is school going to look like in
  September?}\label{what-is-school-going-to-look-like-in-september}}

  \begin{itemize}
  \tightlist
  \item
    It is unlikely that many schools will return to a normal schedule
    this fall, requiring the grind of
    \href{https://www.nytimes3xbfgragh.onion/2020/06/05/us/coronavirus-education-lost-learning.html?action=click\&pgtype=Article\&state=default\&region=MAIN_CONTENT_3\&context=storylines_faq}{online
    learning},
    \href{https://www.nytimes3xbfgragh.onion/2020/05/29/us/coronavirus-child-care-centers.html?action=click\&pgtype=Article\&state=default\&region=MAIN_CONTENT_3\&context=storylines_faq}{makeshift
    child care} and
    \href{https://www.nytimes3xbfgragh.onion/2020/06/03/business/economy/coronavirus-working-women.html?action=click\&pgtype=Article\&state=default\&region=MAIN_CONTENT_3\&context=storylines_faq}{stunted
    workdays} to continue. California's two largest public school
    districts --- Los Angeles and San Diego --- said on July 13, that
    \href{https://www.nytimes3xbfgragh.onion/2020/07/13/us/lausd-san-diego-school-reopening.html?action=click\&pgtype=Article\&state=default\&region=MAIN_CONTENT_3\&context=storylines_faq}{instruction
    will be remote-only in the fall}, citing concerns that surging
    coronavirus infections in their areas pose too dire a risk for
    students and teachers. Together, the two districts enroll some
    825,000 students. They are the largest in the country so far to
    abandon plans for even a partial physical return to classrooms when
    they reopen in August. For other districts, the solution won't be an
    all-or-nothing approach.
    \href{https://bioethics.jhu.edu/research-and-outreach/projects/eschool-initiative/school-policy-tracker/}{Many
    systems}, including the nation's largest, New York City, are
    devising
    \href{https://www.nytimes3xbfgragh.onion/2020/06/26/us/coronavirus-schools-reopen-fall.html?action=click\&pgtype=Article\&state=default\&region=MAIN_CONTENT_3\&context=storylines_faq}{hybrid
    plans} that involve spending some days in classrooms and other days
    online. There's no national policy on this yet, so check with your
    municipal school system regularly to see what is happening in your
    community.
  \end{itemize}
\end{itemize}

``I think this is the culmination of months of conflict between the
health department and City Hall,'' said Councilman Mark Levine, who
heads the Council's health committee. ``This reflects enormous
frustration that global experts in infectious disease are being
marginalized in the middle of a pandemic.''

Most recently, senior health department officials disagreed with Mr. de
Blasio on what to do when staff or students in city schools test
positive for the virus, according to a person with knowledge of the
officials' thinking.

Mr. de Blasio announced on Friday that a given school building could
close, in some cases for 14 days, when two positive cases emerged and
were not linked to the same classroom. The plan gives disease
investigators some discretion on closure. But even with only 1 percent
of tests coming back positive, as is the case in New York City now,
health officials are still worried that the thresholds will lead to many
schools closing at some point in the academic year, the person said.

Perhaps the most consequential debate inside City Hall over the
coronavirus came during the second week in March. The city had a small
number of positive cases, but its public health system was flashing a
warning about the unchecked spread of a flulike virus.

Dr. Barbot and one of her top deputies began urging more restrictions on
gatherings. Mr. de Blasio for a time sided instead with Dr. Katz, who
had been
\href{https://www.nytimes3xbfgragh.onion/2020/05/14/nyregion/coronavirus-de-blasio-mitchell-katz.html}{advising
City Hall against ordering shutdowns.}

Some officials inside the health department
\href{https://www.nytimes3xbfgragh.onion/2020/03/16/nyregion/coronavirus-bill-de-blasio.html}{talked
about quitting} that week, or staging a walkout to force action.
Eventually, top officials and the mayor agreed on the need to lock down
the city to stop the spread of the virus. Mr. de Blasio ordered schools
closed on March 15.

Outside of the administration, some blamed Dr. Barbot for the delays and
confusion, citing her shifting public statements on the virus from late
January to early March. A few elected officials
\href{https://nypost.com/2020/04/04/nyc-pols-urge-de-blasio-to-oust-health-commissioner-over-coronavirus-response/}{called
for her to be fired} in early April.

The turmoil at the top of the city's health agency worsened in May over
the mayor's decision to locate the city's contact-tracing efforts within
its public hospital system and not in the health department.

Under Health + Hospitals, the city's contact-tracing program got off to
a rocky start. Lacking the capability to hire and manage 3,000 new
workers, it outsourced much of the day-to-day management of the call
center at the core of its operations to Optum, a billion-dollar
subsidiary of UnitedHealth Group.

So far,
\href{https://hhinternet.blob.core.windows.net/uploads/2020/07/test-and-trace-data-metrics-20200727.pdf}{fewer
than half} of New Yorkers who have tested positive for the coronavirus
--- some 20,000 people since the program began on June 1 --- have shared
their contacts.

``Right now, cases are popping up all over the place and we are not
linking them to known contacts except in a small proportion of cases,''
Dr. Neil Vora, the director of the trace effort, said at an internal
town-hall-style meeting for tracers last month, a recording of which was
provided to The Times.

Even with the new tracing program, the health department has been called
on to handle more intricate aspects of so-called disease detective work,
particularly in group settings like homeless shelters and nursing homes.
That expanded to include restaurants and other social gatherings last
month.

The mayor said on Friday that outbreaks in schools would also be handled
by the health department, in coordination with the city's new corps of
contact tracers.

Joseph Goldstein and Sharon Otterman contributed reporting.

Advertisement

\protect\hyperlink{after-bottom}{Continue reading the main story}

\hypertarget{site-index}{%
\subsection{Site Index}\label{site-index}}

\hypertarget{site-information-navigation}{%
\subsection{Site Information
Navigation}\label{site-information-navigation}}

\begin{itemize}
\tightlist
\item
  \href{https://help.nytimes3xbfgragh.onion/hc/en-us/articles/115014792127-Copyright-notice}{©~2020~The
  New York Times Company}
\end{itemize}

\begin{itemize}
\tightlist
\item
  \href{https://www.nytco.com/}{NYTCo}
\item
  \href{https://help.nytimes3xbfgragh.onion/hc/en-us/articles/115015385887-Contact-Us}{Contact
  Us}
\item
  \href{https://www.nytco.com/careers/}{Work with us}
\item
  \href{https://nytmediakit.com/}{Advertise}
\item
  \href{http://www.tbrandstudio.com/}{T Brand Studio}
\item
  \href{https://www.nytimes3xbfgragh.onion/privacy/cookie-policy\#how-do-i-manage-trackers}{Your
  Ad Choices}
\item
  \href{https://www.nytimes3xbfgragh.onion/privacy}{Privacy}
\item
  \href{https://help.nytimes3xbfgragh.onion/hc/en-us/articles/115014893428-Terms-of-service}{Terms
  of Service}
\item
  \href{https://help.nytimes3xbfgragh.onion/hc/en-us/articles/115014893968-Terms-of-sale}{Terms
  of Sale}
\item
  \href{https://spiderbites.nytimes3xbfgragh.onion}{Site Map}
\item
  \href{https://help.nytimes3xbfgragh.onion/hc/en-us}{Help}
\item
  \href{https://www.nytimes3xbfgragh.onion/subscription?campaignId=37WXW}{Subscriptions}
\end{itemize}
