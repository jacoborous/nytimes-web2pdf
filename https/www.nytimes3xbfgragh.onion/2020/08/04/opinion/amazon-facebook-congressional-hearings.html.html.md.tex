Sections

SEARCH

\protect\hyperlink{site-content}{Skip to
content}\protect\hyperlink{site-index}{Skip to site index}

\href{https://myaccount.nytimes3xbfgragh.onion/auth/login?response_type=cookie\&client_id=vi}{}

\href{https://www.nytimes3xbfgragh.onion/section/todayspaper}{Today's
Paper}

\href{/section/opinion}{Opinion}\textbar{}What Years of Emails and Texts
Reveal About Your Friendly Tech Companies

\url{https://nyti.ms/3k9Z3ne}

\begin{itemize}
\item
\item
\item
\item
\item
\item
\end{itemize}

Advertisement

\protect\hyperlink{after-top}{Continue reading the main story}

\href{/section/opinion}{Opinion}

Supported by

\protect\hyperlink{after-sponsor}{Continue reading the main story}

\hypertarget{what-years-of-emails-and-texts-reveal-about-your-friendly-tech-companies}{%
\section{What Years of Emails and Texts Reveal About Your Friendly Tech
Companies}\label{what-years-of-emails-and-texts-reveal-about-your-friendly-tech-companies}}

Theatrics distracted from the real payoff of the congressional hearings:
the subpoenaed documents.

\href{https://www.nytimes3xbfgragh.onion/by/tim-wu}{\includegraphics{https://static01.graylady3jvrrxbe.onion/images/2017/04/05/opinion/tim-wu/tim-wu-thumbLarge-v4.jpg}}

By \href{https://www.nytimes3xbfgragh.onion/by/tim-wu}{Tim Wu}

Mr. Wu is the author of ``The Curse of Bigness: Antitrust in the New
Gilded Age.''

\begin{itemize}
\item
  Aug. 4, 2020, 5:00 a.m. ET
\item
  \begin{itemize}
  \item
  \item
  \item
  \item
  \item
  \item
  \end{itemize}
\end{itemize}

\includegraphics{https://static01.graylady3jvrrxbe.onion/images/2020/08/05/opinion/05Wu/04Wu-articleLarge.jpg?quality=75\&auto=webp\&disable=upscale}

The spectacle of the chief executives of Amazon, Apple, Facebook and
Google testifying before Congress last week made for good TV drama. Yet
the theatrics of the showdown distracted from the real payoff of the
hearings: the
\href{https://judiciary.house.gov/online-platforms-and-market-power/}{accompanying
cache} of subpoenaed emails and texts from the past decade and a half.
These documents provide compelling evidence --- long rumored but seldom
established --- that the companies, especially Facebook and Amazon, in
their rise to dominance did not always play by the rules and apparently
violated antitrust laws.

Both public opinion and American law distinguish between two kinds of
dominant company. The first is the monopoly fairly held: a corporation
like Ford Motor that achieves dominance by virtue of its incomparable
greatness. The second, its evil doppelgänger, is the company that
achieves dominance unfairly --- for instance, by suffocating or
absorbing would-be challengers.

The Big Tech companies insist that their rise to power has been the
first story, a saga of ingenuity and courage, and that their market
dominance is a byproduct of continued excellence. They may be giants,
the story goes, but they are friendly giants. Their immense size and
power is simply what is necessary to offer users the best possible
services.

The subpoenaed documents destroy that narrative. No one can deny that
these are well-run companies, loaded with talent, and that each at some
point offered something great. But it appears that without illegal
maneuvers --- without, above all, the anticompetitive buying of
potential rivals --- there might be no Big Tech, but rather a much wider
array of smaller, better, more specialized tech companies.

Exhibit A is Facebook, whose documents are the most damning. Emails from
Mark Zuckerberg, its chief executive, strongly suggest that since about
2008 he has had a method for controlling what in a 2012 email he called
\href{https://papers.ssrn.com/sol3/papers.cfm?abstract_id=3624058}{``nascent''
companies} that posed ``very disruptive'' threats to Facebook. His
method has been the buyout or the aggressive cloning of features to
compel a company to sell itself to Facebook. He foresaw that there would
be a limited number of ``social mechanics,'' or areas of innovation in
social media, each of which would have one winner. ``Instagram can hurt
us,'' he wrote in 2012, right before acquiring the company and
eliminating the threat that its photo- and video-sharing technology
posed to Facebook.

Amazon doesn't come off much better. Its documents show an apparent
willingness to lose money to keep competitors under water. Early on,
because of low pricing, Amazon lost more than \$200 million from diaper
products in a single month. It ran its chief competitor, Quidsi, into
the ground. (Quidsi owned Diapers.com.) Then Amazon bought the weakened
company. This approach, like Facebook's acquiring of competitors, is how
John D. Rockefeller built up Standard Oil in the 1870s. It's ``join us
--- or face extermination.'' Likewise, Amazon has admitted to sometimes
selling its smart speaker, Echo, below cost, presumably on the theory
that collecting huge amounts of data on users and securing direct access
to their homes will present an insurmountable barrier to potential
rivals.

Then there's Google. In the company's early days, its documents suggest,
its executives had little interest in YouTube as a product, but they
feared its rise would threaten Google's monopoly on search. The answer?
Once again, buy away the problem --- rather than compete to see who can
offer users the best service. Google purchased YouTube in 2006 for
\$1.65 billion.

The picture that emerges from these documents is not one of steady
entrepreneurial brilliance. Rather, at points where they might have been
vulnerable to hotter, newer start-ups, Big Tech companies have managed
to avoid the rigors of competition. Their two main tools --- buying
their way out of the problem and a willingness to lose money --- are
both made possible by sky-high Wall Street valuations, which go only
higher with acquisitions of competitors, fueling a cycle of enrichment
and consolidation of power. As Mr. Zuckerberg bluntly boasted in an
email, because of its immense wealth Facebook ``can likely always just
buy any competitive start-ups.''

The greater scandal here may be that the federal government has let
these companies get away with this. Dazzled by the mythology of Silicon
Valley and blinded by a fixation with economic price theory (which
suggested that the only potential problem with an acquisition would be
an increase in prices paid by consumers), the government in the 2010s
allowed more than 500 start-up acquisitions to go unchallenged. This
hands-off approach effectively gave tech executives a green light to
consolidate the industry.

The antitrust subcommittee that held last week's hearings may be helping
shake the law out of a long slumber, but the hearings will be little
more than Kabuki theater unless legal complaints are filed and
anticompetitive mergers are stopped. It may be profitable and savvy to
eliminate rivals to maintain a monopoly, but it remains illegal in this
country under the Sherman Antitrust Act and Standard Oil v. United
States. Unless we re-establish that legal fact, Big Tech will continue
to fight dirty and keep on winning.

Tim Wu (\href{https://twitter.com/superwuster}{@superwuster}) is a law
professor at Columbia University, a contributing Opinion writer and the
author, most recently, of ``The Curse of Bigness: Antitrust in the New
Gilded Age.''

\emph{The Times is committed to publishing}
\href{https://www.nytimes3xbfgragh.onion/2019/01/31/opinion/letters/letters-to-editor-new-york-times-women.html}{\emph{a
diversity of letters}} \emph{to the editor. We'd like to hear what you
think about this or any of our articles. Here are some}
\href{https://help.nytimes3xbfgragh.onion/hc/en-us/articles/115014925288-How-to-submit-a-letter-to-the-editor}{\emph{tips}}\emph{.
And here's our email:}
\href{mailto:letters@NYTimes.com}{\emph{letters@NYTimes.com}}\emph{.}

\emph{Follow The New York Times Opinion section on}
\href{https://www.facebookcorewwwi.onion/nytopinion}{\emph{Facebook}}\emph{,}
\href{http://twitter.com/NYTOpinion}{\emph{Twitter (@NYTopinion)}}
\emph{and}
\href{https://www.instagram.com/nytopinion/}{\emph{Instagram}}\emph{.}

Advertisement

\protect\hyperlink{after-bottom}{Continue reading the main story}

\hypertarget{site-index}{%
\subsection{Site Index}\label{site-index}}

\hypertarget{site-information-navigation}{%
\subsection{Site Information
Navigation}\label{site-information-navigation}}

\begin{itemize}
\tightlist
\item
  \href{https://help.nytimes3xbfgragh.onion/hc/en-us/articles/115014792127-Copyright-notice}{©~2020~The
  New York Times Company}
\end{itemize}

\begin{itemize}
\tightlist
\item
  \href{https://www.nytco.com/}{NYTCo}
\item
  \href{https://help.nytimes3xbfgragh.onion/hc/en-us/articles/115015385887-Contact-Us}{Contact
  Us}
\item
  \href{https://www.nytco.com/careers/}{Work with us}
\item
  \href{https://nytmediakit.com/}{Advertise}
\item
  \href{http://www.tbrandstudio.com/}{T Brand Studio}
\item
  \href{https://www.nytimes3xbfgragh.onion/privacy/cookie-policy\#how-do-i-manage-trackers}{Your
  Ad Choices}
\item
  \href{https://www.nytimes3xbfgragh.onion/privacy}{Privacy}
\item
  \href{https://help.nytimes3xbfgragh.onion/hc/en-us/articles/115014893428-Terms-of-service}{Terms
  of Service}
\item
  \href{https://help.nytimes3xbfgragh.onion/hc/en-us/articles/115014893968-Terms-of-sale}{Terms
  of Sale}
\item
  \href{https://spiderbites.nytimes3xbfgragh.onion}{Site Map}
\item
  \href{https://help.nytimes3xbfgragh.onion/hc/en-us}{Help}
\item
  \href{https://www.nytimes3xbfgragh.onion/subscription?campaignId=37WXW}{Subscriptions}
\end{itemize}
