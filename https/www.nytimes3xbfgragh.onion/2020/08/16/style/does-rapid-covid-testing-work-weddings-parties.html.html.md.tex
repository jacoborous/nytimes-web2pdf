Sections

SEARCH

\protect\hyperlink{site-content}{Skip to
content}\protect\hyperlink{site-index}{Skip to site index}

\href{https://www.nytimes3xbfgragh.onion/section/style}{Style}

\href{https://myaccount.nytimes3xbfgragh.onion/auth/login?response_type=cookie\&client_id=vi}{}

\href{https://www.nytimes3xbfgragh.onion/section/todayspaper}{Today's
Paper}

\href{/section/style}{Style}\textbar{}Rapid Testing Is the New Velvet
Rope

\url{https://nyti.ms/344arv9}

\begin{itemize}
\item
\item
\item
\item
\item
\item
\end{itemize}

\href{https://www.nytimes3xbfgragh.onion/spotlight/at-home?action=click\&pgtype=Article\&state=default\&region=TOP_BANNER\&context=at_home_menu}{At
Home}

\begin{itemize}
\tightlist
\item
  \href{https://www.nytimes3xbfgragh.onion/2020/08/14/dining/lobster-salad-recipe.html?action=click\&pgtype=Article\&state=default\&region=TOP_BANNER\&context=at_home_menu}{Make:
  Lobster Salad}
\item
  \href{https://www.nytimes3xbfgragh.onion/2020/08/15/at-home/coronavirus-at-home-quick-exercises.html?action=click\&pgtype=Article\&state=default\&region=TOP_BANNER\&context=at_home_menu}{Sneak
  In: Exercise}
\item
  \href{https://www.nytimes3xbfgragh.onion/interactive/2020/at-home/even-more-reporters-editors-diaries-lists-recommendations.html?action=click\&pgtype=Article\&state=default\&region=TOP_BANNER\&context=at_home_menu}{See:
  Reporters' Obsessions}
\item
  \href{https://www.nytimes3xbfgragh.onion/2020/08/15/at-home/coronavirus-fall-patio-furniture.html?action=click\&pgtype=Article\&state=default\&region=TOP_BANNER\&context=at_home_menu}{Deck
  Out: Your Porch}
\end{itemize}

Advertisement

\protect\hyperlink{after-top}{Continue reading the main story}

Supported by

\protect\hyperlink{after-sponsor}{Continue reading the main story}

The Great Read

\hypertarget{rapid-testing-is-the-new-velvet-rope}{%
\section{Rapid Testing Is the New Velvet
Rope}\label{rapid-testing-is-the-new-velvet-rope}}

Determined to proceed with parties and events this summer, hosts are
adding screenings at the door. But such measures are hardly a guarantee
of safety, medical experts warn.

\includegraphics{https://static01.graylady3jvrrxbe.onion/images/2020/08/16/fashion/15rapidtesting-parties-martini-v1/15rapidtesting-parties-martini-articleLarge.jpg?quality=75\&auto=webp\&disable=upscale}

By Alyson Krueger

\begin{itemize}
\item
  Published Aug. 16, 2020Updated Aug. 17, 2020
\item
  \begin{itemize}
  \item
  \item
  \item
  \item
  \item
  \item
  \end{itemize}
\end{itemize}

\href{https://www.nytimes3xbfgragh.onion/es/2020/08/18/espanol/estilos-de-vida/prueba-rapida-coronavirus.html}{Leer
en español}

Dr. Asma Rashid, who runs a members-only medical concierge service in
the Hamptons, has received some of the most sought-after party
invitations this summer.

``We've gone to these private, private, private events, where they have
me sign a `nothing you see in this house can be leaked' document,'' she
said. ``This is still a party town.''

Dr. Rashid is there to administer rapid or real-time tests for
coronavirus. She performs the procedure --- either a finger prick or a
nose swab --- in the car, and then lets guests into the house only if
their tests come back negative. The entire procedure takes less than 30
minutes. Consider it a pandemic pregame.

Suffolk County still lacks rapid testing infrastructure, and the private
service is expensive: up to \$500 per test, and not all insurance
companies will cover the cost. Most doctors don't even have kits to do
the tests; patients willing to pay can wait up to a week for an
appointment at the offices that have them in New York City.

For that reason many clients book Dr. Rashid in advance when they
anticipate hosting guests for a sleepover, a barbecue or a wedding. But
some summon her at 2 a.m. for a last-minute test or stop by her office
in a panic after attending a crowded gathering.

``Every time there is an event, a protest or Fourth of July celebration,
there is higher demand,'' she said. A busy day came after a drive-in
``Safe \& Sound'' concert, where the chief executive of Goldman Sachs
performed, at the end of July. Concertgoers, who paid \$1,250 per car to
attend, were supposed to stay in their cars, but social media showed
crowds dancing by the stage. ``I can't even tell you how many requests
we got after that,'' Dr. Rashid said.

The event is
\href{https://www.nytimes3xbfgragh.onion/2020/07/27/nyregion/hamptons-chainsmokers-concert-social-distancing.html}{currently
under investigation}.

Dr. Rashid has tripled her staff to keep up with demand for coronavirus
testing this summer. The last Saturday in July she even opened a new
office in Bridgehampton. ``The way I would describe our growth is
exponential,'' she said.

\includegraphics{https://static01.graylady3jvrrxbe.onion/images/2020/08/16/fashion/15RAPIDTESTING-PARTIES2/merlin_175678128_1d7f2e84-bb86-4897-bcc9-a1c471fc021c-articleLarge.jpg?quality=75\&auto=webp\&disable=upscale}

\hypertarget{no-house-parties-before-grandma}{%
\subsection{No House Parties Before
Grandma}\label{no-house-parties-before-grandma}}

While most of the country waits seven to 14 days for coronavirus test
results, a privileged few have access to rapid tests. There are a few
types --- some detect antibodies, others antigens or
\href{https://www.hopkinsmedicine.org/news/newsroom/news-releases/covid-19-test-that-relies-on-viral-genetic-material-gives-false-negative-results-if-used-too-early-in-those-infected}{viral
genetic material} --- but they all provide an answer in under 30
minutes.

Hosts are hiring doctors to screen guests before they attend their
gatherings, or children coming in from out of town for sleepovers. Other
people are getting tests to provide peace of mind after a particularly
wild night. Event companies are offering rapid testing as a service to
clients alongside catering and music. Instagram influencers are even
touting the service.

Still, these rapid tests aren't totally reliable, said Dr. Demetre
Daskalakis, New York City's deputy commissioner of disease control.
``Negatives are not definitive,'' he said. (And there certainly have
been
\href{https://www.nytimes3xbfgragh.onion/2020/08/09/health/covid-testing.html}{false
positives}.)

``No test is 100 percent,'' Dr. Rashid said. ``A negative test does not
preclude one to not be carrying the virus.''

Indeed, one reason rapid tests aren't in widespread use is that they
require additional testing to confirm. ``We have to retest all of our
negatives, so you're doing two tests for everyone who is negative,''
said Dr. Daskalakis. ``It's a resource issue.''

He also warned that the virus can take some time to show up in a test
result; though some test positive 48 hours after exposure, the two-week
possible incubation period that has dictated quarantine is generally
accepted. So if you were exposed to the virus even 10 days before your
test, the outcome is still uncertain. ``You can't go to a house party
the week before you see Grandma,'' Dr. Daskalakis said. ``That test
doesn't matter.''

Ryan Choura, the founder of Choura, an event and experience production
company in Torrance, Calif., that arranges the tenting and furniture for
the U.S. Open golf tournament and the BeachLife Festival, believes so
strongly that all events should incorporate rapid testing that he
created an arm of his company to do it.

``There is no question that you now have to have rapid testing as a
component to every event no matter what,'' Mr. Choura said.

It's not an easy feat to pull off. ``Temperature is a major issue. You
have to keep tests under 80 degrees,'' he said. ``You need to put up
some kind of tent, like a catering tent.'' He added, ``We have a
100-point checklist for anything we want to build.''

Still, on July 23, the company staged its first event for 90 attendees
(and 30 vendors) using the Rapid Quidel Corporation Sofia SARS Antigen
Test. The good news: Not a single vendor or attendee tested positive.
The bad news: It was a bit of a buzz kill.

Joie Shettler, 51, a singer and co-owner of Rebel Music Entertainment
who lives in Redondo Beach, Calif., attended the event, which included
bottled cocktails and panels on the future of the events industry, after
learning about it on Facebook. She ended up having fun, but the
beginning felt clinical.

``We had to fill out a form beforehand, so that they would be ready for
us when we arrived. We also had to book a time frame of when to
arrive,'' Ms. Shettler said. ``We had instructions to drive up in our
cars, where we would be tested. From there we were instructed to stay in
our car, and they would text us our results within 20 minutes.''

``I was a little worried while waiting for the results,'' she added.
``Both Alysha, my friend, and I were squealing when the results came in.
We were both negative! And we were excited to finally enter the event.''

Mr. Choura said he is planning a 1,000-person live music event, ``most
likely in another state,'' for which every person will be rapid-tested
in the next 30 to 60 days. He is still figuring out the details, but
promises it won't be a drive-in. ``Right now people are bragging about
doing live music, and people are sitting in their cars,'' he said. ``I
don't want to sit in my car and listen to live music. And by the way,
just because you are sitting in your car doesn't mean it's safe.''

Image

Show me your antibodies: The Lodge's guest area.Credit...Joe Carrotta
for The New York Times

\hypertarget{humblebragging-hugging-and-hashtagging}{%
\subsection{Humblebragging, Hugging and
Hashtagging}\label{humblebragging-hugging-and-hashtagging}}

One of the easier places to get a rapid coronavirus test is the Surf
Lodge, a hotel and restaurant in Montauk, N.Y., known for hosting and
entertaining celebrities including John Legend and Bon Jovi.

Jayma Cardoso, one of the hotel's owners, pays Dr. Seth Gordon, her
son's pediatrician, to test all of her employees weekly. ``He has his
own Sofia 2 machine by Quidel so he can do the test'' --- an antigen
test --- ``and get the results very quickly, in 15 minutes,'' she said.

Once a week, the doctor sets up a testing site on the sand-filled deck
overlooking Fort Pond. Raya O'Neal, 24, the hotel's director of
communications, who lives in East Hampton, says while she hates the
test, which takes about 10 seconds, she understands it's an exclusive
perk. ``I have some friends and family members who think I'm too fancy
for them now,'' she said.

Ms. O'Neal feels safer living with her parents knowing she regularly
tests negative for the virus. But it also emboldens her socially. ``I
don't feel as worried as I do about eating or going to public places,''
she said. ``Once I saw my closest friends after not seeing them for
months, and I did the humblebrag about being tested weekly and being
negative, and I just had to hug them.''

But as any public-health expert will tell you, individual test results
are not an all-access pass to Life as It Was Before. ``There is a false
confidence you get when you use a test for social decisions,'' Dr.
Daskalakis said. ``This is one of those things I lose sleep over.''

Nonetheless, receiving rapid testing for the virus has become a mark of
status and, ergo, a trending topic on social media.

Tasha Todd, 40, is a medical assistant in Dallas. When her former
office, a concierge medical group, first received the rapid testing kit,
she posted about it on Instagram, where she has
\href{https://www.instagram.com/fit.tash36/}{nearly 28,000 followers},
to hype up the service. ``I wanted to try to bring more business into
the company,'' Ms. Todd said. ``Not that we could have handled much more
volume. We were seeing 30 people a day, 25 of which were in for Covid
testing.''

``I got a lot of feedback,'' she said. ``A lot of people were messaging
about the prices, where the office was, what the difference was between
that and a regular test, and how quickly the results come in.'' Her
office charges \$150 for a test, but she knows of other clinics in
Dallas that charge \$500 or more.

Ms. Todd said she felt frustrated that many of her followers wouldn't be
able to afford one. ``I would say rapid testing right now is for the
rich. It's too expensive,'' she said. ``Who has 150 to 500 dollars just
lying around in the middle of the recession?''

Image

Forget the bouncer. Private doctors like Dr. Seth Gordon, here
administering a rapid test to Raya O'Neal, director of communications at
the Surf Lodge, are the new essential party personnel.Credit...Joe
Carrotta for The New York Times

Image

The electric slide: A machine delivers the results.Credit...Joe Carrotta
for The New York Times

\hypertarget{temperature-checks-and-colored-leis}{%
\subsection{Temperature Checks and Colored
Leis}\label{temperature-checks-and-colored-leis}}

Some bars and restaurants have tried offering rapid testing to their
customers, an act that brings in publicity. This practice can backfire,
however.

For its reopening weekend at the end of June, Ravel Hotel in the Long
Island City neighborhood of New York worked with an outside doctor to
give all arrivals a test before enter (charging \$35 for those with
insurance and \$50 for those without.) Photos emerged on social media
and then
i\href{https://gothamist.com/arts-entertainment/profundo-ravel-covid-test-rooftop-pandemic-pool-parties-rage-lic}{n
Gothamist} of people who had all tested negative, partying and mingling
closely without masks, producing the inevitable tsks.

Nicole Milazzo, a spokeswoman for the hotel, said that it is no longer
testing guests, after concerns expressed by the mayor's office of
nightlife about ``creating a false sense of security,'' but is
``continuing to follow all rules and regulations put forward by the
C.D.C. and New York City,'' including distancing tables and cabanas by
more than six feet, and providing masks and sanitizer.

Other gatekeepers, unable to get their hands on rapid testing to begin
with, or unwilling to shoulder the cost, have implemented alternate
screening methods.

Thuyen Nguyen, a facialist who has worked with celebrities including the
actress Michelle Williams, is reopening his salon in East Hampton later
this summer. Every person who walks in will get body-scanned by Antlia
Systems thermal technology, which checks for a temperature in under a
second. A little box positioned at reception does all the screening
without touching the person. (Such screens have been criticized by
medical experts because infections can occur without symptoms, including
fevers, and people's baseline temperatures vary.)

Reports on social media have also surfaced of colored bracelets or leis
to moderate contact at weddings, like the spotlight parties from college
where you wore green if you were single, red if you were taken, and
yellow if you weren't sure. Now the idea is green if you are OK with
hugs and high fives, yellow if you are fine with talking but not
touching, and red if you are totally keeping your distance from
everybody.
(``\href{https://twitter.com/Erin_Alexa_/status/1292476550399561736/}{Scares
me to my core},'' tweeted one bride-to-be, of the illusory confidence
such accessories might impart.)

As private citizens improvise at their peril, local governments continue
to try and make rapid testing more accessible. New York City's
Department of Health, for example, put a pop-up site in the Bronx and
now Brooklyn's Sunset Park neighborhood, where
\href{https://www.nydailynews.com/news/politics/ny-nyc-mayor-de-blasio-covid-sunset-park-brooklyn-outbreak-20200812-ems5msaphfcf5is5dq2lmu453e-story.html}{Mayor
Bill de Blasio recently warned of an uptick in coronavirus cases}. The
idea is to identify as many positives early on so the city can help
offer isolation guidance.

``There is the ability of the folks who have and the folks who don't,''
said Dr. Daskalakis. ``We are very consciously pushing the rapid test in
environments where people aren't resource-rich.'' The city plans to
bring the technology to nine clinics that previously focused on sexual
health, transforming them into
\href{https://www.nytimes3xbfgragh.onion/2020/07/23/nyregion/coronavirus-testing-nyc.html}{coronavirus
testing sites}.

Even New Yorkers from more affluent neighborhoods are flocking to these
free testing sites to get quick results.

The first weekend in August Mary Ann Mackey, 31, who lives in Park Slope
and works for the Manhattan Borough President's Office, went to Brooklyn
on a Sunday afternoon to get a rapid test before visiting her parents
for the first time since the pandemic started.

Ms. Mackey arrived at 12:30 and was there all afternoon. She was
surprised by the line. ``By the time I had been through the line it had
been three hours, and the whole block was still full of people,'' she
said. ``Everyone was remarkably patient and calm. People had books and
magazines out. Some people were chatting. I overheard a few people talk
about how they had waited in a long time elsewhere and were excited to
get it over with and know their results.''

This experience felt like a luxury. ``This feels like a weird thing to
say, but it was kind of a cool experience,'' Ms. Mackey said. ``It felt
kind of like a privilege to have results.''

Advertisement

\protect\hyperlink{after-bottom}{Continue reading the main story}

\hypertarget{site-index}{%
\subsection{Site Index}\label{site-index}}

\hypertarget{site-information-navigation}{%
\subsection{Site Information
Navigation}\label{site-information-navigation}}

\begin{itemize}
\tightlist
\item
  \href{https://help.nytimes3xbfgragh.onion/hc/en-us/articles/115014792127-Copyright-notice}{©~2020~The
  New York Times Company}
\end{itemize}

\begin{itemize}
\tightlist
\item
  \href{https://www.nytco.com/}{NYTCo}
\item
  \href{https://help.nytimes3xbfgragh.onion/hc/en-us/articles/115015385887-Contact-Us}{Contact
  Us}
\item
  \href{https://www.nytco.com/careers/}{Work with us}
\item
  \href{https://nytmediakit.com/}{Advertise}
\item
  \href{http://www.tbrandstudio.com/}{T Brand Studio}
\item
  \href{https://www.nytimes3xbfgragh.onion/privacy/cookie-policy\#how-do-i-manage-trackers}{Your
  Ad Choices}
\item
  \href{https://www.nytimes3xbfgragh.onion/privacy}{Privacy}
\item
  \href{https://help.nytimes3xbfgragh.onion/hc/en-us/articles/115014893428-Terms-of-service}{Terms
  of Service}
\item
  \href{https://help.nytimes3xbfgragh.onion/hc/en-us/articles/115014893968-Terms-of-sale}{Terms
  of Sale}
\item
  \href{https://spiderbites.nytimes3xbfgragh.onion}{Site Map}
\item
  \href{https://help.nytimes3xbfgragh.onion/hc/en-us}{Help}
\item
  \href{https://www.nytimes3xbfgragh.onion/subscription?campaignId=37WXW}{Subscriptions}
\end{itemize}
