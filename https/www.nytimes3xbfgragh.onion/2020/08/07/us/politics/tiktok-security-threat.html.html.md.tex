Sections

SEARCH

\protect\hyperlink{site-content}{Skip to
content}\protect\hyperlink{site-index}{Skip to site index}

\href{https://www.nytimes3xbfgragh.onion/section/politics}{Politics}

\href{https://myaccount.nytimes3xbfgragh.onion/auth/login?response_type=cookie\&client_id=vi}{}

\href{https://www.nytimes3xbfgragh.onion/section/todayspaper}{Today's
Paper}

\href{/section/politics}{Politics}\textbar{}Is TikTok More of a
Parenting Problem Than a Security Threat?

\href{https://nyti.ms/2C71fKQ}{https://nyti.ms/2C71fKQ}

\begin{itemize}
\item
\item
\item
\item
\item
\item
\end{itemize}

Advertisement

\protect\hyperlink{after-top}{Continue reading the main story}

Supported by

\protect\hyperlink{after-sponsor}{Continue reading the main story}

White House Memo

\hypertarget{is-tiktok-more-of-a-parenting-problem-than-a-security-threat}{%
\section{Is TikTok More of a Parenting Problem Than a Security
Threat?}\label{is-tiktok-more-of-a-parenting-problem-than-a-security-threat}}

Even as the White House moves against the Chinese social media app, the
intelligence agencies do not see it as a major issue along the lines of
Huawei.

\includegraphics{https://static01.graylady3jvrrxbe.onion/images/2020/08/07/us/politics/07dc-tiktok01/merlin_175379841_7dd8e79d-d3de-46d9-84e6-e7b0c15e8fb4-articleLarge.jpg?quality=75\&auto=webp\&disable=upscale}

\href{https://www.nytimes3xbfgragh.onion/by/david-e-sanger}{\includegraphics{https://static01.graylady3jvrrxbe.onion/images/2018/10/03/multimedia/author-david-e-sanger/author-david-e-sanger-thumbLarge.png}}\href{https://www.nytimes3xbfgragh.onion/by/julian-e-barnes}{\includegraphics{https://static01.graylady3jvrrxbe.onion/images/2019/12/13/reader-center/author-julian-barnes/author-julian-barnes-thumbLarge.png}}

By \href{https://www.nytimes3xbfgragh.onion/by/david-e-sanger}{David E.
Sanger} and
\href{https://www.nytimes3xbfgragh.onion/by/julian-e-barnes}{Julian E.
Barnes}

\begin{itemize}
\item
  Aug. 7, 2020Updated 5:57 p.m. ET
\item
  \begin{itemize}
  \item
  \item
  \item
  \item
  \item
  \item
  \end{itemize}
\end{itemize}

TikTok has long presented a parenting problem, as millions of Americans
raising preteens and teenagers distracted by its viral videos can
attest. But when the C.I.A. was asked recently to assess whether it was
also a national security problem, the answer that came back was highly
equivocal.

Yes, the agency's analysts told the White House, it is possible that the
Chinese intelligence authorities could intercept data or use the app to
bore into smartphones. But there is no evidence they have done so,
despite the calls from President Trump and Secretary of State Mike
Pompeo to neutralize a threat from the app's presence on millions of
American devices.

It made little difference. When Mr. Trump
\href{https://www.nytimes3xbfgragh.onion/2020/08/06/technology/trump-wechat-tiktok-china.html}{issued
an executive order on Thursday} that would
\href{https://www.whitehouse.gov/presidential-actions/executive-order-addressing-threat-posed-tiktok/}{effectively
ban TikTok} from operating in the United States in 45 days --- part of
an effort to force a sale of the app to an American company, most likely
Microsoft --- he declared it threatened ``the national security, foreign
policy and economy of the United States.''

In a surprise addition, he issued a
\href{https://www.whitehouse.gov/presidential-actions/executive-order-addressing-threat-posed-wechat/}{similar
ban on WeChat}, a Chinese social media app on which millions of people,
largely outside the United States, conduct everyday conversations and
financial transactions.

There is no doubt the actions by the administration are the toughest
since it took steps last year to block the use of Huawei
telecommunications equipment by American companies. And there is a fear
running through Silicon Valley that Mr. Trump --- seen as eager to
punish the Chinese, and as angry at
\href{https://www.nytimes3xbfgragh.onion/2020/05/27/arts/television/trump-sarah-cooper.html}{viral
TikTok videos} that mock him and at the app's role in
\href{https://www.nytimes3xbfgragh.onion/2020/06/21/style/tiktok-trump-rally-tulsa.html}{deterring
attendance at his rally in June in Tulsa} --- is opening the door for
countries around the world to declare Facebook and Google to be similar
threats to their own security.

Measuring national security threats has always been tinged as much by
politics as intelligence assessments: Think of John F. Kennedy's warning
of the ``missile gap'' with the Soviets in the 1960 election, or George
W. Bush's declaration of an imminent Iraqi nuclear ability on his
ill-fated march to war 17 years ago.

Mr. Trump's warning about the Chinese threat has been expanding: Days
before the executive order, the State Department announced a
\href{https://www.state.gov/announcing-the-expansion-of-the-clean-network-to-safeguard-americas-assets/}{``Clean
Network'' initiative}, threatening to ban not only apps, but Chinese
undersea cables, telecommunications firms that have operated in the
United States for years and businesses that store information in the
cloud.

Sorting out the real threats from the imagined ones is as complex as the
design of the internet. But the threat TikTok poses, intelligence
officials say, pales to the one created by Huawei, the Chinese
telecommunications giant that was seeking to wire up the United States,
Europe and much of the developing world, using the transition to 5G
networks to control global communications.

``Is TikTok a problem?'' Senator Mark Warner, Democrat of Virginia and
the ranking member of the Senate Intelligence Committee, said on
Thursday at the Aspen Security Forum.

``Yes, but on the hierarchy of problem we are talking about, an app that
allows you to make funny videos'' does not really rank, he said.

Huawei is a far deeper concern, Mr. Warner said, because it seeks to
rebuild internet infrastructure --- putting in the switches and cell
towers on which American communications run. In the 5G era, that will
also be the infrastructure on which manufacturing, gas supply lines,
agriculture and self-driving cars operate. In a time of conflict, China
could, in theory, order the systems shut down or subtly manipulated.

Even so, through much of last year, Mr. Trump talked about Huawei as a
card in trade negotiations, undercutting his own aides, who were
pressing allies to ban China's products.

That has changed since the president began blaming China for spreading
the coronavirus. And recent events --- like the drying up of Chinese
supplies for dealing with the pandemic and the crackdown by China
against the pro-democracy movement in Hong Kong --- have persuaded
allies like Britain to reverse course and ban Huawei's presence in their
networks.

But allies seem unlikely to get as exercised about TikTok. No one is
saying that it can bring the American economy or the NATO alliance to
its knees --- though it can lead to a lot of dinner table arguments over
the time and attention it has sucked away from other things.

That does not mean there is no threat. There are worries about the
location-identifying elements of TikTok, which is why the American
military and intelligence agencies have banned it from official phones
and discouraged its use on personal phones. There is concern about what
other data it could, in theory, pull from the phones.

The White House has never discussed the underlying intelligence. The
most recent C.I.A. assessment, the latest of a series, has been kept
classified, though it has been widely circulated and much discussed in
Washington. Now that intelligence assessment is playing a significant
role in the debate over whether the way to solve this problem is for
Microsoft, or another
\href{https://www.nytimes3xbfgragh.onion/2020/08/03/business/economy/trump-tiktok-china-business.html}{``very
American'' company, to use Mr. Trump's words}, to take ownership.

\includegraphics{https://static01.graylady3jvrrxbe.onion/images/2020/08/07/us/politics/07dc-tiktok02/07dc-tiktok02-articleLarge.jpg?quality=75\&auto=webp\&disable=upscale}

But the problem extends well beyond TikTok.

Mr. Trump's move and the Clean Networks initiative are based on a belief
that the United States can control its internet environment and keep it
out of Chinese hands, including smartphones and the fiber-optic cables
that carry data across the Pacific.

Sue Gordon, the former deputy director of national intelligence, who
left last year after three decades at the C.I.A. and in other posts,
said that even under the best of circumstances, a huge proportion of
American data would flow across Chinese networks.

``That's the reality of an interconnected world,'' she said. ``We have
to learn how to live in dirty networks.''

That will remain the reality even if TikTok's operations are sold to
Microsoft. To ``clean'' the app, the company would most likely have to
rewrite almost all of its software. And the existing code has already
been downloaded more than a 160 million times in the United States, and
two billion times worldwide.

If Huawei's installation of equipment around the world poses a high
risk, and TikTok a comparatively low one, WeChat, according to other
officials, presents a more acute --- though limited --- problem.

American officials believe that China has the ability to monitor
communications on the communications platform.

WeChat is not widely used in the United States, except for by one key
group --- Chinese-born software engineers in Silicon Valley and other
high-tech workforces, according to American officials. They use WeChat
to collaborate on tough mathematical, software or engineering problems,
trading solutions back and forth. Proprietary data can be scooped up by
Chinese intelligence services, an American official said.

A close reading of American statements makes it clear that, so far, much
of the risk attributed to the Chinese apps is theoretical. Mr. Trump's
executive order on TikTok was carefully couched in the future tense.

``This data collection threatens to allow the Chinese Communist Party
access to Americans' personal and proprietary information ---
potentially allowing China to track the locations of Federal employees
and contractors, build dossiers of personal information for blackmail,
and conduct corporate espionage,'' the order said.

Intelligence officials insisted the C.I.A. assessment did not mean the
app was safe or that installing it on a phone was wise. One intelligence
official said he had warned his own family members not to install it,
and some lawmakers also see TikTok more as a surveillance program than a
way to watch dance videos.

``It's all fun and games until the communists start data harvesting,''
Senator Ben Sasse, Republican of Nebraska, said in a statement.

But the threat of TikTok has to be kept in perspective, especially given
all of the personal information being sucked up, sold and shared by
smartphone apps.

Christoph Hebeisen, the director of security intelligence research at
Lookout, a company that focuses on the security of mobile devices,
examined the TikTok app and came to a conclusion close to that of the
intelligence agencies: The Chinese government does not seem to have
access to the company's data on American users, but it could probably
get it if it wanted to.

But he said that he thought that was not too much different than if the
American government obtained a Foreign Intelligence Surveillance Act
warrant to get the data of an American social media company.

``If you are China, you probably don't want your government officials to
have Facebook on their phones,'' he said, ``and if you are the U.S., you
probably don't want your government officials to have TikTok.''

Advertisement

\protect\hyperlink{after-bottom}{Continue reading the main story}

\hypertarget{site-index}{%
\subsection{Site Index}\label{site-index}}

\hypertarget{site-information-navigation}{%
\subsection{Site Information
Navigation}\label{site-information-navigation}}

\begin{itemize}
\tightlist
\item
  \href{https://help.nytimes3xbfgragh.onion/hc/en-us/articles/115014792127-Copyright-notice}{©~2020~The
  New York Times Company}
\end{itemize}

\begin{itemize}
\tightlist
\item
  \href{https://www.nytco.com/}{NYTCo}
\item
  \href{https://help.nytimes3xbfgragh.onion/hc/en-us/articles/115015385887-Contact-Us}{Contact
  Us}
\item
  \href{https://www.nytco.com/careers/}{Work with us}
\item
  \href{https://nytmediakit.com/}{Advertise}
\item
  \href{http://www.tbrandstudio.com/}{T Brand Studio}
\item
  \href{https://www.nytimes3xbfgragh.onion/privacy/cookie-policy\#how-do-i-manage-trackers}{Your
  Ad Choices}
\item
  \href{https://www.nytimes3xbfgragh.onion/privacy}{Privacy}
\item
  \href{https://help.nytimes3xbfgragh.onion/hc/en-us/articles/115014893428-Terms-of-service}{Terms
  of Service}
\item
  \href{https://help.nytimes3xbfgragh.onion/hc/en-us/articles/115014893968-Terms-of-sale}{Terms
  of Sale}
\item
  \href{https://spiderbites.nytimes3xbfgragh.onion}{Site Map}
\item
  \href{https://help.nytimes3xbfgragh.onion/hc/en-us}{Help}
\item
  \href{https://www.nytimes3xbfgragh.onion/subscription?campaignId=37WXW}{Subscriptions}
\end{itemize}
