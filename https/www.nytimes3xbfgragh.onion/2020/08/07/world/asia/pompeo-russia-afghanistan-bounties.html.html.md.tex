Sections

SEARCH

\protect\hyperlink{site-content}{Skip to
content}\protect\hyperlink{site-index}{Skip to site index}

\href{https://www.nytimes3xbfgragh.onion/section/world/asia}{Asia
Pacific}

\href{https://myaccount.nytimes3xbfgragh.onion/auth/login?response_type=cookie\&client_id=vi}{}

\href{https://www.nytimes3xbfgragh.onion/section/todayspaper}{Today's
Paper}

\href{/section/world/asia}{Asia Pacific}\textbar{}Pompeo Warned Russia
Against Bounties on U.S. Troops in Afghanistan

\href{https://nyti.ms/3fzI1v3}{https://nyti.ms/3fzI1v3}

\begin{itemize}
\item
\item
\item
\item
\item
\item
\end{itemize}

Advertisement

\protect\hyperlink{after-top}{Continue reading the main story}

Supported by

\protect\hyperlink{after-sponsor}{Continue reading the main story}

\hypertarget{pompeo-warned-russia-against-bounties-on-us-troops-in-afghanistan}{%
\section{Pompeo Warned Russia Against Bounties on U.S. Troops in
Afghanistan}\label{pompeo-warned-russia-against-bounties-on-us-troops-in-afghanistan}}

Secretary of State Mike Pompeo is said to have sternly discussed payouts
and red lines in a telephone call with Sergey V. Lavrov, Russia's
foreign minister.

\includegraphics{https://static01.graylady3jvrrxbe.onion/images/2020/07/30/us/politics/00dc-bounties/merlin_175110051_aed54ec6-7fc4-4fc3-beca-a13c283adb13-articleLarge.jpg?quality=75\&auto=webp\&disable=upscale}

\href{https://www.nytimes3xbfgragh.onion/by/edward-wong}{\includegraphics{https://static01.graylady3jvrrxbe.onion/images/2018/09/24/multimedia/author-edward-wong/author-edward-wong-thumbLarge-v5.png}}\href{https://www.nytimes3xbfgragh.onion/by/eric-schmitt}{\includegraphics{https://static01.graylady3jvrrxbe.onion/images/2018/06/12/multimedia/author-eric-schmitt/author-eric-schmitt-thumbLarge-v2.png}}

By \href{https://www.nytimes3xbfgragh.onion/by/edward-wong}{Edward Wong}
and \href{https://www.nytimes3xbfgragh.onion/by/eric-schmitt}{Eric
Schmitt}

\begin{itemize}
\item
  Aug. 7, 2020Updated 5:18 p.m. ET
\item
  \begin{itemize}
  \item
  \item
  \item
  \item
  \item
  \item
  \end{itemize}
\end{itemize}

WASHINGTON --- Secretary of State Mike Pompeo has warned Russia's
foreign minister against Moscow
\href{https://www.nytimes3xbfgragh.onion/2020/06/26/us/politics/russia-afghanistan-bounties.html}{paying
bounties} to Taliban-linked militants and other Afghan fighters for
killing American service members, U.S. officials said.

Mr. Pompeo's warning is the first known rebuke from a senior American
official to Russia over the bounties program, and it runs counter to
President Trump's insistence that the intelligence from U.S. government
agencies over the matter is a ``hoax.'' The action indicates that Mr.
Pompeo, who previously served as Mr. Trump's C.I.A. director, believes
the intelligence warranted a stern message.

Mr. Pompeo delivered the warning in a call on July 13 with the minister,
Sergey V. Lavrov, choosing to do so during a conversation that,
officially, was about an unrelated topic --- the possibility of a
meeting of the five permanent members of the United Nations Security
Council, the U.S. officials said in the past week.

The secretary of state did not explicitly point to the covert
\href{https://www.nytimes3xbfgragh.onion/2020/06/26/us/politics/russia-afghanistan-bounties.html}{bounties
scheme} organized
\href{https://www.nytimes3xbfgragh.onion/2020/07/01/world/asia/afghan-russia-bounty-middleman.html}{by
a Russian military intelligence unit} that was first reported in late
June by The New York Times, most likely because the details of what
American intelligence has learned and how it gathered the information
remain classified, one of the officials said. In public, Mr. Pompeo has
carefully avoided answering direct questions about American intelligence
on the Russian bounties. But late last month in congressional testimony,
he said broadly that he had raised with Mr. Lavrov ``all of the issues''
that put American interests at risk.

In the call, Mr. Pompeo made it clear to Mr. Lavrov in language about
payouts and red lines that the United States was strongly opposed to the
program, the official said, adding that the secretary of state had been
livid about what the intelligence had said about the bounties.

The American officials who spoke about Mr. Pompeo's call did so on the
condition of anonymity because of the sensitivity of the matter. Mr.
Pompeo and the State Department have been careful not to reveal any
details of actions he might have taken based on the intelligence over
the bounties. That is perhaps because of both the classified material
and to avoid potential fury from Mr. Trump, who has strongly dismissed
reports of the intelligence and has tried to cultivate a friendship with
President Vladimir V. Putin of Russia.

Mr. Pompeo's private move is the latest example of a common occurrence
in the administration: American officials quietly carrying out actions
that are at odds with Mr. Trump's statements and his stance on important
issues.

The Times reported that senior American officials, including some on the
White House National Security Council, had debated for months over what
to do about the Russian effort. Russian officials have
\href{https://www.nytimes3xbfgragh.onion/2020/07/03/world/europe/russia-bounties-putin-afghanistan.html}{denounced
the reports} on the bounties as lies.

Mr. Trump said last week that
\href{https://www.nytimes3xbfgragh.onion/2020/07/29/us/politics/trump-putin-bounties.html}{he
did not mention} American intelligence assessments of the bounties
program when he spoke this month with Mr. Putin. ``That was a phone call
to discuss other things, and, frankly, that's an issue that many people
said was fake news,'' he said in an interview with ``Axios on HBO,''
even though the C.I.A. had placed medium confidence in the assessment.

Mr. Trump also brushed off past declarations by his own commanders that
Russia had been providing weapons and cash to the Taliban for years, but
the commanders did not specifically cite any bounty program.

He later
\href{https://www.whitehouse.gov/briefings-statements/remarks-president-trump-covid-19-response-storm-preparedness-roundtable-belleair-fl/}{told
reporters} during a trip to Florida that the intelligence was ``another
Russia hoax.''

``They've been giving me the Russia hoax --- Shifty Schiff, all these
characters --- from the day I got here,'' he said, using his nickname
for Representative Adam B. Schiff of California, the top Democrat on the
House Intelligence Committee.

Mr. Schiff complained on Friday that U.S. intelligence officials had so
far failed to provide more detailed information about the suspected
Russian payments, as lawmakers were promised in early July. ``We have
yet to receive this information,'' Mr. Schiff said in a statement.

The bounties operation is overseen by a Russian military intelligence
unit, U.S. officials said. An obvious channel the Americans could use to
address the issue is an important one between the top military officers
in both nations, said
\href{https://carnegieendowment.org/experts/824}{Andrew S. Weiss}, a
former American official and Russia expert at the Carnegie Endowment for
International Peace. But it is not known if the current officers, Gen.
Mark A. Milley, the chairman of the Joint Chiefs of Staff, and Gen.
Valery V. Gerasimov, the chief of the Russian general staff, have spoken
of the bounties via that channel.

``Delivering a very clear and credible message that we will use all
means to protect our people is the only thing that gets the Russians'
attention,'' Mr. Weiss said. ``Sadly, neither Pompeo nor Trump are
credible messengers in that department.''

Mr. Trump and Mr. Putin have spoken eight times this year, according to
a \href{http://kremlin.ru/catalog/persons/498/events/61270}{Kremlin
list} of the Russian president's diplomatic activity --- twice as many
times as they spoke in all of 2019.

Soon after The Times first reported the intelligence assessments about
the suspected Russian bounties on June 26, the State Department prepared
a series of talking points warning Moscow against making payments to
Taliban-linked groups in Afghanistan to kill American soldiers there,
according to a U.S. official, who spoke on the condition of anonymity to
discuss internal administration deliberations on the matter.

The talking points were to be used by top officials in the State
Department, including Mr. Pompeo and John J. Sullivan, the United States
ambassador to Russia, in their discussions with top Russian officials.
But the U.S. official did not know whether Mr. Pompeo or Mr. Sullivan
used the talking points and, if so, how and when they were conveyed.

The comments last week by Mr. Trump were the latest instance of the
president evading a potential point of tension with Mr. Putin, with whom
he has publicly sided in the past in
\href{https://www.nytimes3xbfgragh.onion/2018/07/16/world/europe/trump-putin-election-intelligence.html}{criticizing
his own intelligence agencies' findings} on Russian election
interference in 2016.

Since the intelligence on the bounties became public, the White House
has been criticized for inaction. Intelligence officers put the
information about Russia and the possible bounties
\href{https://www.nytimes3xbfgragh.onion/2020/06/29/us/politics/russian-bounty-trump.html}{in
the President's Daily Brief in February}, though Mr. Trump has said he
was never personally told about it. Mr. Trump rarely reads the written
briefing and instead listens to a verbal summary from a C.I.A. officer.

Mr. Pompeo is the most outwardly loyal of Mr. Trump's cabinet members,
and he makes great effort to avoid the appearance of any differences of
thought or action between them.

But unlike Mr. Trump, he has publicly condemned some of Russia's
policies, including its
\href{https://www.nytimes3xbfgragh.onion/2020/01/31/world/europe/ukraine-pompeo-zelensky-trump.html}{shadow
war in Ukraine} and seizure of Crimea.

Mr. Pompeo is considering a run for president in 2024, his associates
say. His aggressive statements against Russia and China could in part be
an effort to reinforce an image of himself as a traditional Republican
foreign policy hawk, which Mr. Trump is not.

On Afghanistan, Mr. Pompeo has been trying to prod the Afghan government
and the Taliban to
\href{https://www.nytimes3xbfgragh.onion/2020/07/28/world/asia/afghanistan-cease-fire-taliban.html}{negotiate
a peace agreement} that would allow Mr. Trump to withdraw thousands of
American troops.

He has avoided directly answering public questions about specific
actions the State Department might have taken in response to the
intelligence reports on the Russian bounties.

During a Senate Foreign Relations Committee hearing on July 30, Senator
Jeanne Shaheen, Democrat of New Hampshire,
\href{https://www.youtube.com/watch?v=7TeOLjqd5x8\&feature=youtu.be}{asked
Mr. Pompeo} what the government should be doing to push back against
Russia on this issue. He answered in general terms.

He said that ``intelligence collection'' could help prevent the
``tactical event'' --- apparently meaning the killing of American
service members --- from occurring, and that ``our diplomats do make
very clear our expectations and set a set of red lines.''

Asked whether Mr. Trump should condemn the Russian actions to Mr. Putin,
Mr. Pompeo said, ``I'll leave it up to the president what he wants to
say to other leaders.''

Ms. Shaheen cited a
\href{https://www.thedailybeast.com/us-warns-russia-on-bounties-while-trump-cries-fake-news?ref=author}{Daily
Beast article} published the previous day that said the State Department
had warned Russia that there would be repercussions if it paid the
bounties. The report offered no details on how the agency had delivered
the warning, and Mr. Pompeo did not refer to it in his answers.

Also during the hearing, Mr. Pompeo said that he had ``raised all of the
issues that put any American interests at risk'' with Mr. Lavrov,
``whether it's our soldiers on the ground in Syria, soldiers on the
ground in Afghanistan, the activities that are taking place in Libya,
the actions in Ukraine.''

A public summary from the State Department about the July 13 call
between Mr. Pompeo and Mr. Lavrov said the two had ``discussed issues of
mutual concern, including Afghanistan.''

The department declined to comment about the details of the call.

On July 15, Christopher Robinson, a deputy assistant secretary in the
department's Europe and Eurasia bureau, said at a news conference: ``I
think the secretary's been very clear that we have a frank conversation
with Russia with regard to Afghanistan. We have taken steps to address
all threats, and we will continue to do so. Whether it's from Russia or
Iran or any other actor, we will take the steps necessary.''

In response to a senator's question about the bounties during a
\href{https://www.c-span.org/video/?473984-1/senate-foreign-relations-hearing-us-china-relations}{congressional
hearing on China on July 22}, Stephen E. Biegun, the deputy secretary of
state, said that any such actions by Russia ``will be met with the most
severe circumstances most severe consequences.''

He also said that any knowledge of something like the bounties program
would result in at least two things: a notification to commanders that
would be accompanied by the military taking necessary steps to protect
American service members, and ``the subject of a conversation between
very senior officials, and both governments, in no uncertain terms.''

Mr. Biegun did not explicitly confirm the intelligence on the program,
saying he had to be mindful of the fact that the information ``comes
from sensitive sources and methods.''

Robert C. O'Brien, the White House national security adviser, echoed Mr.
Biegun in an
\href{https://www.washingtonpost.com/opinions/2020/08/02/president-trump-is-committed-defending-us-russia-knows-it/}{op-ed
article published last Sunday by The Washington Post} that listed the
Trump administration's attempts to counter various activities by Russia.

``If recently reported allegations of Russian malign activity toward
Americans in Afghanistan prove true,'' he wrote, ``Russia knows from
experience that it will pay a price --- even if that price never becomes
public.''

Advertisement

\protect\hyperlink{after-bottom}{Continue reading the main story}

\hypertarget{site-index}{%
\subsection{Site Index}\label{site-index}}

\hypertarget{site-information-navigation}{%
\subsection{Site Information
Navigation}\label{site-information-navigation}}

\begin{itemize}
\tightlist
\item
  \href{https://help.nytimes3xbfgragh.onion/hc/en-us/articles/115014792127-Copyright-notice}{©~2020~The
  New York Times Company}
\end{itemize}

\begin{itemize}
\tightlist
\item
  \href{https://www.nytco.com/}{NYTCo}
\item
  \href{https://help.nytimes3xbfgragh.onion/hc/en-us/articles/115015385887-Contact-Us}{Contact
  Us}
\item
  \href{https://www.nytco.com/careers/}{Work with us}
\item
  \href{https://nytmediakit.com/}{Advertise}
\item
  \href{http://www.tbrandstudio.com/}{T Brand Studio}
\item
  \href{https://www.nytimes3xbfgragh.onion/privacy/cookie-policy\#how-do-i-manage-trackers}{Your
  Ad Choices}
\item
  \href{https://www.nytimes3xbfgragh.onion/privacy}{Privacy}
\item
  \href{https://help.nytimes3xbfgragh.onion/hc/en-us/articles/115014893428-Terms-of-service}{Terms
  of Service}
\item
  \href{https://help.nytimes3xbfgragh.onion/hc/en-us/articles/115014893968-Terms-of-sale}{Terms
  of Sale}
\item
  \href{https://spiderbites.nytimes3xbfgragh.onion}{Site Map}
\item
  \href{https://help.nytimes3xbfgragh.onion/hc/en-us}{Help}
\item
  \href{https://www.nytimes3xbfgragh.onion/subscription?campaignId=37WXW}{Subscriptions}
\end{itemize}
