Sections

SEARCH

\protect\hyperlink{site-content}{Skip to
content}\protect\hyperlink{site-index}{Skip to site index}

\href{https://www.nytimes3xbfgragh.onion/section/world/asia}{Asia
Pacific}

\href{https://myaccount.nytimes3xbfgragh.onion/auth/login?response_type=cookie\&client_id=vi}{}

\href{https://www.nytimes3xbfgragh.onion/section/todayspaper}{Today's
Paper}

\href{/section/world/asia}{Asia Pacific}\textbar{}Trump Administration
Penalizes Hong Kong Officials for Crackdown on Protesters

\url{https://nyti.ms/2XGM5Um}

\begin{itemize}
\item
\item
\item
\item
\item
\item
\end{itemize}

Advertisement

\protect\hyperlink{after-top}{Continue reading the main story}

Supported by

\protect\hyperlink{after-sponsor}{Continue reading the main story}

\hypertarget{trump-administration-penalizes-hong-kong-officials-for-crackdown-on-protesters}{%
\section{Trump Administration Penalizes Hong Kong Officials for
Crackdown on
Protesters}\label{trump-administration-penalizes-hong-kong-officials-for-crackdown-on-protesters}}

The sanctions are the first punishments brought against officials in
China and Hong Kong for suppressing pro-democracy protests.

\includegraphics{https://static01.graylady3jvrrxbe.onion/images/2020/08/07/us/politics/07dc-china-sanctions/07dc-china-sanctions-articleLarge-v2.jpg?quality=75\&auto=webp\&disable=upscale}

\href{https://www.nytimes3xbfgragh.onion/by/pranshu-verma}{\includegraphics{https://static01.graylady3jvrrxbe.onion/images/2020/07/07/reader-center/author-pranshu-verma/author-pranshu-verma-thumbLarge.png}}\href{https://www.nytimes3xbfgragh.onion/by/edward-wong}{\includegraphics{https://static01.graylady3jvrrxbe.onion/images/2018/09/24/multimedia/author-edward-wong/author-edward-wong-thumbLarge-v5.png}}

By \href{https://www.nytimes3xbfgragh.onion/by/pranshu-verma}{Pranshu
Verma} and
\href{https://www.nytimes3xbfgragh.onion/by/edward-wong}{Edward Wong}

\begin{itemize}
\item
  Aug. 7, 2020Updated 5:32 p.m. ET
\item
  \begin{itemize}
  \item
  \item
  \item
  \item
  \item
  \item
  \end{itemize}
\end{itemize}

WASHINGTON --- The Trump administration imposed sanctions on Friday on
Hong Kong's chief executive, Carrie Lam, and 10 other senior officials
in Hong Kong and mainland China over their roles in cracking down on
political dissent.

The move comes after Beijing announced in June that it was imposing a
\href{https://www.nytimes3xbfgragh.onion/2020/06/30/world/asia/hong-kong-security-law-explain.html}{national
security law in Hong Kong} to grant security agencies expansive powers.

The sanctions are the first against officials in Hong Kong and mainland
China over suppression of
\href{https://www.nytimes3xbfgragh.onion/2020/05/27/world/asia/why-are-hong-kong-protesters.html}{pro-democracy
protests} and dissent in the territory. Last month, President Trump
\href{https://www.nytimes3xbfgragh.onion/2020/07/14/us/politics/trump-news-conference.html}{signed
an executive order} seeking to punish China for its repression in Hong
Kong.

``The Chinese Communist Party has made clear that Hong Kong will never
again enjoy the high degree of autonomy that Beijing itself promised to
the Hong Kong people,'' Secretary of State Mike Pompeo said in a
statement on Friday. ``The United States will therefore treat Hong Kong
as `one country, one system,' and take action against individuals who
have crushed the Hong Kong people's freedoms.''

Mr. Pompeo was referring to ``one country, two systems,'' the term
widely used to describe the promise that China made in an international
treaty signed with Britain in 1984 that Hong Kong would remain
autonomous for 50 years after its transition from British to Chinese
rule in 1997.

Treasury Department officials said Ms. Lam was being penalized because
she was ``directly responsible'' for enacting Beijing's policies that
stifle dissent in Hong Kong.

Others facing sanctions include Chris Tang, the commissioner of the Hong
Kong Police Force, and Stephen Lo, who led the police department until
2019 and oversaw the arrest of 4,000 pro-democracy protesters that led
to over 1,600 people being injured, according to Treasury Department
officials. Many of the protesters were released on bail hours after
their arrests.

Also facing penalties are the leaders of China's liaison office to Hong
Kong and of its Hong Kong and Macau affairs office, both of which enact
Beijing's policies in the territory. The most senior Chinese official
named is Xia Baolong, who oversees the Hong Kong and Macau affairs
office, which is under the State Council, China's cabinet. Mr. Xia is
believed to be trusted by Xi Jinping, China's leader.

The chief of the liaison office, Luo Huining, is the most senior
mainland official in Hong Kong. The Treasury Department also named Zheng
Yanxiong and Eric Chan, two officials taking up senior posts established
by the new national security law.

The actions announced on Friday are symbolic to a degree, since it is
unclear if any of the 11 individuals have assets or property in the
United States that would be frozen under the measure.

The officials are prohibited from traveling to the United States.

Ms. Lam addressed the threat of facing sanctions during a news
conference last month, saying she would ``laugh it off.''

``I'm not worried,'' she said. ``I don't have assets in the United
States and don't really yearn to go to the U.S.''

Dali L. Yang, a political scientist at the University of Chicago, said
the sanctions appeared to be ``very targeted.''

``It's much more about sending a message and demonstrating that the
administration is acting on something that is in violation of and
impinging on the freedom and the democratic potential of Hong Kong,'' he
said.

Pro-democracy protests erupted in Hong Kong in June 2019, and the police
soon began taking increasingly hard-line measures to try to suppress
them, firing tear gas and using batons to beat demonstrators in subway
cars and alleyways. Those tactics escalated the protests, becoming the
focal point of grievances.

The Trump administration has taken a series of measures in recent months
that have heightened tensions between Washington and Beijing.
Administration officials say they are largely taking reciprocal actions
that counter bad behavior by the Chinese Communist Party.

On Thursday, Mr. Trump
\href{https://www.nytimes3xbfgragh.onion/2020/08/06/technology/trump-wechat-tiktok-china.html}{signed
two executive orders} barring American residents and corporations from
making any transactions with the Chinese-owned TikTok and WeChat apps
within 45 days, saying the use of the services posed a national security
risk because they do not keep user data private. Officials did not
detail what the ban would entail. WeChat is widely used in China, and
halting the use of the app in the United States would be onerous for
anyone trying to keep in touch with people in China or make transactions
with Chinese businesses.

Last month, the administration
\href{https://slack-redir.net/link?url=https\%3A\%2F\%2Fwww.nytimes3xbfgragh.onion\%2F2020\%2F07\%2F09\%2Fworld\%2Fasia\%2Ftrump-china-sanctions-uighurs.html}{imposed
sanctions} on senior Chinese officials, including a member of the
Communist Party's ruling Politburo, over human rights abuses against the
largely Muslim Uighur ethnic minority in the Xinjiang region. Officials
have also
\href{https://www.nytimes3xbfgragh.onion/2020/07/22/world/asia/us-china-houston-consulate.html}{shut
down} the Chinese Consulate in Houston, citing economic espionage
efforts by diplomats; banned some students affiliated with Chinese
military institutions from traveling to the United States; and arrested
officers or affiliates of the People's Liberation Army in the United
States on accusations of visa fraud.

The Trump administration has also imposed visa restrictions on Chinese
journalists in the United States and effectively expelled dozens of
them. That has incited Beijing to take harsh actions on American
journalists, including expelling employees of The New York Times, The
Wall Street Journal and The Washington Post.

In May, the administration
\href{https://www.nytimes3xbfgragh.onion/2020/05/09/us/politics/china-journalists-us-visa-crackdown.html}{imposed
a new 90-day limit} on stays for Chinese citizens who are in the United
States on a journalism visa. This required all such journalists to apply
for a visa renewal before Thursday.

U.S. officials are not expected to renew many of the visas, which would
lead to expulsions in the coming days.

Except for starting a trade war in 2018, Mr. Trump
\href{https://www.nytimes3xbfgragh.onion/2020/06/18/us/politics/trump-china-bolton.html}{was
reluctant to take hard actions} against China during much of his time in
office. He constantly praised China's leader and pleaded with Mr. Xi for
help on re-election. But he has soured on that relationship since the
coronavirus spread across the United States, devastating the economy and
his prospects for re-election.

Since Beijing announced the new national security law for Hong Kong,
American officials have debated how to persuade Chinese officials to
roll it back or how to penalize the Chinese government.

Last month, Mr. Trump signed an executive order ending the special
status that the United States grants Hong Kong in diplomatic and trade
relations. U.S. officials are now beginning to treat the territory like
mainland China.

At the time, some administration officials had wanted Mr. Trump to
announce that his government would impose sanctions on Ms. Lam and other
officials in Hong Kong and China. The president declined to do so then.

The current flurry of actions against China dovetails with a core part
of Mr. Trump's campaign strategy. His aides are trying to show Mr. Trump
is hitting China hard to shift the national conversation from his
\href{https://www.nytimes3xbfgragh.onion/2020/08/06/us/coronavirus-us.html}{failures
on managing the pandemic} and the economy.

The outbreak was first detected in central China, and recently, Mr.
Trump has blamed Chinese officials for failing to contain it, though he
\href{https://twitter.com/realdonaldtrump/status/1243407157321560071?lang=en}{praised}
Mr. Xi's efforts this winter.

Some of the China hawks in the administration are trying to
\href{https://www.nytimes3xbfgragh.onion/2020/07/25/world/asia/us-china-trump-xi.html}{set
the two nations} on a course for long-term confrontation and ensure that
relations remain in a state of open rivalry even if the presumptive
Democratic candidate, former Vice President Joseph R. Biden Jr., wins
the November election.

``This administration is weaving together a tapestry of actions, almost
like they're trying to cement a stand that, regardless of what happens
in the elections, would not be taken down easily,'' Mr. Yang said.

Elaine Yu and Austin Ramzy contributed reporting from Hong Kong.

Advertisement

\protect\hyperlink{after-bottom}{Continue reading the main story}

\hypertarget{site-index}{%
\subsection{Site Index}\label{site-index}}

\hypertarget{site-information-navigation}{%
\subsection{Site Information
Navigation}\label{site-information-navigation}}

\begin{itemize}
\tightlist
\item
  \href{https://help.nytimes3xbfgragh.onion/hc/en-us/articles/115014792127-Copyright-notice}{©~2020~The
  New York Times Company}
\end{itemize}

\begin{itemize}
\tightlist
\item
  \href{https://www.nytco.com/}{NYTCo}
\item
  \href{https://help.nytimes3xbfgragh.onion/hc/en-us/articles/115015385887-Contact-Us}{Contact
  Us}
\item
  \href{https://www.nytco.com/careers/}{Work with us}
\item
  \href{https://nytmediakit.com/}{Advertise}
\item
  \href{http://www.tbrandstudio.com/}{T Brand Studio}
\item
  \href{https://www.nytimes3xbfgragh.onion/privacy/cookie-policy\#how-do-i-manage-trackers}{Your
  Ad Choices}
\item
  \href{https://www.nytimes3xbfgragh.onion/privacy}{Privacy}
\item
  \href{https://help.nytimes3xbfgragh.onion/hc/en-us/articles/115014893428-Terms-of-service}{Terms
  of Service}
\item
  \href{https://help.nytimes3xbfgragh.onion/hc/en-us/articles/115014893968-Terms-of-sale}{Terms
  of Sale}
\item
  \href{https://spiderbites.nytimes3xbfgragh.onion}{Site Map}
\item
  \href{https://help.nytimes3xbfgragh.onion/hc/en-us}{Help}
\item
  \href{https://www.nytimes3xbfgragh.onion/subscription?campaignId=37WXW}{Subscriptions}
\end{itemize}
