Sections

SEARCH

\protect\hyperlink{site-content}{Skip to
content}\protect\hyperlink{site-index}{Skip to site index}

\href{https://www.nytimes3xbfgragh.onion/section/sports/football}{Pro
Football}

\href{https://myaccount.nytimes3xbfgragh.onion/auth/login?response_type=cookie\&client_id=vi}{}

\href{https://www.nytimes3xbfgragh.onion/section/todayspaper}{Today's
Paper}

\href{/section/sports/football}{Pro Football}\textbar{}N.F.L.'s Push
Ahead With Season Rankles Workers in the Home Office

\url{https://nyti.ms/2DFGVAK}

\begin{itemize}
\item
\item
\item
\item
\item
\end{itemize}

Advertisement

\protect\hyperlink{after-top}{Continue reading the main story}

Supported by

\protect\hyperlink{after-sponsor}{Continue reading the main story}

\hypertarget{nfls-push-ahead-with-season-rankles-workers-in-the-home-office}{%
\section{N.F.L.'s Push Ahead With Season Rankles Workers in the Home
Office}\label{nfls-push-ahead-with-season-rankles-workers-in-the-home-office}}

Workers who have been ordered back to the league's headquarters are
resisting, arguing that the reopening was rushed and that in some ways
they have been put in an ``impossible situation.''

\includegraphics{https://static01.graylady3jvrrxbe.onion/images/2020/08/07/sports/07nfl-employees01/07nfl-employees01-articleLarge.jpg?quality=75\&auto=webp\&disable=upscale}

\href{https://www.nytimes3xbfgragh.onion/by/kate-kelly}{\includegraphics{https://static01.graylady3jvrrxbe.onion/images/2019/09/13/business/kelly-kate/kelly-kate-thumbLarge.png}}\href{https://www.nytimes3xbfgragh.onion/by/ken-belson}{\includegraphics{https://static01.graylady3jvrrxbe.onion/images/2018/02/16/multimedia/author-ken-belson/author-ken-belson-thumbLarge.jpg}}

By \href{https://www.nytimes3xbfgragh.onion/by/kate-kelly}{Kate Kelly}
and \href{https://www.nytimes3xbfgragh.onion/by/ken-belson}{Ken Belson}

\begin{itemize}
\item
  Published Aug. 7, 2020Updated Aug. 19, 2020
\item
  \begin{itemize}
  \item
  \item
  \item
  \item
  \item
  \end{itemize}
\end{itemize}

Almost alone in the sports world, the N.F.L. has promised to hold its
2020 season as scheduled even as the pandemic forces other sports
leagues to cancel or reschedule games and construct elaborate
self-contained communities to play their seasons.

But the league's push to keep to its ambitious schedule has hit
substantial interference. First, more than 100 players and staff have
tested positive for the coronavirus. Dozens of other players have
decided to sit out this season to reduce their risk of infection.

And now, some N.F.L. office workers are resisting an order to return to
the league's Manhattan headquarters, contending that the reopening has
been rushed and has not been fully thought through, according to two
employees and an internal email. Nearly everyone has been ordered to
begin spending at least some time in the office starting Aug. 17.

In a letter sent to Commissioner Roger Goodell on Wednesday that was
reviewed by The New York Times, representatives from an internal group,
the Parents Initiative Network, said that ``many of us continue to
struggle with the prospect of returning to the office in the midst of
the pandemic.'' The group said its members have ``underlying physical
health concerns, mental health concerns, child care issues, medically
fragile family members, and the list goes on and on.''

League officials have said that allowances will be made for employees
with particular health concerns or family challenges. But the parent
network took issue with that, too. The N.F.L. is requiring workers who
want to continue to work remotely to discuss these requests with human
resources representatives. Workers said that requirement ``puts our
colleagues in an impossible situation'' because of their desire to
maintain their own and their family members' privacy on matters of
physical and mental health, among other reasons.

N.F.L. personnel have handled a range of matters effectively while
working remotely, the letter said, including overseeing the annual
draft, making a plan for coronavirus tracing and developing new sponsor
relationships. Giving employees greater latitude to continue working
remotely, the letter said, would underscore that ``the `F' in NFL also
stands for ```Family.'''

Dasha Smith, the league's chief people officer, said in an interview
that many employees have expressed concerns, including uncertainty
around schools reopening and child care issues. She said the league had
accommodated everyone who had asked for flexibility in returning, and
the only feasible way for employees to address their concerns was with
human resources.

Smith said that workers will only need to return to the office a few
days a week, and that one of the most valuable aspects of being in the
office was the ability to have spontaneous interactions with colleagues,
rather than scheduling them in advance.

``What we're missing out a lot of are those ad hoc, five minute
conversations that are very hard to do remotely,'' she said.

In late June, the N.F.L. brought back about 25 percent of its usual head
count to its headquarters at Manhattan's 345 Park Avenue, where about
800 people work. Workers at its offices in Culver City, Calif., and
Laurel, N.J. have followed a different timetable. Now, league officials
have laid the groundwork for a more complete employee return to Park
Avenue, starting with those who have their own offices on Aug. 17 and
followed by those who work from cubicles on Aug. 24. To maintain social
distancing, no more than half of an office's staff will be present on
any day, with workers alternating days.

\hypertarget{the-coronavirus-outbreak}{%
\subsubsection{The Coronavirus
Outbreak}\label{the-coronavirus-outbreak}}

\hypertarget{sports-and-the-virus}{%
\paragraph{Sports and the Virus}\label{sports-and-the-virus}}

Updated Aug. 21, 2020

Here's what's happening as the world of sports slowly comes back to
life:

\begin{itemize}
\item
  \begin{itemize}
  \tightlist
  \item
    The Western \& Southern Open tennis tournament --- long held near
    Cincinnati --- has
    \href{https://www.nytimes3xbfgragh.onion/2020/08/21/sports/tennis-most-ambitious-doubleheader-in-years-is-underway.html?action=click\&pgtype=Article\&state=default\&region=MAIN_CONTENT_2\&context=storylines_keepup}{been
    moved to Queens} this year, making for an unusual doubleheader with
    the United States Open.
  \item
    The Mets had
    \href{https://www.nytimes3xbfgragh.onion/2020/08/20/sports/baseball/mets-postponed-coronavirus.html?action=click\&pgtype=Article\&state=default\&region=MAIN_CONTENT_2\&context=storylines_keepup}{two
    games postponed} after a player and a staff member tested positive
    for coronavirus.
  \item
    While live sports are back, spectators are not in most cases.
    \href{https://www.nytimes3xbfgragh.onion/2020/08/19/sports/empty-stadiums-live-fans.html?action=click\&pgtype=Article\&state=default\&region=MAIN_CONTENT_2\&context=storylines_keepup}{Readers}
    comment on what they were missing as fans in the stands.
  \end{itemize}
\end{itemize}

The return was announced July 31 in an email from Goodell making the
case to get back to the office, particularly as players have been asked
to report to training camp.

``As our teams and players gear up for the season, it is critical that
we quickly ramp up our physical presence in the workplace too,'' Goodell
wrote in the email, a copy of which was reviewed by The Times.

In his response to the parent network on Thursday, the commissioner said
that many employees have been coming into the office since late June and
``have felt both safe and that our productivity has increased by being
here.'' He said that employees at the league's satellite offices have
also returned to the office ``without incident.''

The tussle at the league's headquarters comes as thousands of players
from around the country return to their team facilities. The league and
players' union spent months negotiating ways to reduce the risk of
infection to players, coaches and team personnel,
\href{https://www.nytimes3xbfgragh.onion/2020/07/14/sports/football/nfl-players-training-camp.html}{approving
plans that} included reconfigured locker rooms, reduced travel schedules
and extensive testing of all employees. Players
\href{https://www.nytimes3xbfgragh.onion/2020/07/24/sports/football/nfl-players-regular-season-start.html}{were
also allowed to skip the season} without penalty. The season is
scheduled to begin Sept. 10 with the Kansas City Chiefs playing at home
against the Houston Texans.

Unlike the players, workers in the league's headquarters, as well as on
individual teams, are not unionized, so they have less leverage to
negotiate work conditions. In that regard, the league's efforts to
accommodate its workers' needs, particularly those of parents, mirror
challenges faced by many companies. The league is also
\href{https://www.nytimes3xbfgragh.onion/2020/07/17/sports/football/sexual-harassment-washington-dan-snyder.html}{aware
of criticism} that it does not treat women fairly, and has
\href{https://www.nfl.com/news/op-ed-nfl-works-hard-to-promote-women-to-leadership-roles}{worked
in recent years to increase the number of women} in its headquarters and
teams.

In May, the league
\href{https://www.cnbc.com/2020/04/29/the-nfl-is-furloughing-league-employees-and-reducing-salaries-due-to-coronavirus.html}{furloughed
employees} who could not do their jobs from home, or those who had their
work significantly reduced. The league also cut salaries for higher-paid
employees, including Goodell.

Goodell now joins the ranks of New York area chief executives,
restaurant owners, school principals and other organizational leaders
trying to run their workplaces during the worst public-health crisis in
recent memory. The tech giant Google has said that most of its employees
won't return to its offices in the United States, including in New York,
until the summer of 2021. Schools in
\href{https://www.nytimes3xbfgragh.onion/2020/07/06/nyregion/nyc-school-reopening-plan.html?searchResultPosition=6}{New
York plan to experiment with a combination of remote}and distance
learning.
\href{https://www.nytimes3xbfgragh.onion/2020/04/09/business/coronvirus-bank-of-america-workers.html}{Bank
of America, JPMorgan Chase} and other banks have maintained a skeletal
staff in their offices since the pandemic first rattled the city, with
mixed results.

The employees who work at N.F.L. headquarters in Manhattan, where the
job descriptions range from handling payroll and events to
communications, legal work and financial operations, are now facing some
of the same concerns as workers in other areas who have experienced wide
Covid-19 outbreaks. They are wary of public transportation, irritated at
the idea of wearing a mask at the office all day while still conducting
much of their business through teleconferencing, and anxious about the
possibility of falling ill themselves or infecting vulnerable family
members. Now, some N.F.L. employees feel torn between staying home to
maintain their personal well-being and reporting to the office to ensure
their job security, according to two current employees and the parent
network's letter.

The start of school in New York City is also the same day the new
football season starts, meaning some N.F.L. parents at the league
headquarters could be busiest at work just as they are juggling limited
in-person class schedules or overseeing another round of distance
learning. And unlike many other cities, who plan to conduct school
entirely remotely this fall, New York's schools are planning to reopen
to many students with one to three days a week in physical classrooms,
complicating child-care schedules.

The N.F.L. has prepared a detailed plan for safeguarding employees who
return to its offices. Cubicles have been spaced out, copiers must be
wiped down after each use and arrows on the floor dictate which routes
to walk. Workers use a virtual app daily to answer questions about their
exposure to the virus and body temperature, and foot-operated buttons
will open doors. ``We've put a lot of thought into this,'' said Smith,
the chief people officer.

Nonetheless, some employees have been roiled by the news. During an
all-hands virtual call held Monday, Smith said she received 300 queries
from employees on the chat function, ranging from requests for
clarification on the safety protocols to expressions of concern about
the repercussions of not returning. There were questions about whether
the league's \$75-per-day reimbursement for using ride-sharing services
to get to work or \$30-per-day parking repayments were generous enough,
said two participants in the call. Others wondered how to handle
child-care issues at a time when school had not yet begun.

Questions were also raised about whether employees would be fired for
not coming in, said the participants. League officials said no.

Advertisement

\protect\hyperlink{after-bottom}{Continue reading the main story}

\hypertarget{site-index}{%
\subsection{Site Index}\label{site-index}}

\hypertarget{site-information-navigation}{%
\subsection{Site Information
Navigation}\label{site-information-navigation}}

\begin{itemize}
\tightlist
\item
  \href{https://help.nytimes3xbfgragh.onion/hc/en-us/articles/115014792127-Copyright-notice}{©~2020~The
  New York Times Company}
\end{itemize}

\begin{itemize}
\tightlist
\item
  \href{https://www.nytco.com/}{NYTCo}
\item
  \href{https://help.nytimes3xbfgragh.onion/hc/en-us/articles/115015385887-Contact-Us}{Contact
  Us}
\item
  \href{https://www.nytco.com/careers/}{Work with us}
\item
  \href{https://nytmediakit.com/}{Advertise}
\item
  \href{http://www.tbrandstudio.com/}{T Brand Studio}
\item
  \href{https://www.nytimes3xbfgragh.onion/privacy/cookie-policy\#how-do-i-manage-trackers}{Your
  Ad Choices}
\item
  \href{https://www.nytimes3xbfgragh.onion/privacy}{Privacy}
\item
  \href{https://help.nytimes3xbfgragh.onion/hc/en-us/articles/115014893428-Terms-of-service}{Terms
  of Service}
\item
  \href{https://help.nytimes3xbfgragh.onion/hc/en-us/articles/115014893968-Terms-of-sale}{Terms
  of Sale}
\item
  \href{https://spiderbites.nytimes3xbfgragh.onion}{Site Map}
\item
  \href{https://help.nytimes3xbfgragh.onion/hc/en-us}{Help}
\item
  \href{https://www.nytimes3xbfgragh.onion/subscription?campaignId=37WXW}{Subscriptions}
\end{itemize}
