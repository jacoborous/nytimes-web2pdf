Sections

SEARCH

\protect\hyperlink{site-content}{Skip to
content}\protect\hyperlink{site-index}{Skip to site index}

\href{https://www.nytimes3xbfgragh.onion/section/health}{Health}

\href{https://myaccount.nytimes3xbfgragh.onion/auth/login?response_type=cookie\&client_id=vi}{}

\href{https://www.nytimes3xbfgragh.onion/section/todayspaper}{Today's
Paper}

\href{/section/health}{Health}\textbar{}New York Is Positioned to Reopen
Schools Safely, Health Experts Say

\url{https://nyti.ms/2XVCM35}

\begin{itemize}
\item
\item
\item
\item
\item
\end{itemize}

\hypertarget{schools-reopening}{%
\subsubsection{\texorpdfstring{\href{https://www.nytimes3xbfgragh.onion/spotlight/schools-reopening?name=styln-coronavirus-schools-reopening\&region=TOP_BANNER\&variant=undefined\&block=storyline_menu_recirc\&action=click\&pgtype=Article\&impression_id=0b947230-e39d-11ea-ae05-3598fe08e800}{Schools
Reopening}}{Schools Reopening}}\label{schools-reopening}}

\begin{itemize}
\tightlist
\item
  \href{https://www.nytimes3xbfgragh.onion/2020/08/19/us/colleges-closing-covid.html?name=styln-coronavirus-schools-reopening\&region=TOP_BANNER\&variant=undefined\&block=storyline_menu_recirc\&action=click\&pgtype=Article\&impression_id=0b947231-e39d-11ea-ae05-3598fe08e800}{Colleges
  Closing}
\item
  \href{https://www.nytimes3xbfgragh.onion/2020/08/20/us/schools-reopening-nurses-covid.html?name=styln-coronavirus-schools-reopening\&region=TOP_BANNER\&variant=undefined\&block=storyline_menu_recirc\&action=click\&pgtype=Article\&impression_id=0b947232-e39d-11ea-ae05-3598fe08e800}{Missing
  School Nurses}
\item
  \href{https://www.nytimes3xbfgragh.onion/2020/08/18/parenting/homeschool-families.html?name=styln-coronavirus-schools-reopening\&region=TOP_BANNER\&variant=undefined\&block=storyline_menu_recirc\&action=click\&pgtype=Article\&impression_id=0b949940-e39d-11ea-ae05-3598fe08e800}{Home-Schooling
  Families}
\item
  \href{https://www.nytimes3xbfgragh.onion/2020/08/05/parenting/parents-distance-learning.html?name=styln-coronavirus-schools-reopening\&region=TOP_BANNER\&variant=undefined\&block=storyline_menu_recirc\&action=click\&pgtype=Article\&impression_id=0b949941-e39d-11ea-ae05-3598fe08e800}{Prepare
  for Distance Learning}
\end{itemize}

Advertisement

\protect\hyperlink{after-top}{Continue reading the main story}

Supported by

\protect\hyperlink{after-sponsor}{Continue reading the main story}

\hypertarget{new-york-is-positioned-to-reopen-schools-safely-health-experts-say}{%
\section{New York Is Positioned to Reopen Schools Safely, Health Experts
Say}\label{new-york-is-positioned-to-reopen-schools-safely-health-experts-say}}

Transmission, even in New York City, is well below thresholds experts
say are safe, but issues like adequate ventilation to combat aerosol
spread of the virus remain.

\includegraphics{https://static01.graylady3jvrrxbe.onion/images/2020/08/07/science/07VIRUS-SCHOOLS-HEALTH1/07VIRUS-SCHOOLS-HEALTH1-articleLarge.jpg?quality=75\&auto=webp\&disable=upscale}

By \href{https://www.nytimes3xbfgragh.onion/by/roni-caryn-rabin}{Roni
Caryn Rabin} and
\href{https://www.nytimes3xbfgragh.onion/by/apoorva-mandavilli}{Apoorva
Mandavilli}

\begin{itemize}
\item
  Aug. 7, 2020
\item
  \begin{itemize}
  \item
  \item
  \item
  \item
  \item
  \end{itemize}
\end{itemize}

New York State, the center of the worst coronavirus outbreak in the
world four months ago, is now one of the few places in the country that
may be able to safely reopen schools, several public health experts said
after Gov. Andrew M. Cuomo gave districts permission to do so.

Mr. Cuomo announced Friday that public school districts across the state
could hold in-person classes this fall, even as districts in many parts
of the country where cases are still rising have abandoned the idea and
will continue with remote learning.

In interviews, doctors, epidemiologists and other public health experts
said that conditions were favorable throughout the state, including New
York City, to bring children back --- as long as safety precautions are
in place. Some expressed concern that the effects of keeping students
home were more worrisome.

As of Wednesday, fewer than 1 percent of coronavirus tests statewide
were positive, well below the 5 percent positivity threshold that both
the Centers for Disease Control and Prevention and the World Health
Organization have targeted as a safe standard for reopening schools.

In New York City, the positivity rate is just slightly higher than the
state average, with 1 percent of tests coming back with positive
results. Mayor Bill de Blasio has said schools in the city would not
open if the metric rises above 3 percent. A report last month from
Harvard's Global Health Institute also recommended opening schools only
when the daily infection rate is less than 3 percent.

``If there's any city that should be opening in the entire country or at
least trying to open, it should be New York City,'' said Dr. Uché
Blackstock, an urgent care physician in Brooklyn and founder of
\href{https://advancinghealthequity.com/}{Advancing Health Equity}, a
health care advocacy group, who has children in the public schools.

Dr. William Schaffner, an infectious disease expert at Vanderbilt
University, said reopening the schools in New York was a bit of a
``social experiment'' or ``trial run,'' but added that the odds of
success were good.

``New York's chances of getting a good result, even though it is a
densely populated metropolitan area, are actually better than in many
rural areas, where they're not nearly as serious about trying to control
the virus,'' he said.

New York also does well by another safety metric, the number of cases
per capita. The Harvard report characterized regions with between one
and 10 new cases a day per 100,000 people as ``yellow zones,'' areas
suitable for in-person classes at all grade levels as long as proper
infection control and social distancing measures are in place. New York
is within this zone, with 23 cases per 100,000 over a seven day period,
about 3.5 cases per 100,000 on a daily basis. The yellow zone
recommendations are to give first priority for reopening to
prekindergarten through fifth grade and last priority to high school.

The C.D.C. and public health experts generally agree on a range of
measures that schools should adopt to protect children, teachers, staff
members and their families, including mask wearing, physical distancing
and improving ventilation in buildings.

Districts across the state have been told to come up with their own
reopening plans, all of which must include those safety measures as well
as plans for testing teachers and students**,** and are subject to
approval by the state's health and education departments.

Most experts agree schools should skip activities like chorus, band and
sports involving physical contact; have children eat in classrooms
instead of a central cafeteria; and take steps to prevent crowding in
hallways.

``If the cases are under control --- and I'm not defining what that
means --- then I think that there are precautions we can take that make
going back to school worthwhile,'' said Linsey Marr, an aerosol
scientist at Virginia Tech. ``If the cases in the community are low,
there's a smaller chance that someone who is infected will show up in
school, and the precautions greatly reduce their risk of transmitting it
to other students and teachers.''

\includegraphics{https://static01.graylady3jvrrxbe.onion/images/2020/08/07/science/07VIRUS-SCHOOLS-HEALTH2/merlin_175277352_f2fe212f-9d31-41d8-b39d-88521eed31d8-articleLarge.jpg?quality=75\&auto=webp\&disable=upscale}

Like many other experts, Dr. Marr expressed concerns about the academic
and developmental costs of keeping children at home, emphasizing the
critical role schools play in their well-being and social and emotional
development, in addition to academic learning. Schools also provide
access to support services and meals for low-income children and
facilitate parents' return to work.

``It is of utmost importance that we educate our kids, and we should do
everything possible to do that in person if community transmission is
under control,'' Dr. Marr said.

School buildings are legally required to have enough air flowing through
to fully replace the air every 20 minutes --- or three full air
exchanges per hour. About 40 percent of school districts in the country
need to update their heating, ventilation and air-conditioning systems,
according to a Government Accountability Office report. Many schools
rely on natural ventilation, meaning just open windows and a roof duct.

Many New York City school buildings are too old and underfunded to
overhaul their ventilation systems, and rely mostly on natural
ventilation. Joseph Allen, director of the Healthy Buildings Program at
Harvard University. said they could institute other measures, such as
opening doors along with windows; adjusting the settings in heating,
ventilation and air-conditioning systems to bring in more outdoor air in
instead of recirculating the indoor air; upgrading filters in the
systems; and placing portable air purifiers in some rooms.

Open windows help as long as enough air is coming through, but the
typical school can support only about one full air exchange per hour,
Dr. Allen said.

A national committee of scientists and educators last month recommended
that younger children and those with special needs
\href{https://www.nytimes3xbfgragh.onion/2020/07/15/health/coronavirus-schools-reopening.html}{return
in person to school} whenever possible. The report by the National
Academies of Science, Engineering and Medicine called for measures like
hand washing, physical distancing and minimizing of group activities
such as lunch and recess.

Image

Students at Stuyvesant High School in Manhattan. One expert said
distinctions should be made between elementary schools and those with
older students.Credit...Bebeto Matthews/Associated Press

Even though a small number of children infected with the virus have
developed a potentially life threatening syndrome, called Multisystem
Inflammatory Syndrome, most studies suggest the virus poses minimal
risks to young children.

\href{https://www.nytimes3xbfgragh.onion/spotlight/schools-reopening?action=click\&pgtype=Article\&state=default\&region=MAIN_CONTENT_3\&context=storylines_keepup}{}

\hypertarget{schools-reopening-}{%
\subsubsection{Schools Reopening ›}\label{schools-reopening-}}

\hypertarget{back-to-school}{%
\paragraph{Back to School}\label{back-to-school}}

Updated Aug. 20, 2020

The latest on how schools are reopening amid the pandemic.

\begin{itemize}
\item
  \begin{itemize}
  \tightlist
  \item
    Much more is
    \href{https://www.nytimes3xbfgragh.onion/2020/08/20/us/schools-reopening-nurses-covid.html?action=click\&pgtype=Article\&state=default\&region=MAIN_CONTENT_3\&context=storylines_keepup}{expected
    of America's school nurses} during the pandemic, but many schools
    don't have one.
  \item
    A vast majority of parents have resigned themselves to
    \href{https://www.nytimes3xbfgragh.onion/2020/08/19/us/colleges-closing-covid.html?action=click\&pgtype=Article\&state=default\&region=MAIN_CONTENT_3\&context=storylines_keepup}{going
    it alone in the pandemic school year}, according to a new survey for
    The New York Times.
  \item
    Alabama is betting that a
    \href{https://www.nytimes3xbfgragh.onion/2020/08/19/business/alabama-uab-coronavirus-tests.html?action=click\&pgtype=Article\&state=default\&region=MAIN_CONTENT_3\&context=storylines_keepup}{robust
    student testing and technology program} will be enough to hinder
    outbreaks on college campuses.
  \item
    We want to hear from teachers making difficult choices. How are you
    thinking about the start of the school year?
    \href{https://www.nytimes3xbfgragh.onion/2020/08/19/us/teachers-school-reopenings.html?action=click\&pgtype=Article\&state=default\&region=MAIN_CONTENT_3\&context=storylines_keepup}{Tell
    us here}.
  \end{itemize}
\end{itemize}

Children harbor
\href{https://www.nytimes3xbfgragh.onion/2020/07/30/health/coronavirus-children.html}{at
least as much virus} as adults do, though children under age 10 are
\href{https://www.nytimes3xbfgragh.onion/2020/07/18/health/coronavirus-children-schools.html}{less
efficient at spreading} it than older children and adults, according to
a recent study from South Korea.

Michael Osterholm, the director of the Center for Infectious Disease
Research and Policy at the University of Minnesota, said elementary
schools could probably be operated safely, but he predicted that high
schools and colleges would face greater challenges staying open. (Mr.
Cuomo's announcement did not address colleges.)

New York will have to continue monitoring cases closely and ``apply the
brakes'' if transmissions start inching up, he said.

Dr. Blackstock said she had seen the number of patients coming in to the
Brooklyn clinics where she practices with Covid-19 symptoms fall
drastically since April, reassuring her that the city has mostly
contained the virus.

As a Black physician who cares for Covid-19 patients, and as a mother of
two young children at a public school in Brooklyn, Dr. Blackstock said
she had a unique perspective on the epidemic in New York and on the
urgent need for children to go back to classrooms.

She said
\href{https://twitter.com/uche_blackstock/status/1291206051094765569}{in
a Tweet on Thursday} that she would send her children back to school for
two or three days a week. The response, accusing her of being
``anti-teacher'' and ``anti-educator,'' startled her.

``Maybe I was naïve and I didn't realize how polarized the discussion
around schools had become,'' she said.

At the public school where Dr. Blackstock's children are students, the
windows don't open fully and ventilation is poor, but on balance, she
said, the benefits of reopening schools outweigh the risks.

``We have to start somewhere,'' she said. ``I feel we have to at least
try.''

Eliza Shapiro contributed reporting.

Advertisement

\protect\hyperlink{after-bottom}{Continue reading the main story}

\hypertarget{site-index}{%
\subsection{Site Index}\label{site-index}}

\hypertarget{site-information-navigation}{%
\subsection{Site Information
Navigation}\label{site-information-navigation}}

\begin{itemize}
\tightlist
\item
  \href{https://help.nytimes3xbfgragh.onion/hc/en-us/articles/115014792127-Copyright-notice}{©~2020~The
  New York Times Company}
\end{itemize}

\begin{itemize}
\tightlist
\item
  \href{https://www.nytco.com/}{NYTCo}
\item
  \href{https://help.nytimes3xbfgragh.onion/hc/en-us/articles/115015385887-Contact-Us}{Contact
  Us}
\item
  \href{https://www.nytco.com/careers/}{Work with us}
\item
  \href{https://nytmediakit.com/}{Advertise}
\item
  \href{http://www.tbrandstudio.com/}{T Brand Studio}
\item
  \href{https://www.nytimes3xbfgragh.onion/privacy/cookie-policy\#how-do-i-manage-trackers}{Your
  Ad Choices}
\item
  \href{https://www.nytimes3xbfgragh.onion/privacy}{Privacy}
\item
  \href{https://help.nytimes3xbfgragh.onion/hc/en-us/articles/115014893428-Terms-of-service}{Terms
  of Service}
\item
  \href{https://help.nytimes3xbfgragh.onion/hc/en-us/articles/115014893968-Terms-of-sale}{Terms
  of Sale}
\item
  \href{https://spiderbites.nytimes3xbfgragh.onion}{Site Map}
\item
  \href{https://help.nytimes3xbfgragh.onion/hc/en-us}{Help}
\item
  \href{https://www.nytimes3xbfgragh.onion/subscription?campaignId=37WXW}{Subscriptions}
\end{itemize}
