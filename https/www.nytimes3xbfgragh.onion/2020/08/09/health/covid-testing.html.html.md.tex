Sections

SEARCH

\protect\hyperlink{site-content}{Skip to
content}\protect\hyperlink{site-index}{Skip to site index}

\href{https://www.nytimes3xbfgragh.onion/section/health}{Health}

\href{https://myaccount.nytimes3xbfgragh.onion/auth/login?response_type=cookie\&client_id=vi}{}

\href{https://www.nytimes3xbfgragh.onion/section/todayspaper}{Today's
Paper}

\href{/section/health}{Health}\textbar{}Ohio Governor Says His Flawed
Virus Test Shouldn't Undercut New, Rapid Methods

\url{https://nyti.ms/3kABzb9}

\begin{itemize}
\item
\item
\item
\item
\item
\end{itemize}

\hypertarget{the-coronavirus-outbreak}{%
\subsubsection{\texorpdfstring{\href{https://www.nytimes3xbfgragh.onion/news-event/coronavirus?name=styln-coronavirus-national\&region=TOP_BANNER\&variant=undefined\&block=storyline_menu_recirc\&action=click\&pgtype=Article\&impression_id=edb0a130-e3ab-11ea-af8a-cf9df229f915}{The
Coronavirus
Outbreak}}{The Coronavirus Outbreak}}\label{the-coronavirus-outbreak}}

\begin{itemize}
\tightlist
\item
  live\href{https://www.nytimes3xbfgragh.onion/2020/08/21/world/covid-19-coronavirus.html?name=styln-coronavirus-national\&region=TOP_BANNER\&variant=undefined\&block=storyline_menu_recirc\&action=click\&pgtype=Article\&impression_id=edb0a131-e3ab-11ea-af8a-cf9df229f915}{Latest
  Updates}
\item
  \href{https://www.nytimes3xbfgragh.onion/interactive/2020/us/coronavirus-us-cases.html?name=styln-coronavirus-national\&region=TOP_BANNER\&variant=undefined\&block=storyline_menu_recirc\&action=click\&pgtype=Article\&impression_id=edb0c840-e3ab-11ea-af8a-cf9df229f915}{Maps
  and Cases}
\item
  \href{https://www.nytimes3xbfgragh.onion/interactive/2020/science/coronavirus-vaccine-tracker.html?name=styln-coronavirus-national\&region=TOP_BANNER\&variant=undefined\&block=storyline_menu_recirc\&action=click\&pgtype=Article\&impression_id=edb0c841-e3ab-11ea-af8a-cf9df229f915}{Vaccine
  Tracker}
\item
  \href{https://www.nytimes3xbfgragh.onion/2020/08/19/us/colleges-closing-covid.html?name=styln-coronavirus-national\&region=TOP_BANNER\&variant=undefined\&block=storyline_menu_recirc\&action=click\&pgtype=Article\&impression_id=edb0c842-e3ab-11ea-af8a-cf9df229f915}{Colleges
  Closing}
\item
  \href{https://www.nytimes3xbfgragh.onion/live/2020/08/21/business/stock-market-today-coronavirus?name=styln-coronavirus-national\&region=TOP_BANNER\&variant=undefined\&block=storyline_menu_recirc\&action=click\&pgtype=Article\&impression_id=edb0c843-e3ab-11ea-af8a-cf9df229f915}{Economy}
\end{itemize}

Advertisement

\protect\hyperlink{after-top}{Continue reading the main story}

Supported by

\protect\hyperlink{after-sponsor}{Continue reading the main story}

\hypertarget{ohio-governor-says-his-flawed-virus-test-shouldnt-undercut-new-rapid-methods}{%
\section{Ohio Governor Says His Flawed Virus Test Shouldn't Undercut
New, Rapid
Methods}\label{ohio-governor-says-his-flawed-virus-test-shouldnt-undercut-new-rapid-methods}}

Several states are trying to increase coronavirus testing with faster
methods, undeterred by less accurate but quicker readouts.

\includegraphics{https://static01.graylady3jvrrxbe.onion/images/2020/08/09/reader-center/09virus-testing1/09virus-testing1-articleLarge.jpg?quality=75\&auto=webp\&disable=upscale}

By
\href{https://www.nytimes3xbfgragh.onion/by/katherine-j--wu}{Katherine
J. Wu}

\begin{itemize}
\item
  Aug. 9, 2020
\item
  \begin{itemize}
  \item
  \item
  \item
  \item
  \item
  \end{itemize}
\end{itemize}

Gov. Mike DeWine of Ohio, who last week
\href{https://www.nytimes3xbfgragh.onion/2020/08/06/us/mike-dewine-coronavirus.html}{tested
positive for the coronavirus, then negative and then negative again},
said on CNN on Sunday that his roller-coaster ride should not be reason
for people to think ``that testing is not reliable or doesn't work.''

His first test result was positive, when he was screened with a rapid
testing method on Thursday before President Trump arrived in Ohio for
campaign appearances.

Mr. DeWine was given
\href{https://www.nytimes3xbfgragh.onion/2020/08/07/us/covid-test-accuracy-governor-dewine-ohio.html}{an
antigen test} made by Quidel, one of two companies that have received
\href{https://www.fda.gov/medical-devices/coronavirus-disease-2019-covid-19-emergency-use-authorizations-medical-devices/vitro-diagnostics-euas\#individual-antigen}{emergency
use authorization from the Food and Drug Administration} for coronavirus
antigen tests.

These tests, while fast and convenient, are known to be less accurate
than PCR tests, which were used to retest Mr. DeWine twice on Thursday
and
\href{https://twitter.com/MikeDeWine/status/1292192192312770561}{once
more on Saturday}. All three PCR tests turned up negative, confirming
that Mr. DeWine was not infected with the virus.

``I don't think that DeWine's results were surprising, per se,'' said
Andrea Prinzi, a clinical microbiologist and diagnostics researcher at
the Anschutz Medical Campus in Colorado. ``We know that the performance
of antigen testing is not as accurate as PCR testing.''

The Ohio governor's experience, however, may raise concerns about how
much states will rely on antigen tests as they seek to augment the forms
of testing, like PCR, that are in short supply or that are mired in
laboratory backlogs, unable to generate results in a timely fashion to
help assess caseloads and dole out treatments.

Mr. DeWine, a Republican, is one of seven governors who announced last
week that they were
\href{https://governor.maryland.gov/2020/08/04/governors-of-maryland-louisiana-massachusetts-michigan-ohio-and-virginia-announce-major-bipartisan-interstate-compact-for-three-million-rapid-antigen-tests/}{banding
together} to buy 3.5 million rapid coronavirus tests, including antigen
tests, to ramp up production.

\hypertarget{latest-updates-the-coronavirus-outbreak}{%
\section{\texorpdfstring{\href{https://www.nytimes3xbfgragh.onion/2020/08/21/world/covid-19-coronavirus.html?action=click\&pgtype=Article\&state=default\&region=MAIN_CONTENT_1\&context=storylines_live_updates}{Latest
Updates: The Coronavirus
Outbreak}}{Latest Updates: The Coronavirus Outbreak}}\label{latest-updates-the-coronavirus-outbreak}}

Updated 2020-08-21T12:38:27.712Z

\begin{itemize}
\tightlist
\item
  \href{https://www.nytimes3xbfgragh.onion/2020/08/21/world/covid-19-coronavirus.html?action=click\&pgtype=Article\&state=default\&region=MAIN_CONTENT_1\&context=storylines_live_updates\#link-6a60a19d}{`Be
  adults': Universities in the U.S. are warning students about
  gatherings as they return to campus.}
\item
  \href{https://www.nytimes3xbfgragh.onion/2020/08/21/world/covid-19-coronavirus.html?action=click\&pgtype=Article\&state=default\&region=MAIN_CONTENT_1\&context=storylines_live_updates\#link-324af071}{As
  he accepts the Democratic nomination, Biden knocks Trump's pandemic
  response.}
\item
  \href{https://www.nytimes3xbfgragh.onion/2020/08/21/world/covid-19-coronavirus.html?action=click\&pgtype=Article\&state=default\&region=MAIN_CONTENT_1\&context=storylines_live_updates\#link-191d44be}{South
  Korea threatens to detain people who obstruct virus-control efforts.}
\end{itemize}

\href{https://www.nytimes3xbfgragh.onion/2020/08/21/world/covid-19-coronavirus.html?action=click\&pgtype=Article\&state=default\&region=MAIN_CONTENT_1\&context=storylines_live_updates}{See
more updates}

More live coverage:
\href{https://www.nytimes3xbfgragh.onion/live/2020/08/21/business/stock-market-today-coronavirus?action=click\&pgtype=Article\&state=default\&region=MAIN_CONTENT_1\&context=storylines_live_updates}{Markets}

Daniel Tierney, the press secretary for Mr. DeWine, noted in an email
that the states involved were considering ``multiple companies and
multiple testing types,'' but did not specify further.

On Sunday, Mr. DeWine said he had already been in touch with Gov. Larry
Hogan of Maryland, a Republican, to talk about the states' agreement to
use their collective ``purchasing power'' for testing and other
supplies.

``If anyone needed a wake-up call with antigens, how careful you have to
be, we certainly saw that with my test,'' Mr. DeWine said. ``And we're
going to be very careful in how we use it.''

A spokesman for Mr. Hogan, Michael Ricci, echoed that sentiment: ``We
are taking this one step at a time.''

Accurate test results are crucial for curbing the spread of disease.
False positives, like the one Mr. DeWine received, can set off an
unnecessary period of self-isolation, depriving people of access to
their workplaces or their own families. False negatives, on the other
hand, can hasten the spread of disease from unwittingly infected people.

PCR tests, like the ones used to determine Mr. DeWine's health status,
are often the best bet for avoiding incorrect results. But these tests
are in
\href{https://www.nytimes3xbfgragh.onion/2020/07/23/health/coronavirus-testing-supply-shortage.html}{short
supply nationwide} as manufacturers and laboratories struggle to meet
the increase in demand that has accompanied recent surges in infections.
Turnaround times for results have stretched past two weeks in some parts
of the country, rendering the information useless for anxious people who
need to know their status immediately so they can self-isolate as needed
and stop the virus from spreading further.

\includegraphics{https://static01.graylady3jvrrxbe.onion/images/2020/08/09/science/09virus-testing2/09virus-testing2-articleLarge.jpg?quality=75\&auto=webp\&disable=upscale}

``Honestly, PCR tests were not designed for this type of mass
screening/testing,'' Ms. Prinzi said. PCR tests, she added, function
best in laboratory environments that are well stocked with chemicals,
high-tech machines and specially trained personnel. Their
high-maintenance ingredient lists and relatively hefty price tags aren't
terribly compatible with quickly getting answers to large numbers of
people.

Rapid tests, on the other hand, can
\href{https://www.nytimes3xbfgragh.onion/2020/08/06/health/rapid-Covid-tests.html}{catch
a majority of active infections if administered frequently}, even if
they're less accurate, many experts have argued.

Compared with PCR tests, antigen tests are more likely to return a false
negative result, mistaking an infected person as virus-free. Quidel's
test, for instance, can miss up to 20 percent of the cases that PCR
detects.

Notably, Mr. DeWine's antigen test produced the opposite error: a false
positive that incorrectly indicated he had been infected.

\href{https://www.nytimes3xbfgragh.onion/news-event/coronavirus?action=click\&pgtype=Article\&state=default\&region=MAIN_CONTENT_3\&context=storylines_faq}{}

\hypertarget{the-coronavirus-outbreak-}{%
\subsubsection{The Coronavirus Outbreak
›}\label{the-coronavirus-outbreak-}}

\hypertarget{frequently-asked-questions}{%
\paragraph{Frequently Asked
Questions}\label{frequently-asked-questions}}

Updated August 17, 2020

\begin{itemize}
\item ~
  \hypertarget{why-does-standing-six-feet-away-from-others-help}{%
  \paragraph{Why does standing six feet away from others
  help?}\label{why-does-standing-six-feet-away-from-others-help}}

  \begin{itemize}
  \tightlist
  \item
    The coronavirus spreads primarily through droplets from your mouth
    and nose, especially when you cough or sneeze. The C.D.C., one of
    the organizations using that measure,
    \href{https://www.nytimes3xbfgragh.onion/2020/04/14/health/coronavirus-six-feet.html?action=click\&pgtype=Article\&state=default\&region=MAIN_CONTENT_3\&context=storylines_faq}{bases
    its recommendation of six feet} on the idea that most large droplets
    that people expel when they cough or sneeze will fall to the ground
    within six feet. But six feet has never been a magic number that
    guarantees complete protection. Sneezes, for instance, can launch
    droplets a lot farther than six feet,
    \href{https://jamanetwork.com/journals/jama/fullarticle/2763852}{according
    to a recent study}. It's a rule of thumb: You should be safest
    standing six feet apart outside, especially when it's windy. But
    keep a mask on at all times, even when you think you're far enough
    apart.
  \end{itemize}
\item ~
  \hypertarget{i-have-antibodies-am-i-now-immune}{%
  \paragraph{I have antibodies. Am I now
  immune?}\label{i-have-antibodies-am-i-now-immune}}

  \begin{itemize}
  \tightlist
  \item
    As of right
    now,\href{https://www.nytimes3xbfgragh.onion/2020/07/22/health/covid-antibodies-herd-immunity.html?action=click\&pgtype=Article\&state=default\&region=MAIN_CONTENT_3\&context=storylines_faq}{that
    seems likely, for at least several months.} There have been
    frightening accounts of people suffering what seems to be a second
    bout of Covid-19. But experts say these patients may have a
    drawn-out course of infection, with the virus taking a slow toll
    weeks to months after initial exposure. People infected with the
    coronavirus typically
    \href{https://www.nature.com/articles/s41586-020-2456-9}{produce}
    immune molecules called antibodies, which are
    \href{https://www.nytimes3xbfgragh.onion/2020/05/07/health/coronavirus-antibody-prevalence.html?action=click\&pgtype=Article\&state=default\&region=MAIN_CONTENT_3\&context=storylines_faq}{protective
    proteins made in response to an
    infection}\href{https://www.nytimes3xbfgragh.onion/2020/05/07/health/coronavirus-antibody-prevalence.html?action=click\&pgtype=Article\&state=default\&region=MAIN_CONTENT_3\&context=storylines_faq}{.
    These antibodies may} last in the body
    \href{https://www.nature.com/articles/s41591-020-0965-6}{only two to
    three months}, which may seem worrisome, but that's perfectly normal
    after an acute infection subsides, said Dr. Michael Mina, an
    immunologist at Harvard University. It may be possible to get the
    coronavirus again, but it's highly unlikely that it would be
    possible in a short window of time from initial infection or make
    people sicker the second time.
  \end{itemize}
\item ~
  \hypertarget{im-a-small-business-owner-can-i-get-relief}{%
  \paragraph{I'm a small-business owner. Can I get
  relief?}\label{im-a-small-business-owner-can-i-get-relief}}

  \begin{itemize}
  \tightlist
  \item
    The
    \href{https://www.nytimes3xbfgragh.onion/article/small-business-loans-stimulus-grants-freelancers-coronavirus.html?action=click\&pgtype=Article\&state=default\&region=MAIN_CONTENT_3\&context=storylines_faq}{stimulus
    bills enacted in March} offer help for the millions of American
    small businesses. Those eligible for aid are businesses and
    nonprofit organizations with fewer than 500 workers, including sole
    proprietorships, independent contractors and freelancers. Some
    larger companies in some industries are also eligible. The help
    being offered, which is being managed by the Small Business
    Administration, includes the Paycheck Protection Program and the
    Economic Injury Disaster Loan program. But lots of folks have
    \href{https://www.nytimes3xbfgragh.onion/interactive/2020/05/07/business/small-business-loans-coronavirus.html?action=click\&pgtype=Article\&state=default\&region=MAIN_CONTENT_3\&context=storylines_faq}{not
    yet seen payouts.} Even those who have received help are confused:
    The rules are draconian, and some are stuck sitting on
    \href{https://www.nytimes3xbfgragh.onion/2020/05/02/business/economy/loans-coronavirus-small-business.html?action=click\&pgtype=Article\&state=default\&region=MAIN_CONTENT_3\&context=storylines_faq}{money
    they don't know how to use.} Many small-business owners are getting
    less than they expected or
    \href{https://www.nytimes3xbfgragh.onion/2020/06/10/business/Small-business-loans-ppp.html?action=click\&pgtype=Article\&state=default\&region=MAIN_CONTENT_3\&context=storylines_faq}{not
    hearing anything at all.}
  \end{itemize}
\item ~
  \hypertarget{what-are-my-rights-if-i-am-worried-about-going-back-to-work}{%
  \paragraph{What are my rights if I am worried about going back to
  work?}\label{what-are-my-rights-if-i-am-worried-about-going-back-to-work}}

  \begin{itemize}
  \tightlist
  \item
    Employers have to provide
    \href{https://www.osha.gov/SLTC/covid-19/standards.html}{a safe
    workplace} with policies that protect everyone equally.
    \href{https://www.nytimes3xbfgragh.onion/article/coronavirus-money-unemployment.html?action=click\&pgtype=Article\&state=default\&region=MAIN_CONTENT_3\&context=storylines_faq}{And
    if one of your co-workers tests positive for the coronavirus, the
    C.D.C.} has said that
    \href{https://www.cdc.gov/coronavirus/2019-ncov/community/guidance-business-response.html}{employers
    should tell their employees} -\/- without giving you the sick
    employee's name -\/- that they may have been exposed to the virus.
  \end{itemize}
\item ~
  \hypertarget{what-is-school-going-to-look-like-in-september}{%
  \paragraph{What is school going to look like in
  September?}\label{what-is-school-going-to-look-like-in-september}}

  \begin{itemize}
  \tightlist
  \item
    It is unlikely that many schools will return to a normal schedule
    this fall, requiring the grind of
    \href{https://www.nytimes3xbfgragh.onion/2020/06/05/us/coronavirus-education-lost-learning.html?action=click\&pgtype=Article\&state=default\&region=MAIN_CONTENT_3\&context=storylines_faq}{online
    learning},
    \href{https://www.nytimes3xbfgragh.onion/2020/05/29/us/coronavirus-child-care-centers.html?action=click\&pgtype=Article\&state=default\&region=MAIN_CONTENT_3\&context=storylines_faq}{makeshift
    child care} and
    \href{https://www.nytimes3xbfgragh.onion/2020/06/03/business/economy/coronavirus-working-women.html?action=click\&pgtype=Article\&state=default\&region=MAIN_CONTENT_3\&context=storylines_faq}{stunted
    workdays} to continue. California's two largest public school
    districts --- Los Angeles and San Diego --- said on July 13, that
    \href{https://www.nytimes3xbfgragh.onion/2020/07/13/us/lausd-san-diego-school-reopening.html?action=click\&pgtype=Article\&state=default\&region=MAIN_CONTENT_3\&context=storylines_faq}{instruction
    will be remote-only in the fall}, citing concerns that surging
    coronavirus infections in their areas pose too dire a risk for
    students and teachers. Together, the two districts enroll some
    825,000 students. They are the largest in the country so far to
    abandon plans for even a partial physical return to classrooms when
    they reopen in August. For other districts, the solution won't be an
    all-or-nothing approach.
    \href{https://bioethics.jhu.edu/research-and-outreach/projects/eschool-initiative/school-policy-tracker/}{Many
    systems}, including the nation's largest, New York City, are
    devising
    \href{https://www.nytimes3xbfgragh.onion/2020/06/26/us/coronavirus-schools-reopen-fall.html?action=click\&pgtype=Article\&state=default\&region=MAIN_CONTENT_3\&context=storylines_faq}{hybrid
    plans} that involve spending some days in classrooms and other days
    online. There's no national policy on this yet, so check with your
    municipal school system regularly to see what is happening in your
    community.
  \end{itemize}
\end{itemize}

But Mr. DeWine might not have been have been the ideal candidate for an
antigen test, said Karissa Culbreath, the scientific director of
infectious disease, research and development at TriCore Reference
Laboratories in New Mexico. Such tests usually perform better on samples
that contain high levels of virus, which tend to come from sicker
patients and people at higher risk of transmitting the infection. When
given within the first five days after coronavirus symptoms start,
Quidel's false negative rate
\href{https://www.businesswire.com/news/home/20200717005579/en/Quidel\%E2\%80\%99s-Sofia\%C2\%AE-SARS-Antigen-FIA-Updates-EUA}{may
drop below 5 percent}, according to the company's
\href{https://www.quidel.com/sites/default/files/product/documents/EF1438902EN00.pdf}{intended
use statement}.

Mr. DeWine, however, had not experienced symptoms,
\href{https://www.nytimes3xbfgragh.onion/2020/08/06/us/mike-dewine-coronavirus.html}{aside
from a headache}.

``If we're testing outside of that intended use, we might expect false
positives or false negatives,'' Dr. Culbreath said, referring to the
five-day window that follows the onset of symptoms.

Allocating tests to people who fit that criteria, she added, will also
eliminate the need for scores of follow-up tests, especially while many
suspected cases across the nation remain undiagnosed.

``Tests are not interchangeable in their usefulness,'' Dr. Culbreath
said. ``We need to look at this as a tool belt and identify the right
tool for the job.''

On Sunday, Mr. DeWine did note that antigen tests function especially
well as ``screening'' tests,
\href{https://www.nytimes3xbfgragh.onion/2020/08/06/health/rapid-Covid-tests.html}{expediently
delivering information to people} while their results are confirmed ---
if necessary --- by the more accurate PCR tests.

He added that it was incumbent upon the companies developing the tests
to demonstrate their accuracy, and that the experience would not deter
him from expanding testing in his state.

``We could use additional money for testing,'' Mr. DeWine said. ``We
have doubled our testing the last four weeks. We need to double it, and
then double it again.''

Ohio was
\href{https://www.nytimes3xbfgragh.onion/interactive/2020/us/states-reopen-map-coronavirus.html}{among
the first states to reopen in May}, but as cases ticked up in mid-June
and July, Mr. DeWine signed a statewide
\href{https://www.nytimes3xbfgragh.onion/interactive/2020/us/states-reopen-map-coronavirus.html}{mandatory
mask mandate} and asked several counties to limit gatherings of any
size. There have been at least 99,969 cases and 3,668 deaths in Ohio
since the beginning of the pandemic, according to a
\href{https://www.nytimes3xbfgragh.onion/interactive/2020/us/ohio-coronavirus-cases.html}{New
York Times database}.

The status of testing in the United States is far from ideal, Ms. Prinzi
said. But for now, it's time to make do with the materials we have, she
said. ``We can argue about diagnostic accuracy all day, but this is a
huge public health crisis right now,'' she added.

Flaws and all, antigen tests are ``a necessary part of our management of
the pandemic,'' Dr. Culbreath said. ``But we have to be very intentional
about how we use these tests.''

Advertisement

\protect\hyperlink{after-bottom}{Continue reading the main story}

\hypertarget{site-index}{%
\subsection{Site Index}\label{site-index}}

\hypertarget{site-information-navigation}{%
\subsection{Site Information
Navigation}\label{site-information-navigation}}

\begin{itemize}
\tightlist
\item
  \href{https://help.nytimes3xbfgragh.onion/hc/en-us/articles/115014792127-Copyright-notice}{©~2020~The
  New York Times Company}
\end{itemize}

\begin{itemize}
\tightlist
\item
  \href{https://www.nytco.com/}{NYTCo}
\item
  \href{https://help.nytimes3xbfgragh.onion/hc/en-us/articles/115015385887-Contact-Us}{Contact
  Us}
\item
  \href{https://www.nytco.com/careers/}{Work with us}
\item
  \href{https://nytmediakit.com/}{Advertise}
\item
  \href{http://www.tbrandstudio.com/}{T Brand Studio}
\item
  \href{https://www.nytimes3xbfgragh.onion/privacy/cookie-policy\#how-do-i-manage-trackers}{Your
  Ad Choices}
\item
  \href{https://www.nytimes3xbfgragh.onion/privacy}{Privacy}
\item
  \href{https://help.nytimes3xbfgragh.onion/hc/en-us/articles/115014893428-Terms-of-service}{Terms
  of Service}
\item
  \href{https://help.nytimes3xbfgragh.onion/hc/en-us/articles/115014893968-Terms-of-sale}{Terms
  of Sale}
\item
  \href{https://spiderbites.nytimes3xbfgragh.onion}{Site Map}
\item
  \href{https://help.nytimes3xbfgragh.onion/hc/en-us}{Help}
\item
  \href{https://www.nytimes3xbfgragh.onion/subscription?campaignId=37WXW}{Subscriptions}
\end{itemize}
