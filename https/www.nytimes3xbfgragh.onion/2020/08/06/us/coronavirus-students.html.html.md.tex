\href{/section/us}{U.S.}\textbar{}`I Was a Little Scared': Inside
America's Reopening Schools

\url{https://nyti.ms/2F0Fldt}

\begin{itemize}
\item
\item
\item
\item
\item
\item
\end{itemize}

\href{https://www.nytimes3xbfgragh.onion/news-event/coronavirus?action=click\&pgtype=Article\&state=default\&region=TOP_BANNER\&context=storylines_menu}{The
Coronavirus Outbreak}

\begin{itemize}
\tightlist
\item
  live\href{https://www.nytimes3xbfgragh.onion/2020/08/07/world/covid-19-news.html?action=click\&pgtype=Article\&state=default\&region=TOP_BANNER\&context=storylines_menu}{Latest
  Updates}
\item
  \href{https://www.nytimes3xbfgragh.onion/interactive/2020/us/coronavirus-us-cases.html?action=click\&pgtype=Article\&state=default\&region=TOP_BANNER\&context=storylines_menu}{Maps
  and Cases}
\item
  \href{https://www.nytimes3xbfgragh.onion/interactive/2020/science/coronavirus-vaccine-tracker.html?action=click\&pgtype=Article\&state=default\&region=TOP_BANNER\&context=storylines_menu}{Vaccine
  Tracker}
\item
  \href{https://www.nytimes3xbfgragh.onion/interactive/2020/world/coronavirus-tips-advice.html?action=click\&pgtype=Article\&state=default\&region=TOP_BANNER\&context=storylines_menu}{F.A.Q.}
\item
  \href{https://www.nytimes3xbfgragh.onion/live/2020/08/07/business/stock-market-today-coronavirus?action=click\&pgtype=Article\&state=default\&region=TOP_BANNER\&context=storylines_menu}{Markets
  \& Economy}
\end{itemize}

\includegraphics{https://static01.graylady3jvrrxbe.onion/images/2020/08/06/us/06virus-students1/06virus-students1-articleLarge.jpg?quality=75\&auto=webp\&disable=upscale}

Sections

\protect\hyperlink{site-content}{Skip to
content}\protect\hyperlink{site-index}{Skip to site index}

\hypertarget{i-was-a-little-scared-inside-americas-reopening-schools}{%
\section{`I Was a Little Scared': Inside America's Reopening
Schools}\label{i-was-a-little-scared-inside-americas-reopening-schools}}

In their first week back, students have faced altered classrooms and
emergency quarantines. Here's what they say school is like in the age of
Covid-19.

Kennedy Heim, a~volleyball player in Elwood, Ind., on her first day of
high school. She tested positive for the coronavirus this
week.Credit...Liz Wright

Supported by

\protect\hyperlink{after-sponsor}{Continue reading the main story}

By Adam Wren and
\href{https://www.nytimes3xbfgragh.onion/by/dan-levin}{Dan Levin}

\begin{itemize}
\item
  Published Aug. 6, 2020Updated Aug. 7, 2020, 2:42 p.m. ET
\item
  \begin{itemize}
  \item
  \item
  \item
  \item
  \item
  \item
  \end{itemize}
\end{itemize}

INDIANAPOLIS --- It was the purple Powerade that convinced her.

Kennedy Heim's first day of high school was last Thursday. By the
weekend, her school in central Indiana had already closed its doors,
after a staff member tested positive for the coronavirus and other
employees were required to quarantine.

Kennedy's mother got a call from a contact tracer saying her daughter, a
14-year-old freshman, might have been exposed. So on Monday, they went
for testing at the National Guard Armory, just down the street from her
school. Wednesday morning, they got the results: Kennedy had tested
positive.

``I just felt like I had a cold,'' she said. But a few hours later,
quarantined in her bedroom, with her mother delivering her meals while
masked, Kennedy sipped on some grape Powerade and realized she had a
classic Covid-19 symptom.

``I was trying to hydrate,'' she said, ``and I was like, `Definitely
can't taste that.'''

As the first students return to American classrooms, many face a
profoundly altered experience, where sitting next to a friend on a long
bus ride or unmasking at a busy table in the cafeteria carries a
heightened level of risk.

Most schools have yet to open, and a growing number --- especially in
the nation's largest districts --- are opting to stay online as
caseloads, hospitalizations and deaths continue to climb in their
states. But in some places, including Indiana, Mississippi, Tennessee
and Georgia, students began streaming back into classrooms as early as
last week, with quarantines quickly following.

Elwood Junior-Senior High School in Indiana, where Kennedy attends,
reverted to remote learning after the positive tests --- which now
include at least two students --- were reported, although it plans to
reopen. Just hours into the first day of the new year at Greenfield
Central Junior High School outside Indianapolis, administrators received
word that
\href{https://www.nytimes3xbfgragh.onion/2020/08/01/us/schools-reopening-indiana-coronavirus.html}{a
student had tested positive}.

In North Paulding High School in Dallas, Ga., a series of widely shared
photos showed
\href{https://www.nytimes3xbfgragh.onion/2020/08/06/us/north-paulding-high-school-coronavirus-georgia.html}{students
crowded into packed hallways} during their first days back to class this
week. Few were wearing masks, and there was little sign of social
distancing, generating criticism and outrage in news reports and on
social media.

A 15-year-old student at North Paulding, Hannah Watters, was suspended
for five days for posting some of those images on Twitter, according to
her mother, Lynne Watters, who said she filed a grievance with the
school on Thursday.

``I expressed my concerns and disagreement with that punishment,'' Ms.
Watters said in a text message. The school's principal could not
immediately be reached for comment, but the district's superintendent
defended its reopening plans, saying the photos had been taken out of
context.

Elsewhere, though, where there has been less controversy or reason for
concern, the familiar school day feels, well, familiar. Some student
pioneers told us what it was like to be among the first American
schoolchildren back in classrooms this fall.

\includegraphics{https://static01.graylady3jvrrxbe.onion/images/2020/08/06/us/06VIRUS-STUDENTS-02/06VIRUS-STUDENTS-02-articleLarge.jpg?quality=75\&auto=webp\&disable=upscale}

\hypertarget{jaleah-walker-16-corinth-high-school-in-mississippi}{%
\subsubsection{Jaleah Walker, 16, Corinth High School in
Mississippi}\label{jaleah-walker-16-corinth-high-school-in-mississippi}}

Jaleah Walker had the option to attend her junior year online, but she
had not seen her friends in months, and she knew her challenging course
load would be easier to manage with a teacher in the room.

``I wanted a sense of normalcy,'' she said. So on July 27, Jaleah went
back to school in northern Mississippi --- one of the earliest students
in America to do so. She and her classmates had their temperatures
checked before entering the building, and everyone was required to wear
masks.

Desks were more spaced out, and there were rules for walking through the
hallways. Students ate in their classrooms instead of in the cafeteria.

Despite the changes, Jaleah said things felt pretty normal, and it was a
huge relief to see her friends again. ``We had been texting and
FaceTiming and just ready to see each other.''

But by the end of the first week, a student had tested positive for the
virus. Everyone who had been in contact with that student was sent home
to quarantine.

``It started to be different as soon as the Covid positives came out,
because the classrooms got way slimmer,'' Jaleah said.

By Thursday, the school had six positive cases, the district said. At
her mom's advice, Jaleah is now taking her classes virtually. She does
not know when or if she will be able to return to school.

Image

Reporting in the pandemic has already been challenging for Austin Lines,
who is~editor in chief of his high school yearbook.Credit...Abby Belgin

\hypertarget{austin-lines-18-new-palestine-high-school-in-indiana}{%
\subsubsection{Austin Lines, 18, New Palestine High School in
Indiana}\label{austin-lines-18-new-palestine-high-school-in-indiana}}

As the editor in chief of his high school yearbook and a photographer
for the school newspaper, returning to school is a big story for Austin
Lines.

\hypertarget{the-coronavirus-outbreak}{%
\subsubsection{The Coronavirus
Outbreak}\label{the-coronavirus-outbreak}}

\hypertarget{back-to-school}{%
\paragraph{Back to School}\label{back-to-school}}

Updated Aug. 7, 2020

The latest highlights as the first students return to U.S. schools.

\begin{itemize}
\item
  \begin{itemize}
  \tightlist
  \item
    Schools are open in parts of the country --- and some are
    \href{https://www.nytimes3xbfgragh.onion/2020/08/03/us/school-closing-coronavirus.html?action=click\&pgtype=Article\&state=default\&region=MAIN_CONTENT_2\&context=storylines_keepup}{already
    closing again}. Students have already faced altered classrooms and
    emergency quarantines. We spoke to some to see what
    \href{https://www.nytimes3xbfgragh.onion/2020/08/06/us/coronavirus-students.html?action=click\&pgtype=Article\&state=default\&region=MAIN_CONTENT_2\&context=storylines_keepup}{school
    is like in the age of Covid-19.}
  \item
    Photos of a crowded high school hallway evoked outrage on social
    media. The student who took them
    \href{https://www.nytimes3xbfgragh.onion/2020/08/06/us/north-paulding-high-school-coronavirus-georgia.html?action=click\&pgtype=Article\&state=default\&region=MAIN_CONTENT_2\&context=storylines_keepup}{says
    she was suspended}.
  \item
    Faced with remote learning or socially distanced classroom options,
    some parents of rising kindergartners
    \href{https://www.nytimes3xbfgragh.onion/2020/07/23/parenting/school-opening-kindergarten-coronavirus.html?action=click\&pgtype=Article\&state=default\&region=MAIN_CONTENT_2\&context=storylines_keepup}{are
    considering holding their kids back.}
  \item
    Teachers,
    \href{https://www.nytimes3xbfgragh.onion/2020/08/05/reader-center/teachers-show-us-how-the-coronavirus-is-changing-your-classroom.html?action=click\&pgtype=Article\&state=default\&region=MAIN_CONTENT_2\&context=storylines_keepup}{show
    us your classrooms}! We want to see how educators are preparing to
    keep themselves and students safe if their schools open this fall.
  \end{itemize}
\end{itemize}

But reporting in the pandemic has already been challenging, he said. The
newspaper and yearbook staffs cannot walk around the school as freely as
before, or march up to strangers to ask them questions.

He does not know if photographers or reporters will be allowed at sports
events, either. ``It presents a lot of questions,'' he said.

On the first day of school on Monday, there was already breaking news:
About 20 students had to quarantine for 14 days because they came into
close contact with a student who had tested positive before school
started.

The student's physician gave the family the incorrect date for when the
student could return to school,
\href{https://www.indystar.com/story/news/education/2020/08/03/new-palestine-high-school-sees-coronavirus-case-first-day-school/5575437002/}{local
media reported}. Some of Austin's friends have had to quarantine as a
result, he said.

``It makes me nervous for how quickly everyone is going to be
quarantined and put out of school,'' he said.

Image

``About half try to behave like everything is normal, and the others are
paranoid,'' Ian Whelahan said.Credit...Ian Whelahan,~Alcoa High School

\hypertarget{ian-whelahan-17-alcoa-high-school-in-blount-county-tenn}{%
\subsubsection{Ian Whelahan, 17, Alcoa High School in Blount County,
Tenn.}\label{ian-whelahan-17-alcoa-high-school-in-blount-county-tenn}}

At his school just south of Knoxville, Ian Whelahan said, students seem
evenly split on the dangers of Covid-19.

``About half try to behave like everything is normal, and the others are
paranoid,'' he said, adding, ``I'm one of the paranoid ones.''

At Alcoa High, on-campus classes are limited to one day a week. ``My day
on campus is Tuesday,'' Ian said, ``so I've been to two classes.'' But
it is nothing like what it was before the pandemic.

``The desks are configured so there is plenty of distance between
them,'' he said. ``And we have to wear masks --- except when we're
seated in the classrooms, and then it's optional. And we're encouraged
to hurry between classes, so there's not really any time to talk about
Covid or anything else.''

He said some students do not pay attention to social distancing
requirements, especially at lunch. ``But there is solo seating, which is
what I do,'' he said.

Ian said he just wanted to stay focused on getting safely through his
senior year, ``so I can start college, hopefully when everything is back
to normal.''

\hypertarget{peter-gunter-15-sequoyah-high-school-in-canton-ga}{%
\subsubsection{Peter Gunter, 15, Sequoyah High School in Canton,
Ga.}\label{peter-gunter-15-sequoyah-high-school-in-canton-ga}}

Peter Gunter's sophomore year began on Monday, and masks --- which are
recommended and worn by most students --- were a big topic of
conversation.

``It was very stressed, on the first day and the second day, that there
will be no bullying to any students who decide to wear masks, or to not
wear masks,'' he said, ``and any bullying that is done there could
result in high penalties.''

Other changes abound: The water fountains are blocked off. Classroom
seats are assigned to aid in possible contact tracing. During lunch
period, classes are staggered into two smaller groups, with every other
seat kept empty and no one eating across the table.

``It is kind of hard because I can't have as many friends at the table
at the same time, but it's OK,'' Peter said.

In his orchestra class, where Peter plays the cello, the chairs are
spaced six feet apart. And there is no band or chorus: Wind instruments
and singing could spread infection.

When the pandemic hit, Peter's mother, a former science teacher, made
masks for health care workers. ``My mom and my dad, they've been
stressing about how careful I need to be with this,'' Peter said. So he
washes his hands between every other class, and before and after eating.
``We're taking this very seriously.''

\hypertarget{kennedy-heim-14-elwood-junior-senior-high-school-in-indiana}{%
\subsubsection{Kennedy Heim, 14, Elwood Junior-Senior High School in
Indiana}\label{kennedy-heim-14-elwood-junior-senior-high-school-in-indiana}}

On her second day of quarantine on Thursday, Kennedy said that she was
feeling fatigued but that her case did not seem as bad as others she has
heard about.

``I was a little scared'' after getting the results on Wednesday, she
said.

At least three or four other students she knows of have also tested
positive at the school, she said, but she is not sure how she might have
been infected --- or if she could have infected anyone else.

``It came out of nowhere, and I don't know who else I was around,'' she
said. She went to volleyball practice the week before starting school,
but no one came within six feet of her, she said.

She also diligently wore her mask during her two days at school, she
said, except while at lunch when eating. Whenever she tucked her mask
beneath her nose, she would make sure others were not nearby.

Kennedy's mother, Liz Wright, also started school last week --- she is a
second-grade teacher. Her school remains open even while the high school
is closed for the week, with the students distance learning.

So now she and her daughter have quarantined from each other.

``I'm not going to lie, I have been skeptical about kids getting it,''
Ms. Wright said. ``But to be a part of this pandemic, it is a real
thing. It's not fun to have to FaceTime your daughter in the other
room.''

Adam Wren reported from Indianapolis, and Dan Levin from New York.
Giulia McDonnell Nieto del Rio contributed reporting from New York, Lucy
Tompkins from Bozeman, Mont., and Chris Wohlwend from Knoxville, Tenn.

Advertisement

\protect\hyperlink{after-bottom}{Continue reading the main story}

\hypertarget{site-index}{%
\subsection{Site Index}\label{site-index}}

\hypertarget{site-information-navigation}{%
\subsection{Site Information
Navigation}\label{site-information-navigation}}

\begin{itemize}
\tightlist
\item
  \href{https://help.nytimes3xbfgragh.onion/hc/en-us/articles/115014792127-Copyright-notice}{©~2020~The
  New York Times Company}
\end{itemize}

\begin{itemize}
\tightlist
\item
  \href{https://www.nytco.com/}{NYTCo}
\item
  \href{https://help.nytimes3xbfgragh.onion/hc/en-us/articles/115015385887-Contact-Us}{Contact
  Us}
\item
  \href{https://www.nytco.com/careers/}{Work with us}
\item
  \href{https://nytmediakit.com/}{Advertise}
\item
  \href{http://www.tbrandstudio.com/}{T Brand Studio}
\item
  \href{https://www.nytimes3xbfgragh.onion/privacy/cookie-policy\#how-do-i-manage-trackers}{Your
  Ad Choices}
\item
  \href{https://www.nytimes3xbfgragh.onion/privacy}{Privacy}
\item
  \href{https://help.nytimes3xbfgragh.onion/hc/en-us/articles/115014893428-Terms-of-service}{Terms
  of Service}
\item
  \href{https://help.nytimes3xbfgragh.onion/hc/en-us/articles/115014893968-Terms-of-sale}{Terms
  of Sale}
\item
  \href{https://spiderbites.nytimes3xbfgragh.onion}{Site Map}
\item
  \href{https://help.nytimes3xbfgragh.onion/hc/en-us}{Help}
\item
  \href{https://www.nytimes3xbfgragh.onion/subscription?campaignId=37WXW}{Subscriptions}
\end{itemize}
