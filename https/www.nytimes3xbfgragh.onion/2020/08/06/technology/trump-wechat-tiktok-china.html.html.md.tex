Sections

SEARCH

\protect\hyperlink{site-content}{Skip to
content}\protect\hyperlink{site-index}{Skip to site index}

\href{https://www.nytimes3xbfgragh.onion/section/technology}{Technology}

\href{https://myaccount.nytimes3xbfgragh.onion/auth/login?response_type=cookie\&client_id=vi}{}

\href{https://www.nytimes3xbfgragh.onion/section/todayspaper}{Today's
Paper}

\href{/section/technology}{Technology}\textbar{}Trump Targets WeChat and
TikTok, in Sharp Escalation With China

\url{https://nyti.ms/3ifwdjl}

\begin{itemize}
\item
\item
\item
\item
\item
\item
\end{itemize}

Advertisement

\protect\hyperlink{after-top}{Continue reading the main story}

Supported by

\protect\hyperlink{after-sponsor}{Continue reading the main story}

\hypertarget{trump-targets-wechat-and-tiktok-in-sharp-escalation-with-china}{%
\section{Trump Targets WeChat and TikTok, in Sharp Escalation With
China}\label{trump-targets-wechat-and-tiktok-in-sharp-escalation-with-china}}

The government cited national security concerns in announcing sweeping
restrictions on two popular Chinese social media networks, a move that
is likely to be met with retaliation.

\includegraphics{https://static01.graylady3jvrrxbe.onion/images/2020/08/06/business/06dc-chinaban1/merlin_175366002_fe1eaecc-4bf3-4dfe-8538-24ad1d4024bb-articleLarge.jpg?quality=75\&auto=webp\&disable=upscale}

\href{https://www.nytimes3xbfgragh.onion/by/ana-swanson}{\includegraphics{https://static01.graylady3jvrrxbe.onion/images/2018/12/10/multimedia/author-ana-swanson/author-ana-swanson-thumbLarge.png}}\href{https://www.nytimes3xbfgragh.onion/by/mike-isaac}{\includegraphics{https://static01.graylady3jvrrxbe.onion/images/2018/02/16/multimedia/author-mike-isaac/author-mike-isaac-thumbLarge.jpg}}\href{https://www.nytimes3xbfgragh.onion/by/paul-mozur}{\includegraphics{https://static01.graylady3jvrrxbe.onion/images/2018/10/15/multimedia/author-paul-mozur/author-paul-mozur-thumbLarge.png}}

By \href{https://www.nytimes3xbfgragh.onion/by/ana-swanson}{Ana
Swanson}, \href{https://www.nytimes3xbfgragh.onion/by/mike-isaac}{Mike
Isaac} and \href{https://www.nytimes3xbfgragh.onion/by/paul-mozur}{Paul
Mozur}

\begin{itemize}
\item
  Published Aug. 6, 2020Updated Aug. 7, 2020, 4:57 a.m. ET
\item
  \begin{itemize}
  \item
  \item
  \item
  \item
  \item
  \item
  \end{itemize}
\end{itemize}

\href{https://cn.nytimes3xbfgragh.onion/usa/20200807/trump-wechat-tiktok-china/}{阅读简体中文版}\href{https://cn.nytimes3xbfgragh.onion/usa/20200807/trump-wechat-tiktok-china/zh-hant/}{閱讀繁體中文版}

WASHINGTON --- The Trump administration has announced sweeping
restrictions on two popular Chinese social media networks,
\href{https://www.nytimes3xbfgragh.onion/2020/08/07/business/trump-china-wechat-tiktok.html}{TikTok
and WeChat}, a sharp escalation of its confrontation with China that is
likely to be met with retaliation.

Two executive orders, released late Thursday and taking effect in 45
days, cited national security concerns to bar any transactions with
WeChat or TikTok by any person or involving any property subject to the
jurisdiction of the United States. The order essentially sets a 45-day
deadline for an acquisition of TikTok, which is
\href{https://www.nytimes3xbfgragh.onion/2020/08/03/technology/trump-tiktok-microsoft.html}{in
talks to be acquired by Microsoft}.

Tensions between the United States and China have already escalated to
levels not seen in decades over rifts in geopolitics, technology and
trade. In recent months, Trump administration officials have challenged
China on its crackdown in Hong Kong, its territorial claims in the South
China Sea and its efforts to produce global tech champions. The campaign
has been provoked in part by China's more assertive posture, but also
President Trump's desire to convince voters that he is tough on China as
the election approaches.

Mr. Trump's advisers have zeroed in on technology companies, which they
say are beholden to the Chinese government through security laws. Many
companies that do business across the Pacific have been left paralyzed
or begun to reconsider their partnerships, unsure of whether these
tensions will spill into a new Cold War. The restrictions announced
Thursday would also represent a further balkanization of the global
internet, as nations continue to cut off foreign technology companies
from one another's markets.

In the announcement, Mr. Trump accused WeChat, made by Tencent, and
TikTok, made by ByteDance, of providing a channel for the Chinese
Communist Party to obtain Americans' proprietary information, keep tabs
on Chinese citizens abroad and carry out disinformation campaigns to
benefit China's interest.

``The spread in the United States of mobile applications developed and
owned by companies in the People's Republic of China (China) continues
to threaten the national security, foreign policy and economy of the
United States,'' the president wrote.

Much remains unclear about the scope of the ban, including precisely
which transactions would be covered. But it appears to have even more
severe consequences for WeChat than for TikTok, which could be rescued
through its talks with an American suitor.

WeChat is used widely around the world, particularly by people of
Chinese descent, to communicate with friends, read news and carry out
business transactions, and such a ban could effectively cut off much
informal communication between people in China and the United States.
Questions remain as to whether the order will affect businesses tied to
Tencent, WeChat's parent company, which is an investor in many popular
American technology and gaming start-ups.

``We are reviewing the executive order to get a full understanding,'' a
Tencent representative said early Friday. The company's shares fell
almost 6 percent in trading on Friday.

A press officer for Microsoft declined to comment.

In a statement on Friday, TikTok said it was ``shocked'' by the
executive order, which it said risked ``undermining global businesses'
trust in the United States' commitment to the rule of law.''

``We will pursue all remedies available to us,'' the statement said,
``in order to ensure that the rule of law is not discarded and that our
company and our users are treated fairly --- if not by the
administration, then by the U.S. courts.''

At a daily news briefing Friday, the Chinese Ministry of Foreign Affairs
spokesman Wang Wenbin called the executive orders a ``nakedly hegemonic
act'' and added, ``on the pretext of national security, the U.S.
frequently abuses national power and unreasonably suppresses relevant
enterprises.''

TikTok is in talks with at least three other American companies,
including Microsoft, regarding a potential acquisition of TikTok's
business. Last week, Microsoft said it planned to pursue the
negotiations for a purchase of TikTok's service in the United States,
Canada, Australia and New Zealand, and would do so by Sept. 15.

Mr. Trump for weeks has been urged to intervene with TikTok, and by a
range of advisers. Many of those advisers, including Treasury Secretary
Steven Mnuchin, had counseled Mr. Trump to follow the recommendations of
a national security panel, the Committee on Foreign Investment in the
United States, and allow Microsoft or another suitor to buy the
Chinese-owned service.

But other advisers, like the White House trade adviser Peter Navarro,
pushed for more sweeping action. By Friday evening, as the president
flew back to Washington from Florida, Mr. Trump told reporters that he
did not want TikTok to be acquired by an American company and that he
would use his presidential authority to bar TikTok from operating in the
United States.

That position did not last long. Mr. Mnuchin and other officials
scrambled to find people who would intervene with the president,
imploring people like Senator Lindsey Graham, Republican from South
Carolina, to explain to the president why the Microsoft deal was a good
option. Mr. Graham and Mr. Mnuchin cautioned Mr. Trump about a risky
political calculation if TikTok simply went dark.

By Sunday,
\href{https://www.nytimes3xbfgragh.onion/2020/08/02/business/economy/trump-tiktok-china-national-security.html}{Mr.
Trump had come around}. But he has never seemed completely settled on
one approach.

The threat of an outright ban on transactions is a serious blow for
ByteDance and Zhang Yiming, the company's chief executive, whose goal
for years has been to connect the world through his various consumer
apps. Nicknamed the
``\href{https://qz.com/1803609/all-the-apps-run-by-tiktoks-chinese-owner-bytedance/\#:~:text=Although\%20most\%20often\%20described\%20as,news\%20aggregator\%20to\%20productivity\%20management}{app
factory}'' in China, ByteDance is home to more than 20 apps, including
personal financial apps and productivity programs.

TikTok is far and away the crown jewel of ByteDance's portfolio. Used by
more than 800 million people globally, TikTok grew popular for its
short, catchy videos that spread quickly and virally over social media
channels. Mr. Zhang took steps to allow TikTok's presence in some of the
world's most important consumer markets, like storing user data on
servers in Virginia and Singapore, and hiring heads of business in the
United States.

For many in China, the ban of WeChat will be a bigger deal. TikTok does
not operate in China, where ByteDance instead offers an equivalent
service, called Douyin.

WeChat, on the other hand, spans Beijing's system of internet filters,
connecting communities within and outside China. Exchange students use
it to keep in touch with their families, investors use it to broker
deals, and diaspora communities rely on it to keep in touch with
relatives. Within China there are few alternatives to WeChat, because
most other international messaging apps are blocked.

The order appeared to ban transactions between U.S. companies and
Tencent, the owner of WeChat. Such a block would become a major
difficulty for American firms in China, which use the ubiquitous WeChat
social media app to do marketing, advertising and after-sales service.

Tencent is also widely invested in American gaming and social media
companies, including Snap, Activision Blizzard and the makers of
Fortnite, Clash of Clans and League of Legends. It's not clear how the
order might affect such investments.

Over the years WeChat went from copycat chat app to a force all its own.
Crammed with services that enable online payments, e-commerce orders and
other services, it grew into an inspiration for Silicon Valley.
Companies like Facebook followed some of its cues in adding features to
their own messaging apps.

Yet WeChat has also long been used by the police in Beijing to track
dissidents. More recently, the app has emerged as a data conduit for the
newly empowered internet police, who examine discussions for signs of
political disloyalty. The app is also heavily censored, turning it into
a sort of state-controlled filter bubble. Rumors not acceptable by
Beijing are quashed, while others are left to spread.

Concern has been growing among Trump administration officials that
WeChat offers the Chinese government not merely a way to gather data and
information within the United States, but also a potent channel for
spreading alternative narratives and disinformation. Matthew Pottinger,
the deputy White House national security adviser, and Mr. Navarro have
both been strong supporters of the executive orders.

But the national security cases against TikTok and WeChat are far from
clear. Even within the national security community --- and the nation's
intelligence agencies --- there are doubts that the United States can
successfully cut its networks and technologies off from China. There is
also a realization that a good number of communications will run over
Chinese-controlled computers, networks and switches no matter what the
U.S. government does.

``While TikTok is being singled out in this executive order, their data
collection and sharing practices are fairly standard in the industry,''
said Kirsten Martin, a professor on technology ethics at the University
of Notre Dame's business school. ``In fact, many fitness apps were
banned from use in the military for tracking location data, but we did
not ban them from all U.S. citizens.''

It's not yet clear how a ban on WeChat or TikTok would be enforced, but
users could adopt the same tactics many take in China when a service is
banned and look for workarounds. If the app is taken down from app
stores, people might find other ways to download it. If its use is
blocked, they can turn to services that mask the origin of an internet
connection. Even so, as Beijing has found out, many lack the savvy and
patience for such technical fixes, and may cease using the service.

Ana Swanson reported from Washington, Mike Isaac from San Francisco, and
Paul Mozur from Taipei, Taiwan. Maggie Haberman, David Sanger and
Raymond Zhong contributed reporting.

Advertisement

\protect\hyperlink{after-bottom}{Continue reading the main story}

\hypertarget{site-index}{%
\subsection{Site Index}\label{site-index}}

\hypertarget{site-information-navigation}{%
\subsection{Site Information
Navigation}\label{site-information-navigation}}

\begin{itemize}
\tightlist
\item
  \href{https://help.nytimes3xbfgragh.onion/hc/en-us/articles/115014792127-Copyright-notice}{©~2020~The
  New York Times Company}
\end{itemize}

\begin{itemize}
\tightlist
\item
  \href{https://www.nytco.com/}{NYTCo}
\item
  \href{https://help.nytimes3xbfgragh.onion/hc/en-us/articles/115015385887-Contact-Us}{Contact
  Us}
\item
  \href{https://www.nytco.com/careers/}{Work with us}
\item
  \href{https://nytmediakit.com/}{Advertise}
\item
  \href{http://www.tbrandstudio.com/}{T Brand Studio}
\item
  \href{https://www.nytimes3xbfgragh.onion/privacy/cookie-policy\#how-do-i-manage-trackers}{Your
  Ad Choices}
\item
  \href{https://www.nytimes3xbfgragh.onion/privacy}{Privacy}
\item
  \href{https://help.nytimes3xbfgragh.onion/hc/en-us/articles/115014893428-Terms-of-service}{Terms
  of Service}
\item
  \href{https://help.nytimes3xbfgragh.onion/hc/en-us/articles/115014893968-Terms-of-sale}{Terms
  of Sale}
\item
  \href{https://spiderbites.nytimes3xbfgragh.onion}{Site Map}
\item
  \href{https://help.nytimes3xbfgragh.onion/hc/en-us}{Help}
\item
  \href{https://www.nytimes3xbfgragh.onion/subscription?campaignId=37WXW}{Subscriptions}
\end{itemize}
