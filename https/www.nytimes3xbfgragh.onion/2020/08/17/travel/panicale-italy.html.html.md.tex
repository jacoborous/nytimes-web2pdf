Sections

SEARCH

\protect\hyperlink{site-content}{Skip to
content}\protect\hyperlink{site-index}{Skip to site index}

\href{/section/travel}{Travel}\textbar{}Quiet Reflections on the
Enchanting Italian Village of Panicale

\url{https://nyti.ms/34aVdo7}

\begin{itemize}
\item
\item
\item
\item
\item
\item
\end{itemize}

\includegraphics{https://static01.graylady3jvrrxbe.onion/images/2020/08/17/travel/17travel-panicale-01/merlin_174878943_8e5ac44c-e9cd-447e-a82d-2e3f3411e9e4-articleLarge.jpg?quality=75\&auto=webp\&disable=upscale}

The World Through a Lens

\hypertarget{quiet-reflections-on-the-enchanting-italian-village-of-panicale}{%
\section{Quiet Reflections on the Enchanting Italian Village of
Panicale}\label{quiet-reflections-on-the-enchanting-italian-village-of-panicale}}

Pensive after the loss of a friend, a photographer finds solace --- and
instruction ---~on a trip to the central Italian region of Umbria.

The hilltop village of Panicale overlooks a vast Umbrian agricultural
valley.Credit...

Supported by

\protect\hyperlink{after-sponsor}{Continue reading the main story}

Photographs and Text by Barry L. Schwartz

\begin{itemize}
\item
  Aug. 17, 2020
\item
  \begin{itemize}
  \item
  \item
  \item
  \item
  \item
  \item
  \end{itemize}
\end{itemize}

\emph{At the onset of the coronavirus pandemic, with travel restrictions
in place worldwide, we launched a new series ---}
\href{https://www.nytimes3xbfgragh.onion/column/the-world-through-a-lens}{\emph{The
World Through a Lens}} \emph{--- in which photojournalists help
transport you, virtually, to some of our planet's most beautiful and
intriguing places. This week, Barry L. Schwartz shares a collection of
images from the central Italian region of Umbria.}

\begin{center}\rule{0.5\linewidth}{\linethickness}\end{center}

In 2017, a few months after we got married, my wife, Maggie, and I took
a six-week trip --- part honeymoon, part yearslong delayed vacation.
Leaving California, our first stop was in Brooklyn to see Maggie's
oldest friend, who we all knew was not going to survive a cancer that
had returned after many years. It was a good visit.

Next, we flew to Barcelona and drove to a small coastal town, Sitges.
While there, I learned that one of my oldest friends had just died, also
from cancer, also at the end of a series of treatments.

A few days later we flew to Florence, driving a few hours south to
Panicale, a small hilltop town in Umbria. A friend --- Steve Siegelman,
a food writer in California --- had lent us his renovated
brick-and-stone rowhouse in what he jokingly refers to as the ``new
neighborhood,'' because it was built in the 1500s, while the town's main
center, the piazza, dates back to the 10th century.

\includegraphics{https://static01.graylady3jvrrxbe.onion/images/2020/08/17/travel/17travel-panicale-04/merlin_174878652_89646c8b-681c-4c1d-b8d5-7900dfae370b-articleLarge.jpg?quality=75\&auto=webp\&disable=upscale}

Image

The town square is the center of life in Panicale. The tourist office,
restaurants and shops are either in the piazza or just outside.

Steve had discovered Panicale on a trip with his parents years before.
Rural hilltop towns all over Europe have been
\href{https://www.nytimes3xbfgragh.onion/2017/09/07/t-magazine/abandoned-italian-towns.html}{emptying
out for decades} as people move away to make a living, leaving houses to
be bought by Americans, Britons and Germans as primary or second homes.
They, along with the locals, help keep Panicale alive.

Image

Originally the town's cistern, this fountain --- the centerpiece of
Piazza Umberto I --- was rebuilt and repurposed in 1903.

Image

Brick and stone walls represent centuries of repairs and modifications.

Early on, we badly needed to do our laundry. The washing machine in the
basement could not be convinced to do the job, resulting in texts to
Steve in California asking whom to call for help. He wrote back that his
local fixer would get a plumber over at some point; in the meantime, he
put us in contact with an expat couple, dear friends of his, Elida and
Guenter, a half-mile away, with an olive grove and a brick house
overlooking a valley. They immediately invited us to come for a meal and
to use their laundry machines, which were set into a hillside like a
wine cellar.

Image

A restored farmhouse just outside the town walls.

Image

Steve supplies his guests with a 21-page manual: how the house works,
where to go, whom to call for advice and help. At the time, there were
three grocers in town, and we were instructed to buy from each, as
everyone in town did, partly to keep them in business and partly because
everyone is so nice. (Iolanda's had great fresh fruits and vegetables.)

Image

At the time, there were three grocers in town, including Iolanda
Mencarelli.

Unlike Maggie, I had never been to Italy. Raised in Los Angeles, I've
had a lifelong obsession with authenticity, an elusive quality in my
hometown. It was a balm to find cobblestone streets and peeling plaster
walls that were not aged by artificial means, and to buy ordinary fruits
and vegetables, not ``heirlooms.''

Image

Streets and stairways follow the contours of the hill, creating
buildings that conform to the landscape and the routes of passage.

Image

Flower pots are a common sight, on both the ground and the walls.

One place in particular I documented at all times of the day: a spot
where four streets converged at a short wall, below which sat a garden.
The wall provided an overlook to the agricultural plain to the north
---~toward the town of Castiglione del Lago, on Lake Trasimeno.

A few days after arriving, we were invited by friends of Elida and
Guenter to a meal in that garden; arriving, I was a little thrilled to
realize I had photographed their garden wall and front door many times.

Image

High above the streets, the loggia of one of the town's most prominent
palazzi looks north to Lake Trasimeno.

Image

The view from the topmost point in town, Palazzo del Podestà.

Aldo Gallo and his wife, Daniela, own
\href{https://goo.gl/maps/qadL6PgxrYhmUg5s6}{Bar Gallo}, the most likely
of all places on the piazza to be open early and late. Their son,
Simone, and his wife, Lorena, have a restaurant across the way. There
are several others places to eat off the piazza, too, and they are all
good. Aldo made us feel welcome from our first day to our last.

Image

Aldo Gallo and his wife, Daniela, own Bar Gallo.

Steve suggested we take an official tour of the town. When we did, we
had the guide to ourselves: a young Italian woman in a graduate art
history program who gave tours as a summer job.

We walked first to a 200-seat proscenium theater,
\href{https://www.teatrostabile.umbria.it/teatro/teatro-caporali/}{Teatro
Caporali}, built in 1786 by 12 families; each had a box. Maggie, a
costume designer, was thrilled. The theater is still used by local and
touring performers.

Image

An Italian guide led us through the town.

Image

One of our stops was at Teatro Caporali, a 200-seat proscenium theater
built in 1786. (Closed in 1958 after being declared unsafe, it received
funding from the Umbrian government and reopened in 1994.)

Image

The theater's classic horseshoe design is based on the grand opera
houses of Italy. Musical and theatrical performances are presented here
year-round.

The \href{https://goo.gl/maps/S8NSxw9ongY7jzqWA}{Tulle Museum}, housed
in a deconsecrated church, Sant'Agostino, is dedicated to the town's
association with lace and embroidery. A concert grand piano stood where
the altar had been. The piano belonged to
\href{https://www.youtube.com/watch?v=RI0B1QOdKww}{Dalia Lazar, a
visiting artist}; she bought it from the family of George Gershwin and
shipped to Panicale for the year of her residency. The museum was the
only place large enough for the piano, and the building was used for
concerts. Later that day we heard Ms. Lazar practicing as we walked by.

Image

Inside the Tulle Museum. Panicale is associated with the production of
lace and embroidery.

We walked to another church, Madonna della Sbarra, the columns of which
are painted to look like marble. The ceiling, a beehive grid, was also
\href{https://www.tate.org.uk/art/art-terms/t/trompe-loeil}{trompe
l'oeil}; I had to look hard to see it was smooth plaster. The altar was
intricately carved and multihued, its paint still bright.

Image

The Madonna della Sbarra. Above the church is a small museum.

Upstairs was a small museum with a few paintings, and behind glass in
the old hermitages were an assortment of artifacts: a Bible, censers,
goblets, well-preserved silk vestments.

Image

The museum houses an assortment of artifacts, including preserved silk
vestments.

Image

Maggie and I married in middle age,~the first marriage for both of us.
In the years preceding our wedding, we each buried our mothers, other
relatives, a few friends. Not unusual at our age. Our ill friends were
part of the inspiration for the trip; while we remained healthy and
ambulatory, it was time to take our version of a Grand Tour.

Image

Buttresses help steady the stone and brick walls and are topped with
tile for protection from the weather.

In that way, wandering around a thousand-year-old town was instructive.
Beyond the too-obvious metaphor of surviving well into old age, there
remained a lot of life and beauty in the old stone walls, in the people
we met, in the sky above the plain, which stretched ---~crowded with
farms --- to the horizon.

Image

Lake Trasimeno, in the distance, at sunset.

\href{https://barryschwartzphoto.com/}{\emph{Barry L. Schwartz}}
\emph{is a photographer based in Napa, Calif. You can follow his work
on}
\href{https://www.instagram.com/barryschwartz1/}{\emph{Instagram}}\emph{.}

\emph{\textbf{Follow New York Times Travel}} \emph{on}
\href{https://www.instagram.com/nytimestravel/}{\emph{Instagram}}\emph{,}
\href{https://twitter.com/nytimestravel}{\emph{Twitter}} \emph{and}
\href{https://www.facebookcorewwwi.onion/nytimestravel/}{\emph{Facebook}}\emph{.
And}
\href{https://www.nytimes3xbfgragh.onion/newsletters/traveldispatch}{\emph{sign
up for our weekly Travel Dispatch newsletter}} \emph{to receive expert
tips on traveling smarter and inspiration for your next vacation.}

Advertisement

\protect\hyperlink{after-bottom}{Continue reading the main story}

\hypertarget{site-index}{%
\subsection{Site Index}\label{site-index}}

\hypertarget{site-information-navigation}{%
\subsection{Site Information
Navigation}\label{site-information-navigation}}

\begin{itemize}
\tightlist
\item
  \href{https://help.nytimes3xbfgragh.onion/hc/en-us/articles/115014792127-Copyright-notice}{©~2020~The
  New York Times Company}
\end{itemize}

\begin{itemize}
\tightlist
\item
  \href{https://www.nytco.com/}{NYTCo}
\item
  \href{https://help.nytimes3xbfgragh.onion/hc/en-us/articles/115015385887-Contact-Us}{Contact
  Us}
\item
  \href{https://www.nytco.com/careers/}{Work with us}
\item
  \href{https://nytmediakit.com/}{Advertise}
\item
  \href{http://www.tbrandstudio.com/}{T Brand Studio}
\item
  \href{https://www.nytimes3xbfgragh.onion/privacy/cookie-policy\#how-do-i-manage-trackers}{Your
  Ad Choices}
\item
  \href{https://www.nytimes3xbfgragh.onion/privacy}{Privacy}
\item
  \href{https://help.nytimes3xbfgragh.onion/hc/en-us/articles/115014893428-Terms-of-service}{Terms
  of Service}
\item
  \href{https://help.nytimes3xbfgragh.onion/hc/en-us/articles/115014893968-Terms-of-sale}{Terms
  of Sale}
\item
  \href{https://spiderbites.nytimes3xbfgragh.onion}{Site Map}
\item
  \href{https://help.nytimes3xbfgragh.onion/hc/en-us}{Help}
\item
  \href{https://www.nytimes3xbfgragh.onion/subscription?campaignId=37WXW}{Subscriptions}
\end{itemize}
