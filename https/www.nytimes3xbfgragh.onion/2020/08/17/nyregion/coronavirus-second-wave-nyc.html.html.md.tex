Sections

SEARCH

\protect\hyperlink{site-content}{Skip to
content}\protect\hyperlink{site-index}{Skip to site index}

\href{https://www.nytimes3xbfgragh.onion/section/nyregion}{New York}

\href{https://myaccount.nytimes3xbfgragh.onion/auth/login?response_type=cookie\&client_id=vi}{}

\href{https://www.nytimes3xbfgragh.onion/section/todayspaper}{Today's
Paper}

\href{/section/nyregion}{New York}\textbar{}New York Has Tamed the
Virus. Can It Hold Off a Second Wave?

\url{https://nyti.ms/2Y8tuAA}

\begin{itemize}
\item
\item
\item
\item
\item
\item
\end{itemize}

\hypertarget{the-coronavirus-outbreak}{%
\subsubsection{\texorpdfstring{\href{https://www.nytimes3xbfgragh.onion/news-event/coronavirus?name=styln-coronavirus-national\&region=TOP_BANNER\&variant=undefined\&block=storyline_menu_recirc\&action=click\&pgtype=Article\&impression_id=01883f40-e395-11ea-a793-353d552722af}{The
Coronavirus
Outbreak}}{The Coronavirus Outbreak}}\label{the-coronavirus-outbreak}}

\begin{itemize}
\tightlist
\item
  live\href{https://www.nytimes3xbfgragh.onion/2020/08/21/world/covid-19-coronavirus.html?name=styln-coronavirus-national\&region=TOP_BANNER\&variant=undefined\&block=storyline_menu_recirc\&action=click\&pgtype=Article\&impression_id=01886650-e395-11ea-a793-353d552722af}{Latest
  Updates}
\item
  \href{https://www.nytimes3xbfgragh.onion/interactive/2020/us/coronavirus-us-cases.html?name=styln-coronavirus-national\&region=TOP_BANNER\&variant=undefined\&block=storyline_menu_recirc\&action=click\&pgtype=Article\&impression_id=01886651-e395-11ea-a793-353d552722af}{Maps
  and Cases}
\item
  \href{https://www.nytimes3xbfgragh.onion/interactive/2020/science/coronavirus-vaccine-tracker.html?name=styln-coronavirus-national\&region=TOP_BANNER\&variant=undefined\&block=storyline_menu_recirc\&action=click\&pgtype=Article\&impression_id=01886652-e395-11ea-a793-353d552722af}{Vaccine
  Tracker}
\item
  \href{https://www.nytimes3xbfgragh.onion/2020/08/19/us/colleges-closing-covid.html?name=styln-coronavirus-national\&region=TOP_BANNER\&variant=undefined\&block=storyline_menu_recirc\&action=click\&pgtype=Article\&impression_id=01886653-e395-11ea-a793-353d552722af}{Colleges
  Closing}
\item
  \href{https://www.nytimes3xbfgragh.onion/live/2020/08/20/business/stock-market-today-coronavirus?name=styln-coronavirus-national\&region=TOP_BANNER\&variant=undefined\&block=storyline_menu_recirc\&action=click\&pgtype=Article\&impression_id=01886654-e395-11ea-a793-353d552722af}{Economy}
\end{itemize}

Advertisement

\protect\hyperlink{after-top}{Continue reading the main story}

Supported by

\protect\hyperlink{after-sponsor}{Continue reading the main story}

\hypertarget{new-york-has-tamed-the-virus-can-it-hold-off-a-second-wave}{%
\section{New York Has Tamed the Virus. Can It Hold Off a Second
Wave?}\label{new-york-has-tamed-the-virus-can-it-hold-off-a-second-wave}}

The sustained low rate of infection has surprised local health
officials. But a resurgence may be inevitable, despite the state's and
city's best efforts.

\includegraphics{https://static01.graylady3jvrrxbe.onion/images/2020/08/03/nyregion/00nyvirus-challenges-1/merlin_174750792_5f7d234e-25e0-46ce-8e0a-f5d63eda27f6-articleLarge.jpg?quality=75\&auto=webp\&disable=upscale}

\href{https://www.nytimes3xbfgragh.onion/by/j-david-goodman}{\includegraphics{https://static01.graylady3jvrrxbe.onion/images/2018/07/18/nyregion/author-j-david-goodman/author-j-david-goodman-thumbLarge.png}}

By \href{https://www.nytimes3xbfgragh.onion/by/j-david-goodman}{J. David
Goodman}

\begin{itemize}
\item
  Aug. 17, 2020
\item
  \begin{itemize}
  \item
  \item
  \item
  \item
  \item
  \item
  \end{itemize}
\end{itemize}

Health experts in New York City thought that coronavirus cases would be
rising again by now. Their models predicted it. They were wrong.

New York State has managed not only to control its outbreak since the
devastation of the early spring, but also to contain it for far longer
than even top officials expected.

Now, as other places struggle to beat back a resurgence and cases climb
in former success-story states like California and Rhode Island, New
York's leaders are consumed by the likelihood that, any day now, their
numbers will begin rising.

The current levels of infection are so remarkable that they have
surprised state and city officials: Around 1 percent of the roughly
30,000 tests each day in the city are positive for the virus. In Los
Angeles, it's 7 percent, while it's 13 percent in Miami-Dade County and
around 15 percent in Houston.

The virus is simply no longer as present in New York as it once was,
epidemiologists and public health officials said.

``New York is like our South Korea now,'' said Dr. Thomas Tsai of the
Harvard Global Health Institute.

But nothing is static about the viral outbreak, experts cautioned. The
question now is whether the state, where 32,000 people have died of the
virus, can keep from being overwhelmed by another wave, as threats loom
from arriving travelers, struggles with contact tracing and rising cases
just over the Hudson River in New Jersey.

\includegraphics{https://static01.graylady3jvrrxbe.onion/images/2020/08/03/nyregion/00nyvirus-challenges-2/merlin_174173817_86f256ab-5632-471f-9536-652406c9a40f-articleLarge.jpg?quality=75\&auto=webp\&disable=upscale}

Officials have also been watching warily as cities once seen as models
in virus containment struggled with new outbreaks. Hong Kong
\href{https://www.bloomberg.com/news/articles/2020-07-27/hong-kong-to-ban-dining-in-require-masks-outside-cable-tv-says}{moved
to ban indoor dining and gatherings of more than two people} in late
July amid a sharp rise in infections. International flights
\href{https://www.melbourneairport.com.au/Passengers/Passenger-information/COVID-19}{were
diverted from Melbourne, Australia}, as cases mounted.

In more than a dozen interviews, epidemiologists, public health
officials and infectious disease specialists said New York owed its
current success in large part to how New Yorkers reacted to the
viciousness with which the virus attacked the state in April.

\hypertarget{latest-updates-the-coronavirus-outbreak}{%
\section{\texorpdfstring{\href{https://www.nytimes3xbfgragh.onion/2020/08/21/world/covid-19-coronavirus.html?action=click\&pgtype=Article\&state=default\&region=MAIN_CONTENT_1\&context=storylines_live_updates}{Latest
Updates: The Coronavirus
Outbreak}}{Latest Updates: The Coronavirus Outbreak}}\label{latest-updates-the-coronavirus-outbreak}}

Updated 2020-08-21T09:57:24.778Z

\begin{itemize}
\tightlist
\item
  \href{https://www.nytimes3xbfgragh.onion/2020/08/21/world/covid-19-coronavirus.html?action=click\&pgtype=Article\&state=default\&region=MAIN_CONTENT_1\&context=storylines_live_updates\#link-4690b6aa}{Shutdowns,
  warnings and scoldings follow gatherings on college campuses.}
\item
  \href{https://www.nytimes3xbfgragh.onion/2020/08/21/world/covid-19-coronavirus.html?action=click\&pgtype=Article\&state=default\&region=MAIN_CONTENT_1\&context=storylines_live_updates\#link-324af071}{As
  he accepts the Democratic nomination, Biden knocks Trump's pandemic
  response.}
\item
  \href{https://www.nytimes3xbfgragh.onion/2020/08/21/world/covid-19-coronavirus.html?action=click\&pgtype=Article\&state=default\&region=MAIN_CONTENT_1\&context=storylines_live_updates\#link-35890b73}{Hundreds
  of doctors in Kenya go on strike over their pay and protective gear.}
\end{itemize}

\href{https://www.nytimes3xbfgragh.onion/2020/08/21/world/covid-19-coronavirus.html?action=click\&pgtype=Article\&state=default\&region=MAIN_CONTENT_1\&context=storylines_live_updates}{See
more updates}

More live coverage:
\href{https://www.nytimes3xbfgragh.onion/live/2020/08/20/business/stock-market-today-coronavirus?action=click\&pgtype=Article\&state=default\&region=MAIN_CONTENT_1\&context=storylines_live_updates}{Markets}

State officials shut down schools and businesses, sacrificing jobs and
weakening the economy to save lives. Adherence to mask wearing has been
strong. Many vulnerable New Yorkers are still sheltering in their
apartments. Others decamped to second homes.

And, critically, Gov. Andrew M. Cuomo and Mayor Bill de Blasio reopened
cautiously, deciding in late June against allowing indoor dining and
bars after seeing those activities connected to outbreaks in other
states.

``People in New York have taken matters much more seriously than in
other places,'' said Dr. Howard Markel, a historian of epidemics at the
University of Michigan. ``And all they're doing is reducing the risk.
They're not extinguishing the virus.''

Image

More New Yorkers are riding the subway, though numbers are still way
below pre-pandemic levels.Credit...Victor J. Blue for The New York Times

Still a resurgence is all but inevitable, public health experts said.

Local beaches have filled on hot weekend days. Diners flock to outdoor
restaurants with plywood patios. More than 1.2 million people took the
subway on a recent Tuesday, down dramatically compared to a year ago,
but more than double what it was on a Tuesday in May.

The same models that predicted an increase in New York City for the
summer now see a rise coming in the early fall. Life can be lived
outside for now, but will move indoors as the weather cools --- just as
the flu season is ramping up. Schools are set to open in September.

And confidence in the good numbers themselves could breed complacency
about masks and distancing. Already, the city has seen a number of
\href{https://www.nytimes3xbfgragh.onion/2020/08/08/nyregion/nyc-illegal-parties.html}{large
illicit dance parties} and a worrisome spike in cases in Sunset Park,
Brooklyn.

``I'm not optimistic about a sustained end to Covid-19 in New York,''
said Dr. Irwin Redlener, the director of the Pandemic Resource and
Response Initiative at Columbia University. ``Even though we had that
horrible peak in April, when we were the epicenter, there are still
millions of people who are vulnerable.''

Among the biggest threats, officials and epidemiologists said, were
out-of-state travelers, who continue to arrive in New York despite a
state-mandated
\href{https://www.nytimes3xbfgragh.onion/2020/06/24/nyregion/ny-coronavirus-states-quarantine.html}{14-day
quarantine}.

The governor instituted the quarantine requirement for anyone coming to
\href{https://www.nytimes3xbfgragh.onion/2020/07/14/nyregion/ny-quarantine-rules.html}{New
York} from a state with high infection rates. Eight states were
initially affected; the list has since grown to 31 states, Puerto Rico
and the Virgin Islands.

More than 160,000 people have been subject to the quarantine since the
start of June, state officials said. But enforcement of the order is
near-impossible, and the state could not say how many have actually
quarantined.

About 20 percent of new positive cases in New York City have been
connected to out-of-state travel, city officials said, with Florida,
Georgia and New Jersey the top departure points. Last week, Mr. de
Blasio said drivers entering the city could be pulled over at random to
be informed of the state's quarantine rules.

The majority of those reached by the city's contact tracers have
\href{https://www.nytimes3xbfgragh.onion/2020/07/29/nyregion/new-york-contact-tracing.html}{not
shared the names of anyone they might have infected}: More than 12,500
people who tested positive did not give their contacts to the city,
\href{https://hhinternet.blob.core.windows.net/uploads/2020/08/test-and-trace-data-metrics-20200803.pdf}{out
of about 22,000 total}. Those who did shared an average of between two
and three contacts.

But city officials could not say how many of those testing positive for
the virus were already known to contact tracers --- in other words, how
many new cases had a connection to a previous positive case. That is
considered by infectious disease experts to be a key metric for gauging
how under control an outbreak is.

Dr. Jay Varma, the mayor's senior adviser for public health, said the
city's program had prevented ``thousands'' of new infections, based on
the number of people identified as symptomatic contacts who said they
were in quarantine. Just over 200 people have isolated themselves in a
city-funded hotel since the start of June.

``I don't think it's correct to insinuate that the work that we're doing
is not having an impact when you clearly see the impact in terms of
disease numbers in New York,'' Dr. Varma said.

Patterns of infection around the state suggest New Yorkers, like most
Americans, are chafing under pandemic restrictions.

\href{https://www.nytimes3xbfgragh.onion/news-event/coronavirus?action=click\&pgtype=Article\&state=default\&region=MAIN_CONTENT_3\&context=storylines_faq}{}

\hypertarget{the-coronavirus-outbreak-}{%
\subsubsection{The Coronavirus Outbreak
›}\label{the-coronavirus-outbreak-}}

\hypertarget{frequently-asked-questions}{%
\paragraph{Frequently Asked
Questions}\label{frequently-asked-questions}}

Updated August 17, 2020

\begin{itemize}
\item ~
  \hypertarget{why-does-standing-six-feet-away-from-others-help}{%
  \paragraph{Why does standing six feet away from others
  help?}\label{why-does-standing-six-feet-away-from-others-help}}

  \begin{itemize}
  \tightlist
  \item
    The coronavirus spreads primarily through droplets from your mouth
    and nose, especially when you cough or sneeze. The C.D.C., one of
    the organizations using that measure,
    \href{https://www.nytimes3xbfgragh.onion/2020/04/14/health/coronavirus-six-feet.html?action=click\&pgtype=Article\&state=default\&region=MAIN_CONTENT_3\&context=storylines_faq}{bases
    its recommendation of six feet} on the idea that most large droplets
    that people expel when they cough or sneeze will fall to the ground
    within six feet. But six feet has never been a magic number that
    guarantees complete protection. Sneezes, for instance, can launch
    droplets a lot farther than six feet,
    \href{https://jamanetwork.com/journals/jama/fullarticle/2763852}{according
    to a recent study}. It's a rule of thumb: You should be safest
    standing six feet apart outside, especially when it's windy. But
    keep a mask on at all times, even when you think you're far enough
    apart.
  \end{itemize}
\item ~
  \hypertarget{i-have-antibodies-am-i-now-immune}{%
  \paragraph{I have antibodies. Am I now
  immune?}\label{i-have-antibodies-am-i-now-immune}}

  \begin{itemize}
  \tightlist
  \item
    As of right
    now,\href{https://www.nytimes3xbfgragh.onion/2020/07/22/health/covid-antibodies-herd-immunity.html?action=click\&pgtype=Article\&state=default\&region=MAIN_CONTENT_3\&context=storylines_faq}{that
    seems likely, for at least several months.} There have been
    frightening accounts of people suffering what seems to be a second
    bout of Covid-19. But experts say these patients may have a
    drawn-out course of infection, with the virus taking a slow toll
    weeks to months after initial exposure. People infected with the
    coronavirus typically
    \href{https://www.nature.com/articles/s41586-020-2456-9}{produce}
    immune molecules called antibodies, which are
    \href{https://www.nytimes3xbfgragh.onion/2020/05/07/health/coronavirus-antibody-prevalence.html?action=click\&pgtype=Article\&state=default\&region=MAIN_CONTENT_3\&context=storylines_faq}{protective
    proteins made in response to an
    infection}\href{https://www.nytimes3xbfgragh.onion/2020/05/07/health/coronavirus-antibody-prevalence.html?action=click\&pgtype=Article\&state=default\&region=MAIN_CONTENT_3\&context=storylines_faq}{.
    These antibodies may} last in the body
    \href{https://www.nature.com/articles/s41591-020-0965-6}{only two to
    three months}, which may seem worrisome, but that's perfectly normal
    after an acute infection subsides, said Dr. Michael Mina, an
    immunologist at Harvard University. It may be possible to get the
    coronavirus again, but it's highly unlikely that it would be
    possible in a short window of time from initial infection or make
    people sicker the second time.
  \end{itemize}
\item ~
  \hypertarget{im-a-small-business-owner-can-i-get-relief}{%
  \paragraph{I'm a small-business owner. Can I get
  relief?}\label{im-a-small-business-owner-can-i-get-relief}}

  \begin{itemize}
  \tightlist
  \item
    The
    \href{https://www.nytimes3xbfgragh.onion/article/small-business-loans-stimulus-grants-freelancers-coronavirus.html?action=click\&pgtype=Article\&state=default\&region=MAIN_CONTENT_3\&context=storylines_faq}{stimulus
    bills enacted in March} offer help for the millions of American
    small businesses. Those eligible for aid are businesses and
    nonprofit organizations with fewer than 500 workers, including sole
    proprietorships, independent contractors and freelancers. Some
    larger companies in some industries are also eligible. The help
    being offered, which is being managed by the Small Business
    Administration, includes the Paycheck Protection Program and the
    Economic Injury Disaster Loan program. But lots of folks have
    \href{https://www.nytimes3xbfgragh.onion/interactive/2020/05/07/business/small-business-loans-coronavirus.html?action=click\&pgtype=Article\&state=default\&region=MAIN_CONTENT_3\&context=storylines_faq}{not
    yet seen payouts.} Even those who have received help are confused:
    The rules are draconian, and some are stuck sitting on
    \href{https://www.nytimes3xbfgragh.onion/2020/05/02/business/economy/loans-coronavirus-small-business.html?action=click\&pgtype=Article\&state=default\&region=MAIN_CONTENT_3\&context=storylines_faq}{money
    they don't know how to use.} Many small-business owners are getting
    less than they expected or
    \href{https://www.nytimes3xbfgragh.onion/2020/06/10/business/Small-business-loans-ppp.html?action=click\&pgtype=Article\&state=default\&region=MAIN_CONTENT_3\&context=storylines_faq}{not
    hearing anything at all.}
  \end{itemize}
\item ~
  \hypertarget{what-are-my-rights-if-i-am-worried-about-going-back-to-work}{%
  \paragraph{What are my rights if I am worried about going back to
  work?}\label{what-are-my-rights-if-i-am-worried-about-going-back-to-work}}

  \begin{itemize}
  \tightlist
  \item
    Employers have to provide
    \href{https://www.osha.gov/SLTC/covid-19/standards.html}{a safe
    workplace} with policies that protect everyone equally.
    \href{https://www.nytimes3xbfgragh.onion/article/coronavirus-money-unemployment.html?action=click\&pgtype=Article\&state=default\&region=MAIN_CONTENT_3\&context=storylines_faq}{And
    if one of your co-workers tests positive for the coronavirus, the
    C.D.C.} has said that
    \href{https://www.cdc.gov/coronavirus/2019-ncov/community/guidance-business-response.html}{employers
    should tell their employees} -\/- without giving you the sick
    employee's name -\/- that they may have been exposed to the virus.
  \end{itemize}
\item ~
  \hypertarget{what-is-school-going-to-look-like-in-september}{%
  \paragraph{What is school going to look like in
  September?}\label{what-is-school-going-to-look-like-in-september}}

  \begin{itemize}
  \tightlist
  \item
    It is unlikely that many schools will return to a normal schedule
    this fall, requiring the grind of
    \href{https://www.nytimes3xbfgragh.onion/2020/06/05/us/coronavirus-education-lost-learning.html?action=click\&pgtype=Article\&state=default\&region=MAIN_CONTENT_3\&context=storylines_faq}{online
    learning},
    \href{https://www.nytimes3xbfgragh.onion/2020/05/29/us/coronavirus-child-care-centers.html?action=click\&pgtype=Article\&state=default\&region=MAIN_CONTENT_3\&context=storylines_faq}{makeshift
    child care} and
    \href{https://www.nytimes3xbfgragh.onion/2020/06/03/business/economy/coronavirus-working-women.html?action=click\&pgtype=Article\&state=default\&region=MAIN_CONTENT_3\&context=storylines_faq}{stunted
    workdays} to continue. California's two largest public school
    districts --- Los Angeles and San Diego --- said on July 13, that
    \href{https://www.nytimes3xbfgragh.onion/2020/07/13/us/lausd-san-diego-school-reopening.html?action=click\&pgtype=Article\&state=default\&region=MAIN_CONTENT_3\&context=storylines_faq}{instruction
    will be remote-only in the fall}, citing concerns that surging
    coronavirus infections in their areas pose too dire a risk for
    students and teachers. Together, the two districts enroll some
    825,000 students. They are the largest in the country so far to
    abandon plans for even a partial physical return to classrooms when
    they reopen in August. For other districts, the solution won't be an
    all-or-nothing approach.
    \href{https://bioethics.jhu.edu/research-and-outreach/projects/eschool-initiative/school-policy-tracker/}{Many
    systems}, including the nation's largest, New York City, are
    devising
    \href{https://www.nytimes3xbfgragh.onion/2020/06/26/us/coronavirus-schools-reopen-fall.html?action=click\&pgtype=Article\&state=default\&region=MAIN_CONTENT_3\&context=storylines_faq}{hybrid
    plans} that involve spending some days in classrooms and other days
    online. There's no national policy on this yet, so check with your
    municipal school system regularly to see what is happening in your
    community.
  \end{itemize}
\end{itemize}

In New York City, neighborhoods with the highest rate of infection are
increasingly found in Manhattan --- Hell's Kitchen or the Financial
District, for example, which are home to wealthier residents --- in
addition to the parts of the Bronx and Queens that have long been hard
hit.

Image

In younger, wealthier parts of the city, infection numbers are higher,
but mask wearing is largely prevalent.Credit...Dave Sanders for The New
York Times

``My concern is complacency,'' the city's former top public health
official, Dr. Oxiris Barbot, said in an interview last month. Dr. Barbot
\href{https://www.nytimes3xbfgragh.onion/2020/08/04/nyregion/oxiris-barbot-health-commissioner-resigns.html}{resigned
this month and voiced ``deep disappointment''} with Mr. de Blasio's
response to the pandemic.

She said the most important factor in New York's success so far has been
broad acceptance of masks and social distancing, adding, ``I think it
would be foolish of us to not plan for an inevitable second wave.''

Fatigue with the rules has already sparked localized outbreaks in parts
of the state: a high school graduation in Chappaqua that resulted in the
infections of 28 people; a July 4 party in Albany that drew 200 people.

Pediatricians in Westchester County became alarmed when families
recently began seeking coronavirus tests in order to go to children's
parties where proof of a negative test was required to attend.

``Unfortunately, people still really don't understand what the testing
means,'' said Dr. Sherlita Amler, the county's health commissioner.
``It's just a moment in time. It's not a get-out-of-jail card.''

While antibody surveys have suggested
\href{https://www.nytimes3xbfgragh.onion/2020/04/23/nyregion/coronavirus-antibodies-test-ny.html}{one
in five New York City residents may have already been exposed}, public
health officials do not believe herd immunity is behind the low numbers,
or could be relied on in the future.

What may protect New Yorkers who decide to buck the rules and gather in
groups without masks is the fact that so many of their neighbors are
still masking up, said Dr. Barbot.

``If a new infection gets introduced into the community, it will be a
terminal transmission, meaning that it won't go any further,'' she said.
``More people wearing face coverings seems to be in line with that.''
She cautioned that it was a theory and the data is not yet there.

And even at the currently low levels, the number of new virus cases in
New York City --- 386 reported positive on Tuesday out of 46,185 tested,
according to state data --- is still too great for its contact tracers
to effectively determine where people are becoming infected, said Dr.
Barbot. The new norms of behavior have to continue for the foreseeable
future, she said.

``I think that we are changing the culture,'' said Dr. Howard Zucker,
the state health commissioner, adding that social distancing and masks
are now almost ``a reflex.''

Such changes have been evident from new behaviors. In Westchester
County, where indoor dining is allowed, patrons are opting to eat
outside. Some restaurants still only offer outdoor seating.

``When the weather turns colder, that will be the test of whether people
are comfortable dining indoors,'' said Noam Bramson, the mayor of New
Rochelle, which had
\href{https://www.nytimes3xbfgragh.onion/2020/03/10/nyregion/coronavirus-new-york-update.html}{the
first reported cluster in New York in March}. The city of 80,000 had 33
active cases at the end of July, he said, down from more than 1,000 in
the spring.

The ways New York has contained the virus are varied, and together offer
a preview of what life may look like for months if not years to come.

``The work force is going to be different for a while,'' said Jim
Malatras, who has been advising Governor Cuomo on the virus response. He
wondered aloud when and how things like gyms or movie theaters would be
able to open safely in the state. (On Monday, the governor announced
\href{https://www.nytimes3xbfgragh.onion/2020/08/17/nyregion/nyc-gyms-reopening.html}{gyms
would be allowed to open} with limited capacity on Aug. 24.)

``We're taking it slow,'' he added. ``Dave Matthews isn't performing at
Madison Square Garden anytime soon.''

Advertisement

\protect\hyperlink{after-bottom}{Continue reading the main story}

\hypertarget{site-index}{%
\subsection{Site Index}\label{site-index}}

\hypertarget{site-information-navigation}{%
\subsection{Site Information
Navigation}\label{site-information-navigation}}

\begin{itemize}
\tightlist
\item
  \href{https://help.nytimes3xbfgragh.onion/hc/en-us/articles/115014792127-Copyright-notice}{©~2020~The
  New York Times Company}
\end{itemize}

\begin{itemize}
\tightlist
\item
  \href{https://www.nytco.com/}{NYTCo}
\item
  \href{https://help.nytimes3xbfgragh.onion/hc/en-us/articles/115015385887-Contact-Us}{Contact
  Us}
\item
  \href{https://www.nytco.com/careers/}{Work with us}
\item
  \href{https://nytmediakit.com/}{Advertise}
\item
  \href{http://www.tbrandstudio.com/}{T Brand Studio}
\item
  \href{https://www.nytimes3xbfgragh.onion/privacy/cookie-policy\#how-do-i-manage-trackers}{Your
  Ad Choices}
\item
  \href{https://www.nytimes3xbfgragh.onion/privacy}{Privacy}
\item
  \href{https://help.nytimes3xbfgragh.onion/hc/en-us/articles/115014893428-Terms-of-service}{Terms
  of Service}
\item
  \href{https://help.nytimes3xbfgragh.onion/hc/en-us/articles/115014893968-Terms-of-sale}{Terms
  of Sale}
\item
  \href{https://spiderbites.nytimes3xbfgragh.onion}{Site Map}
\item
  \href{https://help.nytimes3xbfgragh.onion/hc/en-us}{Help}
\item
  \href{https://www.nytimes3xbfgragh.onion/subscription?campaignId=37WXW}{Subscriptions}
\end{itemize}
