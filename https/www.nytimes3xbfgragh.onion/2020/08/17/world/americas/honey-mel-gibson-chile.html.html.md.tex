Sections

SEARCH

\protect\hyperlink{site-content}{Skip to
content}\protect\hyperlink{site-index}{Skip to site index}

\href{https://www.nytimes3xbfgragh.onion/section/world/americas}{Americas}

\href{https://myaccount.nytimes3xbfgragh.onion/auth/login?response_type=cookie\&client_id=vi}{}

\href{https://www.nytimes3xbfgragh.onion/section/todayspaper}{Today's
Paper}

\href{/section/world/americas}{Americas}\textbar{}She Was Selling Honey
to Survive. Then Mel Gibson Threatened to Sue.

\url{https://nyti.ms/2PYpONq}

\begin{itemize}
\item
\item
\item
\item
\item
\item
\end{itemize}

Advertisement

\protect\hyperlink{after-top}{Continue reading the main story}

Supported by

\protect\hyperlink{after-sponsor}{Continue reading the main story}

\hypertarget{she-was-selling-honey-to-survive-then-mel-gibson-threatened-to-sue}{%
\section{She Was Selling Honey to Survive. Then Mel Gibson Threatened to
Sue.}\label{she-was-selling-honey-to-survive-then-mel-gibson-threatened-to-sue}}

A single mother in Chile began selling organic honey from home during
quarantine, using the actor's name as a play on words. His lawyer was
not amused.

\includegraphics{https://static01.graylady3jvrrxbe.onion/images/2020/08/17/world/17chile-honey/merlin_175830753_51ce69f8-9200-44d1-9807-da6e27b94496-articleLarge.jpg?quality=75\&auto=webp\&disable=upscale}

By \href{https://www.nytimes3xbfgragh.onion/by/ernesto-londono}{Ernesto
Londoño}

\begin{itemize}
\item
  Aug. 17, 2020
\item
  \begin{itemize}
  \item
  \item
  \item
  \item
  \item
  \item
  \end{itemize}
\end{itemize}

It was a venture born of desperation.

Four months into quarantine, Yohana Agurto, an unemployed teacher in
Chile, was scrolling through social media to take her mind off her
dwindling savings and the four children she had to feed.

Inspiration struck when she and her boyfriend came across a post with a
photo of the American actor Mel Gibson.

Ms. Agurto remembered she had a large stash of organic honey in the
pantry. Mel sounds quite similar to the Spanish word for honey, miel. So
on a whim, Ms. Agurto had a graphic designer friend sketch out a logo
with an iconic scene from the movie ``Braveheart,'' printed a handful at
home and glued them onto glass jars.

That was the origin of Miel Gibson, the tiniest and scrappiest of
businesses, which catered, according to the label, ``Only to the
brave.''

She advertised on social media and by word of mouth, picking up enough
orders to keep her reasonably busy and the family's bills paid. Then
last week a most unwelcome email popped into Ms. Agurto's inbox with the
subject line: ``Cease and Desist/Miel Gibson.''

``We are counsel to Mel Gibson,'' said the letter, which was sent by a
Los Angeles attorney whose firm represents several celebrities. ``It has
come to my attention that you are illegally using Mr. Gibson's name
and/or likeness and/or biography to market or sell honey products.''

At first Ms. Agurto entertained the possibility the whole thing could be
a cruel joke, but when it became clear it was a real legal threat, she
panicked.

``I realized I was up against Goliath,'' she said in an interview.

Some friends told Ms. Agurto she should ignore the letter, arguing that
no American lawyer would waste time shutting down a woman selling honey
from her home in Chile's capital during a pandemic.

But others warned her that Americans are tremendously litigious, Ms.
Agurto said, which led her to delete the email account she had created
for the business, hoping the whole thing would go away.

``I was terribly anguished,'' Ms. Agurto, 40, said in an interview. ``I
thought I could end up facing fines. What would happen to my family, to
my finances? I'm a single mother of four and they depend on me.''

Chile's quarantine has been among the longest and strictest in the
world, and the country is still reeling from one of the highest per
capita rates of infection. The government recently authorized Chileans
to dip into their pension plans early to provide a lifeline to millions
who are struggling to make ends meet as several sectors of the economy
remain paralyzed.

After a couple sleepless nights, Ms. Agurto decided she had invested too
much time and effort in her artisanal honey brand to simply shut it
down. She reflected on how much she admired William Wallace, the
Scottish warrior Mel Gibson played in ``Braveheart,'' and decided to go
public about the legal threat.

Her plight got plenty of sympathetic press coverage in Chile and beyond,
free legal advice and a torrent of new orders. On Monday, Ms. Agurto
said messages from would-be customers were streaming in by text message
and on her social media accounts faster than she could read them.

The bulk sellers where she gets her honey are nearly out of stock.

``I'm sleeping three hours a night,'' she said. ``I have hundreds of
messages I haven't responded to.''

María José Arancibia, a lawyer in Chile who represents Ms. Agurto, said
she had reached out to the actor's attorney, hoping to negotiate a
compromise. Ms. Arancibia said Miel Gibson intends to keep its name but
plans to lose the actor's image from the label.

Businesses and celebrities routinely send cease-and-desist letters to
people who are profiting from copyrighted images, names and logos
without authorization. Mr. Gibson's lawyer, Leigh Brecheen, said that
her client is not seeking to put Ms. Agurto out of business.

``None of this is meant to prevent anyone from earning an income or
creating a business,'' she said in an emailed statement. ``But there are
proper channels to contact and clearances you need to go through to make
sure you have the approval for such usage.''

Regardless of how the matter gets resolved, Ms. Agurto said she would
love to send a courtesy sample of Miel Gibson to Mel Gibson. She wants
him to know, she added, that the ordeal has not made her any less of a
fan.

``My motivation was not to profit by using the image of a famous
person,'' she said. ``I was selling honey to survive.''

Advertisement

\protect\hyperlink{after-bottom}{Continue reading the main story}

\hypertarget{site-index}{%
\subsection{Site Index}\label{site-index}}

\hypertarget{site-information-navigation}{%
\subsection{Site Information
Navigation}\label{site-information-navigation}}

\begin{itemize}
\tightlist
\item
  \href{https://help.nytimes3xbfgragh.onion/hc/en-us/articles/115014792127-Copyright-notice}{©~2020~The
  New York Times Company}
\end{itemize}

\begin{itemize}
\tightlist
\item
  \href{https://www.nytco.com/}{NYTCo}
\item
  \href{https://help.nytimes3xbfgragh.onion/hc/en-us/articles/115015385887-Contact-Us}{Contact
  Us}
\item
  \href{https://www.nytco.com/careers/}{Work with us}
\item
  \href{https://nytmediakit.com/}{Advertise}
\item
  \href{http://www.tbrandstudio.com/}{T Brand Studio}
\item
  \href{https://www.nytimes3xbfgragh.onion/privacy/cookie-policy\#how-do-i-manage-trackers}{Your
  Ad Choices}
\item
  \href{https://www.nytimes3xbfgragh.onion/privacy}{Privacy}
\item
  \href{https://help.nytimes3xbfgragh.onion/hc/en-us/articles/115014893428-Terms-of-service}{Terms
  of Service}
\item
  \href{https://help.nytimes3xbfgragh.onion/hc/en-us/articles/115014893968-Terms-of-sale}{Terms
  of Sale}
\item
  \href{https://spiderbites.nytimes3xbfgragh.onion}{Site Map}
\item
  \href{https://help.nytimes3xbfgragh.onion/hc/en-us}{Help}
\item
  \href{https://www.nytimes3xbfgragh.onion/subscription?campaignId=37WXW}{Subscriptions}
\end{itemize}
