Sections

SEARCH

\protect\hyperlink{site-content}{Skip to
content}\protect\hyperlink{site-index}{Skip to site index}

\href{https://myaccount.nytimes3xbfgragh.onion/auth/login?response_type=cookie\&client_id=vi}{}

\href{https://www.nytimes3xbfgragh.onion/section/todayspaper}{Today's
Paper}

\href{/section/upshot}{The Upshot}\textbar{}The Recession Is About to
Slam Cities. Not Just the Blue-State Ones.

\url{https://nyti.ms/3iMfvZ3}

\begin{itemize}
\item
\item
\item
\item
\item
\item
\end{itemize}

\hypertarget{the-coronavirus-outbreak}{%
\subsubsection{\texorpdfstring{\href{https://www.nytimes3xbfgragh.onion/news-event/coronavirus?name=styln-coronavirus-national\&region=TOP_BANNER\&variant=undefined\&block=storyline_menu_recirc\&action=click\&pgtype=Article\&impression_id=868e7b80-e38d-11ea-85e7-9f87a3618d91}{The
Coronavirus
Outbreak}}{The Coronavirus Outbreak}}\label{the-coronavirus-outbreak}}

\begin{itemize}
\tightlist
\item
  live\href{https://www.nytimes3xbfgragh.onion/2020/08/20/world/coronavirus-covid.html?name=styln-coronavirus-national\&region=TOP_BANNER\&variant=undefined\&block=storyline_menu_recirc\&action=click\&pgtype=Article\&impression_id=868e7b81-e38d-11ea-85e7-9f87a3618d91}{Latest
  Updates}
\item
  \href{https://www.nytimes3xbfgragh.onion/interactive/2020/us/coronavirus-us-cases.html?name=styln-coronavirus-national\&region=TOP_BANNER\&variant=undefined\&block=storyline_menu_recirc\&action=click\&pgtype=Article\&impression_id=868ea290-e38d-11ea-85e7-9f87a3618d91}{Maps
  and Cases}
\item
  \href{https://www.nytimes3xbfgragh.onion/interactive/2020/science/coronavirus-vaccine-tracker.html?name=styln-coronavirus-national\&region=TOP_BANNER\&variant=undefined\&block=storyline_menu_recirc\&action=click\&pgtype=Article\&impression_id=868ea291-e38d-11ea-85e7-9f87a3618d91}{Vaccine
  Tracker}
\item
  \href{https://www.nytimes3xbfgragh.onion/2020/08/19/us/colleges-closing-covid.html?name=styln-coronavirus-national\&region=TOP_BANNER\&variant=undefined\&block=storyline_menu_recirc\&action=click\&pgtype=Article\&impression_id=868ea292-e38d-11ea-85e7-9f87a3618d91}{Colleges
  Closing}
\item
  \href{https://www.nytimes3xbfgragh.onion/live/2020/08/20/business/stock-market-today-coronavirus?name=styln-coronavirus-national\&region=TOP_BANNER\&variant=undefined\&block=storyline_menu_recirc\&action=click\&pgtype=Article\&impression_id=868ea293-e38d-11ea-85e7-9f87a3618d91}{Economy}
\end{itemize}

Advertisement

\protect\hyperlink{after-top}{Continue reading the main story}

Upshot

Supported by

\protect\hyperlink{after-sponsor}{Continue reading the main story}

\hypertarget{the-recession-is-about-to-slam-cities-not-just-the-blue-state-ones}{%
\section{The Recession Is About to Slam Cities. Not Just the Blue-State
Ones.}\label{the-recession-is-about-to-slam-cities-not-just-the-blue-state-ones}}

Those with budgets that rely heavily on tourism, sales taxes or direct
state assistance will face particular distress.

\href{https://www.nytimes3xbfgragh.onion/by/emily-badger}{\includegraphics{https://static01.graylady3jvrrxbe.onion/images/2018/02/16/multimedia/author-emily-badger/author-emily-badger-thumbLarge-v2.png}}\href{https://www.nytimes3xbfgragh.onion/by/quoctrung-bui}{\includegraphics{https://static01.graylady3jvrrxbe.onion/images/2018/06/13/multimedia/author-quoctrung-bui/author-quoctrung-bui-thumbLarge-v2.png}}

By \href{https://www.nytimes3xbfgragh.onion/by/emily-badger}{Emily
Badger} and
\href{https://www.nytimes3xbfgragh.onion/by/quoctrung-bui}{Quoctrung
Bui}

\begin{itemize}
\item
  Aug. 17, 2020
\item
  \begin{itemize}
  \item
  \item
  \item
  \item
  \item
  \item
  \end{itemize}
\end{itemize}

Severe scenario

Less severe scenario

Rochester, N.Y.

Buffalo

Syracuse, N.Y.

Detroit

New York City

Orlando, Fla.

Jacksonville, Fla.

Baton Rouge, La.

New Orleans

Philadelphia

Chicago

Las Vegas

Seattle

Oklahoma City

Louisville, Ky.

Cities Could Face

Deep Revenue Shortfalls

Indianapolis

Colorado Springs

Denver

San Francisco

Charlotte, N.C.

CITY IS LOCATED IN A STATE THAT HAS:

Milwaukee

TWO DEMOCRATIC SENATORS

Columbus, Ohio

TWO REPUBLICAN SENATORS

Washington

Phoenix

SENATORS IN BOTH PARTIES

Tucson, Ariz.

Memphis

Fresno, Calif.

Los Angeles

Nashville

San Diego

San Jose, Calif.

Portland, Ore.

Baltimore

Fort Worth

San Antonio

El Paso

Houston

Dallas

Estimated Decline in Annual Revenue for Fiscal Year 2021

Austin, Texas

Albuquerque

Boston

-20\%

-16\%

-12\%

-8\%

-4\%

Severe scenario

Less severe scenario

Rochester, N.Y.

Buffalo

Syracuse, N.Y.

Detroit

New York City

Orlando, Fla.

Jacksonville, Fla.

Baton Rouge, La.

New Orleans

Philadelphia

Chicago

Las Vegas

Seattle

Oklahoma City

Louisville, Ky.

Indianapolis

Colorado Springs

Denver

Cities Could Face

Deep Revenue Shortfalls

San Francisco

Charlotte, N.C.

Milwaukee

Columbus, Ohio

CITY IS LOCATED IN A STATE THAT HAS:

Washington

TWO DEMOCRATIC SENATORS

Phoenix

TWO REPUBLICAN SENATORS

Tucson, Ariz.

Memphis

SENATORS IN BOTH PARTIES

Fresno, Calif.

Los Angeles

Nashville

San Diego

San Jose, Calif.

Portland, Ore.

Baltimore

Fort Worth

San Antonio

El Paso

Houston

Dallas

Estimated Decline in Annual Revenue for Fiscal Year 2021

Austin, Texas

Albuquerque

Boston

-20\%

-16\%

-12\%

-8\%

-4\%

Severe scenario

Less severe scenario

Rochester, N.Y.

Buffalo

Syracuse, N.Y.

Detroit

New York City

Orlando, Fla.

Jacksonville, Fla.

Baton Rouge, La.

New Orleans

Philadelphia

Chicago

Las Vegas

Seattle

Oklahoma City

Louisville, Ky.

Indianapolis

Colorado Springs

Denver

Cities Could Face Deep Revenue Shortfalls

San Francisco

Charlotte, N.C.

Milwaukee

Columbus, Ohio

Washington

CITY IS LOCATED IN A STATE THAT HAS:

Phoenix

TWO DEMOCRATIC SENATORS

Tucson, Ariz.

TWO REPUBLICAN SENATORS

Memphis

SENATORS IN BOTH PARTIES

Fresno, Calif.

Los Angeles

Nashville

San Diego

San Jose, Calif.

Portland, Ore.

Baltimore

Fort Worth

San Antonio

El Paso

Houston

Dallas

Austin, Texas

Estimated Decline in Annual Revenue for Fiscal Year 2021

Albuquerque

Boston

-20\%

-16\%

-12\%

-8\%

-4\%

Severe scenario

Less severe scenario

Rochester, N.Y.

Buffalo

Syracuse, N.Y.

Detroit

New York City

Orlando, Fla.

Jacksonville, Fla.

Baton Rouge, La.

New Orleans

Philadelphia

Chicago

Las Vegas

Seattle

Oklahoma City

Louisville, Ky.

Indianapolis

Colorado Springs

Cities Could Face Deep Revenue Shortfalls

Denver

San Francisco

Charlotte, N.C.

Milwaukee

Columbus, Ohio

Washington

CITY IS LOCATED IN A STATE THAT HAS:

Phoenix

Tucson, Ariz.

TWO DEMOCRATIC SENATORS

Memphis

Fresno, Calif.

TWO Republican SENATORS

Los Angeles

Nashville

SENATORS IN BOTH PARTIES

San Diego

San Jose, Calif.

Portland, Ore.

Baltimore

Fort Worth

San Antonio

El Paso

Houston

Dallas

Austin, Texas

Estimated Decline in Annual Revenue for Fiscal Year 2021

Albuquerque

Boston

-20\%

-16\%

-12\%

-8\%

-4\%

Note: Each city is a Fiscally Standardized City, covering municipal
budgets but also revenues to other government entities providing city
services like school or sewer districts. Source: Estimates from The
Fiscal Effects of The Covid-19 Pandemic on Cities: An Initial
Assessment, National Tax Journal by Howard Chernick, David Copeland and
Andrew Reschovsky

The coronavirus recession will erode city budgets in many insidious
ways. It will slash the casino revenues that Detroit relies on. It will
squeeze the state aid that is a lifeblood to Rochester and Buffalo in
upstate New York. It will cut the sales tax revenue in New Orleans and
Baton Rouge, where a healthy government depends on people buying things.

The crisis has arrived faster than the damage from the Great Recession
ever did. And it will cut deep in the fiscal year ahead, with many
communities likely to lose 10 percent or more of the revenue they would
have seen without the pandemic, according to a new analysis. That's
enough for residents to experience short-staffed libraries, strained
parks departments and fewer road projects. The hardest-hit cities like
Rochester and Buffalo could face 20 percent losses.

``The Great Recession was a story of long, drawn-out fiscal pain ---
this is sharper,'' said Howard Chernick, a professor emeritus of
economics at Hunter College and the Graduate Center at the City
University of New York, who worked on the new analysis estimating
revenue shortfalls for 150 major cities across the nation.

These numbers give a sense of the possible economic pain for cities if
Congress and the White House fail to agree on a new relief package that
includes
\href{https://www.nytimes3xbfgragh.onion/2020/08/14/business/economy/state-local-budget-pain.html}{aid
to state and local governments}. It also rebuts some of the prevailing,
largely Republican arguments that have stalled those negotiations: that
federal help will bail out only blue cities and those that have
mismanaged their finances.

Many cities facing steep losses are in states represented by Republican
senators, like Florida or Louisiana. And the analysis found little
relationship between whether a place was fiscally healthy before the
pandemic and the most dire projections of revenue shortfalls.

\includegraphics{https://static01.graylady3jvrrxbe.onion/images/2020/08/17/upshot/17up-virus-cities/merlin_172402335_8048f9a3-b4b0-48ad-84f8-1a29f874c8c5-articleLarge.jpg?quality=75\&auto=webp\&disable=upscale}

What matters more in this pandemic moment is \emph{how} a city generates
money: Those highly dependent on tourism, on direct state aid or on
volatile sales taxes will hurt the most. Cities like Boston, which rely
heavily on the most stable revenue, property taxes, are in the strongest
position --- for now.

\hypertarget{latest-updates-the-coronavirus-outbreak}{%
\section{\texorpdfstring{\href{https://www.nytimes3xbfgragh.onion/2020/08/20/world/coronavirus-covid.html?action=click\&pgtype=Article\&state=default\&region=MAIN_CONTENT_1\&context=storylines_live_updates}{Latest
Updates: The Coronavirus
Outbreak}}{Latest Updates: The Coronavirus Outbreak}}\label{latest-updates-the-coronavirus-outbreak}}

Updated 2020-08-21T07:46:15.883Z

\begin{itemize}
\tightlist
\item
  \href{https://www.nytimes3xbfgragh.onion/2020/08/20/world/coronavirus-covid.html?action=click\&pgtype=Article\&state=default\&region=MAIN_CONTENT_1\&context=storylines_live_updates\#link-68774d88}{Shutdowns,
  warnings and scoldings follow alarming incidents on college campuses.}
\item
  \href{https://www.nytimes3xbfgragh.onion/2020/08/20/world/coronavirus-covid.html?action=click\&pgtype=Article\&state=default\&region=MAIN_CONTENT_1\&context=storylines_live_updates\#link-26b58724}{Biden
  knocks Trump's pandemic response, and outlines a national strategy.}
\item
  \href{https://www.nytimes3xbfgragh.onion/2020/08/20/world/coronavirus-covid.html?action=click\&pgtype=Article\&state=default\&region=MAIN_CONTENT_1\&context=storylines_live_updates\#link-4e542da3}{U.S.
  health agencies announce moves to confront the flu season and
  plummeting child vaccination rates.}
\end{itemize}

\href{https://www.nytimes3xbfgragh.onion/2020/08/20/world/coronavirus-covid.html?action=click\&pgtype=Article\&state=default\&region=MAIN_CONTENT_1\&context=storylines_live_updates}{See
more updates}

More live coverage:
\href{https://www.nytimes3xbfgragh.onion/live/2020/08/20/business/stock-market-today-coronavirus?action=click\&pgtype=Article\&state=default\&region=MAIN_CONTENT_1\&context=storylines_live_updates}{Markets}

The estimates, to be published in the National Tax Journal by Mr.
Chernick, David Copeland at Georgia State University and Andrew
Reschovsky at the University of Wisconsin, are based on the mix of local
revenue sources, the importance of state aid and the composition of jobs
and wages in each city. The researchers predict average revenue
shortfalls in the 2021 fiscal year of about 5.5 percent in a less severe
scenario, or 9 percent in a more severe one.

These projections cover not just municipal budgets but also
\href{https://www.lincolninst.edu/research-data/data-toolkits/fiscally-standardized-cities}{every
local government entity} that spends money on services to residents in a
given city, including counties and sewer or school districts (those
budgets are adjusted for the share of residents who live within city
borders). As the pandemic has worsened in many parts of the country this
summer, the researchers now believe their severe forecasts are more
likely.

\hypertarget{cities-that-rely-on-more-volatile-revenue-sources-will-hurt-more}{%
\subsection{Cities That Rely on More Volatile Revenue Sources Will Hurt
More}\label{cities-that-rely-on-more-volatile-revenue-sources-will-hurt-more}}

150-City Average

Income tax

Federal aid

State aid

Property tax

Sales tax

Charges and fees

Other

26\%

18\%

8\%

7\%

10\%

24\%

Boston services are funded heavily by property taxes.

Property tax

51\%

Rochester receives about half of its revenue from the state.

State aid

49\%

Miami draws heavily on property taxes, and on charges and fees.

Property tax

Charges and fees

32\%

26\%

New Orleans relies on sales taxes more than most cities.

Sales tax

20\%

150-City Average

Income tax

Fed. aid

State aid

Property tax

Sales

Fees

Other

26\%

7\%

18\%

8\%

10\%

24\%

Boston services are funded heavily by property taxes.

Property tax

51\%

Rochester receives about half of its revenue from the state.

State aid

49\%

Miami draws heavily on property taxes, and on charges and fees.

Property tax

Fees

32\%

26\%

New Orleans relies on sales taxes more than most cities.

Sales Tax

20\%

150-City Average

Income tax

Federal

aid

State aid

Property tax

Sales

Fees

Other

26\%

7\%

18\%

8\%

24\%

Boston services are funded heavily by property taxes.

Property tax

51\%

Rochester receives about half of its revenue from the state.

State aid

49\%

Miami draws heavily on property taxes, and on charges and fees.

Property tax

Fees

32\%

26\%

New Orleans relies on sales taxes more than most cities.

Sales tax

20\%

150-City Average

Income tax

Fed.

aid

State aid

Property tax

Sales

Fees

Other

26\%

7\%

18\%

8\%

24\%

Boston services are funded heavily by property taxes.

Property tax

51\%

Rochester receives about half of its revenue from the state.

State aid

49\%

Miami draws heavily on property taxes, and on charges and fees.

Property tax

Fees

32\%

26\%

New Orleans relies on sales taxes more than most cities.

Sales tax

20\%

Source: Howard Chernick, David Copeland and Andrew Reschovsky

Some of the most vulnerable cities are those like Rochester that rely
heavily on state aid, which is also likely to shrink, as it did
\href{https://www.brookings.edu/articles/state-and-local-budgets-and-the-great-recession/}{in
the Great Recession}.

Rochester already has deferred millions of dollars of nonessential
expenses like new uniforms or fire trucks. It furloughed or reduced the
hours of about one in 10 city workers, many of whom will return as the
city reopens further. Officials delayed an incoming class of new police
recruits and canceled the next class of firefighters.

``We can't produce money, we can't borrow our way out of this, we can't
tax our way out of this,'' Mayor Lovely Warren said. ``But our residents
expect that the trash will be picked up on trash day. They expect that
the snow will be plowed when it snows. They expect that when they call
911 that a police officer will show up.

``For Washington to ignore that reality --- ``it hurts.''

``It's wrong to punish the victim,'' she added. ``The city here is the
victim.''

Other city officials around the country say they have tried to plan
prudently for down times. But the pandemic has brought added costs,
while state laws have limited their ability to raise revenue.

``This is really what the federal government was built to do: to handle
these events that are bigger than the borders of a city and bigger than
the borders of a state,'' said Dave Massaron, the chief financial
officer for the city of Detroit.

In Detroit, one-fifth of the municipal budget typically comes from
casino revenue. And casinos have only just reopened, at reduced
capacity. The city managed to save money when its recreation centers
closed, and it hasn't spent as much as usual managing downtown traffic.
This coming year, the city will also mow the grass less often on vacant
properties it owns.

With such moves, officials believe they will be able to get through
fiscal year 2021 with a balanced budget. But after that the decisions
will get harder, especially without federal help.

\href{https://www.nytimes3xbfgragh.onion/news-event/coronavirus?action=click\&pgtype=Article\&state=default\&region=MAIN_CONTENT_3\&context=storylines_faq}{}

\hypertarget{the-coronavirus-outbreak-}{%
\subsubsection{The Coronavirus Outbreak
›}\label{the-coronavirus-outbreak-}}

\hypertarget{frequently-asked-questions}{%
\paragraph{Frequently Asked
Questions}\label{frequently-asked-questions}}

Updated August 17, 2020

\begin{itemize}
\item ~
  \hypertarget{why-does-standing-six-feet-away-from-others-help}{%
  \paragraph{Why does standing six feet away from others
  help?}\label{why-does-standing-six-feet-away-from-others-help}}

  \begin{itemize}
  \tightlist
  \item
    The coronavirus spreads primarily through droplets from your mouth
    and nose, especially when you cough or sneeze. The C.D.C., one of
    the organizations using that measure,
    \href{https://www.nytimes3xbfgragh.onion/2020/04/14/health/coronavirus-six-feet.html?action=click\&pgtype=Article\&state=default\&region=MAIN_CONTENT_3\&context=storylines_faq}{bases
    its recommendation of six feet} on the idea that most large droplets
    that people expel when they cough or sneeze will fall to the ground
    within six feet. But six feet has never been a magic number that
    guarantees complete protection. Sneezes, for instance, can launch
    droplets a lot farther than six feet,
    \href{https://jamanetwork.com/journals/jama/fullarticle/2763852}{according
    to a recent study}. It's a rule of thumb: You should be safest
    standing six feet apart outside, especially when it's windy. But
    keep a mask on at all times, even when you think you're far enough
    apart.
  \end{itemize}
\item ~
  \hypertarget{i-have-antibodies-am-i-now-immune}{%
  \paragraph{I have antibodies. Am I now
  immune?}\label{i-have-antibodies-am-i-now-immune}}

  \begin{itemize}
  \tightlist
  \item
    As of right
    now,\href{https://www.nytimes3xbfgragh.onion/2020/07/22/health/covid-antibodies-herd-immunity.html?action=click\&pgtype=Article\&state=default\&region=MAIN_CONTENT_3\&context=storylines_faq}{that
    seems likely, for at least several months.} There have been
    frightening accounts of people suffering what seems to be a second
    bout of Covid-19. But experts say these patients may have a
    drawn-out course of infection, with the virus taking a slow toll
    weeks to months after initial exposure. People infected with the
    coronavirus typically
    \href{https://www.nature.com/articles/s41586-020-2456-9}{produce}
    immune molecules called antibodies, which are
    \href{https://www.nytimes3xbfgragh.onion/2020/05/07/health/coronavirus-antibody-prevalence.html?action=click\&pgtype=Article\&state=default\&region=MAIN_CONTENT_3\&context=storylines_faq}{protective
    proteins made in response to an
    infection}\href{https://www.nytimes3xbfgragh.onion/2020/05/07/health/coronavirus-antibody-prevalence.html?action=click\&pgtype=Article\&state=default\&region=MAIN_CONTENT_3\&context=storylines_faq}{.
    These antibodies may} last in the body
    \href{https://www.nature.com/articles/s41591-020-0965-6}{only two to
    three months}, which may seem worrisome, but that's perfectly normal
    after an acute infection subsides, said Dr. Michael Mina, an
    immunologist at Harvard University. It may be possible to get the
    coronavirus again, but it's highly unlikely that it would be
    possible in a short window of time from initial infection or make
    people sicker the second time.
  \end{itemize}
\item ~
  \hypertarget{im-a-small-business-owner-can-i-get-relief}{%
  \paragraph{I'm a small-business owner. Can I get
  relief?}\label{im-a-small-business-owner-can-i-get-relief}}

  \begin{itemize}
  \tightlist
  \item
    The
    \href{https://www.nytimes3xbfgragh.onion/article/small-business-loans-stimulus-grants-freelancers-coronavirus.html?action=click\&pgtype=Article\&state=default\&region=MAIN_CONTENT_3\&context=storylines_faq}{stimulus
    bills enacted in March} offer help for the millions of American
    small businesses. Those eligible for aid are businesses and
    nonprofit organizations with fewer than 500 workers, including sole
    proprietorships, independent contractors and freelancers. Some
    larger companies in some industries are also eligible. The help
    being offered, which is being managed by the Small Business
    Administration, includes the Paycheck Protection Program and the
    Economic Injury Disaster Loan program. But lots of folks have
    \href{https://www.nytimes3xbfgragh.onion/interactive/2020/05/07/business/small-business-loans-coronavirus.html?action=click\&pgtype=Article\&state=default\&region=MAIN_CONTENT_3\&context=storylines_faq}{not
    yet seen payouts.} Even those who have received help are confused:
    The rules are draconian, and some are stuck sitting on
    \href{https://www.nytimes3xbfgragh.onion/2020/05/02/business/economy/loans-coronavirus-small-business.html?action=click\&pgtype=Article\&state=default\&region=MAIN_CONTENT_3\&context=storylines_faq}{money
    they don't know how to use.} Many small-business owners are getting
    less than they expected or
    \href{https://www.nytimes3xbfgragh.onion/2020/06/10/business/Small-business-loans-ppp.html?action=click\&pgtype=Article\&state=default\&region=MAIN_CONTENT_3\&context=storylines_faq}{not
    hearing anything at all.}
  \end{itemize}
\item ~
  \hypertarget{what-are-my-rights-if-i-am-worried-about-going-back-to-work}{%
  \paragraph{What are my rights if I am worried about going back to
  work?}\label{what-are-my-rights-if-i-am-worried-about-going-back-to-work}}

  \begin{itemize}
  \tightlist
  \item
    Employers have to provide
    \href{https://www.osha.gov/SLTC/covid-19/standards.html}{a safe
    workplace} with policies that protect everyone equally.
    \href{https://www.nytimes3xbfgragh.onion/article/coronavirus-money-unemployment.html?action=click\&pgtype=Article\&state=default\&region=MAIN_CONTENT_3\&context=storylines_faq}{And
    if one of your co-workers tests positive for the coronavirus, the
    C.D.C.} has said that
    \href{https://www.cdc.gov/coronavirus/2019-ncov/community/guidance-business-response.html}{employers
    should tell their employees} -\/- without giving you the sick
    employee's name -\/- that they may have been exposed to the virus.
  \end{itemize}
\item ~
  \hypertarget{what-is-school-going-to-look-like-in-september}{%
  \paragraph{What is school going to look like in
  September?}\label{what-is-school-going-to-look-like-in-september}}

  \begin{itemize}
  \tightlist
  \item
    It is unlikely that many schools will return to a normal schedule
    this fall, requiring the grind of
    \href{https://www.nytimes3xbfgragh.onion/2020/06/05/us/coronavirus-education-lost-learning.html?action=click\&pgtype=Article\&state=default\&region=MAIN_CONTENT_3\&context=storylines_faq}{online
    learning},
    \href{https://www.nytimes3xbfgragh.onion/2020/05/29/us/coronavirus-child-care-centers.html?action=click\&pgtype=Article\&state=default\&region=MAIN_CONTENT_3\&context=storylines_faq}{makeshift
    child care} and
    \href{https://www.nytimes3xbfgragh.onion/2020/06/03/business/economy/coronavirus-working-women.html?action=click\&pgtype=Article\&state=default\&region=MAIN_CONTENT_3\&context=storylines_faq}{stunted
    workdays} to continue. California's two largest public school
    districts --- Los Angeles and San Diego --- said on July 13, that
    \href{https://www.nytimes3xbfgragh.onion/2020/07/13/us/lausd-san-diego-school-reopening.html?action=click\&pgtype=Article\&state=default\&region=MAIN_CONTENT_3\&context=storylines_faq}{instruction
    will be remote-only in the fall}, citing concerns that surging
    coronavirus infections in their areas pose too dire a risk for
    students and teachers. Together, the two districts enroll some
    825,000 students. They are the largest in the country so far to
    abandon plans for even a partial physical return to classrooms when
    they reopen in August. For other districts, the solution won't be an
    all-or-nothing approach.
    \href{https://bioethics.jhu.edu/research-and-outreach/projects/eschool-initiative/school-policy-tracker/}{Many
    systems}, including the nation's largest, New York City, are
    devising
    \href{https://www.nytimes3xbfgragh.onion/2020/06/26/us/coronavirus-schools-reopen-fall.html?action=click\&pgtype=Article\&state=default\&region=MAIN_CONTENT_3\&context=storylines_faq}{hybrid
    plans} that involve spending some days in classrooms and other days
    online. There's no national policy on this yet, so check with your
    municipal school system regularly to see what is happening in your
    community.
  \end{itemize}
\end{itemize}

``The city needs only to look back into its past to understand what
happens when you have a structural imbalance and you don't have a way of
addressing it,'' Mr. Massaron said. Before Detroit exited bankruptcy in
2015, emergency response times averaged 18 minutes. All 65,000
streetlights needed replacing.

Other cities
\href{https://papers.ssrn.com/sol3/papers.cfm?abstract_id=3571827}{heavily
dependent on sales taxes} felt the implosion of the economy more
immediately than cities that count on income or property taxes. Revenue
from income taxes will lag behind unemployment; property taxes are set a
year or two in advance. Consumer spending,
\href{https://www.nytimes3xbfgragh.onion/2020/06/17/upshot/coronavirus-spending-rich-poor.html}{particularly
by the biggest spenders}, dropped sharply early in the pandemic. And it
is expected to fall now for millions of workers whose added \$600
federal unemployment benefits
\href{https://www.nytimes3xbfgragh.onion/2020/08/07/upshot/unemployment-benefits-racial-disparity.html}{expired}
at the end of July.

Broad shifts in how Americans eat during the pandemic have affected tax
receipts as well: Restaurant meals are taxed, but
\href{https://www.cbpp.org/research/state-budget-and-tax/states-that-still-impose-sales-taxes-on-groceries-should-consider}{in
most states} the groceries people cook at home are not.

In Colorado Springs, which relies heavily on sales taxes, those revenues
plummeted in late March and April. But they crept back in May and, to
everyone's surprise, the city
\href{https://gazette.com/business/colorado-springs-experiencing-pretty-amazing-economic-turnaround-one-indicator-shows/article_a4f2f636-db50-11ea-8417-9f40749c7ffa.html}{saw
slightly more in sales tax revenues} in June than it did in June of last
year. Mayor John Suthers attributes that to the resilience of the local
military and defense sectors --- and to all the online shopping
residents have been doing.

Thanks to a 2018
\href{https://www.nytimes3xbfgragh.onion/2018/06/21/us/politics/supreme-court-sales-taxes-internet-merchants.html}{court
ruling}, states can now collect sales taxes on purchases through Amazon
or other online retailers, regardless of whether those retailers have a
physical presence locally. That's a silver lining.

``Without the Supreme Court's intervention, in the last three years this
would have been a whole different ballgame for us,'' Mr. Suthers said.

Orlando, Fla., is projected to suffer about as much as Colorado Springs
in these estimates. But with the county responsible for many services,
the Orlando municipal government will be spared the worst of the pain.
Orlando City Hall's revenues rely heavily on property taxes, which were
already set to grow next year. And, like Colorado Springs, the Orlando
area has long benefited from population growth and a construction boom
--- the other side of broad demographic shifts toward the Sun Belt that
have left Northeastern cities like Rochester more vulnerable.

``Maybe Orlando isn't in the same dire situation as other places,'' said
Chris McCullion, the city's chief financial officer. But he, too, is
calling for direct federal aid, as is Mr. Suthers, a Republican mayor.
``This is really, really important for the long-term health of cities
and states,'' Mr. McCullion said.

At risk is not just services for local residents in any given city, but
the possibility that disparities will widen between cities that can
weather this crisis and those that can't, if they are largely left on
their own.

``One legacy of the Great Recession was exposing and increasing
inequities between communities,'' said Amanda Kass, the associate
director of the Government Finance Research Center at the University of
Illinois at Chicago. Now those disparities could grow even wider.

Advertisement

\protect\hyperlink{after-bottom}{Continue reading the main story}

\hypertarget{site-index}{%
\subsection{Site Index}\label{site-index}}

\hypertarget{site-information-navigation}{%
\subsection{Site Information
Navigation}\label{site-information-navigation}}

\begin{itemize}
\tightlist
\item
  \href{https://help.nytimes3xbfgragh.onion/hc/en-us/articles/115014792127-Copyright-notice}{©~2020~The
  New York Times Company}
\end{itemize}

\begin{itemize}
\tightlist
\item
  \href{https://www.nytco.com/}{NYTCo}
\item
  \href{https://help.nytimes3xbfgragh.onion/hc/en-us/articles/115015385887-Contact-Us}{Contact
  Us}
\item
  \href{https://www.nytco.com/careers/}{Work with us}
\item
  \href{https://nytmediakit.com/}{Advertise}
\item
  \href{http://www.tbrandstudio.com/}{T Brand Studio}
\item
  \href{https://www.nytimes3xbfgragh.onion/privacy/cookie-policy\#how-do-i-manage-trackers}{Your
  Ad Choices}
\item
  \href{https://www.nytimes3xbfgragh.onion/privacy}{Privacy}
\item
  \href{https://help.nytimes3xbfgragh.onion/hc/en-us/articles/115014893428-Terms-of-service}{Terms
  of Service}
\item
  \href{https://help.nytimes3xbfgragh.onion/hc/en-us/articles/115014893968-Terms-of-sale}{Terms
  of Sale}
\item
  \href{https://spiderbites.nytimes3xbfgragh.onion}{Site Map}
\item
  \href{https://help.nytimes3xbfgragh.onion/hc/en-us}{Help}
\item
  \href{https://www.nytimes3xbfgragh.onion/subscription?campaignId=37WXW}{Subscriptions}
\end{itemize}
