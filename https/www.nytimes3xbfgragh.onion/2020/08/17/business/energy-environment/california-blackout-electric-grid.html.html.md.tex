Sections

SEARCH

\protect\hyperlink{site-content}{Skip to
content}\protect\hyperlink{site-index}{Skip to site index}

\href{https://www.nytimes3xbfgragh.onion/section/business/energy-environment}{Energy
\& Environment}

\href{https://myaccount.nytimes3xbfgragh.onion/auth/login?response_type=cookie\&client_id=vi}{}

\href{https://www.nytimes3xbfgragh.onion/section/todayspaper}{Today's
Paper}

\href{/section/business/energy-environment}{Energy \&
Environment}\textbar{}California Expresses Frustration as Blackouts
Enter 4th Day

\url{https://nyti.ms/3l15Ujl}

\begin{itemize}
\item
\item
\item
\item
\item
\end{itemize}

Advertisement

\protect\hyperlink{after-top}{Continue reading the main story}

Supported by

\protect\hyperlink{after-sponsor}{Continue reading the main story}

\hypertarget{california-expresses-frustration-as-blackouts-enter-4th-day}{%
\section{California Expresses Frustration as Blackouts Enter 4th
Day}\label{california-expresses-frustration-as-blackouts-enter-4th-day}}

The agency that manages the state's electric grid says rolling blackouts
are needed to balance supply and demand. But the governor said
regulators were not prepared.

\includegraphics{https://static01.graylady3jvrrxbe.onion/images/2020/08/17/business/17CA-BLACKOUT4/merlin_174565470_3be2264d-9de6-44dc-8267-cb5424ad478d-articleLarge.jpg?quality=75\&auto=webp\&disable=upscale}

\href{https://www.nytimes3xbfgragh.onion/by/ivan-penn}{\includegraphics{https://static01.graylady3jvrrxbe.onion/images/2018/06/12/multimedia/author-ivan-penn/author-ivan-penn-thumbLarge.png}}

By \href{https://www.nytimes3xbfgragh.onion/by/ivan-penn}{Ivan Penn}

\begin{itemize}
\item
  Published Aug. 17, 2020Updated Aug. 19, 2020
\item
  \begin{itemize}
  \item
  \item
  \item
  \item
  \item
  \end{itemize}
\end{itemize}

Lawmakers and consumer groups expressed outrage on Monday that the
operator of California's electricity grid had not adequately prepared
for a heat wave and was
\href{https://www.nytimes3xbfgragh.onion/2020/08/16/business/california-blackouts.html}{resorting
to rolling blackouts}.

The blackouts, which started on Friday and were set to continue into
Monday night, were reminiscent of
\href{https://www.nytimes3xbfgragh.onion/2002/09/18/us/california-power-failures-linked-to-energy-companies.html}{an
energy crisis 20 years ago}, when the state's botched deregulation of
the electricity system left millions of people in the dark and drove the
wholesale price of power skyward.

Gov. Gavin Newsom demanded an investigation into why state regulators
had failed to
\href{https://www.nytimes3xbfgragh.onion/2020/08/15/us/california-heat-wave-blackout.html}{prepare
for high temperatures}, which had been forecast for days.

``These blackouts, which occurred without prior warning or enough time
for preparation, are unacceptable and unbefitting of the nation's
largest and most innovative state,'' Mr. Newsom, a Democrat, said in a
letter Monday to the state's three major energy agencies.

\emph{{[}Sign up}
\href{https://www.nytimes3xbfgragh.onion/newsletters/california-today}{\emph{for
California Today}}\emph{, our newsletter from the Golden State.{]}}

Soon after he sent the letter, the state's electric grid manager, the
California Independent System Operator, ordered utilities like Pacific
Gas \& Electric and Southern California Edison to black out as many as
3.3 million people starting as early as 4 p.m. Pacific time to reduce
demand for power.

Steve Berberich, president and chief executive officer of California
I.S.O., said the system could be short about 4,400 megawatts of power in
the late afternoon. ``It's going to be highly disruptive to people,''
Mr. Berberich said. ``We're going to do everything we can to narrow that
gap.''

Sweltering weather has smothered much of the West over the last week and
is expected to strain the electric grid that serves about 80 percent of
California. Temperatures in Death Valley reached 130 degrees.

The heat is expected to continue through Wednesday evening. The
governor, the grid operator and utilities have been asking consumers to
reduce electricity use between 3 and 10 p.m., when power demand
typically peaks in the state.

People who lose power can expect to go without it for an hour at a time
during some of the hottest stretches of the day. The blackouts could
also disrupt learning for many students who began their first day of
virtual classes on Monday because of the coronavirus pandemic.

Last year, Pacific Gas \& Electric, the state's largest utility, shut
the power off to millions of customers, some for days,
\href{https://www.nytimes3xbfgragh.onion/2019/10/12/business/pge-california-outage.html}{to
reduce the risk that its equipment would set off wildfires}.

To increase the supply of electricity, Mr. Newsom has ordered PG\&E and
Southern California Edison to use backup generators that the utilities
installed for when they cut off power as a fire prevention tool.
\href{https://www.nytimes3xbfgragh.onion/2020/08/19/us/california-wildfires-vacaville.html}{Wildfires
have already burned}tens of thousands of acres in the state this summer.

Mark Toney, executive director of the Utility Reform Network, which
represents consumers before the California Public Utilities Commission,
called on lawmakers to investigate California I.S.O. to determine why
the agency did not adequately prepare for the heat wave.

``Why did they not do a better job of managing the grid, which is their
job?'' Mr. Toney said.

State Senator Jerry Hill, who heads a Senate energy subcommittee, said
he had learned that blackouts on Friday took place in part because a
natural gas power plant unexpectedly went offline.

``It failed to produce when called on,'' Mr. Hill said. ``There's
something wrong, and it's up to the Legislature and the governor to find
out.''

Separately, California I.S.O. said that as the wind slowed across much
of the state on Saturday, power from wind turbines dropped sharply. And
a second gas plant could not keep generating the electricity officials
were expecting.

The Federal Energy Regulatory Commission has been monitoring
California's energy troubles. The commission said it had discussed the
electricity demand and wholesale power prices, which spiked in
California over the weekend, with California I.S.O.

On Saturday, wholesale prices on California's electricity market surged,
some above \$3,800 per megawatt-hour, or roughly 100 times the typical
cost of transmitting power over a designated stretch of transmission
line.

The last time prices surged that much was in 2000 and 2001 after
California allowed utilities, Wall Street banks and other players to
trade electricity more freely with one another. Investigators determined
that policymakers had done a poor job deregulating the energy system and
that traders from
\href{https://www.nytimes3xbfgragh.onion/2002/05/07/business/enron-forced-up-california-prices-documents-show.html}{companies
like Enron had manipulated the market} to drive up prices and generate
handsome profits.

Utilities like PG\&E were forced to pay high wholesale prices for power
or black out customers. In 2001, PG\&E sought bankruptcy protection in
part because of the state's energy crisis. The company recently
\href{https://www.nytimes3xbfgragh.onion/2020/06/19/business/energy-environment/pge-bankruptcy-court-approval.html}{resolved
another bankruptcy}, this one caused by its wildfire liabilities.

Federal and state policymakers put safeguards in place to restore order.
But some experts said the recent blackouts and spike in power prices
suggested that the system still had weaknesses.

``Electricity is an essential service and shouldn't be subject to this
type of profiteering, especially during the pandemic,'' Mr. Toney said.

Mr. Berberich said California I.S.O. was exploring the use of existing
power plants that were not in service to help meet high demand in the
days and weeks ahead.

California is struggling to meet its electricity needs because other
states, which provide about 25 percent of the power it uses in a year,
are also seeing record demand.

``We have a perfect storm going on here,'' Mr. Berberich said. ``What we
have is a situation where the entire region is more than hot. It's
extremely hot.''

Advertisement

\protect\hyperlink{after-bottom}{Continue reading the main story}

\hypertarget{site-index}{%
\subsection{Site Index}\label{site-index}}

\hypertarget{site-information-navigation}{%
\subsection{Site Information
Navigation}\label{site-information-navigation}}

\begin{itemize}
\tightlist
\item
  \href{https://help.nytimes3xbfgragh.onion/hc/en-us/articles/115014792127-Copyright-notice}{©~2020~The
  New York Times Company}
\end{itemize}

\begin{itemize}
\tightlist
\item
  \href{https://www.nytco.com/}{NYTCo}
\item
  \href{https://help.nytimes3xbfgragh.onion/hc/en-us/articles/115015385887-Contact-Us}{Contact
  Us}
\item
  \href{https://www.nytco.com/careers/}{Work with us}
\item
  \href{https://nytmediakit.com/}{Advertise}
\item
  \href{http://www.tbrandstudio.com/}{T Brand Studio}
\item
  \href{https://www.nytimes3xbfgragh.onion/privacy/cookie-policy\#how-do-i-manage-trackers}{Your
  Ad Choices}
\item
  \href{https://www.nytimes3xbfgragh.onion/privacy}{Privacy}
\item
  \href{https://help.nytimes3xbfgragh.onion/hc/en-us/articles/115014893428-Terms-of-service}{Terms
  of Service}
\item
  \href{https://help.nytimes3xbfgragh.onion/hc/en-us/articles/115014893968-Terms-of-sale}{Terms
  of Sale}
\item
  \href{https://spiderbites.nytimes3xbfgragh.onion}{Site Map}
\item
  \href{https://help.nytimes3xbfgragh.onion/hc/en-us}{Help}
\item
  \href{https://www.nytimes3xbfgragh.onion/subscription?campaignId=37WXW}{Subscriptions}
\end{itemize}
