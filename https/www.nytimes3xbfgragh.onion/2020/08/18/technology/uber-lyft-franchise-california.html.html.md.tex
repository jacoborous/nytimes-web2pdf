Sections

SEARCH

\protect\hyperlink{site-content}{Skip to
content}\protect\hyperlink{site-index}{Skip to site index}

\href{https://www.nytimes3xbfgragh.onion/section/technology}{Technology}

\href{https://myaccount.nytimes3xbfgragh.onion/auth/login?response_type=cookie\&client_id=vi}{}

\href{https://www.nytimes3xbfgragh.onion/section/todayspaper}{Today's
Paper}

\href{/section/technology}{Technology}\textbar{}Uber and Lyft Consider
Franchise-Like Model in California

\url{https://nyti.ms/2Q4cRBC}

\begin{itemize}
\item
\item
\item
\item
\item
\item
\end{itemize}

Advertisement

\protect\hyperlink{after-top}{Continue reading the main story}

Supported by

\protect\hyperlink{after-sponsor}{Continue reading the main story}

\hypertarget{uber-and-lyft-consider-franchise-like-model-in-california}{%
\section{Uber and Lyft Consider Franchise-Like Model in
California}\label{uber-and-lyft-consider-franchise-like-model-in-california}}

Under pressure to classify their freelance drivers as employees, the
ride-hailing companies are discussing another option.

\includegraphics{https://static01.graylady3jvrrxbe.onion/images/2020/08/18/business/18ridefranchise1/18ridefranchise1-articleLarge-v2.jpg?quality=75\&auto=webp\&disable=upscale}

By \href{https://www.nytimes3xbfgragh.onion/by/kate-conger}{Kate Conger}

\begin{itemize}
\item
  Aug. 18, 2020
\item
  \begin{itemize}
  \item
  \item
  \item
  \item
  \item
  \item
  \end{itemize}
\end{itemize}

OAKLAND, Calif. --- Uber and Lyft, which are facing mounting pressure to
\href{https://www.nytimes3xbfgragh.onion/2020/08/20/technology/uber-lyft-california-shutdown.html}{classify
their freelance drivers as employees} in California, are looking for
another way.

One option that both companies are seriously discussing is licensing
their brands to operators of vehicle fleets in California, according to
three people with knowledge of the plans. The change would resemble an
independently operated franchise, allowing Uber and Lyft to keep an
arms-length association with drivers so that the companies would not
need to employ them and pay their benefits.

The idea would effectively be a return to the days of how groups of
black cars were run. Lyft has presented the plan to its board of
directors, one person said. Uber, which already works with fleet
operators in Germany and Spain, is also familiar with the business
model.

The companies have not committed to the franchise-like plans, said the
people with knowledge of the discussions, who asked to remain anonymous
because the details are confidential. Uber and Lyft are waiting to see
how California's legal situation around drivers, who have been treated
as independent contractors, plays out first, they said.

Matt Kallman, an Uber spokesman, said the work on establishing fleets
was ``exploratory'' and that the company was ``not sure whether a fleet
model would ultimately be viable in California.''

A Lyft spokeswoman, Julie Wood, said the company had looked at
alternative models but favored an approach where drivers ``remain
independent and can work whenever they want while also receiving
additional health care benefits and an earnings guarantee.''

The ride-hailing giants are considering how to retool their businesses
as they grapple with a new California law,
\href{https://www.nytimes3xbfgragh.onion/2019/12/31/technology/california-freelance-gig-workers.html}{Assembly
Bill 5}, which could upend their services. The law, which was designed
to grant employment benefits to gig workers, could force Uber and Lyft
to categorize drivers as employees if it was shown that the drivers'
jobs were part of the companies' core business, among other criteria.

Although the law went into effect in January, Uber and Lyft have not
complied with it, arguing that they are simply tech platforms and are
not transportation businesses. In May,
\href{https://www.nytimes3xbfgragh.onion/2020/05/05/technology/california-uber-lyft-lawsuit.html}{California
sued Uber and Lyft} to enforce the new law.

\includegraphics{https://static01.graylady3jvrrxbe.onion/images/2020/08/18/business/18ridefranchise3/merlin_171653223_5270a76b-8475-483d-a585-3eb06319d33c-articleLarge.jpg?quality=75\&auto=webp\&disable=upscale}

Their clash with the state is set to come to a head this week. This
month, a San Francisco Superior Court judge ordered the companies to
employ their drivers by Thursday. Executives at Uber and Lyft, who have
argued that they cannot meet that deadline, have appealed the decision
and warned that they would be forced to shut down their services as soon
as Friday if the order was not reversed.

``If our efforts here are not successful, it would force us to suspend
operations in California,'' John Zimmer, Lyft's president, said in an
earnings call last week. California accounts for about 16 percent of
Lyft's business, he said.

Dara Khosrowshahi, Uber's chief executive, also said last week in an
\href{https://www.cnbc.com/2020/08/12/uber-may-shut-down-temporarily-in-california.html}{MSNBC
interview} that the company's ride-hailing services in California would
stop, at least temporarily, if the order was not changed.

``It's a fork-in-the-road situation,'' said Dan Ives, a managing
director at Wedbush Securities who tracks the ride-hailing industry.
``These are some of the tough decisions they need to make to save their
business model.''

Uber and Lyft, which are based in San Francisco, have long considered
their drivers to be contractors. That means that drivers are responsible
for their own vehicle and maintenance costs and that Uber and Lyft do
not pay for overtime, unemployment insurance or other expenses.

The companies have argued that this freelance model allows drivers to
drive only when they want to. But critics have said it places
unreasonable financial burdens on drivers and gives Uber and Lyft unfair
advantages over businesses that follow employment laws.

Uber and Lyft have strenuously objected to A.B. 5 and have been fighting
its reach. The companies have poured tens of millions of dollars into a
\href{https://www.nytimes3xbfgragh.onion/2019/08/29/technology/uber-lyft-ballot-initiative.html}{ballot
measure} that would exempt them from the state law. Uber has also made
changes to its product, such as showing fares to drivers upfront and
allowing them to decline rides without facing penalties, to reinforce
their status as independent contractors.

But behind the scenes, officials at Uber and Lyft also began discussing
just-in-case options for their California businesses last year, the
people with knowledge of the plans said.

At Uber, many of the proposed ideas were code-named with the names of
characters from the Mario Bros. video game, like Luigi, the people said.
The
\href{https://www.washingtonpost.com/technology/2020/01/06/ubers-secret-project-bolster-its-case-against-ab-californias-gig-worker-law/}{Washington
Post} reported earlier on Project Luigi, which included the changes to
Uber's app that give drivers more control over fares.

Image

A hub for Lyft drivers in San Francisco. John Zimmer, Lyft's president,
recently said the state's suit could ``force us to suspend operations in
California.''Credit...Jim Wilson/The New York Times

Another option that policy teams at both of the companies floated was
the franchise-like model, the people with knowledge of the plans said.

Under the proposal, Uber and Lyft would invite other businesses to
establish ride-hailing fleets using their platforms. That could bolster
the companies' claims that they were simply tech companies that built
sophisticated dispatch services and that providing transportation was
outside their core business, protecting them from A.B. 5's requirements.

At Uber, the effort drew inspiration from the company's operations in
Germany and Spain, where transportation rules have already forced it to
work with fleets, Mr. Kallman said.

Lyft based its plan on FedEx, which franchises some of its delivery
routes to local operators, current and former employees said.

Uber and Lyft employees said the companies did not collaborate or share
information about their plans with each other.

A franchise-like business can be challenging. Working with a fleet
operator could increase costs because it introduces a third party who
needs to be paid, potentially forcing Uber and Lyft to raise fares or
reduce their service fees, current and former employees said. The
companies would also likely have to surrender some control over driver
behavior, leaving them more vulnerable to reputational damage if a
driver harassed a passenger or a car was dirty.

Another hurdle is that few fleet operators in California are large
enough to absorb Uber's and Lyft's business, partly because Uber and
Lyft previously disrupted taxis, black cars and similar operations.

For now, the companies have staked their primary hopes on the ballot
measure that would exempt them from A.B. 5, employees and financial
analysts said. The initiative, Proposition 22, proposes minimum-wage
standards and limited health benefits for drivers. It will appear on
California's ballot in November.

Whatever changes Uber and Lyft make to their businesses to comply with
A.B. 5 will ultimately be expensive, said Mr. Ives of Wedbush
Securities. He estimated that it would cost Uber \$500 million a year
and Lyft \$200 million a year. Both companies are already unprofitable
and have lost much of their ridership during the coronavirus pandemic.

``This legislation could really be a backbreaker,'' Mr. Ives said.

Advertisement

\protect\hyperlink{after-bottom}{Continue reading the main story}

\hypertarget{site-index}{%
\subsection{Site Index}\label{site-index}}

\hypertarget{site-information-navigation}{%
\subsection{Site Information
Navigation}\label{site-information-navigation}}

\begin{itemize}
\tightlist
\item
  \href{https://help.nytimes3xbfgragh.onion/hc/en-us/articles/115014792127-Copyright-notice}{©~2020~The
  New York Times Company}
\end{itemize}

\begin{itemize}
\tightlist
\item
  \href{https://www.nytco.com/}{NYTCo}
\item
  \href{https://help.nytimes3xbfgragh.onion/hc/en-us/articles/115015385887-Contact-Us}{Contact
  Us}
\item
  \href{https://www.nytco.com/careers/}{Work with us}
\item
  \href{https://nytmediakit.com/}{Advertise}
\item
  \href{http://www.tbrandstudio.com/}{T Brand Studio}
\item
  \href{https://www.nytimes3xbfgragh.onion/privacy/cookie-policy\#how-do-i-manage-trackers}{Your
  Ad Choices}
\item
  \href{https://www.nytimes3xbfgragh.onion/privacy}{Privacy}
\item
  \href{https://help.nytimes3xbfgragh.onion/hc/en-us/articles/115014893428-Terms-of-service}{Terms
  of Service}
\item
  \href{https://help.nytimes3xbfgragh.onion/hc/en-us/articles/115014893968-Terms-of-sale}{Terms
  of Sale}
\item
  \href{https://spiderbites.nytimes3xbfgragh.onion}{Site Map}
\item
  \href{https://help.nytimes3xbfgragh.onion/hc/en-us}{Help}
\item
  \href{https://www.nytimes3xbfgragh.onion/subscription?campaignId=37WXW}{Subscriptions}
\end{itemize}
