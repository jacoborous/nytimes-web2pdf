\href{/section/arts/design}{Art \& Design}\textbar{}For Gregory
Crewdson, Truth Lurks in the Landscape

\url{https://nyti.ms/2Qcqejf}

\begin{itemize}
\item
\item
\item
\item
\item
\end{itemize}

\includegraphics{https://static01.graylady3jvrrxbe.onion/images/2020/08/23/arts/23CREWDSON1/merlin_175281771_8b92d488-8a5e-4287-ad59-5d7b7eb4af81-articleLarge.jpg?quality=75\&auto=webp\&disable=upscale}

Sections

\protect\hyperlink{site-content}{Skip to
content}\protect\hyperlink{site-index}{Skip to site index}

\hypertarget{for-gregory-crewdson-truth-lurks-in-the-landscape}{%
\section{For Gregory Crewdson, Truth Lurks in the
Landscape}\label{for-gregory-crewdson-truth-lurks-in-the-landscape}}

As the artist shot his cinematically constructed images in the
Berkshires, he was afflicted by mysterious maladies. ``I felt my body
betrayed me,'' he said.

Gregory Crewdson at home in a former Methodist church, built in 1890, on
the outskirts of Great Barrington, Mass.Credit...Caroline Tompkins for
The New York Times

Supported by

\protect\hyperlink{after-sponsor}{Continue reading the main story}

By Arthur Lubow

\begin{itemize}
\item
  Aug. 20, 2020
\item
  \begin{itemize}
  \item
  \item
  \item
  \item
  \item
  \end{itemize}
\end{itemize}

GREAT BARRINGTON, Mass. --- The world has caught up to
\href{https://gagosian.com/artists/gregory-crewdson/}{Gregory Crewdson}.
In his large-scale photographs, which are produced with a movie crew in
bravura Hollywood style, the people stare off into space, cloaked in
solipsistic misery. The lighting is so portentous and the isolation and
hopelessness so exaggerated that these scenes have always reminded me of
Technicolor film stills from a 1950s melodrama --- a kitschy imitation
of life.

Until now. In the current locked-down world, which is hollowed out by
economic collapse and fragmented by fear of contagion, Mr. Crewdson's
overwrought images seem like faithful representations of our frazzled
psychological state. ``It's weird how all my pictures have taken on a
new meaning,'' he said.

In a masked interview last month in his Great Barrington studio,
adjacent to the deconsecrated 1890 Methodist church where he lives, he
and Juliane Hiam, his studio manager and romantic partner, hung his
prints, which are more than seven feet wide and four feet high, one by
one on the wall. It was the first studio visit since the pandemic
lockdown in March, and Mr. Crewdson, 57, a stocky, affable man, looked
relaxed in a T-shirt and shorts.

His most recent body of work, a 16-photograph collection he calls
\href{https://aperture.org/shop/gregory-crewdson-an-eclipse-of-moths/}{``An
Eclipse of Moths,''} will be presented in late September at
\href{https://gagosian.com/artists/gregory-crewdson/}{Gagosian} in
Beverly Hills, and subsequently in Paris and New York. He shot the
pictures on the outskirts of Pittsfield, a depressed town 20 miles north
of his home in the Berkshires.

\includegraphics{https://static01.graylady3jvrrxbe.onion/images/2020/08/23/arts/23CREWDSON-02/23CREWDSON-02-articleLarge.jpg?quality=75\&auto=webp\&disable=upscale}

His central decision when he stages a photograph is selecting the site.
``I spend a lot of time driving around, finding locations that can
accommodate a picture,'' Mr. Crewdson said, explaining the elaborate
scouting procedure for his constructed images. ``Returning to these
locations, a story will come into my head.'' He then describes the
concept to Ms. Hiam, who writes a scenario that will provide a template
for the shoot. ``There's no motivation, no plot,'' he said. ``I love the
idea of creating a moment that has no before and after. I do everything
I can to make it as powerful as possible.''

The collection's name alludes to a term for a cluster of the nocturnal
flying insects, which gather around an illumination source, eventually
obscuring it. The hard-bitten figures in his photographs are similarly
attracted --- metaphorically to a promise of grace, literally to
radiance --- in a quest that ends in self-defeat. Driving home that
trope, Mr. Crewdson brought street lamps into many scenes, fitting them
with special lights that cast an eerie glow.

The region has been hit hard by the opioid epidemic. ``On a daily basis
we would see O.D.'s,'' Mr. Crewdson said. He and Ms. Hiam recruited
local people to appear in the pictures, and for some, a dejected air and
a vacant stare did not require acting.

Image

Cranes used in the production of ``Redemption Center.''Credit...Grace
Clark for Crewdson Studio

Embellishing his landscapes, he painted billboards and altered the
street signs. He towed in dilapidated cars and installed an old
telephone booth. ``We age everything,'' he said. ``It should all look
slightly broken. We work closely with the town. I asked them not to pave
any streets, cut any grass. It's all in the effort of making a world
that is beautiful and unsettling. I want it to feel both timeless and of
the moment.'' Everything appears to be outmoded, even the overhead
electric wires.

He shot ``An Eclipse of Moths,'' and previous sequences of
Berkshires-based photographs, during the summer, when the twilight is
prolonged. Conveniently, he is off from his teaching duties at Yale,
where he is the director of graduate studies in photography. ``There is
a New England quality in the majesty of some of the trees,'' said his
friend Deborah Berke, dean of the Yale School of Architecture. ``A
sturdiness and scale and greenness to some of those trees is in high
contrast to the buildings and the cars and the other parts of the
landscape that don't have that staying power.''

The section of Pittsfield where he staged his pictures is near a General
Electric transformer plant
that\href{https://www.wbur.org/radioboston/2016/06/29/ge-and-pittsfield}{poisoned
the environment with PCBs but also employed most of the town}. Ms. Hiam,
who was born in Pittsfield, said, ``My parents worked for GE. Everyone I
knew had parents who came here for GE.'' Pittsfield was devastated in
1987 by the closing of the factory, which now looms over the landscape
like a ruined castle in a European village.

Image

``Funerary Back Lot,'' from Gregory Crewdson's new series, shows a young
woman bathing in a burial vault. The idea came to Mr. Crewdson when he
found the vaults in an abandoned site.Credit... Gregory Crewdson, via
Gagosian

But Mr. Crewdson, who as a boy summered in a cabin in the nearby town of
Becket with his Brooklyn-based family, sees the surroundings from a
different vantage point. ``He definitely feels the area is enchanting,''
Ms. Hiam said. ``It's the feeling a vacationer would have --- the
detached and enchanted view. When he takes me around to show his
locations, I'm often not seeing what he is seeing yet.''

As Mr. Crewdson shot the pictures two years ago, overseeing a large crew
of technicians and actors, he was afflicted by mysterious maladies.
``I'm not even sure it was conscious in terms of the content of the
work, but one of the central themes for me is brokenness,'' he said. ``I
was having a series of serious physical ailments that made going through
life, let alone the production, a real chore.'' He was short of breath,
constantly fatigued, liable to fall asleep at odd times.

``There's always an unconscious connection between my life and my
work,'' he said. ``All my work really does start with a psychological
state of looking inward and projecting outward --- a tension between
something very intimate and something very removed. I was also just
coming out of a difficult divorce and had moved out of New York, trying
to continue my relationship with my children by constant commuting. At
the core, I felt my body had betrayed me.''

He delayed seeing a doctor until he was in postproduction, and then
learned that he was suffering from severe sleep apnea, waking as
frequently as 50 times an hour during the night. Since then, he has
recovered. He follows a strict daily regimen: abstaining from alcohol,
eating the same two meals (heavy on salmon), swimming in a lake for an
hour and a half.

Image

In ``The Cobra,'' a fair ride from yesteryear has been abandoned in a
site that looks like a junkyard. A young woman is in a grim erotic
standoff with a young man in a shipping container.~Credit... Gregory
Crewdson, via Gagosian

Mr. Crewdson's chosen genre, the constructed photograph, is nearly as
old as photography itself, but for a long time it fell out of favor. As
early as the 1850s in Victorian England,
\href{https://www.getty.edu/art/collection/artists/1936/henry-peach-robinson-british-1830-1901/}{Henry
Peach Robinson} and
\href{https://www.getty.edu/art/collection/artists/1650/oscar-gustave-rejlander-british-born-sweden-1813-1875/}{Oscar
G. Rejlander} engineered tableaus with actors and stage sets --- and
even, in a handmade precursor of digital manipulation, montaged multiple
negatives --- to create photographic facsimiles of history paintings or
domestic dramas. But the sentimentality of their images and the
subordination of the camera to the aesthetics of the easel damned those
photographers in the eyes of their modernist descendants.

Over a century later, starting in the late '70s,
\href{https://gagosian.com/artists/jeff-wall/}{Jeff Wall}, who was
deeply schooled in art history and influenced by Conceptual art, brought
the constructed photograph back into critical favor with his large
color-transparency light boxes. Concurrently, but from more of a
pop-culture vantage point (referring to Hitchcock, for instance, rather
than Delacroix and Manet),
\href{https://www.moma.org/learn/moma_learning/cindy-sherman-untitled-film-stills-1977-80/}{Cindy
Sherman's ``Untitled Film Stills''} demonstrated the capability of the
constructed photograph to express an artist's personal obsessions and
(in her case) feminist concerns.

Since then, Mr. Crewdson has been one of many artists, including
\href{https://www.moma.org/artists/7027}{Philip-Lorca diCorcia},
\href{https://www.davidzwirner.com/artists/stan-douglas/biography}{Stan
Douglas} and \href{https://www.alexprager.com/}{Alex Prager}, who
explore various approaches to the practice. ``They have been developing
different models of production in a way that is unmistakably
cinematic,'' Roxana Marcoci, senior curator of photography at MoMA, said
in a phone interview. ``They are as attentive to the furniture and the
location as to the faces and bodies of the people they are portraying.
You have the aesthetic of the film still.''

Image

In Mr. Crewdson's ``Red Star Express,'' the shuttered GE transformer
plant looms over the town like a ruined castle in a European village.
The fire in the truck was set by the production team.Credit... Gregory
Crewdson, via Gagosian

Mr. Crewdson is a filmmaker manqué. ``Almost from day one, I was
interested in the intersection between movies and a still image,'' he
said. ``I love movies maybe above all forms of art --- the dreamlike
quality of going to a movie and watching the light on the screen and
being intoxicated by how that world seems separate from our world. But I
think in single images, always. I'm dyslexic and have trouble with
linear storytelling.''

Early in his career, in the mid-90s, Mr. Crewdson created and
photographed dioramas of out-of-scale insects and animals in natural
settings that evoked the uncanny hothouse atmosphere of David Lynch's
movie ``Blue Velvet.'' As he advanced, he employed actors and film crews
to fabricate scenes with a similar emotional payoff on a grander scale.
During production, fog machines cloud the air and water trucks wet the
streets. Lights beam down from 80-foot-high cranes. Unusually for a
photographer, he employs a camera operator to press the shutter and a
director of photography to arrange the lighting. (Recently, he replaced
his 8 by 10 large-format film camera with a Phase One digital camera,
which, he says, has ``100 or 200 times the definition.'') ``I'm not
comfortable holding the camera even,'' he said. ``I'm interested in what
I see in front of me.''

The pictures in ``An Eclipse of Moths'' traffic less heavily in
otherworldly encounters than in previous sequences, including
\href{https://gagosian.com/exhibitions/2016/gregory-crewdson-cathedral-of-the-pines/}{``Cathedral
of the Pines''} (2016) and
\href{https://gagosian.com/exhibitions/2005/gregory-crewdson-beneath-the-roses/}{``Beneath
the Roses''} (2005). Still, they feature such off-kilter human
activities as a young woman bathing in a concrete burial vault, another
young woman standing on a broken-down fair ride called ``the Cobra'' and
holding an erotic face-off with a shirtless youth in a discarded
shipping container, boys on bicycles close to the abandoned GE plant who
are watching a fire burn in the back of a truck, and a shirtless old man
with a shopping cart full of junk outside a building marked ``Redemption
Center.''

Image

``Taxi Depot''(2018-2019). The photographer creates seamless
compositions from multiple takes. The part of Pittsfield where Mr.
Crewdson shot his pictures has been hit hard by the opioid epidemic.
``On a daily basis, we would see O.D.'s,'' he said.Credit...Gregory
Crewdson, via Gagosian

The even lighting enhances the dreamlike quality. Everything is in
focus, nothing is blurry. In postproduction, he creates seamless
composites from various takes, so that the reflections in the rain
puddles are as sharp as mirror images and the houses in the distance are
as defined as the foreground buildings.

The son of a psychiatrist who treated patients in the family's Park
Slope house, Mr. Crewdson likes to describe how as a boy he would lie on
the floorboards of the living room, hoping to hear what was being said
below. ``I was trying to listen to something that was hidden or
forbidden in the domestic space,'' he explained. He never could make out
the conversation. Like the people in his photographs, these mysterious
strangers tantalized his curiosity but kept their secrets just beyond
his grasp. So the fascination never faded. ``I've always been drawn to
photography because of its inability to tell the full story,'' he said.
``It remains unresolved.''

Advertisement

\protect\hyperlink{after-bottom}{Continue reading the main story}

\hypertarget{site-index}{%
\subsection{Site Index}\label{site-index}}

\hypertarget{site-information-navigation}{%
\subsection{Site Information
Navigation}\label{site-information-navigation}}

\begin{itemize}
\tightlist
\item
  \href{https://help.nytimes3xbfgragh.onion/hc/en-us/articles/115014792127-Copyright-notice}{©~2020~The
  New York Times Company}
\end{itemize}

\begin{itemize}
\tightlist
\item
  \href{https://www.nytco.com/}{NYTCo}
\item
  \href{https://help.nytimes3xbfgragh.onion/hc/en-us/articles/115015385887-Contact-Us}{Contact
  Us}
\item
  \href{https://www.nytco.com/careers/}{Work with us}
\item
  \href{https://nytmediakit.com/}{Advertise}
\item
  \href{http://www.tbrandstudio.com/}{T Brand Studio}
\item
  \href{https://www.nytimes3xbfgragh.onion/privacy/cookie-policy\#how-do-i-manage-trackers}{Your
  Ad Choices}
\item
  \href{https://www.nytimes3xbfgragh.onion/privacy}{Privacy}
\item
  \href{https://help.nytimes3xbfgragh.onion/hc/en-us/articles/115014893428-Terms-of-service}{Terms
  of Service}
\item
  \href{https://help.nytimes3xbfgragh.onion/hc/en-us/articles/115014893968-Terms-of-sale}{Terms
  of Sale}
\item
  \href{https://spiderbites.nytimes3xbfgragh.onion}{Site Map}
\item
  \href{https://help.nytimes3xbfgragh.onion/hc/en-us}{Help}
\item
  \href{https://www.nytimes3xbfgragh.onion/subscription?campaignId=37WXW}{Subscriptions}
\end{itemize}
