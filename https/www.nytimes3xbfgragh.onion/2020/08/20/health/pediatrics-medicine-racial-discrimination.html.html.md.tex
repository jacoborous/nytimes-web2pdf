\href{/section/health}{Health}\textbar{}Pediatrics Group Offers `Long
Overdue' Apology for Racist Past

\url{https://nyti.ms/3ghMlzo}

\begin{itemize}
\item
\item
\item
\item
\item
\end{itemize}

\hypertarget{race-and-america}{%
\subsubsection{\texorpdfstring{\href{https://www.nytimes3xbfgragh.onion/news-event/george-floyd-protests-minneapolis-new-york-los-angeles?name=styln-george-floyd\&region=TOP_BANNER\&variant=undefined\&block=storyline_menu_recirc\&action=click\&pgtype=Article\&impression_id=0a9fe590-e388-11ea-bc3e-7712c2f37c01}{Race
and America}}{Race and America}}\label{race-and-america}}

\begin{itemize}
\tightlist
\item
  \href{https://www.nytimes3xbfgragh.onion/interactive/2020/07/03/us/george-floyd-protests-crowd-size.html?name=styln-george-floyd\&region=TOP_BANNER\&variant=undefined\&block=storyline_menu_recirc\&action=click\&pgtype=Article\&impression_id=0a9fe591-e388-11ea-bc3e-7712c2f37c01}{Black
  Lives Matter Movement}
\item
  \href{https://www.nytimes3xbfgragh.onion/interactive/2020/06/28/us/i-cant-breathe-police-arrest.html?name=styln-george-floyd\&region=TOP_BANNER\&variant=undefined\&block=storyline_menu_recirc\&action=click\&pgtype=Article\&impression_id=0a9fe592-e388-11ea-bc3e-7712c2f37c01}{History
  of `I Can't Breathe'}
\item
  \href{https://www.nytimes3xbfgragh.onion/interactive/2020/06/10/upshot/black-lives-matter-attitudes.html?name=styln-george-floyd\&region=TOP_BANNER\&variant=undefined\&block=storyline_menu_recirc\&action=click\&pgtype=Article\&impression_id=0a9fe593-e388-11ea-bc3e-7712c2f37c01}{How
  Public Opinion Shifted}
\item
  \href{https://www.nytimes3xbfgragh.onion/interactive/2020/07/16/us/black-lives-matter-protests-louisville-breonna-taylor.html?name=styln-george-floyd\&region=TOP_BANNER\&variant=undefined\&block=storyline_menu_recirc\&action=click\&pgtype=Article\&impression_id=0a9fe594-e388-11ea-bc3e-7712c2f37c01}{45
  Days in Louisville}
\end{itemize}

\includegraphics{https://static01.graylady3jvrrxbe.onion/images/2020/08/19/science/19SCI-MEDICALSOCIETIES1/merlin_175899546_d27d163a-c8d3-4839-936f-7f9357612c02-articleLarge.jpg?quality=75\&auto=webp\&disable=upscale}

Sections

\protect\hyperlink{site-content}{Skip to
content}\protect\hyperlink{site-index}{Skip to site index}

\hypertarget{pediatrics-group-offers-long-overdue-apology-for-racist-past}{%
\section{Pediatrics Group Offers `Long Overdue' Apology for Racist
Past}\label{pediatrics-group-offers-long-overdue-apology-for-racist-past}}

The American Academy of Pediatrics recently joined other prominent
medical organizations in confronting its history of discrimination.

Dr. Roland B. Scott in the 1950s, during his time as a professor at
Howard University in Washington.Credit...Howard University, via U.S.
National Library of Medicine

Supported by

\protect\hyperlink{after-sponsor}{Continue reading the main story}

By \href{https://www.nytimes3xbfgragh.onion/by/emma-goldberg}{Emma
Goldberg}

\begin{itemize}
\item
  Aug. 20, 2020
\item
  \begin{itemize}
  \item
  \item
  \item
  \item
  \item
  \end{itemize}
\end{itemize}

Dr. Roland B. Scott was the first African-American to pass the pediatric
board exam, in 1934. He was a faculty member at Howard University, and
went on to establish its center for the study of sickle cell disease; he
gained national acclaim for his research on the blood disorder.

But when he applied for membership with the American Academy of
Pediatrics --- its one criteria for admission was board certification
--- he was rejected multiple times beginning in 1939.

The minutes from the organization's 1944 executive board meeting leave
little room for mystery regarding the group's decision. The group that
considered his application, along with that of another Black physician,
was all-white. ``If they became members they would want to come and eat
with you at the table,'' one academy member said. ``You cannot hold them
down.''

Dr. Scott was accepted a year later along with his Howard professor, Dr.
Alonzo deGrate Smith, another Black pediatrician. But they were only
allowed to join for educational purposes and were not permitted to
attend meetings in the South, ostensibly for their safety.

More than a half-century later, the American Academy of Pediatrics has
formally apologized for its racist actions, including its initial
rejections of Drs. Scott and Smith on the basis of their race. The
statement will be published in the September issue of
\href{https://www.aappublications.org/news/2020/07/29/letter072920}{Pediatrics}.
The group also changed its bylaws to prohibit discrimination on the
basis of race, religion, sexual orientation or gender identity.

``This apology is long overdue,'' said Dr. Sally Goza, the
organization's president, noting that this year marks the group's 90th
anniversary. ``But we must also acknowledge where we have failed to live
up to our ideals.''

Dr. Goza said in an interview that the group learned from the example of
another organization that confronted its racist past: the American
Medical Association.

The American field of medicine has long been
\href{https://www.aamc.org/data-reports/workforce/interactive-data/figure-18-percentage-all-active-physicians-race/ethnicity-2018}{predominantly}
white. Black patients experience
\href{https://www.brookings.edu/blog/usc-brookings-schaeffer-on-health-policy/2020/02/19/there-are-clear-race-based-inequalities-in-health-insurance-and-health-outcomes/}{worse
health} outcomes and
\href{https://www.ncbi.nlm.nih.gov/pmc/articles/PMC4108512/}{higher
rates} of conditions like hypertension and diabetes. Black, Latino and
Native Americans have also suffered
\href{https://www.cdc.gov/coronavirus/2019-ncov/community/health-equity/race-ethnicity.html}{disproportionately}
during the Covid-19
\href{https://www.nytimes3xbfgragh.onion/interactive/2020/07/05/us/coronavirus-latinos-african-americans-cdc-data.html}{pandemic}.

In the last decade, some medical societies and groups have released
statements recognizing the role that systemic racism and discrimination
played in driving these health disparities. Implicit bias
\href{https://www.nytimes3xbfgragh.onion/2020/01/13/upshot/bad-medicine-the-harm-that-comes-from-racism.html}{affects}
the quality of provider services: Living in poverty limits access to
healthy food and preventive care.

After the killing of George Floyd at the hands of the Minneapolis
police, in late May, a flood of medical groups released statements on
racial health disparities: the
\href{https://www.aaem.org/current-news/death-of-george-floyd}{American
Academy of Emergency Medicine}, the
\href{https://www.cardiovascularbusiness.com/topics/healthcare-economics/george-floyd-cardiovascular-denounce-racism-violence}{American
College of Cardiology}, the
\href{https://gi.org/2020/05/30/a-message-from-the-acg-board-of-trustees/}{American
College of Gastroenterology}, the
\href{https://www.aao.org/newsroom/news-releases/detail/statement-on-death-of-george-floyd-its-aftermath}{American
Academy of Ophthalmology}, the
\href{https://www.psychiatry.org/newsroom/news-releases/apa-condemns-racism-in-all-forms-calls-for-end-to-racial-inequalities-in-u-s}{American
Psychiatric Association} and more. The American Public Health
Association released a
\href{https://www.apha.org/topics-and-issues/health-equity/racism-and-health/racism-declarations}{statement}
recognizing racism as a ``public health crisis.''

But few medical organizations have confronted the roles they played in
blocking opportunities for Black advancement in the medical profession
--- until the American Medical Association, and more recently the
American Academy of Pediatrics, formally apologized for their histories.

The A.M.A. issued an
\href{https://www.nytimes3xbfgragh.onion/2008/07/11/health/11ama.html}{apology}
in 2008 for its more than century-long history of discriminating against
African-American physicians. For decades, the organization predicated
its membership on joining a local or state medical society, many of
which excluded Black physicians, especially in the South. Keith Wailoo,
a historian at Princeton University, said the group chose to ``look the
other way'' regarding these exclusionary practices. The A.M.A.'s apology
came in the wake of a paper,
\href{https://jamanetwork.com/data/journals/JAMA/4424/jsc80005_306_313.pdf}{published}
in the Journal of the American Medical Association, that examined a
number of discriminatory aspects of the group's history, including its
efforts to close African-American medical schools.

For some Black physicians, exclusion from the A.M.A. meant the loss of
career advancement opportunities, according to Dr. Wailoo. Others
struggled to gain access to the postgraduate training they needed for
certification in certain medical specialties. As a result, many Black
physicians were limited to becoming general practitioners, especially in
the South. Some facilities also required A.M.A. membership for admitting
privileges to hospitals.

By 1964, the A.M.A. changed its position and refused to certify medical
societies that discriminated on the basis of race, but persistent
segregation in local groups still limited Black physicians' access to
certain hospitals, as well as opportunities for specialty training and
certification.

``Physicians are no different from other Americans who harbor biases,''
said Dr. Wailoo, whose research focuses on race and the history of
medicine. ``We expect doctors to speak on the basis of science, but
they're embedded in culture in the same way everyone else is.''

\includegraphics{https://static01.graylady3jvrrxbe.onion/images/2020/08/19/science/19SCI-MEDICALSOCIETIES2/merlin_175899507_a8777c5f-59c1-4c0b-b90b-44780bd46538-articleLarge.jpg?quality=75\&auto=webp\&disable=upscale}

The A.M.A. also played a role in limiting medical educational
opportunities available to Black physicians. In the early 20th century,
before the medical field held the same prestige it does today, the
A.M.A. commissioned a report assessing the country's medical schools for
their rigor. The report, by educator Abraham Flexner, deemed much of the
country's medical education system substandard. It also recommended
closing all but two of the country's seven Black medical schools. Howard
and Meharry were spared.

As the field became more exclusive, it also became more white, according
to Adam Biggs, a historian at the University of South Carolina. ``When
we talk about how modern medicine came to define what it means to be a
modern practitioner, it was deeply rooted in race,'' Mr. Biggs said.
``Segregation was embedded in the pipeline.''

Between its restrictions on medical education and its exclusionary
membership, the A.M.A. played a role in cultivating the profession's
homogeneity, which it acknowledged in its 2008 statement. It has since
appointed a chief health equity officer and established a center for
health equity. Dr. Goza said that the A.M.A.'s example helped spur the
American Academy of Pediatrics to confront its own history.

There have been some historical examples of efforts to confront racism
in the medical field. In 1997, President Clinton
\href{https://www.cdc.gov/tuskegee/clintonp.htm}{apologized} for the
infamous Tuskegee syphilis study conducted between 1932 and 1972, a
quarter-century after it was first exposed by
\href{https://apnews.com/e9dd07eaa4e74052878a68132cd3803a/AP-WAS-THERE:-Black-men-untreated-in-Tuskegee-Syphilis-Study}{The
Associated Press}. In the early 21st century, a number of state
attorneys general apologized for the forced sterilization of Black,
mentally ill and disabled people, which began in the early 1900s.

But some of the field's future leaders are now demanding change on
medical school campuses.

Dr. Tequilla Manning, a family medicine resident in New York, graduated
from University of Kansas Medical Center three years ago. As a medical
student, she conducted a research project on Dr. Marjorie Cates, who
became the school's first Black female graduate in 1958. She began to
draw parallels between Dr. Cates' experience of discrimination on campus
and her own.

Before graduating in 2017, she gave a presentation on Dr. Cates' story.
Some of the other students in the audience were inspired. They lobbied
University of Kansas to rename a campus medical society for Dr. Cates;
the group previously honored a dean of the school who had advocated for
racially segregated clinical facilities.

Last year Dr. Manning attended the renaming ceremony for the Cates
Society. ``I was crying,'' she said. ``What I experienced is not on the
spectrum of what my ancestors experienced at the hands of white
physicians. But I spent five years at this institution thinking there
was no hope.''

Watching the school publicly honor its first female Black graduate, she
felt a glimmer of optimism: ``I thought, maybe they do give a damn about
the lives of Black students.''

Advertisement

\protect\hyperlink{after-bottom}{Continue reading the main story}

\hypertarget{site-index}{%
\subsection{Site Index}\label{site-index}}

\hypertarget{site-information-navigation}{%
\subsection{Site Information
Navigation}\label{site-information-navigation}}

\begin{itemize}
\tightlist
\item
  \href{https://help.nytimes3xbfgragh.onion/hc/en-us/articles/115014792127-Copyright-notice}{©~2020~The
  New York Times Company}
\end{itemize}

\begin{itemize}
\tightlist
\item
  \href{https://www.nytco.com/}{NYTCo}
\item
  \href{https://help.nytimes3xbfgragh.onion/hc/en-us/articles/115015385887-Contact-Us}{Contact
  Us}
\item
  \href{https://www.nytco.com/careers/}{Work with us}
\item
  \href{https://nytmediakit.com/}{Advertise}
\item
  \href{http://www.tbrandstudio.com/}{T Brand Studio}
\item
  \href{https://www.nytimes3xbfgragh.onion/privacy/cookie-policy\#how-do-i-manage-trackers}{Your
  Ad Choices}
\item
  \href{https://www.nytimes3xbfgragh.onion/privacy}{Privacy}
\item
  \href{https://help.nytimes3xbfgragh.onion/hc/en-us/articles/115014893428-Terms-of-service}{Terms
  of Service}
\item
  \href{https://help.nytimes3xbfgragh.onion/hc/en-us/articles/115014893968-Terms-of-sale}{Terms
  of Sale}
\item
  \href{https://spiderbites.nytimes3xbfgragh.onion}{Site Map}
\item
  \href{https://help.nytimes3xbfgragh.onion/hc/en-us}{Help}
\item
  \href{https://www.nytimes3xbfgragh.onion/subscription?campaignId=37WXW}{Subscriptions}
\end{itemize}
