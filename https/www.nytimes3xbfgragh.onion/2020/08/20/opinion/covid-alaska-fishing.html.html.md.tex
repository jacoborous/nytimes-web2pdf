Sections

SEARCH

\protect\hyperlink{site-content}{Skip to
content}\protect\hyperlink{site-index}{Skip to site index}

\href{https://myaccount.nytimes3xbfgragh.onion/auth/login?response_type=cookie\&client_id=vi}{}

\href{https://www.nytimes3xbfgragh.onion/section/todayspaper}{Today's
Paper}

\href{/section/opinion}{Opinion}\textbar{}Our Normalness Made Me Drunk.
Maybe It Made Me Stupid.

\url{https://nyti.ms/3aF88Qx}

\begin{itemize}
\item
\item
\item
\item
\item
\end{itemize}

Advertisement

\protect\hyperlink{after-top}{Continue reading the main story}

\href{/section/opinion}{Opinion}

Supported by

\protect\hyperlink{after-sponsor}{Continue reading the main story}

\hypertarget{our-normalness-made-me-drunk-maybe-it-made-me-stupid}{%
\section{Our Normalness Made Me Drunk. Maybe It Made Me
Stupid.}\label{our-normalness-made-me-drunk-maybe-it-made-me-stupid}}

A fishing trip is a reminder of the impulses that stop us from
protecting ourselves.

By Julia O'Malley

Ms. O'Malley is a journalist.

\begin{itemize}
\item
  Aug. 20, 2020
\item
  \begin{itemize}
  \item
  \item
  \item
  \item
  \item
  \end{itemize}
\end{itemize}

\includegraphics{https://static01.graylady3jvrrxbe.onion/images/2020/08/20/opinion/20omalley/20omalley-articleLarge.jpg?quality=75\&auto=webp\&disable=upscale}

ANCHORAGE ---~I arrived at fish camp to find a sofa on the beach with a
fire beside it, crackling inside a washing machine drum. A few hours
before, I'd taken my first flight since the pandemic began --- humming
low in a small plane from here, to the small town of Kenai, to the mouth
of the Kasilof River.

My boyfriend, Jack, fishes that river every year. I came to join him.
Like many Alaskans in the summertime, we were after salmon to fill our
freezer. Jack's friend, Kelly, ran the operation, stringing nets into
the water to snare the salmon as they muscle in from the ocean on the
tide. A group of his friends rotated through to help haul nets and cut
fish, all of us taking home fillets for our labor. Most of the time at
fish camp, though, we didn't work. We watched the nets soak and waited
on the tide.

That day, as Tom Waits bellowed from a little speaker, a couple of
Kelly's friends, who were also helping at camp, took a seat across the
fire from me. Wasn't long till we were in one of those conversations
between people on opposite sides of America's political chasm. Antifa.
How transgender athletes are taking over women's sports. The injustice
of affirmative action.

Usually I appreciate how Alaska makes me sort out liking --- or even
loving --- people with different politics. But I knew we'd eventually
talk about the virus. I wasn't sure I could take it.

After an effective shutdown to stem the spread of the coronavirus,
Alaska reopened its businesses in May. By the time I got to Kenai, in
June, people were hosting barbecues and returning to church.
\href{https://www.alaskapublic.org/2020/06/11/as-covid-19-spikes-in-alaska-kenai-peninsula-emerges-as-virus-hotspot/}{Cases
had started increasing in the area}, and would soon rise quickly in
Anchorage, too.

Compared with states like Texas and Arizona, Alaska had too few cases
and too few deaths for most people to have firsthand experience. And the
way Alaskans saw virus risk had grown increasingly politicized,
depending on where people got their news. I had just read an article
about masks in The Anchorage Daily News posted on Facebook. It was
buried in comments, quoting obscure doctors, about how they didn't work.
Commenters, some of whom I knew, kept posting that the virus was the
same as the flu.

I didn't know whether Kelly's friends took it seriously, but I could
guess. They'd already been telling me about what they were reading on
the far-right One America News Network. Turned out, though, I was wrong.

A couple of hours into our wait, the guys got a call. Their co-worker
had tested positive. Suddenly they looked scared and embarrassed. They
didn't want to get us sick. Kelly proposed we mask up and distance
ourselves. His friends left for town to get tested. The rest of us went
out to check the nets.

Gulls dived and screamed as Jack rowed the rubber dinghy out to the
furthest buoy. Kelly hoisted the net over the bow. One salmon appeared,
then another, with iridescent eyeballs and mouths chewing air. Fish
piled around our feet in a pool of pink water and old beer cans.

God, it felt good to be out there: cumulus clouds, wind, the boat
rocking. I smelled fish blood. It smelled like a regular summer.

When we got back in, I started chopping tails, but the knife was dull
and I kept messing it up. By then, the guys were back from getting
tested, and all our masks had fallen down around our necks. One of them
got tired of watching me and grabbed my hand as I chopped. I felt his
breath on my face as his hand guided the knife with extra force.

The next day, flush with fillets, scales bright as silver coins, I
headed into town for groceries for my final dinner before we left.

At Walmart, I was the only customer wearing a mask. A pregnant woman
filled a bag with apples, while a kid in her cart watched a cartoon on
her phone. Witnessing this felt like finding something precious I'd
lost, a wedding ring in the sand. Life was going on as if we were
recovering from the pandemic, rather than right in the middle of it.

I got to the lodge where I was going to cook and picked rhubarb from the
garden. At first, I imagined cooking for just a few, but soon setting a
generous table became as inevitable as the tide. I sent word to everyone
at camp --- including Kelly's friends --- to come eat.

An hour later, we stood in the kitchen with our paper plates, discussing
smoked salmon brine and layoffs on the oil fields. It felt easy. Like a
thousand summer salmon dinners.

About halfway through, I had an urge to wash my hands. Or at least to
put on my mask. But I didn't. Our normalness made me drunk. Maybe it
made me stupid.

The next day, flying home, I thought about how deep the virus slices
into the sacred, seasonal rituals of our lives. Politics aside, it's
hard to fault someone for the desire to preserve some of their
pre-pandemic life, even if it means believing things that don't really
hold up, even if it's risky.

Anchorage appeared below me, the grid of streets reminding me of how
small my universe had become. The screen on my desk. My children outside
my locked bedroom door as I tried to work. The blocks to the grocery
store.

There were so many losses still to come. There were so many things I
wanted to hold on to.

Julia O'Malley
(\href{https://twitter.com/julia_omalley}{@julia\_omalley}) is a
third-generation Alaskan and the author of ``The Whale and the Cupcake:
Subsistence, Longing and Community in Alaska.''

\emph{The Times is committed to publishing}
\href{https://www.nytimes3xbfgragh.onion/2019/01/31/opinion/letters/letters-to-editor-new-york-times-women.html}{\emph{a
diversity of letters}} \emph{to the editor. We'd like to hear what you
think about this or any of our articles. Here are some}
\href{https://help.nytimes3xbfgragh.onion/hc/en-us/articles/115014925288-How-to-submit-a-letter-to-the-editor}{\emph{tips}}\emph{.
And here's our email:}
\href{mailto:letters@NYTimes.com}{\emph{letters@NYTimes.com}}\emph{.}

\emph{Follow The New York Times Opinion section on}
\href{https://www.facebookcorewwwi.onion/nytopinion}{\emph{Facebook}}\emph{,}
\href{http://twitter.com/NYTOpinion}{\emph{Twitter (@NYTopinion)}}
\emph{and}
\href{https://www.instagram.com/nytopinion/}{\emph{Instagram}}\emph{.}

Advertisement

\protect\hyperlink{after-bottom}{Continue reading the main story}

\hypertarget{site-index}{%
\subsection{Site Index}\label{site-index}}

\hypertarget{site-information-navigation}{%
\subsection{Site Information
Navigation}\label{site-information-navigation}}

\begin{itemize}
\tightlist
\item
  \href{https://help.nytimes3xbfgragh.onion/hc/en-us/articles/115014792127-Copyright-notice}{©~2020~The
  New York Times Company}
\end{itemize}

\begin{itemize}
\tightlist
\item
  \href{https://www.nytco.com/}{NYTCo}
\item
  \href{https://help.nytimes3xbfgragh.onion/hc/en-us/articles/115015385887-Contact-Us}{Contact
  Us}
\item
  \href{https://www.nytco.com/careers/}{Work with us}
\item
  \href{https://nytmediakit.com/}{Advertise}
\item
  \href{http://www.tbrandstudio.com/}{T Brand Studio}
\item
  \href{https://www.nytimes3xbfgragh.onion/privacy/cookie-policy\#how-do-i-manage-trackers}{Your
  Ad Choices}
\item
  \href{https://www.nytimes3xbfgragh.onion/privacy}{Privacy}
\item
  \href{https://help.nytimes3xbfgragh.onion/hc/en-us/articles/115014893428-Terms-of-service}{Terms
  of Service}
\item
  \href{https://help.nytimes3xbfgragh.onion/hc/en-us/articles/115014893968-Terms-of-sale}{Terms
  of Sale}
\item
  \href{https://spiderbites.nytimes3xbfgragh.onion}{Site Map}
\item
  \href{https://help.nytimes3xbfgragh.onion/hc/en-us}{Help}
\item
  \href{https://www.nytimes3xbfgragh.onion/subscription?campaignId=37WXW}{Subscriptions}
\end{itemize}
