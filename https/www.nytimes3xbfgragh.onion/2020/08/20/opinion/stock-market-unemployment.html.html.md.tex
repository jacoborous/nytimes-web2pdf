Sections

SEARCH

\protect\hyperlink{site-content}{Skip to
content}\protect\hyperlink{site-index}{Skip to site index}

\href{https://myaccount.nytimes3xbfgragh.onion/auth/login?response_type=cookie\&client_id=vi}{}

\href{https://www.nytimes3xbfgragh.onion/section/todayspaper}{Today's
Paper}

\href{/section/opinion}{Opinion}\textbar{}Stocks Are Soaring. So Is
Misery.

\url{https://nyti.ms/3kXJCia}

\begin{itemize}
\item
\item
\item
\item
\item
\item
\end{itemize}

Advertisement

\protect\hyperlink{after-top}{Continue reading the main story}

\href{/section/opinion}{Opinion}

Supported by

\protect\hyperlink{after-sponsor}{Continue reading the main story}

\hypertarget{stocks-are-soaring-so-is-misery}{%
\section{Stocks Are Soaring. So Is
Misery.}\label{stocks-are-soaring-so-is-misery}}

Optimism about Apple's future profits won't pay this month's rent.

\href{https://www.nytimes3xbfgragh.onion/by/paul-krugman}{\includegraphics{https://static01.graylady3jvrrxbe.onion/images/2018/04/02/opinion/paul-krugman/paul-krugman-thumbLarge.png}}

By \href{https://www.nytimes3xbfgragh.onion/by/paul-krugman}{Paul
Krugman}

Opinion Columnist

\begin{itemize}
\item
  Aug. 20, 2020
\item
  \begin{itemize}
  \item
  \item
  \item
  \item
  \item
  \item
  \end{itemize}
\end{itemize}

\includegraphics{https://static01.graylady3jvrrxbe.onion/images/2020/08/20/opinion/20krugman1/merlin_175179582_a1596206-9d58-4108-92e8-aba76ad6b76a-articleLarge.jpg?quality=75\&auto=webp\&disable=upscale}

On Tuesday, the \href{https://www.cnbc.com/quotes/?symbol=.SPX}{S\&P
500} stock index hit a record high. The next day,
\href{https://www.nytimes3xbfgragh.onion/2020/08/19/technology/apple-2-trillion.html\#:~:text=Apple\%20surpasses\%20\%242\%20trillion\%2C\%20doubling\%20its\%20value\%20in\%20only\%20two\%20years.\&text=Apple\%20reaches\%20\%241\%20trillion\%20in,42\%20years\%20after\%20its\%20founding.\&text=I.P.O.\&text=Apple\%20surpasses\%20\%242\%20trillion\%2C\%20doubling\%20its\%20value\%20in\%20only\%20two\%20years.,-trillion}{Apple}
became the first U.S. company in history to be valued at more than \$2
trillion. Donald Trump is, of course, touting the stock market as proof
that the economy has recovered from the coronavirus; too bad about those
173,000 dead Americans, but as he
\href{https://www.nytimes3xbfgragh.onion/aponline/2020/08/19/us/politics/ap-us-election-2020-dnc-trump-remark.html}{says},
``It is what it is.''

But the economy probably doesn't feel so great to the millions of
workers who still haven't gotten their jobs back and who have just seen
their unemployment benefits slashed. The \$600 a week supplemental
benefit enacted in March has expired, and Trump's purported replacement
is basically a sick joke.

Even before the aid cutoff, the number of parents reporting that they
were having trouble giving their children enough to eat was
\href{https://www.wsj.com/articles/more-americans-go-hungry-amid-coronavirus-pandemic-census-shows-11597570200}{rising
rapidly}. That number will surely soar in the next few weeks. And we're
also about to see a huge
\href{https://www.nytimes3xbfgragh.onion/2020/08/07/business/economy/housing-economy-eviction-renters.html}{wave
of evictions}, both because families are no longer getting the money
they need to pay rent and because a temporary ban on evictions, like
supplemental unemployment benefits, has just expired.

But how can there be such a disconnect between rising stocks and growing
misery? Wall Street types, who do love their letter games, are talking
about a ``K-shaped recovery'': rising stock valuations and individual
wealth at the top, falling incomes and deepening pain at the bottom. But
that's a description, not an explanation. What's going on?

The first thing to note is that the real economy, as opposed to the
financial markets, is still in terrible shape. The Federal Reserve Bank
of New York's
\href{https://www.newyorkfed.org/research/policy/weekly-economic-index\#/interactive}{weekly
economic index} suggests that the economy, although off its low point a
few months ago, is still more deeply depressed than it was at any point
during the recession that followed the 2008 financial crisis.

And this time around, job losses are concentrated among
\href{https://tracktherecovery.org/}{lower-paid workers} --- that is,
precisely those Americans without the financial resources to ride out
bad times.

What about stocks? The truth is that stock prices have never been
closely tied to the state of the economy. As an old economists' joke has
it, the market has predicted nine of the last five recessions.

Stocks do get hit by financial crises, like the disruptions that
followed the fall of Lehman Brothers in September 2008 and the brief
freeze in credit markets back in March. Otherwise, stock prices are
pretty disconnected from things like jobs or even G.D.P.

And these days, the disconnect is even greater than usual.

For the recent rise in the market has been largely driven by a small
number of technology giants. And the market values of these companies
have very little to do with their current profits, let alone the state
of the economy in general. Instead, they're all about investor
perceptions of the fairly distant future.

Take the example of Apple, with its \$2 trillion valuation. Apple has a
\href{https://www.cnbc.com/2020/08/19/apples-2-trillion-value-proof-that-tim-cooks-services-plan-worked.html}{price-earnings
ratio} --- the ratio of its market valuation to its profits --- of about
33. One way to look at that number is that only around 3 percent of the
value investors place on the company reflects the money they expect it
to make over the course of the next year. As long as they expect Apple
to be profitable years from now, they barely care what will happen to
the U.S. economy over the next few quarters.

Furthermore, the profits people expect Apple to make years from now loom
especially large because, after all, where else are they going to put
their money? Yields on U.S. government bonds, for example, are
\href{https://fred.stlouisfed.org/series/DFII10}{well below} the
expected rate of inflation.

And Apple's valuation is actually less extreme than the valuations of
other tech giants, like Amazon or Netflix.

So big tech stocks --- and the people who own them --- are riding high
because investors believe that they'll do very well in the long run. The
depressed economy hardly matters.

Unfortunately, ordinary Americans get very little of their income from
capital gains, and can't live on rosy projections about their future
prospects. Telling your landlord not to worry about your current
inability to pay rent, because you'll surely have a great job five years
from now, will get you nowhere --- or, more accurately, will get you
kicked out of your apartment and put on the street.

So here's the current state of America: Unemployment is still extremely
high, largely because Trump and his allies first refused to take the
coronavirus seriously, then pushed for an early reopening in a nation
that met none of the conditions for resuming business as usual --- and
even now refuse to get firmly behind basic protective strategies like
widespread mask requirements.

Despite this epic failure, the unemployed were kept afloat for months by
federal aid, which helped avert both humanitarian and economic
catastrophe. But now the aid has been cut off, with Trump and allies as
unserious about the looming economic disaster as they were about the
looming epidemiological disaster.

So everything suggests that even if the pandemic subsides --- which is
by no means guaranteed --- we're about to see a huge surge in national
misery.

Oh, and stocks are up. Why, exactly, should we care?

\emph{The Times is committed to publishing}
\href{https://www.nytimes3xbfgragh.onion/2019/01/31/opinion/letters/letters-to-editor-new-york-times-women.html}{\emph{a
diversity of letters}} \emph{to the editor. We'd like to hear what you
think about this or any of our articles. Here are some}
\href{https://help.nytimes3xbfgragh.onion/hc/en-us/articles/115014925288-How-to-submit-a-letter-to-the-editor}{\emph{tips}}\emph{.
And here's our email:}
\href{mailto:letters@NYTimes.com}{\emph{letters@NYTimes.com}}\emph{.}

\emph{Follow The New York Times Opinion section on}
\href{https://www.facebookcorewwwi.onion/nytopinion}{\emph{Facebook}}\emph{,}
\href{http://twitter.com/NYTOpinion}{\emph{Twitter (@NYTopinion)}}
\emph{and}
\href{https://www.instagram.com/nytopinion/}{\emph{Instagram}}\emph{.}

Advertisement

\protect\hyperlink{after-bottom}{Continue reading the main story}

\hypertarget{site-index}{%
\subsection{Site Index}\label{site-index}}

\hypertarget{site-information-navigation}{%
\subsection{Site Information
Navigation}\label{site-information-navigation}}

\begin{itemize}
\tightlist
\item
  \href{https://help.nytimes3xbfgragh.onion/hc/en-us/articles/115014792127-Copyright-notice}{©~2020~The
  New York Times Company}
\end{itemize}

\begin{itemize}
\tightlist
\item
  \href{https://www.nytco.com/}{NYTCo}
\item
  \href{https://help.nytimes3xbfgragh.onion/hc/en-us/articles/115015385887-Contact-Us}{Contact
  Us}
\item
  \href{https://www.nytco.com/careers/}{Work with us}
\item
  \href{https://nytmediakit.com/}{Advertise}
\item
  \href{http://www.tbrandstudio.com/}{T Brand Studio}
\item
  \href{https://www.nytimes3xbfgragh.onion/privacy/cookie-policy\#how-do-i-manage-trackers}{Your
  Ad Choices}
\item
  \href{https://www.nytimes3xbfgragh.onion/privacy}{Privacy}
\item
  \href{https://help.nytimes3xbfgragh.onion/hc/en-us/articles/115014893428-Terms-of-service}{Terms
  of Service}
\item
  \href{https://help.nytimes3xbfgragh.onion/hc/en-us/articles/115014893968-Terms-of-sale}{Terms
  of Sale}
\item
  \href{https://spiderbites.nytimes3xbfgragh.onion}{Site Map}
\item
  \href{https://help.nytimes3xbfgragh.onion/hc/en-us}{Help}
\item
  \href{https://www.nytimes3xbfgragh.onion/subscription?campaignId=37WXW}{Subscriptions}
\end{itemize}
