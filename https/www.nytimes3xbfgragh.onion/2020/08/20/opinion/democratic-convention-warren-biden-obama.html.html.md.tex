Sections

SEARCH

\protect\hyperlink{site-content}{Skip to
content}\protect\hyperlink{site-index}{Skip to site index}

\href{https://myaccount.nytimes3xbfgragh.onion/auth/login?response_type=cookie\&client_id=vi}{}

\href{https://www.nytimes3xbfgragh.onion/section/todayspaper}{Today's
Paper}

\href{/section/opinion}{Opinion}\textbar{}The Democrats Who Rose to the
Moment

\url{https://nyti.ms/3aKcpSL}

\begin{itemize}
\item
\item
\item
\item
\item
\item
\end{itemize}

Advertisement

\protect\hyperlink{after-top}{Continue reading the main story}

\href{/section/opinion}{Opinion}

Supported by

\protect\hyperlink{after-sponsor}{Continue reading the main story}

\hypertarget{the-democrats-who-rose-to-the-moment}{%
\section{The Democrats Who Rose to the
Moment}\label{the-democrats-who-rose-to-the-moment}}

A nation's soldiers in a time of crisis.

\href{https://www.nytimes3xbfgragh.onion/by/david-brooks}{\includegraphics{https://static01.graylady3jvrrxbe.onion/images/2018/04/03/opinion/david-brooks/david-brooks-thumbLarge-v2.png}}

By \href{https://www.nytimes3xbfgragh.onion/by/david-brooks}{David
Brooks}

Opinion Columnist

\begin{itemize}
\item
  Aug. 20, 2020
\item
  \begin{itemize}
  \item
  \item
  \item
  \item
  \item
  \item
  \end{itemize}
\end{itemize}

\includegraphics{https://static01.graylady3jvrrxbe.onion/images/2020/08/20/opinion/20brooks1/20brooks1-articleLarge.jpg?quality=75\&auto=webp\&disable=upscale}

Some people speak from their depths, and some speak from their shallows.
Some speak to make a name in some political game they're playing. But
others speak from wells of a moral conviction. Their words are not
applause lines; they endure.

Barack Obama spoke at the Democratic convention from his depths. His
speech was not just meant to help the Democrats win an election; it was
to identify a historical crisis and address a spiritual need. The former
law professor spoke from his deep love for our Constitution, the whole
intellectual and moral regime that has been built around it and the way
it is now being betrayed by a self-indulgent narcissist.

His speech was fiercely pro-American and fiercely anti-Trump, showing
that, in fact, to be fiercely pro-American you have to be fiercely
anti-Trump.

But Obama went far beyond the election to address the crisis of national
faith beneath the crisis of politics. He spoke from Philadelphia, site
of our true founding that, as flawed as it was, provided the moral
source that points us toward justice.

He spoke to all those young people who, having drawn the lessons from
the doleful events of the past few years and from the propaganda of
their high school curriculums, question whether America is so special
after all. Obama held up, by contrast, those generations of
African-Americans who lived under the lash or the threat of the noose
and who had every reason to lose faith in America but who did not lose
faith and instead redoubled their efforts for its salvation.

His speech was not the only act of devotion at the Democratic convention
this week. Bernie Sanders has served his version of socialism for 50
years. For several weeks last winter, it looked as if he would be the
nominee and this convention would be his. That was snatched from him.

But he put his love of country above his dream and laid it all at the
feet of Joe Biden. In his words, you could hear an old man's awareness
of this crisis of the moment and his surrender of self to the larger
purpose. That was an impressive moral act.

Elizabeth Warren loves her plans, but in her speech you heard not a
wonk's delight in technocracy, but the emotional power of a thousand
wrenching life stories told to her through tears on the campaign trail
--- of mothers defeated by the impossible demands of work and child
care, of young men eviscerated by the self-doubt borne from joblessness.
No politician is as good at translating the arcana of policy to the
language of pain, suffering and relief.

There have been a lot of other speeches, and most of them have been
instantly forgettable --- lacking emotional honesty, philosophic depth
or literary grace. I hope that in some future speech Kamala Harris moves
beyond being a historic symbol and opens her heart and mind. Bill
Clinton didn't need to be there. Jeffrey Epstein's buddy could have
served himself and his party through a year of silence and penance.

And then there have been the ``regular people.'' The virtual convention
is a great equalizer. The people who are usually just members of a
cheering throng are being given more of a chance to tell us about their
lives --- a withering illness, the terrors of a drunken husband slashing
them in the night, even just the awesomeness of fried calamari.

When you let actual people speak, what you get is not angry populism ---
that TV studio concoction --- but hope in the struggle of everyday life.

And this is where I put the Bidens. One way to see Joe Biden is as the
Hubert Humphrey of our day, a party fixture and a conventional pol. But
that's not quite right. The better way to see Biden is as a regular
person who entered into politics but never quite got the game, who is
goofy, heartfelt, unpolished, undisciplined, incapable of being
manipulative. The way a lot of regular people actually are. Jill in a
classroom. Joe on the train.

Some think Biden isn't smart enough to handle the complexities of the
presidency, or is too old and has lost a step. But this convention, the
presidency, and life in general, reveal depths or lack of depths.

Don't underestimate the importance of the depth of Biden's family
values, the depth of his working-class roots, the fact that he is a
person who did not emerge from the valley of grief with empty hands.
Don't underestimate the capacities of a person who does not see
populations in the mass, or subjects in some study, but each person one
by one.

When your democracy is in crisis, you don't need cleverness above all or
dexterity at playing the game. You need someone with the ability to
stick himself down and hold fiercely onto what is precious.

Some young activists give the impression that they invented the struggle
for justice and that everything that came before them is rotten. But the
struggle is as old as America --- 1776, 1860, 1965, 1989. Biden offers a
return to normalcy, but in America the struggle is normalcy.

\emph{The Times is committed to publishing}
\href{https://www.nytimes3xbfgragh.onion/2019/01/31/opinion/letters/letters-to-editor-new-york-times-women.html}{\emph{a
diversity of letters}} \emph{to the editor. We'd like to hear what you
think about this or any of our articles. Here are some}
\href{https://help.nytimes3xbfgragh.onion/hc/en-us/articles/115014925288-How-to-submit-a-letter-to-the-editor}{\emph{tips}}\emph{.
And here's our email:}
\href{mailto:letters@NYTimes.com}{\emph{letters@NYTimes.com}}\emph{.}

\emph{Follow The New York Times Opinion section on}
\href{https://www.facebookcorewwwi.onion/nytopinion}{\emph{Facebook}}\emph{,}
\href{http://twitter.com/NYTOpinion}{\emph{Twitter (@NYTopinion)}}
\emph{and}
\href{https://www.instagram.com/nytopinion/}{\emph{Instagram}}\emph{.}

Advertisement

\protect\hyperlink{after-bottom}{Continue reading the main story}

\hypertarget{site-index}{%
\subsection{Site Index}\label{site-index}}

\hypertarget{site-information-navigation}{%
\subsection{Site Information
Navigation}\label{site-information-navigation}}

\begin{itemize}
\tightlist
\item
  \href{https://help.nytimes3xbfgragh.onion/hc/en-us/articles/115014792127-Copyright-notice}{©~2020~The
  New York Times Company}
\end{itemize}

\begin{itemize}
\tightlist
\item
  \href{https://www.nytco.com/}{NYTCo}
\item
  \href{https://help.nytimes3xbfgragh.onion/hc/en-us/articles/115015385887-Contact-Us}{Contact
  Us}
\item
  \href{https://www.nytco.com/careers/}{Work with us}
\item
  \href{https://nytmediakit.com/}{Advertise}
\item
  \href{http://www.tbrandstudio.com/}{T Brand Studio}
\item
  \href{https://www.nytimes3xbfgragh.onion/privacy/cookie-policy\#how-do-i-manage-trackers}{Your
  Ad Choices}
\item
  \href{https://www.nytimes3xbfgragh.onion/privacy}{Privacy}
\item
  \href{https://help.nytimes3xbfgragh.onion/hc/en-us/articles/115014893428-Terms-of-service}{Terms
  of Service}
\item
  \href{https://help.nytimes3xbfgragh.onion/hc/en-us/articles/115014893968-Terms-of-sale}{Terms
  of Sale}
\item
  \href{https://spiderbites.nytimes3xbfgragh.onion}{Site Map}
\item
  \href{https://help.nytimes3xbfgragh.onion/hc/en-us}{Help}
\item
  \href{https://www.nytimes3xbfgragh.onion/subscription?campaignId=37WXW}{Subscriptions}
\end{itemize}
