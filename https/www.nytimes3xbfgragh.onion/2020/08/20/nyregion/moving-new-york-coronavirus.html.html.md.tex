Sections

SEARCH

\protect\hyperlink{site-content}{Skip to
content}\protect\hyperlink{site-index}{Skip to site index}

\href{https://www.nytimes3xbfgragh.onion/section/nyregion}{New York}

\href{https://myaccount.nytimes3xbfgragh.onion/auth/login?response_type=cookie\&client_id=vi}{}

\href{https://www.nytimes3xbfgragh.onion/section/todayspaper}{Today's
Paper}

\href{/section/nyregion}{New York}\textbar{}Movers in N.Y.C. Are So Busy
They're Turning People Away

\url{https://nyti.ms/3hqozCA}

\begin{itemize}
\item
\item
\item
\item
\item
\end{itemize}

\href{https://www.nytimes3xbfgragh.onion/spotlight/at-home?action=click\&pgtype=Article\&state=default\&region=TOP_BANNER\&context=at_home_menu}{At
Home}

\begin{itemize}
\tightlist
\item
  \href{https://www.nytimes3xbfgragh.onion/2020/08/14/dining/lobster-salad-recipe.html?action=click\&pgtype=Article\&state=default\&region=TOP_BANNER\&context=at_home_menu}{Make:
  Lobster Salad}
\item
  \href{https://www.nytimes3xbfgragh.onion/2020/08/15/at-home/coronavirus-at-home-quick-exercises.html?action=click\&pgtype=Article\&state=default\&region=TOP_BANNER\&context=at_home_menu}{Sneak
  In: Exercise}
\item
  \href{https://www.nytimes3xbfgragh.onion/interactive/2020/at-home/even-more-reporters-editors-diaries-lists-recommendations.html?action=click\&pgtype=Article\&state=default\&region=TOP_BANNER\&context=at_home_menu}{See:
  Reporters' Obsessions}
\item
  \href{https://www.nytimes3xbfgragh.onion/2020/08/15/at-home/coronavirus-fall-patio-furniture.html?action=click\&pgtype=Article\&state=default\&region=TOP_BANNER\&context=at_home_menu}{Deck
  Out: Your Porch}
\end{itemize}

Advertisement

\protect\hyperlink{after-top}{Continue reading the main story}

Supported by

\protect\hyperlink{after-sponsor}{Continue reading the main story}

\hypertarget{movers-in-nyc-are-so-busy-theyre-turning-people-away}{%
\section{Movers in N.Y.C. Are So Busy They're Turning People
Away}\label{movers-in-nyc-are-so-busy-theyre-turning-people-away}}

With so many people fleeing the city, moving companies can barely keep
up with the demand.

\includegraphics{https://static01.graylady3jvrrxbe.onion/images/2020/08/23/nyregion/23nyvirus-movers1/23nyvirus-movers1-articleLarge.jpg?quality=75\&auto=webp\&disable=upscale}

By Julie Satow

\begin{itemize}
\item
  Aug. 20, 2020
\item
  \begin{itemize}
  \item
  \item
  \item
  \item
  \item
  \end{itemize}
\end{itemize}

Squeezing a 350-pound sofa down four flights of a narrow prewar
staircase isn't easy on the best of days. Doing so in a heat wave while
wearing a mask is a lot harder.

``Sweat is dripping down your face, it slips,'' said Vladislav Grigor, a
foreman and dispatcher at Empire Movers in Manhattan. ``It is just
terrible.''

While the work can be merciless, movers are busy this season, and glad
of it.

For Mr. Grigor and his colleagues, it is fair to say this summer has
been like no other. Not only is he having to meet the strenuous physical
demands of his job during a steamy summer, but he is also having to do
so while abiding by the new rules of social distancing.

On top of these challenges is just how overworked movers are. ``It's
nuts out there,'' Mr. Grigor said. ``There is double the volume of
customers --- maybe more --- than last year.''

While the moving industry is fractured among numerous small business
owners, and official statistics are tough to come by, one thing is
clear: From professionals who are downsizing following a job loss, to
students moving back in with their parents, to families fleeing the city
for the suburbs, New Yorkers are changing their addresses in droves.

\includegraphics{https://static01.graylady3jvrrxbe.onion/images/2020/08/23/nyregion/23jpnyvirus-movers2/23jpnyvirus-movers2-articleLarge.jpg?quality=75\&auto=webp\&disable=upscale}

According to FlatRate Moving, the number of moves it has done has
increased more than 46 percent between March 15 and August 15, compared
with the same period last year. The number of those moving outside of
New York City is up 50 percent --- including a nearly 232 percent
increase to Dutchess County and 116 percent increase to Ulster County in
the Hudson Valley.

Image

Cera camps out in the bathroom during a late-June move.Credit...OK
McCausland for The New York Times

Image

On this day, seven moves were scheduled for a single building in Hell's
Kitchen.Credit...OK McCausland for The New York Times

``It felt like move-out day on a college campus,'' said Bobby DelGreco,
who recently vacated his apartment in Stuyvesant Town after nine years
and is now living in a long-term Airbnb in Los Angeles. ``All the doors
were propped open, and there were moving trucks and furniture
everywhere.''

Matt Jahn, who owns the Brooklyn-based Metropolis Moving, said he has
been inundated with customer requests. It's more demand than he can
handle. ``We are turning people away because we just don't have the
capacity,'' he said. ``Normally, in a given summer, we spend a bunch on
advertising. But we cut it this year because we couldn't afford it. And
we have still had amazing demand.''

It certainly didn't start out this way. In mid-March, when the grim
realities of Covid-19 became clear, moving companies braced for a slow
season. ``Right in the beginning, we weren't sure if we were allowed to
work, and a lot of businesses were in limbo,'' said Daniel Norber, the
owner of Imperial Movers, based in the West Village. ``Everyone was
wondering if they should close shop.''

But later that month, Gov. Andrew M. Cuomo announced that moving was
considered an essential service. ``Within 30 minutes of the announcement
I got a flood of calls,'' Mr. Jahn said. Business hasn't slowed since,
and movers expect the trend to continue through the fall.

Image

The moving industry has been overwhelmed by demand in New York
City.Credit...OK McCausland for The New York Times

``The first day we could move, we left,'' said Jaime Welsh-Rajchel. In
mid-March, Dr. Welsh-Rajchel, a dentist, and her young son, Henry, took
refuge from the city with family in Pennsylvania, while her husband,
Todd Rajchel, a dental anesthesiologist at Wyckoff Heights Medical
Center in Bushwick, stayed behind to spend the height of the pandemic
intubating Covid patients.

Dr. Rajchel has since accepted a position at the School of Dentistry at
Creighton University in Omaha, and his wife, Dr. Welsh-Rajchel, returned
to Brooklyn just long enough to help move their items. ``Todd was saying
we need a five-year period to decompress from this experience before we
can come back to New York for a visit,'' she said.

While demand is strong, there are new struggles for moving companies,
mostly because of an industrywide labor shortage. ``Everyone wanted to
flee New York because it was the epicenter, but at the same time, our
movers started getting sick,'' said Mr. Norber of Imperial Movers,
adding that the company lost a dozen workers who either fell ill or were
too afraid to come in. He has since begun using company vans to pick up
workers so they can avoid public transportation. While many have
returned to the job, Mr. Norber's company remains short-staffed and is
operating at about 40 percent capacity.

Moving companies throughout the city told a similar story.

Image

Belongings in a Brooklyn apartment whose tenants are leaving New
York.Credit...OK McCausland for The New York Times

Image

Packing to leave Williamsburg.Credit...OK McCausland for The New York
Times

``We didn't know what the summer would bring, so we didn't ramp up
hiring as quickly,'' said David L. Giampietro, the chief administrative
officer for FlatRate Moving. Then, once it became obvious there would be
a lot of demand for their services, ``all the moving companies were
competing for workers.''

Another factor that made hiring difficult was the \$600 government
stimulus checks that many workers received until the program expired
earlier this month. ``No one said it outright, but it's our assumption
that also was a factor,'' Mr. Giampietro said. ``People didn't want to
come to work because of the program,'' said Mr. Grigor, the foreman.
``Why make \$1,000 working, when you can make the same money not
working?''

Image

The conditions can be brutal, but movers are finding no shortage of
work.Credit...OK McCausland for The New York Times

But there was plenty of work for those who wanted it. This summer, Kiril
Gor has been working for Imperial Movers six out of seven days, taking
in about \$1,500 a week.

``This is dangerous, of course, because of Covid, but we just keep
working because we need the money. I don't want to get some money from
the government,'' added Mr. Gor, who emigrated from Russia. ``I want to
make it myself.''

Three months ago, Sam Hassan joined Imperial Movers as a driver after
his business overseeing logistics for corporate events shut down in the
wake of the pandemic. ``I refuse to feel sorry for myself,'' said Mr.
Hassan, a 57-year-old fitness buff. ``It is a lot of fun. I'm being paid
to work out.''

Image

``It's nuts out there,'' one mover said. ``There is double the volume of
customers --- maybe more --- than last year.''Credit...OK McCausland for
The New York Times

While moving companies have been doing their best to manage the influx
of customers, as the city continues to empty of residents, conflicted
feelings abound, especially for those who have left New York in a hurry.
Dr. Welsh-Rajchel, for instance, wishes her family had more time for
closure. She returned in late June for a few days to help her husband
move out of their apartment, but their hopes for a last hurrah were
quickly dashed. ``It was a bummer. There were things we wanted to do
before we left, but everything was closed,'' she said. ``It was like a
ghost city.''

They were particularly wistful that they couldn't have one last meal at
Crif Dogs on Driggs Avenue in Williamsburg before it closed during the
shutdown. ``It was one of the first places my husband ever went when we
moved here,'' she said. ``He loves hot dogs. It would have been nice to
go back one last time.''

Advertisement

\protect\hyperlink{after-bottom}{Continue reading the main story}

\hypertarget{site-index}{%
\subsection{Site Index}\label{site-index}}

\hypertarget{site-information-navigation}{%
\subsection{Site Information
Navigation}\label{site-information-navigation}}

\begin{itemize}
\tightlist
\item
  \href{https://help.nytimes3xbfgragh.onion/hc/en-us/articles/115014792127-Copyright-notice}{©~2020~The
  New York Times Company}
\end{itemize}

\begin{itemize}
\tightlist
\item
  \href{https://www.nytco.com/}{NYTCo}
\item
  \href{https://help.nytimes3xbfgragh.onion/hc/en-us/articles/115015385887-Contact-Us}{Contact
  Us}
\item
  \href{https://www.nytco.com/careers/}{Work with us}
\item
  \href{https://nytmediakit.com/}{Advertise}
\item
  \href{http://www.tbrandstudio.com/}{T Brand Studio}
\item
  \href{https://www.nytimes3xbfgragh.onion/privacy/cookie-policy\#how-do-i-manage-trackers}{Your
  Ad Choices}
\item
  \href{https://www.nytimes3xbfgragh.onion/privacy}{Privacy}
\item
  \href{https://help.nytimes3xbfgragh.onion/hc/en-us/articles/115014893428-Terms-of-service}{Terms
  of Service}
\item
  \href{https://help.nytimes3xbfgragh.onion/hc/en-us/articles/115014893968-Terms-of-sale}{Terms
  of Sale}
\item
  \href{https://spiderbites.nytimes3xbfgragh.onion}{Site Map}
\item
  \href{https://help.nytimes3xbfgragh.onion/hc/en-us}{Help}
\item
  \href{https://www.nytimes3xbfgragh.onion/subscription?campaignId=37WXW}{Subscriptions}
\end{itemize}
