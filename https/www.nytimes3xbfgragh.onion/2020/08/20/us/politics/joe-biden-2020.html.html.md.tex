Sections

SEARCH

\protect\hyperlink{site-content}{Skip to
content}\protect\hyperlink{site-index}{Skip to site index}

\href{https://www.nytimes3xbfgragh.onion/section/politics}{Politics}

\href{https://myaccount.nytimes3xbfgragh.onion/auth/login?response_type=cookie\&client_id=vi}{}

\href{https://www.nytimes3xbfgragh.onion/section/todayspaper}{Today's
Paper}

\href{/section/politics}{Politics}\textbar{}Joe Biden Finally Got the
Timing Right

\url{https://nyti.ms/2YhrZA3}

\begin{itemize}
\item
\item
\item
\item
\item
\end{itemize}

\begin{itemize}
\item
  \href{https://www.nytimes3xbfgragh.onion/live/2020/08/20/us/dnc-convention-election?action=click\&pgtype=Article\&state=default\&region=TOP_BANNER\&context=storylines_menu}{D.N.C.
  Updates}
\item
  \href{https://www.nytimes3xbfgragh.onion/2020/08/20/us/politics/biden-presidential-nomination-dnc.html?action=click\&pgtype=Article\&state=default\&region=TOP_BANNER\&context=storylines_menu}{Biden's
  Speech}
\item
  \href{https://www.nytimes3xbfgragh.onion/interactive/2019/us/elections/2020-presidential-election-calendar.html?action=click\&pgtype=Article\&state=default\&region=TOP_BANNER\&context=storylines_menu}{Election
  Calendar}
\item
  \href{https://www.nytimes3xbfgragh.onion/interactive/2020/08/11/us/politics/vote-by-mail-us-states.html?action=click\&pgtype=Article\&state=default\&region=TOP_BANNER\&context=storylines_menu}{Voting
  by Mail}
\item
  \href{https://www.nytimes3xbfgragh.onion/newsletters/politics?action=click\&pgtype=Article\&state=default\&region=TOP_BANNER\&context=storylines_menu}{Politics
  Newsletter}
\end{itemize}

Advertisement

\protect\hyperlink{after-top}{Continue reading the main story}

Supported by

\protect\hyperlink{after-sponsor}{Continue reading the main story}

\hypertarget{joe-biden-finally-got-the-timing-right}{%
\section{Joe Biden Finally Got the Timing
Right}\label{joe-biden-finally-got-the-timing-right}}

More than three decades after his first presidential run, in the midst
of a deadly pandemic and without his son Beau at his side, the former
vice president has met his moment.

\includegraphics{https://static01.graylady3jvrrxbe.onion/images/2020/08/20/us/politics/20biden-top/merlin_175970319_cc3ae3f4-386f-4f75-86fa-8cc3c8afbe13-articleLarge.jpg?quality=75\&auto=webp\&disable=upscale}

\href{https://www.nytimes3xbfgragh.onion/by/matt-flegenheimer}{\includegraphics{https://static01.graylady3jvrrxbe.onion/images/2018/10/02/multimedia/author-matt-flegenheimer/author-matt-flegenheimer-thumbLarge.png}}

By \href{https://www.nytimes3xbfgragh.onion/by/matt-flegenheimer}{Matt
Flegenheimer}

\begin{itemize}
\item
  Published Aug. 20, 2020Updated Aug. 21, 2020, 2:44 a.m. ET
\item
  \begin{itemize}
  \item
  \item
  \item
  \item
  \item
  \end{itemize}
\end{itemize}

Joe Biden tells funny stories at funerals and sad ones at campaign
stops.

He has been running for president long enough to lose the 1988
Democratic primary as a hard-charging 40-something pushing generational
change --- and to win the 2020 primary as the white-haired statesman who
has steered through sorrow, who can still sniff it out in any room and
close in like a pang-seeking missile for the stricken.

``He asked if I was OK and gave me a hug,'' a cane-shuffling Iowa man,
Brian Peters, said last winter, blinking away tears after pledging his
support to Mr. Biden on a characteristically misty post-event rope line.
``I told him that I would be.''

Maybe it had to happen this way, friends say, if it was going to happen
at all: After nearly a half-century of public life defined most
viscerally by the forced commingling of politics and personal loss, the
tint of the moment at last matches Mr. Biden's own story: shadowed by
despair, sustained by faith --- in himself; in God; in the human
capacity for resilience, founded or not.

``We all are an accumulation of our life's experiences,'' said Joe
Riley, a friend of Mr. Biden's and the former longtime mayor of
Charleston, S.C.

And Joseph Robinette Biden Jr.'s experiences have delivered him here. He
has at last captured the
\href{https://www.nytimes3xbfgragh.onion/2020/08/20/us/politics/Joe-Biden-accepts-democratic-nomination.html}{Democratic
presidential nomination}, earning the chance to face President Trump
because he is, admirers say, all the things that the incumbent is not:
empathetic, dependable, decent.

\includegraphics{https://static01.graylady3jvrrxbe.onion/images/2020/08/20/us/politics/20dems-ledeall-top1/20dems-ledeall-top1-videoSixteenByNine3000-v2.jpg}

``Character is on the ballot,'' Mr. Biden said in his convention address
Thursday evening, inside a quiet hall in Wilmington, Del. ``Compassion
is on the ballot.''

There is some irony, Democrats concede, in the idea that Mr. Biden
prevailed because voters found him comforting and familiar. Through his
years in presidential politics, his longevity has generally served to
remind his skeptics of all they believe he has gotten wrong: He voted to
authorize the use of military force in Iraq and came to regret it. He
presided over the committee that subjected Anita Hill to demeaning and
invasive questioning in the Supreme Court confirmation hearings for
now-Justice Clarence Thomas. He helped craft tough-on-crime legislation
that many criminal justice experts now associate with mass
incarceration.

In this primary campaign, Mr. Biden, 77, could often appear almost
willfully out of step with the times, telling debate viewers to keep
their record players on at night for children's educational purposes and
warmly remembering his relationships with segregationist senators.

He won anyway, stepping to the lectern on Thursday having reached the
precipice of a prize he has chased for more than three decades and
talked about since grade school.

Yet like many flashes of triumph in his long career, this one is not as
he imagined it, the would-be jubilation laced with an abiding gravity.

His speech skewed sober, befitting the national mood. He did not have a
large crowd to cheer him through it in person, in deference to the
pandemic that has overwhelmed the country he hopes to lead. He did not
have Beau Biden, his son and political heir, who died in 2015 while
pleading with his father not to withdraw from the public arena. A video
tribute to him played instead.

\hypertarget{latest-updates-2020-election}{%
\section{\texorpdfstring{\href{https://www.nytimes3xbfgragh.onion/live/2020/08/19/us/dnc-convention-election?action=click\&pgtype=Article\&state=default\&region=MAIN_CONTENT_1\&context=storylines_live_updates}{Latest
Updates: 2020
Election}}{Latest Updates: 2020 Election}}\label{latest-updates-2020-election}}

\href{https://www.nytimes3xbfgragh.onion/live/2020/08/19/us/dnc-convention-election?action=click\&pgtype=Article\&state=default\&region=MAIN_CONTENT_1\&context=storylines_live_updates\#night-3-featured-more-policy-a-focus-on-women-and-a-full-throated-rejection-of-trump-by-his-predecessor}{7h
ago}

\href{https://www.nytimes3xbfgragh.onion/live/2020/08/19/us/dnc-convention-election?action=click\&pgtype=Article\&state=default\&region=MAIN_CONTENT_1\&context=storylines_live_updates\#night-3-featured-more-policy-a-focus-on-women-and-a-full-throated-rejection-of-trump-by-his-predecessor}{Night
3 featured more policy, a focus on women and a full-throated rejection
of Trump by his predecessor.}

\href{https://www.nytimes3xbfgragh.onion/live/2020/08/19/us/dnc-convention-election?action=click\&pgtype=Article\&state=default\&region=MAIN_CONTENT_1\&context=storylines_live_updates\#trump-live-tweeted-obamas-speech-tonight-hell-appear-on-fox-news-right-before-bidens-tomorrow}{9h
ago}

\href{https://www.nytimes3xbfgragh.onion/live/2020/08/19/us/dnc-convention-election?action=click\&pgtype=Article\&state=default\&region=MAIN_CONTENT_1\&context=storylines_live_updates\#trump-live-tweeted-obamas-speech-tonight-hell-appear-on-fox-news-right-before-bidens-tomorrow}{Trump
live-tweeted Obama's speech tonight. He'll appear on Fox News right
before Biden's tomorrow.}

\href{https://www.nytimes3xbfgragh.onion/live/2020/08/19/us/dnc-convention-election?action=click\&pgtype=Article\&state=default\&region=MAIN_CONTENT_1\&context=storylines_live_updates\#advocates-for-domestic-violence-survivors-praised-biden-in-a-video}{9h
ago}

\href{https://www.nytimes3xbfgragh.onion/live/2020/08/19/us/dnc-convention-election?action=click\&pgtype=Article\&state=default\&region=MAIN_CONTENT_1\&context=storylines_live_updates\#advocates-for-domestic-violence-survivors-praised-biden-in-a-video}{Advocates
for domestic violence survivors praised Biden in a video.}

\href{https://www.nytimes3xbfgragh.onion/live/2020/08/19/us/dnc-convention-election?action=click\&pgtype=Article\&state=default\&region=MAIN_CONTENT_1\&context=storylines_live_updates}{See
more updates}

``Beau should be the one running,'' Mr. Biden said in January, choking
up in a television interview.

But then, the ``should'' constructions have never much cooperated in Mr.
Biden's arc, where the bitter and the sweet tend to find each other in
metronomic succession.

His underdog Senate victory in 1972, as a relentless 29-year-old who did
not know better, came a month before the car crash that killed his wife
and daughter and injured his two sons, the trauma that forever enshrined
him as an avatar of bereavement in the public consciousness.

His debacle of a first presidential bid, for the 1988 Democratic
nomination, collapsed just as he was carrying off one of his signature
congressional achievements: helping to engineer the defeat of a deeply
conservative Supreme Court nominee, Judge Robert H. Bork.

Then an aneurysm nearly killed him.

\includegraphics{https://static01.graylady3jvrrxbe.onion/images/2020/08/20/us/politics/20biden-swear/merlin_175764417_d68794e6-95d2-423b-b06b-d61c2d24caed-articleLarge.jpg?quality=75\&auto=webp\&disable=upscale}

Image

\textbf{June 1987}: Mr. Biden announced his first presidential campaign
with his family, from left, his wife, Jill, and children, Hunter, Ashley
and Beau.Credit...Keith Meyers/The New York Times

Image

\textbf{September 1987}: Mr. Biden spoke with Senator Strom Thurmond of
South Carolina, a segregationist, during the confirmation hearings for
Robert H. Bork.Credit...Jose R. Lopez/The New York Times

The signal promotion of Mr. Biden's career to date --- his elevation to
the vice presidency --- came after another campaign flameout in 2008.
And his eight years as President Barack Obama's chief lieutenant ended
with Mr. Biden a tragic figure once more, burying Beau and deciding
against another run in 2016.

Long fluent in the emotional force of foreboding Irish poetry and
proverb, Mr. Biden has been known to lean on an axiom borrowed from
Daniel Patrick Moynihan, his former Senate colleague: ``I don't think
there's any point in being Irish if you don't know that the world is
going to break your heart eventually.''

So it has gone, on some level, in each chapter of Mr. Biden's biography:
the boy with the stutter; the young man in a hurry; the senator with a
binder of old eulogies in his office, a brimful accounting of his grief.

In one of them, for his father in 2002, he offered up this working
definition of a Biden man: ``a dreamer burdened with reality, a
sensitive spirit layered in stoicism.''

That sounds about right, people close to him say, for better or worse.
He has nurtured his White House dreams and, in his penchant for
exaggeration, occasionally strained to recast reality. He has laid bare
his sensitivity --- he is a hugger and a crier, a walking purveyor of
vulnerability --- and been left to suffer his losses stoically at times,
maintaining a public profile through private anguish.

``He has inordinate strength,'' said Carol Balick, a longtime family
friend whose husband hired Mr. Biden as a young lawyer. ``He doesn't
carry a mythology about himself.''

But he does have his stories, repeated and refashioned through the years
with a homespun sweep calibrated to his audiences.

He was raised in Scranton, Pa., the sort of white working-class hub that
became part of his political coalition, and in Delaware --- the son of a
car-salesman father and a strong-willed mother who encouraged Mr. Biden
through his speech difficulties, telling him he was just so bright that
he couldn't get the thoughts out fast enough.

While his renderings of his youth can feel culled from a Norman
Rockwell, with Mass on Sundays and penny candy for a neighborhood snack,
the Bidens slogged through financial hardships severe enough that they
were forced at one point to move in with his mother's parents.

More distinguished as an athlete than a student, Mr. Biden edged into
adulthood amid the swirl of 1960s activism but found himself at a clear
remove from it. He has at once described civil rights as the animating
cause of his interest in public service and overstated his own
participation in the struggle, compelling advisers years later to gently
remind him that if he did not actually ``march,'' he should probably
stop telling voters that he did.

Image

\textbf{January 2008}: Mr. Biden's second presidential campaign ended
after the Iowa caucuses.Credit...Joshua Lott for The New York Times

Image

\textbf{September 2012}: Mr. Biden and President Barack Obama were
nominated for a second term at the Democratic convention in Charlotte,
N.C.Credit...Doug Mills/The New York Times

Image

\textbf{June 2015}: After his son Beau died of brain cancer, Mr. Biden
decided not to run for president.Credit...Doug Mills/The New York Times

In fact, Mr. Biden's most consequential encounter around this time
happened poolside in the Bahamas, by his account, during a spring break
trip in 1964, his junior year at the University of Delaware.

``I've got the blonde,'' Mr. Biden told a friend, zipping toward a
stranger, Neilia Hunter, and her suntanning companion.

Mr. Biden and Ms. Hunter had dinner that night. They were married two
years later.

And this, Mr. Biden has suggested, is what most informed his throwback
bearing in this period of national upheaval. He was growing up fast: a
family, a burgeoning legal career, a run for county council soon enough.

``I was married, I was in law school,'' he told reporters once,
explaining his psychic distance from the antiwar fervor of his
contemporaries. ``I wore sport coats. I was not part of that. I'm
serious!''

He was. And he did not lack for ambition. Even in his 20s, Mr. Biden was
a plotter, a grinder, a wear-you-down talker.

If he could seem, at times, like an older man in a young man's body, his
next job would only amplify the effect. With his audacious, successful
1972 challenge to the incumbent senator, J. Caleb Boggs, Mr. Biden saw
his future snapping into place. He was a senator-elect before turning
30. He had a wife and three children already.

And then the crash.

Those who knew him then recall those early Senate days as a kind of
rolling thunderstorm, breaking occasionally but never clearing in full.

``Even after it got better --- where after four, five, six months you'd
go and things would seem kind of normal --- then one day it was right
back in the beginning,'' said Ted Kaufman, a longtime friend and aide
who briefly succeeded Mr. Biden in the Senate. ``He'd come into work,
and he was clearly hurting. But he came, and he did it.''

Mr. Biden likes to talk about the people who rescued him in these years:
the senators who looked after him, cementing his lifelong reverence for
the chamber, and the woman --- Jill Jacobs, for a time --- who rebuilt
his family.

``She put us back together,'' Mr. Biden said in a video presentation
during the convention this week. ``She gave me back my life. She gave us
back a family.''

As Mr. Biden's Senate tenure swelled, rumblings about a White House run
became something of a quadrennial tradition.

His first campaign, like this one, was premised as much on his personal
integrity as any signature policy push.

His second --- two decades later, by which time Mr. Biden had spent more
earthly years in the Senate than outside of it --- centered on
experience and judgment, drawing on his grounding in foreign affairs and
his talent for ``God-love-ya'' glad-handing.

That both failed is a matter of political shortcomings, yes, but also of
timing.

Image

\textbf{January 2020}: Mr. Biden's third presidential campaign faltered
at first as he finished fourth in the Iowa caucuses and fifth in the New
Hampshire primary.Credit...Tamir Kalifa for The New York Times

Image

\textbf{February 2020}: Black voters in South Carolina gave Mr. Biden a
commanding win in the state's primary, igniting his political
comeback.Credit...Maddie McGarvey for The New York Times

Image

\textbf{June 2020}: During one of his first events after the coronavirus
led to a suspension of in-person campaigning, Mr. Biden prayed with
community leaders at Bethel A.M.E. Church in Wilmington,
Del.Credit...Erin Schaff/The New York Times

This Joe Biden, the one who won the 2020 primary, is still known best
for all he has lost. He is still liable to misstate, misstep, mishandle.
And he is still often at his strongest offstage, deploying long hugs and
finger-guns among the well-wishers.

If the pandemic crystallized Mr. Biden's rationale for the nomination,
even after he had effectively claimed it, it also reinforced his
longstanding case against Mr. Trump as a national emergency unto
himself.

Often, his supporters' argument has seemed this simple: You need a good
man to defeat a bad man.

Barbara Boxer, a former Democratic senator from California, turned a
phone interview over to her spouse as she worked to summarize her former
colleague's appeal. ``My husband said, `In a word, he's a mensch,''' Ms.
Boxer reported. ``You should say her husband leaned over and said, `He's
a mensch.' But it's true.''

In recent weeks, friends say Mr. Biden has approached his convention
spotlight with a solemnity reflecting the nation's distress. He has said
this is not about ego and never was. He has said he could die happy
without ever hearing ``Hail to the Chief'' play for him.

He has also thought he would be president before --- if never this deep
into a campaign --- only to meet a reversal of fortunes.

``He's been pretty reserved,'' Representative James Clyburn, the South
Carolina Democrat whose endorsement helped revive Mr. Biden's
once-floundering bid, said of the candidate's present outlook.

And why would a polling lead change that? Why would the presidency?

``That's what losing will do for you,'' Mr. Clyburn reasoned.

That is what Joe Biden understands.

\hypertarget{our-2020-election-guide}{%
\section{Our 2020 Election Guide}\label{our-2020-election-guide}}

Updated Aug. 20, 2020

\begin{itemize}
\item
  \begin{center}\rule{0.5\linewidth}{\linethickness}\end{center}

  \hypertarget{convention-recap}{%
  \subsection{Convention Recap}\label{convention-recap}}

  \begin{itemize}
  \tightlist
  \item
    Joe Biden accepted the Democratic nomination, urging Americans to
    have faith that they could
    \href{https://www.nytimes3xbfgragh.onion/2020/08/20/us/politics/Joe-Biden-accepts-democratic-nomination.html?action=click\&pgtype=Article\&state=default\&region=BELOW_MAIN_CONTENT\&context=storylines_guide}{``overcome
    this season of darkness.''}
  \end{itemize}
\item
  \begin{center}\rule{0.5\linewidth}{\linethickness}\end{center}

  \hypertarget{news-analysis}{%
  \subsection{News Analysis}\label{news-analysis}}

  \begin{itemize}
  \tightlist
  \item
    Looming over Mr. Biden's nomination was the ever-present shadow of
    another man who's poised to dominate the campaign:
    \href{https://www.nytimes3xbfgragh.onion/2020/08/20/us/politics/biden-dnc-speech-trump.html?action=click\&pgtype=Article\&state=default\&region=BELOW_MAIN_CONTENT\&context=storylines_guide}{Donald
    J. Trump}.
  \end{itemize}
\item
  \begin{center}\rule{0.5\linewidth}{\linethickness}\end{center}

  \hypertarget{keep-up-with-our-coverage}{%
  \subsection{Keep Up With Our
  Coverage}\label{keep-up-with-our-coverage}}

  \begin{itemize}
  \tightlist
  \item
    Get an
    \href{https://www.nytimes3xbfgragh.onion/newsletters/politics?action=click\&pgtype=Article\&state=default\&region=BELOW_MAIN_CONTENT\&context=storylines_guide}{email}
    recapping the day's news
  \end{itemize}

  \begin{itemize}
  \tightlist
  \item
    Download our mobile app on
    \href{https://apps.apple.com/us/app/nytimes/id284862083?ls=1\&mat_click_id=5c79ae7455014fd1bd66b5610c05b8f2-20191112-16948\&referrer=mat_click_id\%3D5c79ae7455014fd1bd66b5610c05b8f2-20191112-16948\%26link_click_id\%3D722930677036718082}{iOS}
    and
    \href{http://a.localytics.com/android?id=com.nytimes.android\&referrer=utm_source\%3Dother_nyt_mobile_web\%26utm_medium\%3DWeb\%2520page\%26utm_term\%3DGeneral\%2520Mobile\%2520Page\%26utm_campaign\%3DNYT\%2520Mobile\%2520General\%2520Page}{Android}
    and turn on Breaking News and Politics alerts
  \end{itemize}
\end{itemize}

Advertisement

\protect\hyperlink{after-bottom}{Continue reading the main story}

\hypertarget{site-index}{%
\subsection{Site Index}\label{site-index}}

\hypertarget{site-information-navigation}{%
\subsection{Site Information
Navigation}\label{site-information-navigation}}

\begin{itemize}
\tightlist
\item
  \href{https://help.nytimes3xbfgragh.onion/hc/en-us/articles/115014792127-Copyright-notice}{©~2020~The
  New York Times Company}
\end{itemize}

\begin{itemize}
\tightlist
\item
  \href{https://www.nytco.com/}{NYTCo}
\item
  \href{https://help.nytimes3xbfgragh.onion/hc/en-us/articles/115015385887-Contact-Us}{Contact
  Us}
\item
  \href{https://www.nytco.com/careers/}{Work with us}
\item
  \href{https://nytmediakit.com/}{Advertise}
\item
  \href{http://www.tbrandstudio.com/}{T Brand Studio}
\item
  \href{https://www.nytimes3xbfgragh.onion/privacy/cookie-policy\#how-do-i-manage-trackers}{Your
  Ad Choices}
\item
  \href{https://www.nytimes3xbfgragh.onion/privacy}{Privacy}
\item
  \href{https://help.nytimes3xbfgragh.onion/hc/en-us/articles/115014893428-Terms-of-service}{Terms
  of Service}
\item
  \href{https://help.nytimes3xbfgragh.onion/hc/en-us/articles/115014893968-Terms-of-sale}{Terms
  of Sale}
\item
  \href{https://spiderbites.nytimes3xbfgragh.onion}{Site Map}
\item
  \href{https://help.nytimes3xbfgragh.onion/hc/en-us}{Help}
\item
  \href{https://www.nytimes3xbfgragh.onion/subscription?campaignId=37WXW}{Subscriptions}
\end{itemize}
