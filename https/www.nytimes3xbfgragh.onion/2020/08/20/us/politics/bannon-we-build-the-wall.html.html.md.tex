Sections

SEARCH

\protect\hyperlink{site-content}{Skip to
content}\protect\hyperlink{site-index}{Skip to site index}

\href{https://www.nytimes3xbfgragh.onion/section/politics}{Politics}

\href{https://myaccount.nytimes3xbfgragh.onion/auth/login?response_type=cookie\&client_id=vi}{}

\href{https://www.nytimes3xbfgragh.onion/section/todayspaper}{Today's
Paper}

\href{/section/politics}{Politics}\textbar{}How Bannon and His Indicted
Business Partners Cashed In on Trump

\url{https://nyti.ms/2YhxXRu}

\begin{itemize}
\item
\item
\item
\item
\item
\end{itemize}

Advertisement

\protect\hyperlink{after-top}{Continue reading the main story}

Supported by

\protect\hyperlink{after-sponsor}{Continue reading the main story}

\hypertarget{how-bannon-and-his-indicted-business-partners-cashed-in-on-trump}{%
\section{How Bannon and His Indicted Business Partners Cashed In on
Trump}\label{how-bannon-and-his-indicted-business-partners-cashed-in-on-trump}}

The men had a history of monetizing conservative causes long before they
were charged with fraud this week in a scheme to build a private border
wall.

\includegraphics{https://static01.graylady3jvrrxbe.onion/images/2020/08/20/us/politics/20foundation1/merlin_118512152_868bbc23-5118-4504-99c4-8ee28e981d57-articleLarge.jpg?quality=75\&auto=webp\&disable=upscale}

\href{https://www.nytimes3xbfgragh.onion/by/zolan-kanno-youngs}{\includegraphics{https://static01.graylady3jvrrxbe.onion/images/2019/12/13/reader-center/author-zolan-kanno-youngs/author-zolan-kanno-youngs-thumbLarge.png}}\href{https://www.nytimes3xbfgragh.onion/by/eric-lipton}{\includegraphics{https://static01.graylady3jvrrxbe.onion/images/2018/12/06/multimedia/author-eric-lipton/author-eric-lipton-thumbLarge.png}}\href{https://www.nytimes3xbfgragh.onion/by/stephanie-saul}{\includegraphics{https://static01.graylady3jvrrxbe.onion/images/2020/02/06/reader-center/author-stephanie-saul/author-stephanie-saul-thumbLarge.png}}\href{https://www.nytimes3xbfgragh.onion/by/scott-shane}{\includegraphics{https://static01.graylady3jvrrxbe.onion/images/2018/11/02/multimedia/author-scott-shane/author-scott-shane-thumbLarge.png}}

By \href{https://www.nytimes3xbfgragh.onion/by/zolan-kanno-youngs}{Zolan
Kanno-Youngs},
\href{https://www.nytimes3xbfgragh.onion/by/eric-lipton}{Eric Lipton},
\href{https://www.nytimes3xbfgragh.onion/by/stephanie-saul}{Stephanie
Saul} and \href{https://www.nytimes3xbfgragh.onion/by/scott-shane}{Scott
Shane}

\begin{itemize}
\item
  Aug. 20, 2020
\item
  \begin{itemize}
  \item
  \item
  \item
  \item
  \item
  \end{itemize}
\end{itemize}

They all had colorful histories, though not necessarily the kind to
inspire instant confidence. And they had a shared devotion to President
Trump.

Brian Kolfage, a decorated Iraq War veteran and motivational speaker,
had created a string of pro-Trump websites using bogus stories to draw
clicks and sell ads. Timothy Shea sold a Trump-themed energy drink he
marketed as containing ``liberal tears.'' Andrew Badolato had a trail of
failed businesses, unpaid tax bills and sexual misconduct allegations.

They came together early last year with a cherished project --- building
Mr. Trump's long-promised, long-stalled border wall --- and influential
supporters to promote it, including Stephen K. Bannon, an architect of
Mr. Trump's 2016 victory and a former White House strategist. Together,
the four men pitched We Build the Wall as a way for Americans worried
about border security to do an end run around Congress and construct at
least 100 miles of barriers with private money.

But according to a federal indictment unsealed on Thursday, the men
swindled donors, treating the more than \$25 million they raised as a
private piggy bank. Mr. Bannon, through an unnamed nonprofit
organization, received more than \$1 million from the group, according
to the court documents. Mr. Kolfage got a total of \$350,000 that he
spent on ``home renovations, payments toward a boat, a luxury S.U.V., a
golf cart, jewelry, cosmetic surgery, personal tax payments and credit
card debt,'' the indictment claimed. The others collected hundreds of
thousands of dollars for personal expenses, prosecutors said. Mr. Bannon
pleaded not guilty in the case.

Mr. Kolfage, the public face of the effort, had repeatedly assured
donors that 100 percent of the money would go toward construction.
``It's hypocritical and ironic,'' said Javier Perea, the mayor of
Sunland Park, N.M., where work on the private wall began. ``These were
individuals that were selling the idea of enforcing U.S. laws and
enforcing our rules but were the ones who misled the American public on
their intentions, if the allegations are true.''

\includegraphics{https://static01.graylady3jvrrxbe.onion/images/2020/08/20/us/politics/20dc-borderwall1/merlin_155682981_c3546583-a956-40d0-b3a7-2a8e1878e991-articleLarge.jpg?quality=75\&auto=webp\&disable=upscale}

Today, just less than five miles of the private wall have been built,
according to the company's website. After
\href{https://www.propublica.org/article/he-built-a-privately-funded-border-wall-its-already-at-risk-of-falling-down-if-not-fixed}{experts
recently said} that the group's decision to build along the banks of the
Rio Grande in South Texas could cause the barriers to fall into the
river, even Mr. Trump criticized the project.

But the effort earlier drew praise from Homeland Security and Border
Patrol officials, as well as
\href{https://twitter.com/KFILE/status/1296470804058832899?s=20}{the
president's son Donald Jr.} A cast of Trump supporters worked on the
board of We Build the Wall, including Kris Kobach, the former secretary
of state for Kansas; Erik Prince, the defense contractor; Tom Tancredo,
a former Colorado congressman; as well as the frequent Trump defenders
David A. Clarke Jr., a former Wisconsin sheriff, and the former Red Sox
pitcher Curt Schilling.

The project was going forward as recently as Wednesday, when Mr. Kolfage
appeared on Mr. Bannon's
\href{https://pandemic.warroom.org/2020/08/19/ep-342-pandemic-the-unraveling-of-america-w-brian-kolfage-brandon-judd-the-embed-and-wade-davis/}{podcast},
``War Room,'' to tout a fund-raising effort for another cause ---
however vague --- close to the president's rhetoric: a fund ``for the
victims of Black Lives Matter.''

The We Build the Wall operators had tied themselves and their fortunes
to the Trump era and the president's brand, scorning Democrats and the
federal bureaucracy. Mr. Kolfage, 38, who had lost both legs and one arm
in Iraq while serving in the Air Force, has posted multiple
\href{https://twitter.com/BrianKolfage/status/1167114949476388865}{videos}
on Twitter mocking former President Barack Obama's border control
efforts and praising Mr. Trump's.

Mr. Bannon, 66, and Mr. Badolato, 56, a financier from Sarasota, Fla.,
had been business partners since at least 2003, when they joined to
create a number of new ventures, including a nasal spray company called
SinoFresh Healthcare.

But like many business start-ups that Mr. Badolato was involved in, this
one became mired in controversy, as a dispute broke out among executives
over
\href{https://www.sec.gov/Archives/edgar/data/1171596/000095014405003324/g94120e10ksb.htm}{accusations}of
illegal trading of company stock and of using corporate funds for
personal gain.

Mr. Shea, 49, of Castle Rock, Colo., and his wife, Amanda, had built a
growing national profile for themselves using social media to promote
their business ventures. One of those was a company called Winning
Energy, which frequently used images of Mr. Trump to drive sales on
products such as a can of what it described as
\href{https://winning-energy.com/}{12 ounces of ``liberal tears}.'' The
brand sponsored a Trump boat parade in Florida this July.

Image

A Facebook ad for an energy drink company targeted at conservatives.

Mr. Kolfage, who is from Miramar Beach, Fla., and attended a
fund-raising event at Mar-a-Lago last year with the president's son
Eric, spent years operating Facebook pages and websites that sometimes
trafficked in false or exaggerated news stories, many of them pro-Trump.

In an interview with The New York Times in April 2018, he spoke about
those efforts, part of a long-established online industry in which
Facebook pages direct users to the websites that make money by selling
ads.

He admitted spreading fake stories --- ``going at it like in the Wild
West'' --- but blamed his lack of journalism training. ``You'd try to be
as factual as possible while injecting opinion,'' he said. ``There were
a lot of people using sources that were just wrong. That's how it
started going south.''

Mr. Kolfage became business partners with Ms. Shea after she built her
own social-media-based marketing company that Mr. Kolfage used to
increase traffic and profit at one of his websites, Right Wing News.
When Facebook
\href{https://www.facebookcorewwwi.onion/Brian.Kolfage.jr/posts/facebook-lied-they-shut-down-my-page-because-it-was-conservative-powerful-and-th/2211141405766130/}{shut
down that group} in 2018 --- shortly before the congressional elections
--- Mr. Kolfage and the Sheas started a new fund-raising group they
called
\href{https://www.facebookcorewwwi.onion/Fight4Speech/}{Fight4FreeSpeech},
which
\href{https://int.graylady3jvrrxbe.onion/data/documenttools/2018-10-don-trump-jr/2dafd9fc6596ae66/full.pdf}{attracted
support}from nationally known conservatives including Donald Trump Jr.,
and often
featured\href{https://www.facebookcorewwwi.onion/Fight4Speech/videos/734904533974973/}{video
clips} of Mr. Bannon.

Mr. Kolfage suggested that the Trump campaign and presidency were a
critical factor in his business success. ``Connecting on Facebook made
people fell like they were part of something,'' he said. ``Having a
president who's not part of the same old political spectrum really
united people.''

His social media endeavors led to the birth in December 2018 of We Build
the Wall, even as Mr. Trump struggled to secure money from Congress.

Mr. Kolfage repeatedly promised that none of the money would be used to
pay executives involved in the fund-raising efforts. But after it became
clear that there was no mechanism to transfer the money to the
government, Mr. Kolfage said the group would become a private foundation
and build its own wall. It had determined that only \$800,000 of the
approximately \$20 million raised at that point had to be returned to
donors.

``No rules were broken,'' Mr. Kolfage said in an interview last year.
``Ninety-four percent of the donors we have been able to reach are
opting in.''

The private wall did get started, but backlash came almost as soon as
workers began digging into Sunland Park. Mr. Perea said the group failed
to obtain the necessary permits, forcing him to order a temporary halt
on the work.

Mr. Kolfage responded by tweeting that Mr. Perea supported ``open
borders, the sex slaves and illegal drugs coming into their
communities,'' prompting thousands of complaints to Mr. Perea's inbox.

``They were deliberately withholding information,'' said Mr. Perea, who
said county appraisers had valued the privately built barriers at nearly
\$4 million.

The group also built along the riverbank in South Texas, where the
construction of the Trump administration's border wall had been held up
by lawsuits filed by landowners protesting how it would cut through
their property.

Image

Brian Kolfage, left, an Air Force veteran from Florida, started a
private campaign to raise money for President Trump's border wall in
late 2018. Credit...Mark Lambie/The El Paso Times, via Associated Press

Mr. Bannon began to play an increasing role in the organization, the
indictment said, ``including its finances, messaging, donor outreach and
general operations.'' For example, he held
\href{https://www.youtube.com/watch?v=RvhZO5fNr-8}{promotional events}
near El Paso, Texas, to highlight progress on construction and continue
raising funds.

The group hired a North Dakota company, Fisher Sand \& Gravel, to build
the barriers and publicly celebrated its work. The Trump administration
then granted the North Dakota company a \$400 million contract for the
government-built wall, a deal that is currently being investigated by
the inspector general for the Defense Department.

While Mr. Trump distanced himself from the project on Thursday, Mr.
Kobach, the former Kansas secretary of state, told The Times last year
that the group had received the president's blessing.

Robert S. Spalding III, a retired brigadier general who served on Mr.
Trump's National Security Council, said he had agreed to join the board
of directors at the request of Mr. Bannon, whom he knew from the White
House, and had attended a symposium the organization held at the site of
one of its border wall segments.

``All I can say is, it's sad. I did it, to be honest, because I
respected Brian because he had fought for the country and as a fellow
airman I wanted to support him,'' General Spalding said of Mr. Kolfage.
``They just reached out to me and asked if I would do it.'' He added
that he had not been involved in the group's financial operations and
had resigned.

Donors, however, wondered what had happened to the money.

The Florida Department of Agriculture and Consumer Services, which
oversees nonprofit groups in the state, announced last year that it had
opened an investigation of We Build the Wall after complaints from three
consumers.

One of the
\href{https://int.graylady3jvrrxbe.onion/data/documenttools/brian-kolfage-attorney-general-complaints/df78880e5adbc9fb/full.pdf}{complaints,}
from Minnesota Assistant Attorney General Wendy Tien, who said she was
writing in her personal capacity, raised concerns that We Build the Wall
had misled donors in its efforts to raise funds. **** She questioned
whether the group had obtained nonprofit status.

Another complaint came from Harvey Garlotte, a Hattiesburg, Miss., man
who had donated \$60. Mr. Garlotte said he ``felt duped'' because Mr.
Kolfage --- who had originally said he would donate the money to the
border wall --- was redirecting it to a nonprofit organization he
controlled.

``From my side of the road, Mr. Kolfage was simply using a hot-button
topic, a very emotional topic, and his status as a wounded veteran, for
selfish and self-serving reasons and personal financial gain,'' Mr.
Garlotte wrote to The Times.

Mr. Kolfage at the time mocked such concerns. ``This is hilarious,''
\href{https://twitter.com/BrianKolfage/status/1158957245612417029?ref_src=twsrc\%5Etfw\%7Ctwcamp\%5Etweetembed\%7Ctwterm\%5E1158957245612417029\%7Ctwgr\%5E\&ref_url=https\%3A\%2F\%2Fabcnews.go.com\%2FUS\%2Fbuild-wall-group-privately-funded-border-wall-criminal\%2Fstory\%3Fid\%3D64827607}{he
wrote}, retweeting official copies of the complaints.

In a statement on Thursday, the Florida Department of Agriculture and
Consumer Services said that after opening an investigation in May 2019,
it had referred the case to the F.B.I. It was not clear whether the
referral had led to the arrests on Thursday.

After federal and state investigators began to examine the nonprofit
group --- which was created so recently it has not filed even its tax
return detailing how it spent its money --- they began to uncover
evidence that large sums were being transferred to a separate nonprofit
set up by Mr. Shea.

``They did so by using fake invoices and sham vendor arrangements, among
other ways,'' the
\href{https://int.graylady3jvrrxbe.onion/data/documenttools/u-s-v-brian-kolfage-stephen-bannon-et-al/e56f197b430d0fcb/full.pdf}{indictment
says}.

Mr. Kolfage, in a text message recovered by investigators, reminded Mr.
Badolato that his pay, which ultimately totaled more than \$350,000,
needed to remain confidential and on a ``need to know basis,'' the
indictment says.

When they learned in October that they had been targeted by
investigators, Mr. Kolfage and Mr. Badolato began to use encrypted
messaging applications on their phones, and the We Build the Wall
website was revised to remove any reference to Mr. Kolfage's not being
compensated, the indictment says. As of January, it said he would be
paid a salary.

Kitty Bennett contributed research.

Advertisement

\protect\hyperlink{after-bottom}{Continue reading the main story}

\hypertarget{site-index}{%
\subsection{Site Index}\label{site-index}}

\hypertarget{site-information-navigation}{%
\subsection{Site Information
Navigation}\label{site-information-navigation}}

\begin{itemize}
\tightlist
\item
  \href{https://help.nytimes3xbfgragh.onion/hc/en-us/articles/115014792127-Copyright-notice}{©~2020~The
  New York Times Company}
\end{itemize}

\begin{itemize}
\tightlist
\item
  \href{https://www.nytco.com/}{NYTCo}
\item
  \href{https://help.nytimes3xbfgragh.onion/hc/en-us/articles/115015385887-Contact-Us}{Contact
  Us}
\item
  \href{https://www.nytco.com/careers/}{Work with us}
\item
  \href{https://nytmediakit.com/}{Advertise}
\item
  \href{http://www.tbrandstudio.com/}{T Brand Studio}
\item
  \href{https://www.nytimes3xbfgragh.onion/privacy/cookie-policy\#how-do-i-manage-trackers}{Your
  Ad Choices}
\item
  \href{https://www.nytimes3xbfgragh.onion/privacy}{Privacy}
\item
  \href{https://help.nytimes3xbfgragh.onion/hc/en-us/articles/115014893428-Terms-of-service}{Terms
  of Service}
\item
  \href{https://help.nytimes3xbfgragh.onion/hc/en-us/articles/115014893968-Terms-of-sale}{Terms
  of Sale}
\item
  \href{https://spiderbites.nytimes3xbfgragh.onion}{Site Map}
\item
  \href{https://help.nytimes3xbfgragh.onion/hc/en-us}{Help}
\item
  \href{https://www.nytimes3xbfgragh.onion/subscription?campaignId=37WXW}{Subscriptions}
\end{itemize}
