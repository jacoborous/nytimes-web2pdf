Sections

SEARCH

\protect\hyperlink{site-content}{Skip to
content}\protect\hyperlink{site-index}{Skip to site index}

\href{https://www.nytimes3xbfgragh.onion/section/world/asia}{Asia
Pacific}

\href{https://myaccount.nytimes3xbfgragh.onion/auth/login?response_type=cookie\&client_id=vi}{}

\href{https://www.nytimes3xbfgragh.onion/section/todayspaper}{Today's
Paper}

\href{/section/world/asia}{Asia Pacific}\textbar{}Thailand Police Arrest
Activists, Escalating Protest Crackdown

\url{https://nyti.ms/3hgDb7x}

\begin{itemize}
\item
\item
\item
\item
\item
\end{itemize}

Advertisement

\protect\hyperlink{after-top}{Continue reading the main story}

Supported by

\protect\hyperlink{after-sponsor}{Continue reading the main story}

\hypertarget{thailand-police-arrest-activists-escalating-protest-crackdown}{%
\section{Thailand Police Arrest Activists, Escalating Protest
Crackdown}\label{thailand-police-arrest-activists-escalating-protest-crackdown}}

Rappers, a lawyer and other government critics have been accused of
sedition, a crime that can carry a 7-year prison sentence.

\includegraphics{https://static01.graylady3jvrrxbe.onion/images/2020/08/20/world/20thai-arrests-1/merlin_175927605_b967f8e9-f2c9-4eaf-bd93-b14006a30320-articleLarge.jpg?quality=75\&auto=webp\&disable=upscale}

\href{https://www.nytimes3xbfgragh.onion/by/hannah-beech}{\includegraphics{https://static01.graylady3jvrrxbe.onion/images/2018/10/08/multimedia/author-hannah-beech/author-hannah-beech-thumbLarge.png}}

By \href{https://www.nytimes3xbfgragh.onion/by/hannah-beech}{Hannah
Beech}

\begin{itemize}
\item
  Aug. 20, 2020
\item
  \begin{itemize}
  \item
  \item
  \item
  \item
  \item
  \end{itemize}
\end{itemize}

BANGKOK --- His offense was syncopated. And it rhymed.

Dechathorn Bamrungmuang, a member of the Thai collective
\href{https://www.nytimes3xbfgragh.onion/2019/02/15/world/asia/rap-video-thailand-liberate-p.html}{Rap
Against Dictatorship}, was arrested on Thursday on charges of sedition,
human rights lawyers said, part of a mounting crackdown by a government
seemingly allergic to dissent.

A day earlier, the authorities took into custody for the second time a
lawyer who had publicly called for the
\href{https://www.nytimes3xbfgragh.onion/2020/08/13/world/asia/protests-thailand-king-monarchy.html}{Thai
monarchy's powers to be reined in}. At least six pro-democracy activists
were also arrested on Wednesday and Thursday on charges of sedition, a
crime that can carry a seven-year prison sentence, the organization Thai
Lawyers for Human Rights said. Another rapper was taken in, too. Yet
more student activists were served with papers that appeared to indicate
they could be imminently detained.

``The United Nations and concerned governments should speak out publicly
against the rolling political repression in Thailand,'' Brad Adams, Asia
director at Human Rights Watch, said in a statement. ``Thai youth are
increasingly demanding real progress toward democracy and the rule of
law so they can freely express their visions for the future of the
country.''

The legal actions followed weeks of protests by students that culminated
on Sunday in the
\href{https://www.nytimes3xbfgragh.onion/2020/08/16/world/asia/thailand-protests-democracy-monarchy.html}{largest
street rally} in Thailand since a military coup six years ago. Because
of a state of emergency imposed to control the coronavirus, the protests
were technically illegal. But that did not stop more than 10,000 people
from gathering at Democracy Monument in Bangkok.

\includegraphics{https://static01.graylady3jvrrxbe.onion/images/2020/08/20/world/20thai-arrests-2/merlin_175785057_116e0c3a-a866-4d4c-9ef5-5d1d2059ccca-articleLarge.jpg?quality=75\&auto=webp\&disable=upscale}

On Wednesday, hundreds of students, who have been protesting school
rules like
\href{https://www.nytimes3xbfgragh.onion/2020/08/11/world/asia/thailand-student-protest-military.html}{mandatory
haircuts} and the tradition of prostrating themselves to teachers,
gathered at the Education Ministry in Bangkok, calling for an end to the
regimented strictures of Thai society. They raised three fingers in the
air, a symbol of defiance drawn from the ``Hunger Games'' films that was
once banned by the junta behind the 2014 coup.

Thailand has held elections over the decades, but meddling by the
military has regularly upended the people's vote, with a dozen
successful coups since absolute monarchy was abolished in 1932. A former
army chief who led the last coup, Prayuth Chan-ocha, remains prime
minister. Three other retired generals hold cabinet positions.

Mr. Prayuth has said that he is willing to listen to the students. He
has also joked that he would like to execute journalists who veer from
the truth and assailed those who have called for more accountability
over the monarchy.

The junta leaders said protecting the palace was a key reason for their
putsch, and the current government, which took power last year after
elections that independent observers said were flawed, often wraps its
policies in a royal mantle.

\href{https://www.nytimes3xbfgragh.onion/2019/05/04/world/asia/thailand-king-maha-vajiralongkorn.html}{King
Maha Vajiralongkorn Bodindradebayavarangkun} is present in Thailand for
only days at a time, living most of the year in Europe. Since his father
died in 2016, he has consolidated his authority, taking control of royal
coffers and army units that traditionally add their firepower to coups.

Image

A photograph released by Thailand's Bureau of the Royal Household,
showing Prime Minister Prayuth Chan-ocha, center, and other officials
taking their oath in front of the king and queen on Aug.
12.Credit...Bureau of the Royal Household, via Associated Press

Thailand is bound by strict laws that criminalize criticism of the
monarchy, potentially landing offenders in jail for up to 15 years. In
addition to the lèse-majesté laws, the government has also imprisoned
people for contravening sedition and computer crimes laws. Hundreds of
people have been funneled through so-called attitude adjustment camps
that are run out of military compounds.
\href{https://www.nytimes3xbfgragh.onion/2020/06/26/world/asia/thailand-dissidents-disappeared-military.html}{Dissidents
have disappeared}, too, with some of their bodies turning up mutilated.

The transgression of Mr. Dechathorn, who goes by the rap name
Hockhacker, appears to have been a musical performance at a
pro-democracy rally, according to the group of rights lawyers.

In 2018, Rap Against Dictatorship, the Thai musical collective, drew
millions of hits for a widely shared video for a song called
``\href{https://www.youtube.com/watch?v=VZvzvLiGUtw}{What My Country Has
Got}.'' The song referenced a student massacre and called Mr. Prayuth's
government to account for leading a ``country that makes fake promises
like loading bullets, creates a regime and orders us to love it.''

Another song, released this year, took on student hazing. The collective
was awarded the Vaclav Havel Prize for Creative Dissent in 2019.

Sirin Mungcharoen, a pro-democracy activist from Chulalongkorn
University in Bangkok, said she received a police summons on Wednesday
related to a protest in Bangkok on July 18. Ms. Sirin said that she had
attended that student rally, but was not an organizer.

``It's a way to create fear,'' Ms. Sirin said, adding, ``I will keep
fighting.''

Muktita Suhartono contributed reporting from Phuket, Thailand.

Advertisement

\protect\hyperlink{after-bottom}{Continue reading the main story}

\hypertarget{site-index}{%
\subsection{Site Index}\label{site-index}}

\hypertarget{site-information-navigation}{%
\subsection{Site Information
Navigation}\label{site-information-navigation}}

\begin{itemize}
\tightlist
\item
  \href{https://help.nytimes3xbfgragh.onion/hc/en-us/articles/115014792127-Copyright-notice}{©~2020~The
  New York Times Company}
\end{itemize}

\begin{itemize}
\tightlist
\item
  \href{https://www.nytco.com/}{NYTCo}
\item
  \href{https://help.nytimes3xbfgragh.onion/hc/en-us/articles/115015385887-Contact-Us}{Contact
  Us}
\item
  \href{https://www.nytco.com/careers/}{Work with us}
\item
  \href{https://nytmediakit.com/}{Advertise}
\item
  \href{http://www.tbrandstudio.com/}{T Brand Studio}
\item
  \href{https://www.nytimes3xbfgragh.onion/privacy/cookie-policy\#how-do-i-manage-trackers}{Your
  Ad Choices}
\item
  \href{https://www.nytimes3xbfgragh.onion/privacy}{Privacy}
\item
  \href{https://help.nytimes3xbfgragh.onion/hc/en-us/articles/115014893428-Terms-of-service}{Terms
  of Service}
\item
  \href{https://help.nytimes3xbfgragh.onion/hc/en-us/articles/115014893968-Terms-of-sale}{Terms
  of Sale}
\item
  \href{https://spiderbites.nytimes3xbfgragh.onion}{Site Map}
\item
  \href{https://help.nytimes3xbfgragh.onion/hc/en-us}{Help}
\item
  \href{https://www.nytimes3xbfgragh.onion/subscription?campaignId=37WXW}{Subscriptions}
\end{itemize}
