Sections

SEARCH

\protect\hyperlink{site-content}{Skip to
content}\protect\hyperlink{site-index}{Skip to site index}

\href{https://myaccount.nytimes3xbfgragh.onion/auth/login?response_type=cookie\&client_id=vi}{}

\href{https://www.nytimes3xbfgragh.onion/section/todayspaper}{Today's
Paper}

A Home Inscribed With the History of Venice Beach

\url{https://nyti.ms/2BSpdcC}

\begin{itemize}
\item
\item
\item
\item
\item
\end{itemize}

Advertisement

\protect\hyperlink{after-top}{Continue reading the main story}

Supported by

\protect\hyperlink{after-sponsor}{Continue reading the main story}

\hypertarget{a-home-inscribed-with-the-history-of-venice-beach}{%
\section{A Home Inscribed With the History of Venice
Beach}\label{a-home-inscribed-with-the-history-of-venice-beach}}

Within a storied Los Angeles studio complex, Karina Deyko and David de
Rothschild have built an eclectic, freewheeling space in which to live
and work.

\includegraphics{https://static01.graylady3jvrrxbe.onion/images/2020/08/03/t-magazine/design/3tmag-rothschild-slide-2WRX/3tmag-rothschild-slide-2WRX-articleLarge.jpg?quality=75\&auto=webp\&disable=upscale}

By
\href{https://www.nytimes3xbfgragh.onion/by/alice-newell-hanson}{Alice
Newell-Hanson}

\begin{itemize}
\item
  Aug. 3, 2020
\item
  \begin{itemize}
  \item
  \item
  \item
  \item
  \item
  \end{itemize}
\end{itemize}

While the British environmentalist
\href{https://tmagazine.blogs.nytimes3xbfgragh.onion/2010/03/24/profile-in-style-david-de-rothschild/}{David
de Rothschild} was at sea in the spring of 2010, crossing the Pacific
Ocean on the \href{https://theplastiki.com/}{Plastiki} ---~a 60-foot
catamaran built primarily from recycled bottles ---~to raise awareness
about the climate crisis, the actress Karina Deyko, his wife, made her
own voyage of discovery. She had recently met, through a mutual friend
in Los Angeles, the actress Kelly Reilly and the photographer Guy
Webster, who helped establish the tenets of rock 'n' roll portraiture in
the 1960s with his images of Jim Morrison and the Rolling Stones. Reilly
had been renting a small apartment within Webster's sprawling studio
complex in Venice Beach but would soon be traveling to London to make a
film. Webster wondered if Deyko wanted to come see the space and
possibly sublet it. At the time, she was living in Echo Park, on the
east side of the city, and ``I knew that when David was done with the
Plastiki he'd want to be by the beach,'' Deyko says. So she took Webster
up on his offer.

\includegraphics{https://static01.graylady3jvrrxbe.onion/images/2020/08/03/t-magazine/design/3tmag-rothschild-slide-5A2Z/3tmag-rothschild-slide-5A2Z-articleLarge.jpg?quality=75\&auto=webp\&disable=upscale}

Webster's building, a 3,000-square-foot compound set within an
early-20th-century industrial depot just four blocks from the ocean, and
divided into a handful of distinct studio spaces, had served as a shed
for storing boats in the 1910s. Although Deyko didn't learn this history
until later, she immediately sensed it would be a good place for the
couple, who travel widely and often, to drop anchor for a while. She was
drawn, foremost, to the sense of creativity that seemed to emanate from
the gently weathered structure itself, with its worn concrete floors and
high wood-beam ceilings, and from the intriguing people who drifted in
and out of the space. As Venice evolved from a resort town in the 1910s
to a short-lived hub for oil production in the 1930s to a waning
industrial and entertainment district over the following decades, when
its vacant warehouses were repurposed by artists and designers ---
\href{https://www.nytimes3xbfgragh.onion/2020/05/15/arts/ray-charles-eames-artists.html}{Ray
and Charles Eames} established their practice on Abbot Kinney Boulevard
in 1943, and the architects Thom Mayne and Frank Gehry lived and made
work in the neighborhood in the '70s and '80s ---~so too did the
building transform, from a storehouse to a mechanic's shop to a crash
pad for Webster's circle of artist and musician friends. And the
building's past lives are still preserved in its architecture: Low-slung
and partially wrapped in sheets of faded sky-blue corrugated iron, from
the street it could easily be mistaken for a garage (there is even a
peeling Texaco logo painted on the facade). ``Energetically,'' says
Deyko, ``I felt it. It was just an amazing space.'' She sublet the
studio, Reilly never returned to live in Los Angeles (she met her future
husband during that film shoot in England) and Deyko and de Rothschild
have now lived part-time in the building for 10 years.

Image

Working with one of the builders of the Plastiki, de Rothschild's
catamaran made from recycled materials, Deyko constructed a lofted glass
meeting room with a steel staircase in the home's office
area.Credit...Kate Martin

Image

Propped against a wall in the meeting room is a six-foot-tall neon
plastic cutout of a cat that de Rothschild acquired as a gift for Deyko
from the fashion brand Stella McCartney, which had used it in a window
display.Credit...Kate Martin

Image

Deyko built several of the home's kitchen cabinets from parts of a
reclaimed barn door, sourced from E\&K Vintage Wood in Los
Angeles.Credit...Kate Martin

Image

Deyko, by the southern entrance to the studio complex. The building's
facade looks much the same as it did when the space was used as a repair
shop.Credit...Kate Martin

During that decade, the couple purchased the studio and also acquired
two of the neighboring units within the complex when friends moved out.
Today, they inhabit a warren of interconnected spaces that hug a central
paved courtyard and together comprise not only living quarters but also
an office for \href{https://thelostexplorer.com/}{the Lost Explorer},
the environmentally conscious clothing and travel company that de
Rothschild founded in 2015. When they moved in, it was the first home
the pair had shared together and ``we've made our own history here,''
says Deyko. The interiors, which are featured in the designer, store
owner and T contributor Alex Eagle's new book
``\href{https://www.rizzoliusa.com/book/9780847867714}{More Than Just a
House}'' (out in October from Rizzoli), have evolved with the couple,
developing not according to any conscious plan but as a scrapbook might,
being added to as the pair **** acquire souvenirs from their travels and
source and hand-build furniture to suit their changing needs. They favor
objects that, in keeping with their home, bear the marks of unusual
histories. On the white wall above the simple poured-concrete counter of
the kitchen off the main living area, Deyko has pinned baglike woven
jute fishing nets she bought on a trip to Japan. And nearly every accent
or piece of furniture that they didn't make themselves --- ``Karina is
the kind of person who will discover some amazing vintage Japanese
indigo fabric and turn it into bean bags,'' says Eagle --- the couple
found at a flea market or secondhand store, and chose for its faded
upholstery, chipping paint or time-warped wood. The idea behind Eagle's
book, which offers a look into some of her friends' living spaces, is
``to document homes through their objects, the things that people
collect and that make them tick,'' says Eagle. But while many of its
subjects acquire art and design pieces --- whether midcentury Italian
lamps or Nike sneakers --- with the doggedness of a true obsessive,
Deyko and de Rothschild's accumulation of stuff appears more Zenlike, as
if their desire is not to own their possessions but simply to appreciate
them, add to their stories and then pass them on.

Image

In the dining area, a handblown glass Neverending Glory La Scala pendant
light by the Prague-based designers Jan Plecháč and Henry Wielgus hangs
above a vintage wooden table discovered at the Los Angeles antiques
showroom Galerie Half.Credit...Kate Martin

Image

A photograph by the New York-based artist Oliver Clegg, a friend of the
couple's, occupies the wall above their record collection.Credit...Kate
Martin

Image

Deyko and de Rothschild have hosted dinner parties at this custom-made
reclaimed-wood table, framed by bougainvillea in an alley off the main
courtyard.Credit...Kate Martin

Image

A wooden stool found at the antiques store Bazar in Venice sits in front
of an original fireplace in the living space.Credit...Kate Martin

And they inhabit the space with a similarly laid-back spirit. Before the
pandemic hit, guests and collaborators would come and go throughout the
day and occasionally crash on a sofa. In the evenings, the couple would
host lively dinner parties and movie nights. And there is a rusted 1950s
candy machine in the kitchen by the living room that their friends know
is always filled with snacks for the taking if they're nearby and
hungry. ``You never know how your day's going to go or who's going to
turn up at the door,'' says Deyko. While this place is their home for
now, they are aware, too, that they are merely among the stewards of the
building, one whose history will extend beyond them. Though Webster left
Venice in 2014, after suffering a stroke, ``the energy here was always
driven by Guy,'' says de Rothschild, ``It was his photography studio. It
was his place to be creative, and I think it emanated out of that.''
They have thought recently, as they enter their 10th year here, about
whether it might be time ``for someone else to come in and put their
imprint on it,'' Deyko says. ``But I think it will happen organically,''
she adds, as if the couple might simply drift out the door one day and
never come back, allowing a new chapter to begin.

Advertisement

\protect\hyperlink{after-bottom}{Continue reading the main story}

\hypertarget{site-index}{%
\subsection{Site Index}\label{site-index}}

\hypertarget{site-information-navigation}{%
\subsection{Site Information
Navigation}\label{site-information-navigation}}

\begin{itemize}
\tightlist
\item
  \href{https://help.nytimes3xbfgragh.onion/hc/en-us/articles/115014792127-Copyright-notice}{©~2020~The
  New York Times Company}
\end{itemize}

\begin{itemize}
\tightlist
\item
  \href{https://www.nytco.com/}{NYTCo}
\item
  \href{https://help.nytimes3xbfgragh.onion/hc/en-us/articles/115015385887-Contact-Us}{Contact
  Us}
\item
  \href{https://www.nytco.com/careers/}{Work with us}
\item
  \href{https://nytmediakit.com/}{Advertise}
\item
  \href{http://www.tbrandstudio.com/}{T Brand Studio}
\item
  \href{https://www.nytimes3xbfgragh.onion/privacy/cookie-policy\#how-do-i-manage-trackers}{Your
  Ad Choices}
\item
  \href{https://www.nytimes3xbfgragh.onion/privacy}{Privacy}
\item
  \href{https://help.nytimes3xbfgragh.onion/hc/en-us/articles/115014893428-Terms-of-service}{Terms
  of Service}
\item
  \href{https://help.nytimes3xbfgragh.onion/hc/en-us/articles/115014893968-Terms-of-sale}{Terms
  of Sale}
\item
  \href{https://spiderbites.nytimes3xbfgragh.onion}{Site Map}
\item
  \href{https://help.nytimes3xbfgragh.onion/hc/en-us}{Help}
\item
  \href{https://www.nytimes3xbfgragh.onion/subscription?campaignId=37WXW}{Subscriptions}
\end{itemize}
