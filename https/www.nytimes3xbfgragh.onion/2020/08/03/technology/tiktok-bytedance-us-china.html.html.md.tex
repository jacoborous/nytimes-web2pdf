Sections

SEARCH

\protect\hyperlink{site-content}{Skip to
content}\protect\hyperlink{site-index}{Skip to site index}

\href{https://www.nytimes3xbfgragh.onion/section/technology}{Technology}

\href{https://myaccount.nytimes3xbfgragh.onion/auth/login?response_type=cookie\&client_id=vi}{}

\href{https://www.nytimes3xbfgragh.onion/section/todayspaper}{Today's
Paper}

\href{/section/technology}{Technology}\textbar{}How TikTok's Owner
Tried, and Failed, to Cross the U.S.-China Divide

\url{https://nyti.ms/3fo8tb8}

\begin{itemize}
\item
\item
\item
\item
\item
\item
\end{itemize}

Advertisement

\protect\hyperlink{after-top}{Continue reading the main story}

Supported by

\protect\hyperlink{after-sponsor}{Continue reading the main story}

\hypertarget{how-tiktoks-owner-tried-and-failed-to-cross-the-us-china-divide}{%
\section{How TikTok's Owner Tried, and Failed, to Cross the U.S.-China
Divide}\label{how-tiktoks-owner-tried-and-failed-to-cross-the-us-china-divide}}

The founder of ByteDance, Zhang Yiming, dreamed of building a global
tech company based in China. Then the geopolitical reality set in.

\includegraphics{https://static01.graylady3jvrrxbe.onion/images/2020/08/03/world/03china-tiktok1/merlin_175185381_c1318e9d-923c-4ed2-a2a5-d3e4ba3556fe-articleLarge.jpg?quality=75\&auto=webp\&disable=upscale}

\href{https://www.nytimes3xbfgragh.onion/by/raymond-zhong}{\includegraphics{https://static01.graylady3jvrrxbe.onion/images/2018/10/15/multimedia/author-raymond-zhong/author-raymond-zhong-thumbLarge.png}}

By \href{https://www.nytimes3xbfgragh.onion/by/raymond-zhong}{Raymond
Zhong}

\begin{itemize}
\item
  Aug. 3, 2020
\item
  \begin{itemize}
  \item
  \item
  \item
  \item
  \item
  \item
  \end{itemize}
\end{itemize}

\href{https://cn.nytimes3xbfgragh.onion/technology/20200804/tiktok-trump-sale-microsoft/}{阅读简体中文版}\href{https://cn.nytimes3xbfgragh.onion/technology/20200804/tiktok-trump-sale-microsoft/zh-hant/}{閱讀繁體中文版}

The Chinese entrepreneur behind TikTok took ample precautions when he
set out to straddle the tech world's most treacherous divide: the one
separating China's tightly controlled internet from the rest of the
planet.

He made TikTok unavailable in China so the video app's users wouldn't be
subject to the Communist Party's censorship requirements. He stored user
data in Virginia and Singapore. He hired
\href{https://www.nytimes3xbfgragh.onion/2020/05/18/business/media/tiktok-ceo-kevin-mayer.html}{managers
in the United States} to run the app and
\href{https://www.nytimes3xbfgragh.onion/2020/07/15/technology/tiktok-washington-lobbyist.html}{lobbyists
in Washington} to fight for it on Capitol Hill.

None of that counted for much in the end. With TikTok now negotiating
\href{https://www.nytimes3xbfgragh.onion/2020/08/02/business/economy/trump-tiktok-china-national-security.html}{a
sale to Microsoft} under intense pressure from President Trump, who said
on Monday that he was
\href{https://www.nytimes3xbfgragh.onion/2020/08/03/technology/trump-tiktok-microsoft.html?action=click\&module=Top\%20Stories\&pgtype=Homepage}{giving
the go-ahead} to such a deal, the digital wall between China and the
United States is proving to be higher than ever at this moment of
widening conflict between the two countries.

Only this time, it is the U.S. government, not China's, that is putting
up the barricades --- an escalation that could foretell an even more
restrictive time for companies in both nations.

ByteDance, the eight-year-old Chinese social media giant behind TikTok,
is China's first truly global internet success story. The company's
founder, Zhang Yiming, 37, began pushing to expand overseas early on,
believing that only a company with worldwide reach could remain on the
technological edge.

But TikTok ended up resonating with American teenagers when even a
platform for short viral videos is subject to political scrutiny. Under
China's leader, Xi Jinping, the Communist Party has emphasized
\href{https://www.nytimes3xbfgragh.onion/2018/12/18/world/asia/xi-jinping-speech-china.html}{its
ultimate authority} over Chinese people and businesses. Suspicion never
dissipated that TikTok --- no matter how many non-Chinese executives it
put in charge --- might be unable to withstand pressure from Beijing to
surrender user data or manipulate content.

Similar doubts already hang over many other Chinese tech companies.
TikTok's sudden change of fortune could force them to re-evaluate their
own international ambitions.

\includegraphics{https://static01.graylady3jvrrxbe.onion/images/2020/08/03/world/03china-tiktok2/merlin_170449443_1fefecc3-0173-4670-aebf-c0a859e19ef4-articleLarge.jpg?quality=75\&auto=webp\&disable=upscale}

Chibo Tang, a partner in Hong Kong at the venture capital firm Gobi
Partners, said that, increasingly, his advice to Chinese tech companies
was to steer clear of the United States when expanding overseas --- to
follow instead the Chinese government's diplomatic overtures and
investments in places such as Southeast Asia, the Middle East and
Africa.

``If you want to go out and tackle more difficult markets, sure, but
obviously there's consequences and additional costs,'' Mr. Tang said.
``Going forward, Chinese entrepreneurs in these tech companies should be
aware of that.''

One unnamed entrepreneur put ByteDance's position in even starker terms
to \href{https://m.huxiu.com/article/373071.html}{the Chinese tech blog
Huxiu} on Monday: ``Once the U.S. business is lost, half of the space
for thinking about globalization has vanished.''

As uncertainty swirled on Sunday about whether Mr. Trump would allow
Microsoft to continue negotiations with TikTok, ByteDance issued
\href{https://www.toutiao.com/a1673929179426827}{a late-night statement
in China} reiterating its commitment to going global.

``In the process, we are facing all kinds of complex and unimaginable
difficulties,'' the company said. The statement cited the tense
geopolitical environment, culture clashes and, in an unusually direct
jab at a competitor, ``Facebook's plagiarism and smears.''

Facebook is rolling out a TikTok-like feature called Reels on Instagram,
which it owns. The company's chief executive, Mark Zuckerberg, has also
argued that
\href{https://docs.house.gov/meetings/JU/JU05/20200729/110883/HHRG-116-JU05-Wstate-ZuckerbergM-20200729.pdf}{undermining
American tech companies with excess regulation} could allow Chinese
rivals to export their own, very different values to the world. Facebook
declined to comment on ByteDance's statement.

Image

Microsoft said it had received President Trump's go-ahead for pursuing a
TikTok deal.Credit...Hiroko Masuike/The New York Times

For Mr. Zhang of ByteDance, TikTok's run-in with the Trump
administration has been an education in government relations, though
hardly his first.

Mr. Zhang falls on the geekier side of the tech founder spectrum. He
repaired computers in college, and from past interviews, he appears most
at home talking about algorithms and the flow of information. He is not
a Communist Party member,
\href{https://www.theatlantic.com/international/archive/2020/07/tiktok-ban-china-america/614725/}{he
told the Atlantic magazine} recently.

For many years, he echoed Mr. Zuckerberg in saying he ran a tech
company, not a media outlet, which meant he should not be imposing his
own judgments over content.

``I can't accurately decide whether something is good or bad, highbrow
or lowbrow,'' he \href{https://36kr.com/p/1721289883649}{told the
Chinese business magazine Caijing} in 2016.

Mr. Zhang may have thought he was insulating himself in China. But the
perils of that technology-driven approach were made clear in 2018 when
the Chinese authorities shut down one of ByteDance's oldest products,
a\href{https://www.nytimes3xbfgragh.onion/2018/04/11/technology/china-toutiao-bytedance-censor.html}{humor
app called Neihan Duanzi}, for spreading vulgar material.

``For a long time, we put too much emphasis on the role of technology
and didn't realize that technology must be guided by core socialist
values,'' Mr. Zhang wrote in a
\href{https://mp.weixin.qq.com/s/4r6rCwNE7BgTLD37cPJOoA}{public letter
of apology}.

ByteDance's popular news aggregator app, Toutiao, had also been under
fire for saucy content. In response, Toutiao began featuring
\href{https://www.nytimes3xbfgragh.onion/2018/01/02/business/china-toutiao-censorship.html}{more
stories about Mr. Xi} at the top of its feed.

By then, ByteDance had already begun expanding in Japan, India,
Southeast Asia and beyond. TikTok was released in 2017 as the
international edition of Douyin, one of ByteDance's Chinese video apps.

TikTok had some early scrapes with foreign governments. In 2018,
\href{https://www.nytimes3xbfgragh.onion/2018/10/29/technology/bytedance-app-funding-china.html}{Indonesia
temporarily blocked it} for hosting inappropriate content. Despite the
challenges, Mr. Zhang said
\href{https://www.techbuzzchina.com/bytedance/bytedance-ceo-zhang-yiming-at-tsinghua-university-part-3}{at
an event in Beijing that year} that going global was the only way to get
access to the talent and resources needed for long-term success.

He said he had studied another Chinese company's rapid growth overseas
to see how it could be done.

Which company? Huawei.

His choice was prescient in hindsight, though perhaps not in the way he
intended. The Trump administration has for years sought to undermine the
giant Chinese maker of telecommunications equipment and smartphones. It,
too, has been called a national security threat by White House
officials, who fear the Chinese government could use Huawei gear for
espionage.

Image

Mr. Zhang said he had studied Huawei as a model for international
growth.Credit...Lam Yik Fei for The New York Times

International growth was top of mind when Mr. Zhang began courting
Musical.ly, a Chinese-made lip-syncing app that had found success in the
United States and Europe. In late 2017,
\href{https://www.nytimes3xbfgragh.onion/2017/11/10/business/dealbook/musically-sold-app-video.html}{ByteDance
agreed to buy Musical.ly} for around \$1 billion. ByteDance would later
merge the app into TikTok, giving it a toehold in the West that would
eventually propel it to wider success.

According to people with knowledge of the matter, the two parties did
not approach the Committee on Foreign Investment in the United States,
or CFIUS, to seek its blessing beforehand --- a decision that would
later come back to haunt ByteDance.

CFIUS (pronounced SIFF-ee-yuss) typically evaluates foreign deals
involving an American business for possible national security risks. But
it also claims jurisdiction over deals between foreign businesses that
have significant American operations.

As TikTok became a smash hit in the United States, concerns arose about
whether the app was
\href{https://www.theguardian.com/technology/2019/sep/25/revealed-how-tiktok-censors-videos-that-do-not-please-beijing}{censoring
content} that might offend Beijing. Late last year, The New York Times
and others reported that CFIUS was
\href{https://www.nytimes3xbfgragh.onion/2019/11/01/technology/tiktok-national-security-review.html}{looking
into the Musical.ly deal}. Washington politicians also began voicing
fears that TikTok could be a conduit for China to
\href{https://thehill.com/policy/technology/467280-schumer-cotton-request-tiktok-security-assessment}{meddle
in American elections}.

With pressure building, some of Mr. Zhang's investors and advisers
offered ideas for putting distance between TikTok and ByteDance,
including reorganizing TikTok's corporate or legal structure.

In an
\href{https://www.nytimes3xbfgragh.onion/2019/11/18/technology/tiktok-alex-zhu-interview.html}{interview
in November}, Alex Zhu, a founder of Musical.ly who was then the head of
TikTok, said the company wouldn't rule out such changes.

``We continuously look at the company structure and optimize the
structure,'' Mr. Zhu said.

But instead of a major restructuring, Mr. Zhang opted for personnel
changes. This spring, he
\href{https://mp.weixin.qq.com/s/OWrC9iXHxgUZtaLm8GN4ow}{reshuffled
ByteDance executives in China} and said he would personally devote more
time and energy to Europe, the United States and other markets. In May,
Liu Zhen, a former Uber executive in China who had been overseeing
ByteDance's global expansion, left the company. Mr. Zhu was replaced as
TikTok's head by Kevin Mayer,
\href{https://www.nytimes3xbfgragh.onion/2020/05/18/business/media/tiktok-ceo-kevin-mayer.html}{a
veteran Disney executive} in the United States.

ByteDance also embarked upon a
\href{https://www.nytimes3xbfgragh.onion/2020/07/15/technology/tiktok-washington-lobbyist.html}{lobbying
push in Washington} to sell the idea that TikTok's allegiances were with
the United States, not China. In meetings with lawmakers, lobbyists
emphasized the app's light, uplifting fare and the fact that many of its
top leaders were American residents.

Last month, when American technology companies including Facebook and
Google began
\href{https://www.nytimes3xbfgragh.onion/2020/07/07/business/hong-kong-security-law-tech.html}{reassessing
their operations in Hong Kong} in the wake of a new security law that
gave the Chinese government greater powers in the territory, TikTok went
further, announcing that it would stop operating in Hong Kong
completely.

The move let TikTok demonstrate its willingness to stand up to Beijing,
as its head of U.S. public policy later emphasized in an email
newsletter to Capitol Hill. But Hong Kong had not been a major market
for the app, making the decision look more like a publicity stunt than a
self-sacrifice made on principled grounds.

Image

TikTok users took credit for the lackluster attendence at Mr. Trump's
campaign rally in Tulsa, Okla., in June.Credit...Christopher Lee for The
New York Times

The Trump administration's scrutiny continued unabated. After
\href{https://www.nytimes3xbfgragh.onion/2020/06/21/style/tiktok-trump-rally-tulsa.html}{Mr.
Trump failed to draw huge crowds} at a June re-election rally in Tulsa,
Okla., TikTok users claimed to have pulled off a prank by registering
for tickets and then not attending the event. In early July, Secretary
of State Mike Pompeo floated the idea of
\href{https://www.cnbc.com/2020/07/07/us-looking-at-banning-tiktok-and-chinese-social-media-apps-pompeo.html}{banning
the app} over security concerns.

Within weeks,
\href{https://www.nytimes3xbfgragh.onion/2020/08/02/business/economy/trump-tiktok-china-national-security.html}{Microsoft
said} it had received Mr. Trump's go-ahead for pursuing a deal to buy
TikTok's U.S. operations.
\href{https://www.nytimes3xbfgragh.onion/2020/07/31/technology/tiktok-microsoft.html}{CFIUS
had decided} to order ByteDance to divest.

In
\href{https://www.toutiao.com/i6856642212948607502/?timestamp=1596448119\&app=news_article\&group_id=6856642212948607502\&use_new_style=1\&req_id=2020080317483801001404814024245E48}{a
letter to ByteDance's employees} on Monday, Mr. Zhang made the recent
turmoil sound more like a technical matter than an existential threat
brought about by hostile geopolitical forces.

He wrote that the company had repeatedly emphasized that it was willing
to make technical changes to address U.S. concerns, yet the order to
sell was made anyway. ``We do not agree with this decision, because we
have always insisted on guaranteeing users' data security, the
platform's neutrality and transparency.''

Lin Qiqing contributed research.

Advertisement

\protect\hyperlink{after-bottom}{Continue reading the main story}

\hypertarget{site-index}{%
\subsection{Site Index}\label{site-index}}

\hypertarget{site-information-navigation}{%
\subsection{Site Information
Navigation}\label{site-information-navigation}}

\begin{itemize}
\tightlist
\item
  \href{https://help.nytimes3xbfgragh.onion/hc/en-us/articles/115014792127-Copyright-notice}{©~2020~The
  New York Times Company}
\end{itemize}

\begin{itemize}
\tightlist
\item
  \href{https://www.nytco.com/}{NYTCo}
\item
  \href{https://help.nytimes3xbfgragh.onion/hc/en-us/articles/115015385887-Contact-Us}{Contact
  Us}
\item
  \href{https://www.nytco.com/careers/}{Work with us}
\item
  \href{https://nytmediakit.com/}{Advertise}
\item
  \href{http://www.tbrandstudio.com/}{T Brand Studio}
\item
  \href{https://www.nytimes3xbfgragh.onion/privacy/cookie-policy\#how-do-i-manage-trackers}{Your
  Ad Choices}
\item
  \href{https://www.nytimes3xbfgragh.onion/privacy}{Privacy}
\item
  \href{https://help.nytimes3xbfgragh.onion/hc/en-us/articles/115014893428-Terms-of-service}{Terms
  of Service}
\item
  \href{https://help.nytimes3xbfgragh.onion/hc/en-us/articles/115014893968-Terms-of-sale}{Terms
  of Sale}
\item
  \href{https://spiderbites.nytimes3xbfgragh.onion}{Site Map}
\item
  \href{https://help.nytimes3xbfgragh.onion/hc/en-us}{Help}
\item
  \href{https://www.nytimes3xbfgragh.onion/subscription?campaignId=37WXW}{Subscriptions}
\end{itemize}
