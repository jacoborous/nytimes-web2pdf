Sections

SEARCH

\protect\hyperlink{site-content}{Skip to
content}\protect\hyperlink{site-index}{Skip to site index}

\href{https://www.nytimes3xbfgragh.onion/section/nyregion}{New York}

\href{https://myaccount.nytimes3xbfgragh.onion/auth/login?response_type=cookie\&client_id=vi}{}

\href{https://www.nytimes3xbfgragh.onion/section/todayspaper}{Today's
Paper}

\href{/section/nyregion}{New York}\textbar{}Is Riding the Subway Safer
Than Dining Indoors?

\url{https://nyti.ms/3i2v5Q5}

\begin{itemize}
\item
\item
\item
\item
\item
\item
\end{itemize}

\href{https://www.nytimes3xbfgragh.onion/news-event/coronavirus?action=click\&pgtype=Article\&state=default\&region=TOP_BANNER\&context=storylines_menu}{The
Coronavirus Outbreak}

\begin{itemize}
\tightlist
\item
  live\href{https://www.nytimes3xbfgragh.onion/2020/08/04/world/coronavirus-cases.html?action=click\&pgtype=Article\&state=default\&region=TOP_BANNER\&context=storylines_menu}{Latest
  Updates}
\item
  \href{https://www.nytimes3xbfgragh.onion/interactive/2020/us/coronavirus-us-cases.html?action=click\&pgtype=Article\&state=default\&region=TOP_BANNER\&context=storylines_menu}{Maps
  and Cases}
\item
  \href{https://www.nytimes3xbfgragh.onion/interactive/2020/science/coronavirus-vaccine-tracker.html?action=click\&pgtype=Article\&state=default\&region=TOP_BANNER\&context=storylines_menu}{Vaccine
  Tracker}
\item
  \href{https://www.nytimes3xbfgragh.onion/2020/08/02/us/covid-college-reopening.html?action=click\&pgtype=Article\&state=default\&region=TOP_BANNER\&context=storylines_menu}{College
  Reopening}
\item
  \href{https://www.nytimes3xbfgragh.onion/live/2020/08/04/business/stock-market-today-coronavirus?action=click\&pgtype=Article\&state=default\&region=TOP_BANNER\&context=storylines_menu}{Economy}
\end{itemize}

Advertisement

\protect\hyperlink{after-top}{Continue reading the main story}

Supported by

\protect\hyperlink{after-sponsor}{Continue reading the main story}

New York Today

\hypertarget{is-riding-the-subway-safer-than-dining-indoors}{%
\section{Is Riding the Subway Safer Than Dining
Indoors?}\label{is-riding-the-subway-safer-than-dining-indoors}}

\href{https://www.nytimes3xbfgragh.onion/by/mihir-zaveri}{\includegraphics{https://static01.graylady3jvrrxbe.onion/images/2018/07/18/multimedia/author-mihir-zaveri/author-mihir-zaveri-thumbLarge.png}}

By \href{https://www.nytimes3xbfgragh.onion/by/mihir-zaveri}{Mihir
Zaveri}

\begin{itemize}
\item
  Aug. 3, 2020
\item
  \begin{itemize}
  \item
  \item
  \item
  \item
  \item
  \item
  \end{itemize}
\end{itemize}

\emph{{[}Want to get New York Today by email?}
\href{https://www.nytimes3xbfgragh.onion/newsletters/newyorktoday}{\emph{Here's
the sign-up}}\emph{.{]}}

\textbf{It's Monday.}

\textbf{Weather:} ** Sunny today, with a high around 90. Watch out for
thunderstorms and heavy rain this evening as the effects of Tropical
Storm Isaias begin arriving.

\textbf{Alternate-side parking}: In effect until Aug. 15 (Feast of the
Assumption).
\href{https://www1.nyc.gov/html/dot/html/motorist/alternate-side-parking.shtml\#reform}{Read
about the amended regulations here}.

\begin{center}\rule{0.5\linewidth}{\linethickness}\end{center}

\includegraphics{https://static01.graylady3jvrrxbe.onion/images/2020/06/09/nyregion/03nytoday-1/00nysubway-articleLarge.jpg?quality=75\&auto=webp\&disable=upscale}

Not many people are riding the subway these days --- being cooped up
with strangers in a subway car must surely be among the riskiest
activities possible during the coronavirus pandemic.

But that may not be entirely true. Even as riders have returned in
greater numbers to public transit in Europe and Asia, there has been a
lack of corresponding superspreader events,
\href{https://www.nytimes3xbfgragh.onion/2020/08/02/nyregion/nyc-subway-coronavirus-safety.html}{my
colleague Christina Goldbaum reported}. A superspreader is an infected
person who transmits the virus to many others.

So while experts say riding the subway is probably riskier than walking
outdoors, it is also probably safer than dining indoors.

Here are some takeaways from her article:

\textbf{In countries where the outbreak has waned, there have been no
notable superspreader events linked to mass transit.}

In Beijing, subway ridership has risen to about 60 percent of
pre-pandemic levels; in Berlin, ridership on buses and subways is 60 to
70 percent of normal rates; and in Paris, ridership on the Métro has
returned to 45 percent of usual levels.

\hypertarget{latest-updates-global-coronavirus-outbreak}{%
\section{\texorpdfstring{\href{https://www.nytimes3xbfgragh.onion/2020/08/04/world/coronavirus-cases.html?action=click\&pgtype=Article\&state=default\&region=MAIN_CONTENT_1\&context=storylines_live_updates}{Latest
Updates: Global Coronavirus
Outbreak}}{Latest Updates: Global Coronavirus Outbreak}}\label{latest-updates-global-coronavirus-outbreak}}

Updated 2020-08-04T20:08:28.255Z

\begin{itemize}
\tightlist
\item
  \href{https://www.nytimes3xbfgragh.onion/2020/08/04/world/coronavirus-cases.html?action=click\&pgtype=Article\&state=default\&region=MAIN_CONTENT_1\&context=storylines_live_updates\#link-1228a480}{Novavax
  sees encouraging results from two studies of its experimental
  vaccine.}
\item
  \href{https://www.nytimes3xbfgragh.onion/2020/08/04/world/coronavirus-cases.html?action=click\&pgtype=Article\&state=default\&region=MAIN_CONTENT_1\&context=storylines_live_updates\#link-4825b93}{Public
  and private schools in Maryland and elsewhere are divided over
  in-person instruction.}
\item
  \href{https://www.nytimes3xbfgragh.onion/2020/08/04/world/coronavirus-cases.html?action=click\&pgtype=Article\&state=default\&region=MAIN_CONTENT_1\&context=storylines_live_updates\#link-4d1eafa8}{N.Y.C.'s
  health commissioner resigns after clashing with the mayor over the
  virus.}
\end{itemize}

\href{https://www.nytimes3xbfgragh.onion/2020/08/04/world/coronavirus-cases.html?action=click\&pgtype=Article\&state=default\&region=MAIN_CONTENT_1\&context=storylines_live_updates}{See
more updates}

More live coverage:
\href{https://www.nytimes3xbfgragh.onion/live/2020/08/04/business/stock-market-today-coronavirus?action=click\&pgtype=Article\&state=default\&region=MAIN_CONTENT_1\&context=storylines_live_updates}{Markets}

Superspreader events, however, have not emerged.

\textbf{Still, public health experts warn that the evidence so far
should be met with caution.}

Even in cities where ridership is up, it has not reached pre-pandemic
levels. A full return, with jam-packed subway cars, still poses a
danger.

There are also other factors that may affect whether outbreaks occur.
They include the quality of ventilation systems used to filter air, and
the level to which a city has reduced its overall infection rate.

As a result, tracking clusters of coronavirus cases to public transit is
difficult: The chances that infected people would remember the precise
train car they rode in is unlikely, and reaching those who were in that
same car is nearly impossible.

Even looking back at the worst months in New York City, officials are
not clear on the degree to which mass transit contributed to a surge
that killed more than 20,000 people.

\textbf{New York officials are now focused on how to draw riders back
while avoiding crowding at rush hour.}

Subway ridership in the city is still at about 20 percent of
pre-pandemic levels. For the Metropolitan Transportation Authority, the
agency that runs the city's subway and buses and that relies on fare
revenue for 40 percent of its operating budget, that's a problem.

But there are things that can lessen the risk of riding the subway.

In the months since the height of the outbreak in New York, the
authority has invested hundreds of millions of dollars in the daily
disinfection of train cars, distributed over a million masks to riders
and started public service campaigns encouraging riders to maintain
social distance.

New York transit officials say that a recent observational study of more
than 220,000 riders found that over 90 percent were wearing masks.

In New York's subway trains, transit officials say, the air that
circulates through a car is replaced with fresh, filtered air at least
18 times an hour, much higher than the recommended air-exchange rates in
restaurants or in offices.

\begin{center}\rule{0.5\linewidth}{\linethickness}\end{center}

\hypertarget{from-the-times}{%
\subsection{From The Times}\label{from-the-times}}

\href{https://www.nytimes3xbfgragh.onion/2020/08/03/nyregion/nyc-mail-ballots-voting.html}{Why
the Botched N.Y.C. Primary Has Become the November Nightmare}

\href{https://www.nytimes3xbfgragh.onion/2020/08/02/nyregion/liberty-belle-illegal-party.html}{Arrests
Over Illicit Party Boat With 170 Guests Cruising Around N.Y.C.}

\href{https://www.nytimes3xbfgragh.onion/2020/08/03/nyregion/police-shea-de-blasio-nyc.html}{These
Remarks Might Get a Police Chief Fired. Not in New York.}

\href{https://www.nytimes3xbfgragh.onion/2020/08/02/sports/baseball/Yoenis-cespedes-opt-out-rule.html}{Mets'
Yoenis Cespedes Opts Out of 2020 Season}

\textbf{A new podcast:}
``\href{https://www.nytimes3xbfgragh.onion/2020/07/30/podcasts/nice-white-parents-serial.html}{Nice
White Parents}'' from Serial, a New York Times Company, investigates
what happened when a group of white families arrived at a predominately
Black and Latino school in New York City.

Want more news?
\href{https://www.nytimes3xbfgragh.onion/section/nyregion}{Check out our
full coverage}.

\textbf{The Mini Crossword:} Here is
\href{https://www.nytimes3xbfgragh.onion/crosswords/game/mini}{today's
puzzle}.

\begin{center}\rule{0.5\linewidth}{\linethickness}\end{center}

\hypertarget{what-were-reading}{%
\subsection{What we're reading}\label{what-were-reading}}

A rare operation saved the eyesight of a \textbf{6-week-old baby with
cataracts}. {[}\href{https://abc7ny.com/6340873/}{ABC 7}{]}

An \textbf{off-duty New York police officer} fatally shot himself in
Queens.
{[}\href{https://nypost.com/2020/08/02/off-duty-nypd-cop-shoots-himself-in-the-head-in-queens-suicide/}{New
York Post}{]}

\href{https://www.nytimes3xbfgragh.onion/news-event/coronavirus?action=click\&pgtype=Article\&state=default\&region=MAIN_CONTENT_3\&context=storylines_faq}{}

\hypertarget{the-coronavirus-outbreak-}{%
\subsubsection{The Coronavirus Outbreak
›}\label{the-coronavirus-outbreak-}}

\hypertarget{frequently-asked-questions}{%
\paragraph{Frequently Asked
Questions}\label{frequently-asked-questions}}

Updated August 4, 2020

\begin{itemize}
\item ~
  \hypertarget{i-have-antibodies-am-i-now-immune}{%
  \paragraph{I have antibodies. Am I now
  immune?}\label{i-have-antibodies-am-i-now-immune}}

  \begin{itemize}
  \tightlist
  \item
    As of right
    now,\href{https://www.nytimes3xbfgragh.onion/2020/07/22/health/covid-antibodies-herd-immunity.html?action=click\&pgtype=Article\&state=default\&region=MAIN_CONTENT_3\&context=storylines_faq}{that
    seems likely, for at least several months.} There have been
    frightening accounts of people suffering what seems to be a second
    bout of Covid-19. But experts say these patients may have a
    drawn-out course of infection, with the virus taking a slow toll
    weeks to months after initial exposure. People infected with the
    coronavirus typically
    \href{https://www.nature.com/articles/s41586-020-2456-9}{produce}
    immune molecules called antibodies, which are
    \href{https://www.nytimes3xbfgragh.onion/2020/05/07/health/coronavirus-antibody-prevalence.html?action=click\&pgtype=Article\&state=default\&region=MAIN_CONTENT_3\&context=storylines_faq}{protective
    proteins made in response to an
    infection}\href{https://www.nytimes3xbfgragh.onion/2020/05/07/health/coronavirus-antibody-prevalence.html?action=click\&pgtype=Article\&state=default\&region=MAIN_CONTENT_3\&context=storylines_faq}{.
    These antibodies may} last in the body
    \href{https://www.nature.com/articles/s41591-020-0965-6}{only two to
    three months}, which may seem worrisome, but that's perfectly normal
    after an acute infection subsides, said Dr. Michael Mina, an
    immunologist at Harvard University. It may be possible to get the
    coronavirus again, but it's highly unlikely that it would be
    possible in a short window of time from initial infection or make
    people sicker the second time.
  \end{itemize}
\item ~
  \hypertarget{im-a-small-business-owner-can-i-get-relief}{%
  \paragraph{I'm a small-business owner. Can I get
  relief?}\label{im-a-small-business-owner-can-i-get-relief}}

  \begin{itemize}
  \tightlist
  \item
    The
    \href{https://www.nytimes3xbfgragh.onion/article/small-business-loans-stimulus-grants-freelancers-coronavirus.html?action=click\&pgtype=Article\&state=default\&region=MAIN_CONTENT_3\&context=storylines_faq}{stimulus
    bills enacted in March} offer help for the millions of American
    small businesses. Those eligible for aid are businesses and
    nonprofit organizations with fewer than 500 workers, including sole
    proprietorships, independent contractors and freelancers. Some
    larger companies in some industries are also eligible. The help
    being offered, which is being managed by the Small Business
    Administration, includes the Paycheck Protection Program and the
    Economic Injury Disaster Loan program. But lots of folks have
    \href{https://www.nytimes3xbfgragh.onion/interactive/2020/05/07/business/small-business-loans-coronavirus.html?action=click\&pgtype=Article\&state=default\&region=MAIN_CONTENT_3\&context=storylines_faq}{not
    yet seen payouts.} Even those who have received help are confused:
    The rules are draconian, and some are stuck sitting on
    \href{https://www.nytimes3xbfgragh.onion/2020/05/02/business/economy/loans-coronavirus-small-business.html?action=click\&pgtype=Article\&state=default\&region=MAIN_CONTENT_3\&context=storylines_faq}{money
    they don't know how to use.} Many small-business owners are getting
    less than they expected or
    \href{https://www.nytimes3xbfgragh.onion/2020/06/10/business/Small-business-loans-ppp.html?action=click\&pgtype=Article\&state=default\&region=MAIN_CONTENT_3\&context=storylines_faq}{not
    hearing anything at all.}
  \end{itemize}
\item ~
  \hypertarget{what-are-my-rights-if-i-am-worried-about-going-back-to-work}{%
  \paragraph{What are my rights if I am worried about going back to
  work?}\label{what-are-my-rights-if-i-am-worried-about-going-back-to-work}}

  \begin{itemize}
  \tightlist
  \item
    Employers have to provide
    \href{https://www.osha.gov/SLTC/covid-19/standards.html}{a safe
    workplace} with policies that protect everyone equally.
    \href{https://www.nytimes3xbfgragh.onion/article/coronavirus-money-unemployment.html?action=click\&pgtype=Article\&state=default\&region=MAIN_CONTENT_3\&context=storylines_faq}{And
    if one of your co-workers tests positive for the coronavirus, the
    C.D.C.} has said that
    \href{https://www.cdc.gov/coronavirus/2019-ncov/community/guidance-business-response.html}{employers
    should tell their employees} -\/- without giving you the sick
    employee's name -\/- that they may have been exposed to the virus.
  \end{itemize}
\item ~
  \hypertarget{should-i-refinance-my-mortgage}{%
  \paragraph{Should I refinance my
  mortgage?}\label{should-i-refinance-my-mortgage}}

  \begin{itemize}
  \tightlist
  \item
    \href{https://www.nytimes3xbfgragh.onion/article/coronavirus-money-unemployment.html?action=click\&pgtype=Article\&state=default\&region=MAIN_CONTENT_3\&context=storylines_faq}{It
    could be a good idea,} because mortgage rates have
    \href{https://www.nytimes3xbfgragh.onion/2020/07/16/business/mortgage-rates-below-3-percent.html?action=click\&pgtype=Article\&state=default\&region=MAIN_CONTENT_3\&context=storylines_faq}{never
    been lower.} Refinancing requests have pushed mortgage applications
    to some of the highest levels since 2008, so be prepared to get in
    line. But defaults are also up, so if you're thinking about buying a
    home, be aware that some lenders have tightened their standards.
  \end{itemize}
\item ~
  \hypertarget{what-is-school-going-to-look-like-in-september}{%
  \paragraph{What is school going to look like in
  September?}\label{what-is-school-going-to-look-like-in-september}}

  \begin{itemize}
  \tightlist
  \item
    It is unlikely that many schools will return to a normal schedule
    this fall, requiring the grind of
    \href{https://www.nytimes3xbfgragh.onion/2020/06/05/us/coronavirus-education-lost-learning.html?action=click\&pgtype=Article\&state=default\&region=MAIN_CONTENT_3\&context=storylines_faq}{online
    learning},
    \href{https://www.nytimes3xbfgragh.onion/2020/05/29/us/coronavirus-child-care-centers.html?action=click\&pgtype=Article\&state=default\&region=MAIN_CONTENT_3\&context=storylines_faq}{makeshift
    child care} and
    \href{https://www.nytimes3xbfgragh.onion/2020/06/03/business/economy/coronavirus-working-women.html?action=click\&pgtype=Article\&state=default\&region=MAIN_CONTENT_3\&context=storylines_faq}{stunted
    workdays} to continue. California's two largest public school
    districts --- Los Angeles and San Diego --- said on July 13, that
    \href{https://www.nytimes3xbfgragh.onion/2020/07/13/us/lausd-san-diego-school-reopening.html?action=click\&pgtype=Article\&state=default\&region=MAIN_CONTENT_3\&context=storylines_faq}{instruction
    will be remote-only in the fall}, citing concerns that surging
    coronavirus infections in their areas pose too dire a risk for
    students and teachers. Together, the two districts enroll some
    825,000 students. They are the largest in the country so far to
    abandon plans for even a partial physical return to classrooms when
    they reopen in August. For other districts, the solution won't be an
    all-or-nothing approach.
    \href{https://bioethics.jhu.edu/research-and-outreach/projects/eschool-initiative/school-policy-tracker/}{Many
    systems}, including the nation's largest, New York City, are
    devising
    \href{https://www.nytimes3xbfgragh.onion/2020/06/26/us/coronavirus-schools-reopen-fall.html?action=click\&pgtype=Article\&state=default\&region=MAIN_CONTENT_3\&context=storylines_faq}{hybrid
    plans} that involve spending some days in classrooms and other days
    online. There's no national policy on this yet, so check with your
    municipal school system regularly to see what is happening in your
    community.
  \end{itemize}
\end{itemize}

Could the \textbf{Black Lives Matter street paintings} create a First
Amendment predicament for Mayor Bill de Blasio?
{[}\href{https://www.silive.com/news/2020/08/street-murals-may-create-a-first-amendment-predicament-for-de-blasio.html}{Staten
Island Advance}{]}

\begin{center}\rule{0.5\linewidth}{\linethickness}\end{center}

\hypertarget{and-finally-ask-your-questions-about-the-pandemic}{%
\subsection{And finally: Ask your questions about the
pandemic}\label{and-finally-ask-your-questions-about-the-pandemic}}

The United States recorded more than 1.9 million new coronavirus
infections last month, more than double any previous monthly total. In
New York, the city is opening up but taking protective measures.

To help us all understand the recent events and what the future might
hold, we recently asked you to send in your questions about the effects
of the virus on daily life. Readers of The Times's California Today
newsletter, for example, asked
\href{https://www.nytimes3xbfgragh.onion/2020/07/31/us/essential-workers-massage-therapists.html}{how
massage therapists were adapting} to the pandemic and what people should
know before
\href{https://www.nytimes3xbfgragh.onion/2020/07/13/us/california-parks-coronavirus.html}{visiting
state and national parks}.

We're curious about everything --- what do you want to know about the
economy, health care, the environment, the arts, dating, traveling and
more? How are you experiencing life during the pandemic, and how can we
help you make sense of those changes?

\href{https://www.nytimes3xbfgragh.onion/2019/09/17/reader-center/coronavirus-nyc-questions.html}{Using
this form}, tell us what topics you want us to dig into and why.
Specific questions are encouraged. We may answer some of your questions
in future newsletters and stories.

\emph{It's Monday --- be curious.}

\begin{center}\rule{0.5\linewidth}{\linethickness}\end{center}

\hypertarget{metropolitan-diary-giving-directions}{%
\subsection{Metropolitan Diary: Giving
directions}\label{metropolitan-diary-giving-directions}}

Image

Dear Diary:

As I left the No. 6 train station at Bleecker Street, I noticed two
young men on the corner. One was holding a map, and they both had
puzzled looks on their faces as they scanned the nearby street signs.

I asked whether they needed help.

The one with the map said he knew where they were but couldn't find the
spot on the map.

I pointed out Houston Street a block away, and then showed them where it
was on the map. They thanked me.

As I started to walk away, the second man reached out his hand --- not
to shake mine, but to give me a \$1 bill.

It was the only time I'd been offered a tip for giving directions.

\emph{--- John F. Backe}

\begin{center}\rule{0.5\linewidth}{\linethickness}\end{center}

\emph{New York Today is published weekdays around 6 a.m.}
\href{https://www.nytimes3xbfgragh.onion/newsletters/newyorktoday?module=inline}{\emph{Sign
up here}} \emph{to get it by email. You can also find it at}
\href{http://www.nytoday.com/}{\emph{nytoday.com}}\emph{.}

\emph{We're experimenting with the format of New York Today. What would
you like to see more (or less) of? Post a comment or email us:}
\href{mailto:nytoday@NYTimes.com}{\emph{nytoday@NYTimes.com}}\emph{.}

Advertisement

\protect\hyperlink{after-bottom}{Continue reading the main story}

\hypertarget{site-index}{%
\subsection{Site Index}\label{site-index}}

\hypertarget{site-information-navigation}{%
\subsection{Site Information
Navigation}\label{site-information-navigation}}

\begin{itemize}
\tightlist
\item
  \href{https://help.nytimes3xbfgragh.onion/hc/en-us/articles/115014792127-Copyright-notice}{©~2020~The
  New York Times Company}
\end{itemize}

\begin{itemize}
\tightlist
\item
  \href{https://www.nytco.com/}{NYTCo}
\item
  \href{https://help.nytimes3xbfgragh.onion/hc/en-us/articles/115015385887-Contact-Us}{Contact
  Us}
\item
  \href{https://www.nytco.com/careers/}{Work with us}
\item
  \href{https://nytmediakit.com/}{Advertise}
\item
  \href{http://www.tbrandstudio.com/}{T Brand Studio}
\item
  \href{https://www.nytimes3xbfgragh.onion/privacy/cookie-policy\#how-do-i-manage-trackers}{Your
  Ad Choices}
\item
  \href{https://www.nytimes3xbfgragh.onion/privacy}{Privacy}
\item
  \href{https://help.nytimes3xbfgragh.onion/hc/en-us/articles/115014893428-Terms-of-service}{Terms
  of Service}
\item
  \href{https://help.nytimes3xbfgragh.onion/hc/en-us/articles/115014893968-Terms-of-sale}{Terms
  of Sale}
\item
  \href{https://spiderbites.nytimes3xbfgragh.onion}{Site Map}
\item
  \href{https://help.nytimes3xbfgragh.onion/hc/en-us}{Help}
\item
  \href{https://www.nytimes3xbfgragh.onion/subscription?campaignId=37WXW}{Subscriptions}
\end{itemize}
