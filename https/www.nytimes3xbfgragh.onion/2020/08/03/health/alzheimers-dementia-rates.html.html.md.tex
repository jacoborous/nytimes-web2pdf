Sections

SEARCH

\protect\hyperlink{site-content}{Skip to
content}\protect\hyperlink{site-index}{Skip to site index}

\href{https://www.nytimes3xbfgragh.onion/section/health}{Health}

\href{https://myaccount.nytimes3xbfgragh.onion/auth/login?response_type=cookie\&client_id=vi}{}

\href{https://www.nytimes3xbfgragh.onion/section/todayspaper}{Today's
Paper}

\href{/section/health}{Health}\textbar{}Dementia on the Retreat in the
U.S. and Europe

\url{https://nyti.ms/30oTycj}

\begin{itemize}
\item
\item
\item
\item
\item
\end{itemize}

Advertisement

\protect\hyperlink{after-top}{Continue reading the main story}

Supported by

\protect\hyperlink{after-sponsor}{Continue reading the main story}

\hypertarget{dementia-on-the-retreat-in-the-us-and-europe}{%
\section{Dementia on the Retreat in the U.S. and
Europe}\label{dementia-on-the-retreat-in-the-us-and-europe}}

Rates of dementia have steadily fallen over the past 25 years, a new
study finds. But the disease is increasingly common in some parts of the
world.

\includegraphics{https://static01.graylady3jvrrxbe.onion/images/2020/08/03/science/03DEMENTIA-DECLINE/merlin_172056843_fbf69ca0-9414-40b1-9f8f-c9961e9c6706-articleLarge.jpg?quality=75\&auto=webp\&disable=upscale}

\href{https://www.nytimes3xbfgragh.onion/by/gina-kolata}{\includegraphics{https://static01.graylady3jvrrxbe.onion/images/2018/02/16/multimedia/author-gina-kolata/author-gina-kolata-thumbLarge.jpg}}

By \href{https://www.nytimes3xbfgragh.onion/by/gina-kolata}{Gina Kolata}

\begin{itemize}
\item
  Aug. 3, 2020
\item
  \begin{itemize}
  \item
  \item
  \item
  \item
  \item
  \end{itemize}
\end{itemize}

Despite the lack of effective treatments or preventive strategies, the
dementia epidemic
\href{https://n.neurology.org/content/early/2020/07/01/WNL.0000000000010022}{is
on the wane in the United States and Europe}, scientists reported on
Monday.

The risk for a person to develop dementia over a lifetime is now 13
percent lower than it was in 2010. Incidence rates at every age have
steadily declined over the past quarter-century. If the trend continues,
the paper's authors note, there will be 15 million fewer people in
Europe and the United States with dementia than there are now.

The study is the most definitive yet to document a decline in dementia
rates. Its findings counter warnings from advocacy groups of a coming
tsunami of Alzheimer's disease, the most common form of dementia, said
Dr. John Morris, director of the Center for Aging at Washington
University in St. Louis.

It is correct that there are now more people than ever with dementia,
but that is because there are more and more older people in the
population.

The new incidence data are ``hopeful,'' Dr. Morris said. ``It is such a
strong study and such a powerful message. It suggests that the risk is
modifiable.''

Researchers at Harvard University in Cambridge, Mass., reviewed data
from seven large studies with a total of 49,202 individuals. The studies
followed men and women aged 65 and older for at least 15 years, and
included in-person exams and, in many cases, genetic data, brain scans
and information on participants' risk factors for cardiovascular
disease.

The data also include a separate assessment of Alzheimer's disease. Its
incidence, too, has steadily fallen, at a rate of 16 percent per decade,
the researchers found. Their study was published in the journal
Neurology.

In 1995, a 75-year-old man had about a 25 percent chance of developing
dementia in his remaining lifetime. Now that man's chance declined to 18
percent, said Dr. Albert Hofman, chairman of the department of
epidemiology at the Harvard School of Public Health and the lead author
of the new paper.

Although it is often said that women are more likely to get dementia
than men, Dr. Hofman and his colleagues found that men and women have
equal dementia rates.

The reason for the confusion appears to be that there are more older
women than older men in the population. At any age when dementia is
likely, there will be more women with dementia in the population than
men.

One puzzling aspect of the decline is that it seems to be confined to
Europe and the United States --- it was not seen in Asia, South America
or, from limited data, in Africa. There have been reports of increasing
dementia rates in Japan, China and Nigeria, the paper's authors note.

Those increases are puzzling, Dr. Hofman said. The trend may be related
to higher rates of smoking, which makes dementia more likely, in those
countries.

One leading hypothesis for the decline in the United States and Europe
is improved control of cardiovascular risk factors, especially blood
pressure and cholesterol. Nearly all dementia patients have other brain
abnormalities, including blood vessel damage likely to be the result of
high blood pressure.

High blood pressure seems to be most damaging in middle age, Dr. Hofman
said. Those with lower blood pressure earlier in life but higher blood
pressure later tend to have reduced chances of dementia.

Large swings in blood pressure are a risk at any age, he added.

Another possible reason for declining dementia rates might be better
education, which is thought to have a protective effect by giving the
brain more capacity --- for example, a memory cache of more synonyms for
words that were forgotten.

Like control of blood pressure and cholesterol, education levels have
gradually improved over the past few decades. ``There is a theory, but
still not much evidence, that education shifts dementia to a later
age,'' Dr. Hofman said.

The genetic risk factors for dementia cannot have changed, said Dr.
Richard Hodes, director of the National Institute on Aging. ``That means
something in the environment has occurred,'' he added, which ``has to
encourage us.''

But if improved education is the answer, the decline in dementia rates
may be nearing its end, Dr. Hodes noted. He also cautioned against
assuming that factors like blood pressure or education, linked in
observational studies to dementia, might signal cause and effect.

Neither can scientists yet assume that various lifestyle factors linked
to a lower risk of dementia mean they are protective.

``Many are using these reports to recommend better diets and more
exercise,'' Dr. Hodes said. ``I couldn't possibly be opposed to more
social interactions, more activity, better diets, better control of
blood pressure.''

``But we need more research for a greater degree of certainty.''

Advertisement

\protect\hyperlink{after-bottom}{Continue reading the main story}

\hypertarget{site-index}{%
\subsection{Site Index}\label{site-index}}

\hypertarget{site-information-navigation}{%
\subsection{Site Information
Navigation}\label{site-information-navigation}}

\begin{itemize}
\tightlist
\item
  \href{https://help.nytimes3xbfgragh.onion/hc/en-us/articles/115014792127-Copyright-notice}{©~2020~The
  New York Times Company}
\end{itemize}

\begin{itemize}
\tightlist
\item
  \href{https://www.nytco.com/}{NYTCo}
\item
  \href{https://help.nytimes3xbfgragh.onion/hc/en-us/articles/115015385887-Contact-Us}{Contact
  Us}
\item
  \href{https://www.nytco.com/careers/}{Work with us}
\item
  \href{https://nytmediakit.com/}{Advertise}
\item
  \href{http://www.tbrandstudio.com/}{T Brand Studio}
\item
  \href{https://www.nytimes3xbfgragh.onion/privacy/cookie-policy\#how-do-i-manage-trackers}{Your
  Ad Choices}
\item
  \href{https://www.nytimes3xbfgragh.onion/privacy}{Privacy}
\item
  \href{https://help.nytimes3xbfgragh.onion/hc/en-us/articles/115014893428-Terms-of-service}{Terms
  of Service}
\item
  \href{https://help.nytimes3xbfgragh.onion/hc/en-us/articles/115014893968-Terms-of-sale}{Terms
  of Sale}
\item
  \href{https://spiderbites.nytimes3xbfgragh.onion}{Site Map}
\item
  \href{https://help.nytimes3xbfgragh.onion/hc/en-us}{Help}
\item
  \href{https://www.nytimes3xbfgragh.onion/subscription?campaignId=37WXW}{Subscriptions}
\end{itemize}
