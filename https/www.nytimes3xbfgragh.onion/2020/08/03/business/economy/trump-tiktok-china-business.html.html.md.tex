Sections

SEARCH

\protect\hyperlink{site-content}{Skip to
content}\protect\hyperlink{site-index}{Skip to site index}

\href{https://www.nytimes3xbfgragh.onion/section/business/economy}{Economy}

\href{https://myaccount.nytimes3xbfgragh.onion/auth/login?response_type=cookie\&client_id=vi}{}

\href{https://www.nytimes3xbfgragh.onion/section/todayspaper}{Today's
Paper}

\href{/section/business/economy}{Economy}\textbar{}TikTok, Trump and an
Impulse to Act as C.E.O. to Corporate America

\url{https://nyti.ms/2Deypss}

\begin{itemize}
\item
\item
\item
\item
\item
\item
\end{itemize}

Advertisement

\protect\hyperlink{after-top}{Continue reading the main story}

Supported by

\protect\hyperlink{after-sponsor}{Continue reading the main story}

\hypertarget{tiktok-trump-and-an-impulse-to-act-as-ceo-to-corporate-america}{%
\section{TikTok, Trump and an Impulse to Act as C.E.O. to Corporate
America}\label{tiktok-trump-and-an-impulse-to-act-as-ceo-to-corporate-america}}

The president's interventions in company dealings based on his own
instincts are a departure from the arm's-length approach of predecessors
of either party.

\includegraphics{https://static01.graylady3jvrrxbe.onion/images/2020/08/03/business/03DC-Trump-CEO-01/merlin_175268121_b5368dd6-a785-47e2-b64d-e6e4863de9b0-articleLarge.jpg?quality=75\&auto=webp\&disable=upscale}

\href{https://www.nytimes3xbfgragh.onion/by/ana-swanson}{\includegraphics{https://static01.graylady3jvrrxbe.onion/images/2018/12/10/multimedia/author-ana-swanson/author-ana-swanson-thumbLarge.png}}\href{https://www.nytimes3xbfgragh.onion/by/michael-d-shear}{\includegraphics{https://static01.graylady3jvrrxbe.onion/images/2018/06/13/multimedia/author-michael-d-shear/author-michael-d-shear-thumbLarge-v2.png}}

By \href{https://www.nytimes3xbfgragh.onion/by/ana-swanson}{Ana Swanson}
and \href{https://www.nytimes3xbfgragh.onion/by/michael-d-shear}{Michael
D. Shear}

\begin{itemize}
\item
  Aug. 3, 2020
\item
  \begin{itemize}
  \item
  \item
  \item
  \item
  \item
  \item
  \end{itemize}
\end{itemize}

\href{https://cn.nytimes3xbfgragh.onion/business/20200804/trump-tiktok-china-business/}{阅读简体中文版}\href{https://cn.nytimes3xbfgragh.onion/business/20200804/trump-tiktok-china-business/zh-hant/}{閱讀繁體中文版}

WASHINGTON --- President Trump campaigned on a promise to run the
economy like his business empire. And for almost four years, he has
unabashedly wielded the power of the presidency to insert himself into
corporate affairs, helping some companies and punishing others in line
with his instincts and inclinations.

The latest target of his attention is TikTok, the Chinese-owned social
media app under scrutiny for potentially providing the Chinese
government with access to American user data. After threatening on
Friday to ban the app from the United States, Mr. Trump
\href{https://www.nytimes3xbfgragh.onion/2020/08/03/technology/trump-tiktok-microsoft.html}{reversed
course}, saying he would allow TikTok to keep operating if it were sold
to an American owner.

At the White House on Monday, Mr. Trump said that TikTok would be shut
down in the United States on Sept. 15 unless Microsoft or another ``very
American'' company purchased it, and that he had told Microsoft's chief
executive in a call over the weekend to ``go ahead'' with the
acquisition.

He also argued that the United States should receive money in return for
letting the deal happen, without explaining how that would work. ``A
very substantial portion of that price is going to have to come into the
Treasury of the United States, because we're making it possible for this
deal to happen,'' he said.

Given the national security concerns, Mr. Trump had the right to sign
off on a plan to mitigate any risks TikTok posed. But the events
followed a pattern that Mr. Trump set early on in his presidency, in
which some of the world's most powerful companies have found themselves
at his whims.

Daniel Price, a former economics adviser to President George W. Bush,
said Mr. Trump's reversal on TikTok was ``just another example of the
president's undisciplined and impulsive decision-making style, so
bewildering to friend and foe alike.''

``China presents serious security and economic challenges,'' Mr. Price
said. ``But Trump's erratic oscillation from adoration to demonization
has certainly harmed U.S. business interests, and actually diminished
our ability to influence China or rally allies to assist in that
effort.''

Unlike his predecessors, Mr. Trump has frequently waded in to berate or
praise executives and try to influence their operations. He attacked
Carrier and General Motors over plant-closing decisions, badgered Boeing
to lower prices and used Chinese companies as bargaining chips in
negotiations with Beijing.

While past Republican administrations disapproved of government
intervention in the market, Mr. Trump has had no qualms about taking a
heavier hand, favoring industrial policy and
\href{https://www.nytimes3xbfgragh.onion/2019/12/17/business/trump-trade-deals-free-markets.html}{a
more managed approach to trade}.

And when a company's fate is at stake because of government actions ---
as when the Clinton administration filed an antitrust case against
Microsoft, saying it threatened innovation in the nascent internet ---
presidents have usually kept their involvement at arm's length to avoid
charges of political interference.

Mr. Trump has not. He has particularly taken aim at multinational
companies that he says have made fools of past American policymakers.

He signaled his approach even as a candidate. When United Technologies
decided to close its Carrier subsidiary's plant in Indianapolis in 2016
and move furnace production to Mexico, Mr. Trump seized on the incident,
asserting that only he could get companies to stop moving jobs abroad.
He threatened to hit Carrier furnaces from Mexico with 35 percent
tariffs and promised to call the company's executives. In the end, he
predicted, they would capitulate.

As it turned out, saving jobs wasn't as easy as he promised. In exchange
for \$7 million in tax breaks, Carrier kept the plant open and invested
\$16 million in new equipment. But barely half of the 1,350 blue-collar
workers in Indianapolis kept their jobs.

Other corporate leaders have felt the heat. Just weeks after his
election, Mr. Trump strong-armed Boeing into lowering the price of a new
Air Force One, declaring that the plane's costs were ``out of control''
and signaling that he would upend yearslong negotiations.

``\href{https://www.nytimes3xbfgragh.onion/2016/12/06/us/politics/trump-air-force-one-boeing.html}{Cancel
order!}'' he tweeted.

Since then, Mr. Trump has singled out several companies for
confrontation, driven in some cases by personal pique.

He has repeatedly attacked what he calls the ``Amazon Washington Post''
and Jeff Bezos, the Amazon founder who also owns the newspaper. He has
said his yearslong assault on the Postal Service is based on his belief
that the government does not charge Amazon enough to ship its packages.

Mr. Trump's antipathy toward many news organizations has led him to
repeatedly threaten to interfere with media companies' operations. He
twice urged regulators to examine taking away the ``license'' from NBC,
though it was unclear what license he was referring to. He declared as a
candidate that he would not approve AT\&T's acquisition of Time Warner
because the company owned CNN, a network he frequently accuses of
treating him unfairly, and the Justice Department
\href{https://www.nytimes3xbfgragh.onion/2017/11/20/business/dealbook/att-time-warner-merger.html}{later
sued} unsuccessfully to block the deal.

He has also lashed out at companies and their executives for perceived
failures in responding to his desires. After Kenneth C. Frazier, the
chief executive of Merck Pharmaceuticals,
\href{https://www.nytimes3xbfgragh.onion/2017/08/14/business/merck-trump-ceos.html}{resigned
from a presidential advisory council} over Mr. Trump's handling of
violent white nationalist protests in Charlottesville, Va., the
president took after him on Twitter for ``RIPOFF DRUG PRICES.''

Mr. Trump denounced General Motors for closing a car factory in
Lordstown, Ohio, and three other plants in the United States, and
attacked its chief executive, Mary T. Barra, by name. Later, with the
onset of the coronavirus crisis, Mr. Trump criticized Ms. Barra for what
he said was the company's failure to make good on a promise to help make
ventilators.

``Always a mess with Mary B,'' he wrote on Twitter.

``He's been doing this from the outset, using his power to try to
influence corporate deals,'' said Richard W. Painter, a professor at the
University of Minnesota Law School. ``Being president is not the art of
the deal. He's not in a boardroom. He's in the White House.''

But Mr. Trump's efforts to dictate corporate decisions have been
inconsistent, making it harder for executives to anticipate White House
demands or reactions.

As he found himself on the defensive this spring in his handling of the
coronavirus pandemic, Mr. Trump resisted calls to use the Defense
Production Act to pressure industries to make more masks and medical
supplies, saying that such a move would be akin to
\href{https://www.nytimes3xbfgragh.onion/2020/03/31/us/politics/coronavirus-defense-production-act.html}{``nationalizing
our business''} and that the government
\href{https://www.nytimes3xbfgragh.onion/2020/03/20/us/politics/trump-coronavirus-supplies.html}{``was
not a shipping clerk.''}

And even with China, which many in Washington have accused of gaming
America's free-market system by stealing intellectual property and
cheating on trade rules, Mr. Trump has not always intervened to take a
tougher line.

In 2018, he
\href{https://www.nytimes3xbfgragh.onion/2018/06/07/business/us-china-zte-deal.html}{lifted
tough sanctions against the Chinese telecommunications firm ZTE}, over
the objections of Republican lawmakers and his own national security
advisers, in an attempt to win China's help in negotiating with North
Korea. He has alternated between condemning another Chinese technology
giant, Huawei, as a grave security threat and
\href{https://www.nytimes3xbfgragh.onion/2019/11/15/business/us-reprieve-huawei.html}{holding
off on acting against it} in hopes of securing a trade deal.

\includegraphics{https://static01.graylady3jvrrxbe.onion/images/2020/08/03/business/03DC-Trump-CEO-03/merlin_151294254_bd86d497-a4ca-4bf6-95c3-03dacca5a733-articleLarge.jpg?quality=75\&auto=webp\&disable=upscale}

The president's back-and-forth on TikTok offers a new illustration of
how he has made national security decisions by impulse.

A national security panel, called the Committee on Foreign Investment in
the United States, recommended to the president last week that TikTok
sell its assets to an American company to curtail China's potential
influence in the United States, and
\href{https://www.nytimes3xbfgragh.onion/2020/07/31/technology/tiktok-microsoft.html}{Microsoft
had stepped forward} as a potential buyer.

But several China hawks in the Trump administration, including the White
House trade adviser Peter Navarro, argued against the sale, seeing the
moment as an opportunity to take more sweeping action against TikTok and
other Chinese-run internet services.

Mr. Trump took Mr. Navarro's side on Friday, saying that he did not
favor a sale of TikTok and that he planned to ban the app. But after
\href{https://www.nytimes3xbfgragh.onion/2020/08/02/business/economy/trump-tiktok-china-national-security.html}{a
series of calls}, including ones from Senator Lindsey Graham, Republican
of South Carolina, and Satya Nadella, the chief executive of Microsoft,
Mr. Trump appeared to change his mind.

Several of Mr. Trump's aides had warned that a ban could prompt an
intense legal battle, as well as hurt the president's popularity with
younger Americans. TikTok has said it is used by 100 million Americans.

Mr. Trump appeared to object to TikTok's sale in part because it would
funnel money back to China. Speaking to reporters on Monday, the
president argued that the United States should also receive money in
return for permitting the deal to happen, because Microsoft would not
have the right to make the acquisition ``unless we give it to them.''

Explaining his views to reporters, Mr. Trump drew a parallel to his days
in real estate development.

``It's a little bit like the landlord-tenant,'' the president said.
``Without a lease, the tenant has nothing. So they pay what's called key
money.''

``The United States should be reimbursed, or should be paid a
substantial amount of money,'' Mr. Trump said, ``because without the
United States, they don't have anything.''

Neal E. Boudette contributed reporting from Ann Arbor, Mich., Mike Isaac
from San Francisco and Nelson D. Schwartz from New York.

Advertisement

\protect\hyperlink{after-bottom}{Continue reading the main story}

\hypertarget{site-index}{%
\subsection{Site Index}\label{site-index}}

\hypertarget{site-information-navigation}{%
\subsection{Site Information
Navigation}\label{site-information-navigation}}

\begin{itemize}
\tightlist
\item
  \href{https://help.nytimes3xbfgragh.onion/hc/en-us/articles/115014792127-Copyright-notice}{©~2020~The
  New York Times Company}
\end{itemize}

\begin{itemize}
\tightlist
\item
  \href{https://www.nytco.com/}{NYTCo}
\item
  \href{https://help.nytimes3xbfgragh.onion/hc/en-us/articles/115015385887-Contact-Us}{Contact
  Us}
\item
  \href{https://www.nytco.com/careers/}{Work with us}
\item
  \href{https://nytmediakit.com/}{Advertise}
\item
  \href{http://www.tbrandstudio.com/}{T Brand Studio}
\item
  \href{https://www.nytimes3xbfgragh.onion/privacy/cookie-policy\#how-do-i-manage-trackers}{Your
  Ad Choices}
\item
  \href{https://www.nytimes3xbfgragh.onion/privacy}{Privacy}
\item
  \href{https://help.nytimes3xbfgragh.onion/hc/en-us/articles/115014893428-Terms-of-service}{Terms
  of Service}
\item
  \href{https://help.nytimes3xbfgragh.onion/hc/en-us/articles/115014893968-Terms-of-sale}{Terms
  of Sale}
\item
  \href{https://spiderbites.nytimes3xbfgragh.onion}{Site Map}
\item
  \href{https://help.nytimes3xbfgragh.onion/hc/en-us}{Help}
\item
  \href{https://www.nytimes3xbfgragh.onion/subscription?campaignId=37WXW}{Subscriptions}
\end{itemize}
