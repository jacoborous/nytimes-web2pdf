Sections

SEARCH

\protect\hyperlink{site-content}{Skip to
content}\protect\hyperlink{site-index}{Skip to site index}

\href{https://www.nytimes3xbfgragh.onion/section/politics}{Politics}

\href{https://myaccount.nytimes3xbfgragh.onion/auth/login?response_type=cookie\&client_id=vi}{}

\href{https://www.nytimes3xbfgragh.onion/section/todayspaper}{Today's
Paper}

\href{/section/politics}{Politics}\textbar{}White House and Congress
Remain Far From Any Stimulus Deal

\url{https://nyti.ms/2DARQM0}

\begin{itemize}
\item
\item
\item
\item
\item
\end{itemize}

\hypertarget{the-coronavirus-outbreak}{%
\subsubsection{\texorpdfstring{\href{https://www.nytimes3xbfgragh.onion/news-event/coronavirus?name=styln-coronavirus-national\&region=TOP_BANNER\&variant=undefined\&block=storyline_menu_recirc\&action=click\&pgtype=Article\&impression_id=a2abedf0-e38a-11ea-8313-f9307944f60a}{The
Coronavirus
Outbreak}}{The Coronavirus Outbreak}}\label{the-coronavirus-outbreak}}

\begin{itemize}
\tightlist
\item
  live\href{https://www.nytimes3xbfgragh.onion/2020/08/20/world/coronavirus-covid.html?name=styln-coronavirus-national\&region=TOP_BANNER\&variant=undefined\&block=storyline_menu_recirc\&action=click\&pgtype=Article\&impression_id=a2abedf1-e38a-11ea-8313-f9307944f60a}{Latest
  Updates}
\item
  \href{https://www.nytimes3xbfgragh.onion/interactive/2020/us/coronavirus-us-cases.html?name=styln-coronavirus-national\&region=TOP_BANNER\&variant=undefined\&block=storyline_menu_recirc\&action=click\&pgtype=Article\&impression_id=a2abedf2-e38a-11ea-8313-f9307944f60a}{Maps
  and Cases}
\item
  \href{https://www.nytimes3xbfgragh.onion/interactive/2020/science/coronavirus-vaccine-tracker.html?name=styln-coronavirus-national\&region=TOP_BANNER\&variant=undefined\&block=storyline_menu_recirc\&action=click\&pgtype=Article\&impression_id=a2ac1500-e38a-11ea-8313-f9307944f60a}{Vaccine
  Tracker}
\item
  \href{https://www.nytimes3xbfgragh.onion/2020/08/19/us/colleges-closing-covid.html?name=styln-coronavirus-national\&region=TOP_BANNER\&variant=undefined\&block=storyline_menu_recirc\&action=click\&pgtype=Article\&impression_id=a2ac1501-e38a-11ea-8313-f9307944f60a}{Colleges
  Closing}
\item
  \href{https://www.nytimes3xbfgragh.onion/live/2020/08/20/business/stock-market-today-coronavirus?name=styln-coronavirus-national\&region=TOP_BANNER\&variant=undefined\&block=storyline_menu_recirc\&action=click\&pgtype=Article\&impression_id=a2ac1502-e38a-11ea-8313-f9307944f60a}{Economy}
\end{itemize}

Advertisement

\protect\hyperlink{after-top}{Continue reading the main story}

Supported by

\protect\hyperlink{after-sponsor}{Continue reading the main story}

\hypertarget{white-house-and-congress-remain-far-from-any-stimulus-deal}{%
\section{White House and Congress Remain Far From Any Stimulus
Deal}\label{white-house-and-congress-remain-far-from-any-stimulus-deal}}

White House officials and Democratic leaders continued to dig in on
crucial points of any potential deal, jeopardizing additional relief for
small businesses and laid-off workers.

\includegraphics{https://static01.graylady3jvrrxbe.onion/images/2020/08/05/us/politics/05dc-virus-cong/merlin_175344024_07b14c49-ffbb-4576-a213-91e0f41de2dc-articleLarge.jpg?quality=75\&auto=webp\&disable=upscale}

\href{https://www.nytimes3xbfgragh.onion/by/emily-cochrane}{\includegraphics{https://static01.graylady3jvrrxbe.onion/images/2018/11/28/multimedia/author-emily-cochrane/author-emily-cochrane-thumbLarge-v3.png}}\href{https://www.nytimes3xbfgragh.onion/by/jim-tankersley}{\includegraphics{https://static01.graylady3jvrrxbe.onion/images/2018/10/19/multimedia/author-jim-tankersley/author-jim-tankersley-thumbLarge.png}}

By \href{https://www.nytimes3xbfgragh.onion/by/emily-cochrane}{Emily
Cochrane} and
\href{https://www.nytimes3xbfgragh.onion/by/jim-tankersley}{Jim
Tankersley}

\begin{itemize}
\item
  Published Aug. 5, 2020Updated Aug. 17, 2020
\item
  \begin{itemize}
  \item
  \item
  \item
  \item
  \item
  \end{itemize}
\end{itemize}

WASHINGTON --- Top Democrats and White House officials on Wednesday
remained nowhere close to an agreement for a new rescue package to
address the coronavirus's toll on the economy, growing increasingly
pessimistic that they could meet a self-imposed Friday deadline as
President Trump again threatened to act on his own to provide relief.

Even as they vowed to continue talks, negotiators remained dug in on
crucial points of any potential deal, jeopardizing additional relief for
small businesses and laid-off workers --- and all but guaranteeing that
senators who had planned to go home for a scheduled recess next week
would instead stay in Washington awaiting a deal.

Given the number of outstanding policy issues, including the revival of
expanded
\href{https://www.nytimes3xbfgragh.onion/2020/08/06/business/economy/unemployment-claims.html}{unemployment
benefits} and Mr. Trump's rejection of a key Democratic demand for
nearly \$1 trillion for struggling state and local governments, the
prospect of votes on such a package next week appeared remote.

``I feel optimistic that there is a light at the end of the tunnel,''
Speaker Nancy Pelosi of California said after hosting another round of
talks in her Capitol Hill office with Steven Mnuchin, the Treasury
secretary, Mark Meadows, the White House chief of staff, and Senator
Chuck Schumer of New York, the minority leader. ``But how long that
tunnel is remains to be seen.''

Senator Mitch McConnell, Republican of Kentucky and the majority leader,
told reporters on Wednesday that the Senate would ``certainly be in next
week,'' delaying the beginning of the recess in a bid to produce a
legislative framework in the coming days.

Every day of delay risks further damage to an economic recovery that has
stalled --- and, by some measures, begun to regress --- as the number of
cases and deaths from the coronavirus continues to surge in the United
States. Economic forecasters were bracing for the Labor Department's
monthly jobs report on Friday to show a significant deceleration in
hiring from May and June. Any additional help for people and businesses
that lawmakers approve in a new package, including a resumption of
expanded unemployment benefits that have lapsed, could take weeks to
make its way into the economy once Mr. Trump signs a new bill.

``There are no top-line numbers that have been agreed to,'' Mr. Meadows
said after the meeting, charging that Democrats were unwilling to make
significant concessions. ``We continue to be trillions of dollars apart
in terms of what Democrats and Republicans hopefully will ultimately
compromise on.''

\hypertarget{latest-updates-the-coronavirus-outbreak}{%
\section{\texorpdfstring{\href{https://www.nytimes3xbfgragh.onion/2020/08/20/world/coronavirus-covid.html?action=click\&pgtype=Article\&state=default\&region=MAIN_CONTENT_1\&context=storylines_live_updates}{Latest
Updates: The Coronavirus
Outbreak}}{Latest Updates: The Coronavirus Outbreak}}\label{latest-updates-the-coronavirus-outbreak}}

Updated 2020-08-21T07:46:15.883Z

\begin{itemize}
\tightlist
\item
  \href{https://www.nytimes3xbfgragh.onion/2020/08/20/world/coronavirus-covid.html?action=click\&pgtype=Article\&state=default\&region=MAIN_CONTENT_1\&context=storylines_live_updates\#link-68774d88}{Shutdowns,
  warnings and scoldings follow alarming incidents on college campuses.}
\item
  \href{https://www.nytimes3xbfgragh.onion/2020/08/20/world/coronavirus-covid.html?action=click\&pgtype=Article\&state=default\&region=MAIN_CONTENT_1\&context=storylines_live_updates\#link-26b58724}{Biden
  knocks Trump's pandemic response, and outlines a national strategy.}
\item
  \href{https://www.nytimes3xbfgragh.onion/2020/08/20/world/coronavirus-covid.html?action=click\&pgtype=Article\&state=default\&region=MAIN_CONTENT_1\&context=storylines_live_updates\#link-4e542da3}{U.S.
  health agencies announce moves to confront the flu season and
  plummeting child vaccination rates.}
\end{itemize}

\href{https://www.nytimes3xbfgragh.onion/2020/08/20/world/coronavirus-covid.html?action=click\&pgtype=Article\&state=default\&region=MAIN_CONTENT_1\&context=storylines_live_updates}{See
more updates}

More live coverage:
\href{https://www.nytimes3xbfgragh.onion/live/2020/08/20/business/stock-market-today-coronavirus?action=click\&pgtype=Article\&state=default\&region=MAIN_CONTENT_1\&context=storylines_live_updates}{Markets}

``Is Friday a drop-dead date? No,'' he added. ``But my optimism
continues to diminish the closer we get to Friday and certainly falls
off the cliff exponentially after Friday.''

Barring a compromise, Mr. Trump and his top lieutenants on Wednesday
continued to explore the possibility of taking executive action to
address some of the unresolved disputes. Those included reinstating a
weekly federal unemployment benefit that lapsed on Friday, reviving a
federal moratorium on evictions and imposing a payroll tax cut that has
been rejected by lawmakers in both parties.

It is unclear whether Mr. Trump has the legal authority to force the
changes he wants without the consent of Congress. Democrats have sued to
block Mr. Trump from repurposing federal funds for construction of his
border wall. It is also not certain that the orders would work to
bolster the economy as Mr. Trump hopes. For example, companies might not
pass the savings of a suspended payroll tax on to their employees, and
instead continue to withhold them in the event that the tax must be
repaid next year.

\includegraphics{https://static01.graylady3jvrrxbe.onion/images/2020/08/05/us/politics/05dc-virus-cong2/merlin_175354323_d92e0fc0-76b2-4a5c-acd6-97a934f59e91-articleLarge.jpg?quality=75\&auto=webp\&disable=upscale}

``If we can reach a compromise on these big issues, I think everything
else will fall into place,'' Mr. Mnuchin said after briefing Mr.
McConnell on the latest meeting. ``If we can't reach an agreement on
these big issues, then I don't see us coming to an overall deal and then
we'll have to look at the president taking actions under his executive
authority.''

On Wednesday, disputes over funding for the Postal Service also emerged
as a sticking point between Democratic leaders and the Trump
administration, as top officials huddled with the postmaster general,
Louis DeJoy, for more than an hour as part of their negotiations.

Mr. Schumer described a ``heated discussion'' with Mr. DeJoy, who he
said had ignored multiple phone calls over concerns about slow mail
delivery in New York. Democrats and voting rights groups
\href{https://slack-redir.net/link?url=https\%3A\%2F\%2Fwww.nytimes3xbfgragh.onion\%2F2020\%2F07\%2F31\%2Fus\%2Fpolitics\%2Ftrump-usps-mail-delays.html}{have
charged} that cutbacks Mr. DeJoy has put into place are part of a
deliberate effort by Mr. Trump to undermine the Postal Service in an
effort to interfere with
\href{https://www.nytimes3xbfgragh.onion/interactive/2020/08/11/us/politics/vote-by-mail-us-states.html}{mail-in
voting} that will be critical to a safe election in November.

``We told him that elections are sacred and to do cutbacks, at a time
when all ballots have to count --- you can't say, `Whoa, we'll get 94
percent' --- is insufficient,'' Mr. Schumer said after the meeting. ``We
are demanding that the regulations that are put in place, which cut
employment over time, be rescinded, particularly because of Covid and
because of the elections.''

\href{https://www.nytimes3xbfgragh.onion/news-event/coronavirus?action=click\&pgtype=Article\&state=default\&region=MAIN_CONTENT_3\&context=storylines_faq}{}

\hypertarget{the-coronavirus-outbreak-}{%
\subsubsection{The Coronavirus Outbreak
›}\label{the-coronavirus-outbreak-}}

\hypertarget{frequently-asked-questions}{%
\paragraph{Frequently Asked
Questions}\label{frequently-asked-questions}}

Updated August 17, 2020

\begin{itemize}
\item ~
  \hypertarget{why-does-standing-six-feet-away-from-others-help}{%
  \paragraph{Why does standing six feet away from others
  help?}\label{why-does-standing-six-feet-away-from-others-help}}

  \begin{itemize}
  \tightlist
  \item
    The coronavirus spreads primarily through droplets from your mouth
    and nose, especially when you cough or sneeze. The C.D.C., one of
    the organizations using that measure,
    \href{https://www.nytimes3xbfgragh.onion/2020/04/14/health/coronavirus-six-feet.html?action=click\&pgtype=Article\&state=default\&region=MAIN_CONTENT_3\&context=storylines_faq}{bases
    its recommendation of six feet} on the idea that most large droplets
    that people expel when they cough or sneeze will fall to the ground
    within six feet. But six feet has never been a magic number that
    guarantees complete protection. Sneezes, for instance, can launch
    droplets a lot farther than six feet,
    \href{https://jamanetwork.com/journals/jama/fullarticle/2763852}{according
    to a recent study}. It's a rule of thumb: You should be safest
    standing six feet apart outside, especially when it's windy. But
    keep a mask on at all times, even when you think you're far enough
    apart.
  \end{itemize}
\item ~
  \hypertarget{i-have-antibodies-am-i-now-immune}{%
  \paragraph{I have antibodies. Am I now
  immune?}\label{i-have-antibodies-am-i-now-immune}}

  \begin{itemize}
  \tightlist
  \item
    As of right
    now,\href{https://www.nytimes3xbfgragh.onion/2020/07/22/health/covid-antibodies-herd-immunity.html?action=click\&pgtype=Article\&state=default\&region=MAIN_CONTENT_3\&context=storylines_faq}{that
    seems likely, for at least several months.} There have been
    frightening accounts of people suffering what seems to be a second
    bout of Covid-19. But experts say these patients may have a
    drawn-out course of infection, with the virus taking a slow toll
    weeks to months after initial exposure. People infected with the
    coronavirus typically
    \href{https://www.nature.com/articles/s41586-020-2456-9}{produce}
    immune molecules called antibodies, which are
    \href{https://www.nytimes3xbfgragh.onion/2020/05/07/health/coronavirus-antibody-prevalence.html?action=click\&pgtype=Article\&state=default\&region=MAIN_CONTENT_3\&context=storylines_faq}{protective
    proteins made in response to an
    infection}\href{https://www.nytimes3xbfgragh.onion/2020/05/07/health/coronavirus-antibody-prevalence.html?action=click\&pgtype=Article\&state=default\&region=MAIN_CONTENT_3\&context=storylines_faq}{.
    These antibodies may} last in the body
    \href{https://www.nature.com/articles/s41591-020-0965-6}{only two to
    three months}, which may seem worrisome, but that's perfectly normal
    after an acute infection subsides, said Dr. Michael Mina, an
    immunologist at Harvard University. It may be possible to get the
    coronavirus again, but it's highly unlikely that it would be
    possible in a short window of time from initial infection or make
    people sicker the second time.
  \end{itemize}
\item ~
  \hypertarget{im-a-small-business-owner-can-i-get-relief}{%
  \paragraph{I'm a small-business owner. Can I get
  relief?}\label{im-a-small-business-owner-can-i-get-relief}}

  \begin{itemize}
  \tightlist
  \item
    The
    \href{https://www.nytimes3xbfgragh.onion/article/small-business-loans-stimulus-grants-freelancers-coronavirus.html?action=click\&pgtype=Article\&state=default\&region=MAIN_CONTENT_3\&context=storylines_faq}{stimulus
    bills enacted in March} offer help for the millions of American
    small businesses. Those eligible for aid are businesses and
    nonprofit organizations with fewer than 500 workers, including sole
    proprietorships, independent contractors and freelancers. Some
    larger companies in some industries are also eligible. The help
    being offered, which is being managed by the Small Business
    Administration, includes the Paycheck Protection Program and the
    Economic Injury Disaster Loan program. But lots of folks have
    \href{https://www.nytimes3xbfgragh.onion/interactive/2020/05/07/business/small-business-loans-coronavirus.html?action=click\&pgtype=Article\&state=default\&region=MAIN_CONTENT_3\&context=storylines_faq}{not
    yet seen payouts.} Even those who have received help are confused:
    The rules are draconian, and some are stuck sitting on
    \href{https://www.nytimes3xbfgragh.onion/2020/05/02/business/economy/loans-coronavirus-small-business.html?action=click\&pgtype=Article\&state=default\&region=MAIN_CONTENT_3\&context=storylines_faq}{money
    they don't know how to use.} Many small-business owners are getting
    less than they expected or
    \href{https://www.nytimes3xbfgragh.onion/2020/06/10/business/Small-business-loans-ppp.html?action=click\&pgtype=Article\&state=default\&region=MAIN_CONTENT_3\&context=storylines_faq}{not
    hearing anything at all.}
  \end{itemize}
\item ~
  \hypertarget{what-are-my-rights-if-i-am-worried-about-going-back-to-work}{%
  \paragraph{What are my rights if I am worried about going back to
  work?}\label{what-are-my-rights-if-i-am-worried-about-going-back-to-work}}

  \begin{itemize}
  \tightlist
  \item
    Employers have to provide
    \href{https://www.osha.gov/SLTC/covid-19/standards.html}{a safe
    workplace} with policies that protect everyone equally.
    \href{https://www.nytimes3xbfgragh.onion/article/coronavirus-money-unemployment.html?action=click\&pgtype=Article\&state=default\&region=MAIN_CONTENT_3\&context=storylines_faq}{And
    if one of your co-workers tests positive for the coronavirus, the
    C.D.C.} has said that
    \href{https://www.cdc.gov/coronavirus/2019-ncov/community/guidance-business-response.html}{employers
    should tell their employees} -\/- without giving you the sick
    employee's name -\/- that they may have been exposed to the virus.
  \end{itemize}
\item ~
  \hypertarget{what-is-school-going-to-look-like-in-september}{%
  \paragraph{What is school going to look like in
  September?}\label{what-is-school-going-to-look-like-in-september}}

  \begin{itemize}
  \tightlist
  \item
    It is unlikely that many schools will return to a normal schedule
    this fall, requiring the grind of
    \href{https://www.nytimes3xbfgragh.onion/2020/06/05/us/coronavirus-education-lost-learning.html?action=click\&pgtype=Article\&state=default\&region=MAIN_CONTENT_3\&context=storylines_faq}{online
    learning},
    \href{https://www.nytimes3xbfgragh.onion/2020/05/29/us/coronavirus-child-care-centers.html?action=click\&pgtype=Article\&state=default\&region=MAIN_CONTENT_3\&context=storylines_faq}{makeshift
    child care} and
    \href{https://www.nytimes3xbfgragh.onion/2020/06/03/business/economy/coronavirus-working-women.html?action=click\&pgtype=Article\&state=default\&region=MAIN_CONTENT_3\&context=storylines_faq}{stunted
    workdays} to continue. California's two largest public school
    districts --- Los Angeles and San Diego --- said on July 13, that
    \href{https://www.nytimes3xbfgragh.onion/2020/07/13/us/lausd-san-diego-school-reopening.html?action=click\&pgtype=Article\&state=default\&region=MAIN_CONTENT_3\&context=storylines_faq}{instruction
    will be remote-only in the fall}, citing concerns that surging
    coronavirus infections in their areas pose too dire a risk for
    students and teachers. Together, the two districts enroll some
    825,000 students. They are the largest in the country so far to
    abandon plans for even a partial physical return to classrooms when
    they reopen in August. For other districts, the solution won't be an
    all-or-nothing approach.
    \href{https://bioethics.jhu.edu/research-and-outreach/projects/eschool-initiative/school-policy-tracker/}{Many
    systems}, including the nation's largest, New York City, are
    devising
    \href{https://www.nytimes3xbfgragh.onion/2020/06/26/us/coronavirus-schools-reopen-fall.html?action=click\&pgtype=Article\&state=default\&region=MAIN_CONTENT_3\&context=storylines_faq}{hybrid
    plans} that involve spending some days in classrooms and other days
    online. There's no national policy on this yet, so check with your
    municipal school system regularly to see what is happening in your
    community.
  \end{itemize}
\end{itemize}

Democrats are pushing for \$10 billion to be allocated to the agency
over a year, instead of their original proposal for distributing \$25
billion over three years. They have also proposed additional money for
food assistance programs, money for child care, and more than \$900
billion to help states and local governments avoid laying off public
workers as tax revenues fall. Administration officials have offered
\$150 billion in state and local aid, and on Wednesday, Mr. Trump said
he opposed any such money.

``We can't go along with the bailout money,'' he told reporters at the
White House. ``We're not going to go along with that.''

The fate of a \$600-per-week federal unemployment supplement to laid-off
workers, which lapsed last week in the absence of an agreement to extend
them, also remains another significant point of contention. Senate
Republicans want to slash the benefit.

Democrats are pressing to extend the payments through January. On
Tuesday, Republicans countered with a plan to resume them at \$400 per
week through Dec. 15, according to two people with knowledge of the
discussions who spoke on the condition of anonymity to describe them.
Democrats declined the offer, which was first reported by Politico.

Some Senate Republicans, largely removed from the process, have begun
discussing the possibility of holding procedural votes on individual
proposals, forcing Democrats to block them. One of those votes could be
an extension of the Paycheck Protection Program, a popular federal
small-business loan program, which stops taking applications at the end
of the week.

News of a self-imposed deadline did not completely assure senators that
a deal was to be had, though some Republicans said it could compel some
sort of compromise.

``At some point, you have to set a deadline, or just continue this
Kabuki dance every day,'' said Senator Roy Blunt, Republican of
Missouri. ``Nobody wants to do that.''

``There's plenty of time to get a deal if there's a deal to be gotten,''
he added. ``If there's not a deal to be gotten, there's no reason to
continue to act like there is.''

Advertisement

\protect\hyperlink{after-bottom}{Continue reading the main story}

\hypertarget{site-index}{%
\subsection{Site Index}\label{site-index}}

\hypertarget{site-information-navigation}{%
\subsection{Site Information
Navigation}\label{site-information-navigation}}

\begin{itemize}
\tightlist
\item
  \href{https://help.nytimes3xbfgragh.onion/hc/en-us/articles/115014792127-Copyright-notice}{©~2020~The
  New York Times Company}
\end{itemize}

\begin{itemize}
\tightlist
\item
  \href{https://www.nytco.com/}{NYTCo}
\item
  \href{https://help.nytimes3xbfgragh.onion/hc/en-us/articles/115015385887-Contact-Us}{Contact
  Us}
\item
  \href{https://www.nytco.com/careers/}{Work with us}
\item
  \href{https://nytmediakit.com/}{Advertise}
\item
  \href{http://www.tbrandstudio.com/}{T Brand Studio}
\item
  \href{https://www.nytimes3xbfgragh.onion/privacy/cookie-policy\#how-do-i-manage-trackers}{Your
  Ad Choices}
\item
  \href{https://www.nytimes3xbfgragh.onion/privacy}{Privacy}
\item
  \href{https://help.nytimes3xbfgragh.onion/hc/en-us/articles/115014893428-Terms-of-service}{Terms
  of Service}
\item
  \href{https://help.nytimes3xbfgragh.onion/hc/en-us/articles/115014893968-Terms-of-sale}{Terms
  of Sale}
\item
  \href{https://spiderbites.nytimes3xbfgragh.onion}{Site Map}
\item
  \href{https://help.nytimes3xbfgragh.onion/hc/en-us}{Help}
\item
  \href{https://www.nytimes3xbfgragh.onion/subscription?campaignId=37WXW}{Subscriptions}
\end{itemize}
