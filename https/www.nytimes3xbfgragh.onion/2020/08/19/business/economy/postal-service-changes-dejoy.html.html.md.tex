Sections

SEARCH

\protect\hyperlink{site-content}{Skip to
content}\protect\hyperlink{site-index}{Skip to site index}

\href{https://www.nytimes3xbfgragh.onion/section/business/economy}{Economy}

\href{https://myaccount.nytimes3xbfgragh.onion/auth/login?response_type=cookie\&client_id=vi}{}

\href{https://www.nytimes3xbfgragh.onion/section/todayspaper}{Today's
Paper}

\href{/section/business/economy}{Economy}\textbar{}Decision to Halt
Postal Changes Does Little to Quell Election Concerns

\url{https://nyti.ms/2E0Ga61}

\begin{itemize}
\item
\item
\item
\item
\item
\end{itemize}

\begin{itemize}
\item
  \href{https://www.nytimes3xbfgragh.onion/live/2020/08/20/us/dnc-convention-election?action=click\&pgtype=Article\&state=default\&region=TOP_BANNER\&context=storylines_menu}{D.N.C.
  Updates}
\item
  \href{https://www.nytimes3xbfgragh.onion/2020/08/20/us/politics/biden-presidential-nomination-dnc.html?action=click\&pgtype=Article\&state=default\&region=TOP_BANNER\&context=storylines_menu}{Biden's
  Speech}
\item
  \href{https://www.nytimes3xbfgragh.onion/interactive/2019/us/elections/2020-presidential-election-calendar.html?action=click\&pgtype=Article\&state=default\&region=TOP_BANNER\&context=storylines_menu}{Election
  Calendar}
\item
  \href{https://www.nytimes3xbfgragh.onion/interactive/2020/08/11/us/politics/vote-by-mail-us-states.html?action=click\&pgtype=Article\&state=default\&region=TOP_BANNER\&context=storylines_menu}{Voting
  by Mail}
\item
  \href{https://www.nytimes3xbfgragh.onion/newsletters/politics?action=click\&pgtype=Article\&state=default\&region=TOP_BANNER\&context=storylines_menu}{Politics
  Newsletter}
\end{itemize}

Advertisement

\protect\hyperlink{after-top}{Continue reading the main story}

Supported by

\protect\hyperlink{after-sponsor}{Continue reading the main story}

\hypertarget{decision-to-halt-postal-changes-does-little-to-quell-election-concerns}{%
\section{Decision to Halt Postal Changes Does Little to Quell Election
Concerns}\label{decision-to-halt-postal-changes-does-little-to-quell-election-concerns}}

Democrats are calling on Louis DeJoy to step down amid concerns that
changes already made could disenfranchise voters in November.

\includegraphics{https://static01.graylady3jvrrxbe.onion/images/2020/08/19/us/politics/19dc-postal/merlin_175890477_cc46695c-7f4f-4871-b1f1-74d735ddf2e1-articleLarge.jpg?quality=75\&auto=webp\&disable=upscale}

By \href{https://www.nytimes3xbfgragh.onion/by/emily-cochrane}{Emily
Cochrane},
\href{https://www.nytimes3xbfgragh.onion/by/hailey-fuchs}{Hailey Fuchs},
\href{https://www.nytimes3xbfgragh.onion/by/kenneth-p-vogel}{Kenneth P.
Vogel} and
\href{https://www.nytimes3xbfgragh.onion/by/jessica-silver-greenberg}{Jessica
Silver-Greenberg}

\begin{itemize}
\item
  Aug. 19, 2020
\item
  \begin{itemize}
  \item
  \item
  \item
  \item
  \item
  \end{itemize}
\end{itemize}

WASHINGTON --- Louis DeJoy's move to halt changes that were viewed as a
threat to mail-in voting did little to quell the outcry over his
leadership as postmaster general, with lawmakers calling on Wednesday
for his removal and one top Democrat demanding more answers about the
secretive process that led to a major Trump donor running the Postal
Service.

Senator Chuck Schumer of New York, the Democratic leader, called on the
Postal Service board of governors to release information about the
selection process that resulted in
\href{https://www.nytimes3xbfgragh.onion/2020/05/07/us/politics/postmaster-general-louis-dejoy.html}{Mr.
DeJoy's appointment in May}, saying that the changes made under his
watch underscored the need for more details. In a separate letter to Mr.
DeJoy, Mr. Schumer also requested more information about the changes
that had been put in place and which ones would be suspended.

Mr. DeJoy, who will face lawmakers at two separate hearings in the
coming days, said on Tuesday that he would
\href{https://www.nytimes3xbfgragh.onion/2020/08/18/us/politics/postal-service-suspends-changes.html}{suspend
cost-cutting and operational changes} that have slowed mail delivery and
fueled worries about mail-in voting in the November election. But Mr.
DeJoy did not commit to reversing changes already put in place,
including the removal of hundreds of mail-sorting machines, some of
which have already been destroyed, according to union officials and
postal workers.

Speaker Nancy Pelosi of California, who spoke with Mr. DeJoy on
Wednesday, said in a statement that the postmaster general ``frankly
admitted that he had no intention of replacing the sorting machines,
blue mailboxes and other key mail infrastructure that have been removed,
and that plans for adequate overtime, which is critical for the timely
delivery of mail, are not in the works.''

Lawmakers plan to question Mr. DeJoy, who previously ran a logistics and
transportation company, at a Senate committee hearing on Friday and at a
House oversight hearing on Monday. House lawmakers are also expected to
vote Saturday on
\href{https://rules.house.gov/sites/democrats.rules.house.gov/files/BILLS-116HR8015-RCP116-61.pdf}{legislation
that would reverse the changes put into plac}e by Mr. DeJoy, prevent any
further changes before the end of the pandemic and provide \$25 billion
for the beleaguered agency, with \$15 million of that going to the
Postal Service Office of Inspector General.

Top House Republicans, who conferred on Wednesday by phone with Mark
Meadows, the White House chief of staff, are formally urging members of
the conference to vote against the measure, according to three people
familiar with the discussions.

Close to 100 Democrats called for Mr. DeJoy's removal from the position
of postmaster general, writing to the agency's board of governors that
Mr. DeJoy ``has already done considerable damage to the institution, and
we believe his conflicts of interest are insurmountable.''

\hypertarget{latest-updates-2020-election}{%
\section{\texorpdfstring{\href{https://www.nytimes3xbfgragh.onion/live/2020/08/19/us/dnc-convention-election?action=click\&pgtype=Article\&state=default\&region=MAIN_CONTENT_1\&context=storylines_live_updates}{Latest
Updates: 2020
Election}}{Latest Updates: 2020 Election}}\label{latest-updates-2020-election}}

\href{https://www.nytimes3xbfgragh.onion/live/2020/08/19/us/dnc-convention-election?action=click\&pgtype=Article\&state=default\&region=MAIN_CONTENT_1\&context=storylines_live_updates\#night-3-featured-more-policy-a-focus-on-women-and-a-full-throated-rejection-of-trump-by-his-predecessor}{7h
ago}

\href{https://www.nytimes3xbfgragh.onion/live/2020/08/19/us/dnc-convention-election?action=click\&pgtype=Article\&state=default\&region=MAIN_CONTENT_1\&context=storylines_live_updates\#night-3-featured-more-policy-a-focus-on-women-and-a-full-throated-rejection-of-trump-by-his-predecessor}{Night
3 featured more policy, a focus on women and a full-throated rejection
of Trump by his predecessor.}

\href{https://www.nytimes3xbfgragh.onion/live/2020/08/19/us/dnc-convention-election?action=click\&pgtype=Article\&state=default\&region=MAIN_CONTENT_1\&context=storylines_live_updates\#trump-live-tweeted-obamas-speech-tonight-hell-appear-on-fox-news-right-before-bidens-tomorrow}{9h
ago}

\href{https://www.nytimes3xbfgragh.onion/live/2020/08/19/us/dnc-convention-election?action=click\&pgtype=Article\&state=default\&region=MAIN_CONTENT_1\&context=storylines_live_updates\#trump-live-tweeted-obamas-speech-tonight-hell-appear-on-fox-news-right-before-bidens-tomorrow}{Trump
live-tweeted Obama's speech tonight. He'll appear on Fox News right
before Biden's tomorrow.}

\href{https://www.nytimes3xbfgragh.onion/live/2020/08/19/us/dnc-convention-election?action=click\&pgtype=Article\&state=default\&region=MAIN_CONTENT_1\&context=storylines_live_updates\#advocates-for-domestic-violence-survivors-praised-biden-in-a-video}{9h
ago}

\href{https://www.nytimes3xbfgragh.onion/live/2020/08/19/us/dnc-convention-election?action=click\&pgtype=Article\&state=default\&region=MAIN_CONTENT_1\&context=storylines_live_updates\#advocates-for-domestic-violence-survivors-praised-biden-in-a-video}{Advocates
for domestic violence survivors praised Biden in a video.}

\href{https://www.nytimes3xbfgragh.onion/live/2020/08/19/us/dnc-convention-election?action=click\&pgtype=Article\&state=default\&region=MAIN_CONTENT_1\&context=storylines_live_updates}{See
more updates}

Mr. Schumer, along with other Democrats and some Republicans, have
expressed concerns about the changes being put in place that have
resulted in some Americans going days or weeks without receiving mail,
including critical medications and Social Security checks.

In a letter sent Wednesday to Robert M. Duncan, the chairman of the
board of governors, Mr. Schumer demanded more details about Mr. DeJoy's
selection, saying that the board had repeatedly denied lawmakers'
requests to gain access to that information.

Mr. Schumer said the postal board had blocked lawmakers from questioning
the firm involved in the selection of Mr. DeJoy, Russell Reynolds, by
refusing to release the firm from a nondisclosure agreement. Mr. Schumer
said that his office had sought a briefing from Kimberly Archer, a
leader of the firm's global nonprofit practice, and information from the
firm so ``Congress could satisfy its oversight obligations to better
understand the selection of Mr. DeJoy,'' but that the postal board in
July had deemed much of the information sought by lawmakers to be
confidential.

``This administration has repeatedly pointed to the role of Russell
Reynolds to defend the selection of a Republican megadonor with no prior
postal experience as postmaster general while at the same time blocking
the ability of Congress to obtain briefings from the firm and concealing
the role of Secretary Mnuchin and the White House in its search
process,'' Mr. Schumer wrote, referring to Steven Mnuchin, the Treasury
secretary.

A spokesman for the Postal Service directed inquiries about the search
to a news release announcing Mr. DeJoy's selection and to previous
remarks by a Postal Service governor, John M. Barger, who has said
candidates were extensively vetted.

\includegraphics{https://static01.graylady3jvrrxbe.onion/images/2020/08/19/us/politics/19dc-postal2/merlin_175866228_11c5e618-a99a-4f37-8262-7bb3a89f9142-articleLarge.jpg?quality=75\&auto=webp\&disable=upscale}

In an interview, Mr. Barger, who spearheaded the search, said that the
board's decision to select Mr. DeJoy was unanimous and included the
board's sole Democrat at the time, Ron A. Bloom.

``Bloom was a strong yes,'' Mr. Barger said.

One Democratic member of the board, David C. Williams, resigned in
April, shortly before the announcement of Mr. DeJoy's selection, over
concerns that the Postal Service was becoming increasingly politicized
by the Trump administration, according to two people familiar with his
thinking.

Mr. Barger said he was ``surprised'' by the resignation and had ``never
heard an objection from David Williams about any of the candidates,
other than the ones we did not hire.''

``I don't recall him ever having objected to anything,'' Mr. Barger
said, ``or I would have asked him why. And it would have been
considered.'' Mr. Williams is expected to testify on Thursday before
lawmakers from the Congressional Progressive Caucus.

Mounting pressure prompted Mr. DeJoy to backpedal on many of his
changes, saying post office hours would not be shortened,
mail-processing equipment and mailboxes would remain, and overtime would
continue to be approved, as needed. But state officials, postal workers
and union representatives said some damage has already been done.
Hundreds of sorting machines, union officials and workers said, have
already been destroyed.

Since the changes were put in place, large institutional Postal Service
customers have reported mass mailings routinely arriving at home
addresses a day or two later than their intended delivery date, said
Michael Plunkett, the president of the Association for Postal Commerce,
or PostCom.

His group represents catalog makers, banks, phone companies and other
businesses that produce and send large quantities of mail. He attributed
the delivery delays to the cutting of overtime and truck trips between
processing plants and post offices.

``It appears as if they made these changes without taking into account
the effect that it might have on service,'' he said.

More than 20 states announced Tuesday that they would sue the Trump
administration over the changes, claiming that they were unlawful,
disadvantaged residents and disenfranchised voters. Josh Shapiro,
Pennsylvania's attorney general, who plans to file a lawsuit on behalf
of the state on Wednesday, said in an interview that Mr. DeJoy's changes
have caused delays for small businesses and veterans and people who
receive their prescription medicines in the mail. In some cases, he said
that medicines that require temperature control have taken three times
as long as usual to arrive, potentially compromising their potency, and
that the delays have been particularly problematic for veterans.

Mr. Shapiro said the state had found evidence of mail left in boxes and
trucks because of cutbacks on overtime pay for postal workers. The
lawsuit will claim that the changes are illegal because they did not go
through the normal postal regulatory commission process and are
undermining the right of Pennsylvanians to vote.

``I want to see evidence and binding agreements that roll back the
illegal changes they've already made and concrete commitments to not
make any other changes going forward that don't go through the
regulatory process,'' said Mr. Shapiro, a Democrat.

A lawsuit filed Tuesday by Washington and 13 other states included
similar accusations of delays and disenfranchisement. Among the
accusations contained in the lawsuit: Baltimore residents have gone
weeks without mail, and dozens of trailers filled with packages have
been left behind in the Milwaukee area. Rent, food and child support
checks have arrived late. And seniors have received delayed Social
Security checks, the lawsuit states, adding that veterans have
experienced weekslong waits for medications.

In Connecticut, voters across the state received their mail-in ballots
after the August primary. Many were postmarked at a least a week or up
to 10 days prior. Some voters who requested ballots weeks in advance
received them too late to return via mail.

Some Minneapolis residents also found themselves without their ballots,
within days of the state's primary.

The states maintain that they will continue their lawsuit until Mr.
DeJoy agrees to rescind all of the changes that have resulted in
widespread delays. Several attorneys general said that they did not
trust that Mr. DeJoy would restore the Postal Service to what it was
before his changes.

Emily Cochrane, Hailey Fuchs and Kenneth P. Vogel reported from
Washington, and Jessica Silver-Greenberg from New York. Nicholas Fandos,
Catie Edmondson and Alan Rappeport contributed reporting from
Washington.

\hypertarget{our-2020-election-guide}{%
\section{Our 2020 Election Guide}\label{our-2020-election-guide}}

Updated Aug. 20, 2020

\begin{itemize}
\item
  \begin{center}\rule{0.5\linewidth}{\linethickness}\end{center}

  \hypertarget{convention-recap}{%
  \subsection{Convention Recap}\label{convention-recap}}

  \begin{itemize}
  \tightlist
  \item
    Joe Biden accepted the Democratic nomination, urging Americans to
    have faith that they could
    \href{https://www.nytimes3xbfgragh.onion/2020/08/20/us/politics/Joe-Biden-accepts-democratic-nomination.html?action=click\&pgtype=Article\&state=default\&region=BELOW_MAIN_CONTENT\&context=storylines_guide}{``overcome
    this season of darkness.''}
  \end{itemize}
\item
  \begin{center}\rule{0.5\linewidth}{\linethickness}\end{center}

  \hypertarget{news-analysis}{%
  \subsection{News Analysis}\label{news-analysis}}

  \begin{itemize}
  \tightlist
  \item
    Looming over Mr. Biden's nomination was the ever-present shadow of
    another man who's poised to dominate the campaign:
    \href{https://www.nytimes3xbfgragh.onion/2020/08/20/us/politics/biden-dnc-speech-trump.html?action=click\&pgtype=Article\&state=default\&region=BELOW_MAIN_CONTENT\&context=storylines_guide}{Donald
    J. Trump}.
  \end{itemize}
\item
  \begin{center}\rule{0.5\linewidth}{\linethickness}\end{center}

  \hypertarget{keep-up-with-our-coverage}{%
  \subsection{Keep Up With Our
  Coverage}\label{keep-up-with-our-coverage}}

  \begin{itemize}
  \tightlist
  \item
    Get an
    \href{https://www.nytimes3xbfgragh.onion/newsletters/politics?action=click\&pgtype=Article\&state=default\&region=BELOW_MAIN_CONTENT\&context=storylines_guide}{email}
    recapping the day's news
  \end{itemize}

  \begin{itemize}
  \tightlist
  \item
    Download our mobile app on
    \href{https://apps.apple.com/us/app/nytimes/id284862083?ls=1\&mat_click_id=5c79ae7455014fd1bd66b5610c05b8f2-20191112-16948\&referrer=mat_click_id\%3D5c79ae7455014fd1bd66b5610c05b8f2-20191112-16948\%26link_click_id\%3D722930677036718082}{iOS}
    and
    \href{http://a.localytics.com/android?id=com.nytimes.android\&referrer=utm_source\%3Dother_nyt_mobile_web\%26utm_medium\%3DWeb\%2520page\%26utm_term\%3DGeneral\%2520Mobile\%2520Page\%26utm_campaign\%3DNYT\%2520Mobile\%2520General\%2520Page}{Android}
    and turn on Breaking News and Politics alerts
  \end{itemize}
\end{itemize}

Advertisement

\protect\hyperlink{after-bottom}{Continue reading the main story}

\hypertarget{site-index}{%
\subsection{Site Index}\label{site-index}}

\hypertarget{site-information-navigation}{%
\subsection{Site Information
Navigation}\label{site-information-navigation}}

\begin{itemize}
\tightlist
\item
  \href{https://help.nytimes3xbfgragh.onion/hc/en-us/articles/115014792127-Copyright-notice}{©~2020~The
  New York Times Company}
\end{itemize}

\begin{itemize}
\tightlist
\item
  \href{https://www.nytco.com/}{NYTCo}
\item
  \href{https://help.nytimes3xbfgragh.onion/hc/en-us/articles/115015385887-Contact-Us}{Contact
  Us}
\item
  \href{https://www.nytco.com/careers/}{Work with us}
\item
  \href{https://nytmediakit.com/}{Advertise}
\item
  \href{http://www.tbrandstudio.com/}{T Brand Studio}
\item
  \href{https://www.nytimes3xbfgragh.onion/privacy/cookie-policy\#how-do-i-manage-trackers}{Your
  Ad Choices}
\item
  \href{https://www.nytimes3xbfgragh.onion/privacy}{Privacy}
\item
  \href{https://help.nytimes3xbfgragh.onion/hc/en-us/articles/115014893428-Terms-of-service}{Terms
  of Service}
\item
  \href{https://help.nytimes3xbfgragh.onion/hc/en-us/articles/115014893968-Terms-of-sale}{Terms
  of Sale}
\item
  \href{https://spiderbites.nytimes3xbfgragh.onion}{Site Map}
\item
  \href{https://help.nytimes3xbfgragh.onion/hc/en-us}{Help}
\item
  \href{https://www.nytimes3xbfgragh.onion/subscription?campaignId=37WXW}{Subscriptions}
\end{itemize}
