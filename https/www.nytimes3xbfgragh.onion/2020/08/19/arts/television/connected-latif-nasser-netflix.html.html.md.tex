Sections

SEARCH

\protect\hyperlink{site-content}{Skip to
content}\protect\hyperlink{site-index}{Skip to site index}

\href{https://www.nytimes3xbfgragh.onion/section/arts/television}{Television}

\href{https://myaccount.nytimes3xbfgragh.onion/auth/login?response_type=cookie\&client_id=vi}{}

\href{https://www.nytimes3xbfgragh.onion/section/todayspaper}{Today's
Paper}

\href{/section/arts/television}{Television}\textbar{}Latif Nasser,
Harvard Ph.D., on the Rewards of Being Dumb

\url{https://nyti.ms/34gABuW}

\begin{itemize}
\item
\item
\item
\item
\item
\end{itemize}

Advertisement

\protect\hyperlink{after-top}{Continue reading the main story}

Supported by

\protect\hyperlink{after-sponsor}{Continue reading the main story}

\hypertarget{latif-nasser-harvard-phd-on-the-rewards-of-being-dumb}{%
\section{Latif Nasser, Harvard Ph.D., on the Rewards of Being
Dumb}\label{latif-nasser-harvard-phd-on-the-rewards-of-being-dumb}}

They include getting your own Netflix show. Nasser, a science journalist
known for his work on ``Radiolab,'' talked about going on-camera for
``Connected'' and the importance of staying curious.

\includegraphics{https://static01.graylady3jvrrxbe.onion/images/2020/08/20/arts/19nasser1/merlin_175806666_e3ca9f88-d430-49bd-bf8e-a9257a04d7d5-articleLarge.jpg?quality=75\&auto=webp\&disable=upscale}

\href{https://www.nytimes3xbfgragh.onion/by/remy-tumin}{\includegraphics{https://static01.graylady3jvrrxbe.onion/images/2019/01/10/multimedia/author-remy-tumin/author-remy-tumin-thumbLarge.png}}

By \href{https://www.nytimes3xbfgragh.onion/by/remy-tumin}{Remy Tumin}

\begin{itemize}
\item
  Aug. 19, 2020
\item
  \begin{itemize}
  \item
  \item
  \item
  \item
  \item
  \end{itemize}
\end{itemize}

Latif Nasser's new Netflix series,
``\href{https://www.netflix.com/title/81031737}{Connected},'' was only
two days old when he and his wife headed to the hospital to have a baby.
Add a pandemic to the mix and you have what Nasser calls ``the weirdest
time on top of the weirdest time.''

But Nasser is used to weird. In fact, he thrives in it.

``That's sort of my compass,'' he said last week from Los Angeles.
``Surprise and delight and wonder. Those are the things that I gravitate
toward.''

It's the approach he has taken at
\href{https://www.nytimes3xbfgragh.onion/2020/02/03/arts/radiolab-the-other-latif-guantanamo.html}{``Radiolab,''}
the popular WNYC audio program for which he is the director of research,
and now with ``Connected,'' a six-part documentary series that connects
the dots on some of the biggest questions facing science and humanity
today. That includes looking at the power of surveillance through the
lens of migratory birds in Newark, Del.; the complex history of human
excrement in Minde, Portugal; and a literal fishing expedition through
the Sahara.

Nasser's storytelling is known for helping ``Radiolab'' listeners make
sense of the chaos of the world --- and maybe even find a little comfort
and joy in it. With a Ph.D. in the history of science from Harvard,
Nasser sits at an uncommon vantage point to play guide to the world's
deepest curiosities.

``Connected'' debuted on Aug. 2, and since then Nasser has been
splitting his time between work and child care, which often involves
checking Twitter in the middle of the night while pacing back and forth
with his newborn.

\includegraphics{https://static01.graylady3jvrrxbe.onion/images/2020/08/19/arts/19nasser3/merlin_175806696_7b8021f9-9979-401e-a2b8-26c3048a5ed4-articleLarge.jpg?quality=75\&auto=webp\&disable=upscale}

Nasser took a moment from his garage home studio to hop on a Google
Hangout, where he talked about his new project, what made him fall in
love with science and the gravitational pull of true-crime shows. These
are edited excerpts from the conversation.

\textbf{You had two babies --- an actual child and ``Connected'' --- in
the span of a week. How are you holding up?}

The baby thinks day is night and night is day, and then we have a
3-year-old who is certain that day is day and night is night, so between
the two of them, this house is abuzz 24 hours a day.

\textbf{Is it strange for you, being a media guy, to be on the other
side of this all of a sudden?}

On the one hand, the swap from radio to TV is totally normal; I'm just
doing the same job I've been doing, but there's a camera involved. But I
didn't realize I was so dumb! I would watch a documentary and see a host
walking through the desert alone. But now I'm the host in the desert and
wait a second --- there's an enormous camera crew, cooks, security,
fixers, producers and sound people.

It was such a trip because I should know this stuff! I feel like I just
learned to use my eyes. It's so weird.

\textbf{Were you self-conscious going from behind the mic to in front of
the camera?}

I hate listening to myself. I hate watching myself. All I see is my
crooked teeth and bad posture and ask, ``Why are you nodding so much?''
But I realized it's so much more fun than mortifying for me to talk to
people. I'm really excited and curious and want to learn about the
amazing things they're finding. That outpaces the mortification factor.
As much as I hate seeing my own face in the interview, I \emph{love}
watching the other person's face light up. Being able to transfer that
to the viewer is something you can't do in radio. That's really
priceless.

Image

In the episode ``Clouds,'' Nasser interviewed Assaf Anyamba, a research
scientist at NASA, who uses information about the weather to help
predict the spread of disease.Credit...Netflix

\textbf{Did you have an ``ah-ha'' moment when you first fell in love
with science?}

When I was in high school, it felt like someone handed you a big fact
textbook and said, ``Here are a bunch of answers to questions that you
didn't even ask.'' That's how we teach science. I realized in college,
and then more so in grad school, that oh no, no no, they're not the
answers. There are shockingly simple questions that we don't know the
answer to, and we're still figuring it out.

I can vividly remember when I tried to be an archaeologist. I was like,
``Oh my God, archaeology, this is going to be like `Indiana Jones'-type
stuff.'' They brought me in, sat me down and gave me a bucket of what
felt to me like rocks --- and a toothbrush to clean them. I hated it so
much. But once you put it in a big picture, that we're trying to answer
this dynamic question about human history or fundamentals of our
universe, and there are these dramatic stories of individuals trying to
figure it out \ldots{} once you click that in, tooth-brushing those
rocks seems like the most dynamic, interesting thing in the whole world.
But you need to have that other information.

\textbf{We can't be experts at everything, but you do have a Ph.D. in
the history of science. Has that helped you shape your reporting?}

Paradoxically, I think my fancy Harvard Ph.D. has given me the license
to be dumb. I feel like I can walk into a room and I can just ask the
actual question that is actually on my mind without fear of people
thinking I'm an idiot. Because often I am! That's why I love this job.

\textbf{Playing dumb is often one of the best journalistic tools we
have.}

Oh, I feel that so hard! That's my default crouch: I'm an idiot. Explain
it to me. That's how I jump into every interview.

\textbf{People seem to have this attraction to curiosity over authority.
What do you make of that?}

I find this time we live in as a very cynical time and maybe for a good
reason --- people are out there lying to us all the time and spinning
things and selling us stuff. People are quite hesitant to believe you or
go on a story with you. I never pretend to be an authority. I would much
rather be the dumbest guy in the room than the smartest because I think
that's more intellectually honest. There are honest to god authorities
out there that we should be listening to, but on the other hand, let's
just be open minded and listen and think critically and have our own
questions. To me, that's a really important shift, and I think it's a
valuable one. Intellectual humility is a core value for me.

Image

Nasser at his home in Los Angeles. ``There are shockingly simple
questions that we don't know the answer to,'' he said, ``and we're still
figuring it out.''Credit...Joyce Kim for The New York Times

\textbf{So in a way, you're setting up a new way of learning about the
world.}

I hope so. I'm not alone, but I do think the trick is to lead with the
question, not the answer.

People want to be awakened to a question they didn't even realize they
had. And then all of a sudden they're totally possessed by that question
and need to know the answer. There is something deeply satisfying about
that, going all the way back to the riddle of the Sphinx.

It's about creating a little black hole inside people's minds so it has
this gravitational pull --- it wants the information, it's seeking out
the information. This is a weird analogy, but it's the same thing with
cop shows. They always start with the murder. It's a problem: I need to
know how they solved this. The moral order of the universe is off: I
need some resolution. The hard part is building the question in such a
way that you \emph{need} to know the answer.

\textbf{Do you think about your children when you approach the world
through that lens?}

I haven't been a dad for a long time, but when you boil it down, what's
the thing I want this kid to know? In a way, I hope this show is like a
letter to pass on to them. We say we're all connected, but it's in a
scary way now. Hopefully, this is in a beautiful and poetic way that
will make kids' jaws drop, and adults' too, and serve as a way to remind
us this is the way we're fingerprinting on each other's lives.

Advertisement

\protect\hyperlink{after-bottom}{Continue reading the main story}

\hypertarget{site-index}{%
\subsection{Site Index}\label{site-index}}

\hypertarget{site-information-navigation}{%
\subsection{Site Information
Navigation}\label{site-information-navigation}}

\begin{itemize}
\tightlist
\item
  \href{https://help.nytimes3xbfgragh.onion/hc/en-us/articles/115014792127-Copyright-notice}{©~2020~The
  New York Times Company}
\end{itemize}

\begin{itemize}
\tightlist
\item
  \href{https://www.nytco.com/}{NYTCo}
\item
  \href{https://help.nytimes3xbfgragh.onion/hc/en-us/articles/115015385887-Contact-Us}{Contact
  Us}
\item
  \href{https://www.nytco.com/careers/}{Work with us}
\item
  \href{https://nytmediakit.com/}{Advertise}
\item
  \href{http://www.tbrandstudio.com/}{T Brand Studio}
\item
  \href{https://www.nytimes3xbfgragh.onion/privacy/cookie-policy\#how-do-i-manage-trackers}{Your
  Ad Choices}
\item
  \href{https://www.nytimes3xbfgragh.onion/privacy}{Privacy}
\item
  \href{https://help.nytimes3xbfgragh.onion/hc/en-us/articles/115014893428-Terms-of-service}{Terms
  of Service}
\item
  \href{https://help.nytimes3xbfgragh.onion/hc/en-us/articles/115014893968-Terms-of-sale}{Terms
  of Sale}
\item
  \href{https://spiderbites.nytimes3xbfgragh.onion}{Site Map}
\item
  \href{https://help.nytimes3xbfgragh.onion/hc/en-us}{Help}
\item
  \href{https://www.nytimes3xbfgragh.onion/subscription?campaignId=37WXW}{Subscriptions}
\end{itemize}
