Sections

SEARCH

\protect\hyperlink{site-content}{Skip to
content}\protect\hyperlink{site-index}{Skip to site index}

\href{https://www.nytimes3xbfgragh.onion/section/world/americas}{Americas}

\href{https://myaccount.nytimes3xbfgragh.onion/auth/login?response_type=cookie\&client_id=vi}{}

\href{https://www.nytimes3xbfgragh.onion/section/todayspaper}{Today's
Paper}

\href{/section/world/americas}{Americas}\textbar{}Coronavirus Crisis Has
Made Brazil an Ideal Vaccine Laboratory

\url{https://nyti.ms/2Y1WXfK}

\begin{itemize}
\item
\item
\item
\item
\item
\end{itemize}

\hypertarget{the-coronavirus-outbreak}{%
\subsubsection{\texorpdfstring{\href{https://www.nytimes3xbfgragh.onion/news-event/coronavirus?name=styln-coronavirus-national\&region=TOP_BANNER\&variant=undefined\&block=storyline_menu_recirc\&action=click\&pgtype=Article\&impression_id=3e07c1c0-e386-11ea-bbe8-bf70f691015a}{The
Coronavirus
Outbreak}}{The Coronavirus Outbreak}}\label{the-coronavirus-outbreak}}

\begin{itemize}
\tightlist
\item
  live\href{https://www.nytimes3xbfgragh.onion/2020/08/20/world/coronavirus-covid.html?name=styln-coronavirus-national\&region=TOP_BANNER\&variant=undefined\&block=storyline_menu_recirc\&action=click\&pgtype=Article\&impression_id=3e07c1c1-e386-11ea-bbe8-bf70f691015a}{Latest
  Updates}
\item
  \href{https://www.nytimes3xbfgragh.onion/interactive/2020/us/coronavirus-us-cases.html?name=styln-coronavirus-national\&region=TOP_BANNER\&variant=undefined\&block=storyline_menu_recirc\&action=click\&pgtype=Article\&impression_id=3e07c1c2-e386-11ea-bbe8-bf70f691015a}{Maps
  and Cases}
\item
  \href{https://www.nytimes3xbfgragh.onion/interactive/2020/science/coronavirus-vaccine-tracker.html?name=styln-coronavirus-national\&region=TOP_BANNER\&variant=undefined\&block=storyline_menu_recirc\&action=click\&pgtype=Article\&impression_id=3e07c1c3-e386-11ea-bbe8-bf70f691015a}{Vaccine
  Tracker}
\item
  \href{https://www.nytimes3xbfgragh.onion/2020/08/19/us/colleges-closing-covid.html?name=styln-coronavirus-national\&region=TOP_BANNER\&variant=undefined\&block=storyline_menu_recirc\&action=click\&pgtype=Article\&impression_id=3e07e8d0-e386-11ea-bbe8-bf70f691015a}{Colleges
  Closing}
\item
  \href{https://www.nytimes3xbfgragh.onion/live/2020/08/20/business/stock-market-today-coronavirus?name=styln-coronavirus-national\&region=TOP_BANNER\&variant=undefined\&block=storyline_menu_recirc\&action=click\&pgtype=Article\&impression_id=3e07e8d1-e386-11ea-bbe8-bf70f691015a}{Economy}
\end{itemize}

Advertisement

\protect\hyperlink{after-top}{Continue reading the main story}

Supported by

\protect\hyperlink{after-sponsor}{Continue reading the main story}

\hypertarget{coronavirus-crisis-has-made-brazil-an-ideal-vaccine-laboratory}{%
\section{Coronavirus Crisis Has Made Brazil an Ideal Vaccine
Laboratory}\label{coronavirus-crisis-has-made-brazil-an-ideal-vaccine-laboratory}}

Widespread contagion, a deep bench of scientists and a robust
vaccine-making infrastructure have made Brazil an important player in
the quest to find a vaccine.

\includegraphics{https://static01.graylady3jvrrxbe.onion/images/2020/08/16/world/SUB16virus-brazil-vax/merlin_175106148_7983adc4-cfe5-443f-9c34-7dab22b57205-articleLarge.jpg?quality=75\&auto=webp\&disable=upscale}

By Manuela Andreoni and
\href{https://www.nytimes3xbfgragh.onion/by/ernesto-londono}{Ernesto
Londoño}

\begin{itemize}
\item
  Aug. 15, 2020
\item
  \begin{itemize}
  \item
  \item
  \item
  \item
  \item
  \end{itemize}
\end{itemize}

\href{https://www.nytimes3xbfgragh.onion/es/2020/08/17/espanol/america-latina/vacuna-coronavirus-brasil.html}{Leer
en español}

RIO DE JANEIRO --- The chaotic response to the coronavirus in Brazil,
where it has killed more than \href{https://covid.saude.gov.br/}{105,000
people}, made the country's experience a cautionary tale that many
around the world have watched with alarm.

But as the country's caseload soared, vaccine researchers saw a unique
opportunity.

With sustained widespread contagion, a deep bench of immunization
experts, a robust medical manufacturing infrastructure and thousands of
vaccine trial volunteers, Brazil has emerged as a potentially vital
player in the global scramble to end the pandemic.

Three of the most promising and advanced vaccine studies in the world
are relying on scientists and volunteers in Brazil, according to the
World Health Organization's report on the progress of vaccine research.

The embattled government hopes its citizens could be among the first in
the world to be inoculated. And medical experts are imagining the
possibility that Brazil could even manufacture the vaccine and export it
to neighboring countries, a prospect that fills them with something that
has been in short supply this year: pride.

``I'm very optimistic,'' said Dimas Covas, the director of the Butantan
Institute, an internationally renowned biopharmaceutical producer that
is partnering with China's Sinovac on one of the studies that has
reached the third stage of research, during which potential vaccines are
tested on 9,000 people.

``Brazil will be one of the first countries to have the vaccine.''

\includegraphics{https://static01.graylady3jvrrxbe.onion/images/2020/08/15/world/15virus-brazil-vaccine2/merlin_175470378_00df1661-ed97-47ca-98e3-193817ba871c-articleLarge.jpg?quality=75\&auto=webp\&disable=upscale}

Some 5,000 Brazilians have also been recruited to support a vaccine
trial conducted by AstraZeneca, a British-Swedish pharmaceutical company
in partnership with Oxford University. An additional 1,000 volunteers in
Brazil were recruited to test a vaccine developed by New York-based
Pfizer.

Researchers need countries with large enough outbreaks to assess whether
a vaccine will work. Some volunteers are given the potential vaccine
while others are given a placebo, but they have to be in a place where
enough virus is circulating to test the vaccine's efficacy.

Brazil, where the virus has infected more than three million people, has
clear conditions for these trials. And it will be the only country other
than the United States to be playing a major role in three of the
leading studies as an unparalleled quest for a vaccine has led to
unusually fast regulatory approvals and hastily brokered partnerships.

Still, it is far from certain, experts say, that the vaccine trials
underway in Brazil will win the race.

\hypertarget{latest-updates-the-coronavirus-outbreak}{%
\section{\texorpdfstring{\href{https://www.nytimes3xbfgragh.onion/2020/08/20/world/coronavirus-covid.html?action=click\&pgtype=Article\&state=default\&region=MAIN_CONTENT_1\&context=storylines_live_updates}{Latest
Updates: The Coronavirus
Outbreak}}{Latest Updates: The Coronavirus Outbreak}}\label{latest-updates-the-coronavirus-outbreak}}

Updated 2020-08-21T07:46:15.883Z

\begin{itemize}
\tightlist
\item
  \href{https://www.nytimes3xbfgragh.onion/2020/08/20/world/coronavirus-covid.html?action=click\&pgtype=Article\&state=default\&region=MAIN_CONTENT_1\&context=storylines_live_updates\#link-68774d88}{Shutdowns,
  warnings and scoldings follow alarming incidents on college campuses.}
\item
  \href{https://www.nytimes3xbfgragh.onion/2020/08/20/world/coronavirus-covid.html?action=click\&pgtype=Article\&state=default\&region=MAIN_CONTENT_1\&context=storylines_live_updates\#link-26b58724}{Biden
  knocks Trump's pandemic response, and outlines a national strategy.}
\item
  \href{https://www.nytimes3xbfgragh.onion/2020/08/20/world/coronavirus-covid.html?action=click\&pgtype=Article\&state=default\&region=MAIN_CONTENT_1\&context=storylines_live_updates\#link-4e542da3}{U.S.
  health agencies announce moves to confront the flu season and
  plummeting child vaccination rates.}
\end{itemize}

\href{https://www.nytimes3xbfgragh.onion/2020/08/20/world/coronavirus-covid.html?action=click\&pgtype=Article\&state=default\&region=MAIN_CONTENT_1\&context=storylines_live_updates}{See
more updates}

More live coverage:
\href{https://www.nytimes3xbfgragh.onion/live/2020/08/20/business/stock-market-today-coronavirus?action=click\&pgtype=Article\&state=default\&region=MAIN_CONTENT_1\&context=storylines_live_updates}{Markets}

Countries across the world are vying to be among the first to get access
to a vaccine that will be in demand by billions of people. In India, one
of the country's wealthiest families
\href{https://www.nytimes3xbfgragh.onion/2020/08/01/world/asia/coronavirus-vaccine-india.html}{is
taking a gamble} by mass producing the Oxford vaccine in hopes that it
will be the first to clear safety and regulatory hurdles.

Russia
\href{https://www.nytimes3xbfgragh.onion/2020/08/11/world/europe/russia-coronavirus-vaccine-approval.html}{this
week approved a homemade vaccine} which has not yet met the final tests
for safety and efficacy. If it works, it could position the country to
claim it developed the world's first effective coronavirus vaccine.

Brazil's explosive caseload has made it the second hardest-hit nation in
the world after the United States. While other countries in the region
have higher per capita rates, experts have assailed President Jair
Bolsonaro's cavalier handling of the crisis.

The president, who caught the virus in July, has called it a ``measly
flu'' and sabotaged calls for quarantines and lockdowns. He also
appointed an Army general with no medical experience to run the health
ministry after two ministers clashed with the president over his disdain
for science-based approaches.

Because of the country's disorganized response to the virus, Brazilians
have been subjected to travel bans, neighbors have militarized border
crossings and unions representing medical workers recently asked the
International Criminal Court to charge Mr. Bolsonaro for crimes against
humanity, arguing that he has given the virus free rein.

Image

The Sinovac vaccine.Credit...Diego Vara/Reuters

Brazil has a universal public health care system with one of the best
immunization programs in the developing world, which has enabled it to
contain outbreaks of yellow fever, measles and other pathogens.

But in recent years, as the economy has contracted, the program has
suffered, dogged by budget cuts. It has also had to fight disinformation
campaigns that have found a rapt audience on social media.

In 2019, for the first time in 25 years, Brazil didn't fulfill its
vaccination goal for any of the shots it routinely administers.

A coronavirus breakthrough could galvanize the country's vaccine sector.
It could also invigorate its scientific institutions, which employ world
class scientists but have been reeling after years of budget cuts that
have weakened the public health care system and dented the country's
reputation as a research powerhouse.

Katherine O'Brien, the director of immunization at the World Health
Organization, welcomed Brazil's investments in manufacturing vaccines
for Covid-19, the disease caused by the virus. But she said bilateral
deals like the ones Brazil is involved in were still a gamble.

``Some countries are going to be lucky, entering into contracts with a
candidate that's going to demonstrate efficacy,'' Dr. O'Brien said.
``Other countries are going to pursue deals with candidates that are
going to fail and they'll get nothing.''

Image

The laboratory of Bio-Manguinhos, which will produce the Oxford vaccine
in Rio de Janeiro.Credit...Antonio Lacerda/EPA, via Shutterstock

Home to about 210 million people, Brazil has the capacity to make
roughly 500 million vaccines per year. Under the current coronavirus
vaccine deals Brazil has stakes in, Brazilian vaccine plants would
initially handle the final stages of vaccine production after importing
the raw materials, and later produce them entirely.

Brazil has signed two deals to get preferential access to a vaccine.
One, between the São Paulo state's Butantan Institute and Sinovac, would
provide Brazilians with 120 million doses of the vaccine by early 2021.
The second one, between the federal government's Bio-Manguinhos and
AstraZeneca, guarantees access to 100 million doses of the vaccine by
the beginning of next year.

\href{https://www.nytimes3xbfgragh.onion/news-event/coronavirus?action=click\&pgtype=Article\&state=default\&region=MAIN_CONTENT_3\&context=storylines_faq}{}

\hypertarget{the-coronavirus-outbreak-}{%
\subsubsection{The Coronavirus Outbreak
›}\label{the-coronavirus-outbreak-}}

\hypertarget{frequently-asked-questions}{%
\paragraph{Frequently Asked
Questions}\label{frequently-asked-questions}}

Updated August 17, 2020

\begin{itemize}
\item ~
  \hypertarget{why-does-standing-six-feet-away-from-others-help}{%
  \paragraph{Why does standing six feet away from others
  help?}\label{why-does-standing-six-feet-away-from-others-help}}

  \begin{itemize}
  \tightlist
  \item
    The coronavirus spreads primarily through droplets from your mouth
    and nose, especially when you cough or sneeze. The C.D.C., one of
    the organizations using that measure,
    \href{https://www.nytimes3xbfgragh.onion/2020/04/14/health/coronavirus-six-feet.html?action=click\&pgtype=Article\&state=default\&region=MAIN_CONTENT_3\&context=storylines_faq}{bases
    its recommendation of six feet} on the idea that most large droplets
    that people expel when they cough or sneeze will fall to the ground
    within six feet. But six feet has never been a magic number that
    guarantees complete protection. Sneezes, for instance, can launch
    droplets a lot farther than six feet,
    \href{https://jamanetwork.com/journals/jama/fullarticle/2763852}{according
    to a recent study}. It's a rule of thumb: You should be safest
    standing six feet apart outside, especially when it's windy. But
    keep a mask on at all times, even when you think you're far enough
    apart.
  \end{itemize}
\item ~
  \hypertarget{i-have-antibodies-am-i-now-immune}{%
  \paragraph{I have antibodies. Am I now
  immune?}\label{i-have-antibodies-am-i-now-immune}}

  \begin{itemize}
  \tightlist
  \item
    As of right
    now,\href{https://www.nytimes3xbfgragh.onion/2020/07/22/health/covid-antibodies-herd-immunity.html?action=click\&pgtype=Article\&state=default\&region=MAIN_CONTENT_3\&context=storylines_faq}{that
    seems likely, for at least several months.} There have been
    frightening accounts of people suffering what seems to be a second
    bout of Covid-19. But experts say these patients may have a
    drawn-out course of infection, with the virus taking a slow toll
    weeks to months after initial exposure. People infected with the
    coronavirus typically
    \href{https://www.nature.com/articles/s41586-020-2456-9}{produce}
    immune molecules called antibodies, which are
    \href{https://www.nytimes3xbfgragh.onion/2020/05/07/health/coronavirus-antibody-prevalence.html?action=click\&pgtype=Article\&state=default\&region=MAIN_CONTENT_3\&context=storylines_faq}{protective
    proteins made in response to an
    infection}\href{https://www.nytimes3xbfgragh.onion/2020/05/07/health/coronavirus-antibody-prevalence.html?action=click\&pgtype=Article\&state=default\&region=MAIN_CONTENT_3\&context=storylines_faq}{.
    These antibodies may} last in the body
    \href{https://www.nature.com/articles/s41591-020-0965-6}{only two to
    three months}, which may seem worrisome, but that's perfectly normal
    after an acute infection subsides, said Dr. Michael Mina, an
    immunologist at Harvard University. It may be possible to get the
    coronavirus again, but it's highly unlikely that it would be
    possible in a short window of time from initial infection or make
    people sicker the second time.
  \end{itemize}
\item ~
  \hypertarget{im-a-small-business-owner-can-i-get-relief}{%
  \paragraph{I'm a small-business owner. Can I get
  relief?}\label{im-a-small-business-owner-can-i-get-relief}}

  \begin{itemize}
  \tightlist
  \item
    The
    \href{https://www.nytimes3xbfgragh.onion/article/small-business-loans-stimulus-grants-freelancers-coronavirus.html?action=click\&pgtype=Article\&state=default\&region=MAIN_CONTENT_3\&context=storylines_faq}{stimulus
    bills enacted in March} offer help for the millions of American
    small businesses. Those eligible for aid are businesses and
    nonprofit organizations with fewer than 500 workers, including sole
    proprietorships, independent contractors and freelancers. Some
    larger companies in some industries are also eligible. The help
    being offered, which is being managed by the Small Business
    Administration, includes the Paycheck Protection Program and the
    Economic Injury Disaster Loan program. But lots of folks have
    \href{https://www.nytimes3xbfgragh.onion/interactive/2020/05/07/business/small-business-loans-coronavirus.html?action=click\&pgtype=Article\&state=default\&region=MAIN_CONTENT_3\&context=storylines_faq}{not
    yet seen payouts.} Even those who have received help are confused:
    The rules are draconian, and some are stuck sitting on
    \href{https://www.nytimes3xbfgragh.onion/2020/05/02/business/economy/loans-coronavirus-small-business.html?action=click\&pgtype=Article\&state=default\&region=MAIN_CONTENT_3\&context=storylines_faq}{money
    they don't know how to use.} Many small-business owners are getting
    less than they expected or
    \href{https://www.nytimes3xbfgragh.onion/2020/06/10/business/Small-business-loans-ppp.html?action=click\&pgtype=Article\&state=default\&region=MAIN_CONTENT_3\&context=storylines_faq}{not
    hearing anything at all.}
  \end{itemize}
\item ~
  \hypertarget{what-are-my-rights-if-i-am-worried-about-going-back-to-work}{%
  \paragraph{What are my rights if I am worried about going back to
  work?}\label{what-are-my-rights-if-i-am-worried-about-going-back-to-work}}

  \begin{itemize}
  \tightlist
  \item
    Employers have to provide
    \href{https://www.osha.gov/SLTC/covid-19/standards.html}{a safe
    workplace} with policies that protect everyone equally.
    \href{https://www.nytimes3xbfgragh.onion/article/coronavirus-money-unemployment.html?action=click\&pgtype=Article\&state=default\&region=MAIN_CONTENT_3\&context=storylines_faq}{And
    if one of your co-workers tests positive for the coronavirus, the
    C.D.C.} has said that
    \href{https://www.cdc.gov/coronavirus/2019-ncov/community/guidance-business-response.html}{employers
    should tell their employees} -\/- without giving you the sick
    employee's name -\/- that they may have been exposed to the virus.
  \end{itemize}
\item ~
  \hypertarget{what-is-school-going-to-look-like-in-september}{%
  \paragraph{What is school going to look like in
  September?}\label{what-is-school-going-to-look-like-in-september}}

  \begin{itemize}
  \tightlist
  \item
    It is unlikely that many schools will return to a normal schedule
    this fall, requiring the grind of
    \href{https://www.nytimes3xbfgragh.onion/2020/06/05/us/coronavirus-education-lost-learning.html?action=click\&pgtype=Article\&state=default\&region=MAIN_CONTENT_3\&context=storylines_faq}{online
    learning},
    \href{https://www.nytimes3xbfgragh.onion/2020/05/29/us/coronavirus-child-care-centers.html?action=click\&pgtype=Article\&state=default\&region=MAIN_CONTENT_3\&context=storylines_faq}{makeshift
    child care} and
    \href{https://www.nytimes3xbfgragh.onion/2020/06/03/business/economy/coronavirus-working-women.html?action=click\&pgtype=Article\&state=default\&region=MAIN_CONTENT_3\&context=storylines_faq}{stunted
    workdays} to continue. California's two largest public school
    districts --- Los Angeles and San Diego --- said on July 13, that
    \href{https://www.nytimes3xbfgragh.onion/2020/07/13/us/lausd-san-diego-school-reopening.html?action=click\&pgtype=Article\&state=default\&region=MAIN_CONTENT_3\&context=storylines_faq}{instruction
    will be remote-only in the fall}, citing concerns that surging
    coronavirus infections in their areas pose too dire a risk for
    students and teachers. Together, the two districts enroll some
    825,000 students. They are the largest in the country so far to
    abandon plans for even a partial physical return to classrooms when
    they reopen in August. For other districts, the solution won't be an
    all-or-nothing approach.
    \href{https://bioethics.jhu.edu/research-and-outreach/projects/eschool-initiative/school-policy-tracker/}{Many
    systems}, including the nation's largest, New York City, are
    devising
    \href{https://www.nytimes3xbfgragh.onion/2020/06/26/us/coronavirus-schools-reopen-fall.html?action=click\&pgtype=Article\&state=default\&region=MAIN_CONTENT_3\&context=storylines_faq}{hybrid
    plans} that involve spending some days in classrooms and other days
    online. There's no national policy on this yet, so check with your
    municipal school system regularly to see what is happening in your
    community.
  \end{itemize}
\end{itemize}

Both deals include a technology transfer agreement that would allow
Brazil to later manufacture vaccines on its own. Government officials
hope to start vaccinating some Brazilians by the first semester of 2021,
though an exact date depends on the results of ongoing studies and a
future approval process with the local regulatory agency.

Carla Domingues, an epidemiologist who ran the country's immunization
program until last year, said disinformation campaigns about
immunization have hobbled efforts to protect people from HPV, a sexually
transmitted infection.

``Unfortunately, this trend that we have been seeing in other countries
for many years is now here in Brazil,'' she said. ``And we haven't
managed to reverse it.''

Image

Pneumonia vaccines in the laboratory of Bio-Manguinhos, where the Oxford
vaccine will be produced in Brazil.Credit...Antonio Lacerda/EPA, via
Shutterstock

Yet recruiting volunteers for the ongoing studies in Brazil has not been
a challenge, said Soraya Smaili, the president at the Federal University
of São Paulo, which is involved in the AstraZeneca and Oxford study.

``It has not been hard to find volunteers,'' she said. ``People have
stepped forward and everyone wants to be part of the solution. This has
been a lovely social movement.''

Denise Abranches, a dental surgeon who has spent months treating
coronavirus patients with mouth sores in intensive care units, was among
the first to volunteer for a vaccine. She said her only fear was not
getting in line soon enough to get a shot.

``I see this as a way for us to regain a leadership role'' in the global
scientific community, she said. ``The world is looking at us for answers
and this is a vaccine that could help everyone in the world.''

Maurício Zuma, the director at Bio-Manguinhos, one of the manufacturers
that hopes to produce Covid-19 vaccines in Brazil, said that after the
country meets its internal demand, he hopes to export vials to
neighboring countries that also have been struggling with large
caseloads.

``Our intention is to take part in a movement of solidarity,'' he said.
``If we manage to produce the vaccine here and end up with a surplus,
we're obviously going to make sure it's used in other countries in Latin
America.''

But as researchers celebrate Brazil's role in the global vaccine race,
they have also felt compelled to remind citizens that the good news
won't single-handedly put an end to the suffering the virus has
unleashed in the country.

``They should not assume that that's it and they are done,'' said Maria
Elena Bottazzi, a vaccine developer at Baylor College of Medicine.
``There is still a lot of work that Brazil needs to do to strengthen
their public health infrastructure to reduce the transmission of the
virus.''

Advertisement

\protect\hyperlink{after-bottom}{Continue reading the main story}

\hypertarget{site-index}{%
\subsection{Site Index}\label{site-index}}

\hypertarget{site-information-navigation}{%
\subsection{Site Information
Navigation}\label{site-information-navigation}}

\begin{itemize}
\tightlist
\item
  \href{https://help.nytimes3xbfgragh.onion/hc/en-us/articles/115014792127-Copyright-notice}{©~2020~The
  New York Times Company}
\end{itemize}

\begin{itemize}
\tightlist
\item
  \href{https://www.nytco.com/}{NYTCo}
\item
  \href{https://help.nytimes3xbfgragh.onion/hc/en-us/articles/115015385887-Contact-Us}{Contact
  Us}
\item
  \href{https://www.nytco.com/careers/}{Work with us}
\item
  \href{https://nytmediakit.com/}{Advertise}
\item
  \href{http://www.tbrandstudio.com/}{T Brand Studio}
\item
  \href{https://www.nytimes3xbfgragh.onion/privacy/cookie-policy\#how-do-i-manage-trackers}{Your
  Ad Choices}
\item
  \href{https://www.nytimes3xbfgragh.onion/privacy}{Privacy}
\item
  \href{https://help.nytimes3xbfgragh.onion/hc/en-us/articles/115014893428-Terms-of-service}{Terms
  of Service}
\item
  \href{https://help.nytimes3xbfgragh.onion/hc/en-us/articles/115014893968-Terms-of-sale}{Terms
  of Sale}
\item
  \href{https://spiderbites.nytimes3xbfgragh.onion}{Site Map}
\item
  \href{https://help.nytimes3xbfgragh.onion/hc/en-us}{Help}
\item
  \href{https://www.nytimes3xbfgragh.onion/subscription?campaignId=37WXW}{Subscriptions}
\end{itemize}
