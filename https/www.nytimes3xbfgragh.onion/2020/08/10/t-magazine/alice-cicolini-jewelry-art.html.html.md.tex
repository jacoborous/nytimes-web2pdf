Sections

SEARCH

\protect\hyperlink{site-content}{Skip to
content}\protect\hyperlink{site-index}{Skip to site index}

\href{https://myaccount.nytimes3xbfgragh.onion/auth/login?response_type=cookie\&client_id=vi}{}

\href{https://www.nytimes3xbfgragh.onion/section/todayspaper}{Today's
Paper}

The Jewelry Designer Inspired by Ancient Rituals and Artifacts

\begin{itemize}
\item
\item
\item
\item
\item
\end{itemize}

Advertisement

\protect\hyperlink{after-top}{Continue reading the main story}

Supported by

\protect\hyperlink{after-sponsor}{Continue reading the main story}

T Presents

\hypertarget{the-jewelry-designer-inspired-by-ancient-rituals-and-artifacts}{%
\section{The Jewelry Designer Inspired by Ancient Rituals and
Artifacts}\label{the-jewelry-designer-inspired-by-ancient-rituals-and-artifacts}}

Drawing on years of research and working with one of Jaipur's premier
enamel artisans, Alice Cicolini makes pieces destined to become
talismans.

\includegraphics{https://static01.graylady3jvrrxbe.onion/images/2020/08/10/t-magazine/art/Tadobe-slide-4O7J/Tadobe-slide-4O7J-articleLarge.jpg?quality=75\&auto=webp\&disable=upscale}

By \href{https://www.nytimes3xbfgragh.onion/by/aimee-farrell}{Aimee
Farrell}

\begin{itemize}
\item
  Aug. 10, 2020
\item
  \begin{itemize}
  \item
  \item
  \item
  \item
  \item
  \end{itemize}
\end{itemize}

It's telling that it was an artifact, not a stone or a vintage bauble,
that set the London-based designer Alice Cicolini on her path to making
jewelry. Thirteen **** years ago, while working as the director of arts
and culture at the British Council in India, Cicolini traveled to
Mehrangarh, a resplendent **** 15th-century fort and series of palaces
with an accompanying museum in Jodhpur. Among the collection of courtly
textiles, armory and miniature paintings, a maharani jewelry box caught
her eye. ``It would have housed many of the things you need to perform
\emph{solah shringar},'' she explains, referring to the ancient Hindu
practice of a bride on her wedding day wearing 16 traditional
adornments, from bell-embellished anklets to glass and gold bangles.
``The box itself was fairly ordinary looking, but the idea of what its
contents constituted was magic to me.''

Image

In 2018, Cicolini collaborated with the Colombian Muzo emerald mine to
create these drop earrings using rare marquise cabochon hexagonal
stones.Credit...Will Sanders

Image

Pages from Cicolini's graduate sketchbook show the diversity of her
inspirations, which at that time ranged from marbled paper to
18th-century English ceramics.Credit...Will Sanders

The discovery compelled Cicolini to embark on an exacting study of
Indian aesthetic ritual, devouring books on sacred architecture, royal
presentation and the role of cosmology and ceremony in Indian theater.
Some years later, after much research and a move back to her native
England, where she eventually established her studio in East London,
Cicolini's very first jewelry design took shape: The Shinkara pendant,
created in 2010, is a rich, elongated sequence of forms forged in gold,
carved ebony and vitreous enamel **** that replicates the spiked
pinnacle of a temple top. ``It's this hybrid object that stacks
different shapes and patterns from all along
\href{https://www.nytimes3xbfgragh.onion/2020/05/11/t-magazine/the-silk-road.html}{the
Silk Route},'' she explains of the design, which laid the foundation for
an entire line of pieces that draw on religious, natural and
architectural forms --- there's now a dedicated
\href{http://www.alicecicolini.com/index\#/temple}{Temple collection},
inspired by the forms of ancient shrines in Turkey, India and
Uzbekistan, while her
\href{http://www.alicecicolini.com/index\#/summer-snow-collcection}{Summer
Snow collection} was named for the tufts of poplar tree seeds that cover
the streets of Moscow each June --- with an emphasis on artisanal
technique.

\includegraphics{https://static01.graylady3jvrrxbe.onion/images/2020/08/10/t-magazine/tmag-adobe-thumbnail-slide-WWJD/tmag-adobe-thumbnail-slide-WWJD-mediumSquareAt3X.jpg}

``I've always been more interested in the talent of craftspeople than in
some massive emerald,'' says Cicolini, now 47. ``Perceptions of Indian
craft have deteriorated over the last 50 years, but if you think back to
the 18th century, when Gainsborough was painting portraits of
aristocratic women in Kashmiri shawls, it was considered superlative ---
I wanted to remind people that it remains so.'' **** From the beginning,
she has collaborated with
\href{https://www.nytimes3xbfgragh.onion/2015/05/14/fashion/jewelry-an-antique-enamel-technique-from-persia.html}{Kamal
Kumar Meenakar}, one of Jaipur's last remaining masters of
\emph{meenakari} --- the practice of enameling intricate designs on
metal, which was introduced from Persia in the 17th century. ``He's an
artist,'' she says of Meenakar, who takes her sketched designs,
technical drawings and the occasional wax maquette and returns them as
fully formed pieces contained in small wax-sealed tin boxes wrapped in
calico. ``In his hands the enamel becomes a miniature painting.'' While
meenakari was traditionally confined to the underside of a necklace or
earring, Cicolini positions it center stage, conjuring sculptural,
oversize 23.5-karat gold rings hand-painted in enamel with abstract
motifs drawn from patterns in **** textiles, ceramics and antique
Chinese screens and inset with large **** colored stones including
mandarin garnets and pink and green tourmalines in classic Indian uncut
polki forms. Her bold, scholarly style has won her a devoted following
and a coveted place at the fashion boutique Dover Street Market.

Image

A tray of striped rings from the designer's Memphis collection and
geometric, floral pieces from her Summer Snow line.Credit...Will Sanders

Image

The walls of her studio are lined with books on subjects including
gemstones, Russian textiles and Indian and Art Nouveau
jewelry.Credit...Will Sanders

Cicolini lived many lives before becoming a jeweler. The only child of
two teachers, she grew up in a book-filled late 19th-century house on
the suburban fringes of north London. Her mother, a William
Morris-obsessed horticulturalist and amateur poet, instilled in her a
love for the arts. After studying drama and a stint working front of
house **** at London's Young Vic Theater in her early 20s, she became a
project manager for the furniture designer Tom Dixon before completing a
master's degree in fashion history. In 2009, after five years in India,
Cicolini returned to London to earn her master's degree in jewelry
design at Central Saint Martins, where she is now a visiting lecturer.
``It took me a while to understand that my voice as a designer was all
about my ability to curate,'' she says, referring to the way she
showcases the work of artisans through her pieces. ``And if there's a
common thread that runs throughout my creative practice as **** both a
designer and curator, it's storytelling.''

Today, her stories could just as easily stem from a color as a
technique: ``I'll see a particular shade of chartreuse in a carpet and
begin from there,'' she says of her process, which always starts with a
sketch in pencil or gouache. Her ongoing Memphis collection, first
introduced in 2012, can be traced to a holiday snapshot of a
candy-banded Venetian gondola pole, which inspired a deep dive into
chevrons, stripes and the Italian architect
\href{https://www.nytimes3xbfgragh.onion/2017/08/07/t-magazine/ettore-sottsass-designer-memphis-charles-zana.html}{Ettore
Sottsass}, the founder of the colorful Memphis design movement of the
'80s. With their crisp monochrome lines and precise zigzags, these
boldly saturated, stackable lacquered enamel rings and earrings **** are
a stark contrast to her expressive meenakari designs; a difference she
likens to electric light versus a candle. Currently, she's working with
a Geneva-based Colombian goldsmith on a collection themed around
goddesses. Another, set to launch in October, explores regional Indian
sari culture through the medium of marbled enamel.

\includegraphics{https://static01.graylady3jvrrxbe.onion/images/2020/08/10/t-magazine/art/Tadobe-slide-UCBM/Tadobe-slide-UCBM-articleLarge.jpg?quality=75\&auto=webp\&disable=upscale}

Perhaps unsurprisingly, for Cicolini, jewelry is far more than simply
decorative. In fact, she doesn't even wear much of it herself, and
instead collects rings, bracelets and pendants as one would artwork. ``I
just love having them,'' she says of favorite pieces from the
London-based designers Ben Day and Fernando Jorge. Last year, she was
one of a handful of jewelers to collaborate with the Carpenters Workshop
Gallery, a design gallery specializing in contemporary furniture, with
outposts in London, Paris, New York and San Francisco. For the project,
whose only brief was to conceive of a collection to fit the company's
muted aesthetic, she created her Totem collection. It includes two
Totemic rings, each comprising a trio of graphic, stone-free,
monochromatic bands that fit together like a single heraldic shield. The
collaboration perfectly embodies Cicolini's larger philosophy of
jewelry: ``What's fascinating,'' she says, ``is that it exists in this
ever-shifting space between fashion, craft, the body and architecture.''
Whether adorning a wrist, a finger or a mantelpiece, her jewels are
imbued with the stories of centuries of craft, their meaning neatly
encased in their enamel forms like so many maharani jewel boxes ready to
be opened up by generations to come.

\hypertarget{t-presents-15-creative-women-for-our-time}{%
\subsubsection{\texorpdfstring{\href{https://www.nytimes3xbfgragh.onion/interactive/2020/08/10/t-magazine/creative-women-designers-artists-chefs.html}{T
Presents: 15 Creative Women for Our
Time}}{T Presents: 15 Creative Women for Our Time}}\label{t-presents-15-creative-women-for-our-time}}

\href{https://www.nytimes3xbfgragh.onion/section/t-magazine}{}

\href{https://www.nytimes3xbfgragh.onion/2020/08/10/t-magazine/priya-ahluwalia-fashion-menswear.html}{\includegraphics{https://static01.graylady3jvrrxbe.onion/newsgraphics/2020/06/17/tmag-adobe/assets/images/ahluwalia-460.jpg}}

Priya Ahluwalia

Fashion Designer

\href{https://www.nytimes3xbfgragh.onion/2020/08/10/t-magazine/alice-cicolini-jewelry-art.html}{\includegraphics{https://static01.graylady3jvrrxbe.onion/newsgraphics/2020/06/17/tmag-adobe/assets/images/cicolini-460.jpg}}

Alice Cicolini

Jewelry Designer

\href{https://nytimes3xbfgragh.onion/2020/08/10/t-magazine/sonya-clark-flags-art.html}{\includegraphics{https://static01.graylady3jvrrxbe.onion/newsgraphics/2020/06/17/tmag-adobe/assets/images/clark-460.jpg}}

Sonya Clark

Artist

\href{https://www.nytimes3xbfgragh.onion/2020/08/10/t-magazine/pierre-davis-no-sesso.html}{\includegraphics{https://static01.graylady3jvrrxbe.onion/newsgraphics/2020/06/17/tmag-adobe/assets/images/davis-460.jpg}}

Pierre Davis

Fashion Designer

\href{https://www.nytimes3xbfgragh.onion/2020/08/10/t-magazine/paria-farzaneh-fashion-menswear.html}{\includegraphics{https://static01.graylady3jvrrxbe.onion/newsgraphics/2020/06/17/tmag-adobe/assets/images/farzaneh-460.jpg}}

Paria Farzaneh

Fashion Designer

\href{https://www.nytimes3xbfgragh.onion/2020/08/10/t-magazine/elizabeth-garouste-interior-design.html}{\includegraphics{https://static01.graylady3jvrrxbe.onion/newsgraphics/2020/06/17/tmag-adobe/assets/images/garouste-460.jpg}}

Elizabeth Garouste

Furniture Designer and Artist

\href{https://www.nytimes3xbfgragh.onion/2020/08/10/t-magazine/jatovia-gary-film.html}{\includegraphics{https://static01.graylady3jvrrxbe.onion/newsgraphics/2020/06/17/tmag-adobe/assets/images/gary-460.jpg}}

Ja'Tovia Gary

Artist and Filmmaker

\href{https://www.nytimes3xbfgragh.onion/2020/08/10/t-magazine/aiko-hachisuka-art-sculpture.html}{\includegraphics{https://static01.graylady3jvrrxbe.onion/newsgraphics/2020/06/17/tmag-adobe/assets/images/hachisuka-460.jpg}}

Aiko Hachisuka

Artist

\href{https://www.nytimes3xbfgragh.onion/2020/08/10/t-magazine/juliana-huxtable.html}{\includegraphics{https://static01.graylady3jvrrxbe.onion/newsgraphics/2020/06/17/tmag-adobe/assets/images/huxtable-460.jpg}}

Juliana Huxtable

Artist

\href{https://www.nytimes3xbfgragh.onion/2020/08/10/t-magazine/mariam-kamara-architect-design.html}{\includegraphics{https://static01.graylady3jvrrxbe.onion/newsgraphics/2020/06/17/tmag-adobe/assets/images/kamara-460.jpg}}

Mariam Kamara

Architect

\href{https://www.nytimes3xbfgragh.onion/2020/08/10/t-magazine/sophia-moreno-bunge-floral-design.html}{\includegraphics{https://static01.graylady3jvrrxbe.onion/newsgraphics/2020/06/17/tmag-adobe/assets/images/bunge-460.jpg}}

Sophia Moreno-Bunge

Floral Designer

\href{https://www.nytimes3xbfgragh.onion/2020/08/10/t-magazine/marina-moscone-fashion-design.html}{\includegraphics{https://static01.graylady3jvrrxbe.onion/newsgraphics/2020/06/17/tmag-adobe/assets/images/moscone-460.jpg}}

Marina Moscone

Fashion Designer

\href{https://www.nytimes3xbfgragh.onion/2020/08/10/t-magazine/amber-pinkerton-photography.html}{\includegraphics{https://static01.graylady3jvrrxbe.onion/newsgraphics/2020/06/17/tmag-adobe/assets/images/pinkerton-460.jpg}}

Amber Pinkerton

Photographer

\href{https://www.nytimes3xbfgragh.onion/2020/08/10/t-magazine/sonoko-sakai-chef-cooking-soba.html}{\includegraphics{https://static01.graylady3jvrrxbe.onion/newsgraphics/2020/06/17/tmag-adobe/assets/images/sakai-460.jpg}}

Sonoko Sakai

Cookbook Author and Food Activist

\href{https://www.nytimes3xbfgragh.onion/2020/08/10/t-magazine/daniela-soto-innes-cooking-chef.html}{\includegraphics{https://static01.graylady3jvrrxbe.onion/newsgraphics/2020/06/17/tmag-adobe/assets/images/ines-460.jpg}}

Daniela Soto-Innes

Chef

Advertisement

\protect\hyperlink{after-bottom}{Continue reading the main story}

\hypertarget{site-index}{%
\subsection{Site Index}\label{site-index}}

\hypertarget{site-information-navigation}{%
\subsection{Site Information
Navigation}\label{site-information-navigation}}

\begin{itemize}
\tightlist
\item
  \href{https://help.nytimes3xbfgragh.onion/hc/en-us/articles/115014792127-Copyright-notice}{©~2020~The
  New York Times Company}
\end{itemize}

\begin{itemize}
\tightlist
\item
  \href{https://www.nytco.com/}{NYTCo}
\item
  \href{https://help.nytimes3xbfgragh.onion/hc/en-us/articles/115015385887-Contact-Us}{Contact
  Us}
\item
  \href{https://www.nytco.com/careers/}{Work with us}
\item
  \href{https://nytmediakit.com/}{Advertise}
\item
  \href{http://www.tbrandstudio.com/}{T Brand Studio}
\item
  \href{https://www.nytimes3xbfgragh.onion/privacy/cookie-policy\#how-do-i-manage-trackers}{Your
  Ad Choices}
\item
  \href{https://www.nytimes3xbfgragh.onion/privacy}{Privacy}
\item
  \href{https://help.nytimes3xbfgragh.onion/hc/en-us/articles/115014893428-Terms-of-service}{Terms
  of Service}
\item
  \href{https://help.nytimes3xbfgragh.onion/hc/en-us/articles/115014893968-Terms-of-sale}{Terms
  of Sale}
\item
  \href{https://spiderbites.nytimes3xbfgragh.onion}{Site Map}
\item
  \href{https://help.nytimes3xbfgragh.onion/hc/en-us}{Help}
\item
  \href{https://www.nytimes3xbfgragh.onion/subscription?campaignId=37WXW}{Subscriptions}
\end{itemize}
