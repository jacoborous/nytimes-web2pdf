Sections

SEARCH

\protect\hyperlink{site-content}{Skip to
content}\protect\hyperlink{site-index}{Skip to site index}

\href{https://www.nytimes3xbfgragh.onion/section/sports/hockey}{Hockey}

\href{https://myaccount.nytimes3xbfgragh.onion/auth/login?response_type=cookie\&client_id=vi}{}

\href{https://www.nytimes3xbfgragh.onion/section/todayspaper}{Today's
Paper}

\href{/section/sports/hockey}{Hockey}\textbar{}Hockey, Rocked by Racist
Acts, Embraces Black Lives Matter Campaigns

\url{https://nyti.ms/2DKW57Q}

\begin{itemize}
\item
\item
\item
\item
\item
\item
\end{itemize}

\hypertarget{race-and-america}{%
\subsubsection{\texorpdfstring{\href{https://www.nytimes3xbfgragh.onion/news-event/george-floyd-protests-minneapolis-new-york-los-angeles?name=styln-george-floyd\&region=TOP_BANNER\&variant=undefined\&block=storyline_menu_recirc\&action=click\&pgtype=Article\&impression_id=434dc170-e38b-11ea-8c52-fd2265771867}{Race
and America}}{Race and America}}\label{race-and-america}}

\begin{itemize}
\tightlist
\item
  \href{https://www.nytimes3xbfgragh.onion/interactive/2020/07/03/us/george-floyd-protests-crowd-size.html?name=styln-george-floyd\&region=TOP_BANNER\&variant=undefined\&block=storyline_menu_recirc\&action=click\&pgtype=Article\&impression_id=434de880-e38b-11ea-8c52-fd2265771867}{Black
  Lives Matter Movement}
\item
  \href{https://www.nytimes3xbfgragh.onion/interactive/2020/06/28/us/i-cant-breathe-police-arrest.html?name=styln-george-floyd\&region=TOP_BANNER\&variant=undefined\&block=storyline_menu_recirc\&action=click\&pgtype=Article\&impression_id=434de881-e38b-11ea-8c52-fd2265771867}{History
  of `I Can't Breathe'}
\item
  \href{https://www.nytimes3xbfgragh.onion/interactive/2020/06/10/upshot/black-lives-matter-attitudes.html?name=styln-george-floyd\&region=TOP_BANNER\&variant=undefined\&block=storyline_menu_recirc\&action=click\&pgtype=Article\&impression_id=434de882-e38b-11ea-8c52-fd2265771867}{How
  Public Opinion Shifted}
\item
  \href{https://www.nytimes3xbfgragh.onion/interactive/2020/07/16/us/black-lives-matter-protests-louisville-breonna-taylor.html?name=styln-george-floyd\&region=TOP_BANNER\&variant=undefined\&block=storyline_menu_recirc\&action=click\&pgtype=Article\&impression_id=434de883-e38b-11ea-8c52-fd2265771867}{45
  Days in Louisville}
\end{itemize}

Advertisement

\protect\hyperlink{after-top}{Continue reading the main story}

Supported by

\protect\hyperlink{after-sponsor}{Continue reading the main story}

\hypertarget{hockey-rocked-by-racist-acts-embraces-black-lives-matter-campaigns}{%
\section{Hockey, Rocked by Racist Acts, Embraces Black Lives Matter
Campaigns}\label{hockey-rocked-by-racist-acts-embraces-black-lives-matter-campaigns}}

To a degree not seen before, players are kneeling for the United States
national anthem and speaking out publicly on fighting racism, with
backing from the National Hockey League.

\includegraphics{https://static01.graylady3jvrrxbe.onion/images/2020/08/10/sports/10nhl-blm-print/merlin_175211517_4139799c-59b9-4846-8db1-1fef12d256ad-articleLarge.jpg?quality=75\&auto=webp\&disable=upscale}

By Morgan Campbell

\begin{itemize}
\item
  Published Aug. 10, 2020Updated Aug. 11, 2020
\item
  \begin{itemize}
  \item
  \item
  \item
  \item
  \item
  \item
  \end{itemize}
\end{itemize}

Minnesota Wild forward Matt Dumba walked to center ice before an N.H.L.
game this month and addressed the television audience, not about the
game but the need for the sport to fight racism.

``The world woke up to the existence of systematic racism and how deeply
rooted it is,'' Dumba said during a speech before two other teams, the
Chicago Blackhawks and the Edmonton Oilers, took to the ice.

That he spoke with the N.H.L.'s endorsement and while holding a
microphone bearing its logo made the gesture all the more significant
for a sport still grappling with high-profile racist incidents and the
perception that people of color --- Dumba is half Filipino ---
\href{https://www.nytimes3xbfgragh.onion/2020/04/17/sports/hockey/rangers-kandre-miller-zoom-racism.html?action=click\&module=RelatedLinks\&pgtype=Article}{aren't
welcome}.

More than two months after the killing of George Floyd and the protest
movement it has engendered, the N.H.L. has begun a high-profile effort
to make anti-racism part of its identity and, according to the N.H.L.
executive Kim Davis, part of a strategy to appeal to a younger, more
racially diverse audience.

``It's a small shift, but a big shift,'' said Davis, the league's
executive vice president for social impact, who added that she wants
``people to understand that doing the right thing is also right for the
business.''

That has meant scenes and gestures at games once thought unthinkable.

When he finished speaking and ``The Star-Spangled Banner'' began, Dumba
knelt and bowed his head. He wore a hoodie that read, ``Hockey Diversity
Alliance,'' the name of a
\href{https://www.nytimes3xbfgragh.onion/2020/06/08/sports/hockey/nhl-hockey-diversity-alliance.html}{new
initiative} begun by players to combat racism in the sport.

Chicago goaltender Malcolm Subban, who is Black, stood next to Dumba and
laid a hand on his right shoulder. Edmonton forward Darnell Nurse, who
is Black, did the same on Dumba's left as the anthem played and slogans
like ``End Racism'' and ``\#WeSkateForBLACKLIVES'' appeared on enormous
video screens above their heads.

A handful of other players have demonstrated during anthems before
postseason qualifying games. Vegas Golden Knights and Dallas Stars
players knelt last Monday. Dumba raised a fist during the United States
national anthem before the Wild played the Vancouver Canucks on Aug. 2.

The hockey world has been roiled by acts of bigotry.

In April, a group Zoom chat organized by the Rangers to introduce fans
to the prospect K'Andre Miller was derailed by hackers
\href{https://www.nytimes3xbfgragh.onion/2020/04/03/sports/hockey/03rangers-racism-kandre-miller.html}{hurling
racist slurs at him}. Three months earlier, the American Hockey League
suspended Brandon Manning of the Bakersfield Condors for using racist
insults against Bokondji Imama of the Ontario Reign.

Late in 2019, the former N.H.L. player Akim Aliu
\href{https://www.nytimes3xbfgragh.onion/2020/05/23/sports/hockey/akim-aliu-nhl-racism.html?action=click\&module=RelatedLinks\&pgtype=Article}{went
public}with a series of racist incidents --- including a minor league
staff member donning blackface to mock him, and a coach using a racial
slur --- he believes helped short-circuit his career. Aliu's first
revelations came three weeks after the veteran Canadian hockey
commentator Don Cherry was
\href{https://www.nytimes3xbfgragh.onion/2019/11/11/sports/don-cherry.html}{forced
to leave his widely watched ``Coach's Corner'' segment} on the Sportsnet
hockey broadcast after an on-air rant against immigrants.

Aliu also went public less than two weeks after the Toronto Maple Leafs
fired Mike Babcock as coach amid allegations that he fostered a toxic,
bullying workplace. That timing, said Damon Kwame Mason, who produced
and directed a documentary on race and hockey, pushed mainstream hockey
media to report more about racism and helped set the stage for the
N.H.L.'s anti-racism steps this summer.

In June, Aliu joined with six other current and former N.H.L. players to
form the
\href{https://www.nytimes3xbfgragh.onion/2020/06/08/sports/hockey/nhl-hockey-diversity-alliance.html}{Hockey
Diversity Alliance}, which will focus on youth and community engagement
to carry out its mission. In July, a range of professional
\href{https://twitter.com/TheOfficialHDA/status/1288914481456664576?s=20}{athletes
appeared} in a public service announcement for the group calling for an
end to racism in hockey and in society.

``It says to me that we're moving forward,'' said Mason, director of
``Soul on Ice,'' the 2016 documentary on pro hockey's Black history.
``It's almost like the old guard was being told, `Pack your bags.'''

Davis acknowledged that the protest movement in the United States made
the league's anti-racism action more urgent. But she emphasized that she
has been helping the league develop a comprehensive diversity and
inclusion strategy since she was hired in November 2017.

The N.H.L. remains the only major North American sports league not to
volunteer for an audit by \href{https://www.tidesport.org/}{the
Institute for Diversity and Ethics in Sport}, which publishes widely
read reports on race, gender and hiring in sports and the sports media
industry. Davis said the league would work with the institute to
establish baseline diversity statistics, and then set hiring targets.

In the meantime, Davis pointed to pregame demonstrations and the roughly
140 players who have posted on social media in support of the league's
anti-racism message as proo f that the sport's culture can change.

``It's our responsibility to set the tone for the sport regarding
racism,'' Davis said, adding, ``A measure of our success is going to be
the continued activism of our players.''

Before an exhibition game last month against the Arizona Coyotes,
members of the Golden Knights linked arms in solidarity, at the
suggestion of forward Ryan Reaves.

``For a lot of guys, kneeling isn't the way they want to show support,''
Reaves said. ``This was the best way to be able to include everybody in
it.''

Before the Golden Knights played the Stars last Monday, four players
knelt for the anthem --- Reaves and Robin Lehner from Vegas, along with
Tyler Seguin and Jason Dickinson from Dallas.

Before the Aug. 2 game between Vancouver and Minnesota, the Canucks
posted to their Instagram and Twitter accounts a meme of a Venn diagram
comparing the two cities. The circles overlapped at four phrases,
including ``Yet to hoist the cup'' and ``Justice for George Floyd.''
Twitter users quickly criticized the team for its flippant inclusion of
a racial flash point in a pregame social media marketing message.

Yet Mason, who hosts the N.H.L.'s Soul on Ice podcast, was not taken
aback. He said people like those on the Canucks' social media staff were
not yet fluent in the language of racial justice, but he said he hoped
teams would take their fight against racism beyond social media.

``It's not about symbols and messaging,'' he said. ``If I don't see some
action in the next little while, stop with the hashtags.''

Carol Schram contributed reporting.

Advertisement

\protect\hyperlink{after-bottom}{Continue reading the main story}

\hypertarget{site-index}{%
\subsection{Site Index}\label{site-index}}

\hypertarget{site-information-navigation}{%
\subsection{Site Information
Navigation}\label{site-information-navigation}}

\begin{itemize}
\tightlist
\item
  \href{https://help.nytimes3xbfgragh.onion/hc/en-us/articles/115014792127-Copyright-notice}{©~2020~The
  New York Times Company}
\end{itemize}

\begin{itemize}
\tightlist
\item
  \href{https://www.nytco.com/}{NYTCo}
\item
  \href{https://help.nytimes3xbfgragh.onion/hc/en-us/articles/115015385887-Contact-Us}{Contact
  Us}
\item
  \href{https://www.nytco.com/careers/}{Work with us}
\item
  \href{https://nytmediakit.com/}{Advertise}
\item
  \href{http://www.tbrandstudio.com/}{T Brand Studio}
\item
  \href{https://www.nytimes3xbfgragh.onion/privacy/cookie-policy\#how-do-i-manage-trackers}{Your
  Ad Choices}
\item
  \href{https://www.nytimes3xbfgragh.onion/privacy}{Privacy}
\item
  \href{https://help.nytimes3xbfgragh.onion/hc/en-us/articles/115014893428-Terms-of-service}{Terms
  of Service}
\item
  \href{https://help.nytimes3xbfgragh.onion/hc/en-us/articles/115014893968-Terms-of-sale}{Terms
  of Sale}
\item
  \href{https://spiderbites.nytimes3xbfgragh.onion}{Site Map}
\item
  \href{https://help.nytimes3xbfgragh.onion/hc/en-us}{Help}
\item
  \href{https://www.nytimes3xbfgragh.onion/subscription?campaignId=37WXW}{Subscriptions}
\end{itemize}
