Sections

SEARCH

\protect\hyperlink{site-content}{Skip to
content}\protect\hyperlink{site-index}{Skip to site index}

\href{https://www.nytimes3xbfgragh.onion/section/food}{Food}

\href{https://myaccount.nytimes3xbfgragh.onion/auth/login?response_type=cookie\&client_id=vi}{}

\href{https://www.nytimes3xbfgragh.onion/section/todayspaper}{Today's
Paper}

\href{/section/food}{Food}\textbar{}Niven Patel, a Miami Chef, Is Not
Giving Up on 2020

\url{https://nyti.ms/2DuJExj}

\begin{itemize}
\item
\item
\item
\item
\item
\item
\end{itemize}

\hypertarget{the-coronavirus-outbreak}{%
\subsubsection{\texorpdfstring{\href{https://www.nytimes3xbfgragh.onion/news-event/coronavirus?name=styln-coronavirus-national\&region=TOP_BANNER\&variant=undefined\&block=storyline_menu_recirc\&action=click\&pgtype=Article\&impression_id=bcdbc460-e38b-11ea-ab31-cfe12db18fb4}{The
Coronavirus
Outbreak}}{The Coronavirus Outbreak}}\label{the-coronavirus-outbreak}}

\begin{itemize}
\tightlist
\item
  live\href{https://www.nytimes3xbfgragh.onion/2020/08/20/world/coronavirus-covid.html?name=styln-coronavirus-national\&region=TOP_BANNER\&variant=undefined\&block=storyline_menu_recirc\&action=click\&pgtype=Article\&impression_id=bcdbc461-e38b-11ea-ab31-cfe12db18fb4}{Latest
  Updates}
\item
  \href{https://www.nytimes3xbfgragh.onion/interactive/2020/us/coronavirus-us-cases.html?name=styln-coronavirus-national\&region=TOP_BANNER\&variant=undefined\&block=storyline_menu_recirc\&action=click\&pgtype=Article\&impression_id=bcdbeb70-e38b-11ea-ab31-cfe12db18fb4}{Maps
  and Cases}
\item
  \href{https://www.nytimes3xbfgragh.onion/interactive/2020/science/coronavirus-vaccine-tracker.html?name=styln-coronavirus-national\&region=TOP_BANNER\&variant=undefined\&block=storyline_menu_recirc\&action=click\&pgtype=Article\&impression_id=bcdbeb71-e38b-11ea-ab31-cfe12db18fb4}{Vaccine
  Tracker}
\item
  \href{https://www.nytimes3xbfgragh.onion/2020/08/19/us/colleges-closing-covid.html?name=styln-coronavirus-national\&region=TOP_BANNER\&variant=undefined\&block=storyline_menu_recirc\&action=click\&pgtype=Article\&impression_id=bcdbeb72-e38b-11ea-ab31-cfe12db18fb4}{Colleges
  Closing}
\item
  \href{https://www.nytimes3xbfgragh.onion/live/2020/08/20/business/stock-market-today-coronavirus?name=styln-coronavirus-national\&region=TOP_BANNER\&variant=undefined\&block=storyline_menu_recirc\&action=click\&pgtype=Article\&impression_id=bcdbeb73-e38b-11ea-ab31-cfe12db18fb4}{Economy}
\end{itemize}

Advertisement

\protect\hyperlink{after-top}{Continue reading the main story}

Supported by

\protect\hyperlink{after-sponsor}{Continue reading the main story}

\hypertarget{niven-patel-a-miami-chef-is-not-giving-up-on-2020}{%
\section{Niven Patel, a Miami Chef, Is Not Giving Up on
2020}\label{niven-patel-a-miami-chef-is-not-giving-up-on-2020}}

A return to indoor dining may be far off, but he just opened a new,
Caribbean-inspired restaurant and is moving ahead with plans for others.

\includegraphics{https://static01.graylady3jvrrxbe.onion/images/2020/08/12/dining/12virus-patel1/merlin_175113192_dc4f0b5c-b5e0-4422-b7df-f170ded72ff6-articleLarge.jpg?quality=75\&auto=webp\&disable=upscale}

By \href{https://www.nytimes3xbfgragh.onion/by/brett-anderson}{Brett
Anderson}

\begin{itemize}
\item
  Aug. 10, 2020
\item
  \begin{itemize}
  \item
  \item
  \item
  \item
  \item
  \item
  \end{itemize}
\end{itemize}

CORAL GABLES, Fla. --- The last week in July was an especially stressful
one for Niven Patel, and for Floridians in general.

It began with Mr. Patel's decision to permanently close one of two
locations of \href{https://www.gheemiami.com/}{Ghee Indian Kitchen}, the
Miami-area restaurant that brought the chef national acclaim. The week
ended with South Florida
\href{https://www.nytimes3xbfgragh.onion/2020/08/02/us/florida-hurricane-isaias-coronavirus.html}{bracing}
for the arrival of
\href{https://www.nytimes3xbfgragh.onion/2020/08/04/us/isaias-storm-updates.html}{Hurricane
Isaias}, just as the state recorded its fourth straight day of record
reported deaths from Covid-19.

In the midst of all of this --- while still adjusting to having become a
father to twin daughters in June --- Mr. Patel was busy with final
preparations for the opening of a new restaurant called
\href{https://www.thesishotelmiami.com/taste/mamey/}{Mamey}. Doing so in
a pandemic, he conceded, is ``pretty insane.''

\includegraphics{https://static01.graylady3jvrrxbe.onion/images/2020/08/12/dining/12virus-patel4/merlin_175376715_a49a20ce-8a00-4054-8483-0fe8a9d9e155-articleLarge.jpg?quality=75\&auto=webp\&disable=upscale}

Mr. Patel, 36, is one of Miami's best-known and promising young chefs
--- in May, Food \& Wine magazine anointed him one of the country's
\href{https://www.foodandwine.com/chefs/best-new-chefs-2020-niven-patel}{10
Best New Chefs}. At a time when many high-end restaurateurs are putting
their careers and businesses on pause in the face of spiraling health
and economic crises, Mr. Patel is forging ahead --- in his quiet,
cautious way --- with Mamey, in an attempt to salvage what was shaping
up to be a banner year.

The new restaurant, which opened last Wednesday in Coral Gables, serves
food and drinks only for takeout, delivery and limited outdoor dining,
in keeping with Miami-Dade County's ban on indoor service. But even with
those restrictions, Mr. Patel was eager to act.

Image

Outside, near the pool, at the THesis Hotel, where Mamey is located, on
opening night. Miami-Dade County requires that diners wear masks in
restaurants, but they can be taken off at tables.Credit...Alfonso Duran
for The New York Times

``It's honestly been very mentally refreshing, to get into the kitchen
and start creating new dishes that we've envisioned now for a year and a
half,'' he said. ``Where at Ghee it's been about survival, this is
new.''

Mr. Patel has had big plans. Mamey, which takes its name from a fruit
popular in Latin America and South Florida, was originally scheduled to
open in the spring, along with another restaurant, Orno, both of them in
\href{https://www.paseodelariviera.com/}{Paseo de la Riviera}, a new
development in Coral Gables that includes apartments and the
\href{https://www.paseodelariviera.com/thesis-hotel/}{THesis Hotel}.

The openings were put on hold in March, when city and county officials
shut down Miami-area restaurants; so was that of a permanent location
for
\href{https://www.miamiherald.com/miami-com/restaurants/article236003993.html}{Erba},
a pasta restaurant owned by Mr. Patel that had a popular run as a pop-up
in 2019. The photo shoot for a Food \& Wine article on the top chefs was
also in March --- a day after Mr. Patel laid off all 64 of his
employees. ``It's why I'm not smiling,'' he said.

Two of those workers were farmers who tended to
\href{https://www.nytimes3xbfgragh.onion/2017/08/14/dining/florida-chef-niven-patel-backyard-farm.html}{Rancho
Patel, the two-acre farm} that surrounds his family's house in
Homestead, about 30 miles south of Coral Gables. Weeds have since taken
over the swath of tilled land in the backyard where he grew produce and
herbs for Ghee Indian Kitchen.

``We can do some serious damage with this plot,'' Mr. Patel said as he
kicked at rocks in the soil one evening. He recently hired
\href{https://www.tinyfarmmiami.com/}{Roberto Grossman}, a farmer in
Homestead, to revive Rancho Patel. ``We just need to put some love back
into it.''

Image

Taro grows on the side of Mr. Patel's home in Homestead, Fla., which is
also the site of his farm, Rancho Patel.Credit...Alfonso Duran for The
New York Times

A bright spot at Rancho Patel has been the tropical fruit that
flourishes in the blazing-hot Florida summer and that Mr. Patel
harvests, with the help of his father in-law. Much of it --- including
mangoes, lychees and guanábana --- will show up in the food and
cocktails at Mamey, whose opening is the first step in providing the
chef a larger canvas to showcase the full range of his talent.

\hypertarget{latest-updates-the-coronavirus-outbreak}{%
\section{\texorpdfstring{\href{https://www.nytimes3xbfgragh.onion/2020/08/20/world/coronavirus-covid.html?action=click\&pgtype=Article\&state=default\&region=MAIN_CONTENT_1\&context=storylines_live_updates}{Latest
Updates: The Coronavirus
Outbreak}}{Latest Updates: The Coronavirus Outbreak}}\label{latest-updates-the-coronavirus-outbreak}}

Updated 2020-08-21T07:46:15.883Z

\begin{itemize}
\tightlist
\item
  \href{https://www.nytimes3xbfgragh.onion/2020/08/20/world/coronavirus-covid.html?action=click\&pgtype=Article\&state=default\&region=MAIN_CONTENT_1\&context=storylines_live_updates\#link-68774d88}{Shutdowns,
  warnings and scoldings follow alarming incidents on college campuses.}
\item
  \href{https://www.nytimes3xbfgragh.onion/2020/08/20/world/coronavirus-covid.html?action=click\&pgtype=Article\&state=default\&region=MAIN_CONTENT_1\&context=storylines_live_updates\#link-26b58724}{Biden
  knocks Trump's pandemic response, and outlines a national strategy.}
\item
  \href{https://www.nytimes3xbfgragh.onion/2020/08/20/world/coronavirus-covid.html?action=click\&pgtype=Article\&state=default\&region=MAIN_CONTENT_1\&context=storylines_live_updates\#link-4e542da3}{U.S.
  health agencies announce moves to confront the flu season and
  plummeting child vaccination rates.}
\end{itemize}

\href{https://www.nytimes3xbfgragh.onion/2020/08/20/world/coronavirus-covid.html?action=click\&pgtype=Article\&state=default\&region=MAIN_CONTENT_1\&context=storylines_live_updates}{See
more updates}

More live coverage:
\href{https://www.nytimes3xbfgragh.onion/live/2020/08/20/business/stock-market-today-coronavirus?action=click\&pgtype=Article\&state=default\&region=MAIN_CONTENT_1\&context=storylines_live_updates}{Markets}

The cuisine at Ghee, which opened its first location in 2017, reflects
the chef's Indian roots; both his and his wife's families hail from the
Indian state of Gujarat.

Image

The Mamey's Swizzle is a spiced rum drink with coconut, chai and
mamey.Credit...Alfonso Duran for The New York Times

Image

A tray of tropical fruit at Mamey includes star fruit, mango and
mamey.Credit...Alfonso Duran for The New York Times

But Mr. Patel was born in Valdosta, Ga., and raised in Jacksonville,
Fla. He made a name for himself locally while cooking New American food
as chef de cuisine at \href{https://michaelsgenuine.com/}{Michael's
Genuine}, the flagship restaurant of Michael Schwartz, a star Miami
chef, after working in a restaurant on Grand Cayman, in the Caribbean.

``My whole background doesn't have anything to do with Indian food,''
Mr. Patel said. ``I get most excited about exploring all this other kind
of food.''

Mamey's menu combines ideas from the cuisines of island nations --- ``We
don't want boundaries,'' Mr. Patel said --- but with an emphasis on the
Caribbean.

Image

Dishes at Mamey, clockwise from top left: a Caesar salad; wahoo ceviche;
roti stuffed with callaloo and chickpeas; roasted beet and snow pea
salad; and a quinoa and avocado salad.~Credit...Alfonso Duran for The
New York Times

At a menu tasting for his corporate partners in Mamey's dining room a
week before the opening, the chef introduced a dish of plantains as a
personal favorite. Brown and custardy, they had been roasted in ghee,
and finished with pickled onions and fresh cilantro. The chef noted how
the flavors of Jamaican jerk were subtly expressed in the spiced yogurt
beneath the fruit.

Mohamed Alkassar, director of operations for
\href{https://www.nrinternational.com/}{Nolan Reynolds International},
the company that developed Paseo de la Riviera, sat at the end of a long
table in a room behind Mamey's bar. ``What I like about Chef's food the
most, it's the simplest dish on the menu that ends up surprising me the
most,'' he said.

Image

From left, Mr. Patel discussed Mamey's menu with his business
partners~Mohamed Alkassar and Brent Reynolds during a tasting in the
week before opening day.Credit...Alfonso Duran for The New York Times

Mr. Alkassar, 33, had his laptop open to renderings of what the
restaurant will look like by next week, after all of the design elements
are in place: Foliage covers the ceiling above low-hanging light shades
in the two dining rooms, and potted palm trees appear where boxes of
takeout containers were stacked chest-high.

``It was a big investment --- a couple hundred thousand dollars --- when
we were making the last finishing touches on Mamey,'' Mr. Alkassar said.
``One of the conversations we had to have is, do we make that final
investment? When we're not even going to open for inside dining?''

He said the company's decision to proceed with Mamey and Orno, which is
now scheduled to open early next year with a menu focused on local
ingredients cooked in a wood-burning oven, was a vote of confidence in
Mr. Patel, as well as in the hotel. The development is across the street
from the University of Miami, which begins in-person classes on Monday.

Image

Mr. Patel and Mr. Alkassar in Mamey's kitchen. Though the restaurant has
no indoor dining, they hope to sell to students returning to the
University of Miami campus across the street.Credit...Alfonso Duran for
The New York Times

The school's students and faculty, Mr. Alkassar said, ``are a big part
of our revenue model, for both the hotel and the restaurants.''

Mr. Patel runs the hotel's food and beverage programs in partnership
with Nolan Reynolds. The team developed multiple business plans for
Mamey over the spring and summer, as Miami-area restaurants struggled to
navigate changing government restrictions.

The menu was rewritten with takeout in mind. Mr. Patel and his staff
tested how dishes would perform by putting them in cardboard to-go boxes
and waiting 15 minutes.

``Has it lost temperature? Has it lost taste? Does it travel well?''
said Mr. Patel. ``There are all of these factors you have to consider
that before we wouldn't have even thought of.''

Image

Alina Tellez, the assistant general manager of Mamey, holding a takeout
bag with gerbera daisies, which Mr. Patel decided to add to takeout
orders.Credit...Alfonso Duran for The New York Times

Image

Mr. Patel preparing a takeout order of pork belly pad
Thai.Credit...Alfonso Duran for The New York Times

The test included a coconut-lime ceviche of Florida wahoo, summer rolls
filled with pickled shiitakes and fresh avocado, and a jerk chicken
sandwich with jackfruit barbecue sauce.

Tim Piazza, the executive chef for Mr. Patel's restaurant company, Aya
Hospitality, still worried that some dishes were less suited to takeout
than others.

\href{https://www.nytimes3xbfgragh.onion/news-event/coronavirus?action=click\&pgtype=Article\&state=default\&region=MAIN_CONTENT_3\&context=storylines_faq}{}

\hypertarget{the-coronavirus-outbreak-}{%
\subsubsection{The Coronavirus Outbreak
›}\label{the-coronavirus-outbreak-}}

\hypertarget{frequently-asked-questions}{%
\paragraph{Frequently Asked
Questions}\label{frequently-asked-questions}}

Updated August 17, 2020

\begin{itemize}
\item ~
  \hypertarget{why-does-standing-six-feet-away-from-others-help}{%
  \paragraph{Why does standing six feet away from others
  help?}\label{why-does-standing-six-feet-away-from-others-help}}

  \begin{itemize}
  \tightlist
  \item
    The coronavirus spreads primarily through droplets from your mouth
    and nose, especially when you cough or sneeze. The C.D.C., one of
    the organizations using that measure,
    \href{https://www.nytimes3xbfgragh.onion/2020/04/14/health/coronavirus-six-feet.html?action=click\&pgtype=Article\&state=default\&region=MAIN_CONTENT_3\&context=storylines_faq}{bases
    its recommendation of six feet} on the idea that most large droplets
    that people expel when they cough or sneeze will fall to the ground
    within six feet. But six feet has never been a magic number that
    guarantees complete protection. Sneezes, for instance, can launch
    droplets a lot farther than six feet,
    \href{https://jamanetwork.com/journals/jama/fullarticle/2763852}{according
    to a recent study}. It's a rule of thumb: You should be safest
    standing six feet apart outside, especially when it's windy. But
    keep a mask on at all times, even when you think you're far enough
    apart.
  \end{itemize}
\item ~
  \hypertarget{i-have-antibodies-am-i-now-immune}{%
  \paragraph{I have antibodies. Am I now
  immune?}\label{i-have-antibodies-am-i-now-immune}}

  \begin{itemize}
  \tightlist
  \item
    As of right
    now,\href{https://www.nytimes3xbfgragh.onion/2020/07/22/health/covid-antibodies-herd-immunity.html?action=click\&pgtype=Article\&state=default\&region=MAIN_CONTENT_3\&context=storylines_faq}{that
    seems likely, for at least several months.} There have been
    frightening accounts of people suffering what seems to be a second
    bout of Covid-19. But experts say these patients may have a
    drawn-out course of infection, with the virus taking a slow toll
    weeks to months after initial exposure. People infected with the
    coronavirus typically
    \href{https://www.nature.com/articles/s41586-020-2456-9}{produce}
    immune molecules called antibodies, which are
    \href{https://www.nytimes3xbfgragh.onion/2020/05/07/health/coronavirus-antibody-prevalence.html?action=click\&pgtype=Article\&state=default\&region=MAIN_CONTENT_3\&context=storylines_faq}{protective
    proteins made in response to an
    infection}\href{https://www.nytimes3xbfgragh.onion/2020/05/07/health/coronavirus-antibody-prevalence.html?action=click\&pgtype=Article\&state=default\&region=MAIN_CONTENT_3\&context=storylines_faq}{.
    These antibodies may} last in the body
    \href{https://www.nature.com/articles/s41591-020-0965-6}{only two to
    three months}, which may seem worrisome, but that's perfectly normal
    after an acute infection subsides, said Dr. Michael Mina, an
    immunologist at Harvard University. It may be possible to get the
    coronavirus again, but it's highly unlikely that it would be
    possible in a short window of time from initial infection or make
    people sicker the second time.
  \end{itemize}
\item ~
  \hypertarget{im-a-small-business-owner-can-i-get-relief}{%
  \paragraph{I'm a small-business owner. Can I get
  relief?}\label{im-a-small-business-owner-can-i-get-relief}}

  \begin{itemize}
  \tightlist
  \item
    The
    \href{https://www.nytimes3xbfgragh.onion/article/small-business-loans-stimulus-grants-freelancers-coronavirus.html?action=click\&pgtype=Article\&state=default\&region=MAIN_CONTENT_3\&context=storylines_faq}{stimulus
    bills enacted in March} offer help for the millions of American
    small businesses. Those eligible for aid are businesses and
    nonprofit organizations with fewer than 500 workers, including sole
    proprietorships, independent contractors and freelancers. Some
    larger companies in some industries are also eligible. The help
    being offered, which is being managed by the Small Business
    Administration, includes the Paycheck Protection Program and the
    Economic Injury Disaster Loan program. But lots of folks have
    \href{https://www.nytimes3xbfgragh.onion/interactive/2020/05/07/business/small-business-loans-coronavirus.html?action=click\&pgtype=Article\&state=default\&region=MAIN_CONTENT_3\&context=storylines_faq}{not
    yet seen payouts.} Even those who have received help are confused:
    The rules are draconian, and some are stuck sitting on
    \href{https://www.nytimes3xbfgragh.onion/2020/05/02/business/economy/loans-coronavirus-small-business.html?action=click\&pgtype=Article\&state=default\&region=MAIN_CONTENT_3\&context=storylines_faq}{money
    they don't know how to use.} Many small-business owners are getting
    less than they expected or
    \href{https://www.nytimes3xbfgragh.onion/2020/06/10/business/Small-business-loans-ppp.html?action=click\&pgtype=Article\&state=default\&region=MAIN_CONTENT_3\&context=storylines_faq}{not
    hearing anything at all.}
  \end{itemize}
\item ~
  \hypertarget{what-are-my-rights-if-i-am-worried-about-going-back-to-work}{%
  \paragraph{What are my rights if I am worried about going back to
  work?}\label{what-are-my-rights-if-i-am-worried-about-going-back-to-work}}

  \begin{itemize}
  \tightlist
  \item
    Employers have to provide
    \href{https://www.osha.gov/SLTC/covid-19/standards.html}{a safe
    workplace} with policies that protect everyone equally.
    \href{https://www.nytimes3xbfgragh.onion/article/coronavirus-money-unemployment.html?action=click\&pgtype=Article\&state=default\&region=MAIN_CONTENT_3\&context=storylines_faq}{And
    if one of your co-workers tests positive for the coronavirus, the
    C.D.C.} has said that
    \href{https://www.cdc.gov/coronavirus/2019-ncov/community/guidance-business-response.html}{employers
    should tell their employees} -\/- without giving you the sick
    employee's name -\/- that they may have been exposed to the virus.
  \end{itemize}
\item ~
  \hypertarget{what-is-school-going-to-look-like-in-september}{%
  \paragraph{What is school going to look like in
  September?}\label{what-is-school-going-to-look-like-in-september}}

  \begin{itemize}
  \tightlist
  \item
    It is unlikely that many schools will return to a normal schedule
    this fall, requiring the grind of
    \href{https://www.nytimes3xbfgragh.onion/2020/06/05/us/coronavirus-education-lost-learning.html?action=click\&pgtype=Article\&state=default\&region=MAIN_CONTENT_3\&context=storylines_faq}{online
    learning},
    \href{https://www.nytimes3xbfgragh.onion/2020/05/29/us/coronavirus-child-care-centers.html?action=click\&pgtype=Article\&state=default\&region=MAIN_CONTENT_3\&context=storylines_faq}{makeshift
    child care} and
    \href{https://www.nytimes3xbfgragh.onion/2020/06/03/business/economy/coronavirus-working-women.html?action=click\&pgtype=Article\&state=default\&region=MAIN_CONTENT_3\&context=storylines_faq}{stunted
    workdays} to continue. California's two largest public school
    districts --- Los Angeles and San Diego --- said on July 13, that
    \href{https://www.nytimes3xbfgragh.onion/2020/07/13/us/lausd-san-diego-school-reopening.html?action=click\&pgtype=Article\&state=default\&region=MAIN_CONTENT_3\&context=storylines_faq}{instruction
    will be remote-only in the fall}, citing concerns that surging
    coronavirus infections in their areas pose too dire a risk for
    students and teachers. Together, the two districts enroll some
    825,000 students. They are the largest in the country so far to
    abandon plans for even a partial physical return to classrooms when
    they reopen in August. For other districts, the solution won't be an
    all-or-nothing approach.
    \href{https://bioethics.jhu.edu/research-and-outreach/projects/eschool-initiative/school-policy-tracker/}{Many
    systems}, including the nation's largest, New York City, are
    devising
    \href{https://www.nytimes3xbfgragh.onion/2020/06/26/us/coronavirus-schools-reopen-fall.html?action=click\&pgtype=Article\&state=default\&region=MAIN_CONTENT_3\&context=storylines_faq}{hybrid
    plans} that involve spending some days in classrooms and other days
    online. There's no national policy on this yet, so check with your
    municipal school system regularly to see what is happening in your
    community.
  \end{itemize}
\end{itemize}

``You want to be eating ceviche five minutes after you made it, not 45
minutes,'' he said.

Mamey's playful, bright-flavored food is reminiscent of the culturally
omnivorous, fruit-forward cuisine championed in Miami by chefs like
\href{https://normanvanaken.com}{Norman Van Aken},
\href{https://www.miamiherald.com/miami-com/restaurants/article244354337.html}{Cindy
Hutson} and \href{https://chefdouglasrodriguez.com}{Douglas Rodriguez}
in the 1990s and early 2000s. That often busy, Latin-Caribbean-inspired
cooking was a dominant style in local fine dining in the 2010s, when
chefs like Mr. Schwartz and Michelle Bernstein grabbed attention with
simpler dishes at more modest, bistrolike restaurants.

With Mamey, Mr. Patel joins a new generation of chefs --- Michael
Beltran, the 34-year-old chef-owner of
\href{https://arietecoconutgrove.com/}{Ariete}, is another prominent
example --- who are reorienting Miami fine dining around cuisines
brought to South Florida over the years by immigrants from Latin America
and the Caribbean. They're doing so at a moment when others are looking
at the cooking of Miami's recent past with fresh eyes.

``When I started, I said, `I'm not going to put fruit on anything,'''
said Ms. Bernstein, 51, a Miami native whose restaurant in Little
Havana, \href{https://www.cafelatrova.com/}{Café La Trova}, is
temporarily closed because of the coronavirus. ``I'm old enough where
I've come around. Now I can't wait to brown butter and put mango in
it.''

Mr. Patel describes Mamey's menu in practical terms. ``The food fits in
with the demographics of the area,'' he said. ``Plus, I spent
two-and-a-half years in the Cayman Islands. I really love that
cuisine.''

His colleagues invariably bring up his calm demeanor and lack of ego as
defining characteristics. Mr. Patel speaks quietly. His taste for
brightly colored socks --- one afternoon at Racho Patel, he wore a
green, avocado-themed pair --- may be the loudest thing about him.

``He's very collected,'' Mr. Schwartz said. ``That struck me the first
time I met him. A lot of young chefs are kind of frenetic and chaotic.''

Mr. Patel's even temper has certainly been tested during the pandemic.
In June, after seven years of trying to start a family, he and his wife,
Shivani, welcomed twin daughters into their home. Ms. Patel, a co-owner
of Aya Hospitality, stepped away from the restaurants to care for the
newborns, who were carried by a surrogate.

The babies' arrival also raised the level of concern that Mr. Patel
could be exposed to the coronavirus through work --- and could bring it
into the house he shares with his in-laws.

``We're trying to create a culture of being safe,'' Mr. Patel said.
``One person not being safe outside of work could affect the entire
enterprise.''

Image

Tim Piazza, the executive chef of Mr. Patel's restaurant company, has
been quarantining at home, to guarantee there is a healthy chef in case
others become ill.Credit...Alfonso Duran for The New York Times

Brenda Perez took a job as a server at Mamey in part, she said, because
of the cautious way Mr. Patel has managed the original Ghee. The chef
ended indoor dining at the restaurant in early July, just before the
county mandated it.

``The way they responded to the virus this entire time has been very
responsible,'' said Ms. Perez, 37, whose boyfriend works at Ghee. ``They
care about their employees and how they feel.''

In June, as the coronavirus outbreak in Florida grew more worrisome, Mr.
Patel and Mr. Alkassar asked Mr. Piazza, the restaurant group's
executive chef, to quarantine at home. Mr. Piazza's wife had just tested
positive for the coronavirus, as had their nanny and the nanny's
husband.

``When this happened, Niven was like, `Stay home,''' Mr. Piazza said.

Even after tests for his wife came back negative, though, the team
insisted that Mr. Piazza remain home, to guarantee a healthy chef in
case others fell sick.

Mr. Piazza, whom Mr. Patel refers to as ``my right hand,'' has spent the
time testing recipes and writing (and rewriting) menus for Mamey.

```The hotel is a big deal for a lot of people,''' Mr. Piazza said Mr.
Patel told him. ```You're the only guy who I would trust here if
something happened to me.'''

Image

Takeout orders are delivered to THesis Hotel guest rooms. Mr. Patel runs
the food and beverage service for the hotel.Credit...Alfonso Duran for
The New York Times

Mr. Patel's team was still making adjustments on opening night. Salsa
verde was added to the piri piri roast chicken. At the last minute, Mr.
Patel decided to tuck a gerbera daisy into every branded takeout bag.
With the THesis Hotel booked to over 60 percent capacity in its first
week of business, there was even talk of growing Mamey's staff of 34
employees.

Mr. Alkassar stood near the outdoor pool bar on the hotel's third floor,
where diners ate protected from the rain. He has opened more than 20
restaurants, he said, and despite all the challenges, he can't remember
another opening that went as smoothly as Mamey's.

To prove his point, he pulled up a string of texts between Mr. Patel and
himself on his smartphone. One from Mr. Patel read, ``Thanks for letting
me always just be who I am.''

Mamey, in the THesis Hotel, 1350 South Dixie Highway, Coral Gables,
Fla.; 888-304-5055;
\href{https://www.thesishotelmiami.com/taste/mamey/}{thesishotelmiami.com/taste/mamey}.

\emph{Follow} \href{https://twitter.com/nytfood}{\emph{NYT Food on
Twitter}} \emph{and}
\href{https://www.instagram.com/nytcooking/}{\emph{NYT Cooking on
Instagram}}\emph{,}
\href{https://www.facebookcorewwwi.onion/nytcooking/}{\emph{Facebook}}\emph{,}
\href{https://www.youtube.com/nytcooking}{\emph{YouTube}} \emph{and}
\href{https://www.pinterest.com/nytcooking/}{\emph{Pinterest}}\emph{.}
\href{https://www.nytimes3xbfgragh.onion/newsletters/cooking}{\emph{Get
regular updates from NYT Cooking, with recipe suggestions, cooking tips
and shopping advice}}\emph{.}

Advertisement

\protect\hyperlink{after-bottom}{Continue reading the main story}

\hypertarget{site-index}{%
\subsection{Site Index}\label{site-index}}

\hypertarget{site-information-navigation}{%
\subsection{Site Information
Navigation}\label{site-information-navigation}}

\begin{itemize}
\tightlist
\item
  \href{https://help.nytimes3xbfgragh.onion/hc/en-us/articles/115014792127-Copyright-notice}{©~2020~The
  New York Times Company}
\end{itemize}

\begin{itemize}
\tightlist
\item
  \href{https://www.nytco.com/}{NYTCo}
\item
  \href{https://help.nytimes3xbfgragh.onion/hc/en-us/articles/115015385887-Contact-Us}{Contact
  Us}
\item
  \href{https://www.nytco.com/careers/}{Work with us}
\item
  \href{https://nytmediakit.com/}{Advertise}
\item
  \href{http://www.tbrandstudio.com/}{T Brand Studio}
\item
  \href{https://www.nytimes3xbfgragh.onion/privacy/cookie-policy\#how-do-i-manage-trackers}{Your
  Ad Choices}
\item
  \href{https://www.nytimes3xbfgragh.onion/privacy}{Privacy}
\item
  \href{https://help.nytimes3xbfgragh.onion/hc/en-us/articles/115014893428-Terms-of-service}{Terms
  of Service}
\item
  \href{https://help.nytimes3xbfgragh.onion/hc/en-us/articles/115014893968-Terms-of-sale}{Terms
  of Sale}
\item
  \href{https://spiderbites.nytimes3xbfgragh.onion}{Site Map}
\item
  \href{https://help.nytimes3xbfgragh.onion/hc/en-us}{Help}
\item
  \href{https://www.nytimes3xbfgragh.onion/subscription?campaignId=37WXW}{Subscriptions}
\end{itemize}
