Sections

SEARCH

\protect\hyperlink{site-content}{Skip to
content}\protect\hyperlink{site-index}{Skip to site index}

\href{https://www.nytimes3xbfgragh.onion/section/your-money}{Your Money}

\href{https://myaccount.nytimes3xbfgragh.onion/auth/login?response_type=cookie\&client_id=vi}{}

\href{https://www.nytimes3xbfgragh.onion/section/todayspaper}{Today's
Paper}

\href{/section/your-money}{Your Money}\textbar{}New Fee on Some College
Bills: It's for the Virus

\url{https://nyti.ms/346fHOG}

\begin{itemize}
\item
\item
\item
\item
\item
\end{itemize}

\hypertarget{schools-reopening}{%
\subsubsection{\texorpdfstring{\href{https://www.nytimes3xbfgragh.onion/spotlight/schools-reopening?name=styln-coronavirus-schools-reopening\&region=TOP_BANNER\&variant=undefined\&block=storyline_menu_recirc\&action=click\&pgtype=Article\&impression_id=14274e90-e384-11ea-bf7b-edaacc4ea387}{Schools
Reopening}}{Schools Reopening}}\label{schools-reopening}}

\begin{itemize}
\tightlist
\item
  \href{https://www.nytimes3xbfgragh.onion/2020/08/19/us/colleges-closing-covid.html?name=styln-coronavirus-schools-reopening\&region=TOP_BANNER\&variant=undefined\&block=storyline_menu_recirc\&action=click\&pgtype=Article\&impression_id=142775a0-e384-11ea-bf7b-edaacc4ea387}{Colleges
  Closing}
\item
  \href{https://www.nytimes3xbfgragh.onion/2020/08/20/us/schools-reopening-nurses-covid.html?name=styln-coronavirus-schools-reopening\&region=TOP_BANNER\&variant=undefined\&block=storyline_menu_recirc\&action=click\&pgtype=Article\&impression_id=142775a1-e384-11ea-bf7b-edaacc4ea387}{Missing
  School Nurses}
\item
  \href{https://www.nytimes3xbfgragh.onion/2020/08/18/parenting/homeschool-families.html?name=styln-coronavirus-schools-reopening\&region=TOP_BANNER\&variant=undefined\&block=storyline_menu_recirc\&action=click\&pgtype=Article\&impression_id=142775a2-e384-11ea-bf7b-edaacc4ea387}{Home-Schooling
  Families}
\item
  \href{https://www.nytimes3xbfgragh.onion/2020/08/05/parenting/parents-distance-learning.html?name=styln-coronavirus-schools-reopening\&region=TOP_BANNER\&variant=undefined\&block=storyline_menu_recirc\&action=click\&pgtype=Article\&impression_id=142775a3-e384-11ea-bf7b-edaacc4ea387}{Prepare
  for Distance Learning}
\end{itemize}

Advertisement

\protect\hyperlink{after-top}{Continue reading the main story}

Supported by

\protect\hyperlink{after-sponsor}{Continue reading the main story}

your money adviser

\hypertarget{new-fee-on-some-college-bills-its-for-the-virus}{%
\section{New Fee on Some College Bills: It's for the
Virus}\label{new-fee-on-some-college-bills-its-for-the-virus}}

Students are being asked to share the costs of testing and reconfiguring
campus facilities. The fees range from \$50 to \$475 a semester.

\includegraphics{https://static01.graylady3jvrrxbe.onion/images/2020/08/14/business/14Adviser-illo/14Adviser-illo-articleLarge.jpg?quality=75\&auto=webp\&disable=upscale}

By \href{https://www.nytimes3xbfgragh.onion/by/ann-carrns}{Ann Carrns}

\begin{itemize}
\item
  Aug. 14, 2020
\item
  \begin{itemize}
  \item
  \item
  \item
  \item
  \item
  \end{itemize}
\end{itemize}

College students are used to seeing fees on their semester bills:
activity fees, lab fees, athletic fees, technology fees, orientation
fees and so on.

This year, some students are noticing a new item: coronavirus fees.

Faced with extra expenses for screening and testing students for the
virus, and for reconfiguring campus facilities for safety, some colleges
and universities are asking students to pay a share of the cost.

The level of testing and protective steps, and the associated cost, vary
widely by campus. Some colleges are testing all students at the start of
the semester, while others will also test repeatedly throughout the
academic term. Testing is mandatory at some campuses, voluntary at
others. ``It really varies,'' said Lynn Pasquerella, president of the
Association of American Colleges and Universities.

The University of Michigan is charging a \$50-per-term coronavirus fee
this year. Revenue from the fee will help cover the costs of testing and
other pandemic-related health and safety services, a spokesman said.
Details of the measures are still being worked out.

\href{https://www.merrimack.edu/covid-19/tuition-and-financial-aid/}{Merrimack
College}, a private institution in North Andover, Mass., is charging a
``Covid mitigation'' fee of \$475 per semester to all students taking
in-person classes.

The college requires students to test negative for the coronavirus
before moving into their dorms, and plans to conduct weekly surveillance
testing throughout the semester --- with some 4,500 on-campus tests
expected weekly, according to its website. Merrimack is participating in
a college testing protocol offered by the Broad Institute, an initiative
of Harvard and M.I.T. that developed a program to help campuses reopen
safely.

The college didn't respond to requests for comment. But its website said
that even with budget cuts, the ``extraordinary'' costs of testing and
safety measures ``are difficult to absorb,'' so a temporary fee was
necessary.

Other colleges may still be calculating whether and how to charge fees,
since plans for testing and safety protocols are changing daily as the
start of the academic year approaches, health experts say. Students are
already heading back to some campuses, but others won't show up until
after Labor Day.

``This is all still emerging,'' said Elizabeth Marks, senior strategy
consultant with Academic HealthPlans, which provides student health
insurance plans at campuses across the country.

The Centers for Disease Control and Prevention doesn't currently
recommend blanket ``entry testing'' of returning students, faculty and
staff. The agency's website
\href{https://www.cdc.gov/coronavirus/2019-ncov/community/colleges-universities/ihe-testing.html}{notes
that such a step hasn't been systematically studied}, and it is
``unknown'' if it would reduce transmission of the virus beyond what
would be expected by using other prevention measures, like social
distancing, masks and hand washing.

But many colleges that are inviting students back to campus are taking
aggressive steps to avoid outbreaks. Large universities may have the
infrastructure to conduct multiple tests rapidly on thousands of
students, Ms. Pasquerella said, but smaller institutions may lack the
facilities --- or the funds --- to handle a large volume of tests.

\href{https://www.baylor.edu/president/news.php?action=story\&story=219855}{Baylor
University} said it was sending home test kits to all students, and is
requiring negative results before students arrive on the campus in Waco,
Texas. The school will also conduct testing throughout the semester.
Baylor is covering the costs, a spokeswoman said.

\href{https://www.elon.edu/u/ready-and-resilient/health-wellness/required-testing-for-covid-19/\#process-and-instructions}{Elon
University}, a private institution in North Carolina with about 6,300
undergraduates, is also sending home testing kits to incoming students.
The university will charge students \$129, but is giving them time to
seek reimbursement from their health insurer or apply for a fee waiver
before billing them. Students can seek tests elsewhere, at a lower cost,
as long as the test meets Elon's requirements. Random testing will occur
throughout the fall, at the university's expense.

``We know testing is imperfect,'' said Jeff Stein, Elon's vice president
for strategic initiatives. But the school hopes that the tests, combined
with other protective steps, will help contain the virus's spread.

\href{https://www.smcvt.edu/return-to-campus/information-for-students/}{St.
Michael's College} in Vermont is charging all students a
``comprehensive'' testing fee of \$150 for the fall semester, which
includes testing at the start of the semester and repeat tests during
the fall. ``We know that this is a particularly difficult time
financially for many families and we wish we did not have to charge any
fee,'' the college says on its website.

\href{https://www.nytimes3xbfgragh.onion/spotlight/schools-reopening?action=click\&pgtype=Article\&state=default\&region=MAIN_CONTENT_3\&context=storylines_keepup}{}

\hypertarget{schools-reopening-}{%
\subsubsection{Schools Reopening ›}\label{schools-reopening-}}

\hypertarget{back-to-school}{%
\paragraph{Back to School}\label{back-to-school}}

Updated Aug. 20, 2020

The latest on how schools are reopening amid the pandemic.

\begin{itemize}
\item
  \begin{itemize}
  \tightlist
  \item
    Much more is
    \href{https://www.nytimes3xbfgragh.onion/2020/08/20/us/schools-reopening-nurses-covid.html?action=click\&pgtype=Article\&state=default\&region=MAIN_CONTENT_3\&context=storylines_keepup}{expected
    of America's school nurses} during the pandemic, but many schools
    don't have one.
  \item
    A vast majority of parents have resigned themselves to
    \href{https://www.nytimes3xbfgragh.onion/2020/08/19/us/colleges-closing-covid.html?action=click\&pgtype=Article\&state=default\&region=MAIN_CONTENT_3\&context=storylines_keepup}{going
    it alone in the pandemic school year}, according to a new survey for
    The New York Times.
  \item
    Alabama is betting that a
    \href{https://www.nytimes3xbfgragh.onion/2020/08/19/business/alabama-uab-coronavirus-tests.html?action=click\&pgtype=Article\&state=default\&region=MAIN_CONTENT_3\&context=storylines_keepup}{robust
    student testing and technology program} will be enough to hinder
    outbreaks on college campuses.
  \item
    We want to hear from teachers making difficult choices. How are you
    thinking about the start of the school year?
    \href{https://www.nytimes3xbfgragh.onion/2020/08/19/us/teachers-school-reopenings.html?action=click\&pgtype=Article\&state=default\&region=MAIN_CONTENT_3\&context=storylines_keepup}{Tell
    us here}.
  \end{itemize}
\end{itemize}

Brendan Williams, senior director of knowledge at uAspire, a nonprofit
group that advocates college affordability, said in an email that the
group applauded colleges that were being ``transparent'' about the extra
charges, rather than quietly folding them into general fees. But, he
said, ``we don't necessarily agree with passing the costs on to the
student.''

Several colleges noted that if the federal government appropriated money
to help colleges pay for testing programs, they would credit all or part
of their virus fees back to students.

Here are some questions and answers about the fees:

\textbf{Will my health insurance plan reimburse me for college-required
coronavirus tests?}

Maybe. Many insurers, in general, cover tests for the virus only if they
are deemed ``medically necessary,'' which typically means a patient has
symptoms or an order from a physician. Screening tests for people who
don't have symptoms --- which is what many colleges are doing --- may
not be covered at no cost.

St. Michael's College acknowledged that possibility. ``The college can
provide families with evidence of the payment and what it was for so
that they can seek reimbursement from their insurance company,'' the
website says. ``However, our understanding is that most insurance
companies will not reimburse for asymptomatic testing, which is what the
college will be doing in nearly all cases.''

But Stephanie Cohen, an insurance broker near Washington, D.C., said
major health insurers seemed ``likely'' to reimburse for tests required
under formal college testing programs. She advised students to contact
their health plans for clarification. Or students could visit their
doctor to explain the situation, and request a prescription for the
test.

\textbf{If I get sick with the virus while attending college, will my
campus health insurance plan cover my care?}

Health insurance plans, including those created for and sold through
colleges to students, cover coronavirus-related care and treatment in
the same way they cover other illnesses, Ms. Marks said.

Even if a student is sent home because the campus switches from
in-person to remote classes, she said, the health plan generally would
cover care and treatment as long as the student remained eligible, which
typically means the student is enrolled for a minimum number of credit
hours.

\textbf{My college bill includes a ``student health'' fee. Does that
mean I have health insurance?}

No. Most colleges charge all students a mandatory health fee, which
typically covers the cost of primary care, counseling and health
education at a campus health center; a per-visit fee may also be
charged. But the fee doesn't cover more extensive treatment. For that,
you would need insurance coverage, whether through a plan offered on
campus to students or through coverage you have on your own or through a
parent's health plan. (You can remain on your parents' health plan until
age 26.)

Advertisement

\protect\hyperlink{after-bottom}{Continue reading the main story}

\hypertarget{site-index}{%
\subsection{Site Index}\label{site-index}}

\hypertarget{site-information-navigation}{%
\subsection{Site Information
Navigation}\label{site-information-navigation}}

\begin{itemize}
\tightlist
\item
  \href{https://help.nytimes3xbfgragh.onion/hc/en-us/articles/115014792127-Copyright-notice}{©~2020~The
  New York Times Company}
\end{itemize}

\begin{itemize}
\tightlist
\item
  \href{https://www.nytco.com/}{NYTCo}
\item
  \href{https://help.nytimes3xbfgragh.onion/hc/en-us/articles/115015385887-Contact-Us}{Contact
  Us}
\item
  \href{https://www.nytco.com/careers/}{Work with us}
\item
  \href{https://nytmediakit.com/}{Advertise}
\item
  \href{http://www.tbrandstudio.com/}{T Brand Studio}
\item
  \href{https://www.nytimes3xbfgragh.onion/privacy/cookie-policy\#how-do-i-manage-trackers}{Your
  Ad Choices}
\item
  \href{https://www.nytimes3xbfgragh.onion/privacy}{Privacy}
\item
  \href{https://help.nytimes3xbfgragh.onion/hc/en-us/articles/115014893428-Terms-of-service}{Terms
  of Service}
\item
  \href{https://help.nytimes3xbfgragh.onion/hc/en-us/articles/115014893968-Terms-of-sale}{Terms
  of Sale}
\item
  \href{https://spiderbites.nytimes3xbfgragh.onion}{Site Map}
\item
  \href{https://help.nytimes3xbfgragh.onion/hc/en-us}{Help}
\item
  \href{https://www.nytimes3xbfgragh.onion/subscription?campaignId=37WXW}{Subscriptions}
\end{itemize}
