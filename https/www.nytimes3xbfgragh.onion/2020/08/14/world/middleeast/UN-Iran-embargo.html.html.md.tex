Sections

SEARCH

\protect\hyperlink{site-content}{Skip to
content}\protect\hyperlink{site-index}{Skip to site index}

\href{https://www.nytimes3xbfgragh.onion/section/world/middleeast}{Middle
East}

\href{https://myaccount.nytimes3xbfgragh.onion/auth/login?response_type=cookie\&client_id=vi}{}

\href{https://www.nytimes3xbfgragh.onion/section/todayspaper}{Today's
Paper}

\href{/section/world/middleeast}{Middle East}\textbar{}U.N. Security
Council Rejects U.S. Proposal to Extend Arms Embargo on Iran

\url{https://nyti.ms/340RWb2}

\begin{itemize}
\item
\item
\item
\item
\item
\end{itemize}

Advertisement

\protect\hyperlink{after-top}{Continue reading the main story}

Supported by

\protect\hyperlink{after-sponsor}{Continue reading the main story}

\hypertarget{un-security-council-rejects-us-proposal-to-extend-arms-embargo-on-iran}{%
\section{U.N. Security Council Rejects U.S. Proposal to Extend Arms
Embargo on
Iran}\label{un-security-council-rejects-us-proposal-to-extend-arms-embargo-on-iran}}

The defeat underscored America's deepening global isolation on Iran. But
for the Trump administration, the vote could open a separate path to try
to inflict maximum damage on Iran.

\includegraphics{https://static01.graylady3jvrrxbe.onion/images/2020/08/14/world/14un-iran/merlin_175699200_15c7736d-f566-4abe-9011-447834c3cd8e-articleLarge.jpg?quality=75\&auto=webp\&disable=upscale}

\href{https://www.nytimes3xbfgragh.onion/by/michael-schwirtz}{\includegraphics{https://static01.graylady3jvrrxbe.onion/images/2018/02/20/multimedia/author-michael-schwirtz/author-michael-schwirtz-thumbLarge-v2.jpg}}

By \href{https://www.nytimes3xbfgragh.onion/by/michael-schwirtz}{Michael
Schwirtz}

\begin{itemize}
\item
  Aug. 14, 2020
\item
  \begin{itemize}
  \item
  \item
  \item
  \item
  \item
  \end{itemize}
\end{itemize}

The United States suffered an embarrassing diplomatic defeat on Friday
when the United Nations Security Council rejected a proposal to
indefinitely extend an arms embargo on Iran, with even America's
strongest allies refusing to buckle under pressure from the Trump
administration to take a harder line.

The defeat underscored America's deepening global isolation on the issue
of Iran. But for the Trump administration, the vote could open a
separate path to try to inflict maximum damage on Iran ahead of
November's U.S. presidential election.

For months, Trump administration officials have warned that if the vote
to extend the embargo failed, the United States would try to invoke a
provision built into the Obama-era nuclear accord to punish any Iranian
violations by reimposing all sanctions lifted when the deal took effect.
That could include the prohibition of not just arms deals, but also oil
sales and banking agreements. In theory, all U.N. members would have to
adhere to the sanctions.

The provision, known as a snapback, would be devastating for Iran, which
is already struggling with a moribund economy made worse by the
coronavirus. Pursuing the snapback would also put the Trump
administration at odds with America's allies, which vehemently oppose it
as legally dubious and potentially destabilizing to the region.

``The United States has every right to initiate snapback,'' Kelly Craft,
the U.S. ambassador to the United Nations, said Friday night. ``In the
coming days, the United States will follow through on that promise to
stop at nothing to extend the arms embargo.''

That could include trying to enforce the snapback sanctions
unilaterally, without the support of allies.

Secretary of State Mike Pompeo denounced the Security Council's decision
to scuttle the embargo provision, describing it Friday evening as
``inexcusable.'' While he did not say specifically that the United
States would pursue the snapback option, he made clear that the Trump
administration had not given up on the issue of Iranian weapons.

``We will continue to work to ensure that the theocratic terror regime
does not have the freedom to purchase and sell weapons that threaten the
heart of Europe, the Middle East and beyond,'' Mr. Pompeo said.

While Friday's vote was about the duration of the arms embargo, the
heart of the dispute between the United States and its opponents on the
Security Council is the nuclear deal.

Signed in 2015, the deal freed up the Iranian economy by lifting
sanctions in exchange for Iran agreeing to halt its nuclear program. The
deal was President Obama's signature diplomatic achievement, and was
backed by some of America's closest allies, Britain, France and Germany,
as well as its strongest foes, China and Russia.

President Trump came into office vowing to dismantle the deal, insisting
he could get a better one. But when he finally withdrew the United
States from the accord in 2018, it touched off a diplomatic
conflagration that has at times escalated toward war.

Since then, Iran has
\href{https://www.nytimes3xbfgragh.onion/2019/07/01/world/middleeast/iran-uranium-enrichment-limit.html}{exceeded
nuclear enrichment limits} set by the accord and launched covert attacks
on American military targets, while the United States has
\href{https://www.nytimes3xbfgragh.onion/2020/01/02/world/middleeast/qassem-soleimani-iraq-iran-attack.html}{assassinated
Iranian military leaders} and proxies, including Qasem Soleimani, the
leader of Iran's revolutionary guards.

The arms embargo was designed to prevent Iran from buying and selling
weapons, including aircraft and tanks. It was due to expire in October,
at which point Iran would legally be able to begin replenishing its arms
stockpiles, something the Trump administration has said it would not
permit.

``We can't allow the world's biggest state sponsor of terrorism to buy
and sell weapons,'' Mr. Pompeo, the administration's leading voice on
Iran, told reporters in Vienna before Friday's vote. ``I mean, that's
just nuts.''

But Mr. Pompeo's was a lonely voice in support of the measure, which had
been put forth by the United States.

Of the 15 countries on the Security Council, only one, the Dominican
Republic, joined the United States in supporting the proposal. Major
U.S. allies --- Britain, France and Germany --- all abstained from the
vote, making a promised veto by Russia and China unnecessary. Of the 15
countries on the Security Council, Russian and China voted against the
proposal and 11 countries abstained.

In explaining their decision to abstain from the vote on Friday,
America's European allies insisted that they, too, worried about an Iran
with free access to dangerous weapons and expressed hope for further
negotiations on possible restrictions.

But the American proposal to extend the embargo indefinitely, they said,
would never have passed the Security Council because of the threat of a
veto by Russia and China.

The proposal would have needed nine yes votes, and no vetoes from the
five permanent members, to pass.

``It would therefore not contribute to improving security and stability
in the region,'' Jonathan Allen, Britain's permanent representative to
the United Nations, said in a statement after the results were
announced.

Moreover, critics of the United States position question whether the
Trump administration, having withdrawn from the nuclear accord, has
legal standing in any debate over its provisions --- including the arms
embargo and snapback. Having failed to live up to its end of the
agreement, these critics say, the Trump administration cannot insist on
having a say over whether Iran is remaining faithful to the deal.

The State Department
\href{https://www.nytimes3xbfgragh.onion/2020/04/26/world/middleeast/us-iran-nuclear-deal-pompeo.html}{is
prepared to argue}that the United States remains a ``participant state''
in the nuclear accord that Mr. Trump renounced --- but only for the
purposes of invoking the snapback.

In defending its pursuit of the embargo, Trump administration officials
have argued that Iran has been violating the arms restrictions laid out
in the 2015 nuclear agreement.

\href{https://www.nytimes3xbfgragh.onion/interactive/2019/09/16/world/middleeast/trump-saudi-arabia-oil-attack.html}{Western
intelligence agencies} along with U.N. officials have determined that
missiles used in an attack on a Saudi Arabian oil facility last year
were manufactured in Iran, as were weapons intercepted by the U.S. Navy
that were bound for Iran's Houthi allies in Yemen. Iran has dismissed
the allegations.

Though Iran initially adhered to the terms of the agreement following
the U.S. withdrawal, its leaders have more recently signaled their
displeasure by departing from key provisions, including limits on
\href{https://www.nytimes3xbfgragh.onion/2019/07/07/world/middleeast/iran-nuclear-limits-breach.html?searchResultPosition=1}{uranium
enrichment} and the
\href{https://www.nytimes3xbfgragh.onion/2019/07/01/us/politics/iran-nuclear-limit-compliance.html}{stockpiling
of nuclear fuel}.

In advance of the Security Council vote on Friday, Iran's foreign
minister, Javad Zarif, accused the Trump administration of abusing the
Security Council to advance its political agenda.

In a lengthy essay published this week on the website Medium, Mr. Zarif
wrote that the international community faced an important decision: ``Do
we maintain respect for the rule of law, or do we return to the law of
the jungle?''

In a sign of the high diplomatic stakes surrounding Friday's vote,
President Vladimir V. Putin of Russia weighed in beforehand, calling for
a special video summit aimed at preventing ``confrontation and
escalation of the situation in the Security Council.''

The proposed summit, Mr. Putin
\href{https://twitter.com/Dpol_un/status/1294284430647992321}{said in a
statement}, should consist of the leaders of Security Council member
states, plus the heads of Germany and Iran, and seek to establish a
``comprehensive security architecture in the Persian Gulf.''

``No one should resort to blackmail or dictate in this region,'' Mr.
Putin said.

Judging from the statements of Mr. Pompeo and Ms. Craft, the United
States has little patience for continued debate over the embargo.
Rather, Friday's vote could be seen more as a diplomatic formality the
Trump administration felt it had to undertake as part of a broader
effort to achieve its ultimate goal: killing the nuclear deal once and
for all.

``The objective is to bury the deal and pressure the Iranians,'' said
Robert Malley, president of International Crisis Group and a former
coordinator for the Middle East and Persian Gulf region in the Obama
administration.

Farnaz Fassihi, Lara Jakes and David E. Sanger contributed reporting.

Advertisement

\protect\hyperlink{after-bottom}{Continue reading the main story}

\hypertarget{site-index}{%
\subsection{Site Index}\label{site-index}}

\hypertarget{site-information-navigation}{%
\subsection{Site Information
Navigation}\label{site-information-navigation}}

\begin{itemize}
\tightlist
\item
  \href{https://help.nytimes3xbfgragh.onion/hc/en-us/articles/115014792127-Copyright-notice}{©~2020~The
  New York Times Company}
\end{itemize}

\begin{itemize}
\tightlist
\item
  \href{https://www.nytco.com/}{NYTCo}
\item
  \href{https://help.nytimes3xbfgragh.onion/hc/en-us/articles/115015385887-Contact-Us}{Contact
  Us}
\item
  \href{https://www.nytco.com/careers/}{Work with us}
\item
  \href{https://nytmediakit.com/}{Advertise}
\item
  \href{http://www.tbrandstudio.com/}{T Brand Studio}
\item
  \href{https://www.nytimes3xbfgragh.onion/privacy/cookie-policy\#how-do-i-manage-trackers}{Your
  Ad Choices}
\item
  \href{https://www.nytimes3xbfgragh.onion/privacy}{Privacy}
\item
  \href{https://help.nytimes3xbfgragh.onion/hc/en-us/articles/115014893428-Terms-of-service}{Terms
  of Service}
\item
  \href{https://help.nytimes3xbfgragh.onion/hc/en-us/articles/115014893968-Terms-of-sale}{Terms
  of Sale}
\item
  \href{https://spiderbites.nytimes3xbfgragh.onion}{Site Map}
\item
  \href{https://help.nytimes3xbfgragh.onion/hc/en-us}{Help}
\item
  \href{https://www.nytimes3xbfgragh.onion/subscription?campaignId=37WXW}{Subscriptions}
\end{itemize}
