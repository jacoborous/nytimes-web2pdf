Sections

SEARCH

\protect\hyperlink{site-content}{Skip to
content}\protect\hyperlink{site-index}{Skip to site index}

\href{https://www.nytimes3xbfgragh.onion/section/world/americas}{Americas}

\href{https://myaccount.nytimes3xbfgragh.onion/auth/login?response_type=cookie\&client_id=vi}{}

\href{https://www.nytimes3xbfgragh.onion/section/todayspaper}{Today's
Paper}

\href{/section/world/americas}{Americas}\textbar{}Eusebio Leal Spengler,
Who Restored Old Havana, Dies at 77

\url{https://nyti.ms/33Z3WtB}

\begin{itemize}
\item
\item
\item
\item
\item
\end{itemize}

Advertisement

\protect\hyperlink{after-top}{Continue reading the main story}

Supported by

\protect\hyperlink{after-sponsor}{Continue reading the main story}

\hypertarget{eusebio-leal-spengler-who-restored-old-havana-dies-at-77}{%
\section{Eusebio Leal Spengler, Who Restored Old Havana, Dies at
77}\label{eusebio-leal-spengler-who-restored-old-havana-dies-at-77}}

His decades-long effort to preserve a capital's colonial buildings
turned a former slum into a tourist destination and capitalist success
story.

\includegraphics{https://static01.graylady3jvrrxbe.onion/images/2020/08/18/obituaries/12Leal1/12Leal1-articleLarge.jpg?quality=75\&auto=webp\&disable=upscale}

\href{https://www.nytimes3xbfgragh.onion/by/steven-kurutz}{\includegraphics{https://static01.graylady3jvrrxbe.onion/images/2018/09/25/multimedia/author-steven-kurutz/author-steven-kurutz-thumbLarge.png}}

By \href{https://www.nytimes3xbfgragh.onion/by/steven-kurutz}{Steven
Kurutz}

\begin{itemize}
\item
  Aug. 14, 2020
\item
  \begin{itemize}
  \item
  \item
  \item
  \item
  \item
  \end{itemize}
\end{itemize}

Eusebio Leal Spengler, who led an effort to preserve Old Havana,
transforming that historic district from a forgotten slum into an
architectural jewel and tourist destination, died on July 31 in Havana.
He was 77.

His death was
\href{http://en.granma.cu/cuba/2020-07-31/eusebio-leal-spengler-has-died-forever-yours-dear-historian}{reported
by Granma}, the official newspaper of the Cuban Communist Party. In
recent years he had been treated for pancreatic cancer.

In a statement, President Miguel Diaz-Canel of Cuba called him ``the
Cuban who saved Havana.''

Mr. Leal began his preservation efforts in the 1980s, when the old
center of the capital city was a ruin. Residents lived without indoor
plumbing or reliable electricity, garbage piled up on the streets, and
250-year-old buildings sometimes collapsed before their eyes.

As a historian and director of the Havana City Museum, Mr. Leal was
passionate about saving Cuba's architectural history. He once lay down
in front of a steamroller to save a colonial-era wooden street from
being paved over. Through his campaigning, Old Havana was designated a
UNESCO\href{https://whc.unesco.org/en/list/}{World Heritage site} in
1982.

\includegraphics{https://static01.graylady3jvrrxbe.onion/images/2020/08/12/obituaries/12Spengler2/merlin_175159122_5864ae6f-2d67-43cc-a3e9-919f2b199c77-articleLarge.jpg?quality=75\&auto=webp\&disable=upscale}

But lack of funds hampered Mr. Leal's ambitious restoration plans until
the early 1990s, when the Soviet Union collapsed and Cuba lost millions
in subsidies. Out of the economic crisis, the Cuban leader Fidel Castro,
whom Mr. Leal had befriended, gave him unprecedented authority to
collect taxes and profits from tourism in the old center.

Through a state-run company, Habaguanex, Mr. Leal plowed money into
construction projects. He restored elegant 18th-century plazas, baroque
cathedrals and restaurants and hotels, including the pink-washed
\href{https://www.gaviotahotels.com/en/hotels-in-cuba/old-havana/hotel-ambos-mundos}{Hotel
Ambos Mundos}, where Ernest Hemingway wrote ``For Whom the Bell Tolls.''
Paradoxically, Old Havana became a capitalist success story: The
restored buildings drew foreign tourists, whose money then paid for more
restoration work.

By the mid-2000s, about 300 buildings --- roughly a third of those in
Old Havana --- had been refurbished. And Mr. Leal, who employed 3,000
workers as head of the Office of Historian, was hailed as a hero and a
role model for preservationists around the world.

``He was a unique figure for his time,'' Jeffrey DeLaurentis, an
American diplomat who served in Cuba during the Obama administration,
said in an interview. ``He was also given an unusual amount of autonomy.
It was very novel in the system that Cuba has.''

As Old Havana became an economic engine for Cuba, Mr. Leal was known as
its unofficial mayor. He greeted well-wishers on the streets (typically
in his bureaucratic gray guayabera shirt). He starred in a TV series
about the capital's history, ``Andar La Habana'' (``Walk Havana''). He
lectured at American universities and was sought out by journalists as a
leading public intellectual.

Old Havana's renewal came at a cost, however; some residents were
relocated when overcrowded buildings were modernized. Preservation had
its limits, too. Visiting journalists, including one
from\href{https://www.nytimes3xbfgragh.onion/2007/12/06/world/americas/06havana.html}{The
New York Times} in 2007, pointed out that Old Havana had become like a
movie set, a pretty facade, while just blocks away poor Cubans lived in
decrepit buildings. Workers at fancy hotels scraped by on a state salary
of \$20 a month.

As Mr. Leal himself told The Times, revitalizing Cuba's colonial
architecture had done only so much to fix larger social ills. ``It pains
me to see every day the border that divides what has been restored and
what remains to be restored,'' he said.

Image

Mr. Leal gave a tour of Cuba's history to Federica Mogherini, the
European Union's foreign policy chief, in 2018.Credit...Pool photo by
Ernesto Mastrascusa

Eusebio Leal Spengler was born on Sept. 11, 1942, in a working-class
district of Havana. He was reared by his single mother, a washerwoman
and cleaner, and dropped out of school in the sixth grade to help
support the family. No other information was available about his mother
or father. Survivors include his two sons, Javier Leal and Carlos Manuel
Leal.

After the 1959 revolution brought Castro to power, public education in
Cuba became free. In 1975, Mr. Leal earned a bachelor's degree in
history, and later a Ph.D. in historical sciences, from the University
of Havana. But he had early on been self-taught, spending his youth in
libraries reading about history and architecture. In the early 1960s he
was made an apprentice in the Office of Historian, held at the time by
Emilio Roig de Leuchsenring.

When Mr. Roig died in 1967, Mr. Leal assumed the role and oversaw the
renovation of the 18th-century governor's palace into a museum, his
first restoration project.

``By ancient tradition, every old city in Latin America maintains the
institution of `chronicler,' who is named for life to save the memory of
the city,'' he told Smithsonian magazine in a
\href{https://www.smithsonianmag.com/travel/man-who-saved-havana-180968735/?page=2}{2018
profile}.

But Mr. Leal did not want Old Havana to be a mummified city for
tourists. He used a portion of the Habaguanex profits to build schools
and health clinics and to repair apartment buildings so that longtime
residents could remain. As
he\href{https://www.washingtonpost.com/lifestyle/style/eusebio-leal-the-man-who-would-save-old-havana/2012/05/20/gIQAAW31dU_story.html}{told
The Washington Post} in 2000, he took care to create ``spaces of
silence'' in his plan --- living neighborhoods removed from the tourist
hordes.

Image

Mr. Leal showed Secretary of State John Kerry a classic American car
parked in Old Havana in 2015. Mr. Leal's success brought him political
influence.~Credit...Ramon Espinosa/Associated Press

His success brought him political influence. As a practicing Roman
Catholic, he sought to improve the relationship between the Catholic
Church and the Cuban authorities, who had dispelled priests after the
revolution and officially espoused atheism. When Pope Francis visited
Cuba in 2015, Mr. Leal showed him around.

Building by building, block by block, he continued his preservation
efforts until the end of his life. One of his last and most ambitious
projects was the restoration of the National Capitol Building, which,
with its domed roof and neoclassical architecture, resembles the United
States capitol.
\href{https://www.reuters.com/article/us-cuba-capitol/cubas-capitol-reopens-after-years-of-restoration-idUSKCN1GD6L1}{It
opened in 2018} after eight years of work.

``What we're doing here is trying to preserve the patrimony, the memory
of the Cuban nation,'' Mr. Leal told The Times in 2005.

``I won't see the full restoration of the city,'' he added. ``So much is
left to be done, but this is a start.''

Advertisement

\protect\hyperlink{after-bottom}{Continue reading the main story}

\hypertarget{site-index}{%
\subsection{Site Index}\label{site-index}}

\hypertarget{site-information-navigation}{%
\subsection{Site Information
Navigation}\label{site-information-navigation}}

\begin{itemize}
\tightlist
\item
  \href{https://help.nytimes3xbfgragh.onion/hc/en-us/articles/115014792127-Copyright-notice}{©~2020~The
  New York Times Company}
\end{itemize}

\begin{itemize}
\tightlist
\item
  \href{https://www.nytco.com/}{NYTCo}
\item
  \href{https://help.nytimes3xbfgragh.onion/hc/en-us/articles/115015385887-Contact-Us}{Contact
  Us}
\item
  \href{https://www.nytco.com/careers/}{Work with us}
\item
  \href{https://nytmediakit.com/}{Advertise}
\item
  \href{http://www.tbrandstudio.com/}{T Brand Studio}
\item
  \href{https://www.nytimes3xbfgragh.onion/privacy/cookie-policy\#how-do-i-manage-trackers}{Your
  Ad Choices}
\item
  \href{https://www.nytimes3xbfgragh.onion/privacy}{Privacy}
\item
  \href{https://help.nytimes3xbfgragh.onion/hc/en-us/articles/115014893428-Terms-of-service}{Terms
  of Service}
\item
  \href{https://help.nytimes3xbfgragh.onion/hc/en-us/articles/115014893968-Terms-of-sale}{Terms
  of Sale}
\item
  \href{https://spiderbites.nytimes3xbfgragh.onion}{Site Map}
\item
  \href{https://help.nytimes3xbfgragh.onion/hc/en-us}{Help}
\item
  \href{https://www.nytimes3xbfgragh.onion/subscription?campaignId=37WXW}{Subscriptions}
\end{itemize}
