Sections

SEARCH

\protect\hyperlink{site-content}{Skip to
content}\protect\hyperlink{site-index}{Skip to site index}

\href{https://www.nytimes3xbfgragh.onion/section/sports/hockey}{Hockey}

\href{https://myaccount.nytimes3xbfgragh.onion/auth/login?response_type=cookie\&client_id=vi}{}

\href{https://www.nytimes3xbfgragh.onion/section/todayspaper}{Today's
Paper}

\href{/section/sports/hockey}{Hockey}\textbar{}In a Pair of `Hubs,' the
N.H.L. Postseason Has Been Anything but Neutral

\url{https://nyti.ms/3an8MSp}

\begin{itemize}
\item
\item
\item
\item
\item
\end{itemize}

Advertisement

\protect\hyperlink{after-top}{Continue reading the main story}

Supported by

\protect\hyperlink{after-sponsor}{Continue reading the main story}

\hypertarget{in-a-pair-of-hubs-the-nhl-postseason-has-been-anything-but-neutral}{%
\section{In a Pair of `Hubs,' the N.H.L. Postseason Has Been Anything
but
Neutral}\label{in-a-pair-of-hubs-the-nhl-postseason-has-been-anything-but-neutral}}

Fighting has made a comeback, rest is more important than ever and
players are hiding injuries from opponents staying in the same hotel.
``It's been nasty from the beginning,'' said one TV analyst.

\includegraphics{https://static01.graylady3jvrrxbe.onion/images/2020/08/14/sports/13nhl-intensity1-print/13nhl-intensity1-print-articleLarge-v2.jpg?quality=75\&auto=webp\&disable=upscale}

By Curtis Rush and Carol Schram

\begin{itemize}
\item
  Aug. 13, 2020
\item
  \begin{itemize}
  \item
  \item
  \item
  \item
  \item
  \end{itemize}
\end{itemize}

TORONTO --- The N.H.L. restart was less than three minutes old when
Justin Williams, a Carolina Hurricanes forward, offered bare-knuckled
proof that the fight was on to win the Stanley Cup, although the
calendar read Aug. 1, the postseason's start date.

Williams, 38, had had just
\href{https://www.hockeyfights.com/players/5255}{eight fights} in more
than 1,400 career regular-season and playoff games. But at 2 minutes 44
seconds of the first period, in a qualification game against the Rangers
at Scotiabank Arena in Toronto, Williams
\href{https://www.youtube.com/watch?v=RaocSCQWL_8}{dropped the gloves
against Ryan Strome}, leaving Strome's nose bloodied.

Masked because of coronavirus pandemic protocols, the ice-cleaning crew
skated out to scrape the blood off the ice.

By the time the Stanley Cup qualifying round concluded on Sunday, 16
fighting majors had been assessed through 44 games, according to the
N.H.L., a significant increase from the six fighting majors in each of
the last two years of playoffs through the same number of games.

``It's been nasty from the beginning,'' Kevin Bieksa, a former N.H.L.
defenseman and current Sportsnet TV analyst, said.

``I think the format has made these games intense,'' Peter DeBoer, the
Vegas Golden Knights' coach, said after his club clinched the top seed
in the Western Conference round-robin tournament with a 4-3 overtime win
against the Colorado Avalanche. ``That was a playoff game out there
today, as far as intensity went.''

Before the qualifying round, there were fears that playing without live
spectators in two antiseptic neutral environments, in Toronto and
Edmonton, along with the pandemic-induced four-and-a-half-month layoff
between the regular season and the postseason, would deliver sloppy play
and little emotion.

Williams's fight, after his teammate Brady Skjei's crushing check on the
Rangers' Jesper Fast, set a tone that persisted in the two postseason
hubs and that Carolina hoped to carry into its first-round series
against the Boston Bruins.

``I think there is a lot of pent-up energy from a lot of players,''
Williams said after Game 1 against the Rangers. ``Months without playing
a meaningful hockey game is tough for professional athletes. As much as
we love the fans, there was a lot of intensity that was
self-motivated.''

\includegraphics{https://static01.graylady3jvrrxbe.onion/images/2020/08/14/sports/13nhl-intensity2-print/merlin_175591887_cd0b0bcb-cff8-462f-b302-fba1038266c7-articleLarge.jpg?quality=75\&auto=webp\&disable=upscale}

On Tuesday, in Game 1 of their first-round matchup, the
\href{https://www.nytimes3xbfgragh.onion/2020/08/12/sports/hockey/longest-nhl-game-lightning-bluejackets.html}{Tampa
Bay Lightning beat the Columbus Blue Jackets} in five overtimes, the
fourth-longest game in N.H.L. history. On Wednesday, Tampa Bay forward
Tyler Johnson said he woke up ``sore and tired,'' having played more
than 40 minutes the night before.

\hypertarget{the-coronavirus-outbreak}{%
\subsubsection{The Coronavirus
Outbreak}\label{the-coronavirus-outbreak}}

\hypertarget{sports-and-the-virus}{%
\paragraph{Sports and the Virus}\label{sports-and-the-virus}}

Updated Aug. 20, 2020

Here's what's happening as the world of sports slowly comes back to
life:

\begin{itemize}
\item
  \begin{itemize}
  \tightlist
  \item
    While live sports are back, spectators are not in most cases.
    \href{https://www.nytimes3xbfgragh.onion/2020/08/19/sports/empty-stadiums-live-fans.html?action=click\&pgtype=Article\&state=default\&region=MAIN_CONTENT_2\&context=storylines_keepup}{Readers}
    comment on what they were missing as fans in the stands.
  \item
    The Canadian Football League
    \href{https://www.nytimes3xbfgragh.onion/2020/08/18/sports/cfl-season-canceled.html?action=click\&pgtype=Article\&state=default\&region=MAIN_CONTENT_2\&context=storylines_keepup}{became
    the latest casualty} of the pandemic, canceling its 2020 season
    after repeated efforts to play an abbreviated schedule fell through.
  \item
    The United States Open tennis tournament will have the weakest
    women's field in its history after second-ranked
    \href{https://www.nytimes3xbfgragh.onion/2020/08/17/sports/tennis/us-open-women.html?action=click\&pgtype=Article\&state=default\&region=MAIN_CONTENT_2\&context=storylines_keepup}{Simona
    Halep withdrew}.
  \end{itemize}
\end{itemize}

Johnson made plans to rest, take a walk and do a light spin on the bike
to keep his muscles loose in preparation for Game 2 on Thursday. Tampa
Bay defenseman Kevin Shattenkirk, who said he sometimes loses four or
five pounds in a regulation game, played more than 42 minutes against
Columbus. By the fourth overtime period, he said that his muscles
started seizing up and that going to the dressing room and sitting down
made it worse.

The club's medical staff paid special attention to players' groins, hips
and lower backs on Wednesday, and focused on getting players the proper
nutrition.

``Today is kind of going to be their eight-period game,'' Shattenkirk
said of the team's trainers. ``They're going to be working all day on
us.''

Columbus Coach John Tortorella called off a scheduled practice and held
a team meeting instead to rest his team. The Blue Jackets played in
three overtime games in a six-day stretch, including an elimination game
on Sunday in the only play-in series that went to five games.

Long overtimes are an occupational hazard in the N.H.L. playoffs, but
\href{https://www.nytimes3xbfgragh.onion/2020/07/11/sports/hockey/nhl-season-restart-approved.html}{this
year's postseason format} offers players a few unique benefits. The
aches and pains players collected during the regular season are a
distant memory after the league's pause. And not having to travel
between games could help players stay fresher as they advance to later
rounds.

During normal playoffs, injury information is notoriously hard to come
by. This year, secrecy has been taken to a new level. Social distancing
protocols mean that no reporters are allowed to attend practices or
game-day skates to grasp who's hurt and how severe an ailment might be.
Privacy concerns surrounding the disclosure of confirmed coronavirus
cases led the league to mandate a blanket classification of ``unfit to
play'' that makes the old days of ``upper-body'' and ``lower-body''
injury reports seem quaintly detailed.

While the enclosed environments in Toronto and Edmonton have remained
coronavirus-free since teams arrived on July 26, on-ice injuries have
had an impact on some series. The Winnipeg Jets lost their top-line
center, Mark Scheifele, and a key scorer, Patrik Laine, early in their
play-in matchup against the Calgary Flames, and eventually lost in four
games.

The Vancouver Canucks survived a hard-hitting, four-game play-in series
against the Minnesota Wild that included two fights, 140 penalty minutes
and injuries to Vancouver's Tyler Toffoli, Micheal Ferland and Adam
Gaudette, as well as to Minnesota defenseman Ryan Suter.

Though the Canucks and the Wild were housed in close quarters at the
Sutton Place Hotel in Edmonton, they said they had not taken advantage
of the proximity to glean injury information about their opponents.

``It's rare that you run into people with Minnesota masks and shirts
on,'' Vancouver goaltender Jacob Markstrom said. ``You just keep to
yourself and keep to your own teammates.''

``Maybe I should run into them at the elevator a little bit harder,''
said Minnesota's Marcus Foligno, who
\href{https://www.youtube.com/watch?v=okGkhaQ2Xmk}{dropped the gloves
with Ferland} just 1:19 into Game 1 of their play-in series.

``There's only one way out and one way into our hotel, and everybody has
pretty much the same time on game day to be at the rink,'' Foligno said
before his team's elimination. ``You've got to put your head down and
keep walking, and bring it to the ice.''

Curtis Rush reported from Toronto, and Carol Schram from Edmonton,
Alberta.

Advertisement

\protect\hyperlink{after-bottom}{Continue reading the main story}

\hypertarget{site-index}{%
\subsection{Site Index}\label{site-index}}

\hypertarget{site-information-navigation}{%
\subsection{Site Information
Navigation}\label{site-information-navigation}}

\begin{itemize}
\tightlist
\item
  \href{https://help.nytimes3xbfgragh.onion/hc/en-us/articles/115014792127-Copyright-notice}{©~2020~The
  New York Times Company}
\end{itemize}

\begin{itemize}
\tightlist
\item
  \href{https://www.nytco.com/}{NYTCo}
\item
  \href{https://help.nytimes3xbfgragh.onion/hc/en-us/articles/115015385887-Contact-Us}{Contact
  Us}
\item
  \href{https://www.nytco.com/careers/}{Work with us}
\item
  \href{https://nytmediakit.com/}{Advertise}
\item
  \href{http://www.tbrandstudio.com/}{T Brand Studio}
\item
  \href{https://www.nytimes3xbfgragh.onion/privacy/cookie-policy\#how-do-i-manage-trackers}{Your
  Ad Choices}
\item
  \href{https://www.nytimes3xbfgragh.onion/privacy}{Privacy}
\item
  \href{https://help.nytimes3xbfgragh.onion/hc/en-us/articles/115014893428-Terms-of-service}{Terms
  of Service}
\item
  \href{https://help.nytimes3xbfgragh.onion/hc/en-us/articles/115014893968-Terms-of-sale}{Terms
  of Sale}
\item
  \href{https://spiderbites.nytimes3xbfgragh.onion}{Site Map}
\item
  \href{https://help.nytimes3xbfgragh.onion/hc/en-us}{Help}
\item
  \href{https://www.nytimes3xbfgragh.onion/subscription?campaignId=37WXW}{Subscriptions}
\end{itemize}
