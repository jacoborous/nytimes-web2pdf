Sections

SEARCH

\protect\hyperlink{site-content}{Skip to
content}\protect\hyperlink{site-index}{Skip to site index}

\href{https://www.nytimes3xbfgragh.onion/section/business/economy}{Economy}

\href{https://myaccount.nytimes3xbfgragh.onion/auth/login?response_type=cookie\&client_id=vi}{}

\href{https://www.nytimes3xbfgragh.onion/section/todayspaper}{Today's
Paper}

\href{/section/business/economy}{Economy}\textbar{}\$400 Unemployment
Supplement Is Really \$300, and Won't Arrive Soon

\url{https://nyti.ms/2PT3Qvj}

\begin{itemize}
\item
\item
\item
\item
\item
\end{itemize}

\hypertarget{the-coronavirus-outbreak}{%
\subsubsection{\texorpdfstring{\href{https://www.nytimes3xbfgragh.onion/news-event/coronavirus?name=styln-coronavirus-markets\&region=TOP_BANNER\&variant=undefined\&block=storyline_menu_recirc\&action=click\&pgtype=Article\&impression_id=61565bb0-e380-11ea-9fe8-6ddc6eaba552}{The
Coronavirus
Outbreak}}{The Coronavirus Outbreak}}\label{the-coronavirus-outbreak}}

\begin{itemize}
\tightlist
\item
  live\href{https://www.nytimes3xbfgragh.onion/2020/08/20/world/coronavirus-covid.html?name=styln-coronavirus-markets\&region=TOP_BANNER\&variant=undefined\&block=storyline_menu_recirc\&action=click\&pgtype=Article\&impression_id=61565bb1-e380-11ea-9fe8-6ddc6eaba552}{Latest
  Updates}
\item
  \href{https://www.nytimes3xbfgragh.onion/interactive/2020/us/coronavirus-us-cases.html?name=styln-coronavirus-markets\&region=TOP_BANNER\&variant=undefined\&block=storyline_menu_recirc\&action=click\&pgtype=Article\&impression_id=61565bb2-e380-11ea-9fe8-6ddc6eaba552}{Maps
  and Cases}
\item
  \href{https://www.nytimes3xbfgragh.onion/interactive/2020/science/coronavirus-vaccine-tracker.html?name=styln-coronavirus-markets\&region=TOP_BANNER\&variant=undefined\&block=storyline_menu_recirc\&action=click\&pgtype=Article\&impression_id=61565bb3-e380-11ea-9fe8-6ddc6eaba552}{Vaccine
  Tracker}
\item
  \href{https://www.nytimes3xbfgragh.onion/2020/08/19/us/colleges-closing-covid.html?name=styln-coronavirus-markets\&region=TOP_BANNER\&variant=undefined\&block=storyline_menu_recirc\&action=click\&pgtype=Article\&impression_id=61565bb4-e380-11ea-9fe8-6ddc6eaba552}{Colleges
  Closing}
\item
  \href{https://www.nytimes3xbfgragh.onion/live/2020/08/20/business/stock-market-today-coronavirus?name=styln-coronavirus-markets\&region=TOP_BANNER\&variant=undefined\&block=storyline_menu_recirc\&action=click\&pgtype=Article\&impression_id=61565bb5-e380-11ea-9fe8-6ddc6eaba552}{Economy}
\end{itemize}

Advertisement

\protect\hyperlink{after-top}{Continue reading the main story}

Supported by

\protect\hyperlink{after-sponsor}{Continue reading the main story}

\hypertarget{400-unemployment-supplement-is-really-300-and-wont-arrive-soon}{%
\section{\$400 Unemployment Supplement Is Really \$300, and Won't Arrive
Soon}\label{400-unemployment-supplement-is-really-300-and-wont-arrive-soon}}

New state claims fell below one million for the first week since March.
But jobless ranks remain vast, and a White House relief plan faces
hurdles.

\includegraphics{https://static01.graylady3jvrrxbe.onion/images/2020/08/13/business/13virus-jobless2/merlin_174916722_5207af1b-52fd-4295-abf7-6ddabaa31a3b-articleLarge.jpg?quality=75\&auto=webp\&disable=upscale}

\href{https://www.nytimes3xbfgragh.onion/by/ben-casselman}{\includegraphics{https://static01.graylady3jvrrxbe.onion/images/2018/11/09/multimedia/author-ben-casselman/author-ben-casselman-thumbLarge.png}}\href{https://www.nytimes3xbfgragh.onion/by/emily-cochrane}{\includegraphics{https://static01.graylady3jvrrxbe.onion/images/2018/11/28/multimedia/author-emily-cochrane/author-emily-cochrane-thumbLarge-v3.png}}

By \href{https://www.nytimes3xbfgragh.onion/by/ben-casselman}{Ben
Casselman} and
\href{https://www.nytimes3xbfgragh.onion/by/emily-cochrane}{Emily
Cochrane}

\begin{itemize}
\item
  Aug. 13, 2020
\item
  \begin{itemize}
  \item
  \item
  \item
  \item
  \item
  \end{itemize}
\end{itemize}

The
\href{https://www.nytimes3xbfgragh.onion/2020/08/08/us/politics/trump-stimulus-bill-coronavirus.html}{federal
aid to unemployed workers} that President Trump announced last weekend
looks likely to be smaller than initially suggested --- and it remains
unclear when the money will start flowing, how long it will last or how
many workers will benefit.

The uncertainty comes at a delicate time for the economy. New
applications for state unemployment benefits
\href{https://www.nytimes3xbfgragh.onion/live/2020/08/13/business/stock-market-today-coronavirus/963000-filed-state-unemployment-claims-last-week-but-layoffs-remained-high}{fell
below one million last week} for the first time since the pandemic took
hold in March, the Labor Department said Thursday. But filings remain
high by historical standards, and other measures show the economy losing
momentum.

Initial weekly unemployment claims,

both regular and those under the Pandemic Unemployment Assistance
program

6 million

Regular claims fell under one million last week for the first time since
mid-March

5

4

3

2

1

0

Feb.

March

April

May

June

July

Aug.

Initial weekly unemployment claims, both regular and those under the
Pandemic Unemployment Assistance program

6 million

5

Regular claims fell under

one million last week for the

first time since mid-March

4

3

2

1

0

Feb.

March

April

May

June

July

Aug.

Pandemic Unemployment Assistance extends eligibility to some workers who
would not otherwise be able to apply for unemployment benefits, such as
part-time and self-employed workers. Regular claims are seasonally
adjusted but P.U.A. claims are not.

Source: Labor Department

By Ella Koeze

A \$600-a-week federal supplement to unemployment benefits, enacted to
address the pandemic, stopped at the end of July. That has pulled away a
key source of support, not just for the nearly 30 million Americans
receiving benefits but also for the broader economy.

``The status of the financial relief is a huge question mark hanging
over the economy,'' said Daniel Zhao, senior economist for the career
site Glassdoor.

Mr. Trump said Saturday that he was taking executive action to provide
unemployed workers with \$400 a week in extra payments, on top of their
regular state jobless benefits. He did so after talks on a new round of
pandemic relief stalled in Congress.

The Senate adjourned on Thursday until early September, and House
members had already left Washington. The departures all but end any
chance of a quick agreement on sending stimulus checks to American
taxpayers, reviving lapsed unemployment benefits and providing billions
of dollars for schools, testing, child care, small businesses, and state
and local governments.

In the meantime, states are scrambling to figure out how to carry out
Mr. Trump's plan, with unemployed workers wondering whether the money
will arrive in time to
\href{https://www.nytimes3xbfgragh.onion/2020/08/08/business/economy/lost-unemployment-benefits.html}{prevent
lasting financial harm}.

Here is what we know about the program and how it will work.

\hypertarget{the-benefit-will-be-300-for-most-workers-not-400}{%
\subsection{The benefit will be \$300 for most workers, not
\$400.}\label{the-benefit-will-be-300-for-most-workers-not-400}}

When Mr. Trump announced the program, known as Lost Wages Assistance, he
said it would add \$400 to workers' weekly unemployment checks.

But unlike the earlier supplement, which was fully funded by the federal
government, the program called for states to chip in a quarter of the
cost. Governors from both major parties
\href{https://www.nytimes3xbfgragh.onion/2020/08/10/us/politics/virus-stimulus-congress-trump.html}{balked}
at being asked to spend billions of dollars when tax revenues have
plunged because of the economic collapse.

\hypertarget{latest-updates-the-coronavirus-outbreak-and-the-economy}{%
\section{\texorpdfstring{\href{https://www.nytimes3xbfgragh.onion/live/2020/08/20/business/stock-market-today-coronavirus?action=click\&pgtype=Article\&state=default\&region=MAIN_CONTENT_1\&context=storylines_live_updates}{Latest
Updates: The Coronavirus Outbreak and the
Economy}}{Latest Updates: The Coronavirus Outbreak and the Economy}}\label{latest-updates-the-coronavirus-outbreak-and-the-economy}}

\href{https://www.nytimes3xbfgragh.onion/live/2020/08/20/business/stock-market-today-coronavirus?action=click\&pgtype=Article\&state=default\&region=MAIN_CONTENT_1\&context=storylines_live_updates\#the-producer-of-unhinged-makes-a-big-bet-on-audiences-returning-to-theaters}{9h
ago}

\href{https://www.nytimes3xbfgragh.onion/live/2020/08/20/business/stock-market-today-coronavirus?action=click\&pgtype=Article\&state=default\&region=MAIN_CONTENT_1\&context=storylines_live_updates\#the-producer-of-unhinged-makes-a-big-bet-on-audiences-returning-to-theaters}{The
producer of `Unhinged' makes a big bet on audiences returning to
theaters.}

\href{https://www.nytimes3xbfgragh.onion/live/2020/08/20/business/stock-market-today-coronavirus?action=click\&pgtype=Article\&state=default\&region=MAIN_CONTENT_1\&context=storylines_live_updates\#american-airlines-to-stop-flights-to-15-cities-after-government-aid-ends}{18h
ago}

\href{https://www.nytimes3xbfgragh.onion/live/2020/08/20/business/stock-market-today-coronavirus?action=click\&pgtype=Article\&state=default\&region=MAIN_CONTENT_1\&context=storylines_live_updates\#american-airlines-to-stop-flights-to-15-cities-after-government-aid-ends}{American
Airlines to stop flights to 15 cities after government aid ends.}

\href{https://www.nytimes3xbfgragh.onion/live/2020/08/20/business/stock-market-today-coronavirus?action=click\&pgtype=Article\&state=default\&region=MAIN_CONTENT_1\&context=storylines_live_updates\#without-school-plays-and-assemblies-a-technicians-livelihood-withers}{18h
ago}

\href{https://www.nytimes3xbfgragh.onion/live/2020/08/20/business/stock-market-today-coronavirus?action=click\&pgtype=Article\&state=default\&region=MAIN_CONTENT_1\&context=storylines_live_updates\#without-school-plays-and-assemblies-a-technicians-livelihood-withers}{Without
school plays and assemblies, a technician's livelihood withers.}

\href{https://www.nytimes3xbfgragh.onion/live/2020/08/20/business/stock-market-today-coronavirus?action=click\&pgtype=Article\&state=default\&region=MAIN_CONTENT_1\&context=storylines_live_updates}{See
more updates}

More live coverage:
\href{https://www.nytimes3xbfgragh.onion/2020/08/20/world/coronavirus-covid.html?action=click\&pgtype=Article\&state=default\&region=MAIN_CONTENT_1\&context=storylines_live_updates}{Global}

So this week the administration offered new guidance: Rather than adding
\$100 a week on top of existing unemployment benefits, states could
count existing benefits toward their share. In other words, unemployed
workers would get an extra \$300, not \$400.

States still have the option of providing an extra \$100, but few if any
are expected to do so.

``They're stretched,'' said Andrew Stettner, a senior fellow at the
Century Foundation who has been studying the unemployment system. ``They
don't have money for masks for the teachers in their schools. They're
probably not going to come up with an extra \$100 for everyone on
unemployment insurance.''

\hypertarget{the-lowest-paid-workers-wont-qualify-for-the-extra-money}{%
\subsection{The lowest-paid workers won't qualify for the extra
money.}\label{the-lowest-paid-workers-wont-qualify-for-the-extra-money}}

Under
\href{https://wdr.doleta.gov/directives/attach/UIPL/UIPL_27-20.pdf}{guidance}
released by the Labor Department on Wednesday evening, the new program
will be available to people who certify that they are ``unemployed or
partially unemployed due to disruptions caused by Covid-19'' --- but
only if they already qualify for at least \$100 a week in unemployment
benefits.

That provision would exclude roughly one million people, nearly
three-quarters of them women, according to Eliza Forsythe, an economist
at the University of Illinois.

``They're the people who need it the most,'' Ms. Forsythe said. ``They
were low paid to begin with, and then being singled out for not getting
this benefit I think is really cruel.''

It isn't clear why the \$100 minimum was established. Mr. Trump
established the benefit under a federal disaster program that requires
states to cover 25 percent of any costs. But that rule applies to the
overall program, not to individual recipients. People receiving money
under the Pandemic Unemployment Assistance program, for example, qualify
for the \$300 a week even though that program is entirely funded by the
federal government.

\hypertarget{it-could-take-weeks-for-the-money-to-start-flowing}{%
\subsection{It could take weeks for the money to start
flowing.}\label{it-could-take-weeks-for-the-money-to-start-flowing}}

\includegraphics{https://static01.graylady3jvrrxbe.onion/images/2020/08/14/business/13JPvirus-jobless1-print/merlin_175356162_e7728e03-b5d3-4ee4-a5c4-bea2813aefa8-articleLarge.jpg?quality=75\&auto=webp\&disable=upscale}

Even for those who qualify, it could be weeks or even months before they
begin receiving any extra money. States will need to adjust to the new
provisions when they are already overwhelmed by unemployment filings.

It took months for some states to begin paying benefits under the
Pandemic Unemployment Assistance program --- which extended benefits to
cover independent contractors, self-employed workers and others left out
of the standard unemployment insurance system --- in part because of
archaic computer systems that are difficult to reprogram.

``We think it would take months,'' William G. Kunstman, a spokesman for
the Hawaii Department of Labor and Industrial Relations, said in an
email. He cited the difficulty of reprogramming the state's computer
system to comply with federal requirements.

Even states with more modern computer systems said it could take weeks
to get the new supplement started. Bill McCamley, secretary of the New
Mexico Department of Workforce Solutions, said that his state was among
the first to get the pandemic assistance program up and running, but
that it still took nearly a month.

``Even in our system, which is very modern in the unemployment world,
it's still going to take us time to do it right,'' he said.

Mr. McCamley also warned that a murky timeline could prompt further
confusion and distress for people seeking the new benefit. After Mr.
Trump signed the \$2.2 trillion stimulus into law on a Friday in March,
Mr. McCamley said, his office returned Monday to thousands of calls
seeking the aid --- even though it would take about a month to
streamline the new benefits and programs.

``The message went out that this was done, and there was not a
concurrent one saying this doesn't happen at the flip of a light
switch,'' he said.

Trump administration officials contend that the new program will be
faster to put in place because states have gained experience during the
pandemic. But even these officials say it will probably be weeks before
workers start receiving the money.

\hypertarget{the-money-wont-last-long}{%
\subsection{The money won't last long.}\label{the-money-wont-last-long}}

The program is retroactive to Aug. 1, meaning that workers should
eventually receive payments for all of August.

But Mr. Trump's executive action caps spending on the program at \$44
billion, enough to cover five or six weeks of benefits, assuming all
states sign up. That means the program could end almost as soon as it
begins.

It is still possible that Congress could either revive the original
unemployment supplement --- though probably at less than \$600 a week
--- or appropriate more money for Mr. Trump's replacement.

But any deal appears far off. Democrats in the House voted in May to
extend the \$600-a-week enhancement through the end of the year as part
of a \$3.4 trillion stimulus measure, but Senate Republicans have
refused to take up that bill. The \$1 trillion proposal unveiled by
Republicans last month calls for a supplement averaging \$200.

Democrats argue that a legislative solution is the only way to provide
workers with certainty.

``The Labor Department's new guidance leaves many unanswered
questions,'' said Senator Ron Wyden of Oregon, the top Democrat on the
Senate Finance Committee, who helped negotiate the original \$600
benefit. ``Workers struggling to pay rent and buy groceries are not
going to see benefits they were promised any time soon.''

\hypertarget{workers-are-left-in-limbo}{%
\subsection{Workers are left in
limbo.}\label{workers-are-left-in-limbo}}

Image

David Moniz, an out-of-work chef in California, began collecting
unemployment benefits. Then they abruptly stopped.Credit...Marissa
Leshnov for The New York Times

For unemployed workers, the uncertainty over benefits means not knowing
when they will be able to pay down the credit cards, or whether they
will be able to make rent on Sept. 1. For those already struggling to
get help from overwhelmed state unemployment offices, the prospect of
further delays is even more frustrating.

David Moniz started a job in March as a resident chef at Sur La Table,
the kitchen goods retailer, in San Jose, Calif. His timing was terrible:
After he spent one day on the job, the store shut down because of the
virus, and he was furloughed.

It took Mr. Moniz, 29, weeks of calling to get through to California's
employment office and file an unemployment claim. Then, after a few
weeks, his benefits abruptly stopped. His file is shown as ``pending''
on the state website, and despite endless hours of calling, he has been
unable to get through to address the problem. He hasn't received a check
since June 1.

Without any money coming in, Mr. Moniz has burned through his savings
and racked up debt. He has \$28 left before he hits his credit limit, he
said, and owes \$200 in late fees and penalties to his bank, Wells
Fargo.

``Wells Fargo calls me more than anyone in my family does because of my
account right now,'' he said.

Advertisement

\protect\hyperlink{after-bottom}{Continue reading the main story}

\hypertarget{site-index}{%
\subsection{Site Index}\label{site-index}}

\hypertarget{site-information-navigation}{%
\subsection{Site Information
Navigation}\label{site-information-navigation}}

\begin{itemize}
\tightlist
\item
  \href{https://help.nytimes3xbfgragh.onion/hc/en-us/articles/115014792127-Copyright-notice}{©~2020~The
  New York Times Company}
\end{itemize}

\begin{itemize}
\tightlist
\item
  \href{https://www.nytco.com/}{NYTCo}
\item
  \href{https://help.nytimes3xbfgragh.onion/hc/en-us/articles/115015385887-Contact-Us}{Contact
  Us}
\item
  \href{https://www.nytco.com/careers/}{Work with us}
\item
  \href{https://nytmediakit.com/}{Advertise}
\item
  \href{http://www.tbrandstudio.com/}{T Brand Studio}
\item
  \href{https://www.nytimes3xbfgragh.onion/privacy/cookie-policy\#how-do-i-manage-trackers}{Your
  Ad Choices}
\item
  \href{https://www.nytimes3xbfgragh.onion/privacy}{Privacy}
\item
  \href{https://help.nytimes3xbfgragh.onion/hc/en-us/articles/115014893428-Terms-of-service}{Terms
  of Service}
\item
  \href{https://help.nytimes3xbfgragh.onion/hc/en-us/articles/115014893968-Terms-of-sale}{Terms
  of Sale}
\item
  \href{https://spiderbites.nytimes3xbfgragh.onion}{Site Map}
\item
  \href{https://help.nytimes3xbfgragh.onion/hc/en-us}{Help}
\item
  \href{https://www.nytimes3xbfgragh.onion/subscription?campaignId=37WXW}{Subscriptions}
\end{itemize}
