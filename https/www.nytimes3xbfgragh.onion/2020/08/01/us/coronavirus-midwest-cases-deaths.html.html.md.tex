Sections

SEARCH

\protect\hyperlink{site-content}{Skip to
content}\protect\hyperlink{site-index}{Skip to site index}

\href{https://www.nytimes3xbfgragh.onion/section/us}{U.S.}

\href{https://myaccount.nytimes3xbfgragh.onion/auth/login?response_type=cookie\&client_id=vi}{}

\href{https://www.nytimes3xbfgragh.onion/section/todayspaper}{Today's
Paper}

\href{/section/us}{U.S.}\textbar{}After Plummeting, the Virus Soars Back
in the Midwest

\url{https://nyti.ms/319iyDx}

\begin{itemize}
\item
\item
\item
\item
\item
\item
\end{itemize}

\hypertarget{the-coronavirus-outbreak}{%
\subsubsection{\texorpdfstring{\href{https://www.nytimes3xbfgragh.onion/news-event/coronavirus?name=styln-coronavirus-national\&region=TOP_BANNER\&variant=undefined\&block=storyline_menu_recirc\&action=click\&pgtype=Article\&impression_id=8b2e2c40-e396-11ea-b4b7-bf9bf7f05dd6}{The
Coronavirus
Outbreak}}{The Coronavirus Outbreak}}\label{the-coronavirus-outbreak}}

\begin{itemize}
\tightlist
\item
  live\href{https://www.nytimes3xbfgragh.onion/2020/08/21/world/covid-19-coronavirus.html?name=styln-coronavirus-national\&region=TOP_BANNER\&variant=undefined\&block=storyline_menu_recirc\&action=click\&pgtype=Article\&impression_id=8b2e2c41-e396-11ea-b4b7-bf9bf7f05dd6}{Latest
  Updates}
\item
  \href{https://www.nytimes3xbfgragh.onion/interactive/2020/us/coronavirus-us-cases.html?name=styln-coronavirus-national\&region=TOP_BANNER\&variant=undefined\&block=storyline_menu_recirc\&action=click\&pgtype=Article\&impression_id=8b2e5350-e396-11ea-b4b7-bf9bf7f05dd6}{Maps
  and Cases}
\item
  \href{https://www.nytimes3xbfgragh.onion/interactive/2020/science/coronavirus-vaccine-tracker.html?name=styln-coronavirus-national\&region=TOP_BANNER\&variant=undefined\&block=storyline_menu_recirc\&action=click\&pgtype=Article\&impression_id=8b2e5351-e396-11ea-b4b7-bf9bf7f05dd6}{Vaccine
  Tracker}
\item
  \href{https://www.nytimes3xbfgragh.onion/2020/08/19/us/colleges-closing-covid.html?name=styln-coronavirus-national\&region=TOP_BANNER\&variant=undefined\&block=storyline_menu_recirc\&action=click\&pgtype=Article\&impression_id=8b2e5352-e396-11ea-b4b7-bf9bf7f05dd6}{Colleges
  Closing}
\item
  \href{https://www.nytimes3xbfgragh.onion/live/2020/08/20/business/stock-market-today-coronavirus?name=styln-coronavirus-national\&region=TOP_BANNER\&variant=undefined\&block=storyline_menu_recirc\&action=click\&pgtype=Article\&impression_id=8b2e5353-e396-11ea-b4b7-bf9bf7f05dd6}{Economy}
\end{itemize}

Advertisement

\protect\hyperlink{after-top}{Continue reading the main story}

Supported by

\protect\hyperlink{after-sponsor}{Continue reading the main story}

\hypertarget{after-plummeting-the-virus-soars-back-in-the-midwest}{%
\section{After Plummeting, the Virus Soars Back in the
Midwest}\label{after-plummeting-the-virus-soars-back-in-the-midwest}}

States like Missouri, Illinois and Wisconsin are riding a frustrating
seesaw during the pandemic, with new coronavirus cases rising again
after apparent progress.

\includegraphics{https://static01.graylady3jvrrxbe.onion/images/2020/08/02/us/SUB02virus-stateofthevirus/01virus-stateofthevirus01-articleLarge.jpg?quality=75\&auto=webp\&disable=upscale}

By \href{https://www.nytimes3xbfgragh.onion/by/julie-bosman}{Julie
Bosman},
\href{https://www.nytimes3xbfgragh.onion/by/manny-fernandez}{Manny
Fernandez} and
\href{https://www.nytimes3xbfgragh.onion/by/thomas-fuller}{Thomas
Fuller}

\begin{itemize}
\item
  Published Aug. 1, 2020Updated Aug. 4, 2020
\item
  \begin{itemize}
  \item
  \item
  \item
  \item
  \item
  \item
  \end{itemize}
\end{itemize}

CHICAGO --- First, the Pacific Northwest and the Northeast were hit
hardest as the coronavirus tore through the nation. Then it surged
across the South. Now the virus is again picking up dangerous speed in
much of the Midwest --- and in states from Mississippi to Florida to
California that thought they had already seen the worst of it.

As the United States rides what amounts to a second wave of cases, with
daily new infections leveling off at an alarming higher mark, there is a
deepening national sense that the progress made in fighting the pandemic
is coming undone and no patch of America is safe.

In Missouri, Wisconsin and Illinois, distressed government officials are
retightening restrictions on residents and businesses, and sounding
warnings about a surge in coronavirus-related hospitalizations.

In the South and the West, several states are reporting their highest
levels of new coronavirus cases, with outbreaks overwhelming urban and
rural areas alike.

Across the country, communities including Snohomish County, Wash.,
Jackson, Miss., and Baton Rouge, La., have seen coronavirus numbers fall
and then shoot back up --- not unlike the two ends of a seesaw.

In Illinois, Gov. J.B. Pritzker sounded an unusually somber note this
past week as he delivered a warning that reverberated across the state:
Even though Illinoisans had battled an early flood of coronavirus
infections and then managed to reduce the virus's spread, their
successes were fleeting. As of Thursday, the state was averaging more
than 1,400 cases a day, up from about 800 at the start of July.

``We're at a danger point,'' Mr. Pritzker said in Peoria County, where
the total number of cases has doubled in the last month.

Gone is any sense that the country may soon gain control of the
pandemic. Instead, the seven-day average for new infections hovered
around 65,000 for two weeks. Progress in some states has been mostly
offset by growing outbreaks in parts of the South and the Midwest.

\includegraphics{https://static01.graylady3jvrrxbe.onion/images/2020/08/01/us/01virus-stateofthevirus02/01virus-stateofthevirus02-articleLarge.jpg?quality=75\&auto=webp\&disable=upscale}

``There's a sort of collective tiredness and frustration, and of course
I feel it, too --- we all feel it,'' said County Judge Lina Hidalgo, the
top elected official in Harris County, which includes Houston. ``So it's
difficult to know that there's no real end in sight.''

\hypertarget{latest-updates-the-coronavirus-outbreak}{%
\section{\texorpdfstring{\href{https://www.nytimes3xbfgragh.onion/2020/08/21/world/covid-19-coronavirus.html?action=click\&pgtype=Article\&state=default\&region=MAIN_CONTENT_1\&context=storylines_live_updates}{Latest
Updates: The Coronavirus
Outbreak}}{Latest Updates: The Coronavirus Outbreak}}\label{latest-updates-the-coronavirus-outbreak}}

Updated 2020-08-21T10:07:51.693Z

\begin{itemize}
\tightlist
\item
  \href{https://www.nytimes3xbfgragh.onion/2020/08/21/world/covid-19-coronavirus.html?action=click\&pgtype=Article\&state=default\&region=MAIN_CONTENT_1\&context=storylines_live_updates\#link-4690b6aa}{Shutdowns,
  warnings and scoldings follow gatherings on college campuses.}
\item
  \href{https://www.nytimes3xbfgragh.onion/2020/08/21/world/covid-19-coronavirus.html?action=click\&pgtype=Article\&state=default\&region=MAIN_CONTENT_1\&context=storylines_live_updates\#link-324af071}{As
  he accepts the Democratic nomination, Biden knocks Trump's pandemic
  response.}
\item
  \href{https://www.nytimes3xbfgragh.onion/2020/08/21/world/covid-19-coronavirus.html?action=click\&pgtype=Article\&state=default\&region=MAIN_CONTENT_1\&context=storylines_live_updates\#link-35890b73}{Hundreds
  of doctors in Kenya go on strike over their pay and protective gear.}
\end{itemize}

\href{https://www.nytimes3xbfgragh.onion/2020/08/21/world/covid-19-coronavirus.html?action=click\&pgtype=Article\&state=default\&region=MAIN_CONTENT_1\&context=storylines_live_updates}{See
more updates}

More live coverage:
\href{https://www.nytimes3xbfgragh.onion/live/2020/08/20/business/stock-market-today-coronavirus?action=click\&pgtype=Article\&state=default\&region=MAIN_CONTENT_1\&context=storylines_live_updates}{Markets}

On Friday, Dr. Anthony S. Fauci, the nation's top infectious-disease
expert, told Congress he was cautiously optimistic that a safe and
effective coronavirus vaccine would be available by the end of the year
or early 2021, though the federal government's ability to speedily
immunize most Americans was unclear.

Even finding out who has the virus is a challenge, as testing programs
have frustrated many Americans with lengthy delays in providing results.

The picture is similarly depressing overseas, where even governments
that would seem well suited to combating the virus are seeing
resurgences.

New daily infections in Japan, a country with a long tradition of
wearing face masks, rose more than 50 percent in July. Australia, which
can cut itself off from the rest of the world more easily than most, is
battling a wave of infections in and around
\href{https://www.nytimes3xbfgragh.onion/2020/08/04/world/australia/coronavirus-melbourne-lockdown.html}{Melbourne}.
Hong Kong, Israel and Spain are also fighting second waves.

None of those places has an infection rate as high as the United States,
which has the most cases and deaths in the world.

In American communities that saw improvement in June, such as Milwaukee
County in Wisconsin, there was a widespread feeling of relief, said Dr.
Ben Weston, the director of medical services for the Milwaukee County
Office of Emergency Management.

But then mask-wearing and social distancing began to relax.

``There was a sense of complacency, like, `We're finally beyond this,
it's finally getting better,''' he said. ``We were seeing our numbers go
down, but the reason is because of physical distancing. It's because
people were being so careful. There was no reason to think that cases
weren't going to rise.''

Image

Customers leaving a store in Tulsa, Okla., on Thursday.Credit...Chris
Creese for The New York Times

On Thursday, Gov. Tony Evers, a Democrat, made another attempt to get a
handle on the outbreaks in his state, issuing an order that every
Wisconsinite wear a mask indoors in public beginning Saturday.

Many states have traced new outbreaks to the loosening of the
economically costly restrictions aimed at stopping the spread of the
virus.

In California, which has had more than 500,000 coronavirus cases, more
than any other state, the reopening has proved disastrous. When the
pandemic was ravaging the Northeast in March and April, California kept
its daily case count around 2,000, and the state was praised for its
early and aggressive actions to combat the virus.

The state is now averaging more than four times as many cases --- 8,500
a day. Los Angeles County and other Southern California counties account
for the majority of the state's infections, but the virus is now
everywhere.

That notion was reinforced on Tuesday when health officials in one of
the most remote parts of the state, Modoc County, which had been the
last of California's 58 counties without a known case, announced that
the virus had arrived.

A waitress at the Brass Rail, a Basque restaurant and bar, tested
positive, raising concerns about the virus's spread in a tight-knit
county with a population of 8,800 and where cows outnumber people five
to one. (A billboard there warning residents of the coronavirus tells
people to stand one cow's length apart.)

The waitress and her husband recently returned from a trip to the
Central Valley, according to the co-owner of the Brass Rail, Jodie
Larranaga, who said she assumed that the waitress was infected during
her journey.

That the virus is now present in the evergreen forests of the
northeastern corner of the state is testament to its inexorable spread,
say the county's residents. Alturas, the only incorporated city in Modoc
County, is so isolated that its high school football team must drive as
long as five hours to reach its opponents.

\href{https://www.nytimes3xbfgragh.onion/news-event/coronavirus?action=click\&pgtype=Article\&state=default\&region=MAIN_CONTENT_3\&context=storylines_faq}{}

\hypertarget{the-coronavirus-outbreak-}{%
\subsubsection{The Coronavirus Outbreak
›}\label{the-coronavirus-outbreak-}}

\hypertarget{frequently-asked-questions}{%
\paragraph{Frequently Asked
Questions}\label{frequently-asked-questions}}

Updated August 17, 2020

\begin{itemize}
\item ~
  \hypertarget{why-does-standing-six-feet-away-from-others-help}{%
  \paragraph{Why does standing six feet away from others
  help?}\label{why-does-standing-six-feet-away-from-others-help}}

  \begin{itemize}
  \tightlist
  \item
    The coronavirus spreads primarily through droplets from your mouth
    and nose, especially when you cough or sneeze. The C.D.C., one of
    the organizations using that measure,
    \href{https://www.nytimes3xbfgragh.onion/2020/04/14/health/coronavirus-six-feet.html?action=click\&pgtype=Article\&state=default\&region=MAIN_CONTENT_3\&context=storylines_faq}{bases
    its recommendation of six feet} on the idea that most large droplets
    that people expel when they cough or sneeze will fall to the ground
    within six feet. But six feet has never been a magic number that
    guarantees complete protection. Sneezes, for instance, can launch
    droplets a lot farther than six feet,
    \href{https://jamanetwork.com/journals/jama/fullarticle/2763852}{according
    to a recent study}. It's a rule of thumb: You should be safest
    standing six feet apart outside, especially when it's windy. But
    keep a mask on at all times, even when you think you're far enough
    apart.
  \end{itemize}
\item ~
  \hypertarget{i-have-antibodies-am-i-now-immune}{%
  \paragraph{I have antibodies. Am I now
  immune?}\label{i-have-antibodies-am-i-now-immune}}

  \begin{itemize}
  \tightlist
  \item
    As of right
    now,\href{https://www.nytimes3xbfgragh.onion/2020/07/22/health/covid-antibodies-herd-immunity.html?action=click\&pgtype=Article\&state=default\&region=MAIN_CONTENT_3\&context=storylines_faq}{that
    seems likely, for at least several months.} There have been
    frightening accounts of people suffering what seems to be a second
    bout of Covid-19. But experts say these patients may have a
    drawn-out course of infection, with the virus taking a slow toll
    weeks to months after initial exposure. People infected with the
    coronavirus typically
    \href{https://www.nature.com/articles/s41586-020-2456-9}{produce}
    immune molecules called antibodies, which are
    \href{https://www.nytimes3xbfgragh.onion/2020/05/07/health/coronavirus-antibody-prevalence.html?action=click\&pgtype=Article\&state=default\&region=MAIN_CONTENT_3\&context=storylines_faq}{protective
    proteins made in response to an
    infection}\href{https://www.nytimes3xbfgragh.onion/2020/05/07/health/coronavirus-antibody-prevalence.html?action=click\&pgtype=Article\&state=default\&region=MAIN_CONTENT_3\&context=storylines_faq}{.
    These antibodies may} last in the body
    \href{https://www.nature.com/articles/s41591-020-0965-6}{only two to
    three months}, which may seem worrisome, but that's perfectly normal
    after an acute infection subsides, said Dr. Michael Mina, an
    immunologist at Harvard University. It may be possible to get the
    coronavirus again, but it's highly unlikely that it would be
    possible in a short window of time from initial infection or make
    people sicker the second time.
  \end{itemize}
\item ~
  \hypertarget{im-a-small-business-owner-can-i-get-relief}{%
  \paragraph{I'm a small-business owner. Can I get
  relief?}\label{im-a-small-business-owner-can-i-get-relief}}

  \begin{itemize}
  \tightlist
  \item
    The
    \href{https://www.nytimes3xbfgragh.onion/article/small-business-loans-stimulus-grants-freelancers-coronavirus.html?action=click\&pgtype=Article\&state=default\&region=MAIN_CONTENT_3\&context=storylines_faq}{stimulus
    bills enacted in March} offer help for the millions of American
    small businesses. Those eligible for aid are businesses and
    nonprofit organizations with fewer than 500 workers, including sole
    proprietorships, independent contractors and freelancers. Some
    larger companies in some industries are also eligible. The help
    being offered, which is being managed by the Small Business
    Administration, includes the Paycheck Protection Program and the
    Economic Injury Disaster Loan program. But lots of folks have
    \href{https://www.nytimes3xbfgragh.onion/interactive/2020/05/07/business/small-business-loans-coronavirus.html?action=click\&pgtype=Article\&state=default\&region=MAIN_CONTENT_3\&context=storylines_faq}{not
    yet seen payouts.} Even those who have received help are confused:
    The rules are draconian, and some are stuck sitting on
    \href{https://www.nytimes3xbfgragh.onion/2020/05/02/business/economy/loans-coronavirus-small-business.html?action=click\&pgtype=Article\&state=default\&region=MAIN_CONTENT_3\&context=storylines_faq}{money
    they don't know how to use.} Many small-business owners are getting
    less than they expected or
    \href{https://www.nytimes3xbfgragh.onion/2020/06/10/business/Small-business-loans-ppp.html?action=click\&pgtype=Article\&state=default\&region=MAIN_CONTENT_3\&context=storylines_faq}{not
    hearing anything at all.}
  \end{itemize}
\item ~
  \hypertarget{what-are-my-rights-if-i-am-worried-about-going-back-to-work}{%
  \paragraph{What are my rights if I am worried about going back to
  work?}\label{what-are-my-rights-if-i-am-worried-about-going-back-to-work}}

  \begin{itemize}
  \tightlist
  \item
    Employers have to provide
    \href{https://www.osha.gov/SLTC/covid-19/standards.html}{a safe
    workplace} with policies that protect everyone equally.
    \href{https://www.nytimes3xbfgragh.onion/article/coronavirus-money-unemployment.html?action=click\&pgtype=Article\&state=default\&region=MAIN_CONTENT_3\&context=storylines_faq}{And
    if one of your co-workers tests positive for the coronavirus, the
    C.D.C.} has said that
    \href{https://www.cdc.gov/coronavirus/2019-ncov/community/guidance-business-response.html}{employers
    should tell their employees} -\/- without giving you the sick
    employee's name -\/- that they may have been exposed to the virus.
  \end{itemize}
\item ~
  \hypertarget{what-is-school-going-to-look-like-in-september}{%
  \paragraph{What is school going to look like in
  September?}\label{what-is-school-going-to-look-like-in-september}}

  \begin{itemize}
  \tightlist
  \item
    It is unlikely that many schools will return to a normal schedule
    this fall, requiring the grind of
    \href{https://www.nytimes3xbfgragh.onion/2020/06/05/us/coronavirus-education-lost-learning.html?action=click\&pgtype=Article\&state=default\&region=MAIN_CONTENT_3\&context=storylines_faq}{online
    learning},
    \href{https://www.nytimes3xbfgragh.onion/2020/05/29/us/coronavirus-child-care-centers.html?action=click\&pgtype=Article\&state=default\&region=MAIN_CONTENT_3\&context=storylines_faq}{makeshift
    child care} and
    \href{https://www.nytimes3xbfgragh.onion/2020/06/03/business/economy/coronavirus-working-women.html?action=click\&pgtype=Article\&state=default\&region=MAIN_CONTENT_3\&context=storylines_faq}{stunted
    workdays} to continue. California's two largest public school
    districts --- Los Angeles and San Diego --- said on July 13, that
    \href{https://www.nytimes3xbfgragh.onion/2020/07/13/us/lausd-san-diego-school-reopening.html?action=click\&pgtype=Article\&state=default\&region=MAIN_CONTENT_3\&context=storylines_faq}{instruction
    will be remote-only in the fall}, citing concerns that surging
    coronavirus infections in their areas pose too dire a risk for
    students and teachers. Together, the two districts enroll some
    825,000 students. They are the largest in the country so far to
    abandon plans for even a partial physical return to classrooms when
    they reopen in August. For other districts, the solution won't be an
    all-or-nothing approach.
    \href{https://bioethics.jhu.edu/research-and-outreach/projects/eschool-initiative/school-policy-tracker/}{Many
    systems}, including the nation's largest, New York City, are
    devising
    \href{https://www.nytimes3xbfgragh.onion/2020/06/26/us/coronavirus-schools-reopen-fall.html?action=click\&pgtype=Article\&state=default\&region=MAIN_CONTENT_3\&context=storylines_faq}{hybrid
    plans} that involve spending some days in classrooms and other days
    online. There's no national policy on this yet, so check with your
    municipal school system regularly to see what is happening in your
    community.
  \end{itemize}
\end{itemize}

``We all felt very safe for a while,'' said Juan Ledezma, the owner of a
thrift shop on Main Street in Alturas. ``Right now, it's a little bit
scary.''

Businesses across the country have abandoned their own plans to return
to normal in light of the virus's resurgence.

The company that operates a popular water taxi on the Chicago River,
ferrying commuters to work each day, had hoped to reopen by Labor Day.
This week, officials postponed those plans until March.

The historic Berghoff restaurant in Chicago's Loop reopened at the end
of June after months of closure, a sign that the coronavirus curve had
flattened and the city's downtown was ready to start humming again.

This week, as coronavirus infections surged in Illinois, the restaurant
abruptly shut its doors for the second time.

``It broke my heart,'' said Pete Berghoff, whose family has owned the
restaurant since 1898. ``We reopened, and after about three weeks my
enthusiasm was beaten out of me.''

From state to state and region to region, the picture of coronavirus
spread is shifting daily as some communities see gradual improvement and
others suddenly struggle.

Image

Renee Leonard, Delisa Craig and Miriam Girata help one another put on
personal protective equipment at~ a testing site in Orlando, Fla., on
Tuesday.Credit...Eve Edelheit for The New York Times

A few places, including Arizona, South Carolina and Texas, have started
to see new case reports drop after huge surges. California, Florida and
Louisiana continue to report some of their highest daily totals of the
pandemic.

The Rio Grande Valley in Texas is suffering through perhaps the worst
current outbreak in the country, with hundreds of new cases and dozens
of deaths a day. In more than half of states, outbreaks continue to
grow.

In Missouri and Oklahoma, cases have grown to alarming levels, with both
states now averaging more than 1,000 each day. And in Maryland, daily
case numbers are ticking upward again after periods of sustained
progress.

The Northeast, once the virus's biggest hot spot, has improved
considerably since its peak in April, when the region suffered more than
any other region of the country. Yet cases are now increasing slightly
in New Jersey, Rhode Island and Massachusetts, as residents move around
more freely and gather more frequently in groups.

Across the country, deaths from the coronavirus continue to rise. The
country was averaging about 500 per day at the start of July. Over the
last week, it has averaged more than 1,000 daily, with many of those
concentrated in Sun Belt states. On Wednesday, California, Florida and
Texas reported a combined 724 deaths, about half the national total.

Houston, the fourth-largest city in the country, has been adjusting to a
new normal where the only thing certain is that nothing is certain.
After cases and hospitalizations seemed to level off and even decrease
in recent days, Harris County on Friday broke a single-day record with
2,100 new cases.

``I think to a certain extent, we saw a spike because people were
fatigued over it,'' said Alan Rosen, who leads the Harris County
Precinct One constable's office. ``They were fatigued over hearing about
it every day. They were fatigued about being cooped up in their house
and being away from people.''

People there have been coping with the lulls and peaks of a physical,
emotional, fiscal and logistical crisis from an invisible foe nearly
three years after surviving Hurricane Harvey, one of the worst disasters
in American history.

*``*It is a roller coaster,'' said Mr. Rosen, who recovered after
getting infected with the virus in May. ``It's not like a hurricane
that's coming through and we know what to do. We know we got to clean up
and rebuild and everybody is accustomed to the time frame. But with
this, there are just so many unknowns.''

Julie Bosman reported from Chicago, Manny Fernandez from Houston and
Thomas Fuller from Alturas, Calif. Mitch Smith contributed reporting
from Chicago.

Advertisement

\protect\hyperlink{after-bottom}{Continue reading the main story}

\hypertarget{site-index}{%
\subsection{Site Index}\label{site-index}}

\hypertarget{site-information-navigation}{%
\subsection{Site Information
Navigation}\label{site-information-navigation}}

\begin{itemize}
\tightlist
\item
  \href{https://help.nytimes3xbfgragh.onion/hc/en-us/articles/115014792127-Copyright-notice}{©~2020~The
  New York Times Company}
\end{itemize}

\begin{itemize}
\tightlist
\item
  \href{https://www.nytco.com/}{NYTCo}
\item
  \href{https://help.nytimes3xbfgragh.onion/hc/en-us/articles/115015385887-Contact-Us}{Contact
  Us}
\item
  \href{https://www.nytco.com/careers/}{Work with us}
\item
  \href{https://nytmediakit.com/}{Advertise}
\item
  \href{http://www.tbrandstudio.com/}{T Brand Studio}
\item
  \href{https://www.nytimes3xbfgragh.onion/privacy/cookie-policy\#how-do-i-manage-trackers}{Your
  Ad Choices}
\item
  \href{https://www.nytimes3xbfgragh.onion/privacy}{Privacy}
\item
  \href{https://help.nytimes3xbfgragh.onion/hc/en-us/articles/115014893428-Terms-of-service}{Terms
  of Service}
\item
  \href{https://help.nytimes3xbfgragh.onion/hc/en-us/articles/115014893968-Terms-of-sale}{Terms
  of Sale}
\item
  \href{https://spiderbites.nytimes3xbfgragh.onion}{Site Map}
\item
  \href{https://help.nytimes3xbfgragh.onion/hc/en-us}{Help}
\item
  \href{https://www.nytimes3xbfgragh.onion/subscription?campaignId=37WXW}{Subscriptions}
\end{itemize}
