Sections

SEARCH

\protect\hyperlink{site-content}{Skip to
content}\protect\hyperlink{site-index}{Skip to site index}

\href{https://www.nytimes3xbfgragh.onion/section/us}{U.S.}

\href{https://myaccount.nytimes3xbfgragh.onion/auth/login?response_type=cookie\&client_id=vi}{}

\href{https://www.nytimes3xbfgragh.onion/section/todayspaper}{Today's
Paper}

\href{/section/us}{U.S.}\textbar{}Robert Gnaizda, Lawyer Who Fought for
Social Justice, Dies at 83

\url{https://nyti.ms/312DtrW}

\begin{itemize}
\item
\item
\item
\item
\item
\end{itemize}

Advertisement

\protect\hyperlink{after-top}{Continue reading the main story}

Supported by

\protect\hyperlink{after-sponsor}{Continue reading the main story}

\hypertarget{robert-gnaizda-lawyer-who-fought-for-social-justice-dies-at-83}{%
\section{Robert Gnaizda, Lawyer Who Fought for Social Justice, Dies at
83}\label{robert-gnaizda-lawyer-who-fought-for-social-justice-dies-at-83}}

He challenged redlining banks, employers who discriminated in hiring
and, early in his career, Southern counties that thwarted Black voters.

\includegraphics{https://static01.graylady3jvrrxbe.onion/images/2020/07/29/obituaries/Gnaizda-1/Gnaizda-1-articleLarge.jpg?quality=75\&auto=webp\&disable=upscale}

\href{https://www.nytimes3xbfgragh.onion/by/sam-roberts}{\includegraphics{https://static01.graylady3jvrrxbe.onion/images/2018/02/20/multimedia/author-sam-roberts/author-sam-roberts-thumbLarge.jpg}}

By \href{https://www.nytimes3xbfgragh.onion/by/sam-roberts}{Sam Roberts}

\begin{itemize}
\item
  Aug. 1, 2020Updated 2:21 p.m. ET
\item
  \begin{itemize}
  \item
  \item
  \item
  \item
  \item
  \end{itemize}
\end{itemize}

Robert Gnaizda, a lawyer whose deft powers of persuasion in defending
the civil and economic rights of the poor and minority groups often
rendered messy and costly lawsuits against their adversaries a needless
last resort, died on July 11 in San Francisco. He was 83.

The cause was listed as a heart attack. His son Matthew said he had been
in declining health for some time.

Mr. Gnaizda (pronounced guh-NAYZ-duh) risked his life gathering evidence
in the South in the 1960s to help fight the intimidation that kept Black
citizens from registering and voting. He was also an advocate for farm
workers and the rural poor, fought discrimination in hiring by police
and fire departments, and
\href{https://www.nytimes3xbfgragh.onion/2011/04/14/business/14prosecute.html}{successfully
challenged banks} that victimized Black and Hispanic borrowers.

``Bob Gnaizda was, in my opinion, the most imaginative, creative and
consequential public interest lawyer of his generation in the United
States,'' said J. Anthony Kline, presiding justice of the California
Court of Appeal in San Francisco, who, with Mr. Gnaizda and two other
lawyers, formed a pioneering public interest law firm in California in
1971.

The marginalized plaintiffs he represented ``were devoted to him because
he incorporated them in strategic decision making and ensured they
received credit for victories,'' Justice Kline said by email, and ``the
respect he had from many of his adversaries enabled him to negotiate
settlements that became established progressive norms without the need
to engage in expensive and time-consuming litigation.''

Robert Leslie Gnaizda was born in Brooklyn on Aug. 6, 1936, to Sandra
(Ackerman) Gnaizda, the daughter of Russian and Polish immigrants, who
ran a commercial real estate business, and Samuel Gnaizda, a Jewish
immigrant from Russia, who became a pharmacist. He was raised in the
Brownsville section, then a gritty Jewish enclave, where he defended
vulnerable friends against neighborhood bullies and biked to Ebbets
Field to cheer on Jackie Robinson.

``Dodger fans had a reputation for being rude and rowdy,'' his son Matt
said. ``But 10-year-old Bob observed that the Black fans at the stadium
were extremely polite, well behaved and well dressed --- unlike the
other fans. And this made him begin to question the common
stereotypes.''

As a teenager he also worked in a job training program for Black
youngsters, Matt Gnaizda said. He added: ``These experiences --- plus
the discrimination he felt as a Jew --- led him to see that there was
injustice in the world. And he wanted to do something about it.''

After graduating from Stuyvesant High School in Manhattan, Mr. Gnaizda
attended Columbia University, where he was so struck that his classmates
were almost all white and male that he was prompted to write an article
for the Columbia Daily Spectator, the student,

``I asked two questions: One related to women, the other related to
Black students,'' he said in an interview with
\href{https://www.college.columbia.edu/cct/latest/minds/talking-social-advocacy-robert-gnaizda-57-and-rebecca-kee-05}{Columbia
College Today}, an alumni publication, in 2018. ``I wondered why a
school at the edge of Harlem had only one Black student per
undergraduate class. And because my own mother was outstanding at
everything she did, I wondered why we didn't have any women. But the
paper wouldn't print it; they thought it was too inflammatory.''

Mr. Gnaizda graduated from Columbia in 1957. He considered studying
medicine but decided to apply to law school at Harvard and Yale and
become a lawyer if he was admitted by either. He was accepted by both
and chose Yale. He earned his law degree in 1960.

Asked in the 2018 interview what drove him to abandon corporate law when
he was 28 to pursue a social justice agenda, he replied facetiously: ``I
think boredom. I was a tax attorney, and the work didn't interest me
much. I decided to go to Mississippi in early 1965, and there I realized
that I had some natural talents.''

By cultivating local white civic leaders in conversations that dealt as
much with baseball as with voting, he was able to gather enough evidence
during a public hearing on the disenfranchisement of Black people in
Clay County, Miss., to help propel passage of the 1965 Voting Rights
Act.

Mr. Gnaizda lived in San Francisco. In addition to his son Matthew, he
is survived by his wife, Claudia Viek; another son, Joshua; and a
granddaughter. His first marriage ended in divorce, as did his second,
to Ellen Eatough. (Both his sons are from that marriage.)

He founded California Rural Legal Assistance in 1966 and, five years
later, \href{https://www.publicadvocates.org/}{Public Advocates}, a firm
that defended underdogs, with Justice Kline, Sid Wolinsky and Peter
Sitkin. He also helped found the
\href{https://greenlining.org/}{Greenlining Institute} in 1993 to
discourage financial institutions from discriminating against Black and
Hispanic home buyers. He was later general counsel for the
\href{https://www.naac.org/}{National Asian American Coalition} and the
\href{https://www.nationaldiversitycoalition.org/}{National Diversity
Coalition}.

\includegraphics{https://static01.graylady3jvrrxbe.onion/images/2020/07/29/obituaries/Gnaizda-2/merlin_44426521_4ed3fc07-b0f8-4dcd-bc59-52f4a2d9165d-articleLarge.jpg?quality=75\&auto=webp\&disable=upscale}

Through those and other organizations, Mr. Gnaizda accused the Census
Bureau of undercounting Hispanic residents in 1970 and pressured the San
Francisco Police Department to hire more minority recruits in the 1970s.

He fought
\href{https://www.nytimes3xbfgragh.onion/2017/08/24/upshot/how-redlinings-racist-effects-lasted-for-decades.html}{redlining
by banks} that refused to lend to residents of minority neighborhoods.
He fostered investment in those communities by leveraging the lenders'
lack of public-spiritedness against them when they needed government
permission to merge with other banks.

From 1975 to 1976, he was deputy secretary of health and welfare for
Gov. Jerry Brown of California.
\href{https://www.nytimes3xbfgragh.onion/2014/03/15/business/california-sued-over-diversion-of-money-from-national-mortgage-settlement.html}{But
years later, when Mr. Brown was governor for a second time. Mr. Gnaizda
didn't hesitate to sue} him for diverting the proceeds the state
received from lawsuits relating to national mortgage fraud settlements.
The state was eventually ordered to repay the money.

Mr. Gnaizda was interviewed for the Oscar-winning
documentary\href{https://www.nytimes3xbfgragh.onion/2010/10/08/movies/08inside.html}{``Inside
Job''} (2010) about his efforts in the mid-2000s to warn the Federal
Reserve of the
\href{https://www.nytimes3xbfgragh.onion/2007/12/18/business/18subprime.html}{impending
subprime mortgage crisis}.

Image

Mr. Gnaizda, far right, at a news conference in 1975, when was
California's deputy secretary of health and welfare.Credit...via Matt
Gnaizda

His penchant for representing victims of bullying led him into some
unusual cases. There were, for example, the two 7-year-olds in
California who sued Pacific Bell Telephone in 1985 for failing to inform
them that they would incur a 50-cent charge every time they dialed a
\href{https://www.nytimes3xbfgragh.onion/1985/03/28/us/around-the-nation-santa-claus-phone-line-brings-10-million-suit.html}{Santa
Claus} line. (He also founded a Giraffe Appreciation Society whose
directors, all children, paid tribute to the ruminants ``who don't do
any harm and who stick their necks out.'')

What distinguished Mr. Gnaizda as a zealous public interest lawyer, his
son Matt said, was that he ``never saw the other side as evil; he saw
them as people who he disagreed with on certain issues, but could be
convinced to change their behavior.''

Mr. Gnaizda agreed that many of his lawsuits were successful because he
didn't force his opponents into a corner. ``They were just on a
different side, and when they lost, they would usually keep their
word,'' he said in the Columbia College Today interview

Rebecca Kee, who worked with Mr. Gnaizda at the National Diversity
Coalition, put it this way: ``Bob will sue you, and somehow you'll still
like him.''

Advertisement

\protect\hyperlink{after-bottom}{Continue reading the main story}

\hypertarget{site-index}{%
\subsection{Site Index}\label{site-index}}

\hypertarget{site-information-navigation}{%
\subsection{Site Information
Navigation}\label{site-information-navigation}}

\begin{itemize}
\tightlist
\item
  \href{https://help.nytimes3xbfgragh.onion/hc/en-us/articles/115014792127-Copyright-notice}{©~2020~The
  New York Times Company}
\end{itemize}

\begin{itemize}
\tightlist
\item
  \href{https://www.nytco.com/}{NYTCo}
\item
  \href{https://help.nytimes3xbfgragh.onion/hc/en-us/articles/115015385887-Contact-Us}{Contact
  Us}
\item
  \href{https://www.nytco.com/careers/}{Work with us}
\item
  \href{https://nytmediakit.com/}{Advertise}
\item
  \href{http://www.tbrandstudio.com/}{T Brand Studio}
\item
  \href{https://www.nytimes3xbfgragh.onion/privacy/cookie-policy\#how-do-i-manage-trackers}{Your
  Ad Choices}
\item
  \href{https://www.nytimes3xbfgragh.onion/privacy}{Privacy}
\item
  \href{https://help.nytimes3xbfgragh.onion/hc/en-us/articles/115014893428-Terms-of-service}{Terms
  of Service}
\item
  \href{https://help.nytimes3xbfgragh.onion/hc/en-us/articles/115014893968-Terms-of-sale}{Terms
  of Sale}
\item
  \href{https://spiderbites.nytimes3xbfgragh.onion}{Site Map}
\item
  \href{https://help.nytimes3xbfgragh.onion/hc/en-us}{Help}
\item
  \href{https://www.nytimes3xbfgragh.onion/subscription?campaignId=37WXW}{Subscriptions}
\end{itemize}
