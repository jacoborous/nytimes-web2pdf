Sections

SEARCH

\protect\hyperlink{site-content}{Skip to
content}\protect\hyperlink{site-index}{Skip to site index}

\href{https://www.nytimes3xbfgragh.onion/section/us}{U.S.}

\href{https://myaccount.nytimes3xbfgragh.onion/auth/login?response_type=cookie\&client_id=vi}{}

\href{https://www.nytimes3xbfgragh.onion/section/todayspaper}{Today's
Paper}

\href{/section/us}{U.S.}\textbar{}Trader Joe's Defends Product Labels
Criticized as Racist

\url{https://nyti.ms/2XiX7z0}

\begin{itemize}
\item
\item
\item
\item
\item
\end{itemize}

Advertisement

\protect\hyperlink{after-top}{Continue reading the main story}

Supported by

\protect\hyperlink{after-sponsor}{Continue reading the main story}

\hypertarget{trader-joes-defends-product-labels-criticized-as-racist}{%
\section{Trader Joe's Defends Product Labels Criticized as
Racist}\label{trader-joes-defends-product-labels-criticized-as-racist}}

The company had previously said the names of international-themed
products that were intended to promote inclusiveness, such as Trader
José and Trader Ming's, ``may now have the opposite effect.''

\includegraphics{https://static01.graylady3jvrrxbe.onion/images/2020/07/19/multimedia/19xp-trader-joes-pix/19xp-trader-joes-pix-articleLarge.jpg?quality=75\&auto=webp\&disable=upscale}

By \href{https://www.nytimes3xbfgragh.onion/by/allyson-waller}{Allyson
Waller}

\begin{itemize}
\item
  Aug. 1, 2020
\item
  \begin{itemize}
  \item
  \item
  \item
  \item
  \item
  \end{itemize}
\end{itemize}

Weeks after admitting that some of its international-themed product
labels might have fallen short of an ``attempt at inclusiveness,'' the
grocery store chain Trader Joe's is rejecting criticism of the labels
--- some with names like Trader José and Trader Ming's --- as racist.

After an
\href{https://www.nytimes3xbfgragh.onion/2020/07/19/business/trader-joes-petition.html}{online
petition} denounced the company's use of labels such as Arabian Joe's,
Trader Giotto's and Trader Joe San as racist because it ``exoticizes
other cultures,'' Trader Joe's announced that it would keep names that
it felt still resonated with customers.

``We disagree that any of these labels are racist,'' the company said in
a
\href{https://www.traderjoes.com/announcement/a-note-about-our-product-naming}{statement}
on July 24. ``We do not make decisions based on petitions.''

``We thought then --- and still do --- that this naming of products
could be fun and show appreciation for other cultures,'' it said.

Earlier in July, however, Kenya Friend-Daniel, the company's national
director of public relations, said the company was in the process of
updating labels to bear only the Trader Joe's name.

``While this approach to product naming may have been rooted in a
lighthearted attempt at inclusiveness, we recognize that it may now have
the opposite effect --- one that is contrary to the welcoming, rewarding
customer experience we strive to create every day,'' the spokeswoman
said in a statement on July 19. ``With this in mind, we made the
decision several years ago to use only the Trader Joe's name on our
products moving forward.''

On Saturday, she said that she had been referring to new products that
the company introduced after 2017, not products that existed before
then.

``We will continue to evaluate our products, as we always do, and if
certain products/product packaging are not resonating well, changes will
be made,'' Ms. Friend-Daniel said.

Asked about the status of other product labels, Ms. Friend-Daniel said
that ``for the remaining products, we will change packages or
discontinue products that do not resonate'' with customers.

The spokeswoman added that labels such as Arabian Joe's and Armenian
Joe's were no longer in use, and that the label Trader Joe San is
currently used on only about three products.

Briones Bedell, who started the
\href{https://www.change.org/p/trader-joe-s-remove-racist-packaging-from-your-products}{online
petition}that led to renewed scrutiny of the company's labels, said on
Saturday she was ``honestly surprised'' by the company's comments.

``I see it to be a complete reversal to their previous commitment to
removing the labels from the international foods,'' she said.

With her petition, Ms. Bedell, 17, said she wanted to raise awareness of
stereotypes that are of a piece with the larger discussions about race
happening across the country.

``They rely only on characters and kind of vague ideas and not anything
of actual substance or legitimacy,'' Ms. Bedell said of the labels. ``It
becomes a tool of othering.''

Recently, major food companies have committed to re-examining their use
of racist imagery after nationwide protests over police brutality.
Quaker Oats said in June it would retire the
\href{https://www.nytimes3xbfgragh.onion/2020/06/17/business/media/aunt-jemima-racial-stereotype.html}{Aunt
Jemima name and image}. Mars Foods followed suit only hours later,
saying it would ``evolve'' the
\href{https://www.nytimes3xbfgragh.onion/2020/06/17/business/aunt-jemima-mrs-butterworth-uncle-ben.html}{Uncle
Ben's rice} brand.

According to Paul Andrew Galvani, an adjunct professor of marketing at
the University of Houston, Trader Joe's most likely issued its recent
defenses after consulting with customers who are part of its target
market --- a common practice for retail businesses.

``If their consumers are suddenly up in arms and saying, `You know what,
we're not going to shop Trader Joe's unless you change,' then, like any
sensible marketer, they're going to look at that and say, `Well, when it
starts to impact our bottom line, that's when we may have to make a
change,''' Mr. Galvani said.

Advertisement

\protect\hyperlink{after-bottom}{Continue reading the main story}

\hypertarget{site-index}{%
\subsection{Site Index}\label{site-index}}

\hypertarget{site-information-navigation}{%
\subsection{Site Information
Navigation}\label{site-information-navigation}}

\begin{itemize}
\tightlist
\item
  \href{https://help.nytimes3xbfgragh.onion/hc/en-us/articles/115014792127-Copyright-notice}{©~2020~The
  New York Times Company}
\end{itemize}

\begin{itemize}
\tightlist
\item
  \href{https://www.nytco.com/}{NYTCo}
\item
  \href{https://help.nytimes3xbfgragh.onion/hc/en-us/articles/115015385887-Contact-Us}{Contact
  Us}
\item
  \href{https://www.nytco.com/careers/}{Work with us}
\item
  \href{https://nytmediakit.com/}{Advertise}
\item
  \href{http://www.tbrandstudio.com/}{T Brand Studio}
\item
  \href{https://www.nytimes3xbfgragh.onion/privacy/cookie-policy\#how-do-i-manage-trackers}{Your
  Ad Choices}
\item
  \href{https://www.nytimes3xbfgragh.onion/privacy}{Privacy}
\item
  \href{https://help.nytimes3xbfgragh.onion/hc/en-us/articles/115014893428-Terms-of-service}{Terms
  of Service}
\item
  \href{https://help.nytimes3xbfgragh.onion/hc/en-us/articles/115014893968-Terms-of-sale}{Terms
  of Sale}
\item
  \href{https://spiderbites.nytimes3xbfgragh.onion}{Site Map}
\item
  \href{https://help.nytimes3xbfgragh.onion/hc/en-us}{Help}
\item
  \href{https://www.nytimes3xbfgragh.onion/subscription?campaignId=37WXW}{Subscriptions}
\end{itemize}
