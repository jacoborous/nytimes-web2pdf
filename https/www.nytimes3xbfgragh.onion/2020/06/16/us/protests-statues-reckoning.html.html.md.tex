Sections

SEARCH

\protect\hyperlink{site-content}{Skip to
content}\protect\hyperlink{site-index}{Skip to site index}

\href{https://www.nytimes3xbfgragh.onion/section/us}{U.S.}

\href{https://myaccount.nytimes3xbfgragh.onion/auth/login?response_type=cookie\&client_id=vi}{}

\href{https://www.nytimes3xbfgragh.onion/section/todayspaper}{Today's
Paper}

\href{/section/us}{U.S.}\textbar{}Reconsidering the Past, One Statue at
a Time

\url{https://nyti.ms/3d8SnRl}

\begin{itemize}
\item
\item
\item
\item
\item
\item
\end{itemize}

\href{https://www.nytimes3xbfgragh.onion/news-event/george-floyd-protests-minneapolis-new-york-los-angeles?action=click\&pgtype=Article\&state=default\&region=TOP_BANNER\&context=storylines_menu}{Race
and America}

\begin{itemize}
\tightlist
\item
  \href{https://www.nytimes3xbfgragh.onion/2020/07/26/us/protests-portland-seattle-trump.html?action=click\&pgtype=Article\&state=default\&region=TOP_BANNER\&context=storylines_menu}{Protesters
  Return to Other Cities}
\item
  \href{https://www.nytimes3xbfgragh.onion/2020/07/24/us/portland-oregon-protests-white-race.html?action=click\&pgtype=Article\&state=default\&region=TOP_BANNER\&context=storylines_menu}{Portland
  at the Center}
\item
  \href{https://www.nytimes3xbfgragh.onion/2020/07/23/podcasts/the-daily/portland-protests.html?action=click\&pgtype=Article\&state=default\&region=TOP_BANNER\&context=storylines_menu}{Podcast:
  Showdown in Portland}
\item
  \href{https://www.nytimes3xbfgragh.onion/interactive/2020/07/16/us/black-lives-matter-protests-louisville-breonna-taylor.html?action=click\&pgtype=Article\&state=default\&region=TOP_BANNER\&context=storylines_menu}{45
  Days in Louisville}
\end{itemize}

Advertisement

\protect\hyperlink{after-top}{Continue reading the main story}

Supported by

\protect\hyperlink{after-sponsor}{Continue reading the main story}

\hypertarget{reconsidering-the-past-one-statue-at-a-time}{%
\section{Reconsidering the Past, One Statue at a
Time}\label{reconsidering-the-past-one-statue-at-a-time}}

From Virginia to New Mexico, protests over police brutality have brought
hundreds of years of American history bubbling to the surface.

\includegraphics{https://static01.graylady3jvrrxbe.onion/images/2020/06/16/us/16UNREST-SYMBOLS-onate/merlin_173590581_75881914-27e0-4c85-af38-cefa90724d53-articleLarge.jpg?quality=75\&auto=webp\&disable=upscale}

By \href{https://www.nytimes3xbfgragh.onion/by/sarah-mervosh}{Sarah
Mervosh},
\href{https://www.nytimes3xbfgragh.onion/by/simon-romero}{Simon Romero}
and Lucy Tompkins

\begin{itemize}
\item
  Published June 16, 2020Updated June 25, 2020
\item
  \begin{itemize}
  \item
  \item
  \item
  \item
  \item
  \item
  \end{itemize}
\end{itemize}

The boiling anger that exploded in the days after
\href{https://www.nytimes3xbfgragh.onion/article/george-floyd-who-is.html}{George
Floyd} gasped his final breaths is now fueling a national movement to
topple perceived symbols of racism and oppression in the United States,
as protests over police brutality against African-Americans expand to
include demands for a more honest accounting of American history.

In Portland, Ore., demonstrators protesting police killings
\href{https://www.oregonlive.com/portland/2020/06/protesters-take-down-thomas-jefferson-statue-in-front-of-portlands-jefferson-high-school.html}{turned
their ire to Thomas Jefferson}, toppling a statue of the founding father
who also
\href{https://www.monticello.org/slavery/slavery-faqs/property/}{enslaved
more than 600 people}.

In
\href{https://www.nytimes3xbfgragh.onion/2020/06/10/us/christopher-columbus-statue-boston-richmond.html}{Richmond,
Va.,} a statue of the Italian navigator and colonizer Christopher
Columbus was spray-painted, set on fire and thrown into a lake.

And in Albuquerque, tensions over a
\href{https://www.nytimes3xbfgragh.onion/2020/07/15/world/europe/bristol-statue-black-lives-matter.html}{statue}
of Juan de Oñate, a 16th-century colonial governor exiled from New
Mexico over cruel treatment of Native Americans,
\href{https://www.nytimes3xbfgragh.onion/2020/06/15/us/conquistador-onate-albuquerque-new-mexico-unrest.html}{erupted
in street skirmishes and a blast of gunfire} before the monument was
removed on Tuesday.

Across the country, monuments criticized as symbols of historical
oppression have been defaced and brought down at warp speed in recent
days. The movement initially set its sights on
\href{https://www.nytimes3xbfgragh.onion/2020/06/23/style/statue-richmond-lee.html}{Confederate
symbols} and examples of racism against African-Americans, but has since
exploded into a broader cultural moment, forcing a reckoning over such
issues as European colonization and the oppression of Native Americans.

In New Mexico, it has surfaced generations-old tensions among
Indigenous, Hispanic and Anglo residents and brought 400 years of
turbulent history bubbling to the surface.

``We're at this inflection point,'' said Keegan King, a member of Pueblo
of Ácoma, which endured a massacre of 800 or more people directed by
Oñate, the brutal Spanish conquistador and colonial governor. The Black
Lives Matter movement, he said, had encouraged people to examine the
history around them, and not all of it was merely written in books.

``These pieces of systemic racism took the form of monuments and statues
and parks,'' Mr. King said.

The debate over how to represent the uncomfortable parts of American
history has been going on for decades, but the traction for knocking
down monuments seen in recent days raises new questions about whether it
will result in a fundamental shift in how history is taught to new
generations.

``It is a turning point insofar as there are a lot of people now who are
invested in telling the story that historians have been laying down for
decades,'' said Julian Maxwell Hayter, a historian and associate
professor at the University of Richmond.

He said that statues removed from parks and street corners could be
teaching points if they are placed in museums, side-by-side with
documents and first-person accounts from the era.

``Let's say you put a Columbus statue in a museum and you show students
the way Columbus was lionized in a history textbook and you have them
read `Devastation of the Indies' by de Las Casas,'' he said. ``Then you
have to ask, why were people invested in telling this particular version
of Christopher Columbus's history?''

The calls to bring down monuments have spanned far and wide, in large
cities like Philadelphia and rural places like Columbus, Miss., touching
both relatively obscure historical figures and deeply revered cultural
symbols.

In Raleigh, N.C., the statue of a former newspaper publisher who was
also a white supremacist
\href{https://www.newsobserver.com/news/local/article243559272.html?}{was
removed on Tuesday}. In Sacramento, a tribute to John Sutter, a settler
famous for his role in the California gold rush who enslaved and
exploited Native Americans,
\href{https://www.capradio.org/articles/2020/06/15/sutter-health-removes-john-sutter-statue-amid-complaints-about-racial-history/}{was
taken down} this week. And in Dallas, construction crews
\href{https://www.dallasnews.com/arts-entertainment/visual-arts/2020/06/03/the-statue-of-the-texas-ranger-at-love-field-may-be-coming-down/}{recently
removed} a statue of a Texas Ranger, long seen as a mythical figure in
Texas folklore, amid concerns over
\href{https://www.dallasnews.com/arts-entertainment/visual-arts/2020/06/03/the-statue-of-the-texas-ranger-at-love-field-may-be-coming-down/}{historical
episodes of police brutality and racism} within the law enforcement
agency.

The push has largely been welcomed by activists from the Black Lives
Matter movement who see Confederate and other monuments as reminders of
the oppressive history that created the reality they are battling today.
But some of them worried that the focus on historic symbols would do
little to keep attention on the more pressing issue of ending the brutal
treatment of many African-Americans by the police.

``I don't know if I would say a distraction, because I think people
definitely have the ability to be nuanced,'' said **** Alisha Sonnier, a
24-year-old mental health advocate from St. Louis who is concerned that
taking down statues could be an ``easy appeasement.''

``The statue being removed is not going to keep anyone from dying,'' she
said. ``It's not going to save a life.''

Cleon Jones, a 77-year-old activist in Africatown, Ala., formed on
Mobile Bay by the last known shipment of slaves to the United States
from Africa, said he felt frustrated by the notion that progress toward
equality could be stalled by rancor over Confederate monuments.

``We've got to move forward, not look back,'' he said. ``As long as we
are dealing with these statues, we're not moving forward.''

The focus on removing statues has revealed deep civil divisions far
outside the Black Lives Matter movement that are hundreds of years in
the making. It has spurred a backlash among Italian-Americans who have
long regarded Columbus as a point of pride, and also among some
Hispanics in New Mexico, who celebrate an era when Anglos did not
dominate public life.

\includegraphics{https://static01.graylady3jvrrxbe.onion/images/2020/06/16/us/16UNREST-SYMBOLS-davis/merlin_173435376_f442da92-932c-492a-b6d9-478122e8445f-articleLarge.jpg?quality=75\&auto=webp\&disable=upscale}

Image

The head of a statue of Christopher Columbus was pulled off in Boston
last week amid protests over the killing of George Floyd.Credit...Brian
Snyder/Reuters

Image

In Sacramento, a tribute to John Sutter, a settler famous for his role
in the California gold rush who enslaved and exploited Native Americans,
was taken down this week.Credit...Daniel Kim/The Sacramento Bee, via
Associated Press

``We need to have a broader discussion about our history,'' said
Christine Flowers, a 58-year-old Italian-American immigration lawyer,
who was among a group that gathered to protect a statue of Columbus in
Philadelphia.

But she added, ``It is indefensible to try to erase that history by
pulling down something that is very dear and very symbolic for the
culture of Italian-Americans in Philadelphia.''

In Columbus, Miss., a largely African-American town, county officials
voted on Monday to keep a towering monument to Confederate soldiers ---
``our heroes,'' it calls them --- on the courthouse lawn despite
mounting calls for its removal.

``It's a good time to learn some history,'' said Trip Hairston, a white
county supervisor who opposed removing the monument. ``I don't agree
with all that history, of course, but it is what it is --- it's
history.''

It was an argument that left many of those pushing to remove the statue
perplexed. ``It's commemorating and celebrating a lost battle --- I
don't understand,'' said David Horton, 28, a lifelong resident of
Columbus who first fought against a Confederate monument as a 7th grader
at Robert E. Lee Middle School.

``These are things I have to endure all my life as a young
African-American man living in Mississippi,'' he said. ``It's always
made me feel inferior, it's always made me feel like I shouldn't hold my
head up.''

Symbols of the Confederacy and its legacy of slavery have long been at
the center of the reckoning over historic racism in the country.

At least 114 Confederate symbols were removed in the years after a white
supremacist killed nine people at a historic African-American church in
Charleston, S.C., in 2015,
\href{https://www.splcenter.org/20190201/whose-heritage-public-symbols-confederacy}{according
to a 2019 report} by the Southern Poverty Law Center.

The killing of Mr. Floyd in Minneapolis re-energized that movement, as
demonstrators
chipped\href{https://www.nytimes3xbfgragh.onion/2020/06/02/us/george-floyd-birmingham-confederate-statue.html}{away
at a 52-foot Confederate obelisk} in Birmingham, Ala., and
\href{https://www.nytimes3xbfgragh.onion/2020/06/11/us/Jefferson-Davis-Statue-Richmond.html}{toppled
a statue of Jefferson Davis}, the president of the Confederacy, in
Richmond, Va.

The statues debate has once again focused attention on Columbus, the
voyager who emerged as a symbol of Italians' contribution to American
history in the late 1800s, a time when discrimination against Italians
was rampant. But many in recent days are also talking about how his
arrival signaled the beginning of a violent European colonization that
resulted in a cross-Atlantic slave trade and the genocide and
displacement of many Indigenous peoples.

Columbus statues from Boston to Miami have been brought down or defaced
by protesters. A large Columbus statue was defaced with red graffiti in
Kenosha, Wis., and Gov. Andrew M. Cuomo of New York defended a towering
monument to the explorer at Columbus Circle in Manhattan.

In Philadelphia, supporters went to court to block the removal of a
Columbus statue after another statue, of Frank L. Rizzo, a former mayor
known for discriminatory policies, was removed by the city this month in
the middle of the night. ``You just can't let the mob rule,'' said
George Bochetto, a lawyer who filed the petition.

Tensions over the Oñate monument came to a boil Monday night in
Albuquerque, when dozens of protesters engaged in shouting matches, some
seeing the brutal Spanish governor as a symbol of repression, while
others saw him as a positive symbol of a time before Anglos came to
dominate the Southwest. Then a group of white militia members, on a
self-appointed mission to protect the statue, showed up with guns.

In the mayhem that ensued, a man pulled out a weapon and shot one of the
protesters, critically injuring him.

On Tuesday, the authorities in Bernalillo County filed a charge of
aggravated battery with a deadly weapon against the man with the gun,
identified as Steven Baca, 31. Mr. Baca ran unsuccessfully for the
Albuquerque City Council last year.

Image

Protesters gathered at a monument to General Robert E. Lee in Richmond,
Va.Credit...Eze Amos/Getty Images

In Minneapolis, where the demonstrations over Mr. Floyd's death ignited
new protest movements in dozens of cities, many said they never expected
them to grow into an international reckoning over racist symbols. Still,
they said, it was only a matter of time before the latest police killing
of a black man led to something more lasting than previous protests.

``It's kind of like a wound that has a scab,'' said Teron Carter, 49,
standing
\href{https://www.nytimes3xbfgragh.onion/2020/06/15/us/cup-foods-minneapolis-george-floyd.html}{across
the street from Cup Foods}, the deli near Mr. Floyd's fatal encounter
with the police. ``A wound that has a scab is still a wound, it's just
that the scab is on top. And if you scrape that scab a certain way, it
reopens the wound.''

He attributed the scope of the burgeoning movement to built-up grief and
to the energy of young people who simply are not willing to put up with
walking by Confederate and other statues each day.

``It's not just an isolated city event,'' Mr. Carter said. ``Now
everybody saw the opportunity and said, `If we don't get in there and
talk like Minneapolis is talking, then we aren't going to be heard.'''

Reporting was contributed by Nicholas Bogel-Burroughs, Julie Bosman,
John Eligon, Thomas Fuller, Rick Rojas and Matthew Teague.

Advertisement

\protect\hyperlink{after-bottom}{Continue reading the main story}

\hypertarget{site-index}{%
\subsection{Site Index}\label{site-index}}

\hypertarget{site-information-navigation}{%
\subsection{Site Information
Navigation}\label{site-information-navigation}}

\begin{itemize}
\tightlist
\item
  \href{https://help.nytimes3xbfgragh.onion/hc/en-us/articles/115014792127-Copyright-notice}{©~2020~The
  New York Times Company}
\end{itemize}

\begin{itemize}
\tightlist
\item
  \href{https://www.nytco.com/}{NYTCo}
\item
  \href{https://help.nytimes3xbfgragh.onion/hc/en-us/articles/115015385887-Contact-Us}{Contact
  Us}
\item
  \href{https://www.nytco.com/careers/}{Work with us}
\item
  \href{https://nytmediakit.com/}{Advertise}
\item
  \href{http://www.tbrandstudio.com/}{T Brand Studio}
\item
  \href{https://www.nytimes3xbfgragh.onion/privacy/cookie-policy\#how-do-i-manage-trackers}{Your
  Ad Choices}
\item
  \href{https://www.nytimes3xbfgragh.onion/privacy}{Privacy}
\item
  \href{https://help.nytimes3xbfgragh.onion/hc/en-us/articles/115014893428-Terms-of-service}{Terms
  of Service}
\item
  \href{https://help.nytimes3xbfgragh.onion/hc/en-us/articles/115014893968-Terms-of-sale}{Terms
  of Sale}
\item
  \href{https://spiderbites.nytimes3xbfgragh.onion}{Site Map}
\item
  \href{https://help.nytimes3xbfgragh.onion/hc/en-us}{Help}
\item
  \href{https://www.nytimes3xbfgragh.onion/subscription?campaignId=37WXW}{Subscriptions}
\end{itemize}
