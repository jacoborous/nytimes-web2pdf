Sections

SEARCH

\protect\hyperlink{site-content}{Skip to
content}\protect\hyperlink{site-index}{Skip to site index}

\href{https://www.nytimes3xbfgragh.onion/section/world/europe}{Europe}

\href{https://myaccount.nytimes3xbfgragh.onion/auth/login?response_type=cookie\&client_id=vi}{}

\href{https://www.nytimes3xbfgragh.onion/section/todayspaper}{Today's
Paper}

\href{/section/world/europe}{Europe}\textbar{}France Brings 10 Children
of French Jihadists Home From Syria

\url{https://nyti.ms/2CnIDpF}

\begin{itemize}
\item
\item
\item
\item
\item
\item
\end{itemize}

Advertisement

\protect\hyperlink{after-top}{Continue reading the main story}

Supported by

\protect\hyperlink{after-sponsor}{Continue reading the main story}

\hypertarget{france-brings-10-children-of-french-jihadists-home-from-syria}{%
\section{France Brings 10 Children of French Jihadists Home From
Syria}\label{france-brings-10-children-of-french-jihadists-home-from-syria}}

About 270 children of French citizens remain stuck in Syria, according
to rights groups, which argue that leaving them stranded in squalid
detention camps exposes them to illness and radicalization.

\includegraphics{https://static01.graylady3jvrrxbe.onion/images/2020/06/22/world/22france-isis-kids/merlin_152756715_40dabb4d-a4a2-441c-beaa-d72204b57bea-articleLarge.jpg?quality=75\&auto=webp\&disable=upscale}

By \href{https://www.nytimes3xbfgragh.onion/by/constant-meheut}{Constant
Méheut} and \href{https://www.nytimes3xbfgragh.onion/by/ben-hubbard}{Ben
Hubbard}

\begin{itemize}
\item
  June 22, 2020
\item
  \begin{itemize}
  \item
  \item
  \item
  \item
  \item
  \item
  \end{itemize}
\end{itemize}

PARIS --- France on Monday brought home 10 children of French jihadists
who had been stuck in sprawling detention camps in northeastern Syria
since at least the collapse of the Islamic State last year.

The French foreign ministry said it had decided on repatriation because
of ``the situation of these particularly vulnerable young children.''
About 270 children of French citizens remain in Syria, according to
rights groups, which argue that this leaves the children at risk of
illness and radicalization.

About
\href{https://www.nytimes3xbfgragh.onion/2020/05/31/world/middleeast/isis-children-syria-camps.html}{900
children from Western nations} including France, Belgium, Canada and
Australia are still stuck in the camps, which sprung up to hold
relatives of Islamic State fighters who survived the battles with
Kurdish-led fighters and a United States-led military coalition aimed at
destroying the so-called caliphate.

But since the jihadists
\href{https://www.nytimes3xbfgragh.onion/2019/03/23/world/middleeast/isis-syria-caliphate.html}{lost
their final foothold in Syria} in March 2019, many Western nations have
resisted calls from the Kurdish-led forces that run the camps to
repatriate their citizens, saying they don't want to bring home people
who chose to join a terrorist organization.

Rights groups have pressed the governments to at least bring home their
citizens' children, arguing that the minors did not choose to go to
Syria or, in many cases, were born there.

But only small numbers have been repatriated, many because they were
orphans or because they needed lifesaving medical care not available in
Syria.

\includegraphics{https://static01.graylady3jvrrxbe.onion/images/2020/06/22/world/22france-isis-kids4/merlin_152756379_4c3a119c-dbb2-4f24-9799-bb4601cd5beb-articleLarge.jpg?quality=75\&auto=webp\&disable=upscale}

The French foreign ministry announced the arrival of the 10 children in
a statement on Monday, saying they were handed over to judicial
authorities and were under medical supervision and being cared for by
social services.

The statement gave no further information about the children, but
lawyers representing their relatives said they included three orphans
and seven other children from two mothers who had agreed to give up
custody so their children could travel to France.

The repatriated children included two brothers and the twin sister of
Taymia, a
\href{https://www.nytimes3xbfgragh.onion/2020/05/31/world/middleeast/isis-children-syria-camps.html}{7-year-old
French girl} who suffered from a double heart defect and was flown to
France in April for urgent medical care after her health had
deteriorated.

In a phone interview last month from
\href{https://www.nytimes3xbfgragh.onion/2019/03/29/world/middleeast/isis-syria-women-children.html}{the
sprawling Al Hol camp} where the family was stuck in Syria, Taymia's
mother said she had grown so worried about the potential for
\href{https://www.nytimes3xbfgragh.onion/2019/09/03/world/middleeast/isis-alhol-camp-syria.html}{radicalization}
in the camp that she had agreed to allow all of her children to leave.

``That's why I'm ready to separate from them and let France take them
back,'' she said.

It was unclear why France brought the children home now after leaving
them behind in April, but all three arrived safely on Monday as part of
the repatriation operation, said Ludovic Rivière, the family's lawyer.

The New York Times is not publishing Taymia's last name, nor the names
of her mother and siblings, to protect the children's privacy.

The four other non-orphaned children who arrived in France on Monday
were taken from a Kurdish-run detention facility,
\href{https://www.nytimes3xbfgragh.onion/2018/07/04/world/middleeast/islamic-state-families-syria.html}{Roj
Camp}, after their mother agreed to give up custody, said Jean-Charles
Brisard, the director of the Paris-based
\href{http://cat-int.org/?lang=en}{Center for the Analysis of
Terrorism}. She kept her other two children with her in the camp.

Image

Children playing at Roj Camp in 2018.Credit...Ivor Prickett for The New
York Times

``Some women at some point feel compelled to try to separate from their
children because the living conditions in the camps are too difficult,''
said Marie Dosé, a French lawyer who has campaigned for the repatriation
of all former French residents of the Islamic State.

The French government's actions have lagged behind its vows to
repatriate the children of its citizens stranded in Syria.

``They are children, they didn't choose to go to these battlefields,
they didn't choose to join the jihadists' operations,'' France's justice
minister, Nicole Belloubet, told a French radio station on Monday,
adding that, ``when conditions permit,'' France should repatriate all
minors and orphans.

But so far, the government has followed a case-by-case policy that
\href{https://www.nytimes3xbfgragh.onion/2019/03/15/world/europe/france-isis-repatriates-children.html}{prioritizes
orphans} and children whose mothers surrender custody. Under that
policy, 18 children, including 15 orphans, had been brought home before
the 10 who arrived on Monday, leaving about 270 other French children
\href{https://www.nytimes3xbfgragh.onion/2019/05/08/world/middleeast/isis-prisoners-children-women.html}{marooned
in dire conditions in the camps}.

Countries like
\href{https://www.nytimes3xbfgragh.onion/2018/02/24/world/europe/chechnya-russia-isis-children-return.html}{Russia},
Uzbekistan and
\href{https://www.nytimes3xbfgragh.onion/2019/08/10/world/europe/kazakhstan-women-islamic-state-deradicalization.html}{Kazakhstan}
have each repatriated more than 100 of their citizens, many more than
Western nations.

With public opinion firmly against bringing home those who left to fight
with ISIS, France has long sought to
\href{https://www.nytimes3xbfgragh.onion/2019/11/17/world/europe/turkey-isis-fighters-europe.html}{avoid
dealing with French jihadists}, even preferring to
\href{https://www.nytimes3xbfgragh.onion/2020/01/26/world/europe/france-ghost-trials-isis.html}{try
dead fighters rather than the living}.

The French authorities have made it clear that it sees adult women who
joined ISIS as ``fighters'' who must be tried where they committed their
alleged crimes, in Syria or Iraq, suggesting that the mothers were
unlikely to be repatriated with their children.

Image

The former wife of an ISIS fighter and her children at a rehabilitation
center in Kazakhstan, one of the leaders in repatriating women and
children.Credit...Tara Todras-Whitehill for The New York Times

Both Mr. Rivière and Ms. Dosé, the French lawyers, said France had
initially planned to repatriate more than 10 children but that some
mothers, caught off guard by the sudden operation, had decided not to
let their children go.

The 10 children were handed over to a delegation from the French foreign
ministry that traveled to northeastern Syria to meet with Kurdish
officials. The delegation included Éric Chevallier, a former French
ambassador to Syria, according to a
\href{https://twitter.com/abdulkarimomar1/status/1274860769750941696}{photo
of the meeting posted on Twitter} by Abdulkarim Omar, a foreign affairs
official with the Kurdish-led administration.

It was unclear when the photo was taken, but many European officials
have cited the danger of sending diplomats into Syria as one of the
barriers to repatriation.

The grandfather of four other French children who have been in a camp in
Syria with their mother since early 2018 called the arrival of the 10
children ``a glimmer of hope.''

But he said the repatriation process would remain limited as long as
France waited for mothers to give up custody of their children.

``If we want to bring the children back, it's not by waiting for all the
women to give in one by one,'' said the grandfather, who gave only his
last name, Lopez, to protect the family's privacy.

The new repatriations followed a large screening and registration
operation by the Kurdish authorities in the foreigners' section of Al
Hol camp, where about 10,000 women and children from countries other
than Iraq and Syria are held.

The operation last month was intended to collect data about residents of
the camp and identify women who are still active ISIS members and might
try to escape.

Last May, the Center for the Analysis of Terrorism estimated that 13
female French jihadists --- including Hayat Boumeddiene, the wife of
\href{https://www.nytimes3xbfgragh.onion/2015/01/11/world/europe/neighbors-say-suspect-in-french-attacks-and-his-companion-lived-quiet-lives.html}{Amedy
Coulibaly}, who carried out a terror attack in Paris in January 2015 ---
are believed to be on the run.

The Kurdish authorities in northeastern Syria said the screening
operation was meant to ``facilitate coordination with the countries
whose nationals reside in the camp and urge them to assume their
responsibilities toward their citizens.''

The Kurdish forces have long called for the repatriation of all
foreigners, arguing that
\href{https://www.nytimes3xbfgragh.onion/2019/10/13/us/politics/isis-prisoners-kurds.html}{they
cannot detain them indefinitely in an unstable region}.

Constant Méheut reported from Paris, and Ben Hubbard from Beirut,
Lebanon.

Advertisement

\protect\hyperlink{after-bottom}{Continue reading the main story}

\hypertarget{site-index}{%
\subsection{Site Index}\label{site-index}}

\hypertarget{site-information-navigation}{%
\subsection{Site Information
Navigation}\label{site-information-navigation}}

\begin{itemize}
\tightlist
\item
  \href{https://help.nytimes3xbfgragh.onion/hc/en-us/articles/115014792127-Copyright-notice}{©~2020~The
  New York Times Company}
\end{itemize}

\begin{itemize}
\tightlist
\item
  \href{https://www.nytco.com/}{NYTCo}
\item
  \href{https://help.nytimes3xbfgragh.onion/hc/en-us/articles/115015385887-Contact-Us}{Contact
  Us}
\item
  \href{https://www.nytco.com/careers/}{Work with us}
\item
  \href{https://nytmediakit.com/}{Advertise}
\item
  \href{http://www.tbrandstudio.com/}{T Brand Studio}
\item
  \href{https://www.nytimes3xbfgragh.onion/privacy/cookie-policy\#how-do-i-manage-trackers}{Your
  Ad Choices}
\item
  \href{https://www.nytimes3xbfgragh.onion/privacy}{Privacy}
\item
  \href{https://help.nytimes3xbfgragh.onion/hc/en-us/articles/115014893428-Terms-of-service}{Terms
  of Service}
\item
  \href{https://help.nytimes3xbfgragh.onion/hc/en-us/articles/115014893968-Terms-of-sale}{Terms
  of Sale}
\item
  \href{https://spiderbites.nytimes3xbfgragh.onion}{Site Map}
\item
  \href{https://help.nytimes3xbfgragh.onion/hc/en-us}{Help}
\item
  \href{https://www.nytimes3xbfgragh.onion/subscription?campaignId=37WXW}{Subscriptions}
\end{itemize}
