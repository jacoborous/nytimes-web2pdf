Sections

SEARCH

\protect\hyperlink{site-content}{Skip to
content}\protect\hyperlink{site-index}{Skip to site index}

\href{https://www.nytimes3xbfgragh.onion/section/books/review}{Book
Review}

\href{https://myaccount.nytimes3xbfgragh.onion/auth/login?response_type=cookie\&client_id=vi}{}

\href{https://www.nytimes3xbfgragh.onion/section/todayspaper}{Today's
Paper}

\href{/section/books/review}{Book Review}\textbar{}Raina Telgemeier
Can't Wait to Break Bread With Her Friends Again

\url{https://nyti.ms/3cTPALE}

\begin{itemize}
\item
\item
\item
\item
\item
\end{itemize}

\href{https://www.nytimes3xbfgragh.onion/spotlight/at-home?action=click\&pgtype=Article\&state=default\&region=TOP_BANNER\&context=at_home_menu}{At
Home}

\begin{itemize}
\tightlist
\item
  \href{https://www.nytimes3xbfgragh.onion/2020/07/28/books/time-for-a-literary-road-trip.html?action=click\&pgtype=Article\&state=default\&region=TOP_BANNER\&context=at_home_menu}{Take:
  A Literary Road Trip}
\item
  \href{https://www.nytimes3xbfgragh.onion/2020/07/29/magazine/bored-with-your-home-cooking-some-smoky-eggplant-will-fix-that.html?action=click\&pgtype=Article\&state=default\&region=TOP_BANNER\&context=at_home_menu}{Cook:
  Smoky Eggplant}
\item
  \href{https://www.nytimes3xbfgragh.onion/2020/07/27/travel/moose-michigan-isle-royale.html?action=click\&pgtype=Article\&state=default\&region=TOP_BANNER\&context=at_home_menu}{Look
  Out: For Moose}
\item
  \href{https://www.nytimes3xbfgragh.onion/interactive/2020/at-home/even-more-reporters-editors-diaries-lists-recommendations.html?action=click\&pgtype=Article\&state=default\&region=TOP_BANNER\&context=at_home_menu}{Explore:
  Reporters' Obsessions}
\end{itemize}

Advertisement

\protect\hyperlink{after-top}{Continue reading the main story}

Supported by

\protect\hyperlink{after-sponsor}{Continue reading the main story}

\href{/column/by-the-book}{By the Book}

\hypertarget{raina-telgemeier-cant-wait-to-break-bread-with-her-friends-again}{%
\section{Raina Telgemeier Can't Wait to Break Bread With Her Friends
Again}\label{raina-telgemeier-cant-wait-to-break-bread-with-her-friends-again}}

\includegraphics{https://static01.graylady3jvrrxbe.onion/images/2020/06/14/books/review/14ByTheBook/14ByTheBook-articleLarge.jpg?quality=75\&auto=webp\&disable=upscale}

June 11, 2020

\begin{itemize}
\item
\item
\item
\item
\item
\end{itemize}

\textbf{What books are on your nightstand?}

Fittingly for quarantine, I'm reading ``The Martian,'' by Andy Weir, and
Dylan Meconis's graphic novel ``Queen of the Sea,'' about a young girl
living with a convent of nuns on a remote island. There is a plague!
It's accidentally timely!

\textbf{What's the last great book you read?}

I won't stop yelling about Gene Luen Yang's graphic novel ``Dragon
Hoops.'' Gene is a master of the visual craft, weaving history and
sports culture and nerddom and meta-commentary together into a work
unlike anything else.

\textbf{Describe your ideal reading experience (when, where, what,
how).}

Sunny afternoon, couch, windows open, cats dozing at my feet, perpetual
mug of hot tea. A thought-provoking and lovingly-illustrated graphic
novel. Cover to cover.

\textbf{What's your favorite book no one else has heard of?}

One of my favorite illustrators is named Hergie (Ernie Hergenroeder). He
illustrated a series of nonfiction books about social emotional learning
written by Joy Wilt in the late 1970s that I was obsessed with --- and
in many ways were templates for the kinds of childhood emotions and
experiences I write about now. I don't believe I've ever heard Hergie's
name when my peers discuss their influences, but he had a huge impact on
the way I draw kids and environments and body language!

\textbf{Which writers --- novelists, playwrights, critics, journalists,
poets --- working today do you admire most?}

I'm obsessed with Esther Perel --- both her podcast and her published
work. I can't seem to read enough about therapy and human
interconnectedness, and she's got a deft touch with both.

Kiyohiko Azuma, the creator of my favorite manga, ``Yotsuba\&!,'' has
been writing an ongoing series about \ldots{} not much of anything, just
a kid and her dad and their neighbors, for over a decade. It's
slice-of-life and makes tons of space for character interaction, funny
throwaway moments and the enormous emotions of a little kid interacting
with the world. It's brilliant.

And enough good things can't possibly be said about Jason Reynolds. His
work is vital, and his advocacy and role modeling for kids is
awe-inspiring. He was the perfect choice for the role of National
Ambassador for Young People's Literature. We are so lucky to live in a
world where he writes books.

\textbf{What writers are especially good on adolescent life?}

I still look to Lynda Barry for guidance on how to perfectly nail the
teenage voice. She's got an uncanny ability to free-associate through
her characters, who are clearly still learning to speak and write and
use language, but are no less sophisticated in their depth of thought.

\textbf{Which young adult books would you recommend to people who don't
usually read Y.A.?}

Books by Mariko Tamaki and/or Rainbow Rowell. I don't think I've read a
bad work by either of them! ``Laura Dean Keeps Breaking Up With Me''
(illustrated by Rosemary Valero-O'Connell) and ``Pumpkinheads''
(illustrated by Faith Erin Hicks) are respective favorites.

\textbf{What book, if any, most influenced your decision to become a
graphic novelist?}

Keiji Nakazawa's ``Barefoot Gen,'' which was serialized in Japanese
comics magazines and then published as a series of graphic novels over
the course of my adolescence, were pivotal in teaching me the power of
the comics medium. The story centers on a family caught in the atomic
bombing of Hiroshima in 1945 --- a heartbreaking read, but one that
allowed the world to open up in my young mind and injected me with a
lifetime dose of empathy. I learned that chronicling the life stories of
real people, in good times and bad, is an integral piece in preserving
human history.

\textbf{What's the relationship between art and text in your mind?}

They make up a single language for me, which is \emph{cartooning}. I
have a hard time writing text and not peppering every sentence with
emojis and doodles in the margins. I have a hard time drawing a single
image that effectively communicates an idea. But put pictures and words
together, and suddenly I'm able to create stories.

\textbf{Which subjects do you wish more authors would write about?}

I love a good story about a character's inner world. John Green's
``Turtles All the Way Down'' broke me completely in half; to be able to
inhabit Aza's brain, and understand the color and texture and dialogue
of her anxiety, was powerfully affirming. We are all looking for lights
to guide us down the path of this confusing world, and when authors
allow you to walk in someone else's shoes, the path looks a bit
brighter.

\textbf{Which genres do you especially enjoy reading? And which do you
avoid?}

Memoir, nonfiction and realistic fiction are my favorites. I don't
outright avoid genres --- I'll read anything I get a high recommendation
for --- but I tend to shy away from fiction involving heavy
psychological abuse, especially when the antagonist isn't held
accountable for their actions. Enough of that in the real world!

\textbf{How do you organize your books?}

Reading demographic \textgreater{} color \textgreater{} size. I want to
be able to point my friends' kids in the right direction when they
visit, so I keep my picture books, middle grade comics and Y.A. comics
most easily accessible. A good 75 percent of my collection is comics and
illustrated books in various formats and sizes. I read most of my prose
digitally.

\textbf{The last book you read that made you cry?}

``Duck, Death and the Tulip,'' by Wolf Erlbruch. It's a picture book
about a duck and the specter of death, having a conversation. As they
do. My friend brought it over to my house and I read it on the spot, and
was completely caught off guard by how spare and sad and lovely it was.

\textbf{What kind of reader were you as a child? Which childhood books
and authors stick with you most?}

I glommed onto stories about real kids, fast. Ramona Quimby, the
Baby-Sitters Club, and all things Judy Blume were my mainstays. Another
early favorite was ``In the Year of the Boar and Jackie Robinson.'' My
mom loved to read whatever I was reading, and we made lots of time to
discuss.

\textbf{You're organizing a literary dinner party. Which three writers,
dead or alive, do you invite?}

Samin Nosrat, so we can cook (and laugh!) together. Bill Watterson, so
we can sit down while dinner's in the oven and look at old art books and
geek out about traditional inking tools. Toni Morrison, to tell haunting
stories in hushed tones over a long and delicious meal.

\textbf{What do you plan to read next?}

I've got ``Tartine Bread'' on back-order through Indiebound.com, and I
can't wait to get my hands on it. Quarantine is turning out to be about
upping my sourdough game, and I can't wait till this is over to have a
bread-eating party with my friends.

Advertisement

\protect\hyperlink{after-bottom}{Continue reading the main story}

\hypertarget{site-index}{%
\subsection{Site Index}\label{site-index}}

\hypertarget{site-information-navigation}{%
\subsection{Site Information
Navigation}\label{site-information-navigation}}

\begin{itemize}
\tightlist
\item
  \href{https://help.nytimes3xbfgragh.onion/hc/en-us/articles/115014792127-Copyright-notice}{©~2020~The
  New York Times Company}
\end{itemize}

\begin{itemize}
\tightlist
\item
  \href{https://www.nytco.com/}{NYTCo}
\item
  \href{https://help.nytimes3xbfgragh.onion/hc/en-us/articles/115015385887-Contact-Us}{Contact
  Us}
\item
  \href{https://www.nytco.com/careers/}{Work with us}
\item
  \href{https://nytmediakit.com/}{Advertise}
\item
  \href{http://www.tbrandstudio.com/}{T Brand Studio}
\item
  \href{https://www.nytimes3xbfgragh.onion/privacy/cookie-policy\#how-do-i-manage-trackers}{Your
  Ad Choices}
\item
  \href{https://www.nytimes3xbfgragh.onion/privacy}{Privacy}
\item
  \href{https://help.nytimes3xbfgragh.onion/hc/en-us/articles/115014893428-Terms-of-service}{Terms
  of Service}
\item
  \href{https://help.nytimes3xbfgragh.onion/hc/en-us/articles/115014893968-Terms-of-sale}{Terms
  of Sale}
\item
  \href{https://spiderbites.nytimes3xbfgragh.onion}{Site Map}
\item
  \href{https://help.nytimes3xbfgragh.onion/hc/en-us}{Help}
\item
  \href{https://www.nytimes3xbfgragh.onion/subscription?campaignId=37WXW}{Subscriptions}
\end{itemize}
