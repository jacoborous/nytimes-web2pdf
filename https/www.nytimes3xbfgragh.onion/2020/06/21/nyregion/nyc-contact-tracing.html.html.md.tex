Sections

SEARCH

\protect\hyperlink{site-content}{Skip to
content}\protect\hyperlink{site-index}{Skip to site index}

\href{https://www.nytimes3xbfgragh.onion/section/nyregion}{New York}

\href{https://myaccount.nytimes3xbfgragh.onion/auth/login?response_type=cookie\&client_id=vi}{}

\href{https://www.nytimes3xbfgragh.onion/section/todayspaper}{Today's
Paper}

\href{/section/nyregion}{New York}\textbar{}N.Y.C. Hired 3,000 Workers
for Contact Tracing. It's Off to a Slow Start.

\url{https://nyti.ms/2BnBCF5}

\begin{itemize}
\item
\item
\item
\item
\item
\end{itemize}

\href{https://www.nytimes3xbfgragh.onion/news-event/coronavirus?action=click\&pgtype=Article\&state=default\&region=TOP_BANNER\&context=storylines_menu}{The
Coronavirus Outbreak}

\begin{itemize}
\tightlist
\item
  live\href{https://www.nytimes3xbfgragh.onion/2020/08/04/world/coronavirus-cases.html?action=click\&pgtype=Article\&state=default\&region=TOP_BANNER\&context=storylines_menu}{Latest
  Updates}
\item
  \href{https://www.nytimes3xbfgragh.onion/interactive/2020/us/coronavirus-us-cases.html?action=click\&pgtype=Article\&state=default\&region=TOP_BANNER\&context=storylines_menu}{Maps
  and Cases}
\item
  \href{https://www.nytimes3xbfgragh.onion/interactive/2020/science/coronavirus-vaccine-tracker.html?action=click\&pgtype=Article\&state=default\&region=TOP_BANNER\&context=storylines_menu}{Vaccine
  Tracker}
\item
  \href{https://www.nytimes3xbfgragh.onion/2020/08/02/us/covid-college-reopening.html?action=click\&pgtype=Article\&state=default\&region=TOP_BANNER\&context=storylines_menu}{College
  Reopening}
\item
  \href{https://www.nytimes3xbfgragh.onion/live/2020/08/04/business/stock-market-today-coronavirus?action=click\&pgtype=Article\&state=default\&region=TOP_BANNER\&context=storylines_menu}{Economy}
\end{itemize}

Advertisement

\protect\hyperlink{after-top}{Continue reading the main story}

Supported by

\protect\hyperlink{after-sponsor}{Continue reading the main story}

\hypertarget{nyc-hired-3000-workers-for-contact-tracing-its-off-to-a-slow-start}{%
\section{N.Y.C. Hired 3,000 Workers for Contact Tracing. It's Off to a
Slow
Start.}\label{nyc-hired-3000-workers-for-contact-tracing-its-off-to-a-slow-start}}

The program is crucial to the next phase of reopening, which begins on
Monday. But workers have not had much success in getting information
from people who test positive.

\includegraphics{https://static01.graylady3jvrrxbe.onion/images/2020/06/19/nyregion/00nyvirus-contacttracing-1/merlin_173678748_cf1de324-7e94-4f0e-9edd-0f05293c891e-articleLarge.jpg?quality=75\&auto=webp\&disable=upscale}

\href{https://www.nytimes3xbfgragh.onion/by/sharon-otterman}{\includegraphics{https://static01.graylady3jvrrxbe.onion/images/2018/06/14/multimedia/author-sharon-otterman/author-sharon-otterman-thumbLarge.png}}

By \href{https://www.nytimes3xbfgragh.onion/by/sharon-otterman}{Sharon
Otterman}

\begin{itemize}
\item
  June 21, 2020
\item
  \begin{itemize}
  \item
  \item
  \item
  \item
  \item
  \end{itemize}
\end{itemize}

New York City's ambitious contact-tracing program, a crucial initiative
in the effort to curb the coronavirus, has gotten off to a worrisome
start just as the city's
\href{https://www.nytimes3xbfgragh.onion/2020/06/18/nyregion/phase-2-reopening-nyc.html}{reopening
enters a new phase on Monday}, with outdoor dining, in-store shopping
and office work resuming.

The city has hired 3,000 disease detectives and case monitors, who are
supposed to identify anyone who has come into contact with the hundreds
of people who are still testing positive for the virus in the city every
day. But the first statistics from the program, which began on June 1,
indicate that tracers are often unable to locate infected people or
gather information from them.

Only 35 percent of the 5,347 city residents who tested positive or were
presumed positive for the coronavirus in the program's first two weeks
gave information about close contacts to tracers, the city said in
releasing the first statistics. The number ticked up slightly, to 42
percent, during the third week, Avery Cohen, a spokeswoman to Mayor Bill
de Blasio, said on Sunday.

Contact tracing is one of the few tools that public health officials
have to fight Covid-19 in lieu of a vaccine, along with widespread
testing and isolation of those exposed to the coronavirus. The early
results of New York's program raise fresh concerns about the
difficulties in preventing a surge of new cases as states across the
country reopen.

The city has successfully done contact tracing before, with diseases
like tuberculosis and measles. But as with much involving the
coronavirus outbreak, officials have never faced the challenge at this
scale, with so many cases across the five boroughs.

The city's program has so far been limited by a low response rate, scant
use of technology, privacy concerns and a far less sweeping mandate than
that in some other countries, where apartment buildings, stores,
restaurants and other private businesses are often required to collect
visitors' personal information, which makes tracking the spread easier.

China, South Korea and Germany and other countries have set up extensive
tracking programs that have helped officials make major strides in
reducing the outbreak. In South Korea, for example,
\href{https://www.nytimes3xbfgragh.onion/2020/05/09/world/asia/coronavirus-south-korea-second-wave.html}{people
at weddings, funerals, karaoke bars, nightclubs and internet-game
parlors write down} their names and telephone numbers, and the
authorities have been able to draw on cellphone location data, credit
card transactions and even closed-circuit video footage to identify and
isolate potential contacts.

Dr. Ted Long, head of New York City's new
\href{https://www.nychealthandhospitals.org/test-and-trace/}{Test and
Trace Corps}, insisted that the program was going well, but acknowledged
that many people who tested positive had failed to provide information
over the phone to the contact tracers, or left interviews before being
asked. Others told the tracers they had been only at home and had not
put others at risk, and then did not name family members.

Dr. Long said one encouraging sign was that nearly all the people for
whom the city had numbers at least answered the phone. He added that he
believed that the tracers would be more successful when they start going
to people's homes in the next week or two, rather than just relying on
communication over the phone.

``I do think that the program, especially because it is only two weeks
old, is doing an outstanding job,'' he said.

\includegraphics{https://static01.graylady3jvrrxbe.onion/images/2020/06/19/nyregion/00nyvirus-contacttracing-2/merlin_173678676_4e30692c-cae9-40af-ab9b-b012a7f90551-articleLarge.jpg?quality=75\&auto=webp\&disable=upscale}

The city has made major strides in reducing the outbreak since the
shutdown began in March,
\href{https://www.nytimes3xbfgragh.onion/interactive/2020/nyregion/new-york-city-coronavirus-cases.html\#cases}{with
only 327 new cases reported on Thursday}, down from several thousand
cases a day during the peak. But Phase 2 of the reopening on Monday
presents new risks, with 300,000 people likely returning to their jobs.

Perry N. Halkitis, dean of the School of Public Health at Rutgers
University, which is guiding an effort to bring on thousands of tracers
in New Jersey, called New York City's 35 percent rate for eliciting
contacts ``very bad.''

\hypertarget{latest-updates-global-coronavirus-outbreak}{%
\section{\texorpdfstring{\href{https://www.nytimes3xbfgragh.onion/2020/08/04/world/coronavirus-cases.html?action=click\&pgtype=Article\&state=default\&region=MAIN_CONTENT_1\&context=storylines_live_updates}{Latest
Updates: Global Coronavirus
Outbreak}}{Latest Updates: Global Coronavirus Outbreak}}\label{latest-updates-global-coronavirus-outbreak}}

Updated 2020-08-04T21:41:55.934Z

\begin{itemize}
\tightlist
\item
  \href{https://www.nytimes3xbfgragh.onion/2020/08/04/world/coronavirus-cases.html?action=click\&pgtype=Article\&state=default\&region=MAIN_CONTENT_1\&context=storylines_live_updates\#link-2daa96b5}{As
  talks drag on, McConnell signals openness to jobless aid extension
  that Republicans have opposed.}
\item
  \href{https://www.nytimes3xbfgragh.onion/2020/08/04/world/coronavirus-cases.html?action=click\&pgtype=Article\&state=default\&region=MAIN_CONTENT_1\&context=storylines_live_updates\#link-1228a480}{Novavax
  sees encouraging results from two studies of its experimental
  vaccine.}
\item
  \href{https://www.nytimes3xbfgragh.onion/2020/08/04/world/coronavirus-cases.html?action=click\&pgtype=Article\&state=default\&region=MAIN_CONTENT_1\&context=storylines_live_updates\#link-4825b93}{Public
  and private schools in Maryland and elsewhere are divided over
  in-person instruction.}
\end{itemize}

\href{https://www.nytimes3xbfgragh.onion/2020/08/04/world/coronavirus-cases.html?action=click\&pgtype=Article\&state=default\&region=MAIN_CONTENT_1\&context=storylines_live_updates}{See
more updates}

More live coverage:
\href{https://www.nytimes3xbfgragh.onion/live/2020/08/04/business/stock-market-today-coronavirus?action=click\&pgtype=Article\&state=default\&region=MAIN_CONTENT_1\&context=storylines_live_updates}{Markets}

``For each person, you should be in touch with
\href{https://covidlocal.org/assets/documents/COVID\%20Local\%20Metrics\%20overview.pdf}{75
percent of their contacts} within a day,'' he said.

He suggested that the poor showing stemmed in part from the inexperience
of the contact tracers and insufficient hands-on training.

``This is a skill,'' he said. ``You need to practice.''

Across the world, the authorities have rushed to set up contact-tracing
programs, hiring hundreds of thousands of people, including many without
experience doing such work. While the goal is to reach all of a sick
person's contacts, and get them to effectively quarantine for two weeks,
the reality is often much messier.

In Massachusetts, which has one of the most established tracing programs
in the country, health officials said in May that
only\href{https://www.boston25news.com/news/health/more-than-60-percent-mass-contact-tracing-calls-answered/4NK2EEGVEVCDFBMICEKZXFCXR4/}{about
60 percent} of infected patients were picking up the phone. In
\href{https://www.governing.com/now/Louisiana-Struggles-to-Get-Contact-Tracer-Calls-Answered.html}{Louisiana},
less than half were answering. In England, the program has struggled to
\href{https://www.nytimes3xbfgragh.onion/2020/06/17/world/europe/uk-contact-tracing-coronavirus.html}{show
results with a low-paid, inexperienced work force}.

An\href{https://www.wsj.com/articles/coronavirus-contact-tracing-apps-launch-across-europe-amid-hopes-for-broad-adoption-11592319612}{increasing
number of countries} are using phone applications to help track and
trace people who test positive. Several states in the United States,
including North Dakota, that have tried using digital applications have
\href{https://www.washingtonpost.com/technology/2020/05/21/care19-dakota-privacy-coronavirus/}{run
into privacy issues}.

But in New York, as in most of the country, contact tracers are
typically using only low-tech tools like phone calls and a
questionnaire, in part to allay privacy concerns.

The tracers are seeking the names and phone numbers of each person a
confirmed-positive patient has been in close contact with from a few
days before the onset of symptoms, defined as within six feet for at
least 15 minutes. Each contact is then called, told that he or she may
have been exposed to the virus, and asked to quarantine.

The relative silence from virus patients in New York City is one of
several issues troubling the contact-tracing program.

Mr. de Blasio, who has had tense relations with senior officials in his
own Department of Health, stripped the department of oversight for the
program in May, moving it
\href{https://www.nytimes3xbfgragh.onion/2020/05/07/nyregion/coronavirus-contact-tracing-nyc.html}{under
the umbrella of the city's public hospitals agency}. That has led to
concerns among some former health officials that expertise would be lost
in the process.

Dr. Long said 50 experts from the Department of Health --- the city's
contact tracers before Covid-19, who have handled epidemics such as
measles and Ebola --- are guiding the work of the tracing corps, but are
not tracing themselves.

Dr. Long is a primary care physician and vice president of ambulatory
care at the public hospitals corporation. The Health Department's
tracing effort was led by
\href{https://www.publichealth.columbia.edu/people/our-faculty/sa3217}{epidemiologists}.

``I challenge anyone to show me how we are not collaborating,'' he said
of the relationship between the two agencies. ``They have been nothing
short of partners.''

The city has had more success with its testing program, which is ahead
of schedule, with a target of 50,000 tests per day expected to be
reached in July, instead of August, officials said.

But an initiative to set aside hotel rooms for people who have tested
positive to isolate from families is not popular. Though the city rented
1,200 hotel rooms for free use by virus patients, only 60 to 80 rooms
have been occupied in recent weeks, city officials said. And in the two
and a half weeks since tracing began, only 40 patients have requested
rooms through the tracing program, Dr. Long said.

Image

Tents set up for coronavirus testing at Elmhurst Hospital Center in
Queens.~The city has made major strides in reducing the
outbreak.Credit...Juan Arredondo for The New York Times

Over 1,000 virus patients have instead asked for support to isolate at
home, such as assistance with grocery and medicine deliveries, because
they preferred to remain with their families, he said.

In an effort to build a connection between contacts and tracers, half of
all tracers hired live in communities hard-hit by the virus, which are
predominately black and Hispanic, Dr. Long said.

\href{https://www.nytimes3xbfgragh.onion/news-event/coronavirus?action=click\&pgtype=Article\&state=default\&region=MAIN_CONTENT_3\&context=storylines_faq}{}

\hypertarget{the-coronavirus-outbreak-}{%
\subsubsection{The Coronavirus Outbreak
›}\label{the-coronavirus-outbreak-}}

\hypertarget{frequently-asked-questions}{%
\paragraph{Frequently Asked
Questions}\label{frequently-asked-questions}}

Updated August 4, 2020

\begin{itemize}
\item ~
  \hypertarget{i-have-antibodies-am-i-now-immune}{%
  \paragraph{I have antibodies. Am I now
  immune?}\label{i-have-antibodies-am-i-now-immune}}

  \begin{itemize}
  \tightlist
  \item
    As of right
    now,\href{https://www.nytimes3xbfgragh.onion/2020/07/22/health/covid-antibodies-herd-immunity.html?action=click\&pgtype=Article\&state=default\&region=MAIN_CONTENT_3\&context=storylines_faq}{that
    seems likely, for at least several months.} There have been
    frightening accounts of people suffering what seems to be a second
    bout of Covid-19. But experts say these patients may have a
    drawn-out course of infection, with the virus taking a slow toll
    weeks to months after initial exposure. People infected with the
    coronavirus typically
    \href{https://www.nature.com/articles/s41586-020-2456-9}{produce}
    immune molecules called antibodies, which are
    \href{https://www.nytimes3xbfgragh.onion/2020/05/07/health/coronavirus-antibody-prevalence.html?action=click\&pgtype=Article\&state=default\&region=MAIN_CONTENT_3\&context=storylines_faq}{protective
    proteins made in response to an
    infection}\href{https://www.nytimes3xbfgragh.onion/2020/05/07/health/coronavirus-antibody-prevalence.html?action=click\&pgtype=Article\&state=default\&region=MAIN_CONTENT_3\&context=storylines_faq}{.
    These antibodies may} last in the body
    \href{https://www.nature.com/articles/s41591-020-0965-6}{only two to
    three months}, which may seem worrisome, but that's perfectly normal
    after an acute infection subsides, said Dr. Michael Mina, an
    immunologist at Harvard University. It may be possible to get the
    coronavirus again, but it's highly unlikely that it would be
    possible in a short window of time from initial infection or make
    people sicker the second time.
  \end{itemize}
\item ~
  \hypertarget{im-a-small-business-owner-can-i-get-relief}{%
  \paragraph{I'm a small-business owner. Can I get
  relief?}\label{im-a-small-business-owner-can-i-get-relief}}

  \begin{itemize}
  \tightlist
  \item
    The
    \href{https://www.nytimes3xbfgragh.onion/article/small-business-loans-stimulus-grants-freelancers-coronavirus.html?action=click\&pgtype=Article\&state=default\&region=MAIN_CONTENT_3\&context=storylines_faq}{stimulus
    bills enacted in March} offer help for the millions of American
    small businesses. Those eligible for aid are businesses and
    nonprofit organizations with fewer than 500 workers, including sole
    proprietorships, independent contractors and freelancers. Some
    larger companies in some industries are also eligible. The help
    being offered, which is being managed by the Small Business
    Administration, includes the Paycheck Protection Program and the
    Economic Injury Disaster Loan program. But lots of folks have
    \href{https://www.nytimes3xbfgragh.onion/interactive/2020/05/07/business/small-business-loans-coronavirus.html?action=click\&pgtype=Article\&state=default\&region=MAIN_CONTENT_3\&context=storylines_faq}{not
    yet seen payouts.} Even those who have received help are confused:
    The rules are draconian, and some are stuck sitting on
    \href{https://www.nytimes3xbfgragh.onion/2020/05/02/business/economy/loans-coronavirus-small-business.html?action=click\&pgtype=Article\&state=default\&region=MAIN_CONTENT_3\&context=storylines_faq}{money
    they don't know how to use.} Many small-business owners are getting
    less than they expected or
    \href{https://www.nytimes3xbfgragh.onion/2020/06/10/business/Small-business-loans-ppp.html?action=click\&pgtype=Article\&state=default\&region=MAIN_CONTENT_3\&context=storylines_faq}{not
    hearing anything at all.}
  \end{itemize}
\item ~
  \hypertarget{what-are-my-rights-if-i-am-worried-about-going-back-to-work}{%
  \paragraph{What are my rights if I am worried about going back to
  work?}\label{what-are-my-rights-if-i-am-worried-about-going-back-to-work}}

  \begin{itemize}
  \tightlist
  \item
    Employers have to provide
    \href{https://www.osha.gov/SLTC/covid-19/standards.html}{a safe
    workplace} with policies that protect everyone equally.
    \href{https://www.nytimes3xbfgragh.onion/article/coronavirus-money-unemployment.html?action=click\&pgtype=Article\&state=default\&region=MAIN_CONTENT_3\&context=storylines_faq}{And
    if one of your co-workers tests positive for the coronavirus, the
    C.D.C.} has said that
    \href{https://www.cdc.gov/coronavirus/2019-ncov/community/guidance-business-response.html}{employers
    should tell their employees} -\/- without giving you the sick
    employee's name -\/- that they may have been exposed to the virus.
  \end{itemize}
\item ~
  \hypertarget{should-i-refinance-my-mortgage}{%
  \paragraph{Should I refinance my
  mortgage?}\label{should-i-refinance-my-mortgage}}

  \begin{itemize}
  \tightlist
  \item
    \href{https://www.nytimes3xbfgragh.onion/article/coronavirus-money-unemployment.html?action=click\&pgtype=Article\&state=default\&region=MAIN_CONTENT_3\&context=storylines_faq}{It
    could be a good idea,} because mortgage rates have
    \href{https://www.nytimes3xbfgragh.onion/2020/07/16/business/mortgage-rates-below-3-percent.html?action=click\&pgtype=Article\&state=default\&region=MAIN_CONTENT_3\&context=storylines_faq}{never
    been lower.} Refinancing requests have pushed mortgage applications
    to some of the highest levels since 2008, so be prepared to get in
    line. But defaults are also up, so if you're thinking about buying a
    home, be aware that some lenders have tightened their standards.
  \end{itemize}
\item ~
  \hypertarget{what-is-school-going-to-look-like-in-september}{%
  \paragraph{What is school going to look like in
  September?}\label{what-is-school-going-to-look-like-in-september}}

  \begin{itemize}
  \tightlist
  \item
    It is unlikely that many schools will return to a normal schedule
    this fall, requiring the grind of
    \href{https://www.nytimes3xbfgragh.onion/2020/06/05/us/coronavirus-education-lost-learning.html?action=click\&pgtype=Article\&state=default\&region=MAIN_CONTENT_3\&context=storylines_faq}{online
    learning},
    \href{https://www.nytimes3xbfgragh.onion/2020/05/29/us/coronavirus-child-care-centers.html?action=click\&pgtype=Article\&state=default\&region=MAIN_CONTENT_3\&context=storylines_faq}{makeshift
    child care} and
    \href{https://www.nytimes3xbfgragh.onion/2020/06/03/business/economy/coronavirus-working-women.html?action=click\&pgtype=Article\&state=default\&region=MAIN_CONTENT_3\&context=storylines_faq}{stunted
    workdays} to continue. California's two largest public school
    districts --- Los Angeles and San Diego --- said on July 13, that
    \href{https://www.nytimes3xbfgragh.onion/2020/07/13/us/lausd-san-diego-school-reopening.html?action=click\&pgtype=Article\&state=default\&region=MAIN_CONTENT_3\&context=storylines_faq}{instruction
    will be remote-only in the fall}, citing concerns that surging
    coronavirus infections in their areas pose too dire a risk for
    students and teachers. Together, the two districts enroll some
    825,000 students. They are the largest in the country so far to
    abandon plans for even a partial physical return to classrooms when
    they reopen in August. For other districts, the solution won't be an
    all-or-nothing approach.
    \href{https://bioethics.jhu.edu/research-and-outreach/projects/eschool-initiative/school-policy-tracker/}{Many
    systems}, including the nation's largest, New York City, are
    devising
    \href{https://www.nytimes3xbfgragh.onion/2020/06/26/us/coronavirus-schools-reopen-fall.html?action=click\&pgtype=Article\&state=default\&region=MAIN_CONTENT_3\&context=storylines_faq}{hybrid
    plans} that involve spending some days in classrooms and other days
    online. There's no national policy on this yet, so check with your
    municipal school system regularly to see what is happening in your
    community.
  \end{itemize}
\end{itemize}

Sivanthy Vasanthan, 23, who just graduated from Columbia University's
Mailman School of Public Health, said recruiters reached out to her
based on her
\href{https://www.linkedin.com/in/sivanthyvasanthan/}{LinkedIn} profile,
which emphasizes her interest in public health and human rights.

After about two weeks of training, Ms. Vasanthan, who lives in
Manhattan's Washington Heights neighborhood, began calling positive
patients just over a week ago. ``Most of the people who I have talked to
have already been aware of their test results and have been at home,''
she said.

The city gave no metrics for whether it was successfully persuading
those contacted to get tested or to quarantine.

Experts said that while tracing in the city was not where it should be,
the program was clearly beneficial and should push forward.

``It's tough to look at these numbers and say it's a roaring success,''
said Dr. Crystal Watson, an expert on contact tracing at the Johns
Hopkins Bloomberg School of Public Health. ``But I do think it is a
beginning and it will build on itself.''

Dr. Halkitis at Rutgers said he thought the low cooperation rate was
likely due to several factors, including the inexperience of the
tracers; widespread reluctance among Americans to share personal
information with the government; and Mayor de Blasio's decision to shift
the program away from the city's Department of Health.

``You have taken it away from the people who actually know how to do
it,'' he said. ``The D.O.H. people, they are skilled. They know this
stuff.''

On Tuesday, the city laid out strategies to close the gap in tracing.
For the 15 percent of positive cases that have come in without an
accurate phone number, Dr. Long said, tracers have begun reaching out to
doctor's offices and doing database research to get that information.

And for people who have tested positive and are unresponsive to phone
calls, field workers like Daniel Okpare, a public health student in East
Harlem, will soon try to interview them in person.

Mr. Okpare, 30, is still in training, but has been told he will mostly
be visiting patients in Harlem, near where he lives. He said he hoped
\href{https://www.linkedin.com/in/domph/}{his background} as a former
podiatry student who is enrolled in New York University's School of
Global Public Health, as well as his being an immigrant from Nigeria,
would help put people at ease.

Wearing personal protective equipment, and carrying a city-issued iPad
and a cellphone, he will be working alone while knocking on doors.

``It's an opportunity to be part of the front line of response as a
public health professional,'' he said. ``To have eye contact with
someone to say, `Yes you have Covid, but we are going to find every way
possible that you will be safe.'''

Anne Barnard contributed reporting.

Advertisement

\protect\hyperlink{after-bottom}{Continue reading the main story}

\hypertarget{site-index}{%
\subsection{Site Index}\label{site-index}}

\hypertarget{site-information-navigation}{%
\subsection{Site Information
Navigation}\label{site-information-navigation}}

\begin{itemize}
\tightlist
\item
  \href{https://help.nytimes3xbfgragh.onion/hc/en-us/articles/115014792127-Copyright-notice}{©~2020~The
  New York Times Company}
\end{itemize}

\begin{itemize}
\tightlist
\item
  \href{https://www.nytco.com/}{NYTCo}
\item
  \href{https://help.nytimes3xbfgragh.onion/hc/en-us/articles/115015385887-Contact-Us}{Contact
  Us}
\item
  \href{https://www.nytco.com/careers/}{Work with us}
\item
  \href{https://nytmediakit.com/}{Advertise}
\item
  \href{http://www.tbrandstudio.com/}{T Brand Studio}
\item
  \href{https://www.nytimes3xbfgragh.onion/privacy/cookie-policy\#how-do-i-manage-trackers}{Your
  Ad Choices}
\item
  \href{https://www.nytimes3xbfgragh.onion/privacy}{Privacy}
\item
  \href{https://help.nytimes3xbfgragh.onion/hc/en-us/articles/115014893428-Terms-of-service}{Terms
  of Service}
\item
  \href{https://help.nytimes3xbfgragh.onion/hc/en-us/articles/115014893968-Terms-of-sale}{Terms
  of Sale}
\item
  \href{https://spiderbites.nytimes3xbfgragh.onion}{Site Map}
\item
  \href{https://help.nytimes3xbfgragh.onion/hc/en-us}{Help}
\item
  \href{https://www.nytimes3xbfgragh.onion/subscription?campaignId=37WXW}{Subscriptions}
\end{itemize}
