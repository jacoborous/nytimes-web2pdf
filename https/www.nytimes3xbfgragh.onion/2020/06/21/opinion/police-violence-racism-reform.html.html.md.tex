Sections

SEARCH

\protect\hyperlink{site-content}{Skip to
content}\protect\hyperlink{site-index}{Skip to site index}

\href{https://myaccount.nytimes3xbfgragh.onion/auth/login?response_type=cookie\&client_id=vi}{}

\href{https://www.nytimes3xbfgragh.onion/section/todayspaper}{Today's
Paper}

\href{/section/opinion}{Opinion}\textbar{}Policing Is Doing What It Was
Meant to Do. That's the Problem.

\url{https://nyti.ms/37Mu2zG}

\begin{itemize}
\item
\item
\item
\item
\item
\end{itemize}

Advertisement

\protect\hyperlink{after-top}{Continue reading the main story}

\href{/section/opinion}{Opinion}

Supported by

\protect\hyperlink{after-sponsor}{Continue reading the main story}

THE STONE

\hypertarget{policing-is-doing-what-it-was-meant-to-do-thats-the-problem}{%
\section{Policing Is Doing What It Was Meant to Do. That's the
Problem.}\label{policing-is-doing-what-it-was-meant-to-do-thats-the-problem}}

Blaming racist violence on ``bad apples'' misses the point.

By Todd May and George Yancy

Mr. May and Mr. Yancy are philosophers.

\begin{itemize}
\item
  June 21, 2020
\item
  \begin{itemize}
  \item
  \item
  \item
  \item
  \item
  \end{itemize}
\end{itemize}

\includegraphics{https://static01.graylady3jvrrxbe.onion/images/2020/06/21/opinion/21stoneWeb/merlin_173164512_f51c4772-7c5f-4d94-a591-246adb217617-articleLarge.jpg?quality=75\&auto=webp\&disable=upscale}

On June 6, one of us attended a memorial vigil for George Floyd. The
opening speaker first thanked the local Police Department for keeping
the vigil safe and then went on to distinguish between the majority of
police officers who do their job helping and protecting people and the
few who are racist and violent.

His remarks echoed those made by Barack Obama on May 29, in his public
statement on the killing of Mr. Floyd, when he wrote of ``the majority
of men and women in law enforcement who take pride in doing their tough
job the right way, every day.''

We think that making this distinction is a mistake. It is a mistake not
because it underestimates the number of police officers who are racist
and violent, but because the problem of racist policing is not one of
individual actors. It is a mistake because the role of the police in
society must be understood, not individually but structurally.

Like an organ in a human body, a Police Department is part of a
structural whole. It functions to perform a certain task in the body
politic; it is an organ in that body. Seen this way, each police officer
is then like a cell in that organ. Before we can identify any problem in
that organ, we must first understand the job that organ performs.

In the case of the police, the answer might seem obvious. Their function
is to protect the citizenry from crime. At least that's what we're told.
But as any good student of biology or politics knows, it won't help to
ask what an organ is said to do. It is better to observe what it
actually does.

To merely accept the claim that police forces, since their inception,
have protected law-abiding citizens from crime involves the neglect of
several crucial factors. It neglects the long history of police abuse
and the specific intentional abuse of people of color; it neglects the
role that the police have played in breaking strikes, in silencing
dissent and in keeping the social order safe from resistance or change.
It also neglects the early history of policing in the United States that
took the form of
\href{https://lawenforcementmuseum.org/2019/07/10/slave-patrols-an-early-form-of-american-policing/}{slave
patrols} in the 1700s and the enforcement of
\href{https://www.nationalgeographic.org/encyclopedia/black-codes-and-jim-crow-laws/}{Black
Codes and Jim Crow} laws in the 19th and 20th centuries.

In his influential work on prisons, the philosopher and historian Michel
Foucault pointed out the following: We say that the prisons fail at
their task, yet we keep them going. Perhaps we should be asking not why
the prison fails but instead what it actually succeeds at.

That is the question we should be asking of the police. Not why do they
regularly fail to perform their duties correctly and thus need reform,
but rather, what duties are they succeeding at?

Once we ask that question, the answer is entirely clear. They succeed in
keeping people in their place. They succeed in keeping middle-class and
especially upper-class white people safe, so long as they don't get out
of line. They succeed in keeping people of color in their place so that
they don't challenge the social order that privileges middle- and
upper-class white people. And, as we have recently witnessed in many
violent police responses at protests, they succeed in suppressing those
who would question the social order.

If we look at individual police officers divorced from the structure in
which they operate --- if we simply look for the ``bad apples''--- we
fail to see the role of the police as a whole. Whether individual police
officers are racist is not the fundamental issue. The fundamental issue
is whether the police --- the institution of policing as it exists in
the United States --- is racist. And once we look at this clearly, we
understand that the answer must be yes.

As we were thinking about the problems with the ``bad apple'' metaphor
in policing, one of us, on June 13 at 2:46 a.m., received this message:
``Go to HELL, nigger!'' It is one of hundreds of such messages and
threats the author has received in the past several years. It is easy to
say that this individual white person (and we think it fair to assume
that it was a white person) is a racist, a ``bad apple.'' But here, too,
focusing on the individual white person who sent the racist message
obscures our understanding of the white supremacist structure in which
it is generated.

In 2015, during his last year in office, President Barack Obama
addressed the relationship between individual acts of racism and the
larger system of injustice that allows them on an episode of the podcast
``WTF With Marc Maron.'' `` \ldots{} it's not just a matter of it not
being polite,'' he said, to utter the N-word in public. ``That's not the
measure of whether racism still exists or not.'' Earlier in the show, he
observed that racism had not been ``cured'' --- the word for eliminating
a disease that systemically impacts the body --- and that ``the legacy
of slavery, Jim Crow, discrimination'' was ``still part of our DNA'' as
Americans. This slur is also part of our DNA, embedded within the
concept of a ``master race'' and the resulting white-supremacist
violence against black bodies.

In his ``Letter From Birmingham Jail'' in 1963, Martin Luther King Jr.
articulated the horror and pain felt ``when your first name becomes
`nigger' and your middle name becomes `boy' (however old you are).''
W.E.B. Du Bois, in a speech that he delivered in Beijing (then Peking)
at the age of 91, said, ``In my own country for nearly a century I have
been nothing but a `nigger.'''

Both men emphasized how the word is part of the institutional fabric of
black oppression, that individual racist acts are not aberrations but
the products of a larger systemic set of practices that, as the feminist
scholar Barbara Applebaum argues, ``hold structural injustice in
place.'' Central to those practices is policing, and the ``bad apple''
framing fails to confront its role in structural injustice.

One obvious objection to our view here is that by focusing on reforming
or dismantling an entire system, we may end up punishing individual
officers who have not committed racist acts and so bear no
responsibility for them. We acknowledge that this can be a difficult
idea to embrace. Many of us personally know police officers --- family
members, friends, neighborhood officers --- whom we know to be ethical
people; imagining them as responsible for a racist system is a hard leap
to make. We think the influential feminist philosopher Iris Marion Young
argued persuasively against relying on that distinction when she wrote:

\begin{quote}
Structural injustice occurs as a consequence of many individual and
institutions acting in pursuit of their particular goals and interests,
within given institutional rules and accepted norms. All the persons who
participated by their actions in the ongoing schemes of cooperation that
constitute these structures are responsible for them, in the sense that
they are part of the process that causes them. They are not responsible,
however, in the sense of having directed the process or intended its
outcomes.
\end{quote}

Many others have amplified that view. The critical race theorist and
legal scholar Charles Lawrence argued in his 1987 article ``The Id, the
Ego, and Equal Protection: Reckoning With Unconscious Racism'' that the
bad-apple metaphor suggests a ``perpetrator'' model that fails to give
an account of just how systemic racism is ``transmitted by tacit
understandings'' and ``collective unconscious.'' The philosopher Charles
Mills argues, ``the perpetrator {[}of racist actions or beliefs{]}
perspective presupposes a world composed of atomic individuals whose
actions are outside of and apart from the social fabric and without
historical continuity.'' When it comes to racism and policing, we argue
that the bad-apple metaphor places ``bad police officers'' outside of
the social and historical fabric of racism and institutional policing
that affects all of the apples. In fact, in this case, the tree itself
is rotten.

In his book ``The Tears We Cannot Stop: A Sermon to White America,''
Michael Eric Dyson argues:

\begin{quote}
That metaphor of a few bad apples doesn't begin to get at the root of
the problem. Police violence may be more like a poisoned water stream
that pollutes the entire system. To argue that only a few bad cops cause
police terror is like relegating racism to a few bigots. Bigots are
surely a problem, but they are sustained by systems of belief and
perception, by widely held stereotypes and social practice.
\end{quote}

To truly confront problems of racist violence in our society, let's not
once again begin with the question of how to reform the police. Let's
instead start with the question of how to build healthy and safe
communities of mutual respect and see which institutions we need to
reach that goal. If anything that is to be called policing emerges from
that inquiry, it should be at its end rather than assumed at the outset.

Todd May is the author of, most recently,
``\href{https://press.uchicago.edu/ucp/books/book/chicago/D/bo34250692.html}{A
Decent Life:} Morality for the Rest of Us.'' George Yancy is a professor
of philosophy at Emory University. His latest book is
``\href{https://rowman.com/ISBN/9781538131619/Across-Black-Spaces-Essays-and-Interviews-from-an-American-Philosopher}{Across
Black Spaces:} Essays and Interviews From an American Philosopher.''

\emph{\textbf{Now in print:}}
\emph{``}\href{http://bitly.com/1MW2kN3}{\emph{Modern Ethics in 77
Arguments}}\emph{'' and ``}\href{http://bitly.com/1MW2kN3}{\emph{The
Stone Reader: Modern Philosophy in 133 Arguments}}\emph{,'' with essays
from the series, edited by Peter Catapano and Simon Critchley, published
by Liveright Books.}

\emph{The Times is committed to publishing}
\href{https://www.nytimes3xbfgragh.onion/2019/01/31/opinion/letters/letters-to-editor-new-york-times-women.html}{\emph{a
diversity of letters}} \emph{to the editor. We'd like to hear what you
think about this or any of our articles. Here are some}
\href{https://help.nytimes3xbfgragh.onion/hc/en-us/articles/115014925288-How-to-submit-a-letter-to-the-editor}{\emph{tips}}\emph{.
And here's our email:}
\href{mailto:letters@NYTimes.com}{\emph{letters@NYTimes.com}}\emph{.}

\emph{Follow The New York Times Opinion section on}
\href{https://www.facebookcorewwwi.onion/nytopinion}{\emph{Facebook}}\emph{,}
\href{http://twitter.com/NYTOpinion}{\emph{Twitter (@NYTopinion)}}
\emph{and}
\href{https://www.instagram.com/nytopinion/}{\emph{Instagram}}\emph{.}

Advertisement

\protect\hyperlink{after-bottom}{Continue reading the main story}

\hypertarget{site-index}{%
\subsection{Site Index}\label{site-index}}

\hypertarget{site-information-navigation}{%
\subsection{Site Information
Navigation}\label{site-information-navigation}}

\begin{itemize}
\tightlist
\item
  \href{https://help.nytimes3xbfgragh.onion/hc/en-us/articles/115014792127-Copyright-notice}{©~2020~The
  New York Times Company}
\end{itemize}

\begin{itemize}
\tightlist
\item
  \href{https://www.nytco.com/}{NYTCo}
\item
  \href{https://help.nytimes3xbfgragh.onion/hc/en-us/articles/115015385887-Contact-Us}{Contact
  Us}
\item
  \href{https://www.nytco.com/careers/}{Work with us}
\item
  \href{https://nytmediakit.com/}{Advertise}
\item
  \href{http://www.tbrandstudio.com/}{T Brand Studio}
\item
  \href{https://www.nytimes3xbfgragh.onion/privacy/cookie-policy\#how-do-i-manage-trackers}{Your
  Ad Choices}
\item
  \href{https://www.nytimes3xbfgragh.onion/privacy}{Privacy}
\item
  \href{https://help.nytimes3xbfgragh.onion/hc/en-us/articles/115014893428-Terms-of-service}{Terms
  of Service}
\item
  \href{https://help.nytimes3xbfgragh.onion/hc/en-us/articles/115014893968-Terms-of-sale}{Terms
  of Sale}
\item
  \href{https://spiderbites.nytimes3xbfgragh.onion}{Site Map}
\item
  \href{https://help.nytimes3xbfgragh.onion/hc/en-us}{Help}
\item
  \href{https://www.nytimes3xbfgragh.onion/subscription?campaignId=37WXW}{Subscriptions}
\end{itemize}
