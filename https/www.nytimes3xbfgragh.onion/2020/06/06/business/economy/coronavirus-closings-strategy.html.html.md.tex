Sections

SEARCH

\protect\hyperlink{site-content}{Skip to
content}\protect\hyperlink{site-index}{Skip to site index}

\href{https://www.nytimes3xbfgragh.onion/section/business/economy}{Economy}

\href{https://myaccount.nytimes3xbfgragh.onion/auth/login?response_type=cookie\&client_id=vi}{}

\href{https://www.nytimes3xbfgragh.onion/section/todayspaper}{Today's
Paper}

\href{/section/business/economy}{Economy}\textbar{}Coronavirus
Shutdowns: Economists Look for Better Answers

\url{https://nyti.ms/2BFamlt}

\begin{itemize}
\item
\item
\item
\item
\item
\end{itemize}

\hypertarget{the-coronavirus-outbreak}{%
\subsubsection{\texorpdfstring{\href{https://www.nytimes3xbfgragh.onion/news-event/coronavirus?name=styln-coronavirus-markets\&region=TOP_BANNER\&variant=undefined\&block=storyline_menu_recirc\&action=click\&pgtype=Article\&impression_id=e07b1660-e385-11ea-a6bd-919622e83b69}{The
Coronavirus
Outbreak}}{The Coronavirus Outbreak}}\label{the-coronavirus-outbreak}}

\begin{itemize}
\tightlist
\item
  live\href{https://www.nytimes3xbfgragh.onion/2020/08/20/world/coronavirus-covid.html?name=styln-coronavirus-markets\&region=TOP_BANNER\&variant=undefined\&block=storyline_menu_recirc\&action=click\&pgtype=Article\&impression_id=e07b1661-e385-11ea-a6bd-919622e83b69}{Latest
  Updates}
\item
  \href{https://www.nytimes3xbfgragh.onion/interactive/2020/us/coronavirus-us-cases.html?name=styln-coronavirus-markets\&region=TOP_BANNER\&variant=undefined\&block=storyline_menu_recirc\&action=click\&pgtype=Article\&impression_id=e07b1662-e385-11ea-a6bd-919622e83b69}{Maps
  and Cases}
\item
  \href{https://www.nytimes3xbfgragh.onion/interactive/2020/science/coronavirus-vaccine-tracker.html?name=styln-coronavirus-markets\&region=TOP_BANNER\&variant=undefined\&block=storyline_menu_recirc\&action=click\&pgtype=Article\&impression_id=e07b1663-e385-11ea-a6bd-919622e83b69}{Vaccine
  Tracker}
\item
  \href{https://www.nytimes3xbfgragh.onion/2020/08/19/us/colleges-closing-covid.html?name=styln-coronavirus-markets\&region=TOP_BANNER\&variant=undefined\&block=storyline_menu_recirc\&action=click\&pgtype=Article\&impression_id=e07b1664-e385-11ea-a6bd-919622e83b69}{Colleges
  Closing}
\item
  \href{https://www.nytimes3xbfgragh.onion/live/2020/08/20/business/stock-market-today-coronavirus?name=styln-coronavirus-markets\&region=TOP_BANNER\&variant=undefined\&block=storyline_menu_recirc\&action=click\&pgtype=Article\&impression_id=e07b1665-e385-11ea-a6bd-919622e83b69}{Economy}
\end{itemize}

Advertisement

\protect\hyperlink{after-top}{Continue reading the main story}

Supported by

\protect\hyperlink{after-sponsor}{Continue reading the main story}

\hypertarget{coronavirus-shutdowns-economists-look-for-better-answers}{%
\section{Coronavirus Shutdowns: Economists Look for Better
Answers}\label{coronavirus-shutdowns-economists-look-for-better-answers}}

Researchers are developing models for more targeted closings (and
reopenings) that would curb the spread of infection at a less severe
economic cost.

\href{https://www.nytimes3xbfgragh.onion/by/eduardo-porter}{\includegraphics{https://static01.graylady3jvrrxbe.onion/images/2018/02/20/multimedia/author-eduardo-porter/author-eduardo-porter-thumbLarge.jpg}}

By \href{https://www.nytimes3xbfgragh.onion/by/eduardo-porter}{Eduardo
Porter}

\begin{itemize}
\item
  June 6, 2020
\item
  \begin{itemize}
  \item
  \item
  \item
  \item
  \item
  \end{itemize}
\end{itemize}

\href{https://www.nytimes3xbfgragh.onion/es/2020/06/08/espanol/negocios/economia-coronavirus-cierres.html}{Leer
en español}

As Covid-19 cases took off in New York in March, Gov. Andrew M. Cuomo
imposed a lockdown of nonessential businesses to slow the spread of the
coronavirus, calling it
``\href{https://www.nytimes3xbfgragh.onion/2020/03/20/us/ny-ca-stay-home-order.html}{the
most drastic action we can take}.''

Now researchers say more targeted approaches --- in New York and
elsewhere --- might have protected public health with less economic
pain.

Businesses in New York City, where an initial phase of reopening is to
begin on Monday, have been mostly shut down for 11 weeks. But a study
has found that the economic cost could have been reduced by a third or
more by strategically choosing neighborhoods to close, calibrating the
risk of infection for local residents and workers with the impact on
local jobs.

The ZIP codes most affected by the outbreak are not necessarily the
places with the highest concentration of jobs. It would be possible to
keep businesses in certain areas open if the chances of spreading the
virus there were low, especially if the economic cost of closing them
was disproportionately high, the researchers found.

Employment in each ZIP code

209,000

BRONX

100,000

45,000

20,000

MANHATTAN

QUEENS

10,000

0

BROOKLYN

STATEN ISLAND

Covid-19 infections

4,400

Number of cases in each ZIP code as of May 28

BRONX

2,500

1,700

1,00

MANHATTAN

QUEENS

400

0

BROOKLYN

STATEN ISLAND

Employment in

each ZIP code

Covid-19

infections

BRONX

BRONX

QUEENS

QUEENS

MANHATTAN

MANHATTAN

BROOKLYN

BROOKLYN

STATEN ISLAND

STATEN ISLAND

Employment in thousands

Number of cases in each ZIP code as of May 28

0

10

20

45

100

209

0

400

1,000

1,700

2,500

4,400

Employment in

each ZIP code

Covid-19

infections

209,000

4,400

Number of cases in each ZIP code as of May 28

100,000

2,500

BRONX

BRONX

45,000

1,700

20,000

1,000

MANHATTAN

MANHATTAN

QUEENS

QUEENS

10,000

400

0

0

BROOKLYN

BROOKLYN

STATEN ISLAND

STATEN ISLAND

Sources: John R. Birge, Ozan Candogan (Univ. of Chicago) and Yiding Feng
(Northwestern Univ.); New York City Department of Health and Mental
Hygiene

By Karl Russell

``The blunt instrument of a uniform policy causes more economic and
related health harm than is necessary to accomplish the same goal of not
increasing infections,'' said John Birge, a mathematician at the
University of Chicago who was an author of the study.

Other researchers are taking on the problem by assessing the relative
level of risk posed by different businesses.

``The distinction between essential and nonessential businesses is very
arbitrary,'' said Katherine Baicker, an economist at the University of
Chicago involved in some of the research. ``In a world where
policymakers could be nuanced across type of business, geography and
time, the policy response could be much better.''

With daily Covid-19 deaths abating in New York and many other parts of
the country, and cities and states easing lockdowns, researchers are
beginning to assess alternative strategies to manage the spread of the
virus. Much of their work involves using data to devise less restrictive
containment policies, in the United States and elsewhere.

In Europe, France and Spain have adopted versions of
\href{https://www.oecd-forum.org/channels/1642-international-co-operation/posts/toward-a-european-network-of-green-zones-to-avoid-summer-collapse}{a
plan put forth} by Bary S.R. Pradelski, an economist at Oxford
University, and Miquel Oliu-Barton, a mathematician at the Université
Paris-Dauphine, to divide countries into dangerous red and safer green
zones, and allow travel within and between the green while strictly
curtailing it in the red.

\hypertarget{latest-updates-the-coronavirus-outbreak-and-the-economy}{%
\section{\texorpdfstring{\href{https://www.nytimes3xbfgragh.onion/live/2020/08/20/business/stock-market-today-coronavirus?action=click\&pgtype=Article\&state=default\&region=MAIN_CONTENT_1\&context=storylines_live_updates}{Latest
Updates: The Coronavirus Outbreak and the
Economy}}{Latest Updates: The Coronavirus Outbreak and the Economy}}\label{latest-updates-the-coronavirus-outbreak-and-the-economy}}

\href{https://www.nytimes3xbfgragh.onion/live/2020/08/20/business/stock-market-today-coronavirus?action=click\&pgtype=Article\&state=default\&region=MAIN_CONTENT_1\&context=storylines_live_updates\#the-producer-of-unhinged-makes-a-big-bet-on-audiences-returning-to-theaters}{10h
ago}

\href{https://www.nytimes3xbfgragh.onion/live/2020/08/20/business/stock-market-today-coronavirus?action=click\&pgtype=Article\&state=default\&region=MAIN_CONTENT_1\&context=storylines_live_updates\#the-producer-of-unhinged-makes-a-big-bet-on-audiences-returning-to-theaters}{The
producer of `Unhinged' makes a big bet on audiences returning to
theaters.}

\href{https://www.nytimes3xbfgragh.onion/live/2020/08/20/business/stock-market-today-coronavirus?action=click\&pgtype=Article\&state=default\&region=MAIN_CONTENT_1\&context=storylines_live_updates\#american-airlines-to-stop-flights-to-15-cities-after-government-aid-ends}{19h
ago}

\href{https://www.nytimes3xbfgragh.onion/live/2020/08/20/business/stock-market-today-coronavirus?action=click\&pgtype=Article\&state=default\&region=MAIN_CONTENT_1\&context=storylines_live_updates\#american-airlines-to-stop-flights-to-15-cities-after-government-aid-ends}{American
Airlines to stop flights to 15 cities after government aid ends.}

\href{https://www.nytimes3xbfgragh.onion/live/2020/08/20/business/stock-market-today-coronavirus?action=click\&pgtype=Article\&state=default\&region=MAIN_CONTENT_1\&context=storylines_live_updates\#without-school-plays-and-assemblies-a-technicians-livelihood-withers}{19h
ago}

\href{https://www.nytimes3xbfgragh.onion/live/2020/08/20/business/stock-market-today-coronavirus?action=click\&pgtype=Article\&state=default\&region=MAIN_CONTENT_1\&context=storylines_live_updates\#without-school-plays-and-assemblies-a-technicians-livelihood-withers}{Without
school plays and assemblies, a technician's livelihood withers.}

\href{https://www.nytimes3xbfgragh.onion/live/2020/08/20/business/stock-market-today-coronavirus?action=click\&pgtype=Article\&state=default\&region=MAIN_CONTENT_1\&context=storylines_live_updates}{See
more updates}

More live coverage:
\href{https://www.nytimes3xbfgragh.onion/2020/08/20/world/coronavirus-covid.html?action=click\&pgtype=Article\&state=default\&region=MAIN_CONTENT_1\&context=storylines_live_updates}{Global}

The researchers are pushing for the European Union to reopen the tourism
business that is so critical for southern European countries by allowing
travel between green areas that have low infection rates, hospital
capacity to spare, and effective testing and tracing systems.
Governments could focus their resources on the red areas of most
economic importance, increasing testing and adding hospital capacity to
turn them green.

``The impact of not having any tourism in the summer would bring Greece,
Italy and Spain to an economic situation like they experienced in the
Great Recession,'' Mr. Pradelski said. Spain, for instance, has agreed
to open the island of Majorca to tourists from some countries, including
Germany and France, while keeping it closed to travel to and from the
Spanish mainland.

In the United States, a group of researchers including Ms. Baicker is
\href{https://www.nytimes3xbfgragh.onion/interactive/2020/05/06/opinion/coronavirus-us-reopen.html}{following
a different track}: using cellphone data and surveys to identify which
businesses are more crowded, as well as how much of their business is
conducted indoors, and how much interaction it involves, either person
to person or via touching shared surfaces.

Customers tend to linger longer in a Chuck E. Cheese than in a
Chick-fil-A, increasing their risk of contagion if somebody nearby is
infected. Chick-fil-A, however, receives a lot more customers per square
foot, bringing more people in contact with one another. Nail salons
involve more personal interaction than lawn and garden stores. Some
restaurants are packed at certain times, while others receive a steady
trickle throughout the day.

The researchers' idea is that businesses could retrofit in ways suited
to each --- say, spacing out tables or limiting foot traffic --- while
safeguarding health. Moreover, with access to real-time information,
consumers could avoid riskier businesses and shop when their preferred
stores might be less crowded.

The New York City study also relied on cellphone data.
\href{https://papers.ssrn.com/sol3/papers.cfm?abstract_id=3590621}{That
research}, by Mr. Birge and Ozan Candogan of the University of Chicago
and Yiding Feng of Northwestern University, is based on the premise that
residents of one neighborhood can become infected, or infect others,
while at work in another --- depending on how long they spend there and
the infection rate in both. The risk of spreading the virus by opening a
given neighborhood to business also rises with the size of its
susceptible population.

``Mobility plays a key role,'' Mr. Candogan said. ``With smart policies
we could reduce the spread of the disease and minimize job losses.'' The
study identifies the plan that preserves as much employment as possible
even as it contains the disease.

``With appropriate targeting,'' the scholars wrote, ``it is possible to
achieve a reduction in infections with up to 33 percent to 42 percent
lower economic cost than uniform citywide closure policies.''

New York City is not in absolute control of its destiny. If adjoining
areas in Westchester County, Long Island and New Jersey remain open for
business, New Yorkers who are infected while at work there will spread
the virus in the city when they return home.

But even at the peak of contagion in mid-April, assuming 80 percent of
businesses remained open in neighboring counties, judiciously shutting
down businesses would have allowed New York City to reduce infection
rates in every neighborhood while keeping 40 percent of its economy
open, the three researchers found.

The challenge gets easier as the infection rate declines. If New York
reaches a point over the summer in which infection has pretty much faded
and the task is to prevent the virus from reasserting itself, many more
businesses could reopen, according to the researchers.

Share of the economy that could remain open

0

10

48

68

91

100\%

FEWER BUSINESSES OPEN

How much each neighborhood would have to close to reduce the spread of
Covid-19

BRONX

MANHATTAN

High-infection

scenario

QUEENS

BROOKLYN

STATEN ISLAND

BRONX

Low-infection

scenario

MANHATTAN

QUEENS

BROOKLYN

STATEN ISLAND

How much each neighborhood would have to close to reduce the spread of
Covid-19

Share of the economy that could remain open

0

10

48

68

91

100\%

FEWER BUSINESSES OPEN

High-infection

scenario

Low-infection

scenario

BRONX

BRONX

MANHATTAN

MANHATTAN

QUEENS

QUEENS

BROOKLYN

BROOKLYN

STATEN ISLAND

STATEN ISLAND

How much each neighborhood would have to close to reduce the spread of
Covid-19

Share of the economy that could remain open

0

10

48

68

91

100\%

FEWER BUSINESSES OPEN

BRONX

BRONX

High-infection

scenario

Low-infection

scenario

QUEENS

QUEENS

MANHATTAN

MANHATTAN

BROOKLYN

BROOKLYN

STATEN ISLAND

STATEN ISLAND

Source: John R. Birge, Ozan Candogan (University of Chicago) and Yiding
Feng (Northwestern University)

By Karl Russell

In the first case, during the peak of the outbreak, Wall Street would
have had to remain shuttered, as indeed most businesses in Manhattan.
But companies in Manhattan's garment district, where there are lots of
jobs but only a small resident population, could have opened for
business, as well as those across a large swath of the other boroughs.
Their risk of contagion from outsiders coming in to work there was small
enough, and the gain from maintaining their jobs large enough, to
justify keeping them humming.

When New York manages to sharply reduce infection rates, and the
challenge becomes preventing a second wave, the city could keep 87
percent of its economy running even if adjoining counties were operating
at 80 percent capacity, the researchers concluded. Jackson Heights and
Corona in Queens would have to keep most of their businesses closed, as
would some big chunks of Manhattan's Upper East Side and Kingsbridge
Heights in the Bronx. But most neighborhoods could allow business
broadly to resume operation.

It's unclear to what extent these studies could yield useful policy. It
might be difficult for Mr. Cuomo or Mayor Bill de Blasio to close down a
neighborhood and keep the adjoining ones open. One striking feature
arising from this research, in any case, is the extent to which data
could help steer decisions.

Both American studies rely on cellphone data reflecting patterns of
movement and commerce. Researchers do not have access to information
about individual users.

But more data is available. Researchers from Pennsylvania State
University, the University of Chicago and the University of California,
San Diego,
\href{https://bfi.uchicago.edu/working-paper/the-cost-of-privacy-welfare-effects-of-the-disclosure-of-covid-19-cases/}{studied
what could be done} with cellphone tracking data that, instead, detailed
the travel history of people who had become infected. That's what
happened in South Korea.

South Korea
\href{https://www.nytimes3xbfgragh.onion/article/coronavirus-timeline.html}{detected
its first case} of Covid-19 on Jan. 20, one day before the United States
did. By May 26, 295 out of every million Americans had died from the
disease, almost 60 times South Korea's rate. And it contained the
disease without a lockdown. The International Monetary Fund expects
South Korea's economy to contract by only 1.2 percent this year,
compared with 5.9 percent for the United States.

South Korea started testing people much earlier than the United States,
allowing health authorities to track potential routes of contagion and
isolate the infected. But the researchers point out that South Korea's
strategy also relied critically on the publication of their travel
histories.

South Koreans received text messages whenever new cases were discovered
in their neighborhood, as well as information and timelines of infected
people's travel. Though businesses were not shut down, South Koreans
knew which Starbucks had served an infected person, and could stay away
from it for a while.

Using cellphone data to track changes in people's commuting around town,
the economists estimated that in Seoul alone, public disclosure would
reduce the number of cases over two years by 400,000 and the number of
deaths by 13,000. And achieving the same death rate using citywide
lockdowns such as those done in New York would double the economic cost.

``We are not distinguishing between places where the probability of
infection is high and places where the probability is low. We are
lowering social interactions across the board,'' said Chang-Tai Hsieh of
the University of Chicago, a co-author of
\href{https://bfi.uchicago.edu/working-paper/the-cost-of-privacy-welfare-effects-of-the-disclosure-of-covid-19-cases/}{the
study of South Korea's methods}. ``We can do social-distancing in a much
smarter way that's a lot more targeted, in which we get more benefits
and less costs.''

It is unclear whether Americans would tolerate the kind of intrusion
into their personal lives that the Korean strategy would entail. But the
more information is available, the more precise the strategy can be.

``I don't think people have thought that deeply about it,'' Mr. Hsieh
said. ``It is something that we as a society have to decide: What's the
trade-off we are willing to live with?''

Advertisement

\protect\hyperlink{after-bottom}{Continue reading the main story}

\hypertarget{site-index}{%
\subsection{Site Index}\label{site-index}}

\hypertarget{site-information-navigation}{%
\subsection{Site Information
Navigation}\label{site-information-navigation}}

\begin{itemize}
\tightlist
\item
  \href{https://help.nytimes3xbfgragh.onion/hc/en-us/articles/115014792127-Copyright-notice}{©~2020~The
  New York Times Company}
\end{itemize}

\begin{itemize}
\tightlist
\item
  \href{https://www.nytco.com/}{NYTCo}
\item
  \href{https://help.nytimes3xbfgragh.onion/hc/en-us/articles/115015385887-Contact-Us}{Contact
  Us}
\item
  \href{https://www.nytco.com/careers/}{Work with us}
\item
  \href{https://nytmediakit.com/}{Advertise}
\item
  \href{http://www.tbrandstudio.com/}{T Brand Studio}
\item
  \href{https://www.nytimes3xbfgragh.onion/privacy/cookie-policy\#how-do-i-manage-trackers}{Your
  Ad Choices}
\item
  \href{https://www.nytimes3xbfgragh.onion/privacy}{Privacy}
\item
  \href{https://help.nytimes3xbfgragh.onion/hc/en-us/articles/115014893428-Terms-of-service}{Terms
  of Service}
\item
  \href{https://help.nytimes3xbfgragh.onion/hc/en-us/articles/115014893968-Terms-of-sale}{Terms
  of Sale}
\item
  \href{https://spiderbites.nytimes3xbfgragh.onion}{Site Map}
\item
  \href{https://help.nytimes3xbfgragh.onion/hc/en-us}{Help}
\item
  \href{https://www.nytimes3xbfgragh.onion/subscription?campaignId=37WXW}{Subscriptions}
\end{itemize}
