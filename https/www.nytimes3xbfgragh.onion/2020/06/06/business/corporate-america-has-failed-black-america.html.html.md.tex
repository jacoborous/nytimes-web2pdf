\href{/section/business}{Business}\textbar{}`Corporate America Has
Failed Black America'

\url{https://nyti.ms/3eNZH6i}

\begin{itemize}
\item
\item
\item
\item
\item
\end{itemize}

\href{https://www.nytimes3xbfgragh.onion/news-event/george-floyd-protests-minneapolis-new-york-los-angeles?action=click\&pgtype=Article\&state=default\&region=TOP_BANNER\&context=storylines_menu}{Race
and America}

\begin{itemize}
\tightlist
\item
  \href{https://www.nytimes3xbfgragh.onion/2020/07/26/us/protests-portland-seattle-trump.html?action=click\&pgtype=Article\&state=default\&region=TOP_BANNER\&context=storylines_menu}{Protesters
  Return to Other Cities}
\item
  \href{https://www.nytimes3xbfgragh.onion/2020/07/24/us/portland-oregon-protests-white-race.html?action=click\&pgtype=Article\&state=default\&region=TOP_BANNER\&context=storylines_menu}{Portland
  at the Center}
\item
  \href{https://www.nytimes3xbfgragh.onion/2020/07/23/podcasts/the-daily/portland-protests.html?action=click\&pgtype=Article\&state=default\&region=TOP_BANNER\&context=storylines_menu}{Podcast:
  Showdown in Portland}
\item
  \href{https://www.nytimes3xbfgragh.onion/interactive/2020/07/16/us/black-lives-matter-protests-louisville-breonna-taylor.html?action=click\&pgtype=Article\&state=default\&region=TOP_BANNER\&context=storylines_menu}{45
  Days in Louisville}
\end{itemize}

\includegraphics{https://static01.graylady3jvrrxbe.onion/images/2020/06/05/business/05GELLES-01-sub/merlin_172586607_01e832ac-df9c-49cf-b635-e0de31a5bcfe-articleLarge.jpg?quality=75\&auto=webp\&disable=upscale}

Sections

\protect\hyperlink{site-content}{Skip to
content}\protect\hyperlink{site-index}{Skip to site index}

\hypertarget{corporate-america-has-failed-black-america}{%
\section{`Corporate America Has Failed Black
America'}\label{corporate-america-has-failed-black-america}}

For a group of elite black executives, police killings and protests have
unleashed an outpouring of emotion and calls for action.

``My blood boiled a long time ago,'' said Robert Reffkin, the co-founder
of the real estate brokerage Compass, ``when I formed the impression
that most companies don't care.''Credit...George Etheredge for The New
York Times

Supported by

\protect\hyperlink{after-sponsor}{Continue reading the main story}

\href{https://www.nytimes3xbfgragh.onion/by/david-gelles}{\includegraphics{https://static01.graylady3jvrrxbe.onion/images/2018/07/24/multimedia/author-david-gelles/author-david-gelles-thumbLarge.png}}

By \href{https://www.nytimes3xbfgragh.onion/by/david-gelles}{David
Gelles}

\begin{itemize}
\item
  June 6, 2020
\item
  \begin{itemize}
  \item
  \item
  \item
  \item
  \item
  \end{itemize}
\end{itemize}

In the past week, it has seemed like every major company has publicly
condemned racism. All-black squares cover corporate Instagram.
Executives have made multimillion-dollar pledges to anti-discrimination
efforts and programs to support black businesses.

Yet many of the same companies expressing solidarity have contributed to
systemic inequality, targeted the black community with unhealthy
products and services, and failed to hire, promote and fairly compensate
black men and women.

``Corporate America has failed black America,'' said Darren Walker, the
president of the Ford Foundation and a member of the board of Pepsi, and
who is black. ``Even after a generation of Ivy League educations and
extraordinary talented African-Americans going into corporate America,
we seem to have hit a wall.''

With dozens of cities protesting the violent deaths of George Floyd,
Ahmaud Arbery, Breonna Taylor and others, a national conversation about
racism is underway. For black executives, who have spent their lives
excelling at business while overcoming structural discrimination, the
killings and ensuing protests have unleashed an outpouring of emotion.
Many are speaking candidly about their private fears, as well as their
disappointment with the corporate apparatus that made them stars.

Wes Moore, the chief executive of Robin Hood, a New York charity
combating poverty, said he chooses his workout clothes to minimize the
chances anyone will consider him, as a black man, dangerous. ``I pick
the outfit that I wear when I run strategically,'' he said. ``I wear
shirts with my alma mater on it, Johns Hopkins, so people know I'm not a
threat.'' Mr. Arbery was
\href{https://www.nytimes3xbfgragh.onion/article/ahmaud-arbery-shooting-georgia.html}{shot
to death} by apparent vigilantes while out for a jog; three white men
have been charged.

Mr. Moore said he was fed up with being one of just a relatively small
number of black executives in the top tier of American business. ``The
list starts getting very thin very quickly,'' he said. ``There aren't
enough good examples. We've been satisfied with exceptions and
exceptionalism.''

``We've been satisfied by putting John Rogers on every board,'' he
added, referring to the
\href{https://www.arielinvestments.com/content/view/112/1062/}{black
investor} who has been a director at Exelon, McDonald's, Nike and The
New York Times Company. ``But we haven't been deliberate about building
bench and pipeline.''

Robert F. Smith, a private equity billionaire and the
\href{https://www.nytimes3xbfgragh.onion/2019/05/19/business/robert-f-smith-morehouse-vista-equity.html}{richest
black man in America}, said he has been overwhelmed by conflicting
feelings. ``I am saddened, I am angry, I am upset and I am determined,''
he said. ``I run through that wave of emotions every minute.''

Mr. Smith added that for the first time in a long time, he had reason
for optimism. Over the past week, he said, he has been inundated with
calls from other business leaders wanting to know what they can do.
``This is the first time in my life I've seen not just empathy, but
engagement,'' he said. ``This is unacceptable, and other C.E.O.s are
asking how they can get involved.''

\includegraphics{https://static01.graylady3jvrrxbe.onion/images/2020/06/07/business/05GELLES-walker/05GELLES-walker-articleLarge.jpg?quality=75\&auto=webp\&disable=upscale}

Mr. Walker, too, said the severity of this moment seemed to be shocking
some companies into action. ``Corporate America can no longer get away
with token responses to systemic problems,'' said Mr. Walker, who has
been protesting in New York. ``It is going to take a systemic response
to sufficiently address this crisis that has been decades in the
making.''

\hypertarget{its-complete-bs-its-performative}{%
\subsection{`It's complete B.S. It's
performative'}\label{its-complete-bs-its-performative}}

As brands rushed to align themselves with protesters over the past week,
their words often rang hollow, undermined by their own actions.

Amazon called for an end to the ``inequitable'' treatment of black
people. Yet the company has faced sustained criticism for poor working
conditions and low pay. In March, it
\href{https://www.nytimes3xbfgragh.onion/2020/04/03/nyregion/coronavirus-nyc-chris-smalls-amazon.html}{fired
Christian Smalls}, a black employee at a Staten Island warehouse who was
demanding safer conditions while working in a pandemic, and the
company's general counsel
\href{https://www.nytimes3xbfgragh.onion/2020/04/18/technology/athena-mitchell-amazon.html}{disparaged}
him as ``not smart or articulate.'' Amazon has said Mr. Smalls violated
its social distancing policy, and that the executive did not know he was
black.

The commissioner of the National Football League, Roger Goodell, issued
a statement saying the protests express ``the pain, anger and
frustration that so many of us feel.'' But his organization has banned
players ---~most of whom are black --- from kneeling to protest police
brutality, and the quarterback most identified with the gesture, Colin
Kaepernick, has been effectively blacklisted. (On Friday night, Mr.
Goodell appeared to
\href{https://www.nytimes3xbfgragh.onion/2020/06/05/sports/football/trump-anthem-kneeling-kaepernick.html}{reverse
himself}, saying, ``We, the National Football League admit we were
wrong'' and adding, ``I personally protest with you.'')

L'Oréal shared a post that read ``Speaking out is worth it.'' But three
years ago, the makeup company
\href{https://www.nytimes3xbfgragh.onion/2017/09/02/business/munroe-bergdorf-loreal-transgender.html}{dropped
its first transgender model}, Munroe Bergdorf, when she spoke out about
racism after the white nationalist violence in Charlottesville, Va.

``Most of these corporate statements were put together by the marketing
team that was trying not to offend white customers and white
employees,'' said Dorothy A. Brown, a law professor who studies economic
injustice at Emory University in Atlanta. ``It's complete B.S. It's
performative.''

Companies have for the most part addressed racism only in the face of
overwhelming public pressure. In the 1980s, for example, a global
protest movement forced corporations including General Motors and Pepsi
to stop doing business in apartheid South Africa. More often, however,
companies have studiously avoided confronting the legacy of racism.

Members of ``corporate America have generally not distinguished
themselves as moral leaders,'' said Ursula Burns, the former chief
executive of Xerox and a board member at Exxon. ``They generally have
gone along with the flow, and for a long time that's all we expected
them to do. They were responsible to their shareholders.''

Image

Ursula Burns in Davos, Switzerland, in 2018. ``I worry every day if a
policeman is near me,'' she said this week.Credit...Denis
Balibouse/Reuters

Ms. Burns herself, despite leading a gilded life as a successful black
C.E.O., said that law enforcement still makes her nervous. ``I dress
like the one percent. I drive like the one percent. I wear watches and
jewelry like the one percent,'' she said, adding: ``I worry every day if
a policeman is near me. They look at me as first and foremost a threat
to their place in society.''

She added that with police cracking down violently on protesters, ``It
is the scariest moment I have been in, in my entire life.''

Even after the violence in Charlottesville, which led to an
\href{https://www.nytimes3xbfgragh.onion/2017/08/16/business/trumps-council-ceos.html}{abrupt
disbanding} of President Trump's business advisory councils, few
companies made lasting policy changes.

Instead, generations of well-intentioned pledges by businesses have
resulted in
\href{https://www.nytimes3xbfgragh.onion/2020/06/01/business/economy/black-workers-inequality-economic-risks.html}{only
marginal advancement} for the black community. The coronavirus pandemic
has exacerbated grim employment trends, and today fewer than half of
black adults in America have a job. Black workers make less money than
white workers. That is due in part to the fact that they are more likely
to have poorly paying service jobs, but research also shows that highly
educated black employees are paid less than their white peers.

``We don't get paid the same amount for the same work,'' said Mellody
Hobson, the co-chief executive of Ariel Investments and a board member
at JPMorgan and Starbucks. ``We've been disproportionally affected in
layoffs and unemployment.''

\hypertarget{my-blood-boiled-a-long-time-ago}{%
\subsection{`My blood boiled a long time
ago'}\label{my-blood-boiled-a-long-time-ago}}

At many of America's major employers, black men and women are absent
from meaningful leadership roles.

The nation's largest health care company, CVS, has no black people on
its senior leadership team.

In finance, there are no black people on the senior leadership teams of
Bank of America, JPMorgan (where managers in Phoenix branches were
\href{https://www.nytimes3xbfgragh.onion/2019/12/11/business/jpmorgan-banking-racism.html}{recorded
making racist remarks}) or Wells Fargo (which recently faced a federal
lawsuit for discriminating against minority home buyers).

In technology, there are zero black members of the senior leadership
teams of Facebook, Google, Microsoft and Amazon.

In total, there are just four black chief executives among the 500
largest companies in the country.

Many big companies have added black directors to their boards in recent
years. But while board seats can be levers to effect change, they do
little to shift the power centers within companies. Exxon, the largest
U.S. energy company, has two black board members, including Ms. Burns
--- but the management committee is composed entirely of white men.

``We are put into these positions that are honorific, because they want
our presence,'' Mr. Walker said. ``But we are not given authority and
resources.''

With black individuals deeply underrepresented in Silicon Valley and
largely absent at the highest levels of major corporations, little of
the wealth created in the stock market or the technology boom has gone
to black families. Today, typical black households have just one-tenth
the wealth of typical white households, according to Federal Reserve
data.

Image

Mr. Reffkin called on companies to demand that the teams of outside
lawyers, accountants and bankers they use include at least one black
member. Credit...Cate Dingley/Bloomberg

Robert Reffkin, the black co-founder and chief executive of the real
estate brokerage Compass, said that at one of his first jobs, he asked
his employer why they didn't do more to attract and promote black
employees. ``They said, `We tried so hard, but we didn't get the return
on investment,''' he recalled.

It was a clarifying moment for Mr. Reffkin, who grew determined to start
his own company and went on to found Compass, which is valued at \$6.4
billion.

``My blood boiled a long time ago,'' he said, ``when I formed the
impression that most companies don't care.''

\hypertarget{thats-an-indictment}{%
\subsection{`That's an indictment'}\label{thats-an-indictment}}

When companies are forced to confront racism, the responses are often
predictable.

``The playbook is: Issue a statement, get a group of African-American
leaders on a conference call, apologize and have your corporate
foundation make a contribution to the N.A.A.C.P. and the Urban League,''
Mr. Walker said. ``That's not going to work in this crisis.''

While most companies have so far stuck to the script, some have gone
further. SoftBank said it would allocate \$100 million to invest in
minority entrepreneurs, though details were scant. Visa created a \$10
million fund for college-bound black students and said it would
guarantee jobs to those who meet certain requirements. PWC said it would
begin publicly sharing its diversity strategy and results.

Yet many black executives say these efforts, while welcome, will be
insufficient to effect lasting change.

Ryan Williams, the black founder and chief executive of Cadre, a
commercial real estate investing platform, could not name a company he
believed was doing enough to support the black community.

``There is no one that comes to mind that is taking the steps to truly
level the playing field,'' Mr. Williams said. ``That's an indictment of
where we are today.''

Mr. Williams has joined the protests in New York, and said he believed
that little more than luck separated him from the men whose faces are
now found on murals and placards. ``I could have been George Floyd or
Ahmaud Arbery,'' he said.

Mr. Williams is among those black executives agitating for a series of
specific changes to the way companies hire and promote. He called on
companies to first take the basic step of disclosing diversity figures,
so that progress --- or backsliding --- can be measured.

Mr. Smith, the private equity investor, got his first break when Bell
Labs accepted him for an internship. ``An internship changed my life,''
he said. ``Let's change thousands of lives each year.'' He called on
companies to quadruple the size of their internship classes and commit
to giving many of those spots to African-Americans, and then support
them with mentors and sponsors.

``Boards should hold themselves and management accountable for specific
objectives around recruitment, retention and promotion of
African-Americans and other minorities,'' said Mr. Walker. ``Only when
companies and management are accountable in ways that are quantifiable
will we see real systemic transformation of corporate America.''

Few companies are forthcoming about the racial composition of their work
forces. Only 40 percent of companies are transparent about the gender
and racial makeup of their employees, according to
\href{https://www.nytimes3xbfgragh.onion/2015/12/21/business/a-plan-to-rank-just-companies-aims-to-close-the-wealth-gap.html}{Just
Capital}, a nonprofit that tracks corporations' social impact. And just
one company, Intel, has disclosed wage data by gender, ethnic and racial
breakdowns.

Image

Mellody Hobson, the co-C.E.O. of Ariel Investments. ``We don't get paid
the same amount for the same work,'' she said. ``We've been
disproportionally affected in layoffs and unemployment.''Credit...Guerin
Blask for The New York Times

``In business we set targets on everything,'' Ms. Hobson said. ``Only in
the area of diversity have I seen C.E.O.s chronically say, `We're
working on it.'''

When pressed on why their companies lack diversity, many managers fall
back on the argument that there is a pipeline problem; that there simply
aren't enough talented black men and women to fill the roles.

Mr. Moore dismissed that notion outright, arguing that companies simply
aren't looking hard enough, aren't recruiting at historically black
colleges and universities, and have a monoculture that overlooks black
talent.

``It's not about a lowering of standards,'' Mr. Moore said. ``Think
about how I hear that as a black man.''

Mr. Reffkin called on companies to demand that the teams of outside
lawyers, accountants and bankers they use include at least one black
member. Compass has made that commitment, he said, and has also
developed lists of black contractors for other services, such as
photography.

Companies can use their clout to promote diversity in other creative
ways, as well. Earlier this year, Goldman Sachs said it wouldn't take a
company public if it didn't have at least one woman or minority on its
board.

Ms. Hobson was among those who called for companies to tie executive pay
to diversity metrics. A few companies, including Microsoft, Intel and
Johnson \& Johnson have gone that route, but they remain the rare
exceptions.

While board seats are no replacement for executive roles, black
executives said change starts at the top. ``If you do not have blacks on
your board, you're not going to see blacks in the c-suite of that
company,'' Mr. Walker said.

When Vernon Jordan, the civil rights leader, investment banker, lawyer
and political power broker, joined the boards of Xerox and American
Express, both of those companies named black chief executives.

After months of the coronavirus pandemic and weeks of protests, Mr.
Moore said, many Americans have a longing to turn a page, to go back to
a moment before the protests. It is an impulse he cautioned against.

``There seems to be a quest to get back to a level of normalcy. But
that's not good enough, because normalcy meant exclusion, it meant
looking at disparity and shrugging,'' he said. ``The thing we should be
aiming for is a new normal that's grounded in justice --- not just
criminal justice, but economic justice.

``If you're not thinking about how you can use your company to promote
justice,'' he added, ``then you're not doing your job as an executive.''

Advertisement

\protect\hyperlink{after-bottom}{Continue reading the main story}

\hypertarget{site-index}{%
\subsection{Site Index}\label{site-index}}

\hypertarget{site-information-navigation}{%
\subsection{Site Information
Navigation}\label{site-information-navigation}}

\begin{itemize}
\tightlist
\item
  \href{https://help.nytimes3xbfgragh.onion/hc/en-us/articles/115014792127-Copyright-notice}{©~2020~The
  New York Times Company}
\end{itemize}

\begin{itemize}
\tightlist
\item
  \href{https://www.nytco.com/}{NYTCo}
\item
  \href{https://help.nytimes3xbfgragh.onion/hc/en-us/articles/115015385887-Contact-Us}{Contact
  Us}
\item
  \href{https://www.nytco.com/careers/}{Work with us}
\item
  \href{https://nytmediakit.com/}{Advertise}
\item
  \href{http://www.tbrandstudio.com/}{T Brand Studio}
\item
  \href{https://www.nytimes3xbfgragh.onion/privacy/cookie-policy\#how-do-i-manage-trackers}{Your
  Ad Choices}
\item
  \href{https://www.nytimes3xbfgragh.onion/privacy}{Privacy}
\item
  \href{https://help.nytimes3xbfgragh.onion/hc/en-us/articles/115014893428-Terms-of-service}{Terms
  of Service}
\item
  \href{https://help.nytimes3xbfgragh.onion/hc/en-us/articles/115014893968-Terms-of-sale}{Terms
  of Sale}
\item
  \href{https://spiderbites.nytimes3xbfgragh.onion}{Site Map}
\item
  \href{https://help.nytimes3xbfgragh.onion/hc/en-us}{Help}
\item
  \href{https://www.nytimes3xbfgragh.onion/subscription?campaignId=37WXW}{Subscriptions}
\end{itemize}
