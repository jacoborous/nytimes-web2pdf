Sections

SEARCH

\protect\hyperlink{site-content}{Skip to
content}\protect\hyperlink{site-index}{Skip to site index}

\href{https://www.nytimes3xbfgragh.onion/section/business/media}{Media}

\href{https://myaccount.nytimes3xbfgragh.onion/auth/login?response_type=cookie\&client_id=vi}{}

\href{https://www.nytimes3xbfgragh.onion/section/todayspaper}{Today's
Paper}

\href{/section/business/media}{Media}\textbar{}Top Editor of
Philadelphia Inquirer Resigns After `Buildings Matter' Headline

\url{https://nyti.ms/3eYwq98}

\begin{itemize}
\item
\item
\item
\item
\item
\end{itemize}

\href{https://www.nytimes3xbfgragh.onion/news-event/george-floyd-protests-minneapolis-new-york-los-angeles?action=click\&pgtype=Article\&state=default\&region=TOP_BANNER\&context=storylines_menu}{Race
and America}

\begin{itemize}
\tightlist
\item
  \href{https://www.nytimes3xbfgragh.onion/2020/07/26/us/protests-portland-seattle-trump.html?action=click\&pgtype=Article\&state=default\&region=TOP_BANNER\&context=storylines_menu}{Protesters
  Return to Other Cities}
\item
  \href{https://www.nytimes3xbfgragh.onion/2020/07/24/us/portland-oregon-protests-white-race.html?action=click\&pgtype=Article\&state=default\&region=TOP_BANNER\&context=storylines_menu}{Portland
  at the Center}
\item
  \href{https://www.nytimes3xbfgragh.onion/2020/07/23/podcasts/the-daily/portland-protests.html?action=click\&pgtype=Article\&state=default\&region=TOP_BANNER\&context=storylines_menu}{Podcast:
  Showdown in Portland}
\item
  \href{https://www.nytimes3xbfgragh.onion/interactive/2020/07/16/us/black-lives-matter-protests-louisville-breonna-taylor.html?action=click\&pgtype=Article\&state=default\&region=TOP_BANNER\&context=storylines_menu}{45
  Days in Louisville}
\end{itemize}

Advertisement

\protect\hyperlink{after-top}{Continue reading the main story}

Supported by

\protect\hyperlink{after-sponsor}{Continue reading the main story}

\hypertarget{top-editor-of-philadelphia-inquirer-resigns-after-buildings-matter-headline}{%
\section{Top Editor of Philadelphia Inquirer Resigns After `Buildings
Matter'
Headline}\label{top-editor-of-philadelphia-inquirer-resigns-after-buildings-matter-headline}}

Stan Wischnowski, a 20-year veteran of the paper, stepped down days
after the publication of an article that led to a walkout by dozens of
Inquirer journalists.

\includegraphics{https://static01.graylady3jvrrxbe.onion/images/2020/06/06/business/06inquirer-pix/merlin_173293023_ed0b8b4b-6e59-4a65-ba80-253d6d968acd-articleLarge.jpg?quality=75\&auto=webp\&disable=upscale}

\href{https://www.nytimes3xbfgragh.onion/by/marc-tracy}{\includegraphics{https://static01.graylady3jvrrxbe.onion/images/2018/02/20/multimedia/author-marc-tracy/author-marc-tracy-thumbLarge.jpg}}

By \href{https://www.nytimes3xbfgragh.onion/by/marc-tracy}{Marc Tracy}

\begin{itemize}
\item
  Published June 6, 2020Updated June 9, 2020
\item
  \begin{itemize}
  \item
  \item
  \item
  \item
  \item
  \end{itemize}
\end{itemize}

Stan Wischnowski, the top editor of The Philadelphia Inquirer, resigned
on Saturday, days after an article with the headline ``Buildings Matter,
Too,'' on the effects of civil unrest on the city's buildings, led to a
walkout by dozens of staff members.

Lisa Hughes, the publisher of The Inquirer, the 191-year-old daily owned
by the nonprofit Lenfest Institute for Journalism, said Saturday in a
memo to the staff that she had accepted Mr. Wischnowski's decision to
step down after 10 years across two stints as the leader of one of the
country's largest newsrooms.

The headline of the
\href{https://www.inquirer.com/columnists/floyd-protest-center-city-philadelphia-lootings-52nd-street-walnut-chestnut-street-20200601.html}{article}
--- a column by The Inquirer's architecture critic, Inga Saffron, that
was published on Tuesday --- played on the slogan ``Black Lives
Matter,'' long a rallying cry for civil rights activists protesting
police violence against African Americans. It has been a key phrase for
demonstrators in the nearly two weeks of protests across the country and
in cities worldwide
since\href{https://www.nytimes3xbfgragh.onion/2020/05/27/us/george-floyd-minneapolis-death.html}{a
black man in Minneapolis, George Floyd}, died last month after being
handcuffed and pinned to the ground by a white police officer's knee.

The day after the Inquirer article was published, the paper's top
editors, including Mr. Wischnowski, who had worked at the paper for 20
years, issued an apology that appeared on its website.

``The Philadelphia Inquirer
\href{https://www.inquirer.com/resizer/flGyvZx2v2J6HP5LvtQfjXi_vRo=/arc-anglerfish-arc2-prod-pmn/public/6BESMREBLFHEHFZXJPMYNC7BXE.png}{published
a headline} in Tuesday's edition that was deeply offensive. We should
not have printed it,'' the editors wrote. ``We're sorry, and regret that
we did. We also know that an apology on its own is not sufficient.

``The headline accompanied
\href{https://www.inquirer.com/columnists/floyd-protest-center-city-philadelphia-lootings-52nd-street-walnut-chestnut-street-20200601.html}{a
story on the future of Philadelphia's buildings and civic
infrastructure} in the aftermath of this week's protests,'' the apology
continued. ``The headline offensively riffed on the Black Lives Matter
movement, and suggested an equivalence between the loss of buildings and
the lives of black Americans. That is unacceptable.''

The apology noted that the headline had been ``created by one editor''
and reviewed by another before publication.

Staff members, working remotely because of the coronavirus pandemic,
convened for a regularly scheduled videoconference that day. It turned
into an hourslong discussion of newsroom diversity, pay inequity and
other issues, said Diane Mastrull, a weekend editor and the president of
the NewsGuild of Greater Philadelphia union.

``This week, the pain was just so palpable,'' she said in an interview.

On Wednesday, staff members sent a letter to management called
``\href{https://docs.google.com/document/u/1/d/e/2PACX-1vRSXh3ATPo_bjl5iUfrFnTuC-_Z-CQKt8DGtz0LgTzURnRwiPR-SEfNcaWlMMl9PNXXMhQ_nVFGvacK/pub}{An
Open Letter From Journalists of Color at The Philadelphia Inquirer}''
announcing that they would call in sick the following day, and dozens
did.

``We're tired of shouldering the burden of dragging this 200-year-old
institution kicking and screaming into a more equitable age,'' the
letter said. ``We're tired of being told of the progress the company has
made and being served platitudes about `diversity and inclusion' when we
raise our concerns. **** We're tired of seeing our words and photos
twisted to fit a narrative that does not reflect our reality. We're
tired of being told to show both sides of issues there are no two sides
of.''

The letter continued, ``Things need to change.''

Ms. Hughes, the publisher, said on Saturday that management would search
internally and externally for Mr. Wischnowski's replacement. Mr.
Wischnowski declined to comment.

David Boardman, the chairman of the Lenfest Institute board and dean of
Temple University's Klein College of Media and Communication, said in an
emailed statement, ``What Stan was able to accomplish as The Inquirer's
top editor, through a tumultuous turnstile of owners and publishers, has
been remarkable.

``That said,'' he added, ``he leaves behind some decades-old,
deep-seated and vitally important issues around diversity, equity and
inclusion, issues that were not of his creation but that will likely
benefit from a fresh approach.''

Board members of the Philadelphia Association of Black Journalists are
scheduled to meet with leaders of The Inquirer this coming week, said
the group's vice president, Ernest Owens, a writer at large at
Philadelphia Magazine.

``The conversations and concerns we have are beyond just the headline,''
Mr. Owens said, adding, ``This is a majority of color city, and The
Inquirer has fallen short when it comes to engaging the community at
large.''

The discontent at The Inquirer came during a week when more than 800
employees of The New York Times signed a letter protesting the
publication of an
\href{https://www.nytimes3xbfgragh.onion/2020/06/04/business/new-york-times-op-ed-cotton.html}{Op-Ed
article} by Senator Tom Cotton, Republican of Arkansas, calling for a
military response to unrest in American cities.

Times leaders, including the publisher, A. G. Sulzberger, and the
editorial page editor, James Bennet, apologized for publishing the
article in a videoconference meeting with staff members on Friday. Later
that day, The Times appended an editor's note to the Op-Ed.

``After publication, this essay met strong criticism from many readers
(and many Times colleagues), prompting editors to review the piece and
the editing process,'' the note said. ``Based on that review, we have
concluded that the essay fell short of our standards and should not have
been published.''

Advertisement

\protect\hyperlink{after-bottom}{Continue reading the main story}

\hypertarget{site-index}{%
\subsection{Site Index}\label{site-index}}

\hypertarget{site-information-navigation}{%
\subsection{Site Information
Navigation}\label{site-information-navigation}}

\begin{itemize}
\tightlist
\item
  \href{https://help.nytimes3xbfgragh.onion/hc/en-us/articles/115014792127-Copyright-notice}{©~2020~The
  New York Times Company}
\end{itemize}

\begin{itemize}
\tightlist
\item
  \href{https://www.nytco.com/}{NYTCo}
\item
  \href{https://help.nytimes3xbfgragh.onion/hc/en-us/articles/115015385887-Contact-Us}{Contact
  Us}
\item
  \href{https://www.nytco.com/careers/}{Work with us}
\item
  \href{https://nytmediakit.com/}{Advertise}
\item
  \href{http://www.tbrandstudio.com/}{T Brand Studio}
\item
  \href{https://www.nytimes3xbfgragh.onion/privacy/cookie-policy\#how-do-i-manage-trackers}{Your
  Ad Choices}
\item
  \href{https://www.nytimes3xbfgragh.onion/privacy}{Privacy}
\item
  \href{https://help.nytimes3xbfgragh.onion/hc/en-us/articles/115014893428-Terms-of-service}{Terms
  of Service}
\item
  \href{https://help.nytimes3xbfgragh.onion/hc/en-us/articles/115014893968-Terms-of-sale}{Terms
  of Sale}
\item
  \href{https://spiderbites.nytimes3xbfgragh.onion}{Site Map}
\item
  \href{https://help.nytimes3xbfgragh.onion/hc/en-us}{Help}
\item
  \href{https://www.nytimes3xbfgragh.onion/subscription?campaignId=37WXW}{Subscriptions}
\end{itemize}
