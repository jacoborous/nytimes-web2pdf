Sections

SEARCH

\protect\hyperlink{site-content}{Skip to
content}\protect\hyperlink{site-index}{Skip to site index}

\href{https://myaccount.nytimes3xbfgragh.onion/auth/login?response_type=cookie\&client_id=vi}{}

\href{https://www.nytimes3xbfgragh.onion/section/todayspaper}{Today's
Paper}

\href{/section/upshot}{The Upshot}\textbar{}The Rich Cut Their Spending.
That Has Hurt All the Workers Who Count on It.

\url{https://nyti.ms/2N5KilY}

\begin{itemize}
\item
\item
\item
\item
\item
\item
\end{itemize}

\hypertarget{the-coronavirus-outbreak}{%
\subsubsection{\texorpdfstring{\href{https://www.nytimes3xbfgragh.onion/news-event/coronavirus?name=styln-coronavirus-national\&region=TOP_BANNER\&variant=undefined\&block=storyline_menu_recirc\&action=click\&pgtype=Article\&impression_id=a6546ea0-e3ad-11ea-85cc-bf105b368d22}{The
Coronavirus
Outbreak}}{The Coronavirus Outbreak}}\label{the-coronavirus-outbreak}}

\begin{itemize}
\tightlist
\item
  live\href{https://www.nytimes3xbfgragh.onion/2020/08/21/world/covid-19-coronavirus.html?name=styln-coronavirus-national\&region=TOP_BANNER\&variant=undefined\&block=storyline_menu_recirc\&action=click\&pgtype=Article\&impression_id=a65495b0-e3ad-11ea-85cc-bf105b368d22}{Latest
  Updates}
\item
  \href{https://www.nytimes3xbfgragh.onion/interactive/2020/us/coronavirus-us-cases.html?name=styln-coronavirus-national\&region=TOP_BANNER\&variant=undefined\&block=storyline_menu_recirc\&action=click\&pgtype=Article\&impression_id=a65495b1-e3ad-11ea-85cc-bf105b368d22}{Maps
  and Cases}
\item
  \href{https://www.nytimes3xbfgragh.onion/interactive/2020/science/coronavirus-vaccine-tracker.html?name=styln-coronavirus-national\&region=TOP_BANNER\&variant=undefined\&block=storyline_menu_recirc\&action=click\&pgtype=Article\&impression_id=a654bcc0-e3ad-11ea-85cc-bf105b368d22}{Vaccine
  Tracker}
\item
  \href{https://www.nytimes3xbfgragh.onion/2020/08/19/us/colleges-closing-covid.html?name=styln-coronavirus-national\&region=TOP_BANNER\&variant=undefined\&block=storyline_menu_recirc\&action=click\&pgtype=Article\&impression_id=a654bcc1-e3ad-11ea-85cc-bf105b368d22}{Colleges
  Closing}
\item
  \href{https://www.nytimes3xbfgragh.onion/live/2020/08/21/business/stock-market-today-coronavirus?name=styln-coronavirus-national\&region=TOP_BANNER\&variant=undefined\&block=storyline_menu_recirc\&action=click\&pgtype=Article\&impression_id=a654bcc2-e3ad-11ea-85cc-bf105b368d22}{Economy}
\end{itemize}

Advertisement

\protect\hyperlink{after-top}{Continue reading the main story}

Upshot

Supported by

\protect\hyperlink{after-sponsor}{Continue reading the main story}

\hypertarget{the-rich-cut-their-spending-that-has-hurt-all-the-workers-who-count-on-it}{%
\section{The Rich Cut Their Spending. That Has Hurt All the Workers Who
Count on
It.}\label{the-rich-cut-their-spending-that-has-hurt-all-the-workers-who-count-on-it}}

The steepest declines in spending during the coronavirus recession have
come from the highest-income places.

\href{https://www.nytimes3xbfgragh.onion/by/emily-badger}{\includegraphics{https://static01.graylady3jvrrxbe.onion/images/2018/02/16/multimedia/author-emily-badger/author-emily-badger-thumbLarge-v2.png}}\href{https://www.nytimes3xbfgragh.onion/by/alicia-parlapiano}{\includegraphics{https://static01.graylady3jvrrxbe.onion/images/2018/12/10/multimedia/author-alicia-parlapiano/author-alicia-parlapiano-thumbLarge.png}}

By \href{https://www.nytimes3xbfgragh.onion/by/emily-badger}{Emily
Badger} and
\href{https://www.nytimes3xbfgragh.onion/by/alicia-parlapiano}{Alicia
Parlapiano}

\begin{itemize}
\item
  June 17, 2020
\item
  \begin{itemize}
  \item
  \item
  \item
  \item
  \item
  \item
  \end{itemize}
\end{itemize}

March 1

April 1

May 1

June 1

0\%

First stimulus

checks received

Half of states in

process of reopening

ZIP CODE INCOME LEVEL

--5\%

Bottom 25\%

Bottom middle 25\%

--10\%

Top middle 25\%

--15\%

Change in consumer spending during the pandemic

Top 25\%

--20\%

--25\%

--30\%

--35\%

March 1

April 1

May 1

June 1

0\%

First stimulus

checks received

Half of states in

process of reopening

ZIP CODE INCOME LEVEL

--5\%

Bottom 25\%

Bottom middle 25\%

--10\%

Top middle 25\%

--15\%

Change in consumer spending during the pandemic

Top 25\%

--20\%

--25\%

--30\%

--35\%

March 1

April 1

May 1

June 1

0\%

Half of states in

process of reopening

ZIP CODE INCOME LEVEL

--5\%

Bottom 25\%

Bottom middle 25\%

--10\%

Top middle 25\%

--15\%

Top 25\%

--20\%

Change in consumer spending during the pandemic

--25\%

--30\%

--35\%

Change in consumer spending during the pandemic

March 1

April 1

May 1

June 1

ZIP CODE

INCOME

LEVEL

--5\%

Bottom

25\%

Bottom

middle

--10\%

Top

middle

--15\%

Top

25\%

--20\%

--25\%

--30\%

--35\%

First stimulus

checks received

Half of states in

process of reopening

Note: Change calculated from seasonally adjusted January average. Income
groups are based on the median income in the ZIP codes where consumers
live. The data sample reflects about 10 percent of all national credit
and debit card spending.·Source: Analysis of data from Affinity
Solutions by Opportunity Insights

In the Manhattan restaurants around Lincoln Center, the tips often rose
and fell with the changing playbill. A popular classic musical could
mean more preshow diners, and more income. A more famous actress as
Eliza Doolittle could do the same. The end of a big run, like ``My Fair
Lady,'' meant the opposite: Tips would be down for a while.

``We were dependent on how well shows were doing at Lincoln Center, and
we really did pay attention,'' said Emma Craig, who was a server at the
Atlantic Grill a block away before the coronavirus crisis. She has not
returned to that job yet, or to another singing at a private supper club
downtown. In both jobs, she said, ``I am dependent on the trickle
down.''

The recession has crushed this kind of work in particular: service jobs
that depend directly on the spending --- and the whims --- of the
well-off.

\includegraphics{https://static01.graylady3jvrrxbe.onion/images/2020/06/18/business/18up-up-richpooreconomy1/merlin_173569083_7954cc26-d64c-452a-bb5c-60334ad7d787-articleLarge.jpg?quality=75\&auto=webp\&disable=upscale}

Economists at the Harvard-based research group
\href{https://opportunityinsights.org/}{Opportunity Insights} estimate
that the highest-earning quarter of Americans has been responsible for
about half of the decline in consumption during this recession. And that
has \href{https://opportunityinsights.org/paper/tracker/}{wreaked havoc
on the lower-wage service workers} on the other end of many of their
transactions, the researchers say.

``One of the things this crisis has made salient is how interdependent
our health was,'' said Michael Stepner, an economist at the University
of Toronto. ``We're seeing the mirror of that on the economic side.''

As income inequality has grown in America, so
\href{https://pubs.aeaweb.org/doi/pdfplus/10.1257/jep.30.2.3}{has
inequality in consumption}. That means that when the rich spend money,
they drive more of the economy than they did 50 years ago. And more
workers depend on them.

Put another way, this particular economic shock --- one that has halted
much in-person spending, even by rich people who never lost their jobs
--- has been devastating for an economy in which many low-wage workers
count on high-income people spending money.

Mr. Stepner and the economists Raj Chetty, Nathaniel Hendren and John
Friedman have collected data from credit card processors, payroll firms
and other private
companies\href{https://tracker.opportunityinsights.org/}{tracking how
and where people spend their money}, and how businesses and their
workers have been affected as a result. By tying debit and credit card
spending back to the home ZIP codes of millions of anonymized
cardholders, they estimate that households in the bottom quarter of ZIP
codes by income cut their spending by about 30 percent from
pre-coronavirus levels at the lowest point in late March. Now, with the
help of government stimulus, low-income spending is down only about 5
percent.

For the highest-income quarter, spending has recovered much more slowly,
after falling by 36 percent at the lowest point.

\hypertarget{latest-updates-the-coronavirus-outbreak}{%
\section{\texorpdfstring{\href{https://www.nytimes3xbfgragh.onion/2020/08/21/world/covid-19-coronavirus.html?action=click\&pgtype=Article\&state=default\&region=MAIN_CONTENT_1\&context=storylines_live_updates}{Latest
Updates: The Coronavirus
Outbreak}}{Latest Updates: The Coronavirus Outbreak}}\label{latest-updates-the-coronavirus-outbreak}}

Updated 2020-08-21T12:38:27.712Z

\begin{itemize}
\tightlist
\item
  \href{https://www.nytimes3xbfgragh.onion/2020/08/21/world/covid-19-coronavirus.html?action=click\&pgtype=Article\&state=default\&region=MAIN_CONTENT_1\&context=storylines_live_updates\#link-6a60a19d}{`Be
  adults': Universities in the U.S. are warning students about
  gatherings as they return to campus.}
\item
  \href{https://www.nytimes3xbfgragh.onion/2020/08/21/world/covid-19-coronavirus.html?action=click\&pgtype=Article\&state=default\&region=MAIN_CONTENT_1\&context=storylines_live_updates\#link-324af071}{As
  he accepts the Democratic nomination, Biden knocks Trump's pandemic
  response.}
\item
  \href{https://www.nytimes3xbfgragh.onion/2020/08/21/world/covid-19-coronavirus.html?action=click\&pgtype=Article\&state=default\&region=MAIN_CONTENT_1\&context=storylines_live_updates\#link-191d44be}{South
  Korea threatens to detain people who obstruct virus-control efforts.}
\end{itemize}

\href{https://www.nytimes3xbfgragh.onion/2020/08/21/world/covid-19-coronavirus.html?action=click\&pgtype=Article\&state=default\&region=MAIN_CONTENT_1\&context=storylines_live_updates}{See
more updates}

More live coverage:
\href{https://www.nytimes3xbfgragh.onion/live/2020/08/21/business/stock-market-today-coronavirus?action=click\&pgtype=Article\&state=default\&region=MAIN_CONTENT_1\&context=storylines_live_updates}{Markets}

``It's not just that it's somewhat bigger in percentage terms,'' Mr.
Chetty said of shifts by the rich. ``In absolute dollars, that's like
half of the game.''

The researchers point to several curious patterns tied to that fact:
Unemployment claims have been high in rich counties that were largely
immune to the last recession. And lower-income Americans living in those
richer counties have been hit particularly hard. Their spending fell
further than the spending of lower-income workers in poorer counties.

\hypertarget{small-businesses-in-the-richest-neighborhoods-have-had-the-biggest-drops-in-revenue}{%
\subsubsection{Small businesses in the richest neighborhoods have had
the biggest drops in
revenue}\label{small-businesses-in-the-richest-neighborhoods-have-had-the-biggest-drops-in-revenue}}

\hypertarget{change-in-small-business-revenue-during-the-pandemic}{%
\paragraph{Change in small business revenue during the
pandemic}\label{change-in-small-business-revenue-during-the-pandemic}}

March 1

April 1

May 1

June 1

+10\%

First stimulus

checks received

Half of states in

process of reopening

0\%

ZIP CODE INCOME LEVEL

Bottom 25\%

--10\%

Bottom middle 25\%

Top middle 25\%

--20\%

Top 25\%

--30\%

--40\%

--50\%

March 1

April 1

May 1

June 1

+10\%

First stimulus

checks received

Half of states in

process of reopening

0\%

ZIP CODE INCOME LEVEL

Bottom 25\%

--10\%

Bottom middle 25\%

Top middle 25\%

--20\%

Top 25\%

--30\%

--40\%

--50\%

March 1

April 1

May 1

June 1

+10\%

First stimulus

checks received

Half of states in

process of reopening

0\%

ZIP CODE INCOME LEVEL

Bottom 25\%

--10\%

Bottom middle 25\%

Top middle 25\%

--20\%

Top 25\%

--30\%

--40\%

--50\%

March 1

April 1

May 1

June 1

+10\%

ZIP CODE

INCOME

LEVEL

Bottom

25\%

--10\%

Bottom

middle

Top

middle

--20\%

Top

25\%

--30\%

--40\%

First stimulus

checks received

Half of states in

process of reopening

Note: Change is calculated from seasonally adjusted January average.
Income groups are based on the median income in the ZIP codes where the
transactions occurred.·Source: Analysis of credit card transaction data
from Womply by Opportunity Insights

At the ZIP code level for small businesses, the steepest declines in
revenues and hours worked have been in the highest-income neighborhoods.
That's a pattern that can't fully be explained by differences in
coronavirus cases.

In the ZIP code where Ms. Craig worked, near Lincoln Center,
small-business revenue fell by 72 percent at the lowest point. It's
still down by half.

\hypertarget{the-rare-service-sector-recession}{%
\subsection{The rare service-sector
recession}\label{the-rare-service-sector-recession}}

In past recessions, the service sector has been one of the most
resilient parts of the economy. In down times, consumers typically cut
back on big durable goods, like a new washing machine or an upgraded
car. But while you can drive your car a little longer, you may not be
able to stretch out your next trip to the dry cleaners for a year or
two.

The restaurant industry has even been the place where laid-off workers
in other parts of the economy have found work in the past.

So we have never seen anything
\href{https://www.dropbox.com/s/fd1i4r6xqhfjlfx/The\%20Coronavirus\%20crisis\%20will\%20be\%20the\%20first\%20services\%20recession.pdf?dl=0}{that
looks quite like this service-sector recession} --- one where the
bartenders lost their jobs before the construction workers, where
previously thriving restaurants and salons have experienced the steepest
losses.

\hypertarget{spending-on-services-often-steady-during-recessions-has-fallen-sharply}{%
\subsubsection{Spending on services, often steady during recessions, has
fallen
sharply}\label{spending-on-services-often-steady-during-recessions-has-fallen-sharply}}

\hypertarget{personal-consumption-expenditures-seasonally-adjusted-2012-dollars}{%
\paragraph{Personal consumption expenditures, seasonally adjusted 2012
dollars}\label{personal-consumption-expenditures-seasonally-adjusted-2012-dollars}}

\$10 trillion

Great

Recession

\$8

Services

\$6

\$4

Nondurable goods

\$2

Durable goods

2002

2004

2006

2008

2010

2012

2014

2016

2018

2020

\$10

trillion

Great

Recession

\$8

Services

\$6

\$4

Nondurable goods

\$2

Durable goods

'02

'04

'06

'08

'10

'12

'14

'16

'18

'20

Source: U.S. Bureau of Economic Analysis via FRED, Federal Reserve Bank
of St. Louis

The service sector had also been expanding over time,
\href{https://www.nytimes3xbfgragh.onion/2019/01/11/upshot/big-cities-low-skilled-workers-wages.html}{replacing
blue-collar jobs in manufacturing} that were more stable and paid more.
Especially in big, expensive cities, the vast service sector is now the
place where the rich and the poor meet.

``What we've seen with rising inequality of the last few decades is that
more and more modest-income individuals survive because they're serving
where the consumption has been,'' said Lawrence Katz, an economist at
Harvard, who has reviewed his colleagues' findings. And that
consumption, he added, has been in the hands of households at the top.

If we'd had this same kind of economic shock 50 years ago, Mr. Katz
said, the magnitude of the ripple effects from the rich to the poor
would have been much smaller. There simply weren't as many links between
them. (Fifty years ago, the rich also couldn't have counted on working
from home, keeping their incomes intact.)

\hypertarget{low-wage-workers-in-the-richest-neighborhoods-have-had-the-biggest-drop-in-employment}{%
\subsubsection{Low-wage workers in the richest neighborhoods have had
the biggest drop in
employment}\label{low-wage-workers-in-the-richest-neighborhoods-have-had-the-biggest-drop-in-employment}}

\hypertarget{change-in-employment-of-low-wage-workers-during-the-pandemic}{%
\paragraph{Change in employment of low-wage workers during the
pandemic}\label{change-in-employment-of-low-wage-workers-during-the-pandemic}}

March 1

April 1

May 1

May 24

First stimulus

checks received

Half of states in

process of reopening

--10\%

--20\%

--30\%

ZIP CODE INCOME LEVEL

Bottom middle 25\%

Bottom 25\%

Top middle 25\%

--40\%

Top 25\%

--50\%

March 1

April 1

May 1

May 24

--10\%

--20\%

ZIP CODE

INCOME

LEVEL

Bottom

middle

--30\%

Bottom

25\%

Top

middle

--40\%

Top 25\%

--50\%

First stimulus

checks received

Half of states in

process of reopening

Note: Change calculated from January average. Income groups are based on
the median income in the ZIP codes where the workplaces are located. The
median annual income of the workers in the data sample is approximately
\$25,000.·Source: Analysis of payroll data from Earnin and timecard data
from Homebase by Opportunity Insights

Now, cities like Washington that were relatively unscathed by the Great
Recession --- thanks to their high median incomes and all their service
jobs --- stand to be hurt far more deeply in the coronavirus recession.
Initial unemployment data bears this out.

Through April, Washington lost 10 percent of its jobs. During the Great
Recession, the city increased employment by 3 percent.

San Francisco and San Mateo counties have lost 16 percent of their jobs
during the pandemic, roughly in line with job losses nationwide. During
the previous recession, those counties gained jobs while employment in
the rest of the country fell 3 percent over all and nearly 5 percent in
the poorest counties. Unemployment was even more uneven in the 2001 and
1991 recessions, with steeper job losses in poorer counties.

In other words, in good times --- or even in more typical downturns ---
proximity to the rich affords lower-wage workers a higher degree of job
security. In this peculiar coronavirus moment, that arrangement appears
remarkably precarious,
\href{https://www.nytimes3xbfgragh.onion/2020/05/09/us/unemployment-coronavirus-women.html}{particularly
for women} and
\href{https://www.nytimes3xbfgragh.onion/interactive/2020/05/13/upshot/coronavirus-america-job-losses-slowing-tracker.html}{black
and Hispanic workers} disproportionately employed in the service sector.

\href{https://www.nytimes3xbfgragh.onion/news-event/coronavirus?action=click\&pgtype=Article\&state=default\&region=MAIN_CONTENT_3\&context=storylines_faq}{}

\hypertarget{the-coronavirus-outbreak-}{%
\subsubsection{The Coronavirus Outbreak
›}\label{the-coronavirus-outbreak-}}

\hypertarget{frequently-asked-questions}{%
\paragraph{Frequently Asked
Questions}\label{frequently-asked-questions}}

Updated August 17, 2020

\begin{itemize}
\item ~
  \hypertarget{why-does-standing-six-feet-away-from-others-help}{%
  \paragraph{Why does standing six feet away from others
  help?}\label{why-does-standing-six-feet-away-from-others-help}}

  \begin{itemize}
  \tightlist
  \item
    The coronavirus spreads primarily through droplets from your mouth
    and nose, especially when you cough or sneeze. The C.D.C., one of
    the organizations using that measure,
    \href{https://www.nytimes3xbfgragh.onion/2020/04/14/health/coronavirus-six-feet.html?action=click\&pgtype=Article\&state=default\&region=MAIN_CONTENT_3\&context=storylines_faq}{bases
    its recommendation of six feet} on the idea that most large droplets
    that people expel when they cough or sneeze will fall to the ground
    within six feet. But six feet has never been a magic number that
    guarantees complete protection. Sneezes, for instance, can launch
    droplets a lot farther than six feet,
    \href{https://jamanetwork.com/journals/jama/fullarticle/2763852}{according
    to a recent study}. It's a rule of thumb: You should be safest
    standing six feet apart outside, especially when it's windy. But
    keep a mask on at all times, even when you think you're far enough
    apart.
  \end{itemize}
\item ~
  \hypertarget{i-have-antibodies-am-i-now-immune}{%
  \paragraph{I have antibodies. Am I now
  immune?}\label{i-have-antibodies-am-i-now-immune}}

  \begin{itemize}
  \tightlist
  \item
    As of right
    now,\href{https://www.nytimes3xbfgragh.onion/2020/07/22/health/covid-antibodies-herd-immunity.html?action=click\&pgtype=Article\&state=default\&region=MAIN_CONTENT_3\&context=storylines_faq}{that
    seems likely, for at least several months.} There have been
    frightening accounts of people suffering what seems to be a second
    bout of Covid-19. But experts say these patients may have a
    drawn-out course of infection, with the virus taking a slow toll
    weeks to months after initial exposure. People infected with the
    coronavirus typically
    \href{https://www.nature.com/articles/s41586-020-2456-9}{produce}
    immune molecules called antibodies, which are
    \href{https://www.nytimes3xbfgragh.onion/2020/05/07/health/coronavirus-antibody-prevalence.html?action=click\&pgtype=Article\&state=default\&region=MAIN_CONTENT_3\&context=storylines_faq}{protective
    proteins made in response to an
    infection}\href{https://www.nytimes3xbfgragh.onion/2020/05/07/health/coronavirus-antibody-prevalence.html?action=click\&pgtype=Article\&state=default\&region=MAIN_CONTENT_3\&context=storylines_faq}{.
    These antibodies may} last in the body
    \href{https://www.nature.com/articles/s41591-020-0965-6}{only two to
    three months}, which may seem worrisome, but that's perfectly normal
    after an acute infection subsides, said Dr. Michael Mina, an
    immunologist at Harvard University. It may be possible to get the
    coronavirus again, but it's highly unlikely that it would be
    possible in a short window of time from initial infection or make
    people sicker the second time.
  \end{itemize}
\item ~
  \hypertarget{im-a-small-business-owner-can-i-get-relief}{%
  \paragraph{I'm a small-business owner. Can I get
  relief?}\label{im-a-small-business-owner-can-i-get-relief}}

  \begin{itemize}
  \tightlist
  \item
    The
    \href{https://www.nytimes3xbfgragh.onion/article/small-business-loans-stimulus-grants-freelancers-coronavirus.html?action=click\&pgtype=Article\&state=default\&region=MAIN_CONTENT_3\&context=storylines_faq}{stimulus
    bills enacted in March} offer help for the millions of American
    small businesses. Those eligible for aid are businesses and
    nonprofit organizations with fewer than 500 workers, including sole
    proprietorships, independent contractors and freelancers. Some
    larger companies in some industries are also eligible. The help
    being offered, which is being managed by the Small Business
    Administration, includes the Paycheck Protection Program and the
    Economic Injury Disaster Loan program. But lots of folks have
    \href{https://www.nytimes3xbfgragh.onion/interactive/2020/05/07/business/small-business-loans-coronavirus.html?action=click\&pgtype=Article\&state=default\&region=MAIN_CONTENT_3\&context=storylines_faq}{not
    yet seen payouts.} Even those who have received help are confused:
    The rules are draconian, and some are stuck sitting on
    \href{https://www.nytimes3xbfgragh.onion/2020/05/02/business/economy/loans-coronavirus-small-business.html?action=click\&pgtype=Article\&state=default\&region=MAIN_CONTENT_3\&context=storylines_faq}{money
    they don't know how to use.} Many small-business owners are getting
    less than they expected or
    \href{https://www.nytimes3xbfgragh.onion/2020/06/10/business/Small-business-loans-ppp.html?action=click\&pgtype=Article\&state=default\&region=MAIN_CONTENT_3\&context=storylines_faq}{not
    hearing anything at all.}
  \end{itemize}
\item ~
  \hypertarget{what-are-my-rights-if-i-am-worried-about-going-back-to-work}{%
  \paragraph{What are my rights if I am worried about going back to
  work?}\label{what-are-my-rights-if-i-am-worried-about-going-back-to-work}}

  \begin{itemize}
  \tightlist
  \item
    Employers have to provide
    \href{https://www.osha.gov/SLTC/covid-19/standards.html}{a safe
    workplace} with policies that protect everyone equally.
    \href{https://www.nytimes3xbfgragh.onion/article/coronavirus-money-unemployment.html?action=click\&pgtype=Article\&state=default\&region=MAIN_CONTENT_3\&context=storylines_faq}{And
    if one of your co-workers tests positive for the coronavirus, the
    C.D.C.} has said that
    \href{https://www.cdc.gov/coronavirus/2019-ncov/community/guidance-business-response.html}{employers
    should tell their employees} -\/- without giving you the sick
    employee's name -\/- that they may have been exposed to the virus.
  \end{itemize}
\item ~
  \hypertarget{what-is-school-going-to-look-like-in-september}{%
  \paragraph{What is school going to look like in
  September?}\label{what-is-school-going-to-look-like-in-september}}

  \begin{itemize}
  \tightlist
  \item
    It is unlikely that many schools will return to a normal schedule
    this fall, requiring the grind of
    \href{https://www.nytimes3xbfgragh.onion/2020/06/05/us/coronavirus-education-lost-learning.html?action=click\&pgtype=Article\&state=default\&region=MAIN_CONTENT_3\&context=storylines_faq}{online
    learning},
    \href{https://www.nytimes3xbfgragh.onion/2020/05/29/us/coronavirus-child-care-centers.html?action=click\&pgtype=Article\&state=default\&region=MAIN_CONTENT_3\&context=storylines_faq}{makeshift
    child care} and
    \href{https://www.nytimes3xbfgragh.onion/2020/06/03/business/economy/coronavirus-working-women.html?action=click\&pgtype=Article\&state=default\&region=MAIN_CONTENT_3\&context=storylines_faq}{stunted
    workdays} to continue. California's two largest public school
    districts --- Los Angeles and San Diego --- said on July 13, that
    \href{https://www.nytimes3xbfgragh.onion/2020/07/13/us/lausd-san-diego-school-reopening.html?action=click\&pgtype=Article\&state=default\&region=MAIN_CONTENT_3\&context=storylines_faq}{instruction
    will be remote-only in the fall}, citing concerns that surging
    coronavirus infections in their areas pose too dire a risk for
    students and teachers. Together, the two districts enroll some
    825,000 students. They are the largest in the country so far to
    abandon plans for even a partial physical return to classrooms when
    they reopen in August. For other districts, the solution won't be an
    all-or-nothing approach.
    \href{https://bioethics.jhu.edu/research-and-outreach/projects/eschool-initiative/school-policy-tracker/}{Many
    systems}, including the nation's largest, New York City, are
    devising
    \href{https://www.nytimes3xbfgragh.onion/2020/06/26/us/coronavirus-schools-reopen-fall.html?action=click\&pgtype=Article\&state=default\&region=MAIN_CONTENT_3\&context=storylines_faq}{hybrid
    plans} that involve spending some days in classrooms and other days
    online. There's no national policy on this yet, so check with your
    municipal school system regularly to see what is happening in your
    community.
  \end{itemize}
\end{itemize}

``It's completely a house of cards,'' said Ai-jen Poo, the executive
director of the National Domestic Workers Alliance. ``So much of our
essential work force that keeps us safe and literally has kept this
country from collapsing are poverty-wage jobs that were completely
invisible to most people before the pandemic.''

In the restaurant industry, the thread connecting the rich and the poor
is clearer. Higher-income diners hand servers their primary income.

``They are directly, directly, directly reliant on those tips,'' said
Saru Jayaraman, the president of One Fair Wage, a group pushing to end
sub-minimum wages for tipped workers. ``They are so at the mercy of
those upper classes.''

These workers have learned to be familiar with the rhythms of rich
consumption. Bad weather --- too hot, too cold, too rainy --- means
fewer tips. January and February, after the glut of holiday spending,
bring fewer hours and less money. For hotel workers, incomes are tied to
convention season. For bartenders, it may be the theater calendar.

\hypertarget{waiting-for-the-rich-to-spend-again}{%
\subsection{Waiting for the rich to spend
again}\label{waiting-for-the-rich-to-spend-again}}

In recent weeks, spending by the poor has nearly rebounded to pre-crisis
levels, thanks to federal stimulus checks --- low-income consumption
shot up after April 15 the moment they were deposited --- and to
\href{https://www.nytimes3xbfgragh.onion/interactive/2020/04/23/business/economy/unemployment-benefits-stimulus-coronavirus.html}{expanded
unemployment benefits}. But the jobless rate remains at its highest
level since the Great Depression. High-income spending has been much
slower to return. Some consumption has also shifted online, which
doesn't help local businesses.

For Washington policymakers, it is hard to coax more spending out of
people wary of leaving their homes, and to steer that spending to the
businesses and workers affected most.

``If the underlying problem is that people are afraid of interacting in
close proximity, and they're afraid to go shopping in certain ways, then
the only way to get things back to normal is going to be to solve the
public health problem,'' said Mr. Friedman, an economist at Brown and a
researcher on the project.

That may mean expanding the safety net for low-wage workers, the
researchers suggest, to help them survive until that moment comes.

Image

Patricia Namyalo, a restaurant worker in Washington, has been furloughed
because of the pandemic. ``I know how this industry operates,'' she
said. ``I'm not expecting much. I see a lot of doom
ahead.''Credit...Justin T. Gellerson for The New York Times

Patricia Namyalo, a server in a hotel restaurant on Capitol Hill in
Washington, is gloomy about what's ahead. She recalls when business
began to dwindle in early March, before the city's shutdown went into
effect, and well before members of Congress, who sometimes dine at the
hotel, recessed for the crisis.

``There were days when we went to work and had a total of eight tables
come in for brunch,'' said Ms. Namyalo, a 38-year-old immigrant from
Uganda. ``And that's supposed to be shared among four people.''

No one wanted to go home because they were guaranteed \$11 an hour in
her unionized hotel, and they had to keep those hours in an empty
restaurant to make up for lost tips. In retrospect, Ms. Namyalo believes
her customers knew what was coming. They were following the news from
Asia, the headlines out of Europe.

``The upper class was already aware that America was going to follow
suit,'' she said. ``And people like myself --- I didn't quite get it at
the time.''

She suspects the same is true now. Higher-income consumers know they
won't be back to their old levels of dining out or spending any time
soon, even with all the talk of cities reopening. Meanwhile, lower-wage
workers wait, hoping for the call back to work.

\href{https://www.nytimes3xbfgragh.onion/newsletters/upshot}{\emph{Subscribe}}
\emph{to the Upshot newsletter.}

Advertisement

\protect\hyperlink{after-bottom}{Continue reading the main story}

\hypertarget{site-index}{%
\subsection{Site Index}\label{site-index}}

\hypertarget{site-information-navigation}{%
\subsection{Site Information
Navigation}\label{site-information-navigation}}

\begin{itemize}
\tightlist
\item
  \href{https://help.nytimes3xbfgragh.onion/hc/en-us/articles/115014792127-Copyright-notice}{©~2020~The
  New York Times Company}
\end{itemize}

\begin{itemize}
\tightlist
\item
  \href{https://www.nytco.com/}{NYTCo}
\item
  \href{https://help.nytimes3xbfgragh.onion/hc/en-us/articles/115015385887-Contact-Us}{Contact
  Us}
\item
  \href{https://www.nytco.com/careers/}{Work with us}
\item
  \href{https://nytmediakit.com/}{Advertise}
\item
  \href{http://www.tbrandstudio.com/}{T Brand Studio}
\item
  \href{https://www.nytimes3xbfgragh.onion/privacy/cookie-policy\#how-do-i-manage-trackers}{Your
  Ad Choices}
\item
  \href{https://www.nytimes3xbfgragh.onion/privacy}{Privacy}
\item
  \href{https://help.nytimes3xbfgragh.onion/hc/en-us/articles/115014893428-Terms-of-service}{Terms
  of Service}
\item
  \href{https://help.nytimes3xbfgragh.onion/hc/en-us/articles/115014893968-Terms-of-sale}{Terms
  of Sale}
\item
  \href{https://spiderbites.nytimes3xbfgragh.onion}{Site Map}
\item
  \href{https://help.nytimes3xbfgragh.onion/hc/en-us}{Help}
\item
  \href{https://www.nytimes3xbfgragh.onion/subscription?campaignId=37WXW}{Subscriptions}
\end{itemize}
