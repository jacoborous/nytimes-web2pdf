Sections

SEARCH

\protect\hyperlink{site-content}{Skip to
content}\protect\hyperlink{site-index}{Skip to site index}

\href{https://myaccount.nytimes3xbfgragh.onion/auth/login?response_type=cookie\&client_id=vi}{}

\href{https://www.nytimes3xbfgragh.onion/section/todayspaper}{Today's
Paper}

\href{/section/upshot}{The Upshot}\textbar{}Hospitals Sued to Keep
Prices Secret. They Lost.

\url{https://nyti.ms/2NlG8qj}

\begin{itemize}
\item
\item
\item
\item
\item
\item
\end{itemize}

Advertisement

\protect\hyperlink{after-top}{Continue reading the main story}

Upshot

Supported by

\protect\hyperlink{after-sponsor}{Continue reading the main story}

\hypertarget{hospitals-sued-to-keep-prices-secret-they-lost}{%
\section{Hospitals Sued to Keep Prices Secret. They
Lost.}\label{hospitals-sued-to-keep-prices-secret-they-lost}}

The decision was a victory for the Trump administration, which sees
pressure from patients as a way to control health costs.

By \href{https://www.nytimes3xbfgragh.onion/by/sarah-kliff}{Sarah Kliff}
and
\href{https://www.nytimes3xbfgragh.onion/by/margot-sanger-katz}{Margot
Sanger-Katz}

\begin{itemize}
\item
  June 23, 2020
\item
  \begin{itemize}
  \item
  \item
  \item
  \item
  \item
  \item
  \end{itemize}
\end{itemize}

\includegraphics{https://static01.graylady3jvrrxbe.onion/images/2020/06/23/upshot/23up-transparency/23up-transparency-articleLarge.jpg?quality=75\&auto=webp\&disable=upscale}

A federal judge has upheld a Trump administration policy that requires
hospitals and health insurers to publish their negotiated prices for
health services, numbers that are typically kept secret.

The policy is part of a major push by the administration to improve
transparency in health care. Insurers and health providers usually
negotiate deals behind closed doors, and patients rarely know the cost
of services until after the fact.

Administration officials said more price transparency would lead to
lower and more predictable prices in an industry that has huge ranges in
what insurers pay for services.
\href{https://www.nytimes3xbfgragh.onion/2019/04/30/upshot/health-care-huge-price-discrepancies.html}{A
simple blood test}, for example, can cost \$11 or \$1,000.
\href{https://www.nytimes3xbfgragh.onion/2020/06/16/upshot/coronavirus-test-cost-varies-widely.html}{Coronavirus
tests} show a similar variation, with prices from \$27 to \$2,315.

But
\href{https://www.nytimes3xbfgragh.onion/2019/12/04/health/hospitals-trump-prices-transparency.html}{in
a lawsuit}, the American Hospital Association said the administration
did not have the legal authority to require the publication of
negotiated prices, arguing that the publication of the prices could have
perverse effects. On Tuesday, the judge, Carl Nichols, disagreed.

In
\href{https://ecf.dcd.uscourts.gov/cgi-bin/show_public_doc?2019cv3619-35}{his
decision}, Judge Nichols found that the hospitals were ``attacking
transparency measures generally'' in a bid to limit patients' insight
into medical prices.

``Hospitals may be affected by market changes and need to respond to a
market where consumers are more empowered,'' he wrote, stating that was
not reason enough to ``make the rule unlawful.''

Judge Nichols, who was appointed to the D.C. District Court last year,
also rejected the hospitals' other arguments: that the new rules would
create overwhelming administrative burdens and that increased
transparency might actually drive up prices.

``Traditional economic analysis suggested to the agency that informed
customers would put pressure on providers to lower costs and increase
the quality of care,'' Judge Nichols wrote.

Among health economists and other experts, the effects of price
transparency policies remain unsettled. The Trump administration has
argued that published prices will help empower individual patients as
well as employers that buy health insurance for their workers, creating
market pressure to discourage overcharging.

But research on price transparency in health care, which is limited, has
not shown large effects: A study of a New Hampshire law requiring
published prices for common services showed very
\href{https://www.mitpressjournals.org/doi/abs/10.1162/rest_a_00765}{modest
price reductions}. And a body of research from other fields, including
Chilean gasoline and
\href{https://www.nytimes3xbfgragh.onion/2019/06/24/upshot/transparency-medical-prices-could-backfire.html}{Danish
ready-mix concrete}, has found that publishing negotiated prices can
sometimes backfire in markets where there are few competitors, raising
prices.

Judge Nichols conceded that ``the evidence in the record is not
definitive'' in proving that transparency lowers prices, but that it was
``more persuasive than a decades-old case study involving Danish
ready-mixed concrete contracts.''

The hospital rule is part of the administration's bid to control health
costs through transparency, an effort that has become a health policy
priority for President Trump. The hospital rule was preceded by an
executive order on price transparency in health care
\href{https://www.nytimes3xbfgragh.onion/2019/06/24/upshot/health-care-price-transparency-trump.html}{unveiled
at a White House event} where patients spoke about their experiences
with surprise medical bills.

It is not the only part of that transparency effort to encounter legal
obstacles. A federal appeals court has
\href{https://www.nytimes3xbfgragh.onion/aponline/2020/06/17/business/bc-us-drug-prices-tv-ads.html}{thrown
out} another rule that would have required drug makers to include the
price of medications in television advertisements.

Alex Azar, the secretary of Health and Human Services, applauded the
court's decision: ``With today's win, we will continue delivering on the
president's promise to give patients easy access to health care prices.
Especially when patients are seeking needed care during a public health
emergency, it is more important than ever that they have ready access to
the actual prices of health care services.''

The hospital association said it would appeal the decision. ``The
proposal does nothing to help patients understand their out-of-pocket
costs,'' said Melinda Hatton, a senior vice president and general
counsel for the association. ``It also imposes significant burdens on
hospitals at a time when resources are stretched thin.''

The price transparency rule is scheduled to go into effect in January.

Advertisement

\protect\hyperlink{after-bottom}{Continue reading the main story}

\hypertarget{site-index}{%
\subsection{Site Index}\label{site-index}}

\hypertarget{site-information-navigation}{%
\subsection{Site Information
Navigation}\label{site-information-navigation}}

\begin{itemize}
\tightlist
\item
  \href{https://help.nytimes3xbfgragh.onion/hc/en-us/articles/115014792127-Copyright-notice}{©~2020~The
  New York Times Company}
\end{itemize}

\begin{itemize}
\tightlist
\item
  \href{https://www.nytco.com/}{NYTCo}
\item
  \href{https://help.nytimes3xbfgragh.onion/hc/en-us/articles/115015385887-Contact-Us}{Contact
  Us}
\item
  \href{https://www.nytco.com/careers/}{Work with us}
\item
  \href{https://nytmediakit.com/}{Advertise}
\item
  \href{http://www.tbrandstudio.com/}{T Brand Studio}
\item
  \href{https://www.nytimes3xbfgragh.onion/privacy/cookie-policy\#how-do-i-manage-trackers}{Your
  Ad Choices}
\item
  \href{https://www.nytimes3xbfgragh.onion/privacy}{Privacy}
\item
  \href{https://help.nytimes3xbfgragh.onion/hc/en-us/articles/115014893428-Terms-of-service}{Terms
  of Service}
\item
  \href{https://help.nytimes3xbfgragh.onion/hc/en-us/articles/115014893968-Terms-of-sale}{Terms
  of Sale}
\item
  \href{https://spiderbites.nytimes3xbfgragh.onion}{Site Map}
\item
  \href{https://help.nytimes3xbfgragh.onion/hc/en-us}{Help}
\item
  \href{https://www.nytimes3xbfgragh.onion/subscription?campaignId=37WXW}{Subscriptions}
\end{itemize}
