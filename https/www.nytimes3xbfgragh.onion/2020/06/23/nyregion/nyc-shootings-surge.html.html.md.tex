Sections

SEARCH

\protect\hyperlink{site-content}{Skip to
content}\protect\hyperlink{site-index}{Skip to site index}

\href{https://www.nytimes3xbfgragh.onion/section/nyregion}{New York}

\href{https://myaccount.nytimes3xbfgragh.onion/auth/login?response_type=cookie\&client_id=vi}{}

\href{https://www.nytimes3xbfgragh.onion/section/todayspaper}{Today's
Paper}

\href{/section/nyregion}{New York}\textbar{}Gun Violence Spikes in
N.Y.C., Intensifying Debate Over Policing

\url{https://nyti.ms/2BAAY6W}

\begin{itemize}
\item
\item
\item
\item
\item
\end{itemize}

\href{https://www.nytimes3xbfgragh.onion/news-event/george-floyd-protests-minneapolis-new-york-los-angeles?action=click\&pgtype=Article\&state=default\&region=TOP_BANNER\&context=storylines_menu}{Race
and America}

\begin{itemize}
\tightlist
\item
  \href{https://www.nytimes3xbfgragh.onion/2020/07/26/us/protests-portland-seattle-trump.html?action=click\&pgtype=Article\&state=default\&region=TOP_BANNER\&context=storylines_menu}{Protesters
  Return to Other Cities}
\item
  \href{https://www.nytimes3xbfgragh.onion/2020/07/24/us/portland-oregon-protests-white-race.html?action=click\&pgtype=Article\&state=default\&region=TOP_BANNER\&context=storylines_menu}{Portland
  at the Center}
\item
  \href{https://www.nytimes3xbfgragh.onion/2020/07/23/podcasts/the-daily/portland-protests.html?action=click\&pgtype=Article\&state=default\&region=TOP_BANNER\&context=storylines_menu}{Podcast:
  Showdown in Portland}
\item
  \href{https://www.nytimes3xbfgragh.onion/interactive/2020/07/16/us/black-lives-matter-protests-louisville-breonna-taylor.html?action=click\&pgtype=Article\&state=default\&region=TOP_BANNER\&context=storylines_menu}{45
  Days in Louisville}
\end{itemize}

Advertisement

\protect\hyperlink{after-top}{Continue reading the main story}

Supported by

\protect\hyperlink{after-sponsor}{Continue reading the main story}

\hypertarget{gun-violence-spikes-in-nyc-intensifying-debate-over-policing}{%
\section{Gun Violence Spikes in N.Y.C., Intensifying Debate Over
Policing}\label{gun-violence-spikes-in-nyc-intensifying-debate-over-policing}}

More than a dozen people have been fatally shot, including a teenager at
her college graduation party and a clothing designer who was washing his
car.

\includegraphics{https://static01.graylady3jvrrxbe.onion/images/2020/06/22/nyregion/22nyunrest-shootings/22nyunrest-shootings-articleLarge.jpg?quality=75\&auto=webp\&disable=upscale}

\href{https://www.nytimes3xbfgragh.onion/by/ashley-southall}{\includegraphics{https://static01.graylady3jvrrxbe.onion/images/2018/02/20/multimedia/author-ashley-southall/author-ashley-southall-thumbLarge.jpg}}\href{https://www.nytimes3xbfgragh.onion/by/neil-macfarquhar}{\includegraphics{https://static01.graylady3jvrrxbe.onion/images/2018/10/15/multimedia/author-neil-macfarquhar/author-neil-macfarquhar-thumbLarge.png}}

By \href{https://www.nytimes3xbfgragh.onion/by/ashley-southall}{Ashley
Southall} and
\href{https://www.nytimes3xbfgragh.onion/by/neil-macfarquhar}{Neil
MacFarquhar}

\begin{itemize}
\item
  Published June 23, 2020Updated July 17, 2020
\item
  \begin{itemize}
  \item
  \item
  \item
  \item
  \item
  \end{itemize}
\end{itemize}

It has been nearly a quarter century since
\href{https://www.nytimes3xbfgragh.onion/2020/07/15/nyregion/nyc-shootings.html}{New
York City} experienced as much
\href{https://www.nytimes3xbfgragh.onion/2020/07/15/nyregion/nyc-shootings.html}{gun
violence} in the month of June as it has seen this year.

The city logged 125 shootings in the first three weeks of the month,
more than double the number recorded over the same period last year,
police data show. Gunmen opened fire during house parties, barbecues and
dice games, and carried out coldly calculated street executions.

More than a dozen people have been fatally shot, including a teenager at
her college graduation party and
\href{https://www.nydailynews.com/new-york/nyc-crime/ny-man-fatally-shot-brooklyn-gun-violence-soars-20200620-pp3uqtsenbf63n4aei3tw265ou-story.html}{a
clothing designer who was washing his car}. ``You have to go back to
1996 to have a worse start of June,'' Michael LiPetri, the chief of
crime control strategies, said in an interview on Monday.

The rising toll of gun violence has become part of a contentious debate
over the future of policing in the wake of mass protests against police
brutality. Police unions and their supporters have issued shrill
warnings that the city was slipping into a high-crime era reminiscent of
the early 1990s.

The city is not alone. Shootings are on the rise in other big cities
across the country, including Chicago and Minneapolis, a trend that some
conservatives have seized on to argue against the recent demands of
protesters to cut police budgets and rein in officers.

On Monday, Mayor Bill de Blasio announced that the city was sending more
officers into the streets and declared he would not retreat from efforts
to overhaul the Police Department.

\includegraphics{https://static01.graylady3jvrrxbe.onion/images/2020/06/23/nyregion/23nyunrest-shootings2/merlin_173603550_4f4c0b0b-5ec1-408a-ac90-762f68847035-articleLarge.jpg?quality=75\&auto=webp\&disable=upscale}

``We're not going back to the bad old days when there was so much
violence in the city,'' the mayor said at a news conference, ``nor are
we going back to the bad old days where policing was done the wrong way
and, in too many cases, police and community could never connect and
find that mutual respect.''

The mayor's comments came after a particularly bad weekend in which 38
people were shot over 72 hours. The toll continued to mount on Monday,
when a 46-year-old man was shot and killed in Brownsville by a gunman
who ambushed him in the lobby of a public housing building, the police
said. That evening,
\href{https://www.nydailynews.com/new-york/nyc-crime/ny-brooklyn-candlelight-vigil-shooting-20200623-wycautxxpjg5lixa6jvat3kzli-story.html}{five
people were shot} at a candlelight vigil in Crown Heights.

Other cities are seeing similar violence:
\href{https://www.chicagotribune.com/news/breaking/ct-chicago-weekend-violence-shootings-20200622-ghaioius2zhdpdsbsan4k2g2vu-story.html}{In
Chicago, more than 100 people} were shot over the weekend, the most in a
single weekend since 2012, and 14 died. That carnage came just weeks
after 24 people were killed among 85 people shot over the Memorial Day
weekend. Many were caught in crossfire.

In Minneapolis, police said that 111 people have been shot in the four
weeks since the death of George Floyd, a black man who was killed in
police custody, sparked nationwide protests.

Nationally, homicide rates were already rising in 64 large American
cities for the first three months of 2020 over the previous three years,
but on average the pandemic caused them to stall briefly, before ticking
up again in May, said Richard Rosenfeld, a criminologist at the
University of Missouri-St. Louis. Rosenfeld is also a co-author of a
study about homicide rates during the pandemic to be released Thursday
by Arnold Ventures, a philanthropy focused on criminal justice and other
issues.

In
\href{https://www.nytimes3xbfgragh.onion/2020/07/13/nyregion/Davell-Gardner-brooklyn-shooting.html}{New
York}, there were 166 murders through June 21, up from 134 over the same
period last year, the police said.

Some of that increase can be attributed to both the strain of the
pandemic and the recent unrest, although the primary reason cited by
criminologists was the advent of summer --- traditionally a high-crime
season because people are outside for longer and tempers flare in the
heat.

In New York, the police have linked the rise in gun violence to a bail
law enacted this year, which limited judges' ability to keep people in
jail before trial if they had been arrested on certain charges, as well
as the release of thousands of people from jail and prison to help curb
coronavirus. Chief LiPetri said 17 percent of shootings involved people
on probation or parole.

Even veteran observers of the city's crime trends viewed the jump in New
York City as remarkable.

``I have been studying this for a long time. I have never seen that much
of an increase ever,'' said Christopher Herrmann, a professor at the
John Jay College of Criminal Justice who once analyzed crime statistics
for the New York City Police Department.

Mr. Herrmann said the spike in shootings likely stemmed from a
``combination of warmer weather, Covid cabin fever and the traditional
gun violence that we see in June, July and August.''

Some criminologists said there is a precedent for crime rising after
unrest over police killings.

Homicides rose nationally in the aftermath of the unrest ignited by the
2014 killing of Michael Brown in Ferguson, Mo. It went up 15 to 20
percent in the largest cities before subsiding again after two years,
Mr. Rosenfeld said. When discontent with the police ran high, people
were less likely to call them, deciding to settle matters themselves and
driving violence up, he said.

``There is every reason to believe that we will see an increase in
homicides and other violent crimes associated with the current unrest,''
he said.

Eugene O'Donnell, who is also a professor at John Jay, said that while
the rise in shootings signaled a collapse in public safety in New York
neighborhoods most affected by violence, it was too soon to predict a
doomsday scenario.

``The sum total of all of that is grounds to worry, but I have happily
sat back and watched them predict Armageddon that hasn't come for
years,'' Professor O'Donnell, a former city police officer and
prosecutor, said. ``Whether shootings go up or down, I think things have
dramatically changed for the worst in communities affected by this
problem.''

Some of those neighborhoods are now seeing an influx of officers
deployed as part of the police department's annual Summer All Out
strategy, in which officers in desk jobs and other duties are redeployed
to the street to discourage violence.

About 300 officers have been sent this year to neighborhoods like East
Harlem in Manhattan, Mott Haven in the Bronx, East New York in Brooklyn
and Jamaica in Queens. Those neighborhoods struggle with
\href{https://www.nytimes3xbfgragh.onion/2019/12/05/nyregion/queens-gang-shooting-aamir-griffin.html}{conflicts
between crews and gangs} that police say fuel half of the shootings in
the city.

So far this year, police have solved just 28 percent of shootings with
an arrest, Chief LiPetri said, even though that proportion is normally
around half.

Often, investigators know who is responsible for a shooting but lack
sufficient evidence to make an arrest. Victims often refuse to cooperate
with police investigations, and witnesses are afraid of retaliation if
they come forward.

The state's new bail law and the pandemic have made it more difficult to
build those cases, police officials have said, because it is harder for
prosecutors to leverage their power to keep people behind bars.

Chief LiPetri said
\href{https://www.nytimes3xbfgragh.onion/2020/06/22/nyregion/coronavirus-new-york-courts.html}{a
slowdown in court proceedings} because of the virus has also stymied the
efforts to curb violence. Although police arrested more people for gun
possession this year, he said many were released because their cases
could not be presented to a grand jury within six days, the statutory
limit for holding people in custody without an indictment.

About 800 gun cases are still waiting to be presented to grand juries,
he said, and 40 percent of people arrested on gun possession charges had
been released without bail so far this year.

In the past the New York police have relied on plainclothes ``anti-crime
units'' to proactively hunt for people believed to be carrying illegal
guns on the streets, but their aggressive tactics had led to many
complaints and several police shootings. Commissioner Dermot F. Shea
\href{https://www.nytimes3xbfgragh.onion/2020/06/15/nyregion/nypd-plainclothes-cops.html}{disbanded
the units last week}.

The city is instead leaning more heavily on nonprofits that employ
``violence interrupters" --- young men and women with past ties to gangs
who intervene to try to stop street conflicts from escalating.

One of the nonprofits in Brownsville was scrambling on Monday to figure
out why a 46-year-old man was killed and who might want to retaliate.
Anthony Newerls, the program director at Brownsville In, Violence Out,
said it was the second fatal shooting in the neighborhood in 10 days.
``We are completely overwhelmed,'' he said.

Not all of the areas of the city that have struggled with shootings in
the past have seen increases. Iesha Sekou's group, Street Corner
Resources, oversees a stretch of Harlem that has not seen a shooting
since April 25.

Ms. Sekou said the changes brought on by the pandemic as well as the
weather have created a ripe environment for shootings to unfold. Her
group has focused on keeping tabs on who is getting out of jail or
prison, particularly if they have been involved in shootings, because
they might have unfinished business.

Usually, there is a party to celebrate the occasion, and she shows up.
"We would go over there and give out masks and gloves, and we would say,
`We need y'all to make sure there's no violence,''' she said.

Advertisement

\protect\hyperlink{after-bottom}{Continue reading the main story}

\hypertarget{site-index}{%
\subsection{Site Index}\label{site-index}}

\hypertarget{site-information-navigation}{%
\subsection{Site Information
Navigation}\label{site-information-navigation}}

\begin{itemize}
\tightlist
\item
  \href{https://help.nytimes3xbfgragh.onion/hc/en-us/articles/115014792127-Copyright-notice}{©~2020~The
  New York Times Company}
\end{itemize}

\begin{itemize}
\tightlist
\item
  \href{https://www.nytco.com/}{NYTCo}
\item
  \href{https://help.nytimes3xbfgragh.onion/hc/en-us/articles/115015385887-Contact-Us}{Contact
  Us}
\item
  \href{https://www.nytco.com/careers/}{Work with us}
\item
  \href{https://nytmediakit.com/}{Advertise}
\item
  \href{http://www.tbrandstudio.com/}{T Brand Studio}
\item
  \href{https://www.nytimes3xbfgragh.onion/privacy/cookie-policy\#how-do-i-manage-trackers}{Your
  Ad Choices}
\item
  \href{https://www.nytimes3xbfgragh.onion/privacy}{Privacy}
\item
  \href{https://help.nytimes3xbfgragh.onion/hc/en-us/articles/115014893428-Terms-of-service}{Terms
  of Service}
\item
  \href{https://help.nytimes3xbfgragh.onion/hc/en-us/articles/115014893968-Terms-of-sale}{Terms
  of Sale}
\item
  \href{https://spiderbites.nytimes3xbfgragh.onion}{Site Map}
\item
  \href{https://help.nytimes3xbfgragh.onion/hc/en-us}{Help}
\item
  \href{https://www.nytimes3xbfgragh.onion/subscription?campaignId=37WXW}{Subscriptions}
\end{itemize}
