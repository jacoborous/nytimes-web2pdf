Sections

SEARCH

\protect\hyperlink{site-content}{Skip to
content}\protect\hyperlink{site-index}{Skip to site index}

\href{https://myaccount.nytimes3xbfgragh.onion/auth/login?response_type=cookie\&client_id=vi}{}

\href{https://www.nytimes3xbfgragh.onion/section/todayspaper}{Today's
Paper}

\href{/section/opinion}{Opinion}\textbar{}Sex Does Not Mean Gender.
Equating Them Erases Trans Lives.

\url{https://nyti.ms/2B0Nddg}

\begin{itemize}
\item
\item
\item
\item
\item
\end{itemize}

Advertisement

\protect\hyperlink{after-top}{Continue reading the main story}

\href{/section/opinion}{Opinion}

Supported by

\protect\hyperlink{after-sponsor}{Continue reading the main story}

\hypertarget{sex-does-not-mean-gender-equating-them-erases-trans-lives}{%
\section{Sex Does Not Mean Gender. Equating Them Erases Trans
Lives.}\label{sex-does-not-mean-gender-equating-them-erases-trans-lives}}

Embracing the experiences of trans people means leaving old vocabularies
behind.

By Devin Michelle Bunten

Ms. Bunten teaches urban planning at M.I.T.

\begin{itemize}
\item
  June 23, 2020
\item
  \begin{itemize}
  \item
  \item
  \item
  \item
  \item
  \end{itemize}
\end{itemize}

\includegraphics{https://static01.graylady3jvrrxbe.onion/images/2020/06/26/opinion/23bunten1/merlin_162381234_2e047a48-1cb9-417e-b698-2fda980db093-articleLarge.jpg?quality=75\&auto=webp\&disable=upscale}

Men menstruate. Some have even given birth. Women with penises and
prominent larynxes walk the streets and use the ladies' restroom.
Nonbinary people wear binders and use they/them pronouns. It's 2020.

The Trump administration would like to turn back the clock. This month,
the administration
\href{https://www.nytimes3xbfgragh.onion/2020/06/12/us/politics/trump-transgender-rights.html}{finalized
a rule} that would erase nondiscrimination protections for trans people
in the provision of health care. The administration's mode of attack is
linguistic. Trans people live in the space between ``gender'' and
``sex,'' and the new rule aims to erase us by conflating the concepts.

A full embrace of this new trans reality will mean leaving behind old
vocabularies. Some changes are simple: We can speak of trans mothers and
brothers and siblings as easily as of any other family member. Others
are more contested. ``They'' as a singular pronoun is not without its
detractors,
\href{http://itre.cis.upenn.edu/~myl/languagelog/archives/002748.html}{Shakespeare
aside}. And some words will need to be reconfigured entirely. The
``feminine ** products'' aisle offers tampons and pads and diva cups ---
tools for managing the biological function of menstruation. Again, some
men menstruate. So why not simply call these menstrual products?

``Sex'' is a biological framework, a panoply of possibility on its own.
``Sex'' needs precise words like ``male'' and ``female'' and
``intersex'' to describe the origins, components and functions of
bodies. But we can't maintain this precision if we use words about sex
** to describe gender ** --- the social and political roles and
possibilities we take on as women, as men,
\href{https://www.nytimes3xbfgragh.onion/interactive/2018/11/16/magazine/tech-design-instagram-gender.html?action=click\&module=RelatedCoverage\&pgtype=Article\&region=Footer}{as
something else} or
\href{https://www.nytimes3xbfgragh.onion/2019/06/04/magazine/gender-nonbinary.html}{none
of the above}.

That is to say: Stop using ``male'' and ``female'' to refer to men and
women. In fact, stop using sex-based words to refer to people at all.
They're words for bodies, not for people with hearts and souls and
minds.

Anti-trans revanchists have centered their battles in wordplay --- if
you can call it that. J.K. Rowling, in a
\href{https://twitter.com/jk_rowling/status/1269382518362509313}{recent
tweet}, noted that ``people who menstruate'' were once referred to as
``Wumben? Wimpund? Woomud?'' (She meant ``women.'' There's that
wordplay.)

She also argued,
``\href{https://twitter.com/jk_rowling/status/1269389298664701952}{If
sex isn't real,} the lived reality of women globally is erased'' and
``erasing the concept of sex removes the ability of many to meaningfully
discuss their lives.'' Ms. Rowling's linguistic wizardry cloaks her
political goal, to assign gender purely by sex, and therefore relegate
trans-ness to a closet
\href{https://harrypotter.fandom.com/wiki/Cupboard_Under_the_Stairs}{under
the stairs}.

It should be noted that trans people do not generally believe sex is not
real; indeed, discomfort with the sex of our bodies is a frequent
challenge for trans people. Ms. Rowling knows this, since she knows what
the word ``trans'' means. Words hold power, and it's no surprise that
pushback to a rising trans presence has come in the form of definitional
conservatism.

But the battle extends beyond language, and Ms. Rowling's semantic
battle has been taken to new theaters by the Trump administration. From
our
\href{https://www.nytimes3xbfgragh.onion/2018/10/21/us/politics/transgender-trump-administration-sex-definition.html}{schools}
to our
\href{https://www.nytimes3xbfgragh.onion/2020/06/12/us/politics/trump-transgender-rights.html}{hospitals}
to
\href{https://www.nytimes3xbfgragh.onion/2019/12/06/us/politics/trump-transgender-rights.html}{the
federal work force}, the administration has pursued new rules that
define trans people out of existence. This is an attack on trans lives.
As with Ms. Rowling, the language of the proposed rules is the language
of bodies: the social roles of ``man'' and ``woman'' are the only two
available, and we are all assigned one at birth according to our bodies.
(Intersex individuals will note that false binaries are not limited to
social roles.)

Last week, the judiciary offered trans people some relief. The
\href{https://www.supremecourt.gov/opinions/19pdf/17-1618_hfci.pdf}{Supreme
Court ruled}, ``An employer who fires an individual merely for being gay
or transgender violates Title VII,'' which prohibits employers from
discriminating based on sex. Aimee Stephens, a trans woman and a
plaintiff in the case, was fired after notifying her employer she would
be transitioning. As the court argued, she was fired because of her sex.
**** The logic is impeccable. The only difference between a trans woman
and a cisgender woman is the sex assigned to her at birth: Firing a
trans woman but keeping a cis woman \emph{must} be discrimination based
on sex, which is illegal.

In finding for Aimee Stephens, the Supreme Court reinforced the
centrality of bodies to the word ``sex,'' while undermining the
patriarchal belief that our bodies should determine our gender.
Unfortunately, the protections depend on the language of the 1964 Civil
Rights Act and remain as limited as the imaginations of its authors.
While male and female people are protected classes, nonbinary or
genderqueer people may not have enforceable rights --- say, to a
gender-neutral bathroom --- under the act.

Clarity in language provides social and linguistic accommodation for
those of us traditionally denied both. The battle for civil rights
\emph{is} the battle over words. Denying trans people passports because
our gender doesn't match the sex assigned to us at birth limits freedom
of movement. For trans immigrants and
\href{https://www.hrc.org/blog/the-precarious-position-of-transgender-immigrants-and-asylum-seekers}{asylum
seekers}, this restricts access to families abroad. Denying trans people
access to bathrooms on the basis of sex denies us access to public
spaces. (Can you imagine spending a day at school or work without using
the bathroom? If you can't pee, you don't have access.)

When you use words like ``male'' as shorthand for those privileged by
the patriarchy, you leave trans women uncertain whether you have our
backs or --- like the Trump administration and J.K. Rowling --- you are
trying to write us out of existence. It's impossible to dismantle the
patriarchy while wearing a ``pussy hat.''

The anti-trans clique would pursue legal restrictions where nature has
concocted something more anarchic. But we are already here, being trans,
at your job, on your block, in your bathroom. And we deserve no less.
Rooting our social possibilities in our bodies is an abandonment of our
humanity in favor of mere anatomy.

Devin Michelle Bunten (\href{https://twitter.com/devin_mb}{@devin\_mb})
is a writer and an assistant professor of urban economics and housing at
M.I.T.

\emph{The Times is committed to publishing}
\href{https://www.nytimes3xbfgragh.onion/2019/01/31/opinion/letters/letters-to-editor-new-york-times-women.html}{\emph{a
diversity of letters}} \emph{to the editor. We'd like to hear what you
think about this or any of our articles. Here are some}
\href{https://help.nytimes3xbfgragh.onion/hc/en-us/articles/115014925288-How-to-submit-a-letter-to-the-editor}{\emph{tips}}\emph{.
And here's our email:}
\href{mailto:letters@NYTimes.com}{\emph{letters@NYTimes.com}}\emph{.}

\emph{Follow The New York Times Opinion section on}
\href{https://www.facebookcorewwwi.onion/nytopinion}{\emph{Facebook}}\emph{,}
\href{http://twitter.com/NYTOpinion}{\emph{Twitter (@NYTopinion)}}
\emph{and}
\href{https://www.instagram.com/nytopinion/}{\emph{Instagram}}\emph{.}

Advertisement

\protect\hyperlink{after-bottom}{Continue reading the main story}

\hypertarget{site-index}{%
\subsection{Site Index}\label{site-index}}

\hypertarget{site-information-navigation}{%
\subsection{Site Information
Navigation}\label{site-information-navigation}}

\begin{itemize}
\tightlist
\item
  \href{https://help.nytimes3xbfgragh.onion/hc/en-us/articles/115014792127-Copyright-notice}{©~2020~The
  New York Times Company}
\end{itemize}

\begin{itemize}
\tightlist
\item
  \href{https://www.nytco.com/}{NYTCo}
\item
  \href{https://help.nytimes3xbfgragh.onion/hc/en-us/articles/115015385887-Contact-Us}{Contact
  Us}
\item
  \href{https://www.nytco.com/careers/}{Work with us}
\item
  \href{https://nytmediakit.com/}{Advertise}
\item
  \href{http://www.tbrandstudio.com/}{T Brand Studio}
\item
  \href{https://www.nytimes3xbfgragh.onion/privacy/cookie-policy\#how-do-i-manage-trackers}{Your
  Ad Choices}
\item
  \href{https://www.nytimes3xbfgragh.onion/privacy}{Privacy}
\item
  \href{https://help.nytimes3xbfgragh.onion/hc/en-us/articles/115014893428-Terms-of-service}{Terms
  of Service}
\item
  \href{https://help.nytimes3xbfgragh.onion/hc/en-us/articles/115014893968-Terms-of-sale}{Terms
  of Sale}
\item
  \href{https://spiderbites.nytimes3xbfgragh.onion}{Site Map}
\item
  \href{https://help.nytimes3xbfgragh.onion/hc/en-us}{Help}
\item
  \href{https://www.nytimes3xbfgragh.onion/subscription?campaignId=37WXW}{Subscriptions}
\end{itemize}
