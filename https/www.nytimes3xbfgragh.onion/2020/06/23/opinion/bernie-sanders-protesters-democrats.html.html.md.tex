Sections

SEARCH

\protect\hyperlink{site-content}{Skip to
content}\protect\hyperlink{site-index}{Skip to site index}

\href{https://myaccount.nytimes3xbfgragh.onion/auth/login?response_type=cookie\&client_id=vi}{}

\href{https://www.nytimes3xbfgragh.onion/section/todayspaper}{Today's
Paper}

\href{/section/opinion}{Opinion}\textbar{}The Second Defeat of Bernie
Sanders

\url{https://nyti.ms/3dlZIgt}

\begin{itemize}
\item
\item
\item
\item
\item
\item
\end{itemize}

Advertisement

\protect\hyperlink{after-top}{Continue reading the main story}

\href{/section/opinion}{Opinion}

Supported by

\protect\hyperlink{after-sponsor}{Continue reading the main story}

\hypertarget{the-second-defeat-of-bernie-sanders}{%
\section{The Second Defeat of Bernie
Sanders}\label{the-second-defeat-of-bernie-sanders}}

In a revolutionary summer, he may be losing the battle for the future of
the left.

\href{https://www.nytimes3xbfgragh.onion/by/ross-douthat}{\includegraphics{https://static01.graylady3jvrrxbe.onion/images/2018/04/03/opinion/ross-douthat/ross-douthat-thumbLarge.png}}

By \href{https://www.nytimes3xbfgragh.onion/by/ross-douthat}{Ross
Douthat}

Opinion Columnist

\begin{itemize}
\item
  June 23, 2020
\item
  \begin{itemize}
  \item
  \item
  \item
  \item
  \item
  \item
  \end{itemize}
\end{itemize}

\includegraphics{https://static01.graylady3jvrrxbe.onion/images/2020/06/23/opinion/23douthatWeb/merlin_170324484_7efdb4e4-a602-4dfb-89ed-14591a2ca5fc-articleLarge.jpg?quality=75\&auto=webp\&disable=upscale}

Three months ago, Bernie Sanders lost his chance at the Democratic
nomination, after a brief moment in which his socialist revolution
seemed poised to raze the bastions of neoliberal power. But the
developments of the last month, the George Floyd protests and their
cultural repercussions, may prove the more significant defeat for the
Sanders cause. In the winter he merely lost a presidential nomination;
in the summer he may be losing the battle for the future of the left.

Throughout his career, Sanders has stood for the proposition that
left-wing politics lost its way after the 1970s by letting what should
be its central purpose --- the class struggle, the rectification of
economic inequality, the war against the ``millionaires and
billionaires'' --- be obscured by cultural battles and displaced by a
pro-business, pro-Wall Street economic program. This shift has made
left-of-center political parties (in Europe as well as the United
States) steadily more upper middle class and conservatism steadily more
blue collar, but the promise of Sandersism was that the transformation
need not be permanent: A left that recovered the language of class
struggle, that disentangled liberal politics from faculty-lounge elitism
and neoliberal economics, could rally a silent majority against
plutocracy and win.

The 2016 Sanders primary campaign, which won white, working-class voters
who had been drifting from the Democrats, seemed to vindicate this
argument. The 2020 Sanders campaign, however, made it look more dubious,
by illustrating the core challenge facing a socialist revolution: Its
most passionate supporters --- highly educated, economically
disappointed urbanites --- aren't natural coalition partners for a Rust
Belt populism, and the more they tugged Sanders toward the cultural
left, the easier it was for Joe Biden to win blue-collar votes, leaving
Sanders leading an ideological faction rather than a broader
working-class insurgency.

Now, under these strange coronavirus conditions, we're watching a
different sort of insurgency challenge or change liberalism, one founded
on an intersectional vision of left-wing politics that never came
naturally to Sanders. Rather than Medicare for All and taxing
plutocrats, the rallying cry is racial justice and defunding the police.
Instead of finding its nemeses in corporate suites, the intersectional
revolution finds them on antique pedestals and atop the cultural
establishment.

And so far, as my colleague Sydney Ember
\href{https://www.nytimes3xbfgragh.onion/2020/06/19/us/politics/bernie-sanders-protests.html}{noted}
last week, this revolution has been more unifying than Sanders's version
--- uniting the Democratic establishment that once closed ranks against
him, earning support from just about every major corporate and cultural
institution, sending anti-racism titles skyrocketing up the best-seller
list, even bringing Mitt Romney into the streets as a marcher and
inducing Donald Trump to make grudging noises about police reform.

Ember quotes the law professor Kimberlé Crenshaw, the theorist of
intersectionality, marveling at the change: ``You basically have a
moment where every corporation worth its salt is saying something about
structural racism and anti-blackness, and that stuff is even
outdistancing what candidates in the Democratic Party were actually
saying.''

All this, from one perspective, vindicates critics who said Sanders's
vision of revolution was too class-bound and race-blind all along.

But the longer arc of the current revolutionary moment may actually end
up vindicating the socialist critique of post-1970s liberalism --- that
it's obsessed with cultural power at the expense of economic
transformation, and that it puts the language of radicalism in the
service of elitism.

The demand for police reform at the heart of the current protests
doesn't fit this caricature. But much of the action around it, the
anti-racist reckoning unfolding in colleges, media organizations,
corporations and public statuary, may seem more unifying than the
Sanders revolution precisely because it isn't as threatening to power.

The fact that corporations are ``outdistancing'' even politicians, as
Crenshaw puts it, in paying fealty to anti-racism is perhaps the tell.
It's not that corporate America is suddenly deeply committed to racial
equality; even for woke capital, the capitalism comes first. Rather,
it's that anti-racism as a cultural curriculum, a rhetoric of
re-education, is relatively easy to fold into the mechanisms of
managerialism, under the tutelage of the human resources department. The
idea that you need to retrain your employees so that they can work
together without microaggressing isn't Marxism, cultural or otherwise;
it's just a novel form of Fordism, with white-fragility gurus in place
of efficiency experts.

In our cultural institutions, too, the official enthusiasm for the
current radical mood is suggestive of the revolution's limits. The
tumult and protest is obviously a threat to certain people's jobs: The
revolutionaries need scapegoats, examples, wrongthinkers to cast out
\emph{pour encourager les autres}, superannuated figures to retire with
prejudice. But they aren't out to dissolve Harvard or break up Google or
close The New York Times; they're out to rule these institutions, with
more enlightenment than the old guard but the same fundamental powers.
And many of the changes the protesters seek are ones that the
establishment can happily accommodate: I can promise that few powerful
people will feel particularly threatened if the purge of Confederate
monuments widens and some statues of pre-World War II presidents and
Franciscan missionaries come crashing down as well. (Though renaming
Yale might be another matter \ldots{})

So the likely endgame of all this turbulence is the redistribution of
elite jobs, the upward circulation of the more racially diverse younger
generation, the abolition of perceived impediments to the management of
elite diversity (adieu, SAT) and the inculcation of a new elite language
whose academic style will delineate the professional class more
decisively from the unenlightened proles below. (With the possible
long-run consequence that not only the white working class but also some
minority voters will drift toward whatever remains of political
conservatism once Trump is finished with it.)

Yes, serious critics of structural racism have an agenda for economic as
well as cultural reform. But that agenda isn't what's being advanced:
Chuck Schumer will take a knee in kente cloth, but he isn't likely to
pass a major reparations bill, the white liberals buying up the works of
Ibram X. Kendi aren't going to abandon private schools or bus their kids
to minority neighborhoods. And in five years, it's more likely that
2020's legacy will be a cadre of permanently empowered commissars
getting people fired for unwise Twitter likes rather than any dramatic
interracial wealth redistribution.

I am a cynical conservative, so you can dismiss this as the usual
reactionary allergy to the fresh air of revolution. But it's also what
an old-guard leftism, of the sort that Bernie Sanders attempted to
revive, would predict of a revolutionary movement that has so much of
the establishment on board.

The destiny of liberalism, for some time now, has looked like handshake
agreements among corporate, academic and media power centers, with
progressive rhetoric deployed either reassuringly or threateningly,
depending on what's required to keep discontented factions within the
elite in line. The promise of the Sanders campaign was that the insights
of the older left, on class solidarity above all, could alter this
depressing future and make the newer left something more than a
handmaiden of oligarchy, a diversifier of late capitalism's corporate
boards.

The current wave of protests will have unpredictable consequences. But
right now, their revolution's conspicuous elite support seems like
strong evidence that Bernie Sanders failed.

\emph{The Times is committed to publishing}
\href{https://www.nytimes3xbfgragh.onion/2019/01/31/opinion/letters/letters-to-editor-new-york-times-women.html}{\emph{a
diversity of letters}} \emph{to the editor. We'd like to hear what you
think about this or any of our articles. Here are some}
\href{https://help.nytimes3xbfgragh.onion/hc/en-us/articles/115014925288-How-to-submit-a-letter-to-the-editor}{\emph{tips}}\emph{.
And here's our email:}
\href{mailto:letters@NYTimes.com}{\emph{letters@NYTimes.com}}\emph{.}

\emph{Follow The New York Times Opinion section on}
\href{https://www.facebookcorewwwi.onion/nytopinion}{\emph{Facebook}}\emph{,}
\href{http://twitter.com/NYTOpinion}{\emph{Twitter (@NYTOpinion)}}
\emph{and}
\href{https://www.instagram.com/nytopinion/}{\emph{Instagram}}\emph{,
join the Facebook political discussion group,}
\href{https://www.facebookcorewwwi.onion/groups/votingwhilefemale/}{\emph{Voting
While Female}}\emph{.}

Advertisement

\protect\hyperlink{after-bottom}{Continue reading the main story}

\hypertarget{site-index}{%
\subsection{Site Index}\label{site-index}}

\hypertarget{site-information-navigation}{%
\subsection{Site Information
Navigation}\label{site-information-navigation}}

\begin{itemize}
\tightlist
\item
  \href{https://help.nytimes3xbfgragh.onion/hc/en-us/articles/115014792127-Copyright-notice}{©~2020~The
  New York Times Company}
\end{itemize}

\begin{itemize}
\tightlist
\item
  \href{https://www.nytco.com/}{NYTCo}
\item
  \href{https://help.nytimes3xbfgragh.onion/hc/en-us/articles/115015385887-Contact-Us}{Contact
  Us}
\item
  \href{https://www.nytco.com/careers/}{Work with us}
\item
  \href{https://nytmediakit.com/}{Advertise}
\item
  \href{http://www.tbrandstudio.com/}{T Brand Studio}
\item
  \href{https://www.nytimes3xbfgragh.onion/privacy/cookie-policy\#how-do-i-manage-trackers}{Your
  Ad Choices}
\item
  \href{https://www.nytimes3xbfgragh.onion/privacy}{Privacy}
\item
  \href{https://help.nytimes3xbfgragh.onion/hc/en-us/articles/115014893428-Terms-of-service}{Terms
  of Service}
\item
  \href{https://help.nytimes3xbfgragh.onion/hc/en-us/articles/115014893968-Terms-of-sale}{Terms
  of Sale}
\item
  \href{https://spiderbites.nytimes3xbfgragh.onion}{Site Map}
\item
  \href{https://help.nytimes3xbfgragh.onion/hc/en-us}{Help}
\item
  \href{https://www.nytimes3xbfgragh.onion/subscription?campaignId=37WXW}{Subscriptions}
\end{itemize}
