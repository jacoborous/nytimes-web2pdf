Sections

SEARCH

\protect\hyperlink{site-content}{Skip to
content}\protect\hyperlink{site-index}{Skip to site index}

\href{https://www.nytimes3xbfgragh.onion/section/technology}{Technology}

\href{https://myaccount.nytimes3xbfgragh.onion/auth/login?response_type=cookie\&client_id=vi}{}

\href{https://www.nytimes3xbfgragh.onion/section/todayspaper}{Today's
Paper}

\href{/section/technology}{Technology}\textbar{}A Case for Banning
Facial Recognition

\url{https://nyti.ms/2ARmEqs}

\begin{itemize}
\item
\item
\item
\item
\item
\end{itemize}

Advertisement

\protect\hyperlink{after-top}{Continue reading the main story}

Supported by

\protect\hyperlink{after-sponsor}{Continue reading the main story}

on tech

\hypertarget{a-case-for-banning-facial-recognition}{%
\section{A Case for Banning Facial
Recognition}\label{a-case-for-banning-facial-recognition}}

A Google research scientist explains why she thinks the police shouldn't
use facial recognition software.

\includegraphics{https://static01.graylady3jvrrxbe.onion/images/2020/06/09/business/09ontech-videostill/09ontech-videostill-threeByTwoMediumAt2X.png}

\href{https://www.nytimes3xbfgragh.onion/by/shira-ovide}{\includegraphics{https://static01.graylady3jvrrxbe.onion/images/2020/03/18/reader-center/author-shira-ovide/author-shira-ovide-thumbLarge-v2.png}}

By \href{https://www.nytimes3xbfgragh.onion/by/shira-ovide}{Shira Ovide}

\begin{itemize}
\item
  June 9, 2020
\item
  \begin{itemize}
  \item
  \item
  \item
  \item
  \item
  \end{itemize}
\end{itemize}

\emph{This article is part of the On Tech newsletter. You can}
\href{https://www.nytimes3xbfgragh.onion/newsletters/signup/OT}{\emph{sign
up here}} \emph{to receive it weekdays.}

Facial recognition software might be the world's most
\href{https://www.nytimes3xbfgragh.onion/2019/05/15/business/facial-recognition-software-controversy.html}{divisive}
technology.

Law enforcement agencies and some companies use it to
\href{https://www.nytimes3xbfgragh.onion/2019/06/09/opinion/facial-recognition-police-new-york-city.html}{identify}
suspects and victims by matching photos and video with databases like
driver's license records. But civil liberties groups say facial
recognition
\href{https://www.nytimes3xbfgragh.onion/2020/01/18/technology/clearview-privacy-facial-recognition.html}{contributes
to privacy erosion},
\href{https://www.nytimes3xbfgragh.onion/2019/12/19/technology/facial-recognition-bias.html}{reinforces
bias} against black people and is prone to
\href{https://www.washingtonpost.com/technology/2019/04/30/amazons-facial-recognition-technology-is-supercharging-local-police/}{misuse}.

\href{https://www.nytimes3xbfgragh.onion/2019/05/14/us/facial-recognition-ban-san-francisco.html}{San
Francisco} and a major
\href{https://www.nytimes3xbfgragh.onion/2019/06/27/opinion/police-cam-facial-recognition.html}{provider
of police body cameras} have barred its use by law enforcement, and IBM
on Monday
\href{https://www.axios.com/ibm-is-exiting-the-face-recognition-business-62e79f09-34a2-4f1d-a541-caba112415c6.html}{backed
away from its work} in this area. Some proposals to restructure police
departments call for tighter
\href{https://www.protocol.com/police-facial-recognition-legislation}{restrictions}
on their use of facial recognition.

Timnit Gebru, a leader of Google's ethical artificial intelligence team,
explained why she believes that facial recognition is too dangerous to
be used right now for law enforcement purposes. These are edited
excerpts from our virtual discussion at the
\href{http://www.womens-forum.com/}{Women's Forum for the Economy \&
Society} on Monday.

\textbf{Ovide: What are your concerns about facial recognition?}

\textbf{Gebru:} I collaborated with
\href{https://www.ajlunited.org/about}{Joy Buolamwini} at the M.I.T.
Media Lab, who led an \href{http://gendershades.org/}{analysis} that
found
\href{https://www.nytimes3xbfgragh.onion/2018/06/21/opinion/facial-analysis-technology-bias.html}{very
high disparities in error rates} {[}in facial identification systems{]},
especially between lighter-skinned men and darker-skinned women. In
melanoma screenings, imagine that there's a detection technology that
doesn't work for people with darker skin.

I also realized that even perfect facial recognition can be misused. I'm
a black woman living in the U.S. who has dealt with serious consequences
of racism. Facial recognition is being used against the black community.
Baltimore police during the
\href{https://www.nytimes3xbfgragh.onion/2016/04/13/us/baltimore-freddie-gray.html}{Freddie
Gray protests} used
\href{https://www.baltimoresun.com/news/crime/bs-md-facial-recognition-20161017-story.html}{facial
recognition to identify protesters} by linking images to social media
profiles.

\textbf{But a police officer or eyewitness could also look at
surveillance footage and mug shots and misidentify someone as Jim Smith.
Is software more accurate or less biased than humans?}

That depends. Our analysis showed that for many, facial recognition was
way less accurate than humans.

The other problem is something called automation bias. If your intuition
tells you that an image doesn't look like Smith, but the computer model
tells you that it is him with 99 percent accuracy, you're more likely to
believe that model.

There's also an imbalance of power. Facial recognition can be completely
accurate, but it can still be used in a way that is detrimental to
certain groups of people.

The combination of overreliance on technology, misuse and lack of
transparency --- we don't know how widespread the use of this software
is --- is dangerous.

\textbf{A maker of police body cameras recently discussed using
artificial intelligence to}
\textbf{\href{https://www.bloomberg.com/news/newsletters/2020-06-05/should-police-officers-wear-body-cameras}{analyze
video footage}} \textbf{and possibly flag law-enforcement incidents for
review. What's your take on using technology in that way?}

My gut reaction is that a lot of people in technology have the urge to
jump on a tech solution without listening to people who have been
working with community leaders, the police and others proposing
solutions to reform the police.

\textbf{Do you see a way to use facial recognition for law enforcement
and security responsibly?}

It should be banned at the moment. I don't know about the future.

\emph{You can watch our entire conversation about helpful uses of A.I.
and its downsides}
\href{https://www.youtube.com/watch?v=uYlR3OIUQx4\&feature=youtu.be}{\emph{here}}\emph{.}

\begin{center}\rule{0.5\linewidth}{\linethickness}\end{center}

\hypertarget{tip-of-the-week}{%
\subsubsection{Tip of the Week}\label{tip-of-the-week}}

\hypertarget{stopping-trackers-in-their-tracks}{%
\subsection{Stopping trackers in their
tracks}\label{stopping-trackers-in-their-tracks}}

\href{https://www.nytimes3xbfgragh.onion/by/brian-x-chen}{\emph{Brian X.
Chen}}\emph{, a consumer technology writer at the The New York Times,
writes in to explain ways that emails can identify when and where you
click, and how to dial back the tracking.}

Google's Gmail is so popular in large part because its artificial
intelligence is effective at filtering out spam. But it does little to
combat another nuisance: email tracking.

The trackers come in many forms, like an invisible piece of software
inserted into an email or a hyperlink embedded inside text. They are
frequently used to detect when someone opens an email and even a
person's location when the message is opened.

When used legitimately, email trackers help businesses determine what
types of marketing messages to send to you, and how frequently to
communicate with you. This emailed newsletter has some trackers as well
to help us gain insight into the topics you like to read about, among
other metrics.

But from a privacy perspective,
\href{https://www.getrevue.co/profile/themarkup/issues/hello-from-the-markup-199187}{email
tracking} may feel unfair. You didn't opt in to being tracked, and
there's no simple way to opt out.

Fortunately, many email trackers can be thwarted by disabling images
from automatically loading in Gmail messages. Here's how to do that:

\begin{itemize}
\item
  Inside Gmail.com, look in the upper right corner for the icon of a
  gear, click on it, and choose the ``Settings'' option.
\item
  In the settings window, scroll down to ``Images.'' Select ``Ask before
  displaying external images.''
\end{itemize}

With this setting enabled, you can prevent tracking software from
loading automatically. If you choose, you can agree to load the images.
This won't stop all email tracking, but it's better than nothing.

\textbf{Bonus tech tip!} Some readers asked for more help setting up
notifications that can
\href{https://www.nytimes3xbfgragh.onion/2020/06/08/technology/how-to-reduce-credit-card-fraud.html}{alert
you to fraudulent credit card charges}. Signing up for these is not easy
because, let's face it, financial websites are not the simplest to use.

On the apps and websites for the credit cards I have, I found these
alerts in menus labeled ``Profile and Settings'' or ``Help \& Support.''
Look for ``Alerts'' or dig into the privacy and security options. Sign
up for an email or app notification each time your card is used to make
a purchase online and over the phone.

Most of the time, those purchases are from you. But you want to know
right away in the (hopefully) rare times when they're not.

\begin{center}\rule{0.5\linewidth}{\linethickness}\end{center}

\hypertarget{before-we-go-}{%
\subsection{Before we go \ldots{}}\label{before-we-go-}}

\begin{itemize}
\item
  \textbf{Behind the pro-China Twitter campaign:} An analysis by my New
  York Times colleagues found a new and decidedly pro-China presence on
  Twitter, made up of a network of accounts exhibiting
  \href{https://www.nytimes3xbfgragh.onion/2020/06/08/technology/china-twitter-disinformation.html}{seemingly
  coordinated behavior}. The findings add to evidence suggesting that
  Twitter is being manipulated to amplify the Chinese government's
  messaging about the coronavirus and other topics.
\item
  \textbf{Restaurants really aren't fans of those apps:}
  \href{https://www.nytimes3xbfgragh.onion/by/nathaniel-popper}{Nathaniel
  Popper}, a Times tech reporter, explains why
  \href{https://www.nytimes3xbfgragh.onion/2020/06/09/technology/delivery-apps-restaurants-fees-virus.html}{restaurants
  are increasingly unhappy about}the high fees and other aspects of food
  delivery services like Grubhub and Postmates. (I'll have a
  conversation with Nathaniel about this in Wednesday's newsletter.)
\item
  \textbf{The downsides of every gathering of humans:} The neighborhood
  social network Nextdoor has been both a place for people to help one
  another during the pandemic, and a way for neighbors to lash out at
  one another over perceived slights or fan fears about crime. The Verge
  writes about the challenges faced by the volunteers on Nextdoor who
  are
  \href{https://www.theverge.com/21283993/nextdoor-app-racism-community-moderation-guidance-protests}{moderating
  discussions about race} and the recent protests.
\end{itemize}

\textbf{Hugs to this}

\href{https://www.npr.org/podcasts/510282/pop-culture-happy-hour}{NPR's
Pop Culture Happy Hour} recently introduced to me the
\href{https://twitter.com/AnneLouiseAvery/status/1267931929703317505?s=09}{whimsical
mini children's stories} that the writer Anne Louise Avery
\href{https://twitter.com/AnneLouiseAvery/status/1254385383510589441}{composes
on Twitter}.

\begin{center}\rule{0.5\linewidth}{\linethickness}\end{center}

\emph{We want to hear from you. Tell us what you think of this
newsletter and what else you'd like us to explore. You can reach us at}
\href{mailto:ontech@NYTimes.com?subject=On\%20Tech\%20Feedback}{\emph{ontech@NYTimes.com.}}

\emph{Get this newsletter in your inbox every
weekday;}\href{https://www.nytimes3xbfgragh.onion/newsletters/signup/OT}{\emph{please
sign up here}}\emph{.}

Advertisement

\protect\hyperlink{after-bottom}{Continue reading the main story}

\hypertarget{site-index}{%
\subsection{Site Index}\label{site-index}}

\hypertarget{site-information-navigation}{%
\subsection{Site Information
Navigation}\label{site-information-navigation}}

\begin{itemize}
\tightlist
\item
  \href{https://help.nytimes3xbfgragh.onion/hc/en-us/articles/115014792127-Copyright-notice}{©~2020~The
  New York Times Company}
\end{itemize}

\begin{itemize}
\tightlist
\item
  \href{https://www.nytco.com/}{NYTCo}
\item
  \href{https://help.nytimes3xbfgragh.onion/hc/en-us/articles/115015385887-Contact-Us}{Contact
  Us}
\item
  \href{https://www.nytco.com/careers/}{Work with us}
\item
  \href{https://nytmediakit.com/}{Advertise}
\item
  \href{http://www.tbrandstudio.com/}{T Brand Studio}
\item
  \href{https://www.nytimes3xbfgragh.onion/privacy/cookie-policy\#how-do-i-manage-trackers}{Your
  Ad Choices}
\item
  \href{https://www.nytimes3xbfgragh.onion/privacy}{Privacy}
\item
  \href{https://help.nytimes3xbfgragh.onion/hc/en-us/articles/115014893428-Terms-of-service}{Terms
  of Service}
\item
  \href{https://help.nytimes3xbfgragh.onion/hc/en-us/articles/115014893968-Terms-of-sale}{Terms
  of Sale}
\item
  \href{https://spiderbites.nytimes3xbfgragh.onion}{Site Map}
\item
  \href{https://help.nytimes3xbfgragh.onion/hc/en-us}{Help}
\item
  \href{https://www.nytimes3xbfgragh.onion/subscription?campaignId=37WXW}{Subscriptions}
\end{itemize}
