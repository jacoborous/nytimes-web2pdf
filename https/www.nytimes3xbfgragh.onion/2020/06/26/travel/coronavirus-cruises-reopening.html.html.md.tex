Sections

SEARCH

\protect\hyperlink{site-content}{Skip to
content}\protect\hyperlink{site-index}{Skip to site index}

\href{https://www.nytimes3xbfgragh.onion/section/travel}{Travel}

\href{https://myaccount.nytimes3xbfgragh.onion/auth/login?response_type=cookie\&client_id=vi}{}

\href{https://www.nytimes3xbfgragh.onion/section/todayspaper}{Today's
Paper}

\href{/section/travel}{Travel}\textbar{}The Post-Coronavirus Cruise? Not
Ready to Sail

\url{https://nyti.ms/3fZuFsB}

\begin{itemize}
\item
\item
\item
\item
\item
\item
\end{itemize}

\hypertarget{the-coronavirus-outbreak}{%
\subsubsection{\texorpdfstring{\href{https://www.nytimes3xbfgragh.onion/news-event/coronavirus?name=styln-coronavirus-national\&region=TOP_BANNER\&variant=undefined\&block=storyline_menu_recirc\&action=click\&pgtype=Article\&impression_id=32182f30-e386-11ea-941b-313b39d5e3bc}{The
Coronavirus
Outbreak}}{The Coronavirus Outbreak}}\label{the-coronavirus-outbreak}}

\begin{itemize}
\tightlist
\item
  live\href{https://www.nytimes3xbfgragh.onion/2020/08/20/world/coronavirus-covid.html?name=styln-coronavirus-national\&region=TOP_BANNER\&variant=undefined\&block=storyline_menu_recirc\&action=click\&pgtype=Article\&impression_id=32182f31-e386-11ea-941b-313b39d5e3bc}{Latest
  Updates}
\item
  \href{https://www.nytimes3xbfgragh.onion/interactive/2020/us/coronavirus-us-cases.html?name=styln-coronavirus-national\&region=TOP_BANNER\&variant=undefined\&block=storyline_menu_recirc\&action=click\&pgtype=Article\&impression_id=32182f32-e386-11ea-941b-313b39d5e3bc}{Maps
  and Cases}
\item
  \href{https://www.nytimes3xbfgragh.onion/interactive/2020/science/coronavirus-vaccine-tracker.html?name=styln-coronavirus-national\&region=TOP_BANNER\&variant=undefined\&block=storyline_menu_recirc\&action=click\&pgtype=Article\&impression_id=32182f33-e386-11ea-941b-313b39d5e3bc}{Vaccine
  Tracker}
\item
  \href{https://www.nytimes3xbfgragh.onion/2020/08/19/us/colleges-closing-covid.html?name=styln-coronavirus-national\&region=TOP_BANNER\&variant=undefined\&block=storyline_menu_recirc\&action=click\&pgtype=Article\&impression_id=32182f34-e386-11ea-941b-313b39d5e3bc}{Colleges
  Closing}
\item
  \href{https://www.nytimes3xbfgragh.onion/live/2020/08/20/business/stock-market-today-coronavirus?name=styln-coronavirus-national\&region=TOP_BANNER\&variant=undefined\&block=storyline_menu_recirc\&action=click\&pgtype=Article\&impression_id=32182f35-e386-11ea-941b-313b39d5e3bc}{Economy}
\end{itemize}

Advertisement

\protect\hyperlink{after-top}{Continue reading the main story}

Supported by

\protect\hyperlink{after-sponsor}{Continue reading the main story}

\hypertarget{the-post-coronavirus-cruise-not-ready-to-sail}{%
\section{The Post-Coronavirus Cruise? Not Ready to
Sail}\label{the-post-coronavirus-cruise-not-ready-to-sail}}

Data shows that there were far more cases of Covid-19 on cruise ships
than have been reported, but the companies and the C.D.C. have yet to
establish how the boats can come back.

\includegraphics{https://static01.graylady3jvrrxbe.onion/images/2020/06/26/travel/26cruise-comeback2/merlin_173897511_f233f3c1-7b2f-40b2-833d-a1e455f082ba-articleLarge.jpg?quality=75\&auto=webp\&disable=upscale}

By \href{https://www.nytimes3xbfgragh.onion/by/frances-robles}{Frances
Robles}

\begin{itemize}
\item
  Published June 26, 2020Updated Aug. 10, 2020
\item
  \begin{itemize}
  \item
  \item
  \item
  \item
  \item
  \item
  \end{itemize}
\end{itemize}

W. Bradford Gary spent 10 days trapped inside a cruise ship cabin off
the coast of Brazil in March while health authorities in several
countries scrambled to figure out what to do with a vessel full of older
people who had potentially been exposed to the coronavirus.

But when faced with the question of whether he'd ever cruise again, he
doesn't hesitate.

``We are very anxious to get back on board,'' he said, and he believes
he's not alone: ``There are people like us who want to do this.''

Mr. Gary, 70, a retired corporate executive who lives in Palm Beach,
Fla., imagines the cruise ship of the near future equipped with special
disinfecting ultraviolet lights and air flow contraptions commonly used
in sterile laboratories. He envisions larger cabins, fewer passengers
and a lot more outdoor spaces. ``We want to know everything is safe,''
he said.

That is a big order.

With more than 20 million passengers a year, the \$45 billion global
cruise industry has a particularly vexing challenge: Its most loyal
customers, older people, also happen to be the key demographic at risk
for the new illness that has swept the planet, killing more than 450,000
people. Cruises also have the very things that help the coronavirus
spread: large gatherings, confined spaces and workers who live in tight
shared quarters.

More than three months after
the\href{https://www.cdc.gov/quarantine/cruise/index.html}{Centers for
Disease Control and Prevention issued a no sail order}for all United
States cruises, interviews with health officials, loyal passengers,
industry experts, cruise executives and maritime lawyers made clear that
restarting operations would require rethinking cruising itself --- from
the number of passengers onboard to how they are fed, housed and
entertained --- and that the government and the cruise lines are not
close to figuring it out.

Last week, the Cruise Lines International Association (CLIA), the cruise
industry's trade group, said that it was
\href{https://cruising.org/news-and-research/press-room/2020/june/clia-announces-voluntary-suspension-of-cruise-operations-from-us-ports}{voluntarily
extending the no sail period} from U.S. ports until Sept. 15. Earlier,
Carnival Corporation, the world's biggest cruise company, had suggested
that it could start sailings by Aug. 1.

According to Martin Cetron, the C.D.C.'s director for the Division of
Global Migration and Quarantine, cruise ships offer fertile ground for
the ``seeding, amplification, and dissemination'' of Covid-19, worsened
by the fact that crew members often transfer from one ship to another,
taking diseases with them.

Breaking that chain of infection is key.

But as restaurants, casinos, movie theaters and theme parks are poised
to reopen, with plans in place to prevent the spread of the highly
infectious disease, the cruise industry has not publicly laid out its
strategy.

Bari Golin-Blaugrund, a spokeswoman for CLIA, said, ``The cruise
industry is taking a holistic approach to planning for Covid-19 safety,
when sailing is allowed, that would ideally entail a door-to-door
strategy beginning at the time of booking through the passengers' return
home.''

She said cruise ships are already cleaned several times a day, but the
industry is using the time off to rethink everything from offshore
excursions to enhanced medical capabilities onboard and evacuations.

As to setting out highly detailed plans for what a post-pandemic cruise
might look like, ``We're not there yet,'' she said.

\includegraphics{https://static01.graylady3jvrrxbe.onion/images/2020/06/26/travel/26cruise-comeback1/merlin_173897469_e836eaff-8c60-475d-b582-7af47f84a3f1-articleLarge.jpg?quality=75\&auto=webp\&disable=upscale}

\hypertarget{a-lack-of-self-regulation}{%
\subsection{A lack of self-regulation}\label{a-lack-of-self-regulation}}

The coronavirus hit the cruise industry hard. Passengers were stranded
for weeks while people on board got sick and were quarantined in their
staterooms. A
\href{https://www.miamiherald.com/news/business/tourism-cruises/article241914096.html}{Miami
Herald analysis} showed at least 80 people died worldwide.

New data from the C.D.C., released in response to a Freedom of
Information Act request by The Times shows that more than 100 ships in
the U.S. jurisdiction alone had outbreaks on board, sickening nearly
3,000 people, including more than 850 passengers. The C.D.C.'s figures
count cases that were ``clinically compatible'' with Covid-19, but not
confirmed in a lab, and hundreds that occurred among crew after
passengers disembarked.

\hypertarget{latest-updates-the-coronavirus-outbreak}{%
\section{\texorpdfstring{\href{https://www.nytimes3xbfgragh.onion/2020/08/20/world/coronavirus-covid.html?action=click\&pgtype=Article\&state=default\&region=MAIN_CONTENT_1\&context=storylines_live_updates}{Latest
Updates: The Coronavirus
Outbreak}}{Latest Updates: The Coronavirus Outbreak}}\label{latest-updates-the-coronavirus-outbreak}}

Updated 2020-08-21T07:46:15.883Z

\begin{itemize}
\tightlist
\item
  \href{https://www.nytimes3xbfgragh.onion/2020/08/20/world/coronavirus-covid.html?action=click\&pgtype=Article\&state=default\&region=MAIN_CONTENT_1\&context=storylines_live_updates\#link-68774d88}{Shutdowns,
  warnings and scoldings follow alarming incidents on college campuses.}
\item
  \href{https://www.nytimes3xbfgragh.onion/2020/08/20/world/coronavirus-covid.html?action=click\&pgtype=Article\&state=default\&region=MAIN_CONTENT_1\&context=storylines_live_updates\#link-26b58724}{Biden
  knocks Trump's pandemic response, and outlines a national strategy.}
\item
  \href{https://www.nytimes3xbfgragh.onion/2020/08/20/world/coronavirus-covid.html?action=click\&pgtype=Article\&state=default\&region=MAIN_CONTENT_1\&context=storylines_live_updates\#link-4e542da3}{U.S.
  health agencies announce moves to confront the flu season and
  plummeting child vaccination rates.}
\end{itemize}

\href{https://www.nytimes3xbfgragh.onion/2020/08/20/world/coronavirus-covid.html?action=click\&pgtype=Article\&state=default\&region=MAIN_CONTENT_1\&context=storylines_live_updates}{See
more updates}

More live coverage:
\href{https://www.nytimes3xbfgragh.onion/live/2020/08/20/business/stock-market-today-coronavirus?action=click\&pgtype=Article\&state=default\&region=MAIN_CONTENT_1\&context=storylines_live_updates}{Markets}

The C.D.C. records show Carnival, which had 47 ships on which people
fell ill across its nine brands, was most affected. Through a spokesman,
Carnival disputed the figures, saying the C.D.C.'s methodology resulted
in overcounting. If only cases with a lab confirmation were included,
the number of Carnival ships affected would be 15.

Royal Caribbean had 28 ships with coronavirus cases --- almost half its
fleet.

Only 15 of the 121 cruise ships that entered U.S. waters after March 1
did not have the disease on board, the records show.

Cruise companies and public health officials are still evacuating and
repatriating tens of thousands of crew members who have remained at sea
as the pandemic raged, the C.D.C. said.

As of Sunday, there were still 68 ships at sea with 21,506 crew members
on board in the U.S. jurisdiction alone.

The coronavirus marked the first time the agency issued a travel
advisory against a particular mode of travel, as opposed to a geographic
area, Martin Cetron, the C.D.C.'s director for the Division of Global
Migration and Quarantine, said. The C.D.C.'s initial advisory, which
recommended against cruise travel but did not ban it, was issued March
8, after more than 700 people got sick aboard the Diamond Princess ship
in Japan.

The advisory turned into a no sail order after it became clear that
companies were continuing to embark on voyages and passengers were not
canceling in large numbers, Dr. Cetron said. ``This situation wasn't
responding to self-regulation.''

When cruises resume, some changes are likely inevitable, such as thermal
scanners at terminals to check for elevated temperatures, disinfection
foggers to clean boats between cruises and upgraded ventilation systems.
Self-serve buffets may be a thing of the past.

The Times asked 10 of the biggest cruise lines to outline their
preparations for resuming sailings. Only one,
\href{http://gentingcruiselines.com/}{Genting Cruise Lines,}a Hong
Kong-based company that mostly operates in Asia, agreed to talk about
its plans in detail.

While Princess Cruises, part of Carnival Corporation, did not respond
directly, it posted a seven-page
\href{https://www.princess.com/downloads/pdf/plan/Health-Advisory-and-Travel-Safety-Procedures.pdf}{health
advisory}outlining some of the steps it plans to take, including
increasing the temperature of washers and dryers for better disinfection
of bedding and towels. Stateroom surfaces will be cleaned twice a day.

Carnival said it was still too soon to offer any details. In a
conference call with journalists in April, Carnival's chief executive,
Arnold W. Donald, said he would let medical experts decide whether to
require medical clearances for older passengers.

Royal Caribbean's chief executive, Richard Fain,
\href{https://www.fool.com/earnings/call-transcripts/2020/05/20/royal-caribbean-cruises-ltd-rcl-q1-2020-earnings-c.aspx}{said}
in a May earnings call that he planned to unveil a ``healthy return to
service program'' that would focus on upgraded health screenings for
passengers before boarding, as well as new procedures for dealing with
infections on board. But that plan has not yet been released.

Arthur L. Diskin, who was a Carnival cruise ship doctor for 18 years,
said the adjustments will likely be tailored to each company's market.
Longer cruises, for example, tend to have passengers who are much older,
so changes should account for that.

``Maybe they don't have a disco,'' he said. ``Maybe they adapt to have
more outdoor dining venues so people can be adequately spaced.''

But none of those changes address the biggest questions of how to
prevent a disease that has generally been checked by isolation and
social distancing on cruise ships carrying thousands of passengers
expecting to party and enjoy themselves. Will they keep people six feet
apart? Will they make them wear masks?

The first step, said Mattia Jorgensen, a naval architect and marine
engineer with the Finnish company Foreship, which specializes in ship
design and engineering, is a thorough analysis of how the virus gets on
board ships and spreads.

``We need to really look at how we don't get it on board, and in the
event it gets on board, how do we contain it,'' he said.

Mr. Jorgensen said several large cruise ship companies hired his firm to
help design solutions, but the process cannot be completed until the
C.D.C. shares its own guidance.

One of the biggest problems cruise companies are expected to tackle is
how to manage their staff. Crew often transfer from one ship to another
and take illnesses with them. Companies are expected to make changes
that will limit crew movement between ships.

James Walker, a lawyer in Miami whose practice is focused on suing
cruise companies, said ships are going to have to seriously examine how
much time they spend cleaning ships between voyages. Cruise companies do
not make money while in port, and are always anxious to head off to
another voyage, he said.

A review of the C.D.C.'s
\href{https://www.cdc.gov/coronavirus/2019-ncov/travelers/cruise-ship/what-cdc-is-doing.html}{preliminary
list}of affected ships that came to U.S. ports show they sometimes ended
one cruise and began another on the same day. One ship, the
\href{https://www.carnival.com/cruise-ships/carnival-valor.aspx}{Carnival
Valor,} had three back-to-back cruises on which passengers
contractedCovid-19. The ship is designed to carry almost 3,000
passengers in 1,487 staterooms. On quick turnarounds, ships come in to
port at 7 a.m. and head out with new passengers at 4 p.m.

``There's never enough time to clean,'' Mr. Walker said. ``And cruise
lines all know. It's not like they bring in professional cleaning crews
that do exhaustive work. They use the same cabin attendants who have to
be there to greet the guests when they show up.''

Ms. Golin-Blaugrund, the CLIA spokeswoman, said that ``Cruise ships are
cleaned and sanitized, under normal circumstances, with a frequency that
is nearly unparalleled in other settings. Multiple times daily, crews
clean and sanitize surfaces known for transmitting germs, such as
handrails, door handles and faucets. At the end of a voyage and before a
new one begins, ships are cleaned completely from top to bottom.''

As part of the industry's
\href{https://nam11.safelinks.protection.outlook.com/?url=https\%3A\%2F\%2Fwww.cdc.gov\%2Fnceh\%2Fvsp\%2Fdefault.htm\&data=02\%7C01\%7Cpress\%40cruising.org\%7C50fb50d31a3448536ad408d7f7873525\%7C53537f9f6e52441c8cbffc4bae13ff1d\%7C0\%7C0\%7C637250032412483411\&sdata=B1sH\%2FMKUa5XSrjUz5ujv78K37RgqCOoe0ST0dbx\%2FVYo\%3D\&reserved=0}{Vessel
Sanitation Program}, she said, ``Cruise ship crews are trained in
sanitation and public health practices and ships undergo unannounced
inspections twice a year.''

Sheri Griffiths, a video blogger who runs CruiseTipsTV, a YouTube
channel, with her husband, said the industry also needs to rethink its
cancellation policies. In the past, those who canceled had to forfeit
their payments --- often thousands of dollars --- which could lead
people to travel even if they are feeling unwell.

\href{https://www.nytimes3xbfgragh.onion/news-event/coronavirus?action=click\&pgtype=Article\&state=default\&region=MAIN_CONTENT_3\&context=storylines_faq}{}

\hypertarget{the-coronavirus-outbreak-}{%
\subsubsection{The Coronavirus Outbreak
›}\label{the-coronavirus-outbreak-}}

\hypertarget{frequently-asked-questions}{%
\paragraph{Frequently Asked
Questions}\label{frequently-asked-questions}}

Updated August 17, 2020

\begin{itemize}
\item ~
  \hypertarget{why-does-standing-six-feet-away-from-others-help}{%
  \paragraph{Why does standing six feet away from others
  help?}\label{why-does-standing-six-feet-away-from-others-help}}

  \begin{itemize}
  \tightlist
  \item
    The coronavirus spreads primarily through droplets from your mouth
    and nose, especially when you cough or sneeze. The C.D.C., one of
    the organizations using that measure,
    \href{https://www.nytimes3xbfgragh.onion/2020/04/14/health/coronavirus-six-feet.html?action=click\&pgtype=Article\&state=default\&region=MAIN_CONTENT_3\&context=storylines_faq}{bases
    its recommendation of six feet} on the idea that most large droplets
    that people expel when they cough or sneeze will fall to the ground
    within six feet. But six feet has never been a magic number that
    guarantees complete protection. Sneezes, for instance, can launch
    droplets a lot farther than six feet,
    \href{https://jamanetwork.com/journals/jama/fullarticle/2763852}{according
    to a recent study}. It's a rule of thumb: You should be safest
    standing six feet apart outside, especially when it's windy. But
    keep a mask on at all times, even when you think you're far enough
    apart.
  \end{itemize}
\item ~
  \hypertarget{i-have-antibodies-am-i-now-immune}{%
  \paragraph{I have antibodies. Am I now
  immune?}\label{i-have-antibodies-am-i-now-immune}}

  \begin{itemize}
  \tightlist
  \item
    As of right
    now,\href{https://www.nytimes3xbfgragh.onion/2020/07/22/health/covid-antibodies-herd-immunity.html?action=click\&pgtype=Article\&state=default\&region=MAIN_CONTENT_3\&context=storylines_faq}{that
    seems likely, for at least several months.} There have been
    frightening accounts of people suffering what seems to be a second
    bout of Covid-19. But experts say these patients may have a
    drawn-out course of infection, with the virus taking a slow toll
    weeks to months after initial exposure. People infected with the
    coronavirus typically
    \href{https://www.nature.com/articles/s41586-020-2456-9}{produce}
    immune molecules called antibodies, which are
    \href{https://www.nytimes3xbfgragh.onion/2020/05/07/health/coronavirus-antibody-prevalence.html?action=click\&pgtype=Article\&state=default\&region=MAIN_CONTENT_3\&context=storylines_faq}{protective
    proteins made in response to an
    infection}\href{https://www.nytimes3xbfgragh.onion/2020/05/07/health/coronavirus-antibody-prevalence.html?action=click\&pgtype=Article\&state=default\&region=MAIN_CONTENT_3\&context=storylines_faq}{.
    These antibodies may} last in the body
    \href{https://www.nature.com/articles/s41591-020-0965-6}{only two to
    three months}, which may seem worrisome, but that's perfectly normal
    after an acute infection subsides, said Dr. Michael Mina, an
    immunologist at Harvard University. It may be possible to get the
    coronavirus again, but it's highly unlikely that it would be
    possible in a short window of time from initial infection or make
    people sicker the second time.
  \end{itemize}
\item ~
  \hypertarget{im-a-small-business-owner-can-i-get-relief}{%
  \paragraph{I'm a small-business owner. Can I get
  relief?}\label{im-a-small-business-owner-can-i-get-relief}}

  \begin{itemize}
  \tightlist
  \item
    The
    \href{https://www.nytimes3xbfgragh.onion/article/small-business-loans-stimulus-grants-freelancers-coronavirus.html?action=click\&pgtype=Article\&state=default\&region=MAIN_CONTENT_3\&context=storylines_faq}{stimulus
    bills enacted in March} offer help for the millions of American
    small businesses. Those eligible for aid are businesses and
    nonprofit organizations with fewer than 500 workers, including sole
    proprietorships, independent contractors and freelancers. Some
    larger companies in some industries are also eligible. The help
    being offered, which is being managed by the Small Business
    Administration, includes the Paycheck Protection Program and the
    Economic Injury Disaster Loan program. But lots of folks have
    \href{https://www.nytimes3xbfgragh.onion/interactive/2020/05/07/business/small-business-loans-coronavirus.html?action=click\&pgtype=Article\&state=default\&region=MAIN_CONTENT_3\&context=storylines_faq}{not
    yet seen payouts.} Even those who have received help are confused:
    The rules are draconian, and some are stuck sitting on
    \href{https://www.nytimes3xbfgragh.onion/2020/05/02/business/economy/loans-coronavirus-small-business.html?action=click\&pgtype=Article\&state=default\&region=MAIN_CONTENT_3\&context=storylines_faq}{money
    they don't know how to use.} Many small-business owners are getting
    less than they expected or
    \href{https://www.nytimes3xbfgragh.onion/2020/06/10/business/Small-business-loans-ppp.html?action=click\&pgtype=Article\&state=default\&region=MAIN_CONTENT_3\&context=storylines_faq}{not
    hearing anything at all.}
  \end{itemize}
\item ~
  \hypertarget{what-are-my-rights-if-i-am-worried-about-going-back-to-work}{%
  \paragraph{What are my rights if I am worried about going back to
  work?}\label{what-are-my-rights-if-i-am-worried-about-going-back-to-work}}

  \begin{itemize}
  \tightlist
  \item
    Employers have to provide
    \href{https://www.osha.gov/SLTC/covid-19/standards.html}{a safe
    workplace} with policies that protect everyone equally.
    \href{https://www.nytimes3xbfgragh.onion/article/coronavirus-money-unemployment.html?action=click\&pgtype=Article\&state=default\&region=MAIN_CONTENT_3\&context=storylines_faq}{And
    if one of your co-workers tests positive for the coronavirus, the
    C.D.C.} has said that
    \href{https://www.cdc.gov/coronavirus/2019-ncov/community/guidance-business-response.html}{employers
    should tell their employees} -\/- without giving you the sick
    employee's name -\/- that they may have been exposed to the virus.
  \end{itemize}
\item ~
  \hypertarget{what-is-school-going-to-look-like-in-september}{%
  \paragraph{What is school going to look like in
  September?}\label{what-is-school-going-to-look-like-in-september}}

  \begin{itemize}
  \tightlist
  \item
    It is unlikely that many schools will return to a normal schedule
    this fall, requiring the grind of
    \href{https://www.nytimes3xbfgragh.onion/2020/06/05/us/coronavirus-education-lost-learning.html?action=click\&pgtype=Article\&state=default\&region=MAIN_CONTENT_3\&context=storylines_faq}{online
    learning},
    \href{https://www.nytimes3xbfgragh.onion/2020/05/29/us/coronavirus-child-care-centers.html?action=click\&pgtype=Article\&state=default\&region=MAIN_CONTENT_3\&context=storylines_faq}{makeshift
    child care} and
    \href{https://www.nytimes3xbfgragh.onion/2020/06/03/business/economy/coronavirus-working-women.html?action=click\&pgtype=Article\&state=default\&region=MAIN_CONTENT_3\&context=storylines_faq}{stunted
    workdays} to continue. California's two largest public school
    districts --- Los Angeles and San Diego --- said on July 13, that
    \href{https://www.nytimes3xbfgragh.onion/2020/07/13/us/lausd-san-diego-school-reopening.html?action=click\&pgtype=Article\&state=default\&region=MAIN_CONTENT_3\&context=storylines_faq}{instruction
    will be remote-only in the fall}, citing concerns that surging
    coronavirus infections in their areas pose too dire a risk for
    students and teachers. Together, the two districts enroll some
    825,000 students. They are the largest in the country so far to
    abandon plans for even a partial physical return to classrooms when
    they reopen in August. For other districts, the solution won't be an
    all-or-nothing approach.
    \href{https://bioethics.jhu.edu/research-and-outreach/projects/eschool-initiative/school-policy-tracker/}{Many
    systems}, including the nation's largest, New York City, are
    devising
    \href{https://www.nytimes3xbfgragh.onion/2020/06/26/us/coronavirus-schools-reopen-fall.html?action=click\&pgtype=Article\&state=default\&region=MAIN_CONTENT_3\&context=storylines_faq}{hybrid
    plans} that involve spending some days in classrooms and other days
    online. There's no national policy on this yet, so check with your
    municipal school system regularly to see what is happening in your
    community.
  \end{itemize}
\end{itemize}

``That's a huge and significant change to the future of cruising, and
it's critical they do that

For people to feel safe, they need to feel that the passengers around
them won't be boarding the ship sick, and they need to know the crew
will be held to a high standard of wellness,'' Ms. Griffiths said.

Princess Cruises' advisory said that it would issue full refunds or
credits to passengers who had to cancel because they were not well or
had been exposed to the virus.

As for Ms. Griffiths: ``I will get on a cruise when the C.D.C. says that
I don't have to self-quarantine when I get off a ship,'' she said.

\hypertarget{a-plan-for-starting-to-sail}{%
\subsection{A plan for starting to
sail}\label{a-plan-for-starting-to-sail}}

Those looking for a blueprint as to how the cruise companies might
respond could look to\href{http://gentingcruiselines.com/}{Genting
Cruise Lines's plan}.

In April, Genting worked with local port officials and industry trade
groups to come up with an
\href{http://gentingcruiselines.com/media/1267/20200408-genting-cruise-lines-announces-enhanced-preventive-measures-setting-new-standards-for-the-fleet-and-the-cruise-industry.pdf}{eight-part
safety guide} outlining disinfection procedures and rules for social
distancing.

In an interview, Kent Zhu, the company's president, said the guidelines
were designed to market the company's cruises to wary passengers and
provide a template for other companies. ``The passengers needed to know
what actually a cruise line would be able to do to make sure they feel
comfortable,'' he said. ``They want to be reassured that it's safe.''

Mr. Zhu said the company is aiming to begin cruising by the middle of
the summer with limited itineraries to countries that have the virus
under control.

Under the new guidelines, the company will use infrared temperature
screenings on the gangway to weed out sick travelers and require
passengers 70 and older to provide a ``doctor's certificate of fitness
for travel.'' The plan allows guests to cancel up to 48 hours before a
sailing without financial penalty; in an interview, Mr. Zhu said Genting
would offer refunds to those who failed health screenings.

The guide also outlines stringent sanitation protocols, including a
twice-daily wipe down of passenger cabins and hallways. Elevators will
be disinfected every two hours. And in the ship's theater, crew members
will clean each pair of 3-D glasses before and after passengers use them
to watch movies.

Crew members will wear face masks, undergo temperature checks twice a
day and will no longer transfer to different ships.

Epidemiologists say the most important change the cruise lines could
make is more difficult: reducing the number of passengers on each
cruise.

``Of course, that has all kinds of implications for the industry
financially and otherwise,'' said William Schaffner, an infectious
disease expert at Vanderbilt University. ``But the notion of spacing
people apart starts with having fewer people on board.''

Genting is taking that step: Its boats will sail with up to 40 percent
fewer passengers.

``We will not be able to make the same money anymore,'' Mr. Zhu said.
``As long as we can keep our operations going when we resume --- enough
to keep the company going --- we will be happy enough for a few
months.''

Dr. Schaffner called the Genting guidelines ``pretty comprehensive,''
though he said that masks might need to be part of the equation.
(Genting's plan does not require passengers to wear masks or stay any
distance apart from each other onboard.) ``They have decided, and if we
were running the cruise industry, we might come to the same decision: If
we require passengers to wear masks, that's not the cruise experience,
that goes a step too far,'' he said.

But at the very least, he said, companies should offer to provide
passengers with masks when they get off the ship for day trips and
mingle with people on shore.

``That's a time period people should be very much encouraged'' to wear
masks, he said.

\hypertarget{not-getting-back-onboard}{%
\subsection{Not getting back onboard}\label{not-getting-back-onboard}}

Fred Kantrow and his wife were among the passengers who fell ill with
coronavirus after sailing aboard the Celebrity Eclipse, a cruise ship
owned by Royal Caribbean. Their daughter, who picked them up from the
airport on their return home, got sick too.

Mr. Kantrow, 59, a lawyer from Smithtown, N.Y., sued Celebrity after the
experience, saying they had not done enough to prevent the onboard
outbreak of the disease. The Eclipse was forced to sail for two extra
weeks when Chile refused to let passengers disembark; during that time
the ship continued to host crowded parties, photographs included in Mr.
Kantrow's lawsuit show.

The C.D.C. records show 92 people from that ship tested positive for
Covid-19, and Mr. Kantrow's suit claims two died.

``I don't know that they are going to be able to do anything to get me
back,'' he said. ``It's really hard to trust them. In two or three years
will my position change? Maybe. But when we got off ship, my wife said,
`Yeah, I'm not doing that again.'''

Mr. Kantrow's lawyer, Michael Winkleman, who has filed seven
Covid-related lawsuits against cruise companies, said Congress should
use the pandemic to better regulate the cruise industry, which does not
have to abide by U.S. labor laws or pay full corporate taxes, because
\href{https://www.nytimes3xbfgragh.onion/2020/04/08/travel/cruises-coronavirus-stimulus.html}{almost
all of the companies are foreign corporations}. The Death on the High
Seas Act, for example, limits how much families can claim when someone
dies on a cruise.

The industry's track record, he said, shows they will not make
proactively make changes that will hurt their bottom line. (CLIA's
spokeswoman, Ms. Golin-Blaugrund, disputed that idea saying that the
cruise lines ``have repeatedly demonstrated their willingness to invest
in public health and safety onboard and that the industry ``voluntarily
suspended operations globally.'')

``I think there are two forces stronger than the virus,'' Mr. Winkelman
said, ``the love that people have for cruises because it's such a unique
product, and the fact that the companies have so few hurdles and
roadblocks in front of them.''

\emph{David Yaffe-Bellany contributed reporting.}

\begin{center}\rule{0.5\linewidth}{\linethickness}\end{center}

\emph{\textbf{Follow New York Times Travel}}
\emph{on}\href{https://www.instagram.com/nytimestravel/}{\emph{Instagram}}\emph{,}\href{https://twitter.com/nytimestravel}{\emph{Twitter}}
\emph{and}\href{https://www.facebookcorewwwi.onion/nytimestravel/}{\emph{Facebook}}\emph{.
And}\href{https://www.nytimes3xbfgragh.onion/newsletters/traveldispatch}{\emph{sign
up for our weekly Travel Dispatch newsletter}} \emph{to receive expert
tips on traveling smarter and inspiration for your next vacation.}

Advertisement

\protect\hyperlink{after-bottom}{Continue reading the main story}

\hypertarget{site-index}{%
\subsection{Site Index}\label{site-index}}

\hypertarget{site-information-navigation}{%
\subsection{Site Information
Navigation}\label{site-information-navigation}}

\begin{itemize}
\tightlist
\item
  \href{https://help.nytimes3xbfgragh.onion/hc/en-us/articles/115014792127-Copyright-notice}{©~2020~The
  New York Times Company}
\end{itemize}

\begin{itemize}
\tightlist
\item
  \href{https://www.nytco.com/}{NYTCo}
\item
  \href{https://help.nytimes3xbfgragh.onion/hc/en-us/articles/115015385887-Contact-Us}{Contact
  Us}
\item
  \href{https://www.nytco.com/careers/}{Work with us}
\item
  \href{https://nytmediakit.com/}{Advertise}
\item
  \href{http://www.tbrandstudio.com/}{T Brand Studio}
\item
  \href{https://www.nytimes3xbfgragh.onion/privacy/cookie-policy\#how-do-i-manage-trackers}{Your
  Ad Choices}
\item
  \href{https://www.nytimes3xbfgragh.onion/privacy}{Privacy}
\item
  \href{https://help.nytimes3xbfgragh.onion/hc/en-us/articles/115014893428-Terms-of-service}{Terms
  of Service}
\item
  \href{https://help.nytimes3xbfgragh.onion/hc/en-us/articles/115014893968-Terms-of-sale}{Terms
  of Sale}
\item
  \href{https://spiderbites.nytimes3xbfgragh.onion}{Site Map}
\item
  \href{https://help.nytimes3xbfgragh.onion/hc/en-us}{Help}
\item
  \href{https://www.nytimes3xbfgragh.onion/subscription?campaignId=37WXW}{Subscriptions}
\end{itemize}
