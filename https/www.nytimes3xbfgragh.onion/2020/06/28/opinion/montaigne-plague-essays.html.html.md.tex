\href{/section/opinion}{Opinion}\textbar{}Montaigne Fled the Plague, and
Found Himself

\url{https://nyti.ms/3g07Q8a}

\begin{itemize}
\item
\item
\item
\item
\item
\end{itemize}

\includegraphics{https://static01.graylady3jvrrxbe.onion/images/2020/06/26/opinion/stone/stone-articleLarge.jpg?quality=75\&auto=webp\&disable=upscale}

Sections

\protect\hyperlink{site-content}{Skip to
content}\protect\hyperlink{site-index}{Skip to site index}

\href{/section/opinion}{Opinion}

\hypertarget{montaigne-fled-the-plague-and-found-himself}{%
\section{Montaigne Fled the Plague, and Found
Himself}\label{montaigne-fled-the-plague-and-found-himself}}

As disease and war ravaged the nation, he left town and invented the
essay.

Michel de MontaigneCredit...Getty Images

Supported by

\protect\hyperlink{after-sponsor}{Continue reading the main story}

By Robert Zaretsky

Mr. Zaretsky is a historian and author.

\begin{itemize}
\item
  June 28, 2020
\item
  \begin{itemize}
  \item
  \item
  \item
  \item
  \item
  \end{itemize}
\end{itemize}

In the summer of 1585, the mayor of Bordeaux learned, from the comfort
of his nearby chateau, that the bubonic plague had burst upon his city.
Those who could were fleeing, he was told, while those who could not
were ``dying like flies.'' What to do? His term in office, on the one
hand, was nearly over and his last official duty was to attend the
transition ceremony. On the other hand, perhaps his duty was with those
still inside the city walls.

Both hands on the reins of his horse, the mayor rode to the city's edge
and wrote to the municipal council to ask whether his life was worth a
transition ceremony. He did not seem to receive a reply and returned to
his chateau. By the time the plague subsided, more than 14,000 people
--- about a third of the city's population --- had died horrible deaths.
As for the former mayor, he returned to a far more pressing task: the
writing of essays.

The mayor was Michel de Montaigne. Known today as the author of the
``Essays,'' the classic of self-reflection and self-knowing, Montaigne
was better perhaps known in his own lifetime as a man of politics. Yet
his efforts --- quite literally, his \emph{essais ---} at politics and
his \emph{essais} at portraying himself are not unrelated. In both
cases, Montaigne probed the limits of what he could do in the world and
what he could know about himself.

Bordeaux was a hot spot for both bacteriological and theological plagues
in the late 1500s. The wars of religion, a series of eight distinct
conflicts between Catholics and Protestants --- replete with massacres
on both sides --- had ravaged France between 1562 and 1598. As both
mayor and diplomat, Montaigne tried several times to broker accords
between the two sides. He was known (and despised) by both sides as a
\emph{politique}: someone who, for the sake of all, tried to find common
ground in a land savaged by zealotry.

In this, Montaigne never succeeded yet he was not one to waste a plague.
In his essay ``On Physiognomy,'' written in 1585, he described the wars
as ``profitable disasters.'' The mutual butcheries, in effect, prepared
him for the next plague. The cruelty and fury, ambition and avarice that
consumed both sides taught him ``to rely on myself in distress.''

The trick, though, was to first find that self. Or, more accurately, to
\emph{found} that self. In effect, as he wrote and rewrote his essays
until his death in 1592, Montaigne wrote and rewrote his own self. In
``On Giving the Lie,'' he observed the strange alchemy between paper and
person, between writing one's life and becoming that life: ``I have no
more made my book than my book has made me --- a book consubstantial
with its author.''

More than a millennium earlier, thinkers like Epicurus and Seneca had
already mapped out this path. Inscribing their words on the pages of his
essays --- as well as in the roof beams of his library --- Montaigne
grasped that, unlike philosophers in his day (or our own), these
teachers sought not to inform their students, but instead to \emph{form}
them. As the classical scholar Pierre Hadot has argued, Stoicism and
Epicureanism offered not airy abstractions but real-world ``spiritual
exercises.'' Though the methods of these school varied, their mission
was the same: to teach students how to master physics and ethics not as
an end, but as the means to master their own selves and so better deal
with life's daily challenges, no less than its sudden catastrophes.

Yet self-mastery was itself a means to a greater end: the aligning of
the self with the world. The recognition of reality --- of what can and
cannot be changed --- teaches the need for self-control. This ``plague
of the utmost severity'' in 1585 challenged Montaigne's self-mastery
even more than the wars did. When the pestilence reached his estate, he
fled with his family in order to protect them. From the road, he
recalled, he saw peasants digging their own graves.

We will never know what these men and women thought when they saw
Montaigne and his household pass them on their horses and carriages. But
what should \emph{we} think? For many critics, Montaigne was, if not
clearly a coward, less than a hero: Imagine if Mayor Bill De Blasio,
learning that New York City had been struck by the coronavirus while he
was vacationing in the Berkshires, had emailed the City Council to wish
them good luck. Yet we need to remember that Montaigne never pretended
or sought to be a hero. Instead, he sought to do what could be done ---
in this case, save his family --- and sought to find what could be found
in this experience.

In the end, what he found was the essay --- less the masterpiece he had
written, though, than the life he had lived. In the many essays of his
life he discovered the importance of the moderate life. In his final
essay, ``On Experience,'' Montaigne reveals that ``greatness of soul is
not so much pressing upward and forward as knowing how to circumscribe
and set oneself in order.'' What he finds, quite simply, is the
importance of the moderate life. We must then, he writes, ``compose our
character, not compose books.'' There is nothing paradoxical about this
because his literary essays helped him better essay his life. The lesson
he takes from this trial might be relevant for our own trial: ``Our
great and glorious masterpiece is to live properly.''

Robert Zaretsky is a professor at the University of Houston and the
author of, most recently, ``Catherine \& Diderot: The Empress, the
Philosopher, and the Fate of the Enlightenment,'' and the forthcoming
``The Subversive Simone Weil: A Life in Five Ideas.''

\emph{\textbf{Now in print}}*:
``\emph{\href{http://bitly.com/1MW2kN3}{\emph{Modern Ethics in 77
Arguments}}},'' and ``\emph{\href{http://bitly.com/1MW2kN3}{\emph{The
Stone Reader: Modern Philosophy in 133 Arguments}}},'' with essays from
the series, edited by Peter Catapano and Simon Critchley, published by
Liveright Books.*

\emph{The Times is committed to publishing}
\href{https://www.nytimes3xbfgragh.onion/2019/01/31/opinion/letters/letters-to-editor-new-york-times-women.html}{\emph{a
diversity of letters}} \emph{to the editor. We'd like to hear what you
think about this or any of our articles. Here are some}
\href{https://help.nytimes3xbfgragh.onion/hc/en-us/articles/115014925288-How-to-submit-a-letter-to-the-editor}{\emph{tips}}\emph{.
And here's our email:}
\href{mailto:letters@NYTimes.com}{\emph{letters@NYTimes.com}}\emph{.}

\emph{Follow The New York Times Opinion section on}
\href{https://www.facebookcorewwwi.onion/nytopinion}{\emph{Facebook}}\emph{,}
\href{http://twitter.com/NYTOpinion}{\emph{Twitter (@NYTopinion)}}
\emph{and}
\href{https://www.instagram.com/nytopinion/}{\emph{Instagram}}\emph{.}

Advertisement

\protect\hyperlink{after-bottom}{Continue reading the main story}

\hypertarget{site-index}{%
\subsection{Site Index}\label{site-index}}

\hypertarget{site-information-navigation}{%
\subsection{Site Information
Navigation}\label{site-information-navigation}}

\begin{itemize}
\tightlist
\item
  \href{https://help.nytimes3xbfgragh.onion/hc/en-us/articles/115014792127-Copyright-notice}{©~2020~The
  New York Times Company}
\end{itemize}

\begin{itemize}
\tightlist
\item
  \href{https://www.nytco.com/}{NYTCo}
\item
  \href{https://help.nytimes3xbfgragh.onion/hc/en-us/articles/115015385887-Contact-Us}{Contact
  Us}
\item
  \href{https://www.nytco.com/careers/}{Work with us}
\item
  \href{https://nytmediakit.com/}{Advertise}
\item
  \href{http://www.tbrandstudio.com/}{T Brand Studio}
\item
  \href{https://www.nytimes3xbfgragh.onion/privacy/cookie-policy\#how-do-i-manage-trackers}{Your
  Ad Choices}
\item
  \href{https://www.nytimes3xbfgragh.onion/privacy}{Privacy}
\item
  \href{https://help.nytimes3xbfgragh.onion/hc/en-us/articles/115014893428-Terms-of-service}{Terms
  of Service}
\item
  \href{https://help.nytimes3xbfgragh.onion/hc/en-us/articles/115014893968-Terms-of-sale}{Terms
  of Sale}
\item
  \href{https://spiderbites.nytimes3xbfgragh.onion}{Site Map}
\item
  \href{https://help.nytimes3xbfgragh.onion/hc/en-us}{Help}
\item
  \href{https://www.nytimes3xbfgragh.onion/subscription?campaignId=37WXW}{Subscriptions}
\end{itemize}
