Sections

SEARCH

\protect\hyperlink{site-content}{Skip to
content}\protect\hyperlink{site-index}{Skip to site index}

\href{https://www.nytimes3xbfgragh.onion/section/opinion/sunday}{Sunday
Review}

\href{https://myaccount.nytimes3xbfgragh.onion/auth/login?response_type=cookie\&client_id=vi}{}

\href{https://www.nytimes3xbfgragh.onion/section/todayspaper}{Today's
Paper}

\href{/section/opinion/sunday}{Sunday Review}\textbar{}The Police Are
Rioting. We Need to Talk About It.

\url{https://nyti.ms/2A0XsOp}

\begin{itemize}
\item
\item
\item
\item
\item
\item
\end{itemize}

Advertisement

\protect\hyperlink{after-top}{Continue reading the main story}

\href{/section/opinion}{Opinion}

Supported by

\protect\hyperlink{after-sponsor}{Continue reading the main story}

\hypertarget{the-police-are-rioting-we-need-to-talk-about-it}{%
\section{The Police Are Rioting. We Need to Talk About
It.}\label{the-police-are-rioting-we-need-to-talk-about-it}}

It is an attack on civil society and democratic accountability.

\href{https://www.nytimes3xbfgragh.onion/column/jamelle-bouie}{\includegraphics{https://static01.graylady3jvrrxbe.onion/images/2019/01/24/opinion/jamelle-bouie/jamelle-bouie-thumbLarge-v3.png}}

By
\href{https://www.nytimes3xbfgragh.onion/column/jamelle-bouie}{Jamelle
Bouie}

Opinion Columnist

\begin{itemize}
\item
  June 5, 2020
\item
  \begin{itemize}
  \item
  \item
  \item
  \item
  \item
  \item
  \end{itemize}
\end{itemize}

\includegraphics{https://static01.graylady3jvrrxbe.onion/images/2020/06/05/opinion/05bouieWeb/merlin_173012772_17b77530-63c2-4752-8a62-9c205735fd25-articleLarge.jpg?quality=75\&auto=webp\&disable=upscale}

If we're going to speak of rioting protesters, then we need to speak of
rioting police as well. No, they aren't destroying property. But it is
clear from news coverage, as well as countless videos taken by
protesters and bystanders, that many officers are using often
indiscriminate violence against people --- against anyone, including the
peaceful majority of demonstrators, who happens to be in the streets.

Rioting police have
\href{https://twitter.com/stevemullis/status/1266979219566989313}{driven}
vehicles into crowds, reproducing the assault that killed Heather Heyer
in Charlottesville, Va., in 2017. They have
\href{https://twitter.com/imactuallynina/status/1266912627193774080}{surrounded}
a car, smashed the windows, tazed the occupants and dragged them out
onto the ground. Clad in paramilitary gear, they have
\href{https://twitter.com/bubbaprog/status/1266908354821206016}{attacked}
elderly bystanders, pepper-sprayed cooperative protesters and shot
``nonlethal'' rounds
\href{https://www.nytimes3xbfgragh.onion/2020/05/30/us/minneapolis-protests-press.html}{directly}
at reporters, causing serious injuries. In Austin, Texas, a 20-year-old
man is in
\href{https://www.kxan.com/investigations/everything-we-know-about-the-teenager-officers-shot-in-the-head-with-less-lethal-round-at-austin-protest/}{critical
condition} after being shot in the head with a ``less-lethal'' round.
Across the country, rioting police are using tear gas in quantities that
threaten the health and safety of demonstrators, especially in the midst
of a respiratory disease pandemic.

None of this quells disorder. Everything from the militaristic posture
to the attacks themselves does more to inflame and agitate protesters
than it does to calm the situation and bring order to the streets. In
effect, rioting police have done as much to stoke unrest and destabilize
the situation as those responsible for damaged buildings and burning
cars. But where rioting protesters can be held to account for
destruction and violence, rioting police have the imprimatur of the
state.

What we've seen from rioting police, in other words, is an assertion of
power and impunity. In the face of mass anger over police brutality,
they've effectively said \emph{So what?} In the face of demands for
change and reform --- in short, in the face of accountability to the
public they're supposed to serve --- they've bucked their more
conciliatory colleagues with a firm \emph{No.} In which case, if we want
to understand the behavior of the past two weeks, we can't just treat it
as an explosion of wanton violence; we have to treat it as an attack on
civil society and democratic accountability, one rooted in a dispute
over who has the right to hold the police to account.

African-American observers have never had any illusions about who the
police are meant to serve. The police, James Baldwin wrote
\href{https://www.esquire.com/news-politics/a3638/fifth-avenue-uptown/}{in
his 1960 essay} on discontent and unrest in Harlem, ``represent the
force of the white world, and that world's real intentions are simply
for that world's criminal profit and ease, to keep the black man
corralled up here in his place.'' This wasn't because each individual
officer was a bad person, but because he was fundamentally separate from
the black community as a matter of history and culture. ``None of the
police commissioner's men, even with the best will in the world, have
any way of understanding the lives led by the people they swagger about
in twos and threes controlling.''

Go back to the beginning of the 20th century, during America's first age
of progressive reform, as the historian Khalil Gibran Muhammad does in
``\href{https://www.hup.harvard.edu/catalog.php?isbn=9780674238145}{The
Condemnation of Blackness}: Race, Crime, and the Making of Modern Urban
America,'' and you'll find activists describing how ``policemen had
abdicated their responsibility to dispense color-blind service and
protection, resulting in an object lesson for youth: the indiscriminate
mass arrests of blacks being attacked by white mobs.''

The police were ubiquitous in the African-American neighborhoods of the
urban North, but they weren't there to protect black residents as much
as they were there to enforce the racial order, even if it led to actual
disorder in the streets. For example, in the aftermath of the
Philadelphia ``race riot'' of 1918, one black leader complained, ``In
nearly every part of this city peaceable and law-abiding Negroes of the
home-owning type have been set upon by irresponsible hoodlums, their
property damaged and destroyed, while the police seem powerless to
protect.''

If you are trying to understand the function of policing in American
society, then even a cursory glance at the history of the institution
would point you in the direction of social control. And blackness in
particular, the historian Nikhil Pal Singh argues, was a state of being
that required ``permanent supervision and sometimes direct domination.''

The simplest answer to the question ``Why don't the American police
forces act as if they are accountable to black Americans?'' is that they
were never intended to be. And to the extent that the police appear to
be rejecting accountability outright, I think it reflects the extent to
which the polity demanding it is now inclusive of those groups the
police have historically been tasked to control. That polity and its
leaders are simply rejected as legitimate wielders of authority over law
enforcement, especially when they ask for restraint.

A New York Police Department that worked enthusiastically with the
Republican mayors Rudy Giuliani and Michael Bloomberg --- mayors who
found their core support among the white residents of the city --- then
rejected the authority of Bill de Blasio, a Democrat backed by blacks
and Hispanics, who had emphasized police reform when he was a candidate.
Or compare the
\href{https://www.politico.com/story/2016/07/obama-war-on-cops-police-advocacy-group-225291}{contempt}
for President Barack Obama from representatives of law enforcement to
their
\href{https://www.cnn.com/videos/politics/2020/06/01/minneapolis-police-union-president-praise-trump-campaign-rally.cnn}{near-worshipful
posture} toward President Trump.

Yes, some of this reflects partisan politics --- it's in the nature of
policing that many of its practitioners tend to be more conservative
than most --- but I think it's also influenced by a sense that neither
Obama nor his appointees, like Eric Holder or Loretta Lynch, had the
right to criticize them or hold them to account.

If that is the dynamic at work, then we should not be surprised when the
police respond, in the main, with anger and contempt to demands for
change from the policed. Nor should we be surprised by their willingness
to follow the lead of a figure like Trump, who has
\href{https://apple.news/A8GNWDXdkSuuf3BPsi8KKUg}{incited} America's
police forces to be even more violent with protesters (to say nothing of
his
\href{https://www.newyorker.com/news/news-desk/donald-trump-is-serious-when-he-jokes-about-police-brutality}{past
praise} for police abuse).

Trump explicitly rejects the legitimacy of nonwhites as political
actors, having launched his political career on the need for more and
greater racial control of Muslims and Hispanic immigrants. Even without
his tough-guy posturing, Trump is someone who embodies the political and
social order the police have so often been called to defend.

Which is all to say that the nightly clashes between protesters and the
police are, to an extent, a microcosm of larger disputes roiling this
nation: the pressures and conflicts of a diversifying country; the
struggle to escape an exclusive past for a more inclusive future; and
our constant battle over who truly counts --- who can act as a full and
equal member of this society --- and who does not.

\emph{The Times is committed to publishing}
\href{https://www.nytimes3xbfgragh.onion/2019/01/31/opinion/letters/letters-to-editor-new-york-times-women.html}{\emph{a
diversity of letters}} \emph{to the editor. We'd like to hear what you
think about this or any of our articles. Here are some}
\href{https://help.nytimes3xbfgragh.onion/hc/en-us/articles/115014925288-How-to-submit-a-letter-to-the-editor}{\emph{tips}}\emph{.
And here's our email:}
\href{mailto:letters@NYTimes.com}{\emph{letters@NYTimes.com}}\emph{.}

\emph{Follow The New York Times Opinion section on}
\href{https://www.facebookcorewwwi.onion/nytopinion}{\emph{Facebook}}\emph{,}
\href{http://twitter.com/NYTOpinion}{\emph{Twitter (@NYTopinion)}}
\emph{and}
\href{https://www.instagram.com/nytopinion/}{\emph{Instagram}}\emph{.}

Advertisement

\protect\hyperlink{after-bottom}{Continue reading the main story}

\hypertarget{site-index}{%
\subsection{Site Index}\label{site-index}}

\hypertarget{site-information-navigation}{%
\subsection{Site Information
Navigation}\label{site-information-navigation}}

\begin{itemize}
\tightlist
\item
  \href{https://help.nytimes3xbfgragh.onion/hc/en-us/articles/115014792127-Copyright-notice}{©~2020~The
  New York Times Company}
\end{itemize}

\begin{itemize}
\tightlist
\item
  \href{https://www.nytco.com/}{NYTCo}
\item
  \href{https://help.nytimes3xbfgragh.onion/hc/en-us/articles/115015385887-Contact-Us}{Contact
  Us}
\item
  \href{https://www.nytco.com/careers/}{Work with us}
\item
  \href{https://nytmediakit.com/}{Advertise}
\item
  \href{http://www.tbrandstudio.com/}{T Brand Studio}
\item
  \href{https://www.nytimes3xbfgragh.onion/privacy/cookie-policy\#how-do-i-manage-trackers}{Your
  Ad Choices}
\item
  \href{https://www.nytimes3xbfgragh.onion/privacy}{Privacy}
\item
  \href{https://help.nytimes3xbfgragh.onion/hc/en-us/articles/115014893428-Terms-of-service}{Terms
  of Service}
\item
  \href{https://help.nytimes3xbfgragh.onion/hc/en-us/articles/115014893968-Terms-of-sale}{Terms
  of Sale}
\item
  \href{https://spiderbites.nytimes3xbfgragh.onion}{Site Map}
\item
  \href{https://help.nytimes3xbfgragh.onion/hc/en-us}{Help}
\item
  \href{https://www.nytimes3xbfgragh.onion/subscription?campaignId=37WXW}{Subscriptions}
\end{itemize}
