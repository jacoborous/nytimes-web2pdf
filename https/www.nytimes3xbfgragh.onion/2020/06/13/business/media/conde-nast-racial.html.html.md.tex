Sections

SEARCH

\protect\hyperlink{site-content}{Skip to
content}\protect\hyperlink{site-index}{Skip to site index}

\href{https://www.nytimes3xbfgragh.onion/section/business/media}{Media}

\href{https://myaccount.nytimes3xbfgragh.onion/auth/login?response_type=cookie\&client_id=vi}{}

\href{https://www.nytimes3xbfgragh.onion/section/todayspaper}{Today's
Paper}

\href{/section/business/media}{Media}\textbar{}A Reckoning at Condé Nast

\url{https://nyti.ms/2MTFjVc}

\begin{itemize}
\item
\item
\item
\item
\item
\end{itemize}

Advertisement

\protect\hyperlink{after-top}{Continue reading the main story}

Supported by

\protect\hyperlink{after-sponsor}{Continue reading the main story}

\hypertarget{a-reckoning-at-conduxe9-nast}{%
\section{A Reckoning at Condé Nast}\label{a-reckoning-at-conduxe9-nast}}

``It's hard to be a person of color at this company,'' a staff member
said. In response to an uprising, Anna Wintour and the chief executive,
Roger Lynch, offered apologies.

\includegraphics{https://static01.graylady3jvrrxbe.onion/images/2020/06/15/business/12ALTJPUnrest-CondeNast1-print/12Unrest-CondeNast-lede-articleLarge.jpg?quality=75\&auto=webp\&disable=upscale}

\href{https://www.nytimes3xbfgragh.onion/by/edmund-lee}{\includegraphics{https://static01.graylady3jvrrxbe.onion/images/2018/07/10/multimedia/author-edmund-lee/author-edmund-lee-thumbLarge.png}}

By \href{https://www.nytimes3xbfgragh.onion/by/edmund-lee}{Edmund Lee}

\begin{itemize}
\item
  June 13, 2020
\item
  \begin{itemize}
  \item
  \item
  \item
  \item
  \item
  \end{itemize}
\end{itemize}

This was supposed to be Condé Nast's year.

The publisher of Vogue, Vanity Fair and The New Yorker was going to be
profitable again after years of layoffs and losses.

Then advertising revenue suddenly dropped as the coronavirus pandemic
cratered the economy. More recently, as protests against racism and
police violence grew into a worldwide movement, company employees
publicly complained about racism in the workplace and in some Condé Nast
content.

In response, the two leaders of the
\href{https://www.condenast.com/about\#our-executive-leadership-team}{nearly
all-white} executive team --- the artistic director, Anna Wintour, and
the chief executive, Roger Lynch --- offered apologies to the staff.

At an all-hands online meeting on Friday, employees asked if Ms.
Wintour, the top editor of Vogue since 1988 and the company's
\href{https://www.nytimes3xbfgragh.onion/2013/03/13/business/media/conde-nast-creates-new-job-for-anna-wintour.html}{editorial
leader} since 2013, would be leaving. Mr. Lynch and the communications
chief, Danielle Carrig, shot down the question, saying Ms. Wintour was
not going anywhere, said three people who attended the meeting but were
not authorized to discuss it publicly.

\includegraphics{https://static01.graylady3jvrrxbe.onion/images/2020/06/12/business/12Unrest-CondeNast-02/merlin_169616391_de612378-a7ee-49ea-85ab-166c70b9589f-articleLarge.jpg?quality=75\&auto=webp\&disable=upscale}

Tumult has hit Condé Nast, a company built partly on selling a glossy
brand of elitism to the masses, at a time when its financial outlook is
grim. Last year, the U.S. division lost approximately \$100 million on
about \$900 million in revenue, said several people with knowledge of
the company, who were not authorized to speak publicly. The European arm
also had losses.

Mr. Lynch said in an interview Friday that he was ``not familiar'' with
the cited figures, adding that the company's
\href{https://www.nytimes3xbfgragh.onion/2018/11/27/business/conde-nast-ceo-robert-sauerberg.html}{merger}
of its domestic and international operations, part of a recent
restructuring, had been costly.

In April, the company instituted
\href{https://www.nytimes3xbfgragh.onion/2020/04/13/business/media/conde-nast-coronavirus-layoffs.html}{pay
cuts} for anyone making over \$100,000. Then came
\href{https://twitter.com/edmundlee/status/1260571924649869313}{layoffs}
--- 100 jobs gone out of roughly 6,000.

Condé Nast is one of many media organizations, including The New York
Times, whose employees have questioned company leaders as people around
the world have taken part in protests prompted by the killing of George
Floyd, a black man who died last month in Minneapolis after a white
police officer pinned him to the ground.

The company has been led by the Newhouse family since 1959. Steven
Newhouse heads the parent company, Advance, and his cousin Jonathan
Newhouse is chairman of Condé Nast's board. Advance also controls more
than 40 newspapers and news sites across the country. Many of them,
including The Plain Dealer of Cleveland and The Star-Ledger in Newark,
have
\href{https://www.nytimes3xbfgragh.onion/2019/05/12/us/new-orleans-advocate-times-picayune.html}{struggled}.
The Newhouse family has protected itself against losses with significant
investments in the cable giant Charter and the media conglomerate
Discovery.

Before the internet took readers away from print, Condé Nast was known
for thick magazines edited by cultural arbiters who traveled in the same
circles as the people they covered. As digital media rose, Condé Nast
was slow to adapt. Budgets tightened. Magazines including Gourmet,
Mademoiselle and Details folded.

By the time Mr. Lynch, a former head of the music streaming service
Pandora, succeeded Robert A. Sauerberg as the chief executive last year,
Condé Nast was in triage mode. After his arrival, it unloaded three
publications: Brides, Golf Digest and W.

On Monday, Condé Nast reckoned with how the company deals with issues
related to race. Adam Rapoport, the longtime top editor of Bon Appétit,
\href{https://www.nytimes3xbfgragh.onion/2020/06/08/dining/bon-appetit-adam-rapoport.html}{resigned}
after a photo surfaced on social media showing him in a costume that
stereotypically depicted Puerto Rican dress.

Image

Adam Rapoport resigned as Bon Appétit's top editor after a photo of him
in a racially insensitive costume surfaced.Credit...Bryan Bedder/Getty
Images

He apologized to staff members in a videoconference. After Mr. Rapoport
left the call, the staff voiced complaints about the Bon Appétit
workplace. Some minority employees said they had been used as ethnic
props in Bon Appétit's videos, a growing segment of the Condé Nast
business.

``It's so hard to be a person of color at this company,'' said
\href{https://www.instagram.com/ry_who/}{Ryan Walker-Hartshorn}, a black
woman who worked as an assistant to Mr. Rapoport. ``My blood is still
boiling.''

She recalled a 2018 meeting of editors to discuss how to make the
magazine's Instagram account more diverse. In a room of about eight
editors, three were people of color.

``And we're all very junior, no power,'' Ms. Walker-Hartshorn said in an
interview. ``I was like, `You're asking us how to make our Instagram
black without hiring more black people?'''

At a company forum on Tuesday, Mr. Lynch said Bon Appétit employees
should have raised their concerns earlier, a comment that rubbed many
the wrong way. In a closed-door session later that day, he apologized to
a group of staff members who had pushed for Mr. Rapoport's ouster.

``I want you to know I take this personally, and I take personal
responsibility for it,'' he said, according to an audio recording of the
meeting obtained by The New York Times.

A onetime banker at Morgan Stanley, Mr. Lynch spent much of his career
at Dish, the satellite TV service. As a hobby he played
\href{https://twitter.com/rogerlynch/status/885913558512488448?lang=en}{lead
guitar} in a classic-rock cover band, the Merger. He moved from San
Francisco to New York and updated his wardrobe to join Condé Nast.

Mr. Lynch, 57, has emphasized diversity efforts and environmental
programs in emails to the staff. He said in the interview on Friday that
he was developing an overall company strategy as he assembled his
executive team. In December he
\href{https://www.condenast.com/news/new-cfo-and-cmo}{hired} Deirdre
Findlay as the chief marketing officer, making her the company's
highest-ranking black executive.

Image

Mr. Lynch in 2018, when he ran Pandora. He became chief executive of
Condé Nast last spring.Credit...Patrick T. Fallon/Bloomberg

His former executive assistant, Cassie Jones, who is black, quit shortly
after he gave her a gift she considered insulting, three people with
knowledge of the matter said.

In November, after she had spent four months working for him, Mr. Lynch
called Ms. Jones into his office and handed her ``The Elements of
Style,'' a guide to standard English usage by William Strunk Jr. and
E.B. White. Mr. Lynch said he thought she could benefit from it.

With its suggestion that her own language skills were lacking, the gift
struck Ms. Jones as a microaggression, the people said. A few days
later, she quit. Before leaving the headquarters at 1 World Trade in
Lower Manhattan, she placed the book on his desk.

Mr. Lynch said he hadn't meant to insult Ms. Jones, who declined to
comment for this article. ``I really only had the intention --- like
every time I've given it before --- for it to be a helpful resource, as
it has been for me,'' he said. ``I still use it today. I'm really sorry
if she interpreted it that way.''

Before Mr. Lynch's arrival, David Remnick, the editor in chief of The
New Yorker, objected to a plan that would have lowered the magazine's
subscription price and raised ad rates. He has brought aboard a diverse
crew of journalists, including Jia Tolentino, Hua Hsu and Vinson
Cunningham, while adding digital subscriptions.

Three people with knowledge of the company said The New Yorker was
likely to surpass Vogue as Condé Nast's biggest contributor to U.S.
profits by the end of 2020. The people added that about 80 percent of
The New Yorker's revenue came from readers, which helped the magazine
weather the advertising downturn. The magazine did not cut staff during
the recent layoffs.

Image

Condé Nast, with headquarters in Lower Manhattan, has cut the pay of
employees making over \$100,000 and laid off 100
workers.Credit...Vincent Tullo for The New York Times

On June 4, Ms. Wintour sent an apologetic note to the Vogue staff. ``I
want to say this especially to the Black members of our team --- I can
only imagine what these days have been like,'' Ms. Wintour wrote.

She added, ``I want to say plainly that I know Vogue has not found
enough ways to elevate and give space to Black editors, writers,
photographers, designers and other creators. We have made mistakes, too,
publishing images or stories that have been hurtful or intolerant. I
take full responsibility for those mistakes.''

The British-born Ms. Wintour has been credited internally for
championing Radhika Jones, one of
\href{https://runway.blogs.nytimes3xbfgragh.onion/2012/10/04/african-american-chosen-as-editor-at-brides/}{few}
top editors of color in the company's history.

Ms. Jones, the former editorial director of the book department at The
Times who
\href{https://www.nytimes3xbfgragh.onion/2017/11/11/business/media/vanity-fair-editor.html}{took
over} Vanity Fair from
\href{https://www.nytimes3xbfgragh.onion/2017/09/07/business/media/graydon-carter-vanity-fair.html}{Graydon
Carter} in 2017, changed the magazine's identity. The first cover
subject she chose, for the April 2018 issue, was the actress and
producer Lena Waithe, a black woman photographed by Annie Leibovitz in a
plain T-shirt. Later covers featured Michael B. Jordan, Janelle Monae
and Lin-Manuel Miranda. Ms. Jones has put out 16 Vanity Fair covers
featuring people of color.

When Ms. Jones arrived, she was pilloried by fashion insiders who
questioned her style sense. Her choice of legwear --- tights with
illustrated foxes --- drew stares, according to a
\href{https://wwd.com/business-news/media/vanity-fair-fashion-staff-nonplussed-new-editor-personal-style-11051422/}{report}
in Women's Wear Daily. Ms. Wintour later showed her support for Ms.
Jones at a
\href{https://www.instagram.com/p/Bcs_iB3Fktr/?utm_source=ig_embed}{welcome
party} by handing out gifts: tights with foxes on them.

Image

Vanity Fair's top editor, Radhika Jones, sat through a difficult meeting
early in her time at the magazine.Credit...Michael Kovac/Getty Images

At a quarterly meeting of company executives in April 2019, on Mr.
Lynch's second day at Condé Nast, Ms. Jones presented her plan for
Vanity Fair's fall issues, a prime landing spot for fashion and luxury
advertisers. (From September to December last year, the Vanity Fair
covers featured Kristen Stewart, Lupita Nyong'o, Joaquin Phoenix, and
Chrissy Teigen, John Legend and their children.)

Two executives criticized Ms. Jones's plan, according to three people
who were at the meeting and were not authorized to discuss it publicly.
In particular, Susan Plagemann, the chief business officer of Condé
Nast's style division, challenged Ms. Jones at length, saying the plan
would be difficult to sell to advertisers. To defuse the tension, Ms.
Wintour banged her fist on the table, saying, ``We need to move on,''
according to the three people who were at the meeting.

Ms. Plagemann, who is white, joined the company in 2010 as Vogue's chief
business officer and worked closely with Ms. Wintour; in 2018, she was
elevated to her current job. Three people with knowledge of the matter
said she was vocal about her negative view of Vanity Fair under its new
editor.

She had criticized Ms. Jones's choices of cover subjects, telling others
at the company that the magazine should feature ``more people who look
like us,'' two of the people said. A third person said he had heard her
use words expressing a similar sentiment. All the people said they
interpreted the phrase and similar remarks as referring to well-off
white women who adopt an aesthetic common among the fashion set.

Through a Condé Nast spokesman, Ms. Plagemann denied making those
statements and denied expressing a dim view of Ms. Jones's Vanity Fair.

In the interview on Friday, Mr. Lynch addressed Ms. Jones's stewardship
of the magazine more broadly. ``The challenge with her taking that new
direction would be alienating some of the traditional Vanity Fair
audience,'' he said. ``I really applaud what she's done.''

The uprising at Condé Nast was overdue, some staff members said. ``We've
been asking for change for months now,'' Sohla El-Waylly, an assistant
editor at Bon Appétit, said in an interview.

In the Tuesday meeting with Bon Appétit staff members, Mr. Lynch said he
hoped to prove a commitment to diversity with the choice of Mr.
Rapoport's replacement. Later in the call, he suggested that some staff
members wanted to hurt Bon Appétit financially to bring about change, a
comment that irked some in the meeting.

``It felt infantilizing, as if we were teenagers rebelling,'' said Jesse
Sparks, an editorial assistant.

Mr. Lynch said in the interview that he had meant to underscore the
urgency of the matter. ``I wanted to make sure they understood the brand
they worked so hard to build was actually being harmed, and I think I
even apologized to them in that meeting,'' he said.

A Bon Appétit personality, Claire Saffitz, has generated over
\href{https://www.buzzfeednews.com/article/laurenstrapagiel/bon-appetit-test-kitchen-youtube-brad-claire}{200
million views} with ``Gourmet Makes,'' a show in which she makes
homemade versions of Twinkies and other junk food. She represents a new
kind of Condé Nast, one built on a kind of rough-cut authenticity, but
her popularity has drawn attention to the problem of representation.

Image

``We've been asking for change for months now,'' said Sohla El-Waylly,
an assistant editor at Bon Appétit.Credit...Francesco Sapienza for The
New York Times

Ms. El-Waylly, who was a regular guest on the show, said her addition to
``Gourmet Makes'' had been cynically motivated. ``They just want me
there to play the part to make it look like they have people of color on
staff,'' she said.

She said she was not paid for her appearances, as her white counterparts
were. Condé Nast disputed that and said Ms. El-Waylly's salary covered
her video appearances.

On Wednesday, the company's head of video, Matt Duckor,
\href{https://www.nytimes3xbfgragh.onion/2020/06/10/business/conde-nast-matt-duckor.html}{stepped
down}. Several employees had accused him of bias. Many people at the
company are rooting for more change.

``What's crazy is what it took for this stuff to happen,'' Ms.
Walker-Hartshorn said. ``It took George Floyd.''

Advertisement

\protect\hyperlink{after-bottom}{Continue reading the main story}

\hypertarget{site-index}{%
\subsection{Site Index}\label{site-index}}

\hypertarget{site-information-navigation}{%
\subsection{Site Information
Navigation}\label{site-information-navigation}}

\begin{itemize}
\tightlist
\item
  \href{https://help.nytimes3xbfgragh.onion/hc/en-us/articles/115014792127-Copyright-notice}{©~2020~The
  New York Times Company}
\end{itemize}

\begin{itemize}
\tightlist
\item
  \href{https://www.nytco.com/}{NYTCo}
\item
  \href{https://help.nytimes3xbfgragh.onion/hc/en-us/articles/115015385887-Contact-Us}{Contact
  Us}
\item
  \href{https://www.nytco.com/careers/}{Work with us}
\item
  \href{https://nytmediakit.com/}{Advertise}
\item
  \href{http://www.tbrandstudio.com/}{T Brand Studio}
\item
  \href{https://www.nytimes3xbfgragh.onion/privacy/cookie-policy\#how-do-i-manage-trackers}{Your
  Ad Choices}
\item
  \href{https://www.nytimes3xbfgragh.onion/privacy}{Privacy}
\item
  \href{https://help.nytimes3xbfgragh.onion/hc/en-us/articles/115014893428-Terms-of-service}{Terms
  of Service}
\item
  \href{https://help.nytimes3xbfgragh.onion/hc/en-us/articles/115014893968-Terms-of-sale}{Terms
  of Sale}
\item
  \href{https://spiderbites.nytimes3xbfgragh.onion}{Site Map}
\item
  \href{https://help.nytimes3xbfgragh.onion/hc/en-us}{Help}
\item
  \href{https://www.nytimes3xbfgragh.onion/subscription?campaignId=37WXW}{Subscriptions}
\end{itemize}
