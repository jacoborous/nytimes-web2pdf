Sections

SEARCH

\protect\hyperlink{site-content}{Skip to
content}\protect\hyperlink{site-index}{Skip to site index}

\href{https://www.nytimes3xbfgragh.onion/section/us}{U.S.}

\href{https://myaccount.nytimes3xbfgragh.onion/auth/login?response_type=cookie\&client_id=vi}{}

\href{https://www.nytimes3xbfgragh.onion/section/todayspaper}{Today's
Paper}

\href{/section/us}{U.S.}\textbar{}Christopher Columbus Statues in
Boston, Minnesota and Virginia Are Damaged

\url{https://nyti.ms/2AdPxNO}

\begin{itemize}
\item
\item
\item
\item
\item
\end{itemize}

\href{https://www.nytimes3xbfgragh.onion/news-event/george-floyd-protests-minneapolis-new-york-los-angeles?action=click\&pgtype=Article\&state=default\&region=TOP_BANNER\&context=storylines_menu}{Race
and America}

\begin{itemize}
\tightlist
\item
  \href{https://www.nytimes3xbfgragh.onion/2020/07/26/us/protests-portland-seattle-trump.html?action=click\&pgtype=Article\&state=default\&region=TOP_BANNER\&context=storylines_menu}{Protesters
  Return to Other Cities}
\item
  \href{https://www.nytimes3xbfgragh.onion/2020/07/24/us/portland-oregon-protests-white-race.html?action=click\&pgtype=Article\&state=default\&region=TOP_BANNER\&context=storylines_menu}{Portland
  at the Center}
\item
  \href{https://www.nytimes3xbfgragh.onion/2020/07/23/podcasts/the-daily/portland-protests.html?action=click\&pgtype=Article\&state=default\&region=TOP_BANNER\&context=storylines_menu}{Podcast:
  Showdown in Portland}
\item
  \href{https://www.nytimes3xbfgragh.onion/interactive/2020/07/16/us/black-lives-matter-protests-louisville-breonna-taylor.html?action=click\&pgtype=Article\&state=default\&region=TOP_BANNER\&context=storylines_menu}{45
  Days in Louisville}
\end{itemize}

Advertisement

\protect\hyperlink{after-top}{Continue reading the main story}

Supported by

\protect\hyperlink{after-sponsor}{Continue reading the main story}

\hypertarget{christopher-columbus-statues-in-boston-minnesota-and-virginia-are-damaged}{%
\section{Christopher Columbus Statues in Boston, Minnesota and Virginia
Are
Damaged}\label{christopher-columbus-statues-in-boston-minnesota-and-virginia-are-damaged}}

The incidents came as protesters angered by the death of George Floyd
have targeted monuments that they see as symbols of white supremacy.

\includegraphics{https://static01.graylady3jvrrxbe.onion/images/2020/06/12/us/12video/10xp-unrest-columbus1-videoSixteenByNine3000.jpg}

\href{https://www.nytimes3xbfgragh.onion/by/johnny-diaz}{\includegraphics{https://static01.graylady3jvrrxbe.onion/images/2019/11/05/reader-center/author-johnny-diaz/author-johnny-diaz-thumbLarge.png}}

By \href{https://www.nytimes3xbfgragh.onion/by/johnny-diaz}{Johnny Diaz}

\begin{itemize}
\item
  Published June 10, 2020Updated July 24, 2020
\item
  \begin{itemize}
  \item
  \item
  \item
  \item
  \item
  \end{itemize}
\end{itemize}

\href{https://www.nytimes3xbfgragh.onion/2020/06/11/us/Jefferson-Davis-Statue-Richmond.html}{Statues}
of
\href{https://www.nytimes3xbfgragh.onion/2020/07/24/us/christopher-columbus-chicago.html}{Christopher
Columbus} were damaged in Minnesota, Boston and Richmond, Va., as
protesters angered by the death of George Floyd have continued to direct
some of their frustration toward
\href{https://www.nytimes3xbfgragh.onion/2020/06/11/us/Jefferson-Davis-Statue-Richmond.html}{monuments},
\href{https://www.nytimes3xbfgragh.onion/2020/06/03/us/confederate-statues-george-floyd.html}{including
Confederate statues}, that they consider to be symbols of racism.

Outside the State Capitol in St. Paul, Minn., a 10-foot bronze sculpture
of Columbus came toppling down on Wednesday afternoon after a group of
protesters tied ropes around the statue's neck and yanked it from its
pedestal.

The demonstrators kicked the head of the
\href{https://www.nytimes3xbfgragh.onion/2020/06/15/arts/design/fallen-statues-what-next.html}{statue},
which was dedicated in 1931, and danced around it. Minnesota State
Troopers did not intervene.

The group
\href{https://twitter.com/maxnesterak/status/1270839254462201867}{included
local Native Americans}, according to reporters at the scene. Native
Americans around the country
\href{https://www.history.com/news/columbus-day-controversy}{have
rejected the legacy} of Columbus as an explorer.

The Capitol is about 10 miles from where a Minneapolis police officer
pinned his knee on Mr. Floyd's neck for nearly 9 minutes.

In Boston, the head of a statue of Columbus in the city's North End
neighborhood was removed overnight on Tuesday, and pieces of it were
found nearby, Sgt. Detective John Boyle of the Boston police said on
Wednesday. Detectives were investigating the incident, he said.

In Richmond on Tuesday evening, a Columbus statue was torn down and
tossed into a lake in a city park where protesters had gathered for a
demonstration in support of Indigenous peoples.

``We stand in solidarity with black and brown communities that are tired
of being murdered by an out-of-control, militarized and violent police
force,'' the Richmond Indigenous Society, which took part in the rally,
said in a statement on Wednesday.

``As for the statue,'' the society added, ``it seems very appropriate
that it ended up in a lake.''

Mayor Marty Walsh of Boston said at a news conference on Wednesday that
the damaged Columbus statue would be removed and stored while city
officials discuss whether it should be returned to its location in
Christopher Columbus Park, where the six-foot statue had stood atop a
five-foot base since 1979.

``We are going to be taking the statue down this morning and putting it
into storage to assess the damage of the statue,'' Mr. Walsh said.
``This particular statue has been subject of repeated vandalism here in
Boston and, given the conversations that we're certainly having right
now in our city of Boston and throughout the country, we're also going
to take time to assess the historic meaning of the statue.''

About 1,000 protesters attended the rally at Byrd Park in Richmond,
where the Columbus statue was damaged. Demonstrators carried signs that
said ``This land is Powhatan land'' and ``Columbus represents
genocide,'' according to The
\href{https://www.richmond.com/news/local/watch-now-protesters-stand-in-solidarity-with-indigenous-peoples-at-byrd-park-where-columbus-statue/article_8a009c9c-1c5d-5e2a-a3bf-0b015a8a2277.html}{Richmond
Times-Dispatch.}

The statue was spray-painted, set on fire and thrown into a lake in the
\href{http://www.richmondgov.com/parks/parkbyrd.aspx}{287-acre park},
according to
\href{https://www.nbc12.com/2020/06/09/christopher-columbus-statue-torn-down-thrown-lake-by-protesters/}{WWBT-TV}.
Near the bank of the lake where the statue was submerged, someone placed
a sign with a drawing of a headstone and a message: ``Racism, you will
not be missed.''

The statue was
\href{https://twitter.com/OliviaNBC12/status/1270697627731013634}{removed
from the lake} on Wednesday. The city's Police Department did not
immediately respond to an inquiry about the statue on Wednesday.

The statue, which was dedicated in December 1927, was the first statue
of Columbus erected in the South, according to The Times-Dispatch.

Native Americans have been calling for states to replace Columbus Day,
which is celebrated the second Monday in October, with
\href{https://www.nytimes3xbfgragh.onion/2019/10/14/nyregion/indigenous-day.html}{Indigenous
Peoples' Day}. They have said that Columbus's discovery of the New World
led to the genocide of Indigenous populations in the Americas.

Mayor Levar Stoney of Richmond
\href{https://twitter.com/LevarStoney/status/1270794688140849152}{responded
to the incident in Byrd Park on Twitter} on Wednesday, saying that ``the
atrocities inflicted upon Indigenous people by Christopher Columbus are
unconscionable.'' He added that the city began celebrating Indigenous
Peoples' Day rather than Columbus Day last year.

``But the decision \& action to remove a monument should be made in
collaboration w/ the community,'' he wrote. ``Working with Richmond's
History and Culture Commission, we are establishing a process by which
Richmonders can advocate for change to the figures we place on public
pedestals across our city in a legal and peaceful way.''

The Richmond statue was located about two miles from the city's Monument
Avenue, where statues of the Confederate generals J.E.B. Stuart,
\href{https://www.nytimes3xbfgragh.onion/2020/07/02/us/stonewall-jackson-statue-richmond.html}{Stonewall
Jackson} and Robert E. Lee were marked last week.

\includegraphics{https://static01.graylady3jvrrxbe.onion/images/2020/06/10/us/10xp-columbus1/10xp-columbus1-videoSixteenByNineJumbo1600.jpg}

On June 4, Gov. Ralph S. Northam of Virginia
\href{https://www.governor.virginia.gov/newsroom/all-releases/2020/june/headline-857181-en.html}{announced}
plans to remove the Lee statue, which is on state land. Members of the
Richmond City Council this week unanimously voiced their support to
remove
\href{https://www.richmond.com/news/local/its-unanimous-all-nine-richmond-city-council-members-back-removal-of-confederate-monuments-on-monument/article_a639a9e9-6757-5278-8da5-bf498241afb9.html}{four
other Confederate statues} from Monument Avenue, a move that had long
been debated but had found renewed support as the recent protests have
continued.

Many other Confederate monuments across the country have also become
targets in recent weeks, with protesters damaging and toppling statues
from their bases.

In Norfolk, Va., on May 30, protesters climbed a 15-foot figure of a
Confederate soldier and spray-painted its base. In Charleston, S.C.,
``BLM,'' for Black Lives Matter, and ``Traitors'' were spray-painted in
red on the base of \href{https://www.hmdb.org/m.asp?m=120742}{the
Confederate Defenders of Charleston statue}. In North Carolina,
\href{https://www.ncpedia.org/monument/confederate-monument-state}{a
Confederate monument at the State Capitol} in Raleigh was marked with a
black ``X.''

The mayor of Birmingham, Ala., also
\href{https://www.nytimes3xbfgragh.onion/2020/06/02/us/george-floyd-birmingham-confederate-statue.html}{ordered
the removal} of a Confederate statue from a public park last week.

Neil Vigdor contributed reporting.

Advertisement

\protect\hyperlink{after-bottom}{Continue reading the main story}

\hypertarget{site-index}{%
\subsection{Site Index}\label{site-index}}

\hypertarget{site-information-navigation}{%
\subsection{Site Information
Navigation}\label{site-information-navigation}}

\begin{itemize}
\tightlist
\item
  \href{https://help.nytimes3xbfgragh.onion/hc/en-us/articles/115014792127-Copyright-notice}{©~2020~The
  New York Times Company}
\end{itemize}

\begin{itemize}
\tightlist
\item
  \href{https://www.nytco.com/}{NYTCo}
\item
  \href{https://help.nytimes3xbfgragh.onion/hc/en-us/articles/115015385887-Contact-Us}{Contact
  Us}
\item
  \href{https://www.nytco.com/careers/}{Work with us}
\item
  \href{https://nytmediakit.com/}{Advertise}
\item
  \href{http://www.tbrandstudio.com/}{T Brand Studio}
\item
  \href{https://www.nytimes3xbfgragh.onion/privacy/cookie-policy\#how-do-i-manage-trackers}{Your
  Ad Choices}
\item
  \href{https://www.nytimes3xbfgragh.onion/privacy}{Privacy}
\item
  \href{https://help.nytimes3xbfgragh.onion/hc/en-us/articles/115014893428-Terms-of-service}{Terms
  of Service}
\item
  \href{https://help.nytimes3xbfgragh.onion/hc/en-us/articles/115014893968-Terms-of-sale}{Terms
  of Sale}
\item
  \href{https://spiderbites.nytimes3xbfgragh.onion}{Site Map}
\item
  \href{https://help.nytimes3xbfgragh.onion/hc/en-us}{Help}
\item
  \href{https://www.nytimes3xbfgragh.onion/subscription?campaignId=37WXW}{Subscriptions}
\end{itemize}
