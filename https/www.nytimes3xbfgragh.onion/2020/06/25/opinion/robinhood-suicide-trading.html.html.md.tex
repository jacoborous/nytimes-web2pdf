Sections

SEARCH

\protect\hyperlink{site-content}{Skip to
content}\protect\hyperlink{site-index}{Skip to site index}

\href{https://myaccount.nytimes3xbfgragh.onion/auth/login?response_type=cookie\&client_id=vi}{}

\href{https://www.nytimes3xbfgragh.onion/section/todayspaper}{Today's
Paper}

\href{/section/opinion}{Opinion}\textbar{}A Suicide, an App and a Time
for a Reckoning

\url{https://nyti.ms/31gmqV5}

\begin{itemize}
\item
\item
\item
\item
\item
\item
\end{itemize}

Advertisement

\protect\hyperlink{after-top}{Continue reading the main story}

\href{/section/opinion}{Opinion}

Supported by

\protect\hyperlink{after-sponsor}{Continue reading the main story}

\hypertarget{a-suicide-an-app-and-a-time-for-a-reckoning}{%
\section{A Suicide, an App and a Time for a
Reckoning}\label{a-suicide-an-app-and-a-time-for-a-reckoning}}

Companies like the stock-trading app Robinhood can seem not just
careless but also predatory.

\includegraphics{https://static01.graylady3jvrrxbe.onion/images/2018/08/02/opinion/02swisher/02swisher-thumbLarge.png}

By Kara Swisher

Ms. Swisher covers technology and is a contributing Opinion writer.

\begin{itemize}
\item
  June 25, 2020
\item
  \begin{itemize}
  \item
  \item
  \item
  \item
  \item
  \item
  \end{itemize}
\end{itemize}

\includegraphics{https://static01.graylady3jvrrxbe.onion/images/2020/06/25/opinion/25swisher1/merlin_173900673_0b55ad84-931d-43b7-9134-d2c207ea4965-articleLarge.jpg?quality=75\&auto=webp\&disable=upscale}

I spent a lot of time this week trying to come up with the best way to
get those who make things in Silicon Valley to better understand the
suicide of Alex Kearns, a student at the University of Nebraska. He
killed himself after he mistakenly believed that he had a \$730,000
negative balance on the millennial-popular Robinhood app, which he had
downloaded to learn about investing.

The tragedy got a lot of attention, especially after
\href{https://www.forbes.com/sites/sergeiklebnikov/2020/06/17/20-year-old-robinhood-customer-dies-by-suicide-after-seeing-a-730000-negative-balance/\#7b4952da1638}{Forbes
reported} that Mr. Kearns left a note behind asking, ``How was a
20-year-old with no income able to get assigned almost a million dollars
of leverage?''

How, indeed.

Embedded in that query is a much bigger one that has been plaguing the
tech industry and its innovative entrepreneurs for far too long: What is
the reason for their persistent tendency to ignore the potentially
dangerous impact of their creations? These days the companies can seem
not just careless but also predatory.

Is it to make more money? Is it because growth trumps safety? Is it rank
sloppiness? Lack of foresight? A design flaw that could have and, more
to the point, \emph{should} have been anticipated? A laser focus on
innovation? All of the above?

Perhaps the reason hardly matters, since, as Robert Louis Stevenson
wrote, ``Everybody, soon or late, sits down at a banquet of
consequences.''

And that is the ashen meal now in front of Robinhood's co-founders and
co-chief executives, Vlad Tenev and Baiju Bhatt.

The company, based in Menlo Park, Calif., has become a phenom since its
founding in 2013. Using an addictive and hyper-gamified interface,
seamless and instant onboarding to the glories of trading and an
occasional dollop of animated confetti to jack up the fun factor,
Robinhood use has grown fast among young people, many of whom are
newbies to investing.

While there have been troubling outages and worries that the app has
been designed to feel more like a casino than a sober pathway for
important financial decisions, the longtime idea of democratizing
entrance to the stock market --- I covered the advent of popular online
trading companies like eTrade in the late 1990s --- is an important one.
And there is no doubt that Robinhood has been among the most innovative
standouts in the fintech space (especially in removing friction, which
is at the heart of this app, and in offering new features like
fractional shares).

That success has attracted \$1.2 billion in investments from eager
venture capital firms, including \$280 million last month from the
top-tier investor Sequoia Capital, bringing Robinhood's valuation to
\$8.3 billion.

And the pandemic has been a boon to the company, making it a good bet
for investors. Robinhood has added an astonishing three million accounts
in the first quarter of this year, bringing the total to 13 million.

A growth chart that goes up and to the right is all well and good, but
has Robinhood, as one person said to me, ``made the classic Silicon
Valley mistake of applying games and brain hacks to an extremely
important sector,'' even as it underinvested in key parts of the
business like customer service?

Robinhood has certainly doubled down on eliminating friction, which in
Silicon Valley is almost like a religious tenet. Too many clicks are
akin to a major sin to techies, as are too many ``just a sec'' **
warnings, even though most people find stop signs useful if irritating
in real life.

It's not clear yet how things went awry in Mr. Kearns's case, except
that the way the app rendered his account before his death appeared to
make him think that he was deep in a financial hole. Of his own accord,
he was engaged in complex options trading, but without much oversight on
the transactions and without enough information about options trading on
the app. All added up, the calculation proved deadly for him.

Options trading can be very risky --- and is not recommended for the
inexperienced investor like Mr. Kearns. To do it, Robinhood requires an
eligibility questionnaire and for a user to certify his investing
experience, along with signing an acknowledgment of risk and a promise
to read its materials on the topic. Other brokerage firms provide more
substantive interactions on their riskiest financial instruments and
even offer to explain risks in real-time conversations before allowing
an investment to proceed.

One investor I spoke with likened the Robinhood experience to giving a
Ferrari to a kid without a driver's license. That kind of carelessness
is especially problematic when it comes to young men, who studies have
shown are more attracted to online trading, especially because of its
often addictive characteristics, and whose emotional investment can be
too high.

In an interview with me this week, Mr. Tenev said the company could not
comment on the specifics of Mr. Kearns's account because of privacy
concerns. But clearly chastened and shaken by the tragedy, he
acknowledged that fast growth has been a management issue.

``We had our challenges and had not anticipated 2020 shaking out the way
it did,'' Mr. Tenev said. ``Customer support has been strained, but the
team has been hiring.''

A company representative said Robinhood had grown its customer service
department by over 40 percent already this year, which has included
adding registered financial-services professionals. And the company says
it wants to double its customer service staff by the end of the year.

Mr. Tenev stressed again what he and Mr. Bhatt had written about Mr.
Kearns in an unusually \href{https://blog.robinhood.com/}{self-critical
blog post}: Robinhood announced a commitment to more investment in app
education resources, significant changes in its interface around options
and a \$250,000 donation to suicide prevention.

``It is certainly not lost upon us that Robinhood, especially with
retail investing in America, has recognized that we have a huge
responsibility,'' he said. ``We want to be out in front of issues and be
the best and safest options platform.''

Still, some think Robinhood failed Mr. Kearns. Bill Brewster, a
professional investor whose wife is Mr. Kearns's cousin, said the young
man appeared excited and eager to learn about investing.

It was Mr. Brewster who raised Robinhood's failures in the Kearns matter
on Twitter. While he gave the company credit for its response and some
of the moves it says it is going to make, Mr. Brewster remains worried
that Robinhood does not understand or appreciate its responsibilities.

``You are playing with real fire to allow inexperienced people to play
with the riskiest of financial instruments,'' he said. ``There is always
personal responsibility, of course. But as a society, I think we owe
youth some sort of oversight, and it feels like someone was asleep at
the wheel there.''

Mr. Brewster sees in Robinhood elements of what he called ``Pavlovian
gambling associations,'' even as he said he also believes that it is
important to break down some of the barriers to investing. ``You need to
build a platform that you would be comfortable putting in your own kid's
hands,'' he said.

Is it Robinhood's fault that Alex Kearns is dead? No.

Was it the startup's responsibility to do a better job with design,
ethics and friction to better guard against kids like him from going
into a fatal tailspin? Yes. Yes. And yes again.

You could even say a banquet of yeses.

\emph{The Times is committed to publishing}
\href{https://www.nytimes3xbfgragh.onion/2019/01/31/opinion/letters/letters-to-editor-new-york-times-women.html}{\emph{a
diversity of letters}} \emph{to the editor. We'd like to hear what you
think about this or any of our articles. Here are some}
\href{https://help.nytimes3xbfgragh.onion/hc/en-us/articles/115014925288-How-to-submit-a-letter-to-the-editor}{\emph{tips}}\emph{.
And here's our email:}
\href{mailto:letters@NYTimes.com}{\emph{letters@NYTimes.com}}\emph{.}

\emph{Follow The New York Times Opinion section on}
\href{https://www.facebookcorewwwi.onion/nytopinion}{\emph{Facebook}}\emph{,}
\href{http://twitter.com/NYTOpinion}{\emph{Twitter (@NYTopinion)}}
\emph{and}
\href{https://www.instagram.com/nytopinion/}{\emph{Instagram}}\emph{,
and sign up for the}
\href{http://www.nytimes3xbfgragh.onion/newsletters/opiniontoday/}{\emph{Opinion
Today newsletter}}\emph{.}

Advertisement

\protect\hyperlink{after-bottom}{Continue reading the main story}

\hypertarget{site-index}{%
\subsection{Site Index}\label{site-index}}

\hypertarget{site-information-navigation}{%
\subsection{Site Information
Navigation}\label{site-information-navigation}}

\begin{itemize}
\tightlist
\item
  \href{https://help.nytimes3xbfgragh.onion/hc/en-us/articles/115014792127-Copyright-notice}{©~2020~The
  New York Times Company}
\end{itemize}

\begin{itemize}
\tightlist
\item
  \href{https://www.nytco.com/}{NYTCo}
\item
  \href{https://help.nytimes3xbfgragh.onion/hc/en-us/articles/115015385887-Contact-Us}{Contact
  Us}
\item
  \href{https://www.nytco.com/careers/}{Work with us}
\item
  \href{https://nytmediakit.com/}{Advertise}
\item
  \href{http://www.tbrandstudio.com/}{T Brand Studio}
\item
  \href{https://www.nytimes3xbfgragh.onion/privacy/cookie-policy\#how-do-i-manage-trackers}{Your
  Ad Choices}
\item
  \href{https://www.nytimes3xbfgragh.onion/privacy}{Privacy}
\item
  \href{https://help.nytimes3xbfgragh.onion/hc/en-us/articles/115014893428-Terms-of-service}{Terms
  of Service}
\item
  \href{https://help.nytimes3xbfgragh.onion/hc/en-us/articles/115014893968-Terms-of-sale}{Terms
  of Sale}
\item
  \href{https://spiderbites.nytimes3xbfgragh.onion}{Site Map}
\item
  \href{https://help.nytimes3xbfgragh.onion/hc/en-us}{Help}
\item
  \href{https://www.nytimes3xbfgragh.onion/subscription?campaignId=37WXW}{Subscriptions}
\end{itemize}
