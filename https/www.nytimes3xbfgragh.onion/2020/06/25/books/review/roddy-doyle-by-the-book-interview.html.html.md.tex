Sections

SEARCH

\protect\hyperlink{site-content}{Skip to
content}\protect\hyperlink{site-index}{Skip to site index}

\href{https://www.nytimes3xbfgragh.onion/section/books/review}{Book
Review}

\href{https://myaccount.nytimes3xbfgragh.onion/auth/login?response_type=cookie\&client_id=vi}{}

\href{https://www.nytimes3xbfgragh.onion/section/todayspaper}{Today's
Paper}

\href{/section/books/review}{Book Review}\textbar{}Roddy Doyle Scored 8
Out of 10 on a Quiz About Roddy Doyle

\url{https://nyti.ms/2Nqifhn}

\begin{itemize}
\item
\item
\item
\item
\item
\end{itemize}

Advertisement

\protect\hyperlink{after-top}{Continue reading the main story}

Supported by

\protect\hyperlink{after-sponsor}{Continue reading the main story}

\href{/column/by-the-book}{By the Book}

\hypertarget{roddy-doyle-scored-8-out-of-10-on-a-quiz-about-roddy-doyle}{%
\section{Roddy Doyle Scored 8 Out of 10 on a Quiz About Roddy
Doyle}\label{roddy-doyle-scored-8-out-of-10-on-a-quiz-about-roddy-doyle}}

\includegraphics{https://static01.graylady3jvrrxbe.onion/images/2020/06/28/books/review/28ByTheBook/28ByTheBook-articleLarge.jpg?quality=75\&auto=webp\&disable=upscale}

Published June 25, 2020Updated June 26, 2020

\begin{itemize}
\item
\item
\item
\item
\item
\end{itemize}

\textbf{What books are on your nightstand?}

``Notes From an Apocalypse,'' by Mark O'Connell; with that title, I
half-dreaded what I'd find but --- a hundred pages in --- I'm rightly
anxious but laughing. ``Hope in the Dark,'' by Rebecca Solnit, for a bit
of well thought-out optimism last thing at night. I found Nicole
Flattery's story collection, ``Show Them a Good Time,'' under the bed
this morning but it's back on the nightstand where it belongs. The
stories are brilliant, full of surprises.

\textbf{What's the last great book you read?}

``Hamnet,'' by Maggie O'Farrell. There was a spell in mid-March when
every book I picked up seemed to be, by choice or accident, about the
plague. This one was both a choice and an accident; I was keen to read
it but didn't realize the plague would feature so early. We know little
about Shakespeare's son other than the fact that he was called Hamnet,
but Maggie O'Farrell has constructed a glorious novel out of the name,
and the extraordinary final pages remind us of a huge thing that we're
missing right now: live performance.

\textbf{Are there any classic novels that you only recently read for the
first time?}

I reread ``Dr. Jekyll and Mr. Hyde'' recently and wondered, as I had a
quick look at Robert Louis Stevenson's biography, if I'd read
``Kidnapped'' when I was a kid. I started to read it, and I hadn't; it
was brand-new and immediately great. Is there a better storyteller? The
physical detail --- life onboard ship in a storm, the race across the
Scottish Highlands --- is precise and alive. As the narrator, David
Balfour, recounts his adventures he also describes a world that had
largely disappeared by the time Stevenson wrote the book, a place where
having the right surname, speaking the right language, whistling the
right tune were matters of life and death. Irish readers will find it a
bit familiar. A historical adventure story is, somehow, current affairs.

\textbf{What's your favorite book no one else has heard of?}

``The Life and Adventures of Paddy O'Driscoll,'' by Charles Dickens. It
has never been published and exists only in my head, and only as a
title. But it's very good.

\textbf{Which writers --- novelists, playwrights, critics, journalists,
poets --- working today do you admire most?}

Marina Hyde writes brilliantly about football --- real football, soccer.
But her political writing, in The Guardian, seems even better these
days, and more vital. I love her anger and what she does with it. She's
very funny but she's also furious --- the perfect combination. I love
Kate Tempest's poetry and I love the fact that she makes records; she
works with musicians. If I was offered a trip to the theater ---
something I really miss --- and if I could choose the play, I'd gulp and
say, ``Thanks very much --- something by Conor McPherson, please.''

\textbf{What's the most interesting thing you learned from a book
recently?}

Reading Eula Biss's ``On Immunity,'' I learned that milkmaids in
18th-century England bore none of the scars of smallpox on their faces,
unlike so many other people at the time --- because they worked with
cows. The folk belief was that if a milkmaid milked a cow blistered with
cowpox, she would not contract smallpox. It was nice to see the working
classes getting ahead of the toffs for a change --- and women too.

\textbf{What books would you recommend to somebody who wants to learn
more about Irish literature?}

Read Bram Stoker's ``Dracula,'' then read ``This Hostel Life,'' by
Melatu Uche Okorie. You've just read two fine examples of Irish
literature. Stoker was a Dubliner; he grew up a 10-minute walk from
where I live. Okorie's stories capture the language and lives of asylum
seekers who live a half-hour drive from Stoker's house. Ireland is a
small island but there's more than one way to tell an Irish story.

\textbf{What moves you most in a work of literature?}

I suspect that writers reading are like mechanics looking over the
shoulders of other mechanics. What moves me most is what another writer
does with words --- how they create an accent, put hair on a head, give
a man a heart attack, give a woman twins, get a man to tell another man
he loves him --- with the right, surprising line of words.

\textbf{How do you organize your books?}

On shelves, on tables, on ledges, on chairs, beside the bed, under the
bed, on top of books on shelves, on the back seat of the car.

\textbf{What book might people be surprised to find on your shelves?}

The Penguin Rhyming Dictionary. I've never written a poem or a song
lyric but I bought the dictionary when I was translating the libretto of
Mozart's ``Don Giovanni,'' five years ago. Now and again, it came in
handy --- Andy, bandy, candy, sandy, brandy, Gandhi, jaborandi (tropical
shrub), onus probandi (burden of proof). This is the first time I've
consulted the dictionary since I stopped working with Mozart. He was a
nice lad but not great with the deadlines.

\textbf{What kind of reader were you as a child? Which childhood books
and authors stick with you most?}

My mother taught me how to read because she noticed, two years into my
education, that I couldn't. She used comic books and I remember the
first time I recognized a word in a speech bubble over a character's
cartoon head and saw the same word in a different bubble further down
the page. I immediately became a voracious reader, a bit of a cannibal.
``Just William,'' by Richmal Crompton, was my favorite book; the world
of boys was so brilliantly captured. I read it again recently; it's
still great, and very funny. My Uncle Joe, my mother's brother, lived in
Washington, D.C., and he sent boxes of books to us. I remember
illustrated histories of the United States, and a biography of George
Washington for children --- and a great book called ``Benjy,'' by Edwin
O'Connor. The illustrations were wild, and so was the story. Benjy's
father was a TV repairman and whenever he had a row with Benjy's mother
--- a woman with notions and, possibly, a lover --- he'd climb into the
back of the TV with a plate of sandwiches and a deck of cards.

\textbf{You're organizing a literary dinner party. Which three writers,
dead or alive, do you invite?}

Dickens, Emily Dickinson, Dr. Seuss. The words will be visible in the
air above the table. Also, they're all dead, so we won't have to talk
about Covid-19, Facebook or Zoom.

\textbf{Whom would you want to write your life story?}

I haven't a clue. But if there's a movie based on the book, I'd like
Bill Murray's facial expression in ``Lost in Translation'' to play me.
Not Bill --- just his facial expression.

\textbf{What books do you find yourself returning to again and again?}

I reread Raymond Carver's story ``Cathedral'' in March. I actually read
it twice because I'd forgotten how great it was --- and how great he
was. Since then, I've been reading a Carver story every day; they are so
brilliant and, surprisingly perhaps, varied. I read Dickens again and
again. And I read Flann O'Brien's ``At Swim-Two-Birds'' every couple of
years. The really great books change as we get older.

\textbf{What books are you embarrassed not to have read yet?}

I stare at the question but I don't really get it. Why would I be
embarrassed because I haven't read a book? I've read all of my own, so
I'll never get caught out there. On my 60th birthday, an Irish newspaper
\href{https://www.thejournal.ie/roddy-doyle-quiz-3999776-May2018/}{ran a
quiz about me}. I got eight out of 10. But I wasn't embarrassed. I was
actually quite pleased --- and a tiny bit worried.

\textbf{What do you plan to read next?}

Sebastian Barry's new novel, ``A Thousand Moons,'' Richard Ford's new
collection, ``Sorry for Your Trouble,'' and Sarah Bakewell's ``At the
Existentialist Café.''

Advertisement

\protect\hyperlink{after-bottom}{Continue reading the main story}

\hypertarget{site-index}{%
\subsection{Site Index}\label{site-index}}

\hypertarget{site-information-navigation}{%
\subsection{Site Information
Navigation}\label{site-information-navigation}}

\begin{itemize}
\tightlist
\item
  \href{https://help.nytimes3xbfgragh.onion/hc/en-us/articles/115014792127-Copyright-notice}{©~2020~The
  New York Times Company}
\end{itemize}

\begin{itemize}
\tightlist
\item
  \href{https://www.nytco.com/}{NYTCo}
\item
  \href{https://help.nytimes3xbfgragh.onion/hc/en-us/articles/115015385887-Contact-Us}{Contact
  Us}
\item
  \href{https://www.nytco.com/careers/}{Work with us}
\item
  \href{https://nytmediakit.com/}{Advertise}
\item
  \href{http://www.tbrandstudio.com/}{T Brand Studio}
\item
  \href{https://www.nytimes3xbfgragh.onion/privacy/cookie-policy\#how-do-i-manage-trackers}{Your
  Ad Choices}
\item
  \href{https://www.nytimes3xbfgragh.onion/privacy}{Privacy}
\item
  \href{https://help.nytimes3xbfgragh.onion/hc/en-us/articles/115014893428-Terms-of-service}{Terms
  of Service}
\item
  \href{https://help.nytimes3xbfgragh.onion/hc/en-us/articles/115014893968-Terms-of-sale}{Terms
  of Sale}
\item
  \href{https://spiderbites.nytimes3xbfgragh.onion}{Site Map}
\item
  \href{https://help.nytimes3xbfgragh.onion/hc/en-us}{Help}
\item
  \href{https://www.nytimes3xbfgragh.onion/subscription?campaignId=37WXW}{Subscriptions}
\end{itemize}
