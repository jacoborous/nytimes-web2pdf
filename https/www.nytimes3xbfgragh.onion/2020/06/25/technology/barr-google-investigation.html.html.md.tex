Sections

SEARCH

\protect\hyperlink{site-content}{Skip to
content}\protect\hyperlink{site-index}{Skip to site index}

\href{https://www.nytimes3xbfgragh.onion/section/technology}{Technology}

\href{https://myaccount.nytimes3xbfgragh.onion/auth/login?response_type=cookie\&client_id=vi}{}

\href{https://www.nytimes3xbfgragh.onion/section/todayspaper}{Today's
Paper}

\href{/section/technology}{Technology}\textbar{}Barr's Interest in
Google Antitrust Case Keeps It Moving Swiftly

\url{https://nyti.ms/3fXkDbE}

\begin{itemize}
\item
\item
\item
\item
\item
\item
\end{itemize}

Advertisement

\protect\hyperlink{after-top}{Continue reading the main story}

Supported by

\protect\hyperlink{after-sponsor}{Continue reading the main story}

\hypertarget{barrs-interest-in-google-antitrust-case-keeps-it-moving-swiftly}{%
\section{Barr's Interest in Google Antitrust Case Keeps It Moving
Swiftly}\label{barrs-interest-in-google-antitrust-case-keeps-it-moving-swiftly}}

Attorney General William Barr's attention to the Justice Department
investigation shows the high stakes for the agency and for him.

\includegraphics{https://static01.graylady3jvrrxbe.onion/images/2020/06/26/business/25JPantitrustdoj1-print/merlin_165432687_85138655-3d4a-4382-bb2b-db469cfba3df-articleLarge.jpg?quality=75\&auto=webp\&disable=upscale}

By \href{https://www.nytimes3xbfgragh.onion/by/david-mccabe}{David
McCabe} and
\href{https://www.nytimes3xbfgragh.onion/by/cecilia-kang}{Cecilia Kang}

\begin{itemize}
\item
  June 25, 2020
\item
  \begin{itemize}
  \item
  \item
  \item
  \item
  \item
  \item
  \end{itemize}
\end{itemize}

WASHINGTON --- For months, lawyers at the Justice Department have been
marshaling their forces for a possible antitrust lawsuit against Google,
spurred on by the personal interest of Attorney General William P. Barr.

The day-to-day digging of a federal antitrust investigation rarely rises
to the level of the attorney general or the deputy attorney general.

But under Mr. Barr, the agency has made top priority of looking into the
country's biggest tech companies. He receives regular updates on the
Google case from an aide, according to several people close to the
investigations, while an official in the office of his deputy, Jeffrey
Rosen, oversees the investigations into tech companies.

In the latest sign that the Justice Department is moving swiftly, staff
members appear to have begun drafting a case memo to test its legal
argument, three other people connected to the case said. The agency has
assigned a growing number of employees to the inquiry, and it has
brought in an economic expert who could testify at a trial. The details
of the internal maneuvers were gathered from interviews with more than
20 people, most of whom would speak only anonymously because the
deliberations were private.

The attention from top officials shows the high stakes for the Justice
Department and Mr. Barr, and may draw fire from critics who say it shows
how the agency has become politicized. President Trump has repeatedly
chastised the big tech companies, arguing in part that they silence
conservative views. On Wednesday, a lawyer for the department,
testifying as a whistle-blower, told the House Judiciary Committee that
the agency had pursued antitrust investigations either because of Mr.
Barr's personal animus against an industry or the president's political
whims.

Mr. Barr, who has repeatedly said publicly that the tech industry's
power required examination, is expected to decide in the coming months
whether to file a lawsuit accusing Google of abusing its power in the
market for advertising technology and search products. A successful suit
against the company could win plaudits from Mr. Trump. It could also
reshape Google's business, transform a large chunk of the economy and
perhaps even end the era of unfettered growth in Silicon Valley.

But a loss in court could embarrass the Justice Department --- which
\href{https://www.nytimes3xbfgragh.onion/2018/06/12/business/dealbook/att-time-warner-ruling-antitrust-case.html}{suffered
an antitrust defeat in 2018} when it challenged AT\&T's purchase of Time
Warner --- and lead to accusations that the case was based on politics,
not the law. It could also reinforce the tech industry's power.

Deciding not to pursue the case may prove problematic, too: The Federal
Trade Commission faced a flood of criticism in 2012 when it
\href{https://www.nytimes3xbfgragh.onion/2013/01/04/technology/google-agrees-to-changes-in-search-ending-us-antitrust-inquiry.html}{decided
not to sue Google}.

``I think the prevailing winds right now are winds that would result in
more criticism if they decided not to bring a case than if they brought
a weak case and lost,'' said Charles James, who led the Justice
Department's antitrust division in the early 2000s.

Brianna Herlihy, a department spokeswoman, declined to comment on the
tech investigations. In a separate statement, she said the agency
``strongly disagrees'' with the claims of politicization made at
Wednesday's hearing.

Julie Tarallo McAlister, a Google spokeswoman, said the company
continued to cooperate with the Justice Department, ``and we don't have
any updates or comments on speculation.''

The Google investigation began last year, shortly after the Justice
Department and the Federal Trade Commission divided up responsibility
for investigating antitrust complaints about the major tech firms. In
addition to concerns about Google's control over the software that
delivers online ads to consumers, the agency has been examining
allegations that the company
\href{https://www.nytimes3xbfgragh.onion/2020/06/04/technology/google-european-search-menu-antitrust.html}{abused
its dominance} over search, several of the people close to the
investigation said.

The department's investigators have fanned out over the media, tech and
advertising industries, gathering evidence from companies that compete
with Google. Antitrust inquiries often take years, but this one has
moved unusually fast under Mr. Barr.

The agency recently hired 10 to 15 tech fellows to work on the
investigation, one of the people close to the case said. The part of the
antitrust office that is overseeing the inquiry, Technology and
Financial Services, has been told that it will not be taking on any new
matters, a sign that it has narrowed its focus to Google, one person
said.

\includegraphics{https://static01.graylady3jvrrxbe.onion/images/2020/06/12/business/00google-02/merlin_170110569_5bcfeb89-9b1c-4a02-a5bd-0ef923fce62d-articleLarge.jpg?quality=75\&auto=webp\&disable=upscale}

Officials have spent recent months trying to recruit a litigator from a
law firm to join the team for the case, a practice that is not unusual
for major antitrust cases. David Boies, for example, was the star of the
government's 1990s lawsuit against Microsoft. The current search was
\href{https://www.bloomberg.com/news/articles/2020-02-21/doj-solicited-outside-law-firm-for-help-with-tech-antitrust-case}{reported
earlier} by Bloomberg News.

The Justice Department has also hired an economic expert to work on the
investigation, a standard but critical step, two people familiar with
the matter said.

Hired economists play a central role in the courtroom during an
antitrust case, making sense of complicated data and economic principles
for a judge or jury. It was not immediately clear which expert or
experts had been hired.

A case against Google would almost certainly stretch on for years, and a
trial is far from guaranteed: Companies frequently settle with federal
prosecutors, and Justice Department staff could recommend against going
to court. After the Justice Department
\href{https://www.wired.com/2002/11/u-s-v-microsoft-timeline}{began
pursuing} Microsoft in the early 1990s, for example, a final settlement
was reached years later, in 2002.

The Federal Trade Commission closed its yearslong investigation into
Google in 2012 without filing any charges despite hiring a prominent
litigator to work on the case. But since then, European officials have
brought several antitrust cases against the company, and aspects of its
business have attracted the attention of other regulators around the
world.

Mr. Barr signaled early on that he would be one of them. At his
confirmation hearing last year,
\href{https://www.axios.com/bill-barr-confirmation-hearing-big-tech-antitrust-fe535feb-8980-4d47-94fa-15bb564e84de.html}{he
said} that ``a lot of people wonder how such huge behemoths that now
exist in Silicon Valley have taken shape under the nose of the antitrust
enforcers.''

Google is also under investigation by a bipartisan group of state
attorneys general, who could join a Justice Department lawsuit or file
their own.

Few U.S. attorneys general have paid much attention to antitrust law, a
division of the Justice Department that is rarely in the spotlight. But
Mr. Barr, a former lawyer for Verizon and a former board member of Time
Warner, two companies that regularly navigate antitrust law, is steeped
in the subject.

In August, Mr. Barr pulled Lauren Willard, a lawyer from the antitrust
division, to sit within his office to act as his liaison to the cases.
In October, the department hired a veteran antitrust lawyer, Ryan
Shores, to head technology antitrust cases including Google and to
report to Mr. Rosen, Mr. Barr's deputy.

Mr. Shores is working closely with Ms. Willard, who is giving regular
updates to Mr. Barr.

In a sign of how widely he interprets the agency's reach, Mr. Barr
\href{https://www.politico.com/morningtech/}{said} last weekend that
antitrust laws could be used against companies that allegedly restricted
the spread of conservative views. Mr. Trump and conservatives have
become increasingly critical of tech companies, including YouTube, which
is owned by Google, arguing that the companies silence conservative
voices.

``One way this can be addressed,'' Mr. Barr said in an interview with
Fox News, ``is through the antitrust laws and challenging companies that
engage in monopolistic practices.''

Katie Benner contributed reporting.

Advertisement

\protect\hyperlink{after-bottom}{Continue reading the main story}

\hypertarget{site-index}{%
\subsection{Site Index}\label{site-index}}

\hypertarget{site-information-navigation}{%
\subsection{Site Information
Navigation}\label{site-information-navigation}}

\begin{itemize}
\tightlist
\item
  \href{https://help.nytimes3xbfgragh.onion/hc/en-us/articles/115014792127-Copyright-notice}{©~2020~The
  New York Times Company}
\end{itemize}

\begin{itemize}
\tightlist
\item
  \href{https://www.nytco.com/}{NYTCo}
\item
  \href{https://help.nytimes3xbfgragh.onion/hc/en-us/articles/115015385887-Contact-Us}{Contact
  Us}
\item
  \href{https://www.nytco.com/careers/}{Work with us}
\item
  \href{https://nytmediakit.com/}{Advertise}
\item
  \href{http://www.tbrandstudio.com/}{T Brand Studio}
\item
  \href{https://www.nytimes3xbfgragh.onion/privacy/cookie-policy\#how-do-i-manage-trackers}{Your
  Ad Choices}
\item
  \href{https://www.nytimes3xbfgragh.onion/privacy}{Privacy}
\item
  \href{https://help.nytimes3xbfgragh.onion/hc/en-us/articles/115014893428-Terms-of-service}{Terms
  of Service}
\item
  \href{https://help.nytimes3xbfgragh.onion/hc/en-us/articles/115014893968-Terms-of-sale}{Terms
  of Sale}
\item
  \href{https://spiderbites.nytimes3xbfgragh.onion}{Site Map}
\item
  \href{https://help.nytimes3xbfgragh.onion/hc/en-us}{Help}
\item
  \href{https://www.nytimes3xbfgragh.onion/subscription?campaignId=37WXW}{Subscriptions}
\end{itemize}
