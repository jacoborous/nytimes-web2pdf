Sections

SEARCH

\protect\hyperlink{site-content}{Skip to
content}\protect\hyperlink{site-index}{Skip to site index}

\href{https://myaccount.nytimes3xbfgragh.onion/auth/login?response_type=cookie\&client_id=vi}{}

\href{https://www.nytimes3xbfgragh.onion/section/todayspaper}{Today's
Paper}

\href{/section/opinion}{Opinion}\textbar{}Take Back the Streets From the
Automobile

\url{https://nyti.ms/2NfZPj7}

\begin{itemize}
\item
\item
\item
\item
\item
\end{itemize}

Advertisement

\protect\hyperlink{after-top}{Continue reading the main story}

\href{/section/opinion}{Opinion}

Supported by

\protect\hyperlink{after-sponsor}{Continue reading the main story}

\hypertarget{take-back-the-streets-from-the-automobile}{%
\section{Take Back the Streets From the
Automobile}\label{take-back-the-streets-from-the-automobile}}

With people hunkered down at home, cities should act quickly to find a
better balance between cars and pedestrians and cyclists.

By \href{https://www.nytimes3xbfgragh.onion/by/justin-gillis}{Justin
Gillis} and Heather Thompson

Mr. Gillis, a former Times environmental reporter, is a contributing
opinion writer. Ms. Thompson is a transportation planner.

\begin{itemize}
\item
  June 20, 2020
\item
  \begin{itemize}
  \item
  \item
  \item
  \item
  \item
  \end{itemize}
\end{itemize}

\includegraphics{https://static01.graylady3jvrrxbe.onion/images/2020/06/22/opinion/20gillisWeb/merlin_173526783_192e498a-5699-411c-b1e5-64fa738eec17-articleLarge.jpg?quality=75\&auto=webp\&disable=upscale}

Since cities came to exist 5,000 years ago, epidemics have shaped their
fate.

Plagues
\href{https://www.theatlantic.com/science/archive/2016/03/plagues-roman-empire/473862/}{weakened}
the Roman Empire and may have helped bring it down. The sewers that
cleaned up a filthy London in the 19th century were
\href{http://www.bbc.co.uk/history/historic_figures/bazalgette_joseph.shtml}{built}
in direct response to a cholera outbreak. Many of the great urban parks,
including Central Park in New York City, were similarly
\href{https://www.history.com/news/cholera-pandemic-new-york-city-london-paris-green-space}{planned}
after epidemics, to provide more open space.

Today, the coronavirus pandemic, in all its horror, opens the prospect
of sweeping urban change. Cities suddenly see the possibility of
correcting their greatest mistake of the 20th century, the surrender of
too much public space to the automobile.

Cities need to seize this moment and move at lightning speed. We need to
find a better balance between the cars on our streets and the bicyclists
and pedestrians who have, for decades, been neglected and pushed to the
margins.

All over the world, forward-looking cities large and small have already
jumped into action. In Medellin, the innovative Colombian city nestled
in the Andes, workers are seizing traffic lanes and slapping down yellow
paint to signify a change: Cars have been evicted and the lanes are now
reserved for bicyclists. In Kampala, the capital of Uganda, the
authorities have closed streets, encouraged cycling, and sped the
construction of new bike lanes and walkways. In European cities,
``\href{https://www.nytimes3xbfgragh.onion/2020/06/12/business/paris-bicycles-commute-coronavirus.htmlhttps:/www.nytimes3xbfgragh.onion/2020/06/12/business/paris-bicycles-commute-coronavirus.html}{corona
cycleways}'' have become the new norm.

In New York, the city has responded to community demands by pledging to
set aside
\href{https://nyc.streetsblog.org/2020/04/27/breaking-de-blasio-commits-to-100-miles-of-open-streets/}{100
miles of roads} in the next few weeks for people on foot or bike,
largely closing the streets to traffic during daylight hours. Letting
people dine at tables in the middle of the road may help in the
\href{https://www.timeout.com/newyork/news/this-is-what-outdoor-dining-may-soon-look-like-in-nyc-052920}{salvation}
of New York restaurants. Across the country in Oakland, Calif., the city
has
\href{https://www.curbed.com/2020/5/19/21258662/oakland-slow-streets-closures-social-distancing}{decided}
to close nearly 10 percent of its streets. And in the middle of the
country, Kansas City, Mo., was one of the first to limit traffic and
turn parking spots into mini-parks to extend restaurant service.

This is a golden moment for the movement known as
\href{http://tacticalurbanismguide.com/about/}{tactical urbanism}. More
than 200 cities have already announced road closings in response to the
coronavirus pandemic. Thousands of cities have yet to act in any bold
way, however. If they do not, they may miss what could be a
once-in-a-lifetime opportunity.

The circumstances that give rise to this situation are lamentable, of
course, just as were the cholera epidemics that altered cities in the
19th century. Bicycling is
\href{https://www.nytimes3xbfgragh.onion/2020/05/18/nyregion/bike-shortage-coronavirus.html}{booming}
--- bike stores are reporting record sales and order backlogs --- as
people look for easier means to get around and find streets with reduced
traffic to be safer and more congenial. Cities are finding they can make
bold moves to accommodate all the new bikers and walkers because the
drivers who would normally object to street closings are hunkered down
in their homes.

The suppression of automotive traffic is giving us a vivid illustration
of the potential future benefits of cleaning up our cities. Air
pollution, which kills millions of people every year, is down nearly
everywhere. In Mexico City, measurements of the smallest, deadliest
particles have fallen by about half. The Indian government has
\href{https://www.news18.com/news/india/locked-under-blue-skies-air-quality-and-the-pandemic-2612719.html}{publicly
reported} that several pollution measures are down as much as 70 percent
in New Delhi; in some cities, Indian children are able to see distant
mountains for the first time in their lives.

Most of the road closures announced so far have been billed as
temporary, meant to last until the pandemic loosens its grip. The
willingness of drivers to leave their cars parked is certainly not going
to last. What can cities do to make sure they hold on to the recent
gains as the economy reopens?

To answer that, we return to a phrase we used earlier: tactical
urbanism. For the last couple of decades, this movement has been seizing
moments of opportunity to improve urban life.

Sometimes a city government is the instigator, as in 2009, when New York
\href{https://www.nytimes3xbfgragh.onion/2009/05/25/nyregion/25bway.html}{closed
several blocks} of Broadway, one of the busiest streets in the city, to
traffic. Sometimes citizens employ guerrilla tactics --- converting a
vacant lot into a miniature park or garden, for instance, or throwing up
orange traffic cones in the middle of the night to create a bike lane.

The basic idea is to show people the benefits of a change, however
temporary, in order to shift the political dynamic in favor of a more
permanent alteration. You can bet that parents whose bored children are
suddenly able to ride their bikes in the Oakland streets are seeing this
whole set of issues with new eyes.

When Broadway was closed, thousands of New Yorkers flooded the street,
delightedly plopping down in cheap lawn chairs the city had set out on
the pavement. From that moment, the vision of a Broadway for people took
hold, and the blocks of Broadway through Times Square have been closed
to traffic for a decade.

Similarly, tactical urbanist projects
\href{https://www.itdp.org/event/power-of-tactical-urbanism/https:/www.itdp.org/event/power-of-tactical-urbanism/}{all
over the world} have led to closed streets, new parks and many other
amenities. A large majority of these projects entail reclaiming public
space from the automobile. A third or more of the space in any city is
devoted to streets, and in the middle of the last century, much of that
was converted to traffic lanes and free parking spaces.

Today, we have been thrust into perhaps the greatest opportunity ever
for tactical urbanism. With traffic missing from the streets, people are
sensing how completely cars dominate them in normal times, endangering
the lives of the pedestrians and cyclists squeezed into tiny strips
along the margins. This situation was never sensible or moral, but until
now, fixing it was politically impossible in many cities.

A viral twist of fate has given us a chance to alter the balance,
creating streets that work for everyone. Cities that were thinking about
lane changes or street closures before the pandemic should move quickly
to try them out, and the most popular should be made permanent.
Government leaders must pay particular attention to poor neighborhoods,
which tend to be forgotten but whose people have just as much right to
bike and walk as anyone else. Those neighborhoods are often deprived of
parks or sports fields, so a street with few or no cars can be a godsend
for children.

In the end, reclaiming streets will not be enough to lock in improved
air quality and other benefits. Every city needs a comprehensive program
of car control. Some, like London, are already banning the most
polluting vehicles, and a few have gone so far as to declare they will
no longer allow fuel-burning engines after 2030 or 2035. In those towns,
you will drive an electric car if you drive at all.

Cities need to follow the lead of London, Singapore and more recently
\href{https://www.nytimes3xbfgragh.onion/2019/04/24/nyregion/what-is-congestion-pricing.html}{New
York} in enacting stiff congestion charges that discourage unnecessary
driving, with the money plowed into mass transit, as well as more
protected lanes for walking and cycling.

Cities need to be designed for the well-being and health of people, not
for cars. We don't have time to wait. Now is the moment for cities to
imagine that future and start willing it into being.

Heather Thompson (\href{https://twitter.com/hfthompson_}{@hfthompson\_})
is the chief executive officer of the Institute for Transportation \&
Development Policy, which promotes sustainable and equitable transport
worldwide.

Advertisement

\protect\hyperlink{after-bottom}{Continue reading the main story}

\hypertarget{site-index}{%
\subsection{Site Index}\label{site-index}}

\hypertarget{site-information-navigation}{%
\subsection{Site Information
Navigation}\label{site-information-navigation}}

\begin{itemize}
\tightlist
\item
  \href{https://help.nytimes3xbfgragh.onion/hc/en-us/articles/115014792127-Copyright-notice}{©~2020~The
  New York Times Company}
\end{itemize}

\begin{itemize}
\tightlist
\item
  \href{https://www.nytco.com/}{NYTCo}
\item
  \href{https://help.nytimes3xbfgragh.onion/hc/en-us/articles/115015385887-Contact-Us}{Contact
  Us}
\item
  \href{https://www.nytco.com/careers/}{Work with us}
\item
  \href{https://nytmediakit.com/}{Advertise}
\item
  \href{http://www.tbrandstudio.com/}{T Brand Studio}
\item
  \href{https://www.nytimes3xbfgragh.onion/privacy/cookie-policy\#how-do-i-manage-trackers}{Your
  Ad Choices}
\item
  \href{https://www.nytimes3xbfgragh.onion/privacy}{Privacy}
\item
  \href{https://help.nytimes3xbfgragh.onion/hc/en-us/articles/115014893428-Terms-of-service}{Terms
  of Service}
\item
  \href{https://help.nytimes3xbfgragh.onion/hc/en-us/articles/115014893968-Terms-of-sale}{Terms
  of Sale}
\item
  \href{https://spiderbites.nytimes3xbfgragh.onion}{Site Map}
\item
  \href{https://help.nytimes3xbfgragh.onion/hc/en-us}{Help}
\item
  \href{https://www.nytimes3xbfgragh.onion/subscription?campaignId=37WXW}{Subscriptions}
\end{itemize}
