Sections

SEARCH

\protect\hyperlink{site-content}{Skip to
content}\protect\hyperlink{site-index}{Skip to site index}

\href{https://www.nytimes3xbfgragh.onion/section/opinion/sunday}{Sunday
Review}

\href{https://myaccount.nytimes3xbfgragh.onion/auth/login?response_type=cookie\&client_id=vi}{}

\href{https://www.nytimes3xbfgragh.onion/section/todayspaper}{Today's
Paper}

\href{/section/opinion/sunday}{Sunday Review}\textbar{}To Overturn
Trump, We Need to Overturn White Supremacy

\href{https://nyti.ms/2UGsRfM}{https://nyti.ms/2UGsRfM}

\begin{itemize}
\item
\item
\item
\item
\item
\item
\end{itemize}

Advertisement

\protect\hyperlink{after-top}{Continue reading the main story}

\href{/section/opinion}{Opinion}

Supported by

\protect\hyperlink{after-sponsor}{Continue reading the main story}

\hypertarget{to-overturn-trump-we-need-to-overturn-white-supremacy}{%
\section{To Overturn Trump, We Need to Overturn White
Supremacy}\label{to-overturn-trump-we-need-to-overturn-white-supremacy}}

For that to happen, some monuments --- and the historical myths they
supported --- are going to have to come down.

\href{https://www.nytimes3xbfgragh.onion/column/jamelle-bouie}{\includegraphics{https://static01.graylady3jvrrxbe.onion/images/2019/01/24/opinion/jamelle-bouie/jamelle-bouie-thumbLarge-v3.png}}

By
\href{https://www.nytimes3xbfgragh.onion/column/jamelle-bouie}{Jamelle
Bouie}

Opinion Columnist

\begin{itemize}
\item
  June 12, 2020
\item
  \begin{itemize}
  \item
  \item
  \item
  \item
  \item
  \item
  \end{itemize}
\end{itemize}

\includegraphics{https://static01.graylady3jvrrxbe.onion/images/2020/06/14/opinion/12bouieNew/merlin_173424381_14e2672b-bced-4147-9f3c-c1568d894b9e-articleLarge.jpg?quality=75\&auto=webp\&disable=upscale}

It doesn't necessarily follow that a nationwide protest over police
brutality would, for some, become a reason to take action against
Confederate statues and other controversial monuments. But it has. In
just the last week, protesters have knocked down Confederate statues in
Richmond, Va., Nashville and Montgomery, Ala., as well as monuments to
Christopher Columbus in Boston and St. Paul, Minn.

This is because the George Floyd protests are not just about police
violence. They're about structural racism and the persistence of white
supremacy; about the unresolved and unaddressed disadvantages of the
past, as well as the bigotry that has come to dominate far too much of
American politics in the age of Trump. Born of grief and anger, they're
an attempt to turn the country off the path to ruin. And part of this is
necessarily a struggle over our symbols and our public space.

Another way to put this observation is that police brutality, the
proximate cause of these protests, is simply an acute instance of the
many ways in which the lives of black Americans (and other groups) are
degraded and devalued. And while the most consequential form this
degradation takes are material --- the Covid-19 crisis, for example, has
revealed to many Americans the extent to which black lives are still
shaped by a deep racial inequality that leaves them disproportionately
vulnerable to illness and premature death --- there are also many
symbolic statements of black worth, or the lack thereof, out there for
all to see.

Confederate statues like the ones in Richmond, the former capital of the
Confederacy, or the smaller monuments that mark courthouses and town
squares across the South, are visible reminders of a time when white
society was nearly united in its subjugation of blacks. Erected decades
after the end of the Civil War --- as the white South began to codify
segregation and disenfranchisement into Jim Crow --- these statues set
in stone the triumph over Reconstruction and the effort to make the
South, and the nation, a democracy. And they marked the spaces in which
they stood as essentially white territory.

Confederate monuments were erected to exclude, and they continue to
stand for exclusion. In which case it's no surprise that protesters
would vandalize and tear them down. In this moment, to knock over a
statue of Jefferson Davis is to claim the space for black lives against
those who would try to preserve the values of the Confederacy. And to
the extent that other institutions follow suit --- Congress is debating
an amendment that would rename military bases named after Confederate
officers --- it may reflect a belated recognition that these symbols are
not, and cannot be, neutral.

Something similar is happening with the attempt to remove Christopher
Columbus from the public sphere. The Italian explorer became an American
icon in the late 19th century as Italian immigrants fought to assert
their place in American society. But the real-life Columbus was a
brutal, violent man who
\href{https://www.washingtonpost.com/history/2019/10/14/here-are-indigenous-people-christopher-columbus-his-men-could-not-annihilate/}{inaugurated}
the subjugation of natives in the present-day Caribbean and South
America. His legacy is one of slavery and genocide, and that's why
Indigenous people in the United States have long opposed the
commemoration of his voyage. Knocking down statues of the explorer is
also an attempt to reclaim public space on behalf of the excluded and
ignored. (And it's not irrelevant that the only group
\href{https://www.urban.org/urban-wire/native-americans-deserve-more-attention-police-violence-conversation}{more
exposed} to police violence than black Americans is Native Americans.)

It's unclear how Americans feel about the removal of these statues in
this manner, but we do know there has been a sea change in attitudes
toward Black Lives Matter. Over the last two weeks, my colleagues Nate
Cohn and Kevin Quealy
\href{https://www.nytimes3xbfgragh.onion/interactive/2020/06/10/upshot/black-lives-matter-attitudes.html}{note}
in The Upshot that support has ``increased by nearly as much as it had
over the previous two years.'' The majority of Americans, by a 28-point
margin, now support the movement.

Concurrent with this shift is a sharp drop in support for President
Trump. His
\href{https://projects.fivethirtyeight.com/trump-approval-ratings/}{average
job approval} rating is down to 41 percent, two and a half points lower
than it was on the eve of the protests. His average disapproval rating
is up to 55 percent. And against the Democratic nominee for president,
Joe Biden, he is down
\href{https://www.realclearpolitics.com/epolls/2020/president/us/general_election_trump_vs_biden-6247.html}{an
average of eight points}, a substantial decline from May. The Covid-19
crisis has harmed him, but it is his antagonistic handling of the
protests that has accelerated his downward turn.

The reckoning that is toppling Confederate monuments and fueling the
largest sustained protests in 50 years is also, I think, turning the
voting public decisively against the president. The killing of George
Floyd, the racially disparate impact of the pandemic and the violent
\href{https://www.nytimes3xbfgragh.onion/2020/06/05/opinion/sunday/police-riots.html}{police
rioting} against accountability have shown millions of Americans what
the future may hold if we continue along this path of inequality,
exclusion and authoritarianism. And they're pushing back, taking to the
streets to reject this rather than sitting back and letting it happen.
What's more, as election season begins in earnest, Americans are going
to the ballot box as well. In Atlanta on Tuesday, thousands stood in
line for hours to vote. It was at once an example of the voter
suppression that threatens our democracy and a demonstration of will ---
a determination to use the vote to try to push this country off its
current course.

It was this month, 162 years ago, when Abraham Lincoln accepted the
Republican nomination for the U.S. Senate and gave his famous ``House
Divided'' speech in Illinois. This wasn't, as is popularly believed, a
call for unity in the face of division. Just the opposite. It was an
attempt to make clear the stakes of the conflict with the ``slave
power.''

``I believe this government cannot endure, permanently half slave and
half free,'' he said. ``I do not expect the Union to be dissolved --- I
do not expect the house to fall --- but I do expect it will cease to be
divided. It will become all one thing or all the other.''

We cannot be a free and equal democracy \emph{and} a country of
inequality, unaccountable police violence and Trumpist exclusion. We
will have to be either one or the other. The protests represent millions
of Americans announcing their allegiance to the former. It remains to be
seen whether that brings a reaction of similar scope in defense of the
latter.

\emph{The Times is committed to publishing}
\href{https://www.nytimes3xbfgragh.onion/2019/01/31/opinion/letters/letters-to-editor-new-york-times-women.html}{\emph{a
diversity of letters}} \emph{to the editor. We'd like to hear what you
think about this or any of our articles. Here are some}
\href{https://help.nytimes3xbfgragh.onion/hc/en-us/articles/115014925288-How-to-submit-a-letter-to-the-editor}{\emph{tips}}\emph{.
And here's our email:}
\href{mailto:letters@NYTimes.com}{\emph{letters@NYTimes.com}}\emph{.}

\emph{Follow The New York Times Opinion section on}
\href{https://www.facebookcorewwwi.onion/nytopinion}{\emph{Facebook}}\emph{,}
\href{http://twitter.com/NYTOpinion}{\emph{Twitter (@NYTopinion)}}
\emph{and}
\href{https://www.instagram.com/nytopinion/}{\emph{Instagram}}\emph{.}

Advertisement

\protect\hyperlink{after-bottom}{Continue reading the main story}

\hypertarget{site-index}{%
\subsection{Site Index}\label{site-index}}

\hypertarget{site-information-navigation}{%
\subsection{Site Information
Navigation}\label{site-information-navigation}}

\begin{itemize}
\tightlist
\item
  \href{https://help.nytimes3xbfgragh.onion/hc/en-us/articles/115014792127-Copyright-notice}{©~2020~The
  New York Times Company}
\end{itemize}

\begin{itemize}
\tightlist
\item
  \href{https://www.nytco.com/}{NYTCo}
\item
  \href{https://help.nytimes3xbfgragh.onion/hc/en-us/articles/115015385887-Contact-Us}{Contact
  Us}
\item
  \href{https://www.nytco.com/careers/}{Work with us}
\item
  \href{https://nytmediakit.com/}{Advertise}
\item
  \href{http://www.tbrandstudio.com/}{T Brand Studio}
\item
  \href{https://www.nytimes3xbfgragh.onion/privacy/cookie-policy\#how-do-i-manage-trackers}{Your
  Ad Choices}
\item
  \href{https://www.nytimes3xbfgragh.onion/privacy}{Privacy}
\item
  \href{https://help.nytimes3xbfgragh.onion/hc/en-us/articles/115014893428-Terms-of-service}{Terms
  of Service}
\item
  \href{https://help.nytimes3xbfgragh.onion/hc/en-us/articles/115014893968-Terms-of-sale}{Terms
  of Sale}
\item
  \href{https://spiderbites.nytimes3xbfgragh.onion}{Site Map}
\item
  \href{https://help.nytimes3xbfgragh.onion/hc/en-us}{Help}
\item
  \href{https://www.nytimes3xbfgragh.onion/subscription?campaignId=37WXW}{Subscriptions}
\end{itemize}
