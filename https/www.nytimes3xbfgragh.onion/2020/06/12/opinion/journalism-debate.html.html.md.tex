Sections

SEARCH

\protect\hyperlink{site-content}{Skip to
content}\protect\hyperlink{site-index}{Skip to site index}

\href{https://myaccount.nytimes3xbfgragh.onion/auth/login?response_type=cookie\&client_id=vi}{}

\href{https://www.nytimes3xbfgragh.onion/section/todayspaper}{Today's
Paper}

\href{/section/opinion}{Opinion}\textbar{}Roger Cohen: The Outcry Over
`Both Sides' Journalism

\url{https://nyti.ms/37peyS8}

\begin{itemize}
\item
\item
\item
\item
\item
\item
\end{itemize}

Advertisement

\protect\hyperlink{after-top}{Continue reading the main story}

\href{/section/opinion}{Opinion}

Supported by

\protect\hyperlink{after-sponsor}{Continue reading the main story}

\hypertarget{roger-cohen-the-outcry-over-both-sides-journalism}{%
\section{Roger Cohen: The Outcry Over `Both Sides'
Journalism}\label{roger-cohen-the-outcry-over-both-sides-journalism}}

Moral clarity or only one acceptable truth?

\href{https://www.nytimes3xbfgragh.onion/by/roger-cohen}{\includegraphics{https://static01.graylady3jvrrxbe.onion/images/2014/11/01/opinion/cohen-circular/cohen-circular-thumbLarge-v6.png}}

By \href{https://www.nytimes3xbfgragh.onion/by/roger-cohen}{Roger Cohen}

Opinion Columnist

\begin{itemize}
\item
  June 12, 2020
\item
  \begin{itemize}
  \item
  \item
  \item
  \item
  \item
  \item
  \end{itemize}
\end{itemize}

\includegraphics{https://static01.graylady3jvrrxbe.onion/images/2020/06/12/opinion/12cohen1/merlin_163995072_58ccd301-1f57-42f5-a57e-b806e18b4fba-articleLarge.jpg?quality=75\&auto=webp\&disable=upscale}

\emph{The acting editorial page editor, Kathleen Kingsbury,}
\href{https://www.nytimes3xbfgragh.onion/2020/06/12/opinion/tom-cotton-new-york-times.html}{\emph{wrote
about the decision}} \emph{to publish our writers' responses to the Tom
Cotton Op-Ed in Friday's edition of our Opinion Today newsletter.}

\begin{center}\rule{0.5\linewidth}{\linethickness}\end{center}

I have never believed much in the notion of journalistic
``objectivity.'' We all bring our individual sensibilities to bear on
what we write. Great journalism involves the head and the heart, the
lucidity to think and the passion to feel, the two in balance.

If you have lived a privileged white life, as I have, you can and must
make the effort to understand what it is to have lived an oppressed
black life, to know what it's like to walk into a building and be asked
if you are the help, to see the police not as protector but threat, to
know that some view your life as cheap.

In all the places I have worked, from South Africa, where I had spent my
infancy, to Nigeria, to Brazil, I have tried to do that, writing stories
about injustice and the ravages of misery. But I cannot inhabit the
minds of the subjects of those pieces, however hard I tried.

If I have always been skeptical of objectivity I have always believed in
fairness. That is to say, in the attempt to speak to people on both
sides of a question, to report your way to some approximation of the
truth by filtering diverse views.

That is what distinguishes a journalist. Heading toward the storm in the
opposite direction from the crowd, seeking understanding by being there,
in Jerusalem and Gaza, in Tehran and Washington, in Cape Town and
Khayelitsha. Bearing witness involves looking into the eye of strangers
whose lives and ideas seem irreconcilable.

``I'm a writer,'' James Baldwin wrote. ``I like doing things alone.'' To
be alone on deadline is the journalist's lot, facing the many-faceted
world and seeking the means to render it, as closely as possible,
knowing that something in the quiver of life is always ineffable and
will slip through the cracks.

When, in Sarajevo, I covered the war in Bosnia and watched lives blown
away daily by indiscriminate Serb shelling, I made the effort to cross
the lines to speak to the nationalist leaders who had twisted Serbian
victimhood into a license for mass murder of Bosnian Muslims.

Gen. Ratko Mladic and Radovan Karadzic, both since convicted of genocide
by an international court, were delirious in the belief that the Muslims
were the old Ottoman Turk enemy, that the Serbs were victims not
perpetrators. History, I learned, can illuminate but also blind.

These men were heinous. Should I have spoken to them? I thought the
quest for understanding demanded it. I don't think I was objective. My
goal was to describe evil.

Today, a quarter-century later, journalists inhabit a historical fault
line. There is a movement in people's minds. The ancien régime is
crumbling, and when that happens there are decapitations.

The shift was well captured by Wesley Lowery, a black journalist who
left The Washington Post after he clashed with the paper's white
executive editor, Marty Baron, over The Post's social media policy and,
more broadly, what constitutes ``journalistic integrity.'' Lowery, as
\href{https://www.nytimes3xbfgragh.onion/2020/06/07/business/media/new-york-times-washington-post-protests.html}{reported
by my colleague Ben Smith,}
\href{https://twitter.com/WesleyLowery/status/1268366363359354885}{tweeted}
in early June that:

``American view-from-nowhere, `objectivity'-obsessed, both-sides
journalism is a failed experiment. \ldots{} We need to rebuild our
industry as one that operates from a place of moral clarity.''

I still believe in both-sides journalism. ``A place of moral clarity''
can easily mean there is only one truth, and if you deviate from it, you
are done for. The liberal idea that freedom is served by open debate,
even with people holding repugnant views, is worth defending. If
conformity wins, democracy dies.

Lowery's tweet came in response to The Times's publication of an Op-Ed
by Senator Tom Cotton calling for the deployment of troops to quell
civil unrest as demonstrators took to the streets, enraged by the
killing of George Floyd, a black man, by a white police officer. The
piece was odious; the editorial process behind it, flawed. A staff
outcry ensued, driven in part by the view that the article was directly
threatening, especially to African-American journalists. This led to the
resignation of James Bennet, the former editorial page editor, and to
the paper saying that publishing the piece was a mistake.

Cotton's dangerous views are supported by millions of Americans,
including Trump. If he is not publishable --- and, in the current
climate, I believe that even flawlessly executed his Op-Ed would have
provoked fury at The Times --- then an old liberal journalistic
consensus is waning. That feels ominous.

Speaking of truth, I was Bennet's boss when he covered the Second
Intifada with extraordinary bravery and aplomb. He was mine until a few
days ago. He is a man of exceptional honor and decency, humanity and
sensitivity --- a thoughtful, progressive, nuanced, open-minded
colleague for over two decades, ``journalistic integrity'' personified.
This is a terrible loss.

I also recognize another truth: that the Floyd killing illustrated that
racism in the United States is systemic, and white-dominated American
newsrooms are ill-equipped to deal with this reality because only more
diversity can capture multiple perspectives.

\emph{The Times is committed to publishing}
\href{https://www.nytimes3xbfgragh.onion/2019/01/31/opinion/letters/letters-to-editor-new-york-times-women.html}{\emph{a
diversity of letters}} \emph{to the editor. We'd like to hear what you
think about this or any of our articles. Here are some}
\href{https://help.nytimes3xbfgragh.onion/hc/en-us/articles/115014925288-How-to-submit-a-letter-to-the-editor}{\emph{tips}}\emph{.
And here's our email:}
\href{mailto:letters@NYTimes.com}{\emph{letters@NYTimes.com}}\emph{.}

\emph{Follow The New York Times Opinion section on}
\href{https://www.facebookcorewwwi.onion/nytopinion}{\emph{Facebook}}\emph{,}
\href{http://twitter.com/NYTOpinion}{\emph{Twitter (@NYTopinion)}}
\emph{and}
\href{https://www.instagram.com/nytopinion/}{\emph{Instagram}}\emph{.}

Advertisement

\protect\hyperlink{after-bottom}{Continue reading the main story}

\hypertarget{site-index}{%
\subsection{Site Index}\label{site-index}}

\hypertarget{site-information-navigation}{%
\subsection{Site Information
Navigation}\label{site-information-navigation}}

\begin{itemize}
\tightlist
\item
  \href{https://help.nytimes3xbfgragh.onion/hc/en-us/articles/115014792127-Copyright-notice}{©~2020~The
  New York Times Company}
\end{itemize}

\begin{itemize}
\tightlist
\item
  \href{https://www.nytco.com/}{NYTCo}
\item
  \href{https://help.nytimes3xbfgragh.onion/hc/en-us/articles/115015385887-Contact-Us}{Contact
  Us}
\item
  \href{https://www.nytco.com/careers/}{Work with us}
\item
  \href{https://nytmediakit.com/}{Advertise}
\item
  \href{http://www.tbrandstudio.com/}{T Brand Studio}
\item
  \href{https://www.nytimes3xbfgragh.onion/privacy/cookie-policy\#how-do-i-manage-trackers}{Your
  Ad Choices}
\item
  \href{https://www.nytimes3xbfgragh.onion/privacy}{Privacy}
\item
  \href{https://help.nytimes3xbfgragh.onion/hc/en-us/articles/115014893428-Terms-of-service}{Terms
  of Service}
\item
  \href{https://help.nytimes3xbfgragh.onion/hc/en-us/articles/115014893968-Terms-of-sale}{Terms
  of Sale}
\item
  \href{https://spiderbites.nytimes3xbfgragh.onion}{Site Map}
\item
  \href{https://help.nytimes3xbfgragh.onion/hc/en-us}{Help}
\item
  \href{https://www.nytimes3xbfgragh.onion/subscription?campaignId=37WXW}{Subscriptions}
\end{itemize}
