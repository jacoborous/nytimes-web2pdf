Sections

SEARCH

\protect\hyperlink{site-content}{Skip to
content}\protect\hyperlink{site-index}{Skip to site index}

\href{https://www.nytimes3xbfgragh.onion/section/business}{Business}

\href{https://myaccount.nytimes3xbfgragh.onion/auth/login?response_type=cookie\&client_id=vi}{}

\href{https://www.nytimes3xbfgragh.onion/section/todayspaper}{Today's
Paper}

\href{/section/business}{Business}\textbar{}Hospitals Got Bailouts and
Furloughed Thousands While Paying C.E.O.s Millions

\url{https://nyti.ms/2XGp6sG}

\begin{itemize}
\item
\item
\item
\item
\item
\end{itemize}

\href{https://www.nytimes3xbfgragh.onion/news-event/coronavirus?action=click\&pgtype=Article\&state=default\&region=TOP_BANNER\&context=storylines_menu}{The
Coronavirus Outbreak}

\begin{itemize}
\tightlist
\item
  live\href{https://www.nytimes3xbfgragh.onion/2020/08/03/world/coronavirus-covid-19.html?action=click\&pgtype=Article\&state=default\&region=TOP_BANNER\&context=storylines_menu}{Latest
  Updates}
\item
  \href{https://www.nytimes3xbfgragh.onion/interactive/2020/us/coronavirus-us-cases.html?action=click\&pgtype=Article\&state=default\&region=TOP_BANNER\&context=storylines_menu}{Maps
  and Cases}
\item
  \href{https://www.nytimes3xbfgragh.onion/interactive/2020/science/coronavirus-vaccine-tracker.html?action=click\&pgtype=Article\&state=default\&region=TOP_BANNER\&context=storylines_menu}{Vaccine
  Tracker}
\item
  \href{https://www.nytimes3xbfgragh.onion/2020/08/02/us/covid-college-reopening.html?action=click\&pgtype=Article\&state=default\&region=TOP_BANNER\&context=storylines_menu}{College
  Reopening}
\item
  \href{https://www.nytimes3xbfgragh.onion/live/2020/08/03/business/stock-market-today-coronavirus?action=click\&pgtype=Article\&state=default\&region=TOP_BANNER\&context=storylines_menu}{Economy}
\end{itemize}

Advertisement

\protect\hyperlink{after-top}{Continue reading the main story}

Supported by

\protect\hyperlink{after-sponsor}{Continue reading the main story}

\hypertarget{hospitals-got-bailouts-and-furloughed-thousands-while-paying-ceos-millions}{%
\section{Hospitals Got Bailouts and Furloughed Thousands While Paying
C.E.O.s
Millions}\label{hospitals-got-bailouts-and-furloughed-thousands-while-paying-ceos-millions}}

Dozens of top recipients of government aid have laid off, furloughed or
cut the pay of tens of thousands of employees.

\includegraphics{https://static01.graylady3jvrrxbe.onion/images/2020/06/07/business/07virus-hospitalbailouts-01/merlin_173208123_8d8443f5-6b8b-4a49-86cc-3ffa9d7b9f8f-articleLarge.jpg?quality=75\&auto=webp\&disable=upscale}

By
\href{https://www.nytimes3xbfgragh.onion/by/jessica-silver-greenberg}{Jessica
Silver-Greenberg},
\href{https://www.nytimes3xbfgragh.onion/by/jesse-drucker}{Jesse
Drucker} and
\href{https://www.nytimes3xbfgragh.onion/by/david-enrich}{David Enrich}

\begin{itemize}
\item
  June 8, 2020
\item
  \begin{itemize}
  \item
  \item
  \item
  \item
  \item
  \end{itemize}
\end{itemize}

HCA Healthcare is one of the world's wealthiest hospital chains. It
earned more than \$7 billion in profits over the past two years. It is
worth \$36 billion. It paid its chief executive \$26 million in 2019.

But as the coronavirus swept the country, employees at HCA repeatedly
complained that the company was not providing adequate protective gear
to nurses, medical technicians and cleaning staff. Last month, HCA
executives warned that they would lay off thousands of nurses if they
didn't agree to wage freezes and other concessions.

A few weeks earlier, HCA had received about \$1 billion in bailout funds
from the federal government, part of an effort to stabilize hospitals
during the pandemic.

HCA is among a long list of deep-pocketed health care companies that
have received billions of dollars in taxpayer funds but are laying off
or cutting the pay of tens of thousands of doctors, nurses and
lower-paid workers. Many have continued to pay their top executives
millions, although some executives have taken modest pay cuts.

The New York Times analyzed tax and securities filings by 60 of the
country's largest hospital chains, which have received a total of more
than \$15 billion in emergency funds through the economic stimulus
package in the federal CARES Act.

The hospitals --- including publicly traded juggernauts like HCA and
Tenet Healthcare, elite nonprofits like the Mayo Clinic, and regional
chains with thousands of beds and billions in cash --- are collectively
\href{https://www.nytimes3xbfgragh.onion/2020/05/25/business/coronavirus-hospitals-bailout.html}{sitting
on tens of billions of dollars} of cash reserves that are supposed to
help them weather an unanticipated storm. And together, they awarded the
five highest-paid officials at each chain about \$874 million in the
most recent year for which they have disclosed their finances.

At least 36 of those hospital chains have laid off, furloughed or
reduced the pay of employees as they try to save money during the
pandemic.

Industry officials argue that furloughs and pay reductions allow
hospitals to keep providing essential services at a time when the
pandemic has gutted their revenue.

But more than a dozen workers at the wealthy hospitals said in
interviews that their employers had put the heaviest financial burdens
on front-line staff, including **** low-paid cafeteria workers, janitors
and nursing assistants. They said pay cuts and furloughs made it even
harder for members of the medical staff to do their jobs, forcing them
to treat more patients in less time.

Even before the coronavirus swept America, forcing hospitals to stop
providing lucrative nonessential surgery and other services, many
smaller hospitals were on the financial brink. In March, lawmakers
sought to address that with a vast federal economic stimulus package
that included \$175 billion for the Department of Health and Human
Services to hand out in grants to hospitals.

\hypertarget{latest-updates-economy}{%
\section{\texorpdfstring{\href{https://www.nytimes3xbfgragh.onion/live/2020/08/03/business/stock-market-today-coronavirus?action=click\&pgtype=Article\&state=default\&region=MAIN_CONTENT_1\&context=storylines_live_updates}{Latest
Updates:
Economy}}{Latest Updates: Economy}}\label{latest-updates-economy}}

\href{https://www.nytimes3xbfgragh.onion/live/2020/08/03/business/stock-market-today-coronavirus?action=click\&pgtype=Article\&state=default\&region=MAIN_CONTENT_1\&context=storylines_live_updates\#the-chicago-fed-president-says-its-up-to-congress-to-save-the-economy}{11h
ago}

\href{https://www.nytimes3xbfgragh.onion/live/2020/08/03/business/stock-market-today-coronavirus?action=click\&pgtype=Article\&state=default\&region=MAIN_CONTENT_1\&context=storylines_live_updates\#the-chicago-fed-president-says-its-up-to-congress-to-save-the-economy}{The
Chicago Fed president says it's up to Congress to save the economy.}

\href{https://www.nytimes3xbfgragh.onion/live/2020/08/03/business/stock-market-today-coronavirus?action=click\&pgtype=Article\&state=default\&region=MAIN_CONTENT_1\&context=storylines_live_updates\#faa-says-boeing-has-effectively-mitigated-defects-in-the-737-max}{11h
ago}

\href{https://www.nytimes3xbfgragh.onion/live/2020/08/03/business/stock-market-today-coronavirus?action=click\&pgtype=Article\&state=default\&region=MAIN_CONTENT_1\&context=storylines_live_updates\#faa-says-boeing-has-effectively-mitigated-defects-in-the-737-max}{F.A.A.
says Boeing has `effectively mitigated' defects in the 737 Max.}

\href{https://www.nytimes3xbfgragh.onion/live/2020/08/03/business/stock-market-today-coronavirus?action=click\&pgtype=Article\&state=default\&region=MAIN_CONTENT_1\&context=storylines_live_updates\#small-businesses-got-emergency-loans-but-not-what-they-expected}{14h
ago}

\href{https://www.nytimes3xbfgragh.onion/live/2020/08/03/business/stock-market-today-coronavirus?action=click\&pgtype=Article\&state=default\&region=MAIN_CONTENT_1\&context=storylines_live_updates\#small-businesses-got-emergency-loans-but-not-what-they-expected}{Small
businesses got emergency loans, but not what they expected.}

\href{https://www.nytimes3xbfgragh.onion/live/2020/08/03/business/stock-market-today-coronavirus?action=click\&pgtype=Article\&state=default\&region=MAIN_CONTENT_1\&context=storylines_live_updates}{See
more updates}

More live coverage:
\href{https://www.nytimes3xbfgragh.onion/2020/08/03/world/coronavirus-covid-19.html?action=click\&pgtype=Article\&state=default\&region=MAIN_CONTENT_1\&context=storylines_live_updates}{Global}

But the formulas to determine how much money hospitals receive were
based largely on their revenue, not their financial needs. As a result,
hospitals serving wealthier patients have received far more funding than
those that treat low-income patients, according to
\href{https://www.kff.org/health-costs/issue-brief/distribution-of-cares-act-funding-among-hospitals/?utm_campaign=KFF-2020-Health-Costs\&utm_source=hs_email\&utm_medium=email\&utm_content=87886305\&_hsenc=p2ANqtz--5H_dI5ybM9j7dEUcpNHIhhhSIhGZKuiOyCgdpPdMkmJ65llIkF9D8sYDfor_aNPAh8_2CoURE3ISuaDmzFepJuTo2FQ}{a
study} by the Kaiser Family Foundation.

One of the bailout's goals was to avoid job losses in health care, said
Zack Cooper, an associate professor of health policy and economics at
Yale University who is a critic of the formulas used to determine the
payouts. ``However, when you see hospitals laying off or furloughing
staff, it's pretty good evidence the way they designed the policy is not
optimal,'' he added.

\includegraphics{https://static01.graylady3jvrrxbe.onion/images/2020/06/08/business/07JPvirus-hospitalbailout3-print/merlin_173249172_f7641b9d-7166-4114-88d5-dbf042db8575-articleLarge.jpg?quality=75\&auto=webp\&disable=upscale}

The Mayo Clinic, with more than eight months of cash in reserve,
received about \$170 million in bailout funds, according to
\href{https://data.covidstimuluswatch.org/prog.php?agency_sum=\&program_sum=\&parent=mayo-clinic\&major_industry_sum=\&hq_id_sum=\&fedsum=\&company_op=starts\&company=\&agency\%5B\%5D=\&program\%5B\%5D=\&major_industry\%5B\%5D=\&hq_id=\&accountability=\&free_text=\&subsidy_op=\%3E\&subsidy=\&face_loan_op=\%3E\&face_loan=\&subsidy_type\%5B\%5D=\&employees_op=\%3E\&employees=\&state=\&city=}{data}
compiled by Good Jobs First, which researches government subsidies of
companies. The Mayo Clinic is furloughing or reducing the working hours
of about 23,000 employees, according to a spokeswoman, who was among
those who went on furlough. A second spokeswoman said that Mayo Clinic
executives have had their pay cut.

Seven chains that together received more than \$1.5 billion in bailout
funds --- Trinity Health, Beaumont Health and the Henry Ford Health
System in Michigan; SSM Health and Mercy in St. Louis; Fairview Health
in Minneapolis; and Prisma Health in South Carolina --- have furloughed
or laid off more than 30,000 workers, according to company officials and
local news reports.

The bailout money, which hospitals received from the Health and Human
Services Department without having to apply for it, came with few
strings attached.

Katherine McKeogh, a department spokeswoman, said it ``encourages
providers to use these funds to maintain delivery capacity by paying and
protecting doctors, nurses and other health care workers.'' The
legislation restricts hospitals' ability to use the bailout funds to pay
top executives, although it doesn't stop recipients from continuing to
award large bonuses.

The hospitals generally declined to comment on how much they are paying
their top executives this year, although they have reported previous
years' compensation in public filings. But some hospitals furloughing
front-line staff or cutting their salaries have trumpeted their top
executives' decisions to take voluntary pay cuts or to contribute
portions of their salary to help their employees.

The for-profit hospital giant Tenet Healthcare, which has received \$345
million in taxpayer assistance since April, has furloughed roughly
11,000 workers, citing the financial pressures from the pandemic. The
company's chief executive, Ron Rittenmeyer, told analysts in May that he
would donate half of his salary for six months to a fund set up to
assist those furloughed workers.

But Mr. Rittenmeyer's salary last year was a small fraction of his \$24
million pay package, which consists largely of stock options and
bonuses, securities filings show. In total, he will wind up donating
roughly \$375,000 to the fund --- equivalent to about 1.5 percent of his
total pay last year.

A Tenet spokeswoman declined to comment on the precise figures.

Image

HCA's chief executive, Samuel Hazen, donated two months of his salary to
a fund to help HCA's workers. That donation would amount to less than 1
percent of his \$26 million compensation last year.Credit...William
DeShazer for The New York Times

The chief executive at HCA, Samuel Hazen, has donated two months of his
salary to a fund to help HCA's workers. Based on his pay last year, that
donation would amount to about \$237,000 --- or less than 1 percent ---
of his \$26 million compensation.

``The leadership cadre of these organizations are going to need to make
sacrifices that are commensurate with the sacrifices of their work
force, not token sacrifices,'' said Jeff Goldsmith, the president of
Health Futures, an industry consulting firm.

Many large nonprofit hospital chains also pay their senior executives
well into the millions of dollars a year.

Dr. Rod Hochman, the chief executive of the Providence Health System,
for instance, was paid more than \$10 million in 2018, the most recent
year for which records are available. Providence received at least \$509
million in federal bailout funds.

A spokeswoman, Melissa Tizon, said Dr. Hochman would take a voluntary
pay cut of 50 percent for the rest of 2020. But that applies only to his
base salary, which in 2018 was less than 20 percent of his total
compensation.

Some of Providence's physicians and nurses have been told to prepare for
pay cuts of at least 10 percent beginning in July. That includes
employees treating coronavirus patients.

Stanford University's health system collected more than \$100 million in
federal bailout grants, adding to its pile of \$2.4 billion of cash that
it can use for any purpose.

Stanford is temporarily cutting the hours of nursing staff, nursing
assistants, janitorial workers and others at its two hospitals. Julie
Greicius, a spokeswoman for Stanford, said the reduction in hours was
intended ``to keep everyone employed and our staff at full wages with
benefits intact.''

Ms. Greicius said David Entwistle, the chief executive of Stanford's
health system, had the choice of reducing his pay by 20 percent or
taking time off, and chose to reduce his working hours but ``is
maintaining his earning level by using paid time off.'' In 2018, the
latest year for which Stanford has disclosed his compensation, Mr.
Entwistle earned about \$2.8 million. Ms. Greicius said the majority of
employees made the same choice as Mr. Entwistle.

HCA's \$1 billion in federal grants appears to make it the largest
beneficiary of health care bailout funds. But its medical workers have a
long list of complaints about what they see as penny-pinching practices.

Since the pandemic began, medical workers at 19 HCA hospitals have filed
complaints with the Occupational Safety and Health Administration about
the lack of respirator masks and being forced to reuse medical gowns,
according to copies of the complaints reviewed by The Times.

Ed Fishbough, an HCA spokesman, said that despite a global shortage of
masks and other protective gear, the company had ``provided appropriate
P.P.E., including a universal masking policy implemented in March
requiring all staff in all areas to wear masks, including N95s, in line
with C.D.C. guidance.''

Celia Yap-Banago, a nurse at an HCA hospital in Kansas City, Mo., died
from the virus in April, a month after her colleagues complained to OSHA
that she had to treat a patient without wearing protective gear. The
next month, Rosa Luna, who cleaned patient rooms at HCA's hospital in
Riverside, Calif., also died of the virus; her colleagues had warned
executives in emails that workers, especially those cleaning hospital
rooms, weren't provided proper masks.

Around the time of Ms. Luna's death, HCA executives delivered a warning
to officials at the Service Employees International Union and National
Nurses United, which represent many HCA employees. The company would lay
off up to 10 percent of their members, unless the unionized workers
amended their contracts to incorporate wage freezes and the elimination
of company contributions to workers' retirement plans, among other
concessions.

Image

Nurses at more than a dozen HCA hospitals, including one in Trinity,
Fla., staged protests in early April about what they said was a lack of
proper measures to protect them against the coronavirus.Credit...Eve
Edelheit/Bloomberg

Nurses responded by staging protests in front of more than a dozen HCA
hospitals.

``We don't work in a jelly bean factory, where it's OK if we make a blue
jelly bean instead of a red one,'' said Kathy Montanino, a nurse
treating Covid-19 patients at HCA's Riverside hospital. ``We are dealing
with people's lives, and this company puts their profits over patients
and their staff.''

Mr. Fishbough, the spokesman, said HCA ``has not laid off or furloughed
a single caregiver due to the pandemic.'' He said the company had been
paying medical workers 70 percent of their base pay, even if they were
not working. Mr. Fishbough said that executives had taken pay cuts, but
that the unions had refused to take similar steps.

``While we hope to continue to avoid layoffs, the unions' decisions have
made that more difficult ****** for our facilities that are unionized,''
he said. The dispute continues.

Apparently anticipating a strike, a unit of HCA recently created ``a new
line of business focused on staffing strike-related labor shortages,''
according to an email that an HCA recruiter sent to nurses.

The email, reviewed by The Times, said nurses who joined the venture
would earn more than they did in their current jobs: up to \$980 per
shift, plus a \$150 ``Show Up'' bonus and a continental breakfast.

Advertisement

\protect\hyperlink{after-bottom}{Continue reading the main story}

\hypertarget{site-index}{%
\subsection{Site Index}\label{site-index}}

\hypertarget{site-information-navigation}{%
\subsection{Site Information
Navigation}\label{site-information-navigation}}

\begin{itemize}
\tightlist
\item
  \href{https://help.nytimes3xbfgragh.onion/hc/en-us/articles/115014792127-Copyright-notice}{©~2020~The
  New York Times Company}
\end{itemize}

\begin{itemize}
\tightlist
\item
  \href{https://www.nytco.com/}{NYTCo}
\item
  \href{https://help.nytimes3xbfgragh.onion/hc/en-us/articles/115015385887-Contact-Us}{Contact
  Us}
\item
  \href{https://www.nytco.com/careers/}{Work with us}
\item
  \href{https://nytmediakit.com/}{Advertise}
\item
  \href{http://www.tbrandstudio.com/}{T Brand Studio}
\item
  \href{https://www.nytimes3xbfgragh.onion/privacy/cookie-policy\#how-do-i-manage-trackers}{Your
  Ad Choices}
\item
  \href{https://www.nytimes3xbfgragh.onion/privacy}{Privacy}
\item
  \href{https://help.nytimes3xbfgragh.onion/hc/en-us/articles/115014893428-Terms-of-service}{Terms
  of Service}
\item
  \href{https://help.nytimes3xbfgragh.onion/hc/en-us/articles/115014893968-Terms-of-sale}{Terms
  of Sale}
\item
  \href{https://spiderbites.nytimes3xbfgragh.onion}{Site Map}
\item
  \href{https://help.nytimes3xbfgragh.onion/hc/en-us}{Help}
\item
  \href{https://www.nytimes3xbfgragh.onion/subscription?campaignId=37WXW}{Subscriptions}
\end{itemize}
