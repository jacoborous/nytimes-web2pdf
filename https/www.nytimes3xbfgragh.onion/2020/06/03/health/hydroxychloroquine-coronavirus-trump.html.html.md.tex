Sections

SEARCH

\protect\hyperlink{site-content}{Skip to
content}\protect\hyperlink{site-index}{Skip to site index}

\href{https://www.nytimes3xbfgragh.onion/section/health}{Health}

\href{https://myaccount.nytimes3xbfgragh.onion/auth/login?response_type=cookie\&client_id=vi}{}

\href{https://www.nytimes3xbfgragh.onion/section/todayspaper}{Today's
Paper}

\href{/section/health}{Health}\textbar{}Malaria Drug Promoted by Trump
Did Not Prevent Covid Infections, Study Finds

\url{https://nyti.ms/3gUoxn0}

\begin{itemize}
\item
\item
\item
\item
\item
\end{itemize}

\href{https://www.nytimes3xbfgragh.onion/news-event/coronavirus?action=click\&pgtype=Article\&state=default\&region=TOP_BANNER\&context=storylines_menu}{The
Coronavirus Outbreak}

\begin{itemize}
\tightlist
\item
  live\href{https://www.nytimes3xbfgragh.onion/2020/08/01/world/coronavirus-covid-19.html?action=click\&pgtype=Article\&state=default\&region=TOP_BANNER\&context=storylines_menu}{Latest
  Updates}
\item
  \href{https://www.nytimes3xbfgragh.onion/interactive/2020/us/coronavirus-us-cases.html?action=click\&pgtype=Article\&state=default\&region=TOP_BANNER\&context=storylines_menu}{Maps
  and Cases}
\item
  \href{https://www.nytimes3xbfgragh.onion/interactive/2020/science/coronavirus-vaccine-tracker.html?action=click\&pgtype=Article\&state=default\&region=TOP_BANNER\&context=storylines_menu}{Vaccine
  Tracker}
\item
  \href{https://www.nytimes3xbfgragh.onion/interactive/2020/07/29/us/schools-reopening-coronavirus.html?action=click\&pgtype=Article\&state=default\&region=TOP_BANNER\&context=storylines_menu}{What
  School May Look Like}
\item
  \href{https://www.nytimes3xbfgragh.onion/live/2020/07/31/business/stock-market-today-coronavirus?action=click\&pgtype=Article\&state=default\&region=TOP_BANNER\&context=storylines_menu}{Economy}
\end{itemize}

Advertisement

\protect\hyperlink{after-top}{Continue reading the main story}

Supported by

\protect\hyperlink{after-sponsor}{Continue reading the main story}

\hypertarget{malaria-drug-promoted-by-trump-did-not-prevent-covid-infections-study-finds}{%
\section{Malaria Drug Promoted by Trump Did Not Prevent Covid
Infections, Study
Finds}\label{malaria-drug-promoted-by-trump-did-not-prevent-covid-infections-study-finds}}

The first carefully controlled trial of hydroxychloroquine given to
people exposed to the coronavirus did not show any benefit.

\includegraphics{https://static01.graylady3jvrrxbe.onion/images/2020/06/03/science/03VIRUS-HCQ2/03VIRUS-HCQ2-articleLarge.jpg?quality=75\&auto=webp\&disable=upscale}

By \href{https://www.nytimes3xbfgragh.onion/by/denise-grady}{Denise
Grady}

\begin{itemize}
\item
  Published June 3, 2020Updated June 20, 2020
\item
  \begin{itemize}
  \item
  \item
  \item
  \item
  \item
  \end{itemize}
\end{itemize}

The malaria drug
\href{https://www.nytimes3xbfgragh.onion/2020/06/20/health/hydroxychloroquine-coronavirus-trial.html}{hydroxychloroquine}
did not prevent
\href{https://www.nytimes3xbfgragh.onion/2020/06/15/health/fda-hydroxychloroquine-malaria.html}{Covid-19}
in a rigorous study of 821 people who had been exposed to patients
infected with the virus, researchers from the University of Minnesota
and Canada are reporting on Wednesday.

The study was the first large controlled clinical trial of
hydroxychloroquine, a drug that
\href{https://www.nytimes3xbfgragh.onion/2020/05/21/us/politics/trump-fact-check-hydroxychloroquine-coronavirus-.html}{President
Trump has repeatedly promoted} and
\href{https://www.nytimes3xbfgragh.onion/2020/05/18/us/politics/trump-hydroxychloroquine-covid-coronavirus.html}{recently
taken himself}. Conducted in the United States and Canada, this trial
was also the first to test whether the drug could prevent illness in
people who have been exposed to the coronavirus.

This type of study, in which patients are picked at random to receive
either an experimental treatment or a placebo, is considered the most
reliable way to measure the safety and effectiveness of a drug. The
participants were health care workers and people who had been exposed at
home to ill spouses, partners or parents.

``The take-home message for the general public is that if you're exposed
to someone with Covid-19, hydroxychloroquine is not an effective
post-exposure preventive therapy,'' the lead author of the study, Dr.
David R. Boulware, from the University of Minnesota, said in an
interview.

\href{https://www.nejm.org/doi/full/10.1056/NEJMoa2016638}{The results
were published in The New England Journal of Medicine.}

``If we could find something that would ameliorate infection, block it
or make it milder after a solid exposure, that would be quite
wonderful,'' said Dr. Judith Feinberg, the vice chairwoman for research
in medicine at West Virginia University. ``What we want to do is limit
the number of cases. There was great hope riding on this.''

The president's promotion of the drug, and the backlash against it, have
politicized medical questions that would normally have been left to
researchers to answer objectively. Trump supporters and opponents have
accused one another of twisting facts about the drug to make the
president look either right or wrong.

Regardless, Mr. Trump has not stopped touting the drug's potential
benefits. On Sunday, his administration announced that it was sending 2
million doses of the drug to Brazil, to treat patients and help prevent
infection in health care workers. A White House official said the two
countries would collaborate on research into its use.

Early in the pandemic, the drug's use was spurred by anecdotal reports
from China and France of patients who seemed to improve and laboratory
findings of a possible antiviral effect. With no proven treatment for
Covid-19, doctors have been desperate to give severely ill patients some
kind of therapy.

But several studies on sick patients, without control groups, have found
no benefit and even possible harm from the drug. A recent Lancet study
reported increased risks of heart problems and death. Shortly after it
was published, the World Health Organization suspended trials of the
drug.

\href{https://cdf.nejm.org/services/GetOnlineFirstPDF.aspx?DOI=NEJMoa2016638}{But
the Lancet data has been called into question}. On Wednesday, the World
Health Organization said it would resume the trials it had suspended.

\hypertarget{latest-updates-global-coronavirus-outbreak}{%
\section{\texorpdfstring{\href{https://www.nytimes3xbfgragh.onion/2020/08/01/world/coronavirus-covid-19.html?action=click\&pgtype=Article\&state=default\&region=MAIN_CONTENT_1\&context=storylines_live_updates}{Latest
Updates: Global Coronavirus
Outbreak}}{Latest Updates: Global Coronavirus Outbreak}}\label{latest-updates-global-coronavirus-outbreak}}

Updated 2020-08-02T07:42:09.613Z

\begin{itemize}
\tightlist
\item
  \href{https://www.nytimes3xbfgragh.onion/2020/08/01/world/coronavirus-covid-19.html?action=click\&pgtype=Article\&state=default\&region=MAIN_CONTENT_1\&context=storylines_live_updates\#link-34047410}{The
  U.S. reels as July cases more than double the total of any other
  month.}
\item
  \href{https://www.nytimes3xbfgragh.onion/2020/08/01/world/coronavirus-covid-19.html?action=click\&pgtype=Article\&state=default\&region=MAIN_CONTENT_1\&context=storylines_live_updates\#link-780ec966}{Top
  U.S. officials work to break an impasse over the federal jobless
  benefit.}
\item
  \href{https://www.nytimes3xbfgragh.onion/2020/08/01/world/coronavirus-covid-19.html?action=click\&pgtype=Article\&state=default\&region=MAIN_CONTENT_1\&context=storylines_live_updates\#link-2bc8948}{Its
  outbreak untamed, Melbourne goes into even greater lockdown.}
\end{itemize}

\href{https://www.nytimes3xbfgragh.onion/2020/08/01/world/coronavirus-covid-19.html?action=click\&pgtype=Article\&state=default\&region=MAIN_CONTENT_1\&context=storylines_live_updates}{See
more updates}

More live coverage:
\href{https://www.nytimes3xbfgragh.onion/live/2020/07/31/business/stock-market-today-coronavirus?action=click\&pgtype=Article\&state=default\&region=MAIN_CONTENT_1\&context=storylines_live_updates}{Markets}

Dr. Soumya Swaminathan, the deputy director of the W.H.O., said, ``As of
now, there is no evidence that any drug actually reduces the mortality
in patients who have Covid-19, and in fact it is an urgent priority for
all of us to do the needed studies, to do the randomized clinical trials
in order to get that evidence as quickly as possible.''

\includegraphics{https://static01.graylady3jvrrxbe.onion/images/2020/06/03/science/03VIRUS-HCQ/merlin_170749611_f25898fd-a43e-4be0-949d-f0d7344f461f-articleLarge.jpg?quality=75\&auto=webp\&disable=upscale}

Interest in the drug surged after Mr. Trump began advocating it. It is
approved to treat rheumatoid arthritis and lupus, as well as malaria,
and is considered safe for those patients as long as they do not have
underlying abnormalities in their heart rhythm.

Studies in very ill coronavirus patients have linked the drug ---
especially when combined with the antibiotic azithromycin --- to
dangerous heart-rhythm disorders, and
\href{https://www.nytimes3xbfgragh.onion/2020/04/24/health/fda-hydroxychloroquine-coronavirus.html}{both
the Food and Drug Administration} and the National Institute of Allergy
and Infectious Diseases have warned that it should not be used outside
of clinical trials or carefully monitored conditions in a hospital.

Some researchers say that safety concerns about the drug have been
overblown, alarming the public and making it difficult to recruit
participants for the studies needed to determine whether the drug has
any value for treatment or prevention.

The new study included 821 people from across the United States and
parts of Canada who had a either a high-risk or moderate-risk exposure
to a person who had tested positive and was ill from the coronavirus.
None of the participants had symptoms themselves. High-risk exposure
meant they were less than six feet from a patient for more than ten
minutes, with neither a mask nor a face shield. Moderate risk meant they
wore a mask, but no face shield.

About 88 percent had high-risk exposures.

The participants, recruited online, ranged in age from 33 to 50, with a
median age of 40. About half were women, and 66 percent of the total
were health care workers. They were healthy and had no underlying health
problems that would have made hydroxychloroquine dangerous for them.
Most of the rest had been exposed at home, to an infected spouse,
partner or parent.

Within four days of exposure, the participants were picked at random to
receive either hydroxychloroquine or a placebo, and then followed to
determine whether they had either laboratory-confirmed Covid-19, or an
illness consistent with the virus, during the next 14 days.

The drug or placebos were mailed to them, and they then reported their
symptoms online to the researchers, who did not examine them.

Not all the participants could be tested for the virus, because when the
study was being conducted, there was still a shortage of test kits.

There was no meaningful difference between the placebo group and those
who took the drug. Among those taking hydroxychloroquine, 49 of 414, or
11.8 percent, became ill. In the placebo group, 58 or 407, or 14.3
percent, became ill. Analyzed statistically, the difference between
those rates was not significant.

\href{https://www.nytimes3xbfgragh.onion/news-event/coronavirus?action=click\&pgtype=Article\&state=default\&region=MAIN_CONTENT_3\&context=storylines_faq}{}

\hypertarget{the-coronavirus-outbreak-}{%
\subsubsection{The Coronavirus Outbreak
›}\label{the-coronavirus-outbreak-}}

\hypertarget{frequently-asked-questions}{%
\paragraph{Frequently Asked
Questions}\label{frequently-asked-questions}}

Updated July 27, 2020

\begin{itemize}
\item ~
  \hypertarget{should-i-refinance-my-mortgage}{%
  \paragraph{Should I refinance my
  mortgage?}\label{should-i-refinance-my-mortgage}}

  \begin{itemize}
  \tightlist
  \item
    \href{https://www.nytimes3xbfgragh.onion/article/coronavirus-money-unemployment.html?action=click\&pgtype=Article\&state=default\&region=MAIN_CONTENT_3\&context=storylines_faq}{It
    could be a good idea,} because mortgage rates have
    \href{https://www.nytimes3xbfgragh.onion/2020/07/16/business/mortgage-rates-below-3-percent.html?action=click\&pgtype=Article\&state=default\&region=MAIN_CONTENT_3\&context=storylines_faq}{never
    been lower.} Refinancing requests have pushed mortgage applications
    to some of the highest levels since 2008, so be prepared to get in
    line. But defaults are also up, so if you're thinking about buying a
    home, be aware that some lenders have tightened their standards.
  \end{itemize}
\item ~
  \hypertarget{what-is-school-going-to-look-like-in-september}{%
  \paragraph{What is school going to look like in
  September?}\label{what-is-school-going-to-look-like-in-september}}

  \begin{itemize}
  \tightlist
  \item
    It is unlikely that many schools will return to a normal schedule
    this fall, requiring the grind of
    \href{https://www.nytimes3xbfgragh.onion/2020/06/05/us/coronavirus-education-lost-learning.html?action=click\&pgtype=Article\&state=default\&region=MAIN_CONTENT_3\&context=storylines_faq}{online
    learning},
    \href{https://www.nytimes3xbfgragh.onion/2020/05/29/us/coronavirus-child-care-centers.html?action=click\&pgtype=Article\&state=default\&region=MAIN_CONTENT_3\&context=storylines_faq}{makeshift
    child care} and
    \href{https://www.nytimes3xbfgragh.onion/2020/06/03/business/economy/coronavirus-working-women.html?action=click\&pgtype=Article\&state=default\&region=MAIN_CONTENT_3\&context=storylines_faq}{stunted
    workdays} to continue. California's two largest public school
    districts --- Los Angeles and San Diego --- said on July 13, that
    \href{https://www.nytimes3xbfgragh.onion/2020/07/13/us/lausd-san-diego-school-reopening.html?action=click\&pgtype=Article\&state=default\&region=MAIN_CONTENT_3\&context=storylines_faq}{instruction
    will be remote-only in the fall}, citing concerns that surging
    coronavirus infections in their areas pose too dire a risk for
    students and teachers. Together, the two districts enroll some
    825,000 students. They are the largest in the country so far to
    abandon plans for even a partial physical return to classrooms when
    they reopen in August. For other districts, the solution won't be an
    all-or-nothing approach.
    \href{https://bioethics.jhu.edu/research-and-outreach/projects/eschool-initiative/school-policy-tracker/}{Many
    systems}, including the nation's largest, New York City, are
    devising
    \href{https://www.nytimes3xbfgragh.onion/2020/06/26/us/coronavirus-schools-reopen-fall.html?action=click\&pgtype=Article\&state=default\&region=MAIN_CONTENT_3\&context=storylines_faq}{hybrid
    plans} that involve spending some days in classrooms and other days
    online. There's no national policy on this yet, so check with your
    municipal school system regularly to see what is happening in your
    community.
  \end{itemize}
\item ~
  \hypertarget{is-the-coronavirus-airborne}{%
  \paragraph{Is the coronavirus
  airborne?}\label{is-the-coronavirus-airborne}}

  \begin{itemize}
  \tightlist
  \item
    The coronavirus
    \href{https://www.nytimes3xbfgragh.onion/2020/07/04/health/239-experts-with-one-big-claim-the-coronavirus-is-airborne.html?action=click\&pgtype=Article\&state=default\&region=MAIN_CONTENT_3\&context=storylines_faq}{can
    stay aloft for hours in tiny droplets in stagnant air}, infecting
    people as they inhale, mounting scientific evidence suggests. This
    risk is highest in crowded indoor spaces with poor ventilation, and
    may help explain super-spreading events reported in meatpacking
    plants, churches and restaurants.
    \href{https://www.nytimes3xbfgragh.onion/2020/07/06/health/coronavirus-airborne-aerosols.html?action=click\&pgtype=Article\&state=default\&region=MAIN_CONTENT_3\&context=storylines_faq}{It's
    unclear how often the virus is spread} via these tiny droplets, or
    aerosols, compared with larger droplets that are expelled when a
    sick person coughs or sneezes, or transmitted through contact with
    contaminated surfaces, said Linsey Marr, an aerosol expert at
    Virginia Tech. Aerosols are released even when a person without
    symptoms exhales, talks or sings, according to Dr. Marr and more
    than 200 other experts, who
    \href{https://academic.oup.com/cid/article/doi/10.1093/cid/ciaa939/5867798}{have
    outlined the evidence in an open letter to the World Health
    Organization}.
  \end{itemize}
\item ~
  \hypertarget{what-are-the-symptoms-of-coronavirus}{%
  \paragraph{What are the symptoms of
  coronavirus?}\label{what-are-the-symptoms-of-coronavirus}}

  \begin{itemize}
  \tightlist
  \item
    Common symptoms
    \href{https://www.nytimes3xbfgragh.onion/article/symptoms-coronavirus.html?action=click\&pgtype=Article\&state=default\&region=MAIN_CONTENT_3\&context=storylines_faq}{include
    fever, a dry cough, fatigue and difficulty breathing or shortness of
    breath.} Some of these symptoms overlap with those of the flu,
    making detection difficult, but runny noses and stuffy sinuses are
    less common.
    \href{https://www.nytimes3xbfgragh.onion/2020/04/27/health/coronavirus-symptoms-cdc.html?action=click\&pgtype=Article\&state=default\&region=MAIN_CONTENT_3\&context=storylines_faq}{The
    C.D.C. has also} added chills, muscle pain, sore throat, headache
    and a new loss of the sense of taste or smell as symptoms to look
    out for. Most people fall ill five to seven days after exposure, but
    symptoms may appear in as few as two days or as many as 14 days.
  \end{itemize}
\item ~
  \hypertarget{does-asymptomatic-transmission-of-covid-19-happen}{%
  \paragraph{Does asymptomatic transmission of Covid-19
  happen?}\label{does-asymptomatic-transmission-of-covid-19-happen}}

  \begin{itemize}
  \tightlist
  \item
    So far, the evidence seems to show it does. A widely cited
    \href{https://www.nature.com/articles/s41591-020-0869-5}{paper}
    published in April suggests that people are most infectious about
    two days before the onset of coronavirus symptoms and estimated that
    44 percent of new infections were a result of transmission from
    people who were not yet showing symptoms. Recently, a top expert at
    the World Health Organization stated that transmission of the
    coronavirus by people who did not have symptoms was ``very rare,''
    \href{https://www.nytimes3xbfgragh.onion/2020/06/09/world/coronavirus-updates.html?action=click\&pgtype=Article\&state=default\&region=MAIN_CONTENT_3\&context=storylines_faq\#link-1f302e21}{but
    she later walked back that statement.}
  \end{itemize}
\end{itemize}

The drug also did not make the illness any less severe.

Side effects like nausea from hydroxychloroquine were more common than
from placebos, 40.1 percent compared with 16.8 percent, but there were
no problems with heart rhythm or any other serious adverse effects.

Infectious disease experts who were not part of the study said it was
well done and answered an important question, though the results were
disappointing.

Dr. William Schaffner, an infectious disease specialist at Vanderbilt
University, said: ``This was a large, randomized controlled trial done
by very good people. Hydroxychloroquine did not provide a notable
advantage.''

Noting that the drug had shown some ability to prevent the virus from
infecting cells in laboratory studies, Dr. Schaffner said,
``Unfortunately that did not translate into a beneficial effect in
preventing the development of illness.''

An \href{https://www.nejm.org/doi/full/10.1056/NEJMe2020388}{editorial
accompanying the stu}dy pointed out some limitations: the lack of
testing made it impossible to know for sure how many participants
actually had Covid-19, and only 75 percent in the hydroxychloroquine
group took the full course (some quit because of side effects). In
addition, waiting four days after exposure to begin taking the drug may
not have given it a chance to work, and starting it earlier might have
had a different result.

The editorialist, Dr. Myron S. Cohen from the University of North
Carolina, called the results more provocative than definitive, and
wrote, ``the potential prevention benefits of hydroxychloroquine remain
to be determined.''

The study did not address the question of whether hydroxychloroquine can
prevent coronavirus infection if people take it before they are exposed
to sick patients. That possibility is being studied in other clinical
trials involving health care workers and emergency medical technicians
and other emergency medical workers.

At a Senate hearing on the F.D.A.'s oversight of foreign drug
manufacturing on Tuesday, Democrats criticized the agency for its
decision in March to give an emergency use authorization to
hydroxychloroquine.

``The F.D.A., in my view, bowed to the pressure and issued what's called
an `emergency use authorization' for the drug,'' said Sen. Ron Wyden of
Oregon, the ranking Democrat on the Senate Committee on Finance, which
sponsored the hearing. ``Doing so threw open the door to tens of
millions of pills, including some, directly related to this hearing,
manufactured inside facilities in Pakistan and India that have either
failed F.D.A.'s inspection or never been inspected by the F.D.A. at
all.''

Jeremy Kahn, an F.D.A. spokesman, said the agency has done nine
``mission-critical'' drug facility inspections, including overseas and
domestic, since March. But he did not say whether they involved
hydroxychloroquine. He also said the agency would not resume regular
overseas inspections until the State Department has given the go-ahead
for travel.

Sheila Kaplan contributed reporting.

Advertisement

\protect\hyperlink{after-bottom}{Continue reading the main story}

\hypertarget{site-index}{%
\subsection{Site Index}\label{site-index}}

\hypertarget{site-information-navigation}{%
\subsection{Site Information
Navigation}\label{site-information-navigation}}

\begin{itemize}
\tightlist
\item
  \href{https://help.nytimes3xbfgragh.onion/hc/en-us/articles/115014792127-Copyright-notice}{©~2020~The
  New York Times Company}
\end{itemize}

\begin{itemize}
\tightlist
\item
  \href{https://www.nytco.com/}{NYTCo}
\item
  \href{https://help.nytimes3xbfgragh.onion/hc/en-us/articles/115015385887-Contact-Us}{Contact
  Us}
\item
  \href{https://www.nytco.com/careers/}{Work with us}
\item
  \href{https://nytmediakit.com/}{Advertise}
\item
  \href{http://www.tbrandstudio.com/}{T Brand Studio}
\item
  \href{https://www.nytimes3xbfgragh.onion/privacy/cookie-policy\#how-do-i-manage-trackers}{Your
  Ad Choices}
\item
  \href{https://www.nytimes3xbfgragh.onion/privacy}{Privacy}
\item
  \href{https://help.nytimes3xbfgragh.onion/hc/en-us/articles/115014893428-Terms-of-service}{Terms
  of Service}
\item
  \href{https://help.nytimes3xbfgragh.onion/hc/en-us/articles/115014893968-Terms-of-sale}{Terms
  of Sale}
\item
  \href{https://spiderbites.nytimes3xbfgragh.onion}{Site Map}
\item
  \href{https://help.nytimes3xbfgragh.onion/hc/en-us}{Help}
\item
  \href{https://www.nytimes3xbfgragh.onion/subscription?campaignId=37WXW}{Subscriptions}
\end{itemize}
