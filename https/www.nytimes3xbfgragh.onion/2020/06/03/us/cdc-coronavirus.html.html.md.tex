Sections

SEARCH

\protect\hyperlink{site-content}{Skip to
content}\protect\hyperlink{site-index}{Skip to site index}

\href{/section/us}{U.S.}\textbar{}The C.D.C. Waited `Its Entire
Existence for This Moment.' What Went Wrong?

\url{https://nyti.ms/2zLALxv}

\begin{itemize}
\item
\item
\item
\item
\item
\item
\end{itemize}

\hypertarget{the-coronavirus-outbreak}{%
\subsubsection{\texorpdfstring{\href{https://www.nytimes3xbfgragh.onion/news-event/coronavirus?name=styln-coronavirus-national\&region=TOP_BANNER\&variant=undefined\&block=storyline_menu_recirc\&action=click\&pgtype=Article\&impression_id=9cc8e490-e37d-11ea-8f3e-bdd331327cb0}{The
Coronavirus
Outbreak}}{The Coronavirus Outbreak}}\label{the-coronavirus-outbreak}}

\begin{itemize}
\tightlist
\item
  live\href{https://www.nytimes3xbfgragh.onion/2020/08/20/world/coronavirus-covid.html?name=styln-coronavirus-national\&region=TOP_BANNER\&variant=undefined\&block=storyline_menu_recirc\&action=click\&pgtype=Article\&impression_id=9cc90ba0-e37d-11ea-8f3e-bdd331327cb0}{Latest
  Updates}
\item
  \href{https://www.nytimes3xbfgragh.onion/interactive/2020/us/coronavirus-us-cases.html?name=styln-coronavirus-national\&region=TOP_BANNER\&variant=undefined\&block=storyline_menu_recirc\&action=click\&pgtype=Article\&impression_id=9cc90ba1-e37d-11ea-8f3e-bdd331327cb0}{Maps
  and Cases}
\item
  \href{https://www.nytimes3xbfgragh.onion/interactive/2020/science/coronavirus-vaccine-tracker.html?name=styln-coronavirus-national\&region=TOP_BANNER\&variant=undefined\&block=storyline_menu_recirc\&action=click\&pgtype=Article\&impression_id=9cc90ba2-e37d-11ea-8f3e-bdd331327cb0}{Vaccine
  Tracker}
\item
  \href{https://www.nytimes3xbfgragh.onion/2020/08/19/us/colleges-closing-covid.html?name=styln-coronavirus-national\&region=TOP_BANNER\&variant=undefined\&block=storyline_menu_recirc\&action=click\&pgtype=Article\&impression_id=9cc90ba3-e37d-11ea-8f3e-bdd331327cb0}{Colleges
  Closing}
\item
  \href{https://www.nytimes3xbfgragh.onion/live/2020/08/20/business/stock-market-today-coronavirus?name=styln-coronavirus-national\&region=TOP_BANNER\&variant=undefined\&block=storyline_menu_recirc\&action=click\&pgtype=Article\&impression_id=9cc90ba4-e37d-11ea-8f3e-bdd331327cb0}{Economy}
\end{itemize}

\includegraphics{https://static01.graylady3jvrrxbe.onion/images/2020/06/03/multimedia/00virus-cdc-3/merlin_172386342_7071e602-ae66-4b7a-83aa-daa22a02a3e1-articleLarge.jpg?quality=75\&auto=webp\&disable=upscale}

\hypertarget{the-cdc-waited-its-entire-existence-for-this-moment-what-went-wrong}{%
\section{The C.D.C. Waited `Its Entire Existence for This Moment.' What
Went
Wrong?}\label{the-cdc-waited-its-entire-existence-for-this-moment-what-went-wrong}}

The technology was old, the data poor, the bureaucracy slow, the
guidance confusing, the administration not in agreement. The coronavirus
shook the world's premier health agency, creating a loss of confidence
and hampering the U.S. response to the crisis.

Coronavirus patients on ventilators at Elmhurst Hospital in Queens,
N.Y., last month.Credit...Erin Schaff/The New York Times

Supported by

\protect\hyperlink{after-sponsor}{Continue reading the main story}

By \href{https://www.nytimes3xbfgragh.onion/by/eric-lipton}{Eric
Lipton},
\href{https://www.nytimes3xbfgragh.onion/by/abby-goodnough}{Abby
Goodnough},
\href{https://www.nytimes3xbfgragh.onion/by/michael-d-shear}{Michael D.
Shear}, \href{https://www.nytimes3xbfgragh.onion/by/megan-twohey}{Megan
Twohey},
\href{https://www.nytimes3xbfgragh.onion/by/apoorva-mandavilli}{Apoorva
Mandavilli},
\href{https://www.nytimes3xbfgragh.onion/by/sheri-fink}{Sheri Fink} and
Mark Walker

\begin{itemize}
\item
  Published June 3, 2020Updated Aug. 14, 2020
\item
  \begin{itemize}
  \item
  \item
  \item
  \item
  \item
  \item
  \end{itemize}
\end{itemize}

\hypertarget{listen-to-this-article}{%
\subsubsection{Listen to This Article}\label{listen-to-this-article}}

Audio Recording by Audm

\emph{To hear more audio stories from publishers like The New York
Times, download}
\href{https://www.audm.com/?utm_source=nyt\&utm_medium=embed\&utm_campaign=cdc_existence_wrong}{\emph{Audm
for iPhone or Android}}\emph{.}

WASHINGTON --- Americans returning from China landed at U.S. airports by
the thousands in early February, potential carriers of a deadly
\href{https://www.nytimes3xbfgragh.onion/2020/06/15/health/coronavirus-underlying-conditions.html}{virus}
who had been diverted to a handful of cities for screening by the
Centers for Disease Control and Prevention.

Their arrival prompted a frantic scramble by local and state officials
to press the travelers to self-quarantine, and to monitor whether anyone
fell ill. It was one of the earliest tests of whether the public health
system in the United States could contain the contagion.

But the effort was frustrated **** as the C.D.C.'s decades-old
notification system delivered information collected at the airports that
was riddled with duplicative records, bad phone numbers and incomplete
addresses. For weeks, officials tried to track passengers using lists
sent by the C.D.C., scouring information about each flight in separate
spreadsheets.

``It was insane,'' said Dr. Sharon Balter, a director at the Los Angeles
County Department of Public Health. When the system went offline in
mid-February, briefly halting the flow of passenger data, local
officials listened in disbelief on a conference call as the C.D.C.
responded to the possibility that infected travelers might slip away.

``Just let them go,'' two of the health officials recall being told.

The flawed effort was an early revelation for some health departments,
whose confidence in the C.D.C. was shaken as it confronted the most
urgent public health emergency in its 74-year history --- a pathogen
that has penetrated much of the nation, killing more than 100,000
people.

The C.D.C., long considered the world's premier health agency, made
early testing mistakes that contributed to a cascade of problems that
persist today as the country tries to reopen. It failed to provide
timely counts of infections and deaths, hindered by aging technology and
a fractured public health reporting system. And it hesitated in
absorbing the lessons of other countries, including the perils of silent
carriers spreading the infection.

The agency struggled to calibrate its own imperative to be cautious and
the need to move fast as the coronavirus ravaged the country, according
to a review of thousands of emails and interviews with more than 100
state and federal officials, public health experts, C.D.C. employees and
medical workers. In communicating to the public, its leadership was
barely visible, its stream of guidance was often slow and its messages
were sometimes confusing, **** sowing mistrust.

``They let us down,'' said Dr. Stephane Otmezguine, an anesthesiologist
who treated coronavirus patients in Fort Lauderdale, Fla. Richard
Whitley, the top health official in Nevada, wrote to the C.D.C. director
about a communication ``breakdown'' between the states and the agency.
Gov. J.B. Pritzker of Illinois lashed out at the agency over testing,
saying that the government's response would ``go down in history as a
profound failure.''

\includegraphics{https://static01.graylady3jvrrxbe.onion/images/2020/06/03/multimedia/00virus-cdc-tear-03/00virus-cdc-tear1-03-articleLarge.jpg?quality=75\&auto=webp\&disable=upscale}

``The C.D.C. is no longer the reliable go-to place,'' said Dr. Ashish
Jha, the director of the Harvard Global Health Institute.

Even as the virus tested the C.D.C.'s capacity to respond, the agency
and its director, Dr. Robert R. Redfield, faced unprecedented challenges
from President Trump, who repeatedly wished away the pandemic. His
efforts to seize the spotlight from the public health agency reflected
the broader patterns of his erratic presidency: public condemnations on
Twitter, a tendency to dismiss findings from scientists, inconsistent
policy or decision-making and a suspicion that the ``deep state'' inside
the government is working to force him out of office.

Mr. Trump and his top aides have grown increasingly bitter about
perceived leaks from the C.D.C. they say were designed to embarrass the
president and to build support for decisions that ignore broader
concerns about the country's vast social and economic dislocation. At
the same time, some at the C.D.C. have bristled at what they see as
pressure to bend evidence-based recommendations to help Mr. Trump's
political standing.

Located in Atlanta, the C.D.C. is encharged with protecting the nation
against public health threats --- from anthrax to obesity --- and
serving as the unassailable source of information about fighting them.
Given its record and resources, the agency might have become the
undisputed leader in the global fight against the virus.

Instead, the C.D.C. made missteps that undermined America's response.

``Here is an agency that has been waiting its entire existence for this
moment,'' said Dr. Peter Lurie, a former associate commissioner at the
Food and Drug Administration who for years worked closely with the
C.D.C. ``And then they flub it. It is very sad. That is what they were
set up to do.''

The agency's allies say it is just one part of a vast network of state
and local health departments, hospitals, government agencies and
suppliers that were collectively unprepared for the speed, scope and
ferocity of the pandemic. They also point out that lawmakers have long
failed to adequately prioritize funding for the kind of crisis the
country now faces.

Dr. Amy Ray, an infectious disease specialist in Cleveland, said the
C.D.C. did not ``get enough credit,'' adding, ``They are learning at the
same time the world is learning, by watching how this disease
manifests.''

The agency, which declined repeated requests for interviews with its top
officials, said in a statement: ``C.D.C. is at the table as part of the
larger U.S. government response, providing the best, most current data
and scientific understanding we have.''

``It's important to remember that this is a global emergency --- and
it's impacting the entire U.S.,'' the agency said. ``That means it
requires an all-of-government response.''

Image

President Trump made a visit in March to the C.D.C. in Atlanta, speaking
with the agency's director, Dr. Robert R. Redfield, right.Credit...T.J.
Kirkpatrick for The New York Times

\hypertarget{not-our-culture-to-intervene}{%
\subsection{`Not Our Culture to
Intervene'}\label{not-our-culture-to-intervene}}

In early March, Dr. Redfield
\href{https://www.nytimes3xbfgragh.onion/2020/03/06/us/politics/trump-coronavirus-cdc.html}{led
Mr. Trump} on a V.I.P. tour of the high-tech labs at the C.D.C.'s
Atlanta headquarters, standing off to the side as the president spoke.

Wearing a red ``Keep America Great'' cap, Mr. Trump
\href{https://www.whitehouse.gov/briefings-statements/remarks-president-trump-tour-centers-disease-control-prevention-atlanta-ga/}{falsely
asserted} that ``anybody that wants a test can get a test,'' claimed he
had a ``natural ability'' for science and noted that he might hold
campaign rallies even as the virus spread.

``Thank you for your decisive leadership in helping us, you know, put
public health first,'' Dr. Redfield told the president as they posed for
the cameras.

The moment underscored the challenge for the director and his agency. To
combat the virus, he would have to **** manage the mercurial demands of
the president who appointed him and the expectations of the career
scientists he leads.

The sensibilities could not be more different. At one point that month,
administration officials asked the agency to provide feedback on
possible logos --- including ``Make America Healthy Again'' --- for
cloth face masks they
\href{https://www.axios.com/hanes-face-masks-white-house-a09a360f-4d52-4c08-baee-4a85e1f6562f.html}{hoped
to distribute} to millions of Americans. The plan fell through, but not
before C.D.C. leaders agreed to the request, according to one person
familiar with the discussions.

White House aides saw Dr. Redfield, 68, as an ally, but as the
coronavirus crisis intensified, his meandering manner in television
appearances and congressional hearings irritated a president drawn to
big personalities and assertive defenders of his administration.

\href{https://www.hhs.gov/about/leadership/robert-redfield/index.html}{A
former military virologist} who specialized in H.I.V., Dr. Redfield was
Mr. Trump's second choice after his first C.D.C. director
\href{https://www.nytimes3xbfgragh.onion/2018/01/31/health/cdc-brenda-fitzgerald-resigns.html}{resigned}.
He had no experience leading a government agency --- though he had been
considered for jobs in previous Republican administrations --- and often
told associates that he was happiest
\href{http://www.ihv.org/media/SOM/Microsites/IHV/documents/Discovery-Newsletter/Discovery_Winter_2010.pdf}{treating
patients} in Africa or Haiti.

Dr. Robert C. Gallo, who founded the
\href{http://www.ihv.org/about/About-Dr-Robert-C-Gallo/}{Institute of
Human Virology} at the University of Maryland School of Medicine with
Dr. Redfield in 1996, said he had warned him against taking the C.D.C.
post, describing it as ``massive public health, lots of politics, lots
of pressure.''

Image

Dr. Redfield was a virologist focused on H.I.V. before taking the top
job at the C.D.C.Credit...Anna Moneymaker/The New York Times

While praising his friend as ``a terrific, dedicated infectious disease
doctor,'' Dr. Gallo, who also co-founded the
\href{https://gvn.org/}{Global Virus Network}, said in an interview that
Dr. Redfield ``can't do anything communication-wise.'' He added, ``He's
reticent, never wanting the front of anything --- maybe it's extreme
humility.''

The C.D.C., established in the 1940s
\href{https://www.cdc.gov/malaria/about/history/history_cdc.html}{to
control malaria}in the South, has the feel of an academic institution.
There, experts work ``at the speed of science --- you take time doing
it,'' said Dr. Georges C. Benjamin, executive director of the American
Public Health Association.

\hypertarget{latest-updates-the-coronavirus-outbreak}{%
\section{\texorpdfstring{\href{https://www.nytimes3xbfgragh.onion/2020/08/20/world/coronavirus-covid.html?action=click\&pgtype=Article\&state=default\&region=MAIN_CONTENT_1\&context=storylines_live_updates}{Latest
Updates: The Coronavirus
Outbreak}}{Latest Updates: The Coronavirus Outbreak}}\label{latest-updates-the-coronavirus-outbreak}}

Updated 2020-08-21T07:02:52.040Z

\begin{itemize}
\tightlist
\item
  \href{https://www.nytimes3xbfgragh.onion/2020/08/20/world/coronavirus-covid.html?action=click\&pgtype=Article\&state=default\&region=MAIN_CONTENT_1\&context=storylines_live_updates\#link-68774d88}{Shutdowns,
  warnings and scoldings follow alarming incidents on college campuses.}
\item
  \href{https://www.nytimes3xbfgragh.onion/2020/08/20/world/coronavirus-covid.html?action=click\&pgtype=Article\&state=default\&region=MAIN_CONTENT_1\&context=storylines_live_updates\#link-26b58724}{Biden
  knocks Trump's pandemic response, and outlines a national strategy.}
\item
  \href{https://www.nytimes3xbfgragh.onion/2020/08/20/world/coronavirus-covid.html?action=click\&pgtype=Article\&state=default\&region=MAIN_CONTENT_1\&context=storylines_live_updates\#link-4e542da3}{U.S.
  health agencies announce moves to confront the flu season and
  plummeting child vaccination rates.}
\end{itemize}

\href{https://www.nytimes3xbfgragh.onion/2020/08/20/world/coronavirus-covid.html?action=click\&pgtype=Article\&state=default\&region=MAIN_CONTENT_1\&context=storylines_live_updates}{See
more updates}

More live coverage:
\href{https://www.nytimes3xbfgragh.onion/live/2020/08/20/business/stock-market-today-coronavirus?action=click\&pgtype=Article\&state=default\&region=MAIN_CONTENT_1\&context=storylines_live_updates}{Markets}

The agency, a division of the Department of Health and Human Services
with 11,000 employees, cannot make policy, but it guides federal and
state public health systems and advises government leaders.

The C.D.C.'s most fabled experts are the disease detectives of its
\href{https://www.cdc.gov/eis/index.html}{Epidemic Intelligence
Service}, rapid responders who investigate outbreaks. But more broadly,
according to current and former employees and others who worked closely
with the agency, the C.D.C. is risk-averse, perfectionist and ill suited
to improvising in a quickly evolving crisis --- particularly one that
shuts down the country and paralyzes the economy.

``It's not our culture to intervene,'' said Dr. George Schmid, who
worked at the agency off and on for nearly four decades. He described it
as increasingly bureaucratic, weighed down by ``indescribable,
burdensome hierarchy.''

The exacting culture shaped its scientists' ambitions; it also locked
some into a fixed way of thinking, former officials said. And it helped
produce the C.D.C.'s most consequential failure in the crisis: its
inability early on to provide state laboratories around the country with
an effective diagnostic test.

The C.D.C. quickly developed a successful test in January designed to be
highly precise, but it was more complicated to use and turned out to be
no better than versions produced overseas. And in manufacturing test
kits to send to the states, the C.D.C.
\href{https://www.nytimes3xbfgragh.onion/2020/04/18/health/cdc-coronavirus-lab-contamination-testing.html}{contaminated
many of them} through sloppy lab practices. That, along with the
\href{https://www.nytimes3xbfgragh.onion/2020/07/18/us/politics/trump-coronavirus-response-failure-leadership.html}{administration's
failure} to quickly ramp up commercial and academic labs, delayed the
rollout of tests and limited their availability for months.

In late January, the agency sent epidemiologists to Seattle to help
local health officials learn whether what was then the country's first
known patient --- a
\href{https://www.nejm.org/doi/full/10.1056/NEJMoa2001191}{35-year-old
man} who had visited Wuhan, China --- had infected others.

Image

A drive-through testing center in Virginia in March. Delayed testing
hindered the U.S.'s ability to curb the pandemic.Credit...Erin
Schaff/The New York Times

\href{https://www.documentcloud.org/documents/6933936-2020-01-23-SEA-TAC-CASE-Emails-from-the-Start-of.html}{After
an initial round of tests}, the agency imposed restrictive testing
standards. When doctors in Washington State and elsewhere forwarded the
names of about 650 people in January who might have been infected ---
they had contact with a confirmed patient, had been admitted to a
hospital or had other risk factors --- the C.D.C. agreed to test only
256. That group consisted ** primarily of people traveling from Wuhan
and their contacts.

In part because of capacity issues, the agency typically did not
recommend testing people without symptoms --- even though Chinese
doctors were reporting that people could spread the virus without ever
feeling ill. Dr. Redfield mentioned the possibility of asymptomatic
spread in a CNN interview in February, but the C.D.C. did not emphasize
such transmission until late March.

In mid-February, C.D.C. officials announced plans for a national
surveillance effort --- by testing samples from people with flulike
symptoms --- to determine whether the virus was spreading undetected.
The effort was to begin in Seattle, New York and three other cities, but
after disagreements over how to proceed, it did not
\href{https://www.documentcloud.org/documents/6933935-CDC-and-State-Health-Officials-Debate-Start-up.html}{start}.

Later that month, public health officials across the country were
increasingly concerned about visitors streaming into the United States
from South Korea, Japan, Italy and other European countries engulfed by
the virus.

On phone calls with the C.D.C., worried state officials kept asking:
``Are there plans to expand the travel monitoring?'' The response,
according to a participant from New York, was always the same: ``We're
still actively considering that.''

Mr. Trump announced a European travel ban on March 11, a few days after
meeting with Dr. Redfield and others. But it was too late. **** Genomic
tracing
\href{https://www.nytimes3xbfgragh.onion/2020/04/08/science/new-york-coronavirus-cases-europe-genomes.html}{would
later show} that European travelers had brought the virus into New York
as early as mid-February; it multiplied there and elsewhere in the
country. In Seattle, a strain from China had struck nursing homes in
late February.

Image

Health workers with a nursing home patient in Seattle during the
outbreak there in February.Credit...Grant Hindsley for The New York
Times

``\href{https://www.documentcloud.org/documents/6933935-CDC-and-State-Health-Officials-Debate-Start-up.html\#document/p31/a565892}{If
we were able to test early}, we would have recognized earlier'' the
scale of the outbreak, said Dr. Jeffrey Duchin, the chief health officer
in King County, Wash. ``We would have been able to put prevention
measures in place earlier and had fewer cases.''

Part of the C.D.C.'s start-up troubles, current and former employees
said, was that the group in charge of the response initially ---
\href{https://www.cdc.gov/ncird/dvd.html}{the Division of Viral
Diseases} --- is smaller and has far less staff focused on contagious
respiratory diseases than the C.D.C.'s
\href{https://www.cdc.gov/ncird/flu.html}{Influenza Division}, which
eventually took a more leading role. ``They were very quickly
overwhelmed by what they had to do,'' said Dr. Pierre Rollin, a
virologist who left last year.

Now,
\href{https://www.documentcloud.org/documents/6933940-CDC-STAFFING-INFO.html}{more
than 3,000 C.D.C. employees} are aiding the coronavirus response,
analyzing data, performing lab work and deploying to cities where local
health departments need help. **** While other federal agencies are also
involved --- including the F.D.A., which has speeded the use of antibody
tests; the Federal Emergency Management Agency, which has worked to get
ventilators and other supplies; and the National Institutes of Health,
which has studied vaccines and possible treatments --- the C.D.C. is the
reigning expert.

Even before the current crisis, Dr. Redfield had kept a low profile.
Some days he could be spotted in a corner of the cafeteria, sipping
coffee alone.

Although he is on the White House coronavirus task force, Dr. Redfield
found himself eclipsed by Dr. Anthony S. Fauci, the nation's most famous
infectious disease specialist, and Dr. Deborah Birx, an AIDS expert and
former C.D.C. physician.

Meanwhile, his bonds with some of his own staff have frayed. One
associate recounted him saying that the agency's scientists had a
``myopic'' view of their roles, and characterized his relationship with
his top deputy, Dr. Anne Schuchat, a career C.D.C. scientist deeply
respected in the agency, as growing strained.

He has not been in Atlanta recently, shuttling instead between his home
in Baltimore and the West Wing. One person familiar with his thinking
described Dr. Redfield as feeling ``a little bit on an island.''

The C.D.C. still has many defenders who say it has done the best it
could battling a stealthy, previously unknown virus. ``When they do
release something, it does what C.D.C. ought to do --- retain the voice
of credibility,'' said Dr. James A. Town, medical director of the
intensive care unit at Harborview Medical Center in Seattle. ``Even if
it's coming at a slower pace, which can be frustrating, I think they're
pretty thoughtful and trying to make even-keeled investigations.''

Dr. Redfield declined to comment for this article. But in a recent
interview with The Hill, he said, ``I would say C.D.C. has never been
stronger.''

In a briefing last week, he acknowledged that the nation must work to
improve its systems to track disease outbreaks, though he disputed that
the agency was somehow unable to detect when the coronavirus started to
spread in the United States. ``We were never really blind to the
introduction of this virus,'' he said.

Image

The C.D.C.'s headquarters, removed from Washington in Atlanta, has the
feel of an academic institution.Credit...Audra Melton for The New York
Times

\hypertarget{the-data-pipeline}{%
\subsection{The Data Pipeline}\label{the-data-pipeline}}

Inside Building 21, the C.D.C.'s gleaming 12-story headquarters, nothing
has been more critical than getting fast, accurate information on how
the virus is spreading, who is getting sick, how best to treat them and
how quickly the country can reopen.

But that has proved difficult for the agency's antiquated data systems,
many of which rely on information assembled by or shared with local
health officials through phone calls, faxes and thousands of
spreadsheets attached to emails. The data is not integrated,
comprehensive or robust enough, with some exceptions, to depend on in
real time.

The C.D.C. could not produce accurate counts of how many people were
being tested, compile complete demographic information on confirmed
cases or even keep timely tallies of deaths.

The result is an agency that had blind spots at just the wrong moment,
limited in its ability to gather and process information about the
pathogen or share it with those who needed it most: front-line medical
workers, government health officials and policymakers.

``That specific, granular data has huge implications,'' said Julie
Fischer, a professor of microbiology at Georgetown University who
studies community preparedness for emerging diseases. ``We lost precious
time in decision-making and putting public health resources to use.''

When C.D.C. officials urged states to track travelers from China in
February for possible infection, the agency turned to a computer network
called Epi-X. It sent emails to state officials,
\href{https://www.documentcloud.org/documents/6933966-2020-02-Nevada-EpiX-Flight-Arrival-Emails-From-a.html}{one
at a time}, for each arriving flight so they could download a list of
targeted passengers.

In California, state health officers received as many as 146
notification emails a day, forcing them to spend time forwarding them to
the appropriate local health departments. In some cases, the
information, collected for the C.D.C. by the Department of Homeland
Security, listed incorrect dates or times; in other cases, passenger
data was sent to the wrong state or came more than a week after the
travelers had entered the United States.

``We got crappy data,'' said Fran Phillips, Maryland's deputy health
secretary. ``We would call them up and people would say, `Well, I was in
China, but that was three years ago.'''

On Feb. 11, Mr. Whitley, Nevada's top health official, complained to Dr.
Redfield
\href{https://www.documentcloud.org/documents/6933944-2020-02-11-NEVADA-LETTER-to-CDC-DIRECTOR.html}{in
a letter} about ``the breakdown'' in ``communication the states have
received from the C.D.C.'' The agency had said three travelers from
China could ``go along with their normal day-to-day business'' ---
advice that conflicted with the C.D.C.'s message to monitor such
passengers and make sure they were in self-quarantine.

One week later, the C.D.C.'s Epi-X system stopped sending notices
entirely, even though flights kept coming. The agency had temporarily
shut the system down to
``\href{https://www.documentcloud.org/documents/6933945-2020-02-28-CDC-to-States-Re-REVISIONS-to-EPI-X.html}{improve
data quality,}'' it told state officials in an email.

The travel-monitoring program screened at least 268,000 passengers
through mid-April. A
\href{https://www.cdc.gov/mmwr/volumes/69/wr/mm6919e4.htm}{C.D.C.
report} cited 14 Covid cases that were traced back to those passengers,
but lapses and errors in the data made that tally far from conclusive.
The agency went on to say that the program did not stop the disease from
being introduced to California, where incomplete information, high
travel volume and the possibility of asymptomatic spread made it
ineffective.

Once coronavirus cases started developing in earnest in the United
States in March, federal and state officials began demanding information
to make key decisions. Among them: where to move ventilators from the
national stockpile and where to build temporary hospitals.

State and local officials were quickly overwhelmed trying to document
hospitals' needs. Staff at the Los Angeles County Public Health
Department, for example, called each of the 94 county hospitals in the
early weeks of the outbreak, asking nurses how many coronavirus patients
were in intensive care units and how many were on ventilators.

The C.D.C. tried to repurpose one of its data systems to collect the
information directly from hospitals, but it had significant gaps.
Finally, the Department of Health and Human Services in April also
enlisted a private contractor,
\href{https://www.nytimes3xbfgragh.onion/2020/08/14/us/politics/teletracking-technologies-coronavirus-senators.html}{TeleTracking
Technologies}, only to have hospitals struggle to log on to the system.

Hospital executives resorted **** to finding aid themselves. Scott
Malaney, head of Blanchard Valley Health System in Ohio, got a phone
call from an official at a Michigan health care system that was running
short on beds and equipment. It was asking neighboring facilities to
share supplies or take in overflow patients if necessary.

``She said they were looking up the phone book up and down Highway 75 to
see if there were other places that could help,'' Mr. Malaney recalled.

The disconnects in the public health record-keeping system delayed
sharing critical data that could help patients, said Dr. Thomas
Inglesby, director of the Center for Health Security at the Johns
Hopkins Bloomberg School of Public Health.

Hospitals look to the C.D.C. for that information. ``Is it higher risk
to be a healthy person at age 75 with coronavirus or a diabetic with the
disease at age 45?'' Dr. Inglesby said. ``We should have the data to
know the answer to this question quickly, and we should be using it to
make better decisions.''

Image

Health workers in Brooklyn transferred deceased patients to a
refrigerated truck in April as the virus battered
hospitals.Credit...Victor J. Blue for The New York Times

As the number of suspected cases --- and deaths --- mounted, the C.D.C.
struggled to record them accurately. The agency rushed to
\href{https://www.documentcloud.org/documents/6933941-2020-02-13-CDC-Hiring-Staff-to-Go-Through-Filings.html}{hire
extra workers} to process incoming emails from hospitals. Still, many
officials turned to Johns Hopkins University, which became the primary
source for \href{https://coronavirus.jhu.edu/us-map}{up-to-date counts}.
Even the White House cited its numbers instead of the C.D.C.'s lagging
tallies.

Some staff members were mortified when
\href{https://www.seattletimes.com/seattle-news/education/qa-avi-schiffmann-the-washington-state-teen-behind-a-coronavirus-website-with-millions-of-views/}{a
Seattle teenager} managed to compile coronavirus data faster than the
agency itself, creating a website that attracted millions of daily
visitors. ``If a high schooler can do it, someone at C.D.C. should be
able to do it,'' said one longtime employee.

For years, federal and state governments have not invested enough money
to insure that the nation's public health system would have critical
data needed to respond in a pandemic. Since 2010, for example, grants to
help hospitals and states prepare for emergencies have declined.

In 2019, more than 100 public health groups
\href{https://www.documentcloud.org/documents/6933952-2019-03-25-1-Billion-Over-10-Years-Needed-for.html}{pressed
congressional leaders} to allocate \$1 billion over a decade to upgrade
the infrastructure. The C.D.C. received \$50 million toward the effort
this year. Then, as coronavirus cases and deaths mounted in March, the
federal government committed to \$500 million under the emergency CARES
Act.

``The crisis has highlighted the need to continue efforts to modernize
the public health data systems that C.D.C. and states rely on,'' Dr.
Redfield told a Senate committee on May 12. ``Timely and accurate data
are essential as C.D.C. and the nation work to understand the impact of
Covid-19 on all Americans.''

Image

Projected deaths at a presentation by the White House task force in
March.Credit...Erin Schaff/The New York Times

Data is one of the essential tools of public health; Mr. Trump, though,
often appears to see it as a weapon against him. He has suggested that
\href{https://www.politico.com/news/2020/05/14/trump-coronavirus-testing-high-case-numbers-259524}{testing
is ``overrated''} and that it makes the United States look bad by
increasing the number of confirmed cases. **** He has seized on
lower-end projections of the virus's toll, only to see them eclipsed as
the cases and deaths rose.

Recently, the C.D.C. drew criticism after
\href{https://www.nytimes3xbfgragh.onion/2020/05/22/us/politics/coronavirus-tests-cdc.html}{media
reports disclosed} that in tracking how many Americans had been tested,
the agency had breached standard practice by combining data from
antibody tests, which can indicate past infections, with diagnostic
tests. The agency said it was caused by confusion in overworked state
and local health officials reporting results, but the mistake muddied
the picture of the pandemic.

``The scientists at the C.D.C. are still great,'' Dr. Jha said. ``It's
very puzzling to all of us why C.D.C. performance has been so poor.''

Image

Vice President Mike Pence and Mr. Trump at a White House briefing in
March advocating ``15 days to slow the spread.''Credit...Doug Mills/The
New York Times

\hypertarget{a-strained-relationship}{%
\subsection{A Strained Relationship}\label{a-strained-relationship}}

Late in the evening on March 15, the C.D.C. put a bold
\href{https://web.archive.org/web/20200316115537/https://www.cdc.gov/coronavirus/2019-ncov/community/large-events/mass-gatherings-ready-for-covid-19.html}{statement
on its website}: All gatherings of more than 50 people should be
canceled, the agency said, effectively calling for an end to large
public events.

Inside the West Wing, the president's top aides were stunned. Meeting in
the Situation Room, the coronavirus task force was just putting the
finishing touches on
\href{https://www.whitehouse.gov/briefings-statements/remarks-president-trump-vice-president-pence-members-coronavirus-task-force-press-briefing-3/}{its
own guidance}. It limited gatherings to no more than 10 people --- a
fact that C.D.C. officials, including Dr. Redfield, knew from
participating in days of debate on the issue.

\href{https://www.nytimes3xbfgragh.onion/news-event/coronavirus?action=click\&pgtype=Article\&state=default\&region=MAIN_CONTENT_3\&context=storylines_faq}{}

\hypertarget{the-coronavirus-outbreak-}{%
\subsubsection{The Coronavirus Outbreak
›}\label{the-coronavirus-outbreak-}}

\hypertarget{frequently-asked-questions}{%
\paragraph{Frequently Asked
Questions}\label{frequently-asked-questions}}

Updated August 17, 2020

\begin{itemize}
\item ~
  \hypertarget{why-does-standing-six-feet-away-from-others-help}{%
  \paragraph{Why does standing six feet away from others
  help?}\label{why-does-standing-six-feet-away-from-others-help}}

  \begin{itemize}
  \tightlist
  \item
    The coronavirus spreads primarily through droplets from your mouth
    and nose, especially when you cough or sneeze. The C.D.C., one of
    the organizations using that measure,
    \href{https://www.nytimes3xbfgragh.onion/2020/04/14/health/coronavirus-six-feet.html?action=click\&pgtype=Article\&state=default\&region=MAIN_CONTENT_3\&context=storylines_faq}{bases
    its recommendation of six feet} on the idea that most large droplets
    that people expel when they cough or sneeze will fall to the ground
    within six feet. But six feet has never been a magic number that
    guarantees complete protection. Sneezes, for instance, can launch
    droplets a lot farther than six feet,
    \href{https://jamanetwork.com/journals/jama/fullarticle/2763852}{according
    to a recent study}. It's a rule of thumb: You should be safest
    standing six feet apart outside, especially when it's windy. But
    keep a mask on at all times, even when you think you're far enough
    apart.
  \end{itemize}
\item ~
  \hypertarget{i-have-antibodies-am-i-now-immune}{%
  \paragraph{I have antibodies. Am I now
  immune?}\label{i-have-antibodies-am-i-now-immune}}

  \begin{itemize}
  \tightlist
  \item
    As of right
    now,\href{https://www.nytimes3xbfgragh.onion/2020/07/22/health/covid-antibodies-herd-immunity.html?action=click\&pgtype=Article\&state=default\&region=MAIN_CONTENT_3\&context=storylines_faq}{that
    seems likely, for at least several months.} There have been
    frightening accounts of people suffering what seems to be a second
    bout of Covid-19. But experts say these patients may have a
    drawn-out course of infection, with the virus taking a slow toll
    weeks to months after initial exposure. People infected with the
    coronavirus typically
    \href{https://www.nature.com/articles/s41586-020-2456-9}{produce}
    immune molecules called antibodies, which are
    \href{https://www.nytimes3xbfgragh.onion/2020/05/07/health/coronavirus-antibody-prevalence.html?action=click\&pgtype=Article\&state=default\&region=MAIN_CONTENT_3\&context=storylines_faq}{protective
    proteins made in response to an
    infection}\href{https://www.nytimes3xbfgragh.onion/2020/05/07/health/coronavirus-antibody-prevalence.html?action=click\&pgtype=Article\&state=default\&region=MAIN_CONTENT_3\&context=storylines_faq}{.
    These antibodies may} last in the body
    \href{https://www.nature.com/articles/s41591-020-0965-6}{only two to
    three months}, which may seem worrisome, but that's perfectly normal
    after an acute infection subsides, said Dr. Michael Mina, an
    immunologist at Harvard University. It may be possible to get the
    coronavirus again, but it's highly unlikely that it would be
    possible in a short window of time from initial infection or make
    people sicker the second time.
  \end{itemize}
\item ~
  \hypertarget{im-a-small-business-owner-can-i-get-relief}{%
  \paragraph{I'm a small-business owner. Can I get
  relief?}\label{im-a-small-business-owner-can-i-get-relief}}

  \begin{itemize}
  \tightlist
  \item
    The
    \href{https://www.nytimes3xbfgragh.onion/article/small-business-loans-stimulus-grants-freelancers-coronavirus.html?action=click\&pgtype=Article\&state=default\&region=MAIN_CONTENT_3\&context=storylines_faq}{stimulus
    bills enacted in March} offer help for the millions of American
    small businesses. Those eligible for aid are businesses and
    nonprofit organizations with fewer than 500 workers, including sole
    proprietorships, independent contractors and freelancers. Some
    larger companies in some industries are also eligible. The help
    being offered, which is being managed by the Small Business
    Administration, includes the Paycheck Protection Program and the
    Economic Injury Disaster Loan program. But lots of folks have
    \href{https://www.nytimes3xbfgragh.onion/interactive/2020/05/07/business/small-business-loans-coronavirus.html?action=click\&pgtype=Article\&state=default\&region=MAIN_CONTENT_3\&context=storylines_faq}{not
    yet seen payouts.} Even those who have received help are confused:
    The rules are draconian, and some are stuck sitting on
    \href{https://www.nytimes3xbfgragh.onion/2020/05/02/business/economy/loans-coronavirus-small-business.html?action=click\&pgtype=Article\&state=default\&region=MAIN_CONTENT_3\&context=storylines_faq}{money
    they don't know how to use.} Many small-business owners are getting
    less than they expected or
    \href{https://www.nytimes3xbfgragh.onion/2020/06/10/business/Small-business-loans-ppp.html?action=click\&pgtype=Article\&state=default\&region=MAIN_CONTENT_3\&context=storylines_faq}{not
    hearing anything at all.}
  \end{itemize}
\item ~
  \hypertarget{what-are-my-rights-if-i-am-worried-about-going-back-to-work}{%
  \paragraph{What are my rights if I am worried about going back to
  work?}\label{what-are-my-rights-if-i-am-worried-about-going-back-to-work}}

  \begin{itemize}
  \tightlist
  \item
    Employers have to provide
    \href{https://www.osha.gov/SLTC/covid-19/standards.html}{a safe
    workplace} with policies that protect everyone equally.
    \href{https://www.nytimes3xbfgragh.onion/article/coronavirus-money-unemployment.html?action=click\&pgtype=Article\&state=default\&region=MAIN_CONTENT_3\&context=storylines_faq}{And
    if one of your co-workers tests positive for the coronavirus, the
    C.D.C.} has said that
    \href{https://www.cdc.gov/coronavirus/2019-ncov/community/guidance-business-response.html}{employers
    should tell their employees} -\/- without giving you the sick
    employee's name -\/- that they may have been exposed to the virus.
  \end{itemize}
\item ~
  \hypertarget{what-is-school-going-to-look-like-in-september}{%
  \paragraph{What is school going to look like in
  September?}\label{what-is-school-going-to-look-like-in-september}}

  \begin{itemize}
  \tightlist
  \item
    It is unlikely that many schools will return to a normal schedule
    this fall, requiring the grind of
    \href{https://www.nytimes3xbfgragh.onion/2020/06/05/us/coronavirus-education-lost-learning.html?action=click\&pgtype=Article\&state=default\&region=MAIN_CONTENT_3\&context=storylines_faq}{online
    learning},
    \href{https://www.nytimes3xbfgragh.onion/2020/05/29/us/coronavirus-child-care-centers.html?action=click\&pgtype=Article\&state=default\&region=MAIN_CONTENT_3\&context=storylines_faq}{makeshift
    child care} and
    \href{https://www.nytimes3xbfgragh.onion/2020/06/03/business/economy/coronavirus-working-women.html?action=click\&pgtype=Article\&state=default\&region=MAIN_CONTENT_3\&context=storylines_faq}{stunted
    workdays} to continue. California's two largest public school
    districts --- Los Angeles and San Diego --- said on July 13, that
    \href{https://www.nytimes3xbfgragh.onion/2020/07/13/us/lausd-san-diego-school-reopening.html?action=click\&pgtype=Article\&state=default\&region=MAIN_CONTENT_3\&context=storylines_faq}{instruction
    will be remote-only in the fall}, citing concerns that surging
    coronavirus infections in their areas pose too dire a risk for
    students and teachers. Together, the two districts enroll some
    825,000 students. They are the largest in the country so far to
    abandon plans for even a partial physical return to classrooms when
    they reopen in August. For other districts, the solution won't be an
    all-or-nothing approach.
    \href{https://bioethics.jhu.edu/research-and-outreach/projects/eschool-initiative/school-policy-tracker/}{Many
    systems}, including the nation's largest, New York City, are
    devising
    \href{https://www.nytimes3xbfgragh.onion/2020/06/26/us/coronavirus-schools-reopen-fall.html?action=click\&pgtype=Article\&state=default\&region=MAIN_CONTENT_3\&context=storylines_faq}{hybrid
    plans} that involve spending some days in classrooms and other days
    online. There's no national policy on this yet, so check with your
    municipal school system regularly to see what is happening in your
    community.
  \end{itemize}
\end{itemize}

Reporters soon were peppering the White House with questions about
whether it was overruling the C.D.C. Some of Mr. Trump's aides shrugged
it off as a miscommunication. But others viewed it as the C.D.C.
insisting it knew best.

The episode underscored the strained relationship **** between the
health agency and the White House. Veteran officials at the C.D.C. were
not unfamiliar with the ways of Washington. But they had never dealt
with a president like Mr. Trump or a White House like his.

Already under siege for problems with the agency's diagnostic test,
C.D.C. officials watched with growing alarm as Mr. Trump, facing
criticism for his administration's response, repeatedly undermined the
agency.

Though the task force was occasionally ahead of the C.D.C. in its
cautions to the public, Mr. Trump and his aides often expressed
extraordinary skepticism about the coronavirus and the steps required to
combat it. He said the virus would disappear
``\href{https://www.whitehouse.gov/briefings-statements/remarks-president-trump-meeting-african-american-leaders/}{like
a miracle}'' even as C.D.C. scientists described it as a real threat.
When the C.D.C. urged Americans to wear masks, he said,
``\href{https://www.nytimes3xbfgragh.onion/2020/04/09/us/politics/melania-trump-coronavirus.html}{I
don't see it for myself}.''

And when Dr. Redfield
\href{https://www.washingtonpost.com/health/2020/04/21/coronavirus-secondwave-cdcdirector/}{told
The Washington Post}that a second wave of the virus could be ``even more
difficult'' than the first, Mr. Trump insisted that he publicly claim to
have been misquoted during a White House briefing. Dr. Redfield, with
the president standing next to him, scowling, said he had been
misunderstood.

At one point, Mr. Trump even
\href{https://twitter.com/realDonaldTrump/status/1238410044263333894}{complained
about the agency} to his 80 million Twitter followers, saying, ``For
decades the \href{https://twitter.com/CDCgov}{@CDCgov} looked at, and
studied, its testing system, but did nothing about it.''

``There comes a time,'' said Dr. Jeffrey Koplan, who served as C.D.C.
director in the Clinton and Bush administrations, ``when it makes it
very hard to operate effectively, when things are being suggested,
requested, ordered that you think are contrary to the containment of the
pandemic.''

The president and his aides viewed the civil servants at the C.D.C. ---
many of whom had worked under presidents from both parties --- as
disloyal liberals eager to wound Mr. Trump politically by leaking to the
press. In private, some senior administration officials began referring
to agency scientists as members of the ``deep state,'' according to
several people who participated in the conversations but requested
anonymity to discuss the meetings.

As the crisis deepened, tensions between Washington and Atlanta
increased.

Image

Dr. Nancy Messonnier, a leader in the C.D.C.'s fight against the virus,
was sidelined after issuing a stark warning.Credit...Amanda
Voisard/Reuters

In late February, Dr. Nancy Messonnier, who oversees the C.D.C.'s
respiratory diseases center and had been leading the agency's emergency
response, was sidelined after she issued a
\href{https://www.cdc.gov/media/releases/2020/t0225-cdc-telebriefing-covid-19.html}{stark
public} warning that the virus would disrupt American lives. The
comments sent stocks tumbling and infuriated Mr. Trump, who had not been
told in advance. Public health officials, inside and outside the agency,
saw her forced retreat as an effort to silence the truth.

Often, the clashes have centered on the economic consequences of
shutdowns, which have forced 40 million people into unemployment,
companies into bankruptcy and fueled resentment across the country.

In early April,
\href{https://web.archive.org/web/20200409221838/https://www.cdc.gov/quarantine/cruise/index.html}{the
C.D.C. posted} an extension of its ``no sail'' order for cruise ships,
forbidding them from operating through August and warning that the ban
could become indefinite. The White House had supported the original
order, but privately objected to an indefinite ban, fearing lasting harm
to an industry that employs tens of thousands of people.

The posting quickly came down, replaced by an order ending the ban in
July. ``Those things aren't helpful,'' Dr. Redfield would tell his
colleagues when disputes between the C.D.C. and the task force erupted.

The White House was soon put on the defensive when
\href{https://www.usatoday.com/story/travel/cruises/2020/04/13/coronavirus-cruise-ships-saw-red-flags-amid-chaotic-federal-response/2937001001/}{USA
Today} cited internal emails about the pressure. ``Sorry to do this, but
the Office of the Vice President has instructed us to pull the No Sail
Order Extension from the website immediately,'' the paper quoted a
C.D.C. official as writing to agency colleagues.

Image

The Grand Princess, a cruise ship that docked in San Francisco after the
virus sickened passengers and crew members.Credit...John G Mabanglo/EPA,
via Shutterstock

To the president's aides, one of the most frustrating moments came on
May 1, when Dr. Schuchat published one of the agency's regular reports
on morbidity and mortality without giving the White House any notice,
according to two of Mr. Trump's advisers.

Written in dry, scientific language, the report offered
\href{https://www.cdc.gov/mmwr/volumes/69/wr/mm6918e2.htm?s_cid=mm6918e2_w}{a
blunt assessment} of the virus's spread, showing how travel from Europe
and mass gatherings had accelerated it. Dr. Schuchat went further when
interviewed for an
\href{https://apnews.com/a758f05f337736e93dd0c280deff9b10}{Associated
Press} article --- ``Health Official Says U.S. Missed Some Chances to
Slow Virus'' --- **** saying that ``taking action earlier could have
delayed further amplification.''

As the president pushed governors to
``\href{https://twitter.com/realDonaldTrump/status/1251169217531056130}{liberate}''
their states from virus lockdowns, top C.D.C. officials in April
delivered a draft of new guidance full of caveats about lifting the
restrictions. In it, the agency urged schools, churches, child care
centers, day camps, restaurants and bars to take numerous precautions
and move slowly.

Trump aides were furious when they saw the draft. To them, it was more
evidence that the C.D.C. refused to consider political, economic and
social effects in weighing how and when to reopen the country. The
agency's recommendations for houses of worship particularly annoyed some
aides, who resisted the advice that churches stop giving communion.

When the White House sat on the draft guidance for weeks, a copy was
leaked.

While the C.D.C. delayed posting the draft guidance that would allow
churches to reopen, Mr. Trump all but ordered it to do so. During a
visit to Michigan on May 21, the president --- who the next day would
explain, ``In America, we need more prayer, not less'' --- made it clear
the C.D.C. no longer had any choice.

``I said, `You better put it out,''' Mr. Trump
\href{https://www.nytimes3xbfgragh.onion/aponline/2020/05/21/health/ap-us-med-virus-outbreak-churches-reopening.html}{told
reporters}. ``And they're doing it.''

A suggestion in the guidance that houses of worship ``consider
suspending'' the use of choirs and congregant singing because it ``may
contribute to transmission'' was removed. Two federal officials said it
had not been cleared by the White House.

Lawrence Gostin, the director of a legal center at the World Health
Organization, and a former C.D.C. official, chided the White House for
exerting undue pressure on the C.D.C. throughout the crisis.

``Public health is politics. But this is different,'' he said. ``It's
criticizing its public health agencies in public. It's rejecting
guidelines it puts out. It tells them you can't even put guidelines
out.''

``I would expect the C.D.C. to coordinate with the White House,'' he
added. ``But this is not team work. This is not coordination. This is
confrontation.''

Image

Visitors to the board walk of Coney Island in Brooklyn found many
businesses closed over Memorial Day weekend, including the
neighborhood's amusement parks.~Credit...Todd Heisler/The New York Times

\hypertarget{wheres-the-guidance}{%
\subsection{Where's the Guidance?}\label{wheres-the-guidance}}

As the battle against the coronavirus stretches into summer and the
United States lurches toward restarting its economy, the mayor of Miami
Beach wants to know what to do if Covid-19 cases explode after the
city's famous beaches open again.

Doctors and nurses remain desperate for updates on how to protect
themselves. School superintendents and college presidents need to decide
how to hold classes in the fall. And employers want advice about
\href{https://www.nytimes3xbfgragh.onion/2020/05/22/business/employers-coronavirus-testing.html}{whether
to test all of their workers} before returning to business as usual.

The C.D.C. is where they expect to get answers. As the national
clearinghouse for critical public health information, it has dual
missions: to provide medical guidance to health workers while offering
easy-to-understand information for political leaders, business
executives and the general public.

But many say the agency has struggled at times to provide clear and
timely guidance.

At Margaret Mary Community Hospital in rural Batesville, Ind., doctors
and nurses got sick after following C.D.C. guidance in mid-March that
masks were necessary only when treating patients with respiratory
symptoms or fever. The first patients who tested positive for Covid-19
there instead showed up with headaches, fatigue, nausea and diarrhea.

``This virus made it halfway around the world without us having a
heads-up to our providers that this is how the disease can present,''
said Tim Putnam, the hospital's chief executive. ``Over two months after
the disease surfaced, I would have expected better.''

Front-line doctors and nurses have long relied on the agency for advice
on clinical best practices, and many said in interviews that they were
satisfied with the C.D.C.'s advisories, especially given the novelty of
the coronavirus.

The agency has issued
\href{https://www.cdc.gov/coronavirus/2019-ncov/communication/guidance-list.html?Sort=Date\%3A\%3Adesc}{114
advisory documents} for disaster and homeless shelters, retirement
communities, taxis, pediatric clinics and other venues. ``We have issued
countless guidance and recommendations based on the best available
science and data,'' an agency press officer said. Its experts have also
held about a dozen
\href{https://emergency.cdc.gov/coca/calls/2020/index.asp}{calls for
clinicians} about caring for Covid patients, and other calls for medical
groups.

But in interviews with medical practitioners across the country, many
said they now look elsewhere for detailed recommendations about how to
safely care for infected patients, posing questions about the new virus
on mailing lists or scouring online research articles.

In a crisis, one of the C.D.C.'s main roles is to explain its guidance
and reasoning, provide a rationale for when its thinking changes and
acknowledge what it does not know. The agency's routine in past
emergencies was to hold press briefings almost daily; Dr. Thomas
Frieden, Dr. Redfield's predecessor, was highly visible during the Ebola
and Zika crises. But in this case, medical workers and the public were
left to make sense of often-opaque postings on the C.D.C.'s website
after ​its leadership stopped holding regular briefings on March 9.

``Right now, they only have the PDFs that are out there, without any
kind of a conversation,'' said Dr. Jennifer Nuzzo, an epidemiologist at
Johns Hopkins. ``That is a real shortcoming.''

Medical specialty and public health organizations have sometimes taken
it on themselves to identify and highlight updates for their members.

``It would be awesome if C.D.C. could actually announce significant
changes rather than bury it on their website and assume it is done,''
Jim Collins, Michigan's director of communicable diseases, complained to
his colleagues in an email on Jan. 31.

Image

A Michigan health official complained about unclear changes in C.D.C.
guidance.Credit...

The C.D.C., some medical workers complain, has provided limited guidance
on how children transmit the virus, when to ventilate patients and how
to prioritize use of isolation rooms. And it took until April 27 for the
agency to expand its list of possible symptoms to include more than a
dozen signs of illness that some medical specialty societies had
reported weeks earlier.

To many anxious doctors and nurses, some of the C.D.C.'s clinical
guidance often seemed driven by the nationwide shortages of personal
protective equipment, not the best interests of health care workers.

Initially, the C.D.C. recommended that all doctors and nurses coming in
contact with coronavirus patients wear N95 respirators, which filter out
95 percent of all airborne particles.
\href{https://www.washingtonpost.com/health/2020/03/10/face-mask-shortage-prompts-cdc-loosen-coronavirus-guidance/}{But
on March 10}, with supplies dwindling, the C.D.C. announced that less
protective surgical masks were ``an acceptable alternative'' except
during procedures that might aerosolize the virus.
\href{https://www.latimes.com/politics/story/2020-03-21/coronavirus-mask-bandanna-covid-19-bandanna}{Days
later}, the agency
\href{https://www.cdc.gov/coronavirus/2019-ncov/hcp/ppe-strategy/face-masks.html}{said}
health workers could even wear ``homemade masks (e.g., bandanna, scarf)
for care of patients with COVID-19 as a last resort.''

``Mistrust crept in,'' said Lori Freeman, chief executive of the
National Association of County and City Health Officials. ```Are we
really being protected?'''

The relaxed guidance on protective equipment matched advice from the
World Health Organization on surgical masks. But the C.D.C. did not
highlight that fact in its update and gave no public explanation other
than acknowledging the worsening shortages. An analysis published this
week suggests that N95 and other respirator masks are superior to
surgical or cloth masks in
\href{https://www.nytimes3xbfgragh.onion/2020/06/01/health/masks-surgical-N95-coronavirus.html}{protecting
medical workers} against the virus.

Leaders of schools, businesses and other organizations also said they
were confused by the C.D.C.'s advice, which sometimes conflicted with
that of the White House coronavirus task force.

In one such instance on March 16, the White House urged limiting
gatherings to no more than 10 people and ``schooling from home whenever
possible'' for at least the next 15 days. But days earlier, the C.D.C.
had
\href{https://www.cdc.gov/coronavirus/2019-ncov/community/schools-childcare/guidance-for-schools.html}{recommended}that
schools close only if someone in the building tested positive or there
was evidence of ``substantial community transmission.''

On March 17, nearly 2,500 superintendents from around the country were
hoping to get some clarity during an
\href{https://www.aasa.org/content.aspx?id=44539}{online seminar} with
the C.D.C. Why was the C.D.C. recommending most schools could remain
open?

But just 40 minutes before the seminar was to start, the C.D.C.
\href{https://www.documentcloud.org/documents/6933965-2020-03-17-Superintendents-Briefing.html}{canceled
it} without explanation and never rescheduled. The agency
\href{https://www.npr.org/2020/03/18/817606520/coronavirus-closes-schools-its-unclear-when-students-will-return}{later
told reporters} it had decided ``to fully adapt to the new guidance from
White House'' before addressing the superintendents.

In Miami Beach, densely packed with tourists, older residents and
service workers, Mayor Dan Gelber dreads the prospect of new outbreaks.
While he appreciated the reopening guidance that the C.D.C. published
recently, Mr. Gelber, a Democrat, said he wished the agency would also
lay out specific steps to follow if cases surge again.

Image

Workers at the International Taphouse in St. Louis last month after the
city began a partial reopening of businesses.Credit...Whitney Curtis for
The New York Times

``It's almost as if they just said, `Open up and figure out whether it's
a good idea or not afterward,''' he said of the C.D.C. ``We don't have a
net here.''

Noah Weiland contributed reporting.

Advertisement

\protect\hyperlink{after-bottom}{Continue reading the main story}

\hypertarget{site-index}{%
\subsection{Site Index}\label{site-index}}

\hypertarget{site-information-navigation}{%
\subsection{Site Information
Navigation}\label{site-information-navigation}}

\begin{itemize}
\tightlist
\item
  \href{https://help.nytimes3xbfgragh.onion/hc/en-us/articles/115014792127-Copyright-notice}{©~2020~The
  New York Times Company}
\end{itemize}

\begin{itemize}
\tightlist
\item
  \href{https://www.nytco.com/}{NYTCo}
\item
  \href{https://help.nytimes3xbfgragh.onion/hc/en-us/articles/115015385887-Contact-Us}{Contact
  Us}
\item
  \href{https://www.nytco.com/careers/}{Work with us}
\item
  \href{https://nytmediakit.com/}{Advertise}
\item
  \href{http://www.tbrandstudio.com/}{T Brand Studio}
\item
  \href{https://www.nytimes3xbfgragh.onion/privacy/cookie-policy\#how-do-i-manage-trackers}{Your
  Ad Choices}
\item
  \href{https://www.nytimes3xbfgragh.onion/privacy}{Privacy}
\item
  \href{https://help.nytimes3xbfgragh.onion/hc/en-us/articles/115014893428-Terms-of-service}{Terms
  of Service}
\item
  \href{https://help.nytimes3xbfgragh.onion/hc/en-us/articles/115014893968-Terms-of-sale}{Terms
  of Sale}
\item
  \href{https://spiderbites.nytimes3xbfgragh.onion}{Site Map}
\item
  \href{https://help.nytimes3xbfgragh.onion/hc/en-us}{Help}
\item
  \href{https://www.nytimes3xbfgragh.onion/subscription?campaignId=37WXW}{Subscriptions}
\end{itemize}
