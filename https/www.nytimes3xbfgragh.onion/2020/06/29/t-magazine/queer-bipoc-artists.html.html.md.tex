Sections

SEARCH

\protect\hyperlink{site-content}{Skip to
content}\protect\hyperlink{site-index}{Skip to site index}

\href{https://myaccount.nytimes3xbfgragh.onion/auth/login?response_type=cookie\&client_id=vi}{}

\href{https://www.nytimes3xbfgragh.onion/section/todayspaper}{Today's
Paper}

Works for the Now, by Queer Artists of Color

\url{https://nyti.ms/2ZzAWF5}

\begin{itemize}
\item
\item
\item
\item
\item
\end{itemize}

Advertisement

\protect\hyperlink{after-top}{Continue reading the main story}

Supported by

\protect\hyperlink{after-sponsor}{Continue reading the main story}

\hypertarget{works-for-the-now-by-queer-artists-of-color}{%
\section{Works for the Now, by Queer Artists of
Color}\label{works-for-the-now-by-queer-artists-of-color}}

Pride Month may have come to a close, but the wide-ranging pieces shown
here have staying power.

Published June 29, 2020Updated July 4, 2020

\begin{itemize}
\item
\item
\item
\item
\item
\end{itemize}

As the country contends with ongoing violence against queer and BIPOC
communities, it's paramount that voices from those communities are
heard. Not all artists are activists, of course, but they are all keen
observers, ones who invite the viewer to consider their way of seeing
things, whether their chosen subject is as expansive as prison reform or
as singular as their own sense of self. Each work tells a story, and
here, we've asked 15 queer artists of color to elaborate on theirs.
(Look for a coming compilation of works by queer Indigenous artists in
the weeks ahead.)

\emph{These interviews have been edited and condensed.}

\includegraphics{https://static01.graylady3jvrrxbe.onion/images/2020/06/23/t-magazine/art/viewfinder-slide-MH2B/viewfinder-slide-MH2B-articleLarge.jpg?quality=75\&auto=webp\&disable=upscale}

\hypertarget{by-jennifer-packer-36-based-in-new-york}{%
\subsubsection{\texorpdfstring{\textbf{By}
\textbf{\href{https://www.sikkemajenkinsco.com/jennifer-packer}{Jennifer
Packer}, 36, based in New
York}}{By Jennifer Packer, 36, based in New York}}\label{by-jennifer-packer-36-based-in-new-york}}

In his essay ``Fifth Avenue, Uptown: A Letter From Harlem'' (1960),
\href{https://www.nytimes3xbfgragh.onion/2019/09/05/t-magazine/james-baldwin-giovannis-room.html}{James
Baldwin} writes, ``Anyone who has ever struggled with poverty knows how
extremely expensive it is to be poor; and if one is a member of a
captive population, economically speaking, one's feet have simply been
placed on the treadmill forever. One is victimized, economically, in a
thousand ways.'' I think about emotional and moral buoyancy in the face
of various kinds of impoverishment and de facto captivity. To be
bankrupt does not mean that one is alone or without dignity or without
meaningful personal iconography, loving sanctuary or self-respect.

Image

A photo of rafa esparza's 2019 performance ``bust: indestructible
columns.''Credit...Courtesy of the artist and Commonwealth and Council,
Los Angeles. Photo: Natalia Mantini

Image

Another photo of esparza's ``bust: indestructible columns''
(2019).Credit...Courtesy of the artist and Commonwealth and Council, Los
Angeles. Photo: Natalia Mantini

\hypertarget{by-rafa-esparza-38-based-in-los-angeles}{%
\subsubsection{\texorpdfstring{\textbf{By}
\textbf{\href{https://commonwealthandcouncil.com/us/rafa-esparza/biography}{rafa
esparza}, 38, based in Los
Angeles}}{By rafa esparza, 38, based in Los Angeles}}\label{by-rafa-esparza-38-based-in-los-angeles}}

``bust: indestructible columns'' drew on an earlier performance I did in
2015, for which I ensconced myself in a concrete pillar outside of the
Men's Central Jail in downtown Los Angeles and, once the concrete had
dried, freed myself using a hand-chisel and a hammer. The work was a
comment on police violence and the sort of surveillance endured by black
and brown bodies. I intentionally did it in a public place, where I
might have an encounter with the police and where people could bear
witness and be more than a passive audience. Last year, I, along with
\href{https://performancespacenewyork.org/}{Performance Space} and
\href{https://www.ballroommarfa.org/}{Ballroom Marfa} and a number of
individual collaborators, brought a version of the piece to the Ellipse
in Washington, D.C. In this case, I was thinking about the racist
rhetoric of this administration, and the column was a replica of one of
those lining the porticoes at the White House. Afterward, there was a
dinner with a feast prepared by Gerardo Gonzalez and readings of
writers' reflections on the day. The evening evolved into a dance party
and was a reminder of, given the risks and laboriousness of creating
culture and speaking truth to power as queer folks, how important it is
for us to have spaces dedicated to care and joy. Later this week, a new
project that I organized with the artist Cassils will launch --- I can't
say too much, but it involves 80 different artists from all walks of
life coming together to appeal for the abolishment of immigrant
detention.

Image

A detail of Shen Wei's ``Self-portrait (Mochi)''
(2019).Credit...Courtesy of the artist and Flowers Gallery, London, New
York, Hong Kong

Image

A detail of Wei's ``Self-portrait (Origami)'' (2019) .Credit...Courtesy
of the artist and Flowers Gallery, London, New York, Hong Kong

\hypertarget{by-shen-wei-43-based-in-new-york}{%
\subsubsection{\texorpdfstring{\textbf{By}
\textbf{\href{http://shenwei.studio/}{Shen Wei}, 43, based in New
York}}{By Shen Wei, 43, based in New York}}\label{by-shen-wei-43-based-in-new-york}}

Recently, I completed my decade-long self-portrait series ``I Miss You
Already,'' many of the images from which were photographed during my
extensive travels around the world; the series was a diary and a
travelogue for capturing meaningful moments. One of the first things I
do when I walk into any hotel or guest room is set up my camera on a
tripod. That way, when the moment is right, the camera is there and
ready, though often it stands there the entire trip without me taking a
single picture. ``Self-portrait (Mochi)'' (2019) was photographed in a
friend's apartment in Paris. When I'm abroad, I must always stop by an
Asian grocery store to satisfy my longing for homey food. I'm from
Shanghai, where sticky rice is a staple, and felt very nostalgic eating
a piece of mochi cake by a Parisian window shadowed with painterly
bamboo leaves. ``Self-portrait (Origami)'' (2019) was photographed in
Puebla, Mexico, in the morning hour at the dining table in my hotel
room. Whenever I see paper napkins, my natural reaction is to fold them
origami style, an art form I grew up learning. Something about doing a
familiar thing in a foreign environment comforts me. At the end of the
shoot, I folded a fan, a swan and an elephant. It was a perfectly
introverted moment --- I was alone but not lonely.

Image

Sable Elyse Smith's ``Riot I'' (2019).Credit...Courtesy of the artist,
JTT, New York, and Carlos/Ishikawa, London

\hypertarget{by-sable-elyse-smith-33-based-in-new-york}{%
\subsubsection{\texorpdfstring{\textbf{By}
\textbf{\href{http://sableelysesmith.com/}{Sable Elyse Smith}, 33, based
in New
York}}{By Sable Elyse Smith, 33, based in New York}}\label{by-sable-elyse-smith-33-based-in-new-york}}

``Riot I'' (2019) is from an ongoing series of sculptures, and that fact
is important --- the series is as much the work as any of the individual
pieces within it. We are dealing in ---~we are trafficking in ---
systems here. The initial impetus for this series was the site of the
prison visiting room, and the works themselves are altered one-to-one
scale replicas of the furniture within those rooms. This is about touch
and its foreclosure, intimacy and its foreclosure, as well as
performance and commodification and exploitation. It's about economics
both micro and macro --- \emph{defund the {[}expletive{]} police}.
Basically, these structures are emblematic of a system and
infrastructure, a labor and actual people, actual lives. The bodies of
``free'' and ``unfree'' are meant to be affixed to this furniture for
the duration of a visit. And there is so, so, so much more I could say
about all of this. Maybe ``Riot I'' is an epic poem.

Image

Jonathan Lyndon Chase's ``warm seat'' (2019).~Credit...Courtesy of the
artist and Company Gallery, New York

\hypertarget{by-jonathan-lyndon-chase-30-based-in-philadelphia}{%
\subsubsection{\texorpdfstring{\textbf{By}
\textbf{\href{https://www.jonathanlyndonchase.com/}{Jonathan Lyndon
Chase}, 30, based in
Philadelphia}}{By Jonathan Lyndon Chase, 30, based in Philadelphia}}\label{by-jonathan-lyndon-chase-30-based-in-philadelphia}}

This work is based on memories shared with my husband, William. We were
cleaning out the closet, going through mostly clothes and shoe boxes. It
was cold that winter and our hearts and minds were cluttered with
different emotions caused by the outside world that had come into the
interior of our home and bodily homes. Our bedroom was like an icebox,
but having our bodies close together kept us warm as we unpacked that
evening. I wrote the following poem to accompany the image:

\begin{quote}
Kiss some friction on my rubber band\\
Sweet burning lick\\
His nails crush through and through waves\\
Why does his forehead feel like an icy wet dinner plate
\end{quote}

Image

Devan Shimoyama's ``Grandmother's Blessing'' (2019).Credit...Courtesy of
the artist

Image

Shimoyama's ``The Abduction of Ganymede'' (2019).Credit...Courtesy of
the artist

\hypertarget{by-devan-shimoyama-30-based-in-pittsburgh}{%
\subsubsection{\texorpdfstring{\textbf{By}
\textbf{\href{https://www.devanshimoyama.com/}{Devan Shimoyama}, 30,
based in
Pittsburgh}}{By Devan Shimoyama, 30, based in Pittsburgh}}\label{by-devan-shimoyama-30-based-in-pittsburgh}}

I tend to use unconventional materials, specifically in my paintings,
and often borrow from drag culture and the glamorous black women I've
known. I'm thinking about the spaces where we celebrate identity and
construct different fantasies on top of our bodies --- there's a sort of
peacocking associated with wearing one's Sunday best, for instance.
``Grandmother's Blessing'' (2019) shows my own grandmother, with pieces
of costume jewelry affixed to the canvas for her eyes and a blouse made
of a beaded brocade. Whenever I move into a new place, she makes a point
of coming out to visit and bless it, which sort of completes the space
for us both. ``The Abduction of Ganymede'' leans more toward fantasy, a
driving force in my work that allows me to create a brighter alternate
reality. In the Greek myth, Zeus either sends his eagle or turns into an
eagle and then abducts the most beautiful boy. My take is about
self-love, about being able to embrace the narrative that you, with your
black body, are beautiful.

Image

Christina Quarles's ``Oh Dear, Look Whut We've Dun to tha Blues''
(2020).Credit...Courtesy of the artist and Pilar Corrias Gallery, London

Image

Quarles's ``Pull on Thru tha Nite'' (2017).Credit...Courtesy of the
artist and Pilar Corrias Gallery, London

\hypertarget{by-christina-quarles-35-based-in-los-angeles}{%
\subsubsection{\texorpdfstring{\textbf{By}
\textbf{\href{https://www.christinaquarles.com/}{Christina Quarles}, 35,
based in Los
Angeles}}{By Christina Quarles, 35, based in Los Angeles}}\label{by-christina-quarles-35-based-in-los-angeles}}

``Oh Dear, Look Whut We've Dun to tha Blues'' (2020) was one of the last
paintings I made before the United States entered quarantine. I was
feeling rising anxiety all around me, this sense that we were all going
through the motions of business as usual amid a spreading virus. This
doom bubbling just below the veneer of a calm, often decadent facade is
one I have also felt with regard to the urgency of issues like global
warming, racial injustice and class inequality, to name a few. Perhaps
this is what led to me painting one of the figures with a pattern that I
see as oscillating between expensive pink marble terrazzo and a stinky
cheap slice of pimento loaf. ``Pull on Thru tha Nite'' (2017) is titled
for the visual pun of the figure pulling down the night sky, a gesture
that points to the mutability of context and our agency to move and even
play within the confines of fixed meaning. In all my work, I am
interested in the ways in which meaning is derived from an ever-shifting
and always-constructed edge, and how within this boundary we are
confined and contorted but also held and supported. We are subject to
but can also be activated \emph{by} these limits, a paradox outlined by
the writer \href{http://www.joshuagamson.com/}{Joshua Gamson}: Fixed
categories of identity can be used to marginalize but, paradoxically,
can be used by the marginalized to gain visibility and political power.

Image

Guanyu Xu's ``Lighting Up'' (2018).Credit...Courtesy of the artist and
Yancey Richardson

Image

Xu's ``Inside of My Drawer'' (2019).Credit...Courtesy of the artist and
Yancey Richardson

\hypertarget{by-guanyu-xu-27-based-in-chicago}{%
\subsubsection{\texorpdfstring{\textbf{By}
\textbf{\href{https://www.xuguanyu.com/}{Guanyu Xu}, 27, based in
Chicago}}{By Guanyu Xu, 27, based in Chicago}}\label{by-guanyu-xu-27-based-in-chicago}}

``Lighting Up'' (2018) is from my ``One Land to Another'' project, which
is partly a response to the lack of Asian representation in mainstream
culture. I take on different characters and create these cinematic
images --- they're almost like film stills. In 2018 and 2019, I visited
Beijing and installed these images, temporarily, in my childhood home.
My parents don't know I'm gay, and it's an environment where that sort
of expression is not encouraged, so it was a rebellious act, a way of
disrupting the heterosexual structure that's embedded in the home and
familial relations. But it was also an exploration of intersectionality
and the fact that I'm not completely free there or here. In the United
States, I'm a gay person and a foreigner. I'm very cognizant of how
systems of oppression are used as a means of control, and I'm looking to
bridge dialogues about both countries, which in certain ways are quite
similar.

Image

Darryl DeAngelo Terrell's ``I Look Like My Momma (Self-portrait 1980)''
(2019).Credit...Courtesy of the artist

Image

Terrell's ``\#Project20s (Don, 22, Chicago, 2017).''~Credit...Courtesy
of the artist

\hypertarget{by-darryl-deangelo-terrell-29-based-in-detroit}{%
\subsubsection{\texorpdfstring{\textbf{By}
\textbf{\href{http://darryldterrell.com/}{Darryl DeAngelo Terrell}, 29,
based in
Detroit}}{By Darryl DeAngelo Terrell, 29, based in Detroit}}\label{by-darryl-deangelo-terrell-29-based-in-detroit}}

Growing up in Detroit, I never felt I could perform any gender other
than male. But around 2015, during my first year of grad school, I
started experimenting with gender performance and expression with my
alter-ego, Dion, as a way of exploring fat, femme, queer, black bodies.
``I Look Like My Momma (Self-portrait 1980),'' is referencing two images
---
\href{https://www.nytimes3xbfgragh.onion/topic/person/robert-mapplethorpe}{Robert
Mapplethorpe}'s ``Self Portrait, 1980'' and
\href{https://www.nytimes3xbfgragh.onion/2019/02/28/nyregion/harlem-renaissance-james-van-der-zee.html}{James
Van Der Zee}'s ``Couple in Raccoon Coats, 1932.'' I borrowed my mom's
raccoon-and-fox fur coat and gold necklace for it and took the photo in
a peacock wicker chair reminiscent of those I used to see in our family
photos. I sent the image to my mom and she said, ``Damn, boy, you look
good. You look like me.''

``\#Project20s'' started in 2017, when I was doing an independent study
at Ox-bow and around a lot of white people listening to
\href{https://www.nytimes3xbfgragh.onion/topic/person/taylor-swift}{Taylor
Swift}, which was irritating. I, meanwhile, became fixated by the fact
that both
\href{https://www.nytimes3xbfgragh.onion/2017/03/01/t-magazine/beck-tom-waits-kendrick-lamar.html}{Kendrick
Lamar} and
\href{https://www.nytimes3xbfgragh.onion/2015/04/10/t-magazine/kanye-west-adidas-yeezy-fashion-interview.html}{Kanye
West} have songs about the likelihood of black kids not making it to a
certain age in their 20s, while also thinking about how heavily
gentrification was hitting both Chicago and Detroit. My aim is to
photograph upward of 200 black or brown people in their 20s by the time
I turn 30 --- Don, the young person pictured in the second image, was an
art student who died in an ambulance after the EMTs took an unhurried
approach to his asthma attack. I want for this series to live in museums
or gallery spaces where it will confront the people privileged with
leisure time. Gentrification is a form of racial violence, and my
thinking is, ``If you kick us out of our hood, I put us on your white
walls.''

Image

Nina Chanel Abney's ``Issa Saturday Study'' (2019).~Credit...Courtesy of
the artist

\hypertarget{by-nina-chanel-abney-38-based-in-jersey-city-nj}{%
\subsubsection{\texorpdfstring{\textbf{By}
\textbf{\href{https://ninachanel.com/}{Nina Chanel Abney}, 38, based in
Jersey City,
N.J.}}{By Nina Chanel Abney, 38, based in Jersey City, N.J.}}\label{by-nina-chanel-abney-38-based-in-jersey-city-nj}}

``Issa Saturday Study'' (2019) is essentially a study for another
painting I made titled ``Issa Saturday'' for
\href{https://www.norton.org/exhibitions/nina-chanel-abney}{my 2019
exhibition} ``Neon'' at the Norton Museum of Art in West Palm Beach,
Fla. The painting serves as an investigation of how and why we as a
society choose to celebrate the Fourth of July when black people are
still not free from the oppressions of this country.

Image

D'Angelo Lovell Williams's ``Was Blind, But Now I See (Granny)''
(2018).Credit...Courtesy of the artist and Higher Pictures/Janice Guy
Gallery

\hypertarget{by-dangelo-lovell-williams-27-based-in-new-york}{%
\subsubsection{\texorpdfstring{\textbf{By}
\textbf{\href{https://www.dangelolovellwilliams.com/}{D'Angelo Lovell
Williams}, 27, based in New
York}}{By D'Angelo Lovell Williams, 27, based in New York}}\label{by-dangelo-lovell-williams-27-based-in-new-york}}

I treat images like portals. My current exhibition, ``Papa Don't
Preach,'' (open by appointment at \href{https://janiceguy.com/}{Janice
Guy}'s space, in collaboration with
\href{https://higherpictures.com/}{Higher Pictures}, at 520 West 143rd
Street in New York) looks at intersections of blackness, queerness and
family. This image from the exhibition, ``Was Blind, But Now I See
(Granny)'' (2018), narrates a moment between my mom's mom and me. The
title, a line from the song ``Amazing Grace,'' refers to an experience
where my Granny was dead for 10 minutes after going into cardiac arrest
over a decade ago due to smoking cigarettes. She was resuscitated but
was blind, couldn't speak and couldn't walk. It was a sort of rebirth
for her, especially religiously. Over the last decade, my granny has,
with the help of family and friends, slowly regained her speech, her
ability to walk and her sight. When she started speaking again, she
recalled seeing God during the 10 minutes she was dead. The gesture of
her hand over my face is a way of her sharing that testimony with me
through touch. The use of the gaze is prominent in my work, and here,
Granny's gaze is mine and hers at once, even if we don't always agree.
With images existing as these portals, I continue to seek how life as a
black, queer, H.I.V.-positive person intertwines with the depths of
kinship, intimacy and history.

Image

Troy Michie's ``Distorted In the Interest of Design''
(2019).Credit...Courtesy of the artist

Image

Michie's ``Divided Territory'' (2019).Credit...Courtesy of the artist

\hypertarget{by-troy-michie-35-based-in-new-york}{%
\subsubsection{\texorpdfstring{\textbf{By}
\textbf{\href{https://www.troymichie.com/}{Troy Michie}, 35, based in
New
York}}{By Troy Michie, 35, based in New York}}\label{by-troy-michie-35-based-in-new-york}}

I created these works last summer for Frieze London, but they are
extensions of a project I started in 2016, when I delved into the
history of zoot suits and pachuco culture as they existed in my hometown
of El Paso, Texas. The town also has Fort Bliss, the second largest Army
post in the country, so I grew up seeing camouflage everyday, and
collage was a way of unpacking both of these histories while exploring
questions of race, gender and sexuality. As part of my research, I
revisited
\href{https://www.nytimes3xbfgragh.onion/topic/person/ralph-ellison}{Ralph
Ellison}'s ``Invisible Man'' (1952), which describes the zoot suiters as
men who existed outside of their time and also has the line, ``When they
approach me they see only my surroundings.'' ``Distorted In the Interest
of Design'' (2019) looks at what it means to be invisible but also
hyper-visible, which is the reality of being black or brown in this
country. For the series, I mine an archive of '70s-era erotic magazines
with fetishistic images of black men photographed through the gaze of
white photographers. Using the centerfold, I draw clothing onto the nude
models and then cut and weave the magazine pages back together. The
process is additive and subtractive and, above all, a type of
photographic disruption. In ``Divided Territory'' (2019), I affixed
actual pieces of clothing. There are shirt collars, pockets, waistbands
and cuffs --- places where the body connects.

Image

Nikita Gale's ``RECOMMENDATION'' (2018-19).~Credit...Courtesy of the
artist and Commonwealth and Council, Los Angeles. Photo: Ruben Diaz

\hypertarget{by-nikita-gale-36-based-in-los-angeles}{%
\subsubsection{\texorpdfstring{\textbf{By}
\textbf{\href{https://www.nikitagale.com/}{Nikita Gale}, 36, based in
Los
Angeles}}{By Nikita Gale, 36, based in Los Angeles}}\label{by-nikita-gale-36-based-in-los-angeles}}

``RECOMMENDATION'' (2018-19) is a fabricated steel barricade sculpture.
In thinking about the infrastructure of crowd control, I became
interested in the ubiquity of barricades at protests and other large
public gatherings like concerts and political rallies. Barricades have
origins in a very radical material tradition, having been made out of
refuse by the working classes in 19th-century France to block and
redirect the flow of street traffic as a means of protecting themselves
against state violence. These structures also served as social spaces
and ad hoc stages for these citizen insurgents to address one another.
Through the advent of mass production, barricades have become a mobile
architecture that controls how crowds and audiences are allowed to take
up space; they are no longer technologies of the people but technologies
of authority, and the freedom to speak and to listen is negated by the
physical control rendered by the barricades' presence.
``RECOMMENDATION'' considers exclusion and protection, radical
expression and the regulation of speech and listening.

Image

David Alekhuogie's ``roscoe's long beach 34.0407° N, 118.3476° W''
(2018).Credit...Courtesy of the artist and Commonwealth and Council, Los
Angeles. Photo: Ruben Diaz

\hypertarget{by-david-alekhuogie-33-based-in-los-angeles}{%
\subsubsection{\texorpdfstring{\textbf{By}
\textbf{\href{https://www.davidalekhuogie.com/}{David Alekhuogie}, 33,
based in Los
Angeles}}{By David Alekhuogie, 33, based in Los Angeles}}\label{by-david-alekhuogie-33-based-in-los-angeles}}

Though my work draws from my own experiences and emotions, my practice
is fundamentally political. My 2018 ``Pull-Up'' series explores sagging
--- the style of wearing low-hanging pants that was popularized by
hip-hop. The closely cropped compositions abstract the body into
landscape, a political arena where people express their agendas and
fantasies regardless of whether they mean harm or not. The photographs
are often crafted in the studio, borrowing from the language and symbols
of commercial advertising and alluding to the omnipresent
commodification of the assumed bodies. The identities of the models are
hidden, which draws upon the viewer's fantasies. With works like
``roscoe's long beach 34.0407° N, 118.3476° W'' (2018), I rephotographed
the studio shots of the waist area outside, conflating the bodily
landscape with the urban landscape and light of Los Angeles, the city
where I was born and raised, and where young black men have lived and
died.

Image

Elliott Jerome Brown Jr.'s ``Oftentimes, justice for black people takes
the form of forgiveness, allowing them space to reclaim their bodies
from wrongs made against them.'' (2018).Credit...Courtesy of the artist
and Nicelle Beauchene Gallery, New York

Image

Brown's ``Prune and grout'' (2019).Credit...Courtesy of the artist and
Nicelle Beauchene Gallery, New York

\hypertarget{by-elliott-jerome-brown-jr-26-based-in-new-york}{%
\subsubsection{\texorpdfstring{\textbf{By}
\textbf{\href{https://elliottjeromebrownjr.com/}{Elliott Jerome Brown
Jr.}, 26, based in New
York}}{By Elliott Jerome Brown Jr., 26, based in New York}}\label{by-elliott-jerome-brown-jr-26-based-in-new-york}}

The top image is from one of the rare occasions when I was invited to
make photographs in a documentary context. I attended the funeral of a
person who had a storied life, despite having suffered an incredibly
traumatic racial violence early on. This was the only image I felt
comfortable sharing with a wider audience; the guests' identities aren't
disclosed, and yet it communicates why I was there and serves as a way
of paying my respects. I noticed that those who had been closest to this
person moved through the day with ease --- mostly they seemed proud and
at peace --- and it made me think about the power the deceased had and
whether forgiveness was a tool for cultivating that power. A lot of my
work involves interiority, both of physical spaces and of individuals
--- I'm interested in what constitutes their foundation and enables them
to act. Your attacker might not repent and the state might assist in
perpetuating violence, so, in that lack, what tools do you have to
fortify yourself?

``Prune and grout'' (2019) I took last year at a New Orleans bar.
There's a companion piece to this that shows a woman with her head down
on the bar, as though she's mourning someone's absence. This image shows
the logistical setup --- the person pictured here was just helping me
with the other shot, but there was such concern in his eyes. His hand,
which is just outside the frame here, was holding that of the woman.

Advertisement

\protect\hyperlink{after-bottom}{Continue reading the main story}

\hypertarget{site-index}{%
\subsection{Site Index}\label{site-index}}

\hypertarget{site-information-navigation}{%
\subsection{Site Information
Navigation}\label{site-information-navigation}}

\begin{itemize}
\tightlist
\item
  \href{https://help.nytimes3xbfgragh.onion/hc/en-us/articles/115014792127-Copyright-notice}{©~2020~The
  New York Times Company}
\end{itemize}

\begin{itemize}
\tightlist
\item
  \href{https://www.nytco.com/}{NYTCo}
\item
  \href{https://help.nytimes3xbfgragh.onion/hc/en-us/articles/115015385887-Contact-Us}{Contact
  Us}
\item
  \href{https://www.nytco.com/careers/}{Work with us}
\item
  \href{https://nytmediakit.com/}{Advertise}
\item
  \href{http://www.tbrandstudio.com/}{T Brand Studio}
\item
  \href{https://www.nytimes3xbfgragh.onion/privacy/cookie-policy\#how-do-i-manage-trackers}{Your
  Ad Choices}
\item
  \href{https://www.nytimes3xbfgragh.onion/privacy}{Privacy}
\item
  \href{https://help.nytimes3xbfgragh.onion/hc/en-us/articles/115014893428-Terms-of-service}{Terms
  of Service}
\item
  \href{https://help.nytimes3xbfgragh.onion/hc/en-us/articles/115014893968-Terms-of-sale}{Terms
  of Sale}
\item
  \href{https://spiderbites.nytimes3xbfgragh.onion}{Site Map}
\item
  \href{https://help.nytimes3xbfgragh.onion/hc/en-us}{Help}
\item
  \href{https://www.nytimes3xbfgragh.onion/subscription?campaignId=37WXW}{Subscriptions}
\end{itemize}
