Sections

SEARCH

\protect\hyperlink{site-content}{Skip to
content}\protect\hyperlink{site-index}{Skip to site index}

\href{https://www.nytimes3xbfgragh.onion/section/world/asia}{Asia
Pacific}

\href{https://myaccount.nytimes3xbfgragh.onion/auth/login?response_type=cookie\&client_id=vi}{}

\href{https://www.nytimes3xbfgragh.onion/section/todayspaper}{Today's
Paper}

\href{/section/world/asia}{Asia Pacific}\textbar{}New Security Law Gives
China Sweeping Powers Over Hong Kong

\url{https://nyti.ms/3g9Pa5Z}

\begin{itemize}
\item
\item
\item
\item
\item
\item
\end{itemize}

Advertisement

\protect\hyperlink{after-top}{Continue reading the main story}

Supported by

\protect\hyperlink{after-sponsor}{Continue reading the main story}

\hypertarget{new-security-law-gives-china-sweeping-powers-over-hong-kong}{%
\section{New Security Law Gives China Sweeping Powers Over Hong
Kong}\label{new-security-law-gives-china-sweeping-powers-over-hong-kong}}

The law, approved in Beijing with speed and secrecy and signed off by Xi
Jinping, will tighten the Communist Party's grip on Hong Kong after last
year's protests.

\includegraphics{https://static01.graylady3jvrrxbe.onion/images/2020/06/30/world/00hk-rules-HFO1/merlin_174058374_8ab162b6-78b0-4678-b632-6773aa078b2d-articleLarge.jpg?quality=75\&auto=webp\&disable=upscale}

\href{https://www.nytimes3xbfgragh.onion/by/chris-buckley}{\includegraphics{https://static01.graylady3jvrrxbe.onion/images/2018/10/08/multimedia/author-chris-buckley/author-chris-buckley-thumbLarge.png}}\href{https://www.nytimes3xbfgragh.onion/by/keith-bradsher}{\includegraphics{https://static01.graylady3jvrrxbe.onion/images/2018/10/08/multimedia/author-keith-bradsher/author-keith-bradsher-thumbLarge.png}}\href{https://www.nytimes3xbfgragh.onion/by/tiffany-may}{\includegraphics{https://static01.graylady3jvrrxbe.onion/images/2019/12/04/reader-center/author-tiffany-may/author-tiffany-may-thumbLarge.png}}

By \href{https://www.nytimes3xbfgragh.onion/by/chris-buckley}{Chris
Buckley},
\href{https://www.nytimes3xbfgragh.onion/by/keith-bradsher}{Keith
Bradsher} and
\href{https://www.nytimes3xbfgragh.onion/by/tiffany-may}{Tiffany May}

\begin{itemize}
\item
  June 29, 2020
\item
  \begin{itemize}
  \item
  \item
  \item
  \item
  \item
  \item
  \end{itemize}
\end{itemize}

\href{https://cn.nytimes3xbfgragh.onion/china/20200630/china-hong-kong-security-law-rules/}{阅读简体中文版}\href{https://cn.nytimes3xbfgragh.onion/china/20200630/china-hong-kong-security-law-rules/}{閱讀繁體中文版}\href{https://www.nytimes3xbfgragh.onion/es/2020/06/30/espanol/mundo/hong-kong-china-leyes-seguridad.html}{Leer
en español}

China unveiled a
\href{https://www.nytimes3xbfgragh.onion/2020/07/07/business/hong-kong-security-law-tech.html}{contentious
new law for Hong Kong} late Tuesday that grants the authorities sweeping
powers to crack down on opposition to Beijing at home and abroad with
heavy prison sentences for vaguely defined political crimes.

The law's swift approval in Beijing signaled the urgency that the
Communist Party leader, Xi Jinping, has
\href{https://www.nytimes3xbfgragh.onion/2020/05/21/business/economy/coronavirus-china-economy.html}{given
to expanding his control over Hong Kong} to quash pro-democracy protests
that evolved last year into an increasingly confrontational challenge to
Chinese rule.

The Hong Kong government issued the text of the legislation at 11 p.m.
on Tuesday, after weeks of unusual secrecy surrounding the drafting of
the law in Beijing. The law took effect immediately, even though the
public was seeing it in full only for the first time.

The text provided a far-reaching blueprint for the authorities and the
courts to suppress the city's protest movement and for China's national
security apparatus to pervade many layers of Hong Kong's society.

Ambiguously worded offenses of separatism, subversion, terrorism and
collusion with foreign countries carry maximum penalties of life
imprisonment. Inducing residents to hate the government in Beijing or
Hong Kong is defined as a serious crime.

\includegraphics{https://static01.graylady3jvrrxbe.onion/images/2020/06/30/world/30hk-rules-3/merlin_172915911_859041d1-9f6d-46e0-802b-fd436a7f1803-articleLarge.jpg?quality=75\&auto=webp\&disable=upscale}

A new Committee for Safeguarding National Security will be authorized to
operate in total secrecy and be shielded from legal challenges. Its
officials will be given the task of scrutinizing schools, corporations,
nongovernmental organizations, news companies, and foreigners living in
Hong Kong and abroad.

``It's meant to suppress and oppress, and to frighten and intimidate
Hong Kongers,'' Claudia Mo, a pro-democracy lawmaker, said. ``And they
just might succeed in that.''

Other key details in the law include:

\begin{itemize}
\item
  The law takes direct aim at the antigovernment protesters' strategy of
  disruption. Last year, demonstrators
  \href{https://www.nytimes3xbfgragh.onion/2019/08/14/business/hong-kong-economy-airport-protests.html}{paralyzed
  the airport} briefly,
  \href{https://www.nytimes3xbfgragh.onion/2019/10/07/world/asia/hong-kong-protesters-masks-violence.html}{vandalized
  the subway system} and attacked police stations and surrounded
  government buildings. The law describes activities such as damaging
  government buildings and sabotaging public transportation as acts of
  subversion and terrorism, punishable with lengthy jail terms.
\item
  It allows Beijing to seize broad control in security cases, especially
  during crises. Suspects in security cases will mostly be held without
  bail. Trials involving state secrets could be closed to the media and
  the public, with few rights to trial by jury and with only the
  verdicts announced. Suspects in important cases can be sent to face
  trial in mainland China, where courts are opaque and often harsh.
\item
  The law focuses heavily on the perceived role of foreigners in Hong
  Kong's unrest. It will impose harsh penalties on anyone who urges
  foreign countries to criticize or impose sanctions on the government.
  It targets former Hong Kong residents who have acquired foreign
  passports and are outspoken against the government, empowering
  officials to freeze their assets and impose fines.
\end{itemize}

The Chinese legislature approved the law a day before July 1, the
politically charged anniversary of Hong Kong's handover to China in
1997, which regularly draws pro-democracy protests. On the anniversary
last year, a huge, peaceful demonstration gave way to violence when a
small group of activists
\href{https://www.nytimes3xbfgragh.onion/2019/07/01/world/asia/china-hong-kong-protest.html}{broke
into Hong Kong's legislature}, smashing glass walls and spray-painting
slogans on walls.

``Those who have stirred up trouble and broken this type of law in the
past will hopefully watch themselves in the future,'' Tam Yiu-chung,
Hong Kong's representative to the legislative group in China that
reviewed the law, said in a television interview. ``If they continue to
defy the law, they will bear the consequences.''

The unanimous vote on Tuesday by the National People's Congress Standing
Committee, an elite arm of China's party-controlled legislature, came
less than two weeks after the lawmakers
\href{https://www.nytimes3xbfgragh.onion/2020/06/20/world/asia/china-hong-kong-security-law.html}{first
formally considered the legislation}

Breaking from normal procedure, the committee did not release a draft of
the law for public comment. Hong Kong's activists, legal scholars and
officials were left to debate or defend the bill based on details
released by China's state news media earlier this month.

``The fact that the Chinese authorities have now passed this law without
the people of Hong Kong being able to see it tells you a lot about their
intentions,'' said Joshua Rosenzweig, the head of Amnesty
International's China team. ``Their aim is to govern Hong Kong through
fear from this point forward.''

At least two groups that have called for Hong Kong to become an
independent state said they would stop operating in the city. Such
groups remain in the minority in Hong Kong, but
\href{https://www.nytimes3xbfgragh.onion/2018/09/24/world/asia/hong-kong-party-ban-andy-chan.html}{have
drawn government scrutiny}. Activists are also worried that the law
could target those who peacefully
\href{https://www.nytimes3xbfgragh.onion/2019/08/12/world/asia/hong-kong-protests-communist-party.html}{call
for true autonomy for the territory}, as opposed to independence.

``They are doing whatever it takes to crack down on dissent and
opposition here. It's just unthinkable in the year 2020,'' said Ms. Mo,
the pro-democracy lawmaker. ``This is a huge departure from
civilization.''

Four senior members of Demosisto, a political organization in Hong Kong
that has drawn disaffected young people, announced that they were
quitting the group. They included
\href{https://www.nytimes3xbfgragh.onion/2019/10/29/world/asia/joshua-wong-hong-kong-protests.html}{Joshua
Wong, a leader of the 2014 pro-democracy demonstrations} known as the
Umbrella Movement. The group later
\href{https://www.facebookcorewwwi.onion/demosisto/posts/1464823310393153}{said
it would disband}.

``From now on,
\href{https://twitter.com/hashtag/Hongkong?ref_src=twsrc\%5Etfw\%7Ctwcamp\%5Etweetembed\%7Ctwterm\%5E1277794270393643008\%7Ctwgr\%5E\&ref_url=https\%3A\%2F\%2Fscoop.nyt.net\%2Fui\%2Foakarticle\%2F100000007213051\%2Fweb\%2FdqV1egMG0LchnIJbEkRn\&src=hashtag_click}{\#Hongkong}
enters a new era of reign of terror,'' Mr. Wong
\href{https://twitter.com/joshuawongcf/status/1277794270393643008}{wrote
on Twitter}. Announcing his decision to leave Demosisto in a post on
Facebook, he said: ``I will continue to hold fast to my home --- Hong
Kong, until they silence and obliterate me from this land.''

Administrators of chat groups used by protesters on Telegram, a popular
app, sent messages urging users not to panic but also said that they
should purge their devices of data, contacts and photos should they join
any future protests.

The chill spread even to some businesses that have
\href{https://www.nytimes3xbfgragh.onion/2020/01/19/world/asia/hong-kong-protests-yellow-blue.html}{openly
supported the democracy movement}. The Lung Mun Cafe, a well-known
Cantonese diner that provided free meals to student protesters last
year, said on Tuesday that it would no longer be affiliated with the
yellow economy, so named because of the color of umbrellas that
demonstrators once used to defend themselves against streams of tear
gas.

\begin{center}\rule{0.5\linewidth}{\linethickness}\end{center}

Image

Demonstrators supporting the security law in Hong Kong on Tuesday, some
carrying the Chinese flag.~Credit...Lam Yik Fei for The New York Times

``Lung Mun Cafe has more or less accompanied the people of Hong Kong on
this `path against tyranny,''' Cheung Chun-kit, the owner of the cafe
chain, said in a statement posted on Facebook. But he explained that he
was pulling out of the yellow economy because ``the national security
law has made me re-examine my path this year.''

The city's police force has moved quickly to stop peaceful protests
against the security law in recent days, arresting dozens of people,
including
\href{https://news.rthk.hk/rthk/en/component/k2/1534552-20200628.htm}{53
demonstrators on Sunday}. On Tuesday, a small group of protesters
gathered in a luxury mall in Central, the main downtown district, and
chanted: ``We will fight till our last breath!''

A few dozen pro-Beijing supporters wearing white shirts and blue caps
gathered in a park to celebrate the passage of the law. They celebrated
by waving large Chinese flags as they uncorked bottles of sparkling wine
and drank from plastic cups.

The police have
\href{https://news.rthk.hk/rthk/en/component/k2/1534408-20200627.htm}{denied
applications} from three groups to hold protest marches on Wednesday,
the anniversary of the handover, making it the first time the
authorities have refused to allow a demonstration on that date. Some
opposition lawmakers and democracy advocates have urged people to take
to the streets despite the ban.

``The July 1 march tomorrow will show that we will absolutely not accept
this evil national security law,'' Wu Chi-wai, a pro-democracy lawmaker,
said on Tuesday. ``Even if they try to crush us, we will use all kinds
of ways and methods to ensure that Hong Kong people's voices and
opinions can be expressed.''

Austin Ramzy, Elaine Yu and Nick Bruce contributed reporting. Claire Fu
contributed research.

Advertisement

\protect\hyperlink{after-bottom}{Continue reading the main story}

\hypertarget{site-index}{%
\subsection{Site Index}\label{site-index}}

\hypertarget{site-information-navigation}{%
\subsection{Site Information
Navigation}\label{site-information-navigation}}

\begin{itemize}
\tightlist
\item
  \href{https://help.nytimes3xbfgragh.onion/hc/en-us/articles/115014792127-Copyright-notice}{©~2020~The
  New York Times Company}
\end{itemize}

\begin{itemize}
\tightlist
\item
  \href{https://www.nytco.com/}{NYTCo}
\item
  \href{https://help.nytimes3xbfgragh.onion/hc/en-us/articles/115015385887-Contact-Us}{Contact
  Us}
\item
  \href{https://www.nytco.com/careers/}{Work with us}
\item
  \href{https://nytmediakit.com/}{Advertise}
\item
  \href{http://www.tbrandstudio.com/}{T Brand Studio}
\item
  \href{https://www.nytimes3xbfgragh.onion/privacy/cookie-policy\#how-do-i-manage-trackers}{Your
  Ad Choices}
\item
  \href{https://www.nytimes3xbfgragh.onion/privacy}{Privacy}
\item
  \href{https://help.nytimes3xbfgragh.onion/hc/en-us/articles/115014893428-Terms-of-service}{Terms
  of Service}
\item
  \href{https://help.nytimes3xbfgragh.onion/hc/en-us/articles/115014893968-Terms-of-sale}{Terms
  of Sale}
\item
  \href{https://spiderbites.nytimes3xbfgragh.onion}{Site Map}
\item
  \href{https://help.nytimes3xbfgragh.onion/hc/en-us}{Help}
\item
  \href{https://www.nytimes3xbfgragh.onion/subscription?campaignId=37WXW}{Subscriptions}
\end{itemize}
