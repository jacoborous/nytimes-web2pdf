Sections

SEARCH

\protect\hyperlink{site-content}{Skip to
content}\protect\hyperlink{site-index}{Skip to site index}

\href{https://www.nytimes3xbfgragh.onion/section/books/review}{Book
Review}

\href{https://myaccount.nytimes3xbfgragh.onion/auth/login?response_type=cookie\&client_id=vi}{}

\href{https://www.nytimes3xbfgragh.onion/section/todayspaper}{Today's
Paper}

\href{/section/books/review}{Book Review}\textbar{}Who Says Crime
Doesn't Pay? In These Novels, It Does

\url{https://nyti.ms/3fv5YnM}

\begin{itemize}
\item
\item
\item
\item
\item
\end{itemize}

Advertisement

\protect\hyperlink{after-top}{Continue reading the main story}

Supported by

\protect\hyperlink{after-sponsor}{Continue reading the main story}

\href{/column/crime}{Crime}

\hypertarget{who-says-crime-doesnt-pay-in-these-novels-it-does}{%
\section{Who Says Crime Doesn't Pay? In These Novels, It
Does}\label{who-says-crime-doesnt-pay-in-these-novels-it-does}}

\includegraphics{https://static01.graylady3jvrrxbe.onion/images/2020/06/21/books/review/21Crime/21Crime-articleLarge.jpg?quality=75\&auto=webp\&disable=upscale}

By Marilyn Stasio

\begin{itemize}
\item
  June 18, 2020
\item
  \begin{itemize}
  \item
  \item
  \item
  \item
  \item
  \end{itemize}
\end{itemize}

Dave Robicheaux is having visions again. Sometimes, James Lee Burke's
ghost-haunted Cajun detective sees Confederate soldiers in the mist.
Other times, he has apparitions of long-dead slaves pulling the oars on
a phantom galleon making its way up Bayou Teche.

In \textbf{A PRIVATE CATHEDRAL (Simon \& Schuster, 384 pp., \$28),} to
be published in August, Robicheaux confronts an unattractive spirit
named Gideon Richetti (``his skin was green'' and ``his neck looked like
it was dripping scales into his shirt''). Richetti, for reasons known
only to the living dead, is messing in the affairs of the living. Burke
describes him as some kind of time traveler, suggesting that demagogues
transcend their historical eras, discharging waves of toxicity that
survive in a continuum of space and time to infect the generations that
come after them. Somewhere, even as we speak, a baby Benito Mussolini is
being born.

``This is a haunted place, isn't it?'' observes Robicheaux's latest
lover. ``You see things here, things that aren't real, don't you?'' His
old friends take a decidedly less romantic view of the otherworldly
inhabitants of Bayou Teche. (``Don't get back in your spaceship, Robo,''
one of them advises.)

The Louisiana detective is on a mission to shake up the two leading
crime families that uneasily share the local territory. One of these mob
bosses --- either Mark Shondell or Adonis Balangie, but which one? ---
ordered the hit on two journeyman hoods whose bodies were found in the
same barrel floating in Vermilion Bay.

Like families everywhere, these powerful clans are troubled by domestic
headaches that are getting in the way of business. For one thing, two of
their children are in love, ``Romeo and Juliet'' style, and have run
away to cut a record at a famous studio in Muscle Shoals. More worrisome
to Robicheaux and his sidekick, Clete Purcel, the mobsters have been
dabbling in human trafficking.

From these ingredients, Burke has concocted his usual gumbo of thrills
and chills, stirred it with gusto and seasoned it with plenty of local
superstition and rumor. What makes these books so enduring (this is the
23rd Robicheaux novel) and the storytelling so seductive is that Burke
has the voice to do justice to the region's ancient curses and its
modern crimes.

♦

In Heather Young's \textbf{THE DISTANT DEAD (Morrow, 352 pp., \$27.99),}
Adam Merkel, the new middle school math teacher in Lovelock, Nev. ---
``a sandblasted hamlet of ranch houses, prefabs, and mobile homes strung
along a mile of Interstate 80 a hundred miles east of Reno and
seventy-five west of Winnemucca'' --- has been killed, his remains found
smoldering in a fire pit.

But who would want to murder that nice young man, who considered
mathematics ``the one true language of humanity'' and brought home-baked
pies to school on Pi Day? (That's 3/14, or March 14, for the edification
of English majors.)

This book may not be packed with high-octane thrills, but it's honestly
engaging, with its appreciation for the stark beauty of a high desert
landscape and its gentle treatment of underage outcasts like Absalom
(Sal) Prentiss, whose discovery of chess brings joy into his life.

♦

When funky books need to be written, Joe R. Lansdale writes them. He
comes through again with \textbf{MORE BETTER DEALS (Mulholland, 272 pp.,
\$27),} which will be published next month. The plot is pure James M.
Cain's ``The Postman Always Rings Twice,'' but steeped in hillbilly
noir.

Ed Edwards is a crack salesman and repo man who works at Smiling Dave's
used-car lot. He knows he looks like a used-car salesman. But his
aspiration is to look like a guy who owns a Cadillac, so he lets his
lover talk him into killing her husband.

Like Cain's illicit lovers, Ed and Nancy (a pretty woman with
``alligator'' eyes) hatch a plot to murder the husband and collect his
insurance money. The plan is promising, but the real fun is in plot
details like the High-Tone Drive-In, an outdoor movie theater with an
attached pet cemetery that figures in the story and nicely captures
Lansdale's slightly depraved wit.

♦

Jeffery Deaver has more ideas for getting people in and out of tight
spots than Carter has liver pills. \textbf{THE GOODBYE MAN (Putnam, 432
pp., \$28)} is the second book in a new series starring Colter Shaw, an
expert tracker who roams the country in his Winnebago, making a pretty
penny collecting reward money for ``finding the kidnap victim, the
imperiled runaway, the serial killer stalking the salesclerk or college
coed.''

Here, Shaw is after a \$50,000 prize that's been offered for information
leading to the arrest and conviction of two men who defaced a Baptist
church in Gig Harbor, Wash., putting the church janitor in the hospital.
A strange lead sends him undercover at the Osiris Foundation, a mountain
retreat billed as a New Age-y refuge for the depressed and bereaved that
Shaw thinks ``smells like a cult.'' But once he's infiltrated the place,
will he ever be able to leave?

Advertisement

\protect\hyperlink{after-bottom}{Continue reading the main story}

\hypertarget{site-index}{%
\subsection{Site Index}\label{site-index}}

\hypertarget{site-information-navigation}{%
\subsection{Site Information
Navigation}\label{site-information-navigation}}

\begin{itemize}
\tightlist
\item
  \href{https://help.nytimes3xbfgragh.onion/hc/en-us/articles/115014792127-Copyright-notice}{©~2020~The
  New York Times Company}
\end{itemize}

\begin{itemize}
\tightlist
\item
  \href{https://www.nytco.com/}{NYTCo}
\item
  \href{https://help.nytimes3xbfgragh.onion/hc/en-us/articles/115015385887-Contact-Us}{Contact
  Us}
\item
  \href{https://www.nytco.com/careers/}{Work with us}
\item
  \href{https://nytmediakit.com/}{Advertise}
\item
  \href{http://www.tbrandstudio.com/}{T Brand Studio}
\item
  \href{https://www.nytimes3xbfgragh.onion/privacy/cookie-policy\#how-do-i-manage-trackers}{Your
  Ad Choices}
\item
  \href{https://www.nytimes3xbfgragh.onion/privacy}{Privacy}
\item
  \href{https://help.nytimes3xbfgragh.onion/hc/en-us/articles/115014893428-Terms-of-service}{Terms
  of Service}
\item
  \href{https://help.nytimes3xbfgragh.onion/hc/en-us/articles/115014893968-Terms-of-sale}{Terms
  of Sale}
\item
  \href{https://spiderbites.nytimes3xbfgragh.onion}{Site Map}
\item
  \href{https://help.nytimes3xbfgragh.onion/hc/en-us}{Help}
\item
  \href{https://www.nytimes3xbfgragh.onion/subscription?campaignId=37WXW}{Subscriptions}
\end{itemize}
