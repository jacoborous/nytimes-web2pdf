Sections

SEARCH

\protect\hyperlink{site-content}{Skip to
content}\protect\hyperlink{site-index}{Skip to site index}

\href{https://www.nytimes3xbfgragh.onion/section/us}{U.S.}

\href{https://myaccount.nytimes3xbfgragh.onion/auth/login?response_type=cookie\&client_id=vi}{}

\href{https://www.nytimes3xbfgragh.onion/section/todayspaper}{Today's
Paper}

\href{/section/us}{U.S.}\textbar{}Trump Can't Immediately End DACA,
Supreme Court Rules

\url{https://nyti.ms/3fAP86R}

\begin{itemize}
\item
\item
\item
\item
\item
\item
\end{itemize}

Advertisement

\protect\hyperlink{after-top}{Continue reading the main story}

Supported by

\protect\hyperlink{after-sponsor}{Continue reading the main story}

\hypertarget{trump-cant-immediately-end-daca-supreme-court-rules}{%
\section{Trump Can't Immediately End DACA, Supreme Court
Rules}\label{trump-cant-immediately-end-daca-supreme-court-rules}}

The program, Deferred Action for Childhood Arrivals, protects people
brought to the United States as children by shielding them from
deportation and letting them work.

\includegraphics{https://static01.graylady3jvrrxbe.onion/images/2020/06/18/us/politics/18dc-scotus-daca-sub/18dc-scotus-daca-sub-videoSixteenByNine3000.jpg}

\href{https://www.nytimes3xbfgragh.onion/by/adam-liptak}{\includegraphics{https://static01.graylady3jvrrxbe.onion/images/2018/07/13/multimedia/author-adam-liptak/author-adam-liptak-thumbLarge-v3.png}}\href{https://www.nytimes3xbfgragh.onion/by/michael-d-shear}{\includegraphics{https://static01.graylady3jvrrxbe.onion/images/2018/06/13/multimedia/author-michael-d-shear/author-michael-d-shear-thumbLarge-v2.png}}

By \href{https://www.nytimes3xbfgragh.onion/by/adam-liptak}{Adam Liptak}
and \href{https://www.nytimes3xbfgragh.onion/by/michael-d-shear}{Michael
D. Shear}

\begin{itemize}
\item
  June 18, 2020
\item
  \begin{itemize}
  \item
  \item
  \item
  \item
  \item
  \item
  \end{itemize}
\end{itemize}

WASHINGTON --- The Supreme Court
\href{https://www.supremecourt.gov/opinions/19pdf/18-587_5ifl.pdf}{ruled
Thursday} that the Trump administration may not immediately proceed with
its
\href{https://www.nytimes3xbfgragh.onion/2017/09/05/us/politics/trump-daca-dreamers-immigration.html}{plan
to end a program} protecting about 700,000 young immigrants known as
Dreamers from deportation, dealing a surprising setback to one of
President Trump's central campaign promises.

Chief Justice John G. Roberts Jr. wrote the majority opinion, joined by
the court's four more liberal members in upholding the executive action
by President Barack Obama that established the program, Deferred Action
for Childhood Arrivals, or DACA. But the chief justice made clear that
the decision was based on procedural issues and that the Trump
administration could try to redress them.

``We do not decide whether DACA or its rescission are sound policies,''
the chief justice wrote. ``We address only whether the agency complied
with the procedural requirement that it provide a reasoned explanation
for its action.''

Still, the decision was the second this week in which the court reached
a result in a major case that elated liberals. On Monday, it ruled that
\href{https://www.nytimes3xbfgragh.onion/2020/06/15/us/gay-transgender-workers-supreme-court.html}{L.G.B.T.
workers were protected} by a landmark civil rights law. Chief Justice
Roberts was in the majority in that decision, too.

Mr. Trump responded with an angry attack on the court.

``These horrible \& politically charged decisions coming out of the
Supreme Court are shotgun blasts into the face of people that are proud
to call themselves Republicans or Conservatives,''
\href{https://twitter.com/realDonaldTrump/status/1273633632742191106}{he
wrote} on Twitter. And he made clear that he would make the composition
of the court a campaign issue, as he did in 2016.

The court's decision was provisional, and it did not remove the
uncertainty that young immigrants have lived with --- including the
possibility of being forcibly returned to countries many of them cannot
even remember --- since they arrived in the United States as children.
\href{https://www.dhs.gov/news/2012/06/15/secretary-napolitano-announces-deferred-action-process-young-people-who-are-low}{The
DACA program itself} provided only a renewable two-year deferral of
possible deportation, with no pathway to citizenship.

``Today's decision allows Dreamers to breathe a temporary sigh of
relief,'' said Stephen Yale-Loehr, a law professor at Cornell.

But young immigrants said they were surprised and delighted that the
court had ruled in their favor.

``I'm actually still shaking,'' said Joana Cabrera, who came to the
United States from the Philippines at age 9, and at 24 is on a team
researching the use of robots in coronavirus testing in San Francisco.
``I'm unbelievably happy because I was expecting the worst.''

Mr. Trump announced in September 2017 that he would wind down the
program, basing his decision on the argument that creating or
maintaining it was beyond the legal power of any president.

But the justification the government gave, Chief Justice Roberts wrote,
was insufficient or, in legal terms, ``arbitrary and capricious.'' He
said the administration may try again to provide adequate reasons.

But producing such further justifications, and resolving the legal
challenges that will inevitably follow, will take many months if not
years, pushing a final resolution of the case past this year's
elections.

How the court ruled

In Department of Homeland Security v. Regents of the University of
California, the court ruled, 5 to 4, that the Trump administration could
not immediately shut down DACA, a program that shields about 700,000
young immigrants known as Dreamers from deportation and allows them to
work.

Liberal Bloc

\includegraphics{https://static01.graylady3jvrrxbe.onion/newsgraphics/2020/06/09/scotus-key-cases/d2b681e70b06d896ef285c8211c468c460b7c04c/Sotomayor-c.png}

Sotomayor

\includegraphics{https://static01.graylady3jvrrxbe.onion/newsgraphics/2020/06/09/scotus-key-cases/d2b681e70b06d896ef285c8211c468c460b7c04c/Ginsburg-c.png}

Ginsburg

\includegraphics{https://static01.graylady3jvrrxbe.onion/newsgraphics/2020/06/09/scotus-key-cases/d2b681e70b06d896ef285c8211c468c460b7c04c/Kagan-c.png}

Kagan

\includegraphics{https://static01.graylady3jvrrxbe.onion/newsgraphics/2020/06/09/scotus-key-cases/d2b681e70b06d896ef285c8211c468c460b7c04c/Breyer-c.png}

Breyer

Conservative Bloc

\includegraphics{https://static01.graylady3jvrrxbe.onion/newsgraphics/2020/06/09/scotus-key-cases/d2b681e70b06d896ef285c8211c468c460b7c04c/Roberts-c.png}

Roberts

\includegraphics{https://static01.graylady3jvrrxbe.onion/newsgraphics/2020/06/09/scotus-key-cases/d2b681e70b06d896ef285c8211c468c460b7c04c/Kavanaugh-c.png}

Kavanaugh

\includegraphics{https://static01.graylady3jvrrxbe.onion/newsgraphics/2020/06/09/scotus-key-cases/d2b681e70b06d896ef285c8211c468c460b7c04c/Alito-c.png}

Alito

\includegraphics{https://static01.graylady3jvrrxbe.onion/newsgraphics/2020/06/09/scotus-key-cases/d2b681e70b06d896ef285c8211c468c460b7c04c/Gorsuch-c.png}

Gorsuch

\includegraphics{https://static01.graylady3jvrrxbe.onion/newsgraphics/2020/06/09/scotus-key-cases/d2b681e70b06d896ef285c8211c468c460b7c04c/Thomas-c.png}

Thomas

Where the public stands

The DACA program should \textbf{remain}

The DACA program should \textbf{be ended}

All

; 61\% 39\%

Democrats

; 85\% 15\%

Independents

; 61\% 39\%

Republicans

; 30\% 70\%

Question wording: Deferred Action for Childhood Arrivals (DACA) was
created by President Obama to protect undocumented immigrants who have
lived in the U.S. since childhood from deportation. President Trump
wants the Department of Homeland Security to end DACA. What do you
think? \textbar{} Source: SCOTUSPoll, based on an online YouGov survey
of 2,000 U.S. adults conducted April 29 to May 12.

\href{https://www.nytimes3xbfgragh.onion/interactive/2020/06/15/us/supreme-court-major-cases-2020.html}{See
how the court voted on other major cases this term}

In another of his tweets after the court announced its decision on DACA,
Mr. Trump wrote that he had doubts about whether justices were treating
him fairly. ``Do you get the impression that the Supreme Court doesn't
like me?'' \href{https://twitter.com/realDonaldTrump}{he asked} on
Twitter.

There is, at least, good reason to think that a majority of the justices
do not trust the rationales offered by his administration for major
initiatives. Last year, the same five justices
\href{https://www.nytimes3xbfgragh.onion/2019/06/27/us/politics/census-citizenship-question-supreme-court.html}{rejected
its reason} for adding a question on citizenship to the census, with
Chief Justice Roberts saying it ``appears to have been contrived.''

In a dissent in the DACA case, Justice Clarence Thomas, joined by
Justices Samuel A. Alito Jr. and Neil M. Gorsuch, said the majority had
been swayed by sympathy and politics.

``Today's decision must be recognized for what it is: an effort to avoid
a politically controversial but legally correct decision,'' Justice
Thomas wrote. ``The court could have made clear that the solution
respondents seek must come from the legislative branch.''

``It has given the green light,'' he wrote of the court, ``for future
political battles to be fought in this court rather than where they
rightfully belong --- the political branches.''

\href{https://www.dhs.gov/news/2012/06/15/secretary-napolitano-announces-deferred-action-process-young-people-who-are-low}{Established
by Mr. Obama in 2012}, DACA allows young people brought to the United
States as children to apply for a temporary status that shields them
from deportation and allows them to work.

At different times, Mr. Trump has praised the program's goals and
suggested he wanted to preserve it. ``Does anybody really want to throw
out good, educated and accomplished young people who have jobs, some
serving in the military?'' he asked in
\href{https://twitter.com/realdonaldtrump/status/908276308265795585?lang=en}{a
tweet in 2017}.

But Mr. Trump has sometimes struck a different tone. ``Many of the
people in DACA, no longer very young, are far from `angels,'''
\href{https://twitter.com/realDonaldTrump/status/1194219655717642240?ref_src=twsrc\%5Egoogle\%7Ctwcamp\%5Eserp\%7Ctwgr\%5Etweet}{he
wrote on Twitter} last year as the Supreme Court prepared to hear
arguments in the case. ``Some are very tough, hardened criminals.''

In fact, the program has strict requirements. To be eligible, applicants
had to show that they had committed no serious crimes, had arrived in
the United States before they turned 16 and were no older than 30, had
lived in the United States for at least the previous five years, and
were in school, had graduated from high school or received a G.E.D.
certificate, or were an honorably discharged veteran.

``I do not favor punishing children,'' Mr. Trump said in
\href{https://www.whitehouse.gov/briefings-statements/statement-president-donald-j-trump-7/}{his
formal announcement} of the termination. But, he added, ``the program is
unlawful and unconstitutional and cannot be successfully defended in
court.''

Mr. Trump's decision to
\href{https://www.nytimes3xbfgragh.onion/2020/06/19/us/politics/trump-daca.html}{end
the DACA program} was the culmination of a concerted effort by some of
his most anti-immigrant advisers who believed it was imperative that the
president follow through on the promise he made during the campaign.

Stephen K. Bannon, the president's chief strategist during his first six
months in office, worked with Kris Kobach, the former Kansas secretary
of state and an ardent opponent of immigration, to orchestrate a legal
challenge to Mr. Obama's DACA program. Along with Jeff Sessions, then
the attorney general, Mr. Bannon and Mr. Kobach convinced the president
that he should order an end to DACA or the courts would do it for him.

The upshot of the internal debate was a
\href{https://www.dhs.gov/news/2017/09/05/memorandum-rescission-daca}{bare-bones
memo} from Elaine C. Duke, then the acting secretary of homeland
security. She offered no policy reasons for the move, saying only that
DACA was unlawful.

Chief Justice Roberts wrote that Ms. Duke's rationale was inadequate.
She had failed to consider, he wrote, two important points: the degree
to which recipients had come to rely on the program and the possibility
of deferring deportations even if other benefits, like the right to
work, were eliminated.

Young immigrants, Chief Justice Roberts wrote, quoting from
\href{https://www.supremecourt.gov/DocketPDF/18/18-587/117328/20190927151220415_18-587\%20Regents\%20Brief\%20for\%20Respondents.pdf}{a
brief}, had ``enrolled in degree programs, embarked on careers, started
businesses, purchased homes and even married and had children'' in
reliance on the program. Excluding DACA recipients from the work force,
he wrote, could result in the loss of \$215 billion in economic activity
and \$60 billion in federal tax revenues over the next decade.

``Acting Secretary Duke should have considered those matters but did
not,'' Chief Justice Roberts wrote. ``That failure was arbitrary and
capricious.''

Had the administration simply grappled with those issues and nonetheless
declared that it was changing direction as a matter of policy, Chief
Justice Roberts suggested, the termination of DACA would have been a
routine exercise of executive discretion.

Mr. Obama's lawyers had argued that the DACA program was legal because
the government has prosecutorial discretion in deciding whether to
pursue deportation against individual immigrants.

But Mr. Trump said that creating or maintaining the program was beyond
the power of any president, no matter how sympathetic the Dreamers might
be.

In dissent, Justice Thomas agreed. ``The decision to rescind an unlawful
agency action is per se lawful,'' he wrote. He criticized the majority
for ``all but ignoring DACA's substantive legal defect.''

In a separate dissent, Justice Alito said it was shocking that years
have passed without a definitive resolution from the Supreme Court of
whether rescinding DACA was lawful.

``Instead,'' he wrote, ``it tells the Department of Homeland Security to
go back and try again. What this means is that the federal judiciary,
without holding that DACA cannot be rescinded, has prevented that from
occurring during an entire presidential term.''

While the litigation was underway,
\href{https://www.nytimes3xbfgragh.onion/2019/04/07/us/politics/kirstjen-nielsen-dhs-resigns.html}{Kirstjen
Nielsen}, the homeland security secretary at the time, issued a second
set of reasons in
\href{https://www.dhs.gov/sites/default/files/publications/18_0622_S1_Memorandum_DACA.pdf}{a
three-page memo}. It mostly relied on the earlier rationales in Ms.
Duke's memo, but added one more, about the importance of projecting a
message ``that leaves no doubt regarding the clear, consistent and
transparent enforcement of the immigration laws against all classes and
categories of aliens.''

Chief Justice Roberts said the court would not consider the later memo.
Ms. Nielsen's additional justifications, the chief justice wrote, were
``impermissible post hoc rationalizations and thus are not properly
before us.''

``The basic rule here is clear: An agency must defend its actions based
on the reasons it gave when it acted,'' Chief Justice Roberts wrote.
``This is not the case for cutting corners to allow D.H.S. to rely upon
reasons absent from its original decision.''

In his own dissent, Justice Brett M. Kavanaugh said requiring the
administration to provide another set of reasons was a pointless
formality. ``The only practical consequence of the court's decision to
remand appears to be some delay,'' he wrote.

Eight justices rejected a separate argument from lawyers for the young
immigrants, who had said the rescission of the program violated the
Constitution's equal protection clause.

Justice Sonia Sotomayor dissented on that point, saying she would have
let that challenge go forward, based in part on ``statements that
President Trump made both before and after he assumed office'' in which
he called them criminals, drug dealers and rapists.

In a Twitter post Thursday afternoon, Mr. Trump captured the bottom line
of the court's ruling. ``Now we have to start this process all over
again,''
\href{https://twitter.com/realDonaldTrump/status/1273666793362673665}{he
wrote}.

Miriam Jordan contributed reporting from Los Angeles.

Advertisement

\protect\hyperlink{after-bottom}{Continue reading the main story}

\hypertarget{site-index}{%
\subsection{Site Index}\label{site-index}}

\hypertarget{site-information-navigation}{%
\subsection{Site Information
Navigation}\label{site-information-navigation}}

\begin{itemize}
\tightlist
\item
  \href{https://help.nytimes3xbfgragh.onion/hc/en-us/articles/115014792127-Copyright-notice}{©~2020~The
  New York Times Company}
\end{itemize}

\begin{itemize}
\tightlist
\item
  \href{https://www.nytco.com/}{NYTCo}
\item
  \href{https://help.nytimes3xbfgragh.onion/hc/en-us/articles/115015385887-Contact-Us}{Contact
  Us}
\item
  \href{https://www.nytco.com/careers/}{Work with us}
\item
  \href{https://nytmediakit.com/}{Advertise}
\item
  \href{http://www.tbrandstudio.com/}{T Brand Studio}
\item
  \href{https://www.nytimes3xbfgragh.onion/privacy/cookie-policy\#how-do-i-manage-trackers}{Your
  Ad Choices}
\item
  \href{https://www.nytimes3xbfgragh.onion/privacy}{Privacy}
\item
  \href{https://help.nytimes3xbfgragh.onion/hc/en-us/articles/115014893428-Terms-of-service}{Terms
  of Service}
\item
  \href{https://help.nytimes3xbfgragh.onion/hc/en-us/articles/115014893968-Terms-of-sale}{Terms
  of Sale}
\item
  \href{https://spiderbites.nytimes3xbfgragh.onion}{Site Map}
\item
  \href{https://help.nytimes3xbfgragh.onion/hc/en-us}{Help}
\item
  \href{https://www.nytimes3xbfgragh.onion/subscription?campaignId=37WXW}{Subscriptions}
\end{itemize}
