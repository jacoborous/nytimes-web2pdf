Sections

SEARCH

\protect\hyperlink{site-content}{Skip to
content}\protect\hyperlink{site-index}{Skip to site index}

\href{https://www.nytimes3xbfgragh.onion/section/business}{Business}

\href{https://myaccount.nytimes3xbfgragh.onion/auth/login?response_type=cookie\&client_id=vi}{}

\href{https://www.nytimes3xbfgragh.onion/section/todayspaper}{Today's
Paper}

\href{/section/business}{Business}\textbar{}`Banking While Black': How
Cashing a Check Can Be a Minefield

\url{https://nyti.ms/30XgLD5}

\begin{itemize}
\item
\item
\item
\item
\item
\item
\end{itemize}

\href{https://www.nytimes3xbfgragh.onion/news-event/george-floyd-protests-minneapolis-new-york-los-angeles?action=click\&pgtype=Article\&state=default\&region=TOP_BANNER\&context=storylines_menu}{Race
and America}

\begin{itemize}
\tightlist
\item
  \href{https://www.nytimes3xbfgragh.onion/2020/07/26/us/protests-portland-seattle-trump.html?action=click\&pgtype=Article\&state=default\&region=TOP_BANNER\&context=storylines_menu}{Protesters
  Return to Other Cities}
\item
  \href{https://www.nytimes3xbfgragh.onion/2020/07/24/us/portland-oregon-protests-white-race.html?action=click\&pgtype=Article\&state=default\&region=TOP_BANNER\&context=storylines_menu}{Portland
  at the Center}
\item
  \href{https://www.nytimes3xbfgragh.onion/2020/07/23/podcasts/the-daily/portland-protests.html?action=click\&pgtype=Article\&state=default\&region=TOP_BANNER\&context=storylines_menu}{Podcast:
  Showdown in Portland}
\item
  \href{https://www.nytimes3xbfgragh.onion/interactive/2020/07/16/us/black-lives-matter-protests-louisville-breonna-taylor.html?action=click\&pgtype=Article\&state=default\&region=TOP_BANNER\&context=storylines_menu}{45
  Days in Louisville}
\end{itemize}

Advertisement

\protect\hyperlink{after-top}{Continue reading the main story}

Supported by

\protect\hyperlink{after-sponsor}{Continue reading the main story}

\hypertarget{banking-while-black-how-cashing-a-check-can-be-a-minefield}{%
\section{`Banking While Black': How Cashing a Check Can Be a
Minefield}\label{banking-while-black-how-cashing-a-check-can-be-a-minefield}}

Black customers risk being racially profiled on everyday visits to bank
branches. Under federal laws, there is little recourse as long as the
banks ultimately complete their transactions.

\includegraphics{https://static01.graylady3jvrrxbe.onion/images/2020/06/18/business/18bankcustomers2/merlin_173382978_0eff035e-6bfb-44af-ad11-d8fd378efb09-articleLarge.jpg?quality=75\&auto=webp\&disable=upscale}

\href{https://www.nytimes3xbfgragh.onion/by/emily-flitter}{\includegraphics{https://static01.graylady3jvrrxbe.onion/images/2019/06/19/reader-center/author-emily-flitter/author-emily-flitter-thumbLarge.png}}

By \href{https://www.nytimes3xbfgragh.onion/by/emily-flitter}{Emily
Flitter}

\begin{itemize}
\item
  June 18, 2020
\item
  \begin{itemize}
  \item
  \item
  \item
  \item
  \item
  \item
  \end{itemize}
\end{itemize}

Clarice Middleton shook with fear as she stood on the sidewalk outside a
Wells Fargo branch in Atlanta one December morning in 2018. Moments
earlier, she had tried to cash a \$200 check, only to be accused of
fraud by three branch employees, who then called 911.

Ms. Middleton, who is black, remembers thinking: ``I don't want to
die.''

For many black Americans, going to the bank can be a fraught experience.
Something as simple as trying to cash a check or open a bank account can
lead to suspicious employees summoning the police, causing anxiety and
fear --- and sometimes even physical danger --- for the accused
customers.

There is no data on how frequently the police are called on customers
who are making legitimate everyday transactions. The phenomenon has its
own social media hashtag: \#BankingWhileBlack.

Most people who experience an episode of racial profiling don't report
it, lawyers say. Some find it easier to engage in private settlement
negotiations. The few who sue --- as Ms. Middleton did --- are unlikely
to win in court because of loopholes in the law. Now, the police killing
of George Floyd in Minneapolis, which set off nationwide protests
against systemic racism, is prompting more people to speak up.

Ms. Middleton had gone to the Wells Fargo branch in Druid Hills, a
wealthy, mostly white neighborhood in Atlanta, to cash a refund for a
security deposit from a real estate company that had an account with the
bank. Three bank employees examined the check and her identification,
but refused to look at the additional proof Ms. Middleton offered. They
declared the check fraudulent, and one employee called the police,
according to her lawsuit.

When an officer arrived, Ms. Middleton showed him her identification and
the check stub. As a former bank teller, she knew that would be proof
enough that her check was authentic. The officer left without taking
action. The Wells Fargo employees asked Ms. Middleton whether she still
wanted to cash the check.

``I said yes, because they had written all over the back of the check,''
said Ms. Middleton, who sued Wells Fargo last year for racial
discrimination and defamation and sought an unspecified amount of
damages.

Mary Eshet, a Wells Fargo spokeswoman, said Ms. Middleton had begun
yelling ``abusive and profane language'' at the employees when she saw
her ID being scanned.

``Employees tried to address Ms. Middleton's concerns by explaining our
policies, but Ms. Middleton continued to yell profane language,'' Ms
Eshet said. ``She was asked to leave the branch multiple times and
refused, so our employees followed their processes to engage law
enforcement.'' She added that the bank ``appreciates the sensitivities
of engaging law enforcement and the importance of continually reviewing
our training, policies and procedures.''

Ms. Middleton's lawyer, Yechezkel Rodal, said her client had not used
profanity. ``Wells Fargo is in possession of the video surveillance
showing exactly what happened in the branch that morning,'' he said.
``The video will not support Wells Fargo's lies.''

\includegraphics{https://static01.graylady3jvrrxbe.onion/images/2020/06/19/business/00subJPunrest-bankcustomers-print/merlin_173294976_01f9aa3c-49e0-410b-81b8-679bbdd1922b-articleLarge.jpg?quality=75\&auto=webp\&disable=upscale}

Some incidents play out without the involvement of police or courts.

In March 2019, Jabari Bennett wanted to withdraw \$6,400 in cash to buy
a used Toyota Camry from a dealership in Wilmington, Del. He had just
sold his house in Atlanta and moved to Wilmington to live with his
mother. Having been a Wells Fargo customer for four years --- he had
around \$70,000 in his account from the sale of his house --- Mr.
Bennett walked into a nearby branch expecting to be back at the
dealership and in his Camry within minutes.

He came away empty-handed and reeling.

First, a teller refused to accept that he was the account holder,
questioning his out-of-state driver's license, he said --- even though
Mr. Bennett had informed the bank of his new address just two weeks
earlier. Then, a branch manager told Mr. Bennett to leave. He left in
disbelief, then returned to try to complete the transaction. This time,
the manager threatened to call the police. Mr. Bennett left again.

The experience ``made me feel like I was nothing,'' Mr. Bennett said.

He abandoned the deal on the car. A week later, he moved all his money
out of Wells Fargo and then hired Mr. Rodal, who had gained a reputation
for representing black customers against the bank after the story of one
of his clients
\href{https://www.washingtonpost.com/news/business/wp/2018/07/27/a-black-woman-says-wells-fargo-didnt-want-to-cash-her-check-shes-suing-for-discrimination/}{went
viral in 2018}. Mr. Rodal sent Wells Fargo a letter, but negotiations
stalled.

Mr. Bennett decided to share his story publicly in light of the recent
protests: ``I don't want anybody else to go through what I went
through.''

Ms. Eshet, the Wells Fargo spokeswoman, said that branch employees were
trained to spot potential fraud, and that the bank had increased
security protocols to thwart internet scams involving large transfers of
money.

``In this instance, there were enough markers for our team to conduct
extra diligence in order to protect the customer and the bank,'' she
said.

The protests also pushed Benndrick Watson into action.

Last spring, Mr. Watson was driven out of a Wells Fargo branch in
Westchase, a wealthy neighborhood near Tampa, Fla., by what the branch
manager described as a ``slip of the tongue.''

Mr. Watson, who was already a bank customer with a personal checking
account, went to the branch to open a business account for his law firm.

Image

``I felt like I had a knife in my gut,'' Benndrick Watson said of his
experience with a bank branch manager. ``It's a sickening
word.''Credit...Zack Wittman for The New York Times

A banker did a corporate records search and found Mr. Watson's other
business, a record label. Mr. Watson tried to direct the employee to the
records for his law firm instead.

Eventually, the branch manager got involved. He sat down across from Mr.
Watson and watched him enter information, including his Social Security
number, into a keypad.

Then, the man uttered the N-word.

''He just said it --- clear as day, no mistake,'' Mr. Watson said. ``My
jaw just dropped, I dropped the pen, there was silence, he kind of
looked at me, I said: `Did you really just say that?'''

Mr. Watson said the man had immediately begun to protest, saying that he
had not meant to use the word, and that he was deeply sorry. Mr. Watson
did not buy it. He got up and left. The manager followed him to his car,
apologizing profusely, and resigned from the bank shortly afterward.

``I felt like I had a knife in my gut,'' Mr. Watson said. ``It's a
sickening word.''

Mr. Watson turned to Mr. Rodal, who wrote to Wells Fargo seeking an
apology. The bank's regional president, Steve Schultz, responded. ``It
seems that the utterance of the offensive term was unintentional,'' Mr.
Schultz wrote, but said the bank had taken ``corrective action'' against
the branch manager anyway, without providing details. Ms. Eshet of Wells
Fargo said the manager was deemed ineligible for any job with the bank.

Mr. Watson sued Wells Fargo in federal court in Florida on June 4.

In a statement, Ms. Eshet said: ``We deeply apologize to Mr. Watson.
There's no excuse for it, and while we took action to address the
matter, it cannot undo what happened and how he felt. We are very
sorry.''

The problem is hardly confined to Wells Fargo. Last June, Robyn Murphy,
a public relations consultant in Maryland, took her 18-year-old son,
Jason, to a Bank of America branch in Owings Mills, Md., to open a joint
savings account. Ms. Murphy, a 20-year customer of the bank, said she
was shocked when an employee refused to proceed after a computer program
flagged her son's Social Security number as fraudulent.

Ms. Murphy protested: Her son had his own checking account at the bank.
His Social Security number had already been used there without issue.
The Murphys are black. Mr. Murphy, his mother said, is 6-foot-9.

``For all I know, it's fraud,'' the employee told them. Ms. Murphy said
he had asked them to come back with Mr. Murphy's Social Security card.
When Mr. Murphy stood up, the employee yelled: ``Don't get up!''

After Ms. Murphy contacted a senior vice president she knew at the bank,
other officials apologized and offered to open the branch whenever it
was convenient for the Murphys to return and complete the transaction
--- which they did.

``It weighed on us very heavily for a long time,'' Ms. Murphy said.

``We understand the client did not feel she and her son were treated
properly in this interaction with our team, and we regret that,'' Bill
Halldin, a Bank of America spokesman, said in an emailed statement.
``These alerts are designed to protect our clients from fraud and misuse
of their personal information.'' He declined to comment on what, if any,
action the bank had taken against the employee.

Banks say they reject racism of any sort. The country's four largest
banks by asset size, JPMorgan Chase, Wells Fargo, Bank of America and
Citigroup, all require branch employees to complete annual diversity
training, according to the banks' representatives.

Still, banks have not managed to weed out discrimination. The New York
Times reported in December that a JPMorgan Chase employee had described
a customer as being
``\href{https://www.nytimes3xbfgragh.onion/2019/12/11/business/jpmorgan-banking-racism.html}{from
Section 8}'' and therefore undeserving of service. The bank has since
said it would seek to increase its sensitivity to issues surrounding
race.

But little is mandated by law. The Civil Rights Act of 1964 lists
specific businesses that may not treat black customers differently:
movie theaters, hotels, restaurants, and performance and sports venues.
Federal courts have held that because the law identifies the kinds of
businesses to which it applies, those not on the list, such as banks,
cannot be held to it. That loophole makes it hard for victims of racial
profiling to win in court.

There is an additional limitation. In 1866, Congress created new laws to
establish rights for black Americans, including one giving them the
right to enter into agreements to buy goods or services and have those
contracts enforced. Courts have since ruled that the law requires only
that service be granted eventually.

In 2012, for instance, a federal appeals court ruled that a Hispanic man
who had been turned away by a white cashier at a Target store in Florida
did not have a case against Target because he was able to complete his
purchases with a different cashier.

That could stymie Ms. Middleton's case. Wells Fargo is arguing that
because she was eventually able to cash her check, a judge should
dismiss it.

Advertisement

\protect\hyperlink{after-bottom}{Continue reading the main story}

\hypertarget{site-index}{%
\subsection{Site Index}\label{site-index}}

\hypertarget{site-information-navigation}{%
\subsection{Site Information
Navigation}\label{site-information-navigation}}

\begin{itemize}
\tightlist
\item
  \href{https://help.nytimes3xbfgragh.onion/hc/en-us/articles/115014792127-Copyright-notice}{©~2020~The
  New York Times Company}
\end{itemize}

\begin{itemize}
\tightlist
\item
  \href{https://www.nytco.com/}{NYTCo}
\item
  \href{https://help.nytimes3xbfgragh.onion/hc/en-us/articles/115015385887-Contact-Us}{Contact
  Us}
\item
  \href{https://www.nytco.com/careers/}{Work with us}
\item
  \href{https://nytmediakit.com/}{Advertise}
\item
  \href{http://www.tbrandstudio.com/}{T Brand Studio}
\item
  \href{https://www.nytimes3xbfgragh.onion/privacy/cookie-policy\#how-do-i-manage-trackers}{Your
  Ad Choices}
\item
  \href{https://www.nytimes3xbfgragh.onion/privacy}{Privacy}
\item
  \href{https://help.nytimes3xbfgragh.onion/hc/en-us/articles/115014893428-Terms-of-service}{Terms
  of Service}
\item
  \href{https://help.nytimes3xbfgragh.onion/hc/en-us/articles/115014893968-Terms-of-sale}{Terms
  of Sale}
\item
  \href{https://spiderbites.nytimes3xbfgragh.onion}{Site Map}
\item
  \href{https://help.nytimes3xbfgragh.onion/hc/en-us}{Help}
\item
  \href{https://www.nytimes3xbfgragh.onion/subscription?campaignId=37WXW}{Subscriptions}
\end{itemize}
