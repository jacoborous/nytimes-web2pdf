Sections

SEARCH

\protect\hyperlink{site-content}{Skip to
content}\protect\hyperlink{site-index}{Skip to site index}

\href{https://www.nytimes3xbfgragh.onion/section/business}{Business}

\href{https://myaccount.nytimes3xbfgragh.onion/auth/login?response_type=cookie\&client_id=vi}{}

\href{https://www.nytimes3xbfgragh.onion/section/todayspaper}{Today's
Paper}

\href{/section/business}{Business}\textbar{}\$130 Billion Left at
Paycheck Program Deadline, but Senate Acts to Extend It

\url{https://nyti.ms/3dSskhX}

\begin{itemize}
\item
\item
\item
\item
\item
\item
\end{itemize}

Advertisement

\protect\hyperlink{after-top}{Continue reading the main story}

Supported by

\protect\hyperlink{after-sponsor}{Continue reading the main story}

\hypertarget{130-billion-left-at-paycheck-program-deadline-but-senate-acts-to-extend-it}{%
\section{\$130 Billion Left at Paycheck Program Deadline, but Senate
Acts to Extend
It}\label{130-billion-left-at-paycheck-program-deadline-but-senate-acts-to-extend-it}}

Forgivable loans went to nearly five million small businesses that could
use the money to pay workers to stay home. Shifting rules blunted the
effect.

\includegraphics{https://static01.graylady3jvrrxbe.onion/images/2020/06/30/business/30virus-pppend1/merlin_174058341_3d62cc8c-b5d7-4b02-905b-b4e55b9d9ff6-articleLarge.jpg?quality=75\&auto=webp\&disable=upscale}

\href{https://www.nytimes3xbfgragh.onion/by/stacy-cowley}{\includegraphics{https://static01.graylady3jvrrxbe.onion/images/2018/10/03/multimedia/author-stacy-cowley/author-stacy-cowley-thumbLarge.png}}

By \href{https://www.nytimes3xbfgragh.onion/by/stacy-cowley}{Stacy
Cowley}

\begin{itemize}
\item
  June 30, 2020
\item
  \begin{itemize}
  \item
  \item
  \item
  \item
  \item
  \item
  \end{itemize}
\end{itemize}

After a stumbling start, the government's centerpiece
\href{https://www.nytimes3xbfgragh.onion/2020/07/06/us/ppp-small-business-loans.html}{relief
program for small businesses} was closing down on Tuesday --- although
it may turn out to be a temporary hiatus.

In just three months, the
\href{https://www.nytimes3xbfgragh.onion/2020/06/30/us/politics/ppp-extension.html}{Paycheck
Protection Program} handed out \$520 billion in loans meant to preserve
workers' jobs during the coronavirus pandemic. But as
\href{https://www.nytimes3xbfgragh.onion/2020/06/27/us/after-asking-americans-to-sacrifice-in-shutdown-leaders-failed-to-control-virus.html?action=click\&module=Spotlight\&pgtype=Homepage}{new
outbreaks spike} across the country and force many states to rethink
their plans to reopen businesses, the program had more than \$130
billion still in its coffers.

It might not be gone for long, though. Late Tuesday, just a few hours
before the program was scheduled to shut down, the
\href{https://www.nytimes3xbfgragh.onion/2020/06/30/us/politics/coronavirus-senate-small-business-program.html}{Senate
approved a five-week extension}. It wasn't clear when the House might
take up the bill.

John Lettieri, the chief executive of the Economic Innovation Group, a
think tank focused on entrepreneurship, praised the program, calling the
aid it provided to small businesses ``a major achievement.'' But there
was still work to be done, he said.

``We're still in a public health crisis, and we're facing a long, slow,
uneven return,''\\
he said. ``Millions of businesses still have their survival at risk.''

The
\href{https://www.nytimes3xbfgragh.onion/2020/04/26/business/ppp-small-business-loans.html}{hastily
constructed and frequently chaotic aid program}, run by the Small
Business Administration but carried out through banks, handed out money
to nearly five million businesses nationwide, giving them low-interest
loans to cover roughly two and a half months of their typical payroll
costs. Those that use most of the money to pay employees can have their
debt forgiven.

The cash went to a wide variety of companies: manufacturing firms with
hundreds of workers, Main Street retailers with a few dozen employees,
and freelancers working for themselves. The loans ranged from a few
hundred dollars to \$10 million, and allowed businesses to keep paying
employees --- even if they had nothing to do but sit at home.

The program appears to have helped prevent the nation's staggering job
losses from growing even worse. Hiring rebounded more than expected in
May as companies in some of the hardest-hit industries, especially
restaurants,
\href{https://www.nytimes3xbfgragh.onion/2020/06/05/business/economy/jobs-report.html}{restored
millions of jobs} by recalling laid-off workers and hiring new ones.

That's how it played out for Dr. Chris Stansbury, an optometrist who
co-founded \href{https://www.wv-eye.com/}{West Virginia Eye
Consultants}, which has seven offices around the southern part of the
state. He furloughed 40 employees in late March after a statewide
stay-at-home order, when his once-thriving practice was limited to
emergency appointments only. For weeks, its sales were negligible.

The loan he received through the program on April 16 gave him a
financial safety net as he began to reopen --- with a host of new health
precautions --- in early May. Sales are back to around 90 percent of
normal, and Dr. Stansbury said he was cautiously optimistic that the
worst had passed for his business. Nearly all of his workers are back on
the job.

Image

West Virginia Eye Consultants furloughed 40 employees; nearly all are
back on the job.Credit...Kristian Thacker for The New York Times

Image

Without the loan, ``things would have been pretty dire,'' said Dr. Chris
Stansbury, a co-founder of the business.Credit...Kristian Thacker for
The New York Times

``If we hadn't had this money to get us through, things would have been
pretty dire,'' he said. ``I don't think we would have been able to
reopen all of our locations right away.''

Other businesses didn't have such a smooth experience. The program was
marred by technical problems --- like
\href{https://www.nytimes3xbfgragh.onion/2020/04/27/business/sba-loan-system-crash.html}{overtaxed
computer systems that crashed} --- and confusing, frequently revised
rules that frustrated borrowers and lenders alike. Some banks limited
their lending to companies with which they already had relationships.

After a rush of early demand --- the initial \$349 billion set aside for
the program was
\href{https://www.nytimes3xbfgragh.onion/2020/04/16/business/coronavirus-sba-loans-out-of-money.html}{gone
in 13 days} --- borrowing slowed significantly. Any money ultimately
left over will stay with the Treasury.

Before the Senate vote late Tuesday, Treasury Secretary Steven Mnuchin
told members of the House Financial Services Committee that he had been
in discussions with senators from both parties about allowing the
remaining funds to be repurposed. A few hours later, the Senate approved
legislation that would extend the program to Aug. 8.

Lenders cited two main reasons there was money left over. First, most
eligible companies that wanted a loan were ultimately able to obtain
one. (The program limited each applicant to only one loan.) Also, the
program's complicated and shifting requirements dissuaded some qualified
borrowers, who feared they would be unable to get their loan forgiven.

Trying to comply with those rules was a challenge for many businesses.

Tracy Singleton closed her farm-to-table restaurant in Minneapolis, the
\href{https://www.birchwoodcafe.com/}{Birchwood Cafe}, in mid-March and
laid off all but a handful of her 62 workers. She received a \$382,200
loan in early April, a week after the program began, and soon spent it
all --- even though she won't be fully reopening any time soon.

Ms. Singleton said she might have chosen to spend the money more slowly
if things had been different. ``But I had to go with the rules as they
were at the time,'' she said.

When she received the loan, businesses had just eight weeks to spend the
cash if they wanted to have the loan completely forgiven. So Ms.
Singleton, who had switched to curbside pickup sales, brought back
dozens of workers, brainstorming new projects for them to tackle. Her
payroll ballooned from a skeleton crew of eight to a peak of 48
employees.

But as the clock ticked down to the end of her eight weeks of support,
it became clearer to lawmakers that the downturn wasn't ending anytime
soon.

Congress
\href{https://www.nytimes3xbfgragh.onion/2020/06/10/business/Small-business-loans-ppp.html}{amended
the loan program in early June} to give recipients nearly six months to
use their aid money, but Ms. Singleton had already spent most of her
funds. When the money ran out, she laid off workers again. She is down
to a staff of about 20.

``We looked at this as a bridge,'' she said. ``Then our time was up, and
there's no solid ground to stand on yet.''

Congress has been
\href{https://www.nytimes3xbfgragh.onion/2020/06/05/business/jobs-economy-recovery-aid.html}{bitterly
divided about what any new stimulus package should look like}. But
little other help is on offer for small businesses: Companies with fewer
than 500 workers can turn to another Small Business Administration
program, the
\href{https://www.sba.gov/funding-programs/loans/coronavirus-relief-options/economic-injury-disaster-loan-emergency-advance}{Economic
Injury Disaster Loan} fund, but it has
\href{https://www.nytimes3xbfgragh.onion/2020/04/09/business/smallbusiness/small-business-disaster-loans-coronavirus.html}{struggled
with overwhelming demand} and has imposed a \$150,000 cap on its loans.
The Federal Reserve's new Main Street Lending Program offers loans of
\$250,000 and more, but on more onerous terms.

Small businesses employ about half of America's nongovernment workers,
and a fresh wave of deep reductions or permanent closures would quickly
cascade through the national economy. The pain would be even more acute
in
\href{https://www.nytimes3xbfgragh.onion/2020/06/17/upshot/coronavirus-spending-rich-poor.html}{hard-hit
industries, like the service sector, and communities} that
disproportionately rely on small employers.

Some of those employers, like Ms. Singleton, are weighing their bottom
lines against the continued uncertainty of the pandemic.

Even though Minnesota has allowed restaurants to reopen for outdoor
dining and limited indoor seating, Ms. Singleton has stuck with curbside
pickup. She is watching with dread as infection rates soar in areas that
relaxed their restrictions and is wary of putting her employees or
customers at risk.

``I frankly don't think it's safe,'' she said. ``You can really only
reopen once. If we had to open and shut down again, that would be a
nightmare.''

Emily Cochrane contributed reporting.

Advertisement

\protect\hyperlink{after-bottom}{Continue reading the main story}

\hypertarget{site-index}{%
\subsection{Site Index}\label{site-index}}

\hypertarget{site-information-navigation}{%
\subsection{Site Information
Navigation}\label{site-information-navigation}}

\begin{itemize}
\tightlist
\item
  \href{https://help.nytimes3xbfgragh.onion/hc/en-us/articles/115014792127-Copyright-notice}{©~2020~The
  New York Times Company}
\end{itemize}

\begin{itemize}
\tightlist
\item
  \href{https://www.nytco.com/}{NYTCo}
\item
  \href{https://help.nytimes3xbfgragh.onion/hc/en-us/articles/115015385887-Contact-Us}{Contact
  Us}
\item
  \href{https://www.nytco.com/careers/}{Work with us}
\item
  \href{https://nytmediakit.com/}{Advertise}
\item
  \href{http://www.tbrandstudio.com/}{T Brand Studio}
\item
  \href{https://www.nytimes3xbfgragh.onion/privacy/cookie-policy\#how-do-i-manage-trackers}{Your
  Ad Choices}
\item
  \href{https://www.nytimes3xbfgragh.onion/privacy}{Privacy}
\item
  \href{https://help.nytimes3xbfgragh.onion/hc/en-us/articles/115014893428-Terms-of-service}{Terms
  of Service}
\item
  \href{https://help.nytimes3xbfgragh.onion/hc/en-us/articles/115014893968-Terms-of-sale}{Terms
  of Sale}
\item
  \href{https://spiderbites.nytimes3xbfgragh.onion}{Site Map}
\item
  \href{https://help.nytimes3xbfgragh.onion/hc/en-us}{Help}
\item
  \href{https://www.nytimes3xbfgragh.onion/subscription?campaignId=37WXW}{Subscriptions}
\end{itemize}
