Sections

SEARCH

\protect\hyperlink{site-content}{Skip to
content}\protect\hyperlink{site-index}{Skip to site index}

\href{https://www.nytimes3xbfgragh.onion/section/science}{Science}

\href{https://myaccount.nytimes3xbfgragh.onion/auth/login?response_type=cookie\&client_id=vi}{}

\href{https://www.nytimes3xbfgragh.onion/section/todayspaper}{Today's
Paper}

\href{/section/science}{Science}\textbar{}Robert Laughlin, Preserver of
a Mayan Language, Dies at 85

\url{https://nyti.ms/3g6eBWj}

\begin{itemize}
\item
\item
\item
\item
\item
\end{itemize}

\href{https://www.nytimes3xbfgragh.onion/news-event/coronavirus?action=click\&pgtype=Article\&state=default\&region=TOP_BANNER\&context=storylines_menu}{The
Coronavirus Outbreak}

\begin{itemize}
\tightlist
\item
  live\href{https://www.nytimes3xbfgragh.onion/2020/08/04/world/coronavirus-covid-19.html?action=click\&pgtype=Article\&state=default\&region=TOP_BANNER\&context=storylines_menu}{Latest
  Updates}
\item
  \href{https://www.nytimes3xbfgragh.onion/interactive/2020/us/coronavirus-us-cases.html?action=click\&pgtype=Article\&state=default\&region=TOP_BANNER\&context=storylines_menu}{Maps
  and Cases}
\item
  \href{https://www.nytimes3xbfgragh.onion/interactive/2020/science/coronavirus-vaccine-tracker.html?action=click\&pgtype=Article\&state=default\&region=TOP_BANNER\&context=storylines_menu}{Vaccine
  Tracker}
\item
  \href{https://www.nytimes3xbfgragh.onion/2020/08/02/us/covid-college-reopening.html?action=click\&pgtype=Article\&state=default\&region=TOP_BANNER\&context=storylines_menu}{College
  Reopening}
\item
  \href{https://www.nytimes3xbfgragh.onion/live/2020/08/03/business/stock-market-today-coronavirus?action=click\&pgtype=Article\&state=default\&region=TOP_BANNER\&context=storylines_menu}{Economy}
\end{itemize}

Advertisement

\protect\hyperlink{after-top}{Continue reading the main story}

Supported by

\protect\hyperlink{after-sponsor}{Continue reading the main story}

Those We've Lost

\hypertarget{robert-laughlin-preserver-of-a-mayan-language-dies-at-85}{%
\section{Robert Laughlin, Preserver of a Mayan Language, Dies at
85}\label{robert-laughlin-preserver-of-a-mayan-language-dies-at-85}}

His monumental dictionary, after years of field work, documented Tzotzil
in southern Mexico. But that was just the start of his efforts to
preserve the culture.

\includegraphics{https://static01.graylady3jvrrxbe.onion/images/2020/06/25/obituaries/22Laughlin/22Laughlin-articleLarge.jpg?quality=75\&auto=webp\&disable=upscale}

\href{https://www.nytimes3xbfgragh.onion/by/neil-genzlinger}{\includegraphics{https://static01.graylady3jvrrxbe.onion/images/2018/06/13/multimedia/author-neil-genzlinger/author-neil-genzlinger-thumbLarge.jpg}}

By \href{https://www.nytimes3xbfgragh.onion/by/neil-genzlinger}{Neil
Genzlinger}

\begin{itemize}
\item
  Published June 24, 2020Updated June 26, 2020
\item
  \begin{itemize}
  \item
  \item
  \item
  \item
  \item
  \end{itemize}
\end{itemize}

\emph{This obituary is part of a series about people who have died in
the coronavirus pandemic. Read about others}
\href{https://www.nytimes3xbfgragh.onion/interactive/2020/obituaries/people-died-coronavirus-obituaries.html}{\emph{here}}\emph{.}

Robert M. Laughlin, an anthropologist and linguist whose extensive work
in the state of Chiapas in southern Mexico documented and helped
revitalize Mayan languages and culture, died on May 28 in Alexandria,
Va. He was 85.

His son, Reese, said the cause was the new coronavirus.

Dr. Laughlin spent much of his professional life doing field work in
Chiapas, beginning in the late 1950s. He learned the Tzotzil (also
spelled Tsotsil) language as a graduate student with the Harvard Chiapas
Project, a long-term ethnographic field study that had just been started
by Professor Evon Vogt and was focusing on the town of Zinacantán. After
years of painstaking work, in 1975 Dr. Laughlin published The Great
Tzotzil Dictionary of San Lorenzo Zinacantán, with 30,000 entries.

Indigenous languages in the region --- there are many --- had been under
siege since the Spanish conquest, and Dr. Laughlin's dictionary helped
spur a revival of interest in them. The dictionary, published by the
Smithsonian Institution in Washington, where Dr. Laughlin was curator of
Mesoamerican ethnology, was not simply a compilation of which Tzotzil
word equals which English word. It was a deep dive into word origins,
how the language had mutated and more.

``The term `dictionary' hardly does the work justice,'' Judith Aissen,
professor emerita of linguistics at the University of California, Santa
Cruz, said in an email. ``It is a rigorous work of linguistic
scholarship, but through its entries, also the repository of a great
deal of cultural knowledge.''

The dictionary, created with two local collaborators, Romin Teratol and
Anselmo Peres, set an example for the field. ``It has been the
cornerstone of so many efforts in language and knowledge revitalization
ever since,'' Igor Krupnik, chair of the anthropology department at the
Smithsonian's National Museum of Natural History, said by email.

\includegraphics{https://static01.graylady3jvrrxbe.onion/images/2020/06/25/obituaries/22Laughlin3/22Laughlin3-articleLarge.jpg?quality=75\&auto=webp\&disable=upscale}

But it was only the beginning for Dr. Laughlin. He wrote or collaborated
on various collections of folk tales and dreams, an 18th-century Tzotzil
dictionary (with John B. Haviland, an anthropology professor at the
University of California, San Diego), and more. And in 1982, when some
Indigenous friends asked him for help in creating a cultural
association, he became one of the founders of Sna Jtz'ibajom --- or, in
English, the House of the Writer, a collective that promoted local
writings and publications.

An offshoot of that, a few years later, was Monkey Business Theater, a
troupe that performed folk tales and other works. He brought in the
American puppeteer Amy Trompetter to help local participants use puppets
in their storytelling.

``To her distress, the first skit they chose to perform was a folk tale
that tells of a newlywed whose wife's head mysteriously disappears at
night to eat corpses,'' he wrote in ``Monkey Business Theater,'' a 2008
book about the troupe. But the group caught on and was soon in high
demand, performing throughout the region and beyond.

One of Dr. Laughlin's most recent collaborations was ``Mayan Tales From
Chiapas, Mexico'' (2014), in which he and two translators recorded 42
folk tales as told by the same woman, Francisca Hernández Hernández, the
only Tzotzil speaker remaining in her village. The book presented the
stories in English, Spanish and Tzotzil.

In the foreword, Gary H. Gossen, professor emeritus of anthropology and
Latin American studies at the University at Albany, the State University
of New York, wrote of Dr. Laughlin's career: ``He has earnestly and
successfully returned to the native Maya communities of highland Chiapas
a sense of ownership of their own literary legacy.''

Robert Moody Laughlin was born on May 29, 1934, in Princeton, N.J., to
Ledlie and Roberta Howe Laughlin. His father was assistant dean of
admissions at Princeton University, and his mother was a homemaker.

He grew up in Princeton, graduated from South Kent School in Connecticut
in 1952 and earned a bachelor's degree in English literature at
Princeton in 1956. The next year he enrolled in a summer graduate
program in anthropology at the Escuela Nacional de Antropología e
Historia in Mexico City, which included field work among the Mazatec, an
Indigenous people in the state of Oaxaca.

His interest piqued, he enrolled at Harvard, where he received a
master's degree in anthropology in 1961 and a Ph.D. in it in 1963. In
1960 he married Miriam Elizabeth Wolfe, and after he joined the
Smithsonian in 1965, they had alternated between living in Chiapas and
Alexandria, Va., for decades.

Image

Dr. Laughlin in 2005 in his office in the anthropology department of the
Smithsonian Institution. The masks were from Mayan puppet
theater.Credit...James Di Loreto/Smithsonian

Almost as challenging as compiling his monumental 1975 dictionary was
physically producing it, given the complexity of the material, the
multiplicity of symbols and unusual letter combinations, and the
limitations of the relatively primitive computers used to produce it.

``When I went to pick it up,'' Dr. Laughlin wrote in the introduction,
describing the first attempt to print a proof copy, ``I discovered that
the Tzotzil-English section was very much as I had desired. But the
English to Tzotzil section of The Great Tzotzil Dictionary had been
reduced to the lowest common denominator; page after page of one letter
per line arranged in a single column. This was followed by all the Latin
names neatly decapitated and arranged alphabetically according to the
second letter.''

``My dictionary,'' he added, ``became known around the museum as The
Great Tzotzil Disaster.''

Modest efforts to resurrect Indigenous languages had been going on for
several decades when the dictionary appeared, but the Tzotzil language
and its cousins were primarily oral traditions; speakers of such
languages were illiterate in them. The dictionary helped change that.

``A potential audience had slowly been building for material in Tzotzil,
Tzeltal and about 30 other Mayan languages,'' a 1992 article in
Smithsonian magazine noted. ``Laughlin's dictionary contributed a
standardized template for writing down the Mayan sounds.''

Dr. Laughlin died in a hospital in Alexandria. In addition to his son,
he is survived by his wife; a daughter, Liana Laughlin; and three
grandchildren.

When Dr. Laughlin's dictionary was published,
\href{https://www.nytimes3xbfgragh.onion/2005/12/16/us/william-proxmire-maverick-democratic-senator-from-wisconsin-is-dead-at.html?searchResultPosition=1}{Senator
William Proxmire}, the prominent Wisconsin Democrat, gave it one of his
Golden Fleece Awards, which he used to call attention to projects he
considered frivolous. Colleagues said Dr. Laughlin had considered the
award a badge of honor --- ``perhaps out of general contrariness,'' Thor
R. Anderson, his friend and sometimes collaborator, wrote in an
appreciation, ``but also because, at the height of that particular
contretemps, fellow scholars rushed to his defense.''

In 1988, when Dr. Laughlin and Dr. Haviland published their colonial-era
dictionary, ``The Great Tzotzil Dictionary of Santo Domingo Zinacantán,
With Grammatical Analysis and Historical Commentary,'' careful readers
may have noted the dedication on Page 7:

\emph{To William E. Proxmire}

\emph{For the fun of it!}

\href{https://www.nytimes3xbfgragh.onion/interactive/2020/obituaries/people-died-coronavirus-obituaries.html?action=click\&pgtype=Article\&state=default\&region=BELOW_MAIN_CONTENT\&context=covid_obits_promo}{}

\hypertarget{those-weve-lost}{%
\section{Those We've Lost}\label{those-weve-lost}}

The coronavirus pandemic has taken an incalculable death toll. This
series is designed to put names and faces to the numbers.

Read more

\includegraphics{https://static01.graylady3jvrrxbe.onion/images/2020/07/30/obituaries/30Pedro/30Pedro-square640.jpg}

\hypertarget{bernaldina-josuxe9-pedro}{%
\section{Bernaldina José Pedro}\label{bernaldina-josuxe9-pedro}}

d. Boa Vista, Brazil

Leader among the Indigenous Macuxi

\includegraphics{https://static01.graylady3jvrrxbe.onion/images/2020/07/31/obituaries/31Swing/merlin_175167783_8913bc90-0d64-43f3-a655-1bb1bf1601c9-square640.jpg}

\hypertarget{john-eric-swing}{%
\section{John Eric Swing}\label{john-eric-swing}}

d. Fountain Valley, Calif.

Champion of Filipino-Americans

\includegraphics{https://static01.graylady3jvrrxbe.onion/images/2020/07/27/obituaries/27Victor/merlin_175001436_38b11f8e-227a-4e2c-9821-7618af9b2524-square640.jpg}

\hypertarget{victor-victor}{%
\section{Victor Victor}\label{victor-victor}}

d. Santo Domingo, Dominican Republic

Beloved musician of the Dominican Republic

\includegraphics{https://static01.graylady3jvrrxbe.onion/images/2020/07/31/obituaries/31Negron/merlin_175160169_516322ae-fd23-4969-b6b2-193ced371105-square640.jpg}

\hypertarget{dr-eddie-negruxf3n}{%
\section{Dr. Eddie Negrón}\label{dr-eddie-negruxf3n}}

d. Fort Walton Beach, Fla.

Internist on Florida's Emerald Coast

\includegraphics{https://static01.graylady3jvrrxbe.onion/images/2020/07/30/obituaries/30Dobson/merlin_175115928_f6b9271c-8f05-4fe1-a38a-5ca4a58f8935-square640.jpg}

\hypertarget{dobby-dobson}{%
\section{Dobby Dobson}\label{dobby-dobson}}

d. Coral Springs, Fla.

Jamaican singer and songwriter

\includegraphics{https://static01.graylady3jvrrxbe.onion/images/2020/08/01/obituaries/28Gonzalez/merlin_175002771_beb57888-3951-409a-ae13-03a94b2e962e-square640.jpg}

\hypertarget{waldemar-gonzalez}{%
\section{Waldemar Gonzalez}\label{waldemar-gonzalez}}

d. White Plains, N.Y.

Teacher and social worker

Advertisement

\protect\hyperlink{after-bottom}{Continue reading the main story}

\hypertarget{site-index}{%
\subsection{Site Index}\label{site-index}}

\hypertarget{site-information-navigation}{%
\subsection{Site Information
Navigation}\label{site-information-navigation}}

\begin{itemize}
\tightlist
\item
  \href{https://help.nytimes3xbfgragh.onion/hc/en-us/articles/115014792127-Copyright-notice}{©~2020~The
  New York Times Company}
\end{itemize}

\begin{itemize}
\tightlist
\item
  \href{https://www.nytco.com/}{NYTCo}
\item
  \href{https://help.nytimes3xbfgragh.onion/hc/en-us/articles/115015385887-Contact-Us}{Contact
  Us}
\item
  \href{https://www.nytco.com/careers/}{Work with us}
\item
  \href{https://nytmediakit.com/}{Advertise}
\item
  \href{http://www.tbrandstudio.com/}{T Brand Studio}
\item
  \href{https://www.nytimes3xbfgragh.onion/privacy/cookie-policy\#how-do-i-manage-trackers}{Your
  Ad Choices}
\item
  \href{https://www.nytimes3xbfgragh.onion/privacy}{Privacy}
\item
  \href{https://help.nytimes3xbfgragh.onion/hc/en-us/articles/115014893428-Terms-of-service}{Terms
  of Service}
\item
  \href{https://help.nytimes3xbfgragh.onion/hc/en-us/articles/115014893968-Terms-of-sale}{Terms
  of Sale}
\item
  \href{https://spiderbites.nytimes3xbfgragh.onion}{Site Map}
\item
  \href{https://help.nytimes3xbfgragh.onion/hc/en-us}{Help}
\item
  \href{https://www.nytimes3xbfgragh.onion/subscription?campaignId=37WXW}{Subscriptions}
\end{itemize}
