Sections

SEARCH

\protect\hyperlink{site-content}{Skip to
content}\protect\hyperlink{site-index}{Skip to site index}

\href{https://myaccount.nytimes3xbfgragh.onion/auth/login?response_type=cookie\&client_id=vi}{}

\href{https://www.nytimes3xbfgragh.onion/section/todayspaper}{Today's
Paper}

\href{/section/opinion}{Opinion}\textbar{}Trump Speaks! And Speaks. And
Speaks \ldots{}

\url{https://nyti.ms/2YvlOJh}

\begin{itemize}
\item
\item
\item
\item
\item
\item
\end{itemize}

Advertisement

\protect\hyperlink{after-top}{Continue reading the main story}

\href{/section/opinion}{Opinion}

Supported by

\protect\hyperlink{after-sponsor}{Continue reading the main story}

\hypertarget{trump-speaks-and-speaks-and-speaks-}{%
\section{Trump Speaks! And Speaks. And Speaks
\ldots{}}\label{trump-speaks-and-speaks-and-speaks-}}

At least those rallies keep him off the streets.

\href{https://www.nytimes3xbfgragh.onion/by/gail-collins}{\includegraphics{https://static01.graylady3jvrrxbe.onion/images/2018/04/03/opinion/gail-collins/gail-collins-thumbLarge.png}}

By \href{https://www.nytimes3xbfgragh.onion/by/gail-collins}{Gail
Collins}

Opinion Columnist

\begin{itemize}
\item
  June 24, 2020
\item
  \begin{itemize}
  \item
  \item
  \item
  \item
  \item
  \item
  \end{itemize}
\end{itemize}

\includegraphics{https://static01.graylady3jvrrxbe.onion/images/2020/06/25/opinion/25collins_sub_2ndary/merlin_173832819_4a109b41-e5b2-4996-a84c-b1f4655f624f-articleLarge.jpg?quality=75\&auto=webp\&disable=upscale}

Donald Trump thinks we're out to get him.

``You could say 10 speeches. One little word, they'll say: `He's lost
it,''' the president complained during a speech in Phoenix this week.

That would presumably be an \emph{inaccurate} little word. Or something
very weird, like his claim at a famously underattended event in Tulsa
that he'd ordered a slowdown in coronavirus testing to make it seem as
if the infection rate was smaller.

Desperate presidential spinners said that was just a joke. ``I don't
kid,'' Trump retorted.

Tulsa was, according to the president, the beginning of his re-election
campaign. He's actually shot off the starting gun several times before.
But it does feel as if we're in a new phase. Those big rallies are
Trump's very favorite part of being the leader of the most powerful
nation on the globe. He's been locked down for months now, confined
mainly to gatherings in which other people occasionally get to talk.

He needs his screaming fans, even if this is a terrible idea,
healthwise. Six members of Trump's advance team got sick while doing the
planning, and now at least two other staffers tested positive.

You're not going to get this guy to stay home. He needs to compliment
himself in front of thousands of people. Lacing into the Democratic
``elite,'' Trump assured his audience that he is more elite than
anybody. ``I look better than them. Much more handsome. Got better hair
than they do. I got nicer properties. I got nicer houses. I got nicer
apartments. I got nicer everything.''

And, for sure, a bigger ego. After he finished raging to his staff about
the tiers of empty seats in Tulsa, the president announced the night had
been a historical smash hit: ``No. 1 show in Fox history for a Saturday
night.''

Yeah, Fox News announced ``a whopping 7.7 million total viewers'' had
tuned in to listen to Trump speak. Pretty impressive, particularly if
you ignore the fact that most of the nation has been locked up at home
in a world without sports broadcasting, having already rewatched every
episode of ``Star Trek'' and ``Friends.''

Still, many of us will remember Tulsa as That Rally Where Two-Thirds Of
The Seats Were Empty. His next appearance, in Arizona, was much more
Trump's cup of tea: a megachurch packed with cheering fans who generally
ignored all the official pleas for masking.

Most of the audience was young. Having lured them into endangering their
health for his ego, Trump entertained them with tales of his heroic
efforts to drain the political swamp. ``I never knew it was so deep ---
it's deep and thick and a lot of bad characters,'' he confided.

Well, there aren't many swamp critters more appalling than Roger Stone,
the political fixer who spent part of the 2016 presidential campaign
trying to get information for the Trump forces about Hillary Clinton's
emails.

Stone was convicted of lying to Congress and attempting to intimidate a
witness --- in part by threatening to kidnap the guy's therapy dog.

As swamp residents go, Stone would maybe be the equivalent of a
5-foot-11-inch mosquito. But on Wednesday a federal prosecutor told
Congress that he and his associates had been told they could be fired if
they didn't go easy when it came to sentencing. On account of how, you
know, Stone was the president's pal.

Even if they're a little dodgy on the facts side, the rallies are at
least a good way to keep Trump distracted. In Tulsa, he was fretting
about the ongoing demonstrations in Seattle. He asked a congressman who
was traveling with him on the plane whether he ought to ``just go in''
and do something to stop the protesters.

The reply was: ``No, sir, let it simmer for a little while.'' Darned
good advice, although if he'd gone the other way, maybe the congressman
could have added, ``And be sure to bring a Bible.''

One other thing about that story --- it's an example of how Trump likes
to lace his rallies with anecdotes in which people call him ``sir.''
There were 11 ``sirs'' in the Tulsa speech alone.

Daniel Dale, a CNN reporter who's been following this tic for a long
time, theorized that ``sir'' was a hint that whatever anecdote Trump was
telling was actually fictional. But it's also pretty clear that the
president just loves stories in which people are addressing him as if he
were, say, a general.

Trump's been spending a lot of time trying to beat down that image of
him at West Point this month, leaving the stage with an old-guy totter
down the ramp. The fake news, he insisted, cut off all the film that
showed him running --- running! --- for the last 10 feet. ``I looked
very handsome,'' he observed to the crowd.

Later, Trump asked Melania what the reaction to his West Point speech
was. She assured him that the media wasn't saying much about his address
but ``they mention the fact that you may have Parkinson's disease.''

He referred to Melania as ``my wife,'' which is, I guess, nicer than
``the old ball and chain.'' Interesting, though, that she didn't feel
compelled to deliver any good news. Maybe when you have to live with an
ego that large, you try to chip away every little chance you get.

And she didn't call him ``sir.''

\emph{The Times is committed to publishing}
\href{https://www.nytimes3xbfgragh.onion/2019/01/31/opinion/letters/letters-to-editor-new-york-times-women.html}{\emph{a
diversity of letters}} \emph{to the editor. We'd like to hear what you
think about this or any of our articles. Here are some}
\href{https://help.nytimes3xbfgragh.onion/hc/en-us/articles/115014925288-How-to-submit-a-letter-to-the-editor}{\emph{tips}}\emph{.
And here's our email:}
\href{mailto:letters@NYTimes.com}{\emph{letters@NYTimes.com}}\emph{.}

\emph{Follow The New York Times Opinion section on}
\href{https://www.facebookcorewwwi.onion/nytopinion}{\emph{Facebook}}\emph{,}
\href{http://twitter.com/NYTOpinion}{\emph{Twitter (@NYTopinion)}}
\emph{and}
\href{https://www.instagram.com/nytopinion/}{\emph{Instagram}}\emph{.}

Advertisement

\protect\hyperlink{after-bottom}{Continue reading the main story}

\hypertarget{site-index}{%
\subsection{Site Index}\label{site-index}}

\hypertarget{site-information-navigation}{%
\subsection{Site Information
Navigation}\label{site-information-navigation}}

\begin{itemize}
\tightlist
\item
  \href{https://help.nytimes3xbfgragh.onion/hc/en-us/articles/115014792127-Copyright-notice}{©~2020~The
  New York Times Company}
\end{itemize}

\begin{itemize}
\tightlist
\item
  \href{https://www.nytco.com/}{NYTCo}
\item
  \href{https://help.nytimes3xbfgragh.onion/hc/en-us/articles/115015385887-Contact-Us}{Contact
  Us}
\item
  \href{https://www.nytco.com/careers/}{Work with us}
\item
  \href{https://nytmediakit.com/}{Advertise}
\item
  \href{http://www.tbrandstudio.com/}{T Brand Studio}
\item
  \href{https://www.nytimes3xbfgragh.onion/privacy/cookie-policy\#how-do-i-manage-trackers}{Your
  Ad Choices}
\item
  \href{https://www.nytimes3xbfgragh.onion/privacy}{Privacy}
\item
  \href{https://help.nytimes3xbfgragh.onion/hc/en-us/articles/115014893428-Terms-of-service}{Terms
  of Service}
\item
  \href{https://help.nytimes3xbfgragh.onion/hc/en-us/articles/115014893968-Terms-of-sale}{Terms
  of Sale}
\item
  \href{https://spiderbites.nytimes3xbfgragh.onion}{Site Map}
\item
  \href{https://help.nytimes3xbfgragh.onion/hc/en-us}{Help}
\item
  \href{https://www.nytimes3xbfgragh.onion/subscription?campaignId=37WXW}{Subscriptions}
\end{itemize}
