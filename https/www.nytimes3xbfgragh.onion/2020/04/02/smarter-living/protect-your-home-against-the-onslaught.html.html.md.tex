Sections

SEARCH

\protect\hyperlink{site-content}{Skip to
content}\protect\hyperlink{site-index}{Skip to site index}

\href{https://www.nytimes3xbfgragh.onion/section/smarter-living}{Smarter
Living}

\href{https://myaccount.nytimes3xbfgragh.onion/auth/login?response_type=cookie\&client_id=vi}{}

\href{https://www.nytimes3xbfgragh.onion/section/todayspaper}{Today's
Paper}

\href{/section/smarter-living}{Smarter Living}\textbar{}Protect Your
Home Against the Onslaught

\url{https://nyti.ms/2xIa8Ie}

\begin{itemize}
\item
\item
\item
\item
\item
\end{itemize}

\href{https://www.nytimes3xbfgragh.onion/spotlight/at-home?action=click\&pgtype=Article\&state=default\&region=TOP_BANNER\&context=at_home_menu}{At
Home}

\begin{itemize}
\tightlist
\item
  \href{https://www.nytimes3xbfgragh.onion/interactive/2020/08/27/arts/design/jackson-heights-queens-virtual-walk-tour.html?action=click\&pgtype=Article\&state=default\&region=TOP_BANNER\&context=at_home_menu}{Tour:
  Jackson Heights}
\item
  \href{https://www.nytimes3xbfgragh.onion/interactive/2020/at-home/even-more-reporters-editors-diaries-lists-recommendations.html?action=click\&pgtype=Article\&state=default\&region=TOP_BANNER\&context=at_home_menu}{Explore:
  Reporters' Google Docs}
\item
  \href{https://www.nytimes3xbfgragh.onion/2020/08/31/dining/hand-held-picnic-food.html?action=click\&pgtype=Article\&state=default\&region=TOP_BANNER\&context=at_home_menu}{Rethink:
  Your Picnic Menu}
\item
  \href{https://www.nytimes3xbfgragh.onion/article/choosing-a-pediatrician-children.html?action=click\&pgtype=Article\&state=default\&region=TOP_BANNER\&context=at_home_menu}{Choose:
  A Pediatrician}
\end{itemize}

Advertisement

\protect\hyperlink{after-top}{Continue reading the main story}

Supported by

\protect\hyperlink{after-sponsor}{Continue reading the main story}

\hypertarget{protect-your-home-against-the-onslaught}{%
\section{Protect Your Home Against the
Onslaught}\label{protect-your-home-against-the-onslaught}}

With your family at home 24 hours a day, consider these tips to keep
your appliances functioning, the mess to a minimum and the clutter at
bay.

\includegraphics{https://static01.graylady3jvrrxbe.onion/images/2020/04/25/smarter-living/02coronavirus-home-clutter/02coronavirus-home-clutter-articleLarge.jpg?quality=75\&auto=webp\&disable=upscale}

\href{https://www.nytimes3xbfgragh.onion/by/ronda-kaysen}{\includegraphics{https://static01.graylady3jvrrxbe.onion/images/2018/07/16/multimedia/author-ronda-kaysen/author-ronda-kaysen-thumbLarge-v2.png}}

By \href{https://www.nytimes3xbfgragh.onion/by/ronda-kaysen}{Ronda
Kaysen}

\begin{itemize}
\item
  April 2, 2020
\item
  \begin{itemize}
  \item
  \item
  \item
  \item
  \item
  \end{itemize}
\end{itemize}

As millions of Americans hunker down at home for yet another week of
social separation, a new reality is sinking in: All that bread baking
and crafts translate to an overwhelming amount of wear and tear on our
homes.

The kitchen is perpetually full of dishes, the living room is
overwhelmed by abandoned pillow forts, and the laundry baskets are
bafflingly full every day of the week.

``Homes are absolutely on overdrive,'' said Dan DiClerico,
\href{https://www.homeadvisor.com/r/about-dan-diclerico/}{a smart home
strategist for HomeAdvisor.} ``It's like having a newborn in the
house.''

We're using our homes differently now and it shows. The kitchen counter
may double as a home office. The living room may now serve as a
preschool playroom. Reset your thinking about how your home functions,
and it is possible to find a rhythm that reduces the amount of grunt
work you're doing to keep the household running.

\hypertarget{create-zones}{%
\subsection{Create Zones}\label{create-zones}}

Bring order to this new configuration by designating spots for specific
tasks. If the breakfast nook is now a tele-school classroom, dedicate
the space to that task. If space is tight, and the area still needs to
play double duty, clear the table of all schoolwork items before anyone
eats, storing the materials in a bin or basket on a shelf until they
need to be used again. (Of course, no one is coming over, so make do
with what you have in the house.)

``Keep things grouped and corralled and contained --- it helps,'' said
Clea Shearer, \href{https://thehomeedit.com/story/}{a co-owner of The
Home Edit}, a Nashville, Tenn., home organizing company. ``With no
system, it's a free-for-all.''

The same goes for the living or dining rooms. Designate certain areas
for play and others for lounging. Create a reading nook, and stock it
with a cozy blanket and plenty of books. Everyone in the household
should know that this is a space for quiet reading, and only such items
belong there.

If the children need a craft area, choose a table for the activity so
glue sticks and glitter do not end up on the coffee table or sofa. Set
specific times of day and locations for those messier activities, like
finger paints or Play-Doh, and clean up immediately after playtime. If
sticky hands do meet the wall, cleaning products like a Magic Eraser can
help.

\hypertarget{get-control-of-the-kitchen}{%
\subsection{Get Control of the
Kitchen}\label{get-control-of-the-kitchen}}

With everyone home all day, the kitchen may now feel like a 24-hour
restaurant. Put a stop to that. Set times for meals and snacks, so
everyone in the household eats together as much as possible. Make a rule
that anyone who eats at off hours needs to clean their dishes and the
counter before they leave the room.

To keep control of the dishes, limit how many the family uses each day.
Give each family member a set in the morning --- a plate, a bowl, a mug
and a glass --- and they are responsible for rinsing it after each meal
and reusing it throughout the day. With fewer dishes, you'll run the
dishwasher less often, extending the machine's life and your patience.

You are likely using your kitchen differently, with on-the-go breakfasts
replaced with morning pancakes, and takeout night swapped for slow
cooker creations. All this extra cooking means more dishes to wash and
more counters to wipe. Take a step back and reconsider the space and how
you use it now.

Start with the pantry. Items like flour or rolled oats that once lived
in the back may now be daily staples. Move those items to a more
accessible spot and rotate the stuff you're not using much anymore.

``If you're making pancakes every morning, maybe your skillet is kept
right on the burner,'' said Faith Roberson,
\href{https://www.organizewithfaith.com/about}{a Manhattan home
organizer.}

Plan meals ahead of time, taking advantage of multipurpose ingredients.
This will also help you use what you have, and limit your trips to the
supermarket. If you're chopping carrots for a recipe, chop up a few more
and store them in a tightly sealed container for tomorrow, or grate a
few to use in a carrot salad another day. Aim for dishes that freeze
well, too, like chili or stews: Double the recipe and freeze the
leftovers for another dinner the following week. By thinking ahead, you
reduce the work you'll do later, and make the most of the time you do
need to spend on the task at hand.

\hypertarget{give-the-appliances-a-break}{%
\subsection{Give the Appliances a
Break}\label{give-the-appliances-a-break}}

Your dishwasher, washing machine, dryer and water heater may all be
getting more use than they bargained for. Now is not the time for
anything to break down. You need them to be workhorses.

``It's not like everything is going to start breaking down at once,''
said Mr. DiClerico of HomeAdvisor, ``but it is important to stay on top
of preventive maintenance to avoid having to bring a repair man into the
house.''

Your dryer is probably the riskiest appliance because the vent needs to
be periodically cleaned of lint once or twice a year. Clogged dryer
vents can cause house fires. Normally, this is a task best left to
professionals, but Mr. DiClerico recommends using a
\href{https://www.consumerreports.org/cro/magazine/2013/01/claim-check-lint-lizard/index.htm}{Lint
Lizard,} a flexible vacuum cleaner attachment designed to suck up lint,
as a temporary alternative. Your washing machine should be able to
handle the extra loads. However, to sanitize it every few weeks, run an
empty cycle on hot with a cup of bleach.

If your dishwasher is getting extra use, sanitize it every few weeks by
running an empty cycle with a bowl of white vinegar on the bottom rack.
Clean the filter monthly --- or every two weeks, if it's running double
time. If you have a hood over your range, change the filter if you've
been doing a lot of cooking. (Check your manufacturers websites on how
to change filters.)

Your water heater may also be working extra hard. Read the instruction
manual for maintenance recommendations,
\href{https://www.diynetwork.com/how-to/skills-and-know-how/plumbing/how-to-drain-a-water-heater}{as
many heaters should be drained periodically.} This can usually be done
with a hose attachment at the bottom of the cylinder.

\hypertarget{limit-options}{%
\subsection{Limit Options}\label{limit-options}}

Less stuff means less to clean up. In a home with small children, limit
the available toys at any given time, setting the rest away in a closet
to cycle through later. With fewer toys available, you may find that the
children are more likely to focus on one for a longer period of time.

``Be deliberate about what is accessible and what is not,'' said Karri
Bowen-Poole, \href{https://www.smartplayrooms.com/}{the chief executive
of Smart Playrooms,} a Westchester design company. Store the toys in
easily accessible bins so the children can play (and tidy up)
independently. Label the bins with words or pictures to make cleanup
easier for small children.

Resist the lure of online impulse buys. More stuff means more clutter.
Instead, hunt for activities among the items that already exist in your
home. ``Recognizing that less is more can be beneficial,'' Ms.
Bowen-Poole said. ``Put a stapler out for a 6-year-old. Show them how to
make books.''

\hypertarget{set-schedules-and-dole-out-jobs}{%
\subsection{Set Schedules and Dole Out
Jobs}\label{set-schedules-and-dole-out-jobs}}

With no one ever leaving the house, days can bleed from one to the next,
making it difficult to keep on top of chores, activities and tasks. But
people crave order, especially at a time when nearly every routine has
been upended. To stay on top of all that needs to be done, add some
structure to the day, and to how the home will be used.

Figure out the chores and divide them among the members of the household
--- anyone who's old enough to walk is old enough to do something.
**``**Systems and schedules are going to save all of us,'' said
\href{https://www.reginaleeds.com/books}{Regina Leeds, a Los Angeles
organizer and the author of ``One Year to an Organized Life.''} You can
either rotate through responsibilities with a chart, or everyone can
choose ones they like and be responsible for getting those tasks done.
But agree as a household about how you will go about collectively
getting the work done.

Monday could be laundry day and Saturday bathroom cleaning day. Daily
household rules help, too. If you finish an activity, put it away. If
you take cereal out from the cabinet, put the box back and wash the bowl
immediately. Before bedtime, everyone in the family helps tidy up the
living spaces so they're ready for the next day.

``Do a pass at the end of each night,'' Ms. Shearer said. ``It takes
five minutes to tidy up your living space at the end of each day.''

Liken the task to making the bed in the morning --- end the day with a
clean slate and the next one will start more smoothly. Perhaps, with a
little order, the work will feel more manageable.

Advertisement

\protect\hyperlink{after-bottom}{Continue reading the main story}

\hypertarget{site-index}{%
\subsection{Site Index}\label{site-index}}

\hypertarget{site-information-navigation}{%
\subsection{Site Information
Navigation}\label{site-information-navigation}}

\begin{itemize}
\tightlist
\item
  \href{https://help.nytimes3xbfgragh.onion/hc/en-us/articles/115014792127-Copyright-notice}{©~2020~The
  New York Times Company}
\end{itemize}

\begin{itemize}
\tightlist
\item
  \href{https://www.nytco.com/}{NYTCo}
\item
  \href{https://help.nytimes3xbfgragh.onion/hc/en-us/articles/115015385887-Contact-Us}{Contact
  Us}
\item
  \href{https://www.nytco.com/careers/}{Work with us}
\item
  \href{https://nytmediakit.com/}{Advertise}
\item
  \href{http://www.tbrandstudio.com/}{T Brand Studio}
\item
  \href{https://www.nytimes3xbfgragh.onion/privacy/cookie-policy\#how-do-i-manage-trackers}{Your
  Ad Choices}
\item
  \href{https://www.nytimes3xbfgragh.onion/privacy}{Privacy}
\item
  \href{https://help.nytimes3xbfgragh.onion/hc/en-us/articles/115014893428-Terms-of-service}{Terms
  of Service}
\item
  \href{https://help.nytimes3xbfgragh.onion/hc/en-us/articles/115014893968-Terms-of-sale}{Terms
  of Sale}
\item
  \href{https://spiderbites.nytimes3xbfgragh.onion}{Site Map}
\item
  \href{https://help.nytimes3xbfgragh.onion/hc/en-us}{Help}
\item
  \href{https://www.nytimes3xbfgragh.onion/subscription?campaignId=37WXW}{Subscriptions}
\end{itemize}
