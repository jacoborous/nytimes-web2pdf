Sections

SEARCH

\protect\hyperlink{site-content}{Skip to
content}\protect\hyperlink{site-index}{Skip to site index}

\href{https://www.nytimes3xbfgragh.onion/section/health}{Health}

\href{https://myaccount.nytimes3xbfgragh.onion/auth/login?response_type=cookie\&client_id=vi}{}

\href{https://www.nytimes3xbfgragh.onion/section/todayspaper}{Today's
Paper}

\href{/section/health}{Health}\textbar{}Trump Calls This Drug a `Game
Changer.' Doctors Aren't So Sure.

\url{https://nyti.ms/34IoZ1G}

\begin{itemize}
\item
\item
\item
\item
\item
\end{itemize}

\href{https://www.nytimes3xbfgragh.onion/news-event/coronavirus?action=click\&pgtype=Article\&state=default\&region=TOP_BANNER\&context=storylines_menu}{The
Coronavirus Outbreak}

\begin{itemize}
\tightlist
\item
  live\href{https://www.nytimes3xbfgragh.onion/2020/08/01/world/coronavirus-covid-19.html?action=click\&pgtype=Article\&state=default\&region=TOP_BANNER\&context=storylines_menu}{Latest
  Updates}
\item
  \href{https://www.nytimes3xbfgragh.onion/interactive/2020/us/coronavirus-us-cases.html?action=click\&pgtype=Article\&state=default\&region=TOP_BANNER\&context=storylines_menu}{Maps
  and Cases}
\item
  \href{https://www.nytimes3xbfgragh.onion/interactive/2020/science/coronavirus-vaccine-tracker.html?action=click\&pgtype=Article\&state=default\&region=TOP_BANNER\&context=storylines_menu}{Vaccine
  Tracker}
\item
  \href{https://www.nytimes3xbfgragh.onion/interactive/2020/07/29/us/schools-reopening-coronavirus.html?action=click\&pgtype=Article\&state=default\&region=TOP_BANNER\&context=storylines_menu}{What
  School May Look Like}
\item
  \href{https://www.nytimes3xbfgragh.onion/live/2020/07/31/business/stock-market-today-coronavirus?action=click\&pgtype=Article\&state=default\&region=TOP_BANNER\&context=storylines_menu}{Economy}
\end{itemize}

Advertisement

\protect\hyperlink{after-top}{Continue reading the main story}

Supported by

\protect\hyperlink{after-sponsor}{Continue reading the main story}

\hypertarget{trump-calls-this-drug-a-game-changer-doctors-arent-so-sure}{%
\section{Trump Calls This Drug a `Game Changer.' Doctors Aren't So
Sure.}\label{trump-calls-this-drug-a-game-changer-doctors-arent-so-sure}}

The malaria drug hydroxychloroquine has become a political litmus test.
But doctors on the front lines of coronavirus say it's just another tool
in desperate times.

\includegraphics{https://static01.graylady3jvrrxbe.onion/images/2020/04/16/science/00VIRUS-HCQ1/00VIRUS-HCQ1-articleLarge.jpg?quality=75\&auto=webp\&disable=upscale}

By \href{https://www.nytimes3xbfgragh.onion/by/katie-thomas}{Katie
Thomas}

\begin{itemize}
\item
  April 17, 2020
\item
  \begin{itemize}
  \item
  \item
  \item
  \item
  \item
  \end{itemize}
\end{itemize}

Just a month ago, Dr. Bushra Mina had no playbook to treat patients who
were arriving with coronavirus at Lenox Hill Hospital in Manhattan.

His first encounter was with an older man whose health declined quickly.
No drugs were approved to treat this highly infectious virus, and there
was little Dr. Mina could do but provide supportive care.

Weeks later, Dr. Mina, the chief of pulmonary medicine at Lenox Hill, is
on the 16th revision of guidelines shared among doctors as they assess
the ways the virus is emerging and advancing in patients, and what
possible treatments should be applied.

Now most
\href{https://www.nytimes3xbfgragh.onion/news-event/coronavirus?action=click\&pgtype=Article\&state=default\&module=STYLN_coronahub\&variant=show\&region=header\&context=menu}{Covid-19}
patients at Lenox Hill who are not on the verge of dying receive a
five-day regimen of hydroxychloroquine, the long-used malaria drug that
President Trump has repeatedly promoted as a ``what have you got to
lose'' remedy. While his own top health officials are more cautious ---
noting there is limited evidence about the drug's benefits --- doctors
across the country have been prescribing the drug for weeks.

Dr. Mina doesn't know if the hydroxychloroquine is helping his patients.
He is well aware that there are no rigorous clinical trials showing that
the drug works. But he can't wait for the evidence to come in, he said,
when people are dying.

``I think it's a battle, and your options are very limited,'' Dr. Mina
said. ``You're really looking for what you can do with whatever evidence
you have.''

Hydroxychloroquine and a related drug, chloroquine, have been used for
decades to treat and prevent malaria, and hydroxychloroquine has been
used by people with lupus and rheumatoid arthritis because it is known
to calm the immune system. In laboratory tests, it has been shown to
block the coronavirus from invading cells, although it hasn't been
proven in human trials. The drugs are not recommended for people who
have abnormal heart rhythms because it can make them worse.

Almost overnight, the hard-to-pronounce drug has become a litmus test
for support of the president. Conservative talk show hosts
\href{https://www.wsj.com/articles/conservative-group-pushes-for-fda-approval-of-drug-to-treat-coronavirus-11585229415?redirect=amp\#click=https://t.co/5rakc6KA4h}{and
supporters} like his personal lawyer, Rudy Giuliani, praise the drug's
potential, while political opponents have scoffed at what they see as
another way that Mr. Trump has undermined scientific inquiry.

For weeks now, doctors around the country have been giving the drug to
patients at various stages of the disease, and as a preventive measure
to some if they've been exposed by family members or in health care
settings. But even after treating hundreds of patients with the
antimalarial drug, the doctors interviewed did not report clear results
or remarkable recoveries that can be traced to the drug.

In addition to Lenox Hill, other major hospitals in outbreak hot spots
are using hydroxychloroquine as part of their protocol. They include
\href{https://www.massgeneral.org/assets/MGH/pdf/news/coronavirus/mass-general-COVID-19-treatment-guidance.pdf}{Massachusetts
General Hospital} in Boston and Rush University Medical Center in
Chicago, which each recommend it on a case-by-case basis and when
clinical trials are not feasible, and Ochsner Health in Louisiana, which
administers it routinely to coronavirus patients.

\includegraphics{https://static01.graylady3jvrrxbe.onion/images/2020/04/16/science/00VIRUS-HCQ2/merlin_171620142_9fa85116-ba77-4623-8026-f6f00c81ec71-articleLarge.jpg?quality=75\&auto=webp\&disable=upscale}

This week,
\href{https://www.medrxiv.org/content/10.1101/2020.04.10.20060558v1}{researchers
in China} made public the results of another study of
hydroxychloroquine, of 150 hospitalized patients. The study, which has
not been peer-reviewed, found that patients who were given the drug did
not fare significantly better than those who did not receive it, nor did
they experience more serious side effects.

Some medical societies have recently recommended against its regular
use. The
\href{https://www.idsociety.org/practice-guideline/covid-19-guideline-treatment-and-management/}{Infectious
Diseases Society of America} recently advised that use of
hydroxychloroquine be limited to clinical trials,
\href{https://www.thoracic.org/about/newsroom/press-releases/journal/2020/ats-publishes-new-guidance-on-covid-19-management.php}{as
did the American Thoracic Society}.

\hypertarget{latest-updates-global-coronavirus-outbreak}{%
\section{\texorpdfstring{\href{https://www.nytimes3xbfgragh.onion/2020/08/01/world/coronavirus-covid-19.html?action=click\&pgtype=Article\&state=default\&region=MAIN_CONTENT_1\&context=storylines_live_updates}{Latest
Updates: Global Coronavirus
Outbreak}}{Latest Updates: Global Coronavirus Outbreak}}\label{latest-updates-global-coronavirus-outbreak}}

Updated 2020-08-02T07:42:09.613Z

\begin{itemize}
\tightlist
\item
  \href{https://www.nytimes3xbfgragh.onion/2020/08/01/world/coronavirus-covid-19.html?action=click\&pgtype=Article\&state=default\&region=MAIN_CONTENT_1\&context=storylines_live_updates\#link-34047410}{The
  U.S. reels as July cases more than double the total of any other
  month.}
\item
  \href{https://www.nytimes3xbfgragh.onion/2020/08/01/world/coronavirus-covid-19.html?action=click\&pgtype=Article\&state=default\&region=MAIN_CONTENT_1\&context=storylines_live_updates\#link-780ec966}{Top
  U.S. officials work to break an impasse over the federal jobless
  benefit.}
\item
  \href{https://www.nytimes3xbfgragh.onion/2020/08/01/world/coronavirus-covid-19.html?action=click\&pgtype=Article\&state=default\&region=MAIN_CONTENT_1\&context=storylines_live_updates\#link-2bc8948}{Its
  outbreak untamed, Melbourne goes into even greater lockdown.}
\end{itemize}

\href{https://www.nytimes3xbfgragh.onion/2020/08/01/world/coronavirus-covid-19.html?action=click\&pgtype=Article\&state=default\&region=MAIN_CONTENT_1\&context=storylines_live_updates}{See
more updates}

More live coverage:
\href{https://www.nytimes3xbfgragh.onion/live/2020/07/31/business/stock-market-today-coronavirus?action=click\&pgtype=Article\&state=default\&region=MAIN_CONTENT_1\&context=storylines_live_updates}{Markets}

At the Henry Ford Health System in Detroit,
\href{https://www.henryford.com/news/2020/04/hfhs-leads-national-study-to-determine-drugs-effectiveness-in-preventing-covid19}{researchers
are beginning a 3,000-person} clinical trial that will test whether
hydroxychloroquine can prevent infection in health care employees and
other front-line workers. But they have also given it to sick patients,
outside of a trial, when there is little other hope.

``In many ways we feel driven to help patients who are in front of us
--- today --- in the hour of their greatest need,'' said Dr. Steven
Kalkanis, the chief academic officer of Henry Ford Health System. ``And
there is a clamoring to use whatever we have at our disposal.''

But outside of a clinical trial, it can be hard to assess the drug's
value, especially when it is being given to a variety of patients, of
different ages and medical conditions, and at different points in their
disease. Based on the hospital's experience, Dr. Kalkanis said, the
drug's benefits do not appear to be a slam dunk.

``For every anecdotal success story, we hear one where a patient
unfortunately died,'' he said. ``It's not coalescing around, `Oh my
gosh, this is the answer.'''

The
\href{https://www.nytimes3xbfgragh.onion/article/coronavirus-hydroxychloroquine-malaria.html}{drug
has generated excitement} because a laboratory study, with cultured
cells, found that chloroquine could block the coronavirus from invading
cells, which it must do to replicate and cause illness. But drugs that
show promise in the laboratory do not always translate to success in the
human body, and other studies have found that it failed to prevent or
treat influenza and other viral illnesses.

Early reports from doctors in China and
\href{https://www.sciencedirect.com/science/article/pii/S0924857920300996}{France}
have said that hydroxychloroquine, sometimes combined with the
antibiotic azithromycin, seemed to help patients. But the studies were
small and did not use proper control groups --- patients carefully
selected to match those in the experimental group but who are not given
the drug being tested. Research involving few patients and no controls
cannot determine whether a drug works.

At most hospitals in the Ochsner Health system in Louisiana, including
those in New Orleans, infected patients are routinely given a course of
hydroxychloroquine. Patients in the intensive care unit are also given
the drug if they have not received it earlier in their illness.

Dr. Leo Seoane, the chief academic officer at Ochsner Health, said the
health system had declined to participate in research trials that
included a placebo arm, in which some patients would be selected not to
receive the drug. ``We didn't think it was ethical at this point in the
crisis to withhold the therapies that could be beneficial,'' he said.

\href{https://www.nytimes3xbfgragh.onion/news-event/coronavirus?action=click\&pgtype=Article\&state=default\&region=MAIN_CONTENT_3\&context=storylines_faq}{}

\hypertarget{the-coronavirus-outbreak-}{%
\subsubsection{The Coronavirus Outbreak
›}\label{the-coronavirus-outbreak-}}

\hypertarget{frequently-asked-questions}{%
\paragraph{Frequently Asked
Questions}\label{frequently-asked-questions}}

Updated July 27, 2020

\begin{itemize}
\item ~
  \hypertarget{should-i-refinance-my-mortgage}{%
  \paragraph{Should I refinance my
  mortgage?}\label{should-i-refinance-my-mortgage}}

  \begin{itemize}
  \tightlist
  \item
    \href{https://www.nytimes3xbfgragh.onion/article/coronavirus-money-unemployment.html?action=click\&pgtype=Article\&state=default\&region=MAIN_CONTENT_3\&context=storylines_faq}{It
    could be a good idea,} because mortgage rates have
    \href{https://www.nytimes3xbfgragh.onion/2020/07/16/business/mortgage-rates-below-3-percent.html?action=click\&pgtype=Article\&state=default\&region=MAIN_CONTENT_3\&context=storylines_faq}{never
    been lower.} Refinancing requests have pushed mortgage applications
    to some of the highest levels since 2008, so be prepared to get in
    line. But defaults are also up, so if you're thinking about buying a
    home, be aware that some lenders have tightened their standards.
  \end{itemize}
\item ~
  \hypertarget{what-is-school-going-to-look-like-in-september}{%
  \paragraph{What is school going to look like in
  September?}\label{what-is-school-going-to-look-like-in-september}}

  \begin{itemize}
  \tightlist
  \item
    It is unlikely that many schools will return to a normal schedule
    this fall, requiring the grind of
    \href{https://www.nytimes3xbfgragh.onion/2020/06/05/us/coronavirus-education-lost-learning.html?action=click\&pgtype=Article\&state=default\&region=MAIN_CONTENT_3\&context=storylines_faq}{online
    learning},
    \href{https://www.nytimes3xbfgragh.onion/2020/05/29/us/coronavirus-child-care-centers.html?action=click\&pgtype=Article\&state=default\&region=MAIN_CONTENT_3\&context=storylines_faq}{makeshift
    child care} and
    \href{https://www.nytimes3xbfgragh.onion/2020/06/03/business/economy/coronavirus-working-women.html?action=click\&pgtype=Article\&state=default\&region=MAIN_CONTENT_3\&context=storylines_faq}{stunted
    workdays} to continue. California's two largest public school
    districts --- Los Angeles and San Diego --- said on July 13, that
    \href{https://www.nytimes3xbfgragh.onion/2020/07/13/us/lausd-san-diego-school-reopening.html?action=click\&pgtype=Article\&state=default\&region=MAIN_CONTENT_3\&context=storylines_faq}{instruction
    will be remote-only in the fall}, citing concerns that surging
    coronavirus infections in their areas pose too dire a risk for
    students and teachers. Together, the two districts enroll some
    825,000 students. They are the largest in the country so far to
    abandon plans for even a partial physical return to classrooms when
    they reopen in August. For other districts, the solution won't be an
    all-or-nothing approach.
    \href{https://bioethics.jhu.edu/research-and-outreach/projects/eschool-initiative/school-policy-tracker/}{Many
    systems}, including the nation's largest, New York City, are
    devising
    \href{https://www.nytimes3xbfgragh.onion/2020/06/26/us/coronavirus-schools-reopen-fall.html?action=click\&pgtype=Article\&state=default\&region=MAIN_CONTENT_3\&context=storylines_faq}{hybrid
    plans} that involve spending some days in classrooms and other days
    online. There's no national policy on this yet, so check with your
    municipal school system regularly to see what is happening in your
    community.
  \end{itemize}
\item ~
  \hypertarget{is-the-coronavirus-airborne}{%
  \paragraph{Is the coronavirus
  airborne?}\label{is-the-coronavirus-airborne}}

  \begin{itemize}
  \tightlist
  \item
    The coronavirus
    \href{https://www.nytimes3xbfgragh.onion/2020/07/04/health/239-experts-with-one-big-claim-the-coronavirus-is-airborne.html?action=click\&pgtype=Article\&state=default\&region=MAIN_CONTENT_3\&context=storylines_faq}{can
    stay aloft for hours in tiny droplets in stagnant air}, infecting
    people as they inhale, mounting scientific evidence suggests. This
    risk is highest in crowded indoor spaces with poor ventilation, and
    may help explain super-spreading events reported in meatpacking
    plants, churches and restaurants.
    \href{https://www.nytimes3xbfgragh.onion/2020/07/06/health/coronavirus-airborne-aerosols.html?action=click\&pgtype=Article\&state=default\&region=MAIN_CONTENT_3\&context=storylines_faq}{It's
    unclear how often the virus is spread} via these tiny droplets, or
    aerosols, compared with larger droplets that are expelled when a
    sick person coughs or sneezes, or transmitted through contact with
    contaminated surfaces, said Linsey Marr, an aerosol expert at
    Virginia Tech. Aerosols are released even when a person without
    symptoms exhales, talks or sings, according to Dr. Marr and more
    than 200 other experts, who
    \href{https://academic.oup.com/cid/article/doi/10.1093/cid/ciaa939/5867798}{have
    outlined the evidence in an open letter to the World Health
    Organization}.
  \end{itemize}
\item ~
  \hypertarget{what-are-the-symptoms-of-coronavirus}{%
  \paragraph{What are the symptoms of
  coronavirus?}\label{what-are-the-symptoms-of-coronavirus}}

  \begin{itemize}
  \tightlist
  \item
    Common symptoms
    \href{https://www.nytimes3xbfgragh.onion/article/symptoms-coronavirus.html?action=click\&pgtype=Article\&state=default\&region=MAIN_CONTENT_3\&context=storylines_faq}{include
    fever, a dry cough, fatigue and difficulty breathing or shortness of
    breath.} Some of these symptoms overlap with those of the flu,
    making detection difficult, but runny noses and stuffy sinuses are
    less common.
    \href{https://www.nytimes3xbfgragh.onion/2020/04/27/health/coronavirus-symptoms-cdc.html?action=click\&pgtype=Article\&state=default\&region=MAIN_CONTENT_3\&context=storylines_faq}{The
    C.D.C. has also} added chills, muscle pain, sore throat, headache
    and a new loss of the sense of taste or smell as symptoms to look
    out for. Most people fall ill five to seven days after exposure, but
    symptoms may appear in as few as two days or as many as 14 days.
  \end{itemize}
\item ~
  \hypertarget{does-asymptomatic-transmission-of-covid-19-happen}{%
  \paragraph{Does asymptomatic transmission of Covid-19
  happen?}\label{does-asymptomatic-transmission-of-covid-19-happen}}

  \begin{itemize}
  \tightlist
  \item
    So far, the evidence seems to show it does. A widely cited
    \href{https://www.nature.com/articles/s41591-020-0869-5}{paper}
    published in April suggests that people are most infectious about
    two days before the onset of coronavirus symptoms and estimated that
    44 percent of new infections were a result of transmission from
    people who were not yet showing symptoms. Recently, a top expert at
    the World Health Organization stated that transmission of the
    coronavirus by people who did not have symptoms was ``very rare,''
    \href{https://www.nytimes3xbfgragh.onion/2020/06/09/world/coronavirus-updates.html?action=click\&pgtype=Article\&state=default\&region=MAIN_CONTENT_3\&context=storylines_faq\#link-1f302e21}{but
    she later walked back that statement.}
  \end{itemize}
\end{itemize}

But he acknowledged that even though the hospital gives the drug to
nearly everyone who is admitted, the percentage of people who end up in
the intensive care unit --- about a third of those admitted --- is
similar to reports in other places where the drug is not used. ``From a
gut feeling, it's hard for me to know that it is having an impact,''
said Dr. Seoane, who is also a pulmonologist and critical care
physician.

Dr. Sarah Doernberg, an associate professor of infectious disease at
UCSF Medical Center in San Francisco, said she was selective about which
patients were given the drug. ``It's not an established therapy that
everyone should get,'' she said. ``I feel pretty strongly about that.''

A study of its effects in a clinical trial, versus a placebo, was
needed, she said. ``We can figure out the answers to this question, so
that when people get sick months from now, we'll know whether it will
work.''

Those trials are getting underway now,
with\href{https://clinicaltrials.gov/ct2/results?cond=COVID\&term=hydroxychloroquine\&cntry=\&state=\&city=\&dist=}{more
than 100 studies of hydroxychloroquine} in patients with Covid-19 posted
to a federal clinical trials registry.

A placebo-controlled trial financed by the National Institutes of Health
\href{https://www.nih.gov/news-events/news-releases/nih-clinical-trial-hydroxychloroquine-potential-therapy-covid-19-begins}{began
enrolling patients last week}at Vanderbilt University Medical Center in
Nashville. That trial aims to enroll more than 500 people who have been
hospitalized. Several other institutions around the country, including
NYU Langone Health in New York, are testing whether the drug can halt or
prevent the infection in people who are at high risk of getting it, or
have been exposed.

The drug is also being dispersed more loosely through the Strategic
National Stockpile. Hospitals that administer drugs from the stockpile
must report on the patients who use them, but not through a formal
clinical trial.

Manufacturers have donated millions of pills to the stockpile, and are
ramping up production. But Mr. Trump's promotion of the drugs
\href{https://www.nytimes3xbfgragh.onion/2020/03/20/health/coronavirus-chloroquine-trump.html}{has
also led to shortages}, and people who rely on hydroxychloroquine ---
like those with lupus and rheumatoid arthritis --- have had trouble
refilling their prescriptions.

In an interview Thursday
\href{https://www.washingtonpost.com/washington-post-live/2020/04/16/transcript-confronting-covid-19/}{with
the Washington Post}, the F.D.A.'s commissioner, Dr. Stephen M. Hahn,
said he has not felt political pressure to favor hydroxychloroquine. ``I
can promise the American people that F.D.A. will use science and data to
drive our decisions, always,'' he said.

Those who favor conducting trials point to
\href{https://www.bloomberg.com/news/articles/2019-05-30/too-many-medicines-simply-don-t-work}{several
previous drugs or therapies} that were believed to show a benefit, until
more evidence revealed the opposite. In the 1990s, for example,
\href{https://www.cancernetwork.com/breast-cancer/health-insurance-coverage-autologous-bone-marrow-transplantation-breast-cancer}{some
states required insurers to cover} stem cell transplants and high-dose
chemotherapy treatments for breast cancer, under pressure from patient
groups and others. But those interventions were later
\href{https://www.ncbi.nlm.nih.gov/pmc/articles/PMC3393031/}{shown not
to be any better than less-invasive treatments}.

Another risk, some said, was that if a drug were too readily available
--- or portrayed in too positive a light --- people may not want to
chance enrollment in a trial that risks getting a placebo, and not the
drug.

Dr. David Boulware of the University of Minnesota said enrollment is
slowing in the national trial he
\href{https://med.umn.edu/news-events/covid-19-clinical-trial-launches-university-minnesota}{is}overseeing
of up to 1,500 people to test whether hydroxychloroquine works
preventively.

He does not know why participation is slowing, but he added that Mr.
Trump's message is not helping.

``He's just saying, `You should take it, I should take it, everyone
should take it,''' Dr. Boulware said. ``If he was promoting science and
promoting research, we would have had an answer weeks ago.''

Advertisement

\protect\hyperlink{after-bottom}{Continue reading the main story}

\hypertarget{site-index}{%
\subsection{Site Index}\label{site-index}}

\hypertarget{site-information-navigation}{%
\subsection{Site Information
Navigation}\label{site-information-navigation}}

\begin{itemize}
\tightlist
\item
  \href{https://help.nytimes3xbfgragh.onion/hc/en-us/articles/115014792127-Copyright-notice}{©~2020~The
  New York Times Company}
\end{itemize}

\begin{itemize}
\tightlist
\item
  \href{https://www.nytco.com/}{NYTCo}
\item
  \href{https://help.nytimes3xbfgragh.onion/hc/en-us/articles/115015385887-Contact-Us}{Contact
  Us}
\item
  \href{https://www.nytco.com/careers/}{Work with us}
\item
  \href{https://nytmediakit.com/}{Advertise}
\item
  \href{http://www.tbrandstudio.com/}{T Brand Studio}
\item
  \href{https://www.nytimes3xbfgragh.onion/privacy/cookie-policy\#how-do-i-manage-trackers}{Your
  Ad Choices}
\item
  \href{https://www.nytimes3xbfgragh.onion/privacy}{Privacy}
\item
  \href{https://help.nytimes3xbfgragh.onion/hc/en-us/articles/115014893428-Terms-of-service}{Terms
  of Service}
\item
  \href{https://help.nytimes3xbfgragh.onion/hc/en-us/articles/115014893968-Terms-of-sale}{Terms
  of Sale}
\item
  \href{https://spiderbites.nytimes3xbfgragh.onion}{Site Map}
\item
  \href{https://help.nytimes3xbfgragh.onion/hc/en-us}{Help}
\item
  \href{https://www.nytimes3xbfgragh.onion/subscription?campaignId=37WXW}{Subscriptions}
\end{itemize}
