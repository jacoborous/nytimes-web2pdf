Sections

SEARCH

\protect\hyperlink{site-content}{Skip to
content}\protect\hyperlink{site-index}{Skip to site index}

\href{https://myaccount.nytimes3xbfgragh.onion/auth/login?response_type=cookie\&client_id=vi}{}

\href{https://www.nytimes3xbfgragh.onion/section/todayspaper}{Today's
Paper}

\href{/section/opinion}{Opinion}\textbar{}Trump Keeps Putting the Lives
of Lupus Patients at Risk

\url{https://nyti.ms/2JI9eOJ}

\begin{itemize}
\item
\item
\item
\item
\item
\end{itemize}

Advertisement

\protect\hyperlink{after-top}{Continue reading the main story}

\href{/section/opinion}{Opinion}

Supported by

\protect\hyperlink{after-sponsor}{Continue reading the main story}

disability

\hypertarget{trump-keeps-putting-the-lives-of-lupus-patients-at-risk}{%
\section{Trump Keeps Putting the Lives of Lupus Patients at
Risk}\label{trump-keeps-putting-the-lives-of-lupus-patients-at-risk}}

We now have to deal with a shortage of hydroxychloroquine, the drug that
has been keeping me alive for more than two decades.

By Olga Lucia Torres

Ms. Torres teaches narrative medicine.

\begin{itemize}
\item
  April 6, 2020
\item
  \begin{itemize}
  \item
  \item
  \item
  \item
  \item
  \end{itemize}
\end{itemize}

\includegraphics{https://static01.graylady3jvrrxbe.onion/images/2020/04/03/opinion/03disability-torres/03disability-torres-articleLarge.jpg?quality=75\&auto=webp\&disable=upscale}

On March 19, I woke up, made coffee, took my usual 10 pills and turned
on CNN, hoping to see Gov. Andrew Cuomo. I got President Trump instead.
He was talking about the drug
\href{https://www.nytimes3xbfgragh.onion/2020/03/19/health/coronavirus-drugs-chloroquine.html}{hydroxychloroquine
as a potential treatment} for Covid-19. The drug, he said, without
providing any evidence, showed ``tremendous promise.'' My cheeks flushed
with fear and rage.

``I think it's going to be great,'' he went on.

My life, and that of millions of other people who depend on
hydroxychloroquine to treat lupus, changed in that brief moment. Our
lives were suddenly more at risk.

I ran to my computer, almost spilling hot coffee on myself, and looked
up the trials Mr. Trump had referred to. I could only find two small
studies, one from China and one from France, that had proven
inconclusive.

In the days that followed, as the panicked nation started to hoard
hydroxychloroquine, I knew that a run on the drug that had saved my life
could become a death sentence. The American Society of Health-System
Pharmacists soon
\href{https://www.ashp.org/Drug-Shortages/Current-Shortages/Drug-Shortage-Detail.aspx?id=646}{reported
a shortage} of hydroxychloroquine. There have even been
\href{https://www.nytimes3xbfgragh.onion/2020/03/24/business/doctors-buying-coronavirus-drugs.html}{reports
of physicians hoarding the drug} for themselves, which the patient in me
hopes isn't true.

Despite weeks of warnings from medical professionals, including Dr.
Anthony Fauci, the director of the National Institute of Allergy and
Infectious Diseases, that the drug is still unproven, the
\href{https://www.nytimes3xbfgragh.onion/2020/04/05/us/politics/trump-hydroxychloroquine-coronavirus.html}{president
continues to stun experts by defying all scientific advice}. ``What do
you have to lose?'' Mr. Trump said in a press briefing on Saturday. ``I
really think they should take it. But it's their choice. And it's their
doctor's choice or the doctors in the hospital. But hydroxychloroquine.
Try it, if you'd like.''

As a person with lupus, rheumatoid arthritis, three other lesser known
autoimmune diseases, chronic asthma and other health issues (I'm a bit
of an overachiever), I fall into that small percentage of people who are
most likely to die from the coronavirus. For the last two months I have
been isolating in our tiny New York City apartment, terrified that I
would become infected. So far, I have been lucky; I've not been exposed
to it. But now I have to worry about dying from a lack of
hydroxychloroquine.

Back in 1997 I was a young public defender, one of the original eight
attorneys who opened the Bronx Defenders nonprofit. I was working day
and night, too busy to worry when I discovered my hair falling out; I
chalked it up to stress. When my knees started to swell, I figured it
was from standing in the courtroom all day. It wasn't until I couldn't
get out of bed that I finally sought medical attention. I was given a
diagnosis of lupus and immediately put on hydroxychloroquine, along with
some other stronger medications to help calm the flare.

Since then, hydroxychloroquine has been my saving grace. I've been
taking it for more than 22 years. I even took the medication while I was
pregnant. Conclusive studies --- not the kind Mr. Trump is citing ---
found that keeping pregnant women with lupus on hydroxychloroquine
stabilizes the mother and consequently the fetus.

I'm not trying to say that my life matters more than others. But I do
have a young daughter and loving husband, and while I can't practice law
anymore, I do teach narrative medicine, an emerging field in health
care, at two medical schools and try to make a positive difference in
the world. I am not ready for my flame to be snuffed out. Whoever says
that old and sick people would want to die to save our economy is dead
wrong.

On March 29, the Food and Drug Administration issued an emergency
authorization for hydroxychloroquine to be used as a form of coronavirus
treatment. The only silver lining is that it has led to two drug
companies
\href{https://www.hhs.gov/about/news/2020/03/29/hhs-accepts-donations-of-medicine-to-strategic-national-stockpile-as-possible-treatments-for-covid-19-patients.html}{donating
tens of millions of doses} of both hydroxychloroquine and chloroquine to
the Strategic National Stockpile for emergency and trial use. The
Department of Health and Human Services is expected to receive more
donations, and will hopefully grant more access to patients like me who
rely on the drug on a daily basis.

And to help address the shortage of the medication in the meantime,
various organizations have started campaigns demanding that elected
officials protect the supply of the medicine for lupus patients.

I understand the urge to get the medication. But hydroxychloroquine
isn't an over-the-counter drug. The side effects are serious: severe
cardiac toxicity, retinal damage, even permanent blindness. So far
\href{https://couriernewsroom.com/2020/03/23/arizona-man-dies-drug-trump-touted-coronavirus-cure/}{at
least one man has died} (his wife became critically ill) in the United
States after taking a chloroquine phosphate product used to treat
parasites in fish.

Since the F.D.A. granted the emergency approval, friends have sent me
videos of people claiming to be doctors explaining how to take
hydroxychloroquine to cure Covid-19. More terrified people are
self-medicating and putting their lives at risk. Even President Jair
Bolsonaro of Brazil got in trouble for uploading a video to Facebook,
Instagram, Twitter and YouTube in which he says that hydroxychloroquine
is an effective treatment for Covid-19. The companies said the videos
violated their policies of disseminating misleading and harmful
information to the public.

I'm not alone in my concern. Lupus patients all over the world are
panicked about hydroxychloroquine shortages. There have been multiple
reports of lupus patients not being able to fill their prescriptions.
It's not that we don't want the medicine to work on coronavirus
patients. To the contrary. If the medication were studied in proper drug
trials and found to be effective, then enough of the drug could be
manufactured for both lupus and Covid-19 patients. Hopefully, with the
F.D.A. emergency approval, that can happen.

For now, I will wake up, make coffee, take my 10 morning pills, turn on
CNN, hoping to see Governor Cuomo not President Trump, and pray to God I
don't get Covid-19 and that I can get my hydroxychloroquine filled. My
daughter's smile is so beautiful. I want to continue seeing it.

Olga Lucia Torres (@TheOlgaTorres) is a former defense attorney who
teaches narrative medicine at CUNY School of Medicine and Columbia
University.

\emph{Disability is a series of essays, art and opinion by and about
people living with disabilities.}

\emph{\textbf{NOW IN PRINT}}*:
``\emph{\href{https://www.aboutusbook.com/}{\emph{About Us: Essays From
the Disability Series of The New York Times}}},'' edited by Peter
Catapano and Rosemarie Garland-Thomson, published by Liveright, a
collection of 60 essays from this series is now available in book,
e-book and audiobook. T*

\emph{he Times is committed to publishing}
\href{https://www.nytimes3xbfgragh.onion/2019/01/31/opinion/letters/letters-to-editor-new-york-times-women.html}{\emph{a
diversity of letters}} \emph{to the editor. We'd like to hear what you
think about this or any of our articles. Here are some}
\href{https://help.nytimes3xbfgragh.onion/hc/en-us/articles/115014925288-How-to-submit-a-letter-to-the-editor}{\emph{tips}}\emph{.
And here's our email:}
\href{mailto:letters@NYTimes.com}{\emph{letters@NYTimes.com}}\emph{.}

\emph{Follow The New York Times Opinion section on}
\href{https://www.facebookcorewwwi.onion/nytopinion}{\emph{Facebook}}\emph{,}
\href{http://twitter.com/NYTOpinion}{\emph{Twitter (@NYTopinion)}}
\emph{and}
\href{https://www.instagram.com/nytopinion/}{\emph{Instagram}}\emph{.}

Advertisement

\protect\hyperlink{after-bottom}{Continue reading the main story}

\hypertarget{site-index}{%
\subsection{Site Index}\label{site-index}}

\hypertarget{site-information-navigation}{%
\subsection{Site Information
Navigation}\label{site-information-navigation}}

\begin{itemize}
\tightlist
\item
  \href{https://help.nytimes3xbfgragh.onion/hc/en-us/articles/115014792127-Copyright-notice}{©~2020~The
  New York Times Company}
\end{itemize}

\begin{itemize}
\tightlist
\item
  \href{https://www.nytco.com/}{NYTCo}
\item
  \href{https://help.nytimes3xbfgragh.onion/hc/en-us/articles/115015385887-Contact-Us}{Contact
  Us}
\item
  \href{https://www.nytco.com/careers/}{Work with us}
\item
  \href{https://nytmediakit.com/}{Advertise}
\item
  \href{http://www.tbrandstudio.com/}{T Brand Studio}
\item
  \href{https://www.nytimes3xbfgragh.onion/privacy/cookie-policy\#how-do-i-manage-trackers}{Your
  Ad Choices}
\item
  \href{https://www.nytimes3xbfgragh.onion/privacy}{Privacy}
\item
  \href{https://help.nytimes3xbfgragh.onion/hc/en-us/articles/115014893428-Terms-of-service}{Terms
  of Service}
\item
  \href{https://help.nytimes3xbfgragh.onion/hc/en-us/articles/115014893968-Terms-of-sale}{Terms
  of Sale}
\item
  \href{https://spiderbites.nytimes3xbfgragh.onion}{Site Map}
\item
  \href{https://help.nytimes3xbfgragh.onion/hc/en-us}{Help}
\item
  \href{https://www.nytimes3xbfgragh.onion/subscription?campaignId=37WXW}{Subscriptions}
\end{itemize}
