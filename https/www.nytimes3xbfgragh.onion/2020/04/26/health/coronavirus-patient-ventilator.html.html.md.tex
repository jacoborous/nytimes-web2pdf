\href{/section/health}{Health}\textbar{}32 Days on a Ventilator: One
Covid Patient's Fight to Breathe Again

\url{https://nyti.ms/3cPFlsl}

\begin{itemize}
\item
\item
\item
\item
\item
\item
\end{itemize}

\hypertarget{the-coronavirus-outbreak}{%
\subsubsection{\texorpdfstring{\href{https://www.nytimes3xbfgragh.onion/news-event/coronavirus?name=styln-coronavirus-national\&region=TOP_BANNER\&variant=undefined\&block=storyline_menu_recirc\&action=click\&pgtype=Article\&impression_id=49d02900-e3a1-11ea-96fb-13686c3e0279}{The
Coronavirus
Outbreak}}{The Coronavirus Outbreak}}\label{the-coronavirus-outbreak}}

\begin{itemize}
\tightlist
\item
  live\href{https://www.nytimes3xbfgragh.onion/2020/08/21/world/covid-19-coronavirus.html?name=styln-coronavirus-national\&region=TOP_BANNER\&variant=undefined\&block=storyline_menu_recirc\&action=click\&pgtype=Article\&impression_id=49d02901-e3a1-11ea-96fb-13686c3e0279}{Latest
  Updates}
\item
  \href{https://www.nytimes3xbfgragh.onion/interactive/2020/us/coronavirus-us-cases.html?name=styln-coronavirus-national\&region=TOP_BANNER\&variant=undefined\&block=storyline_menu_recirc\&action=click\&pgtype=Article\&impression_id=49d02902-e3a1-11ea-96fb-13686c3e0279}{Maps
  and Cases}
\item
  \href{https://www.nytimes3xbfgragh.onion/interactive/2020/science/coronavirus-vaccine-tracker.html?name=styln-coronavirus-national\&region=TOP_BANNER\&variant=undefined\&block=storyline_menu_recirc\&action=click\&pgtype=Article\&impression_id=49d02903-e3a1-11ea-96fb-13686c3e0279}{Vaccine
  Tracker}
\item
  \href{https://www.nytimes3xbfgragh.onion/2020/08/19/us/colleges-closing-covid.html?name=styln-coronavirus-national\&region=TOP_BANNER\&variant=undefined\&block=storyline_menu_recirc\&action=click\&pgtype=Article\&impression_id=49d02904-e3a1-11ea-96fb-13686c3e0279}{Colleges
  Closing}
\item
  \href{https://www.nytimes3xbfgragh.onion/live/2020/08/20/business/stock-market-today-coronavirus?name=styln-coronavirus-national\&region=TOP_BANNER\&variant=undefined\&block=storyline_menu_recirc\&action=click\&pgtype=Article\&impression_id=49d02905-e3a1-11ea-96fb-13686c3e0279}{Economy}
\end{itemize}

\includegraphics{https://static01.graylady3jvrrxbe.onion/images/2020/04/20/science/00VIRUS-ICU-jim/merlin_171561255_ade5bfc2-8c82-4b16-ac92-4bb48219488e-articleLarge.jpg?quality=75\&auto=webp\&disable=upscale}

Sections

\protect\hyperlink{site-content}{Skip to
content}\protect\hyperlink{site-index}{Skip to site index}

\hypertarget{32-days-on-a-ventilator-one-covid-patients-fight-to-breathe-again}{%
\section{32 Days on a Ventilator: One Covid Patient's Fight to Breathe
Again}\label{32-days-on-a-ventilator-one-covid-patients-fight-to-breathe-again}}

Jim Bello, 49 and healthy, fell gravely ill, highlighting agonizing
mysteries of the coronavirus. Doctors' relentless effort to save him was
a roller-coaster of devastating and triumphant twists.

A family photo of Jim Bello in the bedroom of his daughter Hadley, 13,
pictured.Credit...Kayana Szymczak for The New York Times

Supported by

\protect\hyperlink{after-sponsor}{Continue reading the main story}

\href{https://www.nytimes3xbfgragh.onion/by/pam-belluck}{\includegraphics{https://static01.graylady3jvrrxbe.onion/images/2018/02/16/multimedia/author-pam-belluck/author-pam-belluck-thumbLarge-v2.png}}

By \href{https://www.nytimes3xbfgragh.onion/by/pam-belluck}{Pam Belluck}

\begin{itemize}
\item
  Published April 26, 2020Updated May 8, 2020
\item
  \begin{itemize}
  \item
  \item
  \item
  \item
  \item
  \item
  \end{itemize}
\end{itemize}

HINGHAM, Mass. --- ``Is he going to make it?'' Kim Bello asked,
clutching her phone, alone in her yard.

She had slipped outside so her three children, playing games in the
living room, could be shielded from a wrenching conversation with a
doctor treating her husband, Jim. For two weeks, he had been battling
the coronavirus at Massachusetts General Hospital, on a ventilator and,
for the past nine days, connected to a last-resort artificial heart-lung
machine as well.

The physician, Dr. Emmy Rubin, gently told Ms. Bello that while her
husband had a chance of surviving, ``If you're asking for an honest
opinion, it's more likely than not that he won't.''

Mr. Bello, 49, an athletic and healthy lawyer, had developed a 103
degree fever in early March after a hike in the White Mountains in New
Hampshire and landed in a suburban emergency room six days later,
struggling to breathe.

Now, despite all his doctors had done, his
\href{https://www.nytimes3xbfgragh.onion/interactive/2020/05/08/health/coronavirus-covid-lungs-ventilators.html}{lungs}
looked white as bone on his latest X-ray, with virtually no air-filled
spaces --- ``one of the worst chest X-rays I've ever seen,'' Dr. Paul
Currier, another of his doctors, said.

As he lay in the intensive care unit, even a touch that caused slight
movement to his heavily sedated and chemically paralyzed body could send
his oxygen levels into a tailspin. Doctors worried his heart would stop,
and if it did, they realized they wouldn't be able to resuscitate him.

They had tried everything to help him, including experimental drugs, a
low-tech maneuver of flipping him on his belly to improve airflow and
the most sophisticated life support machine.

They were considering one more ``Hail Mary'' medical maneuver, but
setting it up required cutting the machine-supplied oxygen for 30
seconds, a gap they did not think he could survive.

\includegraphics{https://static01.graylady3jvrrxbe.onion/images/2020/04/20/science/00VIRUS-ICU/merlin_171757299_edcde2d8-0435-4010-9013-bb6cd2f111c2-articleLarge.jpg?quality=75\&auto=webp\&disable=upscale}

``Even if those were things that could help him, trying to do those
would kill him,'' said Dr. Yuval Raz, a key specialist on Mr. Bello's
team.

Mr. Bello's cataclysmic spiral from avid skier, cyclist and runner to
grievously ill patient --- and the heartbreaking and triumphant twists
in doctors' relentless efforts to save him --- underscores the agonizing
challenges confronting even highly trained physicians and well-equipped
hospitals battling a ferociously capricious virus.

Hospitals have never before had, simultaneously, so many patients so
sick that their lungs have basically stopped functioning. And while
doctors are experienced at treating similar respiratory failure, the
path of patients with Covid-19 can be maddeningly unpredictable.

``It's like they fall off a cliff,'' said Dr. Peggy Lai, a critical care
doctor at Mass General. ``You see young patients getting sicker and
sicker by the day despite everything that you know is good standard of
care.''

Without proven therapies to extinguish the infection, doctors ride
roller-coasters of trial and error. They weigh risks of uncertain
treatments and painstakingly adjust machines in hopes of shoring up
patients' lungs enough that their bodies clear the inflammation and
heal.

``The tricky part with this disease,'' Dr. Lai said, ``is that we have
nothing to follow, to know what predicts how sick someone will be and
what predicts them getting better.''

Image

The Bello family, from left: Hadley, Riley, Kim, Taylor and
Jim.Credit...Kayana Szymczak for The New York Times

On March 7, after Mr. Bello hiked Loon Mountain in New Hampshire, where
his family has a condo and skis regularly, he was suddenly struck by a
high fever.

After several feverish days, he developed a cough and chest tightness
and visited a doctor, who prescribed antibiotics for pneumonia. But by
March 13, he had so much trouble breathing that he went to a suburban
Boston hospital's emergency room. Doctors quickly decided he needed a
ventilator.

``What if I don't make it?'' he asked his wife.

After she reassured him, she recalled, ``He winked at me the same way he
winked at me when we first met.''

\hypertarget{latest-updates-the-coronavirus-outbreak}{%
\section{\texorpdfstring{\href{https://www.nytimes3xbfgragh.onion/2020/08/21/world/covid-19-coronavirus.html?action=click\&pgtype=Article\&state=default\&region=MAIN_CONTENT_1\&context=storylines_live_updates}{Latest
Updates: The Coronavirus
Outbreak}}{Latest Updates: The Coronavirus Outbreak}}\label{latest-updates-the-coronavirus-outbreak}}

Updated 2020-08-21T11:05:09.310Z

\begin{itemize}
\tightlist
\item
  \href{https://www.nytimes3xbfgragh.onion/2020/08/21/world/covid-19-coronavirus.html?action=click\&pgtype=Article\&state=default\&region=MAIN_CONTENT_1\&context=storylines_live_updates\#link-4690b6aa}{Shutdowns,
  warnings and scoldings follow gatherings on college campuses.}
\item
  \href{https://www.nytimes3xbfgragh.onion/2020/08/21/world/covid-19-coronavirus.html?action=click\&pgtype=Article\&state=default\&region=MAIN_CONTENT_1\&context=storylines_live_updates\#link-324af071}{As
  he accepts the Democratic nomination, Biden knocks Trump's pandemic
  response.}
\item
  \href{https://www.nytimes3xbfgragh.onion/2020/08/21/world/covid-19-coronavirus.html?action=click\&pgtype=Article\&state=default\&region=MAIN_CONTENT_1\&context=storylines_live_updates\#link-35890b73}{Hundreds
  of doctors in Kenya go on strike over their pay and protective gear.}
\end{itemize}

\href{https://www.nytimes3xbfgragh.onion/2020/08/21/world/covid-19-coronavirus.html?action=click\&pgtype=Article\&state=default\&region=MAIN_CONTENT_1\&context=storylines_live_updates}{See
more updates}

More live coverage:
\href{https://www.nytimes3xbfgragh.onion/live/2020/08/20/business/stock-market-today-coronavirus?action=click\&pgtype=Article\&state=default\&region=MAIN_CONTENT_1\&context=storylines_live_updates}{Markets}

Overnight, Mr. Bello was transferred to Mass General, becoming its first
intubated coronavirus patient. His case initially seemed straightforward
and manageable, said Dr. Currier, his first attending physician.

Like many Covid-19 patients, Mr. Bello had Acute Respiratory Distress
Syndrome, or ARDS. His lungs were so inflamed and flooded with fluid
that the tiny air sacs that transfer oxygen to the blood had become
ineffectual sodden balloons.

Ventilator settings are precisely calibrated and continually adjusted:
oxygen, breathing rate, breath volume and pressure. Doctors work to give
enough pressure to keep airways open but not so much that lungs are
overstretched and further injured.

Image

Mr. Bello's lungs on March 14, his first X-ray at Massachusetts General
Hospital. The white space indicates areas of his lungs that were filled
with fluid and inflammatory cells.Credit...via~Massachusetts General
Hospital

Intubated patients are sedated and often given paralytic drugs so they
don't try to breathe themselves, allowing the machine to take over.

By the end of Mr. Bello's first day at Mass General, the ventilator was
supplying 65 percent oxygen, lower than what he'd needed upon arrival.
The next day, it was further reduced to 35 percent, a good sign, given
that the lowest setting, 21 percent, is equivalent to room air.

``He actually seemed to be improving,'' said Dr. Currier, a pulmonary
and critical care physician.

But then his condition inexplicably worsened, and his
ventilator-supplied oxygen was ratcheted to the maximum, 100 percent.

Alarmed, around 2 a.m. on March 18, the medical team tried a maneuver
called proning, Dr. Currier said. They carefully turned him onto his
stomach to minimize the pressure of his heart against his lungs,
decompressing his airways.

The results were encouraging. ``This is great,'' Dr. Currier thought
before grabbing some sleep. ``We fixed him.''

But as the day progressed, Mr. Bello's blood oxygen levels plummeted.

Doctors had already started him on medications that many hospitals are
trying: hydroxychloroquine, the anti-malarial drug President Trump has
promoted; and a statin, which was eventually stopped because it affected
his liver. He was also enrolled in a clinical trial of an antiviral drug
being tested for Covid-19, Remdesivir, although nobody knew whether he
was receiving it or a placebo.

That afternoon, increasingly concerned about his lung inflammation,
doctors tried an immunosuppressive medication, tocilizumab.

Nothing was working. So doctors turned to an 11th-hour method. An
eight-person team repositioned Mr. Bello onto his back, inserted large
tubes into his neck and leg, and connected him to a specialized
heart-lung bypass machine.

Called extracorporeal membrane oxygenation, or ECMO, the technique
siphons blood out of the patient, runs it through an oxygenator and
pumps it back into the body. It is intricately challenging to manage and
isn't available at many hospitals.

``ECMO is not a benign therapy,'' said Dr. Raz, the medical director of
Mass General's ECMO department. ``There's a lot of bad things that can
happen even with a good outcome.''

Image

A specialized machine that performs extracorporeal membrane oxygenation,
or ECMO.Credit...Kayana Szymczak for The New York Times

Image

Dr. Yuval Raz, medical director of Mass General's ECMO
department.Credit...Kayana Szymczak for The New York Times

Risks can include bleeding complications and strokes. ECMO specialists
must continually ensure that the blood volume circulating through the
machine isn't too low or too high, so that patients don't get too much
fluid and their blood vessels don't collapse.

So far, ECMO has been used for hundreds of coronavirus patients
worldwide, according to the nonprofit
\href{https://www.elso.org/Registry/FullCOVID19RegistryDashboard.aspx}{Extracorporeal
Life Support Organization}. Most are still on the machines, and data is
incomplete, so survival rates are unclear.

``ECMO doesn't fix anything,'' Dr. Raz said. ``It keeps you alive while
other things, hopefully, take place.''

Mr. Bello's lungs were so stiff that his ``lung compliance'' --- a
measure of elasticity that is usually over 100 in healthy people and
about 30 in people with severe respiratory failure --- was in the single
digits.

His lungs could handle breaths only the size of a tablespoon, a tiny
fraction of a normal-size breath. Blood began oozing from around the
tubes, so blood thinners were stopped, Dr. Raz said.

Chest X-rays documented the decline. His first on March 13 showed
significant fluid and inflammation, but ``you could still see the
lungs,'' Dr. Raz said. On March 18, the X-ray was worse, but lung space
was still visible. By March 20, ``he had essentially what we call a
whiteout.''

Image

Mr. Bello's lungs on March 27, a complete ``whiteout,'' in which his
lungs are hardly visible. Some doctors at Mass General said this was one
of the worst chest X-rays they had ever seen.Credit...via~Massachusetts
General Hospital

Daily, doctors and nurses updated Ms. Bello, 48, who took a leave from
her part-time marketing job to help their children --- Hadley, 13, and
twins Riley and Taylor, 11 --- cope with their father's illness. Ms.
Bello also raised thousands of dollars to provide the I.C.U. with meals
from local restaurants, along with other needs.

She and Hadley developed mild symptoms like chest tightness, but doctors
had considered it unnecessary to test them for the coronavirus.

Because visitors are largely prohibited in order to limit the virus's
spread, a nurse, Kerri Voelkel, put the family on speaker phone in Mr.
Bello's room several times daily.

``Hadley would have baked a cake, and she would joke `It didn't come out
so good, Dad, I'm going to try again,''' Ms. Voelkel recalled. ``Taylor
said, `I did my soccer drills out in the backyard.' It's heartbreaking
to be the caregiver standing there and listen to these children talking
to their father.''

As of March 27, Mr. Bello's ninth day on ECMO, there was no improvement.
When nurses tucked pillows under him or subtly shifted him to prevent
bedsores, his oxygen levels would crater.

Dr. Rubin called Ms. Bello to explain the gravity of the situation. If
Mr. Bello went into cardiac arrest, she said, doctors didn't believe
they could revive him. Ms. Bello agreed to a do-not-resuscitate order.

``Be honest,'' she implored Dr. Rubin.

Dr. Rubin assured her they were not giving up and Mr. Bello could still
survive. But, she said, ``Honestly, I think all of our assessment at
that point was that he's probably more likely to die.''

Devastated, Ms. Bello rolled into a ball on the grass.

Image

The doors leading to a Covid-19 I.C.U. wing of Mass
General.Credit...Kayana Szymczak for The New York Times

The following morning, March 28, the medical team dialed down Mr.
Bello's paralytic medication to see if he could manage with less, Ms.
Voelkel said.

The effect was striking. ``Jim woke up,'' she said. He raised his
eyebrows, and ``you could tell he was trying to open his eyes.''

\href{https://www.nytimes3xbfgragh.onion/news-event/coronavirus?action=click\&pgtype=Article\&state=default\&region=MAIN_CONTENT_3\&context=storylines_faq}{}

\hypertarget{the-coronavirus-outbreak-}{%
\subsubsection{The Coronavirus Outbreak
›}\label{the-coronavirus-outbreak-}}

\hypertarget{frequently-asked-questions}{%
\paragraph{Frequently Asked
Questions}\label{frequently-asked-questions}}

Updated August 17, 2020

\begin{itemize}
\item ~
  \hypertarget{why-does-standing-six-feet-away-from-others-help}{%
  \paragraph{Why does standing six feet away from others
  help?}\label{why-does-standing-six-feet-away-from-others-help}}

  \begin{itemize}
  \tightlist
  \item
    The coronavirus spreads primarily through droplets from your mouth
    and nose, especially when you cough or sneeze. The C.D.C., one of
    the organizations using that measure,
    \href{https://www.nytimes3xbfgragh.onion/2020/04/14/health/coronavirus-six-feet.html?action=click\&pgtype=Article\&state=default\&region=MAIN_CONTENT_3\&context=storylines_faq}{bases
    its recommendation of six feet} on the idea that most large droplets
    that people expel when they cough or sneeze will fall to the ground
    within six feet. But six feet has never been a magic number that
    guarantees complete protection. Sneezes, for instance, can launch
    droplets a lot farther than six feet,
    \href{https://jamanetwork.com/journals/jama/fullarticle/2763852}{according
    to a recent study}. It's a rule of thumb: You should be safest
    standing six feet apart outside, especially when it's windy. But
    keep a mask on at all times, even when you think you're far enough
    apart.
  \end{itemize}
\item ~
  \hypertarget{i-have-antibodies-am-i-now-immune}{%
  \paragraph{I have antibodies. Am I now
  immune?}\label{i-have-antibodies-am-i-now-immune}}

  \begin{itemize}
  \tightlist
  \item
    As of right
    now,\href{https://www.nytimes3xbfgragh.onion/2020/07/22/health/covid-antibodies-herd-immunity.html?action=click\&pgtype=Article\&state=default\&region=MAIN_CONTENT_3\&context=storylines_faq}{that
    seems likely, for at least several months.} There have been
    frightening accounts of people suffering what seems to be a second
    bout of Covid-19. But experts say these patients may have a
    drawn-out course of infection, with the virus taking a slow toll
    weeks to months after initial exposure. People infected with the
    coronavirus typically
    \href{https://www.nature.com/articles/s41586-020-2456-9}{produce}
    immune molecules called antibodies, which are
    \href{https://www.nytimes3xbfgragh.onion/2020/05/07/health/coronavirus-antibody-prevalence.html?action=click\&pgtype=Article\&state=default\&region=MAIN_CONTENT_3\&context=storylines_faq}{protective
    proteins made in response to an
    infection}\href{https://www.nytimes3xbfgragh.onion/2020/05/07/health/coronavirus-antibody-prevalence.html?action=click\&pgtype=Article\&state=default\&region=MAIN_CONTENT_3\&context=storylines_faq}{.
    These antibodies may} last in the body
    \href{https://www.nature.com/articles/s41591-020-0965-6}{only two to
    three months}, which may seem worrisome, but that's perfectly normal
    after an acute infection subsides, said Dr. Michael Mina, an
    immunologist at Harvard University. It may be possible to get the
    coronavirus again, but it's highly unlikely that it would be
    possible in a short window of time from initial infection or make
    people sicker the second time.
  \end{itemize}
\item ~
  \hypertarget{im-a-small-business-owner-can-i-get-relief}{%
  \paragraph{I'm a small-business owner. Can I get
  relief?}\label{im-a-small-business-owner-can-i-get-relief}}

  \begin{itemize}
  \tightlist
  \item
    The
    \href{https://www.nytimes3xbfgragh.onion/article/small-business-loans-stimulus-grants-freelancers-coronavirus.html?action=click\&pgtype=Article\&state=default\&region=MAIN_CONTENT_3\&context=storylines_faq}{stimulus
    bills enacted in March} offer help for the millions of American
    small businesses. Those eligible for aid are businesses and
    nonprofit organizations with fewer than 500 workers, including sole
    proprietorships, independent contractors and freelancers. Some
    larger companies in some industries are also eligible. The help
    being offered, which is being managed by the Small Business
    Administration, includes the Paycheck Protection Program and the
    Economic Injury Disaster Loan program. But lots of folks have
    \href{https://www.nytimes3xbfgragh.onion/interactive/2020/05/07/business/small-business-loans-coronavirus.html?action=click\&pgtype=Article\&state=default\&region=MAIN_CONTENT_3\&context=storylines_faq}{not
    yet seen payouts.} Even those who have received help are confused:
    The rules are draconian, and some are stuck sitting on
    \href{https://www.nytimes3xbfgragh.onion/2020/05/02/business/economy/loans-coronavirus-small-business.html?action=click\&pgtype=Article\&state=default\&region=MAIN_CONTENT_3\&context=storylines_faq}{money
    they don't know how to use.} Many small-business owners are getting
    less than they expected or
    \href{https://www.nytimes3xbfgragh.onion/2020/06/10/business/Small-business-loans-ppp.html?action=click\&pgtype=Article\&state=default\&region=MAIN_CONTENT_3\&context=storylines_faq}{not
    hearing anything at all.}
  \end{itemize}
\item ~
  \hypertarget{what-are-my-rights-if-i-am-worried-about-going-back-to-work}{%
  \paragraph{What are my rights if I am worried about going back to
  work?}\label{what-are-my-rights-if-i-am-worried-about-going-back-to-work}}

  \begin{itemize}
  \tightlist
  \item
    Employers have to provide
    \href{https://www.osha.gov/SLTC/covid-19/standards.html}{a safe
    workplace} with policies that protect everyone equally.
    \href{https://www.nytimes3xbfgragh.onion/article/coronavirus-money-unemployment.html?action=click\&pgtype=Article\&state=default\&region=MAIN_CONTENT_3\&context=storylines_faq}{And
    if one of your co-workers tests positive for the coronavirus, the
    C.D.C.} has said that
    \href{https://www.cdc.gov/coronavirus/2019-ncov/community/guidance-business-response.html}{employers
    should tell their employees} -\/- without giving you the sick
    employee's name -\/- that they may have been exposed to the virus.
  \end{itemize}
\item ~
  \hypertarget{what-is-school-going-to-look-like-in-september}{%
  \paragraph{What is school going to look like in
  September?}\label{what-is-school-going-to-look-like-in-september}}

  \begin{itemize}
  \tightlist
  \item
    It is unlikely that many schools will return to a normal schedule
    this fall, requiring the grind of
    \href{https://www.nytimes3xbfgragh.onion/2020/06/05/us/coronavirus-education-lost-learning.html?action=click\&pgtype=Article\&state=default\&region=MAIN_CONTENT_3\&context=storylines_faq}{online
    learning},
    \href{https://www.nytimes3xbfgragh.onion/2020/05/29/us/coronavirus-child-care-centers.html?action=click\&pgtype=Article\&state=default\&region=MAIN_CONTENT_3\&context=storylines_faq}{makeshift
    child care} and
    \href{https://www.nytimes3xbfgragh.onion/2020/06/03/business/economy/coronavirus-working-women.html?action=click\&pgtype=Article\&state=default\&region=MAIN_CONTENT_3\&context=storylines_faq}{stunted
    workdays} to continue. California's two largest public school
    districts --- Los Angeles and San Diego --- said on July 13, that
    \href{https://www.nytimes3xbfgragh.onion/2020/07/13/us/lausd-san-diego-school-reopening.html?action=click\&pgtype=Article\&state=default\&region=MAIN_CONTENT_3\&context=storylines_faq}{instruction
    will be remote-only in the fall}, citing concerns that surging
    coronavirus infections in their areas pose too dire a risk for
    students and teachers. Together, the two districts enroll some
    825,000 students. They are the largest in the country so far to
    abandon plans for even a partial physical return to classrooms when
    they reopen in August. For other districts, the solution won't be an
    all-or-nothing approach.
    \href{https://bioethics.jhu.edu/research-and-outreach/projects/eschool-initiative/school-policy-tracker/}{Many
    systems}, including the nation's largest, New York City, are
    devising
    \href{https://www.nytimes3xbfgragh.onion/2020/06/26/us/coronavirus-schools-reopen-fall.html?action=click\&pgtype=Article\&state=default\&region=MAIN_CONTENT_3\&context=storylines_faq}{hybrid
    plans} that involve spending some days in classrooms and other days
    online. There's no national policy on this yet, so check with your
    municipal school system regularly to see what is happening in your
    community.
  \end{itemize}
\end{itemize}

When prompted, he squeezed both of Ms. Voelkel's hands. He nodded yes or
no to simple questions. And when the nurses said they were going to
adjust his position, he gave a thumbs up.

``We were like, `Oh my gosh, he's in there!''' Ms. Voelkel said.

Ms. Voelkel described the scene to Ms. Bello over the phone. That
afternoon, the family's golden retriever, Bruno, grabbed Mr. Bello's
Boston Celtics cap, holding it in his mouth. Ms. Bello texted Dr. Rubin
a photo of the dog with the cap and wrote, ``Please do everything you
can.''

Dr. Rubin's eyes welled up. ``I give you my word that we are doing
everything we can,'' she texted back.

Image

Ms. Bello was devastated to learn that her husband was so sick, he was
more likely to die than survive. She texted a doctor: ``Please do
everything you can.''Credit...Kayana Szymczak for The New York Times

But later, several hours after the paralytic medication was stopped, Mr.
Bello, alone in the room while nurses monitored from outside, shifted
his body slightly, movement that increased pressure on his blood
vessels. This happens normally when we breathe, but he was too unstable
to withstand it, Dr. Raz said. His oxygen levels nose-dived.

Both Ms. Voelkel and Tyler Texeira, a respiratory therapist, threw on
their protective gear and rushed in. ``We rescued him, we got him
back,'' Ms. Voelkel said.

``This is a man who, his lungs are so bad that we can't have him
awake,'' she said they realized. ``So we had to re-paralyze him in order
to essentially keep him alive.''

Doctors' last option involved trying to drain more fluid by adding
another tube to the heart-lung machine, a maneuver that would require a
brief stoppage of oxygen flow from the machine.

``He was so tenuous that we felt honestly 30 seconds off the ECMO
circuit, he wouldn't survive that,'' said Dr. Rubin, a pulmonary and
critical care physician.

After her shift ended, Ms. Voelkel said, ``I cried the whole way home.''
She thought of the phone calls from Mr. Bello's children, similar in age
to hers. ``The despair I felt that we couldn't save this man was beyond
anything I could comprehend.''

Dr. Rubin called Ms. Bello and suggested that she visit her husband that
night, something she'd been allowed to do only once before. The hospital
hallways felt eerie. She donned protective gear and entered his room.

``I felt like, `Oh my God, if I keep talking to him, if I talk to him
for hours, maybe he'll stabilize, and maybe he'll be OK,''' she said.
``I was just telling him how much we need him, he has to fight this, he
cannot leave us.''

She was told she'd have 15 minutes, but was given more than three hours.

``I'm squeezing your hand right now, I'm holding your arm, I'm laying on
your arm, I'm touching your head,'' she told her husband.

Within three days, an X-ray showed hope --- some clearing in his left
lung.

``Then, it just started improving, slowly,'' Dr. Currier said. ``And
then it just got dramatically better.''

Image

Mr. Bello's chest X-ray on April 8, the lungs visible again, indicating
his system was clearing.Credit...via~Massachusetts General Hospital

On April 4, Mr. Bello's 17th day on ECMO, Todd Mover, a respiratory
therapist, suggested he might be ready to come off the machine. The next
day, Mr. Bello was disconnected from ECMO. He remained on a ventilator,
but began handling reduced oxygen levels supplied by the ventilator, so
doctors started easing paralytic medication and sedation.

Days later, in a milestone, physical therapists sat Mr. Bello on the
edge of the bed. Ms. Voelkel FaceTimed Ms. Bello. She saw her husband
kick his leg.

``I love you, blow me a kiss,'' she cried. Mr. Bello, groggy from
sedation, breathing tube in his mouth, moved his hand to blow his wife a
kiss.

On April 11, nearly a month after her husband's hospitalization, Ms.
Bello sat at their dining room table for another FaceTime session. She
had her daughters sit across the table, to spare them the sight of their
father on the ventilator. They held an iPad so their brother, Riley, who
was in New Hampshire, could also listen and talk.

``Hi Daddy, it's Hadley and Taylor. We miss you so much. Riley's also on
FaceTime with us. We just want to say keep fighting, and you're going to
be OK. We love you so much.''

Image

Ms. Bello in the family's dining room with Hadley, left, and Taylor,
speaking to Mr. Bello and his nurse, Kerri Voelkel, on a computer. Riley
appeared on an iPad. Credit...Kayana Szymczak for The New York Times

Mr. Bello, unable to speak because of the breathing tube, lifted his
head, opened his eyes briefly and waved his hand slightly. ``Love, love,
love,'' his wife said.

Doctors said they did not know why Mr. Bello survived. Their best guess
is time. Although in some cases, people's odds worsen the longer they're
on a ventilator, other patients recover after long intubations. The
doctors don't know if any of the medications worked.

Dr. Currier said he wouldn't be surprised if Ms. Bello's visit helped.

``She was in there for three hours by the bedside,'' he said. ``It was
at its darkest at that point in time. You just can't underestimate how
much a difference something like that makes.''

On April 14, Mr. Bello was disconnected from the ventilator and began
breathing on his own for the first time in 32 days.

This time, when she received a FaceTime call from the hospital, his wife
gathered the children around. On the screen, he whispered the first
words he'd been able to say to his family in a month: ``I love you.''

As he was wheeled out of the I.C.U. to a regular floor, the medical
staff, previously despondent about his case, lined the hospital hallway,
erupting in applause. He waved.

\includegraphics{https://static01.graylady3jvrrxbe.onion/images/2020/04/16/science/00VIRUS-ICU-jim/00VIRUS-ICU-jim-superJumbo.jpg}

``It's phenomenal,'' Dr. Rubin said. Noting Mr. Bello's previous health
and fitness, she added, ``everyone is very optimistic that he'll have a
full recovery.''

In brief comments from a rehabilitation hospital where he was
transferred three days after coming off the ventilator, Mr. Bello said
he was looking forward to getting back to working as a lawyer
representing medical providers. ``I'm alive today because of those very
same people,'' he said.

Already able to eat and to walk, he said he was proud of his wife and
was eager to be back with his family.

Not long after that, on Friday afternoon, Mr. Bello came home.

Image

From left: Taylor, Jim, Riley, Kim and Hadley Bello at their home on
Saturday, a day after Mr. Bello was released from the
hospital.Credit...Bello family photo

Advertisement

\protect\hyperlink{after-bottom}{Continue reading the main story}

\hypertarget{site-index}{%
\subsection{Site Index}\label{site-index}}

\hypertarget{site-information-navigation}{%
\subsection{Site Information
Navigation}\label{site-information-navigation}}

\begin{itemize}
\tightlist
\item
  \href{https://help.nytimes3xbfgragh.onion/hc/en-us/articles/115014792127-Copyright-notice}{©~2020~The
  New York Times Company}
\end{itemize}

\begin{itemize}
\tightlist
\item
  \href{https://www.nytco.com/}{NYTCo}
\item
  \href{https://help.nytimes3xbfgragh.onion/hc/en-us/articles/115015385887-Contact-Us}{Contact
  Us}
\item
  \href{https://www.nytco.com/careers/}{Work with us}
\item
  \href{https://nytmediakit.com/}{Advertise}
\item
  \href{http://www.tbrandstudio.com/}{T Brand Studio}
\item
  \href{https://www.nytimes3xbfgragh.onion/privacy/cookie-policy\#how-do-i-manage-trackers}{Your
  Ad Choices}
\item
  \href{https://www.nytimes3xbfgragh.onion/privacy}{Privacy}
\item
  \href{https://help.nytimes3xbfgragh.onion/hc/en-us/articles/115014893428-Terms-of-service}{Terms
  of Service}
\item
  \href{https://help.nytimes3xbfgragh.onion/hc/en-us/articles/115014893968-Terms-of-sale}{Terms
  of Sale}
\item
  \href{https://spiderbites.nytimes3xbfgragh.onion}{Site Map}
\item
  \href{https://help.nytimes3xbfgragh.onion/hc/en-us}{Help}
\item
  \href{https://www.nytimes3xbfgragh.onion/subscription?campaignId=37WXW}{Subscriptions}
\end{itemize}
