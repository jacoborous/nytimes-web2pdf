Sections

SEARCH

\protect\hyperlink{site-content}{Skip to
content}\protect\hyperlink{site-index}{Skip to site index}

\href{https://www.nytimes3xbfgragh.onion/section/health}{Health}

\href{https://myaccount.nytimes3xbfgragh.onion/auth/login?response_type=cookie\&client_id=vi}{}

\href{https://www.nytimes3xbfgragh.onion/section/todayspaper}{Today's
Paper}

\href{/section/health}{Health}\textbar{}Just What Older People Didn't
Need: More Isolation

\url{https://nyti.ms/2JXZ5xl}

\begin{itemize}
\item
\item
\item
\item
\item
\item
\end{itemize}

\href{https://www.nytimes3xbfgragh.onion/news-event/coronavirus?action=click\&pgtype=Article\&state=default\&region=TOP_BANNER\&context=storylines_menu}{The
Coronavirus Outbreak}

\begin{itemize}
\tightlist
\item
  live\href{https://www.nytimes3xbfgragh.onion/2020/08/01/world/coronavirus-covid-19.html?action=click\&pgtype=Article\&state=default\&region=TOP_BANNER\&context=storylines_menu}{Latest
  Updates}
\item
  \href{https://www.nytimes3xbfgragh.onion/interactive/2020/us/coronavirus-us-cases.html?action=click\&pgtype=Article\&state=default\&region=TOP_BANNER\&context=storylines_menu}{Maps
  and Cases}
\item
  \href{https://www.nytimes3xbfgragh.onion/interactive/2020/science/coronavirus-vaccine-tracker.html?action=click\&pgtype=Article\&state=default\&region=TOP_BANNER\&context=storylines_menu}{Vaccine
  Tracker}
\item
  \href{https://www.nytimes3xbfgragh.onion/interactive/2020/07/29/us/schools-reopening-coronavirus.html?action=click\&pgtype=Article\&state=default\&region=TOP_BANNER\&context=storylines_menu}{What
  School May Look Like}
\item
  \href{https://www.nytimes3xbfgragh.onion/live/2020/07/31/business/stock-market-today-coronavirus?action=click\&pgtype=Article\&state=default\&region=TOP_BANNER\&context=storylines_menu}{Economy}
\end{itemize}

Advertisement

\protect\hyperlink{after-top}{Continue reading the main story}

Supported by

\protect\hyperlink{after-sponsor}{Continue reading the main story}

the new old age

\hypertarget{just-what-older-people-didnt-need-more-isolation}{%
\section{Just What Older People Didn't Need: More
Isolation}\label{just-what-older-people-didnt-need-more-isolation}}

The coronavirus pandemic could sharpen the health risks of loneliness.
But there are ways to connect.

\includegraphics{https://static01.graylady3jvrrxbe.onion/images/2020/04/14/science/14SCI-SPANISOLATION/14SCI-SPANISOLATION-articleLarge.jpg?quality=75\&auto=webp\&disable=upscale}

By \href{https://www.nytimes3xbfgragh.onion/by/paula-span}{Paula Span}

\begin{itemize}
\item
  April 13, 2020
\item
  \begin{itemize}
  \item
  \item
  \item
  \item
  \item
  \item
  \end{itemize}
\end{itemize}

At midmorning, Lisa Carfagna, a marketing staffer for the 92nd Street Y
in Manhattan, took a brief break from working at home on Long Island and
called the Rubins on the Upper East Side.

They were doing fine, Seymour Rubin, 89, assured her over a
speakerphone.

``We try to have a project every day,'' said Shirley Rubin, 84. ``Today,
I'm making a beef stew for the first time in 40 years.''

``If I'm here tomorrow,'' her husband put in, ``you'll know it was
good.''

When the coronavirus outbreak forced the Y to shutter last month ---
leaving participants in its senior program bereft of their usual
lectures, classes and exercise programs --- about 70 staff members
quickly volunteered to make weekly calls to all 650 of them. Ms.
Carfagna regularly checks in with 25.

Like many cultural organizations, the Y has turned to digital technology
--- streamed concerts and lectures, classes on Zoom, discussion groups
via conference call --- to keep its older patrons engaged.

But computers and smartphones can feel daunting; about a third of
Americans over 65 told Pew Research they
\href{https://www.pewresearch.org/internet/2017/05/17/barriers-to-adoption-and-attitudes-towards-technology/}{weren't
confident}about using digital technology. About half said they needed
help in setting up new devices. Some seniors lack broadband connection.

Besides, ``there's something about that live, person-to-person
connection,'' said Julia Zeuner, who directs the senior program and
mobilized the volunteers. ``A friendly voice on the other end of the
line.''

Social isolation and loneliness among older adults have become such a
concern that two years ago, the National Academies of Sciences,
Engineering and Medicine assembled an expert committee to analyze
research findings and make recommendations.

Its
\href{https://www.nationalacademies.org/our-work/the-health-and-medical-dimensions-of-social-isolation-and-loneliness-in-older-adults}{report}
arrived in late February --- as the coronavirus spread.

Now, older people have been sternly warned to adopt the very practices
that, the committee found, can endanger their health. With senior
centers, day programs, theaters, parks, gyms and restaurants closed and
most in-person visiting prohibited, they are enduring a lengthening
period of social separation. Nursing homes and assisted living
facilities are barring family members.

``It's a public health crisis that should be recognized,'' said Colleen
Galambos, a committee member and a gerontologist at the University of
Wisconsin-Milwaukee. ``People who normally wouldn't be considered
isolated or lonely are now experiencing it.''

These are not the same state, the National Academies report pointed out.
Social isolation refers to an objective lack of social contact with
others; loneliness, a subjective sense of being left out and ignored,
can strike even when people are surrounded by others.

\hypertarget{latest-updates-global-coronavirus-outbreak}{%
\section{\texorpdfstring{\href{https://www.nytimes3xbfgragh.onion/2020/08/01/world/coronavirus-covid-19.html?action=click\&pgtype=Article\&state=default\&region=MAIN_CONTENT_1\&context=storylines_live_updates}{Latest
Updates: Global Coronavirus
Outbreak}}{Latest Updates: Global Coronavirus Outbreak}}\label{latest-updates-global-coronavirus-outbreak}}

Updated 2020-08-01T19:54:00.494Z

\begin{itemize}
\tightlist
\item
  \href{https://www.nytimes3xbfgragh.onion/2020/08/01/world/coronavirus-covid-19.html?action=click\&pgtype=Article\&state=default\&region=MAIN_CONTENT_1\&context=storylines_live_updates\#link-3ac56579}{Top
  officials work to break impasse over jobless benefit.}
\item
  \href{https://www.nytimes3xbfgragh.onion/2020/08/01/world/coronavirus-covid-19.html?action=click\&pgtype=Article\&state=default\&region=MAIN_CONTENT_1\&context=storylines_live_updates\#link-8796723}{The
  virus picks up dangerous speed in the Midwest, and in areas that had
  seen success.}
\item
  \href{https://www.nytimes3xbfgragh.onion/2020/08/01/world/coronavirus-covid-19.html?action=click\&pgtype=Article\&state=default\&region=MAIN_CONTENT_1\&context=storylines_live_updates\#link-25930521}{Thousands
  in Berlin protest Germany's coronavirus measures.}
\end{itemize}

\href{https://www.nytimes3xbfgragh.onion/2020/08/01/world/coronavirus-covid-19.html?action=click\&pgtype=Article\&state=default\&region=MAIN_CONTENT_1\&context=storylines_live_updates}{See
more updates}

More live coverage:
\href{https://www.nytimes3xbfgragh.onion/live/2020/07/31/business/stock-market-today-coronavirus?action=click\&pgtype=Article\&state=default\&region=MAIN_CONTENT_1\&context=storylines_live_updates}{Markets}

By itself, aging doesn't create either problem. But it raises the risks.
``Many older people, especially women over 75, are living by themselves
because their spouses died,'' said Dr. Dan Blazer, the committee chair
and a psychiatrist at the Duke University School of Medicine.

Retirement, difficulty driving, hearing and vision loss, cognitive or
physical problems that make getting out difficult --- all contribute to
a troubling tide of disconnection.

About a quarter of people over 65 living independently in their
communities are considered
\href{https://academic.oup.com/psychsocgerontology/article/75/1/107/4953727}{socially
isolated}, and 43 percent of those over 60
\href{https://jamanetwork.com/journals/jamainternalmedicine/fullarticle/1188033}{report}feeling
lonely --- and that was before public health officials instructed older
people, and everyone else, to stay home.

``It's not just touchy-feely stuff,'' said Dr. Ken Covinsky, a
geriatrician at the University of California, San Francisco, who has
been a co-author on studies on loneliness. ``Isolation is a real risk.''

In fact, it's associated with significantly higher rates of
\href{https://heart.bmj.com/content/102/13/987}{heart disease}and
\href{https://www.ncbi.nlm.nih.gov/pubmed/27091846}{stroke} and a 50
percent increased risk of
\href{https://www.ncbi.nlm.nih.gov/pubmed/30452410,}{dementia}, the
National Academies report pointed out. Isolated or lonely seniors report
a greater incidence of
\href{https://bmcpsychiatry.biomedcentral.com/articles/10.1186/s12888-017-1262-x}{depression
and anxiety}.

They suffer a
\href{https://journals.plos.org/plosmedicine/article?id=10.1371/journal.pmed.1000316}{mortality
rate} comparable to that linked to smoking, obesity, excessive alcohol
consumption and physical inactivity.

When the committee looked for promising solutions, it found
\href{https://journals.sagepub.com/doi/abs/10.1177/0733464818807820}{studies}showing
that attending exercise programs helped reduce isolation --- not a
useful approach at the moment. The evidence for much-heralded
technological approaches, from robotic pets and Zoom to voice-activated
assistants, remains thin thus far.

How, then, to help older people maintain their social connections when
they're supposed to be socially, or at least physically, distanced?
Individuals and organizations around the country are proposing and
trying a variety of tactics.

Dr. Covinsky, particularly concerned about restrictions on visitors to
older people at home or in senior facilities, has suggested that as
coronavirus testing becomes more broadly available, family members or
friends who repeatedly test negative could become ``designated
visitors,'' permitted to spend time with their quarantined loved ones.

``We have restricted something that's pretty essential,'' he said. ``We
need to move away from thinking of visitors to old people as optional.''

In Southern California, two PACE programs --- federally funded efforts
to provide full medical and social services for those aging in place ---
have leased tablets called \href{https://www.grandpad.net}{GrandPads}
for their roughly 250 participants. Designed for those over 75, the
devices allow seniors to call the PACE center, receive and reply to
email, and receive video calls from PACE staff members (and play games).

\href{https://www.nytimes3xbfgragh.onion/news-event/coronavirus?action=click\&pgtype=Article\&state=default\&region=MAIN_CONTENT_3\&context=storylines_faq}{}

\hypertarget{the-coronavirus-outbreak-}{%
\subsubsection{The Coronavirus Outbreak
›}\label{the-coronavirus-outbreak-}}

\hypertarget{frequently-asked-questions}{%
\paragraph{Frequently Asked
Questions}\label{frequently-asked-questions}}

Updated July 27, 2020

\begin{itemize}
\item ~
  \hypertarget{should-i-refinance-my-mortgage}{%
  \paragraph{Should I refinance my
  mortgage?}\label{should-i-refinance-my-mortgage}}

  \begin{itemize}
  \tightlist
  \item
    \href{https://www.nytimes3xbfgragh.onion/article/coronavirus-money-unemployment.html?action=click\&pgtype=Article\&state=default\&region=MAIN_CONTENT_3\&context=storylines_faq}{It
    could be a good idea,} because mortgage rates have
    \href{https://www.nytimes3xbfgragh.onion/2020/07/16/business/mortgage-rates-below-3-percent.html?action=click\&pgtype=Article\&state=default\&region=MAIN_CONTENT_3\&context=storylines_faq}{never
    been lower.} Refinancing requests have pushed mortgage applications
    to some of the highest levels since 2008, so be prepared to get in
    line. But defaults are also up, so if you're thinking about buying a
    home, be aware that some lenders have tightened their standards.
  \end{itemize}
\item ~
  \hypertarget{what-is-school-going-to-look-like-in-september}{%
  \paragraph{What is school going to look like in
  September?}\label{what-is-school-going-to-look-like-in-september}}

  \begin{itemize}
  \tightlist
  \item
    It is unlikely that many schools will return to a normal schedule
    this fall, requiring the grind of
    \href{https://www.nytimes3xbfgragh.onion/2020/06/05/us/coronavirus-education-lost-learning.html?action=click\&pgtype=Article\&state=default\&region=MAIN_CONTENT_3\&context=storylines_faq}{online
    learning},
    \href{https://www.nytimes3xbfgragh.onion/2020/05/29/us/coronavirus-child-care-centers.html?action=click\&pgtype=Article\&state=default\&region=MAIN_CONTENT_3\&context=storylines_faq}{makeshift
    child care} and
    \href{https://www.nytimes3xbfgragh.onion/2020/06/03/business/economy/coronavirus-working-women.html?action=click\&pgtype=Article\&state=default\&region=MAIN_CONTENT_3\&context=storylines_faq}{stunted
    workdays} to continue. California's two largest public school
    districts --- Los Angeles and San Diego --- said on July 13, that
    \href{https://www.nytimes3xbfgragh.onion/2020/07/13/us/lausd-san-diego-school-reopening.html?action=click\&pgtype=Article\&state=default\&region=MAIN_CONTENT_3\&context=storylines_faq}{instruction
    will be remote-only in the fall}, citing concerns that surging
    coronavirus infections in their areas pose too dire a risk for
    students and teachers. Together, the two districts enroll some
    825,000 students. They are the largest in the country so far to
    abandon plans for even a partial physical return to classrooms when
    they reopen in August. For other districts, the solution won't be an
    all-or-nothing approach.
    \href{https://bioethics.jhu.edu/research-and-outreach/projects/eschool-initiative/school-policy-tracker/}{Many
    systems}, including the nation's largest, New York City, are
    devising
    \href{https://www.nytimes3xbfgragh.onion/2020/06/26/us/coronavirus-schools-reopen-fall.html?action=click\&pgtype=Article\&state=default\&region=MAIN_CONTENT_3\&context=storylines_faq}{hybrid
    plans} that involve spending some days in classrooms and other days
    online. There's no national policy on this yet, so check with your
    municipal school system regularly to see what is happening in your
    community.
  \end{itemize}
\item ~
  \hypertarget{is-the-coronavirus-airborne}{%
  \paragraph{Is the coronavirus
  airborne?}\label{is-the-coronavirus-airborne}}

  \begin{itemize}
  \tightlist
  \item
    The coronavirus
    \href{https://www.nytimes3xbfgragh.onion/2020/07/04/health/239-experts-with-one-big-claim-the-coronavirus-is-airborne.html?action=click\&pgtype=Article\&state=default\&region=MAIN_CONTENT_3\&context=storylines_faq}{can
    stay aloft for hours in tiny droplets in stagnant air}, infecting
    people as they inhale, mounting scientific evidence suggests. This
    risk is highest in crowded indoor spaces with poor ventilation, and
    may help explain super-spreading events reported in meatpacking
    plants, churches and restaurants.
    \href{https://www.nytimes3xbfgragh.onion/2020/07/06/health/coronavirus-airborne-aerosols.html?action=click\&pgtype=Article\&state=default\&region=MAIN_CONTENT_3\&context=storylines_faq}{It's
    unclear how often the virus is spread} via these tiny droplets, or
    aerosols, compared with larger droplets that are expelled when a
    sick person coughs or sneezes, or transmitted through contact with
    contaminated surfaces, said Linsey Marr, an aerosol expert at
    Virginia Tech. Aerosols are released even when a person without
    symptoms exhales, talks or sings, according to Dr. Marr and more
    than 200 other experts, who
    \href{https://academic.oup.com/cid/article/doi/10.1093/cid/ciaa939/5867798}{have
    outlined the evidence in an open letter to the World Health
    Organization}.
  \end{itemize}
\item ~
  \hypertarget{what-are-the-symptoms-of-coronavirus}{%
  \paragraph{What are the symptoms of
  coronavirus?}\label{what-are-the-symptoms-of-coronavirus}}

  \begin{itemize}
  \tightlist
  \item
    Common symptoms
    \href{https://www.nytimes3xbfgragh.onion/article/symptoms-coronavirus.html?action=click\&pgtype=Article\&state=default\&region=MAIN_CONTENT_3\&context=storylines_faq}{include
    fever, a dry cough, fatigue and difficulty breathing or shortness of
    breath.} Some of these symptoms overlap with those of the flu,
    making detection difficult, but runny noses and stuffy sinuses are
    less common.
    \href{https://www.nytimes3xbfgragh.onion/2020/04/27/health/coronavirus-symptoms-cdc.html?action=click\&pgtype=Article\&state=default\&region=MAIN_CONTENT_3\&context=storylines_faq}{The
    C.D.C. has also} added chills, muscle pain, sore throat, headache
    and a new loss of the sense of taste or smell as symptoms to look
    out for. Most people fall ill five to seven days after exposure, but
    symptoms may appear in as few as two days or as many as 14 days.
  \end{itemize}
\item ~
  \hypertarget{does-asymptomatic-transmission-of-covid-19-happen}{%
  \paragraph{Does asymptomatic transmission of Covid-19
  happen?}\label{does-asymptomatic-transmission-of-covid-19-happen}}

  \begin{itemize}
  \tightlist
  \item
    So far, the evidence seems to show it does. A widely cited
    \href{https://www.nature.com/articles/s41591-020-0869-5}{paper}
    published in April suggests that people are most infectious about
    two days before the onset of coronavirus symptoms and estimated that
    44 percent of new infections were a result of transmission from
    people who were not yet showing symptoms. Recently, a top expert at
    the World Health Organization stated that transmission of the
    coronavirus by people who did not have symptoms was ``very rare,''
    \href{https://www.nytimes3xbfgragh.onion/2020/06/09/world/coronavirus-updates.html?action=click\&pgtype=Article\&state=default\&region=MAIN_CONTENT_3\&context=storylines_faq\#link-1f302e21}{but
    she later walked back that statement.}
  \end{itemize}
\end{itemize}

At the Queens Public Library in New York, program assistants are calling
about 50 homebound patrons each week to remind them of programs
available by phone and to check on their well-being, said Madlyn
Schneider, the older adult coordinator.

In Los Angeles, the Motion Picture and Television Fund has fielded a
groundswell of new volunteers for its
\href{https://mptf.com/daily-call-sheet}{Daily Call Sheet}program, which
matches them with older people who share their entertainment industry
backgrounds.

The volunteers, once vetted and trained, call once or twice a week.
``It's reminiscing and connecting and fundamental human conversation,''
said Dr. Scott Kaiser, the geriatrician who established the program.

The
\href{https://www.ioaging.org/services/all-inclusive-health-care/friendship-line}{Friendship
Line} in San Francisco similarly operates a ``warmline,'' though without
the industry link. It has also seen a surge in users, and so many new
volunteers that it has resorted to a waiting list.

Dr. Blazer reports that the drivers delivering Meals on Wheels to his
96-year-old mother-in-law in suburban Atlanta, instead of just dropping
off food as usual, now tap on her door and chat from across the hallway
of her independent living building. ``They have a conversation from six
feet away,'' he said. ``She says it makes a huge difference.''

However heartened by such efforts, Dr. Blazer cautioned that ``social
isolation isn't going away.'' He worries about what will happen when
some degree of normalcy dawns --- and millions of isolated or lonely
elders recede from public attention.

``We'll find ways to adapt to this, but my hope for older people is that
we don't forget them,'' he said. ``If this epidemic has taught us
anything, it's that we have to reach out.''

Individuals can do that, too. My friend Peg Rosen, a freelance writer in
suburban New Jersey, has begun FaceTiming with her mother-in-law in
Manhattan every morning.

They work the New York Times crossword puzzle together, just as they
used to on Sundays in a city coffee shop, a pleasure now forbidden them.
(To subvert generational stereotypes, it's Ms. Rosen, 56, who prefers
doing the puzzle on paper; the 87-year-old is using her tablet.)

``It's delightful, a highlight of my day,'' Ms. Rosen reported. ``I
think we'll continue when this all ends.''

\textbf{\emph{{[}}\href{http://on.fb.me/1paTQ1h}{\emph{Like the Science
Times page on Facebook.}}} ****** \emph{\textbar{} Sign up for the}
\textbf{\href{http://nyti.ms/1MbHaRU}{\emph{Science Times
newsletter.}}\emph{{]}}}

Advertisement

\protect\hyperlink{after-bottom}{Continue reading the main story}

\hypertarget{site-index}{%
\subsection{Site Index}\label{site-index}}

\hypertarget{site-information-navigation}{%
\subsection{Site Information
Navigation}\label{site-information-navigation}}

\begin{itemize}
\tightlist
\item
  \href{https://help.nytimes3xbfgragh.onion/hc/en-us/articles/115014792127-Copyright-notice}{©~2020~The
  New York Times Company}
\end{itemize}

\begin{itemize}
\tightlist
\item
  \href{https://www.nytco.com/}{NYTCo}
\item
  \href{https://help.nytimes3xbfgragh.onion/hc/en-us/articles/115015385887-Contact-Us}{Contact
  Us}
\item
  \href{https://www.nytco.com/careers/}{Work with us}
\item
  \href{https://nytmediakit.com/}{Advertise}
\item
  \href{http://www.tbrandstudio.com/}{T Brand Studio}
\item
  \href{https://www.nytimes3xbfgragh.onion/privacy/cookie-policy\#how-do-i-manage-trackers}{Your
  Ad Choices}
\item
  \href{https://www.nytimes3xbfgragh.onion/privacy}{Privacy}
\item
  \href{https://help.nytimes3xbfgragh.onion/hc/en-us/articles/115014893428-Terms-of-service}{Terms
  of Service}
\item
  \href{https://help.nytimes3xbfgragh.onion/hc/en-us/articles/115014893968-Terms-of-sale}{Terms
  of Sale}
\item
  \href{https://spiderbites.nytimes3xbfgragh.onion}{Site Map}
\item
  \href{https://help.nytimes3xbfgragh.onion/hc/en-us}{Help}
\item
  \href{https://www.nytimes3xbfgragh.onion/subscription?campaignId=37WXW}{Subscriptions}
\end{itemize}
