Sections

SEARCH

\protect\hyperlink{site-content}{Skip to
content}\protect\hyperlink{site-index}{Skip to site index}

\href{https://www.nytimes3xbfgragh.onion/section/style/self-care/}{Self-Care}

\href{https://myaccount.nytimes3xbfgragh.onion/auth/login?response_type=cookie\&client_id=vi}{}

\href{https://www.nytimes3xbfgragh.onion/section/todayspaper}{Today's
Paper}

\href{/section/style/self-care/}{Self-Care}\textbar{}A Guide to Donating
Clothes Today

\url{https://nyti.ms/2xqi413}

\begin{itemize}
\item
\item
\item
\item
\item
\end{itemize}

Advertisement

\protect\hyperlink{after-top}{Continue reading the main story}

Supported by

\protect\hyperlink{after-sponsor}{Continue reading the main story}

\hypertarget{a-guide-to-donating-clothes-today}{%
\section{A Guide to Donating Clothes
Today}\label{a-guide-to-donating-clothes-today}}

Good news: you can get ready now. Bad news: you may have to wait.

\includegraphics{https://static01.graylady3jvrrxbe.onion/images/2020/04/13/fashion/13HOW-TO-DONATE-CLOTHES-bags/13HOW-TO-DONATE-CLOTHES-bags-articleLarge-v2.jpg?quality=75\&auto=webp\&disable=upscale}

\href{https://www.nytimes3xbfgragh.onion/by/vanessa-friedman}{\includegraphics{https://static01.graylady3jvrrxbe.onion/images/2018/06/12/multimedia/vanessa-friedman/vanessa-friedman-thumbLarge.png}}

By \href{https://www.nytimes3xbfgragh.onion/by/vanessa-friedman}{Vanessa
Friedman}

\begin{itemize}
\item
  April 13, 2020
\item
  \begin{itemize}
  \item
  \item
  \item
  \item
  \item
  \end{itemize}
\end{itemize}

If this is the time when you usually do your spring cleaning, we have
good and bad news for you. The good news: You can still organize and
purge items from your closet. The bad news: You will have to hold on to
most of them until shelter-in-place orders are lifted. Following are
guidelines for how to proceed and the status of the largest national
organizations (which ones you choose to support because of religious or
political affiliations are up to you).

\hypertarget{can-i-donate-my-clothes}{%
\subsection{Can I donate my clothes?}\label{can-i-donate-my-clothes}}

In general, no. It is certainly not advisable to dump bags of garments
outside the doors of outlets like the Salvation Army and Housing Works
in a rare expedition outside. The stores are closed, and the garments
will simply be left where they are, ultimately adding to garbage or
getting destroyed by weather. The network of Goodwill stores across the
country are independently operated, but ``98 percent are closed''
currently, according to Goodwill Industries.

\hypertarget{should-i-still-do-my-spring-cleaning}{%
\subsection{Should I still do my spring
cleaning?}\label{should-i-still-do-my-spring-cleaning}}

Yes. Using this time at home to organize your wardrobe and really think
about what you wear (and why) could be a valuable head-clearing
exercise, especially for the time when we emerge from hibernation and
start to consider what's next.

There is little doubt that buying habits will change after the pandemic,
becoming more deliberate, out of both economic necessity and a shift in
values. The kind of instant gratification represented by so much of fast
fashion increasingly seems simply wasteful. Understanding what you have
that has lasted (and why it has lasted) will help you make better
decisions later.

\includegraphics{https://static01.graylady3jvrrxbe.onion/images/2020/04/13/fashion/13HOW-TO-DONATE-CLOTHES-jeans/merlin_154613487_6caa207b-6ded-44d6-ae11-337acb379ec0-articleLarge.jpg?quality=75\&auto=webp\&disable=upscale}

\hypertarget{things-you-can-do}{%
\subsection{Things you can do:}\label{things-you-can-do}}

\hypertarget{hold-onto-your-donations-until-shelter-in-place-is-over}{%
\subsubsection{Hold onto your donations until shelter-in-place is
over}\label{hold-onto-your-donations-until-shelter-in-place-is-over}}

Elizabeth Koke, the creative director of Housing Works, wrote in an
email that people should hold onto their items to donate once the
shelter-in-place orders are lifted.

``Even though they aren't able to drop off their donations at this time,
they should know that they are still supporting lifesaving services by
keeping a donation pile for us in their homes. We'll be very grateful to
accept their goods when we resume retail operations and will need their
support more than ever.''

Dress for Success, a nonprofit that helps women achieve economic
independence via job placement, is still working with clients remotely,
but in-person clothing donations have been ``temporarily halted,''
according to a company spokeswoman, as have donations from retail
partners. She added, however, that once people feel free to return to
their daily activities, there will be an even greater need for new
donations.

\hypertarget{do-your-research}{%
\subsubsection{Do your research}\label{do-your-research}}

In \href{https://www.goodwillnynj.org/}{New York and New Jersey}, all
Goodwill outlet and donation stores are closed;
\href{https://goodwill-suncoast.org/donation-locations/}{in Florida},
some donation sites are still open on a ``self-service/non-contact
basis.'' For information about the Goodwill in your area, check
\href{https://www.goodwill.org/locator}{here}.

ThredUp, the online resale community, which takes
donation\href{https://www.thredup.com/cleanout}{``clean-out kits''} ---
they send you a bag, you fill it with your castoffs and send it back,
and whatever ThredUp can resell, it resells, with the money going to a
charity --- is still operating, with a special focus on supporting the
Feeding America food banks. Every bag is taking longer to process, said
James Reinhart, the chief executive, because of a reduced work force and
safety precautions that have been taken for the employees.

As for the actual clothes, Mr. Reinhart said the process had not
changed. ``With transit times (seven to 10 days) and processing times
(an additional 14 to 30 days), all the available data says zero chance
the virus can live on clothing/plastic for long enough to be an issue,''
he emailed. Cleaning is still up to the purchaser.

Most textile recycling operations are also on hold.

\hypertarget{donate-essential-items}{%
\subsubsection{Donate essential items}\label{donate-essential-items}}

The Salvation Army, the country's largest social service provider, is
considered an essential operation. They are prioritizing donations of:

\begin{itemize}
\item
  nonperishable food items
\item
  paper products
\item
  baby supplies
\item
  hygiene products
\item
  cleaning and sanitizing items
\end{itemize}

Because this is happening on a case-by-case basis, check the status of
donations in your area at \href{http://satruck.org/}{satruck.org}.

Advertisement

\protect\hyperlink{after-bottom}{Continue reading the main story}

\hypertarget{site-index}{%
\subsection{Site Index}\label{site-index}}

\hypertarget{site-information-navigation}{%
\subsection{Site Information
Navigation}\label{site-information-navigation}}

\begin{itemize}
\tightlist
\item
  \href{https://help.nytimes3xbfgragh.onion/hc/en-us/articles/115014792127-Copyright-notice}{©~2020~The
  New York Times Company}
\end{itemize}

\begin{itemize}
\tightlist
\item
  \href{https://www.nytco.com/}{NYTCo}
\item
  \href{https://help.nytimes3xbfgragh.onion/hc/en-us/articles/115015385887-Contact-Us}{Contact
  Us}
\item
  \href{https://www.nytco.com/careers/}{Work with us}
\item
  \href{https://nytmediakit.com/}{Advertise}
\item
  \href{http://www.tbrandstudio.com/}{T Brand Studio}
\item
  \href{https://www.nytimes3xbfgragh.onion/privacy/cookie-policy\#how-do-i-manage-trackers}{Your
  Ad Choices}
\item
  \href{https://www.nytimes3xbfgragh.onion/privacy}{Privacy}
\item
  \href{https://help.nytimes3xbfgragh.onion/hc/en-us/articles/115014893428-Terms-of-service}{Terms
  of Service}
\item
  \href{https://help.nytimes3xbfgragh.onion/hc/en-us/articles/115014893968-Terms-of-sale}{Terms
  of Sale}
\item
  \href{https://spiderbites.nytimes3xbfgragh.onion}{Site Map}
\item
  \href{https://help.nytimes3xbfgragh.onion/hc/en-us}{Help}
\item
  \href{https://www.nytimes3xbfgragh.onion/subscription?campaignId=37WXW}{Subscriptions}
\end{itemize}
