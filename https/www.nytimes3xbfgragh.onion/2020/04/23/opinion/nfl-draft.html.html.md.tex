Sections

SEARCH

\protect\hyperlink{site-content}{Skip to
content}\protect\hyperlink{site-index}{Skip to site index}

\href{https://myaccount.nytimes3xbfgragh.onion/auth/login?response_type=cookie\&client_id=vi}{}

\href{https://www.nytimes3xbfgragh.onion/section/todayspaper}{Today's
Paper}

\href{/section/opinion}{Opinion}\textbar{}Hubris, a Case Study

\url{https://nyti.ms/2yAXOtO}

\begin{itemize}
\item
\item
\item
\item
\item
\end{itemize}

Advertisement

\protect\hyperlink{after-top}{Continue reading the main story}

\href{/section/opinion}{Opinion}

Supported by

\protect\hyperlink{after-sponsor}{Continue reading the main story}

\hypertarget{hubris-a-case-study}{%
\section{Hubris, a Case Study}\label{hubris-a-case-study}}

Tune in tonight.

\href{https://www.nytimes3xbfgragh.onion/by/david-leonhardt}{\includegraphics{https://static01.graylady3jvrrxbe.onion/images/2018/04/02/opinion/david-leonhardt/david-leonhardt-thumbLarge.png}}

By \href{https://www.nytimes3xbfgragh.onion/by/david-leonhardt}{David
Leonhardt}

Opinion Columnist

\begin{itemize}
\item
  April 23, 2020
\item
  \begin{itemize}
  \item
  \item
  \item
  \item
  \item
  \end{itemize}
\end{itemize}

\includegraphics{https://static01.graylady3jvrrxbe.onion/images/2020/04/23/opinion/23leonhardt-newsletter/merlin_171816735_d2d91c7d-9a52-4cee-b2ef-587cb1dc6910-articleLarge.jpg?quality=75\&auto=webp\&disable=upscale}

\emph{This article is part of David Leonhardt's newsletter. You can}
\href{https://www.nytimes3xbfgragh.onion/newsletters/opiniontoday?action=click\&module=Intentional\&pgtype=Article}{\emph{sign
up here}} \emph{to receive it each weekday.}

This year's National Football League draft will begin tonight. As one of
the first normal sports events in months, it's likely to get
\href{https://www.nytimes3xbfgragh.onion/2020/04/20/sports/football/nfl-draft-virtual-past.html}{huge
television ratings}. I realize that many readers of this newsletter may
not be sports fans, but I want to talk about the draft because it's a
fascinating window into human irrationality, with lessons even for
people who don't like football.

Here's the brief version: Many people --- including experts, with great
resources at their disposal --- are shockingly overconfident about their
ability to forecast the future.

About 15 years ago, two economists, Richard Thaler (who has since
\href{https://www.nobelprize.org/prizes/economic-sciences/2017/thaler/facts/}{won
a Nobel Prize}) and Cade Massey, set out to study the history of the
draft. They analyzed where in the draft order different players were
chosen and then compared the order to the players' later performance.

Thaler and Massey discovered that, despite the time and money that
football teams devoted to studying players, the teams weren't very good
at predicting who would be the best. Those chosen early often had less
impressive careers than those chosen later. The chance that a player at
a given position turns out to be better than the next player drafted at
that same position is only 52 percent, not much better than a coin flip.
Predicting the career paths of 22-year-olds in any field is hard.

Consider that neither of the past two Super Bowl-winning quarterbacks
(Patrick Mahomes and Tom Brady) were chosen at the very top, while
several recent \#1 picks (like Baker Mayfield and Jameis Winston) have
struggled.

And yet N.F.L. teams continue to treat the very top picks as far more
valuable than picks slightly further down. They're often willing to
trade multiple picks later in a draft for a single pick near the top.
It's irrational, and predictably so, but the N.F.L. teams can't help
themselves. Executives remember the exceptions --- the top picks who
turned out to be as good as advertised --- and convince themselves that
they can pull off another one.

``Even the smartest guys in the world, the guys who spend hours with
game film, can't predict this with much success,'' Massey has told me.
``There's no crime in that. The crime is thinking you can predict it.''

The savviest teams have realized they can exploit this irrationality by
trading one of their high picks for multiple picks lower down. They
effectively swap the ability to choose the one player they want for the
ability to take chances on multiple players. They embrace humility. The
Dallas Cowboys
\href{https://www.dallasnews.com/sports/cowboys/2019/10/12/its-been-30-years-since-the-cowboys-traded-herschel-walker-to-the-vikings-fueling-a-dynasty/}{built}
a championship team in the 1990s with this approach, and the New England
Patriots have done so over the past two decades.

Tonight, the teams with the most glaring opportunities to trade down are
the Detroit Lions (who pick third) and the New York Giants (fourth). The
Lions and Giants have also been two of the worst-run franchises in
recent years.

So allow me to suggest some
\href{https://repository.upenn.edu/cgi/viewcontent.cgi?article=1138\&context=oid_papers}{last-minute
reading} for their executives --- and for anyone who enjoys an unusually
well-written academic research paper. The official title is: ``The
Loser's Curse: Decision Making and Market Efficiency in the National
Football League Draft.''

\textbf{For more} \ldots{}

\begin{itemize}
\item
  The Massey-Thaler paper is part of a long line of research on human
  overconfidence. One famous experiment found that psychologists became
  more confident ---
  \href{https://www.nytimes3xbfgragh.onion/2005/04/24/weekinreview/the-nfl-draft-a-study-in-cockeyed-overconfidence.html}{but
  no more accurate} --- in their diagnoses as they received more
  information.
\item
  As
  \href{https://blogs.scientificamerican.com/guest-blog/lessons-from-sherlock-holmes-confidence-is-good-overconfidence-not-so-much/}{Maria
  Konnikova} wrote in Scientific American:
\end{itemize}

\begin{quote}
We tend to be underconfident on easy problems and overconfident on
difficult ones. In other words, we underestimate our ability to do well
when all signs point to success, and we overestimate it when the signs
become much less favorable, failing to adjust enough for the change in
external circumstances. Second, it increases with familiarity. If I'm
doing something for the first time, I will likely be cautious. But if I
do it many times over, I am increasingly likely to trust in my ability
and become complacent, even if the landscape should change
(overconfident drivers, anyone?).
\end{quote}

\begin{itemize}
\item
  I spoke to Thaler yesterday, and he offered a prediction for tonight:
  If any high profile trades happen, the team that moves up in the draft
  will select a quarterback --- and will pay the traditional market
  price in draft picks for doing so. (N.F.L. teams use a chart that
  describes
  \href{https://www.pro-football-reference.com/draft/draft_trade_value.htm}{the
  supposed value} of each draft pick; the chart substantially
  exaggerates the value of top picks.)
\item
  The social psychologist Philip Tetlock has done extensive research
  into forecasting failures, including his book with Dan Gardner,
  \href{https://www.penguinrandomhouse.com/books/227815/superforecasting-by-philip-e-tetlock-and-dan-gardner/}{``Superforecasting.''}
\item
  ``For more than a month, American culture has existed on a split
  screen. On one hand, movies and theater and travel and every other
  sport ground to a halt,'' The Ringer's
  \href{https://www.theringer.com/2020/4/22/21230568/nfl-draft-quarantine-super-bowl-sports-media}{Bryan
  Curtis} writes. On the other hand, the N.F.L. keeps going --- in part
  because it's in its offseason. Curtis describes tonight's draft as
  ``the Quarantine Super Bowl.''
\end{itemize}

\emph{If you are not a subscriber to this newsletter, you can}
\href{https://www.nytimes3xbfgragh.onion/newsletters/david-leonhardt}{\emph{subscribe
here}}\emph{. You can also join me on}
\href{https://twitter.com/DLeonhardt}{\emph{Twitter (@DLeonhardt)}}
\emph{and}
\href{https://www.facebookcorewwwi.onion/DavidRLeonhardt/}{\emph{Facebook}}\emph{.}

\emph{Follow The New York Times Opinion section on}
\href{https://www.facebookcorewwwi.onion/nytopinion}{\emph{Facebook}}\emph{,}
\href{http://twitter.com/NYTOpinion}{\emph{Twitter (@NYTopinion)}}
\emph{and}
\href{https://www.instagram.com/nytopinion/}{\emph{Instagram}}\emph{.}

Advertisement

\protect\hyperlink{after-bottom}{Continue reading the main story}

\hypertarget{site-index}{%
\subsection{Site Index}\label{site-index}}

\hypertarget{site-information-navigation}{%
\subsection{Site Information
Navigation}\label{site-information-navigation}}

\begin{itemize}
\tightlist
\item
  \href{https://help.nytimes3xbfgragh.onion/hc/en-us/articles/115014792127-Copyright-notice}{©~2020~The
  New York Times Company}
\end{itemize}

\begin{itemize}
\tightlist
\item
  \href{https://www.nytco.com/}{NYTCo}
\item
  \href{https://help.nytimes3xbfgragh.onion/hc/en-us/articles/115015385887-Contact-Us}{Contact
  Us}
\item
  \href{https://www.nytco.com/careers/}{Work with us}
\item
  \href{https://nytmediakit.com/}{Advertise}
\item
  \href{http://www.tbrandstudio.com/}{T Brand Studio}
\item
  \href{https://www.nytimes3xbfgragh.onion/privacy/cookie-policy\#how-do-i-manage-trackers}{Your
  Ad Choices}
\item
  \href{https://www.nytimes3xbfgragh.onion/privacy}{Privacy}
\item
  \href{https://help.nytimes3xbfgragh.onion/hc/en-us/articles/115014893428-Terms-of-service}{Terms
  of Service}
\item
  \href{https://help.nytimes3xbfgragh.onion/hc/en-us/articles/115014893968-Terms-of-sale}{Terms
  of Sale}
\item
  \href{https://spiderbites.nytimes3xbfgragh.onion}{Site Map}
\item
  \href{https://help.nytimes3xbfgragh.onion/hc/en-us}{Help}
\item
  \href{https://www.nytimes3xbfgragh.onion/subscription?campaignId=37WXW}{Subscriptions}
\end{itemize}
