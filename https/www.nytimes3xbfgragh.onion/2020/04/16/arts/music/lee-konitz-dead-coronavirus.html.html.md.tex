Sections

SEARCH

\protect\hyperlink{site-content}{Skip to
content}\protect\hyperlink{site-index}{Skip to site index}

\href{https://www.nytimes3xbfgragh.onion/section/arts/music}{Music}

\href{https://myaccount.nytimes3xbfgragh.onion/auth/login?response_type=cookie\&client_id=vi}{}

\href{https://www.nytimes3xbfgragh.onion/section/todayspaper}{Today's
Paper}

\href{/section/arts/music}{Music}\textbar{}Lee Konitz, Jazz Saxophonist
Who Blazed His Own Trail, Dies at 92

\url{https://nyti.ms/2RHLeQ1}

\begin{itemize}
\item
\item
\item
\item
\item
\end{itemize}

\href{https://www.nytimes3xbfgragh.onion/news-event/coronavirus?action=click\&pgtype=Article\&state=default\&region=TOP_BANNER\&context=storylines_menu}{The
Coronavirus Outbreak}

\begin{itemize}
\tightlist
\item
  live\href{https://www.nytimes3xbfgragh.onion/2020/08/04/world/coronavirus-covid-19.html?action=click\&pgtype=Article\&state=default\&region=TOP_BANNER\&context=storylines_menu}{Latest
  Updates}
\item
  \href{https://www.nytimes3xbfgragh.onion/interactive/2020/us/coronavirus-us-cases.html?action=click\&pgtype=Article\&state=default\&region=TOP_BANNER\&context=storylines_menu}{Maps
  and Cases}
\item
  \href{https://www.nytimes3xbfgragh.onion/interactive/2020/science/coronavirus-vaccine-tracker.html?action=click\&pgtype=Article\&state=default\&region=TOP_BANNER\&context=storylines_menu}{Vaccine
  Tracker}
\item
  \href{https://www.nytimes3xbfgragh.onion/2020/08/02/us/covid-college-reopening.html?action=click\&pgtype=Article\&state=default\&region=TOP_BANNER\&context=storylines_menu}{College
  Reopening}
\item
  \href{https://www.nytimes3xbfgragh.onion/live/2020/08/03/business/stock-market-today-coronavirus?action=click\&pgtype=Article\&state=default\&region=TOP_BANNER\&context=storylines_menu}{Economy}
\end{itemize}

Advertisement

\protect\hyperlink{after-top}{Continue reading the main story}

Supported by

\protect\hyperlink{after-sponsor}{Continue reading the main story}

those we've lost

\hypertarget{lee-konitz-jazz-saxophonist-who-blazed-his-own-trail-dies-at-92}{%
\section{Lee Konitz, Jazz Saxophonist Who Blazed His Own Trail, Dies at
92}\label{lee-konitz-jazz-saxophonist-who-blazed-his-own-trail-dies-at-92}}

He was a pioneer of the cool school, but he resisted pigeonholing and
focused on ``making a personal statement.'' He died of complications of
the coronavirus.

\includegraphics{https://static01.graylady3jvrrxbe.onion/images/2020/04/17/obituaries/16Konitz2/00Konitz2-articleLarge.jpg?quality=75\&auto=webp\&disable=upscale}

By Peter Keepnews

\begin{itemize}
\item
  Published April 16, 2020Updated April 20, 2020
\item
  \begin{itemize}
  \item
  \item
  \item
  \item
  \item
  \end{itemize}
\end{itemize}

\emph{This obituary is part of a series about people who have died in
the coronavirus pandemic. Read about others}
\href{https://www.nytimes3xbfgragh.onion/series/people-who-have-died-of-the-coronavirus}{\emph{here}}\emph{.}

Lee Konitz, a prolific and idiosyncratic saxophonist who was one of the
earliest and most admired exponents of the style known as cool jazz,
died on Wednesday in Manhattan. He was 92.

His niece Linda Konitz said the cause was complications of the
coronavirus. She said he also had pneumonia.

Mr. Konitz initially attracted attention as much for the way he didn't
play as for the way he did. Like most of his jazz contemporaries, he
adopted the expanded harmonic vocabulary of his fellow alto saxophonist
\href{https://www.nytimes3xbfgragh.onion/1955/03/15/archives/charlie-parker-jazz-master-dies-a-bebop-founder-and-top-saxophonist.html}{Charlie
Parker}, the leading figure in modern jazz. But his approach departed
from Parker's in significant ways, and he quickly emerged as a role
model for musicians seeking an alternative to Parker's pervasive
influence.

Where modern jazz in the Parker mold, better known as bebop, tended to
be passionate and virtuosic, Mr. Konitz's improvisations were measured
and understated, more thoughtful than heated.

``I knew and loved Charlie Parker and copied his bebop solos like
everyone else,'' Mr. Konitz
\href{https://www.wsj.com/articles/jazzs-king-of-coola-cool-65-1378152141}{told
The Wall Street Journal in 2013}. ``But I didn't want to sound like him.
So I used almost no vibrato and played mostly in the higher register.
That's the heart of my sound.''

Although some musicians and critics dismissed Mr. Konitz's style as
overly cerebral and lacking in emotion, it proved influential in the
development of the so-called cool school. But while cool jazz,
essentially a less heated variation on bebop, was popular for several
years --- and some of its exponents, notably the baritone saxophonist
\href{https://www.nytimes3xbfgragh.onion/1996/01/21/nyregion/gerry-mulligan-a-baritone-saxophonist-and-cool-school-jazz-pioneerdies-at-68.html}{Gerry
Mulligan} and the trumpeter and singer
\href{https://www.nytimes3xbfgragh.onion/1988/05/14/obituaries/chet-baker-jazz-trumpeter-dies-at-59-in-a-fall.html}{Chet
Baker}, both of whom he sometimes worked with, became stars --- Mr.
Konitz for most of his career was a musician's musician, admired by his
peers and jazz aficionados but little known to the general public.

This was in part because of his personality: An introvert by nature, he
was never entirely comfortable in the spotlight. And it was in part
because of his musical philosophy, which valued spontaneity above all
else and often led him to pursue daring improvisational tangents that
could leave his less adventurous listeners feeling a little lost. (His
way of preparing for a performance, he once said, was ``to not be
prepared.'')

``My playing was about making a personal statement --- getting audiences
to pay attention to what I was saying musically rather than giving them
what they wanted to hear, which is entertainment,'' he said in the 2013
interview, referring to his early struggles to find an audience. ``I
wanted to play original music.''

His willingness to take chances was admired by advocates of so-called
free jazz, which, beginning in the late 1950s, defied established rules
of harmony and rhythm. But ultimately no label, not even ``cool,''
really fit Mr. Konitz; he was best characterized as sui generis.

Reviewing a performance in 2000 for The New York Times,
\href{https://www.nytimes3xbfgragh.onion/2000/05/13/arts/jazz-review-writing-their-own-ticket-on-an-original-of-the-past.html}{Ben
Ratliff called Mr. Konitz} ``as original a player as there is in jazz''
and praised the ``boiled-down wisdom'' of his playing, noting that
``even when he is in the heat of improvisation, it sounds like someone
whistling a tune he has known all his life.''

\includegraphics{https://static01.graylady3jvrrxbe.onion/images/2020/04/17/obituaries/17Konitz1/merlin_43024732_641f65f0-f4bb-43ac-9747-6a20efd0f109-articleLarge.jpg?quality=75\&auto=webp\&disable=upscale}

Leon Konitz was born in Chicago on Oct. 13, 1927, the youngest of three
sons of Jewish immigrants. His father, Abraham, who owned a laundry, was
from Austria; his mother, Anna (Getlin) Konitz, was from Russia.

Inspired by
\href{https://www.nytimes3xbfgragh.onion/1986/06/14/obituaries/benny-goodman-king-of-swing-is-dead.html}{Benny
Goodman}, he persuaded his parents to buy him a clarinet when he was 11.
He later switched to saxophone, and in 1945, with the ranks of the
nation's dance bands depleted by the draft and opportunities for young
musicians plentiful, he began his professional career with the
Chicago-based band of Jerry Wald.

His first big break came in 1947 when he joined
\href{https://www.youtube.com/watch?v=oX05oHu63JE}{the Claude Thornhill
orchestra}, whose soft sound and pastel colors meshed well with his
playing style. A
\href{https://www.youtube.com/watch?v=YHxdZARIbSQ}{subsequent stint}
with the more dynamic and aggressive
\href{https://www.nytimes3xbfgragh.onion/1979/08/27/archives/stan-kenton-band-leader-dies-was-center-of-jazz-controversies-in-an.html}{Stan
Kenton} ensemble proved an uneasy musical mix but helped spread his name
in the jazz world.

The recordings that did the most to establish Mr. Konitz's reputation
were made in the late 1940s and early '50s, after he had moved to New
York, under the leadership of two of the most distinctive artists in
modern jazz: the pianist and composer
\href{https://www.nytimes3xbfgragh.onion/1978/11/20/archives/lennie-tristano-at-59-pianist-was-innovator-in-the-cool-jazz-era.html}{Lennie
Tristano}, with whom he studied for several years and whose
\href{https://www.youtube.com/watch?v=bznqGjyYuRk}{unorthodox approach
to improvisation} helped shape his own; and the trumpeter
\href{https://www.nytimes3xbfgragh.onion/1991/09/29/nyregion/miles-davis-trumpeter-dies-jazz-genius-65-defined-cool.html}{Miles
Davis}, whose short-lived but influential
\href{https://www.youtube.com/watch?v=tMPdP6-lrmc}{nine-piece band}
sought to adapt the ethereal Thornhill sound to a bebop context.

Those recordings, and others Mr. Konitz made as a leader in the 1950s,
were widely admired by other musicians. But that admiration did not
translate into work, and he struggled to find bookings; for a brief
period in the '60s he stopped performing altogether.

He did not find steady employment as a musician again until the
mid-'70s, when New York City experienced a small jazz renaissance. He
attracted a loyal audience for his work both with small groups and
with\href{https://www.youtube.com/watch?v=ge6RoRO-3sU}{a nonet} that,
despite its ambitious repertoire and arrangements, ultimately did not
last much longer than the Miles Davis ensemble on which it was partly
modeled.

He had a bigger following in Europe, where for the last several decades
of his life he spent much of his time and did most of his recording. His
European discography ranged in style and format from ``Lone-Lee''
(1974), on which he played unaccompanied, to
\href{https://www.youtube.com/watch?v=eYVqhj7xnIg}{``Saxophone Dreams''}
(1997), on which he was supported by a 61-piece orchestra.

He was named a National Endowment for the Arts
\href{https://www.arts.gov/honors/jazz/lee-konitz}{Jazz Master} in 2009.

While Mr. Konitz rarely maintained a working group for more than a few
months, he performed and recorded as both leader and sideman with an
impressive array of top-rank musicians, ranging from the pianist
\href{https://www.nytimes3xbfgragh.onion/2012/12/06/arts/music/dave-brubeck-jazz-musician-dies-at-91.html}{Dave
Brubeck} (on Mr. Brubeck's 1976 album
\href{https://www.youtube.com/watch?v=2kwAoh8jl0U}{``All the Things We
Are,''} which also featured the avant-garde saxophonist
\href{https://www.nytimes3xbfgragh.onion/2019/01/11/arts/anthony-braxton-composer.html}{Anthony
Braxton}) and the drummer
\href{https://www.nytimes3xbfgragh.onion/2004/05/19/arts/elvin-jones-jazz-drummer-with-coltrane-dies-at-76.html}{Elvin
Jones} (on Mr. Konitz's influential 1961 album
\href{https://www.youtube.com/watch?v=IpNhX-UvIxM}{``Motion,''} an
experiment in spontaneity recorded without planning or rehearsal) to, in
more recent years, the pianist
\href{https://www.youtube.com/watch?v=jAKwaZOWX8c}{Brad Mehldau} and the
guitarist Bill Frisell. In 2003, in a rare foray outside the jazz world,
he played on Elvis Costello's album ``North.''

Despite health problems, Mr. Konitz continued to perform into his 90s.
In recent years he would often stop playing in mid-solo and continue
improvising vocally.

Mr. Konitz was married three times. He is survived by two sons, Josh and
Paul; three daughters, Rebecca Pita, Stephanie Stonefifer and Karen
Kaley; three grandchildren; and one great-granddaughter.

Like many jazz musicians, Mr. Konitz often found himself plying his
trade in bars and nightclubs where the audiences were less than
completely attentive. He professed not to mind.

``Wherever I'm at, I'm happy to have a chance to play,'' he told the
British jazz writer Les Tompkins in 1976. ``People come in and say, `How
can you work in this noisy little joint?' I say: `Very easy. I take the
horn out of the bag, and I put it in my mouth.' I appreciate the
opportunity.''

Julia Carmel contributed reporting.

\href{https://www.nytimes3xbfgragh.onion/interactive/2020/obituaries/people-died-coronavirus-obituaries.html?action=click\&pgtype=Article\&state=default\&region=BELOW_MAIN_CONTENT\&context=covid_obits_promo}{}

\hypertarget{those-weve-lost}{%
\section{Those We've Lost}\label{those-weve-lost}}

The coronavirus pandemic has taken an incalculable death toll. This
series is designed to put names and faces to the numbers.

Read more

\includegraphics{https://static01.graylady3jvrrxbe.onion/images/2020/07/30/obituaries/30Pedro/30Pedro-square640.jpg}

\hypertarget{bernaldina-josuxe9-pedro}{%
\section{Bernaldina José Pedro}\label{bernaldina-josuxe9-pedro}}

d. Boa Vista, Brazil

Leader among the Indigenous Macuxi

\includegraphics{https://static01.graylady3jvrrxbe.onion/images/2020/07/31/obituaries/31Swing/merlin_175167783_8913bc90-0d64-43f3-a655-1bb1bf1601c9-square640.jpg}

\hypertarget{john-eric-swing}{%
\section{John Eric Swing}\label{john-eric-swing}}

d. Fountain Valley, Calif.

Champion of Filipino-Americans

\includegraphics{https://static01.graylady3jvrrxbe.onion/images/2020/07/27/obituaries/27Victor/merlin_175001436_38b11f8e-227a-4e2c-9821-7618af9b2524-square640.jpg}

\hypertarget{victor-victor}{%
\section{Victor Victor}\label{victor-victor}}

d. Santo Domingo, Dominican Republic

Beloved musician of the Dominican Republic

\includegraphics{https://static01.graylady3jvrrxbe.onion/images/2020/07/31/obituaries/31Negron/merlin_175160169_516322ae-fd23-4969-b6b2-193ced371105-square640.jpg}

\hypertarget{dr-eddie-negruxf3n}{%
\section{Dr. Eddie Negrón}\label{dr-eddie-negruxf3n}}

d. Fort Walton Beach, Fla.

Internist on Florida's Emerald Coast

\includegraphics{https://static01.graylady3jvrrxbe.onion/images/2020/07/30/obituaries/30Dobson/merlin_175115928_f6b9271c-8f05-4fe1-a38a-5ca4a58f8935-square640.jpg}

\hypertarget{dobby-dobson}{%
\section{Dobby Dobson}\label{dobby-dobson}}

d. Coral Springs, Fla.

Jamaican singer and songwriter

\includegraphics{https://static01.graylady3jvrrxbe.onion/images/2020/08/01/obituaries/28Gonzalez/merlin_175002771_beb57888-3951-409a-ae13-03a94b2e962e-square640.jpg}

\hypertarget{waldemar-gonzalez}{%
\section{Waldemar Gonzalez}\label{waldemar-gonzalez}}

d. White Plains, N.Y.

Teacher and social worker

Advertisement

\protect\hyperlink{after-bottom}{Continue reading the main story}

\hypertarget{site-index}{%
\subsection{Site Index}\label{site-index}}

\hypertarget{site-information-navigation}{%
\subsection{Site Information
Navigation}\label{site-information-navigation}}

\begin{itemize}
\tightlist
\item
  \href{https://help.nytimes3xbfgragh.onion/hc/en-us/articles/115014792127-Copyright-notice}{©~2020~The
  New York Times Company}
\end{itemize}

\begin{itemize}
\tightlist
\item
  \href{https://www.nytco.com/}{NYTCo}
\item
  \href{https://help.nytimes3xbfgragh.onion/hc/en-us/articles/115015385887-Contact-Us}{Contact
  Us}
\item
  \href{https://www.nytco.com/careers/}{Work with us}
\item
  \href{https://nytmediakit.com/}{Advertise}
\item
  \href{http://www.tbrandstudio.com/}{T Brand Studio}
\item
  \href{https://www.nytimes3xbfgragh.onion/privacy/cookie-policy\#how-do-i-manage-trackers}{Your
  Ad Choices}
\item
  \href{https://www.nytimes3xbfgragh.onion/privacy}{Privacy}
\item
  \href{https://help.nytimes3xbfgragh.onion/hc/en-us/articles/115014893428-Terms-of-service}{Terms
  of Service}
\item
  \href{https://help.nytimes3xbfgragh.onion/hc/en-us/articles/115014893968-Terms-of-sale}{Terms
  of Sale}
\item
  \href{https://spiderbites.nytimes3xbfgragh.onion}{Site Map}
\item
  \href{https://help.nytimes3xbfgragh.onion/hc/en-us}{Help}
\item
  \href{https://www.nytimes3xbfgragh.onion/subscription?campaignId=37WXW}{Subscriptions}
\end{itemize}
