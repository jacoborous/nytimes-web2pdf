Sections

SEARCH

\protect\hyperlink{site-content}{Skip to
content}\protect\hyperlink{site-index}{Skip to site index}

\href{https://www.nytimes3xbfgragh.onion/section/science}{Science}

\href{https://myaccount.nytimes3xbfgragh.onion/auth/login?response_type=cookie\&client_id=vi}{}

\href{https://www.nytimes3xbfgragh.onion/section/todayspaper}{Today's
Paper}

\href{/section/science}{Science}\textbar{}Hubble Marks 30 Years of
Seeing a Universe Being Born and Dying

\url{https://nyti.ms/3eUjt0X}

\begin{itemize}
\item
\item
\item
\item
\item
\end{itemize}

\href{https://www.nytimes3xbfgragh.onion/spotlight/at-home?action=click\&pgtype=Article\&state=default\&region=TOP_BANNER\&context=at_home_menu}{At
Home}

\begin{itemize}
\tightlist
\item
  \href{https://www.nytimes3xbfgragh.onion/2020/08/14/dining/lobster-salad-recipe.html?action=click\&pgtype=Article\&state=default\&region=TOP_BANNER\&context=at_home_menu}{Make:
  Lobster Salad}
\item
  \href{https://www.nytimes3xbfgragh.onion/2020/08/15/at-home/coronavirus-at-home-quick-exercises.html?action=click\&pgtype=Article\&state=default\&region=TOP_BANNER\&context=at_home_menu}{Sneak
  In: Exercise}
\item
  \href{https://www.nytimes3xbfgragh.onion/interactive/2020/at-home/even-more-reporters-editors-diaries-lists-recommendations.html?action=click\&pgtype=Article\&state=default\&region=TOP_BANNER\&context=at_home_menu}{See:
  Reporters' Obsessions}
\item
  \href{https://www.nytimes3xbfgragh.onion/2020/08/15/at-home/coronavirus-fall-patio-furniture.html?action=click\&pgtype=Article\&state=default\&region=TOP_BANNER\&context=at_home_menu}{Deck
  Out: Your Porch}
\end{itemize}

Advertisement

\protect\hyperlink{after-top}{Continue reading the main story}

Supported by

\protect\hyperlink{after-sponsor}{Continue reading the main story}

Out There

\hypertarget{hubble-marks-30-years-of-seeing-a-universe-being-born-and-dying}{%
\section{Hubble Marks 30 Years of Seeing a Universe Being Born and
Dying}\label{hubble-marks-30-years-of-seeing-a-universe-being-born-and-dying}}

A new image reminds us of what the orbiting observatory has let us see.

\includegraphics{https://static01.graylady3jvrrxbe.onion/images/2020/04/24/science/24SCI-OUTTHERE-HUBBLE1/merlin_171897510_2bd1af61-81c9-47ff-b7be-7177c19c76e1-articleLarge.jpg?quality=75\&auto=webp\&disable=upscale}

\href{https://www.nytimes3xbfgragh.onion/by/dennis-overbye}{\includegraphics{https://static01.graylady3jvrrxbe.onion/images/2018/07/30/multimedia/author-dennis-overbye/author-dennis-overbye-thumbLarge.png}}

By \href{https://www.nytimes3xbfgragh.onion/by/dennis-overbye}{Dennis
Overbye}

\begin{itemize}
\item
  April 24, 2020
\item
  \begin{itemize}
  \item
  \item
  \item
  \item
  \item
  \end{itemize}
\end{itemize}

``He not busy being born is busy dying.'' So said Bob Dylan in
``\href{https://www.youtube.com/watch?v=_CJHbfkROow}{It's Alright Ma}
(I'm Only Bleeding).''

As shown in a new picture of stormy star birth in a nearby galaxy,
captured by the Hubble Space Telescope, the cosmos is keeping up the
tradition of both birth and death. Stars are being born out of the ashes
of old ones, forever refreshing the universe.

The picture was released Friday by the Space Telescope Science Institute
in Baltimore, keepers of the Hubble, in honor of the 30th anniversary of
the launch of that telescope on April 24, 1990.

That, too, was a time of darkness and rebirth in our world. Astronomers
had dreamed for half a century of a telescope in space above the
distorting and absorbing effects of the atmosphere, but that dream of a
new rebirth for astronomy almost died when the telescope was launched
and astronomers found they couldn't focus it. The primary mirror had
been ground to the wrong shape.

Hubble was famously reborn in 1993 when astronauts used the space
shuttle to fit it with corrective lenses. Over the next 16 years,
successive servicing missions kept Hubble on top of its game, and
astronomers now are confident that it will still be in prime shape when
its successor, the
\href{https://www.nytimes3xbfgragh.onion/2016/11/22/science/nasa-webb-space-telescope-hubble.html}{James
Webb Space Telescope}, is
\href{https://www.nytimes3xbfgragh.onion/2018/06/27/science/webb-telescope-nasa.html}{at
last launched next year},
\href{https://www.nytimes3xbfgragh.onion/2020/03/19/science/nasa-coronavirus-sls-rocket-moon.html}{we
hope}, giving the world two peerless eyes in the sky.

\includegraphics{https://static01.graylady3jvrrxbe.onion/images/2020/04/24/science/24SCI-HUBBLE2/merlin_145026177_f24c9044-c502-4a84-b4de-3bbe11fecfe5-articleLarge.jpg?quality=75\&auto=webp\&disable=upscale}

Among Hubble's triumphs over the decades have been
\href{https://www.nytimes3xbfgragh.onion/2019/02/25/science/cosmos-hubble-dark-energy.html}{charting
the effects of dark energy on the growth of the universe}, surveying the
weather throughout the solar system and measuring the expansion of the
universe precisely enough to challenge battle-tested theories of
cosmology.

And in pictures such as this 30th birthday portrait, it has documented
the violent births and deaths of stars obeying Mr. Dylan's dictum.
\href{https://hubblesite.org/contents/news-releases/2020/news-2020-16\#section-id-2}{Entitled
``Cosmic Reef,''} in an allusion to the richness of undersea life, it
shows a so-called stellar nursery in the Large Magellanic Cloud, a
satellite galaxy of our Milky Way, about 163,000 light years from here.

The reddish nebula at upper right is full of stars at least 10 times as
massive as the sun. Winds of particles and radiation coming off them
push the gas in the nebula into waves and bubbles from which new stars
will eventually form.

Below it, a single giant star 200,000 times brighter than the sun has
blown its own blue bubble of gas.

And the cosmic beat goes on. And so does Hubble.

\href{https://www.nytimes3xbfgragh.onion/interactive/2015/04/23/science/space/unforgettable-hubble-space-telescope-photos.html}{}

\includegraphics{https://static01.graylady3jvrrxbe.onion/images/2015/04/23/science/hubble_25/hubble_25-videoLarge.jpg}

\hypertarget{unforgettable-hubble-space-telescope-photos}{%
\subsection{Unforgettable Hubble Space Telescope
Photos}\label{unforgettable-hubble-space-telescope-photos}}

On the eve of the 25th anniversary of the launch of the Hubble Space
Telescope, we asked astronomers and others involved in the telescope's
groundbreaking story to tell us about their favorite images.

Advertisement

\protect\hyperlink{after-bottom}{Continue reading the main story}

\hypertarget{site-index}{%
\subsection{Site Index}\label{site-index}}

\hypertarget{site-information-navigation}{%
\subsection{Site Information
Navigation}\label{site-information-navigation}}

\begin{itemize}
\tightlist
\item
  \href{https://help.nytimes3xbfgragh.onion/hc/en-us/articles/115014792127-Copyright-notice}{©~2020~The
  New York Times Company}
\end{itemize}

\begin{itemize}
\tightlist
\item
  \href{https://www.nytco.com/}{NYTCo}
\item
  \href{https://help.nytimes3xbfgragh.onion/hc/en-us/articles/115015385887-Contact-Us}{Contact
  Us}
\item
  \href{https://www.nytco.com/careers/}{Work with us}
\item
  \href{https://nytmediakit.com/}{Advertise}
\item
  \href{http://www.tbrandstudio.com/}{T Brand Studio}
\item
  \href{https://www.nytimes3xbfgragh.onion/privacy/cookie-policy\#how-do-i-manage-trackers}{Your
  Ad Choices}
\item
  \href{https://www.nytimes3xbfgragh.onion/privacy}{Privacy}
\item
  \href{https://help.nytimes3xbfgragh.onion/hc/en-us/articles/115014893428-Terms-of-service}{Terms
  of Service}
\item
  \href{https://help.nytimes3xbfgragh.onion/hc/en-us/articles/115014893968-Terms-of-sale}{Terms
  of Sale}
\item
  \href{https://spiderbites.nytimes3xbfgragh.onion}{Site Map}
\item
  \href{https://help.nytimes3xbfgragh.onion/hc/en-us}{Help}
\item
  \href{https://www.nytimes3xbfgragh.onion/subscription?campaignId=37WXW}{Subscriptions}
\end{itemize}
