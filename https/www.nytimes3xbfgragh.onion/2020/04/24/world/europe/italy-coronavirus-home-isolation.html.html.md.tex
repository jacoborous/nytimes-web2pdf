Sections

SEARCH

\protect\hyperlink{site-content}{Skip to
content}\protect\hyperlink{site-index}{Skip to site index}

\href{https://www.nytimes3xbfgragh.onion/section/world/europe}{Europe}

\href{https://myaccount.nytimes3xbfgragh.onion/auth/login?response_type=cookie\&client_id=vi}{}

\href{https://www.nytimes3xbfgragh.onion/section/todayspaper}{Today's
Paper}

\href{/section/world/europe}{Europe}\textbar{}Isolating the Sick at
Home, Italy Stores Up Family Tragedies

\url{https://nyti.ms/2VxxDx0}

\begin{itemize}
\item
\item
\item
\item
\item
\item
\end{itemize}

\hypertarget{the-coronavirus-outbreak}{%
\subsubsection{\texorpdfstring{\href{https://www.nytimes3xbfgragh.onion/news-event/coronavirus?name=styln-coronavirus-national\&region=TOP_BANNER\&block=storyline_menu_recirc\&action=click\&pgtype=Article\&impression_id=f59e4fd0-efba-11ea-943d-2f0afe93ad3a\&variant=undefined}{The
Coronavirus
Outbreak}}{The Coronavirus Outbreak}}\label{the-coronavirus-outbreak}}

\begin{itemize}
\tightlist
\item
  live\href{https://www.nytimes3xbfgragh.onion/2020/09/05/world/coronavirus-covid.html?name=styln-coronavirus-national\&region=TOP_BANNER\&block=storyline_menu_recirc\&action=click\&pgtype=Article\&impression_id=f59e76e0-efba-11ea-943d-2f0afe93ad3a\&variant=undefined}{Latest
  Updates}
\item
  \href{https://www.nytimes3xbfgragh.onion/interactive/2020/us/coronavirus-us-cases.html?name=styln-coronavirus-national\&region=TOP_BANNER\&block=storyline_menu_recirc\&action=click\&pgtype=Article\&impression_id=f59e76e1-efba-11ea-943d-2f0afe93ad3a\&variant=undefined}{Maps
  and Cases}
\item
  \href{https://www.nytimes3xbfgragh.onion/interactive/2020/science/coronavirus-vaccine-tracker.html?name=styln-coronavirus-national\&region=TOP_BANNER\&block=storyline_menu_recirc\&action=click\&pgtype=Article\&impression_id=f59e76e2-efba-11ea-943d-2f0afe93ad3a\&variant=undefined}{Vaccine
  Tracker}
\item
  \href{https://www.nytimes3xbfgragh.onion/2020/09/02/your-money/eviction-moratorium-covid.html?name=styln-coronavirus-national\&region=TOP_BANNER\&block=storyline_menu_recirc\&action=click\&pgtype=Article\&impression_id=f59e76e3-efba-11ea-943d-2f0afe93ad3a\&variant=undefined}{Eviction
  Moratorium}
\item
  \href{https://www.nytimes3xbfgragh.onion/interactive/2020/09/02/magazine/food-insecurity-hunger-us.html?name=styln-coronavirus-national\&region=TOP_BANNER\&block=storyline_menu_recirc\&action=click\&pgtype=Article\&impression_id=f59e76e4-efba-11ea-943d-2f0afe93ad3a\&variant=undefined}{American
  Hunger}
\end{itemize}

Advertisement

\protect\hyperlink{after-top}{Continue reading the main story}

Supported by

\protect\hyperlink{after-sponsor}{Continue reading the main story}

\hypertarget{isolating-the-sick-at-home-italy-stores-up-family-tragedies}{%
\section{Isolating the Sick at Home, Italy Stores Up Family
Tragedies}\label{isolating-the-sick-at-home-italy-stores-up-family-tragedies}}

``Stay home'' measures have helped Italy control the coronavirus, but
home is also a dangerous place that may be propping up the infection
curve the lockdown was meant to suppress.

\includegraphics{https://static01.graylady3jvrrxbe.onion/images/2020/04/22/world/00virus-italy-home1/merlin_171750207_748373a4-adb4-4190-a827-285c2daf3f18-articleLarge.jpg?quality=75\&auto=webp\&disable=upscale}

By \href{https://www.nytimes3xbfgragh.onion/by/jason-horowitz}{Jason
Horowitz} and
\href{https://www.nytimes3xbfgragh.onion/by/emma-bubola}{Emma Bubola}

\begin{itemize}
\item
  Published April 24, 2020Updated May 6, 2020
\item
  \begin{itemize}
  \item
  \item
  \item
  \item
  \item
  \item
  \end{itemize}
\end{itemize}

ROME --- When her middle-aged son got sick, Ruffina Pompei did what she
had done for decades, bringing vegetable soup and freshly squeezed
orange juice to his room. She slept in an armchair outside his room and
changed his clothes. She told her husband, 89, to steer clear.

But the
\href{https://www.nytimes3xbfgragh.onion/2020/05/06/world/europe/italy-coronavirus-reopening-parents.html}{coronavirus}
tore through the apartment.

Her son died in a hospital in the region of Abruzzo on March 29. Her
husband died the next day in the same hospital. Ms. Pompei, 82, was also
diagnosed with the virus.

``I could not leave him alone,'' she said of her son.

Before everyone else in the West, Italians received and largely obeyed
an order to stay at home. ``I'm staying home'' became a hashtag, then
the name of a national ordinance and then a motto hung from balconies
and windows. But while staying home has worked, reducing the rate of
infections, bringing down the daily toll of the dead and creating
breathing room for hospitals, home has become a dangerous place for many
\href{https://www.nytimes3xbfgragh.onion/2020/05/29/world/europe/italy-young-people-coronavirus.html}{Italians}.

Italian households represent ``the biggest reservoir of infections,''
said Massimo Galli, the director of the infectious diseases department
at Luigi Sacco University Hospital in Milan. He called the cases ``the
possible restarting point of the epidemic in case of a reopening.''

The family acts as a multiplier, said Andrea Crisanti, the top
scientific consultant on the virus in the Veneto region. ``This is a
ticking time bomb,'' he said.

The predicament of home infections is emerging not just in
\href{https://www.nytimes3xbfgragh.onion/2020/05/06/world/europe/italy-coronavirus-reopening-parents.html}{Italy}
but in hot spots across the globe,
\href{https://www.nytimes3xbfgragh.onion/2020/04/09/nyregion/coronavirus-queens-corona-jackson-heights-elmhurst.html}{in
Queens} and the
\href{https://www.nytimes3xbfgragh.onion/2020/04/10/world/europe/coronavirus-paris-suburbs.html}{Paris
suburbs}, as well as the working-class neighborhoods of Rome and Milan.
It is also a problem that local officials and epidemiologists say is
getting too little attention, particularly as the government has
announced tentative steps toward reopening in early May.

\includegraphics{https://static01.graylady3jvrrxbe.onion/images/2020/03/27/world/00virus-italy-home2/merlin_171024219_15758957-b352-4ec9-9a25-6cde5878fbb9-articleLarge.jpg?quality=75\&auto=webp\&disable=upscale}

Italy's leading virologists now consider home infections, alongside
clusters in retirement homes, to be a stubborn source of the country's
contagion. Living together in close quarters and the failure to move the
infected into dedicated quarantine facilities have, they say,
paradoxically propped up the curve of infections that ``stay home''
measures were designed to suppress.

The problem is one the Chinese government bludgeoned quickly. It
\href{https://www.nytimes3xbfgragh.onion/2020/02/06/world/asia/coronavirus-china.html?action=click\&module=RelatedLinks\&pgtype=Article}{ordered
the roundup of all residents in Wuhan infected} with the coronavirus,
warehousing them in quarantine camps, sometimes with little care. While
that approach may have helped contain the virus, ripping people apart
from their homes is anathema to Western democracies, especially Italy,
where tight-knit families are the rule.

Italy, like other Western democracies, has wrestled with the difficulty
of balancing the containment of the virus with the economic, social and
political costs of removing people without symptoms from their own
homes. The government has not made isolating patients outside their
homes a priority.

\hypertarget{latest-updates-the-coronavirus-outbreak}{%
\section{\texorpdfstring{\href{https://www.nytimes3xbfgragh.onion/2020/09/04/world/covid-19-coronavirus.html?action=click\&pgtype=Article\&state=default\&region=MAIN_CONTENT_1\&context=storylines_live_updates}{Latest
Updates: The Coronavirus
Outbreak}}{Latest Updates: The Coronavirus Outbreak}}\label{latest-updates-the-coronavirus-outbreak}}

Updated 2020-09-05T12:05:40.998Z

\begin{itemize}
\tightlist
\item
  \href{https://www.nytimes3xbfgragh.onion/2020/09/04/world/covid-19-coronavirus.html?action=click\&pgtype=Article\&state=default\&region=MAIN_CONTENT_1\&context=storylines_live_updates\#link-1654f6ad}{Research
  connects vaping to a higher chance of catching the virus --- and
  suffering its worst effects.}
\item
  \href{https://www.nytimes3xbfgragh.onion/2020/09/04/world/covid-19-coronavirus.html?action=click\&pgtype=Article\&state=default\&region=MAIN_CONTENT_1\&context=storylines_live_updates\#link-52e4198a}{Another
  college football game won't be played as planned.}
\item
  \href{https://www.nytimes3xbfgragh.onion/2020/09/04/world/covid-19-coronavirus.html?action=click\&pgtype=Article\&state=default\&region=MAIN_CONTENT_1\&context=storylines_live_updates\#link-181cef0}{Pharmaceutical
  companies plan a joint pledge on safety standards as they move
  vaccines to the marketplace.}
\end{itemize}

\href{https://www.nytimes3xbfgragh.onion/2020/09/04/world/covid-19-coronavirus.html?action=click\&pgtype=Article\&state=default\&region=MAIN_CONTENT_1\&context=storylines_live_updates}{See
more updates}

More live coverage:
\href{https://www.nytimes3xbfgragh.onion/live/2020/09/04/business/stock-market-today-coronavirus?action=click\&pgtype=Article\&state=default\&region=MAIN_CONTENT_1\&context=storylines_live_updates}{Markets}

``As a doctor I would say let's put tanks in the streets and let's do a
police state,'' said Guido Marinoni, the president of the Bergamo
doctors' association. ``But the Western world has different realities.''

Italy has not enacted, or articulated, a clear national effort to
prevent contagious people from infecting their households. The country
has essentially accepted a controlled tragedy at home as it focuses on
preventing a contagion from running like wildfire through the society at
large.

``Domestic contagion is the lesser evil,'' said Giorgio Palù, a former
professor of virology and microbiology of the University of Padova and
the former head of the European and Italian Society for Virology.
Compared to unleashing the contagion on the streets, it was better to
keep the virus in the family. ``At home,'' he said, ``I block it.''

Image

An almost empty street in Milan~this month.Credit...Alessandro Grassani
for The New York Times

Silvio Brusaferro, the president of Italy's national health institute,
and one of the leading advisers to the Italian government in the crisis,
acknowledged that homes were ``higher-risk places,'' and on Friday,
demonstrated that family infections accounted for up to 25 percent of
new cases. But he said that what was essential was that home infections
``do not spread further.''

But that offers little solace to the close-knit, multigenerational
families decimated by the virus in their own homes --- to the infected
sisters who lost their father and then saw their grandfather
hospitalized in hard-hit
\href{https://www.nytimes3xbfgragh.onion/interactive/2020/03/27/world/europe/coronavirus-italy-bergamo.html?searchResultPosition=8}{Bergamo};
to
\href{https://www.nytimes3xbfgragh.onion/2020/04/21/world/europe/italy-coronavirus-south.html?searchResultPosition=3}{the
Campanian} hairdresser who lost both her parents; to the metalworker in
Voghera who died days after losing his two sons.

Experts have estimated that more than a million people in Italy could be
infected with the virus at home. The persistence of cases is telling.
Despite a nationwide drop in the number of new infections and deaths,
Lombardy, which remains the Italian epicenter, announced more than 5,000
new cases this week alone, with nearly 900 deaths.

Many public health officials say that the actual number of infections
could be as many as 10 times that.

Federico Ricci-Tersenghi, a scientist at the University of Rome La
Sapienza who specializes in theoretical modeling, said that stemming the
contagion required facilities like in China dedicated to isolating
positive cases.

``Staying home is not the solution, not for the economy or for the
epidemic,'' he said. ``To reopen without having this in place is very
risky. It's probable that the epidemic will start up again.''

When a delegation of Chinese doctors came to Italy in March, they
emphasized the importance of prefabricated structures with a high number
of beds to isolate all the positive cases.

``They explained that it was essential to separate positive cases from
the family,'' recalled Giampietro Rupolo, the president of the Red Cross
in Padua, who was among those who greeted them. ``Otherwise it was
harder to contain.''

But Italian officials have clearly determined that housing the infected
in dedicated facilities is not feasible.

Giovanni Rezza, director of infectious diseases at the national health
institute, said that the government did not think that a centralized
effort was ``feasible, possible, appreciated.''

Image

In Milan, the Michelangelo Hotel was set aside for residents with
positive coronavirus test results.Credit...Alessandro Grassani for The
New York Times

Italian media
\href{https://www.corriere.it/cronache/20_aprile_20/coronavirus-fase-2-condizioni-le-regioni-che-vogliono-partire-prima-2a328f02-82bc-11ea-86b3-8aab0c7cf936.shtml}{have
reported} that the government is considering making the lifting of
lockdowns in Italy's regions contingent on local authorities providing
quarantine facilities for the infected. Experts have also emphasized the
importance of contact tracing and early diagnoses, since contagion can
also happen before a patient starts showing symptoms.

For now, however, Prime Minister Giuseppe Conte has largely steered
clear of the issue of home infection. A government decree in March
allowed local authorities to seize hotels to host patients who cannot
safely self-isolate at home. But, as is the case with many of the
government decrees, it is being interpreted and implemented differently
around the country.

\href{https://www.nytimes3xbfgragh.onion/news-event/coronavirus?action=click\&pgtype=Article\&state=default\&region=MAIN_CONTENT_3\&context=storylines_faq}{}

\hypertarget{the-coronavirus-outbreak-}{%
\subsubsection{The Coronavirus Outbreak
›}\label{the-coronavirus-outbreak-}}

\hypertarget{frequently-asked-questions}{%
\paragraph{Frequently Asked
Questions}\label{frequently-asked-questions}}

Updated September 4, 2020

\begin{itemize}
\item ~
  \hypertarget{what-are-the-symptoms-of-coronavirus}{%
  \paragraph{What are the symptoms of
  coronavirus?}\label{what-are-the-symptoms-of-coronavirus}}

  \begin{itemize}
  \tightlist
  \item
    In the beginning, the coronavirus
    \href{https://www.nytimes3xbfgragh.onion/article/coronavirus-facts-history.html?action=click\&pgtype=Article\&state=default\&region=MAIN_CONTENT_3\&context=storylines_faq\#link-6817bab5}{seemed
    like it was primarily a respiratory illness}~--- many patients had
    fever and chills, were weak and tired, and coughed a lot, though
    some people don't show many symptoms at all. Those who seemed
    sickest had pneumonia or acute respiratory distress syndrome and
    received supplemental oxygen. By now, doctors have identified many
    more symptoms and syndromes. In April,
    \href{https://www.nytimes3xbfgragh.onion/2020/04/27/health/coronavirus-symptoms-cdc.html?action=click\&pgtype=Article\&state=default\&region=MAIN_CONTENT_3\&context=storylines_faq}{the
    C.D.C. added to the list of early signs}~sore throat, fever, chills
    and muscle aches. Gastrointestinal upset, such as diarrhea and
    nausea, has also been observed. Another telltale sign of infection
    may be a sudden, profound diminution of one's
    \href{https://www.nytimes3xbfgragh.onion/2020/03/22/health/coronavirus-symptoms-smell-taste.html?action=click\&pgtype=Article\&state=default\&region=MAIN_CONTENT_3\&context=storylines_faq}{sense
    of smell and taste.}~Teenagers and young adults in some cases have
    developed painful red and purple lesions on their fingers and toes
    --- nicknamed ``Covid toe'' --- but few other serious symptoms.
  \end{itemize}
\item ~
  \hypertarget{why-is-it-safer-to-spend-time-together-outside}{%
  \paragraph{Why is it safer to spend time together
  outside?}\label{why-is-it-safer-to-spend-time-together-outside}}

  \begin{itemize}
  \tightlist
  \item
    \href{https://www.nytimes3xbfgragh.onion/2020/05/15/us/coronavirus-what-to-do-outside.html?action=click\&pgtype=Article\&state=default\&region=MAIN_CONTENT_3\&context=storylines_faq}{Outdoor
    gatherings}~lower risk because wind disperses viral droplets, and
    sunlight can kill some of the virus. Open spaces prevent the virus
    from building up in concentrated amounts and being inhaled, which
    can happen when infected people exhale in a confined space for long
    stretches of time, said Dr. Julian W. Tang, a virologist at the
    University of Leicester.
  \end{itemize}
\item ~
  \hypertarget{why-does-standing-six-feet-away-from-others-help}{%
  \paragraph{Why does standing six feet away from others
  help?}\label{why-does-standing-six-feet-away-from-others-help}}

  \begin{itemize}
  \tightlist
  \item
    The coronavirus spreads primarily through droplets from your mouth
    and nose, especially when you cough or sneeze. The C.D.C., one of
    the organizations using that measure,
    \href{https://www.nytimes3xbfgragh.onion/2020/04/14/health/coronavirus-six-feet.html?action=click\&pgtype=Article\&state=default\&region=MAIN_CONTENT_3\&context=storylines_faq}{bases
    its recommendation of six feet}~on the idea that most large droplets
    that people expel when they cough or sneeze will fall to the ground
    within six feet. But six feet has never been a magic number that
    guarantees complete protection. Sneezes, for instance, can launch
    droplets a lot farther than six feet,
    \href{https://jamanetwork.com/journals/jama/fullarticle/2763852}{according
    to a recent study}. It's a rule of thumb: You should be safest
    standing six feet apart outside, especially when it's windy. But
    keep a mask on at all times, even when you think you're far enough
    apart.
  \end{itemize}
\item ~
  \hypertarget{i-have-antibodies-am-i-now-immune}{%
  \paragraph{I have antibodies. Am I now
  immune?}\label{i-have-antibodies-am-i-now-immune}}

  \begin{itemize}
  \tightlist
  \item
    As of right
    now,\href{https://www.nytimes3xbfgragh.onion/2020/07/22/health/covid-antibodies-herd-immunity.html?action=click\&pgtype=Article\&state=default\&region=MAIN_CONTENT_3\&context=storylines_faq}{~that
    seems likely, for at least several months.}~There have been
    frightening accounts of people suffering what seems to be a second
    bout of Covid-19. But experts say these patients may have a
    drawn-out course of infection, with the virus taking a slow toll
    weeks to months after initial exposure.~People infected with the
    coronavirus typically
    \href{https://www.nature.com/articles/s41586-020-2456-9}{produce}~immune
    molecules called antibodies, which are
    \href{https://www.nytimes3xbfgragh.onion/2020/05/07/health/coronavirus-antibody-prevalence.html?action=click\&pgtype=Article\&state=default\&region=MAIN_CONTENT_3\&context=storylines_faq}{protective
    proteins made in response to an
    infection}\href{https://www.nytimes3xbfgragh.onion/2020/05/07/health/coronavirus-antibody-prevalence.html?action=click\&pgtype=Article\&state=default\&region=MAIN_CONTENT_3\&context=storylines_faq}{.
    These antibodies may}~last in the body
    \href{https://www.nature.com/articles/s41591-020-0965-6}{only two to
    three months}, which may seem worrisome, but that's~perfectly normal
    after an acute infection subsides, said Dr. Michael Mina, an
    immunologist at Harvard University. It may be possible to get the
    coronavirus again, but it's highly unlikely that it would be
    possible in a short window of time from initial infection or make
    people sicker the second time.
  \end{itemize}
\item ~
  \hypertarget{what-are-my-rights-if-i-am-worried-about-going-back-to-work}{%
  \paragraph{What are my rights if I am worried about going back to
  work?}\label{what-are-my-rights-if-i-am-worried-about-going-back-to-work}}

  \begin{itemize}
  \tightlist
  \item
    Employers have to provide
    \href{https://www.osha.gov/SLTC/covid-19/standards.html}{a safe
    workplace}~with policies that protect everyone equally.
    \href{https://www.nytimes3xbfgragh.onion/article/coronavirus-money-unemployment.html?action=click\&pgtype=Article\&state=default\&region=MAIN_CONTENT_3\&context=storylines_faq}{And
    if one of your co-workers tests positive for the coronavirus, the
    C.D.C.}~has said that
    \href{https://www.cdc.gov/coronavirus/2019-ncov/community/guidance-business-response.html}{employers
    should tell their employees}~-\/- without giving you the sick
    employee's name -\/- that they may have been exposed to the virus.
  \end{itemize}
\end{itemize}

In most cases, only people with confirmed cases can go to dedicated
hotel rooms. But the hotel rooms are not available everywhere, and
neither are tests for the virus, which in the hardest-hit places are
given only to people who have been hospitalized.

Officials in Tuscany have urged people who left the hospital with mild
symptoms to isolate in converted hotels. But urge is all they could do,
and by mid-April only 200 had chosen the facilities. Others preferred to
sign a document declaring they would self-isolate at home. In Milan, the
Michelangelo Hotel, set aside for residents with positive coronavirus
test results, was far from full.

In Bergamo, the hardest-hit part of Italy, there are only 400 spots set
aside in local hotels for an infected population at home estimated to be
around 65,000, according to the local doctors' association.

Working-class Italians often face the hardest choices.

In the southern region of Calabria, Paolina Mazza, 63, who has the
coronavirus, self-isolated at home after her husband was hospitalized
with the virus. She expressed frustration that authorities gave her no
alternative but to go home, where she tried to keep some distance from
her son, 39.

``We don't have a second house to isolate and they live in a tiny
apartment,'' her daughter Daniela said. ``We are constantly scared.''

Even some doctors considered the infection of their own families as
inevitable.

Federica Brena, a 35-year-old doctor in the coronavirus ward of
Bergamo's Humanitas Gavazzeni hospital, came down with symptoms and
immediately diagnosed herself and self-isolated at home, with her
husband and 1-year-old son.

``The ideal option would have been to never go back home,'' she said.
``It was obvious that I would infect the two of them.''

Her son came down with a fever and a cough. Her husband also got sick a
few days later, and his fever did not go down for 12 days. ``Living in a
home with other people, for as much as one tries to isolate, it's
hard,'' she said, adding everyone was improving. ``Especially if you
don't live in a palace.''

Self-isolating was also especially difficult for older people, who often
need special care and attention.

In late February, Emanuele Visigalli and his brother looked after their
mother, 79, who had come down with a cough and temperature in the
Lombardy town of Fombio, near the country's initial outbreak.

Image

Emanuele Visigalli surrounded by his family at home in Codogno. ``At the
dinner table, I sit on one side of the table and they all sit on the
other side,'' he said.Credit...Fabio Bucciarelli for The New York Times

He tried to get her hospitalized, but dispatchers from the coronavirus
hotline suggested she self-isolate at home. The idea of abandoning them,
he said, was impossible.

An ambulance first came for his mother. Then for his father, 81. Then, a
week later, another came for him. Both his parents died in the hospital,
but he improved and doctors sent him home.

He said he did not hug his children or kiss his wife when he got back,
and started sleeping in a separate room that his wife disinfected every
day.

``At the dinner table, I sit on one side of the table and they all sit
on the other side,'' he said. Looking out his window, he said he grew
upset by the people walking the streets. ``What is their problem with
staying at home?''

Advertisement

\protect\hyperlink{after-bottom}{Continue reading the main story}

\hypertarget{site-index}{%
\subsection{Site Index}\label{site-index}}

\hypertarget{site-information-navigation}{%
\subsection{Site Information
Navigation}\label{site-information-navigation}}

\begin{itemize}
\tightlist
\item
  \href{https://help.nytimes3xbfgragh.onion/hc/en-us/articles/115014792127-Copyright-notice}{©~2020~The
  New York Times Company}
\end{itemize}

\begin{itemize}
\tightlist
\item
  \href{https://www.nytco.com/}{NYTCo}
\item
  \href{https://help.nytimes3xbfgragh.onion/hc/en-us/articles/115015385887-Contact-Us}{Contact
  Us}
\item
  \href{https://www.nytco.com/careers/}{Work with us}
\item
  \href{https://nytmediakit.com/}{Advertise}
\item
  \href{http://www.tbrandstudio.com/}{T Brand Studio}
\item
  \href{https://www.nytimes3xbfgragh.onion/privacy/cookie-policy\#how-do-i-manage-trackers}{Your
  Ad Choices}
\item
  \href{https://www.nytimes3xbfgragh.onion/privacy}{Privacy}
\item
  \href{https://help.nytimes3xbfgragh.onion/hc/en-us/articles/115014893428-Terms-of-service}{Terms
  of Service}
\item
  \href{https://help.nytimes3xbfgragh.onion/hc/en-us/articles/115014893968-Terms-of-sale}{Terms
  of Sale}
\item
  \href{https://spiderbites.nytimes3xbfgragh.onion}{Site Map}
\item
  \href{https://help.nytimes3xbfgragh.onion/hc/en-us}{Help}
\item
  \href{https://www.nytimes3xbfgragh.onion/subscription?campaignId=37WXW}{Subscriptions}
\end{itemize}
