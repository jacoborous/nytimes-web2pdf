Sections

SEARCH

\protect\hyperlink{site-content}{Skip to
content}\protect\hyperlink{site-index}{Skip to site index}

\href{https://www.nytimes3xbfgragh.onion/section/arts/music}{Music}

\href{https://myaccount.nytimes3xbfgragh.onion/auth/login?response_type=cookie\&client_id=vi}{}

\href{https://www.nytimes3xbfgragh.onion/section/todayspaper}{Today's
Paper}

\href{/section/arts/music}{Music}\textbar{}John Prine, Who Chronicled
the Human Condition in Song, Dies at 73

\url{https://nyti.ms/2Rj7NdB}

\begin{itemize}
\item
\item
\item
\item
\item
\item
\end{itemize}

\href{https://www.nytimes3xbfgragh.onion/news-event/coronavirus?action=click\&pgtype=Article\&state=default\&region=TOP_BANNER\&context=storylines_menu}{The
Coronavirus Outbreak}

\begin{itemize}
\tightlist
\item
  live\href{https://www.nytimes3xbfgragh.onion/2020/08/04/world/coronavirus-covid-19.html?action=click\&pgtype=Article\&state=default\&region=TOP_BANNER\&context=storylines_menu}{Latest
  Updates}
\item
  \href{https://www.nytimes3xbfgragh.onion/interactive/2020/us/coronavirus-us-cases.html?action=click\&pgtype=Article\&state=default\&region=TOP_BANNER\&context=storylines_menu}{Maps
  and Cases}
\item
  \href{https://www.nytimes3xbfgragh.onion/interactive/2020/science/coronavirus-vaccine-tracker.html?action=click\&pgtype=Article\&state=default\&region=TOP_BANNER\&context=storylines_menu}{Vaccine
  Tracker}
\item
  \href{https://www.nytimes3xbfgragh.onion/2020/08/02/us/covid-college-reopening.html?action=click\&pgtype=Article\&state=default\&region=TOP_BANNER\&context=storylines_menu}{College
  Reopening}
\item
  \href{https://www.nytimes3xbfgragh.onion/live/2020/08/03/business/stock-market-today-coronavirus?action=click\&pgtype=Article\&state=default\&region=TOP_BANNER\&context=storylines_menu}{Economy}
\end{itemize}

Advertisement

\protect\hyperlink{after-top}{Continue reading the main story}

Supported by

\protect\hyperlink{after-sponsor}{Continue reading the main story}

Those we've lost

\hypertarget{john-prine-who-chronicled-the-human-condition-in-song-dies-at-73}{%
\section{John Prine, Who Chronicled the Human Condition in Song, Dies at
73}\label{john-prine-who-chronicled-the-human-condition-in-song-dies-at-73}}

A singer and songwriter with a raspy voice and a gift for offbeat humor,
he was revered by his peers, including Bob Dylan. He died of the
coronavirus.

\includegraphics{https://static01.graylady3jvrrxbe.onion/images/2020/04/09/obituaries/09prine1/merlin_36473839_a6429490-8597-41fe-a497-0e482cfd32de-articleLarge.jpg?quality=75\&auto=webp\&disable=upscale}

By \href{https://www.nytimes3xbfgragh.onion/by/william-grimes}{William
Grimes}

\begin{itemize}
\item
  Published April 7, 2020Updated April 16, 2020
\item
  \begin{itemize}
  \item
  \item
  \item
  \item
  \item
  \item
  \end{itemize}
\end{itemize}

\emph{This obituary is part of a series about}
\href{https://www.nytimes3xbfgragh.onion/series/people-who-have-died-of-the-coronavirus}{\emph{people
who have died in the coronavirus pandemic}}\emph{.}

John Prine, the raspy-voiced country-folk singer whose ingenious lyrics
to songs by turns poignant, angry and comic made him a favorite of Bob
Dylan, Kris Kristofferson and others, died on Tuesday in Nashville. He
was 73.

The cause was complications of the coronavirus, his family said.

Mr. Prine
\href{https://oralcancerfoundation.org/people/arts-entertainment/john-prine/}{underwent
cancer surgery in 1998} to remove a tumor in his neck identified as
squamous cell cancer, which had damaged his vocal cords. In 2013, he had
part of one lung removed to treat lung cancer. He had been hospitalized
since late last month.

Mr. Prine was a relative unknown in 1970 when Mr. Kristofferson heard
him play one night at a Chicago club called the Earl of Old Town,
dragged there by the singer-songwriter Steve Goodman. Mr. Kristofferson
was performing in Chicago at the time at the Quiet Knight. Mr. Prine
treated him to a brief after-hours performance of material that, Mr.
Kristofferson later wrote, ``was unlike anything I'd heard before.''

\includegraphics{https://static01.graylady3jvrrxbe.onion/images/2020/03/29/us/00prine-image7/00prine-image7-articleLarge.jpg?quality=75\&auto=webp\&disable=upscale}

A few weeks later, when Mr. Prine was in New York, Mr. Kristofferson
invited him onstage at the Bitter End in Greenwich Village, where he was
appearing with Carly Simon, and introduced him to the audience.

``No way somebody this young can be writing so heavy,'' he said. ``John
Prine is so good, we may have to break his thumbs.''

The record executive Jerry Wexler, who was in the audience, signed Mr.
Prine to a contract with Atlantic Records the next day.

Music writers at the time were eager to crown a successor to Mr. Dylan,
and Mr. Prine, with his nasal, sandpapery voice and literate way with a
song, came ready to order. His debut album, called simply ``John Prine''
and released in 1971, included songs that became his signatures. Some
gained wider fame after being recorded by other artists.

They included ``Sam Stone,'' about a drug-addicted war veteran (with the
unforgettable refrain ``There's a hole in Daddy's arm where all the
money goes''); ``Hello in There,'' a heart-rending evocation of old age
and loneliness; and ``Angel From Montgomery,'' the hard-luck lament of a
middle-aged woman dreaming about a better life, later made famous by
Bonnie Raitt.

``He's a true folk singer in the best folk tradition, cutting right to
the heart of things, as pure and simple as rain,'' Ms. Raitt told
Rolling Stone in 1992.

Mr. Dylan, listing his favorite songwriters in
\href{https://www.huffpost.com/entry/bob-dylan-exclusive-inter_n_187216}{a
2009 interview}, put Mr. Prine front and center. ``Prine's stuff is pure
Proustian existentialism,'' he said. ``Midwestern mind trips to the nth
degree. And he writes beautiful songs.''

Image

Mr. Prine in 1993 with Bonnie Raitt, who made famous his song ``Angel
From Montgomery,'' the hard-luck lament of a middle-aged woman dreaming
about a better life.Credit...Denise Sofranko/Michael Ochs Archives via
Getty Images

John Prine was born on Oct. 10, 1946, in Maywood, Ill., a working-class
suburb of Chicago, to William and Verna (Hamm) Prine. His father, a
tool-and-die maker at the American Can Company, and his mother had moved
from the coal town of Paradise, Ky., in the 1930s.

Mr. Prine later wrote a ruefully bitter song titled
\href{https://www.youtube.com/watch?v=ediaZ5DhYjw}{``Paradise,''} in
which he sang:

\begin{quote}
The coal company came with the world's largest shovel\\
And they tortured the timber and stripped all the land\\
Well, they dug for their coal till the land was forsaken\\
Then they wrote it all down as the progress of man
\end{quote}

John grew up in a country music-loving family. He learned guitar as a
young teenager from his grandfather and brother and began writing songs.

After graduating from high school, he worked for the Post Office for two
years before being drafted into the Army, which sent him to West Germany
in charge of the motor pool at his base. After being discharged, he
resumed his mail route, in and around his hometown, composing songs in
his head.

``I always likened the mail route to a library with no books,'' he wrote
on his \href{http://johnprine.net/}{website}. ``I passed the time each
day making up these little ditties.''

Image

Mr. Prine in 1992 when he won the Grammy Award for best contemporary
folk album.Credit...Rick Maiman/Sygma via Getty Images)

Reluctantly, he took the stage for the first time at an open-mic night
at a small Chicago club called the Fifth Peg, where his performance met
with profound silence from the audience. ``They just sat there,'' Mr.
Prine later wrote. ``They didn't even applaud, they just looked at me.''

Then the clapping began. ``It was like I found out all of a sudden that
I could communicate deep feelings and emotions,'' he wrote. ``And to
find that out all at once was amazing.''

Not long after, Roger Ebert, the film critic for The Chicago Sun-Times,
wandered into the club while Mr. Prine was performing. He liked what he
heard and wrote Mr. Prine's
\href{http://www.rogerebert.com/balder-and-dash/john-prine-american-legend}{first
review}, under the headline ``Singing Mailman Who Delivers a Powerful
Message in a Few Words.''

``He appears onstage with such modesty he almost seems to be backing
into the spotlight,'' Mr. Ebert wrote. ``He sings rather quietly, and
his guitar work is good, but he doesn't show off. He starts slow. But
after a song or two, even the drunks in the room begin to listen to his
lyrics. And then he has you.''

Mr. Prine had a particular gift for offbeat humor, reflected in songs
like ``Jesus, the Missing Years,'' ``Some Humans Ain't Human,'' ``Sabu
Visits the Twin Cities Alone'' and the antiwar screed ``Your Flag Decal
Won't Get You Into Heaven Anymore.''

``I guess what I always found funny was the human condition,'' he told
the British newspaper
\href{http://www.telegraph.co.uk/culture/music/rockandpopfeatures/9853440/John-Prine-I-find-the-human-condition-funny.html}{The
Daily Telegraph} in 2013. ``There is a certain comedy and pathos to
trouble and accidents.''

After recording several albums for Atlantic and Asylum, he started his
own label, Oh Boy Records, in 1984. He never had a hit record, but he
commanded a loyal audience that ensured steady if modest sales for his
albums and a durable concert career.

In 1992, his album
\href{https://www.youtube.com/watch?v=6jKopYJfjeQ\&list=OLAK5uy_lBOa4ysIPMfrphlG1IqSDyeubh6uwC-vk}{``The
Missing Years,''} with guest appearances by Bruce Springsteen, Tom Petty
and other artists, won a Grammy Award for best contemporary folk
recording. He received a second Grammy in the same category in 2006 for
the album ``Fair and Square.''

Mr. Prine, who lived in Nashville, was divorced twice. He is survived by
his wife, Fiona Whelan Prine, a native of Ireland whom he married in
1996; three sons, Jody, Jack and Tommy; two brothers, Dave and Billy;
and three grandchildren.

In 2017, Mr. Prine published ``John Prine Beyond Words,'' a collection
of lyrics, guitar chords, commentary and photographs from his own
archive.

Image

``I guess what I always found funny was the human condition,'' Mr. Prine
once said.Credit...Kyle Dean Reinford for The New York Times

In 2019, he was inducted into the Songwriters Hall of Fame, and his
album
\href{https://www.youtube.com/watch?v=aAYoWePzQ2c\&list=OLAK5uy_mYUVcJHLfobY-NDHtulgiqGaq_vkwc1z4}{``Tree
of Forgiveness''} was nominated for a Grammy for best Americana album.
It was his 19th album and his first of original material in more than a
decade. (The award went to Brandi Carlile, for ``By the Way, I Forgive
You.'')

Mr. Prine went on tour in 2018 to promote ``Tree of Forgiveness,'' and
after a two-night stand at the Ryman Auditorium in Nashville --- known
there as the mother church of country music --- Margaret Renkl, a
contributing opinion writer for The New York Times, wrote, under the
headline
\href{https://www.nytimes3xbfgragh.onion/2018/10/22/opinion/john-prine-american-oracle.html}{``American
Oracle''}:

``The mother church of country music, where the seats are scratched-up
pews and the windows are stained glass, is the place where the new John
Prine --- older now, scarred by cancer surgeries, his voice deeper and
full of gravel --- is most clearly still the old John Prine:
mischievous, delighting in tomfoolery, but also worried about the
world.''

Image

Mr. Prine at the 62nd Grammy Awards in Los Angeles this January.~He
received a 2020 Grammy for lifetime achievement.Credit...Emma
Mcintyre/Getty Images For The Recording Academy

In December, he was chosen to receive
\href{https://www.rollingstone.com/music/music-country/john-prine-grammy-lifetime-achievement-929523/}{a
2020 Grammy for lifetime achievement}.

As a songwriter, Mr. Prine was prolific and quick. In the early days, he
would sometimes dash off a song while driving to a club.

``Sometimes, the best ones come together at the exact same time, and it
takes about as long to write it as it does to sing it,'' he told the
poet Ted Kooser in an
\href{http://www.loc.gov/today/cyberlc/feature_wdesc.php?rec=3677}{interview}
at the Library of Congress in 2005. ``They come along like a dream or
something, and you just got to hurry up and respond to it, because if
you mess around, the song is liable to pass you by.''

Caryn Ganz and Neil Vigdor contributed reporting.

\href{https://www.nytimes3xbfgragh.onion/interactive/2020/obituaries/people-died-coronavirus-obituaries.html?action=click\&pgtype=Article\&state=default\&region=BELOW_MAIN_CONTENT\&context=covid_obits_promo}{}

\hypertarget{those-weve-lost}{%
\section{Those We've Lost}\label{those-weve-lost}}

The coronavirus pandemic has taken an incalculable death toll. This
series is designed to put names and faces to the numbers.

Read more

\includegraphics{https://static01.graylady3jvrrxbe.onion/images/2020/07/30/obituaries/30Pedro/30Pedro-square640.jpg}

\hypertarget{bernaldina-josuxe9-pedro}{%
\section{Bernaldina José Pedro}\label{bernaldina-josuxe9-pedro}}

d. Boa Vista, Brazil

Leader among the Indigenous Macuxi

\includegraphics{https://static01.graylady3jvrrxbe.onion/images/2020/07/31/obituaries/31Swing/merlin_175167783_8913bc90-0d64-43f3-a655-1bb1bf1601c9-square640.jpg}

\hypertarget{john-eric-swing}{%
\section{John Eric Swing}\label{john-eric-swing}}

d. Fountain Valley, Calif.

Champion of Filipino-Americans

\includegraphics{https://static01.graylady3jvrrxbe.onion/images/2020/07/27/obituaries/27Victor/merlin_175001436_38b11f8e-227a-4e2c-9821-7618af9b2524-square640.jpg}

\hypertarget{victor-victor}{%
\section{Victor Victor}\label{victor-victor}}

d. Santo Domingo, Dominican Republic

Beloved musician of the Dominican Republic

\includegraphics{https://static01.graylady3jvrrxbe.onion/images/2020/07/31/obituaries/31Negron/merlin_175160169_516322ae-fd23-4969-b6b2-193ced371105-square640.jpg}

\hypertarget{dr-eddie-negruxf3n}{%
\section{Dr. Eddie Negrón}\label{dr-eddie-negruxf3n}}

d. Fort Walton Beach, Fla.

Internist on Florida's Emerald Coast

\includegraphics{https://static01.graylady3jvrrxbe.onion/images/2020/07/30/obituaries/30Dobson/merlin_175115928_f6b9271c-8f05-4fe1-a38a-5ca4a58f8935-square640.jpg}

\hypertarget{dobby-dobson}{%
\section{Dobby Dobson}\label{dobby-dobson}}

d. Coral Springs, Fla.

Jamaican singer and songwriter

\includegraphics{https://static01.graylady3jvrrxbe.onion/images/2020/08/01/obituaries/28Gonzalez/merlin_175002771_beb57888-3951-409a-ae13-03a94b2e962e-square640.jpg}

\hypertarget{waldemar-gonzalez}{%
\section{Waldemar Gonzalez}\label{waldemar-gonzalez}}

d. White Plains, N.Y.

Teacher and social worker

Advertisement

\protect\hyperlink{after-bottom}{Continue reading the main story}

\hypertarget{site-index}{%
\subsection{Site Index}\label{site-index}}

\hypertarget{site-information-navigation}{%
\subsection{Site Information
Navigation}\label{site-information-navigation}}

\begin{itemize}
\tightlist
\item
  \href{https://help.nytimes3xbfgragh.onion/hc/en-us/articles/115014792127-Copyright-notice}{©~2020~The
  New York Times Company}
\end{itemize}

\begin{itemize}
\tightlist
\item
  \href{https://www.nytco.com/}{NYTCo}
\item
  \href{https://help.nytimes3xbfgragh.onion/hc/en-us/articles/115015385887-Contact-Us}{Contact
  Us}
\item
  \href{https://www.nytco.com/careers/}{Work with us}
\item
  \href{https://nytmediakit.com/}{Advertise}
\item
  \href{http://www.tbrandstudio.com/}{T Brand Studio}
\item
  \href{https://www.nytimes3xbfgragh.onion/privacy/cookie-policy\#how-do-i-manage-trackers}{Your
  Ad Choices}
\item
  \href{https://www.nytimes3xbfgragh.onion/privacy}{Privacy}
\item
  \href{https://help.nytimes3xbfgragh.onion/hc/en-us/articles/115014893428-Terms-of-service}{Terms
  of Service}
\item
  \href{https://help.nytimes3xbfgragh.onion/hc/en-us/articles/115014893968-Terms-of-sale}{Terms
  of Sale}
\item
  \href{https://spiderbites.nytimes3xbfgragh.onion}{Site Map}
\item
  \href{https://help.nytimes3xbfgragh.onion/hc/en-us}{Help}
\item
  \href{https://www.nytimes3xbfgragh.onion/subscription?campaignId=37WXW}{Subscriptions}
\end{itemize}
