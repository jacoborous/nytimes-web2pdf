Sections

SEARCH

\protect\hyperlink{site-content}{Skip to
content}\protect\hyperlink{site-index}{Skip to site index}

\href{/section/nyregion}{New York}\textbar{}`A Tragedy Is Unfolding':
Inside New York's Virus Epicenter

\url{https://nyti.ms/2JTeUFO}

\begin{itemize}
\item
\item
\item
\item
\item
\item
\end{itemize}

\hypertarget{the-coronavirus-outbreak}{%
\subsubsection{\texorpdfstring{\href{https://www.nytimes3xbfgragh.onion/news-event/coronavirus?name=styln-coronavirus-national\&region=TOP_BANNER\&variant=undefined\&block=storyline_menu_recirc\&action=click\&pgtype=Article\&impression_id=28e06040-e39f-11ea-8161-e7709b4b19cf}{The
Coronavirus
Outbreak}}{The Coronavirus Outbreak}}\label{the-coronavirus-outbreak}}

\begin{itemize}
\tightlist
\item
  live\href{https://www.nytimes3xbfgragh.onion/2020/08/21/world/covid-19-coronavirus.html?name=styln-coronavirus-national\&region=TOP_BANNER\&variant=undefined\&block=storyline_menu_recirc\&action=click\&pgtype=Article\&impression_id=28e08750-e39f-11ea-8161-e7709b4b19cf}{Latest
  Updates}
\item
  \href{https://www.nytimes3xbfgragh.onion/interactive/2020/us/coronavirus-us-cases.html?name=styln-coronavirus-national\&region=TOP_BANNER\&variant=undefined\&block=storyline_menu_recirc\&action=click\&pgtype=Article\&impression_id=28e08751-e39f-11ea-8161-e7709b4b19cf}{Maps
  and Cases}
\item
  \href{https://www.nytimes3xbfgragh.onion/interactive/2020/science/coronavirus-vaccine-tracker.html?name=styln-coronavirus-national\&region=TOP_BANNER\&variant=undefined\&block=storyline_menu_recirc\&action=click\&pgtype=Article\&impression_id=28e08752-e39f-11ea-8161-e7709b4b19cf}{Vaccine
  Tracker}
\item
  \href{https://www.nytimes3xbfgragh.onion/2020/08/19/us/colleges-closing-covid.html?name=styln-coronavirus-national\&region=TOP_BANNER\&variant=undefined\&block=storyline_menu_recirc\&action=click\&pgtype=Article\&impression_id=28e08753-e39f-11ea-8161-e7709b4b19cf}{Colleges
  Closing}
\item
  \href{https://www.nytimes3xbfgragh.onion/live/2020/08/20/business/stock-market-today-coronavirus?name=styln-coronavirus-national\&region=TOP_BANNER\&variant=undefined\&block=storyline_menu_recirc\&action=click\&pgtype=Article\&impression_id=28e0ae60-e39f-11ea-8161-e7709b4b19cf}{Economy}
\end{itemize}

\includegraphics{https://static01.graylady3jvrrxbe.onion/images/2020/04/10/nyregion/10nyvirus-queens-p-sub/merlin_171052038_b05a606b-2109-40ea-bf1a-55fa47bd647f-articleLarge.jpg?quality=75\&auto=webp\&disable=upscale}

\hypertarget{a-tragedy-is-unfolding-inside-new-yorks-virus-epicenter}{%
\section{`A Tragedy Is Unfolding': Inside New York's Virus
Epicenter}\label{a-tragedy-is-unfolding-inside-new-yorks-virus-epicenter}}

In a city ravaged by an epidemic, few places have been as hard hit as
central Queens.

Credit...

Supported by

\protect\hyperlink{after-sponsor}{Continue reading the main story}

By \href{https://www.nytimes3xbfgragh.onion/by/annie-correal}{Annie
Correal} and
\href{https://www.nytimes3xbfgragh.onion/by/andrew-jacobs}{Andrew
Jacobs}

Photographs by Ryan Christopher Jones

\begin{itemize}
\item
  Published April 9, 2020Updated Aug. 5, 2020
\item
  \begin{itemize}
  \item
  \item
  \item
  \item
  \item
  \item
  \end{itemize}
\end{itemize}

\href{https://www.nytimes3xbfgragh.onion/es/2020/04/10/espanol/mundo/coronavirus-queens-hospital-elmhurst-corona-jackson-heights.html}{Leer
en español}

Anil Subba, a Nepali Uber driver from Jackson Heights, Queens, died just
hours after doctors at Elmhurst Hospital thought he might be strong
enough to be removed from a ventilator.

In the nearby Corona neighborhood, Edison Forero, 44, a restaurant
worker from Colombia, was still burning with fever when his housemate
demanded he leave his rented room, he said.

Not far away in Jackson Heights, Raziah Begum, a widow and nanny from
Bangladesh, worries she will be ill soon. Two of her three roommates
already have the symptoms of
\href{https://www.nytimes3xbfgragh.onion/2020/08/05/nyregion/nyc-coronavirus-quarantine-checkpoints.html}{Covid-19},
the disease caused by the coronavirus. Everyone in the apartment is
jobless, and they eat one meal a day, she said.

``We are so hungry, but I am more terrified that I will get sick,'' said
Ms. Begum, 53, who has diabetes and high blood pressure.

In a city ravaged by the
\href{https://www.nytimes3xbfgragh.onion/2020/04/10/nyregion/new-york-coronavirus-death-count.html}{coronavirus},
few places have suffered as much as central Queens, where a
seven-square-mile patch of densely packed immigrant enclaves recorded
more than 7,000 cases in the first weeks of the outbreak.

Across
\href{https://www.nytimes3xbfgragh.onion/2020/08/05/nyregion/nyc-coronavirus-quarantine-checkpoints.html}{New
York}, there was a relatively encouraging sign on Thursday:
\href{https://www.nytimes3xbfgragh.onion/2020/04/09/nyregion/ny-coronavirus-hospitalizations-flattening-the-curve.html}{Hospitalizations
remained nearly flat fo}r the first time since the lockdown began.
Still, officials cautioned that it was too early to tell if the trend
would hold.

Deaths have continued to climb, and the state reached a new one-day high
of 799, according to figures released Thursday.

Gov. Philip D. Murphy of New Jersey, which has had more deaths than any
other state besides New York, also said the curve of infection seemed to
be flattening in his state. He and Gov. Andrew M. Cuomo of New York said
that social distancing measures would need to stay in place to keep up
the early progress.

\href{https://www.nytimes3xbfgragh.onion/interactive/2020/04/01/nyregion/nyc-coronavirus-cases-map.html}{In
the month since the virus exploded in New York}, it has claimed rich and
poor, the notable and the anonymous. But
\href{https://www.nytimes3xbfgragh.onion/2020/04/08/nyregion/coronavirus-new-york-update.html?}{as
the death toll has mounted}, the contagion has exposed the city's
stubborn inequities,
\href{https://www.nytimes3xbfgragh.onion/2020/04/07/nyregion/jamaica-hospital-queens-maria-correa-coronavirus.html}{tearing
through working-class immigrant neighborhoods} far more quickly than
others.

\includegraphics{https://static01.graylady3jvrrxbe.onion/images/2020/04/10/nyregion/10queens-jump3/merlin_171052032_12632e50-4c2e-4738-bfd6-c69cc8cad153-articleLarge.jpg?quality=75\&auto=webp\&disable=upscale}

A group of adjoining neighborhoods --- Corona, Elmhurst, East Elmhurst
and Jackson Heights --- have emerged as the epicenter of New York's
raging outbreak.

As of Wednesday, those communities, with a combined population of about
600,000, had recorded more than 7,260 coronavirus cases, according to
data collected by the New York City Department of Health and Mental
Hygiene. Manhattan, with nearly three times as many people, had about
10,860 cases.

Health officials have not released data on the race or ethnicity of the
people who are sick, and officials from the Department of City Planning
cautioned against drawing broad conclusions based on ZIP codes, which is
how the city has released limited information about positive cases.

\hypertarget{latest-updates-the-coronavirus-outbreak}{%
\section{\texorpdfstring{\href{https://www.nytimes3xbfgragh.onion/2020/08/21/world/covid-19-coronavirus.html?action=click\&pgtype=Article\&state=default\&region=MAIN_CONTENT_1\&context=storylines_live_updates}{Latest
Updates: The Coronavirus
Outbreak}}{Latest Updates: The Coronavirus Outbreak}}\label{latest-updates-the-coronavirus-outbreak}}

Updated 2020-08-21T11:05:09.310Z

\begin{itemize}
\tightlist
\item
  \href{https://www.nytimes3xbfgragh.onion/2020/08/21/world/covid-19-coronavirus.html?action=click\&pgtype=Article\&state=default\&region=MAIN_CONTENT_1\&context=storylines_live_updates\#link-4690b6aa}{Shutdowns,
  warnings and scoldings follow gatherings on college campuses.}
\item
  \href{https://www.nytimes3xbfgragh.onion/2020/08/21/world/covid-19-coronavirus.html?action=click\&pgtype=Article\&state=default\&region=MAIN_CONTENT_1\&context=storylines_live_updates\#link-324af071}{As
  he accepts the Democratic nomination, Biden knocks Trump's pandemic
  response.}
\item
  \href{https://www.nytimes3xbfgragh.onion/2020/08/21/world/covid-19-coronavirus.html?action=click\&pgtype=Article\&state=default\&region=MAIN_CONTENT_1\&context=storylines_live_updates\#link-35890b73}{Hundreds
  of doctors in Kenya go on strike over their pay and protective gear.}
\end{itemize}

\href{https://www.nytimes3xbfgragh.onion/2020/08/21/world/covid-19-coronavirus.html?action=click\&pgtype=Article\&state=default\&region=MAIN_CONTENT_1\&context=storylines_live_updates}{See
more updates}

More live coverage:
\href{https://www.nytimes3xbfgragh.onion/live/2020/08/20/business/stock-market-today-coronavirus?action=click\&pgtype=Article\&state=default\&region=MAIN_CONTENT_1\&context=storylines_live_updates}{Markets}

Yet health care workers and community leaders say it is indisputable
that the pandemic has disproportionately affected the Hispanic day
laborers, restaurant workers and cleaners who make up the largest share
of the population in an area often celebrated as one of the most diverse
places on earth.
\href{https://www.nytimes3xbfgragh.onion/2020/04/08/nyregion/coronavirus-race-deaths.html?}{Latinos
comprise 34 percent of the deaths in New York City}, the largest share
for any racial or ethnic group,
\href{https://covid19tracker.health.ny.gov/views/NYS-COVID19-Tracker/NYSDOHCOVID-19Tracker-Fatalities?}{according
to data released by state officials on Wednesday}.

The neighborhoods also have large communities of Indian, Bangladeshi,
Chinese, Filipino and Nepali people, and a score of other ethnicities
that have been devastated by the pandemic.

The city-run
\href{https://www.nytimes3xbfgragh.onion/2020/03/25/nyregion/nyc-coronavirus-hospitals.html}{Elmhurst
Hospital Center} was one of the earliest and hardest-hit by the virus.
Dozens of Covid-19 patients have clogged hallways as they wait for beds,
terrified, alone and often unable to communicate in English.

Image

Elmhurst Hospital Center, a public medical facility, was among the
earliest and hardest hit hospitals in New York.~

``We're the epicenter of the epicenter,'' said Councilman Daniel Dromm,
who represents Elmhurst and Jackson Heights. He became emotional as he
took stock of losses that included
\href{https://twitter.com/Dromm25/status/1245498053890539528?s=20}{five
friends} and more than two dozen constituents. ``This has shaken the
whole neighborhood,'' he said.

In their daily toll of the fallen, city and state health officials have
not disclosed where exactly deaths are occurring. But community leaders
and organizers have kept their own tallies, providing a window into the
virus's disproportionate impact on immigrant communities. Some of the
more prominent names in Queens include the
\href{http://gts.edu/general-news/2020/4/2/in-memoriam-antonio-checo-06}{Rev.
Antonio Checo}, a pastor at St. Mark's Episcopal Church in Jackson
Heights;
\href{https://www.nytimes3xbfgragh.onion/2020/04/01/obituaries/lorena-borjas-dead-coronavirus.html}{Lorena
Borjas}, a transgender activist; and
\href{https://bdnews24.com/people/2020/04/06/kamal-ahmed-chief-of-bangladesh-society-in-new-york-dies-from-coronavirus}{Kamal
Ahmed}, the president of the Bangladesh Society.

The New York Taxi Workers Alliance said 28 drivers had died --- the vast
majority of them immigrants living in Queens --- and Make the Road New
York, an advocacy organization that serves the area's working-class
Latinos, said eight of its members in Queens had died. ``A tragedy is
unfolding,'' said the co-director, Javier H. Valdés.

The crisis has transformed the neighborhood. Roosevelt Avenue, the vital
commercial artery that normally bustles with taquerias, arepa stands,
threading salons and shops selling newspapers in dozens of languages,
has all but shut down. The eerie silence is intermittently broken by
sirens and the clattering of trains on elevated tracks.

A handful of street vendors have returned, but now they sell masks and
dress in Tyvek suits. With churches and mosques closed, families of the
dead can mourn only at home.

Image

Vivien Grullon, far left, sells masks, gloves and cleaning supplies
under the 7 train on Roosevelt Avenue in Jackson Heights, Queens.

\href{https://www.nytimes3xbfgragh.onion/2020/03/23/nyregion/coronavirus-nyc-crowds-density.html}{The
chockablock density that defines this part of Queens} may have also have
been its undoing. Doctors and community leaders say poverty,
\href{https://www.nytimes3xbfgragh.onion/interactive/2019/10/23/nyregion/basements-queens-immigrants.html}{notoriously
overcrowded homes} and
\href{https://www.nytimes3xbfgragh.onion/2020/04/08/nyregion/new-york-coronavirus-response-delays.html?}{government
inaction} left residents especially vulnerable to the virus.

``I don't think the city communicated the level of danger,'' said
Claudia Zamora, the interim deputy director of New Immigrant Community
Empowerment, an advocacy group and worker center in Jackson Heights.

In early March, she said, city health officials sent out fliers with
hand-washing tips, but not the outreach workers and multilingual posters
that might have conveyed the looming peril.

The sick now include laborers like Ángel, 39, a construction worker from
Ecuador who asked that only his first name be used because of his
immigration status.

Like many, he said he worked at a Manhattan construction site until he
fell ill. He said he was turned away from Elmhurst Hospital because his
symptoms were not deemed life-threatening and had been suffering in the
apartment in Corona he shares with three other workers. ``I don't have
anyone to help me,'' he said.

Image

``I don't have anyone to help me,'' said Ángel, a 39-year-old
construction worker from Ecuador.

City officials rejected the suggestion that they left the city's
immigrant neighborhoods to fend for themselves. The Department of
Health, officials said, created coronavirus fact sheets in 15 languages.
Officials mounted multilingual public service campaigns in subways and
on television, and have provided continuous updates to the ethnic media
including on the need for social distancing.

Ronny Barzola, a 28-year-old Ecuadorean-American from nearby Kew Gardens
who works for the food delivery service Caviar, is one of the lucky few
to still have a job. He slathers his hands with sanitizer throughout the
day but worries about his mother and sister, both of whom are sick at
home but have been unable to get tested. ``It's impossible to isolate
when everyone is sharing the same apartment,'' he said.

Image

Cousins Idenia Ferrera, left, and Kimberly Ferrera, right, sit outside
their home in Corona, Queens, obeying directives not to go out during
the pandemic.~

Mr. Subba, a longtime driver for services including Uber and Via, had
stopped driving last month after picking up a sick passenger, said a
cousin, Munindra Nembang, who added that Mr. Subba, 49, had been
diabetic. His wife and two of his children were also infected.

Image

Anil Subba, second from right, a Nepalese Uber driver from Jackson
Heights, died on March 31 after contracting the coronavirus.

Hundreds of other Nepali immigrants are sick, too, he said, including
another Uber driver, who died on Wednesday. ``Some are in I.C.U., some
are on ventilator, some are in the queue,'' Mr. Nembang said. ``We feel
very sad.''

\href{https://www.nytimes3xbfgragh.onion/news-event/coronavirus?action=click\&pgtype=Article\&state=default\&region=MAIN_CONTENT_3\&context=storylines_faq}{}

\hypertarget{the-coronavirus-outbreak-}{%
\subsubsection{The Coronavirus Outbreak
›}\label{the-coronavirus-outbreak-}}

\hypertarget{frequently-asked-questions}{%
\paragraph{Frequently Asked
Questions}\label{frequently-asked-questions}}

Updated August 17, 2020

\begin{itemize}
\item ~
  \hypertarget{why-does-standing-six-feet-away-from-others-help}{%
  \paragraph{Why does standing six feet away from others
  help?}\label{why-does-standing-six-feet-away-from-others-help}}

  \begin{itemize}
  \tightlist
  \item
    The coronavirus spreads primarily through droplets from your mouth
    and nose, especially when you cough or sneeze. The C.D.C., one of
    the organizations using that measure,
    \href{https://www.nytimes3xbfgragh.onion/2020/04/14/health/coronavirus-six-feet.html?action=click\&pgtype=Article\&state=default\&region=MAIN_CONTENT_3\&context=storylines_faq}{bases
    its recommendation of six feet} on the idea that most large droplets
    that people expel when they cough or sneeze will fall to the ground
    within six feet. But six feet has never been a magic number that
    guarantees complete protection. Sneezes, for instance, can launch
    droplets a lot farther than six feet,
    \href{https://jamanetwork.com/journals/jama/fullarticle/2763852}{according
    to a recent study}. It's a rule of thumb: You should be safest
    standing six feet apart outside, especially when it's windy. But
    keep a mask on at all times, even when you think you're far enough
    apart.
  \end{itemize}
\item ~
  \hypertarget{i-have-antibodies-am-i-now-immune}{%
  \paragraph{I have antibodies. Am I now
  immune?}\label{i-have-antibodies-am-i-now-immune}}

  \begin{itemize}
  \tightlist
  \item
    As of right
    now,\href{https://www.nytimes3xbfgragh.onion/2020/07/22/health/covid-antibodies-herd-immunity.html?action=click\&pgtype=Article\&state=default\&region=MAIN_CONTENT_3\&context=storylines_faq}{that
    seems likely, for at least several months.} There have been
    frightening accounts of people suffering what seems to be a second
    bout of Covid-19. But experts say these patients may have a
    drawn-out course of infection, with the virus taking a slow toll
    weeks to months after initial exposure. People infected with the
    coronavirus typically
    \href{https://www.nature.com/articles/s41586-020-2456-9}{produce}
    immune molecules called antibodies, which are
    \href{https://www.nytimes3xbfgragh.onion/2020/05/07/health/coronavirus-antibody-prevalence.html?action=click\&pgtype=Article\&state=default\&region=MAIN_CONTENT_3\&context=storylines_faq}{protective
    proteins made in response to an
    infection}\href{https://www.nytimes3xbfgragh.onion/2020/05/07/health/coronavirus-antibody-prevalence.html?action=click\&pgtype=Article\&state=default\&region=MAIN_CONTENT_3\&context=storylines_faq}{.
    These antibodies may} last in the body
    \href{https://www.nature.com/articles/s41591-020-0965-6}{only two to
    three months}, which may seem worrisome, but that's perfectly normal
    after an acute infection subsides, said Dr. Michael Mina, an
    immunologist at Harvard University. It may be possible to get the
    coronavirus again, but it's highly unlikely that it would be
    possible in a short window of time from initial infection or make
    people sicker the second time.
  \end{itemize}
\item ~
  \hypertarget{im-a-small-business-owner-can-i-get-relief}{%
  \paragraph{I'm a small-business owner. Can I get
  relief?}\label{im-a-small-business-owner-can-i-get-relief}}

  \begin{itemize}
  \tightlist
  \item
    The
    \href{https://www.nytimes3xbfgragh.onion/article/small-business-loans-stimulus-grants-freelancers-coronavirus.html?action=click\&pgtype=Article\&state=default\&region=MAIN_CONTENT_3\&context=storylines_faq}{stimulus
    bills enacted in March} offer help for the millions of American
    small businesses. Those eligible for aid are businesses and
    nonprofit organizations with fewer than 500 workers, including sole
    proprietorships, independent contractors and freelancers. Some
    larger companies in some industries are also eligible. The help
    being offered, which is being managed by the Small Business
    Administration, includes the Paycheck Protection Program and the
    Economic Injury Disaster Loan program. But lots of folks have
    \href{https://www.nytimes3xbfgragh.onion/interactive/2020/05/07/business/small-business-loans-coronavirus.html?action=click\&pgtype=Article\&state=default\&region=MAIN_CONTENT_3\&context=storylines_faq}{not
    yet seen payouts.} Even those who have received help are confused:
    The rules are draconian, and some are stuck sitting on
    \href{https://www.nytimes3xbfgragh.onion/2020/05/02/business/economy/loans-coronavirus-small-business.html?action=click\&pgtype=Article\&state=default\&region=MAIN_CONTENT_3\&context=storylines_faq}{money
    they don't know how to use.} Many small-business owners are getting
    less than they expected or
    \href{https://www.nytimes3xbfgragh.onion/2020/06/10/business/Small-business-loans-ppp.html?action=click\&pgtype=Article\&state=default\&region=MAIN_CONTENT_3\&context=storylines_faq}{not
    hearing anything at all.}
  \end{itemize}
\item ~
  \hypertarget{what-are-my-rights-if-i-am-worried-about-going-back-to-work}{%
  \paragraph{What are my rights if I am worried about going back to
  work?}\label{what-are-my-rights-if-i-am-worried-about-going-back-to-work}}

  \begin{itemize}
  \tightlist
  \item
    Employers have to provide
    \href{https://www.osha.gov/SLTC/covid-19/standards.html}{a safe
    workplace} with policies that protect everyone equally.
    \href{https://www.nytimes3xbfgragh.onion/article/coronavirus-money-unemployment.html?action=click\&pgtype=Article\&state=default\&region=MAIN_CONTENT_3\&context=storylines_faq}{And
    if one of your co-workers tests positive for the coronavirus, the
    C.D.C.} has said that
    \href{https://www.cdc.gov/coronavirus/2019-ncov/community/guidance-business-response.html}{employers
    should tell their employees} -\/- without giving you the sick
    employee's name -\/- that they may have been exposed to the virus.
  \end{itemize}
\item ~
  \hypertarget{what-is-school-going-to-look-like-in-september}{%
  \paragraph{What is school going to look like in
  September?}\label{what-is-school-going-to-look-like-in-september}}

  \begin{itemize}
  \tightlist
  \item
    It is unlikely that many schools will return to a normal schedule
    this fall, requiring the grind of
    \href{https://www.nytimes3xbfgragh.onion/2020/06/05/us/coronavirus-education-lost-learning.html?action=click\&pgtype=Article\&state=default\&region=MAIN_CONTENT_3\&context=storylines_faq}{online
    learning},
    \href{https://www.nytimes3xbfgragh.onion/2020/05/29/us/coronavirus-child-care-centers.html?action=click\&pgtype=Article\&state=default\&region=MAIN_CONTENT_3\&context=storylines_faq}{makeshift
    child care} and
    \href{https://www.nytimes3xbfgragh.onion/2020/06/03/business/economy/coronavirus-working-women.html?action=click\&pgtype=Article\&state=default\&region=MAIN_CONTENT_3\&context=storylines_faq}{stunted
    workdays} to continue. California's two largest public school
    districts --- Los Angeles and San Diego --- said on July 13, that
    \href{https://www.nytimes3xbfgragh.onion/2020/07/13/us/lausd-san-diego-school-reopening.html?action=click\&pgtype=Article\&state=default\&region=MAIN_CONTENT_3\&context=storylines_faq}{instruction
    will be remote-only in the fall}, citing concerns that surging
    coronavirus infections in their areas pose too dire a risk for
    students and teachers. Together, the two districts enroll some
    825,000 students. They are the largest in the country so far to
    abandon plans for even a partial physical return to classrooms when
    they reopen in August. For other districts, the solution won't be an
    all-or-nothing approach.
    \href{https://bioethics.jhu.edu/research-and-outreach/projects/eschool-initiative/school-policy-tracker/}{Many
    systems}, including the nation's largest, New York City, are
    devising
    \href{https://www.nytimes3xbfgragh.onion/2020/06/26/us/coronavirus-schools-reopen-fall.html?action=click\&pgtype=Article\&state=default\&region=MAIN_CONTENT_3\&context=storylines_faq}{hybrid
    plans} that involve spending some days in classrooms and other days
    online. There's no national policy on this yet, so check with your
    municipal school system regularly to see what is happening in your
    community.
  \end{itemize}
\end{itemize}

Many residents struggled with poor health long before the coronavirus
arrived. Dr. Dave Chokshi, chief population health officer for the New
York City Health and Hospitals Corporation, said rates of diabetes, high
blood pressure and other chronic conditions in central Queens were
considerably higher than the city average.

Compounding the crisis, many residents lack health insurance and depend
on public hospitals for even routine procedures, said Diana Ramírez
Barón, a doctor at Grameen VidaSana, a clinic in Jackson Heights for
undocumented women.

``They tell them to stay home and call your physician,'' she said,
referring to public health guidelines for people believed to have the
coronavirus. ``But they don't have a physician. They get scared and they
go to the E.R.''

Image

Some workers have been deemed essential, including deliverymen.

Patricia Rivera, a Mexican immigrant, said she had kept her distance
from her mother's household in East Elmhurst as the virus ripped through
its seven members last month, infecting all but one. But then her
mother, who was struggling to breathe, needed to be taken to the
hospital.

Ms. Rivera, 38, took her to Flushing Hospital Medical Center, but came
home worried she would infect her own crowded household, which includes
a 70-year-old uncle. She found some N95 respirator masks given to a son
on a construction job, and handed them out to her family.

``Fear is what we're all feeling,'' said Ms. Rivera, who is working for
a laundromat, carrying laundry to and from quarantined homes.

For many, the fear of getting sick is heightened by the prospect of
becoming homeless. Johana Marin, 33, a waitress from Jackson Heights,
said she spent several days in the hospital.

``I thought I was going to die and never see my family in Colombia
again,'' she said.

Image

Johana Marin, 33, a waitress from Jackson Heights, said that after she
became ill, she worried she might never see her family again.

When she was discharged, she said, the woman who rented her a room
refused to let her stay. Ms. Marin found refuge in the apartment of an
aunt who she said was now pressing her to leave.

Mr. Dromm, the councilman, said such stories were increasingly common
and he urged the city to convert empty hotel rooms into temporary
housing for those discharged from the hospital or patients with mild
symptoms who were at risk of infecting others. City officials say they
are working to address the problem.

The challenges of dealing with the dead are becoming clear, as officials
discuss
\href{https://www.nytimes3xbfgragh.onion/2020/04/06/nyregion/mass-graves-nyc-parks-coronavirus.html}{digging
temporary graves} and families call on consulates to help them
repatriate the deceased to their home countries.

In the meantime, the needs of the living keep growing. Thousands have
lost jobs, and the undocumented have so far been excluded from federal
government aid.

Image

Bravo Supermarket in Jackson Heights.~

At a food pantry run in nearby Flushing by the nonprofit organization La
Jornada, the vast majority of visitors were, until recently, single
mothers. Now two-thirds are men trying to feed their families, said the
director, Pedro Rodríguez, who worried the number of jobless residents
would soon prove overwhelming. ``A tsunami is coming,'' he said.

Despite the growing despair, many are finding ways to help others.
Mexican grandmothers share recipes for traditional herbal fever
remedies, Pakistani drivers deliver home-cooked meals and Nepali
volunteers --- including Mr. Nembang, the cousin of the driver who died
--- are distributing protective gear to those who must keep working.

For thousands of people, however, life has been reduced to the
dimensions of tiny rented rooms.

Ms. Begum, the former nanny from Bangladesh, said she was riddled with
fear. She spends her days compulsively cleaning the apartment's bathroom
and steering clear of her ailing roommates. The landlord has been
demanding April rent and threatening eviction.

For succor, Ms. Begum turns to the Quran she keeps beside her bed. ``I
am praying every day,'' she said. ``Praying that the coronavirus leaves
America.''

Image

A bodega in Jackson Heights is one of the few types of businesses
allowed to remain open during the pandemic.~

Somini Sengupta, Paula Moura, Jo Corona and Ryan Christopher Jones
contributed reporting.

Advertisement

\protect\hyperlink{after-bottom}{Continue reading the main story}

\hypertarget{site-index}{%
\subsection{Site Index}\label{site-index}}

\hypertarget{site-information-navigation}{%
\subsection{Site Information
Navigation}\label{site-information-navigation}}

\begin{itemize}
\tightlist
\item
  \href{https://help.nytimes3xbfgragh.onion/hc/en-us/articles/115014792127-Copyright-notice}{©~2020~The
  New York Times Company}
\end{itemize}

\begin{itemize}
\tightlist
\item
  \href{https://www.nytco.com/}{NYTCo}
\item
  \href{https://help.nytimes3xbfgragh.onion/hc/en-us/articles/115015385887-Contact-Us}{Contact
  Us}
\item
  \href{https://www.nytco.com/careers/}{Work with us}
\item
  \href{https://nytmediakit.com/}{Advertise}
\item
  \href{http://www.tbrandstudio.com/}{T Brand Studio}
\item
  \href{https://www.nytimes3xbfgragh.onion/privacy/cookie-policy\#how-do-i-manage-trackers}{Your
  Ad Choices}
\item
  \href{https://www.nytimes3xbfgragh.onion/privacy}{Privacy}
\item
  \href{https://help.nytimes3xbfgragh.onion/hc/en-us/articles/115014893428-Terms-of-service}{Terms
  of Service}
\item
  \href{https://help.nytimes3xbfgragh.onion/hc/en-us/articles/115014893968-Terms-of-sale}{Terms
  of Sale}
\item
  \href{https://spiderbites.nytimes3xbfgragh.onion}{Site Map}
\item
  \href{https://help.nytimes3xbfgragh.onion/hc/en-us}{Help}
\item
  \href{https://www.nytimes3xbfgragh.onion/subscription?campaignId=37WXW}{Subscriptions}
\end{itemize}
