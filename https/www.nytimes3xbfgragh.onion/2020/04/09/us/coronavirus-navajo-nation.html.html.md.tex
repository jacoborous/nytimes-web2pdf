Sections

SEARCH

\protect\hyperlink{site-content}{Skip to
content}\protect\hyperlink{site-index}{Skip to site index}

\href{/section/us}{U.S.}\textbar{}Checkpoints, Curfews, Airlifts: Virus
Rips Through Navajo Nation

\url{https://nyti.ms/2VaK14O}

\begin{itemize}
\item
\item
\item
\item
\item
\item
\end{itemize}

\href{https://www.nytimes3xbfgragh.onion/news-event/coronavirus?action=click\&pgtype=Article\&state=default\&region=TOP_BANNER\&context=storylines_menu}{The
Coronavirus Outbreak}

\begin{itemize}
\tightlist
\item
  live\href{https://www.nytimes3xbfgragh.onion/2020/08/04/world/coronavirus-cases.html?action=click\&pgtype=Article\&state=default\&region=TOP_BANNER\&context=storylines_menu}{Latest
  Updates}
\item
  \href{https://www.nytimes3xbfgragh.onion/interactive/2020/us/coronavirus-us-cases.html?action=click\&pgtype=Article\&state=default\&region=TOP_BANNER\&context=storylines_menu}{Maps
  and Cases}
\item
  \href{https://www.nytimes3xbfgragh.onion/interactive/2020/science/coronavirus-vaccine-tracker.html?action=click\&pgtype=Article\&state=default\&region=TOP_BANNER\&context=storylines_menu}{Vaccine
  Tracker}
\item
  \href{https://www.nytimes3xbfgragh.onion/2020/08/02/us/covid-college-reopening.html?action=click\&pgtype=Article\&state=default\&region=TOP_BANNER\&context=storylines_menu}{College
  Reopening}
\item
  \href{https://www.nytimes3xbfgragh.onion/live/2020/08/04/business/stock-market-today-coronavirus?action=click\&pgtype=Article\&state=default\&region=TOP_BANNER\&context=storylines_menu}{Economy}
\end{itemize}

\includegraphics{https://static01.graylady3jvrrxbe.onion/images/2020/04/08/us/00VIRUS-NAVAJO1/merlin_171306282_59bb8294-1263-4fff-a930-80a4c10e0f98-articleLarge.jpg?quality=75\&auto=webp\&disable=upscale}

\hypertarget{checkpoints-curfews-airlifts-virus-rips-through-navajo-nation}{%
\section{Checkpoints, Curfews, Airlifts: Virus Rips Through Navajo
Nation}\label{checkpoints-curfews-airlifts-virus-rips-through-navajo-nation}}

The coronavirus is tearing across the largest Native American
reservation in the United States. Facing a spike in deaths, Navajo
officials are scrambling to respond.

Police officers from the Navajo Police setting up a road block in Window
Rock, Ariz., on April 3.~Credit...Adriana Zehbrauskas for The New York
Times

Supported by

\protect\hyperlink{after-sponsor}{Continue reading the main story}

By \href{https://www.nytimes3xbfgragh.onion/by/simon-romero}{Simon
Romero}

\begin{itemize}
\item
  Published April 9, 2020Updated April 20, 2020
\item
  \begin{itemize}
  \item
  \item
  \item
  \item
  \item
  \item
  \end{itemize}
\end{itemize}

WINDOW ROCK, Ariz. --- When Chad Yazzie joined the Navajo Police
Department just a few months ago, he expected to issue speeding tickets
or break up the occasional fistfight.

But the coronavirus is now tearing across the largest Indian reservation
in the United States. The Navajo Nation's casualty count is eclipsing
that of states with much larger populations, placing the rookie cop on
the front lines.

``My job is to tell our people to take this virus seriously or face the
consequences,'' Officer Yazzie, 24, said as he set up a police roadblock
outside the town of Window Rock to enforce the tribal nation's 8 p.m.
curfew.

utah

25 miles

colorado

navajo nation

arizona

new

mexico

hopi

reservation

Window Rock

Flagstaff

utah

colo.

25 miles

navajo nation

n.M.

hopi

reservation

Window Rock

Flagstaff

ariz.

By The New York Times

Faced with an alarming spike in deaths from what the tribal health
department calls Dikos Ntsaaígíí-19 --- or Covid-19 --- Navajo officials
are putting up checkpoints, assembling field hospitals and threatening
curfew violators with 30 days in jail or a \$1,000 fine.

The measures are part of a scramble to protect more than 150,000 people
on the vast Navajo reservation, which stretches 27,000 square miles
across Arizona, New Mexico and Utah, and tens of thousands of others who
live in towns bordering the Navajo Nation. As of Wednesday night, the
virus had killed 20 people on the reservation, compared with 16 in the
entire state of New Mexico, which has a population 13 times larger.

\includegraphics{https://static01.graylady3jvrrxbe.onion/images/2020/04/08/us/00VIRUS-NAVAJO2/merlin_171306528_729ae8c6-95a6-4e6f-94f6-4a222b075bf3-articleLarge.jpg?quality=75\&auto=webp\&disable=upscale}

Navajo officials, who have traced the surge in the reservation's
coronavirus cases to a March 7 rally held by an evangelical church, warn
that infections will rise in the weeks ahead, potentially reaching a
peak in about a month.

Several factors --- including a high prevalence of diseases like
diabetes, scarcity of running water, and homes with several generations
living under the same roof --- have enabled the virus to spread with
exceptional speed, according to epidemiologists.

While the Navajo Nation may not technically be at war with the virus, it
does not feel at peace either.

The Arizona National Guard this month began airlifting protective masks,
gowns and other equipment, using Blackhawk helicopters to deliver it to
Kayenta, a town of 5,200 people near the sandstone buttes of Monument
Valley.

\hypertarget{latest-updates-global-coronavirus-outbreak}{%
\section{\texorpdfstring{\href{https://www.nytimes3xbfgragh.onion/2020/08/04/world/coronavirus-cases.html?action=click\&pgtype=Article\&state=default\&region=MAIN_CONTENT_1\&context=storylines_live_updates}{Latest
Updates: Global Coronavirus
Outbreak}}{Latest Updates: Global Coronavirus Outbreak}}\label{latest-updates-global-coronavirus-outbreak}}

Updated 2020-08-05T07:58:24.076Z

\begin{itemize}
\tightlist
\item
  \href{https://www.nytimes3xbfgragh.onion/2020/08/04/world/coronavirus-cases.html?action=click\&pgtype=Article\&state=default\&region=MAIN_CONTENT_1\&context=storylines_live_updates\#link-762df92}{As
  talks drag on, McConnell signals openness to jobless aid extension,
  and negotiators agree on a deadline.}
\item
  \href{https://www.nytimes3xbfgragh.onion/2020/08/04/world/coronavirus-cases.html?action=click\&pgtype=Article\&state=default\&region=MAIN_CONTENT_1\&context=storylines_live_updates\#link-1228a480}{Novavax
  sees encouraging results from two studies of its experimental
  vaccine.}
\item
  \href{https://www.nytimes3xbfgragh.onion/2020/08/04/world/coronavirus-cases.html?action=click\&pgtype=Article\&state=default\&region=MAIN_CONTENT_1\&context=storylines_live_updates\#link-794484ed}{Mississippians
  must now wear masks in public, governor says.}
\end{itemize}

\href{https://www.nytimes3xbfgragh.onion/2020/08/04/world/coronavirus-cases.html?action=click\&pgtype=Article\&state=default\&region=MAIN_CONTENT_1\&context=storylines_live_updates}{See
more updates}

More live coverage:
\href{https://www.nytimes3xbfgragh.onion/live/2020/08/04/business/stock-market-today-coronavirus?action=click\&pgtype=Article\&state=default\&region=MAIN_CONTENT_1\&context=storylines_live_updates}{Markets}

Guard members also converted a community center in Chinle into a 50-bed
field hospital for quarantining people who have tested positive for the
virus. And personnel visited a triage tent set up by Tuba City, near the
Navajo Nation's western edge.

Going further, Navajo authorities said their entire nation would be
under curfew for 57 continuous hours, from Friday at 8 p.m. until Monday
at 5 a.m. The holiday weekend offers the opportunity to practice extreme
social distancing, the authorities said. Unlike many stay-at-home orders
around the nation, the Navajo curfews are being enforced with
checkpoints and patrols. Violators can face jail time and hefty fines.

Fearing pushback, the chief of the Navajo Police, Phillip Francisco,
said that anyone knowingly exposing officers to the coronavirus would be
charged with battery against a police officer.

Image

Factors like a high prevalence of diabetes, a scarcity of running water
and multigenerational living arrangements have enabled the virus to
spread with exceptional speed through the Navajo Nation.Credit...Adriana
Zehbrauskas for The New York Times

The Navajo Nation's president, Jonathan Nez, who has begun wearing a
mask in public, said in a telephone interview that authorities were
working under the assumption that the reservation's peak in cases could
be about a month away, in early to mid-May.

Mr. Nez said he was growing exasperated with
\href{https://www.politico.com/news/2020/03/20/coronavirus-american-indian-health-138724}{delays}
in receiving federal emergency funds and by the requirements that tribal
nations, unlike cities and counties, must apply for grants to receive
money from federal stimulus legislation. The extra hurdles have led to
weeks of additional bureaucratic delays, he said.

``We're barely getting bits and pieces,'' Mr. Nez said. ``You have
counties, municipalities, already taking advantage of these funds, and
tribes are over here writing our applications and turning it in and
waiting weeks to get what we need.''

The crisis among the Diné, as many Navajos prefer to call themselves, is
echoing throughout Indian Country. Around the United States, and
especially in New Mexico, tribal leaders have started barring
nonresidents from reservations. In South Dakota, the Oglala Sioux tribe
announced a 72-hour lockdown after a resident of the Pine Ridge
reservation tested positive for the coronavirus. The Blackfeet and the
Northern Cheyenne tribal nations in Montana have announced curfews.

The Hopi reservation, which is surrounded entirely by the Navajo Nation
and includes some of the oldest continuously inhabited places in the
United States, issued its own stay-at-home order. Signaling
vulnerabilities elsewhere in the Southwest, large clusters of cases
emerged this week in San Felipe Pueblo and Zia Pueblo, two of New
Mexico's 23 federally recognized tribal nations.

Image

A grocery store on the Navajo reservation in Window Rock,
Ariz.Credit...Adriana Zehbrauskas for The New York Times

Native American leaders say they are also dealing with the potential for
racist attacks as outsiders try to cast blame for the pandemic on tribal
citizens.

Police in Page, an Arizona town bordering the Navajo Nation, said this
week that they had arrested a 34-year-old man, Daniel Franzen, on
suspicion of attempting to incite an act of terrorism. Mr. Franzen in a
Facebook post had called for using ``lethal force'' against Navajos
because they were, in his view, ``100 percent infected'' with the virus,
the Page Police Department said in a statement.

Infectious disease specialists say the virus is thought to have arrived
on the reservation later than in other parts of the United States. It
began spreading rapidly after it was detected among members of the
Church of the Nazarene, an evangelical congregation in the outpost of
Chilchinbeto near the Arizona-Utah border.

Families traveled from far-flung parts of the Navajo Nation to attend
the rally, which included a prayer service in response to the pandemic
already spreading in parts of the country.

Dr. Laura Hammitt, director of the infectious disease prevention program
at the Johns Hopkins Center for American Indian Health, listed several
factors that have made citizens of the Navajo Nation especially
vulnerable to the coronavirus.

The scarcity of running water on the reservation, she said, makes it
harder to wash hands. There are also pre-existing health conditions,
including respiratory problems caused by indoor pollution because of the
wood and coal used to heat many Navajo homes.

Image

Facing a spike in deaths, Navajo officials are scrambling to
respond.~Credit...Adriana Zehbrauskas for The New York Times

Exceptionally close-knit families, which have helped the Navajos endure
extreme hardships, may also now heighten exposure to the virus.

\href{https://www.nytimes3xbfgragh.onion/news-event/coronavirus?action=click\&pgtype=Article\&state=default\&region=MAIN_CONTENT_3\&context=storylines_faq}{}

\hypertarget{the-coronavirus-outbreak-}{%
\subsubsection{The Coronavirus Outbreak
›}\label{the-coronavirus-outbreak-}}

\hypertarget{frequently-asked-questions}{%
\paragraph{Frequently Asked
Questions}\label{frequently-asked-questions}}

Updated August 4, 2020

\begin{itemize}
\item ~
  \hypertarget{i-have-antibodies-am-i-now-immune}{%
  \paragraph{I have antibodies. Am I now
  immune?}\label{i-have-antibodies-am-i-now-immune}}

  \begin{itemize}
  \tightlist
  \item
    As of right
    now,\href{https://www.nytimes3xbfgragh.onion/2020/07/22/health/covid-antibodies-herd-immunity.html?action=click\&pgtype=Article\&state=default\&region=MAIN_CONTENT_3\&context=storylines_faq}{that
    seems likely, for at least several months.} There have been
    frightening accounts of people suffering what seems to be a second
    bout of Covid-19. But experts say these patients may have a
    drawn-out course of infection, with the virus taking a slow toll
    weeks to months after initial exposure. People infected with the
    coronavirus typically
    \href{https://www.nature.com/articles/s41586-020-2456-9}{produce}
    immune molecules called antibodies, which are
    \href{https://www.nytimes3xbfgragh.onion/2020/05/07/health/coronavirus-antibody-prevalence.html?action=click\&pgtype=Article\&state=default\&region=MAIN_CONTENT_3\&context=storylines_faq}{protective
    proteins made in response to an
    infection}\href{https://www.nytimes3xbfgragh.onion/2020/05/07/health/coronavirus-antibody-prevalence.html?action=click\&pgtype=Article\&state=default\&region=MAIN_CONTENT_3\&context=storylines_faq}{.
    These antibodies may} last in the body
    \href{https://www.nature.com/articles/s41591-020-0965-6}{only two to
    three months}, which may seem worrisome, but that's perfectly normal
    after an acute infection subsides, said Dr. Michael Mina, an
    immunologist at Harvard University. It may be possible to get the
    coronavirus again, but it's highly unlikely that it would be
    possible in a short window of time from initial infection or make
    people sicker the second time.
  \end{itemize}
\item ~
  \hypertarget{im-a-small-business-owner-can-i-get-relief}{%
  \paragraph{I'm a small-business owner. Can I get
  relief?}\label{im-a-small-business-owner-can-i-get-relief}}

  \begin{itemize}
  \tightlist
  \item
    The
    \href{https://www.nytimes3xbfgragh.onion/article/small-business-loans-stimulus-grants-freelancers-coronavirus.html?action=click\&pgtype=Article\&state=default\&region=MAIN_CONTENT_3\&context=storylines_faq}{stimulus
    bills enacted in March} offer help for the millions of American
    small businesses. Those eligible for aid are businesses and
    nonprofit organizations with fewer than 500 workers, including sole
    proprietorships, independent contractors and freelancers. Some
    larger companies in some industries are also eligible. The help
    being offered, which is being managed by the Small Business
    Administration, includes the Paycheck Protection Program and the
    Economic Injury Disaster Loan program. But lots of folks have
    \href{https://www.nytimes3xbfgragh.onion/interactive/2020/05/07/business/small-business-loans-coronavirus.html?action=click\&pgtype=Article\&state=default\&region=MAIN_CONTENT_3\&context=storylines_faq}{not
    yet seen payouts.} Even those who have received help are confused:
    The rules are draconian, and some are stuck sitting on
    \href{https://www.nytimes3xbfgragh.onion/2020/05/02/business/economy/loans-coronavirus-small-business.html?action=click\&pgtype=Article\&state=default\&region=MAIN_CONTENT_3\&context=storylines_faq}{money
    they don't know how to use.} Many small-business owners are getting
    less than they expected or
    \href{https://www.nytimes3xbfgragh.onion/2020/06/10/business/Small-business-loans-ppp.html?action=click\&pgtype=Article\&state=default\&region=MAIN_CONTENT_3\&context=storylines_faq}{not
    hearing anything at all.}
  \end{itemize}
\item ~
  \hypertarget{what-are-my-rights-if-i-am-worried-about-going-back-to-work}{%
  \paragraph{What are my rights if I am worried about going back to
  work?}\label{what-are-my-rights-if-i-am-worried-about-going-back-to-work}}

  \begin{itemize}
  \tightlist
  \item
    Employers have to provide
    \href{https://www.osha.gov/SLTC/covid-19/standards.html}{a safe
    workplace} with policies that protect everyone equally.
    \href{https://www.nytimes3xbfgragh.onion/article/coronavirus-money-unemployment.html?action=click\&pgtype=Article\&state=default\&region=MAIN_CONTENT_3\&context=storylines_faq}{And
    if one of your co-workers tests positive for the coronavirus, the
    C.D.C.} has said that
    \href{https://www.cdc.gov/coronavirus/2019-ncov/community/guidance-business-response.html}{employers
    should tell their employees} -\/- without giving you the sick
    employee's name -\/- that they may have been exposed to the virus.
  \end{itemize}
\item ~
  \hypertarget{should-i-refinance-my-mortgage}{%
  \paragraph{Should I refinance my
  mortgage?}\label{should-i-refinance-my-mortgage}}

  \begin{itemize}
  \tightlist
  \item
    \href{https://www.nytimes3xbfgragh.onion/article/coronavirus-money-unemployment.html?action=click\&pgtype=Article\&state=default\&region=MAIN_CONTENT_3\&context=storylines_faq}{It
    could be a good idea,} because mortgage rates have
    \href{https://www.nytimes3xbfgragh.onion/2020/07/16/business/mortgage-rates-below-3-percent.html?action=click\&pgtype=Article\&state=default\&region=MAIN_CONTENT_3\&context=storylines_faq}{never
    been lower.} Refinancing requests have pushed mortgage applications
    to some of the highest levels since 2008, so be prepared to get in
    line. But defaults are also up, so if you're thinking about buying a
    home, be aware that some lenders have tightened their standards.
  \end{itemize}
\item ~
  \hypertarget{what-is-school-going-to-look-like-in-september}{%
  \paragraph{What is school going to look like in
  September?}\label{what-is-school-going-to-look-like-in-september}}

  \begin{itemize}
  \tightlist
  \item
    It is unlikely that many schools will return to a normal schedule
    this fall, requiring the grind of
    \href{https://www.nytimes3xbfgragh.onion/2020/06/05/us/coronavirus-education-lost-learning.html?action=click\&pgtype=Article\&state=default\&region=MAIN_CONTENT_3\&context=storylines_faq}{online
    learning},
    \href{https://www.nytimes3xbfgragh.onion/2020/05/29/us/coronavirus-child-care-centers.html?action=click\&pgtype=Article\&state=default\&region=MAIN_CONTENT_3\&context=storylines_faq}{makeshift
    child care} and
    \href{https://www.nytimes3xbfgragh.onion/2020/06/03/business/economy/coronavirus-working-women.html?action=click\&pgtype=Article\&state=default\&region=MAIN_CONTENT_3\&context=storylines_faq}{stunted
    workdays} to continue. California's two largest public school
    districts --- Los Angeles and San Diego --- said on July 13, that
    \href{https://www.nytimes3xbfgragh.onion/2020/07/13/us/lausd-san-diego-school-reopening.html?action=click\&pgtype=Article\&state=default\&region=MAIN_CONTENT_3\&context=storylines_faq}{instruction
    will be remote-only in the fall}, citing concerns that surging
    coronavirus infections in their areas pose too dire a risk for
    students and teachers. Together, the two districts enroll some
    825,000 students. They are the largest in the country so far to
    abandon plans for even a partial physical return to classrooms when
    they reopen in August. For other districts, the solution won't be an
    all-or-nothing approach.
    \href{https://bioethics.jhu.edu/research-and-outreach/projects/eschool-initiative/school-policy-tracker/}{Many
    systems}, including the nation's largest, New York City, are
    devising
    \href{https://www.nytimes3xbfgragh.onion/2020/06/26/us/coronavirus-schools-reopen-fall.html?action=click\&pgtype=Article\&state=default\&region=MAIN_CONTENT_3\&context=storylines_faq}{hybrid
    plans} that involve spending some days in classrooms and other days
    online. There's no national policy on this yet, so check with your
    municipal school system regularly to see what is happening in your
    community.
  \end{itemize}
\end{itemize}

``Instead of urban crowding in high-density cities like New York, you
have indoor crowding with several generations living under the same
roof,'' she said.

That explains the special need for field hospitals where patients who
have tested positive for the virus can recuperate away from their
families. Officials are also searching for ways to mitigate the spread
of the virus in the so-called border towns where many Diné live.

For instance, in addition to the field hospitals assembled by the
Arizona National Guard, the iconic El Rancho Hotel in the town of Gallup
near the Navajo Nation is planning to house homeless people who have
developed respiratory problems in one of its buildings.

Despite such measures, fear is building on parts of the reservation, and
some are taking it upon themselves to protect their families.

In a culture prizing communal contact, Julian Parrish, a computer
science high school teacher in Chinle, said he and his girlfriend had
taken the unusual step of going on Facebook to request that visitors
refrain from coming to their home unannounced.

Mr. Parrish, 34, explained that he is prediabetic, his girlfriend is
pregnant and her son has asthma that sometimes requires trips to the
emergency room.

``We don't want to go anywhere near the hospital at this time,'' Mr.
Parrish said. ``No one knows where the hell this virus is going next.''

Image

The 200-strong Navajo Nation police force is now charged with enforcing
the 8 p.m. curfew every night in towns and along lonely stretches of
road that connect far-flung homesteads and sheep
ranches.Credit...Adriana Zehbrauskas for The New York Times

As the death count climbs, the virus is drawing grim comparisons with
previous epidemics that shaped the history of the Diné. From the start
of the European conquest, outbreaks of smallpox, bubonic plague and
typhus ravaged the tribe.

A century ago, the influenza pandemic of 1918 spread to the
\href{https://www.jstor.org/stable/pdf/10.5250/amerindiquar.38.4.0459.pdf?refreqid=excelsior\%3A70667e3ff079d8c69050297f2972d443}{most
remote corners} of the reservation, killing thousands. Estimates put the
mortality rate as high as 10 percent; accounts from that time described
how some survivors died from starvation with no one left to care for
them.

More recently, a
\href{https://www.aaas.org/virus-rocked-four-corners-reemerges}{hantavirus
outbreak} in the region in 1993 stirred fear across the Navajo Nation.
The virus, carried by deer mice, left 13 dead including young, otherwise
healthy people who developed sudden respiratory failure.

Despite the rising death toll from the newest virus, epidemiologists say
the Diné may have advantages in the mitigation fight that other tribal
nations do not.

They point to the nation's relatively large number of diabetes
specialists, who could help with outreach or trace the spread of the
virus. Robust civil society groups within the reservation have also
sprung into action, with volunteers replenishing water tanks for
hundreds of families.

As one of the largest tribal nations in the United States, the Diné, who
number more than 330,000 on the reservation and beyond, can also draw on
resources unavailable to other tribes.

That includes the 200-strong police force now charged with enforcing the
curfew every night in towns and along lonely stretches of road that
connect far-flung homesteads and sheep ranches.

``We have to get the situation under control,'' Officer Yazzie said,
between chasing down curfew violators, writing citations and telling
motorists over a loudspeaker to ``just go home'' where it was safe.

``If we don't do this,'' he said, ``it's our own families at risk.''

Image

Navajo officials warn that infections will surge in the weeks ahead,
potentially reaching a peak in about a month.Credit...Adriana
Zehbrauskas for The New York Times

Advertisement

\protect\hyperlink{after-bottom}{Continue reading the main story}

\hypertarget{site-index}{%
\subsection{Site Index}\label{site-index}}

\hypertarget{site-information-navigation}{%
\subsection{Site Information
Navigation}\label{site-information-navigation}}

\begin{itemize}
\tightlist
\item
  \href{https://help.nytimes3xbfgragh.onion/hc/en-us/articles/115014792127-Copyright-notice}{©~2020~The
  New York Times Company}
\end{itemize}

\begin{itemize}
\tightlist
\item
  \href{https://www.nytco.com/}{NYTCo}
\item
  \href{https://help.nytimes3xbfgragh.onion/hc/en-us/articles/115015385887-Contact-Us}{Contact
  Us}
\item
  \href{https://www.nytco.com/careers/}{Work with us}
\item
  \href{https://nytmediakit.com/}{Advertise}
\item
  \href{http://www.tbrandstudio.com/}{T Brand Studio}
\item
  \href{https://www.nytimes3xbfgragh.onion/privacy/cookie-policy\#how-do-i-manage-trackers}{Your
  Ad Choices}
\item
  \href{https://www.nytimes3xbfgragh.onion/privacy}{Privacy}
\item
  \href{https://help.nytimes3xbfgragh.onion/hc/en-us/articles/115014893428-Terms-of-service}{Terms
  of Service}
\item
  \href{https://help.nytimes3xbfgragh.onion/hc/en-us/articles/115014893968-Terms-of-sale}{Terms
  of Sale}
\item
  \href{https://spiderbites.nytimes3xbfgragh.onion}{Site Map}
\item
  \href{https://help.nytimes3xbfgragh.onion/hc/en-us}{Help}
\item
  \href{https://www.nytimes3xbfgragh.onion/subscription?campaignId=37WXW}{Subscriptions}
\end{itemize}
