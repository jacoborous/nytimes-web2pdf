Sections

SEARCH

\protect\hyperlink{site-content}{Skip to
content}\protect\hyperlink{site-index}{Skip to site index}

\href{https://www.nytimes3xbfgragh.onion/section/theater}{Theater}

\href{https://myaccount.nytimes3xbfgragh.onion/auth/login?response_type=cookie\&client_id=vi}{}

\href{https://www.nytimes3xbfgragh.onion/section/todayspaper}{Today's
Paper}

\href{/section/theater}{Theater}\textbar{}Wynn Handman, Influential
Director and Teacher, Dies at 97

\url{https://nyti.ms/3badClK}

\begin{itemize}
\item
\item
\item
\item
\item
\end{itemize}

\href{https://www.nytimes3xbfgragh.onion/news-event/coronavirus?action=click\&pgtype=Article\&state=default\&region=TOP_BANNER\&context=storylines_menu}{The
Coronavirus Outbreak}

\begin{itemize}
\tightlist
\item
  live\href{https://www.nytimes3xbfgragh.onion/2020/08/04/world/coronavirus-covid-19.html?action=click\&pgtype=Article\&state=default\&region=TOP_BANNER\&context=storylines_menu}{Latest
  Updates}
\item
  \href{https://www.nytimes3xbfgragh.onion/interactive/2020/us/coronavirus-us-cases.html?action=click\&pgtype=Article\&state=default\&region=TOP_BANNER\&context=storylines_menu}{Maps
  and Cases}
\item
  \href{https://www.nytimes3xbfgragh.onion/interactive/2020/science/coronavirus-vaccine-tracker.html?action=click\&pgtype=Article\&state=default\&region=TOP_BANNER\&context=storylines_menu}{Vaccine
  Tracker}
\item
  \href{https://www.nytimes3xbfgragh.onion/2020/08/02/us/covid-college-reopening.html?action=click\&pgtype=Article\&state=default\&region=TOP_BANNER\&context=storylines_menu}{College
  Reopening}
\item
  \href{https://www.nytimes3xbfgragh.onion/live/2020/08/03/business/stock-market-today-coronavirus?action=click\&pgtype=Article\&state=default\&region=TOP_BANNER\&context=storylines_menu}{Economy}
\end{itemize}

Advertisement

\protect\hyperlink{after-top}{Continue reading the main story}

Supported by

\protect\hyperlink{after-sponsor}{Continue reading the main story}

Those We've Lost

\hypertarget{wynn-handman-influential-director-and-teacher-dies-at-97}{%
\section{Wynn Handman, Influential Director and Teacher, Dies at
97}\label{wynn-handman-influential-director-and-teacher-dies-at-97}}

At the American Place Theater, he championed new works. In his acting
classes, he nurtured countless future stars. His death was related to
the coronavirus.

\includegraphics{https://static01.graylady3jvrrxbe.onion/images/2020/04/19/obituaries/19handman-obit1/merlin_171555627_1dcbf573-0ea1-41d9-83b6-547de257bf2e-articleLarge.jpg?quality=75\&auto=webp\&disable=upscale}

\href{https://www.nytimes3xbfgragh.onion/by/neil-genzlinger}{\includegraphics{https://static01.graylady3jvrrxbe.onion/images/2018/06/13/multimedia/author-neil-genzlinger/author-neil-genzlinger-thumbLarge.jpg}}

By \href{https://www.nytimes3xbfgragh.onion/by/neil-genzlinger}{Neil
Genzlinger}

\begin{itemize}
\item
  Published April 14, 2020Updated April 16, 2020
\item
  \begin{itemize}
  \item
  \item
  \item
  \item
  \item
  \end{itemize}
\end{itemize}

\emph{This obituary is part of a series about}
\href{https://www.nytimes3xbfgragh.onion/series/people-who-have-died-of-the-coronavirus}{\emph{people
who have died in the coronavirus pandemic}}\emph{.}

Wynn Handman, a director and acting teacher who shaped the careers of
Dustin Hoffman, Joel Grey, Faye Dunaway, Richard Gere and other stars in
his acting classes and at the influential American Place Theater in
Manhattan, which he co-founded, died on Saturday at his home in
Manhattan. He was 97.

His daughter Laura Handman said the cause was pneumonia related to the
coronavirus.

In addition to mentoring actors, Mr. Handman was an advocate of new
American plays and those who wrote them.

He founded the American Place Theater in 1963 with Michael Tolan, an
actor, and Sidney Lanier, vicar of St. Clement's Episcopal Church on
West 46th Street in Manhattan, where the theater was located in its
early years. Their mission was to promote new voices, approaches and
subjects, an alternative to the often constricted commercial offerings
nearby in the Broadway houses.

\includegraphics{https://static01.graylady3jvrrxbe.onion/images/2020/04/19/obituaries/19handman-obit2/merlin_171554691_9b2f8dd2-619e-457e-a68e-7bf7f7c20930-articleLarge.jpg?quality=75\&auto=webp\&disable=upscale}

``As a producer, Wynn brought the Greenwich Village theater revolution
to spitting distance from Broadway, which, as far as he was concerned,
was the enemy,'' the theater journalist Jeremy Gerard, author of
\href{https://www.americantheatre.org/2014/06/14/wynn-place-show-honors-the-career-of-producer-wynn-handman-2/}{``Wynn
Place Show: A Biased History of the Rollicking Life \& Extreme Times of
Wynn Handman and The American Place Theatre''} (2013), said by email.
The theater, he said, ``shocked audiences --- and many critics --- with
early plays by downtown anarchists (Sam Shepard), Black Power militants
(Ed Bullins) and emerging feminists (María Irene Fornés).''

Mr. Handman, who served as artistic director of the theater --- which
was still producing plays into this century --- admitted that he wasn't
chasing the kind of success most producers and directors craved.

``I was drawn to challenging plays, plays that would not succeed
commercially and therefore needed a home,''
\href{https://www.nytimes3xbfgragh.onion/2013/12/26/arts/former-students-and-wynn-place-show-praise-wynn-handman.html}{he
told The New York Times in 2013}. ``It was never in my mind to do a play
that would become a hit. But that's what most New York theaters are all
about today.''

His greatest hits, it might be said, were the actors who came through
his classes, which he began teaching in the 1950s. Other acting
teachers, like Lee Strasberg, may have been better known, but Mr.
Handman's workshop, for years held in a cramped space near Carnegie
Hall, was just as intense.

``It was a lot of technique, truth, moment-to-moment, how to listen,
improv,'' Burt Reynolds, a student early in his career,
\href{https://www.nytimes3xbfgragh.onion/1981/03/29/magazine/burt-reynolds-going-beyond-macho.html?searchResultPosition=5}{told
The New York Times} in 1981.

In the 2019 documentary
\href{https://www.netflix.com/title/81078456}{``It Takes a Lunatic,''}
directed by Billy Lyons, the actress Marianne Leone Cooper recalled,
``He worked with me for six months on nothing but stillness.''

James Caan, another of the many actors who paid tribute in the
documentary, remembered serious work seriously tackled. ``We didn't
spend a lot of time being trees, you know what I mean?'' he said in the
film.

Image

Mr. Handman, seated, with, from left, Billy Lyons, Robert De Niro and
Michael Douglas at a screening of ``It Takes a Lunatic,'' Mr. Lyons's
documentary about Mr. Handman, at the 2019 Tribeca Film
Festival.Credit...Brent N. Clarke/Invision, via Associated Press

Mr. Handman was still teaching decades later when John Leguizamo tested
out ``Mambo Mouth,'' his breakthrough solo show, which became an Off
Broadway hit in 1990, in one of his classes.

``Wynn sat there laughing and carrying on like any other audience
member,'' Mr. Leguizamo wrote in the foreword to Mr. Gerard's biography,
``but when I was done he cut into it like a surgeon trying to save an
organ without killing the patient.''

Irwin Leo Handman (Wynn had long been his legal name, his daughter said)
was born on May 19, 1922, in Manhattan. His father, Nathan, ran a
printing business, and his mother, Anna (Kemler) Handman, was a
saleswoman at Saks Fifth Avenue.

He grew up in the Inwood section of Manhattan, although that may conjure
a different image to the reader of 2020 than it did almost a century
ago.

``There was a farm across the street,'' Mr. Handman said in the
documentary. ``A real farm. That's true. I had such a happy childhood
that I never wanted to leave Inwood.''

Mr. Handman graduated from DeWitt Clinton High School in the Bronx in
1938 and the City College of New York in 1943, later earning a master's
degree in speech pathology from Teachers College at Columbia University.
After graduating from City College he enlisted in the Coast Guard,
serving on an icebreaker that was assigned to knock out a German weather
station in the Arctic. The mission was a success, and a number of
Germans were taken prisoner.

``When the Germans came aboard the ship, I didn't feel like saluting
them,'' Mr. Handman, who was Jewish, said in the documentary, but his
commander ordered him to follow protocol and do so.

While at sea he would sometimes entertain his shipmates with skits, and
the experience led him to think about acting once the war ended. He
applied to the Neighborhood Playhouse, Sanford Meisner's theater school,
and studied there from 1946 to 1948.

He wanted to act, but Mr. Meisner saw him as a director and in 1949
suggested he lead a summer theater in the Adirondacks where some
Neighborhood Playhouse students were in repertory. Mr. Handman was
reluctant, but Barbara Ann Schlein, whom he would marry the next year,
urged him to try it.

``I found myself, my calling, that summer,'' Mr. Handman told Mr. Gerard
in an interview for the biography.

Mr. Handman taught at the Neighborhood Playhouse from 1948 to 1955, but
in 1952 he also began teaching his own acting classes, and in 1955 he
broke away from Mr. Meisner. His studio across from Carnegie Hall was
furnished with salvaged wooden auditorium seats.

``Its warmth and funkiness were chemical to him,'' said Jonathan Slaff,
a theater publicist who studied with Mr. Handman and represented the
theater in the mid-1990s, ``and he transported its seating and décor
into a studio he established on the eighth floor of Carnegie Hall and,
later, on the 10th floor of 244 West 54th Street.''

Separate from his teaching was the American Place Theater. For the first
year or so it devoted itself to readings. Its first full production, in
November 1964, was ``The Old Glory'' by Robert Lowell, the poet, his
first stage production. It won an
\href{https://www.obieawards.com/events/1960s/year-65/}{Obie Award} for
best American play.

Image

Mr. Handman and others at a rehearsal in 1968 for the American Place
Theater production of ``The Cannibals,'' George Tabori's play about
cannibalism in a Nazi death camp.~Credit...Martha Holmes

The next year the theater staged ``Harry, Noon and Night'' by Ronald
Ribman, with Mr. Grey and Mr. Hoffman in the cast. ``Hogan's Goat'' by
William Alfred was also done that year, with Ms. Dunaway in the cast.

In 1970 the theater moved to a custom-built space on West 46th Street.

Over the decades, the theater's offerings were nothing if not eclectic.
In 1968 there was ``The Cannibals,''
\href{https://www.nytimes3xbfgragh.onion/2007/07/27/theater/27tabori.html}{George
Tabori's} gruesome tale of cannibalism in a Nazi death camp. In 1986
there was Eric Bogosian's ``Drinking in America.'' In 1998 came Aasif
Mandvi's solo show, ``Sakina's Restaurant.''

``He helped foster idiosyncratic work,'' Mr. Bogosian
\href{https://www.nytimes3xbfgragh.onion/2007/05/20/nyregion/20wynn.html}{told
The Times} in 2007. ``He has a great eye for what's good, what's
honest.''

Mr. Handman's wife, known as Bobbie,
\href{https://www.nytimes3xbfgragh.onion/2013/11/15/nyregion/bobbie-handman-a-medal-of-arts-winner-dies-at-85.html}{died
in 2013}. In addition to his daughter Laura, he is survived by another
daughter, Liza Eleanor Handman; two grandchildren; and a
great-granddaughter.

Mr. Handman was still teaching when he contracted the virus.

``As soon as the lockdown was over,'' Mr. Slaff said, ``he would have
been back in class.''

\href{https://www.nytimes3xbfgragh.onion/interactive/2020/obituaries/people-died-coronavirus-obituaries.html?action=click\&pgtype=Article\&state=default\&region=BELOW_MAIN_CONTENT\&context=covid_obits_promo}{}

\hypertarget{those-weve-lost}{%
\section{Those We've Lost}\label{those-weve-lost}}

The coronavirus pandemic has taken an incalculable death toll. This
series is designed to put names and faces to the numbers.

Read more

\includegraphics{https://static01.graylady3jvrrxbe.onion/images/2020/07/30/obituaries/30Pedro/30Pedro-square640.jpg}

\hypertarget{bernaldina-josuxe9-pedro}{%
\section{Bernaldina José Pedro}\label{bernaldina-josuxe9-pedro}}

d. Boa Vista, Brazil

Leader among the Indigenous Macuxi

\includegraphics{https://static01.graylady3jvrrxbe.onion/images/2020/07/31/obituaries/31Swing/merlin_175167783_8913bc90-0d64-43f3-a655-1bb1bf1601c9-square640.jpg}

\hypertarget{john-eric-swing}{%
\section{John Eric Swing}\label{john-eric-swing}}

d. Fountain Valley, Calif.

Champion of Filipino-Americans

\includegraphics{https://static01.graylady3jvrrxbe.onion/images/2020/07/27/obituaries/27Victor/merlin_175001436_38b11f8e-227a-4e2c-9821-7618af9b2524-square640.jpg}

\hypertarget{victor-victor}{%
\section{Victor Victor}\label{victor-victor}}

d. Santo Domingo, Dominican Republic

Beloved musician of the Dominican Republic

\includegraphics{https://static01.graylady3jvrrxbe.onion/images/2020/07/31/obituaries/31Negron/merlin_175160169_516322ae-fd23-4969-b6b2-193ced371105-square640.jpg}

\hypertarget{dr-eddie-negruxf3n}{%
\section{Dr. Eddie Negrón}\label{dr-eddie-negruxf3n}}

d. Fort Walton Beach, Fla.

Internist on Florida's Emerald Coast

\includegraphics{https://static01.graylady3jvrrxbe.onion/images/2020/07/30/obituaries/30Dobson/merlin_175115928_f6b9271c-8f05-4fe1-a38a-5ca4a58f8935-square640.jpg}

\hypertarget{dobby-dobson}{%
\section{Dobby Dobson}\label{dobby-dobson}}

d. Coral Springs, Fla.

Jamaican singer and songwriter

\includegraphics{https://static01.graylady3jvrrxbe.onion/images/2020/08/01/obituaries/28Gonzalez/merlin_175002771_beb57888-3951-409a-ae13-03a94b2e962e-square640.jpg}

\hypertarget{waldemar-gonzalez}{%
\section{Waldemar Gonzalez}\label{waldemar-gonzalez}}

d. White Plains, N.Y.

Teacher and social worker

Advertisement

\protect\hyperlink{after-bottom}{Continue reading the main story}

\hypertarget{site-index}{%
\subsection{Site Index}\label{site-index}}

\hypertarget{site-information-navigation}{%
\subsection{Site Information
Navigation}\label{site-information-navigation}}

\begin{itemize}
\tightlist
\item
  \href{https://help.nytimes3xbfgragh.onion/hc/en-us/articles/115014792127-Copyright-notice}{©~2020~The
  New York Times Company}
\end{itemize}

\begin{itemize}
\tightlist
\item
  \href{https://www.nytco.com/}{NYTCo}
\item
  \href{https://help.nytimes3xbfgragh.onion/hc/en-us/articles/115015385887-Contact-Us}{Contact
  Us}
\item
  \href{https://www.nytco.com/careers/}{Work with us}
\item
  \href{https://nytmediakit.com/}{Advertise}
\item
  \href{http://www.tbrandstudio.com/}{T Brand Studio}
\item
  \href{https://www.nytimes3xbfgragh.onion/privacy/cookie-policy\#how-do-i-manage-trackers}{Your
  Ad Choices}
\item
  \href{https://www.nytimes3xbfgragh.onion/privacy}{Privacy}
\item
  \href{https://help.nytimes3xbfgragh.onion/hc/en-us/articles/115014893428-Terms-of-service}{Terms
  of Service}
\item
  \href{https://help.nytimes3xbfgragh.onion/hc/en-us/articles/115014893968-Terms-of-sale}{Terms
  of Sale}
\item
  \href{https://spiderbites.nytimes3xbfgragh.onion}{Site Map}
\item
  \href{https://help.nytimes3xbfgragh.onion/hc/en-us}{Help}
\item
  \href{https://www.nytimes3xbfgragh.onion/subscription?campaignId=37WXW}{Subscriptions}
\end{itemize}
