Sections

SEARCH

\protect\hyperlink{site-content}{Skip to
content}\protect\hyperlink{site-index}{Skip to site index}

\href{https://www.nytimes3xbfgragh.onion/section/books/review}{Book
Review}

\href{https://myaccount.nytimes3xbfgragh.onion/auth/login?response_type=cookie\&client_id=vi}{}

\href{https://www.nytimes3xbfgragh.onion/section/todayspaper}{Today's
Paper}

\href{/section/books/review}{Book Review}\textbar{}Stashed in the Marsh,
Left in the Road: Victims Pile Up in Four New Crime Novels

\url{https://nyti.ms/346KdGp}

\begin{itemize}
\item
\item
\item
\item
\item
\end{itemize}

Advertisement

\protect\hyperlink{after-top}{Continue reading the main story}

Supported by

\protect\hyperlink{after-sponsor}{Continue reading the main story}

\href{/column/crime}{Crime}

\hypertarget{stashed-in-the-marsh-left-in-the-road-victims-pile-up-in-four-new-crime-novels}{%
\section{Stashed in the Marsh, Left in the Road: Victims Pile Up in Four
New Crime
Novels}\label{stashed-in-the-marsh-left-in-the-road-victims-pile-up-in-four-new-crime-novels}}

\includegraphics{https://static01.graylady3jvrrxbe.onion/images/2020/04/05/books/review/05Crime/05Crime-articleLarge.jpg?quality=75\&auto=webp\&disable=upscale}

By Marilyn Stasio

\begin{itemize}
\item
  April 3, 2020
\item
  \begin{itemize}
  \item
  \item
  \item
  \item
  \item
  \end{itemize}
\end{itemize}

The voices of the dead tell the unhappy story of \textbf{PLEASE SEE US
(Gallery, 341 pp., \$26.99),} Caitlin Mullen's spellbinding debut novel
about a series of murders in Atlantic City. When the narrative opens,
there are two dead women sprawled in the marsh behind the Sunset Motel.
By the end of the story, there will be seven of these Jane Does, all
victims of a killer who targets prostitutes, the most disposable of
women.

Then an older man consults a psychic named Clara, hoping she might lead
him to his 18-year-old niece, Julie Zale, who has been missing for
months. ``I didn't need to be a psychic to know that something bad had
happened to Julie Zale,'' says Clara. ``I had witnessed what happens to
girls who run away and wash up here, like debris dragged in by the
tide.'' Mullen pays these women (who are beginning to pile up behind the
motel) the courtesy of rich histories and sympathetic understanding of
how they met their sad fates.

Janes No. 1 and No. 2 are Jersey girls. They both had jobs on the
boardwalk when they were young and drunk on Springsteen songs, and they
both felt betrayed by the city that had promised them ``grand
destinies.'' Jane No. 3 is a married woman suffering from postpartum
depression who bolted from her husband and their baby girl. Jane No. 4
is Julie Zale, a prostitute's daughter who had recently taken up her
mother's profession.

Even as the murders multiply, ``the city won't treat the cases as the
work of one person,'' Mullen cynically observes. ``Pressure from the
politicians, from the casinos; the words \emph{serial killer} will scare
off whatever tourists are left.''

The pity of it is that, by refusing to link the cases, the city
heartlessly deprives the victims of the only comfort and dignity left to
them --- their kinship as women. ``Having lain there, sisterly, so
close, for so long, the women bristle at the absurdity.''

♦

It's such a cheat to throw suspicion on the clearly blameless
protagonist of your story, yet writers do it all the time. Even a pro
like Julia Spencer-Fleming, whose mysteries featuring Clare Fergusson
and Russ Van Alstyne are such a pleasure, resorts to that narrative
device. In \textbf{HID FROM OUR EYES (Minotaur, 339 pp., \$27.99),} Van
Alstyne, now the police chief of the rugged Adirondacks town of Millers
Kill, came under suspicion in 1972, when a dead woman in a white lace
minidress turned up in the middle of Route 137. Now, another victim ---
``pretty, young, all dressed up, with no shoes or pantyhose'' --- brings
back this lawman's memories of what it's like to be innocent but unable
to prove it.

Because Millers Kill is a small town, the mystery unfolds like a classic
country whodunit, complete with lurid back stories for all the righteous
grown-ups. And because Van Alstyne is married to an Episcopal priest who
is expecting their first child, the human elements cushion the scenes of
violence. Tough, but kind of sweet, if you know what I mean.

♦

Once you've seen a strand of trees choked by the relentless kudzu vines
that grow wild in the South, you can imagine what it must feel like to
be strangled by a python. In \textbf{BLACKWOOD (Little, Brown, 293 pp.,
\$27),} Michael Farris Smith uses the rampant growth of kudzu as a
metaphor for the generational secrets and sins that in 1975 are
suffocating the Mississippi hill of Red Bluff. But when a stranger who
has driven into Red Bluff in a decrepit Cadillac with a woman and a boy
(and minus the baby in a diaper they abandoned at a charity donation
center) peers under a canopy of kudzu, he sees safety.

Red Bluff also offers shelter to Colburn Evans, an ``industrial
sculptor'' who works with found objects, attracted by the free
storefront space given to artists for studios. Once Colburn connects
with Celia, who owns the local bar, and the family of vagrants gets the
town all riled up, the stage is set for a Southern Gothic tale that is
filled with pain and passion and something darker, something evil that
leads Colburn down into a hidden cave and face to face with his own
worst nightmare.

♦

What kind of person would kill someone and then steal his cat? Someone
like the coldblooded murderer in Peter Swanson's devious whodunit,
\textbf{EIGHT PERFECT MURDERS (Morrow/HarperCollins, 270 pp., \$26.99),}
that's who. The chatty storyteller of Swanson's twisty mystery is
Malcolm Kershaw, the proprietor of Old Devils Bookstore on Boston's
Beacon Hill. A trustworthy narrator on all genre matters, Mal once
compiled a list of eight perfect literary murders and posted it on the
store's blog, paying tribute to classics like Patricia Highsmith's
``Strangers on a Train'' and James M. Cain's ``Double Indemnity,'' but
it seems to have inspired a killer to replicate these fictional murders
in real life. Although Swanson doesn't stint on the homicidal details
(``There was a splintery crack as he staggered back, blood falling in a
sheet down over his chin''), I'd go along with the reader who took
exception to Mal's list: ``Anyone who writes a list of perfect murders
that doesn't have at least one John Dickson Carr on it obviously knows
nothing about anything.''

Advertisement

\protect\hyperlink{after-bottom}{Continue reading the main story}

\hypertarget{site-index}{%
\subsection{Site Index}\label{site-index}}

\hypertarget{site-information-navigation}{%
\subsection{Site Information
Navigation}\label{site-information-navigation}}

\begin{itemize}
\tightlist
\item
  \href{https://help.nytimes3xbfgragh.onion/hc/en-us/articles/115014792127-Copyright-notice}{©~2020~The
  New York Times Company}
\end{itemize}

\begin{itemize}
\tightlist
\item
  \href{https://www.nytco.com/}{NYTCo}
\item
  \href{https://help.nytimes3xbfgragh.onion/hc/en-us/articles/115015385887-Contact-Us}{Contact
  Us}
\item
  \href{https://www.nytco.com/careers/}{Work with us}
\item
  \href{https://nytmediakit.com/}{Advertise}
\item
  \href{http://www.tbrandstudio.com/}{T Brand Studio}
\item
  \href{https://www.nytimes3xbfgragh.onion/privacy/cookie-policy\#how-do-i-manage-trackers}{Your
  Ad Choices}
\item
  \href{https://www.nytimes3xbfgragh.onion/privacy}{Privacy}
\item
  \href{https://help.nytimes3xbfgragh.onion/hc/en-us/articles/115014893428-Terms-of-service}{Terms
  of Service}
\item
  \href{https://help.nytimes3xbfgragh.onion/hc/en-us/articles/115014893968-Terms-of-sale}{Terms
  of Sale}
\item
  \href{https://spiderbites.nytimes3xbfgragh.onion}{Site Map}
\item
  \href{https://help.nytimes3xbfgragh.onion/hc/en-us}{Help}
\item
  \href{https://www.nytimes3xbfgragh.onion/subscription?campaignId=37WXW}{Subscriptions}
\end{itemize}
