Sections

SEARCH

\protect\hyperlink{site-content}{Skip to
content}\protect\hyperlink{site-index}{Skip to site index}

\href{https://myaccount.nytimes3xbfgragh.onion/auth/login?response_type=cookie\&client_id=vi}{}

\href{https://www.nytimes3xbfgragh.onion/section/todayspaper}{Today's
Paper}

\href{/section/opinion}{Opinion}\textbar{}Germany Got Testing Right

\url{https://nyti.ms/2W5fxlg}

\begin{itemize}
\item
\item
\item
\item
\item
\end{itemize}

Advertisement

\protect\hyperlink{after-top}{Continue reading the main story}

\href{/section/opinion}{Opinion}

Supported by

\protect\hyperlink{after-sponsor}{Continue reading the main story}

\hypertarget{germany-got-testing-right}{%
\section{Germany Got Testing Right}\label{germany-got-testing-right}}

What can we learn?

\href{https://www.nytimes3xbfgragh.onion/by/david-leonhardt}{\includegraphics{https://static01.graylady3jvrrxbe.onion/images/2020/05/01/multimedia/David-Leonhardt-Headshot-The-Morning/David-Leonhardt-Headshot-The-Morning-thumbLarge-v3.png}}

By \href{https://www.nytimes3xbfgragh.onion/by/david-leonhardt}{David
Leonhardt}

Opinion Columnist

\begin{itemize}
\item
  April 28, 2020
\item
  \begin{itemize}
  \item
  \item
  \item
  \item
  \item
  \end{itemize}
\end{itemize}

\includegraphics{https://static01.graylady3jvrrxbe.onion/images/2020/04/28/opinion/28leonhardt-newsletter/merlin_171978321_f8ab3526-4880-4a48-bf66-50c74fbb3f38-articleLarge.jpg?quality=75\&auto=webp\&disable=upscale}

\emph{This article is part of David Leonhardt's newsletter. You can}
\href{https://www.nytimes3xbfgragh.onion/newsletters/opiniontoday?action=click\&module=Intentional\&pgtype=Article}{\emph{sign
up here}} \emph{to receive it each weekday.}

More than 20,000 people \href{https://coronavirus.jhu.edu/map.html}{have
died} from the
\href{https://www.nytimes3xbfgragh.onion/2020/05/06/world/europe/germany-merkel-coronavirus-reopening.html}{coronavirus}
in each of these European countries: France, Italy, Spain and the United
Kingdom. More than 17,000 have died in New York City.

But in
\href{https://www.nytimes3xbfgragh.onion/2020/05/06/world/europe/germany-merkel-coronavirus-reopening.html}{Germany}
--- which is home to more people than any one of those other European
countries and 10 times as many as New York --- only about 6,000 have
died.

How could that be? There are multiple reasons, but the biggest is
probably the country's approach to testing.

As
\href{https://www.nytimes3xbfgragh.onion/2020/04/04/world/europe/germany-coronavirus-death-rate.html}{Katrin
Bennhold} of The Times has written:

\begin{quote}
By the time Germany recorded its first case of Covid-19 in February,
laboratories across the country had built up a stock of test kits
\ldots{} Early and widespread testing has allowed the authorities to
slow the spread of the pandemic by isolating known cases while they are
infectious. It has also enabled lifesaving treatment to be administered
in a more timely way.
\end{quote}

Testing is the key to every effective strategy for fighting the virus.
It allows the sick to be treated effectively. It enables government
officials and hospitals to focus their resources on the areas that need
it most. And it makes sure that people who have the virus without
symptoms --- and can unknowingly spread it to others --- can be
isolated.

The troubled response to the virus in the United States began with
testing failures, and there are still not nearly enough tests being
conducted. As
\href{https://www.nytimes3xbfgragh.onion/2020/04/28/opinion/coronavirus-testing.html}{Michael
Osterholm and Mark Olshaker} write in a new Op-Ed: ``Far too few tests
are available in the United States. Some are shoddy. Even the ones that
are precise aren't designed to produce the kind of definitive yes-no
results that people expect.''

The testing problem will need to be solved before states can return to
normal --- as some are now taking early steps toward --- without
sparking new outbreaks.

\textbf{For more} \ldots{}

\begin{itemize}
\item
  Scott Gottlieb, of the American Enterprise Institute: ``If we can
  develop a point of care, swabable stick that gives a readable result
  in the doctor's office it can dramatically increase screening.''
\item
  Shan Soe-Lin and Robert Hecht, Yale University: **** ``What's needed
  is widespread testing of people with no known symptoms \ldots{} We
  need to aggressively search for asymptomatic carriers, particularly
  among people who have frequent contact with the public and among
  vulnerable populations. This includes those who are infectious but
  will never develop symptoms and those who will develop them days after
  the test.''
\item
  And
  \href{https://www.nytimes3xbfgragh.onion/2020/04/28/opinion/coronavirus-testing.html}{more
  from} Osterholm, an infectious disease expert, and Olshaker, a writer:
  ``Governments throughout the world and the research, medical-supply
  and clinical-lab industries must unite to vastly increase global
  production of reagents and sampling equipment. Achieving this will
  take months and require building new capacity, presumably with public
  subsidies. The time and costs involved will be considerable, but such
  an effort is the only way to test large populations for this infection
  (and for others in the future).''
\end{itemize}

\emph{If you are not a subscriber to this newsletter, you can}
\href{https://www.nytimes3xbfgragh.onion/newsletters/david-leonhardt}{\emph{subscribe
here}}\emph{. You can also join me on}
\href{https://twitter.com/DLeonhardt}{\emph{Twitter (@DLeonhardt)}}
\emph{and}
\href{https://www.facebookcorewwwi.onion/DavidRLeonhardt/}{\emph{Facebook}}\emph{.}

\emph{Follow The New York Times Opinion section on}
\href{https://www.facebookcorewwwi.onion/nytopinion}{\emph{Facebook}}\emph{,}
\href{http://twitter.com/NYTOpinion}{\emph{Twitter (@NYTopinion)}}
\emph{and}
\href{https://www.instagram.com/nytopinion/}{\emph{Instagram}}\emph{.}

Advertisement

\protect\hyperlink{after-bottom}{Continue reading the main story}

\hypertarget{site-index}{%
\subsection{Site Index}\label{site-index}}

\hypertarget{site-information-navigation}{%
\subsection{Site Information
Navigation}\label{site-information-navigation}}

\begin{itemize}
\tightlist
\item
  \href{https://help.nytimes3xbfgragh.onion/hc/en-us/articles/115014792127-Copyright-notice}{©~2020~The
  New York Times Company}
\end{itemize}

\begin{itemize}
\tightlist
\item
  \href{https://www.nytco.com/}{NYTCo}
\item
  \href{https://help.nytimes3xbfgragh.onion/hc/en-us/articles/115015385887-Contact-Us}{Contact
  Us}
\item
  \href{https://www.nytco.com/careers/}{Work with us}
\item
  \href{https://nytmediakit.com/}{Advertise}
\item
  \href{http://www.tbrandstudio.com/}{T Brand Studio}
\item
  \href{https://www.nytimes3xbfgragh.onion/privacy/cookie-policy\#how-do-i-manage-trackers}{Your
  Ad Choices}
\item
  \href{https://www.nytimes3xbfgragh.onion/privacy}{Privacy}
\item
  \href{https://help.nytimes3xbfgragh.onion/hc/en-us/articles/115014893428-Terms-of-service}{Terms
  of Service}
\item
  \href{https://help.nytimes3xbfgragh.onion/hc/en-us/articles/115014893968-Terms-of-sale}{Terms
  of Sale}
\item
  \href{https://spiderbites.nytimes3xbfgragh.onion}{Site Map}
\item
  \href{https://help.nytimes3xbfgragh.onion/hc/en-us}{Help}
\item
  \href{https://www.nytimes3xbfgragh.onion/subscription?campaignId=37WXW}{Subscriptions}
\end{itemize}
