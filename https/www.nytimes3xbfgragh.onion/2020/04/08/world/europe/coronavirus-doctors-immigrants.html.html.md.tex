Sections

SEARCH

\protect\hyperlink{site-content}{Skip to
content}\protect\hyperlink{site-index}{Skip to site index}

\href{https://www.nytimes3xbfgragh.onion/section/world/europe}{Europe}

\href{https://myaccount.nytimes3xbfgragh.onion/auth/login?response_type=cookie\&client_id=vi}{}

\href{https://www.nytimes3xbfgragh.onion/section/todayspaper}{Today's
Paper}

\href{/section/world/europe}{Europe}\textbar{}Eight U.K. Doctors Died
From Coronavirus. All Were Immigrants.

\url{https://nyti.ms/34jJf9M}

\begin{itemize}
\item
\item
\item
\item
\item
\end{itemize}

\hypertarget{the-coronavirus-outbreak}{%
\subsubsection{\texorpdfstring{\href{https://www.nytimes3xbfgragh.onion/news-event/coronavirus?name=styln-coronavirus-national\&region=TOP_BANNER\&variant=undefined\&block=storyline_menu_recirc\&action=click\&pgtype=Article\&impression_id=4ad93cf0-e393-11ea-918c-0f8207e68c04}{The
Coronavirus
Outbreak}}{The Coronavirus Outbreak}}\label{the-coronavirus-outbreak}}

\begin{itemize}
\tightlist
\item
  live\href{https://www.nytimes3xbfgragh.onion/2020/08/21/world/covid-19-coronavirus.html?name=styln-coronavirus-national\&region=TOP_BANNER\&variant=undefined\&block=storyline_menu_recirc\&action=click\&pgtype=Article\&impression_id=4ad93cf1-e393-11ea-918c-0f8207e68c04}{Latest
  Updates}
\item
  \href{https://www.nytimes3xbfgragh.onion/interactive/2020/us/coronavirus-us-cases.html?name=styln-coronavirus-national\&region=TOP_BANNER\&variant=undefined\&block=storyline_menu_recirc\&action=click\&pgtype=Article\&impression_id=4ad96400-e393-11ea-918c-0f8207e68c04}{Maps
  and Cases}
\item
  \href{https://www.nytimes3xbfgragh.onion/interactive/2020/science/coronavirus-vaccine-tracker.html?name=styln-coronavirus-national\&region=TOP_BANNER\&variant=undefined\&block=storyline_menu_recirc\&action=click\&pgtype=Article\&impression_id=4ad96401-e393-11ea-918c-0f8207e68c04}{Vaccine
  Tracker}
\item
  \href{https://www.nytimes3xbfgragh.onion/2020/08/19/us/colleges-closing-covid.html?name=styln-coronavirus-national\&region=TOP_BANNER\&variant=undefined\&block=storyline_menu_recirc\&action=click\&pgtype=Article\&impression_id=4ad96402-e393-11ea-918c-0f8207e68c04}{Colleges
  Closing}
\item
  \href{https://www.nytimes3xbfgragh.onion/live/2020/08/20/business/stock-market-today-coronavirus?name=styln-coronavirus-national\&region=TOP_BANNER\&variant=undefined\&block=storyline_menu_recirc\&action=click\&pgtype=Article\&impression_id=4ad96403-e393-11ea-918c-0f8207e68c04}{Economy}
\end{itemize}

Advertisement

\protect\hyperlink{after-top}{Continue reading the main story}

Supported by

\protect\hyperlink{after-sponsor}{Continue reading the main story}

\hypertarget{eight-uk-doctors-died-from-coronavirus-all-were-immigrants}{%
\section{Eight U.K. Doctors Died From Coronavirus. All Were
Immigrants.}\label{eight-uk-doctors-died-from-coronavirus-all-were-immigrants}}

In a country where anti-immigrant sentiment gave rise to the Brexit
movement, Britain's health care system depends heavily on foreign
doctors, who are now on the front lines fighting the epidemic.

\includegraphics{https://static01.graylady3jvrrxbe.onion/images/2020/04/06/world/00virus-immigrantdocs3/00virus-immigrantdocs3-articleLarge.jpg?quality=75\&auto=webp\&disable=upscale}

\href{https://www.nytimes3xbfgragh.onion/by/benjamin-mueller}{\includegraphics{https://static01.graylady3jvrrxbe.onion/images/2018/02/20/multimedia/author-benjamin-mueller/author-benjamin-mueller-thumbLarge.jpg}}

By
\href{https://www.nytimes3xbfgragh.onion/by/benjamin-mueller}{Benjamin
Mueller}

\begin{itemize}
\item
  Published April 8, 2020Updated April 13, 2020
\item
  \begin{itemize}
  \item
  \item
  \item
  \item
  \item
  \end{itemize}
\end{itemize}

LONDON --- The eight men moved to
\href{https://www.nytimes3xbfgragh.onion/2020/04/10/technology/coronavirus-5g-uk.html}{Britain}
from different corners of its former empire, all of them doctors or
doctors-to-be, becoming foot soldiers in the effort to build a free
universal health service after World War II.

Now their names have become stacked atop a grim list: the first, and so
far only,
\href{https://www.nytimes3xbfgragh.onion/2020/04/13/us/coronavirus-foreign-doctors-nurses-visas.html}{doctors}
publicly reported to have died after catching
\href{https://www.nytimes3xbfgragh.onion/2020/04/09/podcasts/the-daily/coronavirus-new-orleans.html}{the
coronavirus} in
\href{https://www.nytimes3xbfgragh.onion/2020/04/10/technology/coronavirus-5g-uk.html}{Britain's}
aching National Health Service.

For a country ripped apart in recent years by Brexit and the
anti-immigrant movement that birthed it, the deaths of the eight
\href{https://www.nytimes3xbfgragh.onion/2020/04/13/us/coronavirus-foreign-doctors-nurses-visas.html}{doctors}
--- from Egypt, India, Nigeria, Pakistan, Sri Lanka and Sudan --- attest
to the extraordinary dependence of Britain's treasured health service on
workers from abroad.

It is a story tinged with racism, as white, British doctors have largely
dominated the prestigious disciplines while foreign doctors have
typically found work in places and practices that are apparently putting
them on the
\href{https://www.nytimes3xbfgragh.onion/2020/03/05/world/europe/coronavirus-united-kingdom-national-health-service.html}{dangerous
front lines of the coronavirus pandemic}.

``When people were standing on the street clapping for N.H.S. workers, I
thought, `A year and a half ago, they were talking about Brexit and how
these immigrants have come into our country and want to take our
jobs,''' said Dr. Hisham el-Khidir, whose cousin Dr. Adil el-Tayar, a
transplant surgeon, died on March 25 from the
\href{https://www.nytimes3xbfgragh.onion/2020/04/10/technology/coronavirus-5g-uk.html}{coronavirus}
in western London.

\includegraphics{https://static01.graylady3jvrrxbe.onion/images/2020/04/06/world/00virus-immigrantdocs6/merlin_171322164_05ffcead-1d29-4bb6-afc6-b4d683fec4e9-articleLarge.jpg?quality=75\&auto=webp\&disable=upscale}

``Now today, it's the same immigrants that are trying to work with the
locals,'' said Dr. el-Khidir, a surgeon in Norwich, ``and they are dying
on the front lines.''

By Tuesday, 7,097 people had died in British hospitals from the
\href{https://www.nytimes3xbfgragh.onion/2020/04/10/world/asia/coronavirus-migrants.html}{coronavirus},
the government said on Wednesday, a leap of 938 from the day before, the
largest daily rise in the death toll.

And the victims have included not just the eight doctors but
\href{https://www.bbc.co.uk/news/uk-england-52165167}{a number of
nurses} who worked alongside them,
\href{https://www.theguardian.com/society/2020/apr/07/hong-kong-born-nurse-dies-of-coronavirus-after-44-years-with-nhs?utm_term=Autofeed\&CMP=twt_b-gdnnews\&utm_medium=Social\&utm_source=Twitter\#Echobox=1586279774}{at
least one from overseas.} Health workers are stretched thin as hospitals
across the country are filled with patients, including
\href{https://www.nytimes3xbfgragh.onion/2020/04/06/world/europe/boris-johnson-coronavirus-hospital-intensive-care.html}{Prime
Minister Boris Johnson}, who this week was moved into intensive care
with the coronavirus.

Britain is not the only country reckoning with its debt to foreign
doctors amid the terror and chaos of the pandemic. In the United States,
where
\href{https://www.reuters.com/article/us-health-professions-us-noncitizens/u-s-relies-heavily-on-foreign-born-healthcare-workers-idUSKBN1O32FR}{immigrants
make up more than a quarter of all doctors} but often face long waits
for green cards, New York and New Jersey have already cleared the way
for graduates of overseas medical schools to suit up in the coronavirus
response.

\hypertarget{latest-updates-the-coronavirus-outbreak}{%
\section{\texorpdfstring{\href{https://www.nytimes3xbfgragh.onion/2020/08/21/world/covid-19-coronavirus.html?action=click\&pgtype=Article\&state=default\&region=MAIN_CONTENT_1\&context=storylines_live_updates}{Latest
Updates: The Coronavirus
Outbreak}}{Latest Updates: The Coronavirus Outbreak}}\label{latest-updates-the-coronavirus-outbreak}}

Updated 2020-08-21T09:36:59.270Z

\begin{itemize}
\tightlist
\item
  \href{https://www.nytimes3xbfgragh.onion/2020/08/21/world/covid-19-coronavirus.html?action=click\&pgtype=Article\&state=default\&region=MAIN_CONTENT_1\&context=storylines_live_updates\#link-4690b6aa}{Shutdowns,
  warnings and scoldings follow gatherings on college campuses.}
\item
  \href{https://www.nytimes3xbfgragh.onion/2020/08/21/world/covid-19-coronavirus.html?action=click\&pgtype=Article\&state=default\&region=MAIN_CONTENT_1\&context=storylines_live_updates\#link-324af071}{As
  he accepts the Democratic nomination, Biden knocks Trump's pandemic
  response.}
\item
  \href{https://www.nytimes3xbfgragh.onion/2020/08/21/world/covid-19-coronavirus.html?action=click\&pgtype=Article\&state=default\&region=MAIN_CONTENT_1\&context=storylines_live_updates\#link-35890b73}{Hundreds
  of doctors in Kenya go on strike over their pay and protective gear.}
\end{itemize}

\href{https://www.nytimes3xbfgragh.onion/2020/08/21/world/covid-19-coronavirus.html?action=click\&pgtype=Article\&state=default\&region=MAIN_CONTENT_1\&context=storylines_live_updates}{See
more updates}

More live coverage:
\href{https://www.nytimes3xbfgragh.onion/live/2020/08/20/business/stock-market-today-coronavirus?action=click\&pgtype=Article\&state=default\&region=MAIN_CONTENT_1\&context=storylines_live_updates}{Markets}

But Britain, where
\href{https://www.ons.gov.uk/peoplepopulationandcommunity/populationandmigration/internationalmigration/articles/internationalmigrationandthehealthcareworkforce/2019-08-15}{nearly
a third of doctors in National Health Service hospitals} are immigrants,
has especially strong links to the medical school systems of its former
colonies, making it a natural landing place.

That was true for Dr. el-Tayar, 64, the oldest son of a government clerk
and a housewife from Atbara, Sudan, a railway city on the Nile.

He had 11 siblings, and one left a special impression: Osman, a brother,
who became ill as a child and died without suitable medical treatment.
Though Dr. el-Tayar rarely spoke of his brother's death, he gave the
same name to his oldest son.

``In my mind, I think that's what led him to medicine,'' Dr. el-Khidir
said. ``He didn't want anyone else in his family to feel that.''

Image

Dr. Adil El-Tayar

After graduating from the University of Khartoum, Dr. el-Tayar decided
to help address a tide of kidney disease sweeping across sub-Saharan
Africa. So he moved to Britain in the early 1990s to train as a
specialist transplant surgeon. He returned to Sudan around 2010 and
helped set up a transplant program there.

But the deteriorating political situation in Sudan and the recent birth
of a son persuaded Dr. el-Tayar to settle back in Britain, where he went
to work once again for the health service. Having lost his status as a
senior doctor when he left for Sudan, he had taken up work filling in at
a surgical assessment unit in Herefordshire, northwest of London,
examining patients coming through the emergency room.

It was there that his family believes Dr. el-Tayar, working with only
rudimentary protective gear, contracted the virus. Sequestered in the
western London home where he loved sitting next to his 12-year-old son,
he became so short of breath recently that he could not string together
a sentence. While on a ventilator, his heart failed him.

Had the health service started screening hospital patients for the virus
sooner or supplied doctors with better protective gear, Dr. el-Tayar
might have lived, said his cousin, Dr. el-Khidir.

``In our morbidity analyses, we go through each and every case and ask,
`Was it preventable? Was it avoidable?''' he said. ``I'm trying to
answer this question with my cousin now. Even with all the difficulties,
I've got to say the answer has to be yes.''

Analysts warn that doctor shortages across countries ravaged by the
coronavirus will worsen as the virus spreads. While ventilators may be
the scarcest resource for now, a shortage of doctors and nurses trained
to operate them could leave hospitals struggling to make use even of
what they have.

By recruiting foreign doctors, Britain saves the roughly \$270,000 in
taxpayer money that it costs to train doctors locally, a boon to a
system that does not spend enough on medical education to staff its own
hospitals. That effectively leaves Britain depending on the largess of
countries with weaker health care systems to train its own work force.

Image

Liverpool Town Hall, in northwestern England, illuminated with blue
lights in March in honor of the National Health Service.Credit...Paul
Ellis/Agence France-Presse --- Getty Images

Even so, the doctors are hampered by thousands of dollars in annual visa
fees and, on top of that, a \$500 surcharge for using the very health
service they work for.

Excluded from the most prestigious disciplines, immigrant doctors have
come to dominate so-called Cinderella specialties, like family and
elderly medicine, turning them into pillars of Britain's health system.
And unlike choosier Britain-born doctors, they have historically gone to
work in what one lawmaker in 1961 called ``the rottenest, worst
hospitals in the country,'' the very ones that most needed a doctor.

Those same places are now squarely in the path of the virus.

``Migrant doctors are architects of the N.H.S. --- they're what built it
and held it together and worked in the most unpopular, most difficult
areas, where white British doctors don't want to go and work,'' said Dr.
Aneez Esmail, a professor of general practice at the University of
Manchester. ``It's a hidden story.''

When Dr. el-Tayar moved to Britain in the 1990s, he was following a
pipeline laid by the family of another doctor who has now died after
contracting the coronavirus: Dr. Amged el-Hawrani, 55.

An ear, nose and throat specialist, Dr. el-Hawrani was about 11 when his
father, a radiologist, brought the family in 1975 from Khartoum to
Taunton, a town in southwestern England, and then Bristol, a bigger city
nearby.

Image

Dr. Amged el-HawraniCredit...University Hospitals of Derby and Burton
NHS Foundation Trust/Agence France-Presse --- Getty Images

Many Sudanese doctors at the time were burnishing their skills in
Britain before returning home or moving to Persian Gulf countries for
higher wages. But Dr. el-Hawrani's family turned their home into a
staging post for Sudanese doctors interested in longer-term stays,
hosting their families during exams or house hunts.

``The more the merrier,'' said Amal el-Hawrani, a younger brother of Dr.
el-Hawrani. ``My mum always liked that.''

\href{https://www.nytimes3xbfgragh.onion/news-event/coronavirus?action=click\&pgtype=Article\&state=default\&region=MAIN_CONTENT_3\&context=storylines_faq}{}

\hypertarget{the-coronavirus-outbreak-}{%
\subsubsection{The Coronavirus Outbreak
›}\label{the-coronavirus-outbreak-}}

\hypertarget{frequently-asked-questions}{%
\paragraph{Frequently Asked
Questions}\label{frequently-asked-questions}}

Updated August 17, 2020

\begin{itemize}
\item ~
  \hypertarget{why-does-standing-six-feet-away-from-others-help}{%
  \paragraph{Why does standing six feet away from others
  help?}\label{why-does-standing-six-feet-away-from-others-help}}

  \begin{itemize}
  \tightlist
  \item
    The coronavirus spreads primarily through droplets from your mouth
    and nose, especially when you cough or sneeze. The C.D.C., one of
    the organizations using that measure,
    \href{https://www.nytimes3xbfgragh.onion/2020/04/14/health/coronavirus-six-feet.html?action=click\&pgtype=Article\&state=default\&region=MAIN_CONTENT_3\&context=storylines_faq}{bases
    its recommendation of six feet} on the idea that most large droplets
    that people expel when they cough or sneeze will fall to the ground
    within six feet. But six feet has never been a magic number that
    guarantees complete protection. Sneezes, for instance, can launch
    droplets a lot farther than six feet,
    \href{https://jamanetwork.com/journals/jama/fullarticle/2763852}{according
    to a recent study}. It's a rule of thumb: You should be safest
    standing six feet apart outside, especially when it's windy. But
    keep a mask on at all times, even when you think you're far enough
    apart.
  \end{itemize}
\item ~
  \hypertarget{i-have-antibodies-am-i-now-immune}{%
  \paragraph{I have antibodies. Am I now
  immune?}\label{i-have-antibodies-am-i-now-immune}}

  \begin{itemize}
  \tightlist
  \item
    As of right
    now,\href{https://www.nytimes3xbfgragh.onion/2020/07/22/health/covid-antibodies-herd-immunity.html?action=click\&pgtype=Article\&state=default\&region=MAIN_CONTENT_3\&context=storylines_faq}{that
    seems likely, for at least several months.} There have been
    frightening accounts of people suffering what seems to be a second
    bout of Covid-19. But experts say these patients may have a
    drawn-out course of infection, with the virus taking a slow toll
    weeks to months after initial exposure. People infected with the
    coronavirus typically
    \href{https://www.nature.com/articles/s41586-020-2456-9}{produce}
    immune molecules called antibodies, which are
    \href{https://www.nytimes3xbfgragh.onion/2020/05/07/health/coronavirus-antibody-prevalence.html?action=click\&pgtype=Article\&state=default\&region=MAIN_CONTENT_3\&context=storylines_faq}{protective
    proteins made in response to an
    infection}\href{https://www.nytimes3xbfgragh.onion/2020/05/07/health/coronavirus-antibody-prevalence.html?action=click\&pgtype=Article\&state=default\&region=MAIN_CONTENT_3\&context=storylines_faq}{.
    These antibodies may} last in the body
    \href{https://www.nature.com/articles/s41591-020-0965-6}{only two to
    three months}, which may seem worrisome, but that's perfectly normal
    after an acute infection subsides, said Dr. Michael Mina, an
    immunologist at Harvard University. It may be possible to get the
    coronavirus again, but it's highly unlikely that it would be
    possible in a short window of time from initial infection or make
    people sicker the second time.
  \end{itemize}
\item ~
  \hypertarget{im-a-small-business-owner-can-i-get-relief}{%
  \paragraph{I'm a small-business owner. Can I get
  relief?}\label{im-a-small-business-owner-can-i-get-relief}}

  \begin{itemize}
  \tightlist
  \item
    The
    \href{https://www.nytimes3xbfgragh.onion/article/small-business-loans-stimulus-grants-freelancers-coronavirus.html?action=click\&pgtype=Article\&state=default\&region=MAIN_CONTENT_3\&context=storylines_faq}{stimulus
    bills enacted in March} offer help for the millions of American
    small businesses. Those eligible for aid are businesses and
    nonprofit organizations with fewer than 500 workers, including sole
    proprietorships, independent contractors and freelancers. Some
    larger companies in some industries are also eligible. The help
    being offered, which is being managed by the Small Business
    Administration, includes the Paycheck Protection Program and the
    Economic Injury Disaster Loan program. But lots of folks have
    \href{https://www.nytimes3xbfgragh.onion/interactive/2020/05/07/business/small-business-loans-coronavirus.html?action=click\&pgtype=Article\&state=default\&region=MAIN_CONTENT_3\&context=storylines_faq}{not
    yet seen payouts.} Even those who have received help are confused:
    The rules are draconian, and some are stuck sitting on
    \href{https://www.nytimes3xbfgragh.onion/2020/05/02/business/economy/loans-coronavirus-small-business.html?action=click\&pgtype=Article\&state=default\&region=MAIN_CONTENT_3\&context=storylines_faq}{money
    they don't know how to use.} Many small-business owners are getting
    less than they expected or
    \href{https://www.nytimes3xbfgragh.onion/2020/06/10/business/Small-business-loans-ppp.html?action=click\&pgtype=Article\&state=default\&region=MAIN_CONTENT_3\&context=storylines_faq}{not
    hearing anything at all.}
  \end{itemize}
\item ~
  \hypertarget{what-are-my-rights-if-i-am-worried-about-going-back-to-work}{%
  \paragraph{What are my rights if I am worried about going back to
  work?}\label{what-are-my-rights-if-i-am-worried-about-going-back-to-work}}

  \begin{itemize}
  \tightlist
  \item
    Employers have to provide
    \href{https://www.osha.gov/SLTC/covid-19/standards.html}{a safe
    workplace} with policies that protect everyone equally.
    \href{https://www.nytimes3xbfgragh.onion/article/coronavirus-money-unemployment.html?action=click\&pgtype=Article\&state=default\&region=MAIN_CONTENT_3\&context=storylines_faq}{And
    if one of your co-workers tests positive for the coronavirus, the
    C.D.C.} has said that
    \href{https://www.cdc.gov/coronavirus/2019-ncov/community/guidance-business-response.html}{employers
    should tell their employees} -\/- without giving you the sick
    employee's name -\/- that they may have been exposed to the virus.
  \end{itemize}
\item ~
  \hypertarget{what-is-school-going-to-look-like-in-september}{%
  \paragraph{What is school going to look like in
  September?}\label{what-is-school-going-to-look-like-in-september}}

  \begin{itemize}
  \tightlist
  \item
    It is unlikely that many schools will return to a normal schedule
    this fall, requiring the grind of
    \href{https://www.nytimes3xbfgragh.onion/2020/06/05/us/coronavirus-education-lost-learning.html?action=click\&pgtype=Article\&state=default\&region=MAIN_CONTENT_3\&context=storylines_faq}{online
    learning},
    \href{https://www.nytimes3xbfgragh.onion/2020/05/29/us/coronavirus-child-care-centers.html?action=click\&pgtype=Article\&state=default\&region=MAIN_CONTENT_3\&context=storylines_faq}{makeshift
    child care} and
    \href{https://www.nytimes3xbfgragh.onion/2020/06/03/business/economy/coronavirus-working-women.html?action=click\&pgtype=Article\&state=default\&region=MAIN_CONTENT_3\&context=storylines_faq}{stunted
    workdays} to continue. California's two largest public school
    districts --- Los Angeles and San Diego --- said on July 13, that
    \href{https://www.nytimes3xbfgragh.onion/2020/07/13/us/lausd-san-diego-school-reopening.html?action=click\&pgtype=Article\&state=default\&region=MAIN_CONTENT_3\&context=storylines_faq}{instruction
    will be remote-only in the fall}, citing concerns that surging
    coronavirus infections in their areas pose too dire a risk for
    students and teachers. Together, the two districts enroll some
    825,000 students. They are the largest in the country so far to
    abandon plans for even a partial physical return to classrooms when
    they reopen in August. For other districts, the solution won't be an
    all-or-nothing approach.
    \href{https://bioethics.jhu.edu/research-and-outreach/projects/eschool-initiative/school-policy-tracker/}{Many
    systems}, including the nation's largest, New York City, are
    devising
    \href{https://www.nytimes3xbfgragh.onion/2020/06/26/us/coronavirus-schools-reopen-fall.html?action=click\&pgtype=Article\&state=default\&region=MAIN_CONTENT_3\&context=storylines_faq}{hybrid
    plans} that involve spending some days in classrooms and other days
    online. There's no national policy on this yet, so check with your
    municipal school system regularly to see what is happening in your
    community.
  \end{itemize}
\end{itemize}

Being British-Sudanese in the 1980s was not easy. Race riots flared in
cities across the country. Mosques were scarce. Dr. el-Hawrani went to
school almost exclusively with white British classmates.

The young doctor quietly stood up for his family: When someone once
tried to kill a 100-year-old fern in their garden by cutting out a ring
of bark, Dr. el-Hawrani snapped off branches and nailed them across the
gap so that nutrients could get across.

Still, discrimination bothered him. When it came time to follow his
father into medicine, Dr. el-Hawrani told his brother that he ``wanted
to be an orthopedic surgeon but felt that maybe because of certain
prejudices he didn't get it.''

His resolve only grew stronger after an older brother, Ashraf, a fellow
doctor, died at 29 of causes related to asthma. Dr. el-Hawrani
discovered his brother's body.

Before Dr. el-Hawrani's death, on March 28, he had finally come around
to the idea that his only son, Ashraf, named in his brother's memory,
would study English instead of the family trade. Ashraf said in a
statement that his father ``was dedicated towards his family.''

``Now he has to make his decisions about which university to go to on
his own,'' Amal el-Hawrani said of Ashraf. ``He was expecting to have
his father's help.''

The coronavirus has taken a devastating toll on migrant doctors across
Britain, leaving at least six others dead: Dr. Habib Zaidi, 76, a
longtime general practitioner from Pakistan; Dr. Alfa Sa'adu, 68, a
geriatric doctor from Nigeria; Dr. Jitendra Rathod, 62, a heart surgeon
from India; Dr. Anton Sebastianpillai, in his 70s, a geriatric doctor
from Sri Lanka; Dr. Mohamed Sami Shousha, 79, a breast tissue specialist
from Egypt; and Dr. Syed Haider, in his 80s, a general practitioner from
Pakistan.

Barry Hudson, a longtime patient of Dr. Zaidi in southeastern England,
recalled their exam table conversations about England's cricket team.

``He was a big figure in the community,'' Mr. Hudson said. ``He had a
proper doctor's manner. He didn't rush anybody.''

Image

Dr. Habib ZaidiCredit...NHS Southend CCG

For families that love to gather, grieving at a distance has been
wrenching.

Dr. el-Tayar was buried beside his father and grandfather in Sudan, as
he had wanted. But because only cargo planes were flying there, his wife
and children could not accompany the coffin.

At Dr. el-Hawrani's burial, an imam said a prayer before a small,
spread-out crowd, and the doctor's four living brothers and son lowered
his coffin into the ground. Then they dispersed.

His brother, Amal el-Hawrani, permitted himself a single intimacy: a hug
with his mother, because ``I couldn't turn that away,'' he said.

Then she returned to her home in Bristol, along with a son who had
visited Dr. el-Hawrani in the hospital. Fearful of passing on the virus,
he had to forbid her from his room to keep her from bringing in food.

Advertisement

\protect\hyperlink{after-bottom}{Continue reading the main story}

\hypertarget{site-index}{%
\subsection{Site Index}\label{site-index}}

\hypertarget{site-information-navigation}{%
\subsection{Site Information
Navigation}\label{site-information-navigation}}

\begin{itemize}
\tightlist
\item
  \href{https://help.nytimes3xbfgragh.onion/hc/en-us/articles/115014792127-Copyright-notice}{©~2020~The
  New York Times Company}
\end{itemize}

\begin{itemize}
\tightlist
\item
  \href{https://www.nytco.com/}{NYTCo}
\item
  \href{https://help.nytimes3xbfgragh.onion/hc/en-us/articles/115015385887-Contact-Us}{Contact
  Us}
\item
  \href{https://www.nytco.com/careers/}{Work with us}
\item
  \href{https://nytmediakit.com/}{Advertise}
\item
  \href{http://www.tbrandstudio.com/}{T Brand Studio}
\item
  \href{https://www.nytimes3xbfgragh.onion/privacy/cookie-policy\#how-do-i-manage-trackers}{Your
  Ad Choices}
\item
  \href{https://www.nytimes3xbfgragh.onion/privacy}{Privacy}
\item
  \href{https://help.nytimes3xbfgragh.onion/hc/en-us/articles/115014893428-Terms-of-service}{Terms
  of Service}
\item
  \href{https://help.nytimes3xbfgragh.onion/hc/en-us/articles/115014893968-Terms-of-sale}{Terms
  of Sale}
\item
  \href{https://spiderbites.nytimes3xbfgragh.onion}{Site Map}
\item
  \href{https://help.nytimes3xbfgragh.onion/hc/en-us}{Help}
\item
  \href{https://www.nytimes3xbfgragh.onion/subscription?campaignId=37WXW}{Subscriptions}
\end{itemize}
