Sections

SEARCH

\protect\hyperlink{site-content}{Skip to
content}\protect\hyperlink{site-index}{Skip to site index}

\href{https://www.nytimes3xbfgragh.onion/section/science}{Science}

\href{https://myaccount.nytimes3xbfgragh.onion/auth/login?response_type=cookie\&client_id=vi}{}

\href{https://www.nytimes3xbfgragh.onion/section/todayspaper}{Today's
Paper}

\href{/section/science}{Science}\textbar{}Most New York Coronavirus
Cases Came From Europe, Genomes Show

\url{https://nyti.ms/2V7fBAg}

\begin{itemize}
\item
\item
\item
\item
\item
\item
\end{itemize}

\hypertarget{the-coronavirus-outbreak}{%
\subsubsection{\texorpdfstring{\href{https://www.nytimes3xbfgragh.onion/news-event/coronavirus?name=styln-coronavirus-national\&region=TOP_BANNER\&variant=undefined\&block=storyline_menu_recirc\&action=click\&pgtype=Article\&impression_id=8ad2df20-e391-11ea-bdea-b32ca169ebf5}{The
Coronavirus
Outbreak}}{The Coronavirus Outbreak}}\label{the-coronavirus-outbreak}}

\begin{itemize}
\tightlist
\item
  live\href{https://www.nytimes3xbfgragh.onion/2020/08/21/world/covid-19-coronavirus.html?name=styln-coronavirus-national\&region=TOP_BANNER\&variant=undefined\&block=storyline_menu_recirc\&action=click\&pgtype=Article\&impression_id=8ad2df21-e391-11ea-bdea-b32ca169ebf5}{Latest
  Updates}
\item
  \href{https://www.nytimes3xbfgragh.onion/interactive/2020/us/coronavirus-us-cases.html?name=styln-coronavirus-national\&region=TOP_BANNER\&variant=undefined\&block=storyline_menu_recirc\&action=click\&pgtype=Article\&impression_id=8ad30630-e391-11ea-bdea-b32ca169ebf5}{Maps
  and Cases}
\item
  \href{https://www.nytimes3xbfgragh.onion/interactive/2020/science/coronavirus-vaccine-tracker.html?name=styln-coronavirus-national\&region=TOP_BANNER\&variant=undefined\&block=storyline_menu_recirc\&action=click\&pgtype=Article\&impression_id=8ad30631-e391-11ea-bdea-b32ca169ebf5}{Vaccine
  Tracker}
\item
  \href{https://www.nytimes3xbfgragh.onion/2020/08/19/us/colleges-closing-covid.html?name=styln-coronavirus-national\&region=TOP_BANNER\&variant=undefined\&block=storyline_menu_recirc\&action=click\&pgtype=Article\&impression_id=8ad32d40-e391-11ea-bdea-b32ca169ebf5}{Colleges
  Closing}
\item
  \href{https://www.nytimes3xbfgragh.onion/live/2020/08/20/business/stock-market-today-coronavirus?name=styln-coronavirus-national\&region=TOP_BANNER\&variant=undefined\&block=storyline_menu_recirc\&action=click\&pgtype=Article\&impression_id=8ad32d41-e391-11ea-bdea-b32ca169ebf5}{Economy}
\end{itemize}

Advertisement

\protect\hyperlink{after-top}{Continue reading the main story}

Supported by

\protect\hyperlink{after-sponsor}{Continue reading the main story}

matter

\hypertarget{most-new-york-coronavirus-cases-came-from-europe-genomes-show}{%
\section{Most New York Coronavirus Cases Came From Europe, Genomes
Show}\label{most-new-york-coronavirus-cases-came-from-europe-genomes-show}}

Travelers seeded multiple cases starting as early as mid-February,
genomes show.

\includegraphics{https://static01.graylady3jvrrxbe.onion/images/2020/04/14/science/09VIRUS-MUTATIONS1/09VIRUS-MUTATIONS1-articleLarge.jpg?quality=75\&auto=webp\&disable=upscale}

\href{https://www.nytimes3xbfgragh.onion/by/carl-zimmer}{\includegraphics{https://static01.graylady3jvrrxbe.onion/images/2018/06/12/multimedia/author-carl-zimmer/author-carl-zimmer-thumbLarge.png}}

By \href{https://www.nytimes3xbfgragh.onion/by/carl-zimmer}{Carl Zimmer}

\begin{itemize}
\item
  April 8, 2020
\item
  \begin{itemize}
  \item
  \item
  \item
  \item
  \item
  \item
  \end{itemize}
\end{itemize}

New research indicates that the
\href{https://www.nytimes3xbfgragh.onion/2020/07/04/health/coronavirus-neanderthals.html}{coronavirus}
began to circulate in the New York area by mid-February, weeks before
the first confirmed case, and that travelers brought in the virus mainly
from Europe, not Asia.

``The majority is clearly European,'' said Harm van Bakel, a geneticist
at Icahn School of Medicine at Mount Sinai, who co-wrote a
\href{https://www.medrxiv.org/content/10.1101/2020.04.08.20056929v1}{study}
awaiting peer review.

A separate team at N.Y.U. Grossman School of Medicine came to strikingly
similar conclusions, despite studying a different group of cases. Both
teams analyzed genomes from coronaviruses taken from New Yorkers
starting in mid-March.

The research revealed a previously hidden spread of the virus that might
have been detected if aggressive testing programs had been put in place.

On Jan. 31, President Trump barred foreign nationals from entering the
country if they had been in China during the prior two weeks.

It would not be until late February that Italy would begin locking down
towns and cities, and March 11 when Mr. Trump said he would block
travelers from most European countries. But
\href{https://www.nytimes3xbfgragh.onion/2020/03/19/health/coronavirus-travel-ban.html}{New
Yorkers had already been traveling home with the virus}.

``People were just oblivious,'' said Adriana Heguy, a member of the
N.Y.U. team.

Dr. Heguy and Dr. van Bakel belong to an international guild of viral
historians. They ferret out the history of outbreaks by poring over
clues embedded in the genetic material of viruses taken from thousands
of patients.

Viruses invade a cell and take over its molecular machinery, causing it
to make new viruses.

The process is quick and sloppy. As a result, new viruses can gain a new
mutation that wasn't present in their ancestor. If a new virus manages
to escape its host and infect other people, its descendants will inherit
that mutation.

Tracking viral mutations demands sequencing all the genetic material in
a virus --- its genome. Once researchers have gathered the genomes from
a number of virus samples, they can compare their mutations.

Sophisticated computer programs can then figure out how all of those
mutations arose as viruses descended from a common ancestor. If they get
enough data, they can make rough estimates about how long ago those
ancestors lived. That's because mutations arise at a roughly regular
pace, like a molecular clock.

\includegraphics{https://static01.graylady3jvrrxbe.onion/images/2020/04/14/science/09VIRUS-MUTATIONS2/merlin_169222827_360ed589-a349-44ee-9f20-6480332ee4b4-articleLarge.jpg?quality=75\&auto=webp\&disable=upscale}

Maciej Boni of Penn State University and his colleagues recently used
this method to see where the coronavirus, designated SARS-CoV-2, came
from in the first place. While conspiracy theories might falsely claim
the virus was concocted in a lab, the virus's genome makes clear that it
\href{https://www.biorxiv.org/content/10.1101/2020.03.30.015008v1}{arose
in bats}.

There are many kinds of coronaviruses, which infect both humans and
animals. Dr. Boni and his colleagues found that the genome of the new
virus contains a number of mutations in common with strains of
coronaviruses that infect bats.

\hypertarget{latest-updates-the-coronavirus-outbreak}{%
\section{\texorpdfstring{\href{https://www.nytimes3xbfgragh.onion/2020/08/21/world/covid-19-coronavirus.html?action=click\&pgtype=Article\&state=default\&region=MAIN_CONTENT_1\&context=storylines_live_updates}{Latest
Updates: The Coronavirus
Outbreak}}{Latest Updates: The Coronavirus Outbreak}}\label{latest-updates-the-coronavirus-outbreak}}

Updated 2020-08-21T09:16:44.278Z

\begin{itemize}
\tightlist
\item
  \href{https://www.nytimes3xbfgragh.onion/2020/08/21/world/covid-19-coronavirus.html?action=click\&pgtype=Article\&state=default\&region=MAIN_CONTENT_1\&context=storylines_live_updates\#link-4690b6aa}{Shutdowns,
  warnings and scoldings follow gatherings on college campuses.}
\item
  \href{https://www.nytimes3xbfgragh.onion/2020/08/21/world/covid-19-coronavirus.html?action=click\&pgtype=Article\&state=default\&region=MAIN_CONTENT_1\&context=storylines_live_updates\#link-324af071}{As
  he accepts the Democratic nomination, Biden knocks Trump's pandemic
  response.}
\item
  \href{https://www.nytimes3xbfgragh.onion/2020/08/21/world/covid-19-coronavirus.html?action=click\&pgtype=Article\&state=default\&region=MAIN_CONTENT_1\&context=storylines_live_updates\#link-1c47e0d0}{Postal
  cost-cutting has slowed the delivery of medicine, exacerbating
  pandemic limitations for the chronically ill.}
\end{itemize}

\href{https://www.nytimes3xbfgragh.onion/2020/08/21/world/covid-19-coronavirus.html?action=click\&pgtype=Article\&state=default\&region=MAIN_CONTENT_1\&context=storylines_live_updates}{See
more updates}

More live coverage:
\href{https://www.nytimes3xbfgragh.onion/live/2020/08/20/business/stock-market-today-coronavirus?action=click\&pgtype=Article\&state=default\&region=MAIN_CONTENT_1\&context=storylines_live_updates}{Markets}

The most closely related coronavirus is in a Chinese horseshoe bat, the
researchers found. But the new virus has gained some unique mutations
since splitting off from that bat virus decades ago.

Dr. Boni said that ancestral virus probably gave rise to a number of
strains that infected horseshoe bats, and perhaps sometimes other
animals.

``Very likely there's a vast unsampled diversity,'' he said.

Copying mistakes aren't the only way for new viruses to arise. Sometimes
two kinds of coronaviruses will infect the same cell. Their genetic
material gets mixed up in new viruses.

It's entirely possible, Dr. Boni said, in the past 10 or 20 years, a
hybrid virus arose in some horseshoe bat that was well-suited to infect
humans, too. Later, that virus somehow managed to cross the species
barrier.

``Once in a while, one of these viruses wins the lottery,'' he said.

In January, a team of Chinese and Australian researchers published the
first genome of the new virus. Since then, researchers around the world
have sequenced over 3,000 more. Some are genetically identical to each
other, while others carry distinctive mutations.

That's just a tiny sampling of the full diversity of the virus. As of
April 8, there were 1.5 million
\href{https://gisanddata.maps.arcgis.com/apps/opsdashboard/index.html\#/bda7594740fd40299423467b48e9ecf6}{confirmed
cases} of Covid-19, and the true total is probably many millions more.
But already, the genomes of the virus are revealing previously hidden
outlines of its history over the past few months.

As new genomes come to light, researchers upload them to an online
database called \href{https://www.gisaid.org/}{GISAID}. A team of virus
evolution experts are analyzing the growing collection of genomes in a
project called \href{https://nextstrain.org/}{Nextstrain}. They
continually update the virus family tree.

The deepest branches of the tree all belong to lineages from China. The
Nextstrain team has also used the mutation rate to determine that the
virus probably first moved into humans from an animal host in late 2019.
On Dec. 31, China announced that doctors in Wuhan were treating dozens
of cases of a mysterious new respiratory illness.

Image

An apoptotic cell heavily infected with coronavirus, yellow.Credit...
National Institutes of Health/EPA, via Shutterstock

In January, as the scope of the catastrophe in China became clear, a few
countries started an aggressive testing program. They were able to track
the arrival of the virus on their territory and track its spread through
their populations.

But the United States fumbled in making its first diagnostic kits and
initially limited testing only to people who had come from China and
displayed symptoms of Covid-19.

``It was a disaster that we didn't do testing,'' Dr. Heguy said.

A few cases came to light starting at the end of January. But it was
easy to dismiss them as rare imports that did not lead to local
outbreaks.

The illusion was dashed at the end of February by Trevor Bedford, an
associate professor at the Fred Hutchinson Cancer Research Center and
the University of Washington, and his colleagues.

Using Nextstrain, they
\href{https://www.nytimes3xbfgragh.onion/2020/03/01/health/coronavirus-washington-spread.html}{showed}
that a virus identified in a patient in late February had mutation
shared by one identified in Washington on Jan. 20.

The Washington viruses also shared other mutations in common with ones
isolated in Wuhan, suggesting that a traveler had brought the
coronavirus from China.

With that discovery, Dr. Bedford and his colleagues took the lead in
sequencing coronavirus genomes. Sequencing more genomes around
Washington gave them a better view of how the outbreak there got
started.

\href{https://www.nytimes3xbfgragh.onion/news-event/coronavirus?action=click\&pgtype=Article\&state=default\&region=MAIN_CONTENT_3\&context=storylines_faq}{}

\hypertarget{the-coronavirus-outbreak-}{%
\subsubsection{The Coronavirus Outbreak
›}\label{the-coronavirus-outbreak-}}

\hypertarget{frequently-asked-questions}{%
\paragraph{Frequently Asked
Questions}\label{frequently-asked-questions}}

Updated August 17, 2020

\begin{itemize}
\item ~
  \hypertarget{why-does-standing-six-feet-away-from-others-help}{%
  \paragraph{Why does standing six feet away from others
  help?}\label{why-does-standing-six-feet-away-from-others-help}}

  \begin{itemize}
  \tightlist
  \item
    The coronavirus spreads primarily through droplets from your mouth
    and nose, especially when you cough or sneeze. The C.D.C., one of
    the organizations using that measure,
    \href{https://www.nytimes3xbfgragh.onion/2020/04/14/health/coronavirus-six-feet.html?action=click\&pgtype=Article\&state=default\&region=MAIN_CONTENT_3\&context=storylines_faq}{bases
    its recommendation of six feet} on the idea that most large droplets
    that people expel when they cough or sneeze will fall to the ground
    within six feet. But six feet has never been a magic number that
    guarantees complete protection. Sneezes, for instance, can launch
    droplets a lot farther than six feet,
    \href{https://jamanetwork.com/journals/jama/fullarticle/2763852}{according
    to a recent study}. It's a rule of thumb: You should be safest
    standing six feet apart outside, especially when it's windy. But
    keep a mask on at all times, even when you think you're far enough
    apart.
  \end{itemize}
\item ~
  \hypertarget{i-have-antibodies-am-i-now-immune}{%
  \paragraph{I have antibodies. Am I now
  immune?}\label{i-have-antibodies-am-i-now-immune}}

  \begin{itemize}
  \tightlist
  \item
    As of right
    now,\href{https://www.nytimes3xbfgragh.onion/2020/07/22/health/covid-antibodies-herd-immunity.html?action=click\&pgtype=Article\&state=default\&region=MAIN_CONTENT_3\&context=storylines_faq}{that
    seems likely, for at least several months.} There have been
    frightening accounts of people suffering what seems to be a second
    bout of Covid-19. But experts say these patients may have a
    drawn-out course of infection, with the virus taking a slow toll
    weeks to months after initial exposure. People infected with the
    coronavirus typically
    \href{https://www.nature.com/articles/s41586-020-2456-9}{produce}
    immune molecules called antibodies, which are
    \href{https://www.nytimes3xbfgragh.onion/2020/05/07/health/coronavirus-antibody-prevalence.html?action=click\&pgtype=Article\&state=default\&region=MAIN_CONTENT_3\&context=storylines_faq}{protective
    proteins made in response to an
    infection}\href{https://www.nytimes3xbfgragh.onion/2020/05/07/health/coronavirus-antibody-prevalence.html?action=click\&pgtype=Article\&state=default\&region=MAIN_CONTENT_3\&context=storylines_faq}{.
    These antibodies may} last in the body
    \href{https://www.nature.com/articles/s41591-020-0965-6}{only two to
    three months}, which may seem worrisome, but that's perfectly normal
    after an acute infection subsides, said Dr. Michael Mina, an
    immunologist at Harvard University. It may be possible to get the
    coronavirus again, but it's highly unlikely that it would be
    possible in a short window of time from initial infection or make
    people sicker the second time.
  \end{itemize}
\item ~
  \hypertarget{im-a-small-business-owner-can-i-get-relief}{%
  \paragraph{I'm a small-business owner. Can I get
  relief?}\label{im-a-small-business-owner-can-i-get-relief}}

  \begin{itemize}
  \tightlist
  \item
    The
    \href{https://www.nytimes3xbfgragh.onion/article/small-business-loans-stimulus-grants-freelancers-coronavirus.html?action=click\&pgtype=Article\&state=default\&region=MAIN_CONTENT_3\&context=storylines_faq}{stimulus
    bills enacted in March} offer help for the millions of American
    small businesses. Those eligible for aid are businesses and
    nonprofit organizations with fewer than 500 workers, including sole
    proprietorships, independent contractors and freelancers. Some
    larger companies in some industries are also eligible. The help
    being offered, which is being managed by the Small Business
    Administration, includes the Paycheck Protection Program and the
    Economic Injury Disaster Loan program. But lots of folks have
    \href{https://www.nytimes3xbfgragh.onion/interactive/2020/05/07/business/small-business-loans-coronavirus.html?action=click\&pgtype=Article\&state=default\&region=MAIN_CONTENT_3\&context=storylines_faq}{not
    yet seen payouts.} Even those who have received help are confused:
    The rules are draconian, and some are stuck sitting on
    \href{https://www.nytimes3xbfgragh.onion/2020/05/02/business/economy/loans-coronavirus-small-business.html?action=click\&pgtype=Article\&state=default\&region=MAIN_CONTENT_3\&context=storylines_faq}{money
    they don't know how to use.} Many small-business owners are getting
    less than they expected or
    \href{https://www.nytimes3xbfgragh.onion/2020/06/10/business/Small-business-loans-ppp.html?action=click\&pgtype=Article\&state=default\&region=MAIN_CONTENT_3\&context=storylines_faq}{not
    hearing anything at all.}
  \end{itemize}
\item ~
  \hypertarget{what-are-my-rights-if-i-am-worried-about-going-back-to-work}{%
  \paragraph{What are my rights if I am worried about going back to
  work?}\label{what-are-my-rights-if-i-am-worried-about-going-back-to-work}}

  \begin{itemize}
  \tightlist
  \item
    Employers have to provide
    \href{https://www.osha.gov/SLTC/covid-19/standards.html}{a safe
    workplace} with policies that protect everyone equally.
    \href{https://www.nytimes3xbfgragh.onion/article/coronavirus-money-unemployment.html?action=click\&pgtype=Article\&state=default\&region=MAIN_CONTENT_3\&context=storylines_faq}{And
    if one of your co-workers tests positive for the coronavirus, the
    C.D.C.} has said that
    \href{https://www.cdc.gov/coronavirus/2019-ncov/community/guidance-business-response.html}{employers
    should tell their employees} -\/- without giving you the sick
    employee's name -\/- that they may have been exposed to the virus.
  \end{itemize}
\item ~
  \hypertarget{what-is-school-going-to-look-like-in-september}{%
  \paragraph{What is school going to look like in
  September?}\label{what-is-school-going-to-look-like-in-september}}

  \begin{itemize}
  \tightlist
  \item
    It is unlikely that many schools will return to a normal schedule
    this fall, requiring the grind of
    \href{https://www.nytimes3xbfgragh.onion/2020/06/05/us/coronavirus-education-lost-learning.html?action=click\&pgtype=Article\&state=default\&region=MAIN_CONTENT_3\&context=storylines_faq}{online
    learning},
    \href{https://www.nytimes3xbfgragh.onion/2020/05/29/us/coronavirus-child-care-centers.html?action=click\&pgtype=Article\&state=default\&region=MAIN_CONTENT_3\&context=storylines_faq}{makeshift
    child care} and
    \href{https://www.nytimes3xbfgragh.onion/2020/06/03/business/economy/coronavirus-working-women.html?action=click\&pgtype=Article\&state=default\&region=MAIN_CONTENT_3\&context=storylines_faq}{stunted
    workdays} to continue. California's two largest public school
    districts --- Los Angeles and San Diego --- said on July 13, that
    \href{https://www.nytimes3xbfgragh.onion/2020/07/13/us/lausd-san-diego-school-reopening.html?action=click\&pgtype=Article\&state=default\&region=MAIN_CONTENT_3\&context=storylines_faq}{instruction
    will be remote-only in the fall}, citing concerns that surging
    coronavirus infections in their areas pose too dire a risk for
    students and teachers. Together, the two districts enroll some
    825,000 students. They are the largest in the country so far to
    abandon plans for even a partial physical return to classrooms when
    they reopen in August. For other districts, the solution won't be an
    all-or-nothing approach.
    \href{https://bioethics.jhu.edu/research-and-outreach/projects/eschool-initiative/school-policy-tracker/}{Many
    systems}, including the nation's largest, New York City, are
    devising
    \href{https://www.nytimes3xbfgragh.onion/2020/06/26/us/coronavirus-schools-reopen-fall.html?action=click\&pgtype=Article\&state=default\&region=MAIN_CONTENT_3\&context=storylines_faq}{hybrid
    plans} that involve spending some days in classrooms and other days
    online. There's no national policy on this yet, so check with your
    municipal school system regularly to see what is happening in your
    community.
  \end{itemize}
\end{itemize}

``I'm quite confident that it was not spreading in December in the
United States,'' Dr. Bedford said. ``There may have been a couple other
introductions in January that didn't take off in the same way.''

As new cases arose in other parts of the country, other researchers set
up their own pipelines. The first positive test result in New York came
on March 1, and after a couple of weeks, patients surged into the city's
hospitals.

``I thought, `We need to do this for New York,''' Dr. Heguy said.

Dr. Heguy and her colleagues found some New York viruses that shared
unique mutations not found elsewhere. ``That's when you know you've had
a silent transmission for a while,'' she said.

Dr. Heguy estimated that the virus began circulating in the New York
area a couple of months ago.

And researchers at Mount Sinai started sequencing the genomes of
patients coming through their hospital. They found that the earliest
cases identified in New York were not linked to later ones.

``Two weeks later, we start seeing viruses related to each other,'' said
Ana Silvia Gonzalez-Reiche, a member of the Mount Sinai team.

Dr. Gonzalez-Reiche and her colleagues found that these viruses were
practically identical to viruses found around Europe. They cannot say on
what particular flight a particular virus arrived in New York. But they
write that the viruses reveal ``a period of untracked global
transmission between late January to mid-February.''

So far, the Mount Sinai researchers have identified seven separate
lineages of viruses that entered New York and began circulating. ``We
will probably find more,'' Dr. van Bakel said.

The coronavirus genomes are also revealing hints of early cross-country
travel.

Dr. van Bakel and his colleagues found one New York virus that was
identical to one of the Washington viruses found by Dr. Bedford and his
colleagues. In a
\href{https://www.medrxiv.org/content/10.1101/2020.03.25.20043828v1}{separate
study,} researchers at Yale found another Washington-related virus.
Combined, the two studies hint that the coronavirus has been moving from
coast to coast for several weeks.

Image

Cell samples to be infected with coronavirus at The Icahn School of
Medicine at Mount Sinai in Manhattan.Credit...Victor J. Blue for The New
York Times

Sidney Bell, a computational biologist working with the Nextstrain team,
cautions people not to read too much into these new mutations
themselves. ``Just because something is different doesn't mean it
matters,'' Dr. Bell said.

Mutations do not automatically turn viruses into new, fearsome strains.
They often don't bring about any change at all. ``To me, mutations are
inevitable and kind of boring,'' Dr. Bell said. ``But in the movies, you
get the X-Men.''

Peter Thielen, a molecular biologist at the Johns Hopkins Applied
Physics Laboratory, likes to think of the spread of viruses like a
dandelion seed landing on an empty field.

The flower grows up and produces seeds of its own. Those seeds spread
and sprout. New mutations arise over the generations as the dandelions
fill the field. ``But they're all still dandelions,'' Mr. Thielen said.

While the coronavirus mutations are useful for telling lineages apart,
they don't have any apparent effect on how the virus works.

That's good news for scientists working on a vaccine.

Vaccine developers hope to fight Covid-19 by teaching our bodies to make
antibodies that can grab onto the virus and block its entry into cells.

Some viruses evolve so quickly that they require vaccines that can
produce several different antibodies. That's not the case for Covid-19.
Like other coronaviruses, it has a relatively slow mutation rate
compared to some viruses, like influenza.

As hard as the fight against it may be, its mutations reveal that things
can be a whole lot worse.

Of course, the coronavirus will continue to mutate as long as it still
infects people. It's possible that vaccines will have to change to keep
up with the virus. And that's why scientists need to keep tracking its
history.

\textbf{\emph{{[}}\href{http://on.fb.me/1paTQ1h}{\emph{Like the Science
Times page on Facebook.}}} ****** \emph{\textbar{} Sign up for the}
\textbf{\href{http://nyti.ms/1MbHaRU}{\emph{Science Times
newsletter.}}\emph{{]}}}

Advertisement

\protect\hyperlink{after-bottom}{Continue reading the main story}

\hypertarget{site-index}{%
\subsection{Site Index}\label{site-index}}

\hypertarget{site-information-navigation}{%
\subsection{Site Information
Navigation}\label{site-information-navigation}}

\begin{itemize}
\tightlist
\item
  \href{https://help.nytimes3xbfgragh.onion/hc/en-us/articles/115014792127-Copyright-notice}{©~2020~The
  New York Times Company}
\end{itemize}

\begin{itemize}
\tightlist
\item
  \href{https://www.nytco.com/}{NYTCo}
\item
  \href{https://help.nytimes3xbfgragh.onion/hc/en-us/articles/115015385887-Contact-Us}{Contact
  Us}
\item
  \href{https://www.nytco.com/careers/}{Work with us}
\item
  \href{https://nytmediakit.com/}{Advertise}
\item
  \href{http://www.tbrandstudio.com/}{T Brand Studio}
\item
  \href{https://www.nytimes3xbfgragh.onion/privacy/cookie-policy\#how-do-i-manage-trackers}{Your
  Ad Choices}
\item
  \href{https://www.nytimes3xbfgragh.onion/privacy}{Privacy}
\item
  \href{https://help.nytimes3xbfgragh.onion/hc/en-us/articles/115014893428-Terms-of-service}{Terms
  of Service}
\item
  \href{https://help.nytimes3xbfgragh.onion/hc/en-us/articles/115014893968-Terms-of-sale}{Terms
  of Sale}
\item
  \href{https://spiderbites.nytimes3xbfgragh.onion}{Site Map}
\item
  \href{https://help.nytimes3xbfgragh.onion/hc/en-us}{Help}
\item
  \href{https://www.nytimes3xbfgragh.onion/subscription?campaignId=37WXW}{Subscriptions}
\end{itemize}
