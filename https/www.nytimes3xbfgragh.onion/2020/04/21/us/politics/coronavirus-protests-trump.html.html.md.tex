Sections

SEARCH

\protect\hyperlink{site-content}{Skip to
content}\protect\hyperlink{site-index}{Skip to site index}

\href{https://www.nytimes3xbfgragh.onion/section/politics}{Politics}

\href{https://myaccount.nytimes3xbfgragh.onion/auth/login?response_type=cookie\&client_id=vi}{}

\href{https://www.nytimes3xbfgragh.onion/section/todayspaper}{Today's
Paper}

\href{/section/politics}{Politics}\textbar{}The Quiet Hand of
Conservative Groups in the Anti-Lockdown Protests

\url{https://nyti.ms/3eFiGRd}

\begin{itemize}
\item
\item
\item
\item
\item
\end{itemize}

\begin{itemize}
\item
  \href{https://www.nytimes3xbfgragh.onion/2020/07/31/us/elections/biden-vs-trump.html?action=click\&pgtype=Article\&state=default\&region=TOP_BANNER\&context=storylines_menu}{Election
  Updates}
\item
  \href{https://www.nytimes3xbfgragh.onion/article/biden-vice-president-2020.html?action=click\&pgtype=Article\&state=default\&region=TOP_BANNER\&context=storylines_menu}{Biden's
  V.P. Search}
\item
  \href{https://www.nytimes3xbfgragh.onion/interactive/2020/07/24/us/politics/trump-biden-campaign-donors.html?action=click\&pgtype=Article\&state=default\&region=TOP_BANNER\&context=storylines_menu}{Map
  of Donations}
\item
  \href{https://www.nytimes3xbfgragh.onion/interactive/2020/us/elections/delegate-count-primary-results.html?action=click\&pgtype=Article\&state=default\&region=TOP_BANNER\&context=storylines_menu}{Delegate
  Count}
\item
  \href{https://www.nytimes3xbfgragh.onion/interactive/2019/us/politics/2020-presidential-candidates.html?action=click\&pgtype=Article\&state=default\&region=TOP_BANNER\&context=storylines_menu}{The
  Candidates}
\item
  \href{https://www.nytimes3xbfgragh.onion/newsletters/politics?action=click\&pgtype=Article\&state=default\&region=TOP_BANNER\&context=storylines_menu}{Politics
  Newsletter}
\end{itemize}

Advertisement

\protect\hyperlink{after-top}{Continue reading the main story}

Supported by

\protect\hyperlink{after-sponsor}{Continue reading the main story}

\hypertarget{the-quiet-hand-of-conservative-groups-in-the-anti-lockdown-protests}{%
\section{The Quiet Hand of Conservative Groups in the Anti-Lockdown
Protests}\label{the-quiet-hand-of-conservative-groups-in-the-anti-lockdown-protests}}

Groups in a loose coalition have tapped their networks to drive up
turnout at recent rallies in state capitals and financed lawsuits,
polling and research to combat the stay-at-home orders.

\includegraphics{https://static01.graylady3jvrrxbe.onion/images/2020/04/21/us/politics/21protests-politics/merlin_171732069_a17ff660-2d19-45e9-9d99-4e28417d9ae4-articleLarge.jpg?quality=75\&auto=webp\&disable=upscale}

\href{https://www.nytimes3xbfgragh.onion/by/kenneth-p-vogel}{\includegraphics{https://static01.graylady3jvrrxbe.onion/images/2018/02/20/multimedia/author-kenneth-p-vogel/author-kenneth-p-vogel-thumbLarge-v3.png}}\href{https://www.nytimes3xbfgragh.onion/by/jim-rutenberg}{\includegraphics{https://static01.graylady3jvrrxbe.onion/images/2018/02/20/multimedia/author-jim-rutenberg/author-jim-rutenberg-thumbLarge.jpg}}\href{https://www.nytimes3xbfgragh.onion/by/lisa-lerer}{\includegraphics{https://static01.graylady3jvrrxbe.onion/images/2018/09/11/us/politics/author-lisa-lerer/lisa-lerer-headshot-thumbLarge.png}}

By \href{https://www.nytimes3xbfgragh.onion/by/kenneth-p-vogel}{Kenneth
P. Vogel},
\href{https://www.nytimes3xbfgragh.onion/by/jim-rutenberg}{Jim
Rutenberg} and
\href{https://www.nytimes3xbfgragh.onion/by/lisa-lerer}{Lisa Lerer}

\begin{itemize}
\item
  April 21, 2020
\item
  \begin{itemize}
  \item
  \item
  \item
  \item
  \item
  \end{itemize}
\end{itemize}

WASHINGTON --- An informal coalition of influential conservative leaders
and groups, some with close connections to the White House, has been
quietly working to nurture protests and apply political and legal
pressure to overturn state and local orders intended to stop the spread
of the coronavirus.

The groups have tapped their networks to drive up turnout at
\href{https://www.nytimes3xbfgragh.onion/2020/04/18/us/texas-protests-stay-at-home.html}{recent
rallies} in state capitals, dispatched their lawyers to file lawsuits,
and paid for polling and research to undercut the arguments behind
restrictions that have closed businesses and
\href{https://www.nytimes3xbfgragh.onion/interactive/2020/us/coronavirus-stay-at-home-order.html}{limited
the movement of most Americans}.

Among those fighting the orders are FreedomWorks and Tea Party Patriots,
which played pivotal roles in the beginning of Tea Party protests
starting more than a decade ago. Also involved are a law firm led partly
by former Trump White House officials, a network of state-based
conservative policy groups, and an ad hoc coalition of conservative
leaders known as Save Our Country that has advised the White House on
strategies for a tiered reopening of the economy.

The effort picked up some influential support on Tuesday, when Attorney
General William P. Barr
\href{https://www.hughhewitt.com/attorney-general-william-barr-on-the-crisis/}{expressed
concerns} about state-level restrictions potentially infringing on
constitutional rights, and suggested that, if that occurred, the Justice
Department might weigh in, including by supporting legal challenges by
others. Separately, in Wisconsin, Republicans in the state legislature
sued to block the Democratic governor's order extending stay-at-home
rules through May 26.

Those helping orchestrate the fight against restrictions predict the
effort could energize the right in the same way the Tea Party movement
did in 2009 and 2010, and potentially be helpful to President Trump as
he campaigns for re-election. But the cause has yet to demonstrate that
kind of traction.

\href{https://www.nbcnews.com/politics/meet-the-press/poll-six-10-support-keeping-stay-home-restrictions-fight-coronavirus-n1187011}{Polls
show} a majority of Americans are more concerned about reopening the
country too quickly than they are about the damage to the economy. And
coronavirus protests have drawn smaller crowds ranging from a few dozen
to several thousand at a rally in Michigan last week.

Conditions are hardly ideal for a protest movement related to the virus.
In addition to the health risks, demonstrators potentially
\href{https://www.nj.com/coronavirus/2020/04/woman-charged-for-organizing-protest-of-murphys-coronavirus-stay-at-home-order.html}{face
legal exposure} for violating the very measures they are protesting.
Plus, some key
\href{https://www.nytimes3xbfgragh.onion/2020/04/03/us/politics/maryland-coronavirus.html}{Republican
leaders} have
\href{https://www.nytimes3xbfgragh.onion/2020/03/16/us/politics/virus-primary-2020-ohio.html}{embraced
the types of restrictions} being targeted, while powerful grass-roots
mobilizing groups, including those spearheaded by the billionaire
activist Charles Koch, have so far not embraced the protests.

Still, the fight has emerged as a galvanizing cause for a vocal element
of Mr. Trump's base and others on the political right. Organizers see it
as unifying social conservatives, who view the orders as targeting
religious groups; fiscal conservatives who chafe at the economic
devastation wrought by the restrictions on businesses; and civil
libertarians who contend that the restrictions infringe on
constitutional rights.

``Groups are united in purpose on this,'' said Noah Wall, advocacy
director for FreedomWorks, which in 2009 organized a Tea Party protest
that
drew\href{https://www.politico.com/story/2009/09/a-march-but-is-it-a-movement-027058}{tens
of thousands of people} or
\href{https://www.latimes.com/archives/la-xpm-2009-sep-15-na-crowd15-story.html}{more}
to Washington. He described the current efforts as appealing to a ``much
broader'' group. ``This is about people who want to get back to work and
leave their homes,'' he said.

More than 10 protests are planned for this week, Mr. Wall said, adding
that elected officials ``are going to see a lot of angry activists, and
I think that could change minds.''

\includegraphics{https://static01.graylady3jvrrxbe.onion/images/2020/04/21/us/politics/21protests-politics2/merlin_171758313_38e33560-881a-464d-93ea-aaeac64b95ca-articleLarge.jpg?quality=75\&auto=webp\&disable=upscale}

The protests mostly appear to have been organized by local residents,
and are framed primarily as pushback against what they view as
government overreach. But some rallies have prominently featured
iconography boosting Mr. Trump and Republicans and denouncing Democrats,
as well the occasional Confederate flag and signs promoting conspiracy
theories.

As was the case with the Tea Party movement, established national groups
that generally align with the Republican Party have sought to fuel the
protests, harnessing their energy in a manner that can increase their
profiles and build their membership base and donor rolls.

\hypertarget{latest-updates-2020-election}{%
\section{\texorpdfstring{\href{https://www.nytimes3xbfgragh.onion/2020/07/31/us/elections/biden-vs-trump.html?action=click\&pgtype=Article\&state=default\&region=MAIN_CONTENT_1\&context=storylines_live_updates}{Latest
Updates: 2020
Election}}{Latest Updates: 2020 Election}}\label{latest-updates-2020-election}}

Updated 2020-08-01T01:26:45.732Z

\begin{itemize}
\tightlist
\item
  \href{https://www.nytimes3xbfgragh.onion/2020/07/31/us/elections/biden-vs-trump.html?action=click\&pgtype=Article\&state=default\&region=MAIN_CONTENT_1\&context=storylines_live_updates\#link-29fdff45}{Kamala
  Harris, a top vice-presidential contender, confronts double
  standards.}
\item
  \href{https://www.nytimes3xbfgragh.onion/2020/07/31/us/elections/biden-vs-trump.html?action=click\&pgtype=Article\&state=default\&region=MAIN_CONTENT_1\&context=storylines_live_updates\#link-13ec3d9c}{Karen
  Bass and Susan Rice are rising on Biden's vice-presidential
  shortlist.}
\item
  \href{https://www.nytimes3xbfgragh.onion/2020/07/31/us/elections/biden-vs-trump.html?action=click\&pgtype=Article\&state=default\&region=MAIN_CONTENT_1\&context=storylines_live_updates\#link-49e9a016}{Trump
  says Russian bounties to kill U.S. troops `never took place.'}
\end{itemize}

\href{https://www.nytimes3xbfgragh.onion/2020/07/31/us/elections/biden-vs-trump.html?action=click\&pgtype=Article\&state=default\&region=MAIN_CONTENT_1\&context=storylines_live_updates}{See
more updates}

Nonprofit groups including FreedomWorks and Tea Party Patriots have used
their social media accounts and text and email lists to spread the word
about protests across the country.

Most of FreedomWorks's 40 employees are working remotely on the effort,
helping to connect local protesters and set up websites for them. The
group is considering paid digital advertising to further increase
turnout, and has been conducting weekly tracking polls in swing suburban
districts that it says show support for reopening parts of country. It
is sharing the data with advisers on the president's economic task force
and other conservative allies on Capitol Hill.

While social media has been a key platform for organizing the protests,
those efforts have drawn scrutiny. Facebook
\href{https://www.cnn.com/2020/04/20/politics/facebook-covid-shutdown-protests/index.html}{removed
some posts} devoted to the protests on Monday for encouraging violations
of social distancing laws. And similarities in online organizing efforts
behind different protests have sparked accusations that they are not, in
fact, organic grass-roots campaigns, but ``astroturfing'' efforts that
are manipulated by Washington conservatives to appear locally driven.

Organizers of recent protests in Oklahoma acknowledged that FreedomWorks
helped arrange the events and said they hoped the ``rolling protests,''
which were intended to keep people in their vehicles, helped Mr. Trump
politically. But they stressed that the events reflected real concerns
from real people about the economic damage inflicted by mitigation
measures.

Carol Hefner, an Oklahoma co-chair of Mr. Trump's 2016 campaign who
helped organize a protest last week in Oklahoma City, cited the state's
flat terrain as a factor in any decision to ease restrictions. ``We have
a lot of wind and the wind has pretty much helped us here,'' she said.
``We are in a much better position than many of the other states to go
ahead and open back up.''

Ronda Vuillemont-Smith, an Oklahoma HVAC contractor who helped with the
capital rally and another one on Monday in Tulsa, said she encouraged
protesters to remain in their vehicles. But Ms. Vuillemont-Smith, who
serves on FreedomWorks's activist advisory council, added, ``I see
absolutely no risks whatsoever'' for open-air protests. ``We are adults.
We assume personal responsibility for the decisions that we make,'' she
said.

The Oklahoma organizers and Mr. Wall, as well as the White House and the
Trump campaign, said there was no coordination between the protests and
Mr. Trump's team.

Image

Ronda Vuillemont-Smith, who helped with two protests in Oklahoma, said
she encouraged protesters to remain in their vehicles.Credit...Matt
Barnard/Tulsa World, via Associated Press

But the protests coincide with messages from Mr. Trump, and have been
helped and organized by his supporters, some of whom have begun new
ventures to advance the cause.

One of them is Reopen America Political Action Committee, which aims to
bring small business owners to Washington to lobby lawmakers to reopen,
starting with a 24-hour rally at the White House on May 1 --- the target
Mr. Trump set for reopening.

The group, which was
\href{https://docquery.fec.gov/cgi-bin/forms/C00742890/1402019/}{created
this month}, has yet to
\href{https://docquery.fec.gov/pdf/091/202004159219312091/202004159219312091.pdf}{report
any financial activity}. But its founder, Suzzanne Monk, who is active
on Twitter with the handle
\href{https://twitter.com/Trumpertarian}{@Trumpertarian}, called the
idea for the rally ``pushback against these governors who want to stay
shut down far beyond their economic capacity to do so.''

Support for the protests features more direct ties to the White House
than simply support for Mr. Trump. The administration recently formed an
advisory group for reopening the economy that included Stephen Moore,
the conservative economics commentator. Mr. Moore had been coordinating
with FreedomWorks, the Tea Party Patriots and the American Legislative
Exchange Council in a coalition called
``\href{https://www.washingtonpost.com/business/2020/04/13/trump-reopen-economy-conservative-groups-coronavirus/}{Save
Our Country},'' which was formed to push for a quicker easing of
restrictions.

At the same time, Mr. Moore was communicating with a group of local
activists in Wisconsin involved in organizing a protest at the State
Capitol set for Friday. On a conservative
\href{https://www.youtube.com/watch?v=2h7czFYx6-s\&feature=youtu.be\&t=856}{YouTube
program} that went online the day Mr. Trump named him to the task force,
Mr. Moore said he had ``one big donor in Wisconsin'' who had pledged
financial support for the protesters, telling him, ```Steve, I promise,
I will pay the bail and legal fees of anyone who gets arrested.'''

In an interview with The New York Times, Mr. Moore declined to identify
the donor, but said, ``I do think you're going to see these start to
erupt.''

He said he would probably turn down an invitation to speak at the
protest in Wisconsin, because ``it's important that no one be under the
impression that it's sponsored or directed by national groups in
Washington.''

A legal offensive against the restrictions is also being waged by groups
and individuals supportive of Mr. Trump.

Mr. Barr's comments on Tuesday came a few days after
\href{https://www.documentcloud.org/documents/6842227-Trump-Allies-Call-On-DOJ-To-Block-State.html}{a
letter} sent by groups including FreedomWorks, Tea Party Patriots and
the anti-abortion-rights group Susan B. Anthony List urging the Justice
Department to consider intervening to block restrictions that the
officials said were unconstitutional infringements on civil liberties.

Lawyers aligned with socially conservative causes have filed their own
lawsuits against governors.

Many are focused on allowing smaller churches to keep holding services,
but the objections cover a range of other activities. In Michigan, a
lawsuit is challenging provisions of Gov. Gretchen Whitmer's executive
order banning travel to vacation homes and gatherings of non-household
members.

Image

People protested from their cars outside California's Capitol building
in Sacramento on Monday.Credit...Josh Edelson/Agence France-Presse ---
Getty Images

A law firm
\href{https://www.cnn.com/2019/01/11/politics/trump-organization-oversight-passantino/index.html}{that
advises} the Trump Organization, Michael Best \& Friedrich, is
representing members of a new protest group in North Carolina called
ReOpenNC. Michael Best's ranks include the former Trump chief of staff
Reince Priebus, the former deputy White House counsel Stefan C.
Passantino and the current senior counsel at the Trump campaign,
\href{https://www.politico.com/story/2019/06/03/playbook-birthday-justin-clark-1351103}{Justin
Clark}.

ReOpenNC had told its members that a ``generous donor'' had arranged to
pay for buses to bring protesters to Raleigh from around the state. But,
in a sign of how loath the groups are to be viewed as ``astroturf''
creations, the group said it had scrapped the plan when a news station,
\href{https://www.wral.com/coronavirus/reopennc-rally-planned-for-tuesday-in-downtown-raleigh/19062732/}{WRAL},
asked about it. (Afterward, a former defense contractor and perennial
North Carolina political candidate,
\href{https://www.charlotteobserver.com/news/politics-government/article9100739.html}{Tim
D'Annunzio},
\href{https://twitter.com/search?q=Tim\%20D\%27Annunzio\&src=typed_query}{stepped
forward} on Facebook to say he was the donor and was still hoping to run
the buses.)

On Friday, Anthony J. Biller, a Raleigh-based lawyer with Michael Best,
\href{https://wwwcache.wral.com/asset/news/state/nccapitol/2020/04/19/19062754/Biller_letter_to_Governor_Cooper_and_Comm_Ford_4_18-DMID1-5mk0it4ph.pdf}{wrote}
to Gov. Roy Cooper, a Democrat, on behalf of a ReOpenNC co-founder,
Kristen Elizabeth, and a member who was arrested at a protest last week,
seeking dismissal of the charges. In an interview, Mr. Biller said he
hoped the state would agree to allow ReOpenNC to demonstrate safely
without fear of arrest, adding, ``What is sufficient safety to buy
toilet paper at Costco should be sufficient safety to practice one's
fundamental rights, particularly about these issues.''

He said that he was working pro bono but that there was ``no
coordination with the Trump administration, as some bozos have
implied.''

One force in conservative politics that has kept its distance from the
stay-at-home protests is the network of groups backed by the billionaire
Mr. Koch. The largest Koch-backed group, Americans for Prosperity, which
played a leading role in facilitating the Tea Party movement, has
remained on the sidelines of the coronavirus protests.

GoDaddy records show that a public relations firm tied to the Koch
network,
\href{https://www.politico.com/story/2016/12/koch-brothers-announce-layoffs-232341}{In
Pursuit Of LLC}, registered the domain name ``reopenmississippi.com.''
An official said the group had planned to use the site to highlight a
nuanced approach being developed by the network to reopen the economy
while balancing health concerns.

``The question is --- what is the best way to get people back to work?''
said Emily Seidel, the chief executive of Americans for Prosperity. ``We
don't see protests as the best way to do that,'' she said, adding that
``the choice between full shutdown and immediately opening everything is
a false choice.''

Reid J. Epstein contributed reporting.

\hypertarget{our-2020-election-guide}{%
\section{Our 2020 Election Guide}\label{our-2020-election-guide}}

Updated July 31, 2020

\begin{itemize}
\item
  \begin{center}\rule{0.5\linewidth}{\linethickness}\end{center}

  \hypertarget{the-latest}{%
  \subsection{The Latest}\label{the-latest}}

  \begin{itemize}
  \tightlist
  \item
    President Trump's assault on the Postal Service is intersecting with
    his attacks on mail-in voting.
    \href{https://www.nytimes3xbfgragh.onion/2020/07/31/us/politics/trump-usps-mail-delays.html?action=click\&pgtype=Article\&state=default\&region=BELOW_MAIN_CONTENT\&context=storylines_guide}{Voting
    rights groups say it is a recipe for disaster.}
  \end{itemize}
\item
  \begin{center}\rule{0.5\linewidth}{\linethickness}\end{center}

  \hypertarget{bidens-vp-search}{%
  \subsection{Biden's V.P. Search}\label{bidens-vp-search}}

  \begin{itemize}
  \tightlist
  \item
    \href{https://www.nytimes3xbfgragh.onion/article/biden-vice-president-2020.html?action=click\&pgtype=Article\&state=default\&region=BELOW_MAIN_CONTENT\&context=storylines_guide}{Here
    are 13 women} who have been under consideration to be Joe Biden's
    running mate, and why each might be chosen --- and might not be.
  \end{itemize}
\item
  \begin{center}\rule{0.5\linewidth}{\linethickness}\end{center}

  \hypertarget{keep-up-with-our-coverage}{%
  \subsection{Keep Up With Our
  Coverage}\label{keep-up-with-our-coverage}}

  \begin{itemize}
  \tightlist
  \item
    Get an
    \href{https://www.nytimes3xbfgragh.onion/newsletters/politics?action=click\&pgtype=Article\&state=default\&region=BELOW_MAIN_CONTENT\&context=storylines_guide}{email}
    recapping the day's news
  \end{itemize}

  \begin{itemize}
  \tightlist
  \item
    Download our mobile app on
    \href{https://apps.apple.com/us/app/nytimes/id284862083?ls=1\&mat_click_id=5c79ae7455014fd1bd66b5610c05b8f2-20191112-16948\&referrer=mat_click_id\%3D5c79ae7455014fd1bd66b5610c05b8f2-20191112-16948\%26link_click_id\%3D722930677036718082}{iOS}
    and
    \href{http://a.localytics.com/android?id=com.nytimes.android\&referrer=utm_source\%3Dother_nyt_mobile_web\%26utm_medium\%3DWeb\%2520page\%26utm_term\%3DGeneral\%2520Mobile\%2520Page\%26utm_campaign\%3DNYT\%2520Mobile\%2520General\%2520Page}{Android}
    and turn on Breaking News and Politics alerts
  \end{itemize}
\end{itemize}

Advertisement

\protect\hyperlink{after-bottom}{Continue reading the main story}

\hypertarget{site-index}{%
\subsection{Site Index}\label{site-index}}

\hypertarget{site-information-navigation}{%
\subsection{Site Information
Navigation}\label{site-information-navigation}}

\begin{itemize}
\tightlist
\item
  \href{https://help.nytimes3xbfgragh.onion/hc/en-us/articles/115014792127-Copyright-notice}{©~2020~The
  New York Times Company}
\end{itemize}

\begin{itemize}
\tightlist
\item
  \href{https://www.nytco.com/}{NYTCo}
\item
  \href{https://help.nytimes3xbfgragh.onion/hc/en-us/articles/115015385887-Contact-Us}{Contact
  Us}
\item
  \href{https://www.nytco.com/careers/}{Work with us}
\item
  \href{https://nytmediakit.com/}{Advertise}
\item
  \href{http://www.tbrandstudio.com/}{T Brand Studio}
\item
  \href{https://www.nytimes3xbfgragh.onion/privacy/cookie-policy\#how-do-i-manage-trackers}{Your
  Ad Choices}
\item
  \href{https://www.nytimes3xbfgragh.onion/privacy}{Privacy}
\item
  \href{https://help.nytimes3xbfgragh.onion/hc/en-us/articles/115014893428-Terms-of-service}{Terms
  of Service}
\item
  \href{https://help.nytimes3xbfgragh.onion/hc/en-us/articles/115014893968-Terms-of-sale}{Terms
  of Sale}
\item
  \href{https://spiderbites.nytimes3xbfgragh.onion}{Site Map}
\item
  \href{https://help.nytimes3xbfgragh.onion/hc/en-us}{Help}
\item
  \href{https://www.nytimes3xbfgragh.onion/subscription?campaignId=37WXW}{Subscriptions}
\end{itemize}
