Sections

SEARCH

\protect\hyperlink{site-content}{Skip to
content}\protect\hyperlink{site-index}{Skip to site index}

\href{https://www.nytimes3xbfgragh.onion/section/health}{Health}

\href{https://myaccount.nytimes3xbfgragh.onion/auth/login?response_type=cookie\&client_id=vi}{}

\href{https://www.nytimes3xbfgragh.onion/section/todayspaper}{Today's
Paper}

\href{/section/health}{Health}\textbar{}Remdesivir Shows Modest Benefits
in Coronavirus Trial

\url{https://nyti.ms/2YnIECS}

\begin{itemize}
\item
\item
\item
\item
\item
\item
\end{itemize}

\hypertarget{the-coronavirus-outbreak}{%
\subsubsection{\texorpdfstring{\href{https://www.nytimes3xbfgragh.onion/news-event/coronavirus?name=styln-coronavirus-national\&region=TOP_BANNER\&variant=undefined\&block=storyline_menu_recirc\&action=click\&pgtype=Article\&impression_id=e8659210-e39e-11ea-af8e-ffa98621719b}{The
Coronavirus
Outbreak}}{The Coronavirus Outbreak}}\label{the-coronavirus-outbreak}}

\begin{itemize}
\tightlist
\item
  live\href{https://www.nytimes3xbfgragh.onion/2020/08/21/world/covid-19-coronavirus.html?name=styln-coronavirus-national\&region=TOP_BANNER\&variant=undefined\&block=storyline_menu_recirc\&action=click\&pgtype=Article\&impression_id=e865b920-e39e-11ea-af8e-ffa98621719b}{Latest
  Updates}
\item
  \href{https://www.nytimes3xbfgragh.onion/interactive/2020/us/coronavirus-us-cases.html?name=styln-coronavirus-national\&region=TOP_BANNER\&variant=undefined\&block=storyline_menu_recirc\&action=click\&pgtype=Article\&impression_id=e865b921-e39e-11ea-af8e-ffa98621719b}{Maps
  and Cases}
\item
  \href{https://www.nytimes3xbfgragh.onion/interactive/2020/science/coronavirus-vaccine-tracker.html?name=styln-coronavirus-national\&region=TOP_BANNER\&variant=undefined\&block=storyline_menu_recirc\&action=click\&pgtype=Article\&impression_id=e865b922-e39e-11ea-af8e-ffa98621719b}{Vaccine
  Tracker}
\item
  \href{https://www.nytimes3xbfgragh.onion/2020/08/19/us/colleges-closing-covid.html?name=styln-coronavirus-national\&region=TOP_BANNER\&variant=undefined\&block=storyline_menu_recirc\&action=click\&pgtype=Article\&impression_id=e865b923-e39e-11ea-af8e-ffa98621719b}{Colleges
  Closing}
\item
  \href{https://www.nytimes3xbfgragh.onion/live/2020/08/20/business/stock-market-today-coronavirus?name=styln-coronavirus-national\&region=TOP_BANNER\&variant=undefined\&block=storyline_menu_recirc\&action=click\&pgtype=Article\&impression_id=e865b924-e39e-11ea-af8e-ffa98621719b}{Economy}
\end{itemize}

Advertisement

\protect\hyperlink{after-top}{Continue reading the main story}

Supported by

\protect\hyperlink{after-sponsor}{Continue reading the main story}

\hypertarget{remdesivir-shows-modest-benefits-in-coronavirus-trial}{%
\section{Remdesivir Shows Modest Benefits in Coronavirus
Trial}\label{remdesivir-shows-modest-benefits-in-coronavirus-trial}}

Hope soared nonetheless. The F.D.A. is likely to issue an emergency
approval, a senior official said.

\includegraphics{https://static01.graylady3jvrrxbe.onion/images/2020/04/29/science/29VIRUS-REMDESIVIR/merlin_171898386_366e4d82-f256-44ea-9eeb-c80fba9c7e20-articleLarge.jpg?quality=75\&auto=webp\&disable=upscale}

\href{https://www.nytimes3xbfgragh.onion/by/gina-kolata}{\includegraphics{https://static01.graylady3jvrrxbe.onion/images/2018/02/16/multimedia/author-gina-kolata/author-gina-kolata-thumbLarge.jpg}}\href{https://www.nytimes3xbfgragh.onion/by/peter-baker}{\includegraphics{https://static01.graylady3jvrrxbe.onion/images/2018/06/13/multimedia/peter-baker/peter-baker-thumbLarge-v2.png}}\href{https://www.nytimes3xbfgragh.onion/by/noah-weiland}{\includegraphics{https://static01.graylady3jvrrxbe.onion/images/2019/07/23/reader-center/author-noah-weiland/author-noah-weiland-thumbLarge.png}}

By \href{https://www.nytimes3xbfgragh.onion/by/gina-kolata}{Gina
Kolata}, \href{https://www.nytimes3xbfgragh.onion/by/peter-baker}{Peter
Baker} and
\href{https://www.nytimes3xbfgragh.onion/by/noah-weiland}{Noah Weiland}

\begin{itemize}
\item
  April 29, 2020
\item
  \begin{itemize}
  \item
  \item
  \item
  \item
  \item
  \item
  \end{itemize}
\end{itemize}

Modest results from a federal trial of an experimental drug helped send
the stock market soaring on Wednesday, another sign of the desperation
for a viable treatment against the coronavirus.

Just before markets opened, Gilead, maker of the antiviral drug
remdesivir, announced that it was ``aware of positive data'' about the
drug's performance in a federal trial, sending futures upward. Trading
in the company's shares was briefly halted.

Later, in a briefing at the White House, Dr. Anthony S. Fauci, director
of the National Institute of Allergy and Infectious Diseases, said the
trial had shown that treatment with the drug could modestly speed
recovery in patients infected with the coronavirus.

The improvement in recovery times ``doesn't seem like a knockout 100
percent,'' Dr. Fauci conceded, but ``it is a very important proof of
concept, because what it has proven is that a drug can block this
virus.''

Sitting at Dr. Fauci's side, President Trump said, ``Certainly it's
positive, it's a very positive event.'' In past weeks, he has repeatedly
hailed remdesivir as a potential ``game changer,'' despite spotty
evidence.

Business leaders, scientists and politicians alike are scrambling to
find ways to fight an insidious epidemic and to reopen a devastated
economy. The virus has claimed at least 60,000 lives in the United
States, and more than 200,000 worldwide. There have been precious few
reasons for optimism, and the markets seized on the news.

The trial sponsored by the National Institute of Allergy and Infectious
Diseases enrolled 1,063 patients who were given remdesivir or a placebo.
The time to recovery averaged 11 days among those who got the drug,
compared with 15 days for those who got the placebo.

There were fewer deaths in the remdesivir group, but the result did not
reach statistical significance, Dr. Fauci said. Deaths were not a
primary measure in the trial.

Dr. Fauci cautioned that the results of the study still needed to be
properly peer-reviewed, but he was optimistic that remdesivir would
become ``the standard of care'' for patients with Covid-19.

\hypertarget{latest-updates-the-coronavirus-outbreak}{%
\section{\texorpdfstring{\href{https://www.nytimes3xbfgragh.onion/2020/08/21/world/covid-19-coronavirus.html?action=click\&pgtype=Article\&state=default\&region=MAIN_CONTENT_1\&context=storylines_live_updates}{Latest
Updates: The Coronavirus
Outbreak}}{Latest Updates: The Coronavirus Outbreak}}\label{latest-updates-the-coronavirus-outbreak}}

Updated 2020-08-21T11:05:09.310Z

\begin{itemize}
\tightlist
\item
  \href{https://www.nytimes3xbfgragh.onion/2020/08/21/world/covid-19-coronavirus.html?action=click\&pgtype=Article\&state=default\&region=MAIN_CONTENT_1\&context=storylines_live_updates\#link-4690b6aa}{Shutdowns,
  warnings and scoldings follow gatherings on college campuses.}
\item
  \href{https://www.nytimes3xbfgragh.onion/2020/08/21/world/covid-19-coronavirus.html?action=click\&pgtype=Article\&state=default\&region=MAIN_CONTENT_1\&context=storylines_live_updates\#link-324af071}{As
  he accepts the Democratic nomination, Biden knocks Trump's pandemic
  response.}
\item
  \href{https://www.nytimes3xbfgragh.onion/2020/08/21/world/covid-19-coronavirus.html?action=click\&pgtype=Article\&state=default\&region=MAIN_CONTENT_1\&context=storylines_live_updates\#link-35890b73}{Hundreds
  of doctors in Kenya go on strike over their pay and protective gear.}
\end{itemize}

\href{https://www.nytimes3xbfgragh.onion/2020/08/21/world/covid-19-coronavirus.html?action=click\&pgtype=Article\&state=default\&region=MAIN_CONTENT_1\&context=storylines_live_updates}{See
more updates}

More live coverage:
\href{https://www.nytimes3xbfgragh.onion/live/2020/08/20/business/stock-market-today-coronavirus?action=click\&pgtype=Article\&state=default\&region=MAIN_CONTENT_1\&context=storylines_live_updates}{Markets}

Some scientists were unsettled by the way in which the findings were
reported. The disclosure of trial results in a political setting, before
peer review or publication, is very unusual, said Dr. Steven Nissen, a
cardiologist at the Cleveland Clinic who has conducted dozens of
clinical trials.

``Where are the data?'' he asked. Scientists will need to see figures on
harms associated with the drug in order to assess its benefits, he
added.

``This is too important to be handled in such a sloppy fashion,'' Dr.
Nissen said.

Dr. Michele Barry, a global health expert at Stanford University, said
she had faith in Dr. Fauci's assessment. Still, she added, ``It is
unusual to call a drug the `standard of care' until peer review of data
and publication, and before studies have shown benefit in mortality.''

The Food and Drug Administration is likely at some point to announce an
emergency approval for remdesivir, a senior administration official told
The New York Times. Another drug touted by the president,
hydrochloroquine, also was granted such an approval, but results in
patients have been disappointing.

In one study of veterans with Covid-19, those receiving hydrochloroquine
and an antibiotic died at higher rates than those given ordinary
supportive care.

Mr. Trump also hopes to put in place a crash program to develop a
vaccine, an undertaking being seen with skepticism even inside the
administration. The accelerated process, known internally as Operation
Warp Speed, would aim to produce hundreds of millions of doses by the
end of this year.

But medical experts, including Dr. Fauci, have warned that developing a
vaccine will require a year to 18 months at the earliest, and that
rushing the process could endanger public health.

For now, drug treatment seems a more attainable goal.

``Remdesivir is not a magic bullet, but it's as good as we get right
now,'' said Dr. Peter Chin-Hong, an infectious disease specialist at the
University of California, San Francisco, and one of the trial's
investigators.

``Patients come to the hospital thinking we have a treatment, and by
treatment, they mean a drug,'' he added. ``We have been impotent in not
having any options.''

Remdesivir has never been approved as a treatment for any disease. It
was developed to fight Ebola, but results from a clinical trial in
Africa were disappointing.

But as the coronavirus pandemic took hold, the drug emerged as one of
the more promising potential treatments. It interrupts the production of
the virus in lab studies and seems safe in animals.

\href{https://www.nytimes3xbfgragh.onion/news-event/coronavirus?action=click\&pgtype=Article\&state=default\&region=MAIN_CONTENT_3\&context=storylines_faq}{}

\hypertarget{the-coronavirus-outbreak-}{%
\subsubsection{The Coronavirus Outbreak
›}\label{the-coronavirus-outbreak-}}

\hypertarget{frequently-asked-questions}{%
\paragraph{Frequently Asked
Questions}\label{frequently-asked-questions}}

Updated August 17, 2020

\begin{itemize}
\item ~
  \hypertarget{why-does-standing-six-feet-away-from-others-help}{%
  \paragraph{Why does standing six feet away from others
  help?}\label{why-does-standing-six-feet-away-from-others-help}}

  \begin{itemize}
  \tightlist
  \item
    The coronavirus spreads primarily through droplets from your mouth
    and nose, especially when you cough or sneeze. The C.D.C., one of
    the organizations using that measure,
    \href{https://www.nytimes3xbfgragh.onion/2020/04/14/health/coronavirus-six-feet.html?action=click\&pgtype=Article\&state=default\&region=MAIN_CONTENT_3\&context=storylines_faq}{bases
    its recommendation of six feet} on the idea that most large droplets
    that people expel when they cough or sneeze will fall to the ground
    within six feet. But six feet has never been a magic number that
    guarantees complete protection. Sneezes, for instance, can launch
    droplets a lot farther than six feet,
    \href{https://jamanetwork.com/journals/jama/fullarticle/2763852}{according
    to a recent study}. It's a rule of thumb: You should be safest
    standing six feet apart outside, especially when it's windy. But
    keep a mask on at all times, even when you think you're far enough
    apart.
  \end{itemize}
\item ~
  \hypertarget{i-have-antibodies-am-i-now-immune}{%
  \paragraph{I have antibodies. Am I now
  immune?}\label{i-have-antibodies-am-i-now-immune}}

  \begin{itemize}
  \tightlist
  \item
    As of right
    now,\href{https://www.nytimes3xbfgragh.onion/2020/07/22/health/covid-antibodies-herd-immunity.html?action=click\&pgtype=Article\&state=default\&region=MAIN_CONTENT_3\&context=storylines_faq}{that
    seems likely, for at least several months.} There have been
    frightening accounts of people suffering what seems to be a second
    bout of Covid-19. But experts say these patients may have a
    drawn-out course of infection, with the virus taking a slow toll
    weeks to months after initial exposure. People infected with the
    coronavirus typically
    \href{https://www.nature.com/articles/s41586-020-2456-9}{produce}
    immune molecules called antibodies, which are
    \href{https://www.nytimes3xbfgragh.onion/2020/05/07/health/coronavirus-antibody-prevalence.html?action=click\&pgtype=Article\&state=default\&region=MAIN_CONTENT_3\&context=storylines_faq}{protective
    proteins made in response to an
    infection}\href{https://www.nytimes3xbfgragh.onion/2020/05/07/health/coronavirus-antibody-prevalence.html?action=click\&pgtype=Article\&state=default\&region=MAIN_CONTENT_3\&context=storylines_faq}{.
    These antibodies may} last in the body
    \href{https://www.nature.com/articles/s41591-020-0965-6}{only two to
    three months}, which may seem worrisome, but that's perfectly normal
    after an acute infection subsides, said Dr. Michael Mina, an
    immunologist at Harvard University. It may be possible to get the
    coronavirus again, but it's highly unlikely that it would be
    possible in a short window of time from initial infection or make
    people sicker the second time.
  \end{itemize}
\item ~
  \hypertarget{im-a-small-business-owner-can-i-get-relief}{%
  \paragraph{I'm a small-business owner. Can I get
  relief?}\label{im-a-small-business-owner-can-i-get-relief}}

  \begin{itemize}
  \tightlist
  \item
    The
    \href{https://www.nytimes3xbfgragh.onion/article/small-business-loans-stimulus-grants-freelancers-coronavirus.html?action=click\&pgtype=Article\&state=default\&region=MAIN_CONTENT_3\&context=storylines_faq}{stimulus
    bills enacted in March} offer help for the millions of American
    small businesses. Those eligible for aid are businesses and
    nonprofit organizations with fewer than 500 workers, including sole
    proprietorships, independent contractors and freelancers. Some
    larger companies in some industries are also eligible. The help
    being offered, which is being managed by the Small Business
    Administration, includes the Paycheck Protection Program and the
    Economic Injury Disaster Loan program. But lots of folks have
    \href{https://www.nytimes3xbfgragh.onion/interactive/2020/05/07/business/small-business-loans-coronavirus.html?action=click\&pgtype=Article\&state=default\&region=MAIN_CONTENT_3\&context=storylines_faq}{not
    yet seen payouts.} Even those who have received help are confused:
    The rules are draconian, and some are stuck sitting on
    \href{https://www.nytimes3xbfgragh.onion/2020/05/02/business/economy/loans-coronavirus-small-business.html?action=click\&pgtype=Article\&state=default\&region=MAIN_CONTENT_3\&context=storylines_faq}{money
    they don't know how to use.} Many small-business owners are getting
    less than they expected or
    \href{https://www.nytimes3xbfgragh.onion/2020/06/10/business/Small-business-loans-ppp.html?action=click\&pgtype=Article\&state=default\&region=MAIN_CONTENT_3\&context=storylines_faq}{not
    hearing anything at all.}
  \end{itemize}
\item ~
  \hypertarget{what-are-my-rights-if-i-am-worried-about-going-back-to-work}{%
  \paragraph{What are my rights if I am worried about going back to
  work?}\label{what-are-my-rights-if-i-am-worried-about-going-back-to-work}}

  \begin{itemize}
  \tightlist
  \item
    Employers have to provide
    \href{https://www.osha.gov/SLTC/covid-19/standards.html}{a safe
    workplace} with policies that protect everyone equally.
    \href{https://www.nytimes3xbfgragh.onion/article/coronavirus-money-unemployment.html?action=click\&pgtype=Article\&state=default\&region=MAIN_CONTENT_3\&context=storylines_faq}{And
    if one of your co-workers tests positive for the coronavirus, the
    C.D.C.} has said that
    \href{https://www.cdc.gov/coronavirus/2019-ncov/community/guidance-business-response.html}{employers
    should tell their employees} -\/- without giving you the sick
    employee's name -\/- that they may have been exposed to the virus.
  \end{itemize}
\item ~
  \hypertarget{what-is-school-going-to-look-like-in-september}{%
  \paragraph{What is school going to look like in
  September?}\label{what-is-school-going-to-look-like-in-september}}

  \begin{itemize}
  \tightlist
  \item
    It is unlikely that many schools will return to a normal schedule
    this fall, requiring the grind of
    \href{https://www.nytimes3xbfgragh.onion/2020/06/05/us/coronavirus-education-lost-learning.html?action=click\&pgtype=Article\&state=default\&region=MAIN_CONTENT_3\&context=storylines_faq}{online
    learning},
    \href{https://www.nytimes3xbfgragh.onion/2020/05/29/us/coronavirus-child-care-centers.html?action=click\&pgtype=Article\&state=default\&region=MAIN_CONTENT_3\&context=storylines_faq}{makeshift
    child care} and
    \href{https://www.nytimes3xbfgragh.onion/2020/06/03/business/economy/coronavirus-working-women.html?action=click\&pgtype=Article\&state=default\&region=MAIN_CONTENT_3\&context=storylines_faq}{stunted
    workdays} to continue. California's two largest public school
    districts --- Los Angeles and San Diego --- said on July 13, that
    \href{https://www.nytimes3xbfgragh.onion/2020/07/13/us/lausd-san-diego-school-reopening.html?action=click\&pgtype=Article\&state=default\&region=MAIN_CONTENT_3\&context=storylines_faq}{instruction
    will be remote-only in the fall}, citing concerns that surging
    coronavirus infections in their areas pose too dire a risk for
    students and teachers. Together, the two districts enroll some
    825,000 students. They are the largest in the country so far to
    abandon plans for even a partial physical return to classrooms when
    they reopen in August. For other districts, the solution won't be an
    all-or-nothing approach.
    \href{https://bioethics.jhu.edu/research-and-outreach/projects/eschool-initiative/school-policy-tracker/}{Many
    systems}, including the nation's largest, New York City, are
    devising
    \href{https://www.nytimes3xbfgragh.onion/2020/06/26/us/coronavirus-schools-reopen-fall.html?action=click\&pgtype=Article\&state=default\&region=MAIN_CONTENT_3\&context=storylines_faq}{hybrid
    plans} that involve spending some days in classrooms and other days
    online. There's no national policy on this yet, so check with your
    municipal school system regularly to see what is happening in your
    community.
  \end{itemize}
\end{itemize}

Until now, high expectations for remdesivir have been fueled largely by
anecdotal reports of Covid-19 patients who took the drug and recovered.

\textbf{\emph{{[}}\href{http://on.fb.me/1paTQ1h}{\emph{Like the Science
Times page on Facebook.}}} ****** \emph{\textbar{} Sign up for the}
\textbf{\href{http://nyti.ms/1MbHaRU}{\emph{Science Times
newsletter.}}\emph{{]}}}

Two such reports were published in the prestigious New England Journal
of Medicine, lending credibility to what researchers said were actually
uncertain results.

Without trials comparing the drug to a placebo, it has been impossible
to know whether the drug made a difference or patients got better on
their own with normal supportive care.

A separate study of remdesivir, published on Wednesday in The Lancet,
\href{https://info.thelancet.com/e2t/c/*W5Lhn838fnW_3W4sm_Wg5QQHzp0/*W5r1Ssg7F8LYhW2vgqZF25s9wN0/5/f18dQhb0Sq5G8XYcl9W7cBCZh2qwv15N4cqMnwqd32pMf5gd4XD6prW7cmS1s8pCQ6vW1mbWlS51fy5BW1mZkg96PZgJcW6253v925_ybHN1fkGplKzm2PN5nYjTX6nKnZVtMd0d4gt1GhW7vVV9F1XJ6fhW57jZ6S8_1Q19W49wcLf2LRNVfW4bJ02h4b_rKwW7MbC3t490HkzVWnFLM4LzjSNW7-XB_567h1CPW8r5GVd2GRw4YW3HtrQn2K4XMjW8tzC3F8q-1tCN1KHWqCRFkK4W5G18Kh3y9KYQW3dgtM73TW-bPW44SfhK8LmYtqW61pGpf2Z_1FDN2_yx0RMHdTrVVyvfc3T_7nPW13211g8zzbM9W8LQbGP8k1lZBW8bCgv05CG9_9W8p-_Z43XZzCFW8fHR_k10p3ZcW8bJghY43Rx5xW17pGy38dQmZsW6wydXJ392FRmW4YxKrk4PxC8jW448k6V2YN1mWW3BYbdm3sfwJ-W41Tj081CSMcvW8mhVQY5GJkvkW1TxRJq25y1NZW65rp9c82Z63tW7m73CJ8rdTqcW7C2cLb7D7HxjW1Dy_KC77gPCSf3S9Pl404}{found
no benefit to the drug, compared with placebo}.

``Unfortunately, our trial found that while safe and adequately
tolerated, remdesivir did not provide significant benefits over
placebo,'' said the lead investigator of the new study, Dr. Bin Cao of
the China-Japan Friendship Hospital and Capital Medical University in
Beijing.

``This is not the outcome we hoped for,'' he added.

The results are hard to interpret, because the study was far smaller
than planned --- enrolling 236 patients instead of the 453 that had been
expected, because there were too few severely ill patients now in China.

Dr. Eric Peterson, a clinical trials expert at Duke University, said
that with too few patients, ``all you can say is it doesn't seem to work
in this population.'' If there had been a big effect of the drug, he
added, that would have been seen.

He added that the trial should not be repeated with this population, but
instead in those who are less severely ill.

``This is a flawed study,'' Dr. Barry said, but results might improve if
the drug were given at a higher dose or earlier in the course of the
disease.

Acceding to demands, Gilead has distributed the drug to hundreds of
patients under so-called compassionate use, a regulatory exemption by
which patients may receive a drug apart from a clinical trial.

Gilead itself published reports of uncontrolled studies. On Wednesday,
in another news release, the company announced that a study comparing a
five- to 10-day course of treatment with the drug showed that those
getting the shorter course of treatment did just as well.

That study had no control group and was ``noninformative,'' Dr. Peterson
said.

Sheila Kaplan contributed reporting.

Advertisement

\protect\hyperlink{after-bottom}{Continue reading the main story}

\hypertarget{site-index}{%
\subsection{Site Index}\label{site-index}}

\hypertarget{site-information-navigation}{%
\subsection{Site Information
Navigation}\label{site-information-navigation}}

\begin{itemize}
\tightlist
\item
  \href{https://help.nytimes3xbfgragh.onion/hc/en-us/articles/115014792127-Copyright-notice}{©~2020~The
  New York Times Company}
\end{itemize}

\begin{itemize}
\tightlist
\item
  \href{https://www.nytco.com/}{NYTCo}
\item
  \href{https://help.nytimes3xbfgragh.onion/hc/en-us/articles/115015385887-Contact-Us}{Contact
  Us}
\item
  \href{https://www.nytco.com/careers/}{Work with us}
\item
  \href{https://nytmediakit.com/}{Advertise}
\item
  \href{http://www.tbrandstudio.com/}{T Brand Studio}
\item
  \href{https://www.nytimes3xbfgragh.onion/privacy/cookie-policy\#how-do-i-manage-trackers}{Your
  Ad Choices}
\item
  \href{https://www.nytimes3xbfgragh.onion/privacy}{Privacy}
\item
  \href{https://help.nytimes3xbfgragh.onion/hc/en-us/articles/115014893428-Terms-of-service}{Terms
  of Service}
\item
  \href{https://help.nytimes3xbfgragh.onion/hc/en-us/articles/115014893968-Terms-of-sale}{Terms
  of Sale}
\item
  \href{https://spiderbites.nytimes3xbfgragh.onion}{Site Map}
\item
  \href{https://help.nytimes3xbfgragh.onion/hc/en-us}{Help}
\item
  \href{https://www.nytimes3xbfgragh.onion/subscription?campaignId=37WXW}{Subscriptions}
\end{itemize}
