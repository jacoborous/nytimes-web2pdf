Sections

SEARCH

\protect\hyperlink{site-content}{Skip to
content}\protect\hyperlink{site-index}{Skip to site index}

\href{https://www.nytimes3xbfgragh.onion/section/science}{Science}

\href{https://myaccount.nytimes3xbfgragh.onion/auth/login?response_type=cookie\&client_id=vi}{}

\href{https://www.nytimes3xbfgragh.onion/section/todayspaper}{Today's
Paper}

\href{/section/science}{Science}\textbar{}Hundreds of Scientists
Scramble to Find a Coronavirus Treatment

\url{https://nyti.ms/3da9iEj}

\begin{itemize}
\item
\item
\item
\item
\item
\item
\end{itemize}

\hypertarget{the-coronavirus-outbreak}{%
\subsubsection{\texorpdfstring{\href{https://www.nytimes3xbfgragh.onion/news-event/coronavirus?name=styln-coronavirus-national\&region=TOP_BANNER\&variant=undefined\&block=storyline_menu_recirc\&action=click\&pgtype=Article\&impression_id=31a0c330-e3b0-11ea-b605-f71bd33d99de}{The
Coronavirus
Outbreak}}{The Coronavirus Outbreak}}\label{the-coronavirus-outbreak}}

\begin{itemize}
\tightlist
\item
  live\href{https://www.nytimes3xbfgragh.onion/2020/08/21/world/covid-19-coronavirus.html?name=styln-coronavirus-national\&region=TOP_BANNER\&variant=undefined\&block=storyline_menu_recirc\&action=click\&pgtype=Article\&impression_id=31a0ea40-e3b0-11ea-b605-f71bd33d99de}{Latest
  Updates}
\item
  \href{https://www.nytimes3xbfgragh.onion/interactive/2020/us/coronavirus-us-cases.html?name=styln-coronavirus-national\&region=TOP_BANNER\&variant=undefined\&block=storyline_menu_recirc\&action=click\&pgtype=Article\&impression_id=31a0ea41-e3b0-11ea-b605-f71bd33d99de}{Maps
  and Cases}
\item
  \href{https://www.nytimes3xbfgragh.onion/interactive/2020/science/coronavirus-vaccine-tracker.html?name=styln-coronavirus-national\&region=TOP_BANNER\&variant=undefined\&block=storyline_menu_recirc\&action=click\&pgtype=Article\&impression_id=31a0ea42-e3b0-11ea-b605-f71bd33d99de}{Vaccine
  Tracker}
\item
  \href{https://www.nytimes3xbfgragh.onion/2020/08/19/us/colleges-closing-covid.html?name=styln-coronavirus-national\&region=TOP_BANNER\&variant=undefined\&block=storyline_menu_recirc\&action=click\&pgtype=Article\&impression_id=31a0ea43-e3b0-11ea-b605-f71bd33d99de}{Colleges
  Closing}
\item
  \href{https://www.nytimes3xbfgragh.onion/live/2020/08/21/business/stock-market-today-coronavirus?name=styln-coronavirus-national\&region=TOP_BANNER\&variant=undefined\&block=storyline_menu_recirc\&action=click\&pgtype=Article\&impression_id=31a0ea44-e3b0-11ea-b605-f71bd33d99de}{Economy}
\end{itemize}

Advertisement

\protect\hyperlink{after-top}{Continue reading the main story}

Supported by

\protect\hyperlink{after-sponsor}{Continue reading the main story}

matter

\hypertarget{hundreds-of-scientists-scramble-to-find-a-coronavirus-treatment}{%
\section{Hundreds of Scientists Scramble to Find a Coronavirus
Treatment}\label{hundreds-of-scientists-scramble-to-find-a-coronavirus-treatment}}

In an ambitious international collaboration, researchers have ``mapped''
proteins in the coronavirus and identified 50 drugs to test against it.

\includegraphics{https://static01.graylady3jvrrxbe.onion/images/2020/03/16/science/16VIRUS-ANTIVIRAL1/16VIRUS-ANTIVIRAL1-articleLarge-v2.jpg?quality=75\&auto=webp\&disable=upscale}

\href{https://www.nytimes3xbfgragh.onion/by/carl-zimmer}{\includegraphics{https://static01.graylady3jvrrxbe.onion/images/2018/06/12/multimedia/author-carl-zimmer/author-carl-zimmer-thumbLarge.png}}

By \href{https://www.nytimes3xbfgragh.onion/by/carl-zimmer}{Carl Zimmer}

\begin{itemize}
\item
  March 17, 2020
\item
  \begin{itemize}
  \item
  \item
  \item
  \item
  \item
  \item
  \end{itemize}
\end{itemize}

Working at a breakneck pace, a team of hundreds of scientists has
identified 50 drugs that may be effective treatments for people infected
with
\href{https://www.nytimes3xbfgragh.onion/news-event/coronavirus}{the
coronavirus}.

Many scientists are seeking drugs that attack the virus itself. But the
Quantitative Biosciences Institute Coronavirus Research Group, based at
the University of California, San Francisco, is testing an unusual new
approach.

The researchers are looking for drugs that shield proteins in our own
cells that the coronavirus depends on to thrive and reproduce.

Many of the candidate drugs are already approved to treat diseases, such
as cancer, that would seem to have nothing to do with Covid-19, the
illness caused by the coronavirus.

Scientists at Mount Sinai Hospital in New York and at the Pasteur
Institute in Paris have already begun to test the drugs against the
coronavirus growing in their labs. The far-flung research group is
preparing to release its findings at the end of the week.

There is no antiviral drug proven to be effective against the virus.
When people get infected, the best that doctors can offer is supportive
care --- the patient is getting enough oxygen, managing fever and using
a ventilator to push air into the lungs, if needed --- to give the
immune system time to fight the infection.

If the research effort succeeds, it will be a significant scientific
achievement: an antiviral identified in just months to treat a virus
that no one knew existed until January.

``I'm really impressed at the speed and the scale at which they're
moving,'' said John Young, the global head of infectious diseases at
Roche Pharma Research and Early Development, which is collaborating on
some of the work.

\includegraphics{https://static01.graylady3jvrrxbe.onion/images/2020/03/16/science/16VIRUS-ANTIVIRAL2/16VIRUS-ANTIVIRAL2-articleLarge.jpg?quality=75\&auto=webp\&disable=upscale}

``We think this approach has real potential,'' he said.

Some researchers at the Q.B.I. began studying the coronavirus in
January. But last month, the threat
\href{https://www.nytimes3xbfgragh.onion/2020/02/26/health/coronavirus-cdc-usa.html}{became
more imminent}: A woman in California was found to be infected although
she had not recently traveled outside the country.

That finding suggested that the virus was already circulating in the
community.

``I got to the lab and said we've got to drop everything else,''
recalled Nevan Krogan, director of the Quantitative Biosciences
Institute. ``Everybody has got to work around the clock on this.''

\hypertarget{latest-updates-the-coronavirus-outbreak}{%
\section{\texorpdfstring{\href{https://www.nytimes3xbfgragh.onion/2020/08/21/world/covid-19-coronavirus.html?action=click\&pgtype=Article\&state=default\&region=MAIN_CONTENT_1\&context=storylines_live_updates}{Latest
Updates: The Coronavirus
Outbreak}}{Latest Updates: The Coronavirus Outbreak}}\label{latest-updates-the-coronavirus-outbreak}}

Updated 2020-08-21T13:10:59.634Z

\begin{itemize}
\tightlist
\item
  \href{https://www.nytimes3xbfgragh.onion/2020/08/21/world/covid-19-coronavirus.html?action=click\&pgtype=Article\&state=default\&region=MAIN_CONTENT_1\&context=storylines_live_updates\#link-6a60a19d}{`Be
  adults': Universities in the U.S. are warning students about
  gatherings as they return to campus.}
\item
  \href{https://www.nytimes3xbfgragh.onion/2020/08/21/world/covid-19-coronavirus.html?action=click\&pgtype=Article\&state=default\&region=MAIN_CONTENT_1\&context=storylines_live_updates\#link-324af071}{As
  he accepts the Democratic nomination, Biden knocks Trump's pandemic
  response.}
\item
  \href{https://www.nytimes3xbfgragh.onion/2020/08/21/world/covid-19-coronavirus.html?action=click\&pgtype=Article\&state=default\&region=MAIN_CONTENT_1\&context=storylines_live_updates\#link-191d44be}{South
  Korea threatens to detain people who obstruct virus-control efforts.}
\end{itemize}

\href{https://www.nytimes3xbfgragh.onion/2020/08/21/world/covid-19-coronavirus.html?action=click\&pgtype=Article\&state=default\&region=MAIN_CONTENT_1\&context=storylines_live_updates}{See
more updates}

More live coverage:
\href{https://www.nytimes3xbfgragh.onion/live/2020/08/21/business/stock-market-today-coronavirus?action=click\&pgtype=Article\&state=default\&region=MAIN_CONTENT_1\&context=storylines_live_updates}{Markets}

Dr. Krogan and his colleagues set about finding proteins in our cells
that the coronavirus uses to grow. Normally, such a project might take
two years. But the working group, which includes 22 laboratories,
completed it in a few weeks.

``You have 30 scientists on a Zoom call --- it's the most exhausting,
amazing thing,'' Dr. Krogan said, referring to a teleconferencing
service.

Viruses reproduce by injecting their genes inside a human cell. The
cell's own gene-reading machinery then manufactures viral proteins,
which latch onto cellular proteins to create new viruses. They
eventually escape the cell and infect others.

\href{https://www.nytimes3xbfgragh.onion/interactive/2020/03/11/science/how-coronavirus-hijacks-your-cells.html}{}

\includegraphics{https://static01.graylady3jvrrxbe.onion/images/2020/03/10/us/how-coronavirus-infects-a-cell-promo-1583866148761/how-coronavirus-infects-a-cell-promo-1583866148761-articleLarge-v2.jpg}

\hypertarget{how-coronavirus-hijacks-your-cells}{%
\subsection{How Coronavirus Hijacks Your
Cells}\label{how-coronavirus-hijacks-your-cells}}

The intricate journey of the virus that causes Covid-19.

In 2011, Dr. Krogan and his colleagues developed a way to find all the
human proteins that viruses use to manipulate our cells --- a ``map,''
as Dr. Krogan calls it. They created their first map for H.I.V.

That virus has 18 genes, each of which encodes a protein. The scientists
eventually found that H.I.V. interacts, in one way or another, with 435
proteins in a human cell.

Dr. Krogan and his colleagues went on to make similar maps for viruses
such as Ebola and dengue. Each pathogen hijacks its host cell by
manipulating a different combination of proteins. Once scientists have a
map, they can use it to search for new treatments.

In February, the research group synthesized genes from the coronavirus
and injected them into cells. They uncovered over 400 human proteins
that the virus seems to rely on.

\hypertarget{protein-targets}{%
\subsection{Protein Targets}\label{protein-targets}}

Scientists have mapped hundreds of the proteins in human cells that the
coronavirus uses to grow. Their goal is to find an antiviral drug that
can prevent the virus from replicating.

The SARS-CoV-2 Coronavirus

Some of the Coronavirus proteins and the human proteins they interact
with

M proteins

BRD2 protein

N

proteins

E protein

M

N

Other human proteins

that interact with E

E proteins

orf6

orf7a

nsp1

nsp5

orf3a

orf9b

orf10

nsp5

Protein networks

nsp4

nsp5

orf8

protein14

The SARS-CoV-2 Coronavirus

Some of the Coronavirus proteins and

the human proteins they interact with

M proteins

BRD2 protein

N

proteins

nsp5

E protein

Other human proteins

that interact with E

E proteins

orf6

orf7a

nsp1

orf3a

orf9b

orf10

Protein

network

M

nsp5

nsp4

nsp5

orf8

N

The SARS-CoV-2 Coronavirus

M proteins

N

proteins

E proteins

Some of the Coronavirus proteins and

the human proteins they interact with

BRD2

protein

E protein

orf3a

Other human proteins

that interact with E

nsp5

nsp4

Protein

network

N

M

The SARS-CoV-2 Coronavirus

M proteins

N

proteins

E proteins

Some of the Coronavirus proteins and

the human proteins they interact with

BRD2

protein

E protein

orf3a

Other human proteins

that interact with E

nsp5

nsp4

Protein

network

N

M

By Jonathan Corum \textbar{} Source: Quantitative Biosciences Institute
Coronavirus Research Group

The flulike symptoms observed in infected people are the result of the
coronavirus attacking cells in the respiratory tract. The new map shows
that the virus's proteins travel throughout the human cell, engaging
even with proteins that do not seem to have anything to do with making
new viruses.

One of the viral proteins, for example, latches onto BRD2, a human
protein that tends to our DNA, switching genes on and off. Experts on
proteins are now using the map to figure out why the coronavirus needs
these molecules.

Kevan Shokat, a chemist at U.C.S.F., is poring through 20,000 drugs
approved by the Food and Drug Administration for signs that they may
interact with the proteins on the map created by Dr. Krogan's lab.

Image

Dr. Miorin preparing samples for testing at the Icahn School of Medicine
at Mount Sinai. Researchers in New York and Paris are beginning to test
drug candidates identified with the help of a new ``map'' of viral
proteins.Credit...Victor J. Blue for The New York Times

Dr. Shokat and his colleagues have found 50 promising candidates. The
protein BRD2, for example, can be targeted by a drug called JQ1.
Researchers originally
\href{https://www.ncbi.nlm.nih.gov/pmc/articles/PMC3010259/}{discovered
JQ1 as a potential treatment for several types of cancer}.

\href{https://www.nytimes3xbfgragh.onion/news-event/coronavirus?action=click\&pgtype=Article\&state=default\&region=MAIN_CONTENT_3\&context=storylines_faq}{}

\hypertarget{the-coronavirus-outbreak-}{%
\subsubsection{The Coronavirus Outbreak
›}\label{the-coronavirus-outbreak-}}

\hypertarget{frequently-asked-questions}{%
\paragraph{Frequently Asked
Questions}\label{frequently-asked-questions}}

Updated August 17, 2020

\begin{itemize}
\item ~
  \hypertarget{why-does-standing-six-feet-away-from-others-help}{%
  \paragraph{Why does standing six feet away from others
  help?}\label{why-does-standing-six-feet-away-from-others-help}}

  \begin{itemize}
  \tightlist
  \item
    The coronavirus spreads primarily through droplets from your mouth
    and nose, especially when you cough or sneeze. The C.D.C., one of
    the organizations using that measure,
    \href{https://www.nytimes3xbfgragh.onion/2020/04/14/health/coronavirus-six-feet.html?action=click\&pgtype=Article\&state=default\&region=MAIN_CONTENT_3\&context=storylines_faq}{bases
    its recommendation of six feet} on the idea that most large droplets
    that people expel when they cough or sneeze will fall to the ground
    within six feet. But six feet has never been a magic number that
    guarantees complete protection. Sneezes, for instance, can launch
    droplets a lot farther than six feet,
    \href{https://jamanetwork.com/journals/jama/fullarticle/2763852}{according
    to a recent study}. It's a rule of thumb: You should be safest
    standing six feet apart outside, especially when it's windy. But
    keep a mask on at all times, even when you think you're far enough
    apart.
  \end{itemize}
\item ~
  \hypertarget{i-have-antibodies-am-i-now-immune}{%
  \paragraph{I have antibodies. Am I now
  immune?}\label{i-have-antibodies-am-i-now-immune}}

  \begin{itemize}
  \tightlist
  \item
    As of right
    now,\href{https://www.nytimes3xbfgragh.onion/2020/07/22/health/covid-antibodies-herd-immunity.html?action=click\&pgtype=Article\&state=default\&region=MAIN_CONTENT_3\&context=storylines_faq}{that
    seems likely, for at least several months.} There have been
    frightening accounts of people suffering what seems to be a second
    bout of Covid-19. But experts say these patients may have a
    drawn-out course of infection, with the virus taking a slow toll
    weeks to months after initial exposure. People infected with the
    coronavirus typically
    \href{https://www.nature.com/articles/s41586-020-2456-9}{produce}
    immune molecules called antibodies, which are
    \href{https://www.nytimes3xbfgragh.onion/2020/05/07/health/coronavirus-antibody-prevalence.html?action=click\&pgtype=Article\&state=default\&region=MAIN_CONTENT_3\&context=storylines_faq}{protective
    proteins made in response to an
    infection}\href{https://www.nytimes3xbfgragh.onion/2020/05/07/health/coronavirus-antibody-prevalence.html?action=click\&pgtype=Article\&state=default\&region=MAIN_CONTENT_3\&context=storylines_faq}{.
    These antibodies may} last in the body
    \href{https://www.nature.com/articles/s41591-020-0965-6}{only two to
    three months}, which may seem worrisome, but that's perfectly normal
    after an acute infection subsides, said Dr. Michael Mina, an
    immunologist at Harvard University. It may be possible to get the
    coronavirus again, but it's highly unlikely that it would be
    possible in a short window of time from initial infection or make
    people sicker the second time.
  \end{itemize}
\item ~
  \hypertarget{im-a-small-business-owner-can-i-get-relief}{%
  \paragraph{I'm a small-business owner. Can I get
  relief?}\label{im-a-small-business-owner-can-i-get-relief}}

  \begin{itemize}
  \tightlist
  \item
    The
    \href{https://www.nytimes3xbfgragh.onion/article/small-business-loans-stimulus-grants-freelancers-coronavirus.html?action=click\&pgtype=Article\&state=default\&region=MAIN_CONTENT_3\&context=storylines_faq}{stimulus
    bills enacted in March} offer help for the millions of American
    small businesses. Those eligible for aid are businesses and
    nonprofit organizations with fewer than 500 workers, including sole
    proprietorships, independent contractors and freelancers. Some
    larger companies in some industries are also eligible. The help
    being offered, which is being managed by the Small Business
    Administration, includes the Paycheck Protection Program and the
    Economic Injury Disaster Loan program. But lots of folks have
    \href{https://www.nytimes3xbfgragh.onion/interactive/2020/05/07/business/small-business-loans-coronavirus.html?action=click\&pgtype=Article\&state=default\&region=MAIN_CONTENT_3\&context=storylines_faq}{not
    yet seen payouts.} Even those who have received help are confused:
    The rules are draconian, and some are stuck sitting on
    \href{https://www.nytimes3xbfgragh.onion/2020/05/02/business/economy/loans-coronavirus-small-business.html?action=click\&pgtype=Article\&state=default\&region=MAIN_CONTENT_3\&context=storylines_faq}{money
    they don't know how to use.} Many small-business owners are getting
    less than they expected or
    \href{https://www.nytimes3xbfgragh.onion/2020/06/10/business/Small-business-loans-ppp.html?action=click\&pgtype=Article\&state=default\&region=MAIN_CONTENT_3\&context=storylines_faq}{not
    hearing anything at all.}
  \end{itemize}
\item ~
  \hypertarget{what-are-my-rights-if-i-am-worried-about-going-back-to-work}{%
  \paragraph{What are my rights if I am worried about going back to
  work?}\label{what-are-my-rights-if-i-am-worried-about-going-back-to-work}}

  \begin{itemize}
  \tightlist
  \item
    Employers have to provide
    \href{https://www.osha.gov/SLTC/covid-19/standards.html}{a safe
    workplace} with policies that protect everyone equally.
    \href{https://www.nytimes3xbfgragh.onion/article/coronavirus-money-unemployment.html?action=click\&pgtype=Article\&state=default\&region=MAIN_CONTENT_3\&context=storylines_faq}{And
    if one of your co-workers tests positive for the coronavirus, the
    C.D.C.} has said that
    \href{https://www.cdc.gov/coronavirus/2019-ncov/community/guidance-business-response.html}{employers
    should tell their employees} -\/- without giving you the sick
    employee's name -\/- that they may have been exposed to the virus.
  \end{itemize}
\item ~
  \hypertarget{what-is-school-going-to-look-like-in-september}{%
  \paragraph{What is school going to look like in
  September?}\label{what-is-school-going-to-look-like-in-september}}

  \begin{itemize}
  \tightlist
  \item
    It is unlikely that many schools will return to a normal schedule
    this fall, requiring the grind of
    \href{https://www.nytimes3xbfgragh.onion/2020/06/05/us/coronavirus-education-lost-learning.html?action=click\&pgtype=Article\&state=default\&region=MAIN_CONTENT_3\&context=storylines_faq}{online
    learning},
    \href{https://www.nytimes3xbfgragh.onion/2020/05/29/us/coronavirus-child-care-centers.html?action=click\&pgtype=Article\&state=default\&region=MAIN_CONTENT_3\&context=storylines_faq}{makeshift
    child care} and
    \href{https://www.nytimes3xbfgragh.onion/2020/06/03/business/economy/coronavirus-working-women.html?action=click\&pgtype=Article\&state=default\&region=MAIN_CONTENT_3\&context=storylines_faq}{stunted
    workdays} to continue. California's two largest public school
    districts --- Los Angeles and San Diego --- said on July 13, that
    \href{https://www.nytimes3xbfgragh.onion/2020/07/13/us/lausd-san-diego-school-reopening.html?action=click\&pgtype=Article\&state=default\&region=MAIN_CONTENT_3\&context=storylines_faq}{instruction
    will be remote-only in the fall}, citing concerns that surging
    coronavirus infections in their areas pose too dire a risk for
    students and teachers. Together, the two districts enroll some
    825,000 students. They are the largest in the country so far to
    abandon plans for even a partial physical return to classrooms when
    they reopen in August. For other districts, the solution won't be an
    all-or-nothing approach.
    \href{https://bioethics.jhu.edu/research-and-outreach/projects/eschool-initiative/school-policy-tracker/}{Many
    systems}, including the nation's largest, New York City, are
    devising
    \href{https://www.nytimes3xbfgragh.onion/2020/06/26/us/coronavirus-schools-reopen-fall.html?action=click\&pgtype=Article\&state=default\&region=MAIN_CONTENT_3\&context=storylines_faq}{hybrid
    plans} that involve spending some days in classrooms and other days
    online. There's no national policy on this yet, so check with your
    municipal school system regularly to see what is happening in your
    community.
  \end{itemize}
\end{itemize}

On Thursday, Dr. Shokat and his colleagues filled a box with the first
10 drugs on the list and shipped them overnight to New York to be tested
against the living coronavirus.

The drugs arrived at the lab of Adolfo Garcia-Sastre, director of the
Global Health and Emerging Pathogens Institute at the Icahn School of
Medicine at Mount Sinai Hospital. Dr. Garcia-Sastre recently began
growing the coronavirus in monkey cells.

Over the weekend, the team at the institute began treating infected
cells with the drugs to see if any stop the viruses. ``We have started
experiments, but it will take us a week to get the first data here,''
Dr. Garcia-Sastre said on Tuesday.

The researchers in San Francisco also sent the batch of drugs to the
Pasteur Institute in Paris, where investigators also have begun testing
them against coronaviruses.

If promising drugs are found, investigators plan to try them in an
animal infected with the coronavirus --- perhaps ferrets, because
they're known to get SARS, an illness closely related to Covid-19.

Even if some of these drugs are effective treatments, scientists will
still need to make sure they are safe for treating Covid-19. It may turn
out, for example, that the dose needed to clear the virus from the body
might also lead to dangerous side effects.

This collaboration is far from the only effort
\href{https://pubs.acs.org/doi/10.1021/acscentsci.0c00272}{to find an
antiviral drug effective against the coronavirus}. One of the most
closely watched efforts involves an antiviral called remdesivir.

In past studies on animals, remdesivir blocked a number of viruses. The
drug works by
\href{https://www.jbc.org/content/early/2020/02/24/jbc.AC120.013056}{preventing
viruses from building new genes}.

Image

Dr. Adolfo Garcia-Sastre, director of the Global Health and Emerging
Pathogens Institute of Icahn School of Medicine at Mount
Sinai.Credit...Victor J. Blue for The New York Times

In February, a team of researchers found that remdesivir could
\href{https://www.nature.com/articles/s41422-020-0282-0}{eliminate the
coronavirus from infected cells}. Since then, five clinical trials have
begun to see if the drug will be safe and effective against Covid-19 in
people.

Other researchers have taken startling new approaches. On Saturday,
Stanford University researchers reported using the gene-editing
technology Crispr
\href{https://www.biorxiv.org/content/10.1101/2020.03.13.991307v1}{to
destroy coronavirus genes in infected cells}.

As the
\href{https://www.nytimes3xbfgragh.onion/2020/03/17/us/shelter-in-place-order-bay-area.html}{Bay
Area went into lockdown} on Monday, Dr. Krogan and his colleagues were
finishing their map. They are now preparing a report to post online by
the end of the week, while also submitting it to a journal for
publication.

Their paper will include a list of drugs that the researchers consider
prime candidates to treat people ill with the coronavirus.

``Whoever is capable of trying them, please try them,'' Dr. Krogan said.

Advertisement

\protect\hyperlink{after-bottom}{Continue reading the main story}

\hypertarget{site-index}{%
\subsection{Site Index}\label{site-index}}

\hypertarget{site-information-navigation}{%
\subsection{Site Information
Navigation}\label{site-information-navigation}}

\begin{itemize}
\tightlist
\item
  \href{https://help.nytimes3xbfgragh.onion/hc/en-us/articles/115014792127-Copyright-notice}{©~2020~The
  New York Times Company}
\end{itemize}

\begin{itemize}
\tightlist
\item
  \href{https://www.nytco.com/}{NYTCo}
\item
  \href{https://help.nytimes3xbfgragh.onion/hc/en-us/articles/115015385887-Contact-Us}{Contact
  Us}
\item
  \href{https://www.nytco.com/careers/}{Work with us}
\item
  \href{https://nytmediakit.com/}{Advertise}
\item
  \href{http://www.tbrandstudio.com/}{T Brand Studio}
\item
  \href{https://www.nytimes3xbfgragh.onion/privacy/cookie-policy\#how-do-i-manage-trackers}{Your
  Ad Choices}
\item
  \href{https://www.nytimes3xbfgragh.onion/privacy}{Privacy}
\item
  \href{https://help.nytimes3xbfgragh.onion/hc/en-us/articles/115014893428-Terms-of-service}{Terms
  of Service}
\item
  \href{https://help.nytimes3xbfgragh.onion/hc/en-us/articles/115014893968-Terms-of-sale}{Terms
  of Sale}
\item
  \href{https://spiderbites.nytimes3xbfgragh.onion}{Site Map}
\item
  \href{https://help.nytimes3xbfgragh.onion/hc/en-us}{Help}
\item
  \href{https://www.nytimes3xbfgragh.onion/subscription?campaignId=37WXW}{Subscriptions}
\end{itemize}
