Sections

SEARCH

\protect\hyperlink{site-content}{Skip to
content}\protect\hyperlink{site-index}{Skip to site index}

\href{https://www.nytimes3xbfgragh.onion/section/theater}{Theater}

\href{https://myaccount.nytimes3xbfgragh.onion/auth/login?response_type=cookie\&client_id=vi}{}

\href{https://www.nytimes3xbfgragh.onion/section/todayspaper}{Today's
Paper}

\href{/section/theater}{Theater}\textbar{}Broadway, Symbol of New York
Resilience, Shuts Down Amid Virus Threat

\url{https://nyti.ms/2W5RCnc}

\begin{itemize}
\item
\item
\item
\item
\item
\end{itemize}

\hypertarget{the-coronavirus-outbreak}{%
\subsubsection{\texorpdfstring{\href{https://www.nytimes3xbfgragh.onion/news-event/coronavirus?name=styln-coronavirus-national\&region=TOP_BANNER\&variant=undefined\&block=storyline_menu_recirc\&action=click\&pgtype=Article\&impression_id=ab7b06d0-e3af-11ea-9f78-853988e69a14}{The
Coronavirus
Outbreak}}{The Coronavirus Outbreak}}\label{the-coronavirus-outbreak}}

\begin{itemize}
\tightlist
\item
  live\href{https://www.nytimes3xbfgragh.onion/2020/08/21/world/covid-19-coronavirus.html?name=styln-coronavirus-national\&region=TOP_BANNER\&variant=undefined\&block=storyline_menu_recirc\&action=click\&pgtype=Article\&impression_id=ab7b06d1-e3af-11ea-9f78-853988e69a14}{Latest
  Updates}
\item
  \href{https://www.nytimes3xbfgragh.onion/interactive/2020/us/coronavirus-us-cases.html?name=styln-coronavirus-national\&region=TOP_BANNER\&variant=undefined\&block=storyline_menu_recirc\&action=click\&pgtype=Article\&impression_id=ab7b2de0-e3af-11ea-9f78-853988e69a14}{Maps
  and Cases}
\item
  \href{https://www.nytimes3xbfgragh.onion/interactive/2020/science/coronavirus-vaccine-tracker.html?name=styln-coronavirus-national\&region=TOP_BANNER\&variant=undefined\&block=storyline_menu_recirc\&action=click\&pgtype=Article\&impression_id=ab7b2de1-e3af-11ea-9f78-853988e69a14}{Vaccine
  Tracker}
\item
  \href{https://www.nytimes3xbfgragh.onion/2020/08/19/us/colleges-closing-covid.html?name=styln-coronavirus-national\&region=TOP_BANNER\&variant=undefined\&block=storyline_menu_recirc\&action=click\&pgtype=Article\&impression_id=ab7b2de2-e3af-11ea-9f78-853988e69a14}{Colleges
  Closing}
\item
  \href{https://www.nytimes3xbfgragh.onion/live/2020/08/21/business/stock-market-today-coronavirus?name=styln-coronavirus-national\&region=TOP_BANNER\&variant=undefined\&block=storyline_menu_recirc\&action=click\&pgtype=Article\&impression_id=ab7b2de3-e3af-11ea-9f78-853988e69a14}{Economy}
\end{itemize}

Advertisement

\protect\hyperlink{after-top}{Continue reading the main story}

Supported by

\protect\hyperlink{after-sponsor}{Continue reading the main story}

\hypertarget{broadway-symbol-of-new-york-resilience-shuts-down-amid-virus-threat}{%
\section{Broadway, Symbol of New York Resilience, Shuts Down Amid Virus
Threat}\label{broadway-symbol-of-new-york-resilience-shuts-down-amid-virus-threat}}

Facing restrictions on audience size and concern from actors and
audiences about health risks during the coronavirus pandemic, the
industry announced that shows will be shuttered through April 12.

\includegraphics{https://static01.graylady3jvrrxbe.onion/images/2020/03/12/arts/12virus-broadway-7/merlin_170431644_8c294b5e-1b8e-469a-8447-0ca70e2919e8-articleLarge.jpg?quality=75\&auto=webp\&disable=upscale}

\href{https://www.nytimes3xbfgragh.onion/by/michael-paulson}{\includegraphics{https://static01.graylady3jvrrxbe.onion/images/2018/02/20/multimedia/author-michael-paulson/author-michael-paulson-thumbLarge.jpg}}

By \href{https://www.nytimes3xbfgragh.onion/by/michael-paulson}{Michael
Paulson}

\begin{itemize}
\item
  Published March 12, 2020Updated May 12, 2020
\item
  \begin{itemize}
  \item
  \item
  \item
  \item
  \item
  \end{itemize}
\end{itemize}

The adage is synonymous with
\href{https://www.nytimes3xbfgragh.onion/2020/05/12/theater/broadway-coronavirus.html}{Broadway}
itself: the show must go on.

And for decades, through wars and recessions and all forms of darkness,
Broadway, the heart of America's theater industry and an economic
lifeblood for many artists, has kept its curtains up and its footlights
on.

But on Thursday, facing a widening
\href{https://www.nytimes3xbfgragh.onion/2020/05/12/theater/broadway-coronavirus.html}{coronavirus}
pandemic and new limitations on large gatherings, the industry said it
was suspending all plays and musicals for 32 days, effective
immediately.

``The idea that our venerable, majestic houses are dark, and that there
will be no lights on Broadway --- I'm romanticizing, but that's the
heartbeat of the city, and to think that they've been forced into
darkness is shocking,'' said Patti LuPone, a beloved Broadway titan who
has won two Tony Awards and has been performing in previews for a
revival of ``Company.'' ``I'm shocked that they took this tack, but also
grateful they did, just to keep us healthy.''

The shutdown --- longer than those prompted in recent decades by strikes
and snowstorms and even the terrorist attacks of Sept. 11, 2001 --- will
inevitably cost tens of millions of dollars for investors and artists
and associated businesses, and will likely trigger the collapse of some
plays and musicals that will be unable to survive the delays and losses.

The move came as Gov. Andrew M. Cuomo and Mayor Bill de Blasio enacted
new restrictions to try to stop the spread of the virus, which in the
city has infected nearly 100 people, a number expected to grow.

City and state officials banned most gatherings of more than 500 people,
and required smaller venues to cut their capacity by half; they also
limited nursing home visits. Mr. de Blasio declared a state of
emergency, empowering him to take measures like implementing a curfew or
limiting traffic should conditions worsen.

Public schools, however, were remaining open. The governor noted many
children's resistance to the virus, while the mayor expressed worry
about the disruptions that school closings would create.

Broadway --- central to, and symbolic of, New York --- is not only the
pinnacle of the American theater world, but is also big business:
\href{https://www.nytimes3xbfgragh.onion/2019/05/29/theater/broadway-box-office.html}{Last
season} the industry grossed \$1.8 billion and drew 14.8 million
patrons.

\hypertarget{latest-updates-the-coronavirus-outbreak}{%
\section{\texorpdfstring{\href{https://www.nytimes3xbfgragh.onion/2020/08/21/world/covid-19-coronavirus.html?action=click\&pgtype=Article\&state=default\&region=MAIN_CONTENT_1\&context=storylines_live_updates}{Latest
Updates: The Coronavirus
Outbreak}}{Latest Updates: The Coronavirus Outbreak}}\label{latest-updates-the-coronavirus-outbreak}}

Updated 2020-08-21T13:00:19.184Z

\begin{itemize}
\tightlist
\item
  \href{https://www.nytimes3xbfgragh.onion/2020/08/21/world/covid-19-coronavirus.html?action=click\&pgtype=Article\&state=default\&region=MAIN_CONTENT_1\&context=storylines_live_updates\#link-6a60a19d}{`Be
  adults': Universities in the U.S. are warning students about
  gatherings as they return to campus.}
\item
  \href{https://www.nytimes3xbfgragh.onion/2020/08/21/world/covid-19-coronavirus.html?action=click\&pgtype=Article\&state=default\&region=MAIN_CONTENT_1\&context=storylines_live_updates\#link-324af071}{As
  he accepts the Democratic nomination, Biden knocks Trump's pandemic
  response.}
\item
  \href{https://www.nytimes3xbfgragh.onion/2020/08/21/world/covid-19-coronavirus.html?action=click\&pgtype=Article\&state=default\&region=MAIN_CONTENT_1\&context=storylines_live_updates\#link-191d44be}{South
  Korea threatens to detain people who obstruct virus-control efforts.}
\end{itemize}

\href{https://www.nytimes3xbfgragh.onion/2020/08/21/world/covid-19-coronavirus.html?action=click\&pgtype=Article\&state=default\&region=MAIN_CONTENT_1\&context=storylines_live_updates}{See
more updates}

More live coverage:
\href{https://www.nytimes3xbfgragh.onion/live/2020/08/21/business/stock-market-today-coronavirus?action=click\&pgtype=Article\&state=default\&region=MAIN_CONTENT_1\&context=storylines_live_updates}{Markets}

``The full effects of this on the industry can't possibly be known yet,
but our priority has to be the well-being of audiences and our Broadway
families,'' said Thomas Schumacher, the president of Disney Theatrical
Productions (``The Lion King,'' ``Aladdin'' and ``Frozen'') and the
chairman of the Broadway League, the industry trade group.

Image

Credit...Vincent Tullo for The New York Times

Image

Credit...Vincent Tullo for The New York Times

The shuttering of theaters, which the Broadway League said would
continue through April 12, followed a flood of cultural closings around
the country and around the world.

Earlier on Thursday, several of New York's largest and most prestigious
cultural institutions --- including the Metropolitan Museum of Art, the
Metropolitan Opera, Carnegie Hall and the New York Philharmonic ---
announced that
\href{https://www.nytimes3xbfgragh.onion/2020/03/12/arts/design/met-museum-opera-carnegie-hall-close-coronavirus.html}{they
would temporarily shut down}. At the same time, Live Nation
Entertainment and AEG Presents, the corporate giants that dominate the
concert industry,
\href{https://www.nytimes3xbfgragh.onion/2020/03/10/arts/music/coronavirus-coachella-postponed.html}{suspended
all North American tour} engagements.

In Asia and Europe, many performance spaces had already closed; in the
United States, venues from the \href{https://www.5thavenue.org/}{5th
Avenue Theater} in Seattle to the
\href{https://www.centertheatregroup.org/}{Center Theater Group} in Los
Angeles to the \href{https://www.kennedy-center.org/}{Kennedy Center} in
Washington scrapped shows.

The theater industry had been hoping to avoid mass closings, taking
steps to reduce the risk of infection by adding hand sanitizer
dispensers, more frequently cleaning seats, barring backstage visits and
stage door interactions and even soda refills in used cups.

But Broadway has a lot of risk factors --- many of its shows attract an
older audience, and older people seem to be particularly susceptible to
the coronavirus; it depends heavily on tourism, which is plunging as a
result of the pandemic; and its theaters, lovely as they are, pack
patrons into tight quarters, making the now-recommended social
distancing essentially impossible.

As public health officials increasingly warned about the riskiness of
large gatherings, and after a part-time usher
\href{https://www.nytimes3xbfgragh.onion/2020/03/11/theater/broadway-show-usher-coronavirus.html}{was
diagnosed with the virus}, the drumbeat for closing grew louder.

Actors' Equity Association, a labor union representing 51,000 performers
and stage managers around the country, was becoming more and more
concerned.

``There's no such thing as social distancing for actors --- our jobs
sometimes require that we go to work and kiss our colleagues eight times
a week,'' said the actress
\href{https://www.nytimes3xbfgragh.onion/2016/10/05/theater/union-boss-and-former-miss-america-hits-the-road-in-fun-home.html}{Kate
Shindle,} who is the president of Equity. ``Although nobody wanted to
close the theaters, at the same time people were starting to be scared
to work, and with good reason.''

And theater producers, facing dwindling advance sales and concern for
audiences and employees, came to agree.

``Over the last 48 hours, watching everything escalate and the ground
keep shifting, the Broadway League has done the responsible thing,''
said Sue Frost, a lead producer of the musical
\href{https://www.nytimes3xbfgragh.onion/2017/03/12/theater/come-from-away-review.html}{``Come
From Away,''} which had been set to celebrate its third anniversary
Thursday night. ``We've got to protect our employees and our audiences,
and if this is what we have to do, this is what we have to do.''

\href{https://www.nytimes3xbfgragh.onion/news-event/coronavirus?action=click\&pgtype=Article\&state=default\&region=MAIN_CONTENT_3\&context=storylines_faq}{}

\hypertarget{the-coronavirus-outbreak-}{%
\subsubsection{The Coronavirus Outbreak
›}\label{the-coronavirus-outbreak-}}

\hypertarget{frequently-asked-questions}{%
\paragraph{Frequently Asked
Questions}\label{frequently-asked-questions}}

Updated August 17, 2020

\begin{itemize}
\item ~
  \hypertarget{why-does-standing-six-feet-away-from-others-help}{%
  \paragraph{Why does standing six feet away from others
  help?}\label{why-does-standing-six-feet-away-from-others-help}}

  \begin{itemize}
  \tightlist
  \item
    The coronavirus spreads primarily through droplets from your mouth
    and nose, especially when you cough or sneeze. The C.D.C., one of
    the organizations using that measure,
    \href{https://www.nytimes3xbfgragh.onion/2020/04/14/health/coronavirus-six-feet.html?action=click\&pgtype=Article\&state=default\&region=MAIN_CONTENT_3\&context=storylines_faq}{bases
    its recommendation of six feet} on the idea that most large droplets
    that people expel when they cough or sneeze will fall to the ground
    within six feet. But six feet has never been a magic number that
    guarantees complete protection. Sneezes, for instance, can launch
    droplets a lot farther than six feet,
    \href{https://jamanetwork.com/journals/jama/fullarticle/2763852}{according
    to a recent study}. It's a rule of thumb: You should be safest
    standing six feet apart outside, especially when it's windy. But
    keep a mask on at all times, even when you think you're far enough
    apart.
  \end{itemize}
\item ~
  \hypertarget{i-have-antibodies-am-i-now-immune}{%
  \paragraph{I have antibodies. Am I now
  immune?}\label{i-have-antibodies-am-i-now-immune}}

  \begin{itemize}
  \tightlist
  \item
    As of right
    now,\href{https://www.nytimes3xbfgragh.onion/2020/07/22/health/covid-antibodies-herd-immunity.html?action=click\&pgtype=Article\&state=default\&region=MAIN_CONTENT_3\&context=storylines_faq}{that
    seems likely, for at least several months.} There have been
    frightening accounts of people suffering what seems to be a second
    bout of Covid-19. But experts say these patients may have a
    drawn-out course of infection, with the virus taking a slow toll
    weeks to months after initial exposure. People infected with the
    coronavirus typically
    \href{https://www.nature.com/articles/s41586-020-2456-9}{produce}
    immune molecules called antibodies, which are
    \href{https://www.nytimes3xbfgragh.onion/2020/05/07/health/coronavirus-antibody-prevalence.html?action=click\&pgtype=Article\&state=default\&region=MAIN_CONTENT_3\&context=storylines_faq}{protective
    proteins made in response to an
    infection}\href{https://www.nytimes3xbfgragh.onion/2020/05/07/health/coronavirus-antibody-prevalence.html?action=click\&pgtype=Article\&state=default\&region=MAIN_CONTENT_3\&context=storylines_faq}{.
    These antibodies may} last in the body
    \href{https://www.nature.com/articles/s41591-020-0965-6}{only two to
    three months}, which may seem worrisome, but that's perfectly normal
    after an acute infection subsides, said Dr. Michael Mina, an
    immunologist at Harvard University. It may be possible to get the
    coronavirus again, but it's highly unlikely that it would be
    possible in a short window of time from initial infection or make
    people sicker the second time.
  \end{itemize}
\item ~
  \hypertarget{im-a-small-business-owner-can-i-get-relief}{%
  \paragraph{I'm a small-business owner. Can I get
  relief?}\label{im-a-small-business-owner-can-i-get-relief}}

  \begin{itemize}
  \tightlist
  \item
    The
    \href{https://www.nytimes3xbfgragh.onion/article/small-business-loans-stimulus-grants-freelancers-coronavirus.html?action=click\&pgtype=Article\&state=default\&region=MAIN_CONTENT_3\&context=storylines_faq}{stimulus
    bills enacted in March} offer help for the millions of American
    small businesses. Those eligible for aid are businesses and
    nonprofit organizations with fewer than 500 workers, including sole
    proprietorships, independent contractors and freelancers. Some
    larger companies in some industries are also eligible. The help
    being offered, which is being managed by the Small Business
    Administration, includes the Paycheck Protection Program and the
    Economic Injury Disaster Loan program. But lots of folks have
    \href{https://www.nytimes3xbfgragh.onion/interactive/2020/05/07/business/small-business-loans-coronavirus.html?action=click\&pgtype=Article\&state=default\&region=MAIN_CONTENT_3\&context=storylines_faq}{not
    yet seen payouts.} Even those who have received help are confused:
    The rules are draconian, and some are stuck sitting on
    \href{https://www.nytimes3xbfgragh.onion/2020/05/02/business/economy/loans-coronavirus-small-business.html?action=click\&pgtype=Article\&state=default\&region=MAIN_CONTENT_3\&context=storylines_faq}{money
    they don't know how to use.} Many small-business owners are getting
    less than they expected or
    \href{https://www.nytimes3xbfgragh.onion/2020/06/10/business/Small-business-loans-ppp.html?action=click\&pgtype=Article\&state=default\&region=MAIN_CONTENT_3\&context=storylines_faq}{not
    hearing anything at all.}
  \end{itemize}
\item ~
  \hypertarget{what-are-my-rights-if-i-am-worried-about-going-back-to-work}{%
  \paragraph{What are my rights if I am worried about going back to
  work?}\label{what-are-my-rights-if-i-am-worried-about-going-back-to-work}}

  \begin{itemize}
  \tightlist
  \item
    Employers have to provide
    \href{https://www.osha.gov/SLTC/covid-19/standards.html}{a safe
    workplace} with policies that protect everyone equally.
    \href{https://www.nytimes3xbfgragh.onion/article/coronavirus-money-unemployment.html?action=click\&pgtype=Article\&state=default\&region=MAIN_CONTENT_3\&context=storylines_faq}{And
    if one of your co-workers tests positive for the coronavirus, the
    C.D.C.} has said that
    \href{https://www.cdc.gov/coronavirus/2019-ncov/community/guidance-business-response.html}{employers
    should tell their employees} -\/- without giving you the sick
    employee's name -\/- that they may have been exposed to the virus.
  \end{itemize}
\item ~
  \hypertarget{what-is-school-going-to-look-like-in-september}{%
  \paragraph{What is school going to look like in
  September?}\label{what-is-school-going-to-look-like-in-september}}

  \begin{itemize}
  \tightlist
  \item
    It is unlikely that many schools will return to a normal schedule
    this fall, requiring the grind of
    \href{https://www.nytimes3xbfgragh.onion/2020/06/05/us/coronavirus-education-lost-learning.html?action=click\&pgtype=Article\&state=default\&region=MAIN_CONTENT_3\&context=storylines_faq}{online
    learning},
    \href{https://www.nytimes3xbfgragh.onion/2020/05/29/us/coronavirus-child-care-centers.html?action=click\&pgtype=Article\&state=default\&region=MAIN_CONTENT_3\&context=storylines_faq}{makeshift
    child care} and
    \href{https://www.nytimes3xbfgragh.onion/2020/06/03/business/economy/coronavirus-working-women.html?action=click\&pgtype=Article\&state=default\&region=MAIN_CONTENT_3\&context=storylines_faq}{stunted
    workdays} to continue. California's two largest public school
    districts --- Los Angeles and San Diego --- said on July 13, that
    \href{https://www.nytimes3xbfgragh.onion/2020/07/13/us/lausd-san-diego-school-reopening.html?action=click\&pgtype=Article\&state=default\&region=MAIN_CONTENT_3\&context=storylines_faq}{instruction
    will be remote-only in the fall}, citing concerns that surging
    coronavirus infections in their areas pose too dire a risk for
    students and teachers. Together, the two districts enroll some
    825,000 students. They are the largest in the country so far to
    abandon plans for even a partial physical return to classrooms when
    they reopen in August. For other districts, the solution won't be an
    all-or-nothing approach.
    \href{https://bioethics.jhu.edu/research-and-outreach/projects/eschool-initiative/school-policy-tracker/}{Many
    systems}, including the nation's largest, New York City, are
    devising
    \href{https://www.nytimes3xbfgragh.onion/2020/06/26/us/coronavirus-schools-reopen-fall.html?action=click\&pgtype=Article\&state=default\&region=MAIN_CONTENT_3\&context=storylines_faq}{hybrid
    plans} that involve spending some days in classrooms and other days
    online. There's no national policy on this yet, so check with your
    municipal school system regularly to see what is happening in your
    community.
  \end{itemize}
\end{itemize}

The industry wanted either Mr. Cuomo or Mr. de Blasio to order a
closing, because of a widespread understanding that the shows' insurance
policies would provide coverage only if a closing were
government-mandated.

And on Thursday afternoon, Mr. Cuomo obliged, ordering
\href{https://www.governor.ny.gov/news/during-novel-coronavirus-briefing-governor-cuomo-announces-new-mass-gatherings-regulations}{an
end to all gatherings of more than 500 people}. That encompassed all 41
Broadway theaters --- by definition, Broadway theaters must have more
than 500 seats, and most have more than 1,000.

Mr. Cuomo said the Broadway restriction would go into effect at 5 p.m.
on Thursday, forcing a temporary end to the runs of all 31 plays and
musicals currently in progress, from crowd favorites like ``Hamilton''
and ``The Lion King'' to new musicals like
``\href{https://www.nytimes3xbfgragh.onion/2020/02/27/theater/six-broadway.html}{Six},''
which had been scheduled to open Thursday night. Signs went up at many
theaters with information about refunds (often automatic for those who
bought with credit cards from official theater sites) and exchanges.

Mr. de Blasio, speaking at his own briefing, said the restriction was
necessary but difficult. ``That's really, really painful for the many,
many people who work in that field, let alone so many New Yorkers and
people all over the country who really look forward to these events,
these concerts, these sports events, and it's really going to be kind of
a hole in our lives and it's painful,'' he said. ``It's not something we
would ever want to do but it's something we have to do.''

\includegraphics{https://static01.graylady3jvrrxbe.onion/images/2020/03/12/arts/12virus-broadway-5/12virus-broadway-5-articleLarge-v2.jpg?quality=75\&auto=webp\&disable=upscale}

Broadway theaters did not close for the 1918 flu pandemic. But in more
recent years they have shut down for labor disruptions,
\href{https://nyti.ms/1NtiI84}{storms}, and, on Sept. 11, terrorist
attacks. Most of the closings were short (Broadway, urged to reopen by
city officials, was back in business two days after the 2001 attacks),
but many theaters were shut for 19 days by
\href{https://www.nytimes3xbfgragh.onion/2007/11/29/theater/29broadway.html}{a
stagehands' strike} in 2007 and 25 days for
\href{https://www.nytimes3xbfgragh.onion/1975/10/13/archives/musicians-and-producers-ratify-a-3year-pact-musicians-and-producers.html}{a
musicians' strike} in 1975.

The suspension was announced while one Broadway matinee, for the
long-running ``The Phantom of the Opera,'' was being performed. (Another
matinee scheduled Thursday afternoon, for
\href{https://www.nytimes3xbfgragh.onion/2019/07/25/theater/moulin-rouge-review.html?searchResultPosition=2}{``Moulin
Rouge!''}, had been canceled because a cast member was feeling sick and
the company had become alarmed.)

Inside the lobby of the Majestic Theater, where ``Phantom'' has been
running since 1988, workers sat sullenly, knowing that the performance
would be their last until April. Jim McIntosh, a Broadway bartender,
served his last drink at the matinee, and said he had mixed feelings.
``Even though it hurts me financially, it's kind of a relief,'' he said.
``It's better to be safe than sorry.''

The suspension comes at a difficult time for Broadway, which had 16
openings scheduled between Thursday (``Six'') and April 23 (``Take Me
Out''). The latter date is currently the deadline for shows to open to
be eligible for the June 7 Tony Awards. With the suspension of
performances, industry officials acknowledge that the deadline and the
date of the awards ceremony might have to change.

Given the length of the shutdown, some shows might opt to cut their
losses and never open; industry officials said they also expected that
some running shows --- those already experiencing box office weakness
--- would close rather than try to weather a shutdown and reopen.

The cancellations will invariably be disappointing for tourists and
locals who rely on Broadway for inspiration, entertainment, and
artistry. On Thursday afternoon, two visitors from Brazil, Mariana
Marinho and Barbara Anderaos, popped into the Broadhurst Theater to ask
about tickets for ``\href{https://nyti.ms/34UocKB}{Jagged Little
Pill},'' the musical built around Alanis Morissette's songs. Ms.
Marinho, 36, had loved Ms. Morissette's music since she was a teenager,
and the friends had over a week left to explore New York culture.

As they inquired about seating and ticket prices, a box office employee
broke the news to them: No Broadway shows until April 12.

``I have never seen a Broadway show, so I thought maybe this was my
chance,'' Ms. Marinho said. ``I guess not.''

Julia Jacobs contributed reporting.

Advertisement

\protect\hyperlink{after-bottom}{Continue reading the main story}

\hypertarget{site-index}{%
\subsection{Site Index}\label{site-index}}

\hypertarget{site-information-navigation}{%
\subsection{Site Information
Navigation}\label{site-information-navigation}}

\begin{itemize}
\tightlist
\item
  \href{https://help.nytimes3xbfgragh.onion/hc/en-us/articles/115014792127-Copyright-notice}{©~2020~The
  New York Times Company}
\end{itemize}

\begin{itemize}
\tightlist
\item
  \href{https://www.nytco.com/}{NYTCo}
\item
  \href{https://help.nytimes3xbfgragh.onion/hc/en-us/articles/115015385887-Contact-Us}{Contact
  Us}
\item
  \href{https://www.nytco.com/careers/}{Work with us}
\item
  \href{https://nytmediakit.com/}{Advertise}
\item
  \href{http://www.tbrandstudio.com/}{T Brand Studio}
\item
  \href{https://www.nytimes3xbfgragh.onion/privacy/cookie-policy\#how-do-i-manage-trackers}{Your
  Ad Choices}
\item
  \href{https://www.nytimes3xbfgragh.onion/privacy}{Privacy}
\item
  \href{https://help.nytimes3xbfgragh.onion/hc/en-us/articles/115014893428-Terms-of-service}{Terms
  of Service}
\item
  \href{https://help.nytimes3xbfgragh.onion/hc/en-us/articles/115014893968-Terms-of-sale}{Terms
  of Sale}
\item
  \href{https://spiderbites.nytimes3xbfgragh.onion}{Site Map}
\item
  \href{https://help.nytimes3xbfgragh.onion/hc/en-us}{Help}
\item
  \href{https://www.nytimes3xbfgragh.onion/subscription?campaignId=37WXW}{Subscriptions}
\end{itemize}
