Sections

SEARCH

\protect\hyperlink{site-content}{Skip to
content}\protect\hyperlink{site-index}{Skip to site index}

\href{https://www.nytimes3xbfgragh.onion/section/health}{Health}

\href{https://myaccount.nytimes3xbfgragh.onion/auth/login?response_type=cookie\&client_id=vi}{}

\href{https://www.nytimes3xbfgragh.onion/section/todayspaper}{Today's
Paper}

\href{/section/health}{Health}\textbar{}Doctors and Patients Turn to
Telemedicine in the Coronavirus Outbreak

\url{https://nyti.ms/2Q6nYKN}

\begin{itemize}
\item
\item
\item
\item
\item
\end{itemize}

\href{https://www.nytimes3xbfgragh.onion/news-event/coronavirus?action=click\&pgtype=Article\&state=default\&region=TOP_BANNER\&context=storylines_menu}{The
Coronavirus Outbreak}

\begin{itemize}
\tightlist
\item
  live\href{https://www.nytimes3xbfgragh.onion/2020/08/03/world/coronavirus-covid-19.html?action=click\&pgtype=Article\&state=default\&region=TOP_BANNER\&context=storylines_menu}{Latest
  Updates}
\item
  \href{https://www.nytimes3xbfgragh.onion/interactive/2020/us/coronavirus-us-cases.html?action=click\&pgtype=Article\&state=default\&region=TOP_BANNER\&context=storylines_menu}{Maps
  and Cases}
\item
  \href{https://www.nytimes3xbfgragh.onion/interactive/2020/science/coronavirus-vaccine-tracker.html?action=click\&pgtype=Article\&state=default\&region=TOP_BANNER\&context=storylines_menu}{Vaccine
  Tracker}
\item
  \href{https://www.nytimes3xbfgragh.onion/2020/08/02/us/covid-college-reopening.html?action=click\&pgtype=Article\&state=default\&region=TOP_BANNER\&context=storylines_menu}{College
  Reopening}
\item
  \href{https://www.nytimes3xbfgragh.onion/live/2020/08/03/business/stock-market-today-coronavirus?action=click\&pgtype=Article\&state=default\&region=TOP_BANNER\&context=storylines_menu}{Economy}
\end{itemize}

Advertisement

\protect\hyperlink{after-top}{Continue reading the main story}

Supported by

\protect\hyperlink{after-sponsor}{Continue reading the main story}

\hypertarget{doctors-and-patients-turn-to-telemedicine-in-the-coronavirus-outbreak}{%
\section{Doctors and Patients Turn to Telemedicine in the Coronavirus
Outbreak}\label{doctors-and-patients-turn-to-telemedicine-in-the-coronavirus-outbreak}}

The use of virtual visits climbs as a way of safely treating patients
and containing spread of the infection at hospitals, clinics and medical
offices.

\includegraphics{https://static01.graylady3jvrrxbe.onion/images/2020/03/09/science/09VIRUS-TELEMED1/merlin_170137500_34769c38-2894-4bac-8359-3166a1f9c376-articleLarge.jpg?quality=75\&auto=webp\&disable=upscale}

\href{https://www.nytimes3xbfgragh.onion/by/reed-abelson}{\includegraphics{https://static01.graylady3jvrrxbe.onion/images/2018/07/16/multimedia/author-reed-abelson/author-reed-abelson-thumbLarge.png}}

By \href{https://www.nytimes3xbfgragh.onion/by/reed-abelson}{Reed
Abelson}

\begin{itemize}
\item
  March 11, 2020
\item
  \begin{itemize}
  \item
  \item
  \item
  \item
  \item
  \end{itemize}
\end{itemize}

The man had recently traveled, including a brief stop in Tokyo. He had a
fever and cough about a week ago, but was now feeling fine.

He called the virtual medical line set up by Rush University Medical
Center in Chicago recently to help screen patients for coronavirus.

``He said all the right buzzwords: cough, fever, fatigue,'' said Dr.
Meeta Shah, an emergency room physician at Rush.

After talking with him, Dr. Shah did not think he needed to be admitted
but referred him to the city's health department.

Rush and other large hospitals across the country are quickly expanding
the use of telemedicine to safely screen and treat patients for
coronavirus, and to try to contain the spread of infection while
offering remote services.

``This is a kind of turning point for virtual health,'' Dr. Shah said.
``We're actually seeing how it can be used in a public health crisis.''

While the notion of seeing a doctor via your computer or cellphone is
hardly new, telemedicine has yet to take off widely in the United
States. Health insurance plans do typically offer people the option of
talking to a nurse or doctor online as an alternative to heading to an
emergency room or urgent care center, but most people don't make use of
it. Now doctors, hospital networks and clinics are rethinking how the
technology can be used, to keep the worried well calm and away from
clinical care while steering the most at risk to the proper treatment.

\hypertarget{latest-updates-global-coronavirus-outbreak}{%
\section{\texorpdfstring{\href{https://www.nytimes3xbfgragh.onion/2020/08/03/world/coronavirus-covid-19.html?action=click\&pgtype=Article\&state=default\&region=MAIN_CONTENT_1\&context=storylines_live_updates}{Latest
Updates: Global Coronavirus
Outbreak}}{Latest Updates: Global Coronavirus Outbreak}}\label{latest-updates-global-coronavirus-outbreak}}

Updated 2020-08-04T07:33:06.428Z

\begin{itemize}
\tightlist
\item
  \href{https://www.nytimes3xbfgragh.onion/2020/08/03/world/coronavirus-covid-19.html?action=click\&pgtype=Article\&state=default\&region=MAIN_CONTENT_1\&context=storylines_live_updates\#link-4547638f}{Fauci
  defends Birx after she is criticized by Trump.}
\item
  \href{https://www.nytimes3xbfgragh.onion/2020/08/03/world/coronavirus-covid-19.html?action=click\&pgtype=Article\&state=default\&region=MAIN_CONTENT_1\&context=storylines_live_updates\#link-15e7f995}{Trump
  derides Democrats as lawmakers and administration officials try to
  break stimulus impasse.}
\item
  \href{https://www.nytimes3xbfgragh.onion/2020/08/03/world/coronavirus-covid-19.html?action=click\&pgtype=Article\&state=default\&region=MAIN_CONTENT_1\&context=storylines_live_updates\#link-e5a2cda}{The
  deadline for 2020 census counting has been moved up by a month.}
\end{itemize}

\href{https://www.nytimes3xbfgragh.onion/2020/08/03/world/coronavirus-covid-19.html?action=click\&pgtype=Article\&state=default\&region=MAIN_CONTENT_1\&context=storylines_live_updates}{See
more updates}

More live coverage:
\href{https://www.nytimes3xbfgragh.onion/live/2020/08/03/business/stock-market-today-coronavirus?action=click\&pgtype=Article\&state=default\&region=MAIN_CONTENT_1\&context=storylines_live_updates}{Markets}

``The use of telemedicine is going to be critical for management of this
pandemic,'' said Dr. Stephen Parodi, an infectious disease specialist
and executive with The Permanente Medical Group, the doctors' group
associated with Kaiser Permanente, one of the leaders in the use of
virtual visits for its patients.

Telemedicine got an additional boost
\href{https://www.nytimes3xbfgragh.onion/2020/03/04/us/politics/coronavirus-emergency-aid-congress.html}{under
the \$8.3 billion emergency funding measure from Congress}, which
loosened restrictions on its use to treat people covered under the
federal Medicare program. At a news conference on Monday, Seema Verma,
the administrator of the Centers for Medicare and Medicaid Services,
praised the government's efforts to expand the use of telemedicine under
Medicare, the federal program for people 65 and older.

In a meeting on Tuesday at the White House with President Trump, private
health insurers also said they would pay for the virtual visits for
people who may have coronavirus to improve access to care for their
customers.

By using their phone or computer, patients will be able to get guidance
about whether they need to be seen or tested instead of showing up
unannounced at the emergency room or doctor's office. Patients,
particularly those who would be at high risk for a serious illness if
they were infected, can also opt to substitute a trip to a doctor's
office with a virtual visit when it is a routine check in with a
specialist or a primary care doctor. That way they can avoid crowded
waiting rooms and potential infection.

When Rush admitted a student last week who was believed to have the
virus, the hospital was able to prepare for his arrival by clearing the
ambulance bay of people and vehicles to protect patients and hospital
staff from possible infection. Taken to an isolation room, he was
examined by Dr. Paul Casey, an emergency room physician, and a nurse,
both in protective gear.

An infectious disease specialist was consulted over an iPad. The
patient, who did have the virus, was released last Friday, and Rush was
able to avoid the fate of other hospitals in the United States,
\href{https://www.nytimes3xbfgragh.onion/2020/03/05/us/coronavirus-nurses.html}{where
patients with Covid-19 led to the widespread quarantine of health care
workers}.

``When the news of coronavirus broke last month, we saw the
opportunity,'' Dr. Casey said.

Health systems are racing to adapt and even develop virtual services
that can serve as their front line for patients. ``Telehealth is being
rediscovered,'' said Dr. Peter Antall, the chief medical officer for
AmWell, a company based in Boston that is working with health systems
across the country. ``Everybody recognizes this is an all hands on deck
moment,'' he said. ``We need to scale up wherever we can.''

\includegraphics{https://static01.graylady3jvrrxbe.onion/images/2020/03/09/science/09VIRUS-TELEMED2/merlin_170137467_57c7e007-036b-428b-9d10-8eec2a01528e-articleLarge.jpg?quality=75\&auto=webp\&disable=upscale}

Other systems are also readying their telemedicine offerings. ``The
Covid-19 outbreak is going to serve as an impetus,'' said Dr. Shabana
Khan, the director of telepsychiatry at NYU Langone Health. ``We have no
choice.''

Patients concerned about the coronavirus are being directed to NYU's
virtual urgent care, which they can gain access to via their phone or a
computer.

``Our volumes are showing they are hearing that message loud and
clear,'' said Dr. Paul A. Testa, an emergency medicine doctor who is the
system's chief medical information officer.

NYU is also encouraging its doctors who are self-quarantined because of
recent travel to see patients using video, as well as directing patients
who are particularly vulnerable because of existing medical conditions
to consider a virtual visit instead of heading to a doctor's office.

\href{https://www.nytimes3xbfgragh.onion/news-event/coronavirus?action=click\&pgtype=Article\&state=default\&region=MAIN_CONTENT_3\&context=storylines_faq}{}

\hypertarget{the-coronavirus-outbreak-}{%
\subsubsection{The Coronavirus Outbreak
›}\label{the-coronavirus-outbreak-}}

\hypertarget{frequently-asked-questions}{%
\paragraph{Frequently Asked
Questions}\label{frequently-asked-questions}}

Updated August 3, 2020

\begin{itemize}
\item ~
  \hypertarget{im-a-small-business-owner-can-i-get-relief}{%
  \paragraph{I'm a small-business owner. Can I get
  relief?}\label{im-a-small-business-owner-can-i-get-relief}}

  \begin{itemize}
  \tightlist
  \item
    The
    \href{https://www.nytimes3xbfgragh.onion/article/small-business-loans-stimulus-grants-freelancers-coronavirus.html?action=click\&pgtype=Article\&state=default\&region=MAIN_CONTENT_3\&context=storylines_faq}{stimulus
    bills enacted in March} offer help for the millions of American
    small businesses. Those eligible for aid are businesses and
    nonprofit organizations with fewer than 500 workers, including sole
    proprietorships, independent contractors and freelancers. Some
    larger companies in some industries are also eligible. The help
    being offered, which is being managed by the Small Business
    Administration, includes the Paycheck Protection Program and the
    Economic Injury Disaster Loan program. But lots of folks have
    \href{https://www.nytimes3xbfgragh.onion/interactive/2020/05/07/business/small-business-loans-coronavirus.html?action=click\&pgtype=Article\&state=default\&region=MAIN_CONTENT_3\&context=storylines_faq}{not
    yet seen payouts.} Even those who have received help are confused:
    The rules are draconian, and some are stuck sitting on
    \href{https://www.nytimes3xbfgragh.onion/2020/05/02/business/economy/loans-coronavirus-small-business.html?action=click\&pgtype=Article\&state=default\&region=MAIN_CONTENT_3\&context=storylines_faq}{money
    they don't know how to use.} Many small-business owners are getting
    less than they expected or
    \href{https://www.nytimes3xbfgragh.onion/2020/06/10/business/Small-business-loans-ppp.html?action=click\&pgtype=Article\&state=default\&region=MAIN_CONTENT_3\&context=storylines_faq}{not
    hearing anything at all.}
  \end{itemize}
\item ~
  \hypertarget{what-are-my-rights-if-i-am-worried-about-going-back-to-work}{%
  \paragraph{What are my rights if I am worried about going back to
  work?}\label{what-are-my-rights-if-i-am-worried-about-going-back-to-work}}

  \begin{itemize}
  \tightlist
  \item
    Employers have to provide
    \href{https://www.osha.gov/SLTC/covid-19/standards.html}{a safe
    workplace} with policies that protect everyone equally.
    \href{https://www.nytimes3xbfgragh.onion/article/coronavirus-money-unemployment.html?action=click\&pgtype=Article\&state=default\&region=MAIN_CONTENT_3\&context=storylines_faq}{And
    if one of your co-workers tests positive for the coronavirus, the
    C.D.C.} has said that
    \href{https://www.cdc.gov/coronavirus/2019-ncov/community/guidance-business-response.html}{employers
    should tell their employees} -\/- without giving you the sick
    employee's name -\/- that they may have been exposed to the virus.
  \end{itemize}
\item ~
  \hypertarget{should-i-refinance-my-mortgage}{%
  \paragraph{Should I refinance my
  mortgage?}\label{should-i-refinance-my-mortgage}}

  \begin{itemize}
  \tightlist
  \item
    \href{https://www.nytimes3xbfgragh.onion/article/coronavirus-money-unemployment.html?action=click\&pgtype=Article\&state=default\&region=MAIN_CONTENT_3\&context=storylines_faq}{It
    could be a good idea,} because mortgage rates have
    \href{https://www.nytimes3xbfgragh.onion/2020/07/16/business/mortgage-rates-below-3-percent.html?action=click\&pgtype=Article\&state=default\&region=MAIN_CONTENT_3\&context=storylines_faq}{never
    been lower.} Refinancing requests have pushed mortgage applications
    to some of the highest levels since 2008, so be prepared to get in
    line. But defaults are also up, so if you're thinking about buying a
    home, be aware that some lenders have tightened their standards.
  \end{itemize}
\item ~
  \hypertarget{what-is-school-going-to-look-like-in-september}{%
  \paragraph{What is school going to look like in
  September?}\label{what-is-school-going-to-look-like-in-september}}

  \begin{itemize}
  \tightlist
  \item
    It is unlikely that many schools will return to a normal schedule
    this fall, requiring the grind of
    \href{https://www.nytimes3xbfgragh.onion/2020/06/05/us/coronavirus-education-lost-learning.html?action=click\&pgtype=Article\&state=default\&region=MAIN_CONTENT_3\&context=storylines_faq}{online
    learning},
    \href{https://www.nytimes3xbfgragh.onion/2020/05/29/us/coronavirus-child-care-centers.html?action=click\&pgtype=Article\&state=default\&region=MAIN_CONTENT_3\&context=storylines_faq}{makeshift
    child care} and
    \href{https://www.nytimes3xbfgragh.onion/2020/06/03/business/economy/coronavirus-working-women.html?action=click\&pgtype=Article\&state=default\&region=MAIN_CONTENT_3\&context=storylines_faq}{stunted
    workdays} to continue. California's two largest public school
    districts --- Los Angeles and San Diego --- said on July 13, that
    \href{https://www.nytimes3xbfgragh.onion/2020/07/13/us/lausd-san-diego-school-reopening.html?action=click\&pgtype=Article\&state=default\&region=MAIN_CONTENT_3\&context=storylines_faq}{instruction
    will be remote-only in the fall}, citing concerns that surging
    coronavirus infections in their areas pose too dire a risk for
    students and teachers. Together, the two districts enroll some
    825,000 students. They are the largest in the country so far to
    abandon plans for even a partial physical return to classrooms when
    they reopen in August. For other districts, the solution won't be an
    all-or-nothing approach.
    \href{https://bioethics.jhu.edu/research-and-outreach/projects/eschool-initiative/school-policy-tracker/}{Many
    systems}, including the nation's largest, New York City, are
    devising
    \href{https://www.nytimes3xbfgragh.onion/2020/06/26/us/coronavirus-schools-reopen-fall.html?action=click\&pgtype=Article\&state=default\&region=MAIN_CONTENT_3\&context=storylines_faq}{hybrid
    plans} that involve spending some days in classrooms and other days
    online. There's no national policy on this yet, so check with your
    municipal school system regularly to see what is happening in your
    community.
  \end{itemize}
\item ~
  \hypertarget{is-the-coronavirus-airborne}{%
  \paragraph{Is the coronavirus
  airborne?}\label{is-the-coronavirus-airborne}}

  \begin{itemize}
  \tightlist
  \item
    The coronavirus
    \href{https://www.nytimes3xbfgragh.onion/2020/07/04/health/239-experts-with-one-big-claim-the-coronavirus-is-airborne.html?action=click\&pgtype=Article\&state=default\&region=MAIN_CONTENT_3\&context=storylines_faq}{can
    stay aloft for hours in tiny droplets in stagnant air}, infecting
    people as they inhale, mounting scientific evidence suggests. This
    risk is highest in crowded indoor spaces with poor ventilation, and
    may help explain super-spreading events reported in meatpacking
    plants, churches and restaurants.
    \href{https://www.nytimes3xbfgragh.onion/2020/07/06/health/coronavirus-airborne-aerosols.html?action=click\&pgtype=Article\&state=default\&region=MAIN_CONTENT_3\&context=storylines_faq}{It's
    unclear how often the virus is spread} via these tiny droplets, or
    aerosols, compared with larger droplets that are expelled when a
    sick person coughs or sneezes, or transmitted through contact with
    contaminated surfaces, said Linsey Marr, an aerosol expert at
    Virginia Tech. Aerosols are released even when a person without
    symptoms exhales, talks or sings, according to Dr. Marr and more
    than 200 other experts, who
    \href{https://academic.oup.com/cid/article/doi/10.1093/cid/ciaa939/5867798}{have
    outlined the evidence in an open letter to the World Health
    Organization}.
  \end{itemize}
\end{itemize}

But Dr. Testa emphasized that patients who need to be seen in person
should not hesitate to seek care. ``We're not discouraging anybody from
coming in,'' he said.

Virtual care has its limits, of course, and many of the start-ups and
others promoting their offerings may not be fully equipped to handle
patients who might have the virus. At Zoom+Care, a chain of clinics in
Oregon and Washington, consumers are being encouraged to use the
company's online chat feature so that their risks can be assessed.

``We're being very explicit at Zoom+Care that we can't test you for
Covid-19,'' said Dr. Mark Zeitzer, who is the clinics' medical director
of acute care services. Instead, people may be told to self-quarantine
and keep a careful eye on their symptoms.

But the idea of using telemedicine to prevent further spread of the
virus is being adopted quickly. At Intermountain Healthcare, the Utah
system that cared for an infected patient at its Salt Lake City
hospital, the concern over a potential measles outbreak last year led
executives to consider how to better protect the community from
infectious diseases.

``When coronavirus hit the streets, we took the measles work-flow and
expanded on it,'' said Kerry Palakanis, a nurse practitioner who is the
executive director of Intermountain's initiative, Connect Care.

The system is also thinking about how it can use the same technology to
deliver home health care, particularly for patients who are at high risk
because of chronic medical conditions or have Covid-19 but can be
treated safely at home. People at home could be equipped to take their
blood pressure or test their blood sugars, and a doctor or nurse could
be available over video.

By monitoring more patients virtually, Intermountain will be able to
limit the potential exposure of nurses who conduct home visits. ``Those
nurses are traveling out throughout the community,'' Dr. Palakanis said.

Telemedicine companies say they are getting an increase in the number of
calls, both from those who want to know more about what they can do to
minimize their risk of catching coronavirus and those with worrisome
symptoms. ``We see the whole spectrum of patients,'' said Dr. Kristin
Dean, medical director for Doctor On Demand, a company whose service is
offered to customers of some of the major health insurance companies.

In evaluating whether patients may be safely monitored at home, doctors
take into account people's medical history and the severity of their
symptoms, she said.

``The patients have been appreciative of that switch,'' said Dr. Parodi
of Permanente. ``Many of them don't want to come in and be exposed in a
clinic or office setting.''

Advertisement

\protect\hyperlink{after-bottom}{Continue reading the main story}

\hypertarget{site-index}{%
\subsection{Site Index}\label{site-index}}

\hypertarget{site-information-navigation}{%
\subsection{Site Information
Navigation}\label{site-information-navigation}}

\begin{itemize}
\tightlist
\item
  \href{https://help.nytimes3xbfgragh.onion/hc/en-us/articles/115014792127-Copyright-notice}{©~2020~The
  New York Times Company}
\end{itemize}

\begin{itemize}
\tightlist
\item
  \href{https://www.nytco.com/}{NYTCo}
\item
  \href{https://help.nytimes3xbfgragh.onion/hc/en-us/articles/115015385887-Contact-Us}{Contact
  Us}
\item
  \href{https://www.nytco.com/careers/}{Work with us}
\item
  \href{https://nytmediakit.com/}{Advertise}
\item
  \href{http://www.tbrandstudio.com/}{T Brand Studio}
\item
  \href{https://www.nytimes3xbfgragh.onion/privacy/cookie-policy\#how-do-i-manage-trackers}{Your
  Ad Choices}
\item
  \href{https://www.nytimes3xbfgragh.onion/privacy}{Privacy}
\item
  \href{https://help.nytimes3xbfgragh.onion/hc/en-us/articles/115014893428-Terms-of-service}{Terms
  of Service}
\item
  \href{https://help.nytimes3xbfgragh.onion/hc/en-us/articles/115014893968-Terms-of-sale}{Terms
  of Sale}
\item
  \href{https://spiderbites.nytimes3xbfgragh.onion}{Site Map}
\item
  \href{https://help.nytimes3xbfgragh.onion/hc/en-us}{Help}
\item
  \href{https://www.nytimes3xbfgragh.onion/subscription?campaignId=37WXW}{Subscriptions}
\end{itemize}
