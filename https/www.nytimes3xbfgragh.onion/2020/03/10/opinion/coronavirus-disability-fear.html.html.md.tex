Sections

SEARCH

\protect\hyperlink{site-content}{Skip to
content}\protect\hyperlink{site-index}{Skip to site index}

\href{https://myaccount.nytimes3xbfgragh.onion/auth/login?response_type=cookie\&client_id=vi}{}

\href{https://www.nytimes3xbfgragh.onion/section/todayspaper}{Today's
Paper}

\href{/section/opinion}{Opinion}\textbar{}Afraid of Coronavirus? I Know
What That Fear Is Like

\url{https://nyti.ms/2wFxdei}

\begin{itemize}
\item
\item
\item
\item
\item
\item
\end{itemize}

Advertisement

\protect\hyperlink{after-top}{Continue reading the main story}

\href{/section/opinion}{Opinion}

Supported by

\protect\hyperlink{after-sponsor}{Continue reading the main story}

disability

\hypertarget{afraid-of-coronavirus-i-know-what-that-fear-is-like}{%
\section{Afraid of Coronavirus? I Know What That Fear Is
Like}\label{afraid-of-coronavirus-i-know-what-that-fear-is-like}}

I live with chronic illness. Here is my advice for keeping calm in a
health crisis.

By Kendall Ciesemier

Ms. Ciesemier is the co-host of the health podcast ``That That Don't
Kill Me.''

\begin{itemize}
\item
  March 10, 2020
\item
  \begin{itemize}
  \item
  \item
  \item
  \item
  \item
  \item
  \end{itemize}
\end{itemize}

\includegraphics{https://static01.graylady3jvrrxbe.onion/images/2020/03/10/opinion/10disability-ciesemir/10disability-ciesemir-articleLarge.jpg?quality=75\&auto=webp\&disable=upscale}

I am standing near the door of the subway car as the train on the
Seventh Avenue line comes to an abrupt stop. As I reach out and grab the
metal pole in the middle of the car to steady myself, I look up and find
that all eyes are on me. They are indignant, it seems, that I grabbed
the pole with my hand after just coughing into the crook of my elbow.

My gaze moves from face to face and it strikes me: I am no longer the
only one on this train worried about getting sick. What stuns me isn't
their disgust at my cough or their rush to grab their newly purchased
bottles of hand sanitizer, but rather that the coronavirus outbreak
might be the first time they are fearing for their health and,
subsequently, for their life.

I'm only 27 years old, but I have already confronted and dealt with the
limits of my health. Within my peer group this may be rare, but I'm far
from alone. According to the National Health Council, around
\href{https://nationalhealthcouncil.org/wp-content/uploads/2019/12/AboutChronicDisease.pdf}{133
million Americans live with incurable or chronic diseases} --- that's 40
percent of the public. Our lesson for the corona-concerned is modeled in
our daily actions: we still get out of bed and move through life even
though we often meet new and unpredictable threats to our health. In
many ways, to us, nothing has changed.

\emph{{[}Watch Kendall Ciesemier's}
\href{https://www.instagram.com/p/B9keeraBm_N/}{\emph{Instagram video}}
\emph{on coping with the fear of coronavirus.{]}}

I was born with a rare liver disease resulting in the need for two liver
transplants, both when I was 11 years old. As someone who has lived an
entire life with a chronic illness and survived some steep mortality
odds, I have plenty of empathy for what it's like to question your
safety in the world. It's the invisible weight I carry with me, always.

Packing bags was one way my family prepared for the unexpected but
frequent health emergencies after my transplants. We kept our suitcases
packed and lying around the house, just in case we had to get to the
hospital quickly. When I look at what seems like the world's version of
my very personal journey with health anxiety, I feel acknowledged in a
way that I have not experienced before. Packing bags in case you need to
make an urgent run to the hospital is kind of like stocking up a freezer
full of what my roommate deems her ``coronafood.''

But years of living with transplant complications, including a
compromised immune system, have taught me how to live despite this fear,
how to navigate a world rife with potential danger at every turn. Trust
me, in a strange effort to keep healthier people calm, my credibility
and need for vigilance is made clear by news anchors feeding us updates:
If you are currently living without an illness or disability, my risk of
dying from the coronavirus is higher than yours.

My coronavirus advice, I should say, is also my greater life advice.

Years ago, when my doctor told me I might need to be put on the organ
list for a third transplant, I responded by leaving town the next day
for a summer internship in New York. I was not being reckless. I did
this only after scheduling a visit to meet a local doctor who could take
care of me, if the need should arise. In the days of the coronavirus, I
take the subway to and home from work, but I also diligently wash my
hands afterward. Don't alter your life in dramatic ways, but do make
sure that the risks you take are calculated ones. Be social, but don't
hang around sick people. Get your haircut, but don't touch your face. Go
out to dinner, but don't share food.

Throughout an average day in New York, I share space with hundreds of
other people, all of whom I have no control over. Should a friend come
to my apartment, should I go on a date that ends with a kiss, should I
see Billie Eilish perform when she comes to town, I will lack control
over my exposure to the virus. To respond to this reality, l do what I
normally do: make my calculations and ensure that my risks are minimal,
and then let it be. Because here's the thing: the life that I'm working
so hard to protect is not really a life at all if it is consumed by
fear.

I'm not saying I'm not afraid or that you shouldn't be. My fear is my
constant companion, but I can manage it alongside my desire to live a
full life. I can channel this fear into a boldness for the other things
that scare me. Growing up with an uncertain future, I was able to
achieve some of my wildest goals: giving a TED Talk, starting a
nonprofit and creating a summit to empower young women. Befriending my
fear is my superpower, and it turns out, that I'm not alone here either.

Last week I spoke to some others I know with illnesses that have shaped
or influenced their approaches to life. For Tonya Ingram, who lives with
Lupus, health anxiety has changed her daily decisions: ``My kidneys
failing keeps me present because I have no idea what next year will look
like for me,'' she said. ``Arguments that aren't worth the stress are
released, the extra fries are ordered, the crush is informed, the story
is told. It makes me question what is worth my energy.''

Matthew Cortland, who lives with Crohn's disease, channels his fear into
political action. He believes the system is stacked against disabled
people: ``I have to remind myself, `don't just panic, do something.' And
then I fight for change. By organizing with other chronically ill folks
and reaching out to elected officials with our concerns and needs, I
feel better, and hopefully, accomplish a bit of change that makes things
better for everyone.''

For all three of us, when the fear strikes we push it forward, change
its shape, alter its impact. We have it but we don't let it have us.
Coronavirus panic is natural, but if our fear is going to be there, we
might as well put it to use and make it work for us --- for our goals,
for our perspective and for others.

\emph{Disability is a series of essays, art and opinion by and about
people living with disabilities.}

Kendall Ciesemier (@kciesemier) is the co-host of the health podcast
``That That Don't Kill Me.''

\emph{\textbf{Now in print:}}
\emph{``}\href{https://www.aboutusbook.com/}{\emph{About Us: Essays From
the Disability Series of The New York Times}}\emph{,'' edited by Peter
Catapano and Rosemarie Garland-Thomson, published by Liveright.}

\emph{The Times is committed to publishing}
\href{https://www.nytimes3xbfgragh.onion/2019/01/31/opinion/letters/letters-to-editor-new-york-times-women.html}{\emph{a
diversity of letters}} \emph{to the editor. We'd like to hear what you
think about this or any of our articles. Here are some}
\href{https://help.nytimes3xbfgragh.onion/hc/en-us/articles/115014925288-How-to-submit-a-letter-to-the-editor}{\emph{tips}}\emph{.
And here's our email:}
\href{mailto:letters@NYTimes.com}{\emph{letters@NYTimes.com}}\emph{.}

\emph{Follow The New York Times Opinion section on}
\href{https://www.facebookcorewwwi.onion/nytopinion}{\emph{Facebook}}\emph{,}
\href{http://twitter.com/NYTOpinion}{\emph{Twitter (@NYTopinion)}}
\emph{and}
\href{https://www.instagram.com/nytopinion/}{\emph{Instagram}}\emph{.}

Advertisement

\protect\hyperlink{after-bottom}{Continue reading the main story}

\hypertarget{site-index}{%
\subsection{Site Index}\label{site-index}}

\hypertarget{site-information-navigation}{%
\subsection{Site Information
Navigation}\label{site-information-navigation}}

\begin{itemize}
\tightlist
\item
  \href{https://help.nytimes3xbfgragh.onion/hc/en-us/articles/115014792127-Copyright-notice}{©~2020~The
  New York Times Company}
\end{itemize}

\begin{itemize}
\tightlist
\item
  \href{https://www.nytco.com/}{NYTCo}
\item
  \href{https://help.nytimes3xbfgragh.onion/hc/en-us/articles/115015385887-Contact-Us}{Contact
  Us}
\item
  \href{https://www.nytco.com/careers/}{Work with us}
\item
  \href{https://nytmediakit.com/}{Advertise}
\item
  \href{http://www.tbrandstudio.com/}{T Brand Studio}
\item
  \href{https://www.nytimes3xbfgragh.onion/privacy/cookie-policy\#how-do-i-manage-trackers}{Your
  Ad Choices}
\item
  \href{https://www.nytimes3xbfgragh.onion/privacy}{Privacy}
\item
  \href{https://help.nytimes3xbfgragh.onion/hc/en-us/articles/115014893428-Terms-of-service}{Terms
  of Service}
\item
  \href{https://help.nytimes3xbfgragh.onion/hc/en-us/articles/115014893968-Terms-of-sale}{Terms
  of Sale}
\item
  \href{https://spiderbites.nytimes3xbfgragh.onion}{Site Map}
\item
  \href{https://help.nytimes3xbfgragh.onion/hc/en-us}{Help}
\item
  \href{https://www.nytimes3xbfgragh.onion/subscription?campaignId=37WXW}{Subscriptions}
\end{itemize}
