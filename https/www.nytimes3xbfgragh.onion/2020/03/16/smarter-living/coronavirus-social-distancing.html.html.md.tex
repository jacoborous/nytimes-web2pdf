Sections

SEARCH

\protect\hyperlink{site-content}{Skip to
content}\protect\hyperlink{site-index}{Skip to site index}

\href{https://www.nytimes3xbfgragh.onion/section/smarter-living}{Smarter
Living}

\href{https://myaccount.nytimes3xbfgragh.onion/auth/login?response_type=cookie\&client_id=vi}{}

\href{https://www.nytimes3xbfgragh.onion/section/todayspaper}{Today's
Paper}

\href{/section/smarter-living}{Smarter Living}\textbar{}Wondering About
Social Distancing?

\url{https://nyti.ms/33mJFvv}

\begin{itemize}
\item
\item
\item
\item
\item
\end{itemize}

\hypertarget{the-coronavirus-outbreak}{%
\subsubsection{\texorpdfstring{\href{https://www.nytimes3xbfgragh.onion/news-event/coronavirus?name=styln-coronavirus-national\&region=TOP_BANNER\&block=storyline_menu_recirc\&action=click\&pgtype=Article\&impression_id=912db9f0-efba-11ea-9307-2f39405dea1d\&variant=undefined}{The
Coronavirus
Outbreak}}{The Coronavirus Outbreak}}\label{the-coronavirus-outbreak}}

\begin{itemize}
\tightlist
\item
  live\href{https://www.nytimes3xbfgragh.onion/2020/09/05/world/coronavirus-covid.html?name=styln-coronavirus-national\&region=TOP_BANNER\&block=storyline_menu_recirc\&action=click\&pgtype=Article\&impression_id=912db9f1-efba-11ea-9307-2f39405dea1d\&variant=undefined}{Latest
  Updates}
\item
  \href{https://www.nytimes3xbfgragh.onion/interactive/2020/us/coronavirus-us-cases.html?name=styln-coronavirus-national\&region=TOP_BANNER\&block=storyline_menu_recirc\&action=click\&pgtype=Article\&impression_id=912db9f2-efba-11ea-9307-2f39405dea1d\&variant=undefined}{Maps
  and Cases}
\item
  \href{https://www.nytimes3xbfgragh.onion/interactive/2020/science/coronavirus-vaccine-tracker.html?name=styln-coronavirus-national\&region=TOP_BANNER\&block=storyline_menu_recirc\&action=click\&pgtype=Article\&impression_id=912db9f3-efba-11ea-9307-2f39405dea1d\&variant=undefined}{Vaccine
  Tracker}
\item
  \href{https://www.nytimes3xbfgragh.onion/2020/09/02/your-money/eviction-moratorium-covid.html?name=styln-coronavirus-national\&region=TOP_BANNER\&block=storyline_menu_recirc\&action=click\&pgtype=Article\&impression_id=912de100-efba-11ea-9307-2f39405dea1d\&variant=undefined}{Eviction
  Moratorium}
\item
  \href{https://www.nytimes3xbfgragh.onion/interactive/2020/09/02/magazine/food-insecurity-hunger-us.html?name=styln-coronavirus-national\&region=TOP_BANNER\&block=storyline_menu_recirc\&action=click\&pgtype=Article\&impression_id=912de101-efba-11ea-9307-2f39405dea1d\&variant=undefined}{American
  Hunger}
\end{itemize}

Advertisement

\protect\hyperlink{after-top}{Continue reading the main story}

Supported by

\protect\hyperlink{after-sponsor}{Continue reading the main story}

\hypertarget{wondering-about-social-distancing}{%
\section{Wondering About Social
Distancing?}\label{wondering-about-social-distancing}}

Answers to your most common questions about the best practices for
stemming the tide of the coronavirus pandemic.

\includegraphics{https://static01.graylady3jvrrxbe.onion/images/2020/03/16/travel/16social-distancing-coronavirus1/16social-distancing-coronavirus1-articleLarge.jpg?quality=75\&auto=webp\&disable=upscale}

By Apoorva Mandavilli

\begin{itemize}
\item
  March 16, 2020
\item
  \begin{itemize}
  \item
  \item
  \item
  \item
  \item
  \end{itemize}
\end{itemize}

On Sunday, the Centers for Disease Control and Prevention recommended
against any
\href{https://www.cdc.gov/coronavirus/2019-ncov/community/large-events/mass-gatherings-ready-for-covid-19.html}{gatherings
of 50 or more people} over the next eight weeks, in an effort to contain
the coronavirus pandemic. Many public schools, libraries, universities,
places of worship, and sporting and cultural institutions have also shut
down for at least the next few weeks. These measures are an attempt to
enforce distance between people, a proven way to slow pandemics.

Experts have also been urging people to practice voluntary
``\href{https://www.nytimes3xbfgragh.onion/2020/04/14/us/bishop-gerald-glenn-coronavirus.html}{social
distancing}.'' The term has been trending on Twitter, with even
President Trump
\href{https://twitter.com/realDonaldTrump/status/1238824050924883968}{endorsing
it} on Saturday.

Still, people all over the United States have been out in large numbers
at restaurants, bars and even sporting events, suggesting more than a
little confusion around what social distancing is and who should be
practicing it.

This is deeply worrying, experts said, because even those who become
only mildly ill --- and maybe even those who never even know they are
infected --- can propel the exponential movement of the virus through
the population.

They emphasized that it's important for everyone to practice social
distancing, not just those considered to be at high risk or who are
seriously ill.

``These are not normal times, this is not a drill,'' said Dr. Jeanne
Marrazzo, director of infectious diseases at the University of Alabama
in Birmingham. ``We have never been through anything like this before.''

What exactly is social distancing? We asked experts for practical
guidance.

\hypertarget{what-is-social-distancing}{%
\subsection{What is social
distancing?}\label{what-is-social-distancing}}

Put simply, the idea is to maintain a distance between you and other
people --- in this case, at least six feet.

That also means minimizing contact with people. Avoid public
transportation whenever possible, limit nonessential travel, work from
home and skip social gatherings --- and definitely do not go to crowded
bars and sporting arenas.

``Every single reduction in the number of contacts you have per day with
relatives, with friends, co-workers, in school will have a significant
impact on the ability of the virus to spread in the population,'' said
Dr. Gerardo Chowell, chair of population health sciences at Georgia
State University.

This strategy saved thousands of lives both during the Spanish flu
pandemic of 1918 and, more recently, in Mexico City during the 2009 flu
pandemic.

\hypertarget{latest-updates-the-coronavirus-outbreak}{%
\section{\texorpdfstring{\href{https://www.nytimes3xbfgragh.onion/2020/09/04/world/covid-19-coronavirus.html?action=click\&pgtype=Article\&state=default\&region=MAIN_CONTENT_1\&context=storylines_live_updates}{Latest
Updates: The Coronavirus
Outbreak}}{Latest Updates: The Coronavirus Outbreak}}\label{latest-updates-the-coronavirus-outbreak}}

Updated 2020-09-05T12:05:40.998Z

\begin{itemize}
\tightlist
\item
  \href{https://www.nytimes3xbfgragh.onion/2020/09/04/world/covid-19-coronavirus.html?action=click\&pgtype=Article\&state=default\&region=MAIN_CONTENT_1\&context=storylines_live_updates\#link-1654f6ad}{Research
  connects vaping to a higher chance of catching the virus --- and
  suffering its worst effects.}
\item
  \href{https://www.nytimes3xbfgragh.onion/2020/09/04/world/covid-19-coronavirus.html?action=click\&pgtype=Article\&state=default\&region=MAIN_CONTENT_1\&context=storylines_live_updates\#link-52e4198a}{Another
  college football game won't be played as planned.}
\item
  \href{https://www.nytimes3xbfgragh.onion/2020/09/04/world/covid-19-coronavirus.html?action=click\&pgtype=Article\&state=default\&region=MAIN_CONTENT_1\&context=storylines_live_updates\#link-181cef0}{Pharmaceutical
  companies plan a joint pledge on safety standards as they move
  vaccines to the marketplace.}
\end{itemize}

\href{https://www.nytimes3xbfgragh.onion/2020/09/04/world/covid-19-coronavirus.html?action=click\&pgtype=Article\&state=default\&region=MAIN_CONTENT_1\&context=storylines_live_updates}{See
more updates}

More live coverage:
\href{https://www.nytimes3xbfgragh.onion/live/2020/09/04/business/stock-market-today-coronavirus?action=click\&pgtype=Article\&state=default\&region=MAIN_CONTENT_1\&context=storylines_live_updates}{Markets}

\hypertarget{im-young-and-dont-have-any-risk-factors-can-i-continue-to-socialize}{%
\subsection{I'm young and don't have any risk factors. Can I continue to
socialize?}\label{im-young-and-dont-have-any-risk-factors-can-i-continue-to-socialize}}

Please don't. There is no question that older people and those with
underlying health conditions are most vulnerable to the virus, but young
people are by no means immune.

And there is a greater public health imperative. Even people who show
only mild symptoms may pass the virus to many, many others ---
particularly in the early course of the infection, before they even
realize they are sick. So you might keep the chain of infection going
right to your own older or high-risk relatives. You may also contribute
to the number of people infected, causing the pandemic to grow rapidly
and overwhelm the health care system.

If you ignore the guidance on social distancing, you will essentially
put yourself and everyone else at much higher risk.

Experts acknowledged that social distancing is tough, especially for
young people who are used to gathering in groups. But even cutting down
the number of gatherings, and the number of people in any group, will
help.

\hypertarget{can-i-leave-my-house}{%
\subsection{Can I leave my house?}\label{can-i-leave-my-house}}

Absolutely. The experts were unanimous in their answer to this question.

It's O.K. to go outdoors for fresh air and exercise --- to walk your
dog, go for a hike or ride your bicycle, for example. The point is not
to remain indoors, but to avoid being in close contact with people.

You may also need to leave the house for medicines or other essential
resources. But there are things you can do to keep yourself and others
safe during and after these excursions.

When you do leave your home, wipe down any surfaces you come into
contact with, disinfect your hands with an alcohol-based sanitizer and
avoid touching your face. Above all, frequently wash your hands ---
especially whenever you come in from outside, before you eat or before
you're in contact with the very old or very young.

\hypertarget{can-i-go-to-the-supermarket}{%
\subsection{Can I go to the
supermarket?}\label{can-i-go-to-the-supermarket}}

Yes. But buy as much as you can at a time in order to minimize the
number of trips, and pick a time when the store is least likely to be
crowded.

When you do go, be aware that any surface inside the store may be
contaminated. Use a disinfecting wipe to clean the handle of the grocery
cart, for example. Experts did not recommend wearing gloves, but if you
do use them, make sure you don't touch your face until you have removed
the gloves.

Dr. Caitlin Rivers, an epidemiologist at Johns Hopkins University,
recommends stowing your cellphone in an inaccessible place so that you
don't absent-mindedly reach for it while shopping. ``That could be a
transmission opportunity,'' she said.

If it's a long shopping trip, you may want to bring hand sanitizer with
you and disinfect your hands in between. And when you get home, Dr.
Rivers said, wash your hands right away.

Those at high risk may want to avoid even these outings if they can help
it, especially if they live in densely populated areas.

Dr. Marrazzo said her mother is an ``incredibly healthy'' 93-year-old
who usually drives herself to the store, but she said she has asked her
mother not to go out during this time, because ``the risks are too great
given the age-related mortality we're seeing.''

\hypertarget{can-i-go-out-to-dinner-at-a-restaurant}{%
\subsection{Can I go out to dinner at a
restaurant?}\label{can-i-go-out-to-dinner-at-a-restaurant}}

Some countries have closed down restaurants and bars for the next few
weeks, but there is no specific nationwide guidance yet on this in the
U.S. beyond the C.D.C.'s recommendation against gatherings of more than
50 people.

Before New York City announced it would be
\href{https://www.nytimes3xbfgragh.onion/2020/03/15/nyregion/new-york-coronavirus.html\#link-709bff21}{shutting
down restaurants and bars}, they were supposed to be operating at half
capacity to maintain social distancing and soften the economic impact.
But in small restaurants, that may still mean you're too close to other
diners. It's also not possible to maintain true social distance from the
people preparing or serving the food.

In general, avoid going out to restaurants, Dr. Marrazzo said, but, ``If
you're going to go, go to some place that you trust.'' Choose spacious
restaurants and ones where the staff members likely practice good
hygiene. Better yet, opt for takeout. If you're concerned for the
restaurant's financial future, ask about purchasing gift certificates
you can redeem later.

\href{https://www.nytimes3xbfgragh.onion/news-event/coronavirus?action=click\&pgtype=Article\&state=default\&region=MAIN_CONTENT_3\&context=storylines_faq}{}

\hypertarget{the-coronavirus-outbreak-}{%
\subsubsection{The Coronavirus Outbreak
›}\label{the-coronavirus-outbreak-}}

\hypertarget{frequently-asked-questions}{%
\paragraph{Frequently Asked
Questions}\label{frequently-asked-questions}}

Updated September 4, 2020

\begin{itemize}
\item ~
  \hypertarget{what-are-the-symptoms-of-coronavirus}{%
  \paragraph{What are the symptoms of
  coronavirus?}\label{what-are-the-symptoms-of-coronavirus}}

  \begin{itemize}
  \tightlist
  \item
    In the beginning, the coronavirus
    \href{https://www.nytimes3xbfgragh.onion/article/coronavirus-facts-history.html?action=click\&pgtype=Article\&state=default\&region=MAIN_CONTENT_3\&context=storylines_faq\#link-6817bab5}{seemed
    like it was primarily a respiratory illness}~--- many patients had
    fever and chills, were weak and tired, and coughed a lot, though
    some people don't show many symptoms at all. Those who seemed
    sickest had pneumonia or acute respiratory distress syndrome and
    received supplemental oxygen. By now, doctors have identified many
    more symptoms and syndromes. In April,
    \href{https://www.nytimes3xbfgragh.onion/2020/04/27/health/coronavirus-symptoms-cdc.html?action=click\&pgtype=Article\&state=default\&region=MAIN_CONTENT_3\&context=storylines_faq}{the
    C.D.C. added to the list of early signs}~sore throat, fever, chills
    and muscle aches. Gastrointestinal upset, such as diarrhea and
    nausea, has also been observed. Another telltale sign of infection
    may be a sudden, profound diminution of one's
    \href{https://www.nytimes3xbfgragh.onion/2020/03/22/health/coronavirus-symptoms-smell-taste.html?action=click\&pgtype=Article\&state=default\&region=MAIN_CONTENT_3\&context=storylines_faq}{sense
    of smell and taste.}~Teenagers and young adults in some cases have
    developed painful red and purple lesions on their fingers and toes
    --- nicknamed ``Covid toe'' --- but few other serious symptoms.
  \end{itemize}
\item ~
  \hypertarget{why-is-it-safer-to-spend-time-together-outside}{%
  \paragraph{Why is it safer to spend time together
  outside?}\label{why-is-it-safer-to-spend-time-together-outside}}

  \begin{itemize}
  \tightlist
  \item
    \href{https://www.nytimes3xbfgragh.onion/2020/05/15/us/coronavirus-what-to-do-outside.html?action=click\&pgtype=Article\&state=default\&region=MAIN_CONTENT_3\&context=storylines_faq}{Outdoor
    gatherings}~lower risk because wind disperses viral droplets, and
    sunlight can kill some of the virus. Open spaces prevent the virus
    from building up in concentrated amounts and being inhaled, which
    can happen when infected people exhale in a confined space for long
    stretches of time, said Dr. Julian W. Tang, a virologist at the
    University of Leicester.
  \end{itemize}
\item ~
  \hypertarget{why-does-standing-six-feet-away-from-others-help}{%
  \paragraph{Why does standing six feet away from others
  help?}\label{why-does-standing-six-feet-away-from-others-help}}

  \begin{itemize}
  \tightlist
  \item
    The coronavirus spreads primarily through droplets from your mouth
    and nose, especially when you cough or sneeze. The C.D.C., one of
    the organizations using that measure,
    \href{https://www.nytimes3xbfgragh.onion/2020/04/14/health/coronavirus-six-feet.html?action=click\&pgtype=Article\&state=default\&region=MAIN_CONTENT_3\&context=storylines_faq}{bases
    its recommendation of six feet}~on the idea that most large droplets
    that people expel when they cough or sneeze will fall to the ground
    within six feet. But six feet has never been a magic number that
    guarantees complete protection. Sneezes, for instance, can launch
    droplets a lot farther than six feet,
    \href{https://jamanetwork.com/journals/jama/fullarticle/2763852}{according
    to a recent study}. It's a rule of thumb: You should be safest
    standing six feet apart outside, especially when it's windy. But
    keep a mask on at all times, even when you think you're far enough
    apart.
  \end{itemize}
\item ~
  \hypertarget{i-have-antibodies-am-i-now-immune}{%
  \paragraph{I have antibodies. Am I now
  immune?}\label{i-have-antibodies-am-i-now-immune}}

  \begin{itemize}
  \tightlist
  \item
    As of right
    now,\href{https://www.nytimes3xbfgragh.onion/2020/07/22/health/covid-antibodies-herd-immunity.html?action=click\&pgtype=Article\&state=default\&region=MAIN_CONTENT_3\&context=storylines_faq}{~that
    seems likely, for at least several months.}~There have been
    frightening accounts of people suffering what seems to be a second
    bout of Covid-19. But experts say these patients may have a
    drawn-out course of infection, with the virus taking a slow toll
    weeks to months after initial exposure.~People infected with the
    coronavirus typically
    \href{https://www.nature.com/articles/s41586-020-2456-9}{produce}~immune
    molecules called antibodies, which are
    \href{https://www.nytimes3xbfgragh.onion/2020/05/07/health/coronavirus-antibody-prevalence.html?action=click\&pgtype=Article\&state=default\&region=MAIN_CONTENT_3\&context=storylines_faq}{protective
    proteins made in response to an
    infection}\href{https://www.nytimes3xbfgragh.onion/2020/05/07/health/coronavirus-antibody-prevalence.html?action=click\&pgtype=Article\&state=default\&region=MAIN_CONTENT_3\&context=storylines_faq}{.
    These antibodies may}~last in the body
    \href{https://www.nature.com/articles/s41591-020-0965-6}{only two to
    three months}, which may seem worrisome, but that's~perfectly normal
    after an acute infection subsides, said Dr. Michael Mina, an
    immunologist at Harvard University. It may be possible to get the
    coronavirus again, but it's highly unlikely that it would be
    possible in a short window of time from initial infection or make
    people sicker the second time.
  \end{itemize}
\item ~
  \hypertarget{what-are-my-rights-if-i-am-worried-about-going-back-to-work}{%
  \paragraph{What are my rights if I am worried about going back to
  work?}\label{what-are-my-rights-if-i-am-worried-about-going-back-to-work}}

  \begin{itemize}
  \tightlist
  \item
    Employers have to provide
    \href{https://www.osha.gov/SLTC/covid-19/standards.html}{a safe
    workplace}~with policies that protect everyone equally.
    \href{https://www.nytimes3xbfgragh.onion/article/coronavirus-money-unemployment.html?action=click\&pgtype=Article\&state=default\&region=MAIN_CONTENT_3\&context=storylines_faq}{And
    if one of your co-workers tests positive for the coronavirus, the
    C.D.C.}~has said that
    \href{https://www.cdc.gov/coronavirus/2019-ncov/community/guidance-business-response.html}{employers
    should tell their employees}~-\/- without giving you the sick
    employee's name -\/- that they may have been exposed to the virus.
  \end{itemize}
\end{itemize}

\hypertarget{can-family-come-to-visit}{%
\subsection{Can family come to visit?}\label{can-family-come-to-visit}}

That depends on who is in your family and how healthy they are.

``Certainly, sick family should not visit,'' said Dr. Marrazzo. ``If you
have vulnerable people in your family, or who are very old, then limit
in-person contact.''

But if everyone in the family is young and healthy, then some careful
interaction in small groups is probably OK. ``The smaller the gathering,
the healthier the people are to start with, the lower the risk of the
situation is going to be,'' she said.

At the same time, you don't want family members to feel isolated or not
have the support of loved ones, so check in with them by phone or plan
activities to do with them on video.

\hypertarget{can-i-take-my-kids-to-the-playground}{%
\subsection{Can I take my kids to the
playground?}\label{can-i-take-my-kids-to-the-playground}}

That depends. If your children have any illness, even if it's not
related to the coronavirus, keep them at home.

If they seem healthy and desperately need to burn energy, outdoor
activities such as bike rides are generally OK. But ``people, especially
in higher-risk areas, may want to think twice about trips to
high-traffic public areas like the playground,'' said Dr. Neha
Chaudhary, a psychiatrist at Harvard Medical School.

Kids also tend to touch their mouths, noses and faces constantly, so
parks or playgrounds with few kids and few contaminated surfaces are
ideal. Take hand sanitizer with you and clean any surfaces with
disinfecting wipes before they play.

Serious
\href{https://www.nytimes3xbfgragh.onion/2020/02/05/health/coronavirus-children.html}{illness
from this virus in kids is rare}, so the kids themselves might be safe.
``That doesn't mean they can't come home and give it to Grandma,'' said
Dr. Marazzo.

So kids should wash their hands often, especially before they come into
contact with older or high-risk family members.

\hypertarget{im-scared-to-feel-alone-is-there-anything-i-can-do-to-make-this-easier}{%
\subsection{I'm scared to feel alone. Is there anything I can do to make
this
easier?}\label{im-scared-to-feel-alone-is-there-anything-i-can-do-to-make-this-easier}}

It's a scary and uncertain time. Staying in touch with family and
friends is more important than ever, because we are biologically
hard-wired to seek each other out when we are stressed, said Dr.
Jonathan Kanter, director for the Center for Science of Social
connection at the University of Washington in Seattle.

Dr. Kanter said he was particularly worried about the long-term impact
of social isolation on both the sick and the healthy. The absence of
physical touch can have a profound impact on our stress levels, he said,
and make us feel under threat.

He said even imagining a warm embrace from a loved one can calm the
body's fight-or-flight response.

In the meantime, we are lucky enough to have technologies at hand that
can maintain social connections. ``It's important to note that social
distancing does not mean social isolation,'' Dr. Chaudhary said.

She suggested people stay connected via social media, chat and video. Be
creative: Schedule dinners with friends over FaceTime, participate in
online game nights, plan to watch television shows at the same time,
enroll in remote learning classes. It's especially important to reach
out to those who are sick or to high-risk people who are self-isolating.
``A phone call with a voice is better than text, and a video chat is
better than a telephone call,'' Dr. Kanter said.

\hypertarget{how-long-will-we-need-to-practice-social-distancing}{%
\subsection{How long will we need to practice social
distancing?}\label{how-long-will-we-need-to-practice-social-distancing}}

That is a big unknown, experts said. A lot will depend on how well the
social distancing measures in place work and how much we can slow the
pandemic down. But prepare to hunker down for at least a month, and
possibly much longer.

In Seattle, the recommendations on social distancing have continued to
escalate with the number of infections and deaths, and as the health
system has become increasingly strained.

``For now, it's probably indefinite,'' Dr. Marrazzo said. ``We're in
uncharted territory.''

Advertisement

\protect\hyperlink{after-bottom}{Continue reading the main story}

\hypertarget{site-index}{%
\subsection{Site Index}\label{site-index}}

\hypertarget{site-information-navigation}{%
\subsection{Site Information
Navigation}\label{site-information-navigation}}

\begin{itemize}
\tightlist
\item
  \href{https://help.nytimes3xbfgragh.onion/hc/en-us/articles/115014792127-Copyright-notice}{©~2020~The
  New York Times Company}
\end{itemize}

\begin{itemize}
\tightlist
\item
  \href{https://www.nytco.com/}{NYTCo}
\item
  \href{https://help.nytimes3xbfgragh.onion/hc/en-us/articles/115015385887-Contact-Us}{Contact
  Us}
\item
  \href{https://www.nytco.com/careers/}{Work with us}
\item
  \href{https://nytmediakit.com/}{Advertise}
\item
  \href{http://www.tbrandstudio.com/}{T Brand Studio}
\item
  \href{https://www.nytimes3xbfgragh.onion/privacy/cookie-policy\#how-do-i-manage-trackers}{Your
  Ad Choices}
\item
  \href{https://www.nytimes3xbfgragh.onion/privacy}{Privacy}
\item
  \href{https://help.nytimes3xbfgragh.onion/hc/en-us/articles/115014893428-Terms-of-service}{Terms
  of Service}
\item
  \href{https://help.nytimes3xbfgragh.onion/hc/en-us/articles/115014893968-Terms-of-sale}{Terms
  of Sale}
\item
  \href{https://spiderbites.nytimes3xbfgragh.onion}{Site Map}
\item
  \href{https://help.nytimes3xbfgragh.onion/hc/en-us}{Help}
\item
  \href{https://www.nytimes3xbfgragh.onion/subscription?campaignId=37WXW}{Subscriptions}
\end{itemize}
