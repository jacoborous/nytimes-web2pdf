\href{/section/us}{U.S.}\textbar{}Doctors Fear Bringing Coronavirus
Home: `I Am Sort of a Pariah in My Family'

\url{https://nyti.ms/2IQ66zI}

\begin{itemize}
\item
\item
\item
\item
\item
\end{itemize}

\href{https://www.nytimes3xbfgragh.onion/news-event/coronavirus?action=click\&pgtype=Article\&state=default\&region=TOP_BANNER\&context=storylines_menu}{The
Coronavirus Outbreak}

\begin{itemize}
\tightlist
\item
  live\href{https://www.nytimes3xbfgragh.onion/2020/08/03/world/coronavirus-covid-19.html?action=click\&pgtype=Article\&state=default\&region=TOP_BANNER\&context=storylines_menu}{Latest
  Updates}
\item
  \href{https://www.nytimes3xbfgragh.onion/interactive/2020/us/coronavirus-us-cases.html?action=click\&pgtype=Article\&state=default\&region=TOP_BANNER\&context=storylines_menu}{Maps
  and Cases}
\item
  \href{https://www.nytimes3xbfgragh.onion/interactive/2020/science/coronavirus-vaccine-tracker.html?action=click\&pgtype=Article\&state=default\&region=TOP_BANNER\&context=storylines_menu}{Vaccine
  Tracker}
\item
  \href{https://www.nytimes3xbfgragh.onion/2020/08/02/us/covid-college-reopening.html?action=click\&pgtype=Article\&state=default\&region=TOP_BANNER\&context=storylines_menu}{College
  Reopening}
\item
  \href{https://www.nytimes3xbfgragh.onion/live/2020/08/03/business/stock-market-today-coronavirus?action=click\&pgtype=Article\&state=default\&region=TOP_BANNER\&context=storylines_menu}{Economy}
\end{itemize}

\includegraphics{https://static01.graylady3jvrrxbe.onion/images/2020/03/16/us/16VIRUS-DOCTORS-top/merlin_170602812_2c0d502e-9e20-4b7c-a91f-9122e3b6470d-articleLarge.jpg?quality=75\&auto=webp\&disable=upscale}

Sections

\protect\hyperlink{site-content}{Skip to
content}\protect\hyperlink{site-index}{Skip to site index}

\hypertarget{doctors-fear-bringing-coronavirus-home-i-am-sort-of-a-pariah-in-my-family}{%
\section{Doctors Fear Bringing Coronavirus Home: `I Am Sort of a Pariah
in My
Family'}\label{doctors-fear-bringing-coronavirus-home-i-am-sort-of-a-pariah-in-my-family}}

One doctor dreamed he was surrounded by coughing patients. ``Most
physicians have never seen this level of angst and anxiety in their
careers,'' a veteran emergency room doctor said.

Dr. Stephen Anderson, an emergency room veteran, said there was a
two-day supply of surgical masks at his hospital, MultiCare Auburn
Medical Center near Seattle.Credit...Grant Hindsley for The New York
Times

Supported by

\protect\hyperlink{after-sponsor}{Continue reading the main story}

\href{https://www.nytimes3xbfgragh.onion/by/karen-weise}{\includegraphics{https://static01.graylady3jvrrxbe.onion/images/2019/04/11/multimedia/author-karen-weise/author-karen-weise-thumbLarge.png}}

By \href{https://www.nytimes3xbfgragh.onion/by/karen-weise}{Karen Weise}

\begin{itemize}
\item
  Published March 16, 2020Updated April 13, 2020
\item
  \begin{itemize}
  \item
  \item
  \item
  \item
  \item
  \end{itemize}
\end{itemize}

SEATTLE --- After her shifts in the emergency room, one
\href{https://www.nytimes3xbfgragh.onion/2020/04/13/nyregion/coronavirus-nyc-doctors.html}{doctor}
in Utah strips naked on her porch and runs straight to a shower, trying
not to contaminate her home. In Oregon, an emergency physician talks of
how he was recently bent over a drunk teenager, stapling a head wound,
when he realized with a sudden chill that the patient had a fever and a
cough. A doctor in Washington State woke up one night not long ago with
nightmares of being surrounded by coughing patients.

``Most physicians have never seen this level of angst and anxiety in
their careers,'' said Dr. Stephen Anderson, a 35-year veteran of
emergency rooms in a suburb south of Seattle. ``I am sort of a pariah in
my family. I am dipping myself into the swamp every day.''

As the coronavirus expands around the country,
\href{https://www.nytimes3xbfgragh.onion/2020/04/13/nyregion/coronavirus-nyc-doctors.html}{doctors
and nurses} working in emergency rooms are suddenly wary of everyone
walking in the door with a cough, forced to make quick, harrowing
decisions to help save not only their patients' lives, but their own.

The stress only grew on Sunday, when the American College of Emergency
Physicians revealed that two emergency medicine doctors, in New Jersey
and Washington State, were
\href{https://www.nytimes3xbfgragh.onion/2020/03/15/us/coronavirus-physicians-emergency-rooms.html}{hospitalized
in critical condition} as a result of the coronavirus. Though the virus
is spreading in the community and there was no way of ascertaining
whether they were exposed at work or somewhere else, the two cases
prompted urgent new questions among doctors about how many precautions
are enough.

``Now that we see front-line providers that are on ventilators, it is
really driving it home,'' Dr. Anderson said.

Doctors, nurses and other staff members in a variety of hospital
departments face new uncertainty. In intensive care units, for example,
health care providers must have extended exposure to people who have
contracted the virus. But they know in advance of the risk they face.

In emergency departments, the danger comes from the unknown.

Patients arrive with symptoms but no diagnosis, and staff members must
sometimes tend to urgent needs, such as gaping wounds, before they have
time to screen a patient for Covid-19, the disease caused by the virus.
At times, the protocols they must follow are changing every few hours.

``Many of us have trained for disasters, like Ebola and hurricanes,''
said Dr. Adam Brown, the president of emergency medicine for Envision
Healthcare, the largest provider of contract physicians to emergency
rooms. ``This is different because of the scale and scope of the
disease.''

Add to that the shortage of protective gear and delays in testing, and
health care workers fear they are flying blind.

\hypertarget{latest-updates-global-coronavirus-outbreak}{%
\section{\texorpdfstring{\href{https://www.nytimes3xbfgragh.onion/2020/08/03/world/coronavirus-covid-19.html?action=click\&pgtype=Article\&state=default\&region=MAIN_CONTENT_1\&context=storylines_live_updates}{Latest
Updates: Global Coronavirus
Outbreak}}{Latest Updates: Global Coronavirus Outbreak}}\label{latest-updates-global-coronavirus-outbreak}}

Updated 2020-08-04T07:33:06.428Z

\begin{itemize}
\tightlist
\item
  \href{https://www.nytimes3xbfgragh.onion/2020/08/03/world/coronavirus-covid-19.html?action=click\&pgtype=Article\&state=default\&region=MAIN_CONTENT_1\&context=storylines_live_updates\#link-4547638f}{Fauci
  defends Birx after she is criticized by Trump.}
\item
  \href{https://www.nytimes3xbfgragh.onion/2020/08/03/world/coronavirus-covid-19.html?action=click\&pgtype=Article\&state=default\&region=MAIN_CONTENT_1\&context=storylines_live_updates\#link-15e7f995}{Trump
  derides Democrats as lawmakers and administration officials try to
  break stimulus impasse.}
\item
  \href{https://www.nytimes3xbfgragh.onion/2020/08/03/world/coronavirus-covid-19.html?action=click\&pgtype=Article\&state=default\&region=MAIN_CONTENT_1\&context=storylines_live_updates\#link-e5a2cda}{The
  deadline for 2020 census counting has been moved up by a month.}
\end{itemize}

\href{https://www.nytimes3xbfgragh.onion/2020/08/03/world/coronavirus-covid-19.html?action=click\&pgtype=Article\&state=default\&region=MAIN_CONTENT_1\&context=storylines_live_updates}{See
more updates}

More live coverage:
\href{https://www.nytimes3xbfgragh.onion/live/2020/08/03/business/stock-market-today-coronavirus?action=click\&pgtype=Article\&state=default\&region=MAIN_CONTENT_1\&context=storylines_live_updates}{Markets}

Though the numbers are still low, Envision, which employs 11,000
emergency clinicians across the United States, has five times as many
doctors under quarantine as it did a week ago, Dr. Brown said.

Several providers spoke on the condition of anonymity because their
employers have told them not to talk to the news media.

\includegraphics{https://static01.graylady3jvrrxbe.onion/images/2020/03/16/us/16virus-doctors/merlin_170307846_9930c83d-a58a-4b35-882a-ff3083659ca8-articleLarge.jpg?quality=75\&auto=webp\&disable=upscale}

The personal strain is cascading as the virus reaches new parts of the
country. ``Everybody feels the stress, but everybody is pulling
together,'' said Dr. K. Kay Moody, an emergency room doctor in Olympia,
Wash., who runs a Facebook group with 22,000 emergency physicians.
``That is what is keeping us OK.''

A few doctors said they were talking about bunking up in Airbnbs to
create ``dirty doc'' living quarters to avoid endangering their children
when they go home. Some are showing their partners where to find their
passwords and insurance, should they end up in intensive care. Dr. Moody
said she knew of at least one doctor whose former spouse was threatening
to take their children away if the doctor went to work.

Many emergency physicians work as contractors, not hospital staff, so
they will not necessarily be paid if they are quarantined. ``As it
stands, that is one of the most anxiety-provoking things,'' Dr. Moody
said, ``on top of fear for your life.''

Nurses face similar challenges, though with less pay and support. An
emergency nurse in Milwaukee said she bought her own goggles after
hearing that protective gear was running low. A nurse at a rural
hospital near Lake Tahoe in California said that the hospital was
providing physicians with shower facilities as well as clean scrubs to
wear, but that nurses had to wash their work clothes at home. She said
that the physicians she worked with lobbied the hospital to provide
clean scrubs for the nurses, but that the hospital concluded it would
cost too much.

One doctor, who spoke on condition that the identity of the veterans
hospital where she worked was not revealed, said the protocols have not
kept up with the changing reality on the ground. When determining if a
patient should get a separate room, she said, the emergency department
still asks patients if they have been to high-risk countries, like China
and Italy, even though community transmission of the virus has been well
established.

\href{https://www.nytimes3xbfgragh.onion/news-event/coronavirus?action=click\&pgtype=Article\&state=default\&region=MAIN_CONTENT_3\&context=storylines_faq}{}

\hypertarget{the-coronavirus-outbreak-}{%
\subsubsection{The Coronavirus Outbreak
›}\label{the-coronavirus-outbreak-}}

\hypertarget{frequently-asked-questions}{%
\paragraph{Frequently Asked
Questions}\label{frequently-asked-questions}}

Updated August 3, 2020

\begin{itemize}
\item ~
  \hypertarget{im-a-small-business-owner-can-i-get-relief}{%
  \paragraph{I'm a small-business owner. Can I get
  relief?}\label{im-a-small-business-owner-can-i-get-relief}}

  \begin{itemize}
  \tightlist
  \item
    The
    \href{https://www.nytimes3xbfgragh.onion/article/small-business-loans-stimulus-grants-freelancers-coronavirus.html?action=click\&pgtype=Article\&state=default\&region=MAIN_CONTENT_3\&context=storylines_faq}{stimulus
    bills enacted in March} offer help for the millions of American
    small businesses. Those eligible for aid are businesses and
    nonprofit organizations with fewer than 500 workers, including sole
    proprietorships, independent contractors and freelancers. Some
    larger companies in some industries are also eligible. The help
    being offered, which is being managed by the Small Business
    Administration, includes the Paycheck Protection Program and the
    Economic Injury Disaster Loan program. But lots of folks have
    \href{https://www.nytimes3xbfgragh.onion/interactive/2020/05/07/business/small-business-loans-coronavirus.html?action=click\&pgtype=Article\&state=default\&region=MAIN_CONTENT_3\&context=storylines_faq}{not
    yet seen payouts.} Even those who have received help are confused:
    The rules are draconian, and some are stuck sitting on
    \href{https://www.nytimes3xbfgragh.onion/2020/05/02/business/economy/loans-coronavirus-small-business.html?action=click\&pgtype=Article\&state=default\&region=MAIN_CONTENT_3\&context=storylines_faq}{money
    they don't know how to use.} Many small-business owners are getting
    less than they expected or
    \href{https://www.nytimes3xbfgragh.onion/2020/06/10/business/Small-business-loans-ppp.html?action=click\&pgtype=Article\&state=default\&region=MAIN_CONTENT_3\&context=storylines_faq}{not
    hearing anything at all.}
  \end{itemize}
\item ~
  \hypertarget{what-are-my-rights-if-i-am-worried-about-going-back-to-work}{%
  \paragraph{What are my rights if I am worried about going back to
  work?}\label{what-are-my-rights-if-i-am-worried-about-going-back-to-work}}

  \begin{itemize}
  \tightlist
  \item
    Employers have to provide
    \href{https://www.osha.gov/SLTC/covid-19/standards.html}{a safe
    workplace} with policies that protect everyone equally.
    \href{https://www.nytimes3xbfgragh.onion/article/coronavirus-money-unemployment.html?action=click\&pgtype=Article\&state=default\&region=MAIN_CONTENT_3\&context=storylines_faq}{And
    if one of your co-workers tests positive for the coronavirus, the
    C.D.C.} has said that
    \href{https://www.cdc.gov/coronavirus/2019-ncov/community/guidance-business-response.html}{employers
    should tell their employees} -\/- without giving you the sick
    employee's name -\/- that they may have been exposed to the virus.
  \end{itemize}
\item ~
  \hypertarget{should-i-refinance-my-mortgage}{%
  \paragraph{Should I refinance my
  mortgage?}\label{should-i-refinance-my-mortgage}}

  \begin{itemize}
  \tightlist
  \item
    \href{https://www.nytimes3xbfgragh.onion/article/coronavirus-money-unemployment.html?action=click\&pgtype=Article\&state=default\&region=MAIN_CONTENT_3\&context=storylines_faq}{It
    could be a good idea,} because mortgage rates have
    \href{https://www.nytimes3xbfgragh.onion/2020/07/16/business/mortgage-rates-below-3-percent.html?action=click\&pgtype=Article\&state=default\&region=MAIN_CONTENT_3\&context=storylines_faq}{never
    been lower.} Refinancing requests have pushed mortgage applications
    to some of the highest levels since 2008, so be prepared to get in
    line. But defaults are also up, so if you're thinking about buying a
    home, be aware that some lenders have tightened their standards.
  \end{itemize}
\item ~
  \hypertarget{what-is-school-going-to-look-like-in-september}{%
  \paragraph{What is school going to look like in
  September?}\label{what-is-school-going-to-look-like-in-september}}

  \begin{itemize}
  \tightlist
  \item
    It is unlikely that many schools will return to a normal schedule
    this fall, requiring the grind of
    \href{https://www.nytimes3xbfgragh.onion/2020/06/05/us/coronavirus-education-lost-learning.html?action=click\&pgtype=Article\&state=default\&region=MAIN_CONTENT_3\&context=storylines_faq}{online
    learning},
    \href{https://www.nytimes3xbfgragh.onion/2020/05/29/us/coronavirus-child-care-centers.html?action=click\&pgtype=Article\&state=default\&region=MAIN_CONTENT_3\&context=storylines_faq}{makeshift
    child care} and
    \href{https://www.nytimes3xbfgragh.onion/2020/06/03/business/economy/coronavirus-working-women.html?action=click\&pgtype=Article\&state=default\&region=MAIN_CONTENT_3\&context=storylines_faq}{stunted
    workdays} to continue. California's two largest public school
    districts --- Los Angeles and San Diego --- said on July 13, that
    \href{https://www.nytimes3xbfgragh.onion/2020/07/13/us/lausd-san-diego-school-reopening.html?action=click\&pgtype=Article\&state=default\&region=MAIN_CONTENT_3\&context=storylines_faq}{instruction
    will be remote-only in the fall}, citing concerns that surging
    coronavirus infections in their areas pose too dire a risk for
    students and teachers. Together, the two districts enroll some
    825,000 students. They are the largest in the country so far to
    abandon plans for even a partial physical return to classrooms when
    they reopen in August. For other districts, the solution won't be an
    all-or-nothing approach.
    \href{https://bioethics.jhu.edu/research-and-outreach/projects/eschool-initiative/school-policy-tracker/}{Many
    systems}, including the nation's largest, New York City, are
    devising
    \href{https://www.nytimes3xbfgragh.onion/2020/06/26/us/coronavirus-schools-reopen-fall.html?action=click\&pgtype=Article\&state=default\&region=MAIN_CONTENT_3\&context=storylines_faq}{hybrid
    plans} that involve spending some days in classrooms and other days
    online. There's no national policy on this yet, so check with your
    municipal school system regularly to see what is happening in your
    community.
  \end{itemize}
\item ~
  \hypertarget{is-the-coronavirus-airborne}{%
  \paragraph{Is the coronavirus
  airborne?}\label{is-the-coronavirus-airborne}}

  \begin{itemize}
  \tightlist
  \item
    The coronavirus
    \href{https://www.nytimes3xbfgragh.onion/2020/07/04/health/239-experts-with-one-big-claim-the-coronavirus-is-airborne.html?action=click\&pgtype=Article\&state=default\&region=MAIN_CONTENT_3\&context=storylines_faq}{can
    stay aloft for hours in tiny droplets in stagnant air}, infecting
    people as they inhale, mounting scientific evidence suggests. This
    risk is highest in crowded indoor spaces with poor ventilation, and
    may help explain super-spreading events reported in meatpacking
    plants, churches and restaurants.
    \href{https://www.nytimes3xbfgragh.onion/2020/07/06/health/coronavirus-airborne-aerosols.html?action=click\&pgtype=Article\&state=default\&region=MAIN_CONTENT_3\&context=storylines_faq}{It's
    unclear how often the virus is spread} via these tiny droplets, or
    aerosols, compared with larger droplets that are expelled when a
    sick person coughs or sneezes, or transmitted through contact with
    contaminated surfaces, said Linsey Marr, an aerosol expert at
    Virginia Tech. Aerosols are released even when a person without
    symptoms exhales, talks or sings, according to Dr. Marr and more
    than 200 other experts, who
    \href{https://academic.oup.com/cid/article/doi/10.1093/cid/ciaa939/5867798}{have
    outlined the evidence in an open letter to the World Health
    Organization}.
  \end{itemize}
\end{itemize}

Doctors have begun building plans for how they will ration supplies when
there are more patients than their hospitals can handle. Emergency room
doctors have experience sitting families down to advise discontinuing
care because it would be futile. But in the United States, they are not
used to making such calls based on resources alone.

Some said they were looking to Italy, where doctors on the front line
have sometimes had to ration care in favor of younger patients, or those
without other complicating conditions, who are more likely to benefit
from it.

``If we get it all at once, we don't have the resources, we don't have
the ventilators,'' said Dr. William Jaquis, chair of the American
College of Emergency Physicians.

Last week, Italian media reported that Bergamo, a city northeast of
Milan, saw roughly 50 doctors test positive for the virus. In the region
of Puglia in the south, local media reported that 76 employees had been
quarantined after being exposed to patients who contracted Covid-19.

After the coronavirus broke out at a nursing facility near Seattle, Dr.
Anderson sat with the leaders of his hospital, MultiCare Auburn Medical
Center, to talk about how urgently they should prepare. Their hospital
is ringed by nursing homes and other care facilities, and he rattled off
those most at risk for fatal cases of the virus: males over 60, and
those with cardiac and pulmonary problems. ``I literally stopped what I
was saying and realized that that was me,'' he said.

He said his hospital was down to a two-day supply of surgical masks ---
he wears one per shift. ``Those are supposed to be disposable,'' he
said. Now he must carefully remove and clean the mask each time he takes
it off and on. ``That may sound just like a nuisance, but when you're
potentially touching something that has the virus that could kill you on
it, and you're doing it 25 times a shift, it's kind of nerve-racking,''
he said.

His wife has moved to their mountain cabin, and they have given up on
their retirement cruise in Europe. ``I haven't slept for longer than
three hours in the past two weeks,'' he said.

In the early hours of Monday morning, he could not sleep. More than 200
emails had come into his inbox since he went to bed, including news that
three other health care providers had been admitted to a hospital
overnight, he said.

But he plans to be at his next shift nonetheless.

``I have been doing this for 35 years," he said, ``and I'm not going to
stop now.''

Vanessa Swales contributed reporting from New York.

Advertisement

\protect\hyperlink{after-bottom}{Continue reading the main story}

\hypertarget{site-index}{%
\subsection{Site Index}\label{site-index}}

\hypertarget{site-information-navigation}{%
\subsection{Site Information
Navigation}\label{site-information-navigation}}

\begin{itemize}
\tightlist
\item
  \href{https://help.nytimes3xbfgragh.onion/hc/en-us/articles/115014792127-Copyright-notice}{©~2020~The
  New York Times Company}
\end{itemize}

\begin{itemize}
\tightlist
\item
  \href{https://www.nytco.com/}{NYTCo}
\item
  \href{https://help.nytimes3xbfgragh.onion/hc/en-us/articles/115015385887-Contact-Us}{Contact
  Us}
\item
  \href{https://www.nytco.com/careers/}{Work with us}
\item
  \href{https://nytmediakit.com/}{Advertise}
\item
  \href{http://www.tbrandstudio.com/}{T Brand Studio}
\item
  \href{https://www.nytimes3xbfgragh.onion/privacy/cookie-policy\#how-do-i-manage-trackers}{Your
  Ad Choices}
\item
  \href{https://www.nytimes3xbfgragh.onion/privacy}{Privacy}
\item
  \href{https://help.nytimes3xbfgragh.onion/hc/en-us/articles/115014893428-Terms-of-service}{Terms
  of Service}
\item
  \href{https://help.nytimes3xbfgragh.onion/hc/en-us/articles/115014893968-Terms-of-sale}{Terms
  of Sale}
\item
  \href{https://spiderbites.nytimes3xbfgragh.onion}{Site Map}
\item
  \href{https://help.nytimes3xbfgragh.onion/hc/en-us}{Help}
\item
  \href{https://www.nytimes3xbfgragh.onion/subscription?campaignId=37WXW}{Subscriptions}
\end{itemize}
