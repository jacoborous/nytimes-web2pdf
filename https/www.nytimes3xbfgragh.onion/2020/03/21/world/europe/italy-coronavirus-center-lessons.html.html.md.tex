Sections

SEARCH

\protect\hyperlink{site-content}{Skip to
content}\protect\hyperlink{site-index}{Skip to site index}

\href{https://www.nytimes3xbfgragh.onion/section/world/europe}{Europe}

\href{https://myaccount.nytimes3xbfgragh.onion/auth/login?response_type=cookie\&client_id=vi}{}

\href{https://www.nytimes3xbfgragh.onion/section/todayspaper}{Today's
Paper}

\href{/section/world/europe}{Europe}\textbar{}Italy, Pandemic's New
Epicenter, Has Lessons for the World

\url{https://nyti.ms/3bcTDCz}

\begin{itemize}
\item
\item
\item
\item
\item
\item
\end{itemize}

\hypertarget{the-coronavirus-outbreak}{%
\subsubsection{\texorpdfstring{\href{https://www.nytimes3xbfgragh.onion/news-event/coronavirus?name=styln-coronavirus-national\&region=TOP_BANNER\&block=storyline_menu_recirc\&action=click\&pgtype=Article\&impression_id=2bf985f0-efba-11ea-80ca-fdfa1d63f0db\&variant=undefined}{The
Coronavirus
Outbreak}}{The Coronavirus Outbreak}}\label{the-coronavirus-outbreak}}

\begin{itemize}
\tightlist
\item
  live\href{https://www.nytimes3xbfgragh.onion/2020/09/05/world/coronavirus-covid.html?name=styln-coronavirus-national\&region=TOP_BANNER\&block=storyline_menu_recirc\&action=click\&pgtype=Article\&impression_id=2bf9ad00-efba-11ea-80ca-fdfa1d63f0db\&variant=undefined}{Latest
  Updates}
\item
  \href{https://www.nytimes3xbfgragh.onion/interactive/2020/us/coronavirus-us-cases.html?name=styln-coronavirus-national\&region=TOP_BANNER\&block=storyline_menu_recirc\&action=click\&pgtype=Article\&impression_id=2bf9ad01-efba-11ea-80ca-fdfa1d63f0db\&variant=undefined}{Maps
  and Cases}
\item
  \href{https://www.nytimes3xbfgragh.onion/interactive/2020/science/coronavirus-vaccine-tracker.html?name=styln-coronavirus-national\&region=TOP_BANNER\&block=storyline_menu_recirc\&action=click\&pgtype=Article\&impression_id=2bf9ad02-efba-11ea-80ca-fdfa1d63f0db\&variant=undefined}{Vaccine
  Tracker}
\item
  \href{https://www.nytimes3xbfgragh.onion/2020/09/02/your-money/eviction-moratorium-covid.html?name=styln-coronavirus-national\&region=TOP_BANNER\&block=storyline_menu_recirc\&action=click\&pgtype=Article\&impression_id=2bf9ad03-efba-11ea-80ca-fdfa1d63f0db\&variant=undefined}{Eviction
  Moratorium}
\item
  \href{https://www.nytimes3xbfgragh.onion/interactive/2020/09/02/magazine/food-insecurity-hunger-us.html?name=styln-coronavirus-national\&region=TOP_BANNER\&block=storyline_menu_recirc\&action=click\&pgtype=Article\&impression_id=2bf9ad04-efba-11ea-80ca-fdfa1d63f0db\&variant=undefined}{American
  Hunger}
\end{itemize}

Advertisement

\protect\hyperlink{after-top}{Continue reading the main story}

Supported by

\protect\hyperlink{after-sponsor}{Continue reading the main story}

\hypertarget{italy-pandemics-new-epicenter-has-lessons-for-the-world}{%
\section{Italy, Pandemic's New Epicenter, Has Lessons for the
World}\label{italy-pandemics-new-epicenter-has-lessons-for-the-world}}

The country's experience shows that steps to isolate the coronavirus and
limit people's movement need to be put in place early, with absolute
clarity, then strictly enforced.

\includegraphics{https://static01.graylady3jvrrxbe.onion/images/2020/03/21/world/21italy-virus1sub-copy/21italy-virus1sub-copy-articleLarge-v3.jpg?quality=75\&auto=webp\&disable=upscale}

By \href{https://www.nytimes3xbfgragh.onion/by/jason-horowitz}{Jason
Horowitz}, \href{https://www.nytimes3xbfgragh.onion/by/emma-bubola}{Emma
Bubola} and
\href{https://www.nytimes3xbfgragh.onion/by/elisabetta-povoledo}{Elisabetta
Povoledo}

\begin{itemize}
\item
  March 21, 2020
\item
  \begin{itemize}
  \item
  \item
  \item
  \item
  \item
  \item
  \end{itemize}
\end{itemize}

\href{https://www.nytimes3xbfgragh.onion/it/2020/03/22/world/europe/litalia-pandemia.html}{Leggere
in
italiano}\href{https://www.nytimes3xbfgragh.onion/es/2020/03/22/espanol/coronavirus-lecciones-italia.html}{Leer
en
español}\href{https://cn.nytimes3xbfgragh.onion/world/20200323/italy-coronavirus-center-lessons/}{阅读简体中文版}\href{https://cn.nytimes3xbfgragh.onion/world/20200323/italy-coronavirus-center-lessons/zh-hant/}{閱讀繁體中文版}

ROME --- As Italy's coronavirus infections ticked above 400 cases and
deaths hit the double digits, the leader of the governing Democratic
Party posted a picture of himself clinking glasses for ``an aperitivo in
Milan,'' urging people ``not to change our habits.''

That was on Feb. 27. Not 10 days later, as the toll hit 5,883 infections
and 233 dead, the party boss, Nicola Zingaretti, posted a
\href{https://www.nytimes3xbfgragh.onion/2020/03/07/world/europe/coronavirus-italy.html}{new
video,} this time informing Italy that he, too, had the virus.

Italy now has more than 53,000 recorded infections and more than 4,800
dead, and the rate of increase keeps growing, with more than half the
cases and fatalities coming in the past week. On Saturday, officials
reported 793 additional deaths, by far the largest single-day increase
so far. Italy has surpassed China as the country with the highest death
toll, becoming the epicenter of a shifting pandemic.

The government has sent in the army to enforce the lockdown in Lombardy,
the northern region at the center of the outbreak, where
\href{https://www.nytimes3xbfgragh.onion/2020/03/16/world/europe/italy-coronavirus-funerals.html?searchResultPosition=5}{bodies
have piled up in churches}. On Friday night, the authorities tightened
the nationwide lockdown, closing parks, banning outdoor activities
including walking or jogging far from home.

On Saturday night, Prime Minister Giuseppe Conte announced another
drastic step in response to what he called the country's most difficult
crisis since the Second World War: Italy will close its factories and
all production that is not absolutely essential, an enormous economic
sacrifice intended to contain the virus and protect lives.

``The state is here,'' he said in an effort to reassure the public.

But the tragedy of Italy now stands as a warning to its European
neighbors and the United States, where the virus is coming with equal
velocity. If Italy's experience shows anything, it is that measures to
isolate affected areas and limit the movement of the broader population
need to be taken early, put in place with absolute clarity, then
strictly enforced.

Despite now having some of the toughest measures in the world, Italian
authorities fumbled many of those steps early in the contagion --- when
it most mattered as they sought to preserve basic civil liberties as
well as the economy.

Italy's piecemeal attempts to cut it off ---
\href{https://www.nytimes3xbfgragh.onion/2020/02/23/world/europe/italy-coronavirus.html?searchResultPosition=42}{isolating
towns} first, then
\href{https://www.nytimes3xbfgragh.onion/2020/03/07/world/europe/coronavirus-italy.html?searchResultPosition=28}{regions},
then
\href{https://www.nytimes3xbfgragh.onion/2020/03/09/world/europe/italy-lockdown-coronavirus.html?searchResultPosition=24}{shutting
down} the country in an intentionally porous lockdown --- always lagged
behind the virus's lethal trajectory.

\includegraphics{https://static01.graylady3jvrrxbe.onion/images/2020/03/22/world/22italy-virus-jp5/merlin_170542245_94d21304-7206-4c6d-89ec-b7a8a1c80d71-articleLarge.jpg?quality=75\&auto=webp\&disable=upscale}

``Now we are running after it,'' said Sandra Zampa, the under secretary
at the Ministry of Health, who said Italy did the best it could given
the information it had. ``We closed gradually, as Europe is doing.
France, Spain, Germany, the U.S. are doing the same. Every day you close
a bit, you give up on a bit of normal life. Because the virus does not
allow normal life.''

Some officials gave in to magical thinking, reluctant to make painful
decisions sooner. All the while, the virus fed on that complacency.

Governments beyond Italy are now in danger of following the same path,
repeating familiar mistakes and inviting similar calamity. And unlike
Italy, which navigated uncharted territory for a Western democracy,
other governments have less room for excuses.

Italian officials, for their part, have defended their response,
emphasizing that the crisis is unprecedented in modern times. They
assert that the government responded with speed and competence,
immediately acting on the advice of its scientists and moving more
swiftly on drastic, economically devastating measures than their
European counterparts.

But tracing the record of their actions shows missed opportunities and
critical missteps.

In the critical early days of the outbreak, Mr. Conte and other top
officials sought to down play the threat, creating confusion and a false
sense of security that allowed the virus to spread.

They blamed Italy's high number of infections on
\href{https://www.nytimes3xbfgragh.onion/2020/02/27/world/europe/italy-coronavirus.html?searchResultPosition=35}{aggressive
testing} of people without symptoms in the north, which they argued only
created hysteria and tarnished the country's image abroad.

Image

At Palazzo Marino, headquarters of the municipality of Milan, chairs
were placed outdoors and at a safe distance ahead of a
meeting.Credit...Alessandro Grassani for The New York Times

Even once the Italian government considered a universal lockdown
necessary to defeat the virus,
it\href{https://www.nytimes3xbfgragh.onion/2020/03/08/world/europe/italy-coronavirus-quarantine.html?searchResultPosition=26}{failed
to communicate} the threat powerfully enough to persuade Italians to
abide by the rules, which seemed riddled with loopholes.

``It is not easy in a liberal democracy,'' said Walter Ricciardi, a
World Health Organization board member and a top adviser to the health
ministry, who argued that the Italian government acted on the scientific
evidence made available to it.

\hypertarget{latest-updates-the-coronavirus-outbreak}{%
\section{\texorpdfstring{\href{https://www.nytimes3xbfgragh.onion/2020/09/04/world/covid-19-coronavirus.html?action=click\&pgtype=Article\&state=default\&region=MAIN_CONTENT_1\&context=storylines_live_updates}{Latest
Updates: The Coronavirus
Outbreak}}{Latest Updates: The Coronavirus Outbreak}}\label{latest-updates-the-coronavirus-outbreak}}

Updated 2020-09-05T12:05:40.998Z

\begin{itemize}
\tightlist
\item
  \href{https://www.nytimes3xbfgragh.onion/2020/09/04/world/covid-19-coronavirus.html?action=click\&pgtype=Article\&state=default\&region=MAIN_CONTENT_1\&context=storylines_live_updates\#link-1654f6ad}{Research
  connects vaping to a higher chance of catching the virus --- and
  suffering its worst effects.}
\item
  \href{https://www.nytimes3xbfgragh.onion/2020/09/04/world/covid-19-coronavirus.html?action=click\&pgtype=Article\&state=default\&region=MAIN_CONTENT_1\&context=storylines_live_updates\#link-52e4198a}{Another
  college football game won't be played as planned.}
\item
  \href{https://www.nytimes3xbfgragh.onion/2020/09/04/world/covid-19-coronavirus.html?action=click\&pgtype=Article\&state=default\&region=MAIN_CONTENT_1\&context=storylines_live_updates\#link-181cef0}{Pharmaceutical
  companies plan a joint pledge on safety standards as they move
  vaccines to the marketplace.}
\end{itemize}

\href{https://www.nytimes3xbfgragh.onion/2020/09/04/world/covid-19-coronavirus.html?action=click\&pgtype=Article\&state=default\&region=MAIN_CONTENT_1\&context=storylines_live_updates}{See
more updates}

More live coverage:
\href{https://www.nytimes3xbfgragh.onion/live/2020/09/04/business/stock-market-today-coronavirus?action=click\&pgtype=Article\&state=default\&region=MAIN_CONTENT_1\&context=storylines_live_updates}{Markets}

He said the Italian government had moved at a much faster clip, and took
the threat much more seriously, than its European neighbors or the
United States.

Still, he acknowledged that the health minister had struggled to
persuade his government colleagues to move more quickly and that the
difficulties of navigating Italy's division of powers between Rome and
the regions resulted in a fragmented chain of command and inconsistent
messages.

``In times of war, like an epidemic,'' that system presented grave
problems, he said, adding that it perhaps delayed the imposing of
restrictive measures.

``I would have done them 10 days before, that is the only difference.''

\hypertarget{it-could-never-happen-here}{%
\subsection{It Could Never Happen
Here}\label{it-could-never-happen-here}}

For the coronavirus, 10 days can be a lifetime.

On Jan. 21, as top Chinese officials warned that those hiding virus
cases ``will be nailed on the pillar of shame for eternity,'' Italy's
culture and tourism minister hosted a Chinese delegation for a concert
at the National Academy of Santa Cecilia to inaugurate the year of
Italy-China Culture and Tourism.

Michele Geraci, Italy's former under secretary in the economic
development ministry and a booster of closer relations with China, had a
drink with other politicians but looked around uneasily.

``Are we sure we want to do this?'' he said he asked them. ``Should we
be here today?''

With the benefit of hindsight, Italian officials say certainly not.

Image

In~San Fiorano, one of the original `red zone' towns that were locked
down, residents watched as Prime Minister Giuseppe Conte of Italy
announced travel restrictions on the entire country.Credit...Marzio
Toniolo, via Reuters

Ms. Zampa, the health ministry under secretary, said in retrospect she
would have closed everything immediately. But in real time, it wasn't
that clear.

Politicians across the spectrum worried about the economy and feeding
the country, and found it difficult to accept their impotence in the
face of the virus.

Most importantly, Italy looked at the example of China, Ms. Zampa said,
not as a practical warning, but as a ``science fiction movie that had
nothing to do with us.'' And when the virus exploded, Europe, she said,
``looked at us the same way we looked at China.''

But already in January, some officials on the right were urging Mr.
Conte, their former ally and now political enemy, to quarantine
schoolchildren in the northern regions who were returning from holidays
in China, a measure aimed at protecting schools. Many of those children
were from Chinese immigrant families.

Many liberals criticized the proposal as populist fear-mongering. Mr.
Conte declined the proposal and responded that the northern governors
should trust the judgment of education and health authorities who, he
said, had proposed no such thing.

But Mr. Conte also demonstrated that he was taking the threat of
contagion seriously. On Jan. 30,
he\href{https://www.nytimes3xbfgragh.onion/2020/01/30/world/asia/coronavirus-china.html?searchResultPosition=16}{blocked
all flights} in and out of China.

``We are the first country in Europe to adopt such a precautionary
measure,'' he said.

Over the next month, Italy responded swiftly to coronavirus scares. Two
sick
\href{https://www.nytimes3xbfgragh.onion/2020/01/31/world/asia/coronavirus-china.html?searchResultPosition=22}{Chinese
tourists}and an Italian returning from China received care from a
prominent infectious disease hospital in Rome. A false alarm led
authorities to briefly confine passengers on a
\href{https://www.nytimes3xbfgragh.onion/2020/01/30/world/europe/italy-coronavirus-cruise.html?searchResultPosition=56}{cruise
ship}docked outside of Rome.

\hypertarget{patient-one-super-spreader}{%
\subsection{`Patient One,'
Super-spreader}\label{patient-one-super-spreader}}

When a 38-year-old man went to the emergency room at a hospital in
Codogno, a small town in the Lodi province of Lombardy, with severe flu
symptoms on Feb. 18, the case did not set off alarms.

The patient declined to be hospitalized and went home. He got sicker and
returned to the hospital a few hours later and was admitted to a general
medicine ward.
\href{https://www.repubblica.it/cronaca/2020/03/06/news/l_anestesista_di_codogno_per_mattia_era_tutto_inutile_cosi_ho_avuto_la_folle_idea_di_pensare_al_coronavirus_-250380291/}{On
Feb. 20}, he went into intensive care, where he
\href{https://www.nytimes3xbfgragh.onion/2020/02/21/world/asia/china-coronavirus.html?searchResultPosition=4}{tested
positive} for the virus.

The man, who became known as Patient One, had had a busy month. He
attended at least three dinners, played soccer and ran with a team, all
apparently while contagious and without heavy symptoms.

Image

New beds arriving last month at a hospital in Codogno, near Lodi in
Northern Italy.Credit...Luca Bruno/Associated Press

Mr. Ricciardi said Italy had the bad luck of having a super spreader in
a densely populated and dynamic area who went to the hospital not once,
but twice, infecting hundreds of people, including doctors and nurses.

``He was incredibly active,'' Mr. Ricciardi said.

But he also had not had any direct contacts with China, and experts
suspect he contracted the virus from another European, meaning Italy did
not have an identifiable patient zero or a traceable source of contagion
that could help it contain the virus.

The virus had already been active in Italy for weeks by that time,
experts now say, passed by people without symptoms and often mistaken
for a flu. It spread around Lombardy, the Italian region that has by far
the most trade with China and the home of
\href{https://www.nytimes3xbfgragh.onion/2020/02/24/world/europe/24coronavirus-milan-italy.html?searchResultPosition=40}{Milan},
the country's most culturally vibrant and business-centered city.

``Who we call `Patient One' was probably `Patient 200,' '' said Fabrizio
Pregliasco, an epidemiologist.

On Sunday, Feb. 23, the number of infections clicked past 130 and
\href{https://www.nytimes3xbfgragh.onion/2020/02/23/world/europe/italy-coronavirus.html?searchResultPosition=3}{Italy
sealed off} 11 towns with police and military checkpoints. The last days
of Venice Carnival were canceled. The Lombardy region closed its
schools, museums and movie theaters. The Milanese made a run on the
supermarkets.

Image

Disinfecting around the central train station in Milan last
week.Credit...Alessandro Grassani for The New York Times

But while Mr. Conte again commended Italy for its firm hand, he also
sought to downplay the contagion, attributing the high numbers of
infected to Lombardy's overzealous testing.

``We have been the first ones with the most rigorous and accurate
controls,'' he said on television, adding that more people in Italy
appeared infected because ``we did more tests.''

The next day, as infections surpassed 200, seven people died and the
stock market plunged, Mr. Conte and his health aides doubled down.

He blamed the Codogno hospital for the spread, saying it had handled
things in ``a not-completely-proper way'' and argued that Lombardy and
Veneto, another northern region, were inflating the severity of the
problem by diverging from global guidelines and testing people without
symptoms.

As Lombardy officials scrambled to free up hospital beds, and the number
of infected people rose to 309 with 11 dead, Mr. Conte said on Feb. 25
that ``Italy is a safe country and probably safer than many others.''

On Friday, Mr. Conte's office offered an interview on the condition that
he could answer questions in writing. When sent questions, including
those about his past statements, he declined to respond.

\hypertarget{mixed-messages-sow-confusion}{%
\subsection{Mixed Messages Sow
Confusion}\label{mixed-messages-sow-confusion}}

Reassurances from leaders confused the Italian population.

On Feb. 27, Mr. Zingaretti posted his aperitivo picture. That same day,
the country's foreign minister, Luigi Di Maio, the former leader of one
of the governing parties, the Five Star Movement, held a news conference
in Rome.

``In Italy, we went from the risk of an epidemic to an infodemic,'' Mr.
Di Maio said, disparaging media coverage that highlighted the threat of
the contagion, and adding that only ``0.089 percent'' of the Italian
population was quarantined.

\href{https://www.nytimes3xbfgragh.onion/2020/02/27/world/europe/milan-coronavirus.html?searchResultPosition=6}{In
Milan}, only miles from the center of the outbreak, the mayor, Beppe
Sala, publicized a ``Milan Doesn't Stop'' campaign, and the Duomo, the
city's landmark cathedral that is a draw for tourists, reopened. People
went out.

Image

A crowded wine bar in Milan at the end of February.Credit...Andrea
Mantovani for The New York Times

But on the sixth floor of the regional government headquarters in Milan,
Giacomo Grasselli, who is the coordinator of the intensive care units
throughout Lombardy, saw the numbers going up and quickly realized that
it would be impossible to treat all the sick if the infections continued
unabated.

\href{https://www.nytimes3xbfgragh.onion/news-event/coronavirus?action=click\&pgtype=Article\&state=default\&region=MAIN_CONTENT_3\&context=storylines_faq}{}

\hypertarget{the-coronavirus-outbreak-}{%
\subsubsection{The Coronavirus Outbreak
›}\label{the-coronavirus-outbreak-}}

\hypertarget{frequently-asked-questions}{%
\paragraph{Frequently Asked
Questions}\label{frequently-asked-questions}}

Updated September 4, 2020

\begin{itemize}
\item ~
  \hypertarget{what-are-the-symptoms-of-coronavirus}{%
  \paragraph{What are the symptoms of
  coronavirus?}\label{what-are-the-symptoms-of-coronavirus}}

  \begin{itemize}
  \tightlist
  \item
    In the beginning, the coronavirus
    \href{https://www.nytimes3xbfgragh.onion/article/coronavirus-facts-history.html?action=click\&pgtype=Article\&state=default\&region=MAIN_CONTENT_3\&context=storylines_faq\#link-6817bab5}{seemed
    like it was primarily a respiratory illness}~--- many patients had
    fever and chills, were weak and tired, and coughed a lot, though
    some people don't show many symptoms at all. Those who seemed
    sickest had pneumonia or acute respiratory distress syndrome and
    received supplemental oxygen. By now, doctors have identified many
    more symptoms and syndromes. In April,
    \href{https://www.nytimes3xbfgragh.onion/2020/04/27/health/coronavirus-symptoms-cdc.html?action=click\&pgtype=Article\&state=default\&region=MAIN_CONTENT_3\&context=storylines_faq}{the
    C.D.C. added to the list of early signs}~sore throat, fever, chills
    and muscle aches. Gastrointestinal upset, such as diarrhea and
    nausea, has also been observed. Another telltale sign of infection
    may be a sudden, profound diminution of one's
    \href{https://www.nytimes3xbfgragh.onion/2020/03/22/health/coronavirus-symptoms-smell-taste.html?action=click\&pgtype=Article\&state=default\&region=MAIN_CONTENT_3\&context=storylines_faq}{sense
    of smell and taste.}~Teenagers and young adults in some cases have
    developed painful red and purple lesions on their fingers and toes
    --- nicknamed ``Covid toe'' --- but few other serious symptoms.
  \end{itemize}
\item ~
  \hypertarget{why-is-it-safer-to-spend-time-together-outside}{%
  \paragraph{Why is it safer to spend time together
  outside?}\label{why-is-it-safer-to-spend-time-together-outside}}

  \begin{itemize}
  \tightlist
  \item
    \href{https://www.nytimes3xbfgragh.onion/2020/05/15/us/coronavirus-what-to-do-outside.html?action=click\&pgtype=Article\&state=default\&region=MAIN_CONTENT_3\&context=storylines_faq}{Outdoor
    gatherings}~lower risk because wind disperses viral droplets, and
    sunlight can kill some of the virus. Open spaces prevent the virus
    from building up in concentrated amounts and being inhaled, which
    can happen when infected people exhale in a confined space for long
    stretches of time, said Dr. Julian W. Tang, a virologist at the
    University of Leicester.
  \end{itemize}
\item ~
  \hypertarget{why-does-standing-six-feet-away-from-others-help}{%
  \paragraph{Why does standing six feet away from others
  help?}\label{why-does-standing-six-feet-away-from-others-help}}

  \begin{itemize}
  \tightlist
  \item
    The coronavirus spreads primarily through droplets from your mouth
    and nose, especially when you cough or sneeze. The C.D.C., one of
    the organizations using that measure,
    \href{https://www.nytimes3xbfgragh.onion/2020/04/14/health/coronavirus-six-feet.html?action=click\&pgtype=Article\&state=default\&region=MAIN_CONTENT_3\&context=storylines_faq}{bases
    its recommendation of six feet}~on the idea that most large droplets
    that people expel when they cough or sneeze will fall to the ground
    within six feet. But six feet has never been a magic number that
    guarantees complete protection. Sneezes, for instance, can launch
    droplets a lot farther than six feet,
    \href{https://jamanetwork.com/journals/jama/fullarticle/2763852}{according
    to a recent study}. It's a rule of thumb: You should be safest
    standing six feet apart outside, especially when it's windy. But
    keep a mask on at all times, even when you think you're far enough
    apart.
  \end{itemize}
\item ~
  \hypertarget{i-have-antibodies-am-i-now-immune}{%
  \paragraph{I have antibodies. Am I now
  immune?}\label{i-have-antibodies-am-i-now-immune}}

  \begin{itemize}
  \tightlist
  \item
    As of right
    now,\href{https://www.nytimes3xbfgragh.onion/2020/07/22/health/covid-antibodies-herd-immunity.html?action=click\&pgtype=Article\&state=default\&region=MAIN_CONTENT_3\&context=storylines_faq}{~that
    seems likely, for at least several months.}~There have been
    frightening accounts of people suffering what seems to be a second
    bout of Covid-19. But experts say these patients may have a
    drawn-out course of infection, with the virus taking a slow toll
    weeks to months after initial exposure.~People infected with the
    coronavirus typically
    \href{https://www.nature.com/articles/s41586-020-2456-9}{produce}~immune
    molecules called antibodies, which are
    \href{https://www.nytimes3xbfgragh.onion/2020/05/07/health/coronavirus-antibody-prevalence.html?action=click\&pgtype=Article\&state=default\&region=MAIN_CONTENT_3\&context=storylines_faq}{protective
    proteins made in response to an
    infection}\href{https://www.nytimes3xbfgragh.onion/2020/05/07/health/coronavirus-antibody-prevalence.html?action=click\&pgtype=Article\&state=default\&region=MAIN_CONTENT_3\&context=storylines_faq}{.
    These antibodies may}~last in the body
    \href{https://www.nature.com/articles/s41591-020-0965-6}{only two to
    three months}, which may seem worrisome, but that's~perfectly normal
    after an acute infection subsides, said Dr. Michael Mina, an
    immunologist at Harvard University. It may be possible to get the
    coronavirus again, but it's highly unlikely that it would be
    possible in a short window of time from initial infection or make
    people sicker the second time.
  \end{itemize}
\item ~
  \hypertarget{what-are-my-rights-if-i-am-worried-about-going-back-to-work}{%
  \paragraph{What are my rights if I am worried about going back to
  work?}\label{what-are-my-rights-if-i-am-worried-about-going-back-to-work}}

  \begin{itemize}
  \tightlist
  \item
    Employers have to provide
    \href{https://www.osha.gov/SLTC/covid-19/standards.html}{a safe
    workplace}~with policies that protect everyone equally.
    \href{https://www.nytimes3xbfgragh.onion/article/coronavirus-money-unemployment.html?action=click\&pgtype=Article\&state=default\&region=MAIN_CONTENT_3\&context=storylines_faq}{And
    if one of your co-workers tests positive for the coronavirus, the
    C.D.C.}~has said that
    \href{https://www.cdc.gov/coronavirus/2019-ncov/community/guidance-business-response.html}{employers
    should tell their employees}~-\/- without giving you the sick
    employee's name -\/- that they may have been exposed to the virus.
  \end{itemize}
\end{itemize}

His task force worked to match the sick to beds in intensive-care units
in the nearest possible hospitals and appropriate dwindling resources.

At one of the daily meetings of about 20 health and political officials,
he told the regional president, Attilio Fontana, about the growing
numbers.

An epidemiologist showed the curves of infection. There was a
catastrophe facing the region's well-respected health system.

``We need to do something more,'' Mr. Grasselli told the room.

Mr. Fontana, who had been pressing the central government for tougher
action, agreed. He said that the mixed messages from Rome and the easing
of restrictions had led Italians to believe ``that everything was a
joke, and they kept living as they used to.''

He said he appealed for tougher national measures in video conferences
with the prime minister and other regional presidents, arguing that
climbing numbers of cases threatened to collapse the hospital system in
the north, but that his requests were repeatedly turned down.

``They were convinced that the situation was less serious and they did
not want to hurt our economy too much,'' said Mr. Fontana.

The government started providing some economic assistance, which would
later be followed by a 25 billion euro (\$28 billion) relief package,
but the nation became divided between
\href{https://www.nytimes3xbfgragh.onion/2020/02/28/world/europe/italy-coronavirus-quarantine.html?action=click\&module=RelatedLinks\&pgtype=Article}{those
who saw the threat and those who didn't}.

Ms. Zampa said that it was around that time that government learned that
infections in the town of Vò, the virus epicenter of the Veneto region,
had no epidemiological link to the Codogno outbreak.

Image

A police checkpoint in Viale Porpora in Milan amid a lockdown in Italy
this month.Credit...Alessandro Grassani for The New York Times

She said that the health minister, Mr. Speranza, and Mr. Conte
deliberated about what to do and within the day, they decided to close
down much of the north.

In a surprise 2 a.m. news conference on March **** 8, when 7,375 people
had already tested positive for coronavirus and 366 had died,
\href{https://www.nytimes3xbfgragh.onion/2020/03/07/world/europe/coronavirus-italy.html?action=click\&module=RelatedLinks\&pgtype=Article}{Mr.
Conte announced the extraordinary step} of restricting movement for
about a quarter of the Italian population in the northern regions that
serve as the country's economic engine.

``We are facing an emergency,'' Mr. Conte said at the time. ``A national
emergency.''

A draft of the decree, leaked to Italian media on Saturday night, pushed
many Milan residents to rush to the train station in crowds and attempt
to leave the region, causing what many later considered a dangerous wave
of contagion toward the south.

Yet the following day, most Italians were still confused about the
severity of the restrictions.

To clarify the issue, the interior ministry issued
``auto-certification'' forms that would allow people to travel in and
out of the locked-down area for work, health or ``other'' necessities.

In the meantime, some regional governors independently ordered people
coming from the newly locked-down area to self quarantine. Others
didn't.

The broader restrictions in Lombardy also effectively lifted the
quarantine on Codogno and other ``red zone'' towns linked to the
original outbreak. Checkpoints disappeared. Local mayors complained that
their sacrifices had been wasted.

A day later, on March 9, when the positive cases reached 9,172 and the
death toll climbed to 463, Mr. Conte toughened the restrictions and
\href{https://www.nytimes3xbfgragh.onion/2020/03/09/world/europe/italy-lockdown-coronavirus.html?action=click\&module=RelatedLinks\&pgtype=Article}{extended
them nationally.}

But by then, some experts say, it was already too late.

\hypertarget{local-experiments}{%
\subsection{Local Experiments}\label{local-experiments}}

Italy is still paying the price of those early mixed messages by
scientists and politicians. The people who have died in staggering
numbers recently --- more than 2,300 in the last four days --- were
mostly infected during the confusion of a week or two ago.

Roberto Burioni, a prominent virologist at the San Raffaele University
in Milan, said that people had felt safe to go about their usual
routines and he attributed the spike in cases last week to ``that
behavior.''

Image

Italian military guarding a roadblock leading to the village of
Vo'Euganeo.Credit...Claudio Furlan/LaPresse, via Associated Press

The government has urged national unity in obeying its restrictive
measures. But on Saturday, hundreds of mayors from the hardest-hit areas
told the government those measures were fatally insufficient.

Leaders in the north are desperate for the government to crack down
harder.

On Friday, Mr. Fontana complained that the 114 troops the government
deployed were insignificant, and that at least 1,000 should be sent. On
Saturday, he closed public offices, work sites and banned jogging. ****
He said in an interview that the government needed to stop messing
around and ``apply rigid measures.''

``My idea is that if we had shut everything in the beginning, for two
weeks, probably now we would be celebrating victory,'' he said.

His political ally, Luca Zaia, the president of the Veneto region,
pre-empted the national government with his own crackdown, and said that
Rome needed to enforce ``a more drastic isolation,'' including closing
all stores and prohibiting public activities other than commuting to
work.

``Walks should be banned,'' he said.

Mr. Zaia has some credibility on the issue. As new infections have
proliferated around the country, they have significantly dropped in Vò,
a town of about 3,000 people that was one of the first quarantined and
which had the country's first coronavirus death.

Some government experts attributed that turnaround to the strict
quarantine that had been in place for two weeks. But Mr. Zaia had also
ordered blanket tests there, in defiance of international scientific
guidelines and the national government. The government has argued that
testing people without symptoms is a drain on resources.

``At least this slows down the virus' speed,'' Mr. Zaia said, arguing
that testing helped identify potentially contagious people without
symptoms. ``And slowing down the virus' speed allows the hospitals to
breathe.''

If not, the overwhelming number of patients would crater health care
systems and cause a national catastrophe.

Americans and others, he said, ``need to be ready.''

Image

The Pirellone, an iconic building in Milan, was illuminated with the
message ``Stay at Home.''Credit...Alessandro Grassani for The New York
Times

Advertisement

\protect\hyperlink{after-bottom}{Continue reading the main story}

\hypertarget{site-index}{%
\subsection{Site Index}\label{site-index}}

\hypertarget{site-information-navigation}{%
\subsection{Site Information
Navigation}\label{site-information-navigation}}

\begin{itemize}
\tightlist
\item
  \href{https://help.nytimes3xbfgragh.onion/hc/en-us/articles/115014792127-Copyright-notice}{©~2020~The
  New York Times Company}
\end{itemize}

\begin{itemize}
\tightlist
\item
  \href{https://www.nytco.com/}{NYTCo}
\item
  \href{https://help.nytimes3xbfgragh.onion/hc/en-us/articles/115015385887-Contact-Us}{Contact
  Us}
\item
  \href{https://www.nytco.com/careers/}{Work with us}
\item
  \href{https://nytmediakit.com/}{Advertise}
\item
  \href{http://www.tbrandstudio.com/}{T Brand Studio}
\item
  \href{https://www.nytimes3xbfgragh.onion/privacy/cookie-policy\#how-do-i-manage-trackers}{Your
  Ad Choices}
\item
  \href{https://www.nytimes3xbfgragh.onion/privacy}{Privacy}
\item
  \href{https://help.nytimes3xbfgragh.onion/hc/en-us/articles/115014893428-Terms-of-service}{Terms
  of Service}
\item
  \href{https://help.nytimes3xbfgragh.onion/hc/en-us/articles/115014893968-Terms-of-sale}{Terms
  of Sale}
\item
  \href{https://spiderbites.nytimes3xbfgragh.onion}{Site Map}
\item
  \href{https://help.nytimes3xbfgragh.onion/hc/en-us}{Help}
\item
  \href{https://www.nytimes3xbfgragh.onion/subscription?campaignId=37WXW}{Subscriptions}
\end{itemize}
