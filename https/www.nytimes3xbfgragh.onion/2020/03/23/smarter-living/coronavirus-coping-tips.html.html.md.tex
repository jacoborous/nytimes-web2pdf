Sections

SEARCH

\protect\hyperlink{site-content}{Skip to
content}\protect\hyperlink{site-index}{Skip to site index}

\href{https://www.nytimes3xbfgragh.onion/section/smarter-living}{Smarter
Living}

\href{https://myaccount.nytimes3xbfgragh.onion/auth/login?response_type=cookie\&client_id=vi}{}

\href{https://www.nytimes3xbfgragh.onion/section/todayspaper}{Today's
Paper}

\href{/section/smarter-living}{Smarter Living}\textbar{}How to Stay Sane
When the World Seems Crazy

\url{https://nyti.ms/2UAZ8Ee}

\begin{itemize}
\item
\item
\item
\item
\item
\end{itemize}

\hypertarget{the-coronavirus-outbreak}{%
\subsubsection{\texorpdfstring{\href{https://www.nytimes3xbfgragh.onion/news-event/coronavirus?name=styln-coronavirus-national\&region=TOP_BANNER\&block=storyline_menu_recirc\&action=click\&pgtype=Article\&impression_id=061d5590-efbb-11ea-b9a5-71e5d1ca15bb\&variant=undefined}{The
Coronavirus
Outbreak}}{The Coronavirus Outbreak}}\label{the-coronavirus-outbreak}}

\begin{itemize}
\tightlist
\item
  live\href{https://www.nytimes3xbfgragh.onion/2020/09/05/world/coronavirus-covid.html?name=styln-coronavirus-national\&region=TOP_BANNER\&block=storyline_menu_recirc\&action=click\&pgtype=Article\&impression_id=061d7ca0-efbb-11ea-b9a5-71e5d1ca15bb\&variant=undefined}{Latest
  Updates}
\item
  \href{https://www.nytimes3xbfgragh.onion/interactive/2020/us/coronavirus-us-cases.html?name=styln-coronavirus-national\&region=TOP_BANNER\&block=storyline_menu_recirc\&action=click\&pgtype=Article\&impression_id=061d7ca1-efbb-11ea-b9a5-71e5d1ca15bb\&variant=undefined}{Maps
  and Cases}
\item
  \href{https://www.nytimes3xbfgragh.onion/interactive/2020/science/coronavirus-vaccine-tracker.html?name=styln-coronavirus-national\&region=TOP_BANNER\&block=storyline_menu_recirc\&action=click\&pgtype=Article\&impression_id=061d7ca2-efbb-11ea-b9a5-71e5d1ca15bb\&variant=undefined}{Vaccine
  Tracker}
\item
  \href{https://www.nytimes3xbfgragh.onion/2020/09/02/your-money/eviction-moratorium-covid.html?name=styln-coronavirus-national\&region=TOP_BANNER\&block=storyline_menu_recirc\&action=click\&pgtype=Article\&impression_id=061da3b0-efbb-11ea-b9a5-71e5d1ca15bb\&variant=undefined}{Eviction
  Moratorium}
\item
  \href{https://www.nytimes3xbfgragh.onion/interactive/2020/09/02/magazine/food-insecurity-hunger-us.html?name=styln-coronavirus-national\&region=TOP_BANNER\&block=storyline_menu_recirc\&action=click\&pgtype=Article\&impression_id=061da3b1-efbb-11ea-b9a5-71e5d1ca15bb\&variant=undefined}{American
  Hunger}
\end{itemize}

Advertisement

\protect\hyperlink{after-top}{Continue reading the main story}

Supported by

\protect\hyperlink{after-sponsor}{Continue reading the main story}

\hypertarget{how-to-stay-sane-when-the-world-seems-crazy}{%
\section{How to Stay Sane When the World Seems
Crazy}\label{how-to-stay-sane-when-the-world-seems-crazy}}

Stop and take a breath. The world will keep spinning.

\includegraphics{https://static01.graylady3jvrrxbe.onion/images/2020/03/30/smarter-living/30sl-stay-sane/21sl-stay-sane-articleLarge-v2.jpg?quality=75\&auto=webp\&disable=upscale}

By Allie Volpe

\begin{itemize}
\item
  Published March 23, 2020Updated March 30, 2020
\item
  \begin{itemize}
  \item
  \item
  \item
  \item
  \item
  \end{itemize}
\end{itemize}

Our constant, relentless exposure to news and headlines has a way of
inspiring near-constant dread. As distressing news continually filters
to the top of our feeds, phones and TVs, it isn't uncommon to feel more
than a little
\href{https://www.nbcnews.com/better/health/what-headline-stress-disorder-do-you-have-it-ncna830141}{nervous
about the state of the world}.

And often, many people are. Over 50 percent of Americans want to stay
informed on current events but say following the news is a source of
stress, according to the American Psychological Association's 2019
``\href{https://www.apa.org/news/press/releases/stress/2019/stress-america-2019.pdf}{Stress
In America}'' survey. More broadly, Americans are among the
\href{https://www.nytimes3xbfgragh.onion/2019/04/25/us/americans-stressful.html}{world's
most stressed people}, with 55 percent of adults saying they
experiencing stress during ``a lot of the day'' prior, according to a
Gallup poll.

It's easy to turn on the news and believe the world is ending. When a
large-scale news event --- say, a pandemic --- affects many groups,
people want to discuss it more widely and frequently, said
\href{https://kathleensmith.net/}{Dr. Kathleen Smith}, a therapist and
author of
``\href{https://www.hachettebooks.com/titles/kathleen-smith/everything-isnt-terrible/9780316492553/}{Everything
Isn't Terrible: Conquer Your Insecurities, Interrupt Your Anxiety, and
Finally Calm Down}.''

This constant conversation can lead to a snowballing of negative
thoughts.
\href{https://www.medicalnewstoday.com/articles/320844}{Catastrophizing},
or a pattern of thinking that jumps to the worst-case scenario, is an
evolutionary response to threat, Dr. Smith said.

``Humans are able to imagine the worst-case scenario, which is a trait
most other animals do not have,'' she said. ``That ability to do that
and plan ahead has helped us survive. It has gotten in the way because
we have a lot of reality-based problems today that need solving.''

There are ways to cope when things are rough --- and ways to remind
yourself the world will keep on spinning.

\hypertarget{why-we-catastrophize}{%
\subsection{Why we catastrophize}\label{why-we-catastrophize}}

``When people catastrophize, in many ways, it's a maladaptive way of
trying to regain control,'' said Dr. David Rosmarin, the founder and
director of the \href{http://www.centerforanxiety.org/}{Center for
Anxiety} and an
\href{http://www.mcleanhospital.org/biography/david-h-rosmarin}{assistant
professor in the department of psychiatry at Harvard Medical School}.

\hypertarget{latest-updates-the-coronavirus-outbreak}{%
\section{\texorpdfstring{\href{https://www.nytimes3xbfgragh.onion/2020/09/04/world/covid-19-coronavirus.html?action=click\&pgtype=Article\&state=default\&region=MAIN_CONTENT_1\&context=storylines_live_updates}{Latest
Updates: The Coronavirus
Outbreak}}{Latest Updates: The Coronavirus Outbreak}}\label{latest-updates-the-coronavirus-outbreak}}

Updated 2020-09-05T12:05:40.998Z

\begin{itemize}
\tightlist
\item
  \href{https://www.nytimes3xbfgragh.onion/2020/09/04/world/covid-19-coronavirus.html?action=click\&pgtype=Article\&state=default\&region=MAIN_CONTENT_1\&context=storylines_live_updates\#link-1654f6ad}{Research
  connects vaping to a higher chance of catching the virus --- and
  suffering its worst effects.}
\item
  \href{https://www.nytimes3xbfgragh.onion/2020/09/04/world/covid-19-coronavirus.html?action=click\&pgtype=Article\&state=default\&region=MAIN_CONTENT_1\&context=storylines_live_updates\#link-52e4198a}{Another
  college football game won't be played as planned.}
\item
  \href{https://www.nytimes3xbfgragh.onion/2020/09/04/world/covid-19-coronavirus.html?action=click\&pgtype=Article\&state=default\&region=MAIN_CONTENT_1\&context=storylines_live_updates\#link-181cef0}{Pharmaceutical
  companies plan a joint pledge on safety standards as they move
  vaccines to the marketplace.}
\end{itemize}

\href{https://www.nytimes3xbfgragh.onion/2020/09/04/world/covid-19-coronavirus.html?action=click\&pgtype=Article\&state=default\&region=MAIN_CONTENT_1\&context=storylines_live_updates}{See
more updates}

More live coverage:
\href{https://www.nytimes3xbfgragh.onion/live/2020/09/04/business/stock-market-today-coronavirus?action=click\&pgtype=Article\&state=default\&region=MAIN_CONTENT_1\&context=storylines_live_updates}{Markets}

We try to regulate our emotions when life feels out of control, Dr.
Rosmarin said. But anticipating ultimate doom and gloom as a means of
taking control in uncertain times is not particularly effective. Jumping
to worst-case scenarios breeds poor decision-making, he said: People
tend to adopt a ``who cares'' attitude, which can contribute to
hopelessness and despair.

Sometimes the catastrophic thoughts can become self-fulfilling
prophecies, Dr. Smith said. For example: A widespread panic about a
toilet paper shortage indeed resulted in a mass of shoppers rushing to
buy toilet paper, thus creating a shortage. ``We think we need to fix
the problem, whether it's based in reality or not,'' she said.

\hypertarget{accept-uncertainty}{%
\subsection{Accept uncertainty}\label{accept-uncertainty}}

Although recent history may paint a tumultuous picture, we live in
relatively safe times, Dr. Rosmarin said. Less than a century ago, he
said, real, consistent threats of war were a reality in ways to which
we're now unaccustomed. (And constant news updates weren't even present
to perpetually stoke fear.)

Because of that general feeling of security, we're not used to dealing
with uncertainty, Dr. Rosmarin said. To better accept the unknown, we
have to relinquish control, he said, and maintain trust that the powers
that be are working to solve large-scale issues --- which is what we
subconsciously do any time we use public transit and airplanes, for
example.

``When the cabin door to the cockpit closes and I'm not the one
inside,'' he said, ``I'm happy because I don't know how to fly a plane
and you don't want me flying a plane.''

\hypertarget{stick-to-the-facts}{%
\subsection{Stick to the facts}\label{stick-to-the-facts}}

Anxiety makes us feel powerless, said
\href{https://www.compassionpower.com/about-us/}{Dr. Steven Stosny}, a
therapist who coined the term
``\href{https://www.washingtonpost.com/news/inspired-life/wp/2017/02/06/suffering-from-headline-stress-disorder-since-trumps-win-youre-definitely-not-alone/?utm_term=.66411356313d\&noredirect=on}{Headline
Stress Disorder},'' or the feeling of stress borne from the news. A
sense of powerlessness then breeds fear that we won't be able to handle
the consequences of a terrible event, whether unemployment or sickness.
However, we tend to exaggerate the severity of the threat and
underestimate our ability to cope, he said.

``We cope better than we think we will,'' he said. ``And that's
survival.''

Instead of feeling powerless, evaluate what you know to be true in this
moment --- and don't exaggerate --- to help ground you. Think: \emph{I
have my health, I have my family, I can still make delicious meals}.

Take stock of your reality by asking yourself straightforward questions,
like, ``What are my responsibilities to myself, my family and the larger
community?'' and ``What reality-based problems do I need to solve
today?'' Dr. Smith
\href{https://theanxiousoverachiever.substack.com/p/20-questions-to-help-with-covid-19}{suggested}.

``To me, that's being very responsible because you're responding to
reality and not the nightmare, which is easy to,'' she said. ``If you
jump to the worst-case scenario it doesn't equip you to help yourself in
any way. You freeze up because it becomes unmanageable.''

\hypertarget{avoid-all-or-nothing-thinking}{%
\subsection{Avoid all-or-nothing
thinking}\label{avoid-all-or-nothing-thinking}}

When news and facts are constantly changing, it can be easy to jump to
conclusions and fill in the blanks, Dr. Smith said. However, we
shouldn't rush to process current events with black-and-white thinking.
Absolutist, or all-or-nothing, thinking,
\href{https://www.psychologicalscience.org/publications/observer/obsonline/all-or-nothing-thinking-more-common-in-people-with-anxiety-depression-and-suicidal-ideation.html}{isn't
a healthy way to cope}, and is common among those with depression,
researchers
\href{https://journals.sagepub.com/doi/full/10.1177/2167702617747074}{found
in 2018}.

\href{https://www.nytimes3xbfgragh.onion/news-event/coronavirus?action=click\&pgtype=Article\&state=default\&region=MAIN_CONTENT_3\&context=storylines_faq}{}

\hypertarget{the-coronavirus-outbreak-}{%
\subsubsection{The Coronavirus Outbreak
›}\label{the-coronavirus-outbreak-}}

\hypertarget{frequently-asked-questions}{%
\paragraph{Frequently Asked
Questions}\label{frequently-asked-questions}}

Updated September 4, 2020

\begin{itemize}
\item ~
  \hypertarget{what-are-the-symptoms-of-coronavirus}{%
  \paragraph{What are the symptoms of
  coronavirus?}\label{what-are-the-symptoms-of-coronavirus}}

  \begin{itemize}
  \tightlist
  \item
    In the beginning, the coronavirus
    \href{https://www.nytimes3xbfgragh.onion/article/coronavirus-facts-history.html?action=click\&pgtype=Article\&state=default\&region=MAIN_CONTENT_3\&context=storylines_faq\#link-6817bab5}{seemed
    like it was primarily a respiratory illness}~--- many patients had
    fever and chills, were weak and tired, and coughed a lot, though
    some people don't show many symptoms at all. Those who seemed
    sickest had pneumonia or acute respiratory distress syndrome and
    received supplemental oxygen. By now, doctors have identified many
    more symptoms and syndromes. In April,
    \href{https://www.nytimes3xbfgragh.onion/2020/04/27/health/coronavirus-symptoms-cdc.html?action=click\&pgtype=Article\&state=default\&region=MAIN_CONTENT_3\&context=storylines_faq}{the
    C.D.C. added to the list of early signs}~sore throat, fever, chills
    and muscle aches. Gastrointestinal upset, such as diarrhea and
    nausea, has also been observed. Another telltale sign of infection
    may be a sudden, profound diminution of one's
    \href{https://www.nytimes3xbfgragh.onion/2020/03/22/health/coronavirus-symptoms-smell-taste.html?action=click\&pgtype=Article\&state=default\&region=MAIN_CONTENT_3\&context=storylines_faq}{sense
    of smell and taste.}~Teenagers and young adults in some cases have
    developed painful red and purple lesions on their fingers and toes
    --- nicknamed ``Covid toe'' --- but few other serious symptoms.
  \end{itemize}
\item ~
  \hypertarget{why-is-it-safer-to-spend-time-together-outside}{%
  \paragraph{Why is it safer to spend time together
  outside?}\label{why-is-it-safer-to-spend-time-together-outside}}

  \begin{itemize}
  \tightlist
  \item
    \href{https://www.nytimes3xbfgragh.onion/2020/05/15/us/coronavirus-what-to-do-outside.html?action=click\&pgtype=Article\&state=default\&region=MAIN_CONTENT_3\&context=storylines_faq}{Outdoor
    gatherings}~lower risk because wind disperses viral droplets, and
    sunlight can kill some of the virus. Open spaces prevent the virus
    from building up in concentrated amounts and being inhaled, which
    can happen when infected people exhale in a confined space for long
    stretches of time, said Dr. Julian W. Tang, a virologist at the
    University of Leicester.
  \end{itemize}
\item ~
  \hypertarget{why-does-standing-six-feet-away-from-others-help}{%
  \paragraph{Why does standing six feet away from others
  help?}\label{why-does-standing-six-feet-away-from-others-help}}

  \begin{itemize}
  \tightlist
  \item
    The coronavirus spreads primarily through droplets from your mouth
    and nose, especially when you cough or sneeze. The C.D.C., one of
    the organizations using that measure,
    \href{https://www.nytimes3xbfgragh.onion/2020/04/14/health/coronavirus-six-feet.html?action=click\&pgtype=Article\&state=default\&region=MAIN_CONTENT_3\&context=storylines_faq}{bases
    its recommendation of six feet}~on the idea that most large droplets
    that people expel when they cough or sneeze will fall to the ground
    within six feet. But six feet has never been a magic number that
    guarantees complete protection. Sneezes, for instance, can launch
    droplets a lot farther than six feet,
    \href{https://jamanetwork.com/journals/jama/fullarticle/2763852}{according
    to a recent study}. It's a rule of thumb: You should be safest
    standing six feet apart outside, especially when it's windy. But
    keep a mask on at all times, even when you think you're far enough
    apart.
  \end{itemize}
\item ~
  \hypertarget{i-have-antibodies-am-i-now-immune}{%
  \paragraph{I have antibodies. Am I now
  immune?}\label{i-have-antibodies-am-i-now-immune}}

  \begin{itemize}
  \tightlist
  \item
    As of right
    now,\href{https://www.nytimes3xbfgragh.onion/2020/07/22/health/covid-antibodies-herd-immunity.html?action=click\&pgtype=Article\&state=default\&region=MAIN_CONTENT_3\&context=storylines_faq}{~that
    seems likely, for at least several months.}~There have been
    frightening accounts of people suffering what seems to be a second
    bout of Covid-19. But experts say these patients may have a
    drawn-out course of infection, with the virus taking a slow toll
    weeks to months after initial exposure.~People infected with the
    coronavirus typically
    \href{https://www.nature.com/articles/s41586-020-2456-9}{produce}~immune
    molecules called antibodies, which are
    \href{https://www.nytimes3xbfgragh.onion/2020/05/07/health/coronavirus-antibody-prevalence.html?action=click\&pgtype=Article\&state=default\&region=MAIN_CONTENT_3\&context=storylines_faq}{protective
    proteins made in response to an
    infection}\href{https://www.nytimes3xbfgragh.onion/2020/05/07/health/coronavirus-antibody-prevalence.html?action=click\&pgtype=Article\&state=default\&region=MAIN_CONTENT_3\&context=storylines_faq}{.
    These antibodies may}~last in the body
    \href{https://www.nature.com/articles/s41591-020-0965-6}{only two to
    three months}, which may seem worrisome, but that's~perfectly normal
    after an acute infection subsides, said Dr. Michael Mina, an
    immunologist at Harvard University. It may be possible to get the
    coronavirus again, but it's highly unlikely that it would be
    possible in a short window of time from initial infection or make
    people sicker the second time.
  \end{itemize}
\item ~
  \hypertarget{what-are-my-rights-if-i-am-worried-about-going-back-to-work}{%
  \paragraph{What are my rights if I am worried about going back to
  work?}\label{what-are-my-rights-if-i-am-worried-about-going-back-to-work}}

  \begin{itemize}
  \tightlist
  \item
    Employers have to provide
    \href{https://www.osha.gov/SLTC/covid-19/standards.html}{a safe
    workplace}~with policies that protect everyone equally.
    \href{https://www.nytimes3xbfgragh.onion/article/coronavirus-money-unemployment.html?action=click\&pgtype=Article\&state=default\&region=MAIN_CONTENT_3\&context=storylines_faq}{And
    if one of your co-workers tests positive for the coronavirus, the
    C.D.C.}~has said that
    \href{https://www.cdc.gov/coronavirus/2019-ncov/community/guidance-business-response.html}{employers
    should tell their employees}~-\/- without giving you the sick
    employee's name -\/- that they may have been exposed to the virus.
  \end{itemize}
\end{itemize}

To avoid this thought pattern, give the circumstance nuance. Just
because a handful of events were canceled, for example, doesn't mean the
world is tumbling into isolation --- it means our leaders care about our
safety and are taking precautions. Dr. Smith suggests writing down such
nervous thoughts or giving anxiety a name. ``I call my anxiety Carl,''
she said. ``Carl says the world is probably going to end --- and that
makes me go, \emph{Carl probably doesn't know what he's talking about.}
Sometimes adding a little bit of humor can help.''

\hypertarget{take-care-of-yourself}{%
\subsection{Take care of yourself}\label{take-care-of-yourself}}

\href{https://www.jneurosci.org/content/36/11/3322.abstract}{Research}
has shown anxiety impacts our decision-making skills, and in frenzied
times, you want to make the most informed decisions for yourself and
your family. Keep yourself in tiptop shape with elements of self-care:
Studies have shown that
\href{https://www.health.harvard.edu/blog/can-exercise-help-treat-anxiety-2019102418096}{exercise},
\href{https://www.sciencedaily.com/releases/2019/11/191104124140.htm}{deep
sleep} and
\href{https://www.nytimes3xbfgragh.onion/2017/06/12/well/live/having-friends-is-good-for-you.html}{social
interactions} --- even if it's just a phone call or video chat ---
diminish stress and anxiety. You may also want to step back from social
media or find ways to
\href{https://www.nytimes3xbfgragh.onion/2020/01/15/smarter-living/how-to-fix-social-facebook-instagram-twitter.html}{make
the experience less nerve-racking}.

Perhaps most importantly, cut yourself some slack.

``Don't beat yourself up for worrying,'' Dr. Stosny said. ``That's only
going to make you worry more.''

Even if group gatherings aren't feasible, take part in one-on-one video
hangouts, FaceTime calls and text threads, Dr. Rosmarin suggested.
``Just because we're socially segregated doesn't mean we need to be
socially isolated.''

But remember to turn off the tech eventually. In times of crisis, Dr.
Rosmarin advised avoiding phones and other news sources at least an hour
before bed.

\hypertarget{get-involved}{%
\subsection{Get involved}\label{get-involved}}

Donate or volunteer with an organization you feel is making positive
contributions, whether locally, nationally or internationally. Not only
does volunteer work lower the risk of depression and gives participants
a sense of purpose, it also
\href{https://www.mayoclinichealthsystem.org/hometown-health/speaking-of-health/helping-people-changing-lives-the-6-health-benefits-of-volunteering}{may
reduce stress levels}.

``Anything you do proactively will help,'' Dr. Stosny said. ``It helps
ward off some of the powerlessness or anxiety, even if it's small.''

And it's OK if those charitable efforts end with a virtual happy hour or
dessert as a reward.

Advertisement

\protect\hyperlink{after-bottom}{Continue reading the main story}

\hypertarget{site-index}{%
\subsection{Site Index}\label{site-index}}

\hypertarget{site-information-navigation}{%
\subsection{Site Information
Navigation}\label{site-information-navigation}}

\begin{itemize}
\tightlist
\item
  \href{https://help.nytimes3xbfgragh.onion/hc/en-us/articles/115014792127-Copyright-notice}{©~2020~The
  New York Times Company}
\end{itemize}

\begin{itemize}
\tightlist
\item
  \href{https://www.nytco.com/}{NYTCo}
\item
  \href{https://help.nytimes3xbfgragh.onion/hc/en-us/articles/115015385887-Contact-Us}{Contact
  Us}
\item
  \href{https://www.nytco.com/careers/}{Work with us}
\item
  \href{https://nytmediakit.com/}{Advertise}
\item
  \href{http://www.tbrandstudio.com/}{T Brand Studio}
\item
  \href{https://www.nytimes3xbfgragh.onion/privacy/cookie-policy\#how-do-i-manage-trackers}{Your
  Ad Choices}
\item
  \href{https://www.nytimes3xbfgragh.onion/privacy}{Privacy}
\item
  \href{https://help.nytimes3xbfgragh.onion/hc/en-us/articles/115014893428-Terms-of-service}{Terms
  of Service}
\item
  \href{https://help.nytimes3xbfgragh.onion/hc/en-us/articles/115014893968-Terms-of-sale}{Terms
  of Sale}
\item
  \href{https://spiderbites.nytimes3xbfgragh.onion}{Site Map}
\item
  \href{https://help.nytimes3xbfgragh.onion/hc/en-us}{Help}
\item
  \href{https://www.nytimes3xbfgragh.onion/subscription?campaignId=37WXW}{Subscriptions}
\end{itemize}
