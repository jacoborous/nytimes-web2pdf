Sections

SEARCH

\protect\hyperlink{site-content}{Skip to
content}\protect\hyperlink{site-index}{Skip to site index}

\href{https://www.nytimes3xbfgragh.onion/section/world/europe}{Europe}

\href{https://myaccount.nytimes3xbfgragh.onion/auth/login?response_type=cookie\&client_id=vi}{}

\href{https://www.nytimes3xbfgragh.onion/section/todayspaper}{Today's
Paper}

\href{/section/world/europe}{Europe}\textbar{}Dip in Italy's Cases Does
Not Come Fast Enough for Swamped Hospitals

\url{https://nyti.ms/2UzmlXB}

\begin{itemize}
\item
\item
\item
\item
\item
\item
\end{itemize}

\hypertarget{the-coronavirus-outbreak}{%
\subsubsection{\texorpdfstring{\href{https://www.nytimes3xbfgragh.onion/news-event/coronavirus?name=styln-coronavirus-national\&region=TOP_BANNER\&block=storyline_menu_recirc\&action=click\&pgtype=Article\&impression_id=6fe7a8f0-efba-11ea-b9e2-fd85e1bd9bd3\&variant=undefined}{The
Coronavirus
Outbreak}}{The Coronavirus Outbreak}}\label{the-coronavirus-outbreak}}

\begin{itemize}
\tightlist
\item
  live\href{https://www.nytimes3xbfgragh.onion/2020/09/05/world/coronavirus-covid.html?name=styln-coronavirus-national\&region=TOP_BANNER\&block=storyline_menu_recirc\&action=click\&pgtype=Article\&impression_id=6fe7a8f1-efba-11ea-b9e2-fd85e1bd9bd3\&variant=undefined}{Latest
  Updates}
\item
  \href{https://www.nytimes3xbfgragh.onion/interactive/2020/us/coronavirus-us-cases.html?name=styln-coronavirus-national\&region=TOP_BANNER\&block=storyline_menu_recirc\&action=click\&pgtype=Article\&impression_id=6fe7a8f2-efba-11ea-b9e2-fd85e1bd9bd3\&variant=undefined}{Maps
  and Cases}
\item
  \href{https://www.nytimes3xbfgragh.onion/interactive/2020/science/coronavirus-vaccine-tracker.html?name=styln-coronavirus-national\&region=TOP_BANNER\&block=storyline_menu_recirc\&action=click\&pgtype=Article\&impression_id=6fe7a8f3-efba-11ea-b9e2-fd85e1bd9bd3\&variant=undefined}{Vaccine
  Tracker}
\item
  \href{https://www.nytimes3xbfgragh.onion/2020/09/02/your-money/eviction-moratorium-covid.html?name=styln-coronavirus-national\&region=TOP_BANNER\&block=storyline_menu_recirc\&action=click\&pgtype=Article\&impression_id=6fe7a8f4-efba-11ea-b9e2-fd85e1bd9bd3\&variant=undefined}{Eviction
  Moratorium}
\item
  \href{https://www.nytimes3xbfgragh.onion/interactive/2020/09/02/magazine/food-insecurity-hunger-us.html?name=styln-coronavirus-national\&region=TOP_BANNER\&block=storyline_menu_recirc\&action=click\&pgtype=Article\&impression_id=6fe7a8f5-efba-11ea-b9e2-fd85e1bd9bd3\&variant=undefined}{American
  Hunger}
\end{itemize}

Advertisement

\protect\hyperlink{after-top}{Continue reading the main story}

Supported by

\protect\hyperlink{after-sponsor}{Continue reading the main story}

\hypertarget{dip-in-italys-cases-does-not-come-fast-enough-for-swamped-hospitals}{%
\section{Dip in Italy's Cases Does Not Come Fast Enough for Swamped
Hospitals}\label{dip-in-italys-cases-does-not-come-fast-enough-for-swamped-hospitals}}

Even as new coronavirus infections appear to slow, a backlog is forcing
doctors to make increasingly tough choices about treatment.

\includegraphics{https://static01.graylady3jvrrxbe.onion/images/2020/03/23/world/23virus-italy01/merlin_170588367_5c2442ad-8859-43cb-b490-9657050cf333-articleLarge.jpg?quality=75\&auto=webp\&disable=upscale}

\href{https://www.nytimes3xbfgragh.onion/by/jason-horowitz}{\includegraphics{https://static01.graylady3jvrrxbe.onion/images/2018/10/10/multimedia/author-jason-horowitz/author-jason-horowitz-thumbLarge.png}}\href{https://www.nytimes3xbfgragh.onion/by/david-d-kirkpatrick}{\includegraphics{https://static01.graylady3jvrrxbe.onion/images/2018/10/15/multimedia/author-david-d-kirkpatrick/author-david-d-kirkpatrick-thumbLarge-v2.png}}

By \href{https://www.nytimes3xbfgragh.onion/by/jason-horowitz}{Jason
Horowitz} and
\href{https://www.nytimes3xbfgragh.onion/by/david-d-kirkpatrick}{David
D. Kirkpatrick}

\begin{itemize}
\item
  March 23, 2020
\item
  \begin{itemize}
  \item
  \item
  \item
  \item
  \item
  \item
  \end{itemize}
\end{itemize}

ROME --- The patient had won national swimming championships in his
youth but now had a lot going against him. As he waited for a kidney
transplant, doctors in the northern Italian town of Brescia discovered
he had heart disease and had contracted the coronavirus. But what
ultimately killed him this month was the decision to give his ventilator
to a younger coronavirus patient who had a better shot at survival.

``He died the next day,'' said
\href{https://esc365.escardio.org/Person/1510-prof-metra-marco}{Dr.
Marco Metra}, the chief of cardiology at the University and City
Hospitals in Brescia. ``If a patient has a low likelihood to benefit
from the hospital, we have to not accept them. You send them home.'' He
added, ``This is also what I am seeing every day.''

This is the nightmare situation for doctors throughout the northern
Italian cities at the center of the global coronavirus pandemic. It is
also
\href{https://www.nytimes3xbfgragh.onion/2020/03/21/us/coronavirus-medical-rationing.html}{one
facing the countries} lagging only days behind Italy in the progression
of the pandemic, including Spain, France, Britain and the United States.

Two weeks after the Italian government took strict measures to lock down
the country, officials announced on Monday that, for the second day in a
row, the number of new cases and deaths had declined.

The number of patients in the hospitals in Lombardy, the region that is
by far the hardest hit in Italy, had gone down as well, to 9,266 from
9,439.

``We can say that today is the first positive day,'' said Giulio
Gallera, Lombardy's leading health official. ``It's not the moment to
sing victory, but we finally see light at the end of the tunnel.''

But the first indications of a flattening of the deadly spike in the
contagion have not arrived soon enough for hospitals in the hardest-hit
regions that remain swamped and await a return to relative normalcy.

\includegraphics{https://static01.graylady3jvrrxbe.onion/images/2020/03/23/world/23virus-italy02/merlin_170719524_4357d238-927f-42f7-a783-135076a2d2e3-articleLarge.jpg?quality=75\&auto=webp\&disable=upscale}

The intensive, and lengthy, treatment required to heal hospitalized
patients with the virus means that will not happen anytime soon. Many of
those who have died with the virus were infected weeks ago, before
restrictive measures went into place.

The same is true for many of the patients who are still being treated,
from those being monitored with less serious symptoms to those whose
heads are encased in plastic bubbles filled with oxygen. The worst cases
are unconscious and on respirators.

\hypertarget{latest-updates-the-coronavirus-outbreak}{%
\section{\texorpdfstring{\href{https://www.nytimes3xbfgragh.onion/2020/09/04/world/covid-19-coronavirus.html?action=click\&pgtype=Article\&state=default\&region=MAIN_CONTENT_1\&context=storylines_live_updates}{Latest
Updates: The Coronavirus
Outbreak}}{Latest Updates: The Coronavirus Outbreak}}\label{latest-updates-the-coronavirus-outbreak}}

Updated 2020-09-05T12:05:40.998Z

\begin{itemize}
\tightlist
\item
  \href{https://www.nytimes3xbfgragh.onion/2020/09/04/world/covid-19-coronavirus.html?action=click\&pgtype=Article\&state=default\&region=MAIN_CONTENT_1\&context=storylines_live_updates\#link-1654f6ad}{Research
  connects vaping to a higher chance of catching the virus --- and
  suffering its worst effects.}
\item
  \href{https://www.nytimes3xbfgragh.onion/2020/09/04/world/covid-19-coronavirus.html?action=click\&pgtype=Article\&state=default\&region=MAIN_CONTENT_1\&context=storylines_live_updates\#link-52e4198a}{Another
  college football game won't be played as planned.}
\item
  \href{https://www.nytimes3xbfgragh.onion/2020/09/04/world/covid-19-coronavirus.html?action=click\&pgtype=Article\&state=default\&region=MAIN_CONTENT_1\&context=storylines_live_updates\#link-181cef0}{Pharmaceutical
  companies plan a joint pledge on safety standards as they move
  vaccines to the marketplace.}
\end{itemize}

\href{https://www.nytimes3xbfgragh.onion/2020/09/04/world/covid-19-coronavirus.html?action=click\&pgtype=Article\&state=default\&region=MAIN_CONTENT_1\&context=storylines_live_updates}{See
more updates}

More live coverage:
\href{https://www.nytimes3xbfgragh.onion/live/2020/09/04/business/stock-market-today-coronavirus?action=click\&pgtype=Article\&state=default\&region=MAIN_CONTENT_1\&context=storylines_live_updates}{Markets}

Many of them take an average of weeks to heal. So the new cases, even if
they are fewer, are still added to a bottlenecked system already at
capacity.

In the meantime, patients who can stay at home are urged to do so, while
others requiring intensive care are being flown to other regions or
countries by military aircraft. And doctors in the hardest hit parts of
Italy, and other official voices elsewhere, say the road is narrowing to
increasingly wrenching choices.

In Spain, Fernando Simón, the director of Spain's national health
emergency center, said last Friday that some emergency wards needed to
apply restrictions for admitting patients, because they were saturated.
On Monday, New York City's mayor, Bill de Blasio, issued a similarly
dire warning for his city, predicting that if hospitals did not get more
ventilators this week,
\href{https://www.nytimes3xbfgragh.onion/2020/03/23/nyregion/coronavirus-new-york-update.html}{the
number of deaths would spike}.

In Europe, the shortage of respirators has had tragic consequences. On
Sunday, the Spanish sports director of a Honda motorcycling racing team,
Oscar Haro, released
\href{https://www.youtube.com/watch?time_continue=101\&v=xDN--tZ7Aho\&feature=emb_logo}{a
video} in which he said that his father had died after not being given a
respirator. Both of Mr. Haro's parents tested positive last Monday for
coronavirus, he said, but his father was taken to the hospital while his
mother was kept quarantined at home.

Image

An intensive care unit in Brescia. Fearing hospitals and ambulance
workers could be a source of contagion, some doctors are now urging
patients to stay home.Credit...Claudio Furlan/LaPresse, via Associated
Press

``I cannot understand how a person like my father who had been working
since he was 15, paying into the system, died because there are no
respirators, because they could not treat him anymore,''
\href{https://www.lavanguardia.com/deportes/20200322/4840170014/oscar-haro-padre-fallecido-coronavirus-respiradores-lcr-honda-motogp.html}{Mr.
Haro said}. He blamed the Spanish authorities for failing to prepare
adequately, even after witnessing the spread of the coronavirus in China
and Italy. ``We are allowing to die a generation that built this
country.''

Northern Italy has been on war footing for weeks.

``We are far beyond the tipping point,'' doctors from one Bergamo
hospital wrote in a paper published in a journal from the New England
Journal of Medicine over the weekend, saying that 70 percent of their
intensive care unit beds were reserved for coronavirus patients with ``a
reasonable chance to survive.'' Older patients, they said, ``are not
being resuscitated and
\href{https://www.nytimes3xbfgragh.onion/2020/03/16/world/europe/italy-coronavirus-funerals.html?action=click\&module=RelatedLinks\&pgtype=Article}{die
alone.}''

The near collapse of many of the region's hospitals, and the dearth of
mechanical ventilators, oxygen and personal protective equipment, has
led many doctors to urge patients to stay away. They argue that
overloaded hospitals increasingly seem to be sources of contagion, and
that infected but asymptomatic ambulance workers sent to retrieve
patients in their homes are actually spreading the virus.

In recent days in the area around Bergamo, the families of patients
visited by ambulances said that health workers urged them to stay home
despite their bad symptoms. Given that everyone is supposed to be in a
form of isolation anyway, heavily infected areas like Bergamo have now
essentially moved beyond testing.

``Right now in Lombardy the contagion is so widespread that we should
consider every person potentially positive,'' said Roberto Burioni, a
prominent virologist at the San Raffaele University in Milan.

``At this point in Lombardy swab tests are on the back burner,'' he
added. ``The only way to interrupt the epidemics is to imagine that
every single person, regardless of the test, can be infective.''

Image

A funeral service in the closed cemetery of Seriate, near Bergamo. On
Monday, Italy reported a drop in the number of deaths for the second day
in a row.Credit...Piero Cruciatti/Agence France-Presse --- Getty Images

Pier Luigi Lopalco, an epidemiologist at the University of Siena,
estimated that the total number of infected is ``10 times higher'' than
the one reported to the country every evening at 6 p.m.

Luca Zaia, the president of Veneto, another northern region hit hard by
the outbreak, warned that any country facing the outbreak needed to
prepare, and said that Americans should ``buy all the mechanical
respirators possible to save the lives of these patients.''

He said the experience had taught Italy that coronavirus patients ``are
enormous consumers of oxygen'' and that all of that compressed oxygen
gas had led to the tubes freezing. ``You can have a hospital full of
reanimation beds, but if the tubes aren't adapted, everything freezes,''
he said.

At the Papa Giovani XXIII hospital in Bergamo, Dr. Ivano Riva said that
for now he and his colleagues had not deprived care to anyone who could
have benefited from it.

``The important thing is not to arrive at that point,'' he said, adding,
``No one wants to decide who lives or dies like God.''

He said 26 people on his hospital's medical staff of 101 were out of
work with the virus.

In Brescia, the hospitals have been reporting at least 350 new cases a
day, Dr. Metra said Saturday. Between 10 and 15 percent of the doctors
and nurses are now out sick with the virus, he said. And since the most
serious virus patients require at least two weeks of hospitalization,
practically the only patients who have left the hospital are those who
have died.

At times, he said, his hospital has been forced to choose among multiple
patients with a decent chance of survival in order to use its limited
resources --- mainly ventilators and the trained nurses to run them 24
hours a day --- to save only a few of them.

``We try to be very selective,'' he said, adding that those with a low
chance of survival received morphine directly.

Image

In Brescia, one doctor estimated 10 to 15 percent of doctors and nurses
were out with the coronavirus.~Credit...Filippo Venezia/EPA, via
Shutterstock

At 8 a.m. on Friday, Dr. Metra said, doctors at the hospital had
concluded that a coronavirus patient who had recovered from a heart
attack 15 years ago met the criteria for a ventilator. But by 7 p.m.,
the chief anesthesiologist had been unable to locate one and so the
patient was put on a less intense form of breathing support.

``We don't have the capacity,'' he said. ``When all the ventilators in
the hospital are in use, you have to make these choices.''

\href{https://www.nytimes3xbfgragh.onion/news-event/coronavirus?action=click\&pgtype=Article\&state=default\&region=MAIN_CONTENT_3\&context=storylines_faq}{}

\hypertarget{the-coronavirus-outbreak-}{%
\subsubsection{The Coronavirus Outbreak
›}\label{the-coronavirus-outbreak-}}

\hypertarget{frequently-asked-questions}{%
\paragraph{Frequently Asked
Questions}\label{frequently-asked-questions}}

Updated September 4, 2020

\begin{itemize}
\item ~
  \hypertarget{what-are-the-symptoms-of-coronavirus}{%
  \paragraph{What are the symptoms of
  coronavirus?}\label{what-are-the-symptoms-of-coronavirus}}

  \begin{itemize}
  \tightlist
  \item
    In the beginning, the coronavirus
    \href{https://www.nytimes3xbfgragh.onion/article/coronavirus-facts-history.html?action=click\&pgtype=Article\&state=default\&region=MAIN_CONTENT_3\&context=storylines_faq\#link-6817bab5}{seemed
    like it was primarily a respiratory illness}~--- many patients had
    fever and chills, were weak and tired, and coughed a lot, though
    some people don't show many symptoms at all. Those who seemed
    sickest had pneumonia or acute respiratory distress syndrome and
    received supplemental oxygen. By now, doctors have identified many
    more symptoms and syndromes. In April,
    \href{https://www.nytimes3xbfgragh.onion/2020/04/27/health/coronavirus-symptoms-cdc.html?action=click\&pgtype=Article\&state=default\&region=MAIN_CONTENT_3\&context=storylines_faq}{the
    C.D.C. added to the list of early signs}~sore throat, fever, chills
    and muscle aches. Gastrointestinal upset, such as diarrhea and
    nausea, has also been observed. Another telltale sign of infection
    may be a sudden, profound diminution of one's
    \href{https://www.nytimes3xbfgragh.onion/2020/03/22/health/coronavirus-symptoms-smell-taste.html?action=click\&pgtype=Article\&state=default\&region=MAIN_CONTENT_3\&context=storylines_faq}{sense
    of smell and taste.}~Teenagers and young adults in some cases have
    developed painful red and purple lesions on their fingers and toes
    --- nicknamed ``Covid toe'' --- but few other serious symptoms.
  \end{itemize}
\item ~
  \hypertarget{why-is-it-safer-to-spend-time-together-outside}{%
  \paragraph{Why is it safer to spend time together
  outside?}\label{why-is-it-safer-to-spend-time-together-outside}}

  \begin{itemize}
  \tightlist
  \item
    \href{https://www.nytimes3xbfgragh.onion/2020/05/15/us/coronavirus-what-to-do-outside.html?action=click\&pgtype=Article\&state=default\&region=MAIN_CONTENT_3\&context=storylines_faq}{Outdoor
    gatherings}~lower risk because wind disperses viral droplets, and
    sunlight can kill some of the virus. Open spaces prevent the virus
    from building up in concentrated amounts and being inhaled, which
    can happen when infected people exhale in a confined space for long
    stretches of time, said Dr. Julian W. Tang, a virologist at the
    University of Leicester.
  \end{itemize}
\item ~
  \hypertarget{why-does-standing-six-feet-away-from-others-help}{%
  \paragraph{Why does standing six feet away from others
  help?}\label{why-does-standing-six-feet-away-from-others-help}}

  \begin{itemize}
  \tightlist
  \item
    The coronavirus spreads primarily through droplets from your mouth
    and nose, especially when you cough or sneeze. The C.D.C., one of
    the organizations using that measure,
    \href{https://www.nytimes3xbfgragh.onion/2020/04/14/health/coronavirus-six-feet.html?action=click\&pgtype=Article\&state=default\&region=MAIN_CONTENT_3\&context=storylines_faq}{bases
    its recommendation of six feet}~on the idea that most large droplets
    that people expel when they cough or sneeze will fall to the ground
    within six feet. But six feet has never been a magic number that
    guarantees complete protection. Sneezes, for instance, can launch
    droplets a lot farther than six feet,
    \href{https://jamanetwork.com/journals/jama/fullarticle/2763852}{according
    to a recent study}. It's a rule of thumb: You should be safest
    standing six feet apart outside, especially when it's windy. But
    keep a mask on at all times, even when you think you're far enough
    apart.
  \end{itemize}
\item ~
  \hypertarget{i-have-antibodies-am-i-now-immune}{%
  \paragraph{I have antibodies. Am I now
  immune?}\label{i-have-antibodies-am-i-now-immune}}

  \begin{itemize}
  \tightlist
  \item
    As of right
    now,\href{https://www.nytimes3xbfgragh.onion/2020/07/22/health/covid-antibodies-herd-immunity.html?action=click\&pgtype=Article\&state=default\&region=MAIN_CONTENT_3\&context=storylines_faq}{~that
    seems likely, for at least several months.}~There have been
    frightening accounts of people suffering what seems to be a second
    bout of Covid-19. But experts say these patients may have a
    drawn-out course of infection, with the virus taking a slow toll
    weeks to months after initial exposure.~People infected with the
    coronavirus typically
    \href{https://www.nature.com/articles/s41586-020-2456-9}{produce}~immune
    molecules called antibodies, which are
    \href{https://www.nytimes3xbfgragh.onion/2020/05/07/health/coronavirus-antibody-prevalence.html?action=click\&pgtype=Article\&state=default\&region=MAIN_CONTENT_3\&context=storylines_faq}{protective
    proteins made in response to an
    infection}\href{https://www.nytimes3xbfgragh.onion/2020/05/07/health/coronavirus-antibody-prevalence.html?action=click\&pgtype=Article\&state=default\&region=MAIN_CONTENT_3\&context=storylines_faq}{.
    These antibodies may}~last in the body
    \href{https://www.nature.com/articles/s41591-020-0965-6}{only two to
    three months}, which may seem worrisome, but that's~perfectly normal
    after an acute infection subsides, said Dr. Michael Mina, an
    immunologist at Harvard University. It may be possible to get the
    coronavirus again, but it's highly unlikely that it would be
    possible in a short window of time from initial infection or make
    people sicker the second time.
  \end{itemize}
\item ~
  \hypertarget{what-are-my-rights-if-i-am-worried-about-going-back-to-work}{%
  \paragraph{What are my rights if I am worried about going back to
  work?}\label{what-are-my-rights-if-i-am-worried-about-going-back-to-work}}

  \begin{itemize}
  \tightlist
  \item
    Employers have to provide
    \href{https://www.osha.gov/SLTC/covid-19/standards.html}{a safe
    workplace}~with policies that protect everyone equally.
    \href{https://www.nytimes3xbfgragh.onion/article/coronavirus-money-unemployment.html?action=click\&pgtype=Article\&state=default\&region=MAIN_CONTENT_3\&context=storylines_faq}{And
    if one of your co-workers tests positive for the coronavirus, the
    C.D.C.}~has said that
    \href{https://www.cdc.gov/coronavirus/2019-ncov/community/guidance-business-response.html}{employers
    should tell their employees}~-\/- without giving you the sick
    employee's name -\/- that they may have been exposed to the virus.
  \end{itemize}
\end{itemize}

Routine procedures, such as taking a patient's blood pressure, which in
the past took only a minute, now take half an hour because of
precautions. And the consequences of not taking those precautions,
especially between patients, were all too apparent.

``We have lost some patients just because their bed was close to a sick
patient and we did not know that he was Covid positive,'' he said,
referring to the coronavirus. ``You have these patients who came to the
hospital one week ago to be treated for something else and they were
infected and now they are sick. That is the way that our unit became all
Covid positive.''

Giacomo Grasselli, who is the coordinator of the intensive care units
throughout Lombardy, said in a recent interview that an increase in
cases would require doctors to prioritize ``those with the best chance
for survival.'' But there were also basic decisions a doctor should
always make on who could benefit from care, and when it is futile.

``My father is 84 and I love him very much,'' but it would be
irresponsible, he said, to make him go through the invasive procedures
of an I.C.U.

Earlier this month, Dr. Marco Vergano, a 45-year-old anesthesiologist
based in Turin and chairman of an ethics committee for his medical
specialty, worked late over four nights to draft emergency
recommendations for Italian doctors allocating scarce beds among a
soaring load of patients.

Image

The Italian army patrolling Piazza Duomo in Milan on
Friday.Credit...Alessandro Grassani for The New York Times

The limited supply of ventilators and nurses forced a deadly calculus,
Dr. Vergano said. ``Many patients were coming in and you are not able to
discharge any patients,'' he said. As the beds fill, ``you can admit
more patients only with the death of a patient,'' but ``if a patient
dies in a few days, probably that was a patient who did not deserve to
be admitted. You are wasting very scarce and precious resources.''

He said doctors in the country's hot zone of infections were already
focusing on the needs of the total community, a recommendation made in
the New England Journal of Medicine paper by the Bergamo doctors.

``If you admit an 82-year-old with hypertension, in a situation where
you have two or three patients waiting outside your I.C.U. who have many
more chances of survival that you cannot admit because your I.C.U. is
full, then it becomes really inappropriate, or I would say, immoral,''
he said.

The recommendations were immediately attacked by officials of the main
Italian medical association, many of them outside the hardest-hit areas,
who criticized the guidelines as discriminatory against older patients.
But Dr. Vergano said he received support from doctors on the front
lines, as well as requests for translations from doctors in other
countries.

``They were expecting the same destiny,'' he said. ``Probably with less
resources.''

Jason Horowitz reported from Rome and David D. Kirkpatrick from London.
Raphael Minder contributed reporting from Madrid, and Elisabetta
Povoledo from Rome.

Advertisement

\protect\hyperlink{after-bottom}{Continue reading the main story}

\hypertarget{site-index}{%
\subsection{Site Index}\label{site-index}}

\hypertarget{site-information-navigation}{%
\subsection{Site Information
Navigation}\label{site-information-navigation}}

\begin{itemize}
\tightlist
\item
  \href{https://help.nytimes3xbfgragh.onion/hc/en-us/articles/115014792127-Copyright-notice}{©~2020~The
  New York Times Company}
\end{itemize}

\begin{itemize}
\tightlist
\item
  \href{https://www.nytco.com/}{NYTCo}
\item
  \href{https://help.nytimes3xbfgragh.onion/hc/en-us/articles/115015385887-Contact-Us}{Contact
  Us}
\item
  \href{https://www.nytco.com/careers/}{Work with us}
\item
  \href{https://nytmediakit.com/}{Advertise}
\item
  \href{http://www.tbrandstudio.com/}{T Brand Studio}
\item
  \href{https://www.nytimes3xbfgragh.onion/privacy/cookie-policy\#how-do-i-manage-trackers}{Your
  Ad Choices}
\item
  \href{https://www.nytimes3xbfgragh.onion/privacy}{Privacy}
\item
  \href{https://help.nytimes3xbfgragh.onion/hc/en-us/articles/115014893428-Terms-of-service}{Terms
  of Service}
\item
  \href{https://help.nytimes3xbfgragh.onion/hc/en-us/articles/115014893968-Terms-of-sale}{Terms
  of Sale}
\item
  \href{https://spiderbites.nytimes3xbfgragh.onion}{Site Map}
\item
  \href{https://help.nytimes3xbfgragh.onion/hc/en-us}{Help}
\item
  \href{https://www.nytimes3xbfgragh.onion/subscription?campaignId=37WXW}{Subscriptions}
\end{itemize}
