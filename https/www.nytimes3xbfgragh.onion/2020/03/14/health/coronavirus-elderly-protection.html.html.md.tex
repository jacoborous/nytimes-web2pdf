Sections

SEARCH

\protect\hyperlink{site-content}{Skip to
content}\protect\hyperlink{site-index}{Skip to site index}

\href{https://www.nytimes3xbfgragh.onion/section/health}{Health}

\href{https://myaccount.nytimes3xbfgragh.onion/auth/login?response_type=cookie\&client_id=vi}{}

\href{https://www.nytimes3xbfgragh.onion/section/todayspaper}{Today's
Paper}

\href{/section/health}{Health}\textbar{}How to Protect Older People From
the Coronavirus

\url{https://nyti.ms/2U5lyxk}

\begin{itemize}
\item
\item
\item
\item
\item
\end{itemize}

\hypertarget{the-coronavirus-outbreak}{%
\subsubsection{\texorpdfstring{\href{https://www.nytimes3xbfgragh.onion/news-event/coronavirus?name=styln-coronavirus-national\&region=TOP_BANNER\&variant=undefined\&block=storyline_menu_recirc\&action=click\&pgtype=Article\&impression_id=2c183b40-e3b6-11ea-befc-036b724b9a9b}{The
Coronavirus
Outbreak}}{The Coronavirus Outbreak}}\label{the-coronavirus-outbreak}}

\begin{itemize}
\tightlist
\item
  live\href{https://www.nytimes3xbfgragh.onion/2020/08/21/world/covid-19-coronavirus.html?name=styln-coronavirus-national\&region=TOP_BANNER\&variant=undefined\&block=storyline_menu_recirc\&action=click\&pgtype=Article\&impression_id=2c183b41-e3b6-11ea-befc-036b724b9a9b}{Latest
  Updates}
\item
  \href{https://www.nytimes3xbfgragh.onion/interactive/2020/us/coronavirus-us-cases.html?name=styln-coronavirus-national\&region=TOP_BANNER\&variant=undefined\&block=storyline_menu_recirc\&action=click\&pgtype=Article\&impression_id=2c186250-e3b6-11ea-befc-036b724b9a9b}{Maps
  and Cases}
\item
  \href{https://www.nytimes3xbfgragh.onion/interactive/2020/science/coronavirus-vaccine-tracker.html?name=styln-coronavirus-national\&region=TOP_BANNER\&variant=undefined\&block=storyline_menu_recirc\&action=click\&pgtype=Article\&impression_id=2c186251-e3b6-11ea-befc-036b724b9a9b}{Vaccine
  Tracker}
\item
  \href{https://www.nytimes3xbfgragh.onion/2020/08/19/us/colleges-closing-covid.html?name=styln-coronavirus-national\&region=TOP_BANNER\&variant=undefined\&block=storyline_menu_recirc\&action=click\&pgtype=Article\&impression_id=2c186252-e3b6-11ea-befc-036b724b9a9b}{Colleges
  Closing}
\item
  \href{https://www.nytimes3xbfgragh.onion/live/2020/08/21/business/stock-market-today-coronavirus?name=styln-coronavirus-national\&region=TOP_BANNER\&variant=undefined\&block=storyline_menu_recirc\&action=click\&pgtype=Article\&impression_id=2c186253-e3b6-11ea-befc-036b724b9a9b}{Economy}
\end{itemize}

Advertisement

\protect\hyperlink{after-top}{Continue reading the main story}

Supported by

\protect\hyperlink{after-sponsor}{Continue reading the main story}

\hypertarget{how-to-protect-older-people-from-the-coronavirus}{%
\section{How to Protect Older People From the
Coronavirus}\label{how-to-protect-older-people-from-the-coronavirus}}

People over 60, and especially over 80, are particularly vulnerable to
severe or fatal infection. Here are some steps to reduce their risk.

\includegraphics{https://static01.graylady3jvrrxbe.onion/images/2020/03/14/science/14VIRUS-ELDERLY1/merlin_170400345_d4d71c2c-c866-4ad2-981c-1a50d5d1d82c-articleLarge.jpg?quality=75\&auto=webp\&disable=upscale}

By \href{https://www.nytimes3xbfgragh.onion/by/katie-hafner}{Katie
Hafner}

\begin{itemize}
\item
  March 14, 2020
\item
  \begin{itemize}
  \item
  \item
  \item
  \item
  \item
  \end{itemize}
\end{itemize}

Amid the uncertainty swirling around the coronavirus pandemic stands one
incontrovertible fact: The highest rate of fatalities is among older
people, particularly those with underlying medical conditions.

Of the confirmed cases in
\href{https://informationisbeautiful.net/visualizations/covid-19-coronavirus-infographic-datapack/}{China}
to date, nearly 15 percent of patients over 80 have died. For those
under 50, the death rate was well below 1 percent.

There is no evidence yet that older people are significantly more likely
to acquire the coronavirus than younger people. But medical experts say
that if people over 60 are infected, they are more likely to have
severe, life-threatening disease, even if their general health is good.
Older people with underlying medical conditions are at particularly high
risk. Experts attribute some of the risk to a weakening of the immune
system with age.

This leaves older people and their families wondering what extra
precautions they should take. Several best practices have been
recommended by the Centers for Disease Control and Prevention, the World
Health Organization, geriatricians and infectious diseases specialists.

\hypertarget{familiarize-yourself-with-guidelines-and-follow-them}{%
\subsection{Familiarize yourself with guidelines and follow
them.}\label{familiarize-yourself-with-guidelines-and-follow-them}}

Geriatricians recommend their patients adhere to current recommendations
from the
\href{https://www.cdc.gov/coronavirus/2019-ncov/specific-groups/high-risk-complications.html\#who-is-higher-risk}{C.D.C}.
and W.H.O., a litany of advice that has become all too familiar: Wash
your hands frequently with soap and warm water for 20 seconds (the time
it takes to sing ``Happy Birthday'' twice) or clean them with
alcohol-based hand gel; avoid handshakes; stay away from large
gatherings; clean and disinfect objects that are touched frequently; and
avoid public transportation and crowds. Stock up on supplies.

Cruises are out, as is nonessential travel. Visits with grandchildren
are ill-advised.

``I've had this conversation about a hundred times in the last week,''
said Dr. Elizabeth Eckstrom, chief of geriatrics at Oregon Health \&
Science University in Portland. Dr. Eckstrom said most of the patients
she sees in her clinic are over 80. All of them have made their worry
plain.

And all of her patients, Dr. Eckstrom said, have at least one chronic
condition. ``Most of them have three, four, five or more,'' she added.

\includegraphics{https://static01.graylady3jvrrxbe.onion/images/2020/03/14/science/14VIRUS-ELDERLY4/merlin_170391963_cf35426d-7370-49fb-89a5-a8de7b8e2efb-articleLarge.jpg?quality=75\&auto=webp\&disable=upscale}

People are wrong to assume that if an underlying condition is well
managed with treatment, they're out of danger. Even those with
conditions that are stable should take extra precautions.

``These conditions can limit underlying reserve and lead to worse
outcomes when older people become severely ill, which taxes all organ
systems,'' said Dr. Annie Luetkemeyer, an infectious diseases specialist
at Zuckerberg San Francisco General Hospital.

\hypertarget{latest-updates-the-coronavirus-outbreak}{%
\section{\texorpdfstring{\href{https://www.nytimes3xbfgragh.onion/2020/08/21/world/covid-19-coronavirus.html?action=click\&pgtype=Article\&state=default\&region=MAIN_CONTENT_1\&context=storylines_live_updates}{Latest
Updates: The Coronavirus
Outbreak}}{Latest Updates: The Coronavirus Outbreak}}\label{latest-updates-the-coronavirus-outbreak}}

Updated 2020-08-21T13:48:59.464Z

\begin{itemize}
\tightlist
\item
  \href{https://www.nytimes3xbfgragh.onion/2020/08/21/world/covid-19-coronavirus.html?action=click\&pgtype=Article\&state=default\&region=MAIN_CONTENT_1\&context=storylines_live_updates\#link-6a60a19d}{`Be
  adults': Universities in the U.S. are warning students about
  gatherings as they return to campus.}
\item
  \href{https://www.nytimes3xbfgragh.onion/2020/08/21/world/covid-19-coronavirus.html?action=click\&pgtype=Article\&state=default\&region=MAIN_CONTENT_1\&context=storylines_live_updates\#link-324af071}{As
  he accepts the Democratic nomination, Biden knocks Trump's pandemic
  response.}
\item
  \href{https://www.nytimes3xbfgragh.onion/2020/08/21/world/covid-19-coronavirus.html?action=click\&pgtype=Article\&state=default\&region=MAIN_CONTENT_1\&context=storylines_live_updates\#link-191d44be}{South
  Korea threatens to detain people who obstruct virus-control efforts.}
\end{itemize}

\href{https://www.nytimes3xbfgragh.onion/2020/08/21/world/covid-19-coronavirus.html?action=click\&pgtype=Article\&state=default\&region=MAIN_CONTENT_1\&context=storylines_live_updates}{See
more updates}

More live coverage:
\href{https://www.nytimes3xbfgragh.onion/live/2020/08/21/business/stock-market-today-coronavirus?action=click\&pgtype=Article\&state=default\&region=MAIN_CONTENT_1\&context=storylines_live_updates}{Markets}

``For example, diabetes can make it harder to fight infection, and
underlying heart or lung disease may make it more difficult for those
organs to keep up with demands created by a serious Covid-19
infection,'' she said, referring to the syndrome caused by the new
coronavirus.

Dr. Daniel Winetsky, an infectious diseases fellow at Columbia
University in New York, said his advice to his own parents, who live
across the country in San Francisco, has shifted dramatically. A week
ago, he said, he was reassuring them about their safety, even
encouraging them to go ahead with a trip they were planning to the
Florida Everglades with a small tour group.

Over the weekend, his fears about the pandemic rose, and by Tuesday not
only was he telling them not to go, but he also was advising them to
reduce to a minimum the number of people they came into contact with.
Visits with grandchildren are verboten.

Dr. Winetsky told his mother, Carol, who is 73 and has asthma, to stop
meeting with her biweekly knitting group. And he instructed his father,
Hank, who has had two coronary stents, not to attend either of his two
book group meetings.

His mother continues to go to the grocery store, while avoiding crowded
places like Costco. With her son's permission, she still goes to
physical therapy for a back injury, but she is careful to make sure the
therapist washes her hands and that the equipment gets wiped down with
disinfectant.

\hypertarget{what-about-nonessential-doctors-appointments}{%
\subsection{What about nonessential doctor's
appointments?}\label{what-about-nonessential-doctors-appointments}}

Some experts are recommending that older adults at risk cancel
nonessential doctor's appointments, including wellness visits.
Telemedicine sessions, if available, are often a reasonable substitute.

Dr. Eckstrom generally agrees, but with caveats. While it might be
prudent to cancel wellness and other visits that are not urgent, she
said, ``many older adults have issues that require regular follow up,
such as dementia, Parkinson's disease, falls, heart problems.'' She
worries that skipping visits might allow these conditions to spiral out
of control, but agrees that telemedicine can usually bridge the gap.

Another helpful step: talking to your doctor about stockpiling two or
three months of any critical prescription medicines.

Image

Dr. Paul Casey on a video call at Rush University Medical Center in
Chicago.Credit...Danielle Scruggs for The New York Times

\hypertarget{beware-of-social-isolation}{%
\subsection{Beware of social
isolation.}\label{beware-of-social-isolation}}

Experts warn that social distancing, the cornerstone of epidemic
control, could lead to social isolation, already a problem in the older
population. According to a
\href{https://www.pewresearch.org/fact-tank/2020/03/10/older-people-are-more-likely-to-live-alone-in-the-u-s-than-elsewhere-in-the-world/}{recent
Pew Research Center} study of more than 130 countries and territories,
16 percent of people 60 and older live alone. Loneliness,
\href{https://www.nytimes3xbfgragh.onion/2016/09/06/health/lonliness-aging-health-effects.html}{researchers
have found}, comes with its own set of health hazards.

Dr. Winetsky is aware of the danger, and has suggested to his parents
that they switch to virtual meetings with friends and relatives, with
the benefits of social engagement in mind. ``I've tried to frame it as,
`Don't cancel these things, but change to Zoom or Skype or FaceTime,'''
he said.

April Vollmer, 68, an artist who lives in New York, flew to California
in November for an extended stay with her 91-year-old father, who lives
in Santa Cruz. She has yet to leave.

Just when she was planning last month to fly back to New York, she said,
where she has a husband, friends and a rich cultural life, the
coronavirus hit. Now she oversees her father's home health aides and
takes long walks along the bluffs above the Pacific, a ``virus-free''
activity.

Recently Ms. Vollmer got an email from a friend of her father's who last
year decided to move to assisted living. ``The home has canceled group
events, and residents are eating alone in their rooms,'' Ms. Vollmer
said. ``Seems like a bigger change there than for someone living at
home.''

\hypertarget{have-a-talk-with-home-health-aides}{%
\subsection{Have a talk with home health
aides.}\label{have-a-talk-with-home-health-aides}}

The National Association for Home Care \& Hospice estimates that 12
million ``vulnerable persons of all ages'' in the U.S. receive care in
their homes, delivered by a home care work force of approximately 2.2
million people. For many older adults, that means a steady parade of
home health aides trooping through the door, some more mindful of
hygiene than others.

People should have conversations with their caregivers about hygiene,
suggested Dr. David Nace, president-elect of the Society for Post-Acute
and Long-Term Care Medicine, a professional group that represents
practitioners working in long-term care facilities.

Image

Dolly and Wayne Wong, who have been married more than 70 years, live in
a nursing facility in Sierra Madre, Calif.~The facility recently asked
family members of residents not to visit.Credit...Cruz Fierro

Double-check that aides are washing their hands or using hand gel. Any
equipment they bring in should be wiped down with disinfectant. And make
sure they are feeling healthy.

``If you're by yourself, you may be in a very vulnerable position
because you're dependent upon that person,'' Dr. Nace said. ``It can
feel intimidating. But hopefully there's a good enough relationship that
you can open the conversation.''

\href{https://www.nytimes3xbfgragh.onion/news-event/coronavirus?action=click\&pgtype=Article\&state=default\&region=MAIN_CONTENT_3\&context=storylines_faq}{}

\hypertarget{the-coronavirus-outbreak-}{%
\subsubsection{The Coronavirus Outbreak
›}\label{the-coronavirus-outbreak-}}

\hypertarget{frequently-asked-questions}{%
\paragraph{Frequently Asked
Questions}\label{frequently-asked-questions}}

Updated August 17, 2020

\begin{itemize}
\item ~
  \hypertarget{why-does-standing-six-feet-away-from-others-help}{%
  \paragraph{Why does standing six feet away from others
  help?}\label{why-does-standing-six-feet-away-from-others-help}}

  \begin{itemize}
  \tightlist
  \item
    The coronavirus spreads primarily through droplets from your mouth
    and nose, especially when you cough or sneeze. The C.D.C., one of
    the organizations using that measure,
    \href{https://www.nytimes3xbfgragh.onion/2020/04/14/health/coronavirus-six-feet.html?action=click\&pgtype=Article\&state=default\&region=MAIN_CONTENT_3\&context=storylines_faq}{bases
    its recommendation of six feet} on the idea that most large droplets
    that people expel when they cough or sneeze will fall to the ground
    within six feet. But six feet has never been a magic number that
    guarantees complete protection. Sneezes, for instance, can launch
    droplets a lot farther than six feet,
    \href{https://jamanetwork.com/journals/jama/fullarticle/2763852}{according
    to a recent study}. It's a rule of thumb: You should be safest
    standing six feet apart outside, especially when it's windy. But
    keep a mask on at all times, even when you think you're far enough
    apart.
  \end{itemize}
\item ~
  \hypertarget{i-have-antibodies-am-i-now-immune}{%
  \paragraph{I have antibodies. Am I now
  immune?}\label{i-have-antibodies-am-i-now-immune}}

  \begin{itemize}
  \tightlist
  \item
    As of right
    now,\href{https://www.nytimes3xbfgragh.onion/2020/07/22/health/covid-antibodies-herd-immunity.html?action=click\&pgtype=Article\&state=default\&region=MAIN_CONTENT_3\&context=storylines_faq}{that
    seems likely, for at least several months.} There have been
    frightening accounts of people suffering what seems to be a second
    bout of Covid-19. But experts say these patients may have a
    drawn-out course of infection, with the virus taking a slow toll
    weeks to months after initial exposure. People infected with the
    coronavirus typically
    \href{https://www.nature.com/articles/s41586-020-2456-9}{produce}
    immune molecules called antibodies, which are
    \href{https://www.nytimes3xbfgragh.onion/2020/05/07/health/coronavirus-antibody-prevalence.html?action=click\&pgtype=Article\&state=default\&region=MAIN_CONTENT_3\&context=storylines_faq}{protective
    proteins made in response to an
    infection}\href{https://www.nytimes3xbfgragh.onion/2020/05/07/health/coronavirus-antibody-prevalence.html?action=click\&pgtype=Article\&state=default\&region=MAIN_CONTENT_3\&context=storylines_faq}{.
    These antibodies may} last in the body
    \href{https://www.nature.com/articles/s41591-020-0965-6}{only two to
    three months}, which may seem worrisome, but that's perfectly normal
    after an acute infection subsides, said Dr. Michael Mina, an
    immunologist at Harvard University. It may be possible to get the
    coronavirus again, but it's highly unlikely that it would be
    possible in a short window of time from initial infection or make
    people sicker the second time.
  \end{itemize}
\item ~
  \hypertarget{im-a-small-business-owner-can-i-get-relief}{%
  \paragraph{I'm a small-business owner. Can I get
  relief?}\label{im-a-small-business-owner-can-i-get-relief}}

  \begin{itemize}
  \tightlist
  \item
    The
    \href{https://www.nytimes3xbfgragh.onion/article/small-business-loans-stimulus-grants-freelancers-coronavirus.html?action=click\&pgtype=Article\&state=default\&region=MAIN_CONTENT_3\&context=storylines_faq}{stimulus
    bills enacted in March} offer help for the millions of American
    small businesses. Those eligible for aid are businesses and
    nonprofit organizations with fewer than 500 workers, including sole
    proprietorships, independent contractors and freelancers. Some
    larger companies in some industries are also eligible. The help
    being offered, which is being managed by the Small Business
    Administration, includes the Paycheck Protection Program and the
    Economic Injury Disaster Loan program. But lots of folks have
    \href{https://www.nytimes3xbfgragh.onion/interactive/2020/05/07/business/small-business-loans-coronavirus.html?action=click\&pgtype=Article\&state=default\&region=MAIN_CONTENT_3\&context=storylines_faq}{not
    yet seen payouts.} Even those who have received help are confused:
    The rules are draconian, and some are stuck sitting on
    \href{https://www.nytimes3xbfgragh.onion/2020/05/02/business/economy/loans-coronavirus-small-business.html?action=click\&pgtype=Article\&state=default\&region=MAIN_CONTENT_3\&context=storylines_faq}{money
    they don't know how to use.} Many small-business owners are getting
    less than they expected or
    \href{https://www.nytimes3xbfgragh.onion/2020/06/10/business/Small-business-loans-ppp.html?action=click\&pgtype=Article\&state=default\&region=MAIN_CONTENT_3\&context=storylines_faq}{not
    hearing anything at all.}
  \end{itemize}
\item ~
  \hypertarget{what-are-my-rights-if-i-am-worried-about-going-back-to-work}{%
  \paragraph{What are my rights if I am worried about going back to
  work?}\label{what-are-my-rights-if-i-am-worried-about-going-back-to-work}}

  \begin{itemize}
  \tightlist
  \item
    Employers have to provide
    \href{https://www.osha.gov/SLTC/covid-19/standards.html}{a safe
    workplace} with policies that protect everyone equally.
    \href{https://www.nytimes3xbfgragh.onion/article/coronavirus-money-unemployment.html?action=click\&pgtype=Article\&state=default\&region=MAIN_CONTENT_3\&context=storylines_faq}{And
    if one of your co-workers tests positive for the coronavirus, the
    C.D.C.} has said that
    \href{https://www.cdc.gov/coronavirus/2019-ncov/community/guidance-business-response.html}{employers
    should tell their employees} -\/- without giving you the sick
    employee's name -\/- that they may have been exposed to the virus.
  \end{itemize}
\item ~
  \hypertarget{what-is-school-going-to-look-like-in-september}{%
  \paragraph{What is school going to look like in
  September?}\label{what-is-school-going-to-look-like-in-september}}

  \begin{itemize}
  \tightlist
  \item
    It is unlikely that many schools will return to a normal schedule
    this fall, requiring the grind of
    \href{https://www.nytimes3xbfgragh.onion/2020/06/05/us/coronavirus-education-lost-learning.html?action=click\&pgtype=Article\&state=default\&region=MAIN_CONTENT_3\&context=storylines_faq}{online
    learning},
    \href{https://www.nytimes3xbfgragh.onion/2020/05/29/us/coronavirus-child-care-centers.html?action=click\&pgtype=Article\&state=default\&region=MAIN_CONTENT_3\&context=storylines_faq}{makeshift
    child care} and
    \href{https://www.nytimes3xbfgragh.onion/2020/06/03/business/economy/coronavirus-working-women.html?action=click\&pgtype=Article\&state=default\&region=MAIN_CONTENT_3\&context=storylines_faq}{stunted
    workdays} to continue. California's two largest public school
    districts --- Los Angeles and San Diego --- said on July 13, that
    \href{https://www.nytimes3xbfgragh.onion/2020/07/13/us/lausd-san-diego-school-reopening.html?action=click\&pgtype=Article\&state=default\&region=MAIN_CONTENT_3\&context=storylines_faq}{instruction
    will be remote-only in the fall}, citing concerns that surging
    coronavirus infections in their areas pose too dire a risk for
    students and teachers. Together, the two districts enroll some
    825,000 students. They are the largest in the country so far to
    abandon plans for even a partial physical return to classrooms when
    they reopen in August. For other districts, the solution won't be an
    all-or-nothing approach.
    \href{https://bioethics.jhu.edu/research-and-outreach/projects/eschool-initiative/school-policy-tracker/}{Many
    systems}, including the nation's largest, New York City, are
    devising
    \href{https://www.nytimes3xbfgragh.onion/2020/06/26/us/coronavirus-schools-reopen-fall.html?action=click\&pgtype=Article\&state=default\&region=MAIN_CONTENT_3\&context=storylines_faq}{hybrid
    plans} that involve spending some days in classrooms and other days
    online. There's no national policy on this yet, so check with your
    municipal school system regularly to see what is happening in your
    community.
  \end{itemize}
\end{itemize}

Adam Henick, an investor who lives on the Upper East Side in Manhattan,
said his father, 92, and mother, 88, live in an apartment a block away,
and aides come through every day. Only one wears a mask, he said.

``In a perfect world, no one would enter the apartment without putting a
mask on,'' said Mr. Henick. ``But it's better than being in a nursing
home.''

\hypertarget{the-nursing-home-conundrum}{%
\subsection{The nursing home
conundrum.}\label{the-nursing-home-conundrum}}

Some 1.7 million people, mostly older, are in nursing homes in the U.S.,
a fraction of the 50 million Americans over age 65.

Given the rash of deaths at
\href{https://www.nytimes3xbfgragh.onion/2020/03/07/us/coronavirus-nursing-home.html}{a
nursing home in Kirkland, Wash.}, hit hard by the virus, nursing homes
are on high alert. Many have gone into full lockdown mode.

The federal government is telling nursing homes to bar all visitors,
making exceptions only ``for compassionate care, such as end of life
situations.''

Curtis Wong, 66, a retired Microsoft researcher who lives in the Seattle
area, used to visit his parents often. They are in their 90s and live in
an assisted living facility in Sierra Madre, Calif.

On Thursday, the facility prohibited all nonmedical visits and said it
was changing its building entrance codes. In an email announcing the
measure, the facility's management offered to put residents in touch
with family members via FaceTime.

Three days ago, Mr. Wong said, during a video chat with his father, ``I
worried I might not see him again. Things got very emotional.''

Image

Lori and Michael Spencer, foreground, visited Lori's mother, Judie
Shape, who Lori said had tested positive for coronavirus, at the Life
Care Center of Kirkland, Wash., on Wednesday.Credit...Jason
Redmond/Reuters

Cathy Johnson, who lives outside of Boston, is trying to take matters
into her own hands. Ms. Johnson is the primary caregiver for her
96-year-old father, who lives nearby in an independent living facility
with 2,200 residents. Two cases of coronavirus have been reported in the
area and Ms. Johnson, worried that the facility might shut its doors to
visitors, has been planning to extract her father and bring him to live
at her house.

``I actually think that's not unreasonable, if it's in your community
and you have the ability to care safely for that person in your house,''
said Dr. Nace.

But so far, Ms. Johnson's father, wedded to place and routine, is
refusing to leave the facility.

\hypertarget{stay-active-even-in-a-pandemic}{%
\subsection{Stay active, even in a
pandemic.}\label{stay-active-even-in-a-pandemic}}

Geriatricians fear that social distancing may affect routines in ways
that can compromise the vitality of older adults. They emphasize the
importance of maintaining good habits, including sufficient sleep,
healthful eating and exercise.

Exercise may be beneficial in fighting the effects of coronavirus. It
can help boost the body's immune functions, decrease inflammation and
have mental and emotional benefits. A patient who relies on daily
exercise at the gym but is trying to avoid risky situations might simply
go for a walk.

On Wednesday afternoon, Hank Winetsky, 80, had just returned from a
round of golf with a small group. His foursome ranged in age from 70 to
81. ``Golf is pretty safe when it comes to human contact,'' he said.

But even golf proved not to be a contact-free sport. ``There was a
bottle of water on the cart, and everybody thought it was their own
bottle,'' he said. ``All four of us drank out of it. Now we're all
freaked out.''

Advertisement

\protect\hyperlink{after-bottom}{Continue reading the main story}

\hypertarget{site-index}{%
\subsection{Site Index}\label{site-index}}

\hypertarget{site-information-navigation}{%
\subsection{Site Information
Navigation}\label{site-information-navigation}}

\begin{itemize}
\tightlist
\item
  \href{https://help.nytimes3xbfgragh.onion/hc/en-us/articles/115014792127-Copyright-notice}{©~2020~The
  New York Times Company}
\end{itemize}

\begin{itemize}
\tightlist
\item
  \href{https://www.nytco.com/}{NYTCo}
\item
  \href{https://help.nytimes3xbfgragh.onion/hc/en-us/articles/115015385887-Contact-Us}{Contact
  Us}
\item
  \href{https://www.nytco.com/careers/}{Work with us}
\item
  \href{https://nytmediakit.com/}{Advertise}
\item
  \href{http://www.tbrandstudio.com/}{T Brand Studio}
\item
  \href{https://www.nytimes3xbfgragh.onion/privacy/cookie-policy\#how-do-i-manage-trackers}{Your
  Ad Choices}
\item
  \href{https://www.nytimes3xbfgragh.onion/privacy}{Privacy}
\item
  \href{https://help.nytimes3xbfgragh.onion/hc/en-us/articles/115014893428-Terms-of-service}{Terms
  of Service}
\item
  \href{https://help.nytimes3xbfgragh.onion/hc/en-us/articles/115014893968-Terms-of-sale}{Terms
  of Sale}
\item
  \href{https://spiderbites.nytimes3xbfgragh.onion}{Site Map}
\item
  \href{https://help.nytimes3xbfgragh.onion/hc/en-us}{Help}
\item
  \href{https://www.nytimes3xbfgragh.onion/subscription?campaignId=37WXW}{Subscriptions}
\end{itemize}
