Sections

SEARCH

\protect\hyperlink{site-content}{Skip to
content}\protect\hyperlink{site-index}{Skip to site index}

\href{https://www.nytimes3xbfgragh.onion/section/food}{Food}

\href{https://myaccount.nytimes3xbfgragh.onion/auth/login?response_type=cookie\&client_id=vi}{}

\href{https://www.nytimes3xbfgragh.onion/section/todayspaper}{Today's
Paper}

\href{/section/food}{Food}\textbar{}Le Bernardin, Daniel and Other Top
New York Restaurants Temporarily Close

\url{https://nyti.ms/2UafAvg}

\begin{itemize}
\item
\item
\item
\item
\item
\item
\end{itemize}

Advertisement

\protect\hyperlink{after-top}{Continue reading the main story}

Supported by

\protect\hyperlink{after-sponsor}{Continue reading the main story}

\hypertarget{le-bernardin-daniel-and-other-top-new-york-restaurants-temporarily-close}{%
\section{Le Bernardin, Daniel and Other Top New York Restaurants
Temporarily
Close}\label{le-bernardin-daniel-and-other-top-new-york-restaurants-temporarily-close}}

Others are likely to follow as owners' worries shift from canceled
reservations to survival.

\includegraphics{https://static01.graylady3jvrrxbe.onion/images/2020/03/14/dining/13ny-Restaurants1-print/merlin_118404398_3f6f56f1-10e8-4d81-92cd-8182e9edb979-articleLarge.jpg?quality=75\&auto=webp\&disable=upscale}

\href{https://www.nytimes3xbfgragh.onion/by/julia-moskin}{\includegraphics{https://static01.graylady3jvrrxbe.onion/images/2018/09/25/multimedia/author-julia-moskin/author-julia-moskin-thumbLarge.png}}

By \href{https://www.nytimes3xbfgragh.onion/by/julia-moskin}{Julia
Moskin}

\begin{itemize}
\item
  Published March 13, 2020Updated March 17, 2020
\item
  \begin{itemize}
  \item
  \item
  \item
  \item
  \item
  \item
  \end{itemize}
\end{itemize}

Some of New York City's leading restaurants, including
\href{https://www.nytimes3xbfgragh.onion/2012/05/23/dining/reviews/le-bernardin-in-midtown-manhattan.html}{Le
Bernardin} and
\href{https://www.nytimes3xbfgragh.onion/2013/07/24/dining/reviews/restaurant-review-daniel-on-the-upper-east-side.html}{Daniel},
announced Friday that they would close indefinitely after dinner that
night. All 19 of the \href{https://www.ushgnyc.com/}{Union Square
Hospitality Group}'s restaurants in New York City closed after lunch on
Friday and will remain closed until further notice, said Danny Meyer,
the group's founder and chief executive.

The group's move temporarily shutters some of the city's most popular
and enduring restaurants, like
\href{https://www.nytimes3xbfgragh.onion/2016/08/17/dining/gramercy-tavern-review.html}{Gramercy
Tavern} and
\href{https://www.nytimes3xbfgragh.onion/2017/04/25/dining/union-square-cafe-review.html}{Union
Square Cafe}.

``With all that we now know about federal, state and citywide mandates,
as well as the science that has provided evidence urging everyone to
reduce nonessential social contact, we have made the difficult, but for
us, obvious decision to temporarily close our restaurants in New York
City,'' Mr. Meyer said in a statement. ``I feel it is necessary that
U.S.H.G. do our part to prevent the spread of this pandemic.''

The chef \href{https://www.danielboulud.com/}{Daniel Boulud}, who has
restaurants around the globe, decided to indefinitely close Daniel and
Café Boulud on the Upper East Side, Bar Boulud and Boulud Sud on the
Upper West Side, and four other locations in New York.

``Restaurants are where people go to lift their spirits, so we tried to
say open, but it has become a matter of public safety,'' Mr. Boulud
said. His closings, and the others, included a hiatus on all catering
and events.

One of the city's most enduring fine-dining restaurants,
\href{https://www.nytimes3xbfgragh.onion/2019/11/19/dining/gotham-bar-and-grill-review-pete-wells.html}{Gotham
Bar \& Grill}, announced that it would close permanently after dinner on
Saturday, after 36 years in business and less than a year after it hired
a new chef,
\href{https://www.nytimes3xbfgragh.onion/2019/07/23/dining/gotham-bar-and-grill-chef-victoria-blamey.html}{Victoria
Blamey}. The restaurant notified friends in an email that gave no
reason, but said, ``We now look forward to doing what we can to assist
our industry and help New York City recover.'' A partner, Jerome
Kretchmer, and Ms. Blamey did not immediately respond to messages
seeking comment.

As of Friday afternoon, the vast majority of the city's restaurants
remained open for business. They had barely begun to prepare for
closings that many owners said were seeming increasingly inevitable, at
least temporarily.

A day earlier,
\href{https://www.nytimes3xbfgragh.onion/2020/03/12/dining/restaurants-coronavirus.html}{chefs
were worrying about social distancing}, after Gov. Andrew M. Cuomo
ordered all restaurants that seated 500 customers or fewer to halve
their capacity and make more space between tables. Now they are worrying
about survival.

The city's restaurants have been through emergencies before, and stayed
open. Many fed first responders on Sept. 11, 2001 and ran generators
after Hurricane Sandy, acting as hubs for not only food but also
community news, phone charging and even showers.

\includegraphics{https://static01.graylady3jvrrxbe.onion/images/2020/03/13/dining/13Rest3/merlin_147074466_40f2b6f7-e34e-4490-bd54-111ff59a62ee-articleLarge.jpg?quality=75\&auto=webp\&disable=upscale}

``We are used to cooking our way out of crisis situations,'' said
\href{https://www.nytimes3xbfgragh.onion/2018/11/27/dining/saint-julivert-fisherie-review.html}{Alex
Raij}, co-owner of four small restaurants in Brooklyn and Manhattan,
including Saint Julivert Fisherie in Cobble Hill, Brooklyn. ``But this
is just a big unknown.''

The Union Square group said it would pay affected employees at least
through the end of the current pay cycle. The company also plans to
cover coronavirus-related medical costs for employees not covered by
insurance, but very few restaurants are in a position to do that.

To save cash in advance of the mass closings that many see as
inevitable, part-time employees have been let go, full-time employee
shifts have been cut and salaried managers are working double shifts.

``I think most restaurants will not survive this,'' said
\href{https://www.nytimes3xbfgragh.onion/2016/07/18/t-magazine/food/shuna-lydon-robert-rauschenberg-foundation-cook-recipes.html}{Shuna
Lydon}, the pastry chef at Worthwild, in Chelsea, who was told to stay
home from work Friday. She said that if New Yorkers want their favorite
places to weather this storm, they should visit them now --- but that
conflicting information about the health risks of dining out was making
that a difficult decision.

``You don't know if you should be scared about getting sick, or scared
about the business surviving,'' said Fabián von Hauske Valtierra, a chef
and co-owner of Wildair and Contra in the Lower East Side, on Thursday.

Many restaurateurs were in crisis-planning mode on Thursday when they
heard that as of Friday night, the state would require them to operate
at no more than 50 percent capacity for the foreseeable future.

Today, they said --- with black humor --- that 50 percent of normal
traffic would be a gift, and planning seemed increasingly pointless.

Image

The chef Eric Ripert, shown here in a 2016 photo at his seafood
restaurant Le Bernardin, said it was operating at a
loss.Credit...Timothy A. Clary/Agence France-Presse --- Getty Images

The chef Eric Ripert said his Midtown seafood temple
\href{https://www.nytimes3xbfgragh.onion/2018/06/05/dining/maguy-le-coze-le-bernardin.html}{Le
Bernardin} was currently operating at a loss. He, like many peers, said
this may be the point at which the national lack of support systems for
restaurants will have real consequences in the future. ``I have no
relationship with government at any level, so there is no information,''
he said.

There is no dedicated agency or point person for restaurants in city
government, and few resources specifically for restaurants within the
federal \href{https://www.sba.gov/}{Small Business Administration}. On
Monday, Mayor Bill de Blasio promised that some interest-free loans
would be available through the city's office of
\href{https://www1.nyc.gov/site/sbs/index.page}{Small Business
Services}, but did not specify how many would be made or how the
decisions would be made.

``We cannot overstate how many restaurants are facing a dire future
right now,'' said Melissa Fleischut, chief executive of the New York
State Restaurant Association, in a letter requesting tax relief, payment
extensions and other forms of economic relief sent to Governor Cuomo on
Thursday. Restaurateurs were taken by surprise by the mandatory capacity
reductions, and a few restaurants with a capacity of more than 500
guests had to close entirely, including the banquet hall
\href{https://twitter.com/JingFongNY/status/1238490120795414528}{Jing
Fong} in Chinatown.

``New Yorkers realize what restaurants mean to them,'' Ms. Raij said.
``But the government doesn't realize what an important part of the small
business economy we are.''

``If I were to name one thing which we really need from the city, it is
certainty,'' said Carlos Suarez, an owner of the West Village
restaurants Rosemary's, Claudette and Roey's.

And, of course, in the uncertain future they will need tax relief,
no-interest loans and rent forgiveness. The vast majority of New York
restaurants have no significant cash reserves or credit lines, and
cannot sustain the loss of business for more than a few weeks, let alone
months.

``I am concerned about the impact this will have on the whole food
chain,'' said Dan Kluger, the chef and co-owner of
\href{https://www.nytimes3xbfgragh.onion/2017/05/02/dining/loring-place-review-greenwich-village.html}{Loring
Place} in Greenwich Village, referring to restaurant workers, purveyors,
farmers and everyone whose work touches food. ``It's scary to think
about so many that will lose their livelihoods because of this.''

Amelia Nierenberg contributed reporting.

\emph{Follow} \href{https://twitter.com/nytfood}{\emph{NYT Food on
Twitter}} \emph{and}
\href{https://www.instagram.com/nytcooking/}{\emph{NYT Cooking on
Instagram}}\emph{,}
\href{https://www.facebookcorewwwi.onion/nytcooking/}{\emph{Facebook}}\emph{,}
\href{https://www.youtube.com/nytcooking}{\emph{YouTube}} \emph{and}
\href{https://www.pinterest.com/nytcooking/}{\emph{Pinterest}}\emph{.}
\href{https://www.nytimes3xbfgragh.onion/newsletters/cooking}{\emph{Get
regular updates from NYT Cooking, with recipe suggestions, cooking tips
and shopping advice}}\emph{.}

Advertisement

\protect\hyperlink{after-bottom}{Continue reading the main story}

\hypertarget{site-index}{%
\subsection{Site Index}\label{site-index}}

\hypertarget{site-information-navigation}{%
\subsection{Site Information
Navigation}\label{site-information-navigation}}

\begin{itemize}
\tightlist
\item
  \href{https://help.nytimes3xbfgragh.onion/hc/en-us/articles/115014792127-Copyright-notice}{©~2020~The
  New York Times Company}
\end{itemize}

\begin{itemize}
\tightlist
\item
  \href{https://www.nytco.com/}{NYTCo}
\item
  \href{https://help.nytimes3xbfgragh.onion/hc/en-us/articles/115015385887-Contact-Us}{Contact
  Us}
\item
  \href{https://www.nytco.com/careers/}{Work with us}
\item
  \href{https://nytmediakit.com/}{Advertise}
\item
  \href{http://www.tbrandstudio.com/}{T Brand Studio}
\item
  \href{https://www.nytimes3xbfgragh.onion/privacy/cookie-policy\#how-do-i-manage-trackers}{Your
  Ad Choices}
\item
  \href{https://www.nytimes3xbfgragh.onion/privacy}{Privacy}
\item
  \href{https://help.nytimes3xbfgragh.onion/hc/en-us/articles/115014893428-Terms-of-service}{Terms
  of Service}
\item
  \href{https://help.nytimes3xbfgragh.onion/hc/en-us/articles/115014893968-Terms-of-sale}{Terms
  of Sale}
\item
  \href{https://spiderbites.nytimes3xbfgragh.onion}{Site Map}
\item
  \href{https://help.nytimes3xbfgragh.onion/hc/en-us}{Help}
\item
  \href{https://www.nytimes3xbfgragh.onion/subscription?campaignId=37WXW}{Subscriptions}
\end{itemize}
