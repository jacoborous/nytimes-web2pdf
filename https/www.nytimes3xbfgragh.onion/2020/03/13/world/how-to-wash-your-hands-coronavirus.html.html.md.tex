Sections

SEARCH

\protect\hyperlink{site-content}{Skip to
content}\protect\hyperlink{site-index}{Skip to site index}

\href{https://www.nytimes3xbfgragh.onion/section/world}{World}

\href{https://myaccount.nytimes3xbfgragh.onion/auth/login?response_type=cookie\&client_id=vi}{}

\href{https://www.nytimes3xbfgragh.onion/section/todayspaper}{Today's
Paper}

\href{/section/world}{World}\textbar{}How to Wash Your Hands

\url{https://nyti.ms/3aWwaVU}

\begin{itemize}
\item
\item
\item
\item
\item
\end{itemize}

\hypertarget{the-coronavirus-outbreak}{%
\subsubsection{\texorpdfstring{\href{https://www.nytimes3xbfgragh.onion/news-event/coronavirus?name=styln-coronavirus-national\&region=TOP_BANNER\&block=storyline_menu_recirc\&action=click\&pgtype=Article\&impression_id=4f3f7b10-efb9-11ea-a507-6168886b5d2d\&variant=undefined}{The
Coronavirus
Outbreak}}{The Coronavirus Outbreak}}\label{the-coronavirus-outbreak}}

\begin{itemize}
\tightlist
\item
  live\href{https://www.nytimes3xbfgragh.onion/2020/09/05/world/coronavirus-covid.html?name=styln-coronavirus-national\&region=TOP_BANNER\&block=storyline_menu_recirc\&action=click\&pgtype=Article\&impression_id=4f3f7b11-efb9-11ea-a507-6168886b5d2d\&variant=undefined}{Latest
  Updates}
\item
  \href{https://www.nytimes3xbfgragh.onion/interactive/2020/us/coronavirus-us-cases.html?name=styln-coronavirus-national\&region=TOP_BANNER\&block=storyline_menu_recirc\&action=click\&pgtype=Article\&impression_id=4f3f7b12-efb9-11ea-a507-6168886b5d2d\&variant=undefined}{Maps
  and Cases}
\item
  \href{https://www.nytimes3xbfgragh.onion/interactive/2020/science/coronavirus-vaccine-tracker.html?name=styln-coronavirus-national\&region=TOP_BANNER\&block=storyline_menu_recirc\&action=click\&pgtype=Article\&impression_id=4f3f7b13-efb9-11ea-a507-6168886b5d2d\&variant=undefined}{Vaccine
  Tracker}
\item
  \href{https://www.nytimes3xbfgragh.onion/2020/09/02/your-money/eviction-moratorium-covid.html?name=styln-coronavirus-national\&region=TOP_BANNER\&block=storyline_menu_recirc\&action=click\&pgtype=Article\&impression_id=4f3f7b14-efb9-11ea-a507-6168886b5d2d\&variant=undefined}{Eviction
  Moratorium}
\item
  \href{https://www.nytimes3xbfgragh.onion/interactive/2020/09/02/magazine/food-insecurity-hunger-us.html?name=styln-coronavirus-national\&region=TOP_BANNER\&block=storyline_menu_recirc\&action=click\&pgtype=Article\&impression_id=4f3f7b15-efb9-11ea-a507-6168886b5d2d\&variant=undefined}{American
  Hunger}
\end{itemize}

Advertisement

\protect\hyperlink{after-top}{Continue reading the main story}

Supported by

\protect\hyperlink{after-sponsor}{Continue reading the main story}

\hypertarget{how-to-wash-your-hands}{%
\section{How to Wash Your Hands}\label{how-to-wash-your-hands}}

It's one of the most important things you can do to protect yourself,
your family and your community. And if you're like us, you've probably
been doing it wrong.

\includegraphics{https://static01.graylady3jvrrxbe.onion/images/2020/03/12/reader-center/Handwashing_Cover_7/Handwashing_Cover_7-mediumSquareAt3X.jpg}

Published March 13, 2020Updated March 19, 2020

\begin{itemize}
\item
\item
\item
\item
\item
\end{itemize}

\href{https://www.nytimes3xbfgragh.onion/es/2020/03/19/espanol/ciencia-y-tecnologia/como-lavarse-las-manos-coronavirus.html}{Leer
en español}

\emph{We asked the experts how to wash our hands. After they taught us
the proper technique (above), we had a few more questions:}

\textbf{Will touching the wet faucet really ruin everything?}

Yes.

The faucet may have the same germs you started with! Use a tissue or
paper towel to turn it off once your hands are clean. You don't want to
begin again, do you?

\textbf{How hard do I have to scrub?}

Most people don't rub vigorously enough, said Barbara Smith, a nurse
epidemiologist and infection prevention specialist at Mount Sinai Health
Systems in New York.

When you wash your hands, you are using soap and water to physically
dislodge germs from your skin and then rinse them away.

\textbf{Do I really have to dry my hands all the way?}

Most people don't dry thoroughly enough. **** Germs love moisture. And
don't be afraid to use a little force here too: You are physically
removing whatever germs remain.

\hypertarget{latest-updates-the-coronavirus-outbreak}{%
\section{\texorpdfstring{\href{https://www.nytimes3xbfgragh.onion/2020/09/04/world/covid-19-coronavirus.html?action=click\&pgtype=Article\&state=default\&region=MAIN_CONTENT_1\&context=storylines_live_updates}{Latest
Updates: The Coronavirus
Outbreak}}{Latest Updates: The Coronavirus Outbreak}}\label{latest-updates-the-coronavirus-outbreak}}

Updated 2020-09-05T12:05:40.998Z

\begin{itemize}
\tightlist
\item
  \href{https://www.nytimes3xbfgragh.onion/2020/09/04/world/covid-19-coronavirus.html?action=click\&pgtype=Article\&state=default\&region=MAIN_CONTENT_1\&context=storylines_live_updates\#link-1654f6ad}{Research
  connects vaping to a higher chance of catching the virus --- and
  suffering its worst effects.}
\item
  \href{https://www.nytimes3xbfgragh.onion/2020/09/04/world/covid-19-coronavirus.html?action=click\&pgtype=Article\&state=default\&region=MAIN_CONTENT_1\&context=storylines_live_updates\#link-52e4198a}{Another
  college football game won't be played as planned.}
\item
  \href{https://www.nytimes3xbfgragh.onion/2020/09/04/world/covid-19-coronavirus.html?action=click\&pgtype=Article\&state=default\&region=MAIN_CONTENT_1\&context=storylines_live_updates\#link-181cef0}{Pharmaceutical
  companies plan a joint pledge on safety standards as they move
  vaccines to the marketplace.}
\end{itemize}

\href{https://www.nytimes3xbfgragh.onion/2020/09/04/world/covid-19-coronavirus.html?action=click\&pgtype=Article\&state=default\&region=MAIN_CONTENT_1\&context=storylines_live_updates}{See
more updates}

More live coverage:
\href{https://www.nytimes3xbfgragh.onion/live/2020/09/04/business/stock-market-today-coronavirus?action=click\&pgtype=Article\&state=default\&region=MAIN_CONTENT_1\&context=storylines_live_updates}{Markets}

\textbf{Do I have to use paper towels?}

No. Cloth towels are fine for personal use, but should be washed every
few days --- more if multiple people use the same towel. A sick person
should use a separate towel. Use paper towels for guests!

\textbf{What about an air dryer?}

In terms of hygiene,
\href{https://www.sciencedirect.com/science/article/abs/pii/S002561961200393X}{paper
towels are best}. Hand dryers are OK, so long as you dry your hands
thoroughly. There is inconclusive research that suggests a higher germ
concentration around some hand dryers, but using a hand dryer is
definitely better than wiping your hands on your pants.

\textbf{What's the best way to know you've washed for 20 seconds?}

One one-thousand, two one-thousand, three one-thousand\ldots{}

\textbf{Does it matter what kind of soap I use?}

Liquid soap is best. Bar soap is fine too, just don't let it sit around
in a gloppy dish. Remember: germs love moisture.

\textbf{When should I be washing my hands?}

\begin{itemize}
\item
  Before you leave the house (to protect others from your germs).
\item
  And when you arrive at your destination (to wash off germs you've
  picked up from door knobs, elevator buttons, public transportation,
  etc.)
\item
  Before and after you eat or prepare food.
\item
  Before and after you clean your home.
\item
  After you blow your nose, cough or sneeze.
\item
  After you use the bathroom or change a diaper.
\item
  After you feed or touch a pet.
\end{itemize}

\textbf{What about lotion for dry hands?}

Yes, but use your own personal supply (most lotion does not contain
antibacterial agents, so it should not be shared). And don't forget to
keep the bottle and dispenser clean!

\href{https://www.nytimes3xbfgragh.onion/news-event/coronavirus?action=click\&pgtype=Article\&state=default\&region=MAIN_CONTENT_3\&context=storylines_faq}{}

\hypertarget{the-coronavirus-outbreak-}{%
\subsubsection{The Coronavirus Outbreak
›}\label{the-coronavirus-outbreak-}}

\hypertarget{frequently-asked-questions}{%
\paragraph{Frequently Asked
Questions}\label{frequently-asked-questions}}

Updated September 4, 2020

\begin{itemize}
\item ~
  \hypertarget{what-are-the-symptoms-of-coronavirus}{%
  \paragraph{What are the symptoms of
  coronavirus?}\label{what-are-the-symptoms-of-coronavirus}}

  \begin{itemize}
  \tightlist
  \item
    In the beginning, the coronavirus
    \href{https://www.nytimes3xbfgragh.onion/article/coronavirus-facts-history.html?action=click\&pgtype=Article\&state=default\&region=MAIN_CONTENT_3\&context=storylines_faq\#link-6817bab5}{seemed
    like it was primarily a respiratory illness}~--- many patients had
    fever and chills, were weak and tired, and coughed a lot, though
    some people don't show many symptoms at all. Those who seemed
    sickest had pneumonia or acute respiratory distress syndrome and
    received supplemental oxygen. By now, doctors have identified many
    more symptoms and syndromes. In April,
    \href{https://www.nytimes3xbfgragh.onion/2020/04/27/health/coronavirus-symptoms-cdc.html?action=click\&pgtype=Article\&state=default\&region=MAIN_CONTENT_3\&context=storylines_faq}{the
    C.D.C. added to the list of early signs}~sore throat, fever, chills
    and muscle aches. Gastrointestinal upset, such as diarrhea and
    nausea, has also been observed. Another telltale sign of infection
    may be a sudden, profound diminution of one's
    \href{https://www.nytimes3xbfgragh.onion/2020/03/22/health/coronavirus-symptoms-smell-taste.html?action=click\&pgtype=Article\&state=default\&region=MAIN_CONTENT_3\&context=storylines_faq}{sense
    of smell and taste.}~Teenagers and young adults in some cases have
    developed painful red and purple lesions on their fingers and toes
    --- nicknamed ``Covid toe'' --- but few other serious symptoms.
  \end{itemize}
\item ~
  \hypertarget{why-is-it-safer-to-spend-time-together-outside}{%
  \paragraph{Why is it safer to spend time together
  outside?}\label{why-is-it-safer-to-spend-time-together-outside}}

  \begin{itemize}
  \tightlist
  \item
    \href{https://www.nytimes3xbfgragh.onion/2020/05/15/us/coronavirus-what-to-do-outside.html?action=click\&pgtype=Article\&state=default\&region=MAIN_CONTENT_3\&context=storylines_faq}{Outdoor
    gatherings}~lower risk because wind disperses viral droplets, and
    sunlight can kill some of the virus. Open spaces prevent the virus
    from building up in concentrated amounts and being inhaled, which
    can happen when infected people exhale in a confined space for long
    stretches of time, said Dr. Julian W. Tang, a virologist at the
    University of Leicester.
  \end{itemize}
\item ~
  \hypertarget{why-does-standing-six-feet-away-from-others-help}{%
  \paragraph{Why does standing six feet away from others
  help?}\label{why-does-standing-six-feet-away-from-others-help}}

  \begin{itemize}
  \tightlist
  \item
    The coronavirus spreads primarily through droplets from your mouth
    and nose, especially when you cough or sneeze. The C.D.C., one of
    the organizations using that measure,
    \href{https://www.nytimes3xbfgragh.onion/2020/04/14/health/coronavirus-six-feet.html?action=click\&pgtype=Article\&state=default\&region=MAIN_CONTENT_3\&context=storylines_faq}{bases
    its recommendation of six feet}~on the idea that most large droplets
    that people expel when they cough or sneeze will fall to the ground
    within six feet. But six feet has never been a magic number that
    guarantees complete protection. Sneezes, for instance, can launch
    droplets a lot farther than six feet,
    \href{https://jamanetwork.com/journals/jama/fullarticle/2763852}{according
    to a recent study}. It's a rule of thumb: You should be safest
    standing six feet apart outside, especially when it's windy. But
    keep a mask on at all times, even when you think you're far enough
    apart.
  \end{itemize}
\item ~
  \hypertarget{i-have-antibodies-am-i-now-immune}{%
  \paragraph{I have antibodies. Am I now
  immune?}\label{i-have-antibodies-am-i-now-immune}}

  \begin{itemize}
  \tightlist
  \item
    As of right
    now,\href{https://www.nytimes3xbfgragh.onion/2020/07/22/health/covid-antibodies-herd-immunity.html?action=click\&pgtype=Article\&state=default\&region=MAIN_CONTENT_3\&context=storylines_faq}{~that
    seems likely, for at least several months.}~There have been
    frightening accounts of people suffering what seems to be a second
    bout of Covid-19. But experts say these patients may have a
    drawn-out course of infection, with the virus taking a slow toll
    weeks to months after initial exposure.~People infected with the
    coronavirus typically
    \href{https://www.nature.com/articles/s41586-020-2456-9}{produce}~immune
    molecules called antibodies, which are
    \href{https://www.nytimes3xbfgragh.onion/2020/05/07/health/coronavirus-antibody-prevalence.html?action=click\&pgtype=Article\&state=default\&region=MAIN_CONTENT_3\&context=storylines_faq}{protective
    proteins made in response to an
    infection}\href{https://www.nytimes3xbfgragh.onion/2020/05/07/health/coronavirus-antibody-prevalence.html?action=click\&pgtype=Article\&state=default\&region=MAIN_CONTENT_3\&context=storylines_faq}{.
    These antibodies may}~last in the body
    \href{https://www.nature.com/articles/s41591-020-0965-6}{only two to
    three months}, which may seem worrisome, but that's~perfectly normal
    after an acute infection subsides, said Dr. Michael Mina, an
    immunologist at Harvard University. It may be possible to get the
    coronavirus again, but it's highly unlikely that it would be
    possible in a short window of time from initial infection or make
    people sicker the second time.
  \end{itemize}
\item ~
  \hypertarget{what-are-my-rights-if-i-am-worried-about-going-back-to-work}{%
  \paragraph{What are my rights if I am worried about going back to
  work?}\label{what-are-my-rights-if-i-am-worried-about-going-back-to-work}}

  \begin{itemize}
  \tightlist
  \item
    Employers have to provide
    \href{https://www.osha.gov/SLTC/covid-19/standards.html}{a safe
    workplace}~with policies that protect everyone equally.
    \href{https://www.nytimes3xbfgragh.onion/article/coronavirus-money-unemployment.html?action=click\&pgtype=Article\&state=default\&region=MAIN_CONTENT_3\&context=storylines_faq}{And
    if one of your co-workers tests positive for the coronavirus, the
    C.D.C.}~has said that
    \href{https://www.cdc.gov/coronavirus/2019-ncov/community/guidance-business-response.html}{employers
    should tell their employees}~-\/- without giving you the sick
    employee's name -\/- that they may have been exposed to the virus.
  \end{itemize}
\end{itemize}

\textbf{Can I still paint my nails?}

Yes, but it's best to keep your nails short and your manicure fresh.
Germs can live in cracked and chipped polish.

\textbf{What's the technique with hand sanitizer?}

Use alcohol-based hand sanitizer with at least 60 percent alcohol, and
scrub your hands the same way you would with soap and water. Be sure to
use enough liquid so you can reach every surface of your hands.

\emph{Note:} If you've seen the recipe circulating on social media for
homemade sanitizer using aloe vera gel and rubbing alcohol, we tried it
and it didn't work. You'll just wind up with a batch of diluted alcohol.

\textbf{Is hand-washing really that important?}

``Your hands carry almost all your germs to your respiratory tract.
Keeping them as clean as possible is really helpful,'' said Dr. Adit
Ginde, professor of emergency medicine at the University of Colorado
School of Medicine. ``It would dramatically reduce transmission if
people did it well.''

By Tara Parker-Pope, Shane O'Neill, Jonah Kessel and Alicia DeSantis.

Advertisement

\protect\hyperlink{after-bottom}{Continue reading the main story}

\hypertarget{site-index}{%
\subsection{Site Index}\label{site-index}}

\hypertarget{site-information-navigation}{%
\subsection{Site Information
Navigation}\label{site-information-navigation}}

\begin{itemize}
\tightlist
\item
  \href{https://help.nytimes3xbfgragh.onion/hc/en-us/articles/115014792127-Copyright-notice}{©~2020~The
  New York Times Company}
\end{itemize}

\begin{itemize}
\tightlist
\item
  \href{https://www.nytco.com/}{NYTCo}
\item
  \href{https://help.nytimes3xbfgragh.onion/hc/en-us/articles/115015385887-Contact-Us}{Contact
  Us}
\item
  \href{https://www.nytco.com/careers/}{Work with us}
\item
  \href{https://nytmediakit.com/}{Advertise}
\item
  \href{http://www.tbrandstudio.com/}{T Brand Studio}
\item
  \href{https://www.nytimes3xbfgragh.onion/privacy/cookie-policy\#how-do-i-manage-trackers}{Your
  Ad Choices}
\item
  \href{https://www.nytimes3xbfgragh.onion/privacy}{Privacy}
\item
  \href{https://help.nytimes3xbfgragh.onion/hc/en-us/articles/115014893428-Terms-of-service}{Terms
  of Service}
\item
  \href{https://help.nytimes3xbfgragh.onion/hc/en-us/articles/115014893968-Terms-of-sale}{Terms
  of Sale}
\item
  \href{https://spiderbites.nytimes3xbfgragh.onion}{Site Map}
\item
  \href{https://help.nytimes3xbfgragh.onion/hc/en-us}{Help}
\item
  \href{https://www.nytimes3xbfgragh.onion/subscription?campaignId=37WXW}{Subscriptions}
\end{itemize}
