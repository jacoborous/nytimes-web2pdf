Sections

SEARCH

\protect\hyperlink{site-content}{Skip to
content}\protect\hyperlink{site-index}{Skip to site index}

\href{https://www.nytimes3xbfgragh.onion/section/world/asia}{Asia
Pacific}

\href{https://myaccount.nytimes3xbfgragh.onion/auth/login?response_type=cookie\&client_id=vi}{}

\href{https://www.nytimes3xbfgragh.onion/section/todayspaper}{Today's
Paper}

\href{/section/world/asia}{Asia Pacific}\textbar{}U.S. Limits Chinese
Staff at News Agencies Controlled by Beijing

\href{https://nyti.ms/2VFyo7M}{https://nyti.ms/2VFyo7M}

\begin{itemize}
\item
\item
\item
\item
\item
\end{itemize}

Advertisement

\protect\hyperlink{after-top}{Continue reading the main story}

Supported by

\protect\hyperlink{after-sponsor}{Continue reading the main story}

\hypertarget{us-limits-chinese-staff-at-news-agencies-controlled-by-beijing}{%
\section{U.S. Limits Chinese Staff at News Agencies Controlled by
Beijing}\label{us-limits-chinese-staff-at-news-agencies-controlled-by-beijing}}

The new rule could effectively force 60 Chinese citizens to leave the
United States in a tit-for-tat campaign between Washington and Beijing
that has caught journalists in the crossfire.

\includegraphics{https://static01.graylady3jvrrxbe.onion/images/2020/03/02/us/politics/02dc-chinamedia/merlin_45526087_60b02d26-c369-434b-bd15-1af32dc24279-articleLarge.jpg?quality=75\&auto=webp\&disable=upscale}

\href{https://www.nytimes3xbfgragh.onion/by/lara-jakes}{\includegraphics{https://static01.graylady3jvrrxbe.onion/images/2019/07/25/reader-center/author-lara-jakes/author-lara-jakes-thumbLarge.png}}\href{https://www.nytimes3xbfgragh.onion/by/marc-tracy}{\includegraphics{https://static01.graylady3jvrrxbe.onion/images/2018/02/20/multimedia/author-marc-tracy/author-marc-tracy-thumbLarge.jpg}}

By \href{https://www.nytimes3xbfgragh.onion/by/lara-jakes}{Lara Jakes}
and \href{https://www.nytimes3xbfgragh.onion/by/marc-tracy}{Marc Tracy}

\begin{itemize}
\item
  March 2, 2020
\item
  \begin{itemize}
  \item
  \item
  \item
  \item
  \item
  \end{itemize}
\end{itemize}

\href{https://cn.nytimes3xbfgragh.onion/usa/20200303/china-journalists-diplomats-expulsion/}{阅读简体中文版}\href{https://cn.nytimes3xbfgragh.onion/usa/20200303/china-journalists-diplomats-expulsion/zh-hant/}{閱讀繁體中文版}

WASHINGTON --- The Trump administration on Monday limited to 100 the
number of Chinese citizens who may work in the United States for five
state-controlled Chinese news organizations. The decision is expected to
escalate tensions between Washington and Beijing in a diplomatic feud
that has caught journalists in the crossfire.

The State Department insisted that it was not expelling Chinese
journalists,
\href{https://www.nytimes3xbfgragh.onion/2020/02/19/business/media/china-wall-street-journal.html}{a
step that Beijing took} last month against three Wall Street Journal
reporters --- the first time foreign correspondents had been ordered to
leave China since 1998.

But the new limits could, in effect, force Chinese citizens to leave if
their visas allowing them to work in the United States are tied to the
news organizations that must now choose which employees will remain on
assignment in the United States.

In a statement, Secretary of State Mike Pompeo said the United States
would not restrict the content of what the five organizations report ---
a contrast to what he described as increased surveillance, harassment
and intimidation of foreign journalists in China.

``Our goal is reciprocity,'' Mr. Pompeo said. ``As we have done in other
areas of the U.S.-China relationship, we seek to establish a
long-overdue level playing field. It is our hope that this action will
spur Beijing to adopt a more fair and reciprocal approach to U.S. and
other foreign press in China.''

It was the latest in the tit-for-tat statecraft that has become a
hallmark of the strategic rivalry between the United States and China,
encompassing
\href{https://www.nytimes3xbfgragh.onion/2020/01/15/business/economy/china-trade-deal.html}{trade},
\href{https://www.nytimes3xbfgragh.onion/2020/02/11/world/asia/philippines-united-states-duterte.html}{military}
and
\href{https://www.nytimes3xbfgragh.onion/2020/02/10/us/politics/equifax-hack-china.html}{cybersecurity}
confrontations,
\href{https://www.nytimes3xbfgragh.onion/interactive/2019/11/16/world/asia/china-xinjiang-documents.html}{tensions
over human rights} and the
\href{https://www.nytimes3xbfgragh.onion/2020/02/18/world/asia/china-coronavirus.html}{coronavirus
epidemic}.

In an episode last fall, the Trump administration
\href{https://www.nytimes3xbfgragh.onion/2019/12/15/world/asia/us-china-spies.html}{expelled
two Chinese officials} who trespassed on an important military base in
Virginia, evading the authorities until fire trucks blocked their car.
People close to China's embassy in Washington described it as an
accident by officials who got lost while sightseeing.

The restriction announced on Monday applies only to Chinese citizens
working at five news organizations that the State Department deemed
propaganda outlets controlled by the government in Beijing. It requires
the five organizations --- Xinhua, CGTN, China Radio, China Daily and
People's Daily --- to limit the number of Chinese employees in the
United States to 100, collectively. More than half --- 59 --- have been
allocated to
\href{https://www.facebookcorewwwi.onion/XinhuaNewsAgency/}{Xinhua},
China's official news agency.

Currently, about 160 Chinese citizens work in the United States for
those news outlets, meaning that 60 must either leave by the time the
new policy takes effect on March 13 or ensure they have a visa that will
allow them to stay.

Senior State Department officials said the five organizations would have
broad authority to decide who on staff --- whether journalists, managers
or other employees --- would remain. The officials said the move was not
a result of any specific information or content that the five
organizations have published or broadcast.

Instead, two senior State Department officials said, the new limits seek
to punish Beijing for what they described as systematic stifling of
press freedoms against foreign reporters in China. The two officials
briefed reporters on condition of anonymity as required under State
Department protocol. Separately, a senior official said the Trump
administration might soon limit how long the media visas allow Chinese
journalists to stay in the United States.

The State Department officials would not discuss what steps China might
take in retaliation.

Beyond the expulsions of the Journal reporters, the Chinese government
has repeatedly allowed the visas of foreign correspondents whose work is
perceived as unfriendly to lapse, forcing them to leave the country.

A report released Monday by the Foreign Correspondents' Club of China
pointed to another Wall Street Journal journalist who in the past year
was in effect forced to leave. He had written about potential ties
between organized crime and a cousin of President Xi Jinping of China.

The report said that in at least five other cases, the Chinese
government had similarly declined to renew resident journalists'
long-term visas. Reporters formerly from Al Jazeera English and The
Guardian confirmed to The New York Times that the government had
declined to grant them new visas, without explanation, ending for all
intents and purposes their careers as journalists in China.

The journalists' report, titled ``Control, Halt, Delete: Reporting in
China Under Threat of Expulsion,'' also focused on what it characterized
as a ramped-up practice of issuing truncated long-term visas --- ones
that must be renewed after a short period in an onerous process that
also implicates visa-holders' family members in the country. The report
called this a means of harassing journalists and sending the
not-so-subtle message that the Chinese government was displeased with
their reporting, whether it concerned Mr. Xi, protests in Hong Kong or
the treatment of ethnic minorities.

In all, the report concluded, at least a dozen correspondents received
visas for six months or less in 2019, compared with five the year
before. The standard length for a long-term journalist visa in China,
known as a J-1, is one year.

``Chinese authorities are using visas as weapons against the foreign
press like never before, expanding their deployment of a longtime
intimidation tactic as working conditions for foreign journalists in
China severely deteriorated in 2019,'' said the report, which was based
on a survey of more than 100 journalists in China from 25 countries.

Frédéric Lemaître, a China-based correspondent for the French newspaper
Le Monde, said he was on his second consecutive three-month visa after
he wrote a series about Mr. Xi. Mr. Lemaître said an official in the
press office of the Ministry of Foreign Affairs had told him, ``You are
our guest, and in China, guests must respect the host.''

About 100 American journalists are currently working in China for U.S.
and other foreign news organizations, State Department officials said.
By comparison, the United States issued 425 media visas to Chinese
journalists and their families in 2019 alone.

At a daily news conference on Tuesday, a spokesman for China's Foreign
Ministry, Zhao Lijian, responded to the restrictions announced Monday,
accusing the United States of unwarranted prejudice against the Chinese
news media.

Mr. Zhao took exception with Mr. Pompeo's stated goal of reciprocity,
saying that there were fewer Chinese media organizations operating in
the United States than vice versa, and that visas issued to Chinese
journalists for the United States were more restrictive than those
granted to U.S. reporters in China.

``The United States says reciprocity every time when it opens its mouth,
it is essentially prejudice, discrimination, and exclusion against the
Chinese media,'' Mr. Zhao said. ``It was the United States that broke
the rules of the game first,'' he added.

While the Foreign Ministry formally oversees foreign journalists,
accreditation decisions appear to involve other arms of the government,
including security and propaganda agencies. The decision to expel the
Journal reporters was unlikely to have been made solely by the ministry.

China expelled the Journal reporters the day after the State Department
\href{https://www.nytimes3xbfgragh.onion/2020/02/18/world/asia/china-media-trump.html}{declared
that it would} officially treat the five Chinese news agencies as
foreign government functionaries and therefore subject to similar rules
and restrictions as diplomats stationed in the United States.

In announcing the expulsions, a Foreign Ministry spokesman cited a
controversial headline to an opinion article in the Journal from last
month that had referred to China as the ``Real Sick Man of Asia.'' The
article criticized China's response to the coronavirus outbreak.

Lara Jakes reported from Washington and Marc Tracy from New York. Claire
Fu contributed research from Beijing.

Advertisement

\protect\hyperlink{after-bottom}{Continue reading the main story}

\hypertarget{site-index}{%
\subsection{Site Index}\label{site-index}}

\hypertarget{site-information-navigation}{%
\subsection{Site Information
Navigation}\label{site-information-navigation}}

\begin{itemize}
\tightlist
\item
  \href{https://help.nytimes3xbfgragh.onion/hc/en-us/articles/115014792127-Copyright-notice}{©~2020~The
  New York Times Company}
\end{itemize}

\begin{itemize}
\tightlist
\item
  \href{https://www.nytco.com/}{NYTCo}
\item
  \href{https://help.nytimes3xbfgragh.onion/hc/en-us/articles/115015385887-Contact-Us}{Contact
  Us}
\item
  \href{https://www.nytco.com/careers/}{Work with us}
\item
  \href{https://nytmediakit.com/}{Advertise}
\item
  \href{http://www.tbrandstudio.com/}{T Brand Studio}
\item
  \href{https://www.nytimes3xbfgragh.onion/privacy/cookie-policy\#how-do-i-manage-trackers}{Your
  Ad Choices}
\item
  \href{https://www.nytimes3xbfgragh.onion/privacy}{Privacy}
\item
  \href{https://help.nytimes3xbfgragh.onion/hc/en-us/articles/115014893428-Terms-of-service}{Terms
  of Service}
\item
  \href{https://help.nytimes3xbfgragh.onion/hc/en-us/articles/115014893968-Terms-of-sale}{Terms
  of Sale}
\item
  \href{https://spiderbites.nytimes3xbfgragh.onion}{Site Map}
\item
  \href{https://help.nytimes3xbfgragh.onion/hc/en-us}{Help}
\item
  \href{https://www.nytimes3xbfgragh.onion/subscription?campaignId=37WXW}{Subscriptions}
\end{itemize}
