Sections

SEARCH

\protect\hyperlink{site-content}{Skip to
content}\protect\hyperlink{site-index}{Skip to site index}

\href{https://www.nytimes3xbfgragh.onion/section/health}{Health}

\href{https://myaccount.nytimes3xbfgragh.onion/auth/login?response_type=cookie\&client_id=vi}{}

\href{https://www.nytimes3xbfgragh.onion/section/todayspaper}{Today's
Paper}

\href{/section/health}{Health}\textbar{}When a Drug Study Abruptly Ends,
Volunteers Are Left to Cope

\url{https://nyti.ms/2IeajwX}

\begin{itemize}
\item
\item
\item
\item
\item
\item
\end{itemize}

Advertisement

\protect\hyperlink{after-top}{Continue reading the main story}

Supported by

\protect\hyperlink{after-sponsor}{Continue reading the main story}

the new old age

\hypertarget{when-a-drug-study-abruptly-ends-volunteers-are-left-to-cope}{%
\section{When a Drug Study Abruptly Ends, Volunteers Are Left to
Cope}\label{when-a-drug-study-abruptly-ends-volunteers-are-left-to-cope}}

A participant might commit months or years to a drug trial, only to see
it vanish overnight.

\includegraphics{https://static01.graylady3jvrrxbe.onion/images/2020/03/03/science/03SCI-SPAN1/03SCI-SPAN1-articleLarge-v2.jpg?quality=75\&auto=webp\&disable=upscale}

By \href{https://www.nytimes3xbfgragh.onion/by/paula-span}{Paula Span}

\begin{itemize}
\item
  March 3, 2020
\item
  \begin{itemize}
  \item
  \item
  \item
  \item
  \item
  \item
  \end{itemize}
\end{itemize}

On March 21, 2019, the staff at the Penn Memory Center in Philadelphia
was scrambling to learn more about an early-morning announcement: Two
pharmaceutical companies, Biogen and Eisai, would discontinue their
clinical trial of a drug intended to slow the progression of early
Alzheimer's disease.

A
``\href{http://investors.biogen.com/news-releases/news-release-details/biogen-and-eisai-discontinue-phase-3-engage-and-emerge-trials}{futility
analysis}'' had shown that aducanumab, being studied in more than 3,200
people worldwide, would not prove effective. It was yet another
disheartening result; after decades of drug research,
\href{https://www.nytimes3xbfgragh.onion/2020/02/10/health/alzheimers-amyloid-drug.html}{one
medication after another} --- hundreds of them --- had failed to
prevent, arrest or cure Alzheimer's.

The Penn researchers wanted to be the ones to break the bad news to the
18 participants they had recruited.

``When this effort you contributed months and years to is ending, that's
something you want to hear from people you trust,'' said Emily Largent,
a bioethicist and researcher there.

But the Penn staff was too late to inform John Poritsky, a participant
with early-onset Alzheimer's, and his wife, Debra Morris. The news had
already begun circulating online.

``My friend had sent me a text, `Did you hear that this study is
ending?''' said Dr. Morris, an English professor at the Pennsylvania
College of Technology. ``I was horrified. Floored. I couldn't believe
it.''

For nearly a year, they had regularly traveled three hours from their
home in Williamsport, Pa., to Philadelphia, where Dr. Poritsky had
undergone extensive testing and received monthly infusions without
knowing whether he was receiving the drug or a placebo.

``I'd built up a lot of hope,'' said Dr. Poritsky, 61, a retired English
professor. He wasn't surprised to have developed Alzheimer's; his
father, grandfather and great-uncle all had the disease. But he had
hardly expected a diagnosis before he turned 60.

This drug study, a Phase 3 trial, had allowed him to think not only that
he might benefit personally, but that he could help advance science. ``I
thought, I can be part of something that can cure or arrest this
illness,'' he said. When the plug was suddenly pulled, ``I was just
devastated.''

This scenario occurs with distressing frequency. Most Alzheimer's drug
trials are sponsored by publicly held pharmaceutical companies, which
must follow federal Securities and Exchange Commission regulations when
they disclose information that affects stock prices.

Alerting patients or investigators before notifying shareholders would
violate the companies' legal obligations. So they often issue
early-morning news releases.

Years ago, most patients probably learned about discontinued trials from
researchers and staff whom they had come to know. (The last Alzheimer's
medication to receive F.D.A. approval was Namenda, in 2003.)

But with social media and 24-hour digital reporting, plus keen public
interest in Alzheimer's drugs, ``this has become fast-moving news in a
way it wasn't before,'' Dr. Largent said.

\includegraphics{https://static01.graylady3jvrrxbe.onion/images/2020/03/03/science/03SCI-SPAN2/merlin_169675215_cff83975-087d-4201-b677-36fde8408e51-articleLarge.jpg?quality=75\&auto=webp\&disable=upscale}

So by the time researchers are able to make phone calls, their patients
often have already seen the announcements on Facebook or fielded calls
from worried friends.

``It's akin to a trauma, the news that's devastating and the surprising,
out-of-the-blue way you learn it,'' said Dr. Jason Karlawish, a
geriatrician who co-directs the Penn Memory Center.

For Phil Gutis, 58, a former New York Times reporter diagnosed with
early onset Alzheimer's, ``it was a kick in the stomach.'' Like Dr.
Poritsky, he was enrolled in the aducanumab trial, and had
\href{https://www.nytimes3xbfgragh.onion/2019/03/22/well/mind/alzheimers-drug-trial-study-biogen-dementia-treatment-cure.html}{learned
of its termination} from a friend's text. ``There should be a better
way,'' he said.

Internationally, the Alzheimer's Association calculates that clinical
trials now underway for dementia treatments --- drugs, dietary programs,
devices and other interventions --- aim to enroll more than 56,000
people.

Drug trials for Alzheimer's disease are often a particularly arduous
commitment.

``These are not simple protocols,'' said Dr. Sharon Cohen, a neurologist
and principal investigator at the Toronto Memory Program, which had
enrolled 29 participants in the aducanumab trial. ``The visits are long.
They are frequent. There is in-depth testing. Blood draws. M.R.I. scans
that may recur. PET scans. There may be spinal taps. And the study
partner'' --- a family member or friend --- ``has to attend many of
these as well.''

Why agree to all that, especially when researchers pointedly explain
that the experimental drug may not help and could actually harm
patients? Moreover, in a typical double-blind study, half the
participants won't even get the drug but will instead receive a placebo.

Researchers and patients describe
\href{https://n.neurology.org/content/56/6/789}{a mix of motives}:
desperation, altruism, trust in the investigators and sponsors.

``I thought I was doing this for future generations,'' Mr. Gutis said.
But as he learned more about aducanumab, ``there was definitely optimism
that this possibly could help me.''

Participants also come to value their deepening relationships with the
study staff, who know so much about the disease and take such an
interest in their condition. ``It was almost familial,'' Dr. Poritsky
said.

``We were part of a community and a structure, and it was gone,'' Dr.
Morris said.

The aducanumab study also took an unexpected and uncommon turn. Seven
months after ending the trial, Biogen and Esai
\href{http://investors.biogen.com/news-releases/news-release-details/biogen-plans-regulatory-filing-aducanumab-alzheimers-disease}{announced}
(in another early-morning news release) that a reanalysis, using
additional data, indicated that at high doses the drug appeared to
reduce cognitive decline after all.

They plan to resume an open-label trial (without a placebo) in March and
to seek F.D.A. approval. The development was encouraging, but left study
participants feeling especially whipsawed.

In a recent editorial
\href{https://jamanetwork.com/journals/jamaneurology/fullarticle/10.1001/jamaneurol.2019.4974?guestAccessKey=419fcc84-061c-4f77-b240-7ae202352137\&utm_source=For_The_Media\&utm_medium=referral\&utm_campaign=ftm_links\&utm_content=tfl\&utm_term=022020}{in
JAMA Neurology}, Drs. Karlawish and Largent argued for a more
communicative approach. ``We're trying to change the culture of the way
we run clinical trials in Alzheimer's research,'' Dr. Karlawish said in
an interview.

Lobbying the S.E.C. to change its regulations would be ``infeasible,''
he acknowledged. But the informed consent process, the researchers
wrote, should prepare participants for the possibility of an abrupt end
to the trial.

Participants could opt to receive the pharmaceutical company's news
release, or a letter, as soon as it is issued, ``so you don't feel like
you're the last one to know,'' Dr. Largent said.

The researchers urged companies to share details of what a given study
revealed; even failed experiments provide useful information. ``It's an
important way of demonstrating respect for their contributions,'' Dr.
Largent said.

Finally, study sites should provide support after trials end, by
checking on the well-being of participants and referring them to
counseling or support groups as needed.

Some of those suggestions may take hold. The Alzheimer's Association and
the National Institute on Aging say they plan to meet with drug
manufacturers to discuss improving communication with research
participants.

The Toronto Memory Center has gone a step further. In 2018 it hosted a
lunch for participants and partners after a discontinued trial; the
event included a presentation of the results, ``to show them what their
efforts had led to,'' Dr. Cohen said. She described the participants as
``medical heroes, taking risks to benefit themselves and others.''

Last year, at another lunch, the center presented several participants
with citizen-scientist awards.

Despite the disappointments, participants often remain eager to join
other trials. When aducanumab testing resumes, both Mr. Gutis (who
learned that he had been receiving the drug rather than a placebo, and
thought it had helped him) and Dr. Poritsky (who thought so too, but had
received a placebo) plan to re-enroll.

They will moderate their expectations this time, however.

``You volunteer to be a lab rat,'' Mr. Gutis said. ``But the rat doesn't
have high hopes.''

Advertisement

\protect\hyperlink{after-bottom}{Continue reading the main story}

\hypertarget{site-index}{%
\subsection{Site Index}\label{site-index}}

\hypertarget{site-information-navigation}{%
\subsection{Site Information
Navigation}\label{site-information-navigation}}

\begin{itemize}
\tightlist
\item
  \href{https://help.nytimes3xbfgragh.onion/hc/en-us/articles/115014792127-Copyright-notice}{©~2020~The
  New York Times Company}
\end{itemize}

\begin{itemize}
\tightlist
\item
  \href{https://www.nytco.com/}{NYTCo}
\item
  \href{https://help.nytimes3xbfgragh.onion/hc/en-us/articles/115015385887-Contact-Us}{Contact
  Us}
\item
  \href{https://www.nytco.com/careers/}{Work with us}
\item
  \href{https://nytmediakit.com/}{Advertise}
\item
  \href{http://www.tbrandstudio.com/}{T Brand Studio}
\item
  \href{https://www.nytimes3xbfgragh.onion/privacy/cookie-policy\#how-do-i-manage-trackers}{Your
  Ad Choices}
\item
  \href{https://www.nytimes3xbfgragh.onion/privacy}{Privacy}
\item
  \href{https://help.nytimes3xbfgragh.onion/hc/en-us/articles/115014893428-Terms-of-service}{Terms
  of Service}
\item
  \href{https://help.nytimes3xbfgragh.onion/hc/en-us/articles/115014893968-Terms-of-sale}{Terms
  of Sale}
\item
  \href{https://spiderbites.nytimes3xbfgragh.onion}{Site Map}
\item
  \href{https://help.nytimes3xbfgragh.onion/hc/en-us}{Help}
\item
  \href{https://www.nytimes3xbfgragh.onion/subscription?campaignId=37WXW}{Subscriptions}
\end{itemize}
