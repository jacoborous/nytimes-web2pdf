Sections

SEARCH

\protect\hyperlink{site-content}{Skip to
content}\protect\hyperlink{site-index}{Skip to site index}

\href{https://www.nytimes3xbfgragh.onion/section/well}{Well}

\href{https://myaccount.nytimes3xbfgragh.onion/auth/login?response_type=cookie\&client_id=vi}{}

\href{https://www.nytimes3xbfgragh.onion/section/todayspaper}{Today's
Paper}

\href{/section/well}{Well}\textbar{}When Romance Is a Scam

\url{https://nyti.ms/2wK9pWK}

\begin{itemize}
\item
\item
\item
\item
\item
\item
\end{itemize}

Advertisement

\protect\hyperlink{after-top}{Continue reading the main story}

Supported by

\protect\hyperlink{after-sponsor}{Continue reading the main story}

The New old age

\hypertarget{when-romance-is-a-scam}{%
\section{When Romance Is a Scam}\label{when-romance-is-a-scam}}

More con artists are finding their marks on all manner of social media
platforms, knowing that the payoff from older victims can be big.

\includegraphics{https://static01.graylady3jvrrxbe.onion/images/2020/03/31/science/27SCI-SPAN1/merlin_171011121_11da129e-8299-4483-ae2d-c5752544b7fa-articleLarge.jpg?quality=75\&auto=webp\&disable=upscale}

By \href{https://www.nytimes3xbfgragh.onion/by/paula-span}{Paula Span}

\begin{itemize}
\item
  March 27, 2020
\item
  \begin{itemize}
  \item
  \item
  \item
  \item
  \item
  \item
  \end{itemize}
\end{itemize}

Ellen Floren was not looking for love.

The criminals who lured her into an online scam last summer approached
her not on a dating site, where she might have been wary, but through
the neighborhood hub called Nextdoor.

A man who said his name was James Gibson said he'd noticed her profile
on the site. He also lived in her Chicago neighborhood, he told her,
specifying a street. Could they have a conversation?

``He was very polite: `I hope I'm not out of line. I just found you very
attractive,''' recalled Ms. Floren, who is 67 and a part-time
educational consultant. They chatted on the site for a week or so.
``Then it was, `Is it OK if we email?'''

She agreed. Soon they shifted to phone conversations, often lasting an
hour, and to texting several times a day. ``It became very seductive,''
Ms. Floren said. How could she help sympathizing when he revealed that
his wife and child had been killed in a car crash long ago?

Though they had swapped photos, they hadn't met in person; he said he
was temporarily working in a distant suburb, at a high-level job in
communications systems, and staying at a hotel.

But after a few weeks, when he said he was coming to Chicago, they
arranged to have dinner. ``I thought, this is someone I'm going to enjoy
getting to know,'' Ms. Floren said. She was disappointed when the
supposed Mr. Gibson got in too late to see her, then apologetically said
he had just landed a big job in Europe and had to leave at once,
postponing their date.

The elements of deception and manipulation in Ms. Floren's saga sound
familiar to those knowledgeable about the rise
of\href{https://www.consumer.ftc.gov/articles/what-you-need-know-about-romance-scams}{online
romance swindles}.

Con artists now find victims on any social media platform --- Instagram,
Facebook, games like Words With Friends.

But ``they quickly want to remove you from the platform,'' said Amy
Nofziger, director of the AARP Fraud Watch Network. The romancers ask to
switch to text, phone or messaging apps that offer more intimacy and
less security monitoring. The exchange of personal contact information
also makes both parties appear trusting.

The tragic personal story, the quick professions of love combined with
distance that prevents the parties' ever meeting all fit the pattern,
said Monica Vaca, an associate director in the Bureau of Consumer
Protection at the Federal Trade Commission.

So do the photographs. ``You feel more like you know this person because
you've seen their picture,'' Ms. Vaca said. ``Invariably, it's a photo
of someone else.''

Weeks or months may pass before the scammers --- generally not
individuals, but criminal rings working in shifts (hence their ability
to be wooing online all day) --- make the key move.

They ask for money.

\href{https://www.ftc.gov/reports/consumer-sentinel-network-data-book-2019}{Reports}
collected by the F.T.C. from consumers and local law enforcement show
how sharply online romance fraud is increasing. In 2015, the agency
received 8,500 such complaints. Last year, the number topped 25,000 ---
though Ms. Vaca cautioned that ``this crime is dramatically
underreported.''

But what really drew regulators' attention, given that other fraud
categories generated more complaints, was the money involved. ``It's the
No. 1 fraud category if you look at the total dollars people reported
losing,'' Ms. Vaca said.

In 2015, people reported losing \$33 million to romance scams; last
year, they lost \$201 million --- more than victims lost to fake
lotteries and sweepstakes, impostor frauds or tech support phishing.

Older adults have been particularly hard hit. Anyone --- regardless of
age, gender or education level --- can fall for a romance swindle; in
fact, younger adults are more apt to report losing money to these
frauds. ``But when older people do report losing money, their dollar
losses are much higher,'' Ms. Vaca said.

The median loss for romance fraud victims in their 20s was \$770. People
in their 50s reported losing twice as much. The losses reached \$3,000
for victims in their 60s and \$6,450 for those in their 70s.

``We've heard of people refinancing their homes and cashing out
retirement accounts,'' Ms. Nofziger said. ``Scammers go where the money
is, and criminals know that older adults have the majority of assets in
the United States.''

Last year, federal prosecutors brought a number of alarming romance
cases. A 76-year-old
\href{https://www.justice.gov/usao-ri/pr/five-charged-online-romance-scams-targeting-seniors}{widow
in Rhode Island} transferred more than \$660,000 to bank accounts she
thought belonged to a U.S. Army general in Afghanistan. (Posing as a
military member is another red flag, along with overseas locations.)

In Oklahoma, 10 Nigerian and United States citizens were
\href{https://www.justice.gov/opa/pr/10-men-involved-nigerian-romance-scams-indicted-money-laundering-conspiracy}{indicted}
in a fraud ring targeting multiple victims in three states. A grand jury
in Georgia indicted a
\href{https://www.justice.gov/usao-ndga/pr/georgia-man-indicted-65-million-dollar-online-romance-scam-and-business-email}{man
accused of bilking a Virginia woman}, who had a large trust, of \$6.5
million.

Ms. Floren may qualify as one of the luckier victims. As ``James
Gibson'' was leaving for Europe, he suddenly called, saying his Netflix
card had expired. ``He really wanted to be able to watch movies on the
plane,'' she recalled. ``Would I please go to a Walmart and buy him a
\$100 Netflix card?''

Gift cards, untraceable and available everywhere, have become the
currency of choice for scammers, Ms. Nofziger said. But they may also
ask victims to open a bank account and provide access to it, or to ship
iPhones.

Ms. Floren bought a gift card, reading her apparent suitor the number.
Three days later, he called again, claiming to have left a bag of
expensive tools in a cab. ``He was hysterical on the phone,'' she said.
The tools were worth \$4,000, but he'd found replacements for just
\$2,600. Would she send him the money?

She took a break, had a cup of coffee, wondered why an international
traveler had no credit card or employer willing to help. When the man
called back, she announced, ``You are scamming me,'' tossed in a few
expletives, and hung up, blocking him online and on the phone. Total
financial loss: \$100.

When she posted about the fraud on Nextdoor and Facebook, other women
said they'd been similarly defrauded.

Often, though, victims feel too humiliated to talk about what happened.
A 68-year-old social worker in the Bay Area, for instance, asked not to
be named because she still hasn't told her family about a grifter she
encountered on Our Time, a dating site for singles over 50.

He claimed to be an Air Force pilot, emailed her gushy poetry (probably
copied from romance novels, experts say), then persuaded her to wire
\$1,200 to a location in the Middle East, where he was purportedly
serving.

``Can you believe what a sucker I was?'' the social worker said. ``What
was wrong with me?'' She regrets getting angry with a friend who
questioned the relationship. With hindsight, she blames her
vulnerability on the fact that her mother was dying.

``With this fraud, especially, there is so much emotional trauma,'' Ms.
Nofziger said of its victims. ``They're embarrassed. Their hearts are
broken. They not only lost their money, but this dream they had.''

She advises friends and relatives to treat victims gently. ``Lead with
kindness and empathy,'' she said. ```How could you be so stupid?' is the
worst thing to say.''

She encourages victims to report these crimes to the
\href{https://www.ftccomplaintassistant.gov/}{F.T.C.} or the
\href{https://www.ic3.gov/default.aspx}{F.B.I.} An
\href{https://www.aarp.org/money/scams-fraud/}{AARP helpline} helps
people file complaints.

Bottom line: Anyone who appears to be pursuing romance online, while
somehow never being available to meet in person, may be a fiction
created by criminals. They're patient and skilled, and have
plausible-sounding reasons for asking for money --- but that's the
signal for victims to flee.

``Now,'' Ms. Floren said, ``as soon as anyone would ask me for 10 cents,
I'd say no.''

\textbf{\emph{{[}}\href{https://slack-redir.net/link?url=http\%3A\%2F\%2Fon.fb.me\%2F1paTQ1h\&v=3}{\emph{Like
the Science Times page on Facebook.}}} ****** \emph{\textbar{} Sign up
for the}
\textbf{\href{https://slack-redir.net/link?url=http\%3A\%2F\%2Fnyti.ms\%2F1MbHaRU\&v=3}{\emph{Science
Times newsletter.}}\emph{{]}}}

Advertisement

\protect\hyperlink{after-bottom}{Continue reading the main story}

\hypertarget{site-index}{%
\subsection{Site Index}\label{site-index}}

\hypertarget{site-information-navigation}{%
\subsection{Site Information
Navigation}\label{site-information-navigation}}

\begin{itemize}
\tightlist
\item
  \href{https://help.nytimes3xbfgragh.onion/hc/en-us/articles/115014792127-Copyright-notice}{©~2020~The
  New York Times Company}
\end{itemize}

\begin{itemize}
\tightlist
\item
  \href{https://www.nytco.com/}{NYTCo}
\item
  \href{https://help.nytimes3xbfgragh.onion/hc/en-us/articles/115015385887-Contact-Us}{Contact
  Us}
\item
  \href{https://www.nytco.com/careers/}{Work with us}
\item
  \href{https://nytmediakit.com/}{Advertise}
\item
  \href{http://www.tbrandstudio.com/}{T Brand Studio}
\item
  \href{https://www.nytimes3xbfgragh.onion/privacy/cookie-policy\#how-do-i-manage-trackers}{Your
  Ad Choices}
\item
  \href{https://www.nytimes3xbfgragh.onion/privacy}{Privacy}
\item
  \href{https://help.nytimes3xbfgragh.onion/hc/en-us/articles/115014893428-Terms-of-service}{Terms
  of Service}
\item
  \href{https://help.nytimes3xbfgragh.onion/hc/en-us/articles/115014893968-Terms-of-sale}{Terms
  of Sale}
\item
  \href{https://spiderbites.nytimes3xbfgragh.onion}{Site Map}
\item
  \href{https://help.nytimes3xbfgragh.onion/hc/en-us}{Help}
\item
  \href{https://www.nytimes3xbfgragh.onion/subscription?campaignId=37WXW}{Subscriptions}
\end{itemize}
