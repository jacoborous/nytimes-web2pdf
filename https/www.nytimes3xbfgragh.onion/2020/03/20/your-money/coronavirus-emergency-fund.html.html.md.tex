Sections

SEARCH

\protect\hyperlink{site-content}{Skip to
content}\protect\hyperlink{site-index}{Skip to site index}

\href{https://www.nytimes3xbfgragh.onion/section/your-money}{Your Money}

\href{https://myaccount.nytimes3xbfgragh.onion/auth/login?response_type=cookie\&client_id=vi}{}

\href{https://www.nytimes3xbfgragh.onion/section/todayspaper}{Today's
Paper}

\href{/section/your-money}{Your Money}\textbar{}How to Build an
Emergency Fund in the Middle of an Emergency

\url{https://nyti.ms/2WvZRZW}

\begin{itemize}
\item
\item
\item
\item
\item
\end{itemize}

\hypertarget{the-coronavirus-outbreak}{%
\subsubsection{\texorpdfstring{\href{https://www.nytimes3xbfgragh.onion/news-event/coronavirus?name=styln-coronavirus-national\&region=TOP_BANNER\&block=storyline_menu_recirc\&action=click\&pgtype=Article\&impression_id=f32f8f30-efb9-11ea-a193-7f67ec27f149\&variant=undefined}{The
Coronavirus
Outbreak}}{The Coronavirus Outbreak}}\label{the-coronavirus-outbreak}}

\begin{itemize}
\tightlist
\item
  live\href{https://www.nytimes3xbfgragh.onion/2020/09/05/world/coronavirus-covid.html?name=styln-coronavirus-national\&region=TOP_BANNER\&block=storyline_menu_recirc\&action=click\&pgtype=Article\&impression_id=f32f8f31-efb9-11ea-a193-7f67ec27f149\&variant=undefined}{Latest
  Updates}
\item
  \href{https://www.nytimes3xbfgragh.onion/interactive/2020/us/coronavirus-us-cases.html?name=styln-coronavirus-national\&region=TOP_BANNER\&block=storyline_menu_recirc\&action=click\&pgtype=Article\&impression_id=f32f8f32-efb9-11ea-a193-7f67ec27f149\&variant=undefined}{Maps
  and Cases}
\item
  \href{https://www.nytimes3xbfgragh.onion/interactive/2020/science/coronavirus-vaccine-tracker.html?name=styln-coronavirus-national\&region=TOP_BANNER\&block=storyline_menu_recirc\&action=click\&pgtype=Article\&impression_id=f32f8f33-efb9-11ea-a193-7f67ec27f149\&variant=undefined}{Vaccine
  Tracker}
\item
  \href{https://www.nytimes3xbfgragh.onion/2020/09/02/your-money/eviction-moratorium-covid.html?name=styln-coronavirus-national\&region=TOP_BANNER\&block=storyline_menu_recirc\&action=click\&pgtype=Article\&impression_id=f32f8f34-efb9-11ea-a193-7f67ec27f149\&variant=undefined}{Eviction
  Moratorium}
\item
  \href{https://www.nytimes3xbfgragh.onion/interactive/2020/09/02/magazine/food-insecurity-hunger-us.html?name=styln-coronavirus-national\&region=TOP_BANNER\&block=storyline_menu_recirc\&action=click\&pgtype=Article\&impression_id=f32f8f35-efb9-11ea-a193-7f67ec27f149\&variant=undefined}{American
  Hunger}
\end{itemize}

Advertisement

\protect\hyperlink{after-top}{Continue reading the main story}

Supported by

\protect\hyperlink{after-sponsor}{Continue reading the main story}

your money adviser

\hypertarget{how-to-build-an-emergency-fund-in-the-middle-of-an-emergency}{%
\section{How to Build an Emergency Fund in the Middle of an
Emergency}\label{how-to-build-an-emergency-fund-in-the-middle-of-an-emergency}}

Every little bit can help.

\includegraphics{https://static01.graylady3jvrrxbe.onion/images/2020/03/20/business/20Adviser-illo/20Adviser-illo-articleLarge.jpg?quality=75\&auto=webp\&disable=upscale}

By \href{https://www.nytimes3xbfgragh.onion/by/ann-carrns}{Ann Carrns}

\begin{itemize}
\item
  March 20, 2020
\item
  \begin{itemize}
  \item
  \item
  \item
  \item
  \item
  \end{itemize}
\end{itemize}

The coronavirus is
\href{https://www.nytimes3xbfgragh.onion/2020/03/16/business/economy/coronavirus-us-economy-shutdown.html}{rapidly
slowing} the U.S. economy and disrupting jobs. If you don't have a
rainy-day fund already, it's time to set aside whatever cash you can.

Most people think of
\href{https://www.nytimes3xbfgragh.onion/2019/10/25/your-money/emergency-savings.html}{an
emergency fund} as something saved gradually, over time. But the crisis
is here now for many people --- or soon will be, in the coming weeks ---
so a different approach is needed.

``It's hard to save a lot of money quickly on a modest income,'' said
Stephen Brobeck, senior fellow with the Consumer Federation of America.
And compared with the last downturn, he said, the current economy has
many more people in
\href{https://www.nytimes3xbfgragh.onion/2020/03/18/technology/gig-economy-pandemic.html}{``gig''
or freelance jobs}, who generally aren't eligible for unemployment
benefits and whose fluctuating income makes it hard to save.

Cash assistance may be coming from the federal government as part of its
response to the virus outbreak, but details are uncertain. So making a
plan on your own is wise.

``The answer can't be to do nothing,'' said John Thompson, chief program
officer of the Financial Health Network, a nonprofit focused on
financial innovation.

One reason for hope: Even small cash cushions can help people stave off
disaster.
\href{https://www.urban.org/research/publication/thriving-residents-thriving-cities-family-financial-security-matters-cities}{As
little as \$250} can significantly reduce the risk that a family will
miss paying a utility bill or be evicted, research suggests.

``Each extra dollar saved'' reduces the likelihood of having to skip
bill payments, said Mariel Beasley, a co-founder of Common Cents Lab, a
financial research group at Duke University.

\hypertarget{latest-updates-the-coronavirus-outbreak}{%
\section{\texorpdfstring{\href{https://www.nytimes3xbfgragh.onion/2020/09/04/world/covid-19-coronavirus.html?action=click\&pgtype=Article\&state=default\&region=MAIN_CONTENT_1\&context=storylines_live_updates}{Latest
Updates: The Coronavirus
Outbreak}}{Latest Updates: The Coronavirus Outbreak}}\label{latest-updates-the-coronavirus-outbreak}}

Updated 2020-09-05T12:05:40.998Z

\begin{itemize}
\tightlist
\item
  \href{https://www.nytimes3xbfgragh.onion/2020/09/04/world/covid-19-coronavirus.html?action=click\&pgtype=Article\&state=default\&region=MAIN_CONTENT_1\&context=storylines_live_updates\#link-1654f6ad}{Research
  connects vaping to a higher chance of catching the virus --- and
  suffering its worst effects.}
\item
  \href{https://www.nytimes3xbfgragh.onion/2020/09/04/world/covid-19-coronavirus.html?action=click\&pgtype=Article\&state=default\&region=MAIN_CONTENT_1\&context=storylines_live_updates\#link-52e4198a}{Another
  college football game won't be played as planned.}
\item
  \href{https://www.nytimes3xbfgragh.onion/2020/09/04/world/covid-19-coronavirus.html?action=click\&pgtype=Article\&state=default\&region=MAIN_CONTENT_1\&context=storylines_live_updates\#link-181cef0}{Pharmaceutical
  companies plan a joint pledge on safety standards as they move
  vaccines to the marketplace.}
\end{itemize}

\href{https://www.nytimes3xbfgragh.onion/2020/09/04/world/covid-19-coronavirus.html?action=click\&pgtype=Article\&state=default\&region=MAIN_CONTENT_1\&context=storylines_live_updates}{See
more updates}

More live coverage:
\href{https://www.nytimes3xbfgragh.onion/live/2020/09/04/business/stock-market-today-coronavirus?action=click\&pgtype=Article\&state=default\&region=MAIN_CONTENT_1\&context=storylines_live_updates}{Markets}

Americans' lack of emergency savings is a longtime worry, even during
the strong economy of the past few years. Numerous surveys by the
\href{https://www.federalreserve.gov/publications/2019-economic-well-being-of-us-households-in-2018-dealing-with-unexpected-expenses.htm}{Federal
Reserve} have found that many households would struggle to handle an
unexpected \$400 expense.

\hypertarget{so-what-to-do}{%
\subsection{So what to do?}\label{so-what-to-do}}

First, try to get a handle on your income, Mr. Thompson said. Employers
often use computer systems to schedule shifts weeks in advance, so try
to find out if your hours will be cut so you can estimate how much of a
shortage you're facing.

Next, take stock of possible sources of cash and credit. It's not
advisable to open new credit card accounts, but knowing the credit limit
on each card already in your wallet can help you get an idea of what you
can draw on if needed, Mr. Thompson said.

If you are expecting any sort of lump sum --- whether a bonus or a
commission, or an income tax refund --- set aside as much of it as you
can. Many Americans are now receiving tax refunds, and the amounts can
be substantial, in part because of the earned-income tax credit, which
particularly benefits families with children.

Among families getting a refund, the average is more than \$3,000, or
the equivalent of
\href{https://institute.jpmorganchase.com/institute/research/household-income-spending/report-tax-refunds.htm\#finding-2}{nearly
six weeks} of take-home pay, according to a study of millions of
customer accounts by the JPMorgan Chase Institute, the research arm of
the big bank. (The study looked at data from 2015, 2016 and 2017.
According to I.R.S. statistics, the average refund as of March 5 was
\$3,012).

That could help provide a financial lifeline for the difficult weeks
ahead --- but it isn't a panacea, Mr. Thompson said.

That's because many people have earmarked their refunds for specific
expenditures, like paying down credit card debt or buying household
items. ``For many people, the money is already spent,'' he said.

Still, families getting tax refunds had, on average, more than a quarter
of their refunds remaining six months after receiving them, the Chase
research found. ``A few hundred dollars can make a substantial
difference,'' Mr. Thompson said.

Next, scrutinize spending, and cut where you can. It may feel harsh, but
belt tightening is the idea. Can you postpone a (no doubt much
anticipated) spring trip? Are there subscriptions you can do without
temporarily? (Many
\href{https://www.adweek.com/digital/major-publishers-take-down-paywalls-for-coronavirus-coverage/}{publications}
are offering online coronavirus coverage free of charge.) Can you switch
to a less expensive cellphone plan for a few months?

``Take a really aggressive approach,'' Ms. Beasley said, and direct all
the savings to your emergency fund.

Depending on your circumstances, you may consider temporarily reducing
contributions to your retirement account and redirecting the money to an
emergency fund. It's common for people to contribute to workplace
accounts like 401(k) plans yet lack emergency savings, Ms. Beasley said.
That's because many employers automatically enroll workers in retirement
contributions from their paycheck.

\href{https://www.nytimes3xbfgragh.onion/news-event/coronavirus?action=click\&pgtype=Article\&state=default\&region=MAIN_CONTENT_3\&context=storylines_faq}{}

\hypertarget{the-coronavirus-outbreak-}{%
\subsubsection{The Coronavirus Outbreak
›}\label{the-coronavirus-outbreak-}}

\hypertarget{frequently-asked-questions}{%
\paragraph{Frequently Asked
Questions}\label{frequently-asked-questions}}

Updated September 4, 2020

\begin{itemize}
\item ~
  \hypertarget{what-are-the-symptoms-of-coronavirus}{%
  \paragraph{What are the symptoms of
  coronavirus?}\label{what-are-the-symptoms-of-coronavirus}}

  \begin{itemize}
  \tightlist
  \item
    In the beginning, the coronavirus
    \href{https://www.nytimes3xbfgragh.onion/article/coronavirus-facts-history.html?action=click\&pgtype=Article\&state=default\&region=MAIN_CONTENT_3\&context=storylines_faq\#link-6817bab5}{seemed
    like it was primarily a respiratory illness}~--- many patients had
    fever and chills, were weak and tired, and coughed a lot, though
    some people don't show many symptoms at all. Those who seemed
    sickest had pneumonia or acute respiratory distress syndrome and
    received supplemental oxygen. By now, doctors have identified many
    more symptoms and syndromes. In April,
    \href{https://www.nytimes3xbfgragh.onion/2020/04/27/health/coronavirus-symptoms-cdc.html?action=click\&pgtype=Article\&state=default\&region=MAIN_CONTENT_3\&context=storylines_faq}{the
    C.D.C. added to the list of early signs}~sore throat, fever, chills
    and muscle aches. Gastrointestinal upset, such as diarrhea and
    nausea, has also been observed. Another telltale sign of infection
    may be a sudden, profound diminution of one's
    \href{https://www.nytimes3xbfgragh.onion/2020/03/22/health/coronavirus-symptoms-smell-taste.html?action=click\&pgtype=Article\&state=default\&region=MAIN_CONTENT_3\&context=storylines_faq}{sense
    of smell and taste.}~Teenagers and young adults in some cases have
    developed painful red and purple lesions on their fingers and toes
    --- nicknamed ``Covid toe'' --- but few other serious symptoms.
  \end{itemize}
\item ~
  \hypertarget{why-is-it-safer-to-spend-time-together-outside}{%
  \paragraph{Why is it safer to spend time together
  outside?}\label{why-is-it-safer-to-spend-time-together-outside}}

  \begin{itemize}
  \tightlist
  \item
    \href{https://www.nytimes3xbfgragh.onion/2020/05/15/us/coronavirus-what-to-do-outside.html?action=click\&pgtype=Article\&state=default\&region=MAIN_CONTENT_3\&context=storylines_faq}{Outdoor
    gatherings}~lower risk because wind disperses viral droplets, and
    sunlight can kill some of the virus. Open spaces prevent the virus
    from building up in concentrated amounts and being inhaled, which
    can happen when infected people exhale in a confined space for long
    stretches of time, said Dr. Julian W. Tang, a virologist at the
    University of Leicester.
  \end{itemize}
\item ~
  \hypertarget{why-does-standing-six-feet-away-from-others-help}{%
  \paragraph{Why does standing six feet away from others
  help?}\label{why-does-standing-six-feet-away-from-others-help}}

  \begin{itemize}
  \tightlist
  \item
    The coronavirus spreads primarily through droplets from your mouth
    and nose, especially when you cough or sneeze. The C.D.C., one of
    the organizations using that measure,
    \href{https://www.nytimes3xbfgragh.onion/2020/04/14/health/coronavirus-six-feet.html?action=click\&pgtype=Article\&state=default\&region=MAIN_CONTENT_3\&context=storylines_faq}{bases
    its recommendation of six feet}~on the idea that most large droplets
    that people expel when they cough or sneeze will fall to the ground
    within six feet. But six feet has never been a magic number that
    guarantees complete protection. Sneezes, for instance, can launch
    droplets a lot farther than six feet,
    \href{https://jamanetwork.com/journals/jama/fullarticle/2763852}{according
    to a recent study}. It's a rule of thumb: You should be safest
    standing six feet apart outside, especially when it's windy. But
    keep a mask on at all times, even when you think you're far enough
    apart.
  \end{itemize}
\item ~
  \hypertarget{i-have-antibodies-am-i-now-immune}{%
  \paragraph{I have antibodies. Am I now
  immune?}\label{i-have-antibodies-am-i-now-immune}}

  \begin{itemize}
  \tightlist
  \item
    As of right
    now,\href{https://www.nytimes3xbfgragh.onion/2020/07/22/health/covid-antibodies-herd-immunity.html?action=click\&pgtype=Article\&state=default\&region=MAIN_CONTENT_3\&context=storylines_faq}{~that
    seems likely, for at least several months.}~There have been
    frightening accounts of people suffering what seems to be a second
    bout of Covid-19. But experts say these patients may have a
    drawn-out course of infection, with the virus taking a slow toll
    weeks to months after initial exposure.~People infected with the
    coronavirus typically
    \href{https://www.nature.com/articles/s41586-020-2456-9}{produce}~immune
    molecules called antibodies, which are
    \href{https://www.nytimes3xbfgragh.onion/2020/05/07/health/coronavirus-antibody-prevalence.html?action=click\&pgtype=Article\&state=default\&region=MAIN_CONTENT_3\&context=storylines_faq}{protective
    proteins made in response to an
    infection}\href{https://www.nytimes3xbfgragh.onion/2020/05/07/health/coronavirus-antibody-prevalence.html?action=click\&pgtype=Article\&state=default\&region=MAIN_CONTENT_3\&context=storylines_faq}{.
    These antibodies may}~last in the body
    \href{https://www.nature.com/articles/s41591-020-0965-6}{only two to
    three months}, which may seem worrisome, but that's~perfectly normal
    after an acute infection subsides, said Dr. Michael Mina, an
    immunologist at Harvard University. It may be possible to get the
    coronavirus again, but it's highly unlikely that it would be
    possible in a short window of time from initial infection or make
    people sicker the second time.
  \end{itemize}
\item ~
  \hypertarget{what-are-my-rights-if-i-am-worried-about-going-back-to-work}{%
  \paragraph{What are my rights if I am worried about going back to
  work?}\label{what-are-my-rights-if-i-am-worried-about-going-back-to-work}}

  \begin{itemize}
  \tightlist
  \item
    Employers have to provide
    \href{https://www.osha.gov/SLTC/covid-19/standards.html}{a safe
    workplace}~with policies that protect everyone equally.
    \href{https://www.nytimes3xbfgragh.onion/article/coronavirus-money-unemployment.html?action=click\&pgtype=Article\&state=default\&region=MAIN_CONTENT_3\&context=storylines_faq}{And
    if one of your co-workers tests positive for the coronavirus, the
    C.D.C.}~has said that
    \href{https://www.cdc.gov/coronavirus/2019-ncov/community/guidance-business-response.html}{employers
    should tell their employees}~-\/- without giving you the sick
    employee's name -\/- that they may have been exposed to the virus.
  \end{itemize}
\end{itemize}

In general, it's wise to keep contributing to retirement plans
regularly, because your money is buying more shares when prices are low.
But if your situation is dire, a cut is better than stopping entirely.
Ms. Beasley said one option might be to suspend contributions above any
match from your employer; that way, you're still saving for your
long-term retirement. Just make sure --- set a calendar reminder on your
phone, perhaps --- to resume contributions once the crisis passes.

While it might not be something you have considered in the past, she
said, now is a good time to identify local food banks, or investigate
how to apply for government food
\href{https://www.benefits.gov/benefit/361}{benefits}, like the
Supplemental Nutrition Assistance Program (SNAP) and the Supplemental
Nutrition Program for Women, Infants and Children (WIC).

If you own a home, you could consider opening a home equity line of
credit as a financial backstop. The loans let you draw on your home
equity --- the difference between the value of your home and any
mortgage you already have. At the end of 2019, nearly 45 million
homeowners with mortgages had ``tappable'' home equity, \$119,000 on
average, according to the research firm
\href{https://cdn.blackknightinc.com/wp-content/uploads/2020/02/BKI_MM_Jan2020_Report.pdf}{Black
Knight}.

Lines of credit generally carry lower interest rates than credit cards.
However, the loans are secured by your home, which means you risk
foreclosure if you miss payments. For that reason, Ms. Beasley said,
people should be cautious about using home equity.

Once you have a savings cushion, don't feel bad about using the money if
you need it --- that's what it's for. An emergency fund is different
from retirement savings, which are meant to grow over a long period.
Rainy-day accounts are meant to be drawn down and replenished so you can
use them again.

``You're not saving to create an account you never touch,'' Mr. Thompson
said.

Here are some questions and answers about emergency savings:

\textbf{Do any employers offer emergency savings programs?}

Some employers have begun offering workers the option to save for
emergencies via payroll deduction, but the programs aren't yet
widespread. Some employers offer one-time ``hardship'' grants to workers
who face an unexpected financial setback; amounts vary, but a maximum of
\$2,500 is common. Often, the programs are geared toward helping people
during natural disasters, like hurricanes. Now many are adapting to help
workers affected by the coronavirus, said Doug Stockham, president of
the \href{https://emergencyassistancefdn.org/}{Emergency Assistance
Foundation}, which manages hardship funds for employers.

Workers who can't go to work because of the virus --- perhaps because
they have to stay home with children or a sick member of their family
--- should check with their human resources department to see if grants
are available and how to apply, Mr. Stockham said.

\textbf{Why save when banks are paying low interest rates?}

The main point of an emergency cushion isn't to earn big returns; it's
to have a reserve to cover necessary costs, until things --- hopefully
--- turn around. Some credit unions, including Affinity in Basking
Ridge, N.J., and Canvas in Englewood, Colo., offer ``reverse tier''
savings accounts, which pay higher interest rates on low balances to
encourage saving. (Most banks offer higher rates on bigger balances, to
attract deposits.) At Affinity, for example, you can earn 2 percent on
balances below \$2,500.

\textbf{Can I take a loan or hardship withdrawal from my 401(k)?}

Many companies allow
\href{https://www.irs.gov/retirement-plans/hardships-early-withdrawals-and-loans}{hardship
withdrawals} or loans from 401(k) retirement plans, but doing so puts
your long-term retirement savings at risk. Hardship withdrawals don't
have to be paid back but are taxable as income and may result in
penalties. Loans aren't taxable but must be repaid, and they can be
risky because if you leave your employer you generally have to repay the
loan quickly, said J. Michael Collins, director of the Center for
Financial Security at the University of Wisconsin-Madison.

Advertisement

\protect\hyperlink{after-bottom}{Continue reading the main story}

\hypertarget{site-index}{%
\subsection{Site Index}\label{site-index}}

\hypertarget{site-information-navigation}{%
\subsection{Site Information
Navigation}\label{site-information-navigation}}

\begin{itemize}
\tightlist
\item
  \href{https://help.nytimes3xbfgragh.onion/hc/en-us/articles/115014792127-Copyright-notice}{©~2020~The
  New York Times Company}
\end{itemize}

\begin{itemize}
\tightlist
\item
  \href{https://www.nytco.com/}{NYTCo}
\item
  \href{https://help.nytimes3xbfgragh.onion/hc/en-us/articles/115015385887-Contact-Us}{Contact
  Us}
\item
  \href{https://www.nytco.com/careers/}{Work with us}
\item
  \href{https://nytmediakit.com/}{Advertise}
\item
  \href{http://www.tbrandstudio.com/}{T Brand Studio}
\item
  \href{https://www.nytimes3xbfgragh.onion/privacy/cookie-policy\#how-do-i-manage-trackers}{Your
  Ad Choices}
\item
  \href{https://www.nytimes3xbfgragh.onion/privacy}{Privacy}
\item
  \href{https://help.nytimes3xbfgragh.onion/hc/en-us/articles/115014893428-Terms-of-service}{Terms
  of Service}
\item
  \href{https://help.nytimes3xbfgragh.onion/hc/en-us/articles/115014893968-Terms-of-sale}{Terms
  of Sale}
\item
  \href{https://spiderbites.nytimes3xbfgragh.onion}{Site Map}
\item
  \href{https://help.nytimes3xbfgragh.onion/hc/en-us}{Help}
\item
  \href{https://www.nytimes3xbfgragh.onion/subscription?campaignId=37WXW}{Subscriptions}
\end{itemize}
