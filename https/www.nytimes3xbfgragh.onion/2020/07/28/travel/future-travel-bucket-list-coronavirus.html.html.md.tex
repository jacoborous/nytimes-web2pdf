Sections

SEARCH

\protect\hyperlink{site-content}{Skip to
content}\protect\hyperlink{site-index}{Skip to site index}

\href{https://www.nytimes3xbfgragh.onion/section/travel}{Travel}

\href{https://myaccount.nytimes3xbfgragh.onion/auth/login?response_type=cookie\&client_id=vi}{}

\href{https://www.nytimes3xbfgragh.onion/section/todayspaper}{Today's
Paper}

\href{/section/travel}{Travel}\textbar{}Longer, Slower, Farther:
Savoring the Prospects of Future Travels

\begin{itemize}
\item
\item
\item
\item
\item
\end{itemize}

\href{https://www.nytimes3xbfgragh.onion/news-event/coronavirus?action=click\&pgtype=Article\&state=default\&region=TOP_BANNER\&context=storylines_menu}{The
Coronavirus Outbreak}

\begin{itemize}
\tightlist
\item
  live\href{https://www.nytimes3xbfgragh.onion/2020/08/04/world/coronavirus-cases.html?action=click\&pgtype=Article\&state=default\&region=TOP_BANNER\&context=storylines_menu}{Latest
  Updates}
\item
  \href{https://www.nytimes3xbfgragh.onion/interactive/2020/us/coronavirus-us-cases.html?action=click\&pgtype=Article\&state=default\&region=TOP_BANNER\&context=storylines_menu}{Maps
  and Cases}
\item
  \href{https://www.nytimes3xbfgragh.onion/interactive/2020/science/coronavirus-vaccine-tracker.html?action=click\&pgtype=Article\&state=default\&region=TOP_BANNER\&context=storylines_menu}{Vaccine
  Tracker}
\item
  \href{https://www.nytimes3xbfgragh.onion/2020/08/02/us/covid-college-reopening.html?action=click\&pgtype=Article\&state=default\&region=TOP_BANNER\&context=storylines_menu}{College
  Reopening}
\item
  \href{https://www.nytimes3xbfgragh.onion/live/2020/08/04/business/stock-market-today-coronavirus?action=click\&pgtype=Article\&state=default\&region=TOP_BANNER\&context=storylines_menu}{Economy}
\end{itemize}

Advertisement

\protect\hyperlink{after-top}{Continue reading the main story}

Supported by

\protect\hyperlink{after-sponsor}{Continue reading the main story}

\hypertarget{longer-slower-farther-savoring-the-prospects-of-future-travels}{%
\section{Longer, Slower, Farther: Savoring the Prospects of Future
Travels}\label{longer-slower-farther-savoring-the-prospects-of-future-travels}}

In the travel lull induced by the pandemic, many people are planning
ambitious, once-in-a-lifetime trips. Optimists are targeting 2021. For
others, their next big trip will be in 2022.

\includegraphics{https://static01.graylady3jvrrxbe.onion/images/2020/07/24/travel/00Future-trips-virus01/merlin_169342779_4855ca5b-bcbf-4bb0-a490-d7ef8ef6fa50-articleLarge.jpg?quality=75\&auto=webp\&disable=upscale}

By Elaine Glusac

\begin{itemize}
\item
  July 28, 2020
\item
  \begin{itemize}
  \item
  \item
  \item
  \item
  \item
  \end{itemize}
\end{itemize}

Beth Warren, a middle school history teacher in Lookout Mountain, Ga.,
had been looking forward to a much-anticipated trip this summer to
Egypt, a country she vowed to show her husband and friends after her
first visit several years ago. She was deep into organizing the trip
with \href{https://www.highendjourneys.com/}{High End Journeys} when the
pandemic struck, and has since shifted the visit to summer 2022, in part
to make sure the new Grand Egyptian Museum in Giza is open.

``2022 sounds really far away,'' she said. ``But once I saw Egypt, I
couldn't get enough of it.''

People have always planned big trips months or even a year ahead of
time, but now many are extending that timeline even further. In the
travel stasis induced by the pandemic, future travelers have taken to
tackling their bucket lists with big trips that are more distant and
longer than usual --- and planned further in advance. Optimists are
targeting 2021. For others, their next big trip will be in 2022.

Before the pandemic, according to the American Society of Travel
Advisors, most travelers booked trips six months or more in advance, on
average, and longer for elaborate honeymoons or very special events like
the solar eclipse passing over South America in December. Some travel
companies say longer term bookings have recently rebounded. For
instance, \href{https://www.redsavannah.com/}{Red Savannah}, a British
luxury travel agency that organizes custom trips, says it is up 160
percent over bookings this time last year.

These days, even spontaneous types have more time to think about where
they want to go and put a plan in place.

``I'm trying to go big with my trips,'' said Rayme Gorniak of Chicago,
who is currently laid off from his work managing fitness studio
franchises.

Anything short and normally easy to plan might bring disappointment as
the pandemic continues, he reasoned, but a far-horizon destination ---
he's considering Jordan for June 2021 --- offers hope. The trip also
represents a personal conquest for Mr. Gorniak, who is gay and worried
about the persecution of L.G.B.T. people in some Muslim countries.

``Jordan's been on my radar because of the rich history, and off it
because of the potential risk I would have,'' he said. ``But I've been
doing research on Amman and seeing, as strict religious standards go,
it's a little bit more lax on tradition,'' he said.

For Lori Goldenthal of Wellesley, Mass., changing plans meant changing
the destination. She had originally planned a trip in and around Vietnam
for her husband's upcoming 60th birthday. But after the pandemic hit,
she worked with the agency
\href{https://extraordinaryjourneys.com/}{Extraordinary Journeys} to
book a two-week trip to Namibia for 2021.

``Namibia was on my bucket list and it seemed like a better idea than
going to all these big cities in Asia,'' she said.

``I believe we will go, but who knows,'' she added, noting generous
cancellation policies that made her more comfortable booking the trip.
``Having something to look forward to is fantastic.''

\includegraphics{https://static01.graylady3jvrrxbe.onion/images/2020/07/24/travel/00Future-trips-virus02/merlin_160994223_19dacdde-49b7-410e-8e7f-0e719bc9cc17-articleLarge.jpg?quality=75\&auto=webp\&disable=upscale}

Other forward-looking travelers are simply picking up a year later.

After months of reading about the climate and culture of Greenland, Jill
Hrubecky, a structural engineer based in Brooklyn, was excited for a
cruise she had planned there in August with her mother and an aunt and
uncle. Working with their agency, Huckleberry Travel, they rebooked the
cruise for summer 2021 only after learning that the cancellation policy
is flexible.

``I will not make any nonrefundable, permanent plans for the next couple
of years,'' she said. ``But I'm an optimist. Half the fun of traveling
is planning and getting excited.''

\hypertarget{latest-updates-global-coronavirus-outbreak}{%
\section{\texorpdfstring{\href{https://www.nytimes3xbfgragh.onion/2020/08/04/world/coronavirus-cases.html?action=click\&pgtype=Article\&state=default\&region=MAIN_CONTENT_1\&context=storylines_live_updates}{Latest
Updates: Global Coronavirus
Outbreak}}{Latest Updates: Global Coronavirus Outbreak}}\label{latest-updates-global-coronavirus-outbreak}}

Updated 2020-08-05T07:58:24.076Z

\begin{itemize}
\tightlist
\item
  \href{https://www.nytimes3xbfgragh.onion/2020/08/04/world/coronavirus-cases.html?action=click\&pgtype=Article\&state=default\&region=MAIN_CONTENT_1\&context=storylines_live_updates\#link-762df92}{As
  talks drag on, McConnell signals openness to jobless aid extension,
  and negotiators agree on a deadline.}
\item
  \href{https://www.nytimes3xbfgragh.onion/2020/08/04/world/coronavirus-cases.html?action=click\&pgtype=Article\&state=default\&region=MAIN_CONTENT_1\&context=storylines_live_updates\#link-1228a480}{Novavax
  sees encouraging results from two studies of its experimental
  vaccine.}
\item
  \href{https://www.nytimes3xbfgragh.onion/2020/08/04/world/coronavirus-cases.html?action=click\&pgtype=Article\&state=default\&region=MAIN_CONTENT_1\&context=storylines_live_updates\#link-794484ed}{Mississippians
  must now wear masks in public, governor says.}
\end{itemize}

\href{https://www.nytimes3xbfgragh.onion/2020/08/04/world/coronavirus-cases.html?action=click\&pgtype=Article\&state=default\&region=MAIN_CONTENT_1\&context=storylines_live_updates}{See
more updates}

More live coverage:
\href{https://www.nytimes3xbfgragh.onion/live/2020/08/04/business/stock-market-today-coronavirus?action=click\&pgtype=Article\&state=default\&region=MAIN_CONTENT_1\&context=storylines_live_updates}{Markets}

There are psychological benefits to planning activities in the future,
especially travel, according to Shevaun Neupert, a professor of
psychology at North Carolina State University. Future-oriented thinking
is equated with proactive coping, a means of reducing stress through
detailed planning, such as learning which flights to book to avoid
layovers, and gathering the resources --- including time and money ---
to make it happen.

``Being able to think about and imagine something positive in the future
has benefits in the present,'' she said.

The pandemic, too, may have shown travelers that what they thought they
could always do --- namely, see the world --- isn't such a certainty.

``Maybe they thought it would always be available, which was previously
true. Now we've experienced restrictions and realize, oh, I need to make
this happen,'' she added.

Advance planning is also a practical way to turn vague desires into
concrete plans. The travel adviser network Virtuoso offers a program
called \href{https://www.virtuoso.com/wanderlist/\#/sign-in}{Virtuoso
Wanderlist}, an online survey that friends or family seeking to travel
together take individually. (Since the pandemic, Virtuoso has made the
online planning tool free.)

The program asks where they want to go, their interests and the kinds of
activities they prefer. It then compares the results of those surveyed
to identify mutual preferences and priorities that a travel adviser will
analyze and, in consultation with the clients, come up with a five-year
plan of tackling the bucket list.

Jim Bendt, the managing director of Virtuoso Wanderlist, equates travel
planning with financial planning in the sense that both seek to maximize
precious resources. In the case of travel, the currency is time.

``It takes away the stress,'' said Karen Walkowski, a health care
manager in Eden Prairie, Minn., who took the Wanderlist survey with her
husband. ``It turns a bucket list into a plan.''

\href{https://www.nytimes3xbfgragh.onion/news-event/coronavirus?action=click\&pgtype=Article\&state=default\&region=MAIN_CONTENT_3\&context=storylines_faq}{}

\hypertarget{the-coronavirus-outbreak-}{%
\subsubsection{The Coronavirus Outbreak
›}\label{the-coronavirus-outbreak-}}

\hypertarget{frequently-asked-questions}{%
\paragraph{Frequently Asked
Questions}\label{frequently-asked-questions}}

Updated August 4, 2020

\begin{itemize}
\item ~
  \hypertarget{i-have-antibodies-am-i-now-immune}{%
  \paragraph{I have antibodies. Am I now
  immune?}\label{i-have-antibodies-am-i-now-immune}}

  \begin{itemize}
  \tightlist
  \item
    As of right
    now,\href{https://www.nytimes3xbfgragh.onion/2020/07/22/health/covid-antibodies-herd-immunity.html?action=click\&pgtype=Article\&state=default\&region=MAIN_CONTENT_3\&context=storylines_faq}{that
    seems likely, for at least several months.} There have been
    frightening accounts of people suffering what seems to be a second
    bout of Covid-19. But experts say these patients may have a
    drawn-out course of infection, with the virus taking a slow toll
    weeks to months after initial exposure. People infected with the
    coronavirus typically
    \href{https://www.nature.com/articles/s41586-020-2456-9}{produce}
    immune molecules called antibodies, which are
    \href{https://www.nytimes3xbfgragh.onion/2020/05/07/health/coronavirus-antibody-prevalence.html?action=click\&pgtype=Article\&state=default\&region=MAIN_CONTENT_3\&context=storylines_faq}{protective
    proteins made in response to an
    infection}\href{https://www.nytimes3xbfgragh.onion/2020/05/07/health/coronavirus-antibody-prevalence.html?action=click\&pgtype=Article\&state=default\&region=MAIN_CONTENT_3\&context=storylines_faq}{.
    These antibodies may} last in the body
    \href{https://www.nature.com/articles/s41591-020-0965-6}{only two to
    three months}, which may seem worrisome, but that's perfectly normal
    after an acute infection subsides, said Dr. Michael Mina, an
    immunologist at Harvard University. It may be possible to get the
    coronavirus again, but it's highly unlikely that it would be
    possible in a short window of time from initial infection or make
    people sicker the second time.
  \end{itemize}
\item ~
  \hypertarget{im-a-small-business-owner-can-i-get-relief}{%
  \paragraph{I'm a small-business owner. Can I get
  relief?}\label{im-a-small-business-owner-can-i-get-relief}}

  \begin{itemize}
  \tightlist
  \item
    The
    \href{https://www.nytimes3xbfgragh.onion/article/small-business-loans-stimulus-grants-freelancers-coronavirus.html?action=click\&pgtype=Article\&state=default\&region=MAIN_CONTENT_3\&context=storylines_faq}{stimulus
    bills enacted in March} offer help for the millions of American
    small businesses. Those eligible for aid are businesses and
    nonprofit organizations with fewer than 500 workers, including sole
    proprietorships, independent contractors and freelancers. Some
    larger companies in some industries are also eligible. The help
    being offered, which is being managed by the Small Business
    Administration, includes the Paycheck Protection Program and the
    Economic Injury Disaster Loan program. But lots of folks have
    \href{https://www.nytimes3xbfgragh.onion/interactive/2020/05/07/business/small-business-loans-coronavirus.html?action=click\&pgtype=Article\&state=default\&region=MAIN_CONTENT_3\&context=storylines_faq}{not
    yet seen payouts.} Even those who have received help are confused:
    The rules are draconian, and some are stuck sitting on
    \href{https://www.nytimes3xbfgragh.onion/2020/05/02/business/economy/loans-coronavirus-small-business.html?action=click\&pgtype=Article\&state=default\&region=MAIN_CONTENT_3\&context=storylines_faq}{money
    they don't know how to use.} Many small-business owners are getting
    less than they expected or
    \href{https://www.nytimes3xbfgragh.onion/2020/06/10/business/Small-business-loans-ppp.html?action=click\&pgtype=Article\&state=default\&region=MAIN_CONTENT_3\&context=storylines_faq}{not
    hearing anything at all.}
  \end{itemize}
\item ~
  \hypertarget{what-are-my-rights-if-i-am-worried-about-going-back-to-work}{%
  \paragraph{What are my rights if I am worried about going back to
  work?}\label{what-are-my-rights-if-i-am-worried-about-going-back-to-work}}

  \begin{itemize}
  \tightlist
  \item
    Employers have to provide
    \href{https://www.osha.gov/SLTC/covid-19/standards.html}{a safe
    workplace} with policies that protect everyone equally.
    \href{https://www.nytimes3xbfgragh.onion/article/coronavirus-money-unemployment.html?action=click\&pgtype=Article\&state=default\&region=MAIN_CONTENT_3\&context=storylines_faq}{And
    if one of your co-workers tests positive for the coronavirus, the
    C.D.C.} has said that
    \href{https://www.cdc.gov/coronavirus/2019-ncov/community/guidance-business-response.html}{employers
    should tell their employees} -\/- without giving you the sick
    employee's name -\/- that they may have been exposed to the virus.
  \end{itemize}
\item ~
  \hypertarget{should-i-refinance-my-mortgage}{%
  \paragraph{Should I refinance my
  mortgage?}\label{should-i-refinance-my-mortgage}}

  \begin{itemize}
  \tightlist
  \item
    \href{https://www.nytimes3xbfgragh.onion/article/coronavirus-money-unemployment.html?action=click\&pgtype=Article\&state=default\&region=MAIN_CONTENT_3\&context=storylines_faq}{It
    could be a good idea,} because mortgage rates have
    \href{https://www.nytimes3xbfgragh.onion/2020/07/16/business/mortgage-rates-below-3-percent.html?action=click\&pgtype=Article\&state=default\&region=MAIN_CONTENT_3\&context=storylines_faq}{never
    been lower.} Refinancing requests have pushed mortgage applications
    to some of the highest levels since 2008, so be prepared to get in
    line. But defaults are also up, so if you're thinking about buying a
    home, be aware that some lenders have tightened their standards.
  \end{itemize}
\item ~
  \hypertarget{what-is-school-going-to-look-like-in-september}{%
  \paragraph{What is school going to look like in
  September?}\label{what-is-school-going-to-look-like-in-september}}

  \begin{itemize}
  \tightlist
  \item
    It is unlikely that many schools will return to a normal schedule
    this fall, requiring the grind of
    \href{https://www.nytimes3xbfgragh.onion/2020/06/05/us/coronavirus-education-lost-learning.html?action=click\&pgtype=Article\&state=default\&region=MAIN_CONTENT_3\&context=storylines_faq}{online
    learning},
    \href{https://www.nytimes3xbfgragh.onion/2020/05/29/us/coronavirus-child-care-centers.html?action=click\&pgtype=Article\&state=default\&region=MAIN_CONTENT_3\&context=storylines_faq}{makeshift
    child care} and
    \href{https://www.nytimes3xbfgragh.onion/2020/06/03/business/economy/coronavirus-working-women.html?action=click\&pgtype=Article\&state=default\&region=MAIN_CONTENT_3\&context=storylines_faq}{stunted
    workdays} to continue. California's two largest public school
    districts --- Los Angeles and San Diego --- said on July 13, that
    \href{https://www.nytimes3xbfgragh.onion/2020/07/13/us/lausd-san-diego-school-reopening.html?action=click\&pgtype=Article\&state=default\&region=MAIN_CONTENT_3\&context=storylines_faq}{instruction
    will be remote-only in the fall}, citing concerns that surging
    coronavirus infections in their areas pose too dire a risk for
    students and teachers. Together, the two districts enroll some
    825,000 students. They are the largest in the country so far to
    abandon plans for even a partial physical return to classrooms when
    they reopen in August. For other districts, the solution won't be an
    all-or-nothing approach.
    \href{https://bioethics.jhu.edu/research-and-outreach/projects/eschool-initiative/school-policy-tracker/}{Many
    systems}, including the nation's largest, New York City, are
    devising
    \href{https://www.nytimes3xbfgragh.onion/2020/06/26/us/coronavirus-schools-reopen-fall.html?action=click\&pgtype=Article\&state=default\&region=MAIN_CONTENT_3\&context=storylines_faq}{hybrid
    plans} that involve spending some days in classrooms and other days
    online. There's no national policy on this yet, so check with your
    municipal school system regularly to see what is happening in your
    community.
  \end{itemize}
\end{itemize}

Theirs started with Vietnam and Cambodia last year. This fall, it was to
be a small ship cruise in Greece, which has been postponed a year
because of the virus. The pandemic, she said, reshuffled their
priorities, pushing Tanzania --- originally planned for 2021 --- farther
out, pending a coronavirus vaccine, and moving Alaska up in its place.

``Having a plan takes it from dreaming and conjecturing to actually
having things committed on paper, always with adjustments,'' she said.
``We've moved the chess pieces around.''

In addition to compounding their wanderlust, many travelers and planners
say the pandemic has revealed travel's environmental impact and are
planning more mindfully.

``Our current situation has made me even more committed to focusing
exclusively on sustainability going forward,'' wrote Rose O'Connor, a
travel adviser in Granite Bay, Calif., in an email.

``On one hand, we have seen how tourism can be vital to conservation
efforts in certain destinations,'' she wrote, noting the uptick in
poaching in Africa in the absence of tourism revenue. On the other hand,
she added, traveling from a hot spot like the United States particularly
to remote or developing countries ``is an ethical issue.''

Jeremy Bassetti, a professor of humanities at Valencia College in
Orlando, Fla., has a sabbatical coming up in fall 2021 and plans to use
miles to get to China and then travel overland to Tibet, Nepal and India
for several months. While big trips often accompany sabbaticals, Mr.
Bassetti has rethought his to ``travel longer, farther and more slowly
in 2021,'' he said.

``Why wouldn't we want to travel more to connect more when our
assumptions about being free to travel wherever we want is disappearing
before our eyes?'' he added. ``If you want to experience new cultures,
you can't do it very quickly.''

For others, 2022 presents the possibility of traveling in a time when
the virus may be contained and spontaneity can resume.

High school freshmen Scout Dingman, of Miami, and Sophie Brandimarte, of
Glen Head, N.Y., had been collaborating on a 2021 trip to Europe, making
plans for their families to join. They have marked up maps and are
keeping a Google Doc of destinations where they might branch out to from
Hamburg, where they plan to visit a friend, though they are keeping
their plans loose.

Because of the uncertainly of the virus, and the possibility of having
to cancel and risk deposits, they are delaying the trip to summer 2022
while maintaining their optimism.

``We thought if we pushed it back, then we wouldn't be disappointed,''
Ms. Dingman said.

``We have to think of safety measures now,'' Ms. Brandimarte added.
``But in terms of the actual trip, we really want to keep on the bright
side and not have to worry about that, too.''

\begin{center}\rule{0.5\linewidth}{\linethickness}\end{center}

\emph{\textbf{Follow New York Times Travel}} \emph{on}
\href{https://www.instagram.com/nytimestravel/}{\emph{Instagram}}\emph{,}
\href{https://twitter.com/nytimestravel}{\emph{Twitter}} \emph{and}
\href{https://www.facebookcorewwwi.onion/nytimestravel/}{\emph{Facebook}}\emph{.
And}
\href{https://www.nytimes3xbfgragh.onion/newsletters/traveldispatch}{\emph{sign
up for our weekly Travel Dispatch newsletter}} \emph{to receive expert
tips on traveling smarter and inspiration for your next vacation.}

Advertisement

\protect\hyperlink{after-bottom}{Continue reading the main story}

\hypertarget{site-index}{%
\subsection{Site Index}\label{site-index}}

\hypertarget{site-information-navigation}{%
\subsection{Site Information
Navigation}\label{site-information-navigation}}

\begin{itemize}
\tightlist
\item
  \href{https://help.nytimes3xbfgragh.onion/hc/en-us/articles/115014792127-Copyright-notice}{©~2020~The
  New York Times Company}
\end{itemize}

\begin{itemize}
\tightlist
\item
  \href{https://www.nytco.com/}{NYTCo}
\item
  \href{https://help.nytimes3xbfgragh.onion/hc/en-us/articles/115015385887-Contact-Us}{Contact
  Us}
\item
  \href{https://www.nytco.com/careers/}{Work with us}
\item
  \href{https://nytmediakit.com/}{Advertise}
\item
  \href{http://www.tbrandstudio.com/}{T Brand Studio}
\item
  \href{https://www.nytimes3xbfgragh.onion/privacy/cookie-policy\#how-do-i-manage-trackers}{Your
  Ad Choices}
\item
  \href{https://www.nytimes3xbfgragh.onion/privacy}{Privacy}
\item
  \href{https://help.nytimes3xbfgragh.onion/hc/en-us/articles/115014893428-Terms-of-service}{Terms
  of Service}
\item
  \href{https://help.nytimes3xbfgragh.onion/hc/en-us/articles/115014893968-Terms-of-sale}{Terms
  of Sale}
\item
  \href{https://spiderbites.nytimes3xbfgragh.onion}{Site Map}
\item
  \href{https://help.nytimes3xbfgragh.onion/hc/en-us}{Help}
\item
  \href{https://www.nytimes3xbfgragh.onion/subscription?campaignId=37WXW}{Subscriptions}
\end{itemize}
