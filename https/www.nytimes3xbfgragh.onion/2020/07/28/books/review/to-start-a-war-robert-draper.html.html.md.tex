Sections

SEARCH

\protect\hyperlink{site-content}{Skip to
content}\protect\hyperlink{site-index}{Skip to site index}

\href{https://www.nytimes3xbfgragh.onion/section/books/review}{Book
Review}

\href{https://myaccount.nytimes3xbfgragh.onion/auth/login?response_type=cookie\&client_id=vi}{}

\href{https://www.nytimes3xbfgragh.onion/section/todayspaper}{Today's
Paper}

\href{/section/books/review}{Book Review}\textbar{}Why the United States
Invaded Iraq

\url{https://nyti.ms/3hMKC6f}

\begin{itemize}
\item
\item
\item
\item
\item
\end{itemize}

\href{https://www.nytimes3xbfgragh.onion/spotlight/at-home?action=click\&pgtype=Article\&state=default\&region=TOP_BANNER\&context=at_home_menu}{At
Home}

\begin{itemize}
\tightlist
\item
  \href{https://www.nytimes3xbfgragh.onion/2020/07/28/books/time-for-a-literary-road-trip.html?action=click\&pgtype=Article\&state=default\&region=TOP_BANNER\&context=at_home_menu}{Take:
  A Literary Road Trip}
\item
  \href{https://www.nytimes3xbfgragh.onion/2020/07/29/magazine/bored-with-your-home-cooking-some-smoky-eggplant-will-fix-that.html?action=click\&pgtype=Article\&state=default\&region=TOP_BANNER\&context=at_home_menu}{Cook:
  Smoky Eggplant}
\item
  \href{https://www.nytimes3xbfgragh.onion/2020/07/27/travel/moose-michigan-isle-royale.html?action=click\&pgtype=Article\&state=default\&region=TOP_BANNER\&context=at_home_menu}{Look
  Out: For Moose}
\item
  \href{https://www.nytimes3xbfgragh.onion/interactive/2020/at-home/even-more-reporters-editors-diaries-lists-recommendations.html?action=click\&pgtype=Article\&state=default\&region=TOP_BANNER\&context=at_home_menu}{Explore:
  Reporters' Obsessions}
\end{itemize}

Advertisement

\protect\hyperlink{after-top}{Continue reading the main story}

Supported by

\protect\hyperlink{after-sponsor}{Continue reading the main story}

nonfiction

\hypertarget{why-the-united-states-invaded-iraq}{%
\section{Why the United States Invaded
Iraq}\label{why-the-united-states-invaded-iraq}}

\includegraphics{https://static01.graylady3jvrrxbe.onion/images/2020/08/09/books/review/09Heilbrunn/merlin_11209316_2660760c-7462-4a8a-9032-e76095bdd0d5-articleLarge.jpg?quality=75\&auto=webp\&disable=upscale}

Buy Book ▾

\begin{itemize}
\tightlist
\item
  \href{https://www.amazon.com/gp/search?index=books\&tag=NYTBSREV-20\&field-keywords=To+Start+a+War+Robert+Draper}{Amazon}
\item
  \href{https://du-gae-books-dot-nyt-du-prd.appspot.com/buy?title=To+Start+a+War\&author=Robert+Draper}{Apple
  Books}
\item
  \href{https://www.anrdoezrs.net/click-7990613-11819508?url=https\%3A\%2F\%2Fwww.barnesandnoble.com\%2Fw\%2F\%3Fean\%3D9780525561040}{Barnes
  and Noble}
\item
  \href{https://www.anrdoezrs.net/click-7990613-35140?url=https\%3A\%2F\%2Fwww.booksamillion.com\%2Fp\%2FTo\%2BStart\%2Ba\%2BWar\%2FRobert\%2BDraper\%2F9780525561040}{Books-A-Million}
\item
  \href{https://bookshop.org/a/3546/9780525561040}{Bookshop}
\item
  \href{https://www.indiebound.org/book/9780525561040?aff=NYT}{Indiebound}
\end{itemize}

When you purchase an independently reviewed book through our site, we
earn an affiliate commission.

By Jacob Heilbrunn

\begin{itemize}
\item
  Published July 28, 2020Updated July 29, 2020
\item
  \begin{itemize}
  \item
  \item
  \item
  \item
  \item
  \end{itemize}
\end{itemize}

\textbf{TO START A WAR}\\
\textbf{How the Bush Administration Took America Into Iraq}\\
By Robert Draper

In April 2003, after he had launched the invasion of Iraq, George W.
Bush stood in the Oval Office reception room and watched the televised
liberation of Basra, which serves as the country's main port. Next to
him was Secretary of State Colin Powell, who had warned Bush about the
dangers of ousting Saddam Hussein from power. Smoke rose from the
intelligence service headquarters. The city prison had been opened.
Looters were filching desks, chairs and water tanks from state
buildings. As he looked at the pictures, Bush was perplexed. He asked,
``Why aren't they cheering?''

In ``To Start a War,'' which is filled with such telling scenes,
\href{https://www.nytimes3xbfgragh.onion/by/robert-draper}{Robert
Draper} carefully examines the Bush administration's illusions about
Iraq. Draper is a writer at large for The New York Times Magazine and
the author of
\href{https://www.nytimes3xbfgragh.onion/2007/11/04/books/review/Lewis3-t.html}{``Dead
Certain,''} a study of the Bush administration that relied on numerous
interviews with the president himself. Draper relates that Bush, who was
apparently displeased with his depiction in ``Dead Certain,'' declined
to be interviewed for this book. But Bush did not seek to hinder access
to his former aides and Draper has performed prodigious research,
including conducting interviews with several hundred former national
security officials and scrutinizing recently declassified government
documents. He does not provide any bold revelations, but offers the most
comprehensive account of the administration's road to war, underscoring
that Bush was indeed The Decider when it came to Iraq --- there was
never any debate about \emph{not} overthrowing Hussein.

The basis for conflict, Draper reminds us, had already been prepared in
the late 1990s by what might be called the military-intellectual complex
in Washington. Two key events occurred in 1998: The first was when
Congress passed, and Bill Clinton signed into law, the
\href{https://www.congress.gov/bill/105th-congress/house-bill/4655}{Iraq
Liberation Act}, which the Iraqi expatriate Ahmad Chalabi and his
neoconservative allies like Paul Wolfowitz had championed, and that made
it official American policy to topple Saddam Hussein. The second was the
establishment by Congress of the Rumsfeld Commission. It provided the
former
s\href{https://history.defense.gov/Multimedia/Biographies/Article-View/Article/571280/donald-h-rumsfeld/}{ecretary
of defense Donald Rumsfeld}, Wolfowitz and other hawks with a
high-profile platform to castigate the C.I.A. for its putative
shortsightedness about the looming perils posed by North Korea, Iran and
Iraq. In particular the commission focused on a variety of doomsday
scenarios that might allow Iraq to obtain nuclear weapons and target
America ``in a very short time.''

In those days, none of this seemed to matter that much. But after 9/11,
it did. Drawing on their years of warnings about threats from abroad,
Rumsfeld and Wolfowitz teamed up with Vice President Dick Cheney to push
for war and isolate the reluctant Powell.

Some of Draper's most revealing passages focus on the intense pressure
that Cheney and his chief of staff, I. Lewis Libby, as well as the
Defense Department official Douglas J. Feith, exerted on the
intelligence agencies to buttress and even concoct the case that Saddam
had intimate ties with Al Qaeda and that he possessed weapons of mass
destruction. Draper presents the former C.I.A. director
\href{https://www.cnn.com/2013/08/06/us/george-tenet-fast-facts/index.html}{George
Tenet} in a particularly unflattering light. After being shunted aside
during the Clinton presidency, Tenet was desperate to show Bush that he
was an important and loyal soldier in the new war against terrorism.
``Here we had this precious access,'' one senior analyst told Draper,
``and he didn't want to blow it.'' Tenet and his aides, Draper writes,
``feared the prospect of President Bush being spoon-fed a bouillabaisse
of truths, unverified stories presented as truths and likely falsehoods.
On the other hand, the agency stood to lose its role in helping separate
fact from fiction if it appeared to be close-minded.''

But Tenet ended up displaying canine fealty to Bush. In October 2002,
when asked by the Senate intelligence chairman Bob Graham about whether
any links between Saddam and Osama bin Laden really existed, Draper
writes, Tenet ``issued a reply that Cheney, Libby, Wolfowitz and Feith
could only have dreamed of.'' He declared, among other things, that
there was ``solid reporting of senior level contacts between Iraq and Al
Qaeda going back a decade.''

For all the effort that Cheney and others expended in trying to depict
Iraq as a dire menace, how much did the evidence and details actually
matter? The cold, hard truth is that they didn't. They were political
Play-Doh, to be massaged and molded as Bush's camarilla saw fit. Draper
highlights the famous ``slam dunk'' meeting in the Oval Office in
December 2002, when Tenet assured Bush that the evidence for Colin
Powell's upcoming speech at the United Nations Security Council in
support of an invasion was solid.

In
\href{https://www.nytimes3xbfgragh.onion/2004/04/28/books/review/plan-of-attack-all-the-presidents-mentors.html}{``Plan
of Attack,''}Bob Woodward described Bush as being beset by doubt about
the case for war, and suggested that Tenet's affirmation had been ``very
important.'' Draper disagrees. The issue wasn't the evidence. It was the
spin: ``Tenet's words were `important' only because they helped remove
any doubt as to whether the C.I.A. could mount a solid case.'' Bush's
thinking was as clear as it was simplistic. Saddam was a monster. It
would be a bad idea to leave him in power. According to Draper, Bush's
``increasingly bellicose rhetoric reflected a wartime president who was
no longer tethered to anything other than his own convictions.''

In his 2005 Inaugural Address, Bush tried to turn neoconservative
ideology into official doctrine: ``It is the policy of the United States
to seek and support the growth of democratic movements and institutions
in every nation and culture, with the ultimate goal of ending tyranny in
our world.'' It wasn't until the shellacking that the Republicans
endured in the 2006 midterm elections that Bush began to abandon his
fantasies about spreading peace, love and understanding across the
Middle East. He fired Rumsfeld and shunted Cheney to the side.

If Draper expertly dissects the ferocious turf battles that took place
within the administration over the war, he does not really seek to set
it in a wider context other than to note rather benignly that ``the
story I aim to tell is very much a human narrative of patriotic men and
women who, in the wake of a nightmare, pursued that most elusive of
dreams: finding peace through war.'' But there was more to it than that.
Thanks to Donald Trump's bungling, Bush may be benefiting from a wave of
nostalgia for his presidency. But he was criminally culpable in his
naïveté and incuriosity about the costs and consequences of war. At the
same time, Cheney and Rumsfeld were inveterate schemers whose cynicism
about going to war was exceeded only by their ineptitude in conducting
it.

With American power at its apogee after the fall of the Soviet Union,
their aim was to ensure American primacy, to establish what the
Washington Post
\href{https://www.nytimes3xbfgragh.onion/2018/06/21/obituaries/charles-krauthammer-prominent-conservative-voice-dies-at-68.html}{columnist
Charles Krauthammer}had called America's unipolar moment. Instead, they
squandered the opportunity. In the name of spreading democracy abroad,
they were willing to countenance its degradation at home. Despite the
debacle in Iraq, the very same truculent impulses continue to linger in
the Trump administration, which has been steadily pushing for regime
change in Iran. In this way, Draper provides a timely reminder of the
dangers of embarking upon wars that can imperil America itself.

Advertisement

\protect\hyperlink{after-bottom}{Continue reading the main story}

\hypertarget{site-index}{%
\subsection{Site Index}\label{site-index}}

\hypertarget{site-information-navigation}{%
\subsection{Site Information
Navigation}\label{site-information-navigation}}

\begin{itemize}
\tightlist
\item
  \href{https://help.nytimes3xbfgragh.onion/hc/en-us/articles/115014792127-Copyright-notice}{©~2020~The
  New York Times Company}
\end{itemize}

\begin{itemize}
\tightlist
\item
  \href{https://www.nytco.com/}{NYTCo}
\item
  \href{https://help.nytimes3xbfgragh.onion/hc/en-us/articles/115015385887-Contact-Us}{Contact
  Us}
\item
  \href{https://www.nytco.com/careers/}{Work with us}
\item
  \href{https://nytmediakit.com/}{Advertise}
\item
  \href{http://www.tbrandstudio.com/}{T Brand Studio}
\item
  \href{https://www.nytimes3xbfgragh.onion/privacy/cookie-policy\#how-do-i-manage-trackers}{Your
  Ad Choices}
\item
  \href{https://www.nytimes3xbfgragh.onion/privacy}{Privacy}
\item
  \href{https://help.nytimes3xbfgragh.onion/hc/en-us/articles/115014893428-Terms-of-service}{Terms
  of Service}
\item
  \href{https://help.nytimes3xbfgragh.onion/hc/en-us/articles/115014893968-Terms-of-sale}{Terms
  of Sale}
\item
  \href{https://spiderbites.nytimes3xbfgragh.onion}{Site Map}
\item
  \href{https://help.nytimes3xbfgragh.onion/hc/en-us}{Help}
\item
  \href{https://www.nytimes3xbfgragh.onion/subscription?campaignId=37WXW}{Subscriptions}
\end{itemize}
