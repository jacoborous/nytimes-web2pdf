\href{/section/science}{Science}\textbar{}Too Much Mars? Let's Discuss
Other Worlds

\url{https://nyti.ms/3jORdio}

\begin{itemize}
\item
\item
\item
\item
\item
\item
\end{itemize}

\hypertarget{missions-to-mars}{%
\subsubsection{\texorpdfstring{\href{https://www.nytimes3xbfgragh.onion/news-event/summer-of-mars?name=styln-mars\&region=TOP_BANNER\&variant=undefined\&block=storyline_menu_recirc\&action=click\&pgtype=Article\&impression_id=8a65ad20-e38b-11ea-af50-edb5cee8ee78}{Missions
to Mars}}{Missions to Mars}}\label{missions-to-mars}}

\begin{itemize}
\tightlist
\item
  \href{https://www.nytimes3xbfgragh.onion/2020/07/30/science/nasa-mars-launch.html?name=styln-mars\&region=TOP_BANNER\&variant=undefined\&block=storyline_menu_recirc\&action=click\&pgtype=Article\&impression_id=8a65ad21-e38b-11ea-af50-edb5cee8ee78}{NASA
  Launch Highlights}
\item
  \href{https://www.nytimes3xbfgragh.onion/interactive/2020/science/mars-perseverance-tianwen-hope.html?name=styln-mars\&region=TOP_BANNER\&variant=undefined\&block=storyline_menu_recirc\&action=click\&pgtype=Article\&impression_id=8a65d430-e38b-11ea-af50-edb5cee8ee78}{Meet
  the Spacecraft}
\item
  \href{https://www.nytimes3xbfgragh.onion/2020/07/28/science/nasa-jezero-perseverance.html?name=styln-mars\&region=TOP_BANNER\&variant=undefined\&block=storyline_menu_recirc\&action=click\&pgtype=Article\&impression_id=8a65d431-e38b-11ea-af50-edb5cee8ee78}{NASA's
  Destination}
\item
  \href{https://www.nytimes3xbfgragh.onion/2020/07/28/science/mars-nasa-science.html?name=styln-mars\&region=TOP_BANNER\&variant=undefined\&block=storyline_menu_recirc\&action=click\&pgtype=Article\&impression_id=8a65d432-e38b-11ea-af50-edb5cee8ee78}{Too
  Much Mars?}
\end{itemize}

\includegraphics{https://static01.graylady3jvrrxbe.onion/images/2020/07/28/science/MARSCHAT1/MARSCHAT1-articleLarge.jpg?quality=75\&auto=webp\&disable=upscale}

Sections

\protect\hyperlink{site-content}{Skip to
content}\protect\hyperlink{site-index}{Skip to site index}

\hypertarget{too-much-mars-lets-discuss-other-worlds}{%
\section{Too Much Mars? Let's Discuss Other
Worlds}\label{too-much-mars-lets-discuss-other-worlds}}

Two veteran space journalists discuss why so much attention and budget
seems to be directed to the red planet.

Credit...Miguel Porlan

Supported by

\protect\hyperlink{after-sponsor}{Continue reading the main story}

By Rebecca Boyle and David W. Brown

\begin{itemize}
\item
  Published July 28, 2020Updated July 30, 2020
\item
  \begin{itemize}
  \item
  \item
  \item
  \item
  \item
  \item
  \end{itemize}
\end{itemize}

Three government space agencies around the world are getting ready to
return to Mars this summer. Along with
\href{https://www.nytimes3xbfgragh.onion/2020/07/22/science/china-mars-mission.html}{China}
and the
\href{https://www.nytimes3xbfgragh.onion/2020/07/14/science/mars-united-arab-emirates.html}{United
Arab Emirates}, the United States plans to land
\href{https://www.nytimes3xbfgragh.onion/interactive/2020/science/mars-perseverance-tianwen-hope.html}{the
fifth NASA rover, Perseverance}, on the red planet (along with
\href{https://www.nytimes3xbfgragh.onion/2020/06/23/science/mars-helicopter-nasa.html}{a
small, experimental helicopter, Ingenuity}). But the rover's most
important job will be
\href{https://www.nytimes3xbfgragh.onion/2020/07/28/science/mars-sample-return-mission.html}{scooping
up and caching some samples} that humans or robots may eventually
retrieve.

The planetary science community will cheer these missions. But many
researchers have started asking, more loudly than usual, why we're going
back to Mars yet again. So we invited Rebecca Boyle and David W. Brown,
two journalists who have devoted a fair share of their careers to
interviewing space researchers at NASA and in academia, to discuss why
Mars,
\href{https://www.nytimes3xbfgragh.onion/2020/07/24/science/mars-life-water.html}{a
planet that lost its atmosphere long ago}, seems to absorb so much of
the oxygen --- and budgetary resources --- in the rooms where
\href{https://www.nytimes3xbfgragh.onion/interactive/2020/science/exploring-the-solar-system.html}{explorations
of our solar system} are decided.

\begin{center}\rule{0.5\linewidth}{\linethickness}\end{center}

\textbf{Rebecca Boyle:} So we're going back to Mars. Again, with another
rover. Two, perhaps, if both NASA and China's space agency succeed.

Sigh.

It's not that this is disappointing. But there's a certain level of déjà
vu with NASA's Perseverance mission, modeled so closely after the
successful Curiosity rover in 2011. I have
\href{https://www.scientificamerican.com/article/lost-opportunity-after-a-15-year-odyssey-nasas-trailblazing-mars-rover-approaches-its-end/}{written}
a lot about the value of
\href{https://fivethirtyeight.com/features/everything-about-mars-is-the-worst/}{exploring
Mars} and the particularly Earthlike qualities that
\href{https://www.theatlantic.com/science/archive/2017/01/mars-is-the-best-planet/512654/}{endear
it to us}. But even I can't help but wonder what's next in
\href{https://www.nytimes3xbfgragh.onion/interactive/2020/science/exploring-the-solar-system.html}{our
quest to explore the solar system}, and whether so many journeys to Mars
are blocking other important science.

\href{https://www.nytimes3xbfgragh.onion/interactive/2020/science/mars-perseverance-tianwen-hope.html}{}

\includegraphics{https://static01.graylady3jvrrxbe.onion/images/2020/07/24/science/space/mars-spacecraft-overview-1595636280246/mars-spacecraft-overview-1595636280246-articleLarge.jpg}

\hypertarget{meet-the-3-spacecraft-heading-to-mars-this-summer}{%
\subsection{Meet the 3 Spacecraft Heading to Mars This
Summer}\label{meet-the-3-spacecraft-heading-to-mars-this-summer}}

Three missions are headed to Mars this summer. They carry a wide array
of instruments to explore the red planet.

\textbf{David W. Brown:} There's an entire solar system waiting to be
explored. Since 2001, NASA has flown eight consecutive successful
missions to Mars, including five landers. Humanity now has a library of
Mars data sitting on servers that no one has had a chance to study. Data
collected from brief encounters by spacecraft with
\href{https://www.nytimes3xbfgragh.onion/2018/05/14/science/europa-plumes-water.html}{the
moons of Jupiter}, on the other hand, or
\href{https://www.nytimes3xbfgragh.onion/2019/02/20/science/neptune-moon-hippocamp.html}{the
ice giants, Uranus and Neptune}, have been
\href{https://www.nytimes3xbfgragh.onion/2020/03/27/science/uranus-bubble-voyager.html}{squeezed
dry.}

\textbf{R.B.:} And that has made the story of Mars exploration a
self-fulfilling one. The robots we've sent to Mars have been, on
balance, exceptionally successful, and have told us about
\href{https://www.nytimes3xbfgragh.onion/2020/07/24/science/mars-life-water.html}{vast
waters in Mars' distant memory}, including floodplains where life might
have burbled to the surface. They have told us about Martian ice caps,
carbon dioxide snow, inner geology, tenuous atmosphere and wind-eroded
surface.

Every discovery about Mars brings new questions, which cultivates a
healthy ecosystem of Mars researchers. Some of those researchers wind up
at academic institutions, where they have funding for graduate students
and postdoctoral researchers who continue asking more Mars questions.
And so the search continues.

But the past few years have seen an
\href{https://www.nature.com/articles/d41586-019-01730-5}{increasingly}
\href{https://www.sciencemag.org/news/2017/01/venus-can-wait-jilted-scientists-face-years-without-nasa-return-earth-s-neighbor}{loud}
\href{https://www.theatlantic.com/science/archive/2017/01/venus-lost-generation/513479/}{chorus}
calling for NASA to turn its sights elsewhere.
\href{https://www.nytimes3xbfgragh.onion/2020/01/09/science/venus-volcanoes-active.html}{Maybe
Venus}, maybe
\href{https://www.nytimes3xbfgragh.onion/2020/03/10/science/saturn-titan-moon.html}{Saturn's
moon Titan}, maybe
\href{https://www.nytimes3xbfgragh.onion/2020/03/27/science/uranus-bubble-voyager.html}{the
distant, faceless turquoise orb of Uranus} --- just not Mars, anywhere
but Mars, where we've been
\href{https://www.nytimes3xbfgragh.onion/2019/02/13/science/mars-opportunity-rover-dead.html}{driving
wheeled robots for nearly two decades}.

\textbf{D.W.B.:} For many years, Mars was treated by NASA almost as its
own space program with its own mission lines and priorities. Every two
years, something ** would launch there.

But then in 2010, in part because of budget cuts, the Mars program was
integrated into NASA's overall planetary exploration portfolio. Suddenly
other worlds were being annihilated by a crimson Death Star.

The Curiosity rover went over budget by almost a billion dollars, eating
into other projects. Perseverance went wildly over budget as well, and
if the pattern holds, the
\href{https://www.nytimes3xbfgragh.onion/2020/07/28/science/mars-sample-return-mission.html}{eventual
mission to collect the samples gathered by the rover} would do the same.

While much remains uncertain about the particulars of the sample return
sequence --- will the next be a joint mission with the Europeans, for
example, or will it be NASA astronauts going it alone, or will we pay
SpaceX to do it --- it was deemed by the planetary science community to
be a multi-billion-dollar ``flagship'' mission of the highest priority.

That decision killed off a spacecraft that would have orbited Europa,
Jupiter's ocean moon. (A less expensive design, Europa Clipper, is to
launch this decade.) While the shock was still fresh, NASA selected
\href{https://www.nytimes3xbfgragh.onion/interactive/2018/05/01/science/mars-nasa-insight-ar-3d-ul.html}{the
Mars InSight lander} over a well-regarded mission in its less expensive
Discovery class that would have landed a boat on Titan, Saturn's largest
moon, to sail its seas of liquid methane.

\href{https://www.nytimes3xbfgragh.onion/interactive/2020/science/exploring-the-solar-system.html}{}

\includegraphics{https://static01.graylady3jvrrxbe.onion/images/2020/07/24/us/exploring-the-solar-system-promo-1595620746754/exploring-the-solar-system-promo-1595620746754-articleLarge.png}

\hypertarget{exploring-the-solar-system}{%
\subsection{Exploring the Solar
System}\label{exploring-the-solar-system}}

A guide to the spacecraft beyond Earth's orbit.

\textbf{R.B.}: Rest in peace, Titan Mare Explorer. Meanwhile, as
planetary scientists debated how to pay for their missions, some
geologists salivate for a second look at Venus, the second planet from
the sun.

Venus is about the same size as Earth, it's rocky, it has an atmosphere.
And, it orbits the sun in a zone where temperatures are just right for
liquid water --- and maybe life.

``I got my Ph.D. on Mars volcanoes. I am on three Mars proposals. It's
not like I don't value Mars --- it's an amazing world,'' said Paul
Byrne, a planetary scientist at North Carolina State University. ``But
it's not the only amazing world.''

He told me he is free to study Venus's clouds and atmosphere because he
has a faculty position that covers most of his salary and commitments.
That's not the case for many other scientists, who rely on grants and
federally funded planetary exploration programs to gather their data.

Even in other countries, Venus doesn't get the attention that other
worlds do. Dr. Byrne told me he attended a meeting in Moscow in October
focused on future Venus missions, including a beefed-up, modern version
of the Soviet Venera lander, so far the only spacecraft that has
survived on the Venusian surface (the last one, Venera 12, lasted 110
minutes). Only two Russian geologists turned up to the meeting.

``There isn't a tradition among young Russian scientists to do Venus,
same as here,'' he said. ``There hasn't been much of an effective
selling of Venus. It's being subsumed by Mars. And people who were
funded for Venus have, with very few exceptions, gone away'' to
retirement or other projects, he said.

We know Mars had water at some point in its past, but it's long gone. By
contrast, Venus might have had oceans more recently and for longer
periods, and may have been comfortably livable for billions of years.

``Why are we not putting a fleet there?'' Mr. Byrne said.

Why, indeed?

\textbf{D.W.B.}: Airships sniffing around the upper atmosphere of Venus
are overdue. Certainly since the retirement of the space shuttle,
nothing in the NASA portfolio, save
\href{https://www.nytimes3xbfgragh.onion/2020/05/30/science/spacex-nasa-astronauts.html}{the
recent SpaceX launch to the International Space Station}, has generated
the excitement of the planetary science program.

The
\href{https://www.nytimes3xbfgragh.onion/interactive/2016/03/17/science/pluto-images-charon-moons-new-horizons-flyby.html}{New
Horizons flyby of Pluto}, the
\href{https://www.nytimes3xbfgragh.onion/interactive/2017/09/14/science/cassini-saturn-images.html}{Cassini
mission to Saturn},
\href{https://www.nytimes3xbfgragh.onion/2017/04/07/science/jupiter-photos-hubble-telescope-juno-nasa.html}{Juno's
stunning images of a Jupiter dipped in blue} --- to the extent that
people talk about NASA ``moments,'' they're likely talking about
planetary science.

The exception is perhaps
\href{https://www.nytimes3xbfgragh.onion/2020/04/24/science/hubble-telescope-30th-birthday.html}{the
Hubble Space Telescope}, which delivers so many stunning images that
it's become a kind of ever-present technicolor cultural static. We don't
even notice it anymore, which is a testament to its success.

\includegraphics{https://static01.graylady3jvrrxbe.onion/images/2020/07/28/science/28MARSCHAT2/28MARSCHAT2-articleLarge.jpg?quality=75\&auto=webp\&disable=upscale}

\textbf{R.B.}: I think this also applies to images from Mars, in a
certain sense. We've been receiving them in living color since the late
1990s, and they've become a kind of static too. Curiosity has a
high-resolution camera, and people were captivated by its incredible
images after it landed in August 2012, yet barely notice anymore.

I used to check its downloads occasionally, using a program called
\href{http://www.midnightplanets.com/index.html}{Midnight Planets},
built by Michael Howard, an app developer. He would download raw images
as they were transmitted via the Deep Space Network's antennas. It was
great fun to see what the rovers were seeing, in almost real-time. But
Mr. Howard stopped updating his site almost a year ago, noting that ``I
have moved on to other projects in life.''

I only found this out because I just checked it for the first time in
more than a year. What a statement, right? We can't get excited about
nightly image dumps from Mars. FROM MARS.

Curiosity was the first Mars mission to land in the era of social media,
and people around the world watched on Twitter and live on TV in Times
Square. That was electrifying. But the attention has faded.

Is it because Mars looks so familiar? For now, we'll have to be content
with wondering what the public response would be to something like a
boat on Titan.

\textbf{D.W.B.}: At least there's a
\href{https://www.nytimes3xbfgragh.onion/2019/06/27/science/nasa-titan-dragonfly-caesar.html}{Titan
quadcopter} in the works. The problem for the rest of the solar system
is that the S in NASA doesn't stand for ``science.'' The agency is first
and foremost a human spaceflight organization. That's where its real
money goes, and robotic Mars missions are the natural beneficiaries.
Astronauts can't get to Europa with current technologies, and can't land
on Venus. But humans in spacesuits can survive on Mars, which means
every robotic Mars mission is more than abstruse geology or aeolian
physics. No, every Mars mission is a human precursor mission. Every
dollar spent on Mars rovers reduces the inherent risk of future
astronaut adventures.

Culturally, Mars is deeply important to NASA, and has been since its
infancy. A humans-to-Mars program, conceived before Apollo, was the
lunar program's natural successor. To become multi-planetary, NASA
identified the need for reusable space shuttles, a space station,
rockets at least as powerful as the Saturn V and other space-based
infrastructure. Though the glory days of Apollo funding died, the
fundamental elements of Mars exploration did not: All of those things
were built, though across a much longer timeline.

\textbf{R.B.}: The other reason is private industry. It's relatively
easy to fling something at Mars; every 26 months, the planet is on the
same side of the sun as Earth is, so the journey only takes half a year.

On roughly the same timeline, SpaceX C.E.O.
\href{https://www.nytimes3xbfgragh.onion/2019/09/29/science/elon-musk-spacex-starship.html}{Elon
Musk makes a bold and vague announcement about his plans to launch
cruise ships to Mars}, carrying people who might live there forever. A
few years ago, a Dutch start-up named Mars One even tried launching a
reality-show-based settlement program,
\href{https://www.theverge.com/2019/2/11/18220153/mars-one-bankruptcy-bas-lansdorp-human-settlement}{before
going bankrupt}.

Though Mars is horrendous, it's the least-inhospitable place to go,
except for maybe the moon. It seems attainable. Bootprints on the red
regolith seems feasible. So that's why, in the generations since Apollo,
we keep seeing rover tracks in the red regolith, time and again.

\textbf{D.W.B.}: In some ways, the Perseverance rover confirms the
harshest criticisms of the Mars program. It was sold as a relatively
inexpensive \$1.5 billion dollar reflight of the Curiosity rover and
lander, built using spare parts and with a different payload of science
instruments.

In the end, Perseverance went over budget by more than \$500 million ---
\href{https://www.nytimes3xbfgragh.onion/2020/06/23/science/mars-helicopter-nasa.html}{and
they added a
helicopter}\href{https://www.nytimes3xbfgragh.onion/2020/06/23/science/mars-helicopter-nasa.html}{**}\href{https://www.nytimes3xbfgragh.onion/2020/06/23/science/mars-helicopter-nasa.html}{to
it}!

Almost everything got an upgrade. The rover isn't the same size --- the
wheels, chassis, camera --- all new. Not even the spare heat shield was
used. This ambitious refit could have paid for an entire Discovery-class
mission, and has caused discomfort to other large missions in the NASA
portfolio.

But Russia's experience with Venus explains a big part of NASA's need to
push the envelope of Mars engineering. To cease building Mars landers
--- to stop daring mighty things like helicopters --- is to lose the
institutional knowledge necessary to do the Red Planet successfully. The
U.S. has been launching humans to space since 1961, but
\href{https://www.nytimes3xbfgragh.onion/2020/05/26/science/spacex-launch-nasa.html}{once
we stopped, it took almost a decade to figure out how to do it again}.

\textbf{R.B.}: The confusion and frustration surrounding the Mars
program is a manifestation of NASA's core existential struggle. As you
pointed out, it's always been an astronaut agency, with human
exploration in its DNA. But when presidential administrations change
exploration objectives every four to eight years, it's harder to plan
for the long term. NASA needs a place to go, and Mars has been the
obvious next step since the moon.

But the agency and country should probably be asking: What is the actual
future we want? What's the endgame? Is it a returned chunk of rock that
could tell us more about Mars' early history? Or maybe we get extremely
lucky and bring back a rock that has evidence of fossil bacteria? Or is
it just about adding layers, like Martian sedimentary rock, of evidence
that Mars is interesting enough for humans to risk walking there one
day?

\textbf{D.W.B.}: ``Long term'' is the strength of Mars exploration
overall, and the great mystery of this mission. No one knows when
\href{https://www.nytimes3xbfgragh.onion/2020/07/28/science/mars-sample-return-mission.html}{the
samples bottled up by Perseverance will be brought back to Earth}. They
might collect red dust for fifteen years before another robot snatches
them up and rockets them here for study. The possibility of life on
Mars, extant or extinct, has been raised and dashed and rendered
inconclusive going back to the Viking landers in 1976 through
\href{https://www.nytimes3xbfgragh.onion/2019/11/20/science/mars-oxygen-methane-curiosity-rover.html}{Curiosity's
methane detection in 2019}.

Maybe the samples will answer that question, or maybe, as Dr. Tim McCoy,
the curator of the meteorite collection at the Smithsonian Institution,
tells me, the samples now being targeted by NASA might ultimately not be
too scientifically useful.

``How do you know that those are going to address the pressing questions
that are going to exist 15 or 20 years from now?'' he asked, explaining
that the planetary science community may have moved on to a completely
different set of questions by the time Mars dirt reaches labs on Earth.

\textbf{R.B.}: When Perseverance starts exploring Mars, I hope it
squirrels away the most interesting rock in the solar system. But while
we wait to find out, there are so many other places we haven't even
been.

\href{https://www.nytimes3xbfgragh.onion/2019/07/10/science/moon-facts.html}{As
a moon partisan, I would like to go back there} and pick up samples from
\href{https://www.nytimes3xbfgragh.onion/2020/02/26/science/china-moon-far-side.html}{one
of the solar system's biggest craters, the South Pole-Aitken Basin.} I
would love to see airships forming a Venusian cloud city. And I really,
really want that Titan boat.

\textbf{D.W.B.}: Venus is overdue for a flagship mission. The Earthlike
conditions above its clouds make the discovery of life there not a
matter of if, but when.

The ice giants also need to be explored, and inventive missions like
\href{https://www.nytimes3xbfgragh.onion/2019/03/19/science/triton-neptune-nasa-trident.html}{the
Trident spacecraft to Triton, Neptune's moon}, prove that it can be done
on relatively small budgets.

For now, I love that landing a nuclear-powered car on Mars is somehow
boring. In the 1980s, funding woes left the planetary science community
fighting for survival. Today, we live in a golden age of space
exploration.

\_\_\_\_\_\_\_\_\_

Rebecca Boyle is a science journalist in Colorado and is working on her
first book, a scientific and cultural history of the moon. David W.
Brown is the author of \href{https://amzn.to/35rcgRE}{The Mission}
(Custom House, Jan. 2021), the true story of a team of scientists who
spent decades conspiring to get a spaceship to Europa. He lives in New
Orleans.

\href{https://www.nytimes3xbfgragh.onion/interactive/2020/science/2020-astronomy-space-calendar.html}{}

\includegraphics{https://static01.graylady3jvrrxbe.onion/images/2019/12/04/science/04SUN1/04SUN1-articleLarge.png}

\hypertarget{sync-your-calendar-with-the-solar-system}{%
\subsection{Sync your calendar with the solar
system}\label{sync-your-calendar-with-the-solar-system}}

Never miss an eclipse, a meteor shower, a rocket launch or any other
astronomical and space event that's out of this world.

Advertisement

\protect\hyperlink{after-bottom}{Continue reading the main story}

\hypertarget{site-index}{%
\subsection{Site Index}\label{site-index}}

\hypertarget{site-information-navigation}{%
\subsection{Site Information
Navigation}\label{site-information-navigation}}

\begin{itemize}
\tightlist
\item
  \href{https://help.nytimes3xbfgragh.onion/hc/en-us/articles/115014792127-Copyright-notice}{©~2020~The
  New York Times Company}
\end{itemize}

\begin{itemize}
\tightlist
\item
  \href{https://www.nytco.com/}{NYTCo}
\item
  \href{https://help.nytimes3xbfgragh.onion/hc/en-us/articles/115015385887-Contact-Us}{Contact
  Us}
\item
  \href{https://www.nytco.com/careers/}{Work with us}
\item
  \href{https://nytmediakit.com/}{Advertise}
\item
  \href{http://www.tbrandstudio.com/}{T Brand Studio}
\item
  \href{https://www.nytimes3xbfgragh.onion/privacy/cookie-policy\#how-do-i-manage-trackers}{Your
  Ad Choices}
\item
  \href{https://www.nytimes3xbfgragh.onion/privacy}{Privacy}
\item
  \href{https://help.nytimes3xbfgragh.onion/hc/en-us/articles/115014893428-Terms-of-service}{Terms
  of Service}
\item
  \href{https://help.nytimes3xbfgragh.onion/hc/en-us/articles/115014893968-Terms-of-sale}{Terms
  of Sale}
\item
  \href{https://spiderbites.nytimes3xbfgragh.onion}{Site Map}
\item
  \href{https://help.nytimes3xbfgragh.onion/hc/en-us}{Help}
\item
  \href{https://www.nytimes3xbfgragh.onion/subscription?campaignId=37WXW}{Subscriptions}
\end{itemize}
