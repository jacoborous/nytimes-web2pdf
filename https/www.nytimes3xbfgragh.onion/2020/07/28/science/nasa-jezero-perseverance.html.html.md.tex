Sections

SEARCH

\protect\hyperlink{site-content}{Skip to
content}\protect\hyperlink{site-index}{Skip to site index}

\href{/section/science}{Science}\textbar{}How NASA Found the Ideal Hole
on Mars to Land In

\url{https://nyti.ms/331UCo4}

\begin{itemize}
\item
\item
\item
\item
\item
\end{itemize}

\href{https://www.nytimes3xbfgragh.onion/news-event/summer-of-mars?action=click\&pgtype=Article\&state=default\&region=TOP_BANNER\&context=storylines_menu}{Missions
to Mars}

\begin{itemize}
\tightlist
\item
  \href{https://www.nytimes3xbfgragh.onion/2020/07/30/science/nasa-mars-launch.html?action=click\&pgtype=Article\&state=default\&region=TOP_BANNER\&context=storylines_menu}{NASA
  Launch Highlights}
\item
  \href{https://www.nytimes3xbfgragh.onion/interactive/2020/science/mars-perseverance-tianwen-hope.html?action=click\&pgtype=Article\&state=default\&region=TOP_BANNER\&context=storylines_menu}{Meet
  the Spacecraft}
\item
  \href{https://www.nytimes3xbfgragh.onion/2020/07/28/science/nasa-jezero-perseverance.html?action=click\&pgtype=Article\&state=default\&region=TOP_BANNER\&context=storylines_menu}{NASA's
  Destination}
\item
  \href{https://www.nytimes3xbfgragh.onion/2020/07/28/science/mars-nasa-science.html?action=click\&pgtype=Article\&state=default\&region=TOP_BANNER\&context=storylines_menu}{Too
  Much Mars?}
\end{itemize}

\includegraphics{https://static01.graylady3jvrrxbe.onion/images/2020/07/28/science/28SCI-MARS-JEZERO1/28SCI-MARS-JEZERO1-articleLarge.jpg?quality=75\&auto=webp\&disable=upscale}

\hypertarget{how-nasa-found-the-ideal-hole-on-mars-to-land-in}{%
\section{How NASA Found the Ideal Hole on Mars to Land
In}\label{how-nasa-found-the-ideal-hole-on-mars-to-land-in}}

Jezero crater, the destination of the Perseverance rover, is a promising
place to look for evidence of extinct Martian life.

An image obtained by the Mars Reconnaissance Orbiter in 2015 of the
Jezero Crater.Credit...NASA/JPL-Caltech/University of Arizona

Supported by

\protect\hyperlink{after-sponsor}{Continue reading the main story}

\href{https://www.nytimes3xbfgragh.onion/by/kenneth-chang}{\includegraphics{https://static01.graylady3jvrrxbe.onion/images/2018/02/16/multimedia/author-kenneth-chang/author-kenneth-chang-thumbLarge.jpg}}

By \href{https://www.nytimes3xbfgragh.onion/by/kenneth-chang}{Kenneth
Chang}

\begin{itemize}
\item
  Published July 28, 2020Updated July 30, 2020
\item
  \begin{itemize}
  \item
  \item
  \item
  \item
  \item
  \end{itemize}
\end{itemize}

Sixteen years ago, Caleb Fassett, then a graduate student at Brown
University, spotted an intriguing hole in the ground on Mars.

\href{https://www.nytimes3xbfgragh.onion/2020/07/30/science/nasa-mars-launch.html}{Mars}
today is cold and dry, but it was not always that way. Here was one of
the places with clear signs that liquid water flowed when the planet was
warmer and wetter.

The image, taken by NASA's Odyssey orbiter, showed a sinuous dried-up
river channel leading into one side of the crater. On the other side of
the crater, part of the rim has collapsed, as if it had been swept away
by flowing water.

In between these two features was a large circular depression.

``The only way that could form geometrically was for it to be a lake,''
said Dr. Fassett, now a planetary scientist at NASA's Marshall Space
Flight Center in Huntsville, Ala.

This one-time lake named Jezero, a crater close to 30 miles wide, is the
next stop on NASA's search for possibilities of life elsewhere in the
solar system. On July 30, the space agency's new Mars rover,
Perseverance, is scheduled to launch on a six-and-a-half-month trip to
the red planet, arriving at Jezero in February.

Perseverance is a near clone of Curiosity, the Mars rover that landed
eight years ago and almost immediately discovered unmistakable signs of
a habitable lake. But Perseverance is outfitted with different
instruments designed to answer a more difficult follow-up question:
\href{https://www.nytimes3xbfgragh.onion/2020/07/24/science/mars-life-water.html}{Could
there have been Martians living on Mars long ago}?

Jezero, Mars scientists decided, is the best place to look.

\hypertarget{inside-jezero-crater}{%
\subsection{Inside Jezero Crater}\label{inside-jezero-crater}}

NASA's Perseverance rover will attempt to land in Jezero Crater, an
ancient Martian lake roughly the size of Lake Tahoe. If successful, the
rover will spend years exporing the river delta and making its way to
the crater rim.

Crater

rim

MARS

Jezero

Crater

Shoreline

JEZERO CRATER

River

delta

Possible

path of

rover

Possible

landing site

TARGET

LANDING AREA

1/2 mile

Canyon

carved by

a river

JEZERO CRATER

Shoreline

River

delta

Crater

rim

Possible

path of

rover

Possible

landing site

MARS

Jezero

Crater

TARGET

LANDING AREA

1/2 mile

Canyon

carved by

a river

JEZERO CRATER

Shoreline

River

delta

Crater

rim

Possible

path of

rover

Possible

landing site

MARS

Jezero

Crater

TARGET

LANDING AREA

1/2 mile

By Jonathan Corum \textbar{} Image by NASA, Jet Propulsion Laboratory,
European Space Agency, German Aerospace Center, Freie Universität Berlin
and Justin Cowart. Inset image by NASA and J.P.L.

\hypertarget{happy-homes-for-martians}{%
\subsection{Happy Homes for Martians}\label{happy-homes-for-martians}}

No one expects to find the Martian equivalent of dinosaur fossils, shark
teeth or seashells. If life arose on Mars, it likely resembled what
existed on early Earth ---
\href{https://www.nytimes3xbfgragh.onion/2020/07/24/science/mars-life-water.html}{single-cell
microbes in oceans, lakes and rivers}.

Even on Earth, these microscopic organisms did not leave behind
recognizable fossils. The evidence of this ancient life is difficult to
discern and sharply debated.

Still, Mars scientists think they might be able to detect patterns in
rocks that could have been the work of microbes.

Liquid water is a requirement for life, and Jezero is but one of
hundreds of former lakes on Mars. The feature that drew scientists to
this particular crater was where the river flowed into the lake more
than 3.5 billion years ago.

Even at the modest resolution of that Odyssey photograph, Dr. Fassett
saw a fan of dirt and mud that had been disgorged by the river into the
crater --- similar to the slope of sediments where the Mississippi Delta
slides into the Gulf of Mexico.

Kennda L. Lynch, a scientist at the Lunar and Planetary Institute in
Houston, said this ``beautiful deltaic deposit'' in Jezero could
preserve hints of life from three different environments: from streams
and smaller lakes upstream; from the Jezero lake itself; or in
groundwater pushed to the surface from below.

``We know on Earth that those kinds of deposits preserve organics,''
said Dr. Lynch, who has studied partially dried-up lakes in Utah that
may resemble what Jezero used to look like.

\includegraphics{https://static01.graylady3jvrrxbe.onion/images/2020/07/28/science/28SCI-MARS-JEZERO2/28SCI-MARS-JEZERO2-articleLarge.jpg?quality=75\&auto=webp\&disable=upscale}

Also seen at Jezero along what appears to have been the shoreline are
deposits of minerals known as carbonates, almost like bathtub rings. The
carbonates could be similar to limestone on Earth, which typically forms
out of seafloor sediments and is often chock-full of fossils.

Although chemical reactions not involving biology can create carbonates,
``This carbonate signature could indicate some kind of microbial life,''
Dr. Lynch said.

The fine lake sediments could have been a happy home for tiny Martians.

At least on Earth, layers of microbes can form at the bottom of a lake,
often held together by slime secreted by the organisms. If anything like
that lived within the lake at Jezero, the biological molecules of the
microbes would likely have decayed away by now. But as one layer formed
on top of another, they could have left wavy patterns in the rocks
similar to what has been found in Earth rocks.

''Now, if you look at that rock, you wouldn't know for sure that it was
a potential biosignature,'' said Kathryn Stack Morgan, one of the
mission's deputy project scientists, during a news conference in June.
``But when you couple the textures, as well as the chemical composition,
the mineralogy and the distribution of organic carbon, you can start to
build a case that that rock could only have formed under the influence
of life.''

A camera and a microscope on the rover will be able to see such
patterns. Another instrument, shooting a beam of X-rays into the rock,
could measure the elements within each layer and help determine if the
layers consist of different minerals or just surface smudges.

Part of Perseverance's mission is to
\href{https://www.nytimes3xbfgragh.onion/2020/07/28/science/mars-sample-return-mission.html}{collect
pieces of rocks that a follow-up spacecraft will bring back to Earth}.
Then scientists will be able to peer at the samples in exquisite detail
for signs of past life.

\hypertarget{tournament-of-holes}{%
\subsection{Tournament of Holes}\label{tournament-of-holes}}

Six years ago, Mars scientists began debates over where to send the
rover, starting with more than 30 candidates. For each, scientists
presented their arguments in favor like lawyers laying out a legal case.

Other candidates had names like Eberswalde (another dried-up lake bottom
with a preserved delta), Holden (an old impact crater that was a lake
the size of Lake Huron) and Mawrth Vallis (a mysterious valley that was
likely wet but with no signs of where its water came from).

Timothy A. Goudge, who started as a planetary sciences graduate student
a few years after Dr. Fassett, took the mantle as the champion for
Jezero during the landing site workshops, which were like a Mars Madness
tournament of holes.

``Definitely much more intense and high-energy than a typical scientific
conference,'' he said.

Now a geosciences professor at the University of Texas, Austin, Dr.
Goudge studied whether it would be possible to identify minerals in the
delta sediments and figure out where they had originated upstream. He
said this kind of ``source-to-sink analysis is really common for
understanding systems on Earth.''

If that was possible, then he could investigate whether the minerals
underwent chemical changes along the way. Or were they the pieces of
rock that simply chipped off the outcrops and washed into crater pretty
much unchanged?

Image

A Martian sunset observed in 2005 from the Mars rover Spirit from below
the rim of the Gusev Crater. The crater's Columbia Hills region was a
finalist considered as a landing spot for
Perseverance.~Credit...NASA/JPL/Texas A\&M/Cornell

Dr. Fassett had identified two deltas flowing into Jezero: a large
prominent one to the west and a smaller, more eroded feature on the
northern rim. Dr. Goudge found that the two deltas contained different
minerals, reflecting different origins of the sediments.

The presence of clays and carbonates, minerals that typically form with
the interaction of carbon dioxide and liquid water, strongly suggested
that this was a freshwater lake that was not highly alkaline or highly
acidic: a location that could have been friendly for life.

Bethany L. Ehlmann, a professor at the California Institute of
Technology, initially put forth Jezero as a possible landing site. But
she then suggested that Dr. Goudge present the science at a workshop
reviewing the candidate sites.

``She was like, `I'm presenting a bunch of other stuff. So why don't
you? You've been doing the most recent work on this,''' Dr. Goudge
recalled. ``And so I said yes.''

In 2017, Perseverance mission scientists narrowed the possible landing
sites to three. Jezero made the cut.

``I felt very proud,'' Dr. Goudge said. ``Personal satisfaction of
knowing that the science I'm doing is interesting and that people think
the site is interesting.''

The other two were the Columbia Hills region of Gusev Crater, previously
explored by NASA's Spirit rover, and Northeast Syrtis, an ancient
volcanic site that may have possessed hot springs and melted snow, a
potentially friendly environment for some microbes.

The science team later added a fourth finalist, which was given the
unofficial name of Midway because it was roughly midway between Jezero
and Northeast Syrtis. Midway possesses geology similar to that of
Northeast Syrtis, but it is close enough to Jezero that Perseverance, if
the rover lasts long enough, could explore both --- a twofer science
jackpot.

After a final workshop in October 2018, NASA made its decision a month
later.

Dr. Goudge, who did not receive advance notice of the winner, said he
shut the door to his University of Texas office, sat down to listen to
the announcement on his computer and ``tried to not be too jittery as I
was watching.''

Thomas Zurbuchen, NASA's associate administrator for science, did not
drag out the suspense. ``Hey everybody,'' he said, kicking off the news
conference. ``I selected Jezero crater as the landing site for NASA's
Mars 2020 rover mission.''

Dr. Goudge said he did not yell or jump in joy

``I probably like slumped in my chair, mostly,'' Dr. Goudge said.

A deluge of congratulatory emails and text messages started flooding in.

\hypertarget{practice-for-perseverance}{%
\subsection{Practice for Perseverance}\label{practice-for-perseverance}}

In February, as part of preparations for the mission, scientists
practiced how they would use Perseverance to search for signs of life on
Mars. Of course, the rover and its instruments were not on Mars and not
available; they were being packed up for their interplanetary journey.

Instead, Walker Lake, Nev., which partially dried up tens of thousands
ago, served as a stand-in for Jezero crater. Over two weeks, a team of
seven people played the part of the rover to gather photographs and
readings.

Each day, a team of 150 scientists --- some at NASA's Jet Propulsion
Laboratory in California, which will operate Perseverance, others
scattered around the world --- reviewed the previous day's images and
data and came up with a new set of science observations.

\hypertarget{perseverance}{%
\subsection{Perseverance}\label{perseverance}}

The NASA mission includes Perseverance, a 2,200-pound rover, and
Ingenuity, an experimental Mars helicopter.

Ingenuity Helicopter

The four-pound aircraft will communicate wirelessly with the
Perseverance rover.

Solar Panel

Blades

Four carbon-fiber blades will spin at about 2,400 r.p.m.

Power

The plutonium-based power supply will charge the rover's batteries.

MAST

Instruments will take videos, panoramas and photographs. A laser will
study the chemistry of Martian rocks.

PiXl

Will identify chemical elements to seek signs of past life on Mars.

Antenna

Will transmit data directly to Earth.

Robotic arm

A turret with many instruments is attached to a 7-foot robotic arm. A
drill will extract samples from Martian rocks. The Sherloc device will
identify molecules and minerals to detect potential biosignatures, with
help from the Watson camera.

Perseverance Rover

The 2,200 pound rover will explore Jezero Crater. It has aluminum wheels
and a suspension system to drive over obstacles.

Ingenuity Helicopter

The aircraft will communicate wirelessly with the rover.

Solar Panel

Blades

Power

The plutonium-based power supply will charge the rover's batteries.

MAST

Instruments will take videos, panoramas and photographs. A laser will
study the chemistry of Martian rocks.

PiXl

Will identify chemical elements to seek signs of past life on Mars.

Antenna

Robotic arm

A turret with many instruments is attached to a 7-foot robotic arm. A
drill will extract samples from Martian rocks. The Sherloc device will
identify molecules and minerals to detect potential biosignatures, with
help from the Watson camera.

Perseverance Rover

The 2,200 pound rover will explore Jezero Crater. It has aluminum wheels
and a suspension system to drive over obstacles.

Solar panel

Ingenuity Helicopter

Blades

Power

Mast

PIXL

Antenna

Suspension

Perseverance rover

Robotic arm

A turret with many instruments is attached to a 7-foot robotic arm. A
drill will extract samples from Martian rocks. The Sherloc device will
identify molecules and minerals to detect potential biosignatures, with
help from the Watson camera. PiXl will identify chemical elements to
seek signs of past life on Mars.

By Eleanor Lutz \textbar{} Source: NASA

If the side of a cliff looked intriguing, the scientists might decide
they wanted a closer look. Parts of Walker Lake are studded with
stromatolites --- bulbous structures that contain wavy patterns left
behind by microbial mats.

The off-site scientists sent instructions to Nevada, where the team of
seven set about to mimic the actions that the rover would undertake.

That included pressing a hand-held X-ray tool up to rock to generate
data similar to an instrument on Perseverance, and rolling around a
stroller-like contraption that held a ground-penetrating radar.

``Our job is to treat this like a movie set,'' said Raymond Francis, who
led the team at Walker Lake. ``And don't get anything in front of the
camera that shouldn't be in front of the camera. One of our most
important tools was a broom we bought at the dollar store that we used
to erase footprints.''

Image

Scientists simulated a Perseverance mission in the dry lakebed of Walker
Lake in Nevada in February 2020.Credit...NASA/JPL-Caltech

Image

Stromatolites found in the lakebed during the field exercise in
Nevada.Credit...NASA/JPL-Caltech

Dr. Francis said the remote teams of scientists did well identifying
rocks that deserved a closer look, including a feature --- little black
specks in several layers of sediment --- that he did not expect them to
notice. The distant researchers asked for measurements of the
composition. It was unusual --- high in phosphorous.

The black specks were old fish bones.

``So yeah, they didn't miss much,'' Dr. Francis said. ``If someone finds
fish bones or seashells in Jezero crater, you know that's going to bring
the mission to a screeching halt.''

\hypertarget{whats-in-a-name}{%
\subsection{What's In a Name?}\label{whats-in-a-name}}

One of the mysteries of Jezero has nothing to do with Mars. No one seems
to remember who picked the name Jezero.

Ralph P. Harvey, a professor of geological sciences at Case Western, was
among those who first pushed for the assignment of an official name when
he proposed the crater as a potential landing site for the earlier
Curiosity rover.

``I got tired of calling it `that crater in Nili Fossae,''' he said,
referring to the wider fractured region surrounding the crater.

He turned to the International Astronomical Union, which has conventions
for naming Martian craters. Those that are up to about 50 kilometers in
diameter, or 31 miles, are named after small towns with populations of
100,000 or less. That provides a large pool of potential names that are
generally not controversial. And while a small town that shares its name
with a crater may consider it an honor, the union does not intend the
designation as a commemoration.

Dr. Harvey's suggestions were Kennan after a town in Wisconsin, and
Novelty for one in Ohio.

Dr. Fassett suggested Tida, a town in Egypt, as a water-related pun.

Rita M. Schulz, who chairs the union's working group for planetary
system nomenclature, said her records indicated that the original
proposed name was Stolac, a small Bosnian town, but that ``was not
regarded as a safe choice,'' because of the destruction it suffered
during the war that ravaged the country in the 1990s.

The records did not preserve who had offered the suggestion of Stolac.
But Dr. Schulz said the replacement name of Jezero, another Bosnian
town, must have come from Bradford A. Smith, a planetary scientist who
was then chairman of a group that helped with assigning geographical
names on Mars. Dr. Smith died in 2018.

Neither Dr. Fassett nor Dr. Goudge are among the 375 or so members of
the Perseverance science team.

Dr. Fassett has largely moved on to other places in the solar system,
the Earth's moon in particular, and Dr. Goudge is at a relatively early
stage in his planetary science career. But both have submitted proposals
to work with the Perseverance mission as what NASA calls participating
scientists. Neither has yet heard whether they have been accepted.

Although Dr. Fassett has been excited by all of NASA's Mars missions, he
closely follows Perseverance, and Jezero will be special.

``All of them are great,'' he said, ``but my emotional connection to
this particular site is pretty unusual, right?''

\href{https://www.nytimes3xbfgragh.onion/interactive/2020/science/exploring-the-solar-system.html}{}

\includegraphics{https://static01.graylady3jvrrxbe.onion/images/2020/07/24/us/exploring-the-solar-system-promo-1595620746754/exploring-the-solar-system-promo-1595620746754-articleLarge.png}

\hypertarget{exploring-the-solar-system}{%
\subsection{Exploring the Solar
System}\label{exploring-the-solar-system}}

A guide to the spacecraft beyond Earth's orbit.

\href{https://www.nytimes3xbfgragh.onion/interactive/2020/science/2020-astronomy-space-calendar.html}{}

\includegraphics{https://static01.graylady3jvrrxbe.onion/images/2019/12/04/science/04SUN1/04SUN1-articleLarge.png}

\hypertarget{sync-your-calendar-with-the-solar-system}{%
\subsection{Sync your calendar with the solar
system}\label{sync-your-calendar-with-the-solar-system}}

Never miss an eclipse, a meteor shower, a rocket launch or any other
astronomical and space event that's out of this world.

Advertisement

\protect\hyperlink{after-bottom}{Continue reading the main story}

\hypertarget{site-index}{%
\subsection{Site Index}\label{site-index}}

\hypertarget{site-information-navigation}{%
\subsection{Site Information
Navigation}\label{site-information-navigation}}

\begin{itemize}
\tightlist
\item
  \href{https://help.nytimes3xbfgragh.onion/hc/en-us/articles/115014792127-Copyright-notice}{©~2020~The
  New York Times Company}
\end{itemize}

\begin{itemize}
\tightlist
\item
  \href{https://www.nytco.com/}{NYTCo}
\item
  \href{https://help.nytimes3xbfgragh.onion/hc/en-us/articles/115015385887-Contact-Us}{Contact
  Us}
\item
  \href{https://www.nytco.com/careers/}{Work with us}
\item
  \href{https://nytmediakit.com/}{Advertise}
\item
  \href{http://www.tbrandstudio.com/}{T Brand Studio}
\item
  \href{https://www.nytimes3xbfgragh.onion/privacy/cookie-policy\#how-do-i-manage-trackers}{Your
  Ad Choices}
\item
  \href{https://www.nytimes3xbfgragh.onion/privacy}{Privacy}
\item
  \href{https://help.nytimes3xbfgragh.onion/hc/en-us/articles/115014893428-Terms-of-service}{Terms
  of Service}
\item
  \href{https://help.nytimes3xbfgragh.onion/hc/en-us/articles/115014893968-Terms-of-sale}{Terms
  of Sale}
\item
  \href{https://spiderbites.nytimes3xbfgragh.onion}{Site Map}
\item
  \href{https://help.nytimes3xbfgragh.onion/hc/en-us}{Help}
\item
  \href{https://www.nytimes3xbfgragh.onion/subscription?campaignId=37WXW}{Subscriptions}
\end{itemize}
