Sections

SEARCH

\protect\hyperlink{site-content}{Skip to
content}\protect\hyperlink{site-index}{Skip to site index}

\href{https://www.nytimes3xbfgragh.onion/section/world/australia}{Australia}

\href{https://myaccount.nytimes3xbfgragh.onion/auth/login?response_type=cookie\&client_id=vi}{}

\href{https://www.nytimes3xbfgragh.onion/section/todayspaper}{Today's
Paper}

\href{/section/world/australia}{Australia}\textbar{}Australia Says
Chinese Students Are Targets in `Virtual Kidnapping' Scams

\url{https://nyti.ms/303NYvG}

\begin{itemize}
\item
\item
\item
\item
\item
\item
\end{itemize}

Advertisement

\protect\hyperlink{after-top}{Continue reading the main story}

Supported by

\protect\hyperlink{after-sponsor}{Continue reading the main story}

\hypertarget{australia-says-chinese-students-are-targets-in-virtual-kidnapping-scams}{%
\section{Australia Says Chinese Students Are Targets in `Virtual
Kidnapping'
Scams}\label{australia-says-chinese-students-are-targets-in-virtual-kidnapping-scams}}

Recent cases reveal the evolution of a crime that often exploits worry
over family members abroad with digital savvy and old-fashioned
coercion.

\includegraphics{https://static01.graylady3jvrrxbe.onion/images/2020/07/28/world/28oz-kidnappings-1/merlin_172408368_92b1bc93-97d6-4978-a0ad-cd06cd2b6909-articleLarge.jpg?quality=75\&auto=webp\&disable=upscale}

\href{https://www.nytimes3xbfgragh.onion/by/damien-cave}{\includegraphics{https://static01.graylady3jvrrxbe.onion/images/2018/10/08/multimedia/author-damien-cave/author-damien-cave-thumbLarge.png}}

By \href{https://www.nytimes3xbfgragh.onion/by/damien-cave}{Damien Cave}

\begin{itemize}
\item
  July 28, 2020
\item
  \begin{itemize}
  \item
  \item
  \item
  \item
  \item
  \item
  \end{itemize}
\end{itemize}

\href{https://cn.nytimes3xbfgragh.onion/asia-pacific/20200729/chinese-students-virtual-kidnapping/}{阅读简体中文版}\href{https://cn.nytimes3xbfgragh.onion/asia-pacific/20200729/chinese-students-virtual-kidnapping/zh-hant/}{閱讀繁體中文版}

SYDNEY, Australia --- The young woman's parents in China believed the
video was real. It seemed to show their 21-year-old daughter pleading
for help somewhere in Australia. She had been out of touch for days. She
looked to be in pain, and the perpetrators pointed to only one solution:
a six-figure ransom payment.

The woman's family deposited the money in an offshore bank account. But
it was all a scam. A few hours after the woman's housemate contacted the
police in Sydney on July 14, she was found safe and sound at a hotel,
where she had been lured by the scam artists.

Now, the Australian authorities are warning that ``virtual kidnappings''
could be on the rise as anonymous criminals seek to exploit Chinese
students in the country and their families back home, many of whom are
already on edge and isolated because of the coronavirus pandemic.

On Tuesday,
\href{https://www.facebookcorewwwi.onion/nswpoliceforce/posts/10158124537291185}{the
police in New South Wales} said there had been at least eight confirmed
cases this year, with more than \$2 million paid in ransom for
abductions that never happened.

``The victims of virtual kidnappings we have engaged are traumatized by
what has occurred, believing they have placed themselves, and their
loved ones, in real danger,'' said Peter Thurtell, the assistant
commissioner of the New South Wales police force.

The recent spree points to the evolution of a crime that exploits
oversharing and fear for a distant loved one with digital savvy and
old-fashioned coercion by con artists.
\href{https://theconversation.com/new-virtual-kidnapping-scam-targeting-chinese-students-makes-use-of-data-shared-online-96910}{Since
at least the 1990s}, criminal gangs from Taiwan and China to Mexico and
Cuba have been persuading families to pay ransom for simulated
kidnappings, often with personal information provided intentionally or
unintentionally by the victims.

Last year,
\href{https://www.fbi.gov/contact-us/field-offices/phoenix/news/press-releases/fbi-law-enforcement-partners-warn-of-new-twist-in-virtual-kidnapping-scams}{extortionists
called} hotel rooms on the American side of the U.S.-Mexico border and
convinced guests that armed enforcers were nearby, and that they needed
to drive across the border and switch to a Mexican hotel, where they had
to take a screenshot of themselves that the criminals then used to
persuade loved ones to pay a ransom.

In the Sydney form of the scam, which the authorities said they first
started seeing a few years ago, robocalls deliver messages to thousands
of random phones purporting to be from a messenger service. It says a
package needs to be delivered. Those who continue on the call are
greeted by someone speaking Mandarin who asks for basic identity
information --- name, address, phone number, anything else of import ---
and promises to call back.

For the Chinese students in Australia --- whose ranks have swelled in
recent years, with
\href{https://internationaleducation.gov.au/research/DataVisualisations/Pages/Student-number.aspx}{212,000
enrolled last year} --- the return calls have come from someone who
claims to be from the Chinese government, bearing bad news: The supposed
package to be delivered holds illegal contents or is somehow connected
to a larger crime that could get that person deported or imprisoned, or
get one of their relatives hurt. To be safe, the caller tells the mark,
the person must check into a hotel and turn off the phone. And, oh,
don't tell anyone or else what's already bad will become downright
horrific.

``Especially for Chinese students, here without any support from family,
they get scared when they get information like this,'' said Prof. Lennon
Chang, a senior lecturer in criminal justice at Monash University who
has studied the scam. ``The talented criminals understand this
psychological emotion and use it as a way to lead the students under the
pass.''

The scammers use technology to bolster the fraud. Professor Chang said
they usually mask where they call from, presenting a number from the
Chinese Embassy that can be found online. In some cases, they ask the
victim to send a photo, or alter what they find online to create an
image or video that seems to show the person kidnapped.

The parents, far away, usually receive the ransom demand by phone and
are then sent what appears to be evidence of a crime.

Worried about their children, perhaps after reading about actual
kidnappings of Chinese students
\href{https://edition.cnn.com/2019/03/26/asia/canada-chinese-student-kidnap-intl/index.html}{in
Canada} and \href{https://www.bbc.com/news/world-us-canada-48749420}{in
the United States}, some parents in China comply. In one case from
Sydney last month, a family paid 2 million Australian dollars (\$1.4
million) to the unknown criminals. In the other cases, payments ranged
from a few thousand dollars to more than \$200,000.

``During this period of time, with the pandemic and with less human
contact, the parents might not know who to contact if they get a message
like that, or for the student, they might not be able to talk to people
they trust to verify whether this kind of message is true,'' Professor
Chang said. ``This kind of isolation might create some opportunity for
criminals.''

When the police have been called, it has typically taken them only a few
hours to uncover what had really happened. But the names of victims have
been rarely publicized, and no masterminds have been identified.

On Tuesday, the Australian authorities reminded people to report anyone
they suspected of pretending to be from the Chinese government.

``NSW Police have been assured from the Chinese Consulate-General in
Sydney that no person claiming to be from a Chinese authority such as
police, procuratorates or the courts will contact a student on their
mobile phone and demand monies to be paid or transferred,'' said
Detective Chief Superintendent Darren Bennett.

``If this occurs,'' he added, ``it is a scam.''

Tiffany May contributed reporting from Hong Kong.

Advertisement

\protect\hyperlink{after-bottom}{Continue reading the main story}

\hypertarget{site-index}{%
\subsection{Site Index}\label{site-index}}

\hypertarget{site-information-navigation}{%
\subsection{Site Information
Navigation}\label{site-information-navigation}}

\begin{itemize}
\tightlist
\item
  \href{https://help.nytimes3xbfgragh.onion/hc/en-us/articles/115014792127-Copyright-notice}{©~2020~The
  New York Times Company}
\end{itemize}

\begin{itemize}
\tightlist
\item
  \href{https://www.nytco.com/}{NYTCo}
\item
  \href{https://help.nytimes3xbfgragh.onion/hc/en-us/articles/115015385887-Contact-Us}{Contact
  Us}
\item
  \href{https://www.nytco.com/careers/}{Work with us}
\item
  \href{https://nytmediakit.com/}{Advertise}
\item
  \href{http://www.tbrandstudio.com/}{T Brand Studio}
\item
  \href{https://www.nytimes3xbfgragh.onion/privacy/cookie-policy\#how-do-i-manage-trackers}{Your
  Ad Choices}
\item
  \href{https://www.nytimes3xbfgragh.onion/privacy}{Privacy}
\item
  \href{https://help.nytimes3xbfgragh.onion/hc/en-us/articles/115014893428-Terms-of-service}{Terms
  of Service}
\item
  \href{https://help.nytimes3xbfgragh.onion/hc/en-us/articles/115014893968-Terms-of-sale}{Terms
  of Sale}
\item
  \href{https://spiderbites.nytimes3xbfgragh.onion}{Site Map}
\item
  \href{https://help.nytimes3xbfgragh.onion/hc/en-us}{Help}
\item
  \href{https://www.nytimes3xbfgragh.onion/subscription?campaignId=37WXW}{Subscriptions}
\end{itemize}
