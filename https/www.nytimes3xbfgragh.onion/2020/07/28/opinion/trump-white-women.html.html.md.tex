Sections

SEARCH

\protect\hyperlink{site-content}{Skip to
content}\protect\hyperlink{site-index}{Skip to site index}

\href{https://myaccount.nytimes3xbfgragh.onion/auth/login?response_type=cookie\&client_id=vi}{}

\href{https://www.nytimes3xbfgragh.onion/section/todayspaper}{Today's
Paper}

\href{/section/opinion}{Opinion}\textbar{}Trump Is Dog-Whistling. Are
`Suburban Housewives' Listening?

\url{https://nyti.ms/30ZTM8M}

\begin{itemize}
\item
\item
\item
\item
\item
\item
\end{itemize}

Advertisement

\protect\hyperlink{after-top}{Continue reading the main story}

\href{/section/opinion}{Opinion}

Supported by

\protect\hyperlink{after-sponsor}{Continue reading the main story}

\hypertarget{trump-is-dog-whistling-are-suburban-housewives-listening}{%
\section{Trump Is Dog-Whistling. Are `Suburban Housewives'
Listening?}\label{trump-is-dog-whistling-are-suburban-housewives-listening}}

Or are they too busy organizing protests, posting links to bail funds
and discussing antidotes for tear gas?

\href{https://www.nytimes3xbfgragh.onion/by/jennifer-weiner}{\includegraphics{https://static01.graylady3jvrrxbe.onion/images/2018/05/11/opinion/jennifer-weiner/jennifer-weiner-thumbLarge.png}}

By \href{https://www.nytimes3xbfgragh.onion/by/jennifer-weiner}{Jennifer
Weiner}

Contributing Opinion writer

\begin{itemize}
\item
  July 28, 2020
\item
  \begin{itemize}
  \item
  \item
  \item
  \item
  \item
  \item
  \end{itemize}
\end{itemize}

\includegraphics{https://static01.graylady3jvrrxbe.onion/images/2020/07/28/opinion/28Weiner4/28Weiner4-articleLarge.jpg?quality=75\&auto=webp\&disable=upscale}

\emph{This article has been updated to reflect news developments.}

\href{https://artsandculture.google.com/asset/little-rock-integration/jwEdz9Na2Fy0tQ}{There
she is} in Little Rock, Ark., in 1957, a white woman in a neatly ironed
dress and a short-sleeved, pearl-button cardigan. Her hand is on her hip
as she stands in the universal posture of ``I would like to speak to
your manager'' and stares down a soldier defending the Black students
who'd integrated Little Rock Central High School.

\href{https://images.app.goo.gl/KRH9N5YB1oSNuEum8}{There she is,} in
Queens in 1959, a white woman in a dark skirt, a polka-dot blouse and
matching coat. Her hair is curled, her handbag is looped over her
forearm, and she carries a sign that reads ``Bussing Creates Fussing.''

In South Boston, in 1975, she
\href{https://www.flickr.com/photos/boston_public_library/6326362205}{carries
a sign} that reads ``Whites Have Rights.'' And just last month, in St.
Louis, she was spotted
\href{https://www.washingtonpost.com/nation/2020/07/20/st-louis-couple-who-aimed-guns-protesters-charged-with-felony-weapons-count/}{barefoot
on her front steps}, in black capri pants, brandishing a pistol at Black
Lives Matter protesters.

Donald Trump and his enablers know these women. They know that for as
long as there have been efforts toward desegregation, white women have
defended the status quo. These women have been loud and insistent, maybe
because those protests have long been one of the few places where anger
doesn't immediately render a woman unfeminine and unattractive, strident
or shrill. An angry white woman has always been a ``nasty woman''---
unless she's a mama bear, standing up for her kids and their schools and
neighborhoods.

Thus President Trump's recent provocative
\href{https://twitter.com/realDonaldTrump/status/1286372175117791236?s=20}{tweet}
to ``the Suburban Housewives of America,'' warning ``Biden will destroy
your neighborhood and your American Dream.''

The tweet linked to a column in The New York Post by former Lt. Gov.
Betsy McCaughey of New York, which gave dire warnings about an
Obama-Biden initiative called Affirmatively Furthering Fair Housing. Ms.
McCaughey wrote that it would require towns to ``make it possible for
low-income minorities to choose suburban living.'' Women, she added,
need to recognize the danger and ``focus on what's at stake for their
families.''

Then, on Wednesday, Mr. Trump did it again,
\href{https://twitter.com/realDonaldTrump/status/1288509568578777088}{tweeting}
``I am happy to inform all of the people living their Suburban Lifestyle
Dream that you will no longer be bothered or financially hurt by having
low income housing built in your neighborhood'' --- because he's
rescinding the housing initiative. ``Your housing prices will go up
based on the market, and crime will go down,'' he wrote, adding,
``Enjoy!''

These weren't even dog whistles. These were someone standing on the
porch and bellowing, ``Here, Fido!''

But are the ``Suburban Housewives'' listening?''

I grew up in the kind of place Mr. Trump and Ms. McCaughey probably
picture when they think about the suburbs. In the 1970s and '80s my
hometown, Simsbury, Conn., was an affluent community of 20,000. Today
it's
\href{https://www.census.gov/quickfacts/simsburytownhartfordcountyconnecticut}{more
than 90 percent} white; when I lived there, it was whiter.

The handful of nonwhite students came almost entirely from two places:
the ABC House, where boys from big cities lived in Simsbury while they
attended its high school (ABC stood for A Better Chance), and Hartford,
where kids came as part of an initiative called Project Concern, one of
the first voluntary school desegregation programs in the United States.
After it began, in 1966,
\href{https://www.courant.com/courant-250/moments-in-history/hc-xpm-2014-06-29-hc-250-project-concern-20140625-story.html\#:~:text=Glastonbury's\%20school\%20board\%20rejected\%20Project\%20Concern.\&text=One\%20Vernon\%20resident\%2C\%20who\%20spoke,this\%20with\%20anyone\%20from\%20Hartford.}{thousands
of residents in five Connecticut towns} crowded school board meetings to
argue against it.

In 1968, a resident of nearby Vernon was applauded after she told the
school board: ``Vernon is a nice, wonderful, middle-class town, and I do
not wish to share this with anyone from Hartford. What we have, we have
earned and want to keep. What is mine is mine.''

Mr. Trump's re-election depends, at least in part, on white suburban
women still feeling that way.

But if recent protests and nonfiction best-seller lists are any
indication, at least some ``housewives'' have arrived at a more nuanced
understanding of racial dynamics and have harnessed the potent, symbolic
power of white motherhood to advocate for change. Simsbury, for example,
now has a Facebook group called
\href{https://www.facebookcorewwwi.onion/groups/holdingthedooropen/?ref=share}{Holding
the Door Open}, ``for people within greater Simsbury CT who are open,
ready and willing to discuss, share and grow in their understanding of
racism, white privilege and exclusion in our community and within
America.''

On Facebook, Wall of Moms groups are popping up, not just in the big
cities where you'd expect to see them, but also in
\href{https://www.facebookcorewwwi.onion/groups/1170791863291062}{Montgomery
County and Bucks County} and
\href{https://www.facebookcorewwwi.onion/groups/3261681747203912}{Delaware
County} outside of Philadelphia. There are Walls of Moms in Maine and
New Mexico and Michigan.

\includegraphics{https://static01.graylady3jvrrxbe.onion/images/2020/07/28/opinion/28Weiner1/merlin_174894195_22dd2b9f-b296-4dc5-a75c-0057f34689df-articleLarge.jpg?quality=75\&auto=webp\&disable=upscale}

The moms are organizing protests and reading groups, posting links to
bail funds, discussing antidotes for tear gas. They're starting groups
in their kids' schools to talk about white privilege and how to continue
the fight once the current wave of protests is done. Many feature some
version of the sign ``When George Floyd Called for His Mama, He Summoned
All the Mamas.''

What led to the change?

For Rene Daguerre-Bradford, who organized one of Simsbury's two Black
Lives Matter protests, it was seeing the video of George Floyd's death.
``When I saw how George Floyd died --- when I heard him begging them to
stop because he couldn't breathe --- I broke out into tears. How can any
human being treat someone like that?'' she told me.

To some, white mothers on the front lines looks like a case of too
little, too late. In The Washington Post, the president of the Portland,
Ore., N.A.A.C.P. branch wrote ``the `Wall of Moms', while perhaps
well-intentioned, ends up redirecting attention away from the urgent
issue of murdered Black bodies.''

Still, Donald Trump has to be uneasy as he considers the white women who
have woken up --- however belatedly --- to the reality of the moment,
women who have the tools and, thanks to the pandemic, the time to do the
work. As one mom told me, ``We're a bunch of mad white ladies with
nothing but time.''

\emph{The Times is committed to publishing}
\href{https://www.nytimes3xbfgragh.onion/2019/01/31/opinion/letters/letters-to-editor-new-york-times-women.html}{\emph{a
diversity of letters}} \emph{to the editor. We'd like to hear what you
think about this or any of our articles. Here are some}
\href{https://help.nytimes3xbfgragh.onion/hc/en-us/articles/115014925288-How-to-submit-a-letter-to-the-editor}{\emph{tips}}\emph{.
And here's our email:}
\href{mailto:letters@NYTimes.com}{\emph{letters@NYTimes.com}}\emph{.}

\emph{Follow The New York Times Opinion section on}
\href{https://www.facebookcorewwwi.onion/nytopinion}{\emph{Facebook}}\emph{,}
\href{http://twitter.com/NYTOpinion}{\emph{Twitter (@NYTopinion)}}
\emph{and}
\href{https://www.instagram.com/nytopinion/}{\emph{Instagram}}\emph{.}

Advertisement

\protect\hyperlink{after-bottom}{Continue reading the main story}

\hypertarget{site-index}{%
\subsection{Site Index}\label{site-index}}

\hypertarget{site-information-navigation}{%
\subsection{Site Information
Navigation}\label{site-information-navigation}}

\begin{itemize}
\tightlist
\item
  \href{https://help.nytimes3xbfgragh.onion/hc/en-us/articles/115014792127-Copyright-notice}{©~2020~The
  New York Times Company}
\end{itemize}

\begin{itemize}
\tightlist
\item
  \href{https://www.nytco.com/}{NYTCo}
\item
  \href{https://help.nytimes3xbfgragh.onion/hc/en-us/articles/115015385887-Contact-Us}{Contact
  Us}
\item
  \href{https://www.nytco.com/careers/}{Work with us}
\item
  \href{https://nytmediakit.com/}{Advertise}
\item
  \href{http://www.tbrandstudio.com/}{T Brand Studio}
\item
  \href{https://www.nytimes3xbfgragh.onion/privacy/cookie-policy\#how-do-i-manage-trackers}{Your
  Ad Choices}
\item
  \href{https://www.nytimes3xbfgragh.onion/privacy}{Privacy}
\item
  \href{https://help.nytimes3xbfgragh.onion/hc/en-us/articles/115014893428-Terms-of-service}{Terms
  of Service}
\item
  \href{https://help.nytimes3xbfgragh.onion/hc/en-us/articles/115014893968-Terms-of-sale}{Terms
  of Sale}
\item
  \href{https://spiderbites.nytimes3xbfgragh.onion}{Site Map}
\item
  \href{https://help.nytimes3xbfgragh.onion/hc/en-us}{Help}
\item
  \href{https://www.nytimes3xbfgragh.onion/subscription?campaignId=37WXW}{Subscriptions}
\end{itemize}
