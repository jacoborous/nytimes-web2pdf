Sections

SEARCH

\protect\hyperlink{site-content}{Skip to
content}\protect\hyperlink{site-index}{Skip to site index}

\href{https://www.nytimes3xbfgragh.onion/section/health}{Health}

\href{https://myaccount.nytimes3xbfgragh.onion/auth/login?response_type=cookie\&client_id=vi}{}

\href{https://www.nytimes3xbfgragh.onion/section/todayspaper}{Today's
Paper}

\href{/section/health}{Health}\textbar{}`Amazing, Isn't It?' Long-Sought
Blood Test for Alzheimer's in Reach

\url{https://nyti.ms/30TUyEl}

\begin{itemize}
\item
\item
\item
\item
\item
\item
\end{itemize}

Advertisement

\protect\hyperlink{after-top}{Continue reading the main story}

Supported by

\protect\hyperlink{after-sponsor}{Continue reading the main story}

\hypertarget{amazing-isnt-it-long-sought-blood-test-for-alzheimers-in-reach}{%
\section{`Amazing, Isn't It?' Long-Sought Blood Test for Alzheimer's in
Reach}\label{amazing-isnt-it-long-sought-blood-test-for-alzheimers-in-reach}}

Scientists say such tests could be available in a few years, speeding
research for treatments and providing a diagnosis for dementia patients
who want to know if they have Alzheimer's disease.

\includegraphics{https://static01.graylady3jvrrxbe.onion/images/2020/07/28/science/28ALZHEIMERS/merlin_175031418_8e9f1c00-577f-4717-aec9-e3c1d5a53dfd-articleLarge.jpg?quality=75\&auto=webp\&disable=upscale}

\href{https://www.nytimes3xbfgragh.onion/by/pam-belluck}{\includegraphics{https://static01.graylady3jvrrxbe.onion/images/2018/02/16/multimedia/author-pam-belluck/author-pam-belluck-thumbLarge-v2.png}}

By \href{https://www.nytimes3xbfgragh.onion/by/pam-belluck}{Pam Belluck}

\begin{itemize}
\item
  July 28, 2020
\item
  \begin{itemize}
  \item
  \item
  \item
  \item
  \item
  \item
  \end{itemize}
\end{itemize}

A newly developed blood test for Alzheimer's has diagnosed the disease
as accurately as methods that are far more expensive or invasive,
scientists reported on Tuesday, a significant step toward a longtime
goal for patients, doctors and dementia researchers. The test has the
potential to make diagnosis simpler, more affordable and widely
available.

The test determined whether people with dementia had Alzheimer's instead
of another condition. And it identified signs of the degenerative,
deadly disease 20 years before memory and thinking problems were
expected in people with a genetic mutation that causes Alzheimer's,
according to
\href{https://jamanetwork.com/journals/jama/fullarticle/10.1001/jama.2020.12134?guestAccessKey=42d098cb-7eca-4a1c-9d7b-9951b104b003\&utm_source=For_The_Media\&utm_medium=referral\&utm_campaign=ftm_links\&utm_content=tfl\&utm_term=072820}{research
published in JAMA} and presented at the Alzheimer's Association
International Conference.

Such a test could be available for clinical use in as little as two to
three years, the researchers and other experts estimated, providing a
readily accessible way to diagnose whether people with cognitive issues
were experiencing Alzheimer's, rather than another type of dementia that
might require different treatment or have a different prognosis. A blood
test like this might also eventually be used to predict whether someone
with no symptoms would develop Alzheimer's.

``This blood test very, very accurately predicts who's got Alzheimer's
disease in their brain, including people who seem to be normal,'' said
Dr. Michael Weiner, an Alzheimer's disease researcher at the University
of California, San Francisco, who was not involved in the study. ``It's
not a cure, it's not a treatment, but you can't treat the disease
without being able to diagnose it. And accurate, low-cost diagnosis is
really exciting, so it's a breakthrough.''

Nearly
\href{https://www.alz.org/media/documents/alzheimers-facts-and-figures-2019-r.pdf}{six
million people in the United States} and roughly
\href{https://www.ncbi.nlm.nih.gov/pmc/articles/PMC6936673/}{30 million
worldwide} have Alzheimer's, and their ranks are expected to more than
double by 2050 as the population ages.

Blood tests for Alzheimer's, which are being developed by several
research teams, would provide some hope in a field that has experienced
failure after failure in its search for ways to treat and prevent a
devastating disease that robs people of their memories and ability to
function independently.

Experts said blood tests would accelerate the search for new therapies
by making it faster and cheaper to screen participants for clinical
trials, a process that now often takes years and costs millions of
dollars because it relies on expensive methods like PET scans of the
brain and spinal taps for cerebrospinal fluid.

But the ability to diagnose Alzheimer's with a quick blood test would
also intensify ethical and emotional dilemmas for people deciding
whether they wanted to know they had a disease that does not yet have a
cure or treatment.

The test, which measures a form of the tau protein found in tangles that
spread throughout the brain in Alzheimer's, proved remarkably accurate
in a study of 1,402 people from three different groups in Sweden,
Colombia and the United States. It performed better than MRI brain
scans, was as good as PET scans or spinal taps and was nearly as
accurate as the most definitive diagnostic method: autopsies that found
strong evidence of Alzheimer's in people's brains after they died.

``Based on the data, it's a big step forward,'' said Rudolph Tanzi, a
professor of neurology at Massachusetts General Hospital and Harvard
Medical School, who was not involved in the research.

He and other experts said that the results would need to be replicated
in clinical trials in more populations, including those reflecting more
racial and ethnic diversity. The test will also need to be refined and
standardized so results can consistently be analyzed in labs, and will
need approval by federal regulators.

Currently, Alzheimer's diagnoses are made mostly with clinical
assessments of memory and cognitive impairment, as well as interviews
with patients' family members and caregivers. The diagnoses are often
inaccurate because doctors have trouble distinguishing Alzheimer's from
other dementias and physical conditions that involve cognitive
impairment.

Measures like PET scans and spinal taps --- costly and often unavailable
--- can detect elevated levels of amyloid protein, which clumps into
plaques in the brains of people with Alzheimer's, and there has been
recent progress on blood tests for amyloid. But amyloid alone isn't
enough to diagnose Alzheimer's because some people with high levels
don't develop the disease.

``Just saying you have amyloid in the brain through a PET scan today
does not tell you they have tau, and that's why it is not a diagnostic
for Alzheimer's,'' said Maria Carrillo, chief science officer at the
Alzheimer's Association. By contrast, the tau blood test appears to
register the presence of amyloid plaques and tau tangles, both of which
are in brains of people with confirmed Alzheimer's, she said.

``This test really opens up the possibility of being able to use a blood
test in the clinic to diagnose someone more definitely with
Alzheimer's,'' Dr. Carrillo said. ``Amazing, isn't it? I mean, really,
five years ago, I would have told you it was science fiction.''

Detecting tau may also be valuable for predicting how quickly a person's
cognitive abilities will decline, because, unlike amyloid, tau tends to
increase as dementia worsens, she said.

\includegraphics{https://static01.graylady3jvrrxbe.onion/images/2020/07/28/science/28ALZHEIMERS2/28ALZHEIMERS2-articleLarge.jpg?quality=75\&auto=webp\&disable=upscale}

The test was 96 percent accurate in determining whether people with
dementia had Alzheimer's rather than other neurodegenerative disorders,
said Dr. Oskar Hansson, a senior author of the study and a professor of
clinical memory research at Lund University in Sweden. That performance,
in a group of nearly 700 people from Sweden, was similar to PET scans
and spinal taps, and it was better than MRI scans and blood tests for
amyloid, another form of tau and a third type of neurological biomarker
called neurofilament light chain.

People with Alzheimer's had seven times more of the tau protein, called
p-tau217, that the test measured than people without any dementia or
those with other neurological disorders, like frontotemporal dementia,
vascular dementia or Parkinson's disease, Dr. Hansson said.

``This is so specific for Alzheimer's disease,'' he said.

The study also compared findings of brain autopsies of donors from
Arizona with test results on blood that they donated before they died.
It found the blood test was 98 percent as accurate in diagnosing
Alzheimer's as autopsies of people found to have had a high likelihood
of the disease because they had both amyloid plaques and extensive tau
tangles in their brains, said Dr. Eric Reiman, another senior author and
the executive director of the Banner Alzheimer's Institute in Phoenix.
The test was 89 percent as accurate as autopsies of brains that
contained plaques but had fewer tau tangles and were considered
moderately likely to have had Alzheimer's, he said.

And in over 600 members of
\href{https://www.nytimes3xbfgragh.onion/2010/06/02/health/02alzheimers.html}{the
world's largest family with genetic early-onset Alzheimer's}, the test
essentially identified who would develop the disease 20 years before
dementia symptoms would surface. In this extended family in Colombia of
about 6,000 people, some have a mutation that causes cognitive
impairment beginning in their mid-40s. The test could distinguish
between those with and without the mutation in people as young as 25.

The most immediate uses of blood tests would be to speed up and lower
the cost of clinical trials and to allow doctors to diagnose or rule out
Alzheimer's in patients with dementia if they and their families sought
that information to help them plan for what lay ahead.

``The certainty of a diagnosis could help patients, family caregivers
and physicians themselves cope,'' Dr. Reiman said.

Blood tests could eventually be used earlier, allowing people who were
beginning to have mild memory issues to learn whether they would develop
Alzheimer's or instead had another condition that might be less
aggressive or fast-moving, Dr. Weiner said.

And, Dr. Tanzi said, in the future blood tests might be given to people
without any impairment, perhaps as initial screening tools to be
followed with PET scans if worrisome levels of biomarkers were detected.

``It has the promise to make early detection of the disease possible,
before we have symptoms,'' Dr. Tanzi said, something the field would
only recommend for clinical use if there were effective ways to prevent
or treat Alzheimer's.

Dr. Hansson said his lab was studying whether the test could predict
dementia in people with no impairments or those with mild memory
problems.

The test in the JAMA study used a method called an immunoassay to detect
compounds that bind to antibodies. Several such assays are being
developed. The particular assay in the study was developed by Eli Lilly
and Company, which provided materials and three employees to conduct the
assays; the company was allowed to review the manuscript but not veto
anything in it, the authors reported. Most of the funding for the study
came from government agencies and foundations in Sweden and the United
States.

At the Alzheimer's Association conference, Dr. Hansson and a co-author,
Dr. Kaj Blennow, presented their findings, as did two other research
teams working on tau blood tests.

One test, developed by a team at Washington University in St. Louis that
included Dr. Randall Bateman, Dr. Suzanne Schindler and Nicolas
Barthélemy, used a method called mass spectrometry, which detects entire
molecules of tau or amyloid. In a
\href{https://rupress.org/jem/article-lookup/doi/10.1084/jem.20200861}{study
published} on Tuesday in the Journal of Experimental Medicine, that team
found that the same form of tau in the JAMA study, p-tau217, correlated
more closely to amyloid buildup in the brain than another form, p-tau
181, that some researchers have been focusing on. Dr. Schindler, an
assistant professor of neurology, said that might be because p-tau217
emerges earlier in the Alzheimer's disease process.

``I personally find it very reassuring that these different groups are
using different types of assays and getting the same result,'' Dr.
Schindler said. ``It looks real. It looks like 217 has tremendous
promise as a blood test for Alzheimer's disease, and it is likely to
correspond with the symptoms.''

In another study presented at the conference, Dr. Adam Boxer, a
neurologist at U.C.S.F., and Elisabeth Thijssen, a visiting graduate
student, used the same immunoassay in the JAMA study and found both
forms of tau could distinguish Alzheimer's from
\href{https://www.nytimes3xbfgragh.onion/2012/05/06/health/a-rare-form-of-dementia-tests-a-vow-of-for-better-for-worse.html}{frontotemporal
dementia}, showing how specific these proteins are for detecting tau
associated with Alzheimer's, Dr. Boxer said.

Several researchers are working with companies or, like Dr. Bateman and
Dr. Reiman, have formed their own. Ultimately, various methods may be
approved for medical use. Dr. Carrillo said mass spectrometry had the
advantage of relying on a machine that was already in use, but the
disadvantage of being more expensive and requiring more expertise than
immunoassays, which are easily analyzed by laboratories that routinely
run blood tests.

``Within a few years, it's very possible that there will be certified
laboratory tests for these proteins and others, and maybe tests will be
developed for Parkinson's disease and so forth,'' Dr. Weiner said.
``It's a new world.''

\textbf{\emph{{[}}\href{http://on.fb.me/1paTQ1h}{\emph{Like the Science
Times page on Facebook.}}} ****** \emph{\textbar{} Sign up for the}
\textbf{\href{http://nyti.ms/1MbHaRU}{\emph{Science Times
newsletter.}}\emph{{]}}}

Advertisement

\protect\hyperlink{after-bottom}{Continue reading the main story}

\hypertarget{site-index}{%
\subsection{Site Index}\label{site-index}}

\hypertarget{site-information-navigation}{%
\subsection{Site Information
Navigation}\label{site-information-navigation}}

\begin{itemize}
\tightlist
\item
  \href{https://help.nytimes3xbfgragh.onion/hc/en-us/articles/115014792127-Copyright-notice}{©~2020~The
  New York Times Company}
\end{itemize}

\begin{itemize}
\tightlist
\item
  \href{https://www.nytco.com/}{NYTCo}
\item
  \href{https://help.nytimes3xbfgragh.onion/hc/en-us/articles/115015385887-Contact-Us}{Contact
  Us}
\item
  \href{https://www.nytco.com/careers/}{Work with us}
\item
  \href{https://nytmediakit.com/}{Advertise}
\item
  \href{http://www.tbrandstudio.com/}{T Brand Studio}
\item
  \href{https://www.nytimes3xbfgragh.onion/privacy/cookie-policy\#how-do-i-manage-trackers}{Your
  Ad Choices}
\item
  \href{https://www.nytimes3xbfgragh.onion/privacy}{Privacy}
\item
  \href{https://help.nytimes3xbfgragh.onion/hc/en-us/articles/115014893428-Terms-of-service}{Terms
  of Service}
\item
  \href{https://help.nytimes3xbfgragh.onion/hc/en-us/articles/115014893968-Terms-of-sale}{Terms
  of Sale}
\item
  \href{https://spiderbites.nytimes3xbfgragh.onion}{Site Map}
\item
  \href{https://help.nytimes3xbfgragh.onion/hc/en-us}{Help}
\item
  \href{https://www.nytimes3xbfgragh.onion/subscription?campaignId=37WXW}{Subscriptions}
\end{itemize}
