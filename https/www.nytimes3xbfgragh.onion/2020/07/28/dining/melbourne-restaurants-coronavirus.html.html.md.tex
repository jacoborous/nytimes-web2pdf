Sections

SEARCH

\protect\hyperlink{site-content}{Skip to
content}\protect\hyperlink{site-index}{Skip to site index}

\href{https://www.nytimes3xbfgragh.onion/section/food}{Food}

\href{https://myaccount.nytimes3xbfgragh.onion/auth/login?response_type=cookie\&client_id=vi}{}

\href{https://www.nytimes3xbfgragh.onion/section/todayspaper}{Today's
Paper}

\href{/section/food}{Food}\textbar{}The Pandemic Could End the Age of
Midpriced Dining

\url{https://nyti.ms/2X2pbqb}

\begin{itemize}
\item
\item
\item
\item
\item
\item
\end{itemize}

\href{https://www.nytimes3xbfgragh.onion/spotlight/at-home?action=click\&pgtype=Article\&state=default\&region=TOP_BANNER\&context=at_home_menu}{At
Home}

\begin{itemize}
\tightlist
\item
  \href{https://www.nytimes3xbfgragh.onion/2020/08/03/well/family/the-benefits-of-talking-to-strangers.html?action=click\&pgtype=Article\&state=default\&region=TOP_BANNER\&context=at_home_menu}{Talk:
  To Strangers}
\item
  \href{https://www.nytimes3xbfgragh.onion/2020/08/01/at-home/coronavirus-make-pizza-on-a-grill.html?action=click\&pgtype=Article\&state=default\&region=TOP_BANNER\&context=at_home_menu}{Make:
  Grilled Pizza}
\item
  \href{https://www.nytimes3xbfgragh.onion/2020/07/31/arts/television/goldbergs-abc-stream.html?action=click\&pgtype=Article\&state=default\&region=TOP_BANNER\&context=at_home_menu}{Watch:
  'The Goldbergs'}
\item
  \href{https://www.nytimes3xbfgragh.onion/interactive/2020/at-home/even-more-reporters-editors-diaries-lists-recommendations.html?action=click\&pgtype=Article\&state=default\&region=TOP_BANNER\&context=at_home_menu}{Explore:
  Reporters' Google Docs}
\end{itemize}

Advertisement

\protect\hyperlink{after-top}{Continue reading the main story}

Supported by

\protect\hyperlink{after-sponsor}{Continue reading the main story}

Critic's Notebook

\hypertarget{the-pandemic-could-end-the-age-of-midpriced-dining}{%
\section{The Pandemic Could End the Age of Midpriced
Dining}\label{the-pandemic-could-end-the-age-of-midpriced-dining}}

When Melbourne restaurants reopened after lockdown, owners got creative,
and dinner got far more expensive.

\includegraphics{https://static01.graylady3jvrrxbe.onion/images/2020/07/29/dining/28fare1/merlin_174758739_a4877efc-01ca-4572-83a6-9213dc8aab01-articleLarge.jpg?quality=75\&auto=webp\&disable=upscale}

By Besha Rodell

\begin{itemize}
\item
  Published July 28, 2020Updated Aug. 4, 2020
\item
  \begin{itemize}
  \item
  \item
  \item
  \item
  \item
  \item
  \end{itemize}
\end{itemize}

MELBOURNE, Australia --- When Victor Liong reopened his restaurant on
AC/DC Lane in
\href{https://www.nytimes3xbfgragh.onion/2020/08/04/world/australia/coronavirus-melbourne-lockdown.html}{Melbourne}'s
city center in June, after months of a coronavirus shutdown order, he
carefully considered his options.

Since its opening in 2013, \href{https://www.leehofook.com.au/}{Lee Ho
Fook} had been a restaurant that could cater to just about any occasion.
You could stop in at the small bar tucked into the ground floor for
chicken cracklings and a cocktail. Or you could head upstairs to the
dining room, where your options ran from modern Chinese small plates to
a grand feast of a whole roast duck served with a star anise and
cinnamon sauce.

But with government distancing restrictions and a precarious financial
situation, the calculation of doing business was not the same as it had
been before the pandemic. In fact, the closing compelled Mr. Liong to
reassess everything.

What he landed on was totally different from the casual excellence for
which he'd been known. While still offering a robust to-go menu of
fan-favorite dishes, Lee Ho Fook became a tasting-menu restaurant where
the price of admission is \$160 per person.

\includegraphics{https://static01.graylady3jvrrxbe.onion/images/2020/07/29/dining/28fare2/merlin_174758766_aaa94772-6acd-4667-8022-48838fc34167-articleLarge.jpg?quality=75\&auto=webp\&disable=upscale}

``The mechanics of a tasting menu ensures a financial position that we
can plan for,'' Mr. Liong wrote in an email. ``I never wanted to create
such a restaurant, but I feel the previous model wasn't exactly a
winning model.''

Mr. Liong is
\href{https://www.nytimes3xbfgragh.onion/2020/07/28/dining/ever-chicago-restaurant-coronavirus.html}{not
alone} in his belief that this is the most viable way forward.
Restaurant owners are desperately looking for a lifeline amid the
limitations of takeout, delivery and spaced-out tables.

It was not long ago that eating out in a nice restaurant was widely
derided as a pompous activity of the very wealthy. I absorbed this
through pop culture, like the cartoons in my parents' copies of The New
Yorker, where the waiter and tablecloth provided immediate,
class-conscious context. Back then, good food made by well-known chefs
was expensive, available only to those with plenty of disposable income.
It was considered inherently elitist.

People can, and do, debate endlessly about which factors over the last
two decades have given restaurants their global cultural relevance,
morphing from an indulgence for the rich into a shared obsession across
many demographics. I'd argue that the advent of casual, creative,
high-quality dining is what brought more food fanatics into the fold.

But even prepandemic, restaurants like Lee Ho Fook were only just
scraping by. Profit margins were minuscule; any small disaster could
sabotage years of work and a lifetime of literal and creative capital.

Melbourne's food scene thrives primarily thanks to the casual gastronomy
found in its cafes, pubs and
\href{https://www.nytimes3xbfgragh.onion/2019/10/30/dining/melbourne-wine-bars-restaurants.html}{wine
bars}. But I saw a distinct trend in the opposite direction as the
restaurant industry emerged battered from months of closings. And this
wealthy, creative, diverse city --- with access to all kinds of fresh
food --- could be a bellwether for other cities around the world.

Melbourne had only a few glorious weeks of eating out before rising
coronavirus numbers put the city
\href{https://www.nytimes3xbfgragh.onion/2020/07/10/world/australia/melbourne-lockdown.html}{back
into lockdown on July 8}, forcing restaurants and bars to return to
to-go service or close altogether. Australia's virus numbers are still
relatively tiny, everyone has access to testing and health care, and the
latest shutdown orders came after a record number of cases were detected
on one day: 191 in a state of 6.3 million people.

Image

Hot-and-sour Murray cod with green tea, fermented chile and herbs at Lee
Ho Fook.Credit...Kristoffer Paulsen for The New York Times

During the weeks between lockdowns, dining out here looked and felt very
different from its pre-coronavirus incarnation. And it was far more
expensive.

I ate out voraciously and often, reveling in food I didn't have to cook,
dishes I didn't have to wash and the friendly faces of people I hadn't
been cooped up with for months on end.

I have to admit that I wasn't prepared for the meal I ate at Lee Ho Fook
during our brief and wonderful respite from lockdown. My husband and I
booked a table without knowing about the change in format, and what was
meant to be a low-key Wednesday night dinner ended up costing us over
\$400.

Image

Dry-aged slow-roasted duck as part of Lee Ho Fook's \$160-per-person
tasting menu.Credit...Kristoffer Paulsen for The New York Times

It was also one of the best meals I've eaten in Melbourne. It showcased
Mr. Liong's talent in a way I'd been unable to grasp when he ran a much
more casual restaurant. The pacing was beautiful, the progression of
dishes flawless.

Now that we're back in lockdown, my memories of that meal are helping me
get through: noodles dressed in housemade XO sauce topped with a
glistening, raw scarlet prawn and its roe; hot-and-sour Murray cod with
fermented chile and drifts of herbs; a few slices of dry-aged
slow-roasted duck with taro and caramelized onion soy rice. It was
perfect.

Mr. Liong's team even sent us home with a tiny gift bag, as is often the
custom in expensive tasting-menu restaurants, with spiced macadamia nuts
and a small bottle of bespoke hand sanitizer.

Image

The gift bag sent home with guests included spiced macadamia nuts and a
bottle of hand sanitizer.Credit...Kristoffer Paulsen for The New York
Times

At a favorite neighborhood wine bar,
\href{https://littleandorra.com.au/}{Little Andorra}, where I once would
stop by the bar for a glass of Croatian wine and a plate of cured
kingfish with smoked butter and basil, I now had to book and pay in
advance for a full-course meal. At \$60 per person for five courses and
bread (wine was extra), it was an absolute bargain, but this visit
fulfilled a vastly different role in my social and financial life than
Little Andorra has in the past.

At the cocktail bar \href{https://www.blackpearlbar.com.au/}{Black
Pearl}, the upstairs room was transformed on Saturday nights to
accommodate sit-down diners. I paid \$120 in advance for a fantastic
meal cooked by the owners of \href{https://www.tipo00.com.au/}{Tipo 00},
one of the city's best Italian restaurants, along with an included
series of spritzes made by the Black Pearl team.

It was a lovely example of the adaptability and creativity of the
industry --- a night that guaranteed social-distancing measures and
provided a far more predictable stream of revenue than either business
might achieve otherwise.

I've been thinking a lot about
\href{https://www.goodfood.com.au/eat-out/news/adam-liaws-forecast-for-fine-dining-in-the-future-20200603-h1oibb}{a
recent article in Good Food by the Australian chef Adam Liaw}, which
foresees a future of dining that is built on this type of collaboration
and added value. It's already happening here:
\href{https://www.attica.com.au/}{Attica}, Melbourne's most famous
restaurant, has turned into a bakery, home-delivery service, soup
kitchen and T-shirt company, among other things.

But the fictional guest in Mr. Liaw's article --- who buys tickets to a
prepaid dinner, adds takeout specialty products to his bill, orders food
packages for family members and pays for future dinner reservations ---
spends somewhere in the vicinity of \$1,000 in one night. Are there
enough diners with that kind of disposable income to support an industry
built on high prices?

Restaurants around the world are facing multiple reckonings: how to
remain open in the most economically challenging era of our lifetime;
how to support and protect workers who are often among the most
vulnerable in a society; how to move forward with a business model that
actually makes sense.

In Melbourne, at least, I'm seeing more and more owners decide that food
should be served either quickly and casually and cheaply, or in a format
that is lengthy and expensive. The middle ground is falling away.

Perhaps the dream of excellent midpriced dining was just that, a dream.

\emph{Follow} \href{https://twitter.com/nytfood}{\emph{NYT Food on
Twitter}} \emph{and}
\href{https://www.instagram.com/nytcooking/}{\emph{NYT Cooking on
Instagram}}\emph{,}
\href{https://www.facebookcorewwwi.onion/nytcooking/}{\emph{Facebook}}\emph{,}
\href{https://www.youtube.com/nytcooking}{\emph{YouTube}} \emph{and}
\href{https://www.pinterest.com/nytcooking/}{\emph{Pinterest}}\emph{.}
\href{https://www.nytimes3xbfgragh.onion/newsletters/cooking}{\emph{Get
regular updates from NYT Cooking, with recipe suggestions, cooking tips
and shopping advice}}\emph{.}

Advertisement

\protect\hyperlink{after-bottom}{Continue reading the main story}

\hypertarget{site-index}{%
\subsection{Site Index}\label{site-index}}

\hypertarget{site-information-navigation}{%
\subsection{Site Information
Navigation}\label{site-information-navigation}}

\begin{itemize}
\tightlist
\item
  \href{https://help.nytimes3xbfgragh.onion/hc/en-us/articles/115014792127-Copyright-notice}{©~2020~The
  New York Times Company}
\end{itemize}

\begin{itemize}
\tightlist
\item
  \href{https://www.nytco.com/}{NYTCo}
\item
  \href{https://help.nytimes3xbfgragh.onion/hc/en-us/articles/115015385887-Contact-Us}{Contact
  Us}
\item
  \href{https://www.nytco.com/careers/}{Work with us}
\item
  \href{https://nytmediakit.com/}{Advertise}
\item
  \href{http://www.tbrandstudio.com/}{T Brand Studio}
\item
  \href{https://www.nytimes3xbfgragh.onion/privacy/cookie-policy\#how-do-i-manage-trackers}{Your
  Ad Choices}
\item
  \href{https://www.nytimes3xbfgragh.onion/privacy}{Privacy}
\item
  \href{https://help.nytimes3xbfgragh.onion/hc/en-us/articles/115014893428-Terms-of-service}{Terms
  of Service}
\item
  \href{https://help.nytimes3xbfgragh.onion/hc/en-us/articles/115014893968-Terms-of-sale}{Terms
  of Sale}
\item
  \href{https://spiderbites.nytimes3xbfgragh.onion}{Site Map}
\item
  \href{https://help.nytimes3xbfgragh.onion/hc/en-us}{Help}
\item
  \href{https://www.nytimes3xbfgragh.onion/subscription?campaignId=37WXW}{Subscriptions}
\end{itemize}
