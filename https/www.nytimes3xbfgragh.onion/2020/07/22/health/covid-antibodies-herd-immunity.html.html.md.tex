Sections

SEARCH

\protect\hyperlink{site-content}{Skip to
content}\protect\hyperlink{site-index}{Skip to site index}

\href{https://www.nytimes3xbfgragh.onion/section/health}{Health}

\href{https://myaccount.nytimes3xbfgragh.onion/auth/login?response_type=cookie\&client_id=vi}{}

\href{https://www.nytimes3xbfgragh.onion/section/todayspaper}{Today's
Paper}

\href{/section/health}{Health}\textbar{}Can You Get Covid-19 Again? It's
Very Unlikely, Experts Say

\url{https://nyti.ms/2BloYGM}

\begin{itemize}
\item
\item
\item
\item
\item
\end{itemize}

\href{https://www.nytimes3xbfgragh.onion/news-event/coronavirus?action=click\&pgtype=Article\&state=default\&region=TOP_BANNER\&context=storylines_menu}{The
Coronavirus Outbreak}

\begin{itemize}
\tightlist
\item
  live\href{https://www.nytimes3xbfgragh.onion/2020/08/01/world/coronavirus-covid-19.html?action=click\&pgtype=Article\&state=default\&region=TOP_BANNER\&context=storylines_menu}{Latest
  Updates}
\item
  \href{https://www.nytimes3xbfgragh.onion/interactive/2020/us/coronavirus-us-cases.html?action=click\&pgtype=Article\&state=default\&region=TOP_BANNER\&context=storylines_menu}{Maps
  and Cases}
\item
  \href{https://www.nytimes3xbfgragh.onion/interactive/2020/science/coronavirus-vaccine-tracker.html?action=click\&pgtype=Article\&state=default\&region=TOP_BANNER\&context=storylines_menu}{Vaccine
  Tracker}
\item
  \href{https://www.nytimes3xbfgragh.onion/interactive/2020/07/29/us/schools-reopening-coronavirus.html?action=click\&pgtype=Article\&state=default\&region=TOP_BANNER\&context=storylines_menu}{What
  School May Look Like}
\item
  \href{https://www.nytimes3xbfgragh.onion/live/2020/07/31/business/stock-market-today-coronavirus?action=click\&pgtype=Article\&state=default\&region=TOP_BANNER\&context=storylines_menu}{Economy}
\end{itemize}

Advertisement

\protect\hyperlink{after-top}{Continue reading the main story}

Supported by

\protect\hyperlink{after-sponsor}{Continue reading the main story}

\hypertarget{can-you-get-covid-19-again-its-very-unlikely-experts-say}{%
\section{Can You Get Covid-19 Again? It's Very Unlikely, Experts
Say}\label{can-you-get-covid-19-again-its-very-unlikely-experts-say}}

Reports of reinfection instead may be cases of drawn-out illness. A
decline in antibodies is normal after a few weeks, and people are
protected from the coronavirus in other ways.

\includegraphics{https://static01.graylady3jvrrxbe.onion/images/2020/07/20/science/00VIRUS-REINFECTION1/merlin_174703059_d8cefca6-857f-481c-aa7f-6802d23fc7c0-articleLarge.jpg?quality=75\&auto=webp\&disable=upscale}

By
\href{https://www.nytimes3xbfgragh.onion/by/apoorva-mandavilli}{Apoorva
Mandavilli}

\begin{itemize}
\item
  July 22, 2020
\item
  \begin{itemize}
  \item
  \item
  \item
  \item
  \item
  \end{itemize}
\end{itemize}

\href{https://www.nytimes3xbfgragh.onion/es/2020/07/24/espanol/ciencia-y-tecnologia/reinfeccion-coronavirus.html}{Leer
en español}

The anecdotes are alarming. A woman in Los Angeles
\href{https://www.foxla.com/news/southern-california-woman-tests-positive-for-covid-19-for-second-time-after-initial-recovery}{seemed
to recover} from Covid-19, but weeks later took a turn for the worse and
tested positive again. A New Jersey doctor
\href{https://dailyvoice.com/new-jersey/monmouth/news/central-jersey-doctor-reports-patients-reinfected-with-coronavirus/790555/}{claimed}
several patients healed from one bout only to become reinfected with the
coronavirus. And another doctor said a second round of illness was a
reality for some people, and was much more severe.

These recent accounts tap into people's deepest anxieties that they are
destined to succumb to Covid-19 over and over, feeling progressively
sicker, and will never emerge from this nightmarish pandemic. And these
stories fuel fears that we won't be able to reach herd immunity --- the
ultimate destination where the virus can no longer find enough victims
to pose a deadly threat.

But the anecdotes are just that --- stories without evidence of
reinfections, according to nearly a dozen experts who study viruses. ``I
haven't heard of a case where it's been truly unambiguously
demonstrated,'' said Marc Lipsitch, an epidemiologist at the Harvard
T.H. Chan School of Public Health.

Other experts were even more reassuring. While little is definitively
known about the coronavirus, just seven months into the pandemic, the
new virus is behaving like most others, they said, lending credence to
the belief that herd immunity can be achieved with a vaccine.

It may be possible for the coronavirus to strike the same person twice,
but it's highly unlikely that it would do so in such a short window or
to make people sicker the second time, they said. What's more likely is
that some people have a drawn-out course of infection, with the virus
taking a slow toll weeks to months after their initial exposure.

People infected with the coronavirus typically
\href{https://www.nature.com/articles/s41586-020-2456-9}{produce} immune
molecules called antibodies. Several teams have recently reported that
the levels of these antibodies decline in
\href{https://www.medrxiv.org/content/10.1101/2020.07.09.20149633v1?\%253fcollection=}{two}
to \href{https://www.nature.com/articles/s41591-020-0965-6}{three
months}, causing some consternation. But a drop in antibodies is
perfectly normal after an acute infection subsides, said Dr. Michael
Mina, an immunologist at Harvard University.

Many clinicians are ``scratching their heads saying, `What an
extraordinarily odd virus that it's not leading to robust immunity,' but
they're totally wrong,''' Dr. Mina said. ``It doesn't get more textbook
than this.''

Antibodies are not the only form of protection against pathogens. The
coronavirus also provokes a
\href{https://www.biorxiv.org/content/10.1101/2020.06.29.174888v1}{vigorous}
\href{https://www.medrxiv.org/content/10.1101/2020.04.11.20062349v2?\%253fcollection=}{defense}
from
\href{https://www.medrxiv.org/content/10.1101/2020.05.13.20100636v1?\%253fcollection=}{immune}
cells that \href{https://pubmed.ncbi.nlm.nih.gov/32473127/}{can kill the
virus} and quickly rouse reinforcements for future battles. Less is
known about how long these so-called memory T cells persist --- those
that recognize other coronaviruses may linger for life --- but they can
buttress defenses against the new coronavirus.

``If those are maintained, and especially if they're maintained within
the lung and the respiratory tract, then I think they can do a pretty
good job of stopping an infection from spreading,'' said Akiko Iwasaki,
an immunologist at Yale University.

Megan Kent, 37, a medical speech pathologist who lives just outside
Boston, first tested positive for the virus on March 30, after her
boyfriend became ill. She couldn't smell or taste anything, she
recalled, but otherwise felt fine. After a 14-day quarantine, she went
back to work at Melrose Wakefield Hospital and also helped out at a
nursing home.

\hypertarget{latest-updates-global-coronavirus-outbreak}{%
\section{\texorpdfstring{\href{https://www.nytimes3xbfgragh.onion/2020/08/01/world/coronavirus-covid-19.html?action=click\&pgtype=Article\&state=default\&region=MAIN_CONTENT_1\&context=storylines_live_updates}{Latest
Updates: Global Coronavirus
Outbreak}}{Latest Updates: Global Coronavirus Outbreak}}\label{latest-updates-global-coronavirus-outbreak}}

Updated 2020-08-01T19:54:00.494Z

\begin{itemize}
\tightlist
\item
  \href{https://www.nytimes3xbfgragh.onion/2020/08/01/world/coronavirus-covid-19.html?action=click\&pgtype=Article\&state=default\&region=MAIN_CONTENT_1\&context=storylines_live_updates\#link-3ac56579}{Top
  officials work to break impasse over jobless benefit.}
\item
  \href{https://www.nytimes3xbfgragh.onion/2020/08/01/world/coronavirus-covid-19.html?action=click\&pgtype=Article\&state=default\&region=MAIN_CONTENT_1\&context=storylines_live_updates\#link-8796723}{The
  virus picks up dangerous speed in the Midwest, and in areas that had
  seen success.}
\item
  \href{https://www.nytimes3xbfgragh.onion/2020/08/01/world/coronavirus-covid-19.html?action=click\&pgtype=Article\&state=default\&region=MAIN_CONTENT_1\&context=storylines_live_updates\#link-25930521}{Thousands
  in Berlin protest Germany's coronavirus measures.}
\end{itemize}

\href{https://www.nytimes3xbfgragh.onion/2020/08/01/world/coronavirus-covid-19.html?action=click\&pgtype=Article\&state=default\&region=MAIN_CONTENT_1\&context=storylines_live_updates}{See
more updates}

More live coverage:
\href{https://www.nytimes3xbfgragh.onion/live/2020/07/31/business/stock-market-today-coronavirus?action=click\&pgtype=Article\&state=default\&region=MAIN_CONTENT_1\&context=storylines_live_updates}{Markets}

On May 8, Ms. Kent suddenly felt ill. ``I felt like a Mack truck hit
me,'' she said. She slept the whole weekend and went to the hospital on
Monday, convinced she had mononucleosis. The next day she tested
positive for the coronavirus --- again. She was unwell for nearly a
month, and has since learned she has antibodies.

``This time around was a hundred times worse,'' she said. ``Was I
reinfected?''

There are other, more plausible explanations for what Ms. Kent
experienced, experts said. ``I'm not saying it can't happen. But from
what I've seen so far, that would be an uncommon phenomenon,'' said Dr.
Peter Hotez, the dean of the National School of Tropical Medicine at
Baylor College of Medicine.

Ms. Kent may not have fully recovered, even though she felt better, for
example. The virus may have secreted itself into certain parts of the
body --- as the Ebola virus is known to do --- and then resurfaced. She
did not get tested between the two positives, but even if she had,
faulty tests and low viral levels can produce a false negative.

Given these more likely scenarios, Dr. Mina had choice words for the
physicians who caused the panic over reports of reinfections. ``This is
so bad, people have lost their minds,'' he said. ``It's just
sensationalist click bait.''

In the early weeks of the pandemic, some people in China, Japan and
South Korea tested positive twice,
\href{https://www.nytimes3xbfgragh.onion/2020/02/29/health/coronavirus-reinfection.html}{sparking
similar fears}.

South Korea's Centers for Disease Control and Prevention
\href{https://www.cdc.go.kr/board/board.es?mid=a30402000000\&bid=0030}{investigated
285 of those cases}, and found that several of the second positives came
two months after the first, and in one case 82 days later. Nearly half
of the people had symptoms at the second test. But the researchers were
unable to grow live virus from any of the samples, and the infected
people hadn't spread the virus to others.

``It was pretty solid epidemiological and virological evidence that
reinfection was not happening, at least in those people,'' said Angela
Rasmussen, a virologist at Columbia University in New York.

Most people who are exposed to the coronavirus
\href{https://www.nytimes3xbfgragh.onion/2020/05/07/health/coronavirus-antibody-prevalence.html}{make
antibodies} that can destroy the virus; the more severe the symptoms,
the stronger the response. (A few people don't produce the antibodies,
but that's true for any virus.) Worries about reinfection have been
fueled by recent studies suggesting that these antibody levels plummet.

\includegraphics{https://static01.graylady3jvrrxbe.onion/images/2020/07/20/science/00VIRUS-REINFECTION2/merlin_172461039_33c5b1a6-f9c1-414b-b9ff-9c601d9c45b7-articleLarge.jpg?quality=75\&auto=webp\&disable=upscale}

For example, a study published in June found that antibodies to one part
of the virus
\href{https://www.nytimes3xbfgragh.onion/2020/06/18/health/coronavirus-antibodies.html}{fell
to undetectable levels}within three months in 40 percent of asymptomatic
people. Last week, a study that has not yet been published in a
peer-reviewed journal showed that neutralizing antibodies --- the
powerful subtype that can stop the virus from infecting cells ---
\href{https://www.medrxiv.org/content/10.1101/2020.07.09.20148429v1}{declined
sharply} within a month.

``It's actually incredibly depressing,'' said Michael Malim, a
virologist at King's College London. ``It's a huge drop.''

But other work suggests that the antibody levels decline --- and then
stabilize. In
\href{https://www.medrxiv.org/content/10.1101/2020.07.14.20151126v1}{a
study of nearly 20,000 people} posted to the online server MedRxiv on
July 17, the vast majority made plentiful antibodies, and half of those
with low levels still had antibodies that could destroy the virus.

``None of this is really surprising from a biological point of view,''
said Florian Krammer, an immunologist at the Icahn Mount Sinai School of
Medicine who led that study.

Dr. Mina agreed. ``This is a famous dynamic of how antibodies develop
after infection: They go very, very high, and then they come back down,"
he said.

He elaborated: The first cells that secrete antibodies during an
infection are called plasmablasts, which expand exponentially into a
pool of millions. But the body can't sustain those levels. Once the
infection wanes, a small fraction of the cells enters the bone marrow
and sets up shop to create long-term immunity memory, which can churn
out antibodies when they're needed again. The rest of the plasmablasts
wither and die.

\href{https://www.nytimes3xbfgragh.onion/news-event/coronavirus?action=click\&pgtype=Article\&state=default\&region=MAIN_CONTENT_3\&context=storylines_faq}{}

\hypertarget{the-coronavirus-outbreak-}{%
\subsubsection{The Coronavirus Outbreak
›}\label{the-coronavirus-outbreak-}}

\hypertarget{frequently-asked-questions}{%
\paragraph{Frequently Asked
Questions}\label{frequently-asked-questions}}

Updated July 27, 2020

\begin{itemize}
\item ~
  \hypertarget{should-i-refinance-my-mortgage}{%
  \paragraph{Should I refinance my
  mortgage?}\label{should-i-refinance-my-mortgage}}

  \begin{itemize}
  \tightlist
  \item
    \href{https://www.nytimes3xbfgragh.onion/article/coronavirus-money-unemployment.html?action=click\&pgtype=Article\&state=default\&region=MAIN_CONTENT_3\&context=storylines_faq}{It
    could be a good idea,} because mortgage rates have
    \href{https://www.nytimes3xbfgragh.onion/2020/07/16/business/mortgage-rates-below-3-percent.html?action=click\&pgtype=Article\&state=default\&region=MAIN_CONTENT_3\&context=storylines_faq}{never
    been lower.} Refinancing requests have pushed mortgage applications
    to some of the highest levels since 2008, so be prepared to get in
    line. But defaults are also up, so if you're thinking about buying a
    home, be aware that some lenders have tightened their standards.
  \end{itemize}
\item ~
  \hypertarget{what-is-school-going-to-look-like-in-september}{%
  \paragraph{What is school going to look like in
  September?}\label{what-is-school-going-to-look-like-in-september}}

  \begin{itemize}
  \tightlist
  \item
    It is unlikely that many schools will return to a normal schedule
    this fall, requiring the grind of
    \href{https://www.nytimes3xbfgragh.onion/2020/06/05/us/coronavirus-education-lost-learning.html?action=click\&pgtype=Article\&state=default\&region=MAIN_CONTENT_3\&context=storylines_faq}{online
    learning},
    \href{https://www.nytimes3xbfgragh.onion/2020/05/29/us/coronavirus-child-care-centers.html?action=click\&pgtype=Article\&state=default\&region=MAIN_CONTENT_3\&context=storylines_faq}{makeshift
    child care} and
    \href{https://www.nytimes3xbfgragh.onion/2020/06/03/business/economy/coronavirus-working-women.html?action=click\&pgtype=Article\&state=default\&region=MAIN_CONTENT_3\&context=storylines_faq}{stunted
    workdays} to continue. California's two largest public school
    districts --- Los Angeles and San Diego --- said on July 13, that
    \href{https://www.nytimes3xbfgragh.onion/2020/07/13/us/lausd-san-diego-school-reopening.html?action=click\&pgtype=Article\&state=default\&region=MAIN_CONTENT_3\&context=storylines_faq}{instruction
    will be remote-only in the fall}, citing concerns that surging
    coronavirus infections in their areas pose too dire a risk for
    students and teachers. Together, the two districts enroll some
    825,000 students. They are the largest in the country so far to
    abandon plans for even a partial physical return to classrooms when
    they reopen in August. For other districts, the solution won't be an
    all-or-nothing approach.
    \href{https://bioethics.jhu.edu/research-and-outreach/projects/eschool-initiative/school-policy-tracker/}{Many
    systems}, including the nation's largest, New York City, are
    devising
    \href{https://www.nytimes3xbfgragh.onion/2020/06/26/us/coronavirus-schools-reopen-fall.html?action=click\&pgtype=Article\&state=default\&region=MAIN_CONTENT_3\&context=storylines_faq}{hybrid
    plans} that involve spending some days in classrooms and other days
    online. There's no national policy on this yet, so check with your
    municipal school system regularly to see what is happening in your
    community.
  \end{itemize}
\item ~
  \hypertarget{is-the-coronavirus-airborne}{%
  \paragraph{Is the coronavirus
  airborne?}\label{is-the-coronavirus-airborne}}

  \begin{itemize}
  \tightlist
  \item
    The coronavirus
    \href{https://www.nytimes3xbfgragh.onion/2020/07/04/health/239-experts-with-one-big-claim-the-coronavirus-is-airborne.html?action=click\&pgtype=Article\&state=default\&region=MAIN_CONTENT_3\&context=storylines_faq}{can
    stay aloft for hours in tiny droplets in stagnant air}, infecting
    people as they inhale, mounting scientific evidence suggests. This
    risk is highest in crowded indoor spaces with poor ventilation, and
    may help explain super-spreading events reported in meatpacking
    plants, churches and restaurants.
    \href{https://www.nytimes3xbfgragh.onion/2020/07/06/health/coronavirus-airborne-aerosols.html?action=click\&pgtype=Article\&state=default\&region=MAIN_CONTENT_3\&context=storylines_faq}{It's
    unclear how often the virus is spread} via these tiny droplets, or
    aerosols, compared with larger droplets that are expelled when a
    sick person coughs or sneezes, or transmitted through contact with
    contaminated surfaces, said Linsey Marr, an aerosol expert at
    Virginia Tech. Aerosols are released even when a person without
    symptoms exhales, talks or sings, according to Dr. Marr and more
    than 200 other experts, who
    \href{https://academic.oup.com/cid/article/doi/10.1093/cid/ciaa939/5867798}{have
    outlined the evidence in an open letter to the World Health
    Organization}.
  \end{itemize}
\item ~
  \hypertarget{what-are-the-symptoms-of-coronavirus}{%
  \paragraph{What are the symptoms of
  coronavirus?}\label{what-are-the-symptoms-of-coronavirus}}

  \begin{itemize}
  \tightlist
  \item
    Common symptoms
    \href{https://www.nytimes3xbfgragh.onion/article/symptoms-coronavirus.html?action=click\&pgtype=Article\&state=default\&region=MAIN_CONTENT_3\&context=storylines_faq}{include
    fever, a dry cough, fatigue and difficulty breathing or shortness of
    breath.} Some of these symptoms overlap with those of the flu,
    making detection difficult, but runny noses and stuffy sinuses are
    less common.
    \href{https://www.nytimes3xbfgragh.onion/2020/04/27/health/coronavirus-symptoms-cdc.html?action=click\&pgtype=Article\&state=default\&region=MAIN_CONTENT_3\&context=storylines_faq}{The
    C.D.C. has also} added chills, muscle pain, sore throat, headache
    and a new loss of the sense of taste or smell as symptoms to look
    out for. Most people fall ill five to seven days after exposure, but
    symptoms may appear in as few as two days or as many as 14 days.
  \end{itemize}
\item ~
  \hypertarget{does-asymptomatic-transmission-of-covid-19-happen}{%
  \paragraph{Does asymptomatic transmission of Covid-19
  happen?}\label{does-asymptomatic-transmission-of-covid-19-happen}}

  \begin{itemize}
  \tightlist
  \item
    So far, the evidence seems to show it does. A widely cited
    \href{https://www.nature.com/articles/s41591-020-0869-5}{paper}
    published in April suggests that people are most infectious about
    two days before the onset of coronavirus symptoms and estimated that
    44 percent of new infections were a result of transmission from
    people who were not yet showing symptoms. Recently, a top expert at
    the World Health Organization stated that transmission of the
    coronavirus by people who did not have symptoms was ``very rare,''
    \href{https://www.nytimes3xbfgragh.onion/2020/06/09/world/coronavirus-updates.html?action=click\&pgtype=Article\&state=default\&region=MAIN_CONTENT_3\&context=storylines_faq\#link-1f302e21}{but
    she later walked back that statement.}
  \end{itemize}
\end{itemize}

In children, each subsequent exposure to a virus --- or to a vaccine ---
boosts immunity until, by adulthood, the antibody response is steady and
strong.

What's unusual in the current pandemic, Dr. Mina said, is to see how
this dynamic plays out in adults, because they so rarely experience a
virus for the first time.

Even after the first surge of immunity fades, there is likely to be some
residual protection. And while antibodies have received all the
attention because they are easier to study and detect, memory T cells
and B cells are also powerful immune warriors in a fight against any
pathogen.

A study published July 15, for example, looked at three different
groups. In one, each of 36 people exposed to the new virus had
\href{https://www.nature.com/articles/s41586-020-2550-z}{T cells that
recognize} a protein that looks similar in all coronaviruses. In
another, 23 people infected with the SARS virus in 2003 also had these T
cells, as did 37 people in the third group who were never exposed to
either pathogen.

``A level of pre-existing immunity against SARS-CoV2 appears to exist in
the general population,'' said Dr. Antonio Bertoletti, a virologist at
Duke NUS Medical School in Singapore.

The immunity may have been stimulated by
\href{https://immunology.sciencemag.org/content/5/48/eabd2071}{prior
exposure} to coronaviruses that cause common colds. These T cells may
not thwart infection, but they would blunt the illness and may explain
why some people with Covid-19 have mild to no symptoms. ``I believe that
cellular and antibody immunity will be equally important,'' Dr.
Bertoletti said.

Vaccine trials that closely track volunteers may deliver more
information about the nature of immunity to the new coronavirus and the
level needed to block reinfection. Research in
\href{https://science.sciencemag.org/content/early/2020/07/01/science.abc5343}{monkeys
offers hope}: In a study of
\href{https://science.sciencemag.org/content/early/2020/05/19/science.abc4776}{nine
rhesus macaques}, for example, exposure to the virus induced immunity
that was
\href{https://www.nytimes3xbfgragh.onion/2020/05/20/health/coronavirus-vaccine-harvard.html}{strong
enough to prevent} a second infection.

Researchers are tracking infected monkeys to determine how long this
protection lasts. ``Durability studies by their nature take time,'' said
Dr. Dan Barouch, a virologist at Beth Israel Deaconess Medical Center in
Boston who led the study.

Dr. Barouch and other experts rejected fears that herd immunity might
never be reached.

``We achieve herd immunity all the time with less than perfect
vaccines,'' said Dr. Saad Omer, the director of the Yale Institute for
Global Health. ``It's very rare in fact to have vaccines that are
100-percent effective.''

A vaccine that protects just half of the people who receive it is
considered moderately effective, and one that covers more than 80
percent highly effective. Even a vaccine that only suppresses the levels
of virus would deter its spread to others.

The experts said reinfection had occurred with other pathogens including
influenza --- but they emphasized that those cases were exceptions, and
the new coronavirus was likely to be no different.

``I would say reinfection is possible, though not likely, and I'd think
it would be rare,'' Dr. Rasmussen said. ``But even rare occurrences
might seem alarmingly frequent when a huge number of people have been
infected.''

Advertisement

\protect\hyperlink{after-bottom}{Continue reading the main story}

\hypertarget{site-index}{%
\subsection{Site Index}\label{site-index}}

\hypertarget{site-information-navigation}{%
\subsection{Site Information
Navigation}\label{site-information-navigation}}

\begin{itemize}
\tightlist
\item
  \href{https://help.nytimes3xbfgragh.onion/hc/en-us/articles/115014792127-Copyright-notice}{©~2020~The
  New York Times Company}
\end{itemize}

\begin{itemize}
\tightlist
\item
  \href{https://www.nytco.com/}{NYTCo}
\item
  \href{https://help.nytimes3xbfgragh.onion/hc/en-us/articles/115015385887-Contact-Us}{Contact
  Us}
\item
  \href{https://www.nytco.com/careers/}{Work with us}
\item
  \href{https://nytmediakit.com/}{Advertise}
\item
  \href{http://www.tbrandstudio.com/}{T Brand Studio}
\item
  \href{https://www.nytimes3xbfgragh.onion/privacy/cookie-policy\#how-do-i-manage-trackers}{Your
  Ad Choices}
\item
  \href{https://www.nytimes3xbfgragh.onion/privacy}{Privacy}
\item
  \href{https://help.nytimes3xbfgragh.onion/hc/en-us/articles/115014893428-Terms-of-service}{Terms
  of Service}
\item
  \href{https://help.nytimes3xbfgragh.onion/hc/en-us/articles/115014893968-Terms-of-sale}{Terms
  of Sale}
\item
  \href{https://spiderbites.nytimes3xbfgragh.onion}{Site Map}
\item
  \href{https://help.nytimes3xbfgragh.onion/hc/en-us}{Help}
\item
  \href{https://www.nytimes3xbfgragh.onion/subscription?campaignId=37WXW}{Subscriptions}
\end{itemize}
