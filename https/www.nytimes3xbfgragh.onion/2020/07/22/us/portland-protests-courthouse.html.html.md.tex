Sections

SEARCH

\protect\hyperlink{site-content}{Skip to
content}\protect\hyperlink{site-index}{Skip to site index}

\href{https://www.nytimes3xbfgragh.onion/section/us}{U.S.}

\href{https://myaccount.nytimes3xbfgragh.onion/auth/login?response_type=cookie\&client_id=vi}{}

\href{https://www.nytimes3xbfgragh.onion/section/todayspaper}{Today's
Paper}

\href{/section/us}{U.S.}\textbar{}Portland Clashes Converge on
Courthouse Named for an Antiwar Republican

\url{https://nyti.ms/3jvjvi3}

\begin{itemize}
\item
\item
\item
\item
\item
\end{itemize}

\href{https://www.nytimes3xbfgragh.onion/news-event/george-floyd-protests-minneapolis-new-york-los-angeles?action=click\&pgtype=Article\&state=default\&region=TOP_BANNER\&context=storylines_menu}{Race
and America}

\begin{itemize}
\tightlist
\item
  \href{https://www.nytimes3xbfgragh.onion/2020/07/26/us/protests-portland-seattle-trump.html?action=click\&pgtype=Article\&state=default\&region=TOP_BANNER\&context=storylines_menu}{Protesters
  Return to Other Cities}
\item
  \href{https://www.nytimes3xbfgragh.onion/2020/07/24/us/portland-oregon-protests-white-race.html?action=click\&pgtype=Article\&state=default\&region=TOP_BANNER\&context=storylines_menu}{Portland
  at the Center}
\item
  \href{https://www.nytimes3xbfgragh.onion/2020/07/23/podcasts/the-daily/portland-protests.html?action=click\&pgtype=Article\&state=default\&region=TOP_BANNER\&context=storylines_menu}{Podcast:
  Showdown in Portland}
\item
  \href{https://www.nytimes3xbfgragh.onion/interactive/2020/07/16/us/black-lives-matter-protests-louisville-breonna-taylor.html?action=click\&pgtype=Article\&state=default\&region=TOP_BANNER\&context=storylines_menu}{45
  Days in Louisville}
\end{itemize}

Advertisement

\protect\hyperlink{after-top}{Continue reading the main story}

Supported by

\protect\hyperlink{after-sponsor}{Continue reading the main story}

\hypertarget{portland-clashes-converge-on-courthouse-named-for-an-antiwar-republican}{%
\section{Portland Clashes Converge on Courthouse Named for an Antiwar
Republican}\label{portland-clashes-converge-on-courthouse-named-for-an-antiwar-republican}}

The site where federal officers have fired tear gas and projectiles at
protesters is named after a senator, Mark O. Hatfield, who described
himself as ``close'' to being a pacifist.

\includegraphics{https://static01.graylady3jvrrxbe.onion/images/2020/07/21/us/21portland-courthouse/merlin_174794292_b72af956-a82b-4d0b-bef6-337cdd0e31d9-articleLarge.jpg?quality=75\&auto=webp\&disable=upscale}

By
\href{https://www.nytimes3xbfgragh.onion/by/nicholas-bogel-burroughs}{Nicholas
Bogel-Burroughs}

\begin{itemize}
\item
  Published July 22, 2020Updated July 29, 2020
\item
  \begin{itemize}
  \item
  \item
  \item
  \item
  \item
  \end{itemize}
\end{itemize}

At the federal court in
\href{https://www.nytimes3xbfgragh.onion/interactive/2020/07/22/us/portland-protests.html}{Portland,
Ore.}, the action usually takes place inside sleek, wood-paneled
courtrooms. That is where an imprisoned C.I.A. spy
\href{https://www.oregonlive.com/portland/2010/11/former_cia_spy_jim_nicholson_p_1.html}{admitted
to smuggling notes to Russia} through his son, where a judge
\href{https://www.cnn.com/2002/LAW/04/17/oregon.assisted.suicide/}{upheld
Oregon's landmark assisted-suicide law}, and where a jury
\href{https://www.nytimes3xbfgragh.onion/2016/10/28/us/bundy-brothers-acquitted-in-takeover-of-oregon-wildlife-refuge.html}{acquitted
seven people} who participated in the armed takeover of a federal
wildlife sanctuary.

In recent weeks, though, it is the exterior of the Mark O. Hatfield U.S.
Courthouse in downtown Portland that
\href{https://www.nytimes3xbfgragh.onion/2020/07/21/us/portland-protests.html}{has
become a battleground during chaotic, nightly protests} that continued
through Wednesday morning. Federal agents have fired projectiles and
tear gas
\href{https://twitter.com/ByMikeBaker/status/1285490693486473216?s=20}{from
inside the building} as protesters lobbed water bottles, set fires and
spray-painted the walls with phrases including ``Feds go home.''

The federal courthouse --- named after a longtime Oregon lawmaker who
said he was ``close'' to being a pacifist --- is perhaps an unlikely
target for the demonstrations, which have more often focused on
\href{https://www.oregonlive.com/news/2020/06/portlands-justice-center-fence-a-daunting-place-after-dark-for-both-protesters-and-police.html}{the
building next door}, which houses the Portland Police Bureau, a county
jail and other local law enforcement operations.

But the Trump administration's decision to send
\href{https://www.nytimes3xbfgragh.onion/2020/07/29/us/protests-portland-federal-withdrawal.html}{federal
agents to Portland} and other cities to stamp out protests and defend
U.S. property, like the courthouse, has in recent days
\href{https://www.nytimes3xbfgragh.onion/2020/07/20/us/portland-protests-navy-christopher-david.html}{galvanized
protesters} against the federal government.

The courthouse, which takes up an entire city block and rises to 16
floors, began hearing cases in 1997, replacing a nearby building that
had been home to the federal Oregon District Court for 64 years. It cost
\$129 million to build, according to
\href{https://usdchs.org/wp-content/uploads/2020/03/1998-Spring-Oregon-Benchmarks.pdf}{an
article published in 1998} in Oregon Benchmarks, a newsletter for the
court's historical society, which said the building's design had
received ``more plaudits than pans.''

The courthouse is named after
\href{https://www.nytimes3xbfgragh.onion/2011/08/08/us/politics/08hatfield.html}{Mark
Odom Hatfield}, a Republican who served as Oregon's governor from 1959
to 1967 and then as one of its senators from 1967 to 1997.

Before his political career, he served as a Navy lieutenant in World War
II and visited Hiroshima not long after the United States dropped the
atomic bomb, experiences that shaped his worldview and turned him
against many future American wars, including in Vietnam. Many described
him as a pacifist --- he
\href{https://www.senate.gov/senators/FeaturedBios/Featured_Bio_HatfieldMark.htm}{never
voted} for a military authorization bill --- but he told The Associated
Press in 1991 that he was not one, only ``close.''

``I find that war, with exceptions, has really not settled very many
things that were the causes of the wars in the first place,'' Mr.
Hatfield \href{https://www.newspapers.com/newspage/442485597/}{said in
that interview}. Later in 1991, he
\href{https://www.congress.gov/bill/102nd-congress/senate-bill/1155}{proposed
a bill} that would have required federal executions to take place in
public and in a place where they could be broadcast on television,
\href{https://www.nytimes3xbfgragh.onion/1991/05/03/opinion/pictures-at-an-execution.html}{believing
that doing so} would lead the public to oppose the death penalty. The
Senate never voted on the bill.

Mr. Hatfield was also known for his devout Christianity and willingness
to work with Democrats. As a state lawmaker in 1953, he
\href{https://oregonhistoryproject.org/articles/historical-records/signing-oregon39s-civil-rights-bill-1953}{worked
to pass the state's civil rights bill} outlawing racial discrimination
by hotels and other businesses. Mr. Hatfield
\href{https://www.nytimes3xbfgragh.onion/2011/08/08/us/politics/08hatfield.html}{died
in 2011 at age 89}.

Julie Engbloom, a lawyer and president of the federal court's historical
society, wrote in the group's
\href{https://usdchs.org/wp-content/uploads/2020/07/benchmarks_spring2020_rev-final-web.pdf}{most
recent newsletter} about graffiti on the courthouse that had covered a
quotation from Thomas Jefferson. Etched into the building's stone is
\href{http://www.digitalhistory.uh.edu/disp_textbook.cfm?smtID=3\&psid=230}{a
phrase that Jefferson wrote} in opposition to adding Missouri to the
United States as a free state: ``The boisterous sea of liberty is never
without a wave.''

``The graffiti painted on our beloved courthouse is difficult to take
in,''
\href{https://usdchs.org/wp-content/uploads/2020/07/benchmarks_spring2020_rev-final-web.pdf}{she
wrote}. ``But we must.''

Ms. Engbloom added that the Minneapolis police killing of
\href{https://www.nytimes3xbfgragh.onion/article/george-floyd-who-is.html}{George
Floyd} in May had unleashed a
\href{https://www.nytimes3xbfgragh.onion/news-event/george-floyd-protests-minneapolis-new-york-los-angeles}{wave
of protests} that were ``rolling toward us like a tsunami.'' The
protesters were demanding ``more from our institutions,'' she wrote,
``including the courts.''

Advertisement

\protect\hyperlink{after-bottom}{Continue reading the main story}

\hypertarget{site-index}{%
\subsection{Site Index}\label{site-index}}

\hypertarget{site-information-navigation}{%
\subsection{Site Information
Navigation}\label{site-information-navigation}}

\begin{itemize}
\tightlist
\item
  \href{https://help.nytimes3xbfgragh.onion/hc/en-us/articles/115014792127-Copyright-notice}{©~2020~The
  New York Times Company}
\end{itemize}

\begin{itemize}
\tightlist
\item
  \href{https://www.nytco.com/}{NYTCo}
\item
  \href{https://help.nytimes3xbfgragh.onion/hc/en-us/articles/115015385887-Contact-Us}{Contact
  Us}
\item
  \href{https://www.nytco.com/careers/}{Work with us}
\item
  \href{https://nytmediakit.com/}{Advertise}
\item
  \href{http://www.tbrandstudio.com/}{T Brand Studio}
\item
  \href{https://www.nytimes3xbfgragh.onion/privacy/cookie-policy\#how-do-i-manage-trackers}{Your
  Ad Choices}
\item
  \href{https://www.nytimes3xbfgragh.onion/privacy}{Privacy}
\item
  \href{https://help.nytimes3xbfgragh.onion/hc/en-us/articles/115014893428-Terms-of-service}{Terms
  of Service}
\item
  \href{https://help.nytimes3xbfgragh.onion/hc/en-us/articles/115014893968-Terms-of-sale}{Terms
  of Sale}
\item
  \href{https://spiderbites.nytimes3xbfgragh.onion}{Site Map}
\item
  \href{https://help.nytimes3xbfgragh.onion/hc/en-us}{Help}
\item
  \href{https://www.nytimes3xbfgragh.onion/subscription?campaignId=37WXW}{Subscriptions}
\end{itemize}
