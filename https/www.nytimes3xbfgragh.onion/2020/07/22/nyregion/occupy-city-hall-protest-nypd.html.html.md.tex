Sections

SEARCH

\protect\hyperlink{site-content}{Skip to
content}\protect\hyperlink{site-index}{Skip to site index}

\href{https://www.nytimes3xbfgragh.onion/section/nyregion}{New York}

\href{https://myaccount.nytimes3xbfgragh.onion/auth/login?response_type=cookie\&client_id=vi}{}

\href{https://www.nytimes3xbfgragh.onion/section/todayspaper}{Today's
Paper}

\href{/section/nyregion}{New York}\textbar{}`Occupy City Hall'
Encampment Taken Down in Pre-Dawn Raid by N.Y.P.D.

\url{https://nyti.ms/39i3qYc}

\begin{itemize}
\item
\item
\item
\item
\item
\item
\end{itemize}

\hypertarget{race-and-america}{%
\subsubsection{\texorpdfstring{\href{https://www.nytimes3xbfgragh.onion/news-event/george-floyd-protests-minneapolis-new-york-los-angeles?name=styln-george-floyd\&region=TOP_BANNER\&variant=undefined\&block=storyline_menu_recirc\&action=click\&pgtype=Article\&impression_id=8dfd67f0-e38d-11ea-9429-99c7ddc8e385}{Race
and America}}{Race and America}}\label{race-and-america}}

\begin{itemize}
\tightlist
\item
  \href{https://www.nytimes3xbfgragh.onion/interactive/2020/07/03/us/george-floyd-protests-crowd-size.html?name=styln-george-floyd\&region=TOP_BANNER\&variant=undefined\&block=storyline_menu_recirc\&action=click\&pgtype=Article\&impression_id=8dfd67f1-e38d-11ea-9429-99c7ddc8e385}{Black
  Lives Matter Movement}
\item
  \href{https://www.nytimes3xbfgragh.onion/interactive/2020/06/28/us/i-cant-breathe-police-arrest.html?name=styln-george-floyd\&region=TOP_BANNER\&variant=undefined\&block=storyline_menu_recirc\&action=click\&pgtype=Article\&impression_id=8dfd8f00-e38d-11ea-9429-99c7ddc8e385}{History
  of `I Can't Breathe'}
\item
  \href{https://www.nytimes3xbfgragh.onion/interactive/2020/06/10/upshot/black-lives-matter-attitudes.html?name=styln-george-floyd\&region=TOP_BANNER\&variant=undefined\&block=storyline_menu_recirc\&action=click\&pgtype=Article\&impression_id=8dfd8f01-e38d-11ea-9429-99c7ddc8e385}{How
  Public Opinion Shifted}
\item
  \href{https://www.nytimes3xbfgragh.onion/interactive/2020/07/16/us/black-lives-matter-protests-louisville-breonna-taylor.html?name=styln-george-floyd\&region=TOP_BANNER\&variant=undefined\&block=storyline_menu_recirc\&action=click\&pgtype=Article\&impression_id=8dfd8f02-e38d-11ea-9429-99c7ddc8e385}{45
  Days in Louisville}
\end{itemize}

Advertisement

\protect\hyperlink{after-top}{Continue reading the main story}

Supported by

\protect\hyperlink{after-sponsor}{Continue reading the main story}

\hypertarget{occupy-city-hall-encampment-taken-down-in-pre-dawn-raid-by-nypd}{%
\section{`Occupy City Hall' Encampment Taken Down in Pre-Dawn Raid by
N.Y.P.D.}\label{occupy-city-hall-encampment-taken-down-in-pre-dawn-raid-by-nypd}}

Officers in riot gear cleared the makeshift camp in City Hall Park,
which began as a protest against police abuses but then turned into a
gathering of homeless people.

\includegraphics{https://static01.graylady3jvrrxbe.onion/images/2020/07/22/nyregion/22occupy-video/22occupy-video-videoSixteenByNine3000.jpg}

By \href{https://www.nytimes3xbfgragh.onion/by/alan-feuer}{Alan Feuer}
and Juliana Kim

\begin{itemize}
\item
  July 22, 2020
\item
  \begin{itemize}
  \item
  \item
  \item
  \item
  \item
  \item
  \end{itemize}
\end{itemize}

Police officers in riot gear cleared out the
\href{https://www.nytimes3xbfgragh.onion/2020/06/28/nyregion/occupy-city-hall-nyc.html}{``Occupy
City Hall'' encampment} in City Hall Park near dawn on Wednesday,
shutting down a monthlong demonstration against police brutality that
recently had attracted numerous
\href{https://www.nytimes3xbfgragh.onion/2020/07/09/nyregion/occupy-city-hall-nyc-homeless.html}{homeless
people}.

A phalanx of officers in helmets started closing in on dozens of
protesters and homeless people shortly before 4 a.m., moving in
lock-step behind a wall of plastic shields, according to protesters and
videos posted on social media.

Seven people were arrested after sporadic clashes erupted between
officers and residents of the camp, officials said. One protester was
taken into custody after the police said he threw a brick at an officer.

\begin{quote}
NYPD riot police have driven City Hall occupants, at least half of whom
have no other housing options. Tents and signs have been trashed.
\href{https://t.co/BUkXIRfQOd}{pic.twitter.com/BUkXIRfQOd}

--- NYC Protest Updates 2020 (@protest\_nyc)
\href{https://twitter.com/protest_nyc/status/1285846628012896258?ref_src=twsrc\%5Etfw}{July
22, 2020}
\end{quote}

As the police moved through the camp, officers took down a series of
tarps and makeshift tents that demonstrators and several homeless people
had been living in for weeks and tossed them into city garbage trucks.

By 8 a.m., city cleaning crews had arrived to scrub anti-police graffiti
from sidewalks, subway entrances and the walls of several historic
buildings in the area, including the Manhattan Surrogate's Court across
the street from City Hall and the Tweed Courthouse, home of the
Department of Education.

The graffiti on the courthouse and other government landmarks --- much
of it obscene --- became a graphic symbol of the continuing unrest in
the city. Police officials said the cleanup at and around the protest
site could take weeks to finish.

Police union leaders and some local residents complained that the city
had waited too long to take the graffiti down.

\includegraphics{https://static01.graylady3jvrrxbe.onion/images/2020/07/22/nyregion/22nyoccupyNEW/22nyoccupyNEW-articleLarge.jpg?quality=75\&auto=webp\&disable=upscale}

``The graffiti is just another manifestation of the city in decline,''
Gov. Andrew M. Cuomo said on Wednesday. Mr. Cuomo urged city officials
to clean it up as quickly as possible, saying, ``What does it take? It's
spray paint.''

Speaking to reporters Wednesday morning, Mayor Bill de Blasio said that
the decision to shut down the camp had been made around 10 p.m. on
Tuesday because of ``health and safety.''

``What we saw change over the last few weeks was the gathering there got
smaller and smaller, was less and less about protest and more and more
of an area where homeless folks were gathering,'' the mayor said.

At a separate news conference, Raymond Spinella, the police department's
chief of support services, said that several hundred officers moved into
the park at about 3:40 a.m. after giving the protesters camped there a
10-minute warning.

He said the decision to move in on Wednesday had been reached with City
Hall and was based on the relatively few number of people in the park.

A similar raid was launched almost a decade ago to dismantle the Occupy
Wall Street camp in Zuccotti Park in Lower Manhattan.

In November 2011, dozens of officers marched into the park at 1 a.m.,
rousting protesters who had been there since September and removing
their tents, tarps and belongings.

On Wednesday, Yessenia Benitez, 29, of Harlem, said she saw about 100
officers converge on City Hall Park well before sunrise, announcing to
protesters that they were breaking the law and ordering them to leave at
once.

Most people dispersed, she said, but a small group watched the police
operation from Foley Square, a few blocks to the north.

A few protesters said the police had told them that they would be able
to return to the park to retrieve their belongings. But when they went
back, everything --- their water, clothing and personal effects --- had
been tossed into sanitation trucks, they said.

Image

The scene near City Hall in late June, where protesters set up an
encampment in Manhattan.Credit...Amr Alfiky/The New York Times

The occupation began on June 23 when about 100 people set up camp on a
small patch of grass to the east of City Hall with the mission of
bringing pressure on the City Council to cut the New York Police
Department's funding at an upcoming vote before the July 1 budget
deadline.

Within a week, the small squatters' colony grew into a ramshackle
community with food service, a hand-sanitizing station and even a
library where campers could go to hear lectures on mass incarceration
and the school-to-prison pipeline.

Hundreds slept in the park each night, festooning benches and fences
with signs denouncing racism and police brutality.

The project reached its peak on June 30 when thousands crowded into the
plaza after dark to watch the Council vote on a giant video screen.
While the Council ultimately
\href{https://www.nytimes3xbfgragh.onion/2020/06/30/nyregion/nypd-budget.html}{decided
to shift nearly \$1 billion away from the police}, many of the
protesters expressed disappointment, wanting deeper cuts. Most went home
within days.

Those that remained quickly assumed a new responsibility: caring for the
dozens of homeless people who had flocked to the site --- which
protesters started calling Abolition Park --- for its free meals,
open-air camping and communal sensibility.

While organizers said they felt a duty to tend to some of the city's
most vulnerable residents, problems soon arose. Fights broke out.
Passers-by were harassed.

Some local residents, even those who said they supported the project's
politics, started to complain that the once-peaceful compound had turned
into a shantytown marred by violence and disorder.

The camp, just feet from City Hall, had presented a thorny political
problem for Mayor de Blasio. He has been routinely criticized by the
demonstrators
\href{https://www.nytimes3xbfgragh.onion/2020/06/12/nyregion/de-blasio-blacks-protest.html}{and
his Black supporters} since the larger, citywide protests, prompted by
the death of George Floyd in Minneapolis, started in late May.

The decision to close the encampment came only days after President
Trump sent teams of
\href{https://www.nytimes3xbfgragh.onion/2020/07/21/us/portland-protests.html}{heavily
armed federal agents to Portland} to protect federal property and to
subdue protests there that have turned violent on occasion.

Mr. Trump has also threatened to send agents to New York and other
cities.

Jawanza James Williams, one of the original organizers of the encampment
in June, said the use of hundreds of police officers to disperse several
dozen people was ``a reminder of why we made the demands we did in the
beginning.''

Mr. Williams also disputed Mr. de Blasio's argument that the homeless
encampment was a health concern, pointing to the Center for Disease
Control's warning that breaking up homeless encampments could risk a
spread of the coronavirus. ``The park was the safest place for homeless
people besides having their own permanent housing,'' he said.

Though the occupation at City Hall was over, some protesters said it was
not a fatal blow to their cause.

``It's a reminder that the fight isn't over, and you know what, I'm glad
that they reminded us now,'' said Gabe Quinones, 22, who worked at the
camp as a volunteer ``de-escalator,'' settling disputes and soothing
frayed tempers. ``We'll go somewhere else with everything that we
learned here and continue our work.''

Image

Protesters on Broadway near City Hall Wednesday night.Credit...Dave
Sanders for The New York Times

In a sign that some people were not quite ready to give up on City Hall
Park altogether, around 300 protesters marched there from Union Square
Park on Wednesday in hopes of staging another demonstration.

After being met by barricades and dozens of officers in riot gear, the
group marched off into Lower Manhattan, a small armada of police vans
not far behind.

Advertisement

\protect\hyperlink{after-bottom}{Continue reading the main story}

\hypertarget{site-index}{%
\subsection{Site Index}\label{site-index}}

\hypertarget{site-information-navigation}{%
\subsection{Site Information
Navigation}\label{site-information-navigation}}

\begin{itemize}
\tightlist
\item
  \href{https://help.nytimes3xbfgragh.onion/hc/en-us/articles/115014792127-Copyright-notice}{©~2020~The
  New York Times Company}
\end{itemize}

\begin{itemize}
\tightlist
\item
  \href{https://www.nytco.com/}{NYTCo}
\item
  \href{https://help.nytimes3xbfgragh.onion/hc/en-us/articles/115015385887-Contact-Us}{Contact
  Us}
\item
  \href{https://www.nytco.com/careers/}{Work with us}
\item
  \href{https://nytmediakit.com/}{Advertise}
\item
  \href{http://www.tbrandstudio.com/}{T Brand Studio}
\item
  \href{https://www.nytimes3xbfgragh.onion/privacy/cookie-policy\#how-do-i-manage-trackers}{Your
  Ad Choices}
\item
  \href{https://www.nytimes3xbfgragh.onion/privacy}{Privacy}
\item
  \href{https://help.nytimes3xbfgragh.onion/hc/en-us/articles/115014893428-Terms-of-service}{Terms
  of Service}
\item
  \href{https://help.nytimes3xbfgragh.onion/hc/en-us/articles/115014893968-Terms-of-sale}{Terms
  of Sale}
\item
  \href{https://spiderbites.nytimes3xbfgragh.onion}{Site Map}
\item
  \href{https://help.nytimes3xbfgragh.onion/hc/en-us}{Help}
\item
  \href{https://www.nytimes3xbfgragh.onion/subscription?campaignId=37WXW}{Subscriptions}
\end{itemize}
