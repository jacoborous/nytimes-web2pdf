Sections

SEARCH

\protect\hyperlink{site-content}{Skip to
content}\protect\hyperlink{site-index}{Skip to site index}

\href{https://www.nytimes3xbfgragh.onion/section/climate}{Climate}

\href{https://myaccount.nytimes3xbfgragh.onion/auth/login?response_type=cookie\&client_id=vi}{}

\href{https://www.nytimes3xbfgragh.onion/section/todayspaper}{Today's
Paper}

\href{/section/climate}{Climate}\textbar{}Study Links Gas Flares to
Preterm Births, With Hispanic Women at High Risk

\url{https://nyti.ms/2WOVwRc}

\begin{itemize}
\item
\item
\item
\item
\item
\end{itemize}

\href{https://www.nytimes3xbfgragh.onion/section/climate?action=click\&pgtype=Article\&state=default\&region=TOP_BANNER\&context=storylines_menu}{Climate
and Environment}

\begin{itemize}
\tightlist
\item
  \href{https://www.nytimes3xbfgragh.onion/2020/07/30/climate/sea-level-inland-floods.html?action=click\&pgtype=Article\&state=default\&region=TOP_BANNER\&context=storylines_menu}{Rising
  Seas}
\item
  \href{https://www.nytimes3xbfgragh.onion/interactive/2020/climate/trump-environment-rollbacks.html?action=click\&pgtype=Article\&state=default\&region=TOP_BANNER\&context=storylines_menu}{Trump's
  Changes}
\item
  \href{https://www.nytimes3xbfgragh.onion/interactive/2020/04/19/climate/climate-crash-course-1.html?action=click\&pgtype=Article\&state=default\&region=TOP_BANNER\&context=storylines_menu}{Climate
  101}
\item
  \href{https://www.nytimes3xbfgragh.onion/interactive/2018/08/30/climate/how-much-hotter-is-your-hometown.html?action=click\&pgtype=Article\&state=default\&region=TOP_BANNER\&context=storylines_menu}{Is
  Your Hometown Hotter?}
\item
  \href{https://www.nytimes3xbfgragh.onion/newsletters/climate-change?action=click\&pgtype=Article\&state=default\&region=TOP_BANNER\&context=storylines_menu}{Newsletter}
\end{itemize}

Advertisement

\protect\hyperlink{after-top}{Continue reading the main story}

Supported by

\protect\hyperlink{after-sponsor}{Continue reading the main story}

\hypertarget{study-links-gas-flares-to-preterm-births-with-hispanic-women-at-high-risk}{%
\section{Study Links Gas Flares to Preterm Births, With Hispanic Women
at High
Risk}\label{study-links-gas-flares-to-preterm-births-with-hispanic-women-at-high-risk}}

Expectant mothers who lived near flaring sites had higher odds of giving
birth prematurely than those who did not, researchers found. The adverse
outcomes fell entirely on Hispanic women.

\includegraphics{https://static01.graylady3jvrrxbe.onion/images/2020/07/22/climate/22CLI-PREGNANCY/merlin_174566307_5566d581-ff81-477e-87f4-e197b0eca175-articleLarge.jpg?quality=75\&auto=webp\&disable=upscale}

By Julia Rosen

\begin{itemize}
\item
  Published July 22, 2020Updated July 23, 2020
\item
  \begin{itemize}
  \item
  \item
  \item
  \item
  \item
  \end{itemize}
\end{itemize}

Across the United States, gas flares light the night skies over oil and
gas fields --- visible symbols of the country's energy boom. They also
\href{https://www.worldbank.org/en/programs/gasflaringreduction\#1}{emit
greenhouse gases}, making them symbols of climate change that many
environmental groups would like to see snuffed out.

Now, \href{https://ehp.niehs.nih.gov/doi/10.1289/EHP6394}{a new study
points to another problem}: Pregnant women who lived near areas where
flaring is common had 50 percent greater odds of giving birth
prematurely than those who did not. These births occurred before 37
weeks of gestation, when incomplete development raises a baby's chance
of numerous disorders, even death.

``It's on par with the increased risk you see for women who smoke,''
said Lara Cushing, an assistant professor of environmental health
sciences at the University of California, Los Angeles, and lead author
of the study. Unlike smoking, however, ``it's not really something you
can do much about on an individual level,'' she said.

The analysis also found that the impacts of flaring fell entirely on
Hispanic mothers, raising concerns about environmental injustice at a
time when questions of racial inequality have gripped the nation.

Past research has shown that living near oil and gas wells increases the
odds of adverse birth outcomes. The study, published last week in
Environmental Health Perspectives, is the first to look specifically at
flaring.

\href{https://www.nytimes3xbfgragh.onion/section/climate?action=click\&pgtype=Article\&state=default\&region=MAIN_CONTENT_1\&context=storylines_keepup}{}

\hypertarget{climate-and-environment-}{%
\subsubsection{Climate and Environment
›}\label{climate-and-environment-}}

\hypertarget{keep-up-on-the-latest-climate-news}{%
\paragraph{Keep Up on the Latest Climate
News}\label{keep-up-on-the-latest-climate-news}}

Updated July 30, 2020

Here's what you need to know about the latest climate change news this
week:

\begin{itemize}
\item
  \begin{itemize}
  \tightlist
  \item
    \href{https://www.nytimes3xbfgragh.onion/2020/07/30/climate/bangladesh-floods.html?action=click\&pgtype=Article\&state=default\&region=MAIN_CONTENT_1\&context=storylines_keepup}{Floods
    in}\href{https://www.nytimes3xbfgragh.onion/2020/07/30/climate/bangladesh-floods.html?action=click\&pgtype=Article\&state=default\&region=MAIN_CONTENT_1\&context=storylines_keepup}{Bangladesh}
    are punishing the people least responsible for climate change.
  \item
    As climate change raises sea levels,
    \href{https://www.nytimes3xbfgragh.onion/2020/07/30/climate/sea-level-inland-floods.html?action=click\&pgtype=Article\&state=default\&region=MAIN_CONTENT_1\&context=storylines_keepup}{storm
    surges and high tides} are likely to push farther inland.
  \item
    The E.P.A. inspector general plans to investigate whether a rollback
    of fuel efficiency standards
    \href{https://www.nytimes3xbfgragh.onion/2020/07/27/climate/trump-fuel-efficiency-rule.html?action=click\&pgtype=Article\&state=default\&region=MAIN_CONTENT_1\&context=storylines_keepup}{violated
    government rules}.
  \end{itemize}
\end{itemize}

Oil and gas producers flare natural gas when it is too abundant to
capture and sell, or when low prices make doing so unprofitable. Burning
the gas prevents methane, a potent greenhouse gas, from escaping to the
atmosphere, but it still releases planet-warming carbon dioxide and
other
\href{https://www.nytimes3xbfgragh.onion/2020/07/16/world/middleeast/iraq-gas-flaring-cancer-environment.html}{harmful
chemicals}.

Dr. Cushing and her colleagues analyzed satellite images to track
nightly flare activity across the Eagle Ford Shale in Texas, which can
be
\href{https://earthobservatory.nasa.gov/images/87725/shale-revolution-as-clear-as-night-and-day}{seen
from space} as a crescent of twinkling lights between Laredo and San
Antonio. In an
\href{https://pubs.acs.org/doi/abs/10.1021/acs.est.8b05355}{earlier
study}, the researchers counted 43,000 flares between 2012 and 2016.

Over that same period, women in the region gave birth to 23,500 babies.
The study found that the odds of preterm birth were 30 percent higher
for mothers who lived within three miles of an oil and gas well compared
with those who did not, and 50 percent higher for women who were exposed
to 10 or more flares over the course of their pregnancies.

It can be hard to tease out cause and effect in retrospective studies
such as this, said Dr. Heather Burris, a neonatologist at the University
of Pennsylvania's Perelman School of Medicine who was not involved in
the work. But Dr. Burris said the researchers did their best to rule out
factors that might make some women prone to preterm birth, like age,
smoking habits, socioeconomic status and access to prenatal care.

The Texas Oil and Gas Association took issue with the study. ``The
researchers used proximity as a surrogate for exposure,'' said Todd
Staples, president of the association and a member of the Texas Methane
and Flaring Coalition, in a statement. He added that ``oil and natural
gas companies continue to make great strides in environmental
progress.''

Scientists do not know exactly why some women give birth prematurely,
Dr. Burris said. But the new study adds to growing evidence that
environmental factors play an important role.

In the case of flaring, researchers say the mechanism may involve
particulate matter, volatile organic compounds and other toxic
substances. ``It seems pretty plausible that it would have an effect on
premature birth given that air pollution and preterm birth are well
linked,'' said Elaine Hill, a health economist at the University of
Rochester Medical Center who was not involved in the study.

The results highlight stark racial disparities in environmental health
because the connection between flaring and preterm birth only emerged
among Hispanic women, who made up a majority of the study population.
Flaring did not increase the risk of preterm birth for non-Hispanic
white women, who accounted for about a third of mothers in the study.

Dr. Cushing said there are several potential explanations. On average,
Hispanic women experienced more flaring, and it's possible that the
effects only manifest above a certain threshold of exposure. Other
studies have also shown that women of color are more susceptible to
pollution. That may be because their bodies are already worn down by
longtime health problems, exposure to other contaminants or chronic
stress caused by discrimination, Dr. Cushing said.

Although the study didn't address it, economics could also provide part
of the answer, Dr. Hill said. If white women in the study were more
likely to own land, and thus mineral rights, then the income bumps they
received from oil and gas extraction could have offset negative health
effects, she said.

Whatever the reason, Dr. Burris said the study suggests that flaring
poses a danger to expectant mothers. ``I wouldn't go as far as to say
that it's safe for some women and not others,'' she said. ``No way.''

Flaring has
\href{https://www.worldbank.org/en/news/press-release/2019/06/12/increased-shale-oil-production-and-political-conflict-contribute-to-increase-in-global-gas-flaring}{increased
in the U.S.} in recent years, but there are efforts to curb the
practice. Last week,
\href{https://www.nytimes3xbfgragh.onion/reuters/2020/07/16/us/16reuters-usa-methane-judge.html}{a
federal court} blocked the Trump administration's attempt to roll back
Obama-era regulations that discouraged flaring. The Texas Railroad
Commission, which oversees the state's oil and gas industry, is also
considering
\href{https://www.reuters.com/article/us-climate-change-flaring-idUSKBN23N3BQ}{tightening
flaring regulations}.

Diana Lopez, executive director of the Southwest Workers Union in San
Antonio, which advocates for environmental justice, said she hoped the
study would bring new urgency to the issue by showing how vulnerable
populations bear the collateral costs of fossil fuel extraction.

``That's just a classic example of environmental racism,'' she said.

Advertisement

\protect\hyperlink{after-bottom}{Continue reading the main story}

\hypertarget{site-index}{%
\subsection{Site Index}\label{site-index}}

\hypertarget{site-information-navigation}{%
\subsection{Site Information
Navigation}\label{site-information-navigation}}

\begin{itemize}
\tightlist
\item
  \href{https://help.nytimes3xbfgragh.onion/hc/en-us/articles/115014792127-Copyright-notice}{©~2020~The
  New York Times Company}
\end{itemize}

\begin{itemize}
\tightlist
\item
  \href{https://www.nytco.com/}{NYTCo}
\item
  \href{https://help.nytimes3xbfgragh.onion/hc/en-us/articles/115015385887-Contact-Us}{Contact
  Us}
\item
  \href{https://www.nytco.com/careers/}{Work with us}
\item
  \href{https://nytmediakit.com/}{Advertise}
\item
  \href{http://www.tbrandstudio.com/}{T Brand Studio}
\item
  \href{https://www.nytimes3xbfgragh.onion/privacy/cookie-policy\#how-do-i-manage-trackers}{Your
  Ad Choices}
\item
  \href{https://www.nytimes3xbfgragh.onion/privacy}{Privacy}
\item
  \href{https://help.nytimes3xbfgragh.onion/hc/en-us/articles/115014893428-Terms-of-service}{Terms
  of Service}
\item
  \href{https://help.nytimes3xbfgragh.onion/hc/en-us/articles/115014893968-Terms-of-sale}{Terms
  of Sale}
\item
  \href{https://spiderbites.nytimes3xbfgragh.onion}{Site Map}
\item
  \href{https://help.nytimes3xbfgragh.onion/hc/en-us}{Help}
\item
  \href{https://www.nytimes3xbfgragh.onion/subscription?campaignId=37WXW}{Subscriptions}
\end{itemize}
