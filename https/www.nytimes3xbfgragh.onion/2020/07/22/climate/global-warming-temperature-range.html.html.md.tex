Sections

SEARCH

\protect\hyperlink{site-content}{Skip to
content}\protect\hyperlink{site-index}{Skip to site index}

\href{https://www.nytimes3xbfgragh.onion/section/climate}{Climate}

\href{https://myaccount.nytimes3xbfgragh.onion/auth/login?response_type=cookie\&client_id=vi}{}

\href{https://www.nytimes3xbfgragh.onion/section/todayspaper}{Today's
Paper}

\href{/section/climate}{Climate}\textbar{}How Much Will the Planet Warm
if Carbon Dioxide Levels Double?

\href{https://nyti.ms/2ZNQbeM}{https://nyti.ms/2ZNQbeM}

\begin{itemize}
\item
\item
\item
\item
\item
\item
\end{itemize}

\href{https://www.nytimes3xbfgragh.onion/section/climate?action=click\&pgtype=Article\&state=default\&region=TOP_BANNER\&context=storylines_menu}{Climate
and Environment}

\begin{itemize}
\tightlist
\item
  \href{https://www.nytimes3xbfgragh.onion/2020/07/30/climate/sea-level-inland-floods.html?action=click\&pgtype=Article\&state=default\&region=TOP_BANNER\&context=storylines_menu}{Rising
  Seas}
\item
  \href{https://www.nytimes3xbfgragh.onion/interactive/2020/climate/trump-environment-rollbacks.html?action=click\&pgtype=Article\&state=default\&region=TOP_BANNER\&context=storylines_menu}{Trump's
  Changes}
\item
  \href{https://www.nytimes3xbfgragh.onion/interactive/2020/04/19/climate/climate-crash-course-1.html?action=click\&pgtype=Article\&state=default\&region=TOP_BANNER\&context=storylines_menu}{Climate
  101}
\item
  \href{https://www.nytimes3xbfgragh.onion/interactive/2018/08/30/climate/how-much-hotter-is-your-hometown.html?action=click\&pgtype=Article\&state=default\&region=TOP_BANNER\&context=storylines_menu}{Is
  Your Hometown Hotter?}
\item
  \href{https://www.nytimes3xbfgragh.onion/newsletters/climate-change?action=click\&pgtype=Article\&state=default\&region=TOP_BANNER\&context=storylines_menu}{Newsletter}
\end{itemize}

Advertisement

\protect\hyperlink{after-top}{Continue reading the main story}

Supported by

\protect\hyperlink{after-sponsor}{Continue reading the main story}

\hypertarget{how-much-will-the-planet-warm-if-carbon-dioxide-levels-double}{%
\section{How Much Will the Planet Warm if Carbon Dioxide Levels
Double?}\label{how-much-will-the-planet-warm-if-carbon-dioxide-levels-double}}

Past estimates spanned a wide range.

+1.5°C

likely Range

+4.5°

+1°

HIGHLY LIKELY Range

+6°

A new study has narrowed it.

+4.1°

+2.6°C

+4.9°

+2.2°

+6°

+3°

+4°

+5°

+1°C

+2°

Past estimates spanned a wide range.

+1.5°C

likely Range

+4.5°

+1°

+6°

HIGHLY LIKELY Range

A new study has narrowed it.

+4.1°

+2.6°C

+4.9°

+2.2°

+6°

+3°

+4°

+5°

+1°C

+2°

Sources: Sherwood et al. in Reviews of Geophysics, IPCC

By The New York Times

\href{https://www.nytimes3xbfgragh.onion/by/john-schwartz}{\includegraphics{https://static01.graylady3jvrrxbe.onion/images/2018/02/16/multimedia/author-john-schwartz/author-john-schwartz-thumbLarge.jpg}}

By \href{https://www.nytimes3xbfgragh.onion/by/john-schwartz}{John
Schwartz}

\begin{itemize}
\item
  July 22, 2020
\item
  \begin{itemize}
  \item
  \item
  \item
  \item
  \item
  \item
  \end{itemize}
\end{itemize}

How much, exactly, will greenhouse gases heat the planet?

For more than 40 years, scientists have expressed the answer as a range
of possible temperature increases, between 1.5 and 4.5 degrees Celsius,
that will result from carbon dioxide levels doubling from preindustrial
times. Now, a team of researchers has sharply narrowed the range of
temperatures, tightening it to between 2.6 and 4.1 degrees Celsius.

Steven Sherwood, a climate scientist at the University of New South
Wales in Sydney, Australia, and an author of the new report said that
the group's research suggested that these temperature shifts, which are
referred to as ``climate sensitivity'' because they reflect how
sensitive the planet is to rising carbon dioxide levels, are now
unlikely below the low end of the range. The research also suggests that
the ``alarmingly high sensitivities'' of 5 degrees Celsius or higher are
less likely, though they are ``not impossible,'' Dr. Sherwood said.

What remains, however, is still
\href{https://www.nytimes3xbfgragh.onion/2018/10/07/climate/ipcc-climate-report-2040.html}{an
array of effects that mean worldwide disaster} if emissions are not
sharply reduced in coming years.

Masahiro Watanabe, a professor in the atmosphere and ocean research
institute at the University of Tokyo and an author of the report, said
that determining an accurate range of temperatures was critically
important for international efforts to address global warming, like the
\href{https://www.nytimes3xbfgragh.onion/2015/12/13/world/europe/climate-change-accord-paris.html?action=click\&block=associated_collection_recirc\&impression_id=958581844\&index=1\&pgtype=Article\&region=footer}{Paris
climate agreement}, and for mitigating the effects of climate change.

``Narrowing the uncertainty is relevant not only for climate science but
also for society that is responsible for solid decision making,'' he
said.

\href{https://www.nytimes3xbfgragh.onion/section/climate?action=click\&pgtype=Article\&state=default\&region=MAIN_CONTENT_1\&context=storylines_keepup}{}

\hypertarget{climate-and-environment-}{%
\subsubsection{Climate and Environment
›}\label{climate-and-environment-}}

\hypertarget{keep-up-on-the-latest-climate-news}{%
\paragraph{Keep Up on the Latest Climate
News}\label{keep-up-on-the-latest-climate-news}}

Updated July 30, 2020

Here's what you need to know about the latest climate change news this
week:

\begin{itemize}
\item
  \begin{itemize}
  \tightlist
  \item
    \href{https://www.nytimes3xbfgragh.onion/2020/07/30/climate/bangladesh-floods.html?action=click\&pgtype=Article\&state=default\&region=MAIN_CONTENT_1\&context=storylines_keepup}{Floods
    in}\href{https://www.nytimes3xbfgragh.onion/2020/07/30/climate/bangladesh-floods.html?action=click\&pgtype=Article\&state=default\&region=MAIN_CONTENT_1\&context=storylines_keepup}{Bangladesh}
    are punishing the people least responsible for climate change.
  \item
    As climate change raises sea levels,
    \href{https://www.nytimes3xbfgragh.onion/2020/07/30/climate/sea-level-inland-floods.html?action=click\&pgtype=Article\&state=default\&region=MAIN_CONTENT_1\&context=storylines_keepup}{storm
    surges and high tides} are likely to push farther inland.
  \item
    The E.P.A. inspector general plans to investigate whether a rollback
    of fuel efficiency standards
    \href{https://www.nytimes3xbfgragh.onion/2020/07/27/climate/trump-fuel-efficiency-rule.html?action=click\&pgtype=Article\&state=default\&region=MAIN_CONTENT_1\&context=storylines_keepup}{violated
    government rules}.
  \end{itemize}
\end{itemize}

The new paper,
\href{https://agupubs.onlinelibrary.wiley.com/doi/abs/10.1029/2019RG000678}{published
on Wednesday in the journal Reviews of Geophysics}, narrowed the range
of temperatures considerably and shifted it toward warmer outcomes. The
researchers determined that there was less than a 5 percent chance of a
temperature shift below two degrees, but a 6 to 18 percent chance of a
higher temperature change than 4.5 degrees Celsius, or 8.1 degrees
Fahrenheit.

If the effects of carbon dioxide are at the low end of the range or even
below it, then climate change will be affected less by emissions and the
planet will warm more slowly. If the Earth's climate is more highly
sensitive to carbon dioxide levels, then the expected results are not
only more imminent, but also more catastrophic.

The scientists noted that the Earth's temperature is already about 1.2
degrees Celsius above preindustrial levels, and that, if current
emissions trends continue, the doubling of atmospheric carbon dioxide
could happen well before the end of this century.

Andrew Dessler, a climate scientist at Texas A\&M University, who was
not an author of the report but who was one of its earlier outside
reviewers, called the paper ``a real tour de force,'' adding that ``this
is probably the most important paper I've read in years.''

\includegraphics{https://static01.graylady3jvrrxbe.onion/images/2020/07/22/climate/22CLI-GLOBALTEMPS-sub/22CLI-GLOBALTEMPS-sub-articleLarge.jpg?quality=75\&auto=webp\&disable=upscale}

For many years, those who wished to underplay the threat of climate
change have tried to say that the sensitivity is low, and so rising
greenhouse gases would have little effect. And some recent climate
models
\href{https://www.yaleclimateconnections.org/2020/07/some-new-climate-models-are-projecting-extreme-warming-are-they-correct/}{have
suggested warming could be frighteningly worse}.

The value of the paper, Dr. Dessler said, lies in the way that it
narrows the probable range of temperatures the world can expect. ``There
were a number of people who were arguing the climate sensitivity was
much lower, and a smaller number claiming it was much higher,'' he said,
``and I think the case for either of those positions is a lot weaker now
that this paper has been published.''

That means that those who undercut the seriousness of climate change and
the need for action have a much harder case to make now, Dr. Dessler
said. ``It would be great if the skeptics were right,'' he said. ``But
it's pretty clear that the data don't support that contention.''

The paper, produced under an international science organization, the
\href{https://www.wcrp-climate.org/}{World Climate Research Program},
brought together three broad fields of climate evidence: temperature
records since the industrial revolution, records of prehistoric
temperatures preserved in things like sediment samples and tree rings,
as well as satellite observations and computer models of the climate
system. None alone could determine the range, but the researchers found
ways mathematically to reconcile the three disciplines to reach their
conclusions.

``This paper is really the first to try and include all of those
disparate sources of observational evidence in a coherent package that
actually makes sense,'' said Gavin A. Schmidt, director of the NASA
Goddard Institute for Space Studies and an author of the paper.

Another author on the paper, Gabriele Hegerl, a professor of climate
system science at the University of Edinburgh, said that the way the
threads of research came together was surprising: ``We don't expect
these three lines of evidence to agree completely,'' she said, but hoped
they would ``overlap.'' And they did, she said, so ``our research is
more robust than I initially expected.''

Not everyone is prepared to accept the new results. Nicholas Lewis, an
independent scientist who has been critical of aspects of mainstream
climate research and who has found flaws in the work of others that led
to the
\href{https://www.sciencemag.org/news/2018/11/high-profile-ocean-warming-paper-get-correction}{retraction
last year of a major study on ocean warming}, questioned the new paper's
reliance on computer models to interpret the lines of evidence, as well
as the group's definition of climate sensitivity itself. He also
suggested that the paper ignored some possible complications from
changes in clouds and convection.

Dr. Schmidt said that the new paper made all of the data and methodology
available. ``This is a real challenge to people who think the experts
are wrong to go in, change the assumptions, run the code and show us
their results,'' he said.

Some degree of uncertainty about planetary warming is inevitable, said
Zeke Hausfather, a scientist with The Breakthrough Institute and an
author of the paper. But the current range is ``not a good amount of
warming at all,'' he said, noting that eliminating the extremes still
leaves a middle range that means climate disaster. ``You don't need five
degrees of warming to justify climate action,'' he said. ``Three degrees
is plenty bad.''

William Collins, a climate scientist at Lawrence Berkeley National
Laboratory who was not involved with the study, praised the effort to
tie together so much research into a single paper, but said that further
advances in computing and data gathering would continue to drive the
quest for answers. He compared climate sensitivity research to climbing
Mount Everest and said: ``This is an extremely important base camp. We
are not at the pinnacle yet.''

Advertisement

\protect\hyperlink{after-bottom}{Continue reading the main story}

\hypertarget{site-index}{%
\subsection{Site Index}\label{site-index}}

\hypertarget{site-information-navigation}{%
\subsection{Site Information
Navigation}\label{site-information-navigation}}

\begin{itemize}
\tightlist
\item
  \href{https://help.nytimes3xbfgragh.onion/hc/en-us/articles/115014792127-Copyright-notice}{©~2020~The
  New York Times Company}
\end{itemize}

\begin{itemize}
\tightlist
\item
  \href{https://www.nytco.com/}{NYTCo}
\item
  \href{https://help.nytimes3xbfgragh.onion/hc/en-us/articles/115015385887-Contact-Us}{Contact
  Us}
\item
  \href{https://www.nytco.com/careers/}{Work with us}
\item
  \href{https://nytmediakit.com/}{Advertise}
\item
  \href{http://www.tbrandstudio.com/}{T Brand Studio}
\item
  \href{https://www.nytimes3xbfgragh.onion/privacy/cookie-policy\#how-do-i-manage-trackers}{Your
  Ad Choices}
\item
  \href{https://www.nytimes3xbfgragh.onion/privacy}{Privacy}
\item
  \href{https://help.nytimes3xbfgragh.onion/hc/en-us/articles/115014893428-Terms-of-service}{Terms
  of Service}
\item
  \href{https://help.nytimes3xbfgragh.onion/hc/en-us/articles/115014893968-Terms-of-sale}{Terms
  of Sale}
\item
  \href{https://spiderbites.nytimes3xbfgragh.onion}{Site Map}
\item
  \href{https://help.nytimes3xbfgragh.onion/hc/en-us}{Help}
\item
  \href{https://www.nytimes3xbfgragh.onion/subscription?campaignId=37WXW}{Subscriptions}
\end{itemize}
