Sections

SEARCH

\protect\hyperlink{site-content}{Skip to
content}\protect\hyperlink{site-index}{Skip to site index}

\href{https://www.nytimes3xbfgragh.onion/section/business/media}{Media}

\href{https://myaccount.nytimes3xbfgragh.onion/auth/login?response_type=cookie\&client_id=vi}{}

\href{https://www.nytimes3xbfgragh.onion/section/todayspaper}{Today's
Paper}

\href{/section/business/media}{Media}\textbar{}Hearst Employees Say
Magazine Boss Led Toxic Culture

\href{https://nyti.ms/3jwnKcY}{https://nyti.ms/3jwnKcY}

\begin{itemize}
\item
\item
\item
\item
\item
\item
\end{itemize}

Advertisement

\protect\hyperlink{after-top}{Continue reading the main story}

Supported by

\protect\hyperlink{after-sponsor}{Continue reading the main story}

\hypertarget{hearst-employees-say-magazine-boss-led-toxic-culture}{%
\section{Hearst Employees Say Magazine Boss Led Toxic
Culture}\label{hearst-employees-say-magazine-boss-led-toxic-culture}}

At Cosmopolitan staff meetings, workers complain of discrimination and
tokenism. Former employees say the Hearst Magazines president, Troy
Young, made sexually offensive remarks.

\includegraphics{https://static01.graylady3jvrrxbe.onion/images/2020/07/02/business/00Unrest-Hearst-young/00Unrest-Hearst-young-articleLarge.jpg?quality=75\&auto=webp\&disable=upscale}

By \href{https://www.nytimes3xbfgragh.onion/by/katie-robertson}{Katie
Robertson} and
\href{https://www.nytimes3xbfgragh.onion/by/ben-smith}{Ben Smith}

\begin{itemize}
\item
  Published July 22, 2020Updated July 23, 2020
\item
  \begin{itemize}
  \item
  \item
  \item
  \item
  \item
  \item
  \end{itemize}
\end{itemize}

\href{https://www.nytimes3xbfgragh.onion/2020/07/23/business/troy-young-hearst-magazines.html}{\emph{Update:
Troy Young has resigned}}\emph{, the chief executive of Hearst told
employees.}

For decades Hearst magazines have advised American women on how they
should conduct themselves in the home (Good Housekeeping, Redbook), in
society (Harper's Bazaar, Town \& Country) and in the bedroom
(Cosmopolitan).

This is the company whose stars have included
\href{https://www.nytimes3xbfgragh.onion/2000/04/03/business/winfrey-breaks-new-ground-with-magazine.html}{Oprah
Winfrey}, the head of O: The Oprah Magazine, which Hearst has helped run
since 2000; and Helen Gurley Brown, the groundbreaking editor who
transformed the once-staid Cosmopolitan into a racy monthly that angered
conservatives
\href{https://www.newyorker.com/books/page-turner/owning-your-desire-remembering-helen-gurley-brown}{and}
\href{https://www.nytimes3xbfgragh.onion/1974/08/11/archives/two-faces-of-ms-versus-cosmo.html}{feminists}
alike while selling big on newsstands.

But inside the Hearst Tower in Midtown Manhattan, the Hearst Magazines
leader, Troy Young, has drawn complaints from people who said he had
made lewd, sexist remarks at work. And in recent weeks, inspired by the
civil rights movement, current and former employees at Cosmopolitan and
another Hearst women's title, Marie Claire, have spoken out on social
media and during staff meetings on what they describe as a toxic
environment.

Mr. Young, a former advertising executive,
\href{https://www.nytimes3xbfgragh.onion/2013/05/09/business/media/hearst-magazines-hires-troy-young-as-digital-media-chief.html}{joined
Hearst} in 2013 as its first head of digital media. He quickly changed
the corporate structure so that the editors of the magazines' websites
reported to him, rather than to the editors of individual publications.
As part of his plan, digital editors with relatively low salaries
replaced high-priced veteran print editors.

His work impressed Steven R. Swartz, the chief executive of Hearst
Communications, and
\href{https://www.nytimes3xbfgragh.onion/2018/07/25/business/hearst-magazines-president.html}{Mr.
Young succeeded David Carey} as Hearst Magazines president in 2018,
winning the job over the high-profile former editor and magazine
executive
\href{https://www.nytimes3xbfgragh.onion/2018/08/06/business/media/joanna-coles-troy-young-hearst.html}{Joanna
Coles}.

That promotion came after at least four employees had complained about
what they described as Mr. Young's bullying or harassing behavior to the
human resources department or senior executives, according to four
former Hearst employees, who spoke on the condition of anonymity because
they feared retaliation.

One incident involving Mr. Young occurred during a visit to the
Cosmopolitan office when he was the digital head, according to two
people who were present. Mr. Young picked up one of the sex toys that
had been sent to the magazine and asked if he could keep it, the people
said. Referring to the openings of two toys, he said he would
``definitely need the bigger one,'' the people said.

Mr. Young also emailed pornography to a high-level Hearst editor, Jay
Fielden, according to three people with knowledge of what happened. Mr.
Fielden complained to Mr. Carey, who was then the division president,
the people said. In May last year,
\href{https://www.nytimes3xbfgragh.onion/2019/05/23/business/media/esquires-editor-fielden-hearst-magazines.html}{Mr.
Fielden left} Hearst, where he had been the top editor of Esquire and
Town \& Country. He declined to comment for this article.

At a Cosmopolitan holiday party in 2013, Mr. Young joined a group in
which a young staff member was describing a bad date with a man who
complained of an ex-girlfriend's odor. The woman, who spoke on the
condition of anonymity to describe a sensitive conversation, said Mr.
Young had told her that she should have inserted her fingers into
herself and asked her date if he liked her smell. The woman said she was
shocked by his comment and walked away.

Two Esquire staff members witnessed the incident: Nate Hopper, an
assistant editor at the time, and Ben Collins, an editor who is now a
reporter for NBC. Both confirmed the Cosmopolitan staff member's
recollection.

``I think he violated the decency of what was otherwise a friendly
conversation,'' Mr. Hopper said. ``It has been something that I wish I
had done something about in the moment for a very long time.''

Mr. Young, 52, addressed the former Hearst employees' complaints in a
statement for this article: ``Specific allegations raised by my
detractors are either untrue, greatly exaggerated or taken out of
context. The pace of evolving our business and the strength of my
commitment is ambitious, and I sincerely regret the toll it has taken on
some in our organization.''

As for the holiday party, he said in a separate statement, ``Candid
conversations about sex defined the Cosmo brand for decades, and those
who worked there discussed it openly.'' He did not address the other
specific allegations.

A Hearst Magazines spokeswoman said that, during Mr. Young's years as
digital chief, his ``relentless pursuit of excellence was at times
combined with a brash demeanor that rubbed some the wrong way.'' The
spokeswoman added, ``Since being named president of the division, he has
worked to develop a more inclusive management style.''

\includegraphics{https://static01.graylady3jvrrxbe.onion/images/2020/07/02/business/00Unrest-Hearst-pels/merlin_156275811_03d9a513-a4de-4b8c-b041-c92c66ee9da0-articleLarge.jpg?quality=75\&auto=webp\&disable=upscale}

As part of the shake-up on Mr. Young's watch,
\href{https://www.nytimes3xbfgragh.onion/2019/04/05/style/cosmopolitan-magazine-jessica-pels.html}{Jessica
Pels} became Cosmopolitan's youngest top editor in 2018. Before that,
she had run the magazine's digital side and was the digital head of
Marie Claire. During the recent weeks of protests against racism and
police violence, Ms. Pels has faced staff members' demands for action on
what they described as a culture of discrimination that has long been
ignored.

Ms. Pels held staff videoconferences in the wake of social media
comments posted last month by Jazmin Jones, who had worked under Ms.
Pels as a video editor at Marie Claire. In an Instagram
\href{https://www.instagram.com/p/CBGjWu1jXQC/?utm_source=ig_web_copy_link}{post},
Ms. Jones, who is Black, accused the company of racial discrimination,
saying she was made to feel uncomfortable in threads that touched on
race in the interoffice communications app Slack.

A screen shot of a Slack conversation posted by Ms. Jones shows an
editor, whom she identified as Ms. Pels, commenting disparagingly on the
hair and makeup of a staff member of color during an on-camera
appearance for a Marie Claire video. Ms. Pels sprinkled the Slack
conversation with remarks that she was committing a human resources
violation by making the complaint.

In an interview, Ms. Jones, 30, said, ``Hearst doesn't care about you if
you're not a skinny white lady.''

During a videoconference last month for the Cosmopolitan staff, a woman
of color confronted Ms. Pels over being pulled into meetings she would
not normally have been part of when camera crews were present. She said
her inclusion was evidence of the company's attempt to promote a false
appearance of staff diversity, according to a recording of the meeting
obtained by The New York Times.

Prachi Gupta, who covered politics for the Cosmopolitan site during the
2016 presidential campaign, before Ms. Pels became editor, said she felt
that Black and brown women were made to ``feel less than equal'' at the
company.

``Because there were no women of color in leadership positions, I was
not able to seek advice or counsel when I was pushed into some of the
uncomfortable positions,'' she said.

In a June 6
\href{https://twitter.com/prachigu/status/1269362844522426371?s=20}{Twitter
post}, Ms. Gupta, who is Indian-American, wrote: ``From the get-go, I
was tokenized. A white P.R. person at Hearst told me that it would be
easy to book me for media appearances because my look was `very on
trend,' and it was clear she meant that I wasn't white.''

Image

The former Cosmopolitan reporter Prachi Gupta, shown here at the Writers
Guild Awards in New York this year, said she had felt ``tokenized'' at
Hearst.Credit...Jamie McCarthy/Getty Images

Ms. Jones's and Ms. Gupta's descriptions of their experiences were
echoed by 10 former and current Hearst Magazines staff members in
interviews with The Times.

In a videoconference staff meeting, Ms. Pels offered tearful apologies.
``I have not done enough to correct imbalances,'' she said, according to
an audio recording of the meeting obtained by The Times.

In a statement for this article, Ms. Pels said diversity was a
``career-long priority for me.''

``At this pivotal moment, my team and I have been making real changes
and having extensive, honest and passionate discussions about the
progress that needs to be made, and the work I can do as a leader to
actively facilitate it,'' she said in the statement.

As Cosmopolitan's top editor, Ms. Pels has conducted interviews with
Democratic presidential candidates and published an essay in favor of
the Black Lives Matter movement by Senator
\href{https://www.cosmopolitan.com/politics/a32766156/kamala-harris-black-lives-matter-protests/}{Kamala
Harris} of California.

Image

Hearst Tower, in New York, is the home of the magazines Good
Housekeeping, Harper's Bazaar, Town \& Country, Cosmopolitan, and
more.Credit...Hiroko Masuike/The New York Times

Hearst employees have questioned company leadership at a time when
employees at its more glamorous rival,
\href{https://www.nytimes3xbfgragh.onion/2020/06/13/business/media/conde-nast-racial.html}{Condé
Nast}, have done the same. There have also been staff revolts at other
media organizations, including
\href{https://www.nytimes3xbfgragh.onion/2020/06/07/business/media/james-bennet-resigns-nytimes-op-ed.html}{The
Times},
\href{https://www.nytimes3xbfgragh.onion/2020/06/06/business/media/editor-philadephia-inquirer-resigns.html}{The
Philadelphia Inquirer},
\href{https://www.nytimes3xbfgragh.onion/2020/07/10/business/media/wall-street-journal-staff.html}{The
Wall Street Journal} and
\href{https://www.nytimes3xbfgragh.onion/2020/06/08/business/media/refinery-29-christene-barberich.html}{Refinery29}.

Last month Hearst Magazines appointed
\href{https://www.nytimes3xbfgragh.onion/2020/06/09/business/media/harpers-bazaar-editor-samira-nasr.html}{Samira
Nasr}, previously Vanity Fair's fashion director, as the top editor of
the U.S. edition of Harper's Bazaar. She is the first woman of color to
hold the post. And Cosmopolitan started an initiative, ``Cosmo Can Do
Better,'' that calls for the hiring of more Black people and people of
color. As part of it, the magazine released staff statistics, saying its
work force was made up of 29 percent Black people and people of color,
61 percent white employees, with 10 percent undisclosed. Its leadership
comprised 21 percent people of color, the survey said. A Hearst
spokeswoman said the company is committed to diversity at all levels.

Michelle Ruiz, a former senior editor at Cosmopolitan, said the messages
of inclusion and empowerment from many Hearst publications were at odds
with company leadership. She described an encounter with Mr. Young at
the Hearst cafeteria that took place when she was heavily pregnant.
``So, is the baby mine?'' he said, as she recalled it.

``For an executive at the company to suggest that he'd impregnated me
was clearly inappropriate,'' said Ms. Ruiz, now a contributing editor at
Vogue.com. ``There's a real hypocrisy to elevating this man to lead a
company populated with magazines that are preaching women's empowerment
on their covers.''

Advertisement

\protect\hyperlink{after-bottom}{Continue reading the main story}

\hypertarget{site-index}{%
\subsection{Site Index}\label{site-index}}

\hypertarget{site-information-navigation}{%
\subsection{Site Information
Navigation}\label{site-information-navigation}}

\begin{itemize}
\tightlist
\item
  \href{https://help.nytimes3xbfgragh.onion/hc/en-us/articles/115014792127-Copyright-notice}{©~2020~The
  New York Times Company}
\end{itemize}

\begin{itemize}
\tightlist
\item
  \href{https://www.nytco.com/}{NYTCo}
\item
  \href{https://help.nytimes3xbfgragh.onion/hc/en-us/articles/115015385887-Contact-Us}{Contact
  Us}
\item
  \href{https://www.nytco.com/careers/}{Work with us}
\item
  \href{https://nytmediakit.com/}{Advertise}
\item
  \href{http://www.tbrandstudio.com/}{T Brand Studio}
\item
  \href{https://www.nytimes3xbfgragh.onion/privacy/cookie-policy\#how-do-i-manage-trackers}{Your
  Ad Choices}
\item
  \href{https://www.nytimes3xbfgragh.onion/privacy}{Privacy}
\item
  \href{https://help.nytimes3xbfgragh.onion/hc/en-us/articles/115014893428-Terms-of-service}{Terms
  of Service}
\item
  \href{https://help.nytimes3xbfgragh.onion/hc/en-us/articles/115014893968-Terms-of-sale}{Terms
  of Sale}
\item
  \href{https://spiderbites.nytimes3xbfgragh.onion}{Site Map}
\item
  \href{https://help.nytimes3xbfgragh.onion/hc/en-us}{Help}
\item
  \href{https://www.nytimes3xbfgragh.onion/subscription?campaignId=37WXW}{Subscriptions}
\end{itemize}
