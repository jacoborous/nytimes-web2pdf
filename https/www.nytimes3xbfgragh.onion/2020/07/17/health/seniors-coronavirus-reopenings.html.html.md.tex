Sections

SEARCH

\protect\hyperlink{site-content}{Skip to
content}\protect\hyperlink{site-index}{Skip to site index}

\href{https://www.nytimes3xbfgragh.onion/section/health}{Health}

\href{https://myaccount.nytimes3xbfgragh.onion/auth/login?response_type=cookie\&client_id=vi}{}

\href{https://www.nytimes3xbfgragh.onion/section/todayspaper}{Today's
Paper}

\href{/section/health}{Health}\textbar{}You're a Senior. How Do You
Calculate Coronavirus Risk Right Now?

\url{https://nyti.ms/32pQlum}

\begin{itemize}
\item
\item
\item
\item
\item
\item
\end{itemize}

\href{https://www.nytimes3xbfgragh.onion/spotlight/at-home?action=click\&pgtype=Article\&state=default\&region=TOP_BANNER\&context=at_home_menu}{At
Home}

\begin{itemize}
\tightlist
\item
  \href{https://www.nytimes3xbfgragh.onion/2020/07/28/books/time-for-a-literary-road-trip.html?action=click\&pgtype=Article\&state=default\&region=TOP_BANNER\&context=at_home_menu}{Take:
  A Literary Road Trip}
\item
  \href{https://www.nytimes3xbfgragh.onion/2020/07/29/magazine/bored-with-your-home-cooking-some-smoky-eggplant-will-fix-that.html?action=click\&pgtype=Article\&state=default\&region=TOP_BANNER\&context=at_home_menu}{Cook:
  Smoky Eggplant}
\item
  \href{https://www.nytimes3xbfgragh.onion/2020/07/27/travel/moose-michigan-isle-royale.html?action=click\&pgtype=Article\&state=default\&region=TOP_BANNER\&context=at_home_menu}{Look
  Out: For Moose}
\item
  \href{https://www.nytimes3xbfgragh.onion/interactive/2020/at-home/even-more-reporters-editors-diaries-lists-recommendations.html?action=click\&pgtype=Article\&state=default\&region=TOP_BANNER\&context=at_home_menu}{Explore:
  Reporters' Obsessions}
\end{itemize}

Advertisement

\protect\hyperlink{after-top}{Continue reading the main story}

Supported by

\protect\hyperlink{after-sponsor}{Continue reading the main story}

The New Old Age

\hypertarget{youre-a-senior-how-do-you-calculate-coronavirus-risk-right-now}{%
\section{You're a Senior. How Do You Calculate Coronavirus Risk Right
Now?}\label{youre-a-senior-how-do-you-calculate-coronavirus-risk-right-now}}

Early in the pandemic, older adults were told to stay at home. With
different states reopening or re-closing, weighing the risks is more
complicated.

\includegraphics{https://static01.graylady3jvrrxbe.onion/images/2020/07/21/science/17SCI-SPAN/merlin_174505056_72b2a307-760a-44b3-8c31-8100fb181412-articleLarge.jpg?quality=75\&auto=webp\&disable=upscale}

By \href{https://www.nytimes3xbfgragh.onion/by/paula-span}{Paula Span}

\begin{itemize}
\item
  July 17, 2020
\item
  \begin{itemize}
  \item
  \item
  \item
  \item
  \item
  \item
  \end{itemize}
\end{itemize}

What to do about Lake Placid?

For weeks, Dave and Nancy Nathan had been debating whether to proceed
with a long-planned family trip to a lodge there next month, marking his
80th birthday.

``It looked dreamy, mountains and lakes,'' said Nancy, 74. Besides, they
hadn't gathered their clan --- three daughters and their families, a
dozen people in all --- for a year. She thought she and Dave could
manage the drive from their home in Bethesda, Md., to upstate New York.

He wasn't so sure.

Both retirees, they'd been cautious through the pandemic, mindful that
while neither had health conditions that would make Covid-19 especially
dangerous,
\href{https://www.nytimes3xbfgragh.onion/2020/07/08/health/coronavirus-risk-factors.html}{age
alone put them at higher risk}. They had avoided supermarkets, relying
on grocery delivery services and take out food. Dave wore gloves on the
tennis court.

``I've been dubious about travel,'' he said. ``I have no need to be more
daring.'' Worried, too, about the family members flying from Oregon and
Florida for his birthday, he called himself Dr. No.

``It's not fun for him, or anyone, if he's always looking over his
shoulder,'' Nancy said, sympathizing. Still, she hoped they could go.

Early on in the pandemic, most public health officials warned older
adults to simply stay at home, except to buy food or medicine or
exercise outdoors apart from others. Now, with states and cities
reopening (and some re-closing) at varying paces, the calculations grow
steadily more complicated.

``Lots of people are really agonizing about what to do and whom to have
faith in,'' said Dr. William Schaffner, an infectious disease specialist
at Vanderbilt University.

The Centers for Disease Control and Prevention has reported, based on
March data, that Covid-19
\href{https://www.cdc.gov/mmwr/volumes/69/wr/mm6915e3.htm}{hospitalizations
rise with age}, from about 12 per 100,000 people among those 65 to 74
years old to 17 per 100,000 for those over 85. And
\href{https://www.nytimes3xbfgragh.onion/2020/07/08/health/coronavirus-risk-factors.html}{a
large study} from England has reported that patients over 80 are at
least 20 times more likely to die than those in their 50s.

While the risk of contracting the new coronavirus appears no higher for
people over 65, ``once you get an infection, the virus is much
nastier,'' said Dr. Schaffner, an older adult himself.

``Even if we recover, there's the possibility that we never get back to
the same level of physical and mental competence we had,'' he added.

Given that prospect, do you get a haircut? Dr. Schaffner has decided he
will, wearing a surgical mask and knowing his longtime stylist will take
``meticulous'' precautions.

The Nathans' book group has been meeting on Zoom. Can the four couples
now meet in a back yard? The members agreed, as long as everyone
distanced.

``The least risky thing is to stay home, lock the door and seal yourself
in Saran Wrap,'' Dr. Schaffner said.

Though he was being sardonic, economists at M.I.T. came close to
endorsing that strategy (minus the plastic wrap) in a recent
\href{https://www.nber.org/papers/w27102}{paper suggesting age-targeted
lockdowns}. They proposed protecting people over 65 by having them
isolate for an estimated 18 months until a vaccine becomes available;
younger people, facing less health risk, would return to work.

``We could have both way fewer deaths and way less economic pain,'' said
Michael Whinston, a co-author. In March, when he and three colleagues
developed their model, they wanted to avert two extreme prospects: a
projected 2 million American deaths if the country didn't shut down;
economic devastation, if it did.

But their approach also assumes that older adults' only interest lies in
not dying.

``We have to find a balance between preserving safety and living,'' said
Dr. Linda Fried, a geriatrician and the dean of the Mailman School of
Public Health at Columbia University. ``We all need to do some things to
maintain our mental health and well-being.''

\textbf{\emph{{[}}\href{http://on.fb.me/1paTQ1h}{\emph{Like the Science
Times page on Facebook.}}} ****** \emph{\textbar{} Sign up for the}
\textbf{\href{http://nyti.ms/1MbHaRU}{\emph{Science Times
newsletter.}}\emph{{]}}}

Normally, Dr. Fried pointed out, seniors would find decision-making less
knotty because the C.D.C. would be providing detailed,
\href{https://www.cdc.gov/coronavirus/2019-ncov/need-extra-precautions/older-adults.html}{science-based
guidance for at-risk groups}, updated weekly.

``It's immensely atypical, I believe unprecedented, that we're not
seeing this,'' she said. Without that leadership, seniors confront a
crazy quilt of changing state and local policies, and ``everyone's on
their own.''

That means older people need to consider their individual health status
when deciding which risks to take. Their less robust immune systems make
it harder to bounce back from serious infection. They're also more apt
to have the underlying conditions --- diabetes, serious heart, lung or
kidney disease --- shown to increase severe illness and
hospitalizations. People of color, obese people and men face higher
risk.

``If you're a vibrant older person without chronic illnesses, you're
probably a little more resilient,'' said Dr. Fried, quickly adding that
``there are no guarantees.''

A \href{https://riskcalc.org/COVID19/}{calculator}
\href{https://pubmed.ncbi.nlm.nih.gov/32533957/}{developed by
researchers at the Cleveland Clinic} may provide a clearer sense of
individual risk.

Geography matters too. Older people in New Hampshire or Maine --- where
new cases were falling last week --- may reasonably opt for less
restrictive behavior than those in Florida and Arizona, where Covid has
been surging. Pay attention to
\href{https://www.nytimes3xbfgragh.onion/interactive/2020/us/coronavirus-us-cases.html}{which
counties are seeing cases rise} and which are doing
\href{https://www.unacast.com/covid19/social-distancing-scoreboard}{a
good job at observing guidelines}.)

``You base what you do on where you are,'' said Dr. Nathaniel Hupert,
the co-director of the Institute for Disease and Disaster Preparedness
at Weill Cornell Medicine, who advises New York State's Covid task
force.

Personal calculations also include what seems most important. Dr.
Schaffner and his wife visited her son, daughter-in-law and two
grandchildren recently for the first time in months. They sat indoors
for two hours --- Nashville was too hot for outside socializing ---
wearing masks, sitting apart, not eating or drinking. But they're nixing
restaurants for now, along with their annual summer trip to Florida.

Even with individual decisions, the basic precautions that public health
leaders have urged on everyone, old and young, still apply. Wearing
masks in public, maintaining at least six feet of distance from others,
avoiding crowds, washing hands --- all help protect oneself and others.

\href{https://www.nytimes3xbfgragh.onion/2020/05/15/us/coronavirus-what-to-do-outside.html}{Outdoor
activities are safer,} but ``if you have to be indoors, open a window or
crack a door,'' Dr. Hupert said. ``Even a small change in the seal of a
room dramatically changes the way aerosols travel and land.''

He also urges his older patients to stay up-to-date on
\href{https://www.cdc.gov/vaccines/schedules/hcp/imz/adult.html\#table-age}{recommended
vaccinations}, which appear to boost immune responses against illnesses
beyond those for which they're intended. ``A whole host of changes in
the body potentially can benefit you for months by ramping up the body's
response to viral diseases,'' Dr. Hupert said.

Older people I spoke with talked about developing individual strategies.
Going places early, before crowds develop. Reserving an establishment's
first appointment of the day. Bringing their own plates, cutlery and
drinks to distanced outdoor dinners.

Dr. Schaffner advocates discussing ground rules before issuing or
accepting invitations, ``so you don't have to make a decision on the
fly.'' Call beforehand to ask about a funeral/barbecue/birthday party.
Will it be indoor or outdoors? Is there room to distance? Will
participants wear masks? If you don't like the answers, you may elect to
meet more safely another time.

In the end, the Nathans abandoned their Lake Placid plans. Why push it,
Nancy said, ``if the birthday boy doesn't want to go.''

But she's holding onto airline and hotel reservations for a September
trip to London. It may not come off --- two of three shows she booked
have already canceled performances --- but ``I have two months to
daydream about it.''

Advertisement

\protect\hyperlink{after-bottom}{Continue reading the main story}

\hypertarget{site-index}{%
\subsection{Site Index}\label{site-index}}

\hypertarget{site-information-navigation}{%
\subsection{Site Information
Navigation}\label{site-information-navigation}}

\begin{itemize}
\tightlist
\item
  \href{https://help.nytimes3xbfgragh.onion/hc/en-us/articles/115014792127-Copyright-notice}{©~2020~The
  New York Times Company}
\end{itemize}

\begin{itemize}
\tightlist
\item
  \href{https://www.nytco.com/}{NYTCo}
\item
  \href{https://help.nytimes3xbfgragh.onion/hc/en-us/articles/115015385887-Contact-Us}{Contact
  Us}
\item
  \href{https://www.nytco.com/careers/}{Work with us}
\item
  \href{https://nytmediakit.com/}{Advertise}
\item
  \href{http://www.tbrandstudio.com/}{T Brand Studio}
\item
  \href{https://www.nytimes3xbfgragh.onion/privacy/cookie-policy\#how-do-i-manage-trackers}{Your
  Ad Choices}
\item
  \href{https://www.nytimes3xbfgragh.onion/privacy}{Privacy}
\item
  \href{https://help.nytimes3xbfgragh.onion/hc/en-us/articles/115014893428-Terms-of-service}{Terms
  of Service}
\item
  \href{https://help.nytimes3xbfgragh.onion/hc/en-us/articles/115014893968-Terms-of-sale}{Terms
  of Sale}
\item
  \href{https://spiderbites.nytimes3xbfgragh.onion}{Site Map}
\item
  \href{https://help.nytimes3xbfgragh.onion/hc/en-us}{Help}
\item
  \href{https://www.nytimes3xbfgragh.onion/subscription?campaignId=37WXW}{Subscriptions}
\end{itemize}
