Sections

SEARCH

\protect\hyperlink{site-content}{Skip to
content}\protect\hyperlink{site-index}{Skip to site index}

\href{https://www.nytimes3xbfgragh.onion/section/business/economy}{Economy}

\href{https://myaccount.nytimes3xbfgragh.onion/auth/login?response_type=cookie\&client_id=vi}{}

\href{https://www.nytimes3xbfgragh.onion/section/todayspaper}{Today's
Paper}

\href{/section/business/economy}{Economy}\textbar{}How to Save a
Half-Open Economy

\url{https://nyti.ms/394HorM}

\begin{itemize}
\item
\item
\item
\item
\item
\end{itemize}

\hypertarget{the-coronavirus-outbreak}{%
\subsubsection{\texorpdfstring{\href{https://www.nytimes3xbfgragh.onion/news-event/coronavirus?name=styln-coronavirus-markets\&region=TOP_BANNER\&variant=undefined\&block=storyline_menu_recirc\&action=click\&pgtype=Article\&impression_id=118e8600-e3aa-11ea-98f9-47cef13f5a6f}{The
Coronavirus
Outbreak}}{The Coronavirus Outbreak}}\label{the-coronavirus-outbreak}}

\begin{itemize}
\tightlist
\item
  live\href{https://www.nytimes3xbfgragh.onion/2020/08/21/world/covid-19-coronavirus.html?name=styln-coronavirus-markets\&region=TOP_BANNER\&variant=undefined\&block=storyline_menu_recirc\&action=click\&pgtype=Article\&impression_id=118ead10-e3aa-11ea-98f9-47cef13f5a6f}{Latest
  Updates}
\item
  \href{https://www.nytimes3xbfgragh.onion/interactive/2020/us/coronavirus-us-cases.html?name=styln-coronavirus-markets\&region=TOP_BANNER\&variant=undefined\&block=storyline_menu_recirc\&action=click\&pgtype=Article\&impression_id=118ead11-e3aa-11ea-98f9-47cef13f5a6f}{Maps
  and Cases}
\item
  \href{https://www.nytimes3xbfgragh.onion/interactive/2020/science/coronavirus-vaccine-tracker.html?name=styln-coronavirus-markets\&region=TOP_BANNER\&variant=undefined\&block=storyline_menu_recirc\&action=click\&pgtype=Article\&impression_id=118ead12-e3aa-11ea-98f9-47cef13f5a6f}{Vaccine
  Tracker}
\item
  \href{https://www.nytimes3xbfgragh.onion/2020/08/19/us/colleges-closing-covid.html?name=styln-coronavirus-markets\&region=TOP_BANNER\&variant=undefined\&block=storyline_menu_recirc\&action=click\&pgtype=Article\&impression_id=118ead13-e3aa-11ea-98f9-47cef13f5a6f}{Colleges
  Closing}
\item
  \href{https://www.nytimes3xbfgragh.onion/live/2020/08/21/business/stock-market-today-coronavirus?name=styln-coronavirus-markets\&region=TOP_BANNER\&variant=undefined\&block=storyline_menu_recirc\&action=click\&pgtype=Article\&impression_id=118ead14-e3aa-11ea-98f9-47cef13f5a6f}{Economy}
\end{itemize}

Advertisement

\protect\hyperlink{after-top}{Continue reading the main story}

Supported by

\protect\hyperlink{after-sponsor}{Continue reading the main story}

\hypertarget{how-to-save-a-half-open-economy}{%
\section{How to Save a Half-Open
Economy}\label{how-to-save-a-half-open-economy}}

It may take months or even years to recover its vigor. Here's how
economists say the government could help.

\includegraphics{https://static01.graylady3jvrrxbe.onion/images/2020/07/17/business/17virus-econ-explain/merlin_174644976_330d7212-aa84-40f1-ad4c-6da0067f1503-articleLarge.jpg?quality=75\&auto=webp\&disable=upscale}

\href{https://www.nytimes3xbfgragh.onion/by/ben-casselman}{\includegraphics{https://static01.graylady3jvrrxbe.onion/images/2018/11/09/multimedia/author-ben-casselman/author-ben-casselman-thumbLarge.png}}\href{https://www.nytimes3xbfgragh.onion/by/jim-tankersley}{\includegraphics{https://static01.graylady3jvrrxbe.onion/images/2018/10/19/multimedia/author-jim-tankersley/author-jim-tankersley-thumbLarge.png}}

By \href{https://www.nytimes3xbfgragh.onion/by/ben-casselman}{Ben
Casselman} and
\href{https://www.nytimes3xbfgragh.onion/by/jim-tankersley}{Jim
Tankersley}

\begin{itemize}
\item
  July 17, 2020
\item
  \begin{itemize}
  \item
  \item
  \item
  \item
  \item
  \end{itemize}
\end{itemize}

When states began to order businesses to close and residents to stay
home as the coronavirus outbreak spread, economists likened the policy
to a medically induced coma: shutting down all but the most vital
functions to focus on the underlying affliction.

Now the patient is awake, but the malady remains.

A surge in coronavirus cases has forced several states to reimpose
restrictions and
\href{https://www.nytimes3xbfgragh.onion/2020/07/15/business/economy/economic-recovery-coronavirus-resurgence.html}{dashed
hopes of a rapid economic rebound}. But a widespread return to the
shutdown policies that dominated in March and April seems unlikely.

Instead, the economy looks likely to remain in a sort of limbo, neither
fully open nor fully shut, for months or even years.

For certain workers in certain industries in certain locations, life
again seems somewhat normal. But for many others --- those whose age or
health conditions make them especially vulnerable to the virus, or who
have young children at home, or who work in high-risk industries, or who
live in places where cases are rising rapidly --- the pandemic remains a
major disruption.

This new phase poses a unique challenge for policymakers. Economists
across the political spectrum say it would be a mistake for the federal
government to cut off support for workers and businesses while the
economy remained weak. But those policies may need to be revamped to
help the worst-hit industries and regions --- and will have to change as
the crisis evolves.

``We don't know how the pandemic is going to unfurl, and we don't know
where the hot spots are going to be,'' said Wendy Edelberg, a former
chief economist for the Congressional Budget Office and now the director
of the Hamilton Project, an economic policy arm of the Brookings
Institution. ``That's going to demand that the policy response be a lot
more nimble.''

Still, economists and other experts say there are steps that government,
at all levels, can take to mitigate the economic damage.

\hypertarget{prioritize-public-health}{%
\subsection{Prioritize public health.}\label{prioritize-public-health}}

\includegraphics{https://static01.graylady3jvrrxbe.onion/images/2020/07/17/business/17virus-econ-explain5/merlin_174526011_bf9b5692-c384-4aed-8a8b-bf2d164338f9-articleLarge.jpg?quality=75\&auto=webp\&disable=upscale}

In the political debate over reopening, economic and public health
considerations are often portrayed as being at odds. But economists have
said since the beginning of the crisis that the two go hand in hand: The
\href{https://www.nytimes3xbfgragh.onion/2020/07/07/business/sweden-economy-coronavirus.html}{economy}
cannot recover until the virus is in check.

``One thing we've learned thus far is that a halfway commitment to
public health measures just isn't very effective,'' said David Wilcox, a
former Federal Reserve official who is an economist at the Peterson
Institute for International Economics. ``It's not effective in arresting
the virus, and you still incur tremendous economic damage. In order to
build the foundation of a secure recovery, the imperative is to bring
the virus under control.''

\hypertarget{latest-updates-the-coronavirus-outbreak-and-the-economy}{%
\section{\texorpdfstring{\href{https://www.nytimes3xbfgragh.onion/live/2020/08/21/business/stock-market-today-coronavirus?action=click\&pgtype=Article\&state=default\&region=MAIN_CONTENT_1\&context=storylines_live_updates}{Latest
Updates: The Coronavirus Outbreak and the
Economy}}{Latest Updates: The Coronavirus Outbreak and the Economy}}\label{latest-updates-the-coronavirus-outbreak-and-the-economy}}

\href{https://www.nytimes3xbfgragh.onion/live/2020/08/21/business/stock-market-today-coronavirus?action=click\&pgtype=Article\&state=default\&region=MAIN_CONTENT_1\&context=storylines_live_updates\#government-debt-in-britain-exceeds-2-trillion-for-the-first-time}{2h
ago}

\href{https://www.nytimes3xbfgragh.onion/live/2020/08/21/business/stock-market-today-coronavirus?action=click\&pgtype=Article\&state=default\&region=MAIN_CONTENT_1\&context=storylines_live_updates\#government-debt-in-britain-exceeds-2-trillion-for-the-first-time}{Government
debt in Britain exceeds £2 trillion for the first time.}

\href{https://www.nytimes3xbfgragh.onion/live/2020/08/21/business/stock-market-today-coronavirus?action=click\&pgtype=Article\&state=default\&region=MAIN_CONTENT_1\&context=storylines_live_updates\#us-stock-futures-waver-as-global-markets-fall}{2h
ago}

\href{https://www.nytimes3xbfgragh.onion/live/2020/08/21/business/stock-market-today-coronavirus?action=click\&pgtype=Article\&state=default\&region=MAIN_CONTENT_1\&context=storylines_live_updates\#us-stock-futures-waver-as-global-markets-fall}{U.S.
stock futures waver as global markets fall.}

\href{https://www.nytimes3xbfgragh.onion/live/2020/08/21/business/stock-market-today-coronavirus?action=click\&pgtype=Article\&state=default\&region=MAIN_CONTENT_1\&context=storylines_live_updates\#americas-national-debt-has-surpassed-the-gdp-but-no-one-seems-to-care}{2h
ago}

\href{https://www.nytimes3xbfgragh.onion/live/2020/08/21/business/stock-market-today-coronavirus?action=click\&pgtype=Article\&state=default\&region=MAIN_CONTENT_1\&context=storylines_live_updates\#americas-national-debt-has-surpassed-the-gdp-but-no-one-seems-to-care}{America's
national debt has surpassed the G.D.P., but no one seems to care.}

\href{https://www.nytimes3xbfgragh.onion/live/2020/08/21/business/stock-market-today-coronavirus?action=click\&pgtype=Article\&state=default\&region=MAIN_CONTENT_1\&context=storylines_live_updates}{See
more updates}

More live coverage:
\href{https://www.nytimes3xbfgragh.onion/2020/08/21/world/covid-19-coronavirus.html?action=click\&pgtype=Article\&state=default\&region=MAIN_CONTENT_1\&context=storylines_live_updates}{Global}

If political leaders want businesses to reopen and customers to return,
Mr. Wilcox said, they need to invest in widespread testing and tracing
to make consumers confident that they are safe. And they need to avoid
encouraging businesses to reopen before it is safe to do so.

\hypertarget{extend-unemployment-benefits}{%
\subsection{Extend unemployment
benefits.}\label{extend-unemployment-benefits}}

Image

A line for unemployment benefits last month in San Juan, P.R. A federal
program providing an extra \$600 a week is set to expire on July
31.Credit...Dennis M. Rivera Pichardo for The New York Times

More than 20 million Americans are getting an extra \$600 a week in
their unemployment checks because of the federal aid package passed in
March, but that provision is scheduled to expire this month. While some
economists say the enhanced benefits could be scaled back or modified,
most say it would be a mistake to let them lapse altogether.

Unemployment benefits are serving three purposes. In states where the
virus is raging, they help residents afford to stay home, which is
crucial to overcoming the pandemic. In all states, they help jobless
workers avoid hunger, eviction and financial ruin. And by providing
billions of dollars to the people most likely to spend it, they
stimulate the economy.

The extra \$600 means that many low-wage workers are earning more on
unemployment than they were on the job, which Republicans in Congress
worry could discourage returning to work. Economists say that is a valid
concern --- when the unemployment rate is low and workers are scarce.
Right now, the situation is the opposite: In May, there were roughly
five million open jobs and 20 million unemployed workers.

``There's not enough jobs for everybody anyway,'' said Erik Hurst, an
economist at the University of Chicago who has been studying the
economic effects of the pandemic.

Some economists, particularly on the right, say it may make sense to
reduce the weekly supplement as the economy improves, and some have
suggested tweaks like a ``back-to-work bonus'' that rewards people for
finding jobs. But few think it makes sense to scrap the enhanced
benefits.

\hypertarget{spend-what-it-takes-to-reopen-schools-safely}{%
\subsection{Spend what it takes to reopen schools
safely.}\label{spend-what-it-takes-to-reopen-schools-safely}}

Image

Dividers on desks for social distancing at St. Benedict School in
Montebello, Calif.Credit...Lucy Nicholson/Reuters

Whether or not to reopen schools this fall has become a political point
of contention in recent days. But economists say there is no doubt about
one thing: The economy can't get back to normal while millions who would
otherwise be working must stay at home caring for their school-age
children.

Epidemiologists and public health experts are unsure that in-person
classes can be held safely in places where the virus is out of control,
like Florida and Arizona. But in other places, the biggest obstacle is
money: It would cost billions of dollars to retrofit classrooms,
overhaul ventilation systems, buy protective equipment and add staff
members to ensure that both children and adults were safe.

``Schools are going to need a lot more resources to get open safely,
given we haven't gotten the virus under control in a lot of places,''
said Melissa Kearney, a University of Maryland economist who heads the
Economic Strategy Group at the Aspen Institute. ``The less control we
have over this virus, the more expensive it's going to be.''

State and local governments, reeling from plummeting tax revenues, don't
have the resources for such changes. But the federal government does.
And it could be money well spent: Allowing schools to reopen safely
would free up adults for work and allow other economic activity to
resume.

\hypertarget{keep-businesses-alive}{%
\subsection{Keep businesses alive.}\label{keep-businesses-alive}}

Image

New approaches to save businesses are needed from the federal
government, some experts say.Credit...Travis Dove for The New York Times

Even in states where the virus is less prevalent, some businesses, like
indoor bars, movie theaters and concert venues, may not be able to open
safely for a long time. Others, like restaurants, will have to operate
at a capacity unlikely to turn a profit.

That means that without government help, thousands of businesses are
likely to fail in the months ahead. That could have devastating economic
consequences, turning temporary furloughs into permanent job losses and
slowing the eventual recovery.

Lost jobs ``are going to come back very slowly --- it's going to be
months and months of hard work,'' said Betsey Stevenson, a University of
Michigan economist who was on President Barack Obama's Council of
Economic Advisers. ``The question is, do we have 30 million people who
are going to go through that process, or do we have five million? We
don't have the answer to that yet, but every month it goes on, that
number grows larger.''

Experts say Congress needs a new approach to save businesses.

The Economic Innovation Group, a Washington think tank focused on
entrepreneurship, has proposed giving
\href{https://eig.org/news/main-street-rescue-and-resiliency-program}{interest-free
loans} to small employers. Rather than providing a temporary injection
of cash, they argued, a loan program could let companies invest in
improving their long-term prospects. A retailer could buy a building it
had been renting, for example, bringing down monthly costs. Or a
restaurant could add outdoor space, reducing dependence on indoor
dining.

Mr. Wilcox of the Peterson Institute has
\href{https://thehill.com/opinion/finance/493656-unclogging-the-financial-pipeline-to-us-workers-and-businesses}{recommended}
a more expansive --- and expensive --- approach, essentially having the
government fill in the revenue shortfall created by the pandemic through
direct grants to businesses. The government has effectively forced
business owners to take a hit, he said, so it should help them survive.

``Start from a social agreement that the government is going to take
onto its shoulders the cost of sustaining businesses through the period
of intense public health crisis,'' he said.

\hypertarget{provide-some-certainty}{%
\subsection{Provide some certainty.}\label{provide-some-certainty}}

Image

A store in downtown Manhattan. It's hard for businesses to plan
effectively, an economist said.Credit...Hiroko Masuike/The New York
Times

No one knows where and when cases will surge, how long the pandemic will
last, or when a vaccine will be ready. That makes it harder for both
businesses and policymakers to plan effectively, said Martha Gimbel, an
economist and a labor market expert at Schmidt Futures, a philanthropic
initiative.

``If we knew we were going to have a vaccine in January, we could make
decisions,'' she said. ``If we knew we were going to have a vaccine in
January 2022, we could make decisions. But we don't know, and economies
don't do well when there's uncertainty.''

Economic policy can't eliminate that uncertainty. But right now, it is
making it worse: Jobless workers don't know whether their extra benefits
will run out in a matter of days. Businesses don't know if they will be
able to apply for a new round of federal loans, or have to enroll in a
new program, or get nothing at all. State and local governments are
trying to plug multibillion-dollar budget holes with no idea whether
they will get federal help, or how much.

Economists have urged Congress to answer some of those questions --- not
just now, but for the future. Benefits could be linked to the
unemployment rate, for example, so that workers would not have to worry
about losing benefits before the job market improved. Similar steps,
linked to different metrics, could make businesses and state and local
governments confident that government support won't evaporate without
warning.

Brinkmanship, on the other hand, could have economic costs even if
Congress ends up extending support at the last moment.

Advertisement

\protect\hyperlink{after-bottom}{Continue reading the main story}

\hypertarget{site-index}{%
\subsection{Site Index}\label{site-index}}

\hypertarget{site-information-navigation}{%
\subsection{Site Information
Navigation}\label{site-information-navigation}}

\begin{itemize}
\tightlist
\item
  \href{https://help.nytimes3xbfgragh.onion/hc/en-us/articles/115014792127-Copyright-notice}{©~2020~The
  New York Times Company}
\end{itemize}

\begin{itemize}
\tightlist
\item
  \href{https://www.nytco.com/}{NYTCo}
\item
  \href{https://help.nytimes3xbfgragh.onion/hc/en-us/articles/115015385887-Contact-Us}{Contact
  Us}
\item
  \href{https://www.nytco.com/careers/}{Work with us}
\item
  \href{https://nytmediakit.com/}{Advertise}
\item
  \href{http://www.tbrandstudio.com/}{T Brand Studio}
\item
  \href{https://www.nytimes3xbfgragh.onion/privacy/cookie-policy\#how-do-i-manage-trackers}{Your
  Ad Choices}
\item
  \href{https://www.nytimes3xbfgragh.onion/privacy}{Privacy}
\item
  \href{https://help.nytimes3xbfgragh.onion/hc/en-us/articles/115014893428-Terms-of-service}{Terms
  of Service}
\item
  \href{https://help.nytimes3xbfgragh.onion/hc/en-us/articles/115014893968-Terms-of-sale}{Terms
  of Sale}
\item
  \href{https://spiderbites.nytimes3xbfgragh.onion}{Site Map}
\item
  \href{https://help.nytimes3xbfgragh.onion/hc/en-us}{Help}
\item
  \href{https://www.nytimes3xbfgragh.onion/subscription?campaignId=37WXW}{Subscriptions}
\end{itemize}
