Sections

SEARCH

\protect\hyperlink{site-content}{Skip to
content}\protect\hyperlink{site-index}{Skip to site index}

\href{https://www.nytimes3xbfgragh.onion/section/world/australia}{Australia}

\href{https://myaccount.nytimes3xbfgragh.onion/auth/login?response_type=cookie\&client_id=vi}{}

\href{https://www.nytimes3xbfgragh.onion/section/todayspaper}{Today's
Paper}

\href{/section/world/australia}{Australia}\textbar{}Melbourne Lockdown
Elicits Hope, Anger and Vows of Resilience

\url{https://nyti.ms/2Cn57rl}

\begin{itemize}
\item
\item
\item
\item
\item
\end{itemize}

\href{https://www.nytimes3xbfgragh.onion/news-event/coronavirus?action=click\&pgtype=Article\&state=default\&region=TOP_BANNER\&context=storylines_menu}{The
Coronavirus Outbreak}

\begin{itemize}
\tightlist
\item
  live\href{https://www.nytimes3xbfgragh.onion/2020/08/01/world/coronavirus-covid-19.html?action=click\&pgtype=Article\&state=default\&region=TOP_BANNER\&context=storylines_menu}{Latest
  Updates}
\item
  \href{https://www.nytimes3xbfgragh.onion/interactive/2020/us/coronavirus-us-cases.html?action=click\&pgtype=Article\&state=default\&region=TOP_BANNER\&context=storylines_menu}{Maps
  and Cases}
\item
  \href{https://www.nytimes3xbfgragh.onion/interactive/2020/science/coronavirus-vaccine-tracker.html?action=click\&pgtype=Article\&state=default\&region=TOP_BANNER\&context=storylines_menu}{Vaccine
  Tracker}
\item
  \href{https://www.nytimes3xbfgragh.onion/interactive/2020/07/29/us/schools-reopening-coronavirus.html?action=click\&pgtype=Article\&state=default\&region=TOP_BANNER\&context=storylines_menu}{What
  School May Look Like}
\item
  \href{https://www.nytimes3xbfgragh.onion/live/2020/07/31/business/stock-market-today-coronavirus?action=click\&pgtype=Article\&state=default\&region=TOP_BANNER\&context=storylines_menu}{Economy}
\end{itemize}

Advertisement

\protect\hyperlink{after-top}{Continue reading the main story}

Supported by

\protect\hyperlink{after-sponsor}{Continue reading the main story}

Letter 167

\hypertarget{melbourne-lockdown-elicits-hope-anger-and-vows-of-resilience}{%
\section{Melbourne Lockdown Elicits Hope, Anger and Vows of
Resilience}\label{melbourne-lockdown-elicits-hope-anger-and-vows-of-resilience}}

Readers respond to the situation in Australia.

\includegraphics{https://static01.graylady3jvrrxbe.onion/images/2020/07/17/world/17ausletter167-1/17ausletter167-1-articleLarge-v2.jpg?quality=75\&auto=webp\&disable=upscale}

By Besha Rodell

\begin{itemize}
\item
  July 17, 2020
\item
  \begin{itemize}
  \item
  \item
  \item
  \item
  \item
  \end{itemize}
\end{itemize}

\href{https://www.nytimes3xbfgragh.onion/series/nyt-australia-newsletter?module=inline}{\emph{The
Australia Letter}} \emph{is a weekly newsletter from our Australia
bureau.}

\begin{center}\rule{0.5\linewidth}{\linethickness}\end{center}

\href{https://www.nytimes3xbfgragh.onion/2020/07/10/world/australia/melbourne-lockdown.html}{\emph{Last
week's letter, about the mood in Melbourne}} \emph{during our second
strict lockdown, prompted a flood of feedback from readers. People
shared with us how they're coping, thanked us for mirroring their
perspective and criticized us for a lack of self-awareness. Here is a
sampling of those reader responses:}

During the first lockdown, we were all truly ``in this together.'' Now
it's just metro Melbourne and the comments from the rest of the country
do make it feel like we've done something wrong here, although I don't
think Melburnians have been more complacent than other parts of the
country. Having gyms close again was really rough. The first time I went
to the gym since they shut, I felt better than I had since March.

That said, we're going to get through this stronger. It was pretty
likely that this was going to happen, so if we're first in the country
to get our second wave (I don't think we'll be the last), let's get on
with it. Last time, the never-ending uncertainty was rough, while at
least now I can circle a date in the calendar.

--- \textbf{Kissairis Muñoz}

---

I missed my 75th birthday celebrations with my extended family, and our
50th wedding anniversary celebrations, thanks to COVID-19 isolation. So
sad. But then, there is next year to celebrate. Hopefully.

--- \textbf{Joe Slaven, Townsville, Australia}

---

The piece on Melbourne was complete tripe that appears to be written by
some poor soul with no resilience and nothing to do but lay about
wallowing in their own self-pity. This is not reflective of all
Melburnians and I'd suggest not really indicative of most Melbourne
residents. Cozy ``fires at the pub,'' country trips in winter --- what
an idealistic load of rubbish doing nothing other than to paint an
unrealistic picture of this city and it's inhabitants!

--- \textbf{Mark Pepper}

---

There is a stark difference between Lockdown 1.0 and Lockdown 2.0. The
first time around it seemed that everyone was making an effort to find
the funny, irreverent and touching moments to brighten all our days.
Then as lockdown started to ease, ugliness started to creep in; in
emotions, in behavior, in words, in government rhetoric. Now in
Melbourne there is very deep anger, anxiety and a sense of division.

So what do I do to try and remain positive?

I am spending less time watching the news, catching highlights and then
turning it off.

I am dipping into loved books and making my way through a backlog of new
potential favorites.

I am walking --- no matter the weather. This is not negotiable.

\hypertarget{latest-updates-global-coronavirus-outbreak}{%
\section{\texorpdfstring{\href{https://www.nytimes3xbfgragh.onion/2020/08/01/world/coronavirus-covid-19.html?action=click\&pgtype=Article\&state=default\&region=MAIN_CONTENT_1\&context=storylines_live_updates}{Latest
Updates: Global Coronavirus
Outbreak}}{Latest Updates: Global Coronavirus Outbreak}}\label{latest-updates-global-coronavirus-outbreak}}

Updated 2020-08-01T19:54:00.494Z

\begin{itemize}
\tightlist
\item
  \href{https://www.nytimes3xbfgragh.onion/2020/08/01/world/coronavirus-covid-19.html?action=click\&pgtype=Article\&state=default\&region=MAIN_CONTENT_1\&context=storylines_live_updates\#link-3ac56579}{Top
  officials work to break impasse over jobless benefit.}
\item
  \href{https://www.nytimes3xbfgragh.onion/2020/08/01/world/coronavirus-covid-19.html?action=click\&pgtype=Article\&state=default\&region=MAIN_CONTENT_1\&context=storylines_live_updates\#link-8796723}{The
  virus picks up dangerous speed in the Midwest, and in areas that had
  seen success.}
\item
  \href{https://www.nytimes3xbfgragh.onion/2020/08/01/world/coronavirus-covid-19.html?action=click\&pgtype=Article\&state=default\&region=MAIN_CONTENT_1\&context=storylines_live_updates\#link-25930521}{Thousands
  in Berlin protest Germany's coronavirus measures.}
\end{itemize}

\href{https://www.nytimes3xbfgragh.onion/2020/08/01/world/coronavirus-covid-19.html?action=click\&pgtype=Article\&state=default\&region=MAIN_CONTENT_1\&context=storylines_live_updates}{See
more updates}

More live coverage:
\href{https://www.nytimes3xbfgragh.onion/live/2020/07/31/business/stock-market-today-coronavirus?action=click\&pgtype=Article\&state=default\&region=MAIN_CONTENT_1\&context=storylines_live_updates}{Markets}

I am trying to pick up the phone and have conversations instead of
relying on texting and emails. I am touching base with my neighbors. I
am much less inclined to do this than last time, which means it is more
necessary.

I am giving myself the freedom to drink as much coffee as I need to get
through this.

--- \textbf{Michelle}

---

Greetings from Houston, Texas, USA. I normally love these dispatches but
this one gives me pause. We here would love concerns about oysters and
champagne at home, missed sunny vacations and stalled pool laps. But
instead we are grappling with issues associated with a corrupt federal
government which refuses to protect its citizens, political debate
pertaining to masks, a surging virus and economic collapse. The author's
complaints seem trite in comparison. I wish her concerns were the
problems of this country.

--- \textbf{Dana M. Gannon}

---

It was great to read the letter from Besha because something shines out
when somebody writes what is clearly the truth, the exact way she is
feeling.

I hope it engenders in all her readers, as it did for me, a feeling of
sympathy, a kind of loving feeling --- and I hope that she and her
fellow Victorians discover something special and important in the next
few weeks.

Then, when it is all over, come up here to Queensland for a big hug.

--- \textbf{Michael}

---

I have always found winter to be daunting (I grew up on the Connecticut
shore --- talk about gray, cold winters) and have generally huddled
inside for 12 Melbourne winters. But this year is something different.
After the vicious summer bushfires, the rain started, generous but not
imposing. It has kept on for months, and now Melbourne is green and lush
and sooooo gorgeous. With the help of recently purchased electric bikes,
my partner and I have discovered a vast network of bike and walking
paths from our base. Now those chilly days are simply *days* that we
might discover a new route or revisit a favorite.

I grieve for the lost connections and routines, and I am also deeply
grateful to live in a country that cares about people, and I'm knocked
off balance, and I'm worried about our collective mental health. What
helps right the balance for me are the joys of wending along a lush path
on a cool, bright afternoon, of seeing the full sky at sunset, and of
breathing and moving through the world. And having someone to share both
the lockdown and the exploration with --- that makes most things just
fine.

--- \textbf{Katherine Russell}

---

Today's letter, bathed in self-pity and almost entirely lacking
perspective, is a paean to middle class self-centeredness. Boo hoo, you
can't have fake Christmas. Where is the concern for people who are less
fortunate and therefore likelier to get sick and die? Oh, it rains a lot
in winter? Maybe we should take up a collection for your suffering soul!

\href{https://www.nytimes3xbfgragh.onion/news-event/coronavirus?action=click\&pgtype=Article\&state=default\&region=MAIN_CONTENT_3\&context=storylines_faq}{}

\hypertarget{the-coronavirus-outbreak-}{%
\subsubsection{The Coronavirus Outbreak
›}\label{the-coronavirus-outbreak-}}

\hypertarget{frequently-asked-questions}{%
\paragraph{Frequently Asked
Questions}\label{frequently-asked-questions}}

Updated July 27, 2020

\begin{itemize}
\item ~
  \hypertarget{should-i-refinance-my-mortgage}{%
  \paragraph{Should I refinance my
  mortgage?}\label{should-i-refinance-my-mortgage}}

  \begin{itemize}
  \tightlist
  \item
    \href{https://www.nytimes3xbfgragh.onion/article/coronavirus-money-unemployment.html?action=click\&pgtype=Article\&state=default\&region=MAIN_CONTENT_3\&context=storylines_faq}{It
    could be a good idea,} because mortgage rates have
    \href{https://www.nytimes3xbfgragh.onion/2020/07/16/business/mortgage-rates-below-3-percent.html?action=click\&pgtype=Article\&state=default\&region=MAIN_CONTENT_3\&context=storylines_faq}{never
    been lower.} Refinancing requests have pushed mortgage applications
    to some of the highest levels since 2008, so be prepared to get in
    line. But defaults are also up, so if you're thinking about buying a
    home, be aware that some lenders have tightened their standards.
  \end{itemize}
\item ~
  \hypertarget{what-is-school-going-to-look-like-in-september}{%
  \paragraph{What is school going to look like in
  September?}\label{what-is-school-going-to-look-like-in-september}}

  \begin{itemize}
  \tightlist
  \item
    It is unlikely that many schools will return to a normal schedule
    this fall, requiring the grind of
    \href{https://www.nytimes3xbfgragh.onion/2020/06/05/us/coronavirus-education-lost-learning.html?action=click\&pgtype=Article\&state=default\&region=MAIN_CONTENT_3\&context=storylines_faq}{online
    learning},
    \href{https://www.nytimes3xbfgragh.onion/2020/05/29/us/coronavirus-child-care-centers.html?action=click\&pgtype=Article\&state=default\&region=MAIN_CONTENT_3\&context=storylines_faq}{makeshift
    child care} and
    \href{https://www.nytimes3xbfgragh.onion/2020/06/03/business/economy/coronavirus-working-women.html?action=click\&pgtype=Article\&state=default\&region=MAIN_CONTENT_3\&context=storylines_faq}{stunted
    workdays} to continue. California's two largest public school
    districts --- Los Angeles and San Diego --- said on July 13, that
    \href{https://www.nytimes3xbfgragh.onion/2020/07/13/us/lausd-san-diego-school-reopening.html?action=click\&pgtype=Article\&state=default\&region=MAIN_CONTENT_3\&context=storylines_faq}{instruction
    will be remote-only in the fall}, citing concerns that surging
    coronavirus infections in their areas pose too dire a risk for
    students and teachers. Together, the two districts enroll some
    825,000 students. They are the largest in the country so far to
    abandon plans for even a partial physical return to classrooms when
    they reopen in August. For other districts, the solution won't be an
    all-or-nothing approach.
    \href{https://bioethics.jhu.edu/research-and-outreach/projects/eschool-initiative/school-policy-tracker/}{Many
    systems}, including the nation's largest, New York City, are
    devising
    \href{https://www.nytimes3xbfgragh.onion/2020/06/26/us/coronavirus-schools-reopen-fall.html?action=click\&pgtype=Article\&state=default\&region=MAIN_CONTENT_3\&context=storylines_faq}{hybrid
    plans} that involve spending some days in classrooms and other days
    online. There's no national policy on this yet, so check with your
    municipal school system regularly to see what is happening in your
    community.
  \end{itemize}
\item ~
  \hypertarget{is-the-coronavirus-airborne}{%
  \paragraph{Is the coronavirus
  airborne?}\label{is-the-coronavirus-airborne}}

  \begin{itemize}
  \tightlist
  \item
    The coronavirus
    \href{https://www.nytimes3xbfgragh.onion/2020/07/04/health/239-experts-with-one-big-claim-the-coronavirus-is-airborne.html?action=click\&pgtype=Article\&state=default\&region=MAIN_CONTENT_3\&context=storylines_faq}{can
    stay aloft for hours in tiny droplets in stagnant air}, infecting
    people as they inhale, mounting scientific evidence suggests. This
    risk is highest in crowded indoor spaces with poor ventilation, and
    may help explain super-spreading events reported in meatpacking
    plants, churches and restaurants.
    \href{https://www.nytimes3xbfgragh.onion/2020/07/06/health/coronavirus-airborne-aerosols.html?action=click\&pgtype=Article\&state=default\&region=MAIN_CONTENT_3\&context=storylines_faq}{It's
    unclear how often the virus is spread} via these tiny droplets, or
    aerosols, compared with larger droplets that are expelled when a
    sick person coughs or sneezes, or transmitted through contact with
    contaminated surfaces, said Linsey Marr, an aerosol expert at
    Virginia Tech. Aerosols are released even when a person without
    symptoms exhales, talks or sings, according to Dr. Marr and more
    than 200 other experts, who
    \href{https://academic.oup.com/cid/article/doi/10.1093/cid/ciaa939/5867798}{have
    outlined the evidence in an open letter to the World Health
    Organization}.
  \end{itemize}
\item ~
  \hypertarget{what-are-the-symptoms-of-coronavirus}{%
  \paragraph{What are the symptoms of
  coronavirus?}\label{what-are-the-symptoms-of-coronavirus}}

  \begin{itemize}
  \tightlist
  \item
    Common symptoms
    \href{https://www.nytimes3xbfgragh.onion/article/symptoms-coronavirus.html?action=click\&pgtype=Article\&state=default\&region=MAIN_CONTENT_3\&context=storylines_faq}{include
    fever, a dry cough, fatigue and difficulty breathing or shortness of
    breath.} Some of these symptoms overlap with those of the flu,
    making detection difficult, but runny noses and stuffy sinuses are
    less common.
    \href{https://www.nytimes3xbfgragh.onion/2020/04/27/health/coronavirus-symptoms-cdc.html?action=click\&pgtype=Article\&state=default\&region=MAIN_CONTENT_3\&context=storylines_faq}{The
    C.D.C. has also} added chills, muscle pain, sore throat, headache
    and a new loss of the sense of taste or smell as symptoms to look
    out for. Most people fall ill five to seven days after exposure, but
    symptoms may appear in as few as two days or as many as 14 days.
  \end{itemize}
\item ~
  \hypertarget{does-asymptomatic-transmission-of-covid-19-happen}{%
  \paragraph{Does asymptomatic transmission of Covid-19
  happen?}\label{does-asymptomatic-transmission-of-covid-19-happen}}

  \begin{itemize}
  \tightlist
  \item
    So far, the evidence seems to show it does. A widely cited
    \href{https://www.nature.com/articles/s41591-020-0869-5}{paper}
    published in April suggests that people are most infectious about
    two days before the onset of coronavirus symptoms and estimated that
    44 percent of new infections were a result of transmission from
    people who were not yet showing symptoms. Recently, a top expert at
    the World Health Organization stated that transmission of the
    coronavirus by people who did not have symptoms was ``very rare,''
    \href{https://www.nytimes3xbfgragh.onion/2020/06/09/world/coronavirus-updates.html?action=click\&pgtype=Article\&state=default\&region=MAIN_CONTENT_3\&context=storylines_faq\#link-1f302e21}{but
    she later walked back that statement.}
  \end{itemize}
\end{itemize}

In parts of the US, we've been self-isolating since early March. Six
weeks sounds like a minute to me.

The author sees this new lockdown as something imposed rather than a
call to social good.

She's swimming, all right. Swimming in blind privilege and self-pity.

--- \textbf{Cathy Harding}

---

I too dread Melbourne winters. I too plan special events or holidays
away to get us through it. I too am dreading another period of remote
learning for my daughter and I too celebrate a significant anniversary
and had planned on being in Spain, but then we thought Queensland, and
now it appears we will be in luck if we can go out for dinner.

There is something different about this lockdown than the first.
Something darker. I also feel the undercurrent among family and friends
--- and yes, all those strangers on the internet --- of anger,
frustration, darkness and fear.

I greatly appreciated your reminder that once this is done, it will
almost be spring.

--- \textbf{Kate Elizabeth Cotter}

\emph{Here are this week's stories.}

\begin{center}\rule{0.5\linewidth}{\linethickness}\end{center}

\hypertarget{australia-and-new-zealand}{%
\subsection{\texorpdfstring{\href{https://www.nytimes3xbfgragh.onion/section/world/australia}{Australia
and New
Zealand}}{Australia and New Zealand}}\label{australia-and-new-zealand}}

\includegraphics{https://static01.graylady3jvrrxbe.onion/images/2020/07/17/world/17ausletter167-2/17ausletter167-2-articleLarge.jpg?quality=75\&auto=webp\&disable=upscale}

\begin{itemize}
\tightlist
\item
  \textbf{\href{https://www.nytimes3xbfgragh.onion/2020/07/13/science/chlamydia-koalas-vaccines.html}{How
  Koalas With an S.T.D. Could Help Humanity}.} When it comes to finding
  a vaccine for chlamydia, the world's most common sexually transmitted
  infection, koalas may prove a key ally.
\end{itemize}

\begin{itemize}
\tightlist
\item
  \textbf{\href{https://www.nytimes3xbfgragh.onion/2020/07/16/realestate/the-most-popular-listings-of-june.html}{The
  Most Popular Listings of June}.} The most viewed listings in June
  included a luxurious farm retreat in Connecticut and a private
  Australian island for about \$1 million.
\end{itemize}

\begin{itemize}
\item
  \textbf{\href{https://www.nytimes3xbfgragh.onion/2020/07/14/science/earthquake-dna-genes-kelp.html}{Scientists
  Find an Earthquake's Toll in an Organism's DNA}.} Along a coastline in
  New Zealand, kelp seems to contain a genetic record of the planet's
  geological upheaval.
\item
  \textbf{\href{https://www.nytimes3xbfgragh.onion/2020/07/13/world/australia/christchurch-mosque-killings-sentencing.html}{White
  Supremacist Who Admitted Christchurch Killings Plans to Represent
  Himself}.} There is concern that the Australian who pleaded guilty to
  killing 51 worshipers at two mosques in New Zealand last year would
  use the occasion to spout his views.
\item
  \textbf{\href{https://www.nytimes3xbfgragh.onion/2020/07/13/technology/google-ads-antitrust.html}{How
  May Google Fight an Antitrust Case? Look at This Little-Noticed
  Paper}.} A document sent by the search giant to Australian regulators
  argues that the company doesn't control enough of the digital ad
  industry to overcharge customers or block competitors.
\item
  \textbf{\href{https://www.nytimes3xbfgragh.onion/2020/07/10/world/australia/kfc-party-fines-australia.html}{KFC
  Birthday Party Costs \$18,000 in Covid-19 Fines in Australia}.}
  ``That's a heck of a birthday party to recall,'' said Chief
  Commissioner Shane Patton of the Victoria police. Officers were led to
  a house where they found people hiding in the backyard, garage and
  under beds.
\end{itemize}

\begin{center}\rule{0.5\linewidth}{\linethickness}\end{center}

\hypertarget{around-the-times}{%
\subsection{Around the Times}\label{around-the-times}}

Image

Monks from Wat Matchanthikaram, wearing masks and face shields to
protect them, received alms from Bangkok residents in
April.Credit...Adam Dean for The New York Times

\begin{itemize}
\item
  \textbf{\href{https://www.nytimes3xbfgragh.onion/2020/07/16/world/asia/coronavirus-thailand-photos.html?action=click\&module=Editors\%20Picks\&pgtype=Homepage}{No
  One Knows What Thailand Is Doing Right, but So Far, It's Working}.}
  Can the country's low rate of Covid-19 infections be attributed to
  culture? Genetics? Face masks? Or a combination of all three?
\item
  \textbf{\href{https://www.nytimes3xbfgragh.onion/2020/07/16/us/politics/trump-republicans.html?action=click\&module=Top\%20Stories\&pgtype=Homepage}{A
  Club of G.O.P. Political Heirs Push Back on Trump.}} **** Mitt Romney,
  Larry Hogan and Liz Cheney --- descendants of sometimes rebellious or
  resolute Republicans of the past --- are dissenting voices on a
  president who has taken over their party.
\item
  **\href{https://www.nytimes3xbfgragh.onion/2020/07/15/parenting/kids-covid-19-test.html?surface=home-living-vi\&fellback=false\&req_id=284471978\&algo=identity\&imp_id=619355117\&action=click\&module=Smarter\%20Living\&pgtype=Homepage}{Coronavirus
  Tests Can Be Scary For Kids. Here's How to Make Them Easier.}**A nasal
  swab is invasive and uncomfortable for anyone. For kids, knowing what
  to expect can help ease the anxiety.
\item
  \textbf{\href{https://www.nytimes3xbfgragh.onion/2020/07/14/health/cornavirus-vaccine-moderna.html?surface=home-discovery-vi-prg\&fellback=false\&req_id=482187727\&algo=identity\&imp_id=143330218\&action=click\&module=Science\%20\%20Technology\&pgtype=Homepage}{First
  Coronavirus Vaccine Tested in Humans Shows Early Promise}.} The
  vaccine, developed by government scientists and Moderna, a biotech
  company, appeared safe and provoked an immune response in 45 people in
  a study.
\end{itemize}

\begin{center}\rule{0.5\linewidth}{\linethickness}\end{center}

Enjoying the Australia Letter?
\href{https://www.nytimes3xbfgragh.onion/newsletters/australia-letter?utm_source=ausend}{Sign
up here} or forward to a friend.

For more Australia coverage and discussion, start your day with your
local
\href{https://www.nytimes3xbfgragh.onion/interactive/2018/briefing/global-morning-briefing-newsletter-signup.html?utm_source=ausend}{Morning
Briefing} and join us in our
\href{https://www.facebookcorewwwi.onion/groups/nytaustralia/}{Facebook
group}.

Advertisement

\protect\hyperlink{after-bottom}{Continue reading the main story}

\hypertarget{site-index}{%
\subsection{Site Index}\label{site-index}}

\hypertarget{site-information-navigation}{%
\subsection{Site Information
Navigation}\label{site-information-navigation}}

\begin{itemize}
\tightlist
\item
  \href{https://help.nytimes3xbfgragh.onion/hc/en-us/articles/115014792127-Copyright-notice}{©~2020~The
  New York Times Company}
\end{itemize}

\begin{itemize}
\tightlist
\item
  \href{https://www.nytco.com/}{NYTCo}
\item
  \href{https://help.nytimes3xbfgragh.onion/hc/en-us/articles/115015385887-Contact-Us}{Contact
  Us}
\item
  \href{https://www.nytco.com/careers/}{Work with us}
\item
  \href{https://nytmediakit.com/}{Advertise}
\item
  \href{http://www.tbrandstudio.com/}{T Brand Studio}
\item
  \href{https://www.nytimes3xbfgragh.onion/privacy/cookie-policy\#how-do-i-manage-trackers}{Your
  Ad Choices}
\item
  \href{https://www.nytimes3xbfgragh.onion/privacy}{Privacy}
\item
  \href{https://help.nytimes3xbfgragh.onion/hc/en-us/articles/115014893428-Terms-of-service}{Terms
  of Service}
\item
  \href{https://help.nytimes3xbfgragh.onion/hc/en-us/articles/115014893968-Terms-of-sale}{Terms
  of Sale}
\item
  \href{https://spiderbites.nytimes3xbfgragh.onion}{Site Map}
\item
  \href{https://help.nytimes3xbfgragh.onion/hc/en-us}{Help}
\item
  \href{https://www.nytimes3xbfgragh.onion/subscription?campaignId=37WXW}{Subscriptions}
\end{itemize}
