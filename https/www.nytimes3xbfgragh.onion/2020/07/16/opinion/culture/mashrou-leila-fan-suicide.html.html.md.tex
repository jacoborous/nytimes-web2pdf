Sections

SEARCH

\protect\hyperlink{site-content}{Skip to
content}\protect\hyperlink{site-index}{Skip to site index}

\href{https://myaccount.nytimes3xbfgragh.onion/auth/login?response_type=cookie\&client_id=vi}{}

\href{https://www.nytimes3xbfgragh.onion/section/todayspaper}{Today's
Paper}

\href{/section/opinion}{Opinion}\textbar{}She Waved a Rainbow Flag at
Our Cairo Show. Tragedy Followed.

\url{https://nyti.ms/2C8CiPr}

\begin{itemize}
\item
\item
\item
\item
\item
\end{itemize}

Advertisement

\protect\hyperlink{after-top}{Continue reading the main story}

\href{/section/opinion}{Opinion}

Supported by

\protect\hyperlink{after-sponsor}{Continue reading the main story}

\hypertarget{she-waved-a-rainbow-flag-at-our-cairo-show-tragedy-followed}{%
\section{She Waved a Rainbow Flag at Our Cairo Show. Tragedy
Followed.}\label{she-waved-a-rainbow-flag-at-our-cairo-show-tragedy-followed}}

That Sarah Hegazi felt safe enough to honor our music with her bravery
is thrilling; that such a simple act forever altered and then ended her
life brings me great sorrow.

By Haig Papazian

Mr. Papazian is a founding member and violinist for the band Mashrou'
Leila.

\begin{itemize}
\item
  July 16, 2020
\item
  \begin{itemize}
  \item
  \item
  \item
  \item
  \item
  \end{itemize}
\end{itemize}

\includegraphics{https://static01.graylady3jvrrxbe.onion/images/2020/07/16/opinion/16Papazian2/16Papazian2-articleLarge.jpg?quality=75\&auto=webp\&disable=upscale}

Last month, Sarah Hegazi, a 30-year-old Egyptian L.G.B.T.Q. rights
activist, took her own life in Canada. Far away from Cairo, her home,
she was profoundly haunted by what had happened to her there over the
past two and a half years, having been
\href{https://www.nytimes3xbfgragh.onion/2020/06/15/world/middleeast/egypt-gay-suicide-sarah-hegazi.html}{arrested,
tortured} and hounded into exile. Her transgression? She raised the
rainbow flag --- unabashedly and joyously --- at a concert in Cairo.

I was onstage that fated night, Sept. 22, 2017, with my band Mashrou'
Leila. We're an indie group from Beirut and have played across the
Middle East and beyond for more than a decade now. Our Arabic lyrics
tell stories of love, hope, loss, inequality and corruption, speaking to
the ills that plague our region.

Performing onstage has given me my proudest memories. From where I
usually stand, I look out at the sea of dreamers holding signs, waving
flags, laughing, screaming and singing their hearts out. Collectively,
we --- band and fans --- do what the Middle East's leaders won't: create
a home for all of us. Class, race, sexuality, gender, politics and
religion all fade away for two hours. Such a version of what the Arab
world could be is a powerful rebuke and a threat to what the dictators,
Islamists and sectarians have been offering us instead for decades.

Though we have performed at some of the most iconic venues across the
globe, that concert in Cairo was our largest ever, with 35,000 people in
attendance. To perform for so many in the soul of the Arab world, as
Egypt is considered, was a milestone for us and a testament to that
hunger for change.

Our band came together in 2008, in a series of late-night jam sessions
in Lebanon. We were architecture students, thinking we'd build a better
world through the houses, museums and cities we'd design. Instead,
through our music and the people it brought together, we ended up
building a community, one that transcends the tribal identities that
have long held us back. What we do share is a belief in the
possibilities of fairer, brighter and more resilient futures.

While I never met Ms. Hegazi, I feel I knew her. A photo from that night
immortalizes her, the same one that would seal her fate when it went
viral. She's aloft on the shoulders of a friend, gloriously raising the
rainbow flag; it almost gives her wings.

\includegraphics{https://static01.graylady3jvrrxbe.onion/images/2020/07/16/opinion/16Papazian1/16Papazian1-articleLarge.jpg?quality=75\&auto=webp\&disable=upscale}

I visit her at that concert in my memories: Lights dimmed, an intimate
and safe darkness transports us all to Marrikh, Arabic for Mars and the
name of one of our songs. We performed it under the stars in Cairo to a
constellation of swaying cellphone lights. I've also sought videos
posted online of that night, shot from different angles with shaky
cameras: pixelated dreams and noise-distorted recordings of the emotions
of thousands.

At first, Ms. Hegazi's picture was greeted online as a triumphant
exclamation of pride. Within days, it was used to whip up public
hysteria and justify a homophobic arrest campaign. The Egyptian
government imprisoned and tortured Ms. Hegazi, and many others, mostly
on the basis of their real or perceived sexual orientation or gender
identity --- as it has done for decades and continues to do.

That Ms. Hegazi felt safe enough to honor our music with her bravery is
thrilling; that such a simple act forever altered and then ended her
life brings me great sorrow. That plummet --- from hope to despair ---
is familiar to anyone who dared to believe in the Arab Spring.

In those early days of hope, our band dedicated a video to the
``generation of the revolution.'' In 2011, we performed our first shows
ever in the countries that led the Arab Spring, Tunisia and Egypt. We
used our platform to amplify Arab women's voices, played fund-raisers
for Syrian refugees, campaigned for sustainable environmental projects
and advocated L.G.B.T.Q. rights and sexual-health awareness. Our lead
singer has always been open about his queerness. As we started touring
the world, we met many inspiring queer Arab activists. Their courage and
resilience taught me to be more at ease with my own sexual identity and
queerness.

But the old guard quickly reasserted itself across the Middle East,
answering the youthful uprisings with a brutal and oppressive
counterrevolution.

Image

The author, second from left, performing with Mashrou' Leila in
Washington, D.C.Credit...Yeganeh Torbati/Reuters

We became a target for cynical politicians and pundits who stoked
religious fervor (be it Christian or Muslim) for their own gain,
accusing us of everything from Satanism to debauchery to a lack of
authenticity, campaigns often fueled by fake news. Last summer, our
10th-anniversary show in our own country
\href{https://www.nytimes3xbfgragh.onion/2019/07/31/world/middleeast/lebanon-mashrou-leila-blasphemy.html\#:~:text=Lebanese\%20Band's\%20Concert\%20Is\%20Canceled\%20After\%20It's\%20Accused\%20of\%20Blasphemy,-Mashrou'\%20Leila\%20has\&text=A\%20Lebanese\%20music\%20festival\%20has,Madonna\%20as\%20the\%20Virgin\%20Mary.}{was
canceled} after death threats. We were forbidden to play in many places
in the Middle East and after that 2017 concert, barred from ever
performing in Egypt again. These injustices pale compared with what the
local regimes regularly do to their own citizens.

I decided to follow love, and I too moved away. But though in exile our
houses are safe, their walls are bare. Here, our dreams are sheltered,
but no memories are to be found.

Two years after seeking asylum in Canada, Ms. Hegazi left us with this
note: ``To my siblings: I have tried to find salvation and I have
failed. Forgive me. To my friends: The journey was cruel and I am too
weak to resist. Forgive me. To the world: You were horrifically cruel,
but I forgive.''

Ms. Hegazi's words of forgiveness remind me why having queer voices and
public representation in the region is so important as we seek
compassion and courage to unite us in our dangerous, often lethal, fight
to be ourselves.

In a fairer Arab future, our history books will speak of the young
Egyptian woman who raised a rainbow flag at a concert in Cairo. In a
more resilient future, we will rebuild our home so that everyone in the
region, from Beirut to Damascus, Amman to Cairo, Tunis to Riyadh,
Jerusalem to Baghdad, can be who they are, unabashedly and joyously.

Haig Papazian
(\href{https://www.instagram.com/haigpapa/?hl=en}{@haigpapa}) is the
violinist for the Lebanese alternative rock band Mashrou' Leila.

\emph{The Times is committed to publishing}
\href{https://www.nytimes3xbfgragh.onion/2019/01/31/opinion/letters/letters-to-editor-new-york-times-women.html}{\emph{a
diversity of letters}} \emph{to the editor. We'd like to hear what you
think about this or any of our articles. Here are some}
\href{https://help.nytimes3xbfgragh.onion/hc/en-us/articles/115014925288-How-to-submit-a-letter-to-the-editor}{\emph{tips}}\emph{.
And here's our email:}
\href{mailto:letters@NYTimes.com}{\emph{letters@NYTimes.com}}\emph{.}

\emph{Follow The New York Times Opinion section on}
\href{https://www.facebookcorewwwi.onion/nytopinion}{\emph{Facebook}}\emph{,}
\href{http://twitter.com/NYTOpinion}{\emph{Twitter (@NYTopinion)}}
\emph{and}
\href{https://www.instagram.com/nytopinion/}{\emph{Instagram}}\emph{.}

Advertisement

\protect\hyperlink{after-bottom}{Continue reading the main story}

\hypertarget{site-index}{%
\subsection{Site Index}\label{site-index}}

\hypertarget{site-information-navigation}{%
\subsection{Site Information
Navigation}\label{site-information-navigation}}

\begin{itemize}
\tightlist
\item
  \href{https://help.nytimes3xbfgragh.onion/hc/en-us/articles/115014792127-Copyright-notice}{©~2020~The
  New York Times Company}
\end{itemize}

\begin{itemize}
\tightlist
\item
  \href{https://www.nytco.com/}{NYTCo}
\item
  \href{https://help.nytimes3xbfgragh.onion/hc/en-us/articles/115015385887-Contact-Us}{Contact
  Us}
\item
  \href{https://www.nytco.com/careers/}{Work with us}
\item
  \href{https://nytmediakit.com/}{Advertise}
\item
  \href{http://www.tbrandstudio.com/}{T Brand Studio}
\item
  \href{https://www.nytimes3xbfgragh.onion/privacy/cookie-policy\#how-do-i-manage-trackers}{Your
  Ad Choices}
\item
  \href{https://www.nytimes3xbfgragh.onion/privacy}{Privacy}
\item
  \href{https://help.nytimes3xbfgragh.onion/hc/en-us/articles/115014893428-Terms-of-service}{Terms
  of Service}
\item
  \href{https://help.nytimes3xbfgragh.onion/hc/en-us/articles/115014893968-Terms-of-sale}{Terms
  of Sale}
\item
  \href{https://spiderbites.nytimes3xbfgragh.onion}{Site Map}
\item
  \href{https://help.nytimes3xbfgragh.onion/hc/en-us}{Help}
\item
  \href{https://www.nytimes3xbfgragh.onion/subscription?campaignId=37WXW}{Subscriptions}
\end{itemize}
