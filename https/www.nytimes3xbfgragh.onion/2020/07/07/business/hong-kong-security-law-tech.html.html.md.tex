Sections

SEARCH

\protect\hyperlink{site-content}{Skip to
content}\protect\hyperlink{site-index}{Skip to site index}

\href{https://www.nytimes3xbfgragh.onion/section/business}{Business}

\href{https://myaccount.nytimes3xbfgragh.onion/auth/login?response_type=cookie\&client_id=vi}{}

\href{https://www.nytimes3xbfgragh.onion/section/todayspaper}{Today's
Paper}

\href{/section/business}{Business}\textbar{}In Hong Kong, a Proxy Battle
Over Internet Freedom Begins

\href{https://nyti.ms/2O4iI92}{https://nyti.ms/2O4iI92}

\begin{itemize}
\item
\item
\item
\item
\item
\item
\end{itemize}

Advertisement

\protect\hyperlink{after-top}{Continue reading the main story}

Supported by

\protect\hyperlink{after-sponsor}{Continue reading the main story}

\hypertarget{in-hong-kong-a-proxy-battle-over-internet-freedom-begins}{%
\section{In Hong Kong, a Proxy Battle Over Internet Freedom
Begins}\label{in-hong-kong-a-proxy-battle-over-internet-freedom-begins}}

As the city grapples with new restrictions on online speech, American
tech giants are on the front line of a clash between China and the
United States over the internet's future.

\includegraphics{https://static01.graylady3jvrrxbe.onion/images/2020/07/07/world/07hk-tech1-sub/07hk-tech1-sub-articleLarge-v2.jpg?quality=75\&auto=webp\&disable=upscale}

\href{https://www.nytimes3xbfgragh.onion/by/paul-mozur}{\includegraphics{https://static01.graylady3jvrrxbe.onion/images/2018/10/15/multimedia/author-paul-mozur/author-paul-mozur-thumbLarge.png}}

By \href{https://www.nytimes3xbfgragh.onion/by/paul-mozur}{Paul Mozur}

\begin{itemize}
\item
  Published July 7, 2020Updated July 13, 2020
\item
  \begin{itemize}
  \item
  \item
  \item
  \item
  \item
  \item
  \end{itemize}
\end{itemize}

\href{https://cn.nytimes3xbfgragh.onion/business/20200708/hong-kong-security-law-tech/}{阅读简体中文版}\href{https://cn.nytimes3xbfgragh.onion/business/20200708/hong-kong-security-law-tech/zh-hant/}{閱讀繁體中文版}

As Hong Kong grapples with a draconian
\href{https://www.nytimes3xbfgragh.onion/2020/07/13/world/asia/hong-kong-elections-security.html}{new
security law}, the tiny territory is emerging as the front line in a
global fight between the United States and China over censorship,
surveillance and the future of the internet.

\href{https://www.nytimes3xbfgragh.onion/2020/07/01/world/asia/hong-kong-security-law-china.html}{Long
a bastion} of online freedom on the digital border of China's tightly
managed internet, Hong Kong's uneasy status changed radically in just a
week. The new law mandates police censorship and
\href{https://www.nytimes3xbfgragh.onion/2020/07/02/world/asia/hong-kong-security-china.html}{covert
digital surveillance}, rules that can be applied to online speech across
the world.

Now,
\href{https://www.nytimes3xbfgragh.onion/2020/07/13/world/asia/hong-kong-elections-security.html}{the
Hong Kong government} is crafting web controls to appease the most
prolific censor on the planet, the Chinese Communist Party. And the
changes threaten to further inflame tensions between China and the
United States, in which technology itself has become a means by which
the two economic superpowers seek to spread influence and undercut each
other.

Caught in the middle are the city's seven million residents, online
records of rollicking political debate --- some of which may now be
illegal --- and the world's largest internet companies, which host, and
by extension guard, that data.

A standoff is already brewing. Many big tech companies, including
Facebook, Google, Twitter,
\href{https://hongkongfp.com/2020/07/07/breaking-zoom-suspends-data-requests-from-hong-kong-govt-over-national-security-law-concerns/}{Zoom}
and
\href{https://qz.com/1877636/chinas-great-firewall-arrives-in-hong-kongs-internet/}{LinkedIn},
have said in the past two days that they would temporarily stop
complying with requests for user data from the Hong Kong authorities.
The Hong Kong government, in turn, has made it clear that the penalty
for noncompliance with the law could include jail time for company
employees.

\href{https://www.nytimes3xbfgragh.onion/2020/07/06/technology/tiktok-google-facebook-twitter-hong-kong.html}{TikTok},
which despite being owned by the Chinese internet giant ByteDance has
its eye on the U.S. market, went even further than its American rivals.
The video app said late Monday it would withdraw from stores in Hong
Kong and make it inoperable to users there within a few days. The
company has said that managers outside China call the shots on key
aspects of its business, including rules about data.

Based on the law, the Hong Kong authorities can dictate the way people
around the world talk about the city's contested politics. A Facebook
employee could potentially be arrested in Hong Kong if the company
failed to hand over user data on someone based in the United States whom
Chinese authorities deemed a threat to national security.

``If Facebook refuses to give national security data, its service may be
terminated in Hong Kong, and it will lose access to the Hong Kong
market,'' said Glacier Kwong of Keyboard Frontline, a nongovernmental
organization that monitors digital rights in Hong Kong.

``It's not impossible that this will happen,'' Ms. Kwong added. ``China
often uses its market and boycotting to make foreign companies listen to
their demands.''

\includegraphics{https://static01.graylady3jvrrxbe.onion/images/2020/07/07/world/07hk-tech2/merlin_174111945_6cd72ad6-5747-491f-875d-19f651557701-articleLarge.jpg?quality=75\&auto=webp\&disable=upscale}

While it is not clear how widely Hong Kong's government will enforce the
law, the looming legal fights could determine whether the city falls
behind China's digital Iron Curtain or becomes a hybrid where online
speech and communications are selectively policed.

The technological Cold War between China and the United States is
playing out on various fronts around the world. The trade war has
ensnared Chinese tech giants like Huawei and ZTE, while American
companies complain of industrial policies that favor Chinese businesses
at home and subsidize them abroad. Beijing's severe digital controls
have kept companies like Google and Facebook from operating their
services in mainland China.

Though U.S. internet companies still earn billions of dollars in Chinese
ad revenue, a decision to go along with the Hong Kong rules would risk
the ire of Washington, where there has been bipartisan condemnation of
the security law. New restrictions on American businesses could also
trigger retaliation.

The American secretary of state, Mike Pompeo, said on Monday that the
Trump administration was considering blocking some Chinese apps, which
he has called a threat to national security.

``I don't want to get out in front of the president, but it's something
we are looking at,'' he said in an interview on Fox News.

A Chinese Ministry of Foreign Affairs spokesman, Zhao Lijian, defended
the law at a news conference on Tuesday, saying that it would make a
more ``stable and harmonious'' Hong Kong.

``The horses will run faster, the horses will run happier, the stocks
will sizzle hotter, and the dancers will dance better. We have full
confidence in Hong Kong,'' he said, alluding to a quote from the late
Chinese leader Deng Xiaoping about the city.

Google's experience over the past year shows the fraught position of the
largest U.S. internet companies. As the Hong Kong police struggled to
contain protests across the city in 2019, they turned to internet
companies for help. Overall data requests and
\href{https://www.charlesmok.hk/legco/council-question-requests-made-to-information-and-communication-technology-companies-for-disclosure-and-removal-of-information/}{orders
from police} to remove content more than doubled in the second half of
2019 from the first half to over 7,000 requests, according to a
pro-democracy lawmaker, Charles Mok.

Image

Protesters in June of last year. Fearing prosecution under the new
security law, many people have taken down social media posts and deleted
accounts.Credit...Kin Cheung/Associated Press

The police asked Google to take down a number of posts, including a
confidential police manual that had leaked online, a YouTube video from
the hacking group Anonymous supporting the protests, and links to a
website that let the public look up personal details about police
officers, according to a company report.

In each case, Google said no.

The new law could punish the company with fines, equipment seizures and
arrests if it again declines such requests. It also would allow the
police to potentially seize equipment from companies that host such
content.

``We see the trend. It's not just that they're making more requests,
it's the growing power in the hands of the authorities to do this
arbitrarily,'' Mr. Mok said, adding that ``some of the local smaller
platforms will be worried about the legal consequences and they may
comply'' with government requests.

Several small local apps associated with the protest movement have
already shut down. Eat With You, which labeled restaurants based on
their political affiliation, stopped operating the day after the
security law was enacted last week. On Sunday, another service that
mapped pro-protester and pro-police businesses on Google Maps
\href{https://t.me/yellowshoppromotion/1044}{suspended its services},
citing ``changing social circumstances.''

Image

With some slogans now potentially illegal, people in Hong Kong have
begun demonstrating with blank sheets of paper.~Credit...Lam Yik Fei for
The New York Times

Individuals, as well, have taken to self-censorship. Many have taken
down posts, removed ``likes'' for some pro-democracy pages and even
deleted accounts on platforms like Twitter, according to activists.
Fears that WhatsApp would hand over data also drove people to switch to
downloading a rival encrypted chat app, Signal. WhatsApp, though, had no
recent data requests from Hong Kong police, according to a person
familiar with the matter.

People in Hong Kong have also quickly embraced the types of coded online
speech that flourish in China, where internet police and censors patrol
the web. One slogan, which the authorities have said could be illegal,
was changed from ``liberate Hong Kong, revolution of our time'' to
``shopping in Hong Kong, Times Square,'' a reference to a local shopping
mall.

In other cases, posters abbreviated the slogan based on Cantonese
phonetics, writing simply ``GFHG, SDGM.'' The unofficial anthem of the
protests, ``Glory to Hong Kong,'' has had its lyrics converted into
numbers that sound roughly like the lyrics. ``05 432 680, 04 640 0242,''
\href{https://www.facebookcorewwwi.onion/glorytohkdgx/posts/199562861505570}{goes
the opening}.

Companies, meanwhile, have the option of shifting data away from Hong
Kong. Lento Yip, chairman of the Hong Kong Internet Service Providers
Association, said he noticed more businesses relocating servers out of
the city in June, though he wasn't certain about their motivations.

``From a business aspect, some websites or content providers might just
move to other places. It doesn't cost much and it's pretty easy,'' he
said.

For companies like Amazon and Google, which have large data centers in
Hong Kong, such a move would be neither cheap nor easy. And their other
options are equally complicated. Moving all employees out of the city
would insulate firms from arrests, but it may not be feasible.

There are potential technical maneuvers that companies could use to
guard against the law, said Edmon Chung, a member of the board of
directors of the Internet Society of Hong Kong, a nonprofit dedicated to
the open development of the internet.

\href{https://www.info.gov.hk/gia/general/202007/06/P2020070600784.htm?fontSize=1}{The
rules stipulate} that tech companies may avoid requests to take down
data if the technology necessary to comply with some rules is ``not
reasonably available,'' which Mr. Chung said opened up the possibility
of using encryption, storing content in multiple places and other
methods of avoiding scrutiny.

He added that the response from both abroad and inside Hong Kong could
still go a long way toward shaping the law and how it is applied.

``If the people in Hong Kong stand up to this, it might not be as bad as
what is in mainland China,'' Mr. Chung said.

``How Hong Kong people get around it and circumvent it and create new
types of speech to continually challenge the party line is something
that remains to be seen, and I remain hopeful that the free spirit in
Hong Kong will hold up,'' he said.

Raymond Zhong contributed reporting. Lin Qiqing and Claire Fu
contributed research.

Advertisement

\protect\hyperlink{after-bottom}{Continue reading the main story}

\hypertarget{site-index}{%
\subsection{Site Index}\label{site-index}}

\hypertarget{site-information-navigation}{%
\subsection{Site Information
Navigation}\label{site-information-navigation}}

\begin{itemize}
\tightlist
\item
  \href{https://help.nytimes3xbfgragh.onion/hc/en-us/articles/115014792127-Copyright-notice}{©~2020~The
  New York Times Company}
\end{itemize}

\begin{itemize}
\tightlist
\item
  \href{https://www.nytco.com/}{NYTCo}
\item
  \href{https://help.nytimes3xbfgragh.onion/hc/en-us/articles/115015385887-Contact-Us}{Contact
  Us}
\item
  \href{https://www.nytco.com/careers/}{Work with us}
\item
  \href{https://nytmediakit.com/}{Advertise}
\item
  \href{http://www.tbrandstudio.com/}{T Brand Studio}
\item
  \href{https://www.nytimes3xbfgragh.onion/privacy/cookie-policy\#how-do-i-manage-trackers}{Your
  Ad Choices}
\item
  \href{https://www.nytimes3xbfgragh.onion/privacy}{Privacy}
\item
  \href{https://help.nytimes3xbfgragh.onion/hc/en-us/articles/115014893428-Terms-of-service}{Terms
  of Service}
\item
  \href{https://help.nytimes3xbfgragh.onion/hc/en-us/articles/115014893968-Terms-of-sale}{Terms
  of Sale}
\item
  \href{https://spiderbites.nytimes3xbfgragh.onion}{Site Map}
\item
  \href{https://help.nytimes3xbfgragh.onion/hc/en-us}{Help}
\item
  \href{https://www.nytimes3xbfgragh.onion/subscription?campaignId=37WXW}{Subscriptions}
\end{itemize}
