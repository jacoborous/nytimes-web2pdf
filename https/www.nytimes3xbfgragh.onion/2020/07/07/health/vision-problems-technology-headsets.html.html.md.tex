Sections

SEARCH

\protect\hyperlink{site-content}{Skip to
content}\protect\hyperlink{site-index}{Skip to site index}

\href{https://www.nytimes3xbfgragh.onion/section/health}{Health}

\href{https://myaccount.nytimes3xbfgragh.onion/auth/login?response_type=cookie\&client_id=vi}{}

\href{https://www.nytimes3xbfgragh.onion/section/todayspaper}{Today's
Paper}

\href{/section/health}{Health}\textbar{}Technology Bridges the Gap to
Better Sight

\href{https://nyti.ms/3gwBSkj}{https://nyti.ms/3gwBSkj}

\begin{itemize}
\item
\item
\item
\item
\item
\item
\end{itemize}

Advertisement

\protect\hyperlink{after-top}{Continue reading the main story}

Supported by

\protect\hyperlink{after-sponsor}{Continue reading the main story}

\hypertarget{technology-bridges-the-gap-to-better-sight}{%
\section{Technology Bridges the Gap to Better
Sight}\label{technology-bridges-the-gap-to-better-sight}}

More than 6 million Americans have vision problems that cannot be
corrected by glasses or contact lenses. Companies like IrisVision are
creating headsets to help them see better.

\includegraphics{https://static01.graylady3jvrrxbe.onion/images/2020/07/08/multimedia/08sp-ff-vision1/08sp-ff-vision1-articleLarge.jpg?quality=75\&auto=webp\&disable=upscale}

By Janet Morrissey

\begin{itemize}
\item
  Published July 7, 2020Updated July 14, 2020
\item
  \begin{itemize}
  \item
  \item
  \item
  \item
  \item
  \item
  \end{itemize}
\end{itemize}

\emph{This article is part of our continuing}
\href{https://www.nytimes3xbfgragh.onion/spotlight/fast-forward}{\emph{Fast
Forward series}}\emph{, which examines technological, economic, social
and cultural shifts that happen as businesses evolve.}

When Jimmy Blakley signed up to serve his country during the Vietnam
War, his health and vision were pristine. But shortly after finishing
his service in 1971, Mr. Blakley's vision began to sharply deteriorate.

He said doctors discovered signs of Agent Orange, the toxic herbicide
used during the war, in his blood, and told him it was likely the cause
of his vision loss. Over the years he had surgery and eye injections,
but became legally blind in 1999. He used magnifiers on top of glasses
to read and needed to sit two feet away to watch his 65-inch TV. He was
frustrated.

Then Mr. Blakley, now 72 and living in College Station, Tex., learned
about \href{https://irisvision.com/}{IrisVision}, a device that uses a
smartphone, virtual reality headset and algorithms to help people with
poor vision see and read clearly. For Mr. Blakley, it was life-changing.

``I was amazed,'' he said. ``I was seeing the bottom of the eye chart;
it was like being 19 again.''

Among companies using technology to address vision problems, IrisVision
is working to go a step further: Its device, in addition to improving
sight, is expected to be able to diagnose conditions and test and even
treat patients remotely within the next two years.

``It's not aesthetically pleasing but it does something that really
provides a useful function for people,'' said
\href{https://www.nei.nih.gov/about/news-and-events/press-contacts-and-spokespeople/jerome-r-wujek-phd}{Jerome
Wujek}, program director and research resources officer at the National
Eye Institute, part of the National Institutes of Health. ``We are
visual animals. If you lose your vision, you lose a big chunk of your
life.'' He said devices that used artificial intelligence to test,
diagnose and treat remotely were the future as they would allow
ophthalmologists, optometrists and others to treat people in rural and
underserved areas.

The market for this technology could be huge. About 6.4 million
Americans have low vision that cannot be corrected by glasses or contact
lenses, according to a
\href{https://pubmed.ncbi.nlm.nih.gov/27656731/}{2016 National Academy
of Sciences report}. Of those, 4.2 million are older than 40. And as the
``silver tsunami'' of aging baby boomers sweeps the nation,
\href{https://www.afb.org/research-and-initiatives/statistics/statistics-seniors-vision-loss}{age-related
eye diseases and conditions}, like macular degeneration, glaucoma and
diabetic retinopathy, are expected to surge.

The
\href{https://www.nytimes3xbfgragh.onion/news-event/coronavirus}{coronavirus}outbreak
has made it particularly challenging for the visually impaired. People
with low vision rely heavily on touch and sound to identify objects and
people, and this can be tough while following hand sanitizing and social
distancing practices, and listening to people through masks, said Dr. L.
Penny Rosenblum, director of research at the American Foundation for the
Blind, who has congenital cataracts and glaucoma. Those with low vision
can also find themselves even more isolated if they are unable to use
Zoom and other technology to stay connected, Dr. Rosenblum added.

\includegraphics{https://static01.graylady3jvrrxbe.onion/images/2020/07/08/multimedia/08sp-ff-vision2/08sp-ff-vision2-articleLarge.jpg?quality=75\&auto=webp\&disable=upscale}

Frank Werblin, co-founder and chief scientist at IrisVision and a
professor of neuroscience at the University of California, Berkeley, for
more than 40 years, came up with the idea for IrisVision in 2014. While
speaking at a Foundation Fighting Blindness conference, a board trustee,
J. Lynn Dougan, approached him, offering to provide funding if Dr.
Werblin could invent a wearable device that could help his
vision-impaired daughter. Dr. Werblin said he had been fascinated by the
relationship between brain neurons, the retina and vision, and eagerly
accepted the challenge.

Along the way, Dr. Werblin brought in a tech-savvy mobile app developer,
Ammad Khan, as a partner. ``Imagining a concept that would transform
somebody's life was very compelling for me,'' Mr. Khan said.

With \$1.5 million from Mr. Dougan, a \$1.5 million grant from the
National Eye Institute, and about \$4 million from angel investors and
venture capitalists, the co-founders got to work. They collaborated with
experts from Stanford and Johns Hopkins universities, and partnered with
Samsung, which provided the smartphone, the augmented, virtual and mixed
reality technologies, and the mobile artificial intelligence platform.
By 2017, the device was ready.

IrisVision helps restore visual function to people with such conditions
as macular degeneration, diabetic retinopathy, retinitis pigmentosa,
Stargardt disease, glaucoma and optic atrophy --- but not cataracts. It
can help someone with visual acuity as low as 20/1000. (Acuity measures
sharpness: A 20/200 acuity, for example, means a person can make out an
object at a distance of 20 feet that a normal vision person could see
from 200 feet.)

So how does it work?

IrisVision helps the brain use parts of the eyes that still function
properly. The smartphone's camera captures an image, and then the
virtual reality, or V.R., headset and algorithms enhance the image by
providing enough information to fill in the gaps and remap the scene to
provide a complete picture.

The company recently added voice control, video streaming, Alexa and
other interactive features that give those with low vision easy access
to news, weather updates, YouTube videos and TV shows viewed within the
headset. The person can adjust the headset, by voice command or button,
to zoom in or adjust the color, contrast or brightness.

Still, the device has its shortcomings: It is large and clunky, and the
user needs to be stationary --- not walking around --- to use it. In
addition, at \$2,950, it is expensive. And unless you are a veteran,
health insurers do not cover the cost.

Image

Within two years, Mr. Khan said, the plan is that an updated model will
be able to diagnose and treat eye conditions remotely.Credit...Cayce
Clifford for The New York Times

IrisVision has seen an increase in sales during the pandemic, rising 50
percent in June from the same month a year ago, according to Mr. Khan.
He said most of the devices were now being sold directly to consumers,
as clinics where people often purchased them had shut down during the
outbreak. IrisVision ships the devices to homes and then helps buyers
set up the technology remotely.

A number of other companies are also working to address issues with low
vision, like NuEyes, eSight, Patriot Vision, OrCam, Vispero and Aira.
Patriot's ViewPoint, Vispero's Compact 6 HD Wear, and NuEyes e2 all
offer V.R. headsets similar to IrisVision's, although magnification,
field of view and other features vary among models. Some devices, like
e2 and Compact Wear, are less expensive but offer fewer features.

They also vary in size and weight. NuEyes Pro, eSight, and Vispero's
Jordy resemble oversized sunglasses or ski goggles and are designed for
walking down the street, visiting museums or shopping, which IrisVision
is not.

``We're still in the very, very early days,'' said Kjell Carlsson, a
senior analyst at Forrester Research. He said the size and price would
need to come down before the devices became mainstream.

The differences in devices mean that one may suit a person or a
condition better than another.

``The IrisVision was too pixilated and the image `too big' for me,'' Dr.
Rosenblum said, adding that her friend's 91-year old mother, who had
macular degeneration, loved IrisVision. ``One device doesn't fit all.''

IrisVision has the backing of Brook Byers, who invested in 2017 and has
since pumped more than \$3 million into the company. Mr. Byers, a senior
partner and a co-founder at Kleiner Perkins, a venture capital firm that
invested early in Amazon, Google and Twitter, has retinal issues in one
eye stemming from an accident when he was 11.

``I tried it on in my office. I have a really, really bad eye, and wow,
I could see with it,'' he said. ``There was an aha moment.''

Mr. Byers said he was particularly excited about the company's next
steps. IrisVision's updated model, expected later this year, will be
able to test and diagnose eye conditions remotely. And, Mr. Khan said,
the treatment component should be added within two years.

``The great value of this device isn't that it simply restores vision,''
Dr. Werblin said. ``But it brings people back to life.''

Advertisement

\protect\hyperlink{after-bottom}{Continue reading the main story}

\hypertarget{site-index}{%
\subsection{Site Index}\label{site-index}}

\hypertarget{site-information-navigation}{%
\subsection{Site Information
Navigation}\label{site-information-navigation}}

\begin{itemize}
\tightlist
\item
  \href{https://help.nytimes3xbfgragh.onion/hc/en-us/articles/115014792127-Copyright-notice}{©~2020~The
  New York Times Company}
\end{itemize}

\begin{itemize}
\tightlist
\item
  \href{https://www.nytco.com/}{NYTCo}
\item
  \href{https://help.nytimes3xbfgragh.onion/hc/en-us/articles/115015385887-Contact-Us}{Contact
  Us}
\item
  \href{https://www.nytco.com/careers/}{Work with us}
\item
  \href{https://nytmediakit.com/}{Advertise}
\item
  \href{http://www.tbrandstudio.com/}{T Brand Studio}
\item
  \href{https://www.nytimes3xbfgragh.onion/privacy/cookie-policy\#how-do-i-manage-trackers}{Your
  Ad Choices}
\item
  \href{https://www.nytimes3xbfgragh.onion/privacy}{Privacy}
\item
  \href{https://help.nytimes3xbfgragh.onion/hc/en-us/articles/115014893428-Terms-of-service}{Terms
  of Service}
\item
  \href{https://help.nytimes3xbfgragh.onion/hc/en-us/articles/115014893968-Terms-of-sale}{Terms
  of Sale}
\item
  \href{https://spiderbites.nytimes3xbfgragh.onion}{Site Map}
\item
  \href{https://help.nytimes3xbfgragh.onion/hc/en-us}{Help}
\item
  \href{https://www.nytimes3xbfgragh.onion/subscription?campaignId=37WXW}{Subscriptions}
\end{itemize}
