Sections

SEARCH

\protect\hyperlink{site-content}{Skip to
content}\protect\hyperlink{site-index}{Skip to site index}

\href{https://www.nytimes3xbfgragh.onion/section/science}{Science}

\href{https://myaccount.nytimes3xbfgragh.onion/auth/login?response_type=cookie\&client_id=vi}{}

\href{https://www.nytimes3xbfgragh.onion/section/todayspaper}{Today's
Paper}

\href{/section/science}{Science}\textbar{}Mars Mission From United Arab
Emirates Embarks on 7-Month Journey

\url{https://nyti.ms/3eFEgUB}

\begin{itemize}
\item
\item
\item
\item
\item
\end{itemize}

\hypertarget{missions-to-mars}{%
\subsubsection{\texorpdfstring{\href{https://www.nytimes3xbfgragh.onion/news-event/summer-of-mars?name=styln-mars\&region=TOP_BANNER\&variant=undefined\&block=storyline_menu_recirc\&action=click\&pgtype=Article\&impression_id=a7e61be0-e391-11ea-9027-41a96f0fd4e9}{Missions
to Mars}}{Missions to Mars}}\label{missions-to-mars}}

\begin{itemize}
\tightlist
\item
  \href{https://www.nytimes3xbfgragh.onion/2020/07/30/science/nasa-mars-launch.html?name=styln-mars\&region=TOP_BANNER\&variant=undefined\&block=storyline_menu_recirc\&action=click\&pgtype=Article\&impression_id=a7e642f0-e391-11ea-9027-41a96f0fd4e9}{NASA
  Launch Highlights}
\item
  \href{https://www.nytimes3xbfgragh.onion/interactive/2020/science/mars-perseverance-tianwen-hope.html?name=styln-mars\&region=TOP_BANNER\&variant=undefined\&block=storyline_menu_recirc\&action=click\&pgtype=Article\&impression_id=a7e642f1-e391-11ea-9027-41a96f0fd4e9}{Meet
  the Spacecraft}
\item
  \href{https://www.nytimes3xbfgragh.onion/2020/07/28/science/nasa-jezero-perseverance.html?name=styln-mars\&region=TOP_BANNER\&variant=undefined\&block=storyline_menu_recirc\&action=click\&pgtype=Article\&impression_id=a7e642f2-e391-11ea-9027-41a96f0fd4e9}{NASA's
  Destination}
\item
  \href{https://www.nytimes3xbfgragh.onion/2020/07/28/science/mars-nasa-science.html?name=styln-mars\&region=TOP_BANNER\&variant=undefined\&block=storyline_menu_recirc\&action=click\&pgtype=Article\&impression_id=a7e642f3-e391-11ea-9027-41a96f0fd4e9}{Too
  Much Mars?}
\end{itemize}

Advertisement

\protect\hyperlink{after-top}{Continue reading the main story}

Supported by

\protect\hyperlink{after-sponsor}{Continue reading the main story}

\hypertarget{mars-mission-from-united-arab-emirates-embarks-on-7-month-journey}{%
\section{Mars Mission From United Arab Emirates Embarks on 7-Month
Journey}\label{mars-mission-from-united-arab-emirates-embarks-on-7-month-journey}}

Lifting off from Tanegashima Space Center in Japan, it is the first of
three missions headed to the red planet this summer.

\includegraphics{https://static01.graylady3jvrrxbe.onion/images/2020/07/19/science/19marslaunch1/19marslaunch1-videoSixteenByNineJumbo1600.jpg}

By The New York Times

\begin{itemize}
\item
  Published July 19, 2020Updated July 22, 2020
\item
  \begin{itemize}
  \item
  \item
  \item
  \item
  \item
  \end{itemize}
\end{itemize}

You'll be hearing a lot about Mars in the weeks to come this summer.
Three missions are launching toward the red planet, taking advantage of
the way Earth and its neighbor get closer every 26 months or so,
allowing a relatively short trip between the two worlds. If they launch
successfully, the spacecraft will arrive at Mars early next year.

The first of the three missions, built by the United Arab Emirates,
lifted off on Monday morning from a launch site in Japan (it was the end
of Sunday afternoon in the United States). Carried into calm skies by a
Mitsubishi H-IIA rocket, the spacecraft separated from the rocket about
an hour later and began a journey to Mars that will last until February.
The trip to the red planet begins a bold entry into interplanetary
exploration by a small country that has previously only sent a few small
satellites to orbit.

\begin{quote}
We have lift-off. H2A, the rocket carrying the Hope Probe to space, has
launched from the Tanegashima Space Centre in
Japan.\href{https://twitter.com/hashtag/HopeMarsMission?src=hash\&ref_src=twsrc\%5Etfw}{\#HopeMarsMission}
\href{https://t.co/pRKZLOL7NT}{pic.twitter.com/pRKZLOL7NT}

--- Hope Mars Mission (@HopeMarsMission)
\href{https://twitter.com/HopeMarsMission/status/1284978102058258434?ref_src=twsrc\%5Etfw}{July
19, 2020}
\end{quote}

\hypertarget{what-is-the-uae-sending-to-mars-and-what-will-it-do}{%
\subsection{What is the U.A.E. sending to Mars and what will it
do?}\label{what-is-the-uae-sending-to-mars-and-what-will-it-do}}

The Emirates Mars Mission, also known as Hope, is an orbiter that will
study Mars from above the planet. It will join a fleet of six other
spacecrafts studying the red planet from space, three operated by NASA,
two by the European Space Agency (one shared with Russia) and one by
India. Each contains different instruments to help further research of
the Martian atmosphere and surface.

The Hope orbiter is carrying three instruments: an infrared
spectrometer, an ultraviolet spectrometer and a camera. From its high
orbit --- varying from 12,400 miles to 27,000 miles above the surface
--- the spacecraft will give planetary scientists their first global
view of Martian weather at all times of day. Over its two-year mission,
it will investigate how dust storms and other weather phenomena near the
Martian surface speed or slow the loss of the planet's atmosphere into
space.

\hypertarget{the-hope-spacecraft}{%
\subsection{The Hope Spacecraft}\label{the-hope-spacecraft}}

The United Arab Emirates' first Mars mission aims to investigate the
Martian atmosphere. The orbiter's giant solar panel wings will deploy in
space after launch.

ANTENNA

A six-foot antenna will communicate with Earth

THERMAL BLANKET

A protective layer of insulation around the orbiter gives it a gold
appearance

SOLAR PANELS

Will unfurl after launch and charge the onboard battery

Hope is about as tall as a person and weighs almost 3,000 pounds

CAMERA

Will capture high-resolution images of Mars

Infrared Spectrometer

Will study dust, ice clouds, water vapor and temperature in the lower
atmosphere

ULTRAVIOLET SPECTROMETER

Will investigate carbon monoxide, hydrogen and oxygen in the upper
atmosphere

ANTENNA

A six-foot antenna will communicate with Earth

THERMAL BLANKET

To protect from extreme temperatures

Hope is about as tall as a person and weighs almost 3,000 pounds

SOLAR PANELS

Will charge the onboard battery

Camera

Infrared Spectrometer

Ultraviolet Spectrometer

Antenna

Thermal

Blanket

Hope is about as tall as a person

Solar

Panels

Camera

Infrared Spectrometer

Ultraviolet Spectrometer

By Eleanor Lutz \textbar{} Source: Mohammed Bin Rashid Space Center

\hypertarget{how-extensive-is-the-emirati-space-program}{%
\subsection{How extensive is the Emirati space
program?}\label{how-extensive-is-the-emirati-space-program}}

The Emirates previously built and launched three earth observing
satellites, gaining experience from a collaboration with a South Korean
company. The country also has a nascent human spaceflight program. Last
year, its first astronaut,
\href{https://www.nytimes3xbfgragh.onion/2019/09/25/science/emirati-astronaut-uae-international-space-station.html}{Hazzaa
al-Mansoori, who completed an eight-day stay at the International Space
Station}, was carried there aboard a Russian rocket.

For the Mars mission, the country took a similar approach to the earlier
satellites by working with the Laboratory for Atmospheric and Space
Physics at the University of Colorado,
\href{https://www.nytimes3xbfgragh.onion/2020/02/15/science/mars-united-arab-emirates.html}{where
Hope was built} before being sent to Dubai for testing.

Emirati engineers worked side by side with their counterparts in
Boulder, Colo., learning and doing as they designed and assembled the
spacecraft.

\hypertarget{what-else-is-launching-to-mars-this-summer}{%
\subsection{What else is launching to Mars this
summer?}\label{what-else-is-launching-to-mars-this-summer}}

Two other missions are headed to Mars in the weeks to come.

The next expected launch will be
\href{https://www.nytimes3xbfgragh.onion/2020/07/22/science/china-mars-mission.html}{China's
Tianwen-1}, which could occur
\href{http://www.xinhuanet.com/english/2020-07/17/c_139219529.htm}{between
later this week through early August}.

The Chinese mission includes an orbiter, a lander and a rover that will
\href{https://nssdc.gsfc.nasa.gov/nmc/spacecraft/display.action?id=HUOXING+1}{study
the Martian soil's water and ice content, among other research targets}.
This will be China's second attempt to get to Mars. Its first,
Yinghuo-1, failed to escape Earth in 2011 when the Russian rocket that
was carrying it malfunctioned. In the years since that mission, China
has completed a number of successful crewed missions in low earth orbit,
and it
\href{https://www.nytimes3xbfgragh.onion/2020/02/26/science/china-moon-far-side.html}{landed
a rover on the far side of the moon}, the only spacecraft that has ever
accomplished that feat.

On July 30,
\href{https://www.nytimes3xbfgragh.onion/2020/03/05/science/mars-2020-rover-name.html}{NASA
is scheduled to launch Perseverance, a robotic rover} that will be the
fifth wheeled American vehicle to explore Mars. It will land in a crater
called Jezero, seeking to find signs of ancient, extinct life that might
have once thrived when the crater was a lake.

Early in its mission, Perseverance will release
\href{https://www.nytimes3xbfgragh.onion/2020/06/23/science/mars-helicopter-nasa.html}{a
small experimental helicopter, Ingenuity}. It will attempt short flights
in the thin Martian atmosphere, aiming to demonstrate that the
technology can extend the reach of missions beyond the limited range of
robotic rovers.

A fourth mission,
\href{https://www.nytimes3xbfgragh.onion/2020/03/12/science/mars-rover-coronavirus.html}{the
joint Russian-European Rosalind Franklin rover}, was to launch this
summer, too. But technical hurdles, aggravated by the coronavirus
pandemic, could not be overcome in time to meet the launch window. It is
now scheduled to launch in 2022.

Advertisement

\protect\hyperlink{after-bottom}{Continue reading the main story}

\hypertarget{site-index}{%
\subsection{Site Index}\label{site-index}}

\hypertarget{site-information-navigation}{%
\subsection{Site Information
Navigation}\label{site-information-navigation}}

\begin{itemize}
\tightlist
\item
  \href{https://help.nytimes3xbfgragh.onion/hc/en-us/articles/115014792127-Copyright-notice}{©~2020~The
  New York Times Company}
\end{itemize}

\begin{itemize}
\tightlist
\item
  \href{https://www.nytco.com/}{NYTCo}
\item
  \href{https://help.nytimes3xbfgragh.onion/hc/en-us/articles/115015385887-Contact-Us}{Contact
  Us}
\item
  \href{https://www.nytco.com/careers/}{Work with us}
\item
  \href{https://nytmediakit.com/}{Advertise}
\item
  \href{http://www.tbrandstudio.com/}{T Brand Studio}
\item
  \href{https://www.nytimes3xbfgragh.onion/privacy/cookie-policy\#how-do-i-manage-trackers}{Your
  Ad Choices}
\item
  \href{https://www.nytimes3xbfgragh.onion/privacy}{Privacy}
\item
  \href{https://help.nytimes3xbfgragh.onion/hc/en-us/articles/115014893428-Terms-of-service}{Terms
  of Service}
\item
  \href{https://help.nytimes3xbfgragh.onion/hc/en-us/articles/115014893968-Terms-of-sale}{Terms
  of Sale}
\item
  \href{https://spiderbites.nytimes3xbfgragh.onion}{Site Map}
\item
  \href{https://help.nytimes3xbfgragh.onion/hc/en-us}{Help}
\item
  \href{https://www.nytimes3xbfgragh.onion/subscription?campaignId=37WXW}{Subscriptions}
\end{itemize}
