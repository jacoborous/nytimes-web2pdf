Sections

SEARCH

\protect\hyperlink{site-content}{Skip to
content}\protect\hyperlink{site-index}{Skip to site index}

\href{https://www.nytimes3xbfgragh.onion/section/us}{U.S.}

\href{https://myaccount.nytimes3xbfgragh.onion/auth/login?response_type=cookie\&client_id=vi}{}

\href{https://www.nytimes3xbfgragh.onion/section/todayspaper}{Today's
Paper}

\href{/section/us}{U.S.}\textbar{}From Antifa to Mothers in Helmets,
Diverse Elements Fuel Portland Protests

\url{https://nyti.ms/32yFL4v}

\begin{itemize}
\item
\item
\item
\item
\item
\item
\end{itemize}

\href{https://www.nytimes3xbfgragh.onion/news-event/george-floyd-protests-minneapolis-new-york-los-angeles?action=click\&pgtype=Article\&state=default\&region=TOP_BANNER\&context=storylines_menu}{Race
and America}

\begin{itemize}
\tightlist
\item
  \href{https://www.nytimes3xbfgragh.onion/2020/07/26/us/protests-portland-seattle-trump.html?action=click\&pgtype=Article\&state=default\&region=TOP_BANNER\&context=storylines_menu}{Protesters
  Return to Other Cities}
\item
  \href{https://www.nytimes3xbfgragh.onion/2020/07/24/us/portland-oregon-protests-white-race.html?action=click\&pgtype=Article\&state=default\&region=TOP_BANNER\&context=storylines_menu}{Portland
  at the Center}
\item
  \href{https://www.nytimes3xbfgragh.onion/2020/07/23/podcasts/the-daily/portland-protests.html?action=click\&pgtype=Article\&state=default\&region=TOP_BANNER\&context=storylines_menu}{Podcast:
  Showdown in Portland}
\item
  \href{https://www.nytimes3xbfgragh.onion/interactive/2020/07/16/us/black-lives-matter-protests-louisville-breonna-taylor.html?action=click\&pgtype=Article\&state=default\&region=TOP_BANNER\&context=storylines_menu}{45
  Days in Louisville}
\end{itemize}

Advertisement

\protect\hyperlink{after-top}{Continue reading the main story}

Supported by

\protect\hyperlink{after-sponsor}{Continue reading the main story}

\hypertarget{from-antifa-to-mothers-in-helmets-diverse-elements-fuel-portland-protests}{%
\section{From Antifa to Mothers in Helmets, Diverse Elements Fuel
Portland
Protests}\label{from-antifa-to-mothers-in-helmets-diverse-elements-fuel-portland-protests}}

Protesters have been in the streets for more than 50 consecutive days.
Federal agents deployed to Portland have hardened their resolve to stay
there.

\includegraphics{https://static01.graylady3jvrrxbe.onion/images/2020/07/19/us/19unrest-portland-1/merlin_174739467_d8a9b93e-8600-4724-8e85-a9f05dc7ec39-videoSixteenByNineJumbo1600.jpg}

By Sergio Olmos,
\href{https://www.nytimes3xbfgragh.onion/by/rick-rojas}{Rick Rojas} and
\href{https://www.nytimes3xbfgragh.onion/by/mike-baker}{Mike Baker}

\begin{itemize}
\item
  Published July 19, 2020Updated July 29, 2020
\item
  \begin{itemize}
  \item
  \item
  \item
  \item
  \item
  \item
  \end{itemize}
\end{itemize}

PORTLAND, Ore. --- Angela Foster started showing up in the early days of
the
\href{https://www.nytimes3xbfgragh.onion/2020/07/29/us/protests-portland-federal-withdrawal.html}{protests
in Portland} as one of the novice activists standing off to the side
with no gear to protect herself.

Roughly 40 demonstrations later, she has moved toward the front, wearing
a mask, goggles and a helmet, and bracing for law enforcement officers
to charge at her.

``We're not leaving,'' Ms. Foster said in an interview on Sunday.

While President Trump on Sunday described the unrest in
\href{https://www.nytimes3xbfgragh.onion/2020/07/20/us/portland-protests-navy-christopher-david.html}{Portland}
as a national threat involving ``anarchists and agitators,'' the
protests have featured a wide array of demonstrators, many now
galvanized by federal officers exemplifying the militarized enforcement
that protesters have long denounced. Gatherings over the weekend grew to
upward of 1,000 people ---~the largest crowds in weeks.

Some protesters have exhibited the lawless behavior that federal
officials have cited to justify their crackdown: Some have thrown cans
and bottles, shot fireworks or pointed lasers at officers. One was
recently accused of hitting a federal officer with a hammer. On
Saturday, protesters set a fire in the
\href{https://www.nytimes3xbfgragh.onion/2020/07/21/us/detroit-police-shooting-journalists.html}{police}
union headquarters.

\includegraphics{https://static01.graylady3jvrrxbe.onion/images/2020/07/19/us/19unrest-portland-2/merlin_174737520_5eefd351-ddf3-4dd8-81a2-f03d994a8e15-articleLarge.jpg?quality=75\&auto=webp\&disable=upscale}

But many others have demonstrated in the streets through peaceful means,
appalled by the aggressive responses by federal officers that have left
some protesters injured and the air inflamed with tear gas. They have
held signs and marched. At times when people have thrown bottles, other
demonstrators have rushed to try to stop them. On Saturday, a group of
women locked arms and
\href{https://twitter.com/JoshuaPotash/status/1284704272282800128}{chanted:
``Feds stay clear. Moms are here.''}

Attending the protests for the first time over the weekend was
Christopher David, 53, a former Navy civil engineering corps officer and
a 1988 graduate of the U.S. Naval Academy.

``I wasn't even paying attention to the protests at all until the feds
came in,'' Mr. David said. ``When that video came out of those two
unmarked guys in camouflage abducting people and putting them in
minivans, that's when I became aware.''

He had taken a bus to the Portland courthouse and was about to leave
around 10:45 p.m. when federal officers emerged and began advancing on
the protesters. He said he felt the need to ask the officers, Why were
they violating their oath to the Constitution?

Instead of getting an answer on Saturday, Mr. David, a 6-foot-2,
280-pound former Navy varsity wrestler, found himself being beaten with
a baton by a federal officer dressed in camouflage fatigues as another
doused him with pepper spray,
\href{https://twitter.com/PDXzane/status/1284726088187310080}{according
to video of the encounter}.

Mr. David was taken to a nearby hospital, where a specialist said his
right hand was broken and would require surgery to install pins, screws
and plates.

``I'm appalled and disappointed at the feds' behavior --- that whoever
led them and trained them allowed them to become this way,'' Mr. David
said. ``This is a failure of leadership more than it is a failure of
their own individual behavior towards me.''

Luis Enrique Marquez, a self-described anti-fascist who has been a
fixture at protests in Portland for years, said the purpose of the
federal officers' arrival had appeared to be to scare the protesters.
But he said the officers had instead galvanized them by displaying the
types of actions that have concerned protesters for years.

``With every act of violence they commit, our numbers seem to grow,
people seem to get more angry,'' Mr. Marquez said.

Demonstrators in Portland, including some who identify as antifa, the
loose coalition of self-described anti-fascist activists, have had years
of conflict with law enforcement. But after the killing of George Floyd
in Minneapolis set off a nationwide movement for racial justice and
police accountability, the protest in Portland drew thousands to the
streets.

That created powerful scenes including images of protesters blanketing
the Burnside Bridge, each lying face down on the pavement for
\href{https://www.nytimes3xbfgragh.onion/2020/06/18/us/george-floyd-timing.html}{eight
minutes and 46 seconds in remembrance of Mr. Floyd}.

Image

Dozens of protesters at the front carry homemade shields made out of
materials such as 55-gallon drums.Credit...Mason Trinca for The New York
Times

While those initial mass crowds have waned, hundreds of protesters have
continued on with near-nightly confrontations with law enforcement.

Unlike demonstrators in Seattle at the Capitol Hill Organized Protest,
or CHOP, in which they established a permanent location that created
\href{https://www.nytimes3xbfgragh.onion/2020/07/01/us/seattle-protest-zone-CHOP-CHAZ-unrest.html}{tensions
over how the police should handle unrest inside the area}, protesters in
Portland have brought the same feel of communal support throughout the
downtown area. Volunteers wearing red crosses hand out ear plugs, eye
wash and hand sanitizer. A mobile snack van provides Gatorade and food.

Jeremy Vajko, who operates the snack van, said he initially operated in
the CHOP zone in Seattle and then came to Portland to support the people
on the streets.

``I noticed there was problems with nutrition,'' he said. ``People are
sleep deprived.''

During the daytime, the protests can draw families, businesspeople and
political leaders such as Jo Ann Hardesty, a city commissioner. At
night, the crowd is made up mostly of young people. Dozens of protesters
at the front carry homemade shields made out of materials such as
55-gallon drums. Others stand farther back, shining lasers or gathering
materials for building barricades.

But protesters' tactics have strained the city. Business owners, already
struggling because of the coronavirus pandemic, have cited the protests
as a reason residents have been staying away from downtown.

Susan Landa, who for almost 31 years has owned a business selling gems
and minerals downtown, said she supports peaceful protests and even
defunding and shifting funds from the police.

But she said some of the protesters seemed like ``vandals'' and restless
young people who were ``taking out rage because of the pandemic.''

She added: ``Most of downtown is boarded up. We don't feel safe enough
to open up. It's killing our businesses.''

Some leaders in the Black community have also questioned the tactics,
suggesting that some demonstrators have seized the moment in the
aftermath of Mr. Floyd's killing to advance their own causes.

Last month, officers from the Portland Police Bureau repeatedly fired
tear gas and made arrests of protesters, who have variously called for
the abolishment or defunding of the bureau, and for more accountability
for law enforcement officers. The city's officers now operate with new
limits on the use of tear gas after a judge ordered it to only be used
if it's needed to keep people safe.

Protesters have focused much of their attention on Mayor Ted Wheeler,
who also serves as police commissioner. Crowds have at times gathered
late at night outside Mr. Wheeler's condo building, shining lights and
chanting about the perceived failures of his administration.

Image

Local leaders have grown increasingly vocal in opposition to the federal
presence, likening it to what happens in authoritarian
regimes.~Credit...Mason Trinca for The New York Times

For weeks, Mr. Wheeler has called for an end to destructive
demonstrations, saying he is concerned about ``groups who continue to
perpetrate violence and vandalism on our streets.'' But as federal
agencies have moved in to play a role in combating the unrest, Mr.
Wheeler has said he told the federal officials to stay away.

City police leaders have said they are not coordinating with federal
agencies on the protests. But at one point early Saturday morning, a
line of federal officers was moving up one street while a line of local
police officers was moving up another, both advancing to keep protesters
on the move. It was unclear what level of coordination was involved in
that effort.

Mr. Trump said in a Twitter post on Sunday that federal officials were
``trying to help Portland, not hurt it.'' Mr. Trump, who has said states
need to ``dominate'' protesters, said Portland officials had lost
control.

``They are missing in action,'' Mr. Trump wrote. ``We must protect
Federal property, AND OUR PEOPLE.''

Local leaders have grown increasingly vocal in opposition to the federal
presence after one protester appeared to have been shot in the head with
what was described as a less-lethal munition, severely injuring him in a
bloody scene that was captured on video. Federal officers have operated
from unmarked vans, at times seizing protesters and pulling them into
the vehicles.

Image

A protester on Saturday night.~Credit...Mason Trinca for The New York
Times

Joel B. Barker, who runs a marketing agency, said that he had frequently
participated in protests during the day near the Justice Center, which
includes the county jail, and that he usually left before 9 p.m. at the
latest. He said that the protests drew a diverse crowd, reflecting a
range of racial backgrounds, age and socioeconomic statuses, and that
there was a sense of unity.

He lives about a mile away, and the demonstrations have not had any
repercussions close to his home. The demonstrators, he said, were
largely peaceful and not there to foment disorder.

Mr. Barker said he felt rage that the city was being used for what he
believed was a ploy for the president in an election year.

``It's really terrible,'' he added, ``and I want America to understand
how terrible it is to feel like a city you love is being occupied by
your own federal government, because that's how it feels.''

Oregon's attorney general, Ellen Rosenblum, has filed a lawsuit seeking
to halt some of the detainment tactics used by federal officers. Her
office has also opened a criminal investigation into the case of the
protester who sustained a head injury.

Lisa Reynolds, a pediatrician who is running as a Democrat for a seat in
the Oregon House of Representatives, said she had tried to keep her
distance from the protests, largely because of the coronavirus crisis.
But on Sunday, she said, she was going to be fitted for a respirator so
she would be safer at protests where tear gas is used.

``I think my fear kept me away,'' she said. ``I think this is a step
where I need to put myself out there a little more.''

Sergio Olmos reported from Portland, Rick Rojas from Atlanta and Mike
Baker from Seattle. John Ismay contributed reporting from Arlington, Va.

Advertisement

\protect\hyperlink{after-bottom}{Continue reading the main story}

\hypertarget{site-index}{%
\subsection{Site Index}\label{site-index}}

\hypertarget{site-information-navigation}{%
\subsection{Site Information
Navigation}\label{site-information-navigation}}

\begin{itemize}
\tightlist
\item
  \href{https://help.nytimes3xbfgragh.onion/hc/en-us/articles/115014792127-Copyright-notice}{©~2020~The
  New York Times Company}
\end{itemize}

\begin{itemize}
\tightlist
\item
  \href{https://www.nytco.com/}{NYTCo}
\item
  \href{https://help.nytimes3xbfgragh.onion/hc/en-us/articles/115015385887-Contact-Us}{Contact
  Us}
\item
  \href{https://www.nytco.com/careers/}{Work with us}
\item
  \href{https://nytmediakit.com/}{Advertise}
\item
  \href{http://www.tbrandstudio.com/}{T Brand Studio}
\item
  \href{https://www.nytimes3xbfgragh.onion/privacy/cookie-policy\#how-do-i-manage-trackers}{Your
  Ad Choices}
\item
  \href{https://www.nytimes3xbfgragh.onion/privacy}{Privacy}
\item
  \href{https://help.nytimes3xbfgragh.onion/hc/en-us/articles/115014893428-Terms-of-service}{Terms
  of Service}
\item
  \href{https://help.nytimes3xbfgragh.onion/hc/en-us/articles/115014893968-Terms-of-sale}{Terms
  of Sale}
\item
  \href{https://spiderbites.nytimes3xbfgragh.onion}{Site Map}
\item
  \href{https://help.nytimes3xbfgragh.onion/hc/en-us}{Help}
\item
  \href{https://www.nytimes3xbfgragh.onion/subscription?campaignId=37WXW}{Subscriptions}
\end{itemize}
