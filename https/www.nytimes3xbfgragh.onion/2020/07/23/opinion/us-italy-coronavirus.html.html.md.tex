Sections

SEARCH

\protect\hyperlink{site-content}{Skip to
content}\protect\hyperlink{site-index}{Skip to site index}

\href{https://myaccount.nytimes3xbfgragh.onion/auth/login?response_type=cookie\&client_id=vi}{}

\href{https://www.nytimes3xbfgragh.onion/section/todayspaper}{Today's
Paper}

\href{/section/opinion}{Opinion}\textbar{}Why Can't Trump's America Be
Like Italy?

\url{https://nyti.ms/32M156E}

\begin{itemize}
\item
\item
\item
\item
\item
\item
\end{itemize}

Advertisement

\protect\hyperlink{after-top}{Continue reading the main story}

\href{/section/opinion}{Opinion}

Supported by

\protect\hyperlink{after-sponsor}{Continue reading the main story}

\hypertarget{why-cant-trumps-america-be-like-italy}{%
\section{Why Can't Trump's America Be Like
Italy?}\label{why-cant-trumps-america-be-like-italy}}

On the coronavirus, the ``sick man of Europe'' puts us to shame.

\href{https://www.nytimes3xbfgragh.onion/by/paul-krugman}{\includegraphics{https://static01.graylady3jvrrxbe.onion/images/2018/04/02/opinion/paul-krugman/paul-krugman-thumbLarge.png}}

By \href{https://www.nytimes3xbfgragh.onion/by/paul-krugman}{Paul
Krugman}

Opinion Columnist

\begin{itemize}
\item
  July 23, 2020
\item
  \begin{itemize}
  \item
  \item
  \item
  \item
  \item
  \item
  \end{itemize}
\end{itemize}

\includegraphics{https://static01.graylady3jvrrxbe.onion/images/2020/07/23/opinion/23krugmanWeb/merlin_173582502_69386d01-ba07-4dc5-a3c3-74023fb4ad87-articleLarge.jpg?quality=75\&auto=webp\&disable=upscale}

A few days ago The Times published a long, damning
\href{https://www.nytimes3xbfgragh.onion/2020/07/18/us/politics/trump-coronavirus-response-failure-leadership.html}{article}
about how the Trump administration managed to fail so completely in
responding to the coronavirus. Much of the content confirmed what anyone
following the debacle suspected. One thing I didn't see coming, however,
was the apparently central role played by Italy's experience.

Italy, you see, was the first Western nation to experience a major wave
of infections. Hospitals were overwhelmed; partly as a result, the
initial death toll was terrible. Yet cases peaked after a few weeks and
began a steep decline. And White House officials were seemingly
confident that America would follow a similar track.

We didn't. U.S. cases plateaued for a couple of months, then began
rising rapidly. Death rates followed with a lag. At this point we can
only look longingly at Italy's success in containing the coronavirus:
Restaurants and cafes are
\href{https://www.washingtonpost.com/travel/2020/06/20/rome-is-ready-visitors-again-its-famous-restaurants-cafes-look-little-different/}{open},
albeit with restrictions, much of normal life has resumed, yet Italy's
current
\href{https://ourworldindata.org/coronavirus-data-explorer?zoomToSelection=true\&casesMetric=true\&dailyFreq=true\&perCapita=true\&smoothing=7\&country=USA~ITA\&pickerMetric=location\&pickerSort=asc}{death
rate} is less than a 10th of America's. On a typical recent day, more
than 800 Americans but only around a dozen Italians died from Covid-19.

Although Donald Trump keeps boasting that we've had the best coronavirus
response in the world, and some credulous supporters may actually
believe him, my guess is that many people are aware that our handling of
the virus has fallen tragically short compared with, say, that of
Germany. It may not seem surprising, however, that German discipline and
competence have paid off (although we used to think that we were
\href{https://www.washingtonpost.com/health/2019/10/24/none-these-countries-us-included-is-fully-prepared-pandemic-report-says/}{better
prepared} than anyone else to deal with a pandemic). But how can America
be doing so much worse than Italy?

I don't mean to peddle facile national stereotypes. For all its
problems, Italy is a serious and sophisticated country, not a
comic-opera stage set. Still, Italy entered this pandemic with major
disadvantages compared with the United States.

After all, Italy's bureaucracy isn't famed for its efficiency, nor are
its citizens known for their willingness to follow rules. The nation's
government is deeply in debt, and this debt matters because Italy
doesn't have its own currency; this means that it can't do
\href{https://www.nytimes3xbfgragh.onion/2020/07/22/opinion/economy-spending-modern-monetary-theory.html}{what
we do}, and print lots of money in a crisis.

Unfavorable demography and economic troubles are also major Italian
disadvantages. The
\href{https://data.oecd.org/pop/elderly-population.htm}{ratio} of
seniors to working-age adults is the highest in the Western world.
Italy's growth record is deeply disappointing: Per capita G.D.P. has
\href{https://fred.stlouisfed.org/series/NYGDPPCAPKDITA}{stagnated} for
two decades.

When it came to dealing with Covid-19, however, all these Italian
disadvantages were outweighed by one huge advantage: Italy wasn't
burdened with America's disastrous leadership.

After a terrible start, Italy quickly moved to do what was necessary to
deal with the coronavirus. It instituted a very severe lockdown, and
kept to it. Government aid helped sustain workers and businesses through
the lockdown. The safety net had holes in it, but top officials tried to
make it work; in a supreme case of non-Trumpism, the prime minister even
\href{https://www.bloomberg.com/news/articles/2020-05-01/europe-s-wary-reopening-begins-with-apology-by-italian-leader?sref=qzusa8bC}{apologized}
for delays in aid.

\includegraphics{https://static01.graylady3jvrrxbe.onion/images/2020/07/23/opinion/23krugman2/merlin_171138090_5cef07c3-5488-40e5-ae48-90be8c6458ae-articleLarge.jpg?quality=75\&auto=webp\&disable=upscale}

And, crucially, Italy crushed the curve: It kept the lockdown in place
until cases were relatively few, and it was cautious about reopening.

America could have followed the same path. In fact, the Covid-19
trajectory in the
\href{https://www.nytimes3xbfgragh.onion/2020/07/23/briefing/home-schooling-coronavirus-portland-your-thursday-briefing.html}{Northeast},
which was hard-hit in the beginning but took the outbreak seriously,
actually does look a lot like Italy's.

But the Trump administration and its allies pushed for rapid reopening,
ignoring warnings from epidemiologists. Because we didn't do what Italy
did, we didn't crush the curve; quite the opposite. Matters were made
worse by pathological opposition to things like wearing masks, the way
even obvious precautions became battlegrounds in the culture wars.

So cases and then deaths surged. Even the promised economic payoff from
rapid, what-me-worry reopening was a mirage: many states are reimposing
partial lockdowns, and there is growing evidence that the jobs recovery
is stalling, if not going into
\href{https://twitter.com/bencasselman/status/1285964009326477315}{reverse}.

Incredibly, Trump and his allies seem to have given no thought at all
about what to do if the overwhelming view of experts was right, and
their gamble on ignoring the coronavirus didn't pan out. A miraculous
boom was Plan A; there was no Plan B.

In particular, tens of millions of workers are about to lose crucial
unemployment benefits, and Republicans haven't even settled on a bad
response. On Wednesday Senate Republicans floated the idea of reducing
supplemental benefits from \$600 a week to
\href{https://www.cnbc.com/2020/07/22/coronavirus-stimulus-republicans-consider-unemployment-insurance-extension.html}{just
\$100}, which would spell disaster for many families.

For someone like Trump, all this must be humiliating --- or would be if
anyone dared tell him about it. After three and a half years of Making
America Great Again, we've become a pathetic figure on the world stage,
a cautionary tale about pride going before a fall.

These days Americans can only envy Italy's success in weathering the
coronavirus, its rapid return to a kind of normalcy that is a distant
dream in a nation that used to congratulate itself for its can-do
culture. Italy is often referred to as ``the sick man of Europe''; what
does that make us?

\emph{The Times is committed to publishing}
\href{https://www.nytimes3xbfgragh.onion/2019/01/31/opinion/letters/letters-to-editor-new-york-times-women.html}{\emph{a
diversity of letters}} \emph{to the editor. We'd like to hear what you
think about this or any of our articles. Here are some}
\href{https://help.nytimes3xbfgragh.onion/hc/en-us/articles/115014925288-How-to-submit-a-letter-to-the-editor}{\emph{tips}}\emph{.
And here's our email:}
\href{mailto:letters@NYTimes.com}{\emph{letters@NYTimes.com}}\emph{.}

\emph{Follow The New York Times Opinion section on}
\href{https://www.facebookcorewwwi.onion/nytopinion}{\emph{Facebook}}\emph{,}
\href{http://twitter.com/NYTOpinion}{\emph{Twitter (@NYTopinion)}}
\emph{and}
\href{https://www.instagram.com/nytopinion/}{\emph{Instagram}}\emph{.}

Advertisement

\protect\hyperlink{after-bottom}{Continue reading the main story}

\hypertarget{site-index}{%
\subsection{Site Index}\label{site-index}}

\hypertarget{site-information-navigation}{%
\subsection{Site Information
Navigation}\label{site-information-navigation}}

\begin{itemize}
\tightlist
\item
  \href{https://help.nytimes3xbfgragh.onion/hc/en-us/articles/115014792127-Copyright-notice}{©~2020~The
  New York Times Company}
\end{itemize}

\begin{itemize}
\tightlist
\item
  \href{https://www.nytco.com/}{NYTCo}
\item
  \href{https://help.nytimes3xbfgragh.onion/hc/en-us/articles/115015385887-Contact-Us}{Contact
  Us}
\item
  \href{https://www.nytco.com/careers/}{Work with us}
\item
  \href{https://nytmediakit.com/}{Advertise}
\item
  \href{http://www.tbrandstudio.com/}{T Brand Studio}
\item
  \href{https://www.nytimes3xbfgragh.onion/privacy/cookie-policy\#how-do-i-manage-trackers}{Your
  Ad Choices}
\item
  \href{https://www.nytimes3xbfgragh.onion/privacy}{Privacy}
\item
  \href{https://help.nytimes3xbfgragh.onion/hc/en-us/articles/115014893428-Terms-of-service}{Terms
  of Service}
\item
  \href{https://help.nytimes3xbfgragh.onion/hc/en-us/articles/115014893968-Terms-of-sale}{Terms
  of Sale}
\item
  \href{https://spiderbites.nytimes3xbfgragh.onion}{Site Map}
\item
  \href{https://help.nytimes3xbfgragh.onion/hc/en-us}{Help}
\item
  \href{https://www.nytimes3xbfgragh.onion/subscription?campaignId=37WXW}{Subscriptions}
\end{itemize}
