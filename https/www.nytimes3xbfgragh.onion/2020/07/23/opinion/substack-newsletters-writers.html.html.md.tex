Sections

SEARCH

\protect\hyperlink{site-content}{Skip to
content}\protect\hyperlink{site-index}{Skip to site index}

\href{https://myaccount.nytimes3xbfgragh.onion/auth/login?response_type=cookie\&client_id=vi}{}

\href{https://www.nytimes3xbfgragh.onion/section/todayspaper}{Today's
Paper}

\href{/section/opinion}{Opinion}\textbar{}The Future of Nonconformity

\url{https://nyti.ms/2WPhfbD}

\begin{itemize}
\item
\item
\item
\item
\item
\item
\end{itemize}

Advertisement

\protect\hyperlink{after-top}{Continue reading the main story}

\href{/section/opinion}{Opinion}

Supported by

\protect\hyperlink{after-sponsor}{Continue reading the main story}

\hypertarget{the-future-of-nonconformity}{%
\section{The Future of
Nonconformity}\label{the-future-of-nonconformity}}

Where freethinkers go to fight.

\href{https://www.nytimes3xbfgragh.onion/by/david-brooks}{\includegraphics{https://static01.graylady3jvrrxbe.onion/images/2018/04/03/opinion/david-brooks/david-brooks-thumbLarge-v2.png}}

By \href{https://www.nytimes3xbfgragh.onion/by/david-brooks}{David
Brooks}

Opinion Columnist

\begin{itemize}
\item
  July 23, 2020
\item
  \begin{itemize}
  \item
  \item
  \item
  \item
  \item
  \item
  \end{itemize}
\end{itemize}

\includegraphics{https://static01.graylady3jvrrxbe.onion/images/2020/07/23/opinion/23brooksNew/23brooksNew-articleLarge.jpg?quality=75\&auto=webp\&disable=upscale}

Like other realms, American intellectual life has been marked by a
series of exclusions. The oldest and vastest was the exclusion of people
of color from the commanding institutions of our culture.

Today, there's the exclusion of conservatives from academic life. Then
there's the exclusion of working-class voices from mainstream media. Our
profession didn't used to be all coastal yuppies, but now it mostly is.
Then there's the marginalization of those with radical critiques ---
from say, the Marxist left and the theological right.

Intellectual exclusion and segregation have been terrible for America,
poisoning both the right and the left.

Conservatives were told their voices didn't matter, and many reacted in
a childish way that seemed to justify that exclusion. A corrosive spirit
of resentment and victimhood spread across the American right --- an
intellectual inferiority complex combined with a moral superiority
complex.

\includegraphics{https://static01.graylady3jvrrxbe.onion/images/2020/04/24/autossell/24Brooks-twitter-thumb/24Brooks-twitter-thumb-videoSixteenByNineJumbo1600.jpg}

For many on the right the purpose of thinking changed. Thinking was no
longer for understanding. Thinking was for belonging. Right-wing talk
radio is the endless repetition of familiar mantras to reassure
listeners that they are all on the same team. Thinking was for conquest:
Those liberals think they're better than us, but we own the libs.

Thinking itself became suspect. Sarah Palin and Donald Trump
reintroduced anti-intellectualism into the American right: a distrust of
the media, expertise and facts. A president who dispenses with the pen
inevitably takes up the club.

Intellectual segregation has been bad for the left, too. It produced
insularity. Progressives are often blindsided by reality --- blindsided
that Trump won the presidency; blindsided that Joe Biden clinched the
Democratic presidential nomination. The second consequence is fragility.
When you make politics the core of your religious identity, and you
shield yourself from heresy, then any glimpse of that heresy is going to
provoke an extreme emotional reaction. The third consequence is
conformity. Writers are now expected to write as a representative of a
group, in order to affirm the self-esteem of the group. Predictability
is the point.

In some ways the left has become even more conformist than the right.
The liberal New Republic has less viewpoint diversity than the
conservative National Review --- a reversal of historical patterns.
Christopher Hitchens was one of the great essayists in America. He would
be unemployable today because there was no set of priors he wasn't
willing to offend.

Now the boundaries of exclusion are shifting again. What we erroneously
call ``cancel culture'' is an attempt to shift the boundaries of the
sayable so it excludes not only conservatives but liberals and the
heterodox as well. Hence the attacks on, say, Steven Pinker and Andrew
Sullivan.

This is not just an elite or rare phenomenon. Sixty-two percent of
Americans say they are afraid to share things they believe, according to
a
\href{https://www.cato.org/publications/survey-reports/poll-62-americans-say-they-have-political-views-theyre-afraid-share\#_blank}{poll}
for the Cato Institute. A majority of staunch progressives say they feel
free to share their political views, but majorities of liberals,
moderates and conservatives are afraid to.

Happily, there's a growing rebellion against groupthink and exclusion. A
Politico
\href{https://www.politico.com/news/2020/07/22/americans-cancel-culture-377412}{poll}
found that 49 percent of Americans say the cancel culture has a negative
impact on society and only 27 say it has a positive impact. This month
Yascha Mounk started
\href{https://www.persuasion.community/}{Persuasion}, an online
community to celebrate viewpoint diversity and it already has more than
25,000 subscribers.

After being pushed out from New York magazine, Sullivan established his
own newsletter, \href{https://andrewsullivan.substack.com/subscribe}{The
Weekly Dish}, on Substack, a platform that makes it easy for readers to
pay writers for their work. He now has 60,000 subscribers, instantly
making his venture financially viable.

Other heterodox writers are already on Substack.
\href{https://taibbi.substack.com/}{Matt Taibbi} and Judd Legum are
iconoclastic left-wing writers with large subscriber bases.
\href{https://thedispatch.com/}{The Dispatch} is a conservative
publication featuring Jonah Goldberg, David French and Stephen F. Hayes,
superb writers but too critical of Trump for the orthodox right. The
Dispatch is reportedly making about \$2 million a year on Substack.

The first good thing about Substack is there's no canceling. A young,
talented heterodox thinker doesn't have to worry that less talented
conformists in his or her organization will use ideology as an outlet
for their resentments. The next good thing is there are no ads, just
subscription revenue. Online writers don't have to chase clicks by
writing about whatever Trump tweeted 15 seconds ago. They can build deep
relationships with the few rather than trying to affirm or titillate the
many.

It's possible that the debate now going on stupidly on Twitter can
migrate to newsletters. It's possible that writers will bundle, with
established writers promoting promising ones. It's possible that those
of us at the great remaining mainstream outlets will be enmeshed in
conversations that are more freewheeling and thoughtful.

Mostly I'm hopeful that the long history of intellectual exclusion and
segregation will seem disgraceful. It will seem disgraceful if you're at
a university and only 1.5 percent of the faculty members are
conservative. (I'm looking at you,
\href{https://www.thecrimson.com/article/2020/3/3/faculty-support-warren-president/}{Harvard}).
A person who ideologically self-segregates will seem pathetic. I'm
hoping the definition of a pundit changes --- not a foot soldier out for
power, but a person who argues in order to come closer to understanding.

\emph{The Times is committed to publishing}
\href{https://www.nytimes3xbfgragh.onion/2019/01/31/opinion/letters/letters-to-editor-new-york-times-women.html}{\emph{a
diversity of letters}} \emph{to the editor. We'd like to hear what you
think about this or any of our articles. Here are some}
\href{https://help.nytimes3xbfgragh.onion/hc/en-us/articles/115014925288-How-to-submit-a-letter-to-the-editor}{\emph{tips}}\emph{.
And here's our email:}
\href{mailto:letters@NYTimes.com}{\emph{letters@NYTimes.com}}\emph{.}

\emph{Follow The New York Times Opinion section on}
\href{https://www.facebookcorewwwi.onion/nytopinion}{\emph{Facebook}}\emph{,}
\href{http://twitter.com/NYTOpinion}{\emph{Twitter (@NYTopinion)}}
\emph{and}
\href{https://www.instagram.com/nytopinion/}{\emph{Instagram}}\emph{.}

Advertisement

\protect\hyperlink{after-bottom}{Continue reading the main story}

\hypertarget{site-index}{%
\subsection{Site Index}\label{site-index}}

\hypertarget{site-information-navigation}{%
\subsection{Site Information
Navigation}\label{site-information-navigation}}

\begin{itemize}
\tightlist
\item
  \href{https://help.nytimes3xbfgragh.onion/hc/en-us/articles/115014792127-Copyright-notice}{©~2020~The
  New York Times Company}
\end{itemize}

\begin{itemize}
\tightlist
\item
  \href{https://www.nytco.com/}{NYTCo}
\item
  \href{https://help.nytimes3xbfgragh.onion/hc/en-us/articles/115015385887-Contact-Us}{Contact
  Us}
\item
  \href{https://www.nytco.com/careers/}{Work with us}
\item
  \href{https://nytmediakit.com/}{Advertise}
\item
  \href{http://www.tbrandstudio.com/}{T Brand Studio}
\item
  \href{https://www.nytimes3xbfgragh.onion/privacy/cookie-policy\#how-do-i-manage-trackers}{Your
  Ad Choices}
\item
  \href{https://www.nytimes3xbfgragh.onion/privacy}{Privacy}
\item
  \href{https://help.nytimes3xbfgragh.onion/hc/en-us/articles/115014893428-Terms-of-service}{Terms
  of Service}
\item
  \href{https://help.nytimes3xbfgragh.onion/hc/en-us/articles/115014893968-Terms-of-sale}{Terms
  of Sale}
\item
  \href{https://spiderbites.nytimes3xbfgragh.onion}{Site Map}
\item
  \href{https://help.nytimes3xbfgragh.onion/hc/en-us}{Help}
\item
  \href{https://www.nytimes3xbfgragh.onion/subscription?campaignId=37WXW}{Subscriptions}
\end{itemize}
