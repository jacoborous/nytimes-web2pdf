Sections

SEARCH

\protect\hyperlink{site-content}{Skip to
content}\protect\hyperlink{site-index}{Skip to site index}

\href{https://www.nytimes3xbfgragh.onion/section/books/review}{Book
Review}

\href{https://myaccount.nytimes3xbfgragh.onion/auth/login?response_type=cookie\&client_id=vi}{}

\href{https://www.nytimes3xbfgragh.onion/section/todayspaper}{Today's
Paper}

\href{/section/books/review}{Book Review}\textbar{}The Last Book That
Made Aimee Bender Laugh Out Loud

\url{https://nyti.ms/32NEOFE}

\begin{itemize}
\item
\item
\item
\item
\item
\end{itemize}

Advertisement

\protect\hyperlink{after-top}{Continue reading the main story}

Supported by

\protect\hyperlink{after-sponsor}{Continue reading the main story}

\href{/column/by-the-book}{By the Book}

\hypertarget{the-last-book-that-made-aimee-bender-laugh-out-loud}{%
\section{The Last Book That Made Aimee Bender Laugh Out
Loud}\label{the-last-book-that-made-aimee-bender-laugh-out-loud}}

\includegraphics{https://static01.graylady3jvrrxbe.onion/images/2020/07/26/books/review/26ByTheBook/26ByTheBook-articleLarge.jpg?quality=75\&auto=webp\&disable=upscale}

July 23, 2020

\begin{itemize}
\item
\item
\item
\item
\item
\end{itemize}

\emph{```Trust Exercise,' by Susan Choi, was many remarkable things,
including really funny,'' says Bender, whose new novel is ``The
Butterfly Lampshade.'' ``Especially the most amazingly weird and right
sentence with `lasagna' in it.''}

\textbf{What books are on your nightstand?}

I like having lots of options so the nightstand is crowded! The lamp is
getting edged out.

``Freshwater,'' Akwaeke Emezi.

``The Life of Things, the Love of Things,'' Remo Bodei.

``Minor Feelings,'' Cathy Park Hong.

``The Fire This Time,'' Jesmyn Ward anthology.

``feeld,'' Jos Charles.

``Essential Earthman,'' Henry Mitchell.

``Jazz,'' Toni Morrison.

``Why Does the World Exist?'' Jim Holt.

``This Is Pleasure,'' Mary Gaitskill.

``American Sonnets for My Past and Future Assassin,'' Terrance Hayes.

``On Imagination,'' Mary Ruefle.

``The Collected Schizophrenias,'' Esmé Weijun Wang.

``The Dream of the Moving Statue,'' Kenneth Gross.

\textbf{What's the last great book you read?}

``The Story of a Brief Marriage,'' by Anuk Arudpragasam, a short novel
set over the course of a few days in a refugee camp during the Sri
Lankan civil war. He inhabits the writing so deeply on a sensory level,
taking his time, bringing us right there into tremendous pain and
yearning for connection, and I think ultimately captures something
profound about trauma.

\textbf{Are there any classic novels that you only recently read for the
first time?}

``War and Peace,'' as guided by the quarantine-motivated blog posts on A
Public Space of the magnificent fiction writer Yiyun Li, whose entries
were so illuminating and endlessly charming! The blog has come and gone;
I'm still reading this book. How did Tolstoy do it? He seemed to be able
to hold breadth and subtle internal shifts so comfortably in his own
mind all at once. Internal and external so balanced. Like reading an
entire brain on the page, unfettered.

\textbf{Describe your ideal reading experience (when, where, what,
how).}

Post-nap so that I won't fall asleep, on a couch, window open, breeze,
beverage.

\textbf{What's your favorite book no one else has heard of?}

Here are a couple:

``How to Set a Fire and Why'' --- the voice Jesse Ball creates in this
book is unbelievable --- so intense and alive, of an incredibly smart,
honest, anarchic teenage girl. There's something fearless in here, and I
try to read whatever Ball writes.

``The Old Child,'' by Jenny Erpenbeck. I had this in my purse for
awhile, as my purse-book, and when I'd take it out and read a couple
pages I'd always have to dive back into the purse to find a pen because
I'd just underline and underline. All I do is underline this book. Who
on earth is this odd and riveting child?

\textbf{What book should nobody read until the age of 40?}

I was unsure what to do with Alice Munro until about 35 and then she
cracked open my entire sense of story. Since she is a master of how to
use time, it makes some sense to me that you have to have lived through
a certain amount of time to absorb what she's doing with narrative.

\textbf{Do you have any comfort reads?}

Kay Ryan's poetry is comforting to me: Her poems are short and
mathematical and ordered, yet plunge somewhere messy, light up something
about mystery and meaning. Each small poem enacts this balance. So it
feels like a way to dip into a Bach sonata or something, made into
language.

\textbf{What's the last book you read that made you laugh?}

``Trust Exercise,'' by Susan Choi, was many remarkable things, including
really funny, especially the most amazingly weird and right sentence
with ``lasagna'' in it that I plan on teaching in the fall as a tiny
microcosm model of voice.

Also ``New People,'' by Danzy Senna, has a narrative voice so funny and
cutting and revealing.

\textbf{The last book you read that made you cry?}

``A Constellation of Vital Phenomena,'' Anthony Marra, the ending
chapter, and poems by Jericho Brown in ``The Tradition.''

\textbf{Has a book ever brought you closer to another person, or come
between you?}

Once I was praising ``Time's Arrow,'' by Martin Amis, and a friend
called it ``gimmicky'' with a dismissive wave of the hand --- the novel
is told backward, as in all the characters actually move backward, as if
they are on rewind, which flips the meaning of many things, and in my
view, this change allowed for the emotional force and surprise of the
book. So that was a bummer of a conversation. But we got past it.

\textbf{Do you think any canonical books are widely misunderstood?}

I don't really think ``Ulysses'' is widely misunderstood, but I did have
my own narrow misunderstanding about it; I hadn't known that the first
three chapters focus on Stephen Dedalus, the same Stephen from ``A
Portrait of the Artist as a Young Man,'' a book I hadn't connected with,
but in Chapter 4, Dedalus moves aside and in comes big feeler Leopold
Bloom and the writing style changes dramatically. I took a wonderful
course with Wilton Barnhardt in graduate school, and he was the most
joyful guide to this book and let us soak in the language, and allow
that as our entry instead of trying to crack every allusion, which is
never my strength. Had I picked up the book on my own, without Wilton, I
would've felt so daunted by Dedalus and never found Bloom.

\textbf{What's the most interesting thing you learned from a book
recently?}

The Bodei book on my nightstand is all about how we navigate the world
of objects, written in such friendly prose --- orphaned objects,
clusters of objects that define a generation, objects that used to be
trackable to the humans that made them (cobbler, tailor, etc.) and now
aren't, and all the consequences of this.

\textbf{Do you prefer books that reach you emotionally, or
intellectually?}

I love this question. It's really both. When I read someone like Borges,
I feel keenly that through his thought experiments, he is getting to
emotion. And I feel this in a totally different way with Maggie Nelson:
She is able to wrangle with concepts in part through an emotional
palette, which allows me to understand theory under her guidance much
more than I might through a purely cerebral writer. I love seeing these
elements work together on the page. Whatever it is, I just want some
part of a book to be abstracted in some way so that the emotion can
sidle in unexpectedly --- it could be a kind of strangeness, or a formal
experimentation, or whatever, but it seems to be that once my thinky
self is occupied with that, I am freed up to feel.

\textbf{Which genres do you especially enjoy reading? And which do you
avoid?}

I am a fan of all genres! It ultimately comes down to language. When
something is told in a way that feels specific to that particular
writer, then the story can go anywhere and I will happily follow. I
don't think I read for plot or even character --- I read for something
living below the words that drives both of those things and shows up in
the sentences.

\textbf{How do you organize your books?}

Pile 1, Pile 2, Pile 3 \ldots{} Pile 10, and then some shelves.

\textbf{What kind of reader were you as a child? Which childhood books
and authors stick with you most?}

I was a big reader, and gobbled up especially the books of wondrous
lands. I must've read Baum's Oz books 50 times each. And all the
L'Engle, magical family and nonmagical family. Also ``The Four-Story
Mistake,'' ``A Tree Grows in Brooklyn,'' ``Cheaper by the Dozen,'' Julie
Andrews's marvel ``The Last of the Really Great Whangdoodles,'' fairy
tales from around the world, often collected in the Lang books ---
Lilac, Ochre and more --- ``The Phantom Tollbooth,'' Stuart Little
crashing his little light wooden hammer on the faucet to get enough
water to brush his teeth.

\textbf{What do you plan to read next?}

``Signs Preceding the End of the World,'' Yuri Herrera --- very excited
to dive into this one! I've heard great things.

Advertisement

\protect\hyperlink{after-bottom}{Continue reading the main story}

\hypertarget{site-index}{%
\subsection{Site Index}\label{site-index}}

\hypertarget{site-information-navigation}{%
\subsection{Site Information
Navigation}\label{site-information-navigation}}

\begin{itemize}
\tightlist
\item
  \href{https://help.nytimes3xbfgragh.onion/hc/en-us/articles/115014792127-Copyright-notice}{©~2020~The
  New York Times Company}
\end{itemize}

\begin{itemize}
\tightlist
\item
  \href{https://www.nytco.com/}{NYTCo}
\item
  \href{https://help.nytimes3xbfgragh.onion/hc/en-us/articles/115015385887-Contact-Us}{Contact
  Us}
\item
  \href{https://www.nytco.com/careers/}{Work with us}
\item
  \href{https://nytmediakit.com/}{Advertise}
\item
  \href{http://www.tbrandstudio.com/}{T Brand Studio}
\item
  \href{https://www.nytimes3xbfgragh.onion/privacy/cookie-policy\#how-do-i-manage-trackers}{Your
  Ad Choices}
\item
  \href{https://www.nytimes3xbfgragh.onion/privacy}{Privacy}
\item
  \href{https://help.nytimes3xbfgragh.onion/hc/en-us/articles/115014893428-Terms-of-service}{Terms
  of Service}
\item
  \href{https://help.nytimes3xbfgragh.onion/hc/en-us/articles/115014893968-Terms-of-sale}{Terms
  of Sale}
\item
  \href{https://spiderbites.nytimes3xbfgragh.onion}{Site Map}
\item
  \href{https://help.nytimes3xbfgragh.onion/hc/en-us}{Help}
\item
  \href{https://www.nytimes3xbfgragh.onion/subscription?campaignId=37WXW}{Subscriptions}
\end{itemize}
