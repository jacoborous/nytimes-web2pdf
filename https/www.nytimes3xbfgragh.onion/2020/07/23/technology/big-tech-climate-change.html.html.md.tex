Sections

SEARCH

\protect\hyperlink{site-content}{Skip to
content}\protect\hyperlink{site-index}{Skip to site index}

\href{https://www.nytimes3xbfgragh.onion/section/technology}{Technology}

\href{https://myaccount.nytimes3xbfgragh.onion/auth/login?response_type=cookie\&client_id=vi}{}

\href{https://www.nytimes3xbfgragh.onion/section/todayspaper}{Today's
Paper}

\href{/section/technology}{Technology}\textbar{}Big Tech Versus Climate
Change

\url{https://nyti.ms/2D4S1Pe}

\begin{itemize}
\item
\item
\item
\item
\item
\end{itemize}

Advertisement

\protect\hyperlink{after-top}{Continue reading the main story}

Supported by

\protect\hyperlink{after-sponsor}{Continue reading the main story}

on tech

\hypertarget{big-tech-versus-climate-change}{%
\section{Big Tech Versus Climate
Change}\label{big-tech-versus-climate-change}}

How tech companies and all of us can help slow global warming.

\includegraphics{https://static01.graylady3jvrrxbe.onion/images/2020/07/23/business/23ontech-videostill/23ontech-videostill-mediumSquareAt3X.png}

\href{https://www.nytimes3xbfgragh.onion/by/shira-ovide}{\includegraphics{https://static01.graylady3jvrrxbe.onion/images/2020/03/18/reader-center/author-shira-ovide/author-shira-ovide-thumbLarge-v2.png}}

By \href{https://www.nytimes3xbfgragh.onion/by/shira-ovide}{Shira Ovide}

\begin{itemize}
\item
  July 23, 2020
\item
  \begin{itemize}
  \item
  \item
  \item
  \item
  \item
  \end{itemize}
\end{itemize}

\emph{This article is part of the On Tech newsletter. You can}
\href{https://www.nytimes3xbfgragh.onion/newsletters/signup/OT}{\emph{sign
up here}} \emph{to receive it weekdays.}

A growing share of Americans are
\href{https://www.nytimes3xbfgragh.onion/2020/07/17/briefing/coronavirus-brian-kemp-washington-redskins-your-friday-briefing.html}{concerned
about the environment}, and the big U.S. tech companies would seem to be
in a position to lead the way on fighting climate change.

They're rich and staffed with smart people, and they have generally
\href{https://www.nytimes3xbfgragh.onion/2020/07/21/climate/apple-emissions-pledge.html}{pledged
to do more to reduce the carbon emissions} that warm the planet.

My colleague
\href{https://www.nytimes3xbfgragh.onion/by/somini-sengupta}{Somini
Sengupta}, who writes about climate change and used to cover the tech
industry, walked me through confusing climate change terms and how tech
companies and all of us can help slow global warming.

\textbf{Shira: What does it mean when a company pledges to go
``\href{https://www.apple.com/newsroom/2020/07/apple-commits-to-be-100-percent-carbon-neutral-for-its-supply-chain-and-products-by-2030/}{carbon
neutral}'' or
``\href{https://blogs.microsoft.com/on-the-issues/2020/07/21/carbon-negative-transform-to-net-zero/}{carbon
negative}?''}

\textbf{Somini:} A company will still produce carbon emissions, but it
will offset that by doing things that absorb emissions from the
atmosphere --- like planting forests. Trees are great! They absorb
carbon dioxide. At least some portion of Amazon's and Apple's climate
action plans involve reforestation.

But that's not enough. Climate scientists say global emissions must be
cut by half by 2030 if we stand a chance of averting the worst impacts
of warming.

\textbf{How do tech companies contribute to climate change, and how are
they helping?}

First, the industry uses lots of electricity, including for
\href{https://www.nytimes3xbfgragh.onion/2020/07/08/technology/internet-infrastructure.html}{computer
data centers}. If much of that comes from coal, it creates a boatload of
planet-warming emissions. This is a relatively easy problem to solve if
companies
\href{https://aws.amazon.com/about-aws/sustainability/sustainability-timeline/}{use
renewable energy}, which is expanding fast and getting cheaper.

\href{https://www.nytimes3xbfgragh.onion/2019/04/10/technology/amazon-climate-change-letter.html}{Amazon},
Google and Microsoft have also gotten
\href{https://www.greenpeace.org/usa/reports/oil-in-the-cloud/}{attention}
for selling technology to help the oil and gas industry extract fossil
fuels, which are a major source of greenhouse gas emissions. Google
\href{https://www.cnbc.com/2020/05/20/google-ai-greenpeace-oil-gas.html}{promised
to stop}.

Other areas to watch: Can Apple, Amazon and Google compel manufacturers
of their devices to reduce factory emissions and switch to cleaner
energy? And can they reuse and recycle the materials inside of devices?
In general, recycled materials are better for the environment.

Then there's the question of how much internet companies like Facebook
are helping spread
\href{https://heated.world/p/fact-check-of-viral-climate-misinformation}{disinformation
on climate science}.

\textbf{Is it effective for}
\textbf{\href{https://www.nytimes3xbfgragh.onion/2020/01/23/business/corporate-climate-davos.html}{companies
to pick their own paths}} \textbf{on climate change? What about
governments?}

As a former technology reporter, this moment reminds me of when big U.S.
tech companies didn't want
\href{https://www.nytimes3xbfgragh.onion/2013/10/31/technology/no-us-action-so-states-move-on-privacy-law.html}{regulations}
on data privacy. They changed their privacy policies and promised to do
better.

It's possible that big tech companies are again setting themselves
voluntary targets to forestall national legislation, like on emissions
standards. Both
\href{https://www.gov.uk/government/news/uk-becomes-first-major-economy-to-pass-net-zero-emissions-law}{Britain}
and the
\href{https://www.wsj.com/articles/eu-to-cut-greenhouse-gas-emissions-to-zero-by-2050-11576203017}{European
Union} now require their countries to achieve net zero emissions by
2050. That's bound to affect tech and every other industry.

\textbf{What can we do as consumers of technology?}

We can
\href{https://www.nytimes3xbfgragh.onion/guides/year-of-living-better/how-to-reduce-your-carbon-footprint}{educate
ourselves} on what goes into the technology we buy, what the climate
impacts are and how long a product might last.

We can also
\href{https://www.nytimes3xbfgragh.onion/2020/07/08/technology/personaltech/tech-that-lasts.html}{think
about what we buy} in the first place. Making shiny new things
contributes to global warming. So does shipping, delivering and
returning stuff. We can help by making our existing products or devices
\href{https://www.nytimes3xbfgragh.onion/2020/07/01/technology/personaltech/make-your-tech-last-longer.html}{last
longer} by replacing the battery or making a repair, or
\href{https://www.nytimes3xbfgragh.onion/2016/04/28/technology/personaltech/taking-the-stigma-out-of-buying-usedelectronics.html}{buying
used}.

\emph{If you don't already get this newsletter in your inbox,}
\href{https://www.nytimes3xbfgragh.onion/newsletters/signup/OT}{\emph{please
sign up here}}\emph{.}

\begin{center}\rule{0.5\linewidth}{\linethickness}\end{center}

\hypertarget{three-ways-to-enjoy-scary-technology}{%
\subsection{Three ways to enjoy scary
technology}\label{three-ways-to-enjoy-scary-technology}}

\emph{Instead of}
\href{https://www.nytimes3xbfgragh.onion/2020/07/16/technology/coronavirus-doomscrolling.html}{\emph{doomscrolling}}
\emph{today, how about flopping on the sofa to take in great
entertainment about \ldots{} uh \ldots{} nightmarish technology?}

\emph{Margot Harrison, a fiction writer and editor at the Vermont
newspaper Seven Days, offered us three recommendations for works dealing
with malevolent technology. Her latest novel,
``}\href{https://www.lbyr.com/titles/margot-harrison/the-glare/9781368005654/}{\emph{The
Glare}}\emph{,'' was released this month. Also check out}
\href{https://www.nytimes3xbfgragh.onion/2020/07/18/books/review/cyber-horror-virtual-life-uncanny-valley.html}{\emph{her
recent essay}} \emph{in The New York Times.}

\textbf{\href{https://us.macmillan.com/books/9780374175412}{``Infinite
Detail: A Novel''}} \textbf{by Tim Maughan}

In this 2019 dystopian novel, the only thing scarier than the
all-pervasive presence of the internet is its abrupt disappearance. The
story is told in alternating sections labeled ``Before'' and ``After.''
In the former, anarchist hackers unravel the web that holds us all; in
the latter, they deal with the consequences of succeeding beyond their
wildest dreams.

While depicting many all-too-plausible extensions of control and
surveillance technology, Maughan suggests that it's impossible to take a
simple stand for or against the machines with which our ways of life are
already fused.

\textbf{\href{https://www.candlewick.com/cat.asp?mode=book\&isbn=0763662623}{``Feed''}}
\textbf{by M.T. Anderson}

This was the book that convinced me that young adult fiction might be
especially open to exploring technological anxieties because teens have
never known a world offline.

Anderson envisions a future in which everyone has an implant feeding
them entertainment, social interactions and micro-targeted advertising.
The concept isn't new, but Anderson's narrator has an unforgettable
voice: Holden Caulfield with a near-lethal injection of swaggering
early-aughts MTV.

\textbf{The}
\textbf{\href{https://www.nytimes3xbfgragh.onion/2016/10/21/arts/television/review-black-mirror-finds-terror-and-soul-in-the-machine.html}{``Nosedive''
episode}} \textbf{of ``Black Mirror''}

No piece of fiction has channeled my personal anxieties about social
media quite as effectively as this.

In a near future in which people's status and livelihood depend directly
on the ratings others give them, a young woman makes a fatal series of
small mistakes that zero out her social credit. It's a nightmare that
might convince you to put down the phone.

\begin{center}\rule{0.5\linewidth}{\linethickness}\end{center}

\hypertarget{before-we-go-}{%
\subsection{Before we go \ldots{}}\label{before-we-go-}}

\begin{itemize}
\item
  \textbf{The cyberattack deterrence isn't working:} To fight
  cyberattacks from China and Russia, the U.S. government for years has
  tried to name, shame and indict those behind them, and sometimes even
  counterattacked. But those punishments
  \href{https://www.nytimes3xbfgragh.onion/2020/07/22/us/politics/china-russia-trump-cyberattacks.html}{haven't
  been sufficient to deter continued cyberattacks and disinformation
  operations}, reported David E. Sanger, the Times national security
  correspondent.
\item
  \textbf{You can't pry this phone out of my hand:} Just about every
  tech company in the world considers India the emerging internet gold
  rush, but the companies are finding one big barrier: Many millions of
  Indians opt for basic cellphones over smartphones. This makes life
  harder for Netflix, Facebook and WeChat. The Chinese tech publication
  Abacus looks at why
  \href{https://www.scmp.com/abacus/tech/article/3093597/can-google-and-jio-convert-indias-feature-phone-diehards-android?utm_medium=email\&utm_source=mailchimp\&utm_campaign=enlz-abacusnews_chatbox\&utm_content=20200722\&MCUID=91b7d58ca3\&MCCampaignID=88e0c9f17d\&MCAccountID=3775521f5f542047246d9c827\&tc=18}{the
  basic cellphone in India is far more appealing} than that Nokia you
  had in the early 2000s.
\item
  \textbf{Our national cake obsession didn't last long:} Five minutes
  ago, it was impossible to avoid surreal social media videos of
  \href{https://www.nytimes3xbfgragh.onion/2020/07/14/style/what-is-the-cake-meme.html}{cakes
  disguised as Crocs, pickles or human heads.} Now the craze is dying,
  \href{https://www.nbcnews.com/pop-culture/pop-culture-news/internet-went-crazy-over-cake-going-mainstream-can-end-trend-n1234588}{NBC
  News reported}. Like any fun thing, weird cake was ruined because The
  Olds got into it. (I am An Old as well. I swear. It's fine.)
\end{itemize}

\hypertarget{hugs-to-this}{%
\subsubsection{Hugs to this}\label{hugs-to-this}}

Please enjoy very good dog
\href{https://twitter.com/BlairBraverman/status/1286085108005507077}{Spike
romping in a meadow}. (And if you don't already, follow the dog sledder
and author \href{https://twitter.com/BlairBraverman}{Blair Braverman on
Twitter} for lots of very good dogs.)

\begin{center}\rule{0.5\linewidth}{\linethickness}\end{center}

\emph{We want to hear from you. Tell us what you think of this
newsletter and what else you'd like us to explore. You can reach us at}
\href{mailto:ontech@NYTimes.com?subject=On\%20Tech\%20Feedback}{\emph{ontech@NYTimes.com.}}
**

\emph{If you don't already get this newsletter in your inbox,}
\href{https://www.nytimes3xbfgragh.onion/newsletters/signup/OT}{\emph{please
sign up here}}\emph{.}

Advertisement

\protect\hyperlink{after-bottom}{Continue reading the main story}

\hypertarget{site-index}{%
\subsection{Site Index}\label{site-index}}

\hypertarget{site-information-navigation}{%
\subsection{Site Information
Navigation}\label{site-information-navigation}}

\begin{itemize}
\tightlist
\item
  \href{https://help.nytimes3xbfgragh.onion/hc/en-us/articles/115014792127-Copyright-notice}{©~2020~The
  New York Times Company}
\end{itemize}

\begin{itemize}
\tightlist
\item
  \href{https://www.nytco.com/}{NYTCo}
\item
  \href{https://help.nytimes3xbfgragh.onion/hc/en-us/articles/115015385887-Contact-Us}{Contact
  Us}
\item
  \href{https://www.nytco.com/careers/}{Work with us}
\item
  \href{https://nytmediakit.com/}{Advertise}
\item
  \href{http://www.tbrandstudio.com/}{T Brand Studio}
\item
  \href{https://www.nytimes3xbfgragh.onion/privacy/cookie-policy\#how-do-i-manage-trackers}{Your
  Ad Choices}
\item
  \href{https://www.nytimes3xbfgragh.onion/privacy}{Privacy}
\item
  \href{https://help.nytimes3xbfgragh.onion/hc/en-us/articles/115014893428-Terms-of-service}{Terms
  of Service}
\item
  \href{https://help.nytimes3xbfgragh.onion/hc/en-us/articles/115014893968-Terms-of-sale}{Terms
  of Sale}
\item
  \href{https://spiderbites.nytimes3xbfgragh.onion}{Site Map}
\item
  \href{https://help.nytimes3xbfgragh.onion/hc/en-us}{Help}
\item
  \href{https://www.nytimes3xbfgragh.onion/subscription?campaignId=37WXW}{Subscriptions}
\end{itemize}
