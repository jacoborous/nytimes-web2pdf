Sections

SEARCH

\protect\hyperlink{site-content}{Skip to
content}\protect\hyperlink{site-index}{Skip to site index}

\href{https://www.nytimes3xbfgragh.onion/section/food}{Food}

\href{https://myaccount.nytimes3xbfgragh.onion/auth/login?response_type=cookie\&client_id=vi}{}

\href{https://www.nytimes3xbfgragh.onion/section/todayspaper}{Today's
Paper}

\href{/section/food}{Food}\textbar{}A Brooklyn Restaurant's Answer to
Cabin Fever: Summer Camp

\href{https://nyti.ms/39oN5Be}{https://nyti.ms/39oN5Be}

\begin{itemize}
\item
\item
\item
\item
\item
\item
\end{itemize}

\href{https://www.nytimes3xbfgragh.onion/spotlight/at-home?action=click\&pgtype=Article\&state=default\&region=TOP_BANNER\&context=at_home_menu}{At
Home}

\begin{itemize}
\tightlist
\item
  \href{https://www.nytimes3xbfgragh.onion/2020/07/28/books/time-for-a-literary-road-trip.html?action=click\&pgtype=Article\&state=default\&region=TOP_BANNER\&context=at_home_menu}{Take:
  A Literary Road Trip}
\item
  \href{https://www.nytimes3xbfgragh.onion/2020/07/29/magazine/bored-with-your-home-cooking-some-smoky-eggplant-will-fix-that.html?action=click\&pgtype=Article\&state=default\&region=TOP_BANNER\&context=at_home_menu}{Cook:
  Smoky Eggplant}
\item
  \href{https://www.nytimes3xbfgragh.onion/2020/07/27/travel/moose-michigan-isle-royale.html?action=click\&pgtype=Article\&state=default\&region=TOP_BANNER\&context=at_home_menu}{Look
  Out: For Moose}
\item
  \href{https://www.nytimes3xbfgragh.onion/interactive/2020/at-home/even-more-reporters-editors-diaries-lists-recommendations.html?action=click\&pgtype=Article\&state=default\&region=TOP_BANNER\&context=at_home_menu}{Explore:
  Reporters' Obsessions}
\end{itemize}

Advertisement

\protect\hyperlink{after-top}{Continue reading the main story}

Supported by

\protect\hyperlink{after-sponsor}{Continue reading the main story}

Critic's Notebook

\hypertarget{a-brooklyn-restaurants-answer-to-cabin-fever-summer-camp}{%
\section{A Brooklyn Restaurant's Answer to Cabin Fever: Summer
Camp}\label{a-brooklyn-restaurants-answer-to-cabin-fever-summer-camp}}

Olmsted, like many of its New York City peers, is trying to make the
most of a strange season by serving up fun and games along with the
distancing.

\includegraphics{https://static01.graylady3jvrrxbe.onion/images/2020/07/29/dining/23Dinecamp1/merlin_174744861_fb9eb52c-8f51-47ca-95c3-f19ce18a88d2-articleLarge.jpg?quality=75\&auto=webp\&disable=upscale}

By \href{https://www.nytimes3xbfgragh.onion/by/pete-wells}{Pete Wells}

\begin{itemize}
\item
  July 23, 2020
\item
  \begin{itemize}
  \item
  \item
  \item
  \item
  \item
  \item
  \end{itemize}
\end{itemize}

The other night a watermelon was brought to my table in the backyard of
\href{https://www.nytimes3xbfgragh.onion/2016/08/10/dining/olmsted-restaurant-review.html}{Olmsted},
in Brooklyn. The first thing I noticed was its unusually compact size,
slightly smaller than my teenage son's head. The more unusual thing,
though, was the copper spigot sticking out of its midsection. When I
turned the spigot, out trickled a stream of cold, pink watermelon punch.

Even before the alcohol --- both aquavit and clairin, the clear,
small-batch rum from Haiti, swam around in watermelon juice seasoned by
lemongrass and a subliminal amount of fish sauce --- had a chance to
work its way to my chafed nerves, I was already glad I'd left the house.

Yes, it's impossible to eat outdoors at a restaurant --- and outdoors is
the only way we can eat at New York City restaurants at the moment ---
without some degree of worry. Many people won't do it, whether or not
the governor and mayor say it's safe. But for those who risk it, there's
a decent chance you'll encounter a watermelon samovar, or something like
it.

Since getting clearance to
\href{https://www.nytimes3xbfgragh.onion/2020/06/23/dining/outdoor-restaurants-nyc-coronavirus.html}{begin
serving at outdoor tables} in late June, the city's restaurants have
been in a manic phase. They are almost as desperate to prove that life
is
\href{https://www.nytimes3xbfgragh.onion/2020/07/09/dining/outdoor-dining-design-nyc-coronavirus.html}{one
big alfresco party} as New Yorkers are to believe it.

Summer is typically a slow season in the city. With a return to indoor
dining nowhere in sight, though, owners have concluded that the summer
may well be the only season they're going to get. So they've tried to
make eating out seem like a combination of a tropical vacation and a
family picnic at which a few members of the family have been drinking
steadily since breakfast.

Image

Punch flows from watermelons.Credit...Jenny Huang for The New York Times

Image

Reservations have been hard to come by.Credit...Jenny Huang for The New
York Times

Olmsted, in the Prospect Heights neighborhood, has gone so far as to
name its backyard incarnation \href{http://www.olmstednyc.com/}{Olmsted
Summer Camp}. A few picnic tables have joined the regular outdoor
furniture, which is widely spaced around a container garden where
tomatoes and peppers are starting to ripen. (Raised planter beds turn
out to be ideal for social distancing.)

To keep the campers entertained, the restaurant has been programming
music nights, movie nights, trivia nights, magic nights and comedy
nights. There was nothing going on when I was there, which is a good
thing, because nothing would send me back to self-quarantine faster than
having to eat during a stand-up comic's routine.

The host had seated me near several shelves of games. She must have
guessed, correctly, that I was one of those campers who stayed in the
tent when everybody else was out running the obstacle course. According
to an email Olmsted sent July 1, the night before Summer Camp began, all
the checkers, Jenga pieces, playing cards and Connect 4 sets are
``sanitized after use!''

This is reassuring, but only up to a point. From what we know about the
coronavirus, it is more likely to float through the air than cling to
surfaces when it jumps from person to person. And the longer you stay in
one place, the higher your risk of infection. While scientists believe
that mingling with other people is
\href{https://www.nytimes3xbfgragh.onion/2020/05/15/dining/restaurant-opening-safety-coronavirus.html}{safer
outdoors than inside}, the prevailing wisdom is that you should still
try to keep your mingling brief. If you choose to go out for a meal,
it's probably wise not to settle in for a match of canasta after
dessert.

Not that anyone was playing games when I was there. They did all the
things people used to do at restaurants, taking selfies and arguing
about subjects you can't believe people really argue about. I was
unhappy to see that almost nobody except the staff wore masks. My own
protocol was to tie mine on any time I didn't have food in front of me.

That was the plan, at least. I got a bit absent-minded about it once the
watermelon punch had me in its clutches. This is not very surprising,
but it does make me skeptical about restaurants' finding a safe way to
serve indoors before Covid-19 is brought to heel.

\includegraphics{https://static01.graylady3jvrrxbe.onion/images/2020/07/29/dining/23DineCamp3/merlin_174744936_ee24c174-6a10-4624-89f9-8498dd1ce740-articleLarge.jpg?quality=75\&auto=webp\&disable=upscale}

Greg Baxtrom, the chef and one of Olmsted's owners, stopped by the table
to talk for a few minutes. His mouth was hidden behind a stretchy
Spiderman fabric.

``I'm a huge Marvel Comics guy,'' he said. He had wanted to get Marvel
versions of all the games, he said, but balked when he saw how much more
they cost.

Mr. Baxtrom had ditched his regular menu except for the frozen yogurt
with lavender honey, a dessert that dares to ask the question: What if
the Dairy Queen made a state visit to Greece?

Before the pandemic, a typical Olmsted plate had at least half a dozen
different but complementary things going on. The summer camp dishes are
vastly simpler. Each is similar enough to the American summer-foods
canon to make you glad you're eating it, but dissimilar enough to keep
you from asking why you are eating it at a restaurant.

Image

Fried chicken comes with sweet pickles.Credit...Jenny Huang for The New
York Times

Image

Shiso and kombu dress a watermelon salad.Credit...Jenny Huang for The
New York Times

My meal started with a salad, sort of: sliced watermelon, on the rind,
under ribbons of kombu and purple shiso. I could eat it every night
until the end of September without complaining. There was fried chicken,
which the skeleton crew in the kitchen endows with a resounding crunch,
and a kebab of pineapple, spring onions and New York strip that had a
neo-tiki swivel in its hips.

I liked both, but they were kicked to the back seat when the smoked
spareribs arrived, gorgeous bands of meat under a barbecue sauce
inspired by the fruity, salty, savory paste that comes with tonkatsu in
Japanese restaurants.

Image

Tonkatsu sauce glazes the smoked ribs.Credit...Jenny Huang for The New
York Times

Like many restaurateurs, Mr. Baxtrom has kept afloat since March by
trying a little of this and a little of that. With a grant from
\href{https://leeinitiative.org/}{the Lee Initiative}, he and a small
crew cooked and gave away meals and groceries to out-of-work restaurant
employees and others. They also sell groceries at Olmsted Trading Post,
which used to be Olmsted's private dining room. There are fresh pretzel
rolls, vegetables from Mr. Baxtrom's preferred farms and condiments made
on site, including a ranch so good it made me understand why the first
bottled salad dressings had been such a big deal.

Image

Olmsted set up a store that sells baked goods and summer
produce.Credit...Jenny Huang for The New York Times

Across Vanderbilt Avenue, Mr. Baxtrom has a second restaurant, Maison
Yaki, that is too small to operate safely now. He has turned it into
\href{https://resy.com/cities/ny/maison-yaki?date=2020-07-21\&seats=2}{a
showcase} where Black food entrepreneurs will set up shop for two weeks
at a time until the end of the year or longer. This is how I joined the
very large fan base for the salt-flecked vegan miso-chocolate chip
cookies baked by \href{https://www.lanihalliday.com/}{Lani Halliday},
whose business is called Brutus Bakeshop.

I enjoyed my night among the tomato plants and watermelon samovars. It
may be an accident of timing, but there's something joyful about
returning to restaurants while summer is beating down around us --- it
demands a relaxed style of cooking and eating that happens to be the
only style that seems to make any sense right now. Will it make sense
for the restaurant business in the long run?

Image

The baker Lani Halliday ran a pop-up shop at Olmsted's sister
restaurant, Maison Yaki.Credit...Jenny Huang for The New York Times

Image

Ms. Halliday's miso chocolate-chip cooks have a devoted
following.Credit...Jenny Huang for The New York Times

In a phone conversation the next day, Mr. Baxtrom said reservation and
walk-in traffic for summer camp had been strong. Olmsted now has more
seats outside than it did inside.

But tables go empty each time a thunderstorm rolls in. Because the food
is more casual and apt to be shared, the average check is lower than it
was last year. Still, Mr. Baxtrom plans to keep feeding backyard campers
until the fall.

``We're just trying to make it through this year,'' he said.

The chefs he knows talk to each other about a coronavirus vaccine, and
about how that would allow them to run their businesses the way they
used to. They are fixated on surviving until April, when they seem to
believe a vaccine will become available.

``Spring 2021,'' Mr. Baxtrom said. ``Just get there without losing
everything. That's the mind-set that I and a lot of others have now.
Just get to the vaccine.''

Olmsted Summer Camp, Wednesdays through Sundays at 659 Vanderbilt Avenue
(Park Place), Prospect Heights, Brooklyn; 718-552-2610;
\href{http://olmstednyc.com/}{olmstednyc.com}.

\emph{Follow} \href{https://twitter.com/nytfood}{\emph{NYT Food on
Twitter}} \emph{and}
\href{https://www.instagram.com/nytcooking/}{\emph{NYT Cooking on
Instagram}}\emph{,}
\href{https://www.facebookcorewwwi.onion/nytcooking/}{\emph{Facebook}}\emph{,}
\href{https://www.youtube.com/nytcooking}{\emph{YouTube}} \emph{and}
\href{https://www.pinterest.com/nytcooking/}{\emph{Pinterest}}\emph{.}
\href{https://www.nytimes3xbfgragh.onion/newsletters/cooking}{\emph{Get
regular updates from NYT Cooking, with recipe suggestions, cooking tips
and shopping advice}}\emph{.}

Advertisement

\protect\hyperlink{after-bottom}{Continue reading the main story}

\hypertarget{site-index}{%
\subsection{Site Index}\label{site-index}}

\hypertarget{site-information-navigation}{%
\subsection{Site Information
Navigation}\label{site-information-navigation}}

\begin{itemize}
\tightlist
\item
  \href{https://help.nytimes3xbfgragh.onion/hc/en-us/articles/115014792127-Copyright-notice}{©~2020~The
  New York Times Company}
\end{itemize}

\begin{itemize}
\tightlist
\item
  \href{https://www.nytco.com/}{NYTCo}
\item
  \href{https://help.nytimes3xbfgragh.onion/hc/en-us/articles/115015385887-Contact-Us}{Contact
  Us}
\item
  \href{https://www.nytco.com/careers/}{Work with us}
\item
  \href{https://nytmediakit.com/}{Advertise}
\item
  \href{http://www.tbrandstudio.com/}{T Brand Studio}
\item
  \href{https://www.nytimes3xbfgragh.onion/privacy/cookie-policy\#how-do-i-manage-trackers}{Your
  Ad Choices}
\item
  \href{https://www.nytimes3xbfgragh.onion/privacy}{Privacy}
\item
  \href{https://help.nytimes3xbfgragh.onion/hc/en-us/articles/115014893428-Terms-of-service}{Terms
  of Service}
\item
  \href{https://help.nytimes3xbfgragh.onion/hc/en-us/articles/115014893968-Terms-of-sale}{Terms
  of Sale}
\item
  \href{https://spiderbites.nytimes3xbfgragh.onion}{Site Map}
\item
  \href{https://help.nytimes3xbfgragh.onion/hc/en-us}{Help}
\item
  \href{https://www.nytimes3xbfgragh.onion/subscription?campaignId=37WXW}{Subscriptions}
\end{itemize}
