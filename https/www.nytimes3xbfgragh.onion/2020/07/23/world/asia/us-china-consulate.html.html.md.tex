Sections

SEARCH

\protect\hyperlink{site-content}{Skip to
content}\protect\hyperlink{site-index}{Skip to site index}

\href{https://www.nytimes3xbfgragh.onion/section/world/asia}{Asia
Pacific}

\href{https://myaccount.nytimes3xbfgragh.onion/auth/login?response_type=cookie\&client_id=vi}{}

\href{https://www.nytimes3xbfgragh.onion/section/todayspaper}{Today's
Paper}

\href{/section/world/asia}{Asia Pacific}\textbar{}Beijing's Tightrope:
Stand Tough, but Avoid a Full Rupture With the U.S.

\begin{itemize}
\item
\item
\item
\item
\item
\end{itemize}

Advertisement

\protect\hyperlink{after-top}{Continue reading the main story}

Supported by

\protect\hyperlink{after-sponsor}{Continue reading the main story}

\hypertarget{beijings-tightrope-stand-tough-but-avoid-a-full-rupture-with-the-us}{%
\section{Beijing's Tightrope: Stand Tough, but Avoid a Full Rupture With
the
U.S.}\label{beijings-tightrope-stand-tough-but-avoid-a-full-rupture-with-the-us}}

After the closure of the Houston consulate and other actions, Chinese
officials are trying to respond without endangering the economy or
inviting further global isolation.

\includegraphics{https://static01.graylady3jvrrxbe.onion/images/2020/07/23/world/23china-retaliate01sub/merlin_172914465_1e785506-ed96-4d17-b243-5d45bbbde532-articleLarge.jpg?quality=75\&auto=webp\&disable=upscale}

\href{https://www.nytimes3xbfgragh.onion/by/keith-bradsher}{\includegraphics{https://static01.graylady3jvrrxbe.onion/images/2018/10/08/multimedia/author-keith-bradsher/author-keith-bradsher-thumbLarge.png}}\href{https://www.nytimes3xbfgragh.onion/by/steven-lee-myers}{\includegraphics{https://static01.graylady3jvrrxbe.onion/images/2018/10/15/multimedia/author-steven-lee-myers/author-steven-lee-myers-thumbLarge.png}}

By \href{https://www.nytimes3xbfgragh.onion/by/keith-bradsher}{Keith
Bradsher} and
\href{https://www.nytimes3xbfgragh.onion/by/steven-lee-myers}{Steven Lee
Myers}

\begin{itemize}
\item
  Published July 23, 2020Updated July 24, 2020
\item
  \begin{itemize}
  \item
  \item
  \item
  \item
  \item
  \end{itemize}
\end{itemize}

\href{https://cn.nytimes3xbfgragh.onion/china/20200724/us-china-consulate/}{阅读简体中文版}\href{https://cn.nytimes3xbfgragh.onion/china/20200724/us-china-consulate/zh-hant/}{閱讀繁體中文版}

BEIJING --- Two weeks ago, China's foreign minister, Wang Yi, pleaded
with the United States to step back
\href{https://www.nytimes3xbfgragh.onion/2020/07/14/world/asia/cold-war-china-us.html}{from
the brink} and find ways to work together. Just days later, he
complained to his Russian counterpart that the United States had ``lost
its mind, morals and credibility.''

The question now is what
\href{https://www.nytimes3xbfgragh.onion/2020/07/24/world/asia/china-us-consulate-chengdu.html}{China
can do about it}. The Trump administration's broad assault on China has
left its leadership with few options that would not threaten a complete
breach in relations. If that happened, it could leave Beijing even more
isolated at a time when China is also
\href{https://www.nytimes3xbfgragh.onion/2020/06/17/world/asia/india-china-border-clashes.html}{clashing
with India}, Britain, Canada, Australia and many other countries. It
could also hurt the Chinese economy when it is already reeling from the
coronavirus pandemic and the global fallout.

The order on Tuesday to close the Chinese Consulate in Houston with only
72 hours' notice was only the latest action by the administration that
has infuriated officials in China. In a matter of weeks, Beijing has
endured a stepped-up campaign against its 5G wireless technology,
sanctions against officials overseeing Hong Kong and the largely Muslim
region of Xinjiang, and now accusations that China has dispatched scores
of soldiers under cover to steal commercial, military and even medical
secrets.

A spokesman for China's foreign ministry vowed again on Thursday that
the government would retaliate in kind to the closure. He dismissed the
administration's accusations as ``a malicious smearing.''

On Friday, Beijing hit back, telling the United States it
must\href{https://www.nytimes3xbfgragh.onion/2020/07/24/world/asia/china-us-consulate-chengdu.html}{shut
down its consulate in Chengdu,} the westernmost of the five American
consulates in mainland China.

The furor is inflaming anti-American sentiment in China and emboldening
more hawkish voices. Nationalists are calling for China to go further
than a measured tit-for-tat response and even consider shutting down the
American Consulate in Hong Kong.

``Let them sweat,'' Hu Xijin, the editor of The Global Times, a
nationalist Communist Party newspaper, wrote of American diplomats in
the embassy and six consulates. He said the mission in Hong Kong was
``obviously'' an intelligence center, while vastly exaggerating the size
of the staff.

He then referred to the frantic reaction to the closure at the Houston
consulate, where people could be seen in a courtyard burning documents
in metal bins. ``Have each of their consulates make an emergency plan,
pack up all the files and prepare to burn.''

\includegraphics{https://static01.graylady3jvrrxbe.onion/images/2020/07/23/world/23china-retaliate05/merlin_174830739_070a2e47-7235-4e82-89f5-90862536586d-articleLarge.jpg?quality=75\&auto=webp\&disable=upscale}

Behind the scenes, senior Chinese officials seem to have little desire
to escalate the tensions even further, concerned that any moves could
play into President Trump's hands as he mounts his re-election campaign.
A highly visible showdown with China could distract Americans from Mr.
Trump's botched response to the pandemic and allow him to campaign as a
leader who is defending his country against a foreign power.

``This is a classic game, to find an external distraction and rouse the
people behind the president,'' said Lau Siu-kai, a senior Beijing
adviser on Hong Kong issues.

At the same time, Beijing cannot afford to appear weak in the face of
the barrage of attacks from the United States. A rising sense of
national pride, instilled by the country's schools and amplified by
state media, demands that Chinese leaders stand strong when challenged
from abroad.

``China needs to protect its own honor and sovereignty,'' said Shen
Dingli, a professor of international relations at Fudan University in
Shanghai.

Wang Wenbin, the foreign ministry's spokesman, made clear at the
ministry's daily briefing on Thursday that Chinese officials were
acutely aware of their dilemma.

``We are not interested in interfering in the U.S. election; we also
hope that the U.S. side will not play the China card in the election,''
he said, while immediately cautioning the Trump administration as well.
``We advise the U.S. side not to make mistakes again and again,
otherwise China will certainly make a legitimate and necessary
response.''

Rising tensions with Washington have laid bare divisions in Beijing over
how to respond to a confrontation that has become even broader and more
aggressive than Chinese officials expected only weeks ago.

Secretary of State Mike Pompeo, delivering a speech Thursday at the
Richard Nixon Presidential Library in California, said the Trump
administration would persist in challenging China around the globe. He
did not mention any issues of common interest or grounds for
constructive engagement.

``General Secretary Xi Jinping is a true believer in a bankrupt
totalitarian ideology,'' Mr. Pompeo said, referring to the Chinese
leader. He added, ``I call on every leader of every nation to start by
doing what America has done --- to simply insist on reciprocity, to
insist on transparency, and on accountability from the Chinese Communist
Party.''

On one side are officials in the country's security services and the
military who oppose any conciliatory stance that might be interpreted by
the United States as weakness, according to several people involved in
Chinese policymaking who spoke on the condition of anonymity, given the
diplomatic sensitivities. Other officials, generally those focused on
the economy, have sought a more measured response to the American
actions --- keeping the trade truce intact, for example.

Even after the closing of the Houston consulate, China remains committed
to the so-called Phase 1 trade agreement with the United States signed
on Jan. 15, the people familiar with Chinese policymaking said.

If China wanted to hurt Mr. Trump in the election campaign, Beijing
could halt the large purchases of American food that it agreed to make
as part of the trade truce negotiated before the pandemic broke out.
That would penalize American farmers, who could prove an important
voting bloc in some states.

So far, China has kept buying large quantities of American corn, wheat,
sorghum and pork this summer, said Darin Friedrichs, an agricultural
commodities specialist in the Shanghai office of INTL FCStone, a large
Chicago trading firm. Less than two weeks ago, China made its largest
single purchase ever of American corn, just four days after another
major transaction.

Image

The United States and China signed a limited trade agreement in January.
China is said to remain committed to the pact.Credit...Pete Marovich for
The New York Times

China's leader, Xi Jinping, remains the ultimate arbiter of the policy
debate in Beijing, but has not spoken out on the deterioration in
relations. On Wednesday, when the closing of the consulate became
public, he toured distant Jilin Province, seemingly undisturbed by the
diplomatic confrontation. On Thursday, he visited the Air Force Aviation
University, talking instead about China's national celebration of its
military in August.

``Beijing's policy has always been adjusted by Xi himself,'' said Wu
Qiang, an independent analyst in Beijing. ``He steps on the gas himself
and then hits the brake himself.''

The Chinese appear to have been taken aback by the sharp deterioration
in relations. In a speech on July 9, the foreign minister, Wang Yi,
appeared to outline a path for stabilizing relations.

``President Xi Jinping has underlined on many occasions that we have a
thousand reasons to make the China-U.S. relationship a success, and none
whatsoever to wreck it,'' he said. ``As long as both sides have the
positive will to improve and grow this relationship, we will find ways
to steer this relationship out of the difficulties and bring it back to
the right track.''

Instead, the Chinese faced confrontation on a multitude of new fronts.
In the latest salvo over the consulate, the Trump administration accused
Chinese diplomats of aiding economic espionage and the attempted theft
of scientific research in numerous cases across the United States.

Chinese officials angrily denounced the closure of the consulate,
calling it a provocation that would further undermine already soured
relations. Cai Wei, China's top diplomat in Houston, said the move
against the consulate, the first Beijing opened in the United States
after re-establishing ties in 1979, was ``very damaging.''

In previous tense moments, the two leaders, Mr. Trump and Mr. Xi,
sometimes smoothed over differences with a long phone call or a meeting.
It has happened in the past when trade fights boiled over, as well as
early in the coronavirus outbreak, when the rhetoric between both sides
intensified.

The tone now in Washington, though, has worsened. And Mr. Trump no
longer seems interested in defusing the crisis.

``Xi Jinping could take the initiative instead,'' said Susan L. Shirk,
the chairwoman of the 21st Century China Center at the University of
California, San Diego. ``Xi could also demonstrate China's benign
intentions by inviting the U.S. to join with it in leading an
international effort to plan now for the testing, manufacture and fair
distribution of Covid vaccine.''

The tough policies and tougher rhetoric from Washington indicate that
the United States, not China, is setting the ever more confrontational
tone in the bilateral relationship. ``I think originally you could have
faulted the Chinese for much of the imbalance,'' said Orville Schell,
the director of the Center on U.S.-China Relations at the Asia Society,
``but now the U.S. doesn't seem as ardent about leaving the door open
for remedy, as it is arching its back against China.''

Given the breadth of American actions and increasingly bipartisan
support, it is not clear that China can hope for an improvement even if
Mr. Trump's challenger, Joseph R. Biden Jr., wins the election.

Mr. Schell noted that as vice president, Mr. Biden met with Mr. Xi
frequently, even traveling together.

``There's a kind of symmetry there that he could use to recast the
relationship,'' he said. ``The real question is whether Xi can respond
the same way --- whether giving a little to get a little is seen as
weakness.''

``I do think that Biden and his people are perfectly capable of working
out a new balance point,'' he added. ``I have much less confidence that
China will find that easy to do.''

Keith Bradsher reported from Beijing, and Steven Lee Myers from Seoul,
South Korea. Rick Gladstone contributed reporting from Eastham, Mass.,
and Edward Wong from Washington. Claire Fu and Coral Yang contributed
research.

Advertisement

\protect\hyperlink{after-bottom}{Continue reading the main story}

\hypertarget{site-index}{%
\subsection{Site Index}\label{site-index}}

\hypertarget{site-information-navigation}{%
\subsection{Site Information
Navigation}\label{site-information-navigation}}

\begin{itemize}
\tightlist
\item
  \href{https://help.nytimes3xbfgragh.onion/hc/en-us/articles/115014792127-Copyright-notice}{©~2020~The
  New York Times Company}
\end{itemize}

\begin{itemize}
\tightlist
\item
  \href{https://www.nytco.com/}{NYTCo}
\item
  \href{https://help.nytimes3xbfgragh.onion/hc/en-us/articles/115015385887-Contact-Us}{Contact
  Us}
\item
  \href{https://www.nytco.com/careers/}{Work with us}
\item
  \href{https://nytmediakit.com/}{Advertise}
\item
  \href{http://www.tbrandstudio.com/}{T Brand Studio}
\item
  \href{https://www.nytimes3xbfgragh.onion/privacy/cookie-policy\#how-do-i-manage-trackers}{Your
  Ad Choices}
\item
  \href{https://www.nytimes3xbfgragh.onion/privacy}{Privacy}
\item
  \href{https://help.nytimes3xbfgragh.onion/hc/en-us/articles/115014893428-Terms-of-service}{Terms
  of Service}
\item
  \href{https://help.nytimes3xbfgragh.onion/hc/en-us/articles/115014893968-Terms-of-sale}{Terms
  of Sale}
\item
  \href{https://spiderbites.nytimes3xbfgragh.onion}{Site Map}
\item
  \href{https://help.nytimes3xbfgragh.onion/hc/en-us}{Help}
\item
  \href{https://www.nytimes3xbfgragh.onion/subscription?campaignId=37WXW}{Subscriptions}
\end{itemize}
