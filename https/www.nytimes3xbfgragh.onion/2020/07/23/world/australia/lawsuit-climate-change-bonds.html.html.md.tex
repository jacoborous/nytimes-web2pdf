Sections

SEARCH

\protect\hyperlink{site-content}{Skip to
content}\protect\hyperlink{site-index}{Skip to site index}

\href{https://www.nytimes3xbfgragh.onion/section/world/australia}{Australia}

\href{https://myaccount.nytimes3xbfgragh.onion/auth/login?response_type=cookie\&client_id=vi}{}

\href{https://www.nytimes3xbfgragh.onion/section/todayspaper}{Today's
Paper}

\href{/section/world/australia}{Australia}\textbar{}Australian Student
Sues Government Over Financial Risks of Climate Change

\url{https://nyti.ms/2WND2R8}

\begin{itemize}
\item
\item
\item
\item
\item
\end{itemize}

Advertisement

\protect\hyperlink{after-top}{Continue reading the main story}

Supported by

\protect\hyperlink{after-sponsor}{Continue reading the main story}

\hypertarget{australian-student-sues-government-over-financial-risks-of-climate-change}{%
\section{Australian Student Sues Government Over Financial Risks of
Climate
Change}\label{australian-student-sues-government-over-financial-risks-of-climate-change}}

A 23-year-old law student filed a class-action suit accusing Australia
of failing to disclose financial risks from climate change. Experts say
it is the first of its kind.

\includegraphics{https://static01.graylady3jvrrxbe.onion/images/2020/07/23/world/23oz-climate-suit1/merlin_168211200_75f1f46d-43fb-49cc-9ece-3b02acf08bf5-articleLarge.jpg?quality=75\&auto=webp\&disable=upscale}

\href{https://www.nytimes3xbfgragh.onion/by/isabella-kwai}{\includegraphics{https://static01.graylady3jvrrxbe.onion/images/2019/09/17/reader-center/author-isabella-kwai/author-isabella-kwai-thumbLarge.png}}

By \href{https://www.nytimes3xbfgragh.onion/by/isabella-kwai}{Isabella
Kwai}

\begin{itemize}
\item
  July 23, 2020
\item
  \begin{itemize}
  \item
  \item
  \item
  \item
  \item
  \end{itemize}
\end{itemize}

SYDNEY, Australia --- Katta O'Donnell grew up with a fear of fire. As a
child, she remembers burning bark falling from the air because of
wildfires. This year, she worried that
\href{https://www.nytimes3xbfgragh.onion/2020/02/15/world/australia/fires-climate-change.html?searchResultPosition=2}{the
blazes sweeping across regional Australia,} fueled by climate change,
could destroy her home outside Melbourne, the same way they had turned
thousands of acres into ash.

Now, Ms. O'Donnell, 23, is leading a class-action lawsuit filed on
Wednesday that accuses the Australian government of failing to disclose
the material risks of climate change to those investing in government
bonds. The suit accuses the government and the treasury of breaching its
duty by not disclosing the risks of global warming and their material
impact on investors.

It is the first time, experts say, that such a climate change case has
been brought against a sovereign nation.

Ms. O'Donnell is joining
\href{https://www.nytimes3xbfgragh.onion/interactive/2020/07/21/magazine/teenage-activist-climate-change.html}{a
wave of young climate activists} who have stepped on to the world stage
in recent years. The Swedish teenager Greta Thunberg, for example, has
spurred a global protest movement, testified before the United States
Congress and
\href{https://www.nytimes3xbfgragh.onion/2020/03/04/world/europe/eu-climate-law-greta-thunberg.html?searchResultPosition=4}{the
European Parliament}, scolded world leaders in
\href{https://www.nytimes3xbfgragh.onion/2020/04/22/climate/earth-day-climate-coronavirus.html?searchResultPosition=2}{a
fiery speech at the United Nations}for not doing enough and sounded that
alarm at the World Economic Forum in Davos, declaring,
\href{https://www.nytimes3xbfgragh.onion/2020/01/21/climate/greta-thunberg-davos.html?searchResultPosition=9}{``Our
house is still on fire.''}

But Ms. O'Donnell's case takes a unique tack by focusing on government
bonds and the investment environment, said Jacqueline Peel, a law
professor at University of Melbourne.

``My personal experience with climate change makes everything I read
about climate change more tangible,'' Ms. O'Donnell, a fifth-year law
student at La Trobe University in Melbourne, said in a recent interview.
``I want my government acting with honesty and telling the truth about
climate risks.''

Simply put: Any risks to the country's economic growth, value of its
currency or international relations, to name a few factors, might change
the value of her investment, her suit states.

Ms. O'Donnell, backed by a team including two prominent lawyers, is not
asking for damages, but wants the government to step up on its climate
change policies. The suit seeks an injunction stopping the government
from further marketing bonds until they add those disclosures.

Image

Katta O'Donnell, who is leading a class-action lawsuit against the
Australian government.Credit...Molly Townsend

``The claim asks for disclosure of risks --- it doesn't tell the
government what to do or how to act,'' said David Barnden, one of three
lawyers representing Ms. O'Donnell. All took her case free, they said.

But experts say that the case's strategy is interesting given that the
government has the power to legislate on climate change and control, in
part, that risk.

The Australian government has not publicly responded to the lawsuit.
Reached for comment, a spokeswoman for the Treasury Department said in a
statement that it did not comment on current court proceedings.

Australia is physically vulnerable to climate change, which has helped
\href{https://www.nytimes3xbfgragh.onion/2019/12/08/world/australia/water-drought-climate.html}{drive
drought},
\href{https://www.nytimes3xbfgragh.onion/2019/12/18/world/australia/record-heat.html}{broken
temperature records} and led to the
\href{https://www.nytimes3xbfgragh.onion/2020/04/06/world/australia/great-barrier-reefs-bleaching-dying.html}{bleaching
of the Great Barrier reef}, so the financial risks of investing in the
country have raised concerns. In 2019, Sweden's central bank said
\href{https://www.smh.com.au/business/markets/sweden-dumps-aussie-bonds-as-country-not-known-for-good-climate-work-20191114-p53agw.html}{it
was letting go of Western Australian and Queensland government bonds} in
part because the greenhouse emissions from both were too high.

In recent years, the country's financial and corporate regulator have
pressured financial institutions that issue bonds to disclose their
plans to measure and mitigate the risks related to climate change.

``One of the major issuers of securities on the global financial markets
is not leading from the front,'' Rob Henderson, the former chief
economist for National Australia Bank, said of the government's lack of
disclosure.

Ms. O'Donnell's case builds on an emerging trend of climate litigation,
with calls for private companies to take responsibility for their part
in the growing threat to the planet.

A Peruvian man chose to sue Germany's largest energy company because, he
said, melting glaciers exacerbated by
\href{https://www.nytimes3xbfgragh.onion/interactive/2019/04/09/magazine/climate-change-peru-law.html}{climate
change are threatening his home}. Other nations, including the Pacific
island of Vanuatu, which are facing a threat to their very existences
because of climate change, have said they are considering
\href{https://edition.cnn.com/2018/12/17/world/vanuatu-cop-climate-change-intl/index.html}{taking
legal action against the world's biggest fossil-fuel companies}.

In all, 1,587 climate litigation cases have been brought worldwide since
1986 and May this year, with Australia second only to the United States,
according to the
\href{http://www.lse.ac.uk/granthaminstitute/publication/global-trends-in-climate-change-litigation-2020-snapshot/}{Grantham
Institute of Research on Climate Change and the Environment}. The cases
have been filed ``as a way of either advancing or delaying effective
action on climate change,'' the institute says.

It is unclear if Ms. O'Donnell will be successful. But with many private
corporations measuring --- and promising to mitigate --- their
contributions to climate change, there is ``strong acceptance of the
simple argument that climate change poses material and financial
risks,'' said Anita Foerster, a senior lecturer in business law at
Monash University.

Ms. O' Donnell, who bought her first government-issued bonds this year,
says her interest in climate law and its effect on investors began when
she heard Mr. Barnden, now her lawyer, speak at a lecture last year. She
said she chose her legal strategy because she wanted to educate herself
and others who bought such bonds of the potential financial risks of
climate change.

\includegraphics{https://static01.graylady3jvrrxbe.onion/images/2020/07/23/world/23oz-climate-suit2/merlin_169207422_a4f018aa-a7c6-47ed-bad8-b0be0a82297a-articleLarge.jpg?quality=75\&auto=webp\&disable=upscale}

``All routes are crucial, and we will need to unite.'' she said. ``But
investment and the economies and the climate are all so closely linked,
and that really needs to be highlighted.''

``The government knows about the problem,'' she added. ``They know the
solutions, and they know what they need to do but they're not doing
it.''

Mr. Henderson said he expected the case to prompt those in other nations
to follow suit: ``Other people will be saying, hang on what about our
government?''

Advertisement

\protect\hyperlink{after-bottom}{Continue reading the main story}

\hypertarget{site-index}{%
\subsection{Site Index}\label{site-index}}

\hypertarget{site-information-navigation}{%
\subsection{Site Information
Navigation}\label{site-information-navigation}}

\begin{itemize}
\tightlist
\item
  \href{https://help.nytimes3xbfgragh.onion/hc/en-us/articles/115014792127-Copyright-notice}{©~2020~The
  New York Times Company}
\end{itemize}

\begin{itemize}
\tightlist
\item
  \href{https://www.nytco.com/}{NYTCo}
\item
  \href{https://help.nytimes3xbfgragh.onion/hc/en-us/articles/115015385887-Contact-Us}{Contact
  Us}
\item
  \href{https://www.nytco.com/careers/}{Work with us}
\item
  \href{https://nytmediakit.com/}{Advertise}
\item
  \href{http://www.tbrandstudio.com/}{T Brand Studio}
\item
  \href{https://www.nytimes3xbfgragh.onion/privacy/cookie-policy\#how-do-i-manage-trackers}{Your
  Ad Choices}
\item
  \href{https://www.nytimes3xbfgragh.onion/privacy}{Privacy}
\item
  \href{https://help.nytimes3xbfgragh.onion/hc/en-us/articles/115014893428-Terms-of-service}{Terms
  of Service}
\item
  \href{https://help.nytimes3xbfgragh.onion/hc/en-us/articles/115014893968-Terms-of-sale}{Terms
  of Sale}
\item
  \href{https://spiderbites.nytimes3xbfgragh.onion}{Site Map}
\item
  \href{https://help.nytimes3xbfgragh.onion/hc/en-us}{Help}
\item
  \href{https://www.nytimes3xbfgragh.onion/subscription?campaignId=37WXW}{Subscriptions}
\end{itemize}
