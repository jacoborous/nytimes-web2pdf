Sections

SEARCH

\protect\hyperlink{site-content}{Skip to
content}\protect\hyperlink{site-index}{Skip to site index}

\href{/section/us}{U.S.}\textbar{}Federal Agencies Agree to Withdraw
From Portland, With Conditions

\url{https://nyti.ms/2P2jDHM}

\begin{itemize}
\item
\item
\item
\item
\item
\item
\end{itemize}

\href{https://www.nytimes3xbfgragh.onion/news-event/george-floyd-protests-minneapolis-new-york-los-angeles?action=click\&pgtype=Article\&state=default\&region=TOP_BANNER\&context=storylines_menu}{Race
and America}

\begin{itemize}
\tightlist
\item
  \href{https://www.nytimes3xbfgragh.onion/2020/07/26/us/protests-portland-seattle-trump.html?action=click\&pgtype=Article\&state=default\&region=TOP_BANNER\&context=storylines_menu}{Protesters
  Return to Other Cities}
\item
  \href{https://www.nytimes3xbfgragh.onion/2020/07/24/us/portland-oregon-protests-white-race.html?action=click\&pgtype=Article\&state=default\&region=TOP_BANNER\&context=storylines_menu}{Portland
  at the Center}
\item
  \href{https://www.nytimes3xbfgragh.onion/2020/07/23/podcasts/the-daily/portland-protests.html?action=click\&pgtype=Article\&state=default\&region=TOP_BANNER\&context=storylines_menu}{Podcast:
  Showdown in Portland}
\item
  \href{https://www.nytimes3xbfgragh.onion/interactive/2020/07/16/us/black-lives-matter-protests-louisville-breonna-taylor.html?action=click\&pgtype=Article\&state=default\&region=TOP_BANNER\&context=storylines_menu}{45
  Days in Louisville}
\end{itemize}

\includegraphics{https://static01.graylady3jvrrxbe.onion/images/2020/07/29/us/29portland-withdrawal/29portland-withdrawal-articleLarge.jpg?quality=75\&auto=webp\&disable=upscale}

\hypertarget{federal-agencies-agree-to-withdraw-from-portland-with-conditions}{%
\section{Federal Agencies Agree to Withdraw From Portland, With
Conditions}\label{federal-agencies-agree-to-withdraw-from-portland-with-conditions}}

Gov. Kate Brown of Oregon said the teams would begin a withdrawal on
Thursday. Federal officials cautioned that they would withdraw only when
they were confident the federal courthouse could be secured.

The arrival of federal officers triggered a dramatic escalation in
protests in downtown Portland, Ore.Credit...Mason Trinca for The New
York Times

Supported by

\protect\hyperlink{after-sponsor}{Continue reading the main story}

\href{https://www.nytimes3xbfgragh.onion/by/mike-baker}{\includegraphics{https://static01.graylady3jvrrxbe.onion/images/2020/05/19/reader-center/author-mike-baker/author-mike-baker-thumbLarge.png}}\href{https://www.nytimes3xbfgragh.onion/by/zolan-kanno-youngs}{\includegraphics{https://static01.graylady3jvrrxbe.onion/images/2019/12/13/reader-center/author-zolan-kanno-youngs/author-zolan-kanno-youngs-thumbLarge.png}}

By \href{https://www.nytimes3xbfgragh.onion/by/mike-baker}{Mike Baker}
and
\href{https://www.nytimes3xbfgragh.onion/by/zolan-kanno-youngs}{Zolan
Kanno-Youngs}

\begin{itemize}
\item
  Published July 29, 2020Updated July 31, 2020
\item
  \begin{itemize}
  \item
  \item
  \item
  \item
  \item
  \item
  \end{itemize}
\end{itemize}

For days, as fireworks and tear gas erupted in the streets of Portland,
Ore., during the deployment of federal tactical teams cracking down on
raucous demonstrations, President Trump campaigned against protesters he
described as ``sick and deranged anarchists \& agitators'' who he said
had threatened to leave Portland ``burned and beaten to the ground.''

But even as the president was doubling down, Vice President Mike Pence
and other senior administration officials were negotiating an agreement
with Oregon's governor, Kate Brown, to begin withdrawing the federal
tactical teams from Portland.

On Wednesday, Ms. Brown announced that the federal law enforcement
agents guarding the federal courthouse in downtown Portland would begin
withdrawing as early as Thursday. ``We know where we are headed,'' she
said. ``Complete withdrawal of federal troops from the city and the
state.''

Federal officials confirmed an agreement but hedged on the timing,
cautioning that a departure would depend on the success of the state's
promise to secure the area.

``Our entire law enforcement presence that was currently in Portland
yesterday and the previous week will remain in Portland until we are
assured that the courthouse and other federal facilities will no longer
be attacked nightly,'' Chad F. Wolf, the acting secretary of homeland
security, told reporters on Wednesday.

The agreement, although tenuous and framed by political divisions,
marked a stark turnaround for an administration that had aggressively
defended the presence of the federal forces. Federal agents more prone
to investigating drug smugglers than handling demonstrations had come to
the city without the support of local leaders and found themselves mired
in an endless cycle of clashes with demonstrators who opposed their
presence.

While Mr. Trump has used images of tactical agents cracking down on
protesters in his campaign videos, there was an increasing sense in the
administration that the violent scenes of unrest linked to federal
agents in Portland could risk becoming a liability, an administration
official said. Among the thousands of protesters who had joined
demonstrators in recent weeks were a Wall of Moms, nurses in scrubs and
military veterans.

The agreement to hand over responsibility to the Oregon State Police
represented a tactical retreat from the continuing confrontations while
allowing the administration to save face by saying it had accomplished
its main objective, the security of federal properties.

``President Trump and his administration have been consistent in our
message throughout the violence in Portland: The violent criminal
activity directed towards federal properties and law enforcement will
not be tolerated,'' Mr. Wolf said. ``State and local leaders must step
forward and police their communities.''

Mr. Trump cast some doubt on Wednesday about the administration's
willingness to leave.

``You hear all sorts of reports about us leaving,'' Mr. Trump said hours
before the announcement of the agreement. ``We're not leaving until
they've secured their city. We told the governor. We told the mayor.
Secure your city. If they don't secure their city soon, we have no
choice. We're going to have to go in and clean it out.''

Later in the day,
\href{https://twitter.com/realDonaldTrump/status/1288599151349923840}{the
president said on Twitter} that Fox News had reported ``incorrectly''
about what was happening in Portland, though he was not specific. ``We
are demanding that the Governor \& Mayor do their job or we will do it
for them,'' he wrote.

Officials in Oregon said they still expected the withdrawal to be
carried out in the coming days.

State and federal officials had largely not been communicating over the
past two weeks as the protests continued to escalate, filling the void
with public denouncements of one another.

The move toward a resolution began last week, when Ms. Brown reached out
to Mr. Pence, her closest contact in the White House.

Ms. Brown had spent months working with Mr. Pence on the coronavirus
pandemic, at times pleading for more federal support, but this time she
came with a request for less federal involvement, telling him that the
deployment of U.S. tactical teams on the streets of Portland needed to
end.

\includegraphics{https://static01.graylady3jvrrxbe.onion/images/2020/08/20/us/20portland-withdrawal06/merlin_175080561_f21d9d93-6d8f-491e-8771-70c9338510f0-articleLarge.jpg?quality=75\&auto=webp\&disable=upscale}

After contacting Mr. Pence's office last week, the two had a phone
conversation on Monday, which led to further conversation with the White
House chief of staff, Mark Meadows, according to Ms. Brown and
administration officials. Mr. Pence also contacted Mr. Wolf, letting him
know about the possibility of an agreement.

Later that day, Ms. Brown met in Portland with officials from the F.B.I.
and the Department of Homeland Security; she offered the possibility of
using the Oregon State Police to help secure the federal buildings.

Advisers to Ms. Brown said she acted in order to give the Trump
administration ``an exit strategy,'' as one put it, from an increasingly
volatile situation. The meeting marked the first substantial progress
after weeks of an apparent stalemate.

Image

Hundreds gathered at the federal courthouse in Portland on Tuesday
night.Credit...Brandon Bell for The New York Times

The deployment of federal law enforcement officers in Portland came as
demonstrations there, which were started to protest the death of George
Floyd in Minneapolis police custody, persisted through June. With
protests boiling around the country, Mr. Trump issued an executive order
to protect statues and federal property, prompting the Department of
Homeland Security to send teams to the federal courthouse in Portland.

The militarized tactical teams that arrived around the July 4 weekend
immediately began to employ aggressive tactics to keep demonstrators
away from federal property. One protester was shot in the head with a
crowd-control munition, and
\href{https://www.nytimes3xbfgragh.onion/2020/07/20/us/portland-protests-navy-christopher-david.html}{a
Navy veteran was hit repeatedly with a baton} as he stood still. In a
tactic that was challenged in court by the Oregon attorney general, the
federal officers
\href{https://www.nytimes3xbfgragh.onion/2020/07/17/us/portland-protests.html}{used
unmarked vans while arresting protesters}.

While the political officials traded insults, some demonstrators turned
their frustration to the presence of the tactical teams. The Trump
administration defended the deployment by citing a federal statute that
allows the homeland security secretary to deputize agents to protect
federal property. Those officials can also conduct investigations into
crimes against the property or federal officers.

But the agents, which included teams from Immigration and Customs
Enforcement, the U.S. Marshals and the Border Patrol's equivalent of a
SWAT team, also pursued protesters through the streets, at times with
tear gas, into areas where the courthouse was no longer visible.

\href{https://www.nytimes3xbfgragh.onion/2020/07/25/us/portland-federal-legal-jurisdiction-courts.html}{The
tactics of the agents} prompted investigations by the inspectors general
for the Departments of Homeland Security and Justice. But city and state
officials made no progress until this week in ending the deployment.

The weekslong breakdown in communication is especially detrimental to a
Homeland Security Department that serves as the conduit between state
governments and the Trump administration not just for law enforcement
matters, but also for responding to the pandemic and securing the
election.

Michael Chertoff, a former homeland security secretary in the George W.
Bush administration, said that while the agreement was a ``positive step
forward,'' the past month of heightened tensions and traded insults
should serve as ``a wake-up call to the department and the state and
locals about the importance of keeping these relationships warm.''

Not doing so can slow the response or make it too aggressive, Mr.
Chertoff said.

The announcement of an imminent withdrawal in Portland came a day after
officials in Washington State announced the departure of a federal
tactical team that had arrived in Seattle last week. Leaders in Seattle
have dealt with their own protests, including one over the weekend ---
in solidarity with Portland --- that included protesters burning
buildings and breaking windows and local police firing crowd-dispersal
weapons.

Under the agreement between Ms. Brown and Mr. Wolf, the governor's
office said the Oregon State Police would provide security for the
exterior of the city's federal courthouse, while the usual team of
federal officers that protects the courthouse year-round would continue
to provide security for the interior of the building.

The agreement sets up a risky situation for Ms. Brown and the Oregon
State Police, who will now be tasked with keeping calm at the
courthouse. Demonstrations have occurred nightly for more than 60 days,
with much of the ire during that time focused on the local Portland
Police Bureau.

Image

Federal agents in Portland have been mired in an endless cycle of
clashes with demonstrators who saw their presence and tactics as
troubling evidence of a federal administration with authoritarian
tendencies.Credit...Brandon Bell for The New York Times

In an email to State Police officers on Wednesday, Superintendent Travis
Hampton said he was ``very reluctant'' to expose his tactical teams to
protesters, some of whom may use violent tactics. He called the
situation in Portland ``dire'' but said the community and law
enforcement needed the assistance of the state officers.

``They will have the appropriate means to do their jobs and stay as safe
as possible --- but all eyes of the nation will be on us, particularly
when we supplant federal officers at the courthouse in an effort to
bring down the protest temperature,'' Mr. Hampton said. ``It is not a
stage we wished to be on, but we will do our part for Oregon. We'll do
our best.''

Image

Peter Buck, 74, struggled to breath after tear gas was deployed during a
protest in Portland last week.~Federal forces have employed aggressive
tactics to keep demonstrators away from federal property.Credit...Mason
Trinca for The New York Times

The news of an agreement for withdrawal found a positive reception among
some protesters on Wednesday.

Peter Buck, 74, joined the protests after federal agents arrived in
Portland, toting a leaf blower to help clear away tear gas. He said he
was delighted by the announcement that the agents might leave.

``There's no way the protesters are going to wear down or get
frightened,'' Mr. Buck said. ``There's a lot of enthusiasm and it's
going to be amazing if they retreat.''

Mr. Buck, who lives in Washington State, drove to Portland several times
to join protests against the federal presence. ``The thing that
motivated me to go to that city was the idea of sending federal troops
against the citizens of this country,'' he said.

If the federal agents left Portland, he said, he would support Black
Lives Matter events closer to his home, or attend protests in other
cities if Mr. Trump deployed federal agents there. ``The next place he
sends federal troops, I'll probably go there with my leaf blower,'' Mr.
Buck said.

Maggie Haberman, Kate Conger and Jonathan Martin contributed reporting.

Image

After dispersing a crowd of protesters overnight, officers retreated to
protect the federal courthouse.Credit...Brandon Bell for The New York
Times

Advertisement

\protect\hyperlink{after-bottom}{Continue reading the main story}

\hypertarget{site-index}{%
\subsection{Site Index}\label{site-index}}

\hypertarget{site-information-navigation}{%
\subsection{Site Information
Navigation}\label{site-information-navigation}}

\begin{itemize}
\tightlist
\item
  \href{https://help.nytimes3xbfgragh.onion/hc/en-us/articles/115014792127-Copyright-notice}{©~2020~The
  New York Times Company}
\end{itemize}

\begin{itemize}
\tightlist
\item
  \href{https://www.nytco.com/}{NYTCo}
\item
  \href{https://help.nytimes3xbfgragh.onion/hc/en-us/articles/115015385887-Contact-Us}{Contact
  Us}
\item
  \href{https://www.nytco.com/careers/}{Work with us}
\item
  \href{https://nytmediakit.com/}{Advertise}
\item
  \href{http://www.tbrandstudio.com/}{T Brand Studio}
\item
  \href{https://www.nytimes3xbfgragh.onion/privacy/cookie-policy\#how-do-i-manage-trackers}{Your
  Ad Choices}
\item
  \href{https://www.nytimes3xbfgragh.onion/privacy}{Privacy}
\item
  \href{https://help.nytimes3xbfgragh.onion/hc/en-us/articles/115014893428-Terms-of-service}{Terms
  of Service}
\item
  \href{https://help.nytimes3xbfgragh.onion/hc/en-us/articles/115014893968-Terms-of-sale}{Terms
  of Sale}
\item
  \href{https://spiderbites.nytimes3xbfgragh.onion}{Site Map}
\item
  \href{https://help.nytimes3xbfgragh.onion/hc/en-us}{Help}
\item
  \href{https://www.nytimes3xbfgragh.onion/subscription?campaignId=37WXW}{Subscriptions}
\end{itemize}
