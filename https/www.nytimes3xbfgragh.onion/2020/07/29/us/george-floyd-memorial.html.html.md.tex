Sections

SEARCH

\protect\hyperlink{site-content}{Skip to
content}\protect\hyperlink{site-index}{Skip to site index}

\href{https://www.nytimes3xbfgragh.onion/section/us}{U.S.}

\href{https://myaccount.nytimes3xbfgragh.onion/auth/login?response_type=cookie\&client_id=vi}{}

\href{https://www.nytimes3xbfgragh.onion/section/todayspaper}{Today's
Paper}

\href{/section/us}{U.S.}\textbar{}Where George Floyd Was Killed: Solemn
by Day, Violent by Night

\url{https://nyti.ms/2P4iDmp}

\begin{itemize}
\item
\item
\item
\item
\item
\item
\end{itemize}

\href{https://www.nytimes3xbfgragh.onion/news-event/george-floyd-protests-minneapolis-new-york-los-angeles?action=click\&pgtype=Article\&state=default\&region=TOP_BANNER\&context=storylines_menu}{Race
and America}

\begin{itemize}
\tightlist
\item
  \href{https://www.nytimes3xbfgragh.onion/2020/07/26/us/protests-portland-seattle-trump.html?action=click\&pgtype=Article\&state=default\&region=TOP_BANNER\&context=storylines_menu}{Protesters
  Return to Other Cities}
\item
  \href{https://www.nytimes3xbfgragh.onion/2020/07/24/us/portland-oregon-protests-white-race.html?action=click\&pgtype=Article\&state=default\&region=TOP_BANNER\&context=storylines_menu}{Portland
  at the Center}
\item
  \href{https://www.nytimes3xbfgragh.onion/2020/07/23/podcasts/the-daily/portland-protests.html?action=click\&pgtype=Article\&state=default\&region=TOP_BANNER\&context=storylines_menu}{Podcast:
  Showdown in Portland}
\item
  \href{https://www.nytimes3xbfgragh.onion/interactive/2020/07/16/us/black-lives-matter-protests-louisville-breonna-taylor.html?action=click\&pgtype=Article\&state=default\&region=TOP_BANNER\&context=storylines_menu}{45
  Days in Louisville}
\end{itemize}

Advertisement

\protect\hyperlink{after-top}{Continue reading the main story}

Supported by

\protect\hyperlink{after-sponsor}{Continue reading the main story}

\hypertarget{where-george-floyd-was-killed-solemn-by-day-violent-by-night}{%
\section{Where George Floyd Was Killed: Solemn by Day, Violent by
Night}\label{where-george-floyd-was-killed-solemn-by-day-violent-by-night}}

Two months after the killing of Mr. Floyd, a South Minneapolis
neighborhood remains a police-free zone.

\includegraphics{https://static01.graylady3jvrrxbe.onion/images/2020/07/23/us/00floyd-memorial1/00floyd-memorial1-articleLarge-v2.jpg?quality=75\&auto=webp\&disable=upscale}

By \href{https://www.nytimes3xbfgragh.onion/by/tim-arango}{Tim Arango}
and Matt Furber

\begin{itemize}
\item
  July 29, 2020
\item
  \begin{itemize}
  \item
  \item
  \item
  \item
  \item
  \item
  \end{itemize}
\end{itemize}

MINNEAPOLIS --- Still, they come --- from all over the country and
beyond. From New Mexico, Texas, Massachusetts. From Africa and Europe.
One man even said he
\href{https://www.npr.org/sections/live-updates-protests-for-racial-justice/2020/07/12/890261223/man-treks-1-000-miles-from-alabama-to-minnesota-for-change-justice-and-equality}{walked
a thousand miles}, all the way from Alabama, just to be in
\href{https://www.nytimes3xbfgragh.onion/2020/07/29/us/george-floyd-memorial.html}{Minneapolis}.

``They are coming to feel the energy and pay tribute,'' said Bianca
Dawkins, 28, a local resident who has met many of the visitors.

Two months after the police killing of George Floyd, the four-block area
of South Minneapolis where he gasped his last breaths remains a sacred
space, a no-go zone for officers. There is a neatly trimmed garden,
anchored by a sculpture of a raised fist. There are colorful murals and
the words ``I can't breathe'' painted across the pavement, as well as
the names of dozens of other Black people killed by the police.

At night, though, the space is increasingly a battleground, with
shootings and drug overdoses. The area has had an uptick in gun violence
similar to what
\href{https://www.nytimes3xbfgragh.onion/2020/07/05/us/chicago-shootings.html}{other}
\href{https://www.nytimes3xbfgragh.onion/2020/06/23/nyregion/nyc-shootings-surge.html}{cities}
have seen in the wake of protests.

At all times, the neighborhood brims with emotion. In its totality, it
feels like the raw center of America's reckoning with racial injustice.

The chaos at night has presented city officials with the challenge of
how to reassert control of the space without setting off new waves of
anger, all while maintaining it as a solemn place to honor Mr. Floyd. In
Ferguson, Mo., where the police killing of Michael Brown set off
protests in 2014,
\href{https://www.theguardian.com/us-news/2014/dec/26/ferguson-michael-brown-memorial-destroyed}{tensions
were reignited} when officers moved to clear out a memorial.

But in Minneapolis, at least for now, the city is moving cautiously.

``Opening up too quickly will have a devastating effect on people still
mourning,'' said Angela Conley, a Hennepin County commissioner, who has
been leading community discussions about the future of the area where
Mr. Floyd was killed.

\includegraphics{https://static01.graylady3jvrrxbe.onion/images/2020/07/23/us/00floyd-memorial2/merlin_173568471_115b8919-256c-45a8-a48d-a81a822da2fd-articleLarge.jpg?quality=75\&auto=webp\&disable=upscale}

Even so, elected officials are fielding a growing number of calls from
residents concerned about the violence and loud noise at night in the
area, where, among several incidents,
\href{https://www.startribune.com/pregnant-woman-fatally-shot-in-south-minneapolis-baby-delivered/571641972/}{a
pregnant woman was recently killed}.

``What people aren't recognizing is that people who live there are
having a very, very challenging time from the unlawfulness that is
occurring after the sun goes down,'' said Andrea Jenkins, a member of
the City Council whose district includes the memorial space. ``There are
constant gunshots every night. Emergency vehicles can't get in. Disabled
people are not able to access their medications, their appointments,
their food deliveries, et cetera. It's a very challenging situation.''

Ms. Jenkins, who noted that the area has historically been plagued by
gang violence, has also been taking a leading role in discussions over
how to memorialize Mr. Floyd's killing. One proposal suggests making the
garden permanent. Other ideas include a civil rights museum and renaming
Chicago Ave. in honor of Mr. Floyd. Activists are finding ways to
preserve the street art that was painted over the plywood boards that
went up to protect businesses during the protests.

From Baltimore to Ferguson to New York, organic memorials to mark where
Black men have been killed by the police have taken shape in recent
years.

But time has also worn many of them away. On Staten Island, there is a
plaque where Eric Garner was killed by the police in 2014, but a
memorial with flowers and candles was
\href{http://www.mtv.com/news/2054937/eric-garner-memorial-burned/}{set
on fire} a year after his death. In Baltimore, where Freddie Gray's
death in a police van sparked protests, a
\href{https://wtop.com/baltimore/2020/04/housing-project-where-freddie-gray-lived-being-torn-down/}{graffiti
mural} in honor of him was destroyed when the housing project where he
lived was torn down this year. In Ferguson last year, on the fifth
anniversary of Mr. Brown's death, a memorial that had been taken down
\href{https://www.stltoday.com/news/local/crime-and-courts/photos-michael-brown-memorial-rebuilt-in-ferguson/collection_61703922-5214-5cc4-98a6-b8fb6b80ed59.html\#1}{was
rebuilt}.

Even before the killing of Mr. Floyd, Ms. Jenkins and other activists in
South Minneapolis said they had hoped to build a site to recognize the
history of racial injustice in the city. ``I've been talking about a
museum for the last three years,'' Ms. Jenkins said. ``My top priority
is to build a center for racial healing in the city of Minneapolis
because Black people have been in pain for hundreds of years.''

The conversation over what to do with the space comes as many activists
in the city are fighting to defund the Police Department and reimagine
public safety. But that push for reform is happening amid a rise in
violence. Many Black residents of South Minneapolis, especially those
who live near Cup Foods, the convenience store where Mr. Floyd was
accused of using a fake \$20 bill to buy cigarettes before he was
killed, say they are caught between two emotions: anger at the police
but fearful for their safety now that officers have pulled back from the
area.

Ms. Dawkins lives a few doors down from Cup Foods, and on a recent
afternoon was selling candy and drinks and promoting a GoFundMe campaign
to raise money to avoid foreclosure. When the pandemic hit, she was
furloughed from her job at Nordstrom, and her fiancé is also out of
work.

But financial worries are only one thing on her mind. She has two
children, including a 6-week-old baby. She says the daytime is fine, and
she has met many people who have traveled to pay their respects to Mr.
Floyd.

``But when the other crowd comes at night, I can't call the police, and
that scares the hell out of me,'' she said. Ms. Dawkins pointed to a
gunshot in the windshield of her car, a gold sedan.

``We have kids in this home, so I do want police to protect families,''
she said. ``It's a hard balance. I'm happy this incident brought change,
but I want to feel safe.''

As the protests gained momentum in late May, Dr. Jackie Kawiecki set up
a medic station near Cup Foods, administering first aid to injured
protesters. Since then, she has maintained a group of medics who treat
minor ailments like abrasions and heat exhaustion during the daytime.

``Sunset to sundown is very different from sundown to sunset,'' she
said. ``My nighttime world, after sunset, I have taken care of double
gunshot wounds, drug overdoses.'' One night a man wounded by gunfire
drove a bicycle past the barricades, she said, before collapsing outside
her tent and yelling: ``I'm shot! I'm shot!''

After the pregnant woman was killed nearby in early July, and having
dodged gunfire herself, Dr. Kawiecki limited the hours of her medic
station from 9 a.m. to 9 p.m.

But residents say that nothing happening at night diminishes the
atmosphere that prevails in daylight. The other day, Anna Raeker, 25,
was greeting visitors to the space and directing people to a pamphlet
taped to a table: ``This is a space community members want to
decentralize white feelings and prioritize Black pain.''

``I think it's important that white people are intentional with the ways
they are using this space,'' said Ms. Raeker, who is white.

Nearby, next to a large mural of Mr. Floyd, a local rapper, Jordan
Wallingford, was taking a break from filming a music video. Mr.
Wallingford,
\href{https://www.startribune.com/meet-haphduzn-the-new-guy-on-atmosphere-s-welcome-to-minnesota-tour/193927531/}{who
performs under the name Haphduzn}, said he at least wanted to see the
garden, roundabout and raised fist sculpture stay permanently. ``Because
that was the spark that changed the world,'' he said.

Deborah Straub, who for weeks has been handing out snacks to children
from the neighborhood --- ``free candy, free chips, free popcorn, free
hot dogs,'' she said --- said the area should be preserved.

``Leave everything the way it is,'' she said.

Advertisement

\protect\hyperlink{after-bottom}{Continue reading the main story}

\hypertarget{site-index}{%
\subsection{Site Index}\label{site-index}}

\hypertarget{site-information-navigation}{%
\subsection{Site Information
Navigation}\label{site-information-navigation}}

\begin{itemize}
\tightlist
\item
  \href{https://help.nytimes3xbfgragh.onion/hc/en-us/articles/115014792127-Copyright-notice}{©~2020~The
  New York Times Company}
\end{itemize}

\begin{itemize}
\tightlist
\item
  \href{https://www.nytco.com/}{NYTCo}
\item
  \href{https://help.nytimes3xbfgragh.onion/hc/en-us/articles/115015385887-Contact-Us}{Contact
  Us}
\item
  \href{https://www.nytco.com/careers/}{Work with us}
\item
  \href{https://nytmediakit.com/}{Advertise}
\item
  \href{http://www.tbrandstudio.com/}{T Brand Studio}
\item
  \href{https://www.nytimes3xbfgragh.onion/privacy/cookie-policy\#how-do-i-manage-trackers}{Your
  Ad Choices}
\item
  \href{https://www.nytimes3xbfgragh.onion/privacy}{Privacy}
\item
  \href{https://help.nytimes3xbfgragh.onion/hc/en-us/articles/115014893428-Terms-of-service}{Terms
  of Service}
\item
  \href{https://help.nytimes3xbfgragh.onion/hc/en-us/articles/115014893968-Terms-of-sale}{Terms
  of Sale}
\item
  \href{https://spiderbites.nytimes3xbfgragh.onion}{Site Map}
\item
  \href{https://help.nytimes3xbfgragh.onion/hc/en-us}{Help}
\item
  \href{https://www.nytimes3xbfgragh.onion/subscription?campaignId=37WXW}{Subscriptions}
\end{itemize}
