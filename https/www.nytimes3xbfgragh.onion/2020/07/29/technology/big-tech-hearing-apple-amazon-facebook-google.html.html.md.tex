Sections

SEARCH

\protect\hyperlink{site-content}{Skip to
content}\protect\hyperlink{site-index}{Skip to site index}

\href{https://www.nytimes3xbfgragh.onion/section/technology}{Technology}

\href{https://myaccount.nytimes3xbfgragh.onion/auth/login?response_type=cookie\&client_id=vi}{}

\href{https://www.nytimes3xbfgragh.onion/section/todayspaper}{Today's
Paper}

\href{/section/technology}{Technology}\textbar{}Lawmakers, United in
Their Ire, Lash Out at Big Tech's Leaders

\url{https://nyti.ms/3hNC8vL}

\begin{itemize}
\item
\item
\item
\item
\item
\end{itemize}

Advertisement

\protect\hyperlink{after-top}{Continue reading the main story}

Supported by

\protect\hyperlink{after-sponsor}{Continue reading the main story}

\hypertarget{lawmakers-united-in-their-ire-lash-out-at-big-techs-leaders}{%
\section{Lawmakers, United in Their Ire, Lash Out at Big Tech's
Leaders}\label{lawmakers-united-in-their-ire-lash-out-at-big-techs-leaders}}

The chiefs of Amazon, Apple, Google and Facebook faced withering
questions from Democrats about anti-competitive practices and from
Republicans about anti-conservative bias.

\includegraphics{https://static01.graylady3jvrrxbe.onion/images/2020/07/30/reader-center/29TECHHEARING-A1/merlin_175077825_5ebc931b-baa1-489a-960c-34e4d845e997-articleLarge.jpg?quality=75\&auto=webp\&disable=upscale}

By \href{https://www.nytimes3xbfgragh.onion/by/cecilia-kang}{Cecilia
Kang} and
\href{https://www.nytimes3xbfgragh.onion/by/david-mccabe}{David McCabe}

\begin{itemize}
\item
  Published July 29, 2020Updated July 31, 2020
\item
  \begin{itemize}
  \item
  \item
  \item
  \item
  \item
  \end{itemize}
\end{itemize}

WASHINGTON --- The chief executives of
\href{https://www.nytimes3xbfgragh.onion/2020/07/30/podcasts/the-daily/congress-facebook-amazon-google-apple.html}{Amazon,
Apple, Google and Facebook}, four tech giants worth nearly \$5 trillion
combined,
\href{https://www.nytimes3xbfgragh.onion/live/2020/07/29/technology/tech-ceos-hearing-testimony}{faced
withering questions}from Republican and Democratic lawmakers alike on
Wednesday for the tactics and market dominance that had made their
enterprises successful.

For more than five hours, the 15 members of an antitrust panel in the
House lobbed questions and repeatedly interrupted and talked over Jeff
Bezos of Amazon, Tim Cook of Apple, Mark Zuckerberg of Facebook and
Sundar Pichai of Google.

It was the first congressional hearing for some time where Democrats and
Republicans acted as if they had a common foe, though for different
reasons. Democratic lawmakers criticized the tech companies for buying
start-ups to stifle them and for unfairly using their data hoards to
clone and kill off competitors, while Republicans questioned whether the
platforms had muzzled conservative viewpoints and were unpatriotic.

``As gatekeepers to the digital economy, these platforms enjoy the power
to pick winners and losers, shake down small businesses and enrich
themselves while choking off competitors,'' said Representative David
Cicilline, Democrat of Rhode Island and chairman of the House Judiciary
Committee's antitrust subcommittee. ``Our founders would not bow before
a king. Nor should we bow before the emperors of the online economy.''

In response, Mr. Pichai, Mr. Zuckerberg, Mr. Cook and Mr. Bezos, who
testified via videoconference because of the coronavirus pandemic, were
forced to strike a more humble chord. They presented themselves as
participants in enormously competitive and fast-changing digital
marketplaces, and they evaded questions about the decisions that turned
their companies into giants.

``We approach this process with respect and humility, but we make no
concession on the facts,'' said Mr. Cook at the outset of his testimony.

Not since
\href{https://www.nytimes3xbfgragh.onion/1999/02/27/business/microsoft-rests-its-case-ending-on-a-misstep.html}{Microsoft
stood trial in the late 1990s} for antitrust charges have tech chief
executives been under such a microscope for the power of their
businesses. With echoes of the trustbusting of U.S. Steel and Standard
Oil more than a century ago and AT\&T in 1984, the hearing underlined
the government's recognition that this cohort of tech companies ---
which wield immense control over commerce, communications and public
discourse --- had become the new trusts of the internet age.

President Trump also used the event to rail against tech power. In a
\href{https://twitter.com/realDonaldTrump/status/1288506554585505793}{post
on Twitter} before the hearing began, he said that he would issue
executive orders to rein in the companies if Congress did not.

From its conception, the House antitrust hearing was set to be a
spectacle, lining up four of the world's most powerful executives ---
with two of them among the planet's richest individuals --- to answer
largely hostile questions together. While the joint appearance limited
sustained questioning of any one executive, it created a side-by-side
image that recalled the
\href{https://www.nytimes3xbfgragh.onion/1994/04/15/us/tobacco-chiefs-say-cigarettes-aren-t-addictive.html}{1994
congressional}hearing of top American tobacco executives, who said they
did not believe that cigarettes were addictive.

House lawmakers, who had opened an investigation into the tech companies
in June 2019, made the most of it. Representative Jerry Nadler, Democrat
of New York, confronted Mr. Zuckerberg with the C.E.O.'s own emails,
saying they showed a plot to take out a young competitor. Representative
Jim Jordan, Republican of Ohio, said Google was biased and asked Mr.
Pichai whether the company would change its products to help elect
Joseph R. Biden for president.

In one of the sharpest exchanges, Representative Pramila Jayapal, a
Washington Democrat, confronted Mr. Bezos on accusations that an Amazon
lawyer had lied to the committee about how the company develops its own
products. She asked him to answer whether it misused data with a yes or
no.

``I can't answer that question yes or no,'' said Mr. Bezos, appearing
rattled.

Yet while the hearing was ripe with theater, any impact will be limited
by antitrust laws that were created a century ago and that are
\href{https://www.nytimes3xbfgragh.onion/2018/09/07/technology/monopoly-antitrust-lina-khan-amazon.html}{imperfect
for corralling internet firms}. Since the 1980s, enforcement officials
have used the notion of consumer welfare as the predominant test for
antitrust violations --- generally meaning that if prices are not going
up, the markets are most likely competitive enough. The tech giants have
generally not driven up prices of digital services or consumer goods;
many do not charge at all for services like Google Maps or Instagram.

While Democrats at the hearing indicated they were more inclined to
change antitrust law, Representative Jim **** Sensenbrenner, Republican
of Wisconsin, said he did not think the laws needed to change. ****
Ultimately, Congress doesn't have the power to break up the companies.

Still, the proceedings provided fuel to other investigations of the tech
companies by the Justice Department, the Federal Trade Commission and
state attorneys general. The Justice Department is expected to
\href{https://www.nytimes3xbfgragh.onion/2020/06/25/technology/barr-google-investigation.html}{soon
announce charges against Google}accusing it of abusing its dominance in
online advertising, people with knowledge of the investigation have
said. The
\href{https://www.nytimes3xbfgragh.onion/2020/07/17/technology/ftc-facebook-investigation.html}{F.T.C.
is preparing to question Mr. Zuckerberg} under oath in its investigation
of Facebook's grip over social networking and acquisitions of nascent
rivals.

``This is a critical juncture in how the Washington policy clash with
the titans of Silicon Valley unfolds,'' said Gene Kimmelman, a former
Justice Department antitrust official and a special adviser to the
consumer advocacy group Public Knowledge.

Regulators around the world are also moving to limit the power of the
tech giants. Europe has led the charge with antitrust investigations and
Margrethe Vestager, the region's top trustbuster,
\href{https://www.nytimes3xbfgragh.onion/2019/11/19/technology/tech-regulator-europe.html}{recently
vowed to take a harder line} on the companies. On Wednesday,
\href{https://www.nytimes3xbfgragh.onion/2020/07/29/world/europe/turkey-social-media-control.html}{Turkey
passed legislation} giving its government sweeping new powers to
regulate social media content.

The hearing on Wednesday was a turnabout from just a few years ago, when
Facebook, Google, Amazon and Apple were emblems of national pride for
their innovation and growth. But the expanding reach of the four ---
which are involved in everything from smartphones to e-commerce to
digital payments --- and their stumbles
in\href{https://www.nytimes3xbfgragh.onion/2020/07/28/technology/virus-video-trump.html}{misinformation},
privacy,
\href{https://www.nytimes3xbfgragh.onion/2018/02/17/technology/indictment-russian-tech-facebook.html}{election
interference} and labor issues have increasingly raised hackles.

Even so, the companies have continued growing as more people live their
lives online, with all of them expected to post solid financial
performances when they report quarterly earnings on Thursday.

The hearing was made more bizarre by Mr. Bezos, Mr. Cook, Mr. Pichai and
Mr. Zuckerberg dialing in remotely using Cisco's Webex videoconferencing
service. Lawmakers --- who mostly appeared in person wearing masks in a
House hearing room --- faced empty chairs and a jumbo screen with the
faces of the executives, who looked soberly into their cameras.

Lawmakers nonetheless drilled down on key moments when the companies had
gained power or allegedly squeezed consumers, competitors and small
businesses. They directed most of their questions to Mr. Zuckerberg and
Mr. Pichai, then to Mr. Bezos, according
to\href{https://www.nytimes3xbfgragh.onion/live/2020/07/29/technology/tech-ceos-hearing-testimony/heres-which-tech-ceo-is-getting-asked-the-most-questions-by-lawmakers}{a
tally} by The New York Times. Mr. Cook was asked the fewest questions.

The tone of the hearing was set with Mr. Cicilline's very first
question, directed at Mr. Pichai. ``Why does Google steal content from
honest businesses?'' Mr. Cicilline asked. Mr. Pichai replied: ``Mr.
Chairman, with respect, I disagree with that characterization.''

Mr. Pichai was repeatedly asked about Google's dominance in search and
how the company was potentially trying to keep users within ``a walled
garden.'' He said Google had many competitors for specific categories of
search, such as shopping.

Mr. Zuckerberg was asked about Facebook emails where executives
discussed the company's 2012 acquisition of Instagram as a possible
strategy to take out a nascent competitor. Mr. Zuckerberg said that, in
fact, Instagram's success had never been guaranteed and was the result
of Facebook's investment in the product.

When lawmakers asked Mr. Bezos if Amazon had bullied small merchants, he
said that it was ``not how we operate the business'' --- before being
confronted by an audio recording of a bookseller begging him directly
for relief.

In response to questions about whether Apple favored some app developers
over others, Mr. Cook said there were ``open and transparent rules''
that applied ``evenly to everyone.''

David Heinemeier Hansson, the co-founder of Basecamp, a
project-management company that has battled with both Google and Apple
over their market power, said the hearing would be irrelevant if the
government did not act to rein in the tech giants.

``What we ultimately need is relief. We don't just need a historic
moment. We need this to lead to legislation and regulation and
enforcement,'' he said.

But, Mr. Heinemeier Hansson added, ``thankfully I've never been more
optimistic for that than I am right now.''

\includegraphics{https://static01.graylady3jvrrxbe.onion/images/2017/01/29/podcasts/the-daily-album-art/the-daily-album-art-articleInline-v2.jpg?quality=75\&auto=webp\&disable=upscale}

\hypertarget{listen-to-the-daily-the-big-tech-hearing}{%
\subsubsection{Listen to `The Daily': The Big Tech
Hearing}\label{listen-to-the-daily-the-big-tech-hearing}}

A grilling on the power of digital giants in the internet age.

transcript

Back to The Daily

bars

0:00/35:19

-35:19

transcript

\hypertarget{listen-to-the-daily-the-big-tech-hearing-1}{%
\subsection{Listen to `The Daily': The Big Tech
Hearing}\label{listen-to-the-daily-the-big-tech-hearing-1}}

\hypertarget{hosted-by-michael-barbaro-produced-by-eric-krupke-and-robert-jimison-with-help-from-annie-brown-and-edited-by-lisa-tobin-and-dave-shaw}{%
\subsubsection{Hosted by Michael Barbaro; produced by Eric Krupke and
Robert Jimison; with help from Annie Brown; and edited by Lisa Tobin and
Dave
Shaw}\label{hosted-by-michael-barbaro-produced-by-eric-krupke-and-robert-jimison-with-help-from-annie-brown-and-edited-by-lisa-tobin-and-dave-shaw}}

\hypertarget{a-grilling-on-the-power-of-digital-giants-in-the-internet-age}{%
\paragraph{A grilling on the power of digital giants in the internet
age.}\label{a-grilling-on-the-power-of-digital-giants-in-the-internet-age}}

\begin{itemize}
\item
  michael barbaro\\
  From The New York Times, I'm Michael Barbaro. This is ``The Daily.''
\item
  {[}music{]}\\
  Today: The C.E.O.s of the nation's most influential technology
  companies --- Amazon, Apple, Google and Facebook --- are brought
  before Congress to answer a question. Are they the too powerful and
  too dominant monopolies of the internet age? My colleague, Cecilia
  Kang, was in the room.

  It's Thursday, July 30.

  Cecilia, we are talking just ahead of the start of what is probably
  the most anticipated hearing in the history of the tech industry. So,
  just to begin, why is this hearing happening at all?
\item
  cecilia kang\\
  This hearing is happening because there is a recognition across
  government that these four very powerful and very important companies
  to the economy have become so dominant that they are harming consumers
  and harming competition.

  So Congress has summoned the C.E.O.s of the corporations --- Jeff
  Bezos of Amazon, Tim Cook of Apple, Mark Zuckerberg of Facebook and
  Sundar Pichai of Google --- to ask them and interrogate them on their
  business practices, and find out if these internet giants that have
  become, in many ways, the new trusts of our economy, if they are
  harming consumers and competition.
\item
  michael barbaro\\
  So how exactly did we get to this point where these four executives
  are being summoned before Congress and being forced to confront that
  question?
\item
  cecilia kang\\
  I think you can start with the 2016 presidential election. That really
  was a wake up call in Washington and across the world, really, about
  the power of these social media platforms to be used for harm, not
  just for entertainment and good. The presidential election of 2020 in
  the United States then really picked up on this feeling of concern.
\item
  archived recording (elizabeth warren)\\
  I'm deeply concerned right now that the space around companies like
  Amazon, Facebook, Google, is now referred to by venture capitalists as
  the ``kill zone.''
\end{itemize}

cecilia kang

And we saw for the first time a political candidate, Elizabeth Warren,
announced her promise to break up big tech.

\begin{itemize}
\tightlist
\item
  archived recording (elizabeth warren)\\
  Break those things apart and we will have a much more competitive
  robust market in America. That's how capitalism should work.
\end{itemize}

cecilia kang

This was the first time that even the real term ``big tech'' became sort
of part of our lexicon.

michael barbaro

Right, and almost like a kind of epithet.

cecilia kang

Absolutely.

\begin{itemize}
\item
  archived recording\\
  You have criticized a lot of big banks. Today, you're talking about
  breaking up big tech. Why?
\item
  archived recording (elizabeth warren)\\
  So here's the deal. We need real competition in this field. And
  there's a problem.
\end{itemize}

cecilia kang

And there was a domino effect after that.

\begin{itemize}
\tightlist
\item
  archived recording (donald trump)\\
  So I think that Google and Twitter and Facebook ---
\end{itemize}

cecilia kang

Donald Trump ---

\begin{itemize}
\item
  archived recording (donald trump)\\
  --- they're really treading on very, very troubled territory. And they
  have to be careful. It's not fair to large portions of the population.
\item
  archived recording (hillary clinton)\\
  I'm also really disappointed in a lot of the tech companies.
\end{itemize}

cecilia kang

--- Hillary Clinton, Bernie Sanders ---

\begin{itemize}
\tightlist
\item
  archived recording (bernie sanders)\\
  I think we need vigorous antitrust legislation in this country.
\end{itemize}

cecilia kang

--- all articulated similar concerns about big tech.

\begin{itemize}
\tightlist
\item
  archived recording (bernie sanders)\\
  They have incredible power over the economy, over the political life
  of this country in a very dangerous sense.
\end{itemize}

cecilia kang

Soon after, the ground moved underneath the technology companies in
Washington. In the span of one week in June 2019, the Department of
Justice, the Federal Trade Commission and state attorneys generals all
announced that they had opened investigations into the biggest
technology companies. It was unprecedented. These companies have not
been investigated, except for Google, on antitrust grounds before in the
United States. And beyond that, it was a real recognition to put these
four companies into a cohort.

michael barbaro

I wonder if you can briefly walk us through how each of these companies'
behavior has gotten them to this point, into this hearing room. Because
these four companies are all big and they're all tech, but they're
actually all pretty different, right? So what exactly has each done that
people during this hearing are going to be confronting them about?

cecilia kang

So, in the case of Amazon, the accusation is that it is both a retailer
and it is a platform for third-party sellers --- in other words, another
small business that might sell, let's say, a face mask or tissue paper
--- that will sell their goods on Amazon. And the accusation is that
Amazon abuses its position and its clout to make sure that their own
products will always perform better than these third parties. And they
also use the data and the intelligence they have to suppress and to
charge these third-party sellers more.

In the case of Apple, the accusation is that it unfairly uses its clout
over the App Store. The App Store is huge. It has more than a million
apps on it. And it uses its power over the platform to block rivals and
to force the apps that are on the App Store to pay high commissions.

In the case of Facebook, the accusation is that it is a monopoly in
social networking and that it has acquired rivals like Instagram and
WhatsApp to maintain its monopoly. And in the process, really killed off
competition on the internet.

In the case of Google, the accusation is that it uses its dominant
position in search and online advertising, and in the Android smartphone
market, to crush rivals and to continue to maintain its dominance in all
of those marketplaces.

michael barbaro

So the common theme here is size, dominance and basically, monopolistic
conduct.

cecilia kang

Yeah, and I would say that they are different companies, and they do
have different business models. But the one thing I would say they have
in common is that they are gatekeepers. They are actually the
chokepoints on the internet, because they control commerce, and they
control communications, and they control the discovery of information on
the internet. These are sort of unprecedented in their scope. And
they're global. Everyone around the world uses them.

michael barbaro

So, listening to you talk about this hearing --- the stakes of it, the
questions around influence and conduct --- I am inevitably reminded of
---

\begin{itemize}
\tightlist
\item
  archived recording\\
  I'd like to ask our guests to please take your seats.
\end{itemize}

michael barbaro

--- what was probably the most famous congressional hearing of my
lifetime, which was the tobacco industry hearings of 1994.

\begin{itemize}
\tightlist
\item
  archived recording\\
  For the first time ever, the chief executive officers of our nation's
  tobacco companies are testifying together before the United States
  Congress.
\end{itemize}

michael barbaro

And of course, tobacco, smoking, health are different than technology.
But it was a moment when the top executives of billion dollar companies,
and very powerful companies in the economy ---

\begin{itemize}
\tightlist
\item
  archived recording\\
  The truth is that cigarettes are the single most dangerous consumer
  product ever sold.
\end{itemize}

michael barbaro

--- were summoned before Congress and really held to account in a highly
public way.

cecilia kang

I similarly see the comparison to the tobacco hearings. I mean, this is
the moment when you have the heads, the captains of the biggest
companies in technology, just like we saw the heads, the captains of the
biggest companies of the tobacco industry, have to come before Congress
---

\begin{itemize}
\tightlist
\item
  archived recording\\
  If you raise your right hand.
\end{itemize}

cecilia kang

--- stand up ---

\begin{itemize}
\tightlist
\item
  archived recording\\
  Do you swear ---
\end{itemize}

cecilia kang

--- raise their right hand, swear in ---

\begin{itemize}
\tightlist
\item
  archived recording\\
  --- the whole truth and nothing but the truth?
\end{itemize}

cecilia kang

--- and really defend themselves as companies that are potentially
harmful to society.

michael barbaro

Hm.

\begin{itemize}
\item
  archived recording (congressman)\\
  First, I'd like to just go down the row. Yes or no, do you believe
  nicotine is not addictive?
\item
  archived recording (executive 1)\\
  I believe nicotine is not addictive, yes.
\item
  archived recording (congressman)\\
  Mr. Johnston.
\item
  archived recording (executive 2)\\
  Uh, Congressman, cigarettes and nicotine clearly do not meet the
  classic definitions of addiction. There is no intoxication.
\item
  archived recording (congressman)\\
  We'll take that as a no, and again ---
\end{itemize}

cecilia kang

The moment of reckoning is similar for the tech industry in the way that
that was a moment of reckoning for the tobacco industry.

michael barbaro

And of course, in that case, those tobacco hearings, they were the
beginning of very serious changes in how the United States regulated
tobacco companies. There were big reforms. There were big fines. It was
a turning point. And if this hearing ends up feeling like a turning
point for the technology industry, I wonder what the basis for whatever
regulation flows from this would be.

cecilia kang

The hearing is going to be a real test of whether antitrust laws and
competition laws that were first created in 1890 can actually apply to
internet companies, where the companies of Silicon Valley are just so
different than rail, sugar, steel --- the trusts that, at that time,
were the inspiration for trust busters like Theodore Roosevelt and
others that were trying to contain the power of the big industrialists
at that time. So a lot of the conventional tests that have been used on
whether a company has violated antitrust laws may not apply.

And one of the biggest tests is this test known as the consumer welfare
test. This is a standard that's been used for about 40 years now and
very much permeates antitrust thinking in this country. And that
question is, are consumers harmed? It's really hard to prove harm with a
company like Google or Facebook, when they can say, well, at the end of
the day, our products are free. And at the end of the day, if you don't
like us, we're one click away from an alternative.

michael barbaro

Mm. Well, this is going to be an interesting hearing.

cecilia kang

It certainly will.

{[}music{]}

michael barbaro

We'll be right back.

\begin{itemize}
\item
  cecilia kang\\
  Hi, this is Cecilia Kang. It is noon. I am in the House Judiciary
  Committee's hearing room in the Rayburn Building. Right now, we have
  some lawmakers, and many of their aides are shuffling in, all with
  their face masks on. And there is, in the middle of the room, which
  would normally be about five rows of chairs very tightly packed
  together, they're all spread apart, about six feet each. It's an odd
  scene.

  There's a lot of cleansing of desks and microphones. We have right now
  a cleanup crew coming in with their gloves and face masks, cleaning
  off the microphones with some alcohol, and everybody being handed
  Purell and hand wipes. So that's the scene a few minutes before we
  begin.
\item
  archived recording\\
  The subcommittee will come to order.
\end{itemize}

michael barbaro

So Cecilia, tell us how this hearing starts on Wednesday.

cecilia kang

So the hearing started just a little bit late, about one hour late. And
the lawmakers, 15 of them, looked towards the back of the room at a big
jumbotron type screen from their dais. And they saw the faces of the
C.E.O.s streamed from the homes and offices of Silicon Valley and
presumably Seattle with Jeff Bezos. And ---

\begin{itemize}
\tightlist
\item
  archived recording\\
  Will you please unmute your microphones and raise your right hands? Do
  you swear or affirm under penalty of perjury that the testimony you
  ought to give is true and correct to the best of your knowledge?
\end{itemize}

cecilia kang

--- after Chairman Cicilline gavels in the hearing, he begins to ask the
C.E.O.s to introduce themselves.

michael barbaro

And Cecilia, I found these introductions, these five-minute kind of
testimonials, surprisingly personal.

cecilia kang

I think the C.E.O.s really wanted to accomplish a lot in these opening
remarks.

\begin{itemize}
\tightlist
\item
  archived recording (jeff bezos)\\
  Thank you, Chairman Cicilline, Ranking Member Sensenbrenner, and
  members of the subcommittee. I was born into great wealth, not
  monetary wealth, but it is said the wealth of a loving family.
\end{itemize}

cecilia kang

You heard Jeff Bezos and Sundar Pichai in particular really emphasize
their humble roots.

\begin{itemize}
\item
  archived recording (jeff bezos)\\
  My mom, Jackie, had me when she was a 17-year-old high school student
  in Albuquerque. Being pregnant in high school was not popular.
\item
  archived recording (sundar pichai)\\
  I didn't have much access to a computer growing up in India. So you
  can imagine my amazement when I arrived in the U.S. for graduate
  school and saw an entire lab of computers to use whenever I wanted.
\end{itemize}

cecilia kang

They are known as the richest individuals in the world, and that's
certainly the case with Jeff Bezos and Mark Zuckerberg. And I think they
wanted to be more relatable. Each of these companies wanted to start
off, right off the bat, by explaining how they were scrappy for so long,
and they continue to have that scrappy spirit. And then, in some cases,
they aren't monopolies.

\begin{itemize}
\tightlist
\item
  archived recording (mark zuckerburg)\\
  Many of our competitors have hundreds of millions or billions of
  users. Some are upstarts, but others are gatekeepers with the power to
  decide if we can even release our apps in their app stores to compete
  with them.
\end{itemize}

cecilia kang

And that, in fact, there's competition all over the world.

\begin{itemize}
\tightlist
\item
  archived recording (mark zuckerburg)\\
  And history shows that if we don't keep innovating, someone will
  replace every company here today. And that change can often happen
  faster than you expect.
\end{itemize}

cecilia kang

So they wanted to set a line early on to just dispel the notion that
there is a big tech kind of threat right now in our economy, and that
they, as individual companies, are part of a very vibrant, competitive
marketplace that's changing very quickly.

michael barbaro

So then we get to the questions from lawmakers to these four C.E.O.s.
And what do you think characterize those questions overall?

cecilia kang

Well, it was fascinating, Michael. The Democrats and Republicans were
very much split in their approaches.

\begin{itemize}
\tightlist
\item
  archived recording (david cicilline)\\
  So my first question, Mr. Pichai, is why does Google steal content
  from honest businesses?
\end{itemize}

cecilia kang

Right off the bat, the Democrats launched into very specific questions
about antitrust, and the antitrust cases against each of these
companies.

\begin{itemize}
\tightlist
\item
  archived recording (sundar pichai)\\
  Mr. Chairman, with respect, I disagree with that characterization.
\end{itemize}

cecilia kang

They really used this opportunity to show off the most damning evidence
that they had collected over 13 months, the hundreds of hours of
interviews that they've held with employees and rivals.

\begin{itemize}
\tightlist
\item
  archived recording (james sensenbrenner)\\
  Conservatives are consumers, too.
\end{itemize}

cecilia kang

The Republicans were very coordinated as well on one particular message.

\begin{itemize}
\tightlist
\item
  archived recording (james sensenbrenner)\\
  Censorship of conservative viewpoints ---
\end{itemize}

cecilia kang

They believe that tech companies represented are so powerful that
they're censoring public discourse. They're censoring speech.

\begin{itemize}
\tightlist
\item
  archived recording (james sensenbrenner)\\
  You know, I'm concerned that the people who manage the net --- and the
  four of you manage a big part of the net --- and are ending up using
  this as a political screen.
\end{itemize}

michael barbaro

I was wondering throughout that line of questioning, Cecilia, is there a
case that the Republicans focusing on this idea of conservative bias
kind of is related to anti-trust --- that ultimately, these companies
have a monopoly on the market of ideas? Or, is this really just
Republicans using this opportunity, this face time with these
executives, to focus on political grievances, and kind of ignoring the
intention of this hearing?

cecilia kang

I do think that there's a sincere belief that antitrust is related to
their concerns about censorship. Because they believe that the companies
have become so powerful, the social media companies, and that they have,
right now, the biggest marketplace of ideas and the biggest exchanges of
information. And so they see the problem of censorship as a symptom of
companies that are too big and powerful.

michael barbaro

Mm-hmm, that's interesting. But we didn't really see much Republican
focus on the more traditional idea of antitrust, meaning a business has
gotten so big that it's hurting kind of all consumers, and it's
anti-competitive to other businesses. Is that an indication that the
Republicans are less concerned about big tech as an economic threat than
Democrats are?

cecilia kang

Yeah, I was surprised actually by how little the Republican side went
into the specific debates on antitrust around these companies. And you
did hear, for example, James Sensenbrenner, who is the ranking member of
the antitrust subcommittee, say that ---

\begin{itemize}
\tightlist
\item
  archived recording (james sensenbrenner)\\
  I think the law's good.
\end{itemize}

cecilia kang

--- actually, right now, the market should work itself out.

\begin{itemize}
\tightlist
\item
  archived recording (james sensenbrenner)\\
  And we don't need to throw it all in the wastebasket.
\end{itemize}

cecilia kang

And that things are OK right now. The laws do not need to change.

\begin{itemize}
\tightlist
\item
  archived recording (james sensenbrenner)\\
  Let me ask Mr. Bezos. You know, say you are required to spin stuff off
  so you might have no more of a one-stop shop. How are the consumers
  helped by that?
\end{itemize}

cecilia kang

And said that this test --- the consumer welfare standard, this test
that whether prices go up and if there are fewer options for consumers
--- that should remain the big test for even big tech and these tech
companies.

\begin{itemize}
\item
  archived recording (jeff bezos)\\
  Sir, thank you. They would not be.
\item
  archived recording (james sensenbrenner)\\
  Right.
\item
  archived recording (jeff bezos)\\
  Very clear.
\end{itemize}

michael barbaro

Right, he said the laws don't need to change ---

\begin{itemize}
\tightlist
\item
  archived recording (james sensenbrenner)\\
  --- of enforcement of those anti-trust laws ---
\end{itemize}

michael barbaro

--- but that enforcement does.

cecilia kang

Indeed. He said that enforcement is appropriate. The laws just simply
don't need to change. And also, we should be careful, he said. In his
words, he said, being big ---

\begin{itemize}
\tightlist
\item
  archived recording (james sensenbrenner)\\
  Being big is not inherently bad.
\end{itemize}

cecilia kang

--- doesn't inherently mean that you're bad.

michael barbaro

OK, so let's talk about the most memorable exchanges involving each
company when the focus was on the more traditional aspects of antitrust
and the evidence that had been dug up in the course of this
investigation. Did they focus these lines of questioning on what you had
predicted? For example, Facebook, you had said, was going to be asked
about its tendency to buy up competitors. Is that what happened?

cecilia kang

I definitely expected the issue of buying up competitors to come up.
What I did not expect is the level of specificity that was included in
the line of questioning.

\begin{itemize}
\tightlist
\item
  archived recording (david cicilline)\\
  And thank you, John. I now recognize the distinguished chair of the
  full Judiciary Committee, Mr. Nadler from New York.
\end{itemize}

cecilia kang

I was really surprised, for example, that Jerry Nadler ---

\begin{itemize}
\tightlist
\item
  archived recording (jerry nadler)\\
  Mr. Zuckerberg, I want to thank you for providing this information
  during our investigation.
\end{itemize}

cecilia kang

--- brought up and read directly from emails from the top executives at
Facebook during the time when they wanted to purchase Instagram.

\begin{itemize}
\tightlist
\item
  archived recording (jerry nadler)\\
  However, the documents you provided tell a very disturbing story.
\end{itemize}

cecilia kang

And quoted from these emails the intent to, for example, neutralize
competitors ---

\begin{itemize}
\tightlist
\item
  archived recording (jerry nadler)\\
  You have written that Facebook can likely always just buy any
  competitive startups.
\end{itemize}

cecilia kang

--- and the concern articulated in these emails ---

\begin{itemize}
\tightlist
\item
  archived recording (jerry nadler)\\
  When Facebook contemplated acquiring Instagram, a competitive startup,
  you told your C.F.O. that the nascent Instagram could be very
  disruptive to us. And in the weeks leading up to the deal ---
\end{itemize}

cecilia kang

--- that Instagram was going to be a big threat.

\begin{itemize}
\tightlist
\item
  archived recording (jerry nadler)\\
  --- saying that, quote, ``Instagram can meaningfully hurt us without
  becoming a huge business.''
\end{itemize}

cecilia kang

And Mark Zuckerberg responded by saying ---

\begin{itemize}
\tightlist
\item
  archived recording (mark zuckerberg)\\
  Yes, I've been clear that Instagram was a competitor.
\end{itemize}

cecilia kang

--- well, yes, Instagram is a competitor. And we clearly thought they're
our competitor. And by the way ---

\begin{itemize}
\tightlist
\item
  archived recording (mark zuckerberg)\\
  I think the F.T.C. had all of these documents and reviewed this and
  unanimously voted at the time not to challenge the acquisition.
\end{itemize}

cecilia kang

--- the F.T.C. in 2011 approved this merger. So let's be clear that this
has been vetted by the federal government. He also said that if not for
Facebook and the resources that Facebook had, Instagram perhaps would
not be the company it is today, the app that it is today, which is a
wildly popular global app. And Jerry Nadler responded to that ---

\begin{itemize}
\tightlist
\item
  archived recording (jerry nadler)\\
  Mr. Zuckerberg, you're making my point.
\end{itemize}

cecilia kang

--- I think you're proving my point. He's saying, you do take nascent
competitors, and you gobble them up. And then you turn them into
important parts of the Facebook ecosystem. But those are really
interesting exchanges, in particular because they were so specific, and
they were taking the words of the executives in these emails and in
these documents straight back to the executives, and asking them
directly to respond and defend themselves. And that was something that
these executives aren't used to having to do, and certainly not in front
of the public.

michael barbaro

OK, let's move on to Google. You had predicted that Google would be
asked about the downside of its dominance in search. Is that what
happened?

cecilia kang

Yes, David Cicilline asked Sundar Pichai about his search practices. He
said ---

\begin{itemize}
\tightlist
\item
  archived recording (david cicilline)\\
  We heard throughout this investigation that Google has stolen content
  to build your own business.
\end{itemize}

cecilia kang

--- you steal content. You surface search results that aren't
necessarily the best search results, but that are the best search
results for you and your services.

\begin{itemize}
\tightlist
\item
  archived recording (david cicilline)\\
  These are consistent reports. And so, your testimony that that doesn't
  happen is really inconsistent with what we've learned during the
  course of the investigation.
\end{itemize}

cecilia kang

And you steal content from companies specifically like Yelp, which is a
restaurant review site. And you use that content to help lift and
benefit other Google services. David Cicilline's accusation was that
Google has a walled garden of all kinds of services, and they just want
users to be on their services as much as possible. And as a consequence
of that, any rival is either being used or being blocked entirely from
this important gateway, which is this Google search engine.

michael barbaro

And I noticed that when the chairman tried to press the C.E.O. of Google
on, for example, this allegation of stealing content from Yelp ---

\begin{itemize}
\tightlist
\item
  archived recording (david cicilline)\\
  Mr. Pichai, isn't that anti-competitive?
\end{itemize}

michael barbaro

--- the C.E.O. of Google did not respond.

cecilia kang

Sundar Pichai, throughout his whole testimony, was very reserved.

\begin{itemize}
\tightlist
\item
  archived recording (sundar pichai)\\
  Congressman, you know, when I run the company, I'm really focused on
  giving users what they want. We conduct ourselves to the highest
  standard.
\end{itemize}

cecilia kang

And often, he did not reply to specific accusations. And that was the
case this time as well.

\begin{itemize}
\tightlist
\item
  archived recording (sundar pichai)\\
  Happy to engage and understand the specifics and answer your questions
  further.
\end{itemize}

cecilia kang

He was deflecting.

michael barbaro

And on Amazon, Cecilia, you had said that Jeff Bezos would be challenged
about the way that company treats third party vendors. How did that play
out?

cecilia kang

Several lawmakers questioned Jeff Bezos about its treatment of third
party vendors.

\begin{itemize}
\item
  archived recording (pramila jayapal)\\
  Does Amazon ever access and use third-party seller data when making
  business decisions? And just a yes or no will suffice, sir.
\item
  archived recording (jeff bezos)\\
  I can't answer that question yes or no.
\end{itemize}

cecilia kang

Representative Lucy McBath ---

\begin{itemize}
\tightlist
\item
  archived recording (lucy mcbath)\\
  And we've interviewed many small businesses, and they use the words
  like ``bullying,'' ``fear,'' and ``panic'' to describe their
  relationship with Amazon.
\end{itemize}

cecilia kang

--- aired the recording from one of her constituents in her district who
was a bookseller on Amazon.

\begin{itemize}
\tightlist
\item
  archived recording (bookseller)\\
  And as we grew, we were shrinking Amazon's market share in the
  textbooks category.
\end{itemize}

cecilia kang

And this bookseller was delisted from the marketplace.

\begin{itemize}
\tightlist
\item
  archived recording (bookseller)\\
  So now in retaliation, Amazon started restricting us from selling.
\end{itemize}

cecilia kang

And in this recording, we heard the bookseller talk about how being
delisted essentially crippled her business entirely.

\begin{itemize}
\tightlist
\item
  archived recording (bookseller)\\
  We haven't sold a single book from the past 10 months. We were never
  given a reason. Amazon didn't even provide us with a notice as to why
  we were being restricted. There was no warning. There was no plan.
\end{itemize}

michael barbaro

What did that anecdote, that audiotape, illustrate about Amazon and this
question of antitrust?

cecilia kang

I think it demonstrated that Amazon is so big.

\begin{itemize}
\tightlist
\item
  archived recording (lucy mcbath)\\
  Do you think this is an acceptable way to treat someone that you
  described as both a partner and a customer?
\end{itemize}

cecilia kang

And it spoke to the fact that Amazon, in a way, has become its own
economy.

\begin{itemize}
\tightlist
\item
  archived recording (jeff bezos)\\
  No, congresswoman, and I appreciate you showing me that anecdote.
\end{itemize}

cecilia kang

You could say the same thing with Apple, too, and its App Store. There
are so many other companies that depend on these economies and
platforms, if you will, for their livelihoods. And in a way, they become
their own subeconomies. All four of them, actually.

michael barbaro

Mm-hmm. You just mentioned Apple. And it was your prediction that this
hearing would be about the App Store and not much else. Was that true?

cecilia kang

That was the case. With Apple, Tim Cook was asked about Apple's control
over its App Store. And Representative Hank Johnson, a Democrat from
Georgia, asked Tim Cook ---

\begin{itemize}
\tightlist
\item
  archived recording (hank johnson)\\
  Mr. Cook, does Apple not treat all app developers equally?
\end{itemize}

cecilia kang

--- if he treated all apps fairly. He asked, why is it that developers
have to get permission, and why is it that you charge developers 30
percent commission on average for simply operating on iPhones?

\begin{itemize}
\item
  archived recording (tim cook)\\
  If you look back at history ---
\item
  archived recording (hank johnson)\\
  What's to stop Apple from increasing its commission to 50 percent?
\item
  archived recording (tim cook)\\
  Sir, we have never increased commissions in the store since the first
  day it operated in 2008.
\item
  archived recording (hank johnson)\\
  There's nothing to stop you from doing so, is there?
\end{itemize}

cecilia kang

What's to stop you from raising that commission price? The line of
questioning really was about how Apple maintains its monopoly over that
App Store and makes sure that it stays ahead of rivals by that dominant
gateway position that they have as the controller of App Store.

\begin{itemize}
\tightlist
\item
  archived recording (tim cook)\\
  So we had fierce competition at the developer side and the customer
  side, which is essentially it's so competitive, I would describe it as
  a street fight for market share in the smartphone business.
\end{itemize}

cecilia kang

And Tim Cook really didn't have a great answer.

\begin{itemize}
\item
  archived recording (david cicilline)\\
  As a great American Supreme Court justice Louis Brandeis once said,
  ``We must make our choice. We may have democracy, or we may have
  wealth concentrated in the hands of a few, but we can't have both.''
  This concludes today's hearing. Without objection, this hearing is
  adjourned.
\item
  archived recording\\
  {[}GAVEL{]}
\end{itemize}

michael barbaro

So let's talk about how this all went. This was supposed to be big
tech's big tobacco moment, as you said. But Democrats and Republicans
were up to two very different things in this hearing. And I'm mindful
that lawmakers have been ridiculed in the past for their kind of shallow
understanding of technology in hearings like this. And this is for tech
companies that are distinctly difficult to understand because of their
vastness, pinning down their inner workings. And so, I'm wondering if
you think that this did feel like a big tobacco moment. Did this hearing
accomplish what lawmakers had intended?

cecilia kang

This hearing felt like big tech's big tobacco moment in that, for the
first time, the four C.E.O.s of the four biggest technology companies
had to defend themselves from accusations that were pretty tough, that
presented these companies in a pretty dark and negative light --- as
brutal, dominant enterprises that are willing to squash competition and
harm consumers along the way to maintain their dominance. In that way,
the hearing presented them through a lens that the companies had not
before been viewed through by consumers or the public. The hearing
really presented the companies as something different than just tech
startups. They presented them as big enterprises, very similar to the
trusts of the late 1800s and the early 1900s.

michael barbaro

Right.

cecilia kang

At that time, the same sort of debates were swirling around, whether it
was good for U.S. steel or for standard oil to be such sprawling
enterprises, and to be such big actors and have so much influence.

michael barbaro

So where does this leave us now? I mean, this combination of a 13-month
investigation, this spectacle of this hearing, what happens next?

cecilia kang

So that's the big question, Michael. I think that what you saw was
agreement among the Republicans and Democrats that they were angry at
these technology companies, and they had a lot of concerns. But where
you're going to see disagreement is what comes next in terms of
legislative change, what comes next also in terms of recommendations to
enforcement agencies that are actually investigating these companies at
this time. So there's going to be a lot of disagreement as to what the
path forward is going ahead.

What does change is that these companies now really can't shake this
image that they have an antitrust problem --- that all of them are, in
some way, dominant and have abused their monopoly power to harm
competition and potentially to harm consumers as well. And that's not
the kind of tag that any of these companies want attached to them.

{[}music{]}

michael barbaro

In other words, once you have been tagged as a trust and a monopoly,
it's probably just a question of what the regulatory answer to that is.

cecilia kang

Yeah, I think it's just a question of time.

{[}music{]}

michael barbaro

Thank you, Cecilia.

cecilia kang

Thank you.

michael barbaro

We'll be right back.

Here's what else you need to know today. The Times reports that more
than 150,000 people have died from the coronavirus in the U.S., a new
milestone in the pandemic. The death rate, which had briefly fallen over
the summer, is now rising in 23 different states, especially in Arizona,
South Carolina and Mississippi. On average, the virus has killed 1,000
people a day over the past week alone.

And the governor of Oregon, Kate Brown, said that federal officers would
begin to withdraw from the city of Portland today. Under an agreement
between the governor and the Trump administration, Oregon state police
will provide security for the exterior of the city's federal courthouse,
replacing the federal officers, who had repeatedly clashed with and tear
gassed protesters there.

\begin{itemize}
\tightlist
\item
  archived recording (donald trump)\\
  You hear all sorts of reports about us leaving. We're not leaving
  until they have secured their city. We told the governor, we told the
  mayor, secure your city.
\end{itemize}

michael barbaro

Even as the negotiations to leave were underway, President Trump
threatened that federal agents would remain in Portland or return there.

\begin{itemize}
\tightlist
\item
  archived recording (donald trump)\\
  If they don't secure their city soon, we have no choice. We're going
  to have to go in and clean it out. We'll do it very easily. We're all
  prepared to do it. So in Portland, they either clean out their city
  and do the job ---
\end{itemize}

{[}music{]}

michael barbaro

That's it for ``The Daily.'' I'm Michael Barbaro. See you tomorrow.

Reporting was contributed by Jack Nicas, Mike Isaac, Daisuke
Wakabayashi, Karen Weise and Kellen Browning.

Advertisement

\protect\hyperlink{after-bottom}{Continue reading the main story}

\hypertarget{site-index}{%
\subsection{Site Index}\label{site-index}}

\hypertarget{site-information-navigation}{%
\subsection{Site Information
Navigation}\label{site-information-navigation}}

\begin{itemize}
\tightlist
\item
  \href{https://help.nytimes3xbfgragh.onion/hc/en-us/articles/115014792127-Copyright-notice}{©~2020~The
  New York Times Company}
\end{itemize}

\begin{itemize}
\tightlist
\item
  \href{https://www.nytco.com/}{NYTCo}
\item
  \href{https://help.nytimes3xbfgragh.onion/hc/en-us/articles/115015385887-Contact-Us}{Contact
  Us}
\item
  \href{https://www.nytco.com/careers/}{Work with us}
\item
  \href{https://nytmediakit.com/}{Advertise}
\item
  \href{http://www.tbrandstudio.com/}{T Brand Studio}
\item
  \href{https://www.nytimes3xbfgragh.onion/privacy/cookie-policy\#how-do-i-manage-trackers}{Your
  Ad Choices}
\item
  \href{https://www.nytimes3xbfgragh.onion/privacy}{Privacy}
\item
  \href{https://help.nytimes3xbfgragh.onion/hc/en-us/articles/115014893428-Terms-of-service}{Terms
  of Service}
\item
  \href{https://help.nytimes3xbfgragh.onion/hc/en-us/articles/115014893968-Terms-of-sale}{Terms
  of Sale}
\item
  \href{https://spiderbites.nytimes3xbfgragh.onion}{Site Map}
\item
  \href{https://help.nytimes3xbfgragh.onion/hc/en-us}{Help}
\item
  \href{https://www.nytimes3xbfgragh.onion/subscription?campaignId=37WXW}{Subscriptions}
\end{itemize}
