Sections

SEARCH

\protect\hyperlink{site-content}{Skip to
content}\protect\hyperlink{site-index}{Skip to site index}

\href{https://www.nytimes3xbfgragh.onion/section/world/americas}{Americas}

\href{https://myaccount.nytimes3xbfgragh.onion/auth/login?response_type=cookie\&client_id=vi}{}

\href{https://www.nytimes3xbfgragh.onion/section/todayspaper}{Today's
Paper}

\href{/section/world/americas}{Americas}\textbar{}One in Three Children
Have Unacceptably High Lead Levels, Study Says

\url{https://nyti.ms/30YGnho}

\begin{itemize}
\item
\item
\item
\item
\item
\end{itemize}

Advertisement

\protect\hyperlink{after-top}{Continue reading the main story}

Supported by

\protect\hyperlink{after-sponsor}{Continue reading the main story}

\hypertarget{one-in-three-children-have-unacceptably-high-lead-levels-study-says}{%
\section{One in Three Children Have Unacceptably High Lead Levels, Study
Says}\label{one-in-three-children-have-unacceptably-high-lead-levels-study-says}}

``Children around the world are being poisoned by lead on a massive and
previously unrecognized scale,'' according to the study, a collaboration
of UNICEF and Pure Earth, an advocacy group.

\includegraphics{https://static01.graylady3jvrrxbe.onion/images/2020/07/29/world/29children-leadpoisoning/29children-leadpoisoning-articleLarge.jpg?quality=75\&auto=webp\&disable=upscale}

By \href{https://www.nytimes3xbfgragh.onion/by/rick-gladstone}{Rick
Gladstone}

\begin{itemize}
\item
  July 29, 2020
\item
  \begin{itemize}
  \item
  \item
  \item
  \item
  \item
  \end{itemize}
\end{itemize}

Lead contamination has long been recognized as a health hazard,
particularly for the young. But a new
\href{https://www.unicef.org/reports/toxic-truth-childrens-exposure-to-lead-pollution-2020}{study}
asserts that the extent of the problem is far bigger than previously
thought, with one in three children worldwide --- about 800 million in
all --- threatened by unacceptably high lead levels in their blood.

The ubiquity of lead --- in dust and fumes from smelters and fires,
vehicle batteries, old peeling paint, old water pipes, electronics
junkyards, and even cosmetics and lead-infused spices --- represents an
enormous and understated risk to the mental and physical development of
a generation of children, according to the study, released late
Wednesday.

The danger is particularly acute in poor and middle-income countries
where industrial pollution safeguards are poorly enforced or
nonexistent.

``The unequivocal conclusion of this research is that children around
the world are being poisoned by lead on a massive and previously
unrecognized scale,'' said the study, a collaboration of UNICEF and
\href{https://www.pureearth.org/}{Pure Earth}, a nonprofit that seeks to
help poor countries threatened by toxic pollutants.

The study also said that nearly one million adults a year die
prematurely because of lead exposure.

The authors said they based their analysis and statistical conclusions
on research compiled by United Nations agencies including the World
Health Organization, as well as by numerous universities and nonprofit
groups. The authors also used modeling techniques from the
\href{http://www.healthdata.org/,}{Institute for Health Metrics and
Evaluation}, an independent health research center that is part of the
University of Washington.

Their primary conclusion was that one-third of the world's children, up
to the age of 19, have blood lead levels at or exceeding five micrograms
per deciliter, a threshold that both the W.H.O. and the U.S. Centers for
Disease Control and Prevention have determined is a cause for action.

Most children in the United States and many other developed countries
have lead levels well below this threshold, the study said, although in
some areas they are ``dangerously high.'' The vast majority of affected
children, it said, ``live in poor countries where they are exposed to
lead through multiple routes.''

Nicholas Rees, a policy specialist on climate and environment at UNICEF
and one of the study's co-authors, said the consequences are dire.

``When you're talking about a third of the world's children, you're
talking about a potential loss of learning opportunities, an impact on
future wages, you're talking about a tremendous burden on society,'' he
said.

A major contributor to lead poisoning is a surge in the recycling of
lead in automotive batteries to satisfy soaring growth in the numbers of
cars and trucks, particularly in the developing world, the study said.
While lead recycling for batteries is heavily regulated in the United
States, it is often done haphazardly in poor and middle-income
countries.

A global industry group of lead battery manufacturers and recyclers said
it did not dispute the study's conclusions on battery recycling, which
accounts for the vast majority of all lead use.

``We want to see an end to all informal and unregulated recycling as
documented by Pure Earth and UNICEF,'' the group said in a statement.

At the same time, the group said, ``we cannot do this alone.'' It
emphasized its own efforts in helping member companies ensure that
``inappropriately recycled lead does not enter our supply chain.'' The
group also said it provides consulting to countries on improving
\href{https://www.nytimes3xbfgragh.onion/2019/12/08/world/asia/e-waste-thailand-southeast-asia.html?searchResultPosition=10}{recycling}
standards.

``For many people in low- and middle-income countries, informal and
unregulated recycling is a subsistence issue, and the materials they are
handling have a high economic value,'' it said. ``Governments and
regulators in these countries must incentivize high-performing,
regulated recyclers and crack down on the informal sector and its
practices.''

Perry Gottesfeld, an expert on lead poisoning prevention who is
executive director of
\href{http://www.okinternational.org/}{Occupational Knowledge
International}, a nonprofit that seeks to reduce industrial pollutants,
said that, if anything, the new study may have underestimated the number
of adults who die of exposure to lead poisoning. Mr. Gottesfeld also
said the study did not hold lead recycling companies sufficiently
accountable for the contamination problem.

While the industry is ``much more regulated in the United States than
anywhere else,'' he said, it would be misleading to describe lead
recycling as safe even if done properly.

Lead has been known as a potent neurotoxin for hundreds of years ---
Benjamin Franklin
\href{http://environmentaleducation.com/wp-content/uploads/userfiles/Ben\%20Franklin\%20Letter\%20on\%20EEA(1).pdf}{wrote
of its harm in 1786} --- but the most insidious effects have become
clearer only in recent decades.

The exposure of children to lead is linked to reductions in I.Q. scores,
shortened attention spans and potentially violent and criminal behavior.
Fetuses and children under 5 are at the greatest risk of lifelong
damage.

Richard Fuller, president of Pure Earth and a co-author of the study,
said he believed it was not just coincidence that the violence and
instability in many parts of the developing world are in regions where
lead contamination is relatively high.

``I think getting rid of lead is going to reduce violence,'' he said.
``I just wonder if this might be one of the most important things.''

But he also acknowledged the lack of a substitute for lead-acid
batteries. ``Honestly, I would love if we didn't have these things in
the world,'' Mr. Fuller said.

Advertisement

\protect\hyperlink{after-bottom}{Continue reading the main story}

\hypertarget{site-index}{%
\subsection{Site Index}\label{site-index}}

\hypertarget{site-information-navigation}{%
\subsection{Site Information
Navigation}\label{site-information-navigation}}

\begin{itemize}
\tightlist
\item
  \href{https://help.nytimes3xbfgragh.onion/hc/en-us/articles/115014792127-Copyright-notice}{©~2020~The
  New York Times Company}
\end{itemize}

\begin{itemize}
\tightlist
\item
  \href{https://www.nytco.com/}{NYTCo}
\item
  \href{https://help.nytimes3xbfgragh.onion/hc/en-us/articles/115015385887-Contact-Us}{Contact
  Us}
\item
  \href{https://www.nytco.com/careers/}{Work with us}
\item
  \href{https://nytmediakit.com/}{Advertise}
\item
  \href{http://www.tbrandstudio.com/}{T Brand Studio}
\item
  \href{https://www.nytimes3xbfgragh.onion/privacy/cookie-policy\#how-do-i-manage-trackers}{Your
  Ad Choices}
\item
  \href{https://www.nytimes3xbfgragh.onion/privacy}{Privacy}
\item
  \href{https://help.nytimes3xbfgragh.onion/hc/en-us/articles/115014893428-Terms-of-service}{Terms
  of Service}
\item
  \href{https://help.nytimes3xbfgragh.onion/hc/en-us/articles/115014893968-Terms-of-sale}{Terms
  of Sale}
\item
  \href{https://spiderbites.nytimes3xbfgragh.onion}{Site Map}
\item
  \href{https://help.nytimes3xbfgragh.onion/hc/en-us}{Help}
\item
  \href{https://www.nytimes3xbfgragh.onion/subscription?campaignId=37WXW}{Subscriptions}
\end{itemize}
