Sections

SEARCH

\protect\hyperlink{site-content}{Skip to
content}\protect\hyperlink{site-index}{Skip to site index}

\href{https://www.nytimes3xbfgragh.onion/section/realestate}{Real
Estate}

\href{https://myaccount.nytimes3xbfgragh.onion/auth/login?response_type=cookie\&client_id=vi}{}

\href{https://www.nytimes3xbfgragh.onion/section/todayspaper}{Today's
Paper}

\href{/section/realestate}{Real Estate}\textbar{}House Hunting in Nova
Scotia: A Sprawling Seaside Villa for \$2 Million

\url{https://nyti.ms/2EvnQ4K}

\begin{itemize}
\item
\item
\item
\item
\item
\item
\end{itemize}

Advertisement

\protect\hyperlink{after-top}{Continue reading the main story}

Supported by

\protect\hyperlink{after-sponsor}{Continue reading the main story}

International real estate

\hypertarget{house-hunting-in-nova-scotia-a-sprawling-seaside-villa-for-2-million}{%
\section{House Hunting in Nova Scotia: A Sprawling Seaside Villa for \$2
Million}\label{house-hunting-in-nova-scotia-a-sprawling-seaside-villa-for-2-million}}

The eastern Canadian province has managed to make up for the loss of
foreign buyers with more domestic transactions, staving off the worst of
the pandemic.

\href{https://www.nytimes3xbfgragh.onion/slideshow/2020/07/29/realestate/cape-cod-style-on-the-shores-of-nova-scotia.html}{}

\hypertarget{cape-cod-style-on-the-shores-of-nova-scotia}{%
\subsection{Cape Cod Style on the Shores of Nova
Scotia}\label{cape-cod-style-on-the-shores-of-nova-scotia}}

19 Photos

View Slide Show ›

\includegraphics{https://static01.graylady3jvrrxbe.onion/images/2020/08/02/realestate/29IHH-NOVASCOTIA-slide-CTJD/29IHH-NOVASCOTIA-slide-CTJD-articleLarge.jpg?quality=75\&auto=webp\&disable=upscale}

Tim Robichaud/Harbour Town Photography

By Alison Gregor

\begin{itemize}
\item
  July 29, 2020
\item
  \begin{itemize}
  \item
  \item
  \item
  \item
  \item
  \item
  \end{itemize}
\end{itemize}

\hypertarget{a-five-bedroom-estate-on-the-coast-of-nova-scotia}{%
\subsection{A Five-Bedroom Estate on the Coast of Nova
Scotia}\label{a-five-bedroom-estate-on-the-coast-of-nova-scotia}}

\hypertarget{2-million-275-million-canadian-dollars}{%
\subsubsection{**\$2 MILLION (**2.75 MILLION CANADIAN
DOLLARS)}\label{2-million-275-million-canadian-dollars}}

This five-bedroom seaside
\href{https://novascotia.evrealestate.com/ListingDetails/26-Fredas-Point-Extension-Chester-NS-B0J-1J0/201909280}{mansion}
with Atlantic Ocean views is perched on a headland jutting into Mahone
Bay in the village of Chester, on the South Shore of Nova Scotia,
Canada.

Designed by the Toronto-based architect James Wright and built in 2000,
the 5,850-square-foot home blends into a neighborhood of century-old
Cape Cods with its classic gray siding, white trim, dormers and a
cedar-shake roof. An octagonal wing with a walkway around it evokes a
lighthouse, and the home's round windows accentuate the nautical feel.

The property consists of two parcels: a 1.5-acre lot that includes the
two-story house, and a separate 2,500-square-foot lot with 30 feet of
shoreline and a 90-foot dock on Mahone Bay.

``The property is at the pinnacle of the peninsula, so it has commanding
views over the harbor,'' said Margaret Craig, an agent with Engel \&
Volkers Nova Scotia Halifax, which has the listing. The home's dock
marks the finish line of Chester's annual sailing regatta, she said.

\includegraphics{https://static01.graylady3jvrrxbe.onion/images/2020/07/29/realestate/29IHH-NOVASCOTIA-slide-RXTO/29IHH-NOVASCOTIA-slide-RXTO-articleLarge.jpg?quality=75\&auto=webp\&disable=upscale}

A gravel drive approaches the home's main entrance, which opens to a
light-filled foyer with a staircase. The foyer leads to a great room
flanked by two wood-burning fireplaces --- one of beach stone in the
living area, and one with a wood mantel in the dining area. The great
room has soaring beamed ceilings, pegged oak floors, paneled walls and
built-in window seats with views of Mahone Bay. French doors open to a
slate veranda. The furniture is not included in the asking price, but is
negotiable, Ms. Craig said.

From the dining area, two hallways lead to a powder room and the
kitchen, which has a vaulted ceiling, large island, glass cooktop, and
French doors opening to the stone terrace. There is also a dining area
with a large window seat built into a bow window. From the living area
in the great room, a slate breezeway leads to a wing with two en suite
bedrooms, including the master, which has a dressing room and connects
to an octagonal sitting room with a fireplace and French doors to the
terrace.

The staircase ascends to a second-floor landing with access to two en
suite bedrooms, each opening to a large balcony overlooking Mahone Bay.
The staircase descends to the home's lower level, which has a home
theater and recreation room with a fireplace and a bar with a granite
top. A hallway leads to a full bathroom and an octagonal bedroom with
doors opening to the lawn.

The property, which includes a terraced rock garden and is landscaped
with rhododendrons, is a 10-minute walk from the heart of Chester, which
is about 40 miles west of Nova Scotia's capital, Halifax. Chester, with
about 2,350 residents, offers shopping, restaurants, a marina, a theater
and an 18-hole golf course, Ms. Craig said. The home is about 50 miles
southwest of Halifax International Airport.

\hypertarget{market-overview}{%
\subsection{Market Overview}\label{market-overview}}

The South Shore of Nova Scotia, about 150 miles off the east coast of
Maine, stretches from south of Halifax down to Shelburne, with white
sandy beaches, colorful waterfront villages and picturesque lighthouses.
The area has long been popular with foreign home buyers, particularly
Americans.

However, the coronavirus pandemic, which prompted Canada to close its
borders in March to all nonessential travel, has put a damper on that
segment of the housing market, said Allan Mosher, a broker with the
Lunenburg office of Keller Williams Select Realty. (Nova Scotia province
had reported 1,067 confirmed Covid-19 cases and 63 deaths as of July 28,
\href{https://novascotia.ca/coronavirus/data/}{according to the
government}.)

Image

An octagonal wing with a walkway around it evokes a lighthouse, and the
home's round windows accentuate the nautical feel.Credit...Tim
Robichaud/Harbour Town Photography

``We don't have our American clients, and that absolutely is affecting
our real estate market,'' Mr. Mosher said. ``Not too many people will go
online on FaceTime or Zoom or one of those sites and buy an \$800,000
property.''

Other brokers said they are handling more sales of properties virtually,
as buyers grow accustomed to digital listing tools. Ms. Craig said she
has continued to get inquiries and offers from some Americans who are
comfortable buying properties sight unseen.

``But it's a big decision, and some buyers are pushing pause on their
plans,'' she said. ``Most summers, we would see a number of buyers who
are taking advantage of the opportunity to view some properties in
person while they're in Nova Scotia on summer vacation. We haven't seen
these buyers this summer.''

The data, however, doesn't reflect the weak sales to Americans and other
foreign buyers. Home sales to local buyers in Nova Scotia appear to be
filling the void, along with sales to groups such as millennials moving
to Halifax, Nova Scotia natives moving home for retirement, and
Canadians moving from the west who can telecommute, Ms. Craig said.

``During Covid, there was an increase in the number of people searching
for homes online, and not surprisingly, there was a trend toward more
rural properties as people contemplated an exodus from dense cities,''
she said.

Across Nova Scotia province, home sales in June totaled 1,428 units, an
increase of 10.4 percent from June 2019, according to data from the
\href{https://www.nsrealtors.ca/}{Nova Scotia Association of Realtors}.
This June, the average price of homes sold was 286,227 Canadian dollars
(\$214,000), up 10.1 percent from June 2019.

Image

A second-floor balcony with sea views is accessed through two upstairs
bedrooms.Credit...Tim Robichaud/Harbour Town Photography

On the South Shore of Nova Scotia, the residential average price in June
was 214,121 Canadian dollars (\$160,000), up 9.1 percent from June 2019,
according to the Nova Scotia Association of Realtors.

Most foreign home buyers seek properties with a waterfront location or a
sea view, which can add tens or even hundreds of thousands of dollars to
a property value, Ms. Craig said, adding that the broad spectrum of home
prices on the South Shore is one of its attractions.

``There is a wide range of waterfront homes available on the South
Shore, ranging from 350,000 Canadian dollars (\$260,000) to 3.5 million
Canadian dollars (\$2.6 million) plus, depending on the size of the
home, location, quality of frontage,'' Ms. Craig said.

\hypertarget{who-buys-in-nova-scotia}{%
\subsection{Who Buys in Nova Scotia}\label{who-buys-in-nova-scotia}}

Americans have traditionally been the predominant foreign buyers on the
South Shore of Nova Scotia, but in recent years a growing number have
come from Germany and the United Kingdom, Mr. Mosher said.

Halifax, the provincial capital and home to several universities and a
strong agricultural economy, typically sees many buyers from Asian
countries, such as China, often the parents of children who are
attending one of the city's schools, said Moneesha Sinha, a real estate
lawyer with Blois, Nickerson \& Bryson LLP.

In recent decades, many Nova Scotian natives who moved to western Canada
for economic reasons have returned to buy a retirement home, Ms. Craig
said.

Image

The master suite has French doors to a terrace and is connected to the
octagonal sitting room.Credit...Tim Robichaud/Harbour Town Photography

``We're seeing a lot of people who can work from home move east because
of the quality of life and affordable homes,'' she said. ``It will be
interesting going forward if we'll see more of this post-Covid as work
from home becomes an option.''

\hypertarget{buying-basics}{%
\subsection{Buying Basics}\label{buying-basics}}

There are no restrictions on real estate purchases by foreigners in Nova
Scotia.

Foreign buyers must hire a local lawyer to handle the transaction, and
currently because of the pandemic, document signings can be handled
virtually, Ms. Sinha said. Legal fees are usually around 900 Canadian
dollars (\$670), she said.

Transactions are done in Canadian dollars. The seller usually pays the
real estate agent commission, which is typically 4 to 5 percent.

Closing costs include a deed transfer tax, which is charged at 1.5
percent of the sale price in most municipalities, plus about 1,600
Canadian dollars (\$1,200) in fees, which include recording fees and
title insurance, Ms. Craig said.

Home inspections can range from 500 to 2,000 Canadian dollars (\$375 to
\$1,500) plus tax, depending on the size of the property and other
factors, she said.

Financing is available to foreign home buyers, though the terms may not
be as favorable as those offered Canadian buyers, brokers said.

Image

The home's dock marks the finish line of the Chester Yacht Club's annual
sailing regatta on Mahone Bay.Credit...Tim Robichaud/Harbour Town
Photography

\hypertarget{websites}{%
\subsection{Websites}\label{websites}}

\begin{itemize}
\item
  Nova Scotia tourism site:
  \href{http://novascotia.com/}{novascotia.com}
\item
  Nova Scotia Association of Realtors:
  \href{http://nsrealtors.ca/}{nsrealtors.ca}
\item
  Government of Canada: \href{http://canada.ca/}{canada.ca}
\end{itemize}

\hypertarget{languages-and-currency}{%
\subsection{Languages and Currency}\label{languages-and-currency}}

English, French; Canadian dollar (1 Canadian dollar = \$0.75)

\hypertarget{taxes-and-fees}{%
\subsection{Taxes and Fees}\label{taxes-and-fees}}

Annual estimated taxes on this property are about 9,700 Canadian dollars
(\$7,200).

\hypertarget{contact}{%
\subsection{Contact}\label{contact}}

Margaret Craig, Engel \& Volkers Nova Scotia Halifax, 902-233-0227,
\href{http://novascotia.evrealestate.com/}{novascotia.evrealestate.com}

For weekly email updates on residential real estate news,
\href{http://www.nytimes3xbfgragh.onion/newsletters/realestate/}{sign up
here}. Follow us on Twitter:
\href{https://twitter.com/nytrealestate}{@nytrealestate}.

Advertisement

\protect\hyperlink{after-bottom}{Continue reading the main story}

\hypertarget{site-index}{%
\subsection{Site Index}\label{site-index}}

\hypertarget{site-information-navigation}{%
\subsection{Site Information
Navigation}\label{site-information-navigation}}

\begin{itemize}
\tightlist
\item
  \href{https://help.nytimes3xbfgragh.onion/hc/en-us/articles/115014792127-Copyright-notice}{©~2020~The
  New York Times Company}
\end{itemize}

\begin{itemize}
\tightlist
\item
  \href{https://www.nytco.com/}{NYTCo}
\item
  \href{https://help.nytimes3xbfgragh.onion/hc/en-us/articles/115015385887-Contact-Us}{Contact
  Us}
\item
  \href{https://www.nytco.com/careers/}{Work with us}
\item
  \href{https://nytmediakit.com/}{Advertise}
\item
  \href{http://www.tbrandstudio.com/}{T Brand Studio}
\item
  \href{https://www.nytimes3xbfgragh.onion/privacy/cookie-policy\#how-do-i-manage-trackers}{Your
  Ad Choices}
\item
  \href{https://www.nytimes3xbfgragh.onion/privacy}{Privacy}
\item
  \href{https://help.nytimes3xbfgragh.onion/hc/en-us/articles/115014893428-Terms-of-service}{Terms
  of Service}
\item
  \href{https://help.nytimes3xbfgragh.onion/hc/en-us/articles/115014893968-Terms-of-sale}{Terms
  of Sale}
\item
  \href{https://spiderbites.nytimes3xbfgragh.onion}{Site Map}
\item
  \href{https://help.nytimes3xbfgragh.onion/hc/en-us}{Help}
\item
  \href{https://www.nytimes3xbfgragh.onion/subscription?campaignId=37WXW}{Subscriptions}
\end{itemize}
