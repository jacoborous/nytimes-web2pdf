Sections

SEARCH

\protect\hyperlink{site-content}{Skip to
content}\protect\hyperlink{site-index}{Skip to site index}

\href{https://myaccount.nytimes3xbfgragh.onion/auth/login?response_type=cookie\&client_id=vi}{}

\href{https://www.nytimes3xbfgragh.onion/section/todayspaper}{Today's
Paper}

\href{/section/opinion}{Opinion}\textbar{}Proud of the Kids

\url{https://nyti.ms/2X9Lk5U}

\begin{itemize}
\item
\item
\item
\item
\item
\end{itemize}

Advertisement

\protect\hyperlink{after-top}{Continue reading the main story}

\href{/section/opinion}{Opinion}

Supported by

\protect\hyperlink{after-sponsor}{Continue reading the main story}

letters

\hypertarget{proud-of-the-kids}{%
\section{Proud of the Kids}\label{proud-of-the-kids}}

A mother recounts what her 12-year-old daughter and her friends are
saying about returning to school and notes, ``I was awed by them.''
Also: The story behind the flour.

July 29, 2020

\begin{itemize}
\item
\item
\item
\item
\item
\end{itemize}

\hypertarget{more-from-our-inbox}{%
\subsubsection{More from our inbox:}\label{more-from-our-inbox}}

\begin{itemize}
\tightlist
\item
  \protect\hyperlink{link-4c6cf7c0}{The Secret's in the Flour}
\end{itemize}

\includegraphics{https://static01.graylady3jvrrxbe.onion/images/2020/07/26/opinion/sunday/25opreaders-kids/25opreaders-kids-articleLarge.jpg?quality=75\&auto=webp\&disable=upscale}

\textbf{To the Editor:}

Re
``\href{https://www.nytimes3xbfgragh.onion/2020/07/25/opinion/sunday/coronavirus-kids-school.html}{What
Do the Kids Think?}'' (Sunday Review, July 26), a selection of comments
from students about going back to school:

I am the mother of a 12-year-old girl. During this past spring and now
summer I have had the opportunity to ``listen in'' on some of my
daughter's FaceTime and group chats with other 12- and 13-year-old kids.
I have been truly enlightened.

At first I thought they were just snarky, opinionated tweens. But I have
come to realize they are bright and compassionate thinkers.

As I passed by my daughter's room one day, I overheard these various
statements among her and her friends: ``We shouldn't be made political
pawns,'' ``we're being used for the economy,'' ``how could I live with
myself if I brought the virus home to my granny?,'' ``learning online
isn't that hard if it saves my family and friends,'' ``why do we not get
a voice?,'' ``we're scared too,'' ``how can that president use us like
we don't matter?,'' ``I wish I could vote,'' ``we're not lab rats.''

I was awed by them. Proud. Would I have been this evolved at her age? I
don't know. But what I do know, and your article confirmed it, is that
kids are listening. They are learning. They are communicating. They are
thinking. And they want to be heard.

Kathryn Morris\\
Powder Springs, Ga.

\hypertarget{the-secrets-in-the-flour}{%
\subsection{The Secret's in the Flour}\label{the-secrets-in-the-flour}}

Image

~Credit...Hilary Swift for The New York Times

\textbf{To the Editor:}

It was a pleasure to read
``\href{https://www.nytimes3xbfgragh.onion/2020/07/24/opinion/us-grain-industry.html?searchResultPosition=1}{That
Flour You Bought Could Foretell Our Economy},'' by Tim Wu (Op-Ed, July
27), about smaller scale production's important place in restoring
American manufacturing.

My grandfather was ahead of his time in the 1940s, when he restored an
old water-driven mill in Virginia. He produced stone ground cornmeal,
white flour and whole grain flours.

His flour sacks read: ``Remember the miller when you eat your daily
bread.'' All his grandchildren have a sack with this saying framed on
our walls, and most of us cook with flour from King Arthur, one of the
smaller sellers your article mentions.

Maybe now is a time all Americans can remember the miller, the weaver,
the seamstress, the welder.

Sherry Thomas\\
East Marion, N.Y.

\textbf{To the Editor:}

Now you've done it. By publishing Tim Wu's Op-Ed article, you have
revealed the secret to why my husband's home-baked challah rolls are
coveted by everyone lucky enough to be the beneficiary of his largess
after he finishes his weekly baking on Friday afternoons.

King Arthur Flour has made those rolls the talk of our neighborhood!

Ruth Solomon\\
Chicago

Advertisement

\protect\hyperlink{after-bottom}{Continue reading the main story}

\hypertarget{site-index}{%
\subsection{Site Index}\label{site-index}}

\hypertarget{site-information-navigation}{%
\subsection{Site Information
Navigation}\label{site-information-navigation}}

\begin{itemize}
\tightlist
\item
  \href{https://help.nytimes3xbfgragh.onion/hc/en-us/articles/115014792127-Copyright-notice}{©~2020~The
  New York Times Company}
\end{itemize}

\begin{itemize}
\tightlist
\item
  \href{https://www.nytco.com/}{NYTCo}
\item
  \href{https://help.nytimes3xbfgragh.onion/hc/en-us/articles/115015385887-Contact-Us}{Contact
  Us}
\item
  \href{https://www.nytco.com/careers/}{Work with us}
\item
  \href{https://nytmediakit.com/}{Advertise}
\item
  \href{http://www.tbrandstudio.com/}{T Brand Studio}
\item
  \href{https://www.nytimes3xbfgragh.onion/privacy/cookie-policy\#how-do-i-manage-trackers}{Your
  Ad Choices}
\item
  \href{https://www.nytimes3xbfgragh.onion/privacy}{Privacy}
\item
  \href{https://help.nytimes3xbfgragh.onion/hc/en-us/articles/115014893428-Terms-of-service}{Terms
  of Service}
\item
  \href{https://help.nytimes3xbfgragh.onion/hc/en-us/articles/115014893968-Terms-of-sale}{Terms
  of Sale}
\item
  \href{https://spiderbites.nytimes3xbfgragh.onion}{Site Map}
\item
  \href{https://help.nytimes3xbfgragh.onion/hc/en-us}{Help}
\item
  \href{https://www.nytimes3xbfgragh.onion/subscription?campaignId=37WXW}{Subscriptions}
\end{itemize}
