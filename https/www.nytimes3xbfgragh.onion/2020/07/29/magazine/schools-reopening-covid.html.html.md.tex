Sections

SEARCH

\protect\hyperlink{site-content}{Skip to
content}\protect\hyperlink{site-index}{Skip to site index}

\href{https://myaccount.nytimes3xbfgragh.onion/auth/login?response_type=cookie\&client_id=vi}{}

\href{https://www.nytimes3xbfgragh.onion/section/todayspaper}{Today's
Paper}

Why Is There No Consensus About Reopening Schools?

\href{https://nyti.ms/2BCjBDc}{https://nyti.ms/2BCjBDc}

\begin{itemize}
\item
\item
\item
\item
\item
\item
\end{itemize}

\href{https://www.nytimes3xbfgragh.onion/news-event/coronavirus?action=click\&pgtype=Article\&state=default\&region=TOP_BANNER\&context=storylines_menu}{The
Coronavirus Outbreak}

\begin{itemize}
\tightlist
\item
  live\href{https://www.nytimes3xbfgragh.onion/2020/08/04/world/coronavirus-cases.html?action=click\&pgtype=Article\&state=default\&region=TOP_BANNER\&context=storylines_menu}{Latest
  Updates}
\item
  \href{https://www.nytimes3xbfgragh.onion/interactive/2020/us/coronavirus-us-cases.html?action=click\&pgtype=Article\&state=default\&region=TOP_BANNER\&context=storylines_menu}{Maps
  and Cases}
\item
  \href{https://www.nytimes3xbfgragh.onion/interactive/2020/science/coronavirus-vaccine-tracker.html?action=click\&pgtype=Article\&state=default\&region=TOP_BANNER\&context=storylines_menu}{Vaccine
  Tracker}
\item
  \href{https://www.nytimes3xbfgragh.onion/2020/08/02/us/covid-college-reopening.html?action=click\&pgtype=Article\&state=default\&region=TOP_BANNER\&context=storylines_menu}{College
  Reopening}
\item
  \href{https://www.nytimes3xbfgragh.onion/live/2020/08/04/business/stock-market-today-coronavirus?action=click\&pgtype=Article\&state=default\&region=TOP_BANNER\&context=storylines_menu}{Economy}
\end{itemize}

Advertisement

\protect\hyperlink{after-top}{Continue reading the main story}

Supported by

\protect\hyperlink{after-sponsor}{Continue reading the main story}

\href{/column/studies-show}{Studies Show}

\hypertarget{why-is-there-no-consensus-about-reopening-schools}{%
\section{Why Is There No Consensus About Reopening
Schools?}\label{why-is-there-no-consensus-about-reopening-schools}}

\includegraphics{https://static01.graylady3jvrrxbe.onion/images/2020/08/02/magazine/02mag-studies-1/02mag-studies-1-articleLarge.jpg?quality=75\&auto=webp\&disable=upscale}

By Kim Tingley

\begin{itemize}
\item
  Published July 29, 2020Updated Aug. 4, 2020, 8:22 a.m. ET
\item
  \begin{itemize}
  \item
  \item
  \item
  \item
  \item
  \item
  \end{itemize}
\end{itemize}

Is it possible to
\href{https://www.nytimes3xbfgragh.onion/2020/08/03/us/school-closing-coronavirus.html}{reopen
school buildings} in the fall in a way that keeps kids, educators, staff
and their families and communities safe from Covid-19? Is it possible
\emph{not} to do so without harming them in other ways? Already,
\href{https://www.nytimes3xbfgragh.onion/2020/07/29/health/covid-school-reopening.html}{school
closures} have set children behind academically. More than 20 million
children rely on school breakfasts and lunches. Too many parents face
the choice between losing their jobs or leaving their children at home
unsupervised. Vaccination rates for various dangerous diseases,
typically required before students can attend school, have plummeted.
Isolating children from their peers exacts social and emotional costs,
which differ by age group and are nearly impossible to quantify. And
whether schools reopen or remain closed, the risks are borne
disproportionately by low-income communities and people of color. ``This
is really one of the most perplexing and complex issues I've ever faced
in 40 years,'' says Dan M. Cooper, a professor of pediatrics at the
University of California, Irvine.

A flood of guidance has been issued in recent weeks, much of it urging
\href{https://www.nytimes3xbfgragh.onion/2020/08/03/business/how-schools-reopen.html}{schools
to reopen} and suggesting safety precautions. Media outlets as well have
relayed reams of often conflicting expert advice on how to weigh risks
and benefits, to individuals and to society. In every case, that
calculation is constrained by major gaps in our understanding of how
Covid affects children and those in contact with them. Strong evidence
suggests that children are much less likely than adults to get sick or
die from the virus. (By July 9, data from most of the U.S. showed that
\href{https://services.aap.org/en/pages/2019-novel-coronavirus-covid-19-infections/children-and-covid-19-state-level-data-report/}{nearly
242,000 children} had tested positive for Covid, representing 8 percent
of cases, the American Academy of Pediatrics reports; they account for
fewer than 3 percent of hospitalizations and fewer than 1 percent of
deaths.)

But are children less likely to be infected, or just less likely to show
symptoms? Does the virus behave differently in grade-schoolers than in
high-schoolers? What factors determine whether children become seriously
ill? And, perhaps most crucial for schools, what are the odds that
students will transmit the virus to one another or to adults?

One reason scientists have a lesser understanding of how the coronavirus
acts on children is that in March, at the outset of the pandemic, most
stricken countries simultaneously closed schools, shuttered businesses
and urged people to stay home, making it nearly impossible to separate
out the effect of school closures on rates of transmission in the
community. During a stay-at-home advisory in Switzerland, researchers
from Geneva University Hospitals tried to determine how vulnerable
various age groups were to infection. Beginning in April, they adapted a
health study already underway to test residents for coronavirus
antibodies. Subjects came in weekly and were invited to bring everyone
they lived with who was at least 5 to be tested, too. The results from
more than 2,700 participants over five weeks,
\href{https://www.thelancet.com/journals/lancet/article/PIIS0140-6736(20)31304-0/fulltext}{published
in The Lancet} in June, showed that children ages 5 to 9 and adults over
65 were significantly less likely to test positive than those between
the ages of 10 and 64. Of the 123 children in that age group, 21 were
exposed to an infected household member, but only one developed
antibodies.

\includegraphics{https://static01.graylady3jvrrxbe.onion/images/2020/08/02/magazine/02mag-studies-2/02mag-studies-2-articleLarge.jpg?quality=75\&auto=webp\&disable=upscale}

Large-scale randomized testing and contact tracing over time, which
would give a more complete picture of who transmits the virus and how,
hasn't been done yet in schools. In July, in the journal Emerging
Infectious Diseases, researchers from the Korea Centers for Disease
Control and Prevention published
\href{https://wwwnc.cdc.gov/eid/article/26/10/20-1315_article}{the
results of tracing more than 59,000 contacts} of 5,706 coronavirus
patients. Children younger than 10 were found to have transmitted the
virus much less than did those between 10 and 19, whose transmission
rate was equivalent to that of adults. But only 3 percent of patients in
that initial cohort were 19 and younger, and their having been tested
probably means they presented symptoms. It's still unclear how
asymptomatic children, who are hard to identify, might spread the virus;
it's also unclear if there are differences in transmission between the
ages of 10 and 19.

``A lot of the data we're getting from different sources is messy and
not necessarily pointing in the same direction,'' says Nicholas Davies,
an epidemiologist at the London School of Hygiene \& Tropical Medicine.
He and colleagues used a statistical method called Bayesian inference to
test several hypotheses about children and Covid. These included the
premises that kids are being infected but not showing symptoms and that
kids are less susceptible to infection. Based on epidemiological data
from China, Italy, Japan, Singapore, Canada and South Korea, the
researchers concluded that both premises were probably true, to an
extent. Their findings, published in Nature Medicine in June, estimate
that \href{https://www.nature.com/articles/s41591-020-0962-9}{people
under 20 are about half as likely} as older age groups to become
infected, and that among 10- to 19-year-olds who do get the virus, only
21 percent will have clinical symptoms. They couldn't make finer age
distinctions, nor say how likely any children are to infect others.

None of these studies directly addresses the impact of reopening schools
on the spread of Covid. In fact, when researchers from the University of
Washington departments of global health and epidemiology began compiling
a summary of models from 15 other countries where students have
returned, they found ``very few'' scientific publications on the topic
and relied primarily on news reports. In nearly all countries, they
observed, schools adopted safety measures, including face masks and
social distancing. None of the countries (except Sweden, which kept many
schools open) resumed classes before national rates of infection had
significantly declined; there is no evidence to say what the outcome of
opening schools would be in areas of the U.S. where the virus is
surging. In Germany, where infection rates were higher than in other
European countries, the return of older students accompanied an
\href{https://www.medrxiv.org/content/10.1101/2020.06.24.20139634v1.full.pdf}{increase
in infections among one another but not staff,} according to a preprint
led by researchers at the University of Manchester and Public Health
England.

In Israel, students and staff wore masks after schools reopened in early
May. But several weeks later, those rules were relaxed. According to
Haaretz, outbreaks began soon after, exposing thousands at schools to
infection, causing many of them to close down again. There's ``not clear
cause and effect'' between the removal of masks and the outbreaks, says
the summary's lead author, Brandon Guthrie, but it's ``circumstantial
evidence'' that they offer some protection in classrooms. It also
reveals how unenforceable the health guidance schools are receiving can
be.

Cooper, co-author of a commentary in The Journal of Pediatrics in May
that highlights
\href{https://www.jpeds.com/article/S0022-3476(20)30608-9/fulltext\#\%20}{the
need for collaboration} between local schools and public health
officials, believes, in general, that ``schools need to reopen, and we
need to study what happens in the schools very, very carefully.'' The
C.D.C. could be ``quite helpful,'' according to Anita Cicero, deputy
director at the Johns Hopkins Center for Health Security, if it ``put
out a model protocol'' for researchers trying to track Covid cases that
emerge in schools ``so everyone is collecting data the same way.''

In May, the N.I.H. initiated a study to test thousands of children and
their families over six months to see who gets the virus, whether it's
transmitted within the household and who develops Covid, while
collecting information about participants' recent activities. That's the
kind of detailed data collection needed to help determine under what
conditions schools are likely to endure outbreaks or contribute to
community spread. But none of that data will help us in time for the
start of the school year. Instead, without the ability to consistently
test students, get quick results and trace contacts, it will be
impossible for schools to tell who has the virus and whether it's
circulating on campus; when students and staff inevitably get sick,
individual schools will have to debate shutting down or staying open
without any more useful information to guide them than they have now. To
all of America's failures in the Covid-19 crisis, we should surely add
this one: the inability to get schools the tools and data they need to
strike the best possible balance between education and health.

Advertisement

\protect\hyperlink{after-bottom}{Continue reading the main story}

\hypertarget{site-index}{%
\subsection{Site Index}\label{site-index}}

\hypertarget{site-information-navigation}{%
\subsection{Site Information
Navigation}\label{site-information-navigation}}

\begin{itemize}
\tightlist
\item
  \href{https://help.nytimes3xbfgragh.onion/hc/en-us/articles/115014792127-Copyright-notice}{©~2020~The
  New York Times Company}
\end{itemize}

\begin{itemize}
\tightlist
\item
  \href{https://www.nytco.com/}{NYTCo}
\item
  \href{https://help.nytimes3xbfgragh.onion/hc/en-us/articles/115015385887-Contact-Us}{Contact
  Us}
\item
  \href{https://www.nytco.com/careers/}{Work with us}
\item
  \href{https://nytmediakit.com/}{Advertise}
\item
  \href{http://www.tbrandstudio.com/}{T Brand Studio}
\item
  \href{https://www.nytimes3xbfgragh.onion/privacy/cookie-policy\#how-do-i-manage-trackers}{Your
  Ad Choices}
\item
  \href{https://www.nytimes3xbfgragh.onion/privacy}{Privacy}
\item
  \href{https://help.nytimes3xbfgragh.onion/hc/en-us/articles/115014893428-Terms-of-service}{Terms
  of Service}
\item
  \href{https://help.nytimes3xbfgragh.onion/hc/en-us/articles/115014893968-Terms-of-sale}{Terms
  of Sale}
\item
  \href{https://spiderbites.nytimes3xbfgragh.onion}{Site Map}
\item
  \href{https://help.nytimes3xbfgragh.onion/hc/en-us}{Help}
\item
  \href{https://www.nytimes3xbfgragh.onion/subscription?campaignId=37WXW}{Subscriptions}
\end{itemize}
