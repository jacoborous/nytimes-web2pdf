Sections

SEARCH

\protect\hyperlink{site-content}{Skip to
content}\protect\hyperlink{site-index}{Skip to site index}

\href{https://www.nytimes3xbfgragh.onion/section/us}{U.S.}

\href{https://myaccount.nytimes3xbfgragh.onion/auth/login?response_type=cookie\&client_id=vi}{}

\href{https://www.nytimes3xbfgragh.onion/section/todayspaper}{Today's
Paper}

\href{/section/us}{U.S.}\textbar{}For Maine Lobstermen, a Perfect Storm
Threatens the Summer Season

\url{https://nyti.ms/2ZwHMuW}

\begin{itemize}
\item
\item
\item
\item
\item
\end{itemize}

\hypertarget{the-coronavirus-outbreak}{%
\subsubsection{\texorpdfstring{\href{https://www.nytimes3xbfgragh.onion/news-event/coronavirus?name=styln-coronavirus-national\&region=TOP_BANNER\&variant=undefined\&block=storyline_menu_recirc\&action=click\&pgtype=Article\&impression_id=a487c290-e108-11ea-b959-2b70faf7b631}{The
Coronavirus
Outbreak}}{The Coronavirus Outbreak}}\label{the-coronavirus-outbreak}}

\begin{itemize}
\tightlist
\item
  live\href{https://www.nytimes3xbfgragh.onion/2020/08/17/world/coronavirus-covid.html?name=styln-coronavirus-national\&region=TOP_BANNER\&variant=undefined\&block=storyline_menu_recirc\&action=click\&pgtype=Article\&impression_id=a487c291-e108-11ea-b959-2b70faf7b631}{Latest
  Updates}
\item
  \href{https://www.nytimes3xbfgragh.onion/interactive/2020/us/coronavirus-us-cases.html?name=styln-coronavirus-national\&region=TOP_BANNER\&variant=undefined\&block=storyline_menu_recirc\&action=click\&pgtype=Article\&impression_id=a487c292-e108-11ea-b959-2b70faf7b631}{Maps
  and Cases}
\item
  \href{https://www.nytimes3xbfgragh.onion/interactive/2020/science/coronavirus-vaccine-tracker.html?name=styln-coronavirus-national\&region=TOP_BANNER\&variant=undefined\&block=storyline_menu_recirc\&action=click\&pgtype=Article\&impression_id=a487c293-e108-11ea-b959-2b70faf7b631}{Vaccine
  Tracker}
\item
  \href{https://www.nytimes3xbfgragh.onion/2020/08/17/us/k-12-schools-reopening.html?name=styln-coronavirus-national\&region=TOP_BANNER\&variant=undefined\&block=storyline_menu_recirc\&action=click\&pgtype=Article\&impression_id=a487c294-e108-11ea-b959-2b70faf7b631}{State
  of Play for K-12}
\item
  \href{https://www.nytimes3xbfgragh.onion/live/2020/08/17/business/stock-market-today-coronavirus?name=styln-coronavirus-national\&region=TOP_BANNER\&variant=undefined\&block=storyline_menu_recirc\&action=click\&pgtype=Article\&impression_id=a487c295-e108-11ea-b959-2b70faf7b631}{Markets
  \& Economy}
\end{itemize}

Advertisement

\protect\hyperlink{after-top}{Continue reading the main story}

Supported by

\protect\hyperlink{after-sponsor}{Continue reading the main story}

\hypertarget{for-maine-lobstermen-a-perfect-storm-threatens-the-summer-season}{%
\section{For Maine Lobstermen, a Perfect Storm Threatens the Summer
Season}\label{for-maine-lobstermen-a-perfect-storm-threatens-the-summer-season}}

The state's lobster industry, already struggling before the virus, could
be crippled as tourism dries up, leaving boatloads of crustaceans and no
one to eat them.

\includegraphics{https://static01.graylady3jvrrxbe.onion/images/2020/07/04/us/politics/00JPdc-lobster4-print/merlin_174036114_11514c6e-0e04-4f90-a185-4dcd1132cae0-articleLarge.jpg?quality=75\&auto=webp\&disable=upscale}

\href{https://www.nytimes3xbfgragh.onion/by/thomas-gibbons-neff}{\includegraphics{https://static01.graylady3jvrrxbe.onion/images/2018/07/12/multimedia/author-thomas-gibbons-neff/author-thomas-gibbons-neff-thumbLarge.png}}

By
\href{https://www.nytimes3xbfgragh.onion/by/thomas-gibbons-neff}{Thomas
Gibbons-Neff}

\begin{itemize}
\item
  July 3, 2020
\item
  \begin{itemize}
  \item
  \item
  \item
  \item
  \item
  \end{itemize}
\end{itemize}

OFF THE COAST OF MAINE --- As he pulled alongside one of his lobster
pots, marked by a red and yellow buoy on the Penobscot Bay, Mike
Hutchings extracted and measured several of the crustaceans that would
contribute to his 130-pound catch that day. It was a decent haul but his
assessment of the fishing season was grim: ``The worst it's ever been.''

Mr. Hutchings's catch on the final Saturday in June came as the lobster
trade approached its money-making time. With the Fourth of July holiday
around the corner, Mr. Hutchings and his fellow lobstermen were supposed
to be gearing up for a major payday as out-of-staters, cruise ships,
warmer weather and bounties of lobsters, having just molted their shells
and been lured into the thousands of traps anchored on the rocky bottom
of Maine's coastal waters, came together in a seasonal windfall.

But like many businesses across the country, the Maine lobster industry,
which makes up the bulk of the fishing revenue the state brings in every
year, is being battered by the coronavirus, which has crushed the
tourism trade that Mr. Hutchings and his fellow fishermen rely on for a
living.

With fewer tourists expected to descend in search of lobster rolls, the
immediate problem for Mr. Hutchings is simple: too many lobsters and not
enough people to eat them. That has sent the price of lobsters plunging.

Image

The price of lobster has plunged this year.Credit...Tristan Spinski for
The New York Times

Image

Lobster bait on Mr. Hutchings's boat.Credit...Tristan Spinski for The
New York Times

The pain is particularly unwelcome for an industry that has spent the
past several years caught in the middle of political fights, including
President Trump's trade war with China, looming restrictions to protect
an endangered whale species and bait quotas. And then there are the
region's warming waters, spurred by climate change, which have slowly
shifted the areas conducive to lobster reproduction
\href{https://www.nytimes3xbfgragh.onion/2018/06/21/climate/maine-lobsters.html}{away
from the coast}.

The effect of the virus on the Maine lobster trade is the latest
indication of how the disease is upending nearly all corners of business
activity and inflicting economic pain poised to last longer than many
had predicted. Last month, after groups of fishermen outlined their
concerns for Mr. Trump at an event in Bangor, Maine, the president
directed the Agriculture Department to provide federal assistance to
lobster harvesters.

But that assistance, which has yet to be detailed or allocated, may come
too late.

More than 30 million people typically visit Maine each year. The
majority come in the summer months for the pleasant air of coastal New
England, as well as for the lobster, a high-priced specialty that is a
staple of tourist meals.

But the normal influx of visitors has been derailed by the virus, which
is surging in some parts of the country, contributing to the general
unease many Americans share when it comes to traveling. Further
compounding the situation are the quarantine restrictions that Gov.
Janet T. Mills, a Democrat, put in place for out-of-state travelers.
(Maine has had about 3,300 virus cases, one of the lowest numbers in the
country, according to
\href{https://www.nytimes3xbfgragh.onion/interactive/2020/us/maine-coronavirus-cases.html}{data
compiled by The New York Times}.)

\includegraphics{https://static01.graylady3jvrrxbe.onion/images/2020/07/04/us/politics/00JPdc-lobster1-print/merlin_174035634_837a566e-74b9-465b-b044-8c1661351ec8-articleLarge.jpg?quality=75\&auto=webp\&disable=upscale}

Unlike previous years, this summer will bring no cruise ships and few
``no vacancy'' signs. The typical rainbow of out-of-state license plates
idling in bumper-to-bumper traffic on the bridge to Wiscasset is
unlikely to materialize.

\hypertarget{latest-updates-the-coronavirus-outbreak}{%
\section{\texorpdfstring{\href{https://www.nytimes3xbfgragh.onion/2020/08/17/world/coronavirus-covid.html?action=click\&pgtype=Article\&state=default\&region=MAIN_CONTENT_1\&context=storylines_live_updates}{Latest
Updates: The Coronavirus
Outbreak}}{Latest Updates: The Coronavirus Outbreak}}\label{latest-updates-the-coronavirus-outbreak}}

Updated 2020-08-18T03:48:43.730Z

\begin{itemize}
\tightlist
\item
  \href{https://www.nytimes3xbfgragh.onion/2020/08/17/world/coronavirus-covid.html?action=click\&pgtype=Article\&state=default\&region=MAIN_CONTENT_1\&context=storylines_live_updates\#link-6fdbc8ef}{U.S.
  college campuses grapple with coronavirus fears, outbreaks and
  protests.}
\item
  \href{https://www.nytimes3xbfgragh.onion/2020/08/17/world/coronavirus-covid.html?action=click\&pgtype=Article\&state=default\&region=MAIN_CONTENT_1\&context=storylines_live_updates\#link-7e47207}{For
  primary and secondary school students and staff, it's been a difficult
  back-to-school season.}
\item
  \href{https://www.nytimes3xbfgragh.onion/2020/08/17/world/coronavirus-covid.html?action=click\&pgtype=Article\&state=default\&region=MAIN_CONTENT_1\&context=storylines_live_updates\#link-44c3fee2}{At
  the Democratic National Convention, Cuomo and others assail Trump's
  handling of the virus.}
\end{itemize}

\href{https://www.nytimes3xbfgragh.onion/2020/08/17/world/coronavirus-covid.html?action=click\&pgtype=Article\&state=default\&region=MAIN_CONTENT_1\&context=storylines_live_updates}{See
more updates}

More live coverage:
\href{https://www.nytimes3xbfgragh.onion/live/2020/08/17/business/stock-market-today-coronavirus?action=click\&pgtype=Article\&state=default\&region=MAIN_CONTENT_1\&context=storylines_live_updates}{Markets}

For Mr. Hutchings, 66, whose hands are worn from both line and lobster
after fishing Maine waters for more than 50 years, the effect of the
pandemic boils down to whether he can make enough money to keep his boat
profitable.

His expenses include bait, fuel and his crew's wages. And as the cost of
a pound of lobster steadily drops, he has been weighing almost daily
whether to leave his harbor in Lincolnville for good.

``If the price gets so low, I won't go,'' Mr. Hutchings said, standing
as his stern man, Eddie Hustus, quickly moved herring and pogies into
mesh bait bags below his boat. ``I'm not going to do it for nothing.''

In the waning days of June, Mr. Hutchings said he was selling the more
costly hard-shell lobsters at around \$4.50 a pound, roughly half of
what he was able to get for them a year ago. In Lincolnville Harbor,
only three of eight boats in the cove had lobster traps in the water,
Mr. Hutchings explained. The captains of the others were patiently
waiting to see how prices shift.

Image

Raising a lobster trap.Credit...Tristan Spinski for The New York Times

Image

An undersize lobster being returned to the ocean.Credit...Tristan
Spinski for The New York Times

The economic hit to lobstermen seemed a far cry from Mr. Trump's
declaration on social media just a few days earlier.

``Pres. Obama destroyed the lobster and fishing industry in Maine. Now
it's back, bigger and better than anyone ever thought possible,''
\href{https://twitter.com/realDonaldTrump/status/1275951862269829121}{the
president said on Twitter}. ``Enjoy your `lobstering' and fishing! Make
lots of money!''

Maine's lobster industry hit its peak in 2016, the last year of
President Barack Obama's second term, with 132 million pounds caught at
a value of \$540 million, according to state data. Maine's fishermen
sold less than \$500 million during each of the first three years of the
Trump administration, on par with Mr. Obama's first term. In 2019, a
particularly bad haul pushed the price per pound of lobster to \$4.82,
the highest since Maine
\href{https://www.maine.gov/dmr/commercial-fishing/landings/documents/lobster.table.pdf}{began
recording the data in 1880}.

After his meeting in Bangor last month, Mr. Trump issued a proclamation
directing the agriculture secretary to find ways to assist the lobster
industry, which he said had been unfairly targeted with retaliatory
tariffs by China.

Image

The usual influx of summer tourists in Maine has been derailed by the
coronavirus pandemic.Credit...Tristan Spinski for The New York Times

``From 2015 to 2018, American lobster was the most valuable single
seafood species harvested in the United States, with Maine accounting
for approximately 80 percent of that value each year,'' Mr. Trump said
in the proclamation, adding that his administration would ``mitigate the
effects of unfair retaliatory trade practices on this important
industry.''

Mr. Hutchings, who supports Mr. Trump, called the Bangor event ``a photo
op,'' but said he appreciated the president's decision to sit down in
Maine with the fishing industry, which he believed to be a presidential
first, at least in his lifetime.

``Whether something good comes out of it, who knows,'' Mr. Hutchings
said.

Mr. Trump's move to help Maine fishermen is aimed at strengthening his
blue-collar bona fides during an election year. Yet for those affected,
no number of presidential round tables adorned with lobster traps is
likely to change what could be a terrible summer for Maine's fisheries.

``I think there's obviously a lot of uncertainty for local businesses
and a lot of concern for fishermen and for everyone else who relies on
tourist business,'' said Marianne LaCroix, the executive director of the
Maine Lobster Marketing Collaborative.

Image

Mr. Hutchings's expenses include bait, fuel and his crew's
wages.Credit...Tristan Spinski for The New York Times

Image

``If you're a fisherman, you have to make it work,'' Mr. Hutchings
said.Credit...Tristan Spinski for The New York Times

Raymond Young, 55, a third-generation lobsterman who grew up putting
wood plugs in the claws of crustaceans and owns Young's Lobster Pound, a
Belfast, Maine, staple, has spent the past several years trying to
adjust his business as Mr. Trump's trade policies changed the market.

\href{https://www.nytimes3xbfgragh.onion/news-event/coronavirus?action=click\&pgtype=Article\&state=default\&region=MAIN_CONTENT_3\&context=storylines_faq}{}

\hypertarget{the-coronavirus-outbreak-}{%
\subsubsection{The Coronavirus Outbreak
›}\label{the-coronavirus-outbreak-}}

\hypertarget{frequently-asked-questions}{%
\paragraph{Frequently Asked
Questions}\label{frequently-asked-questions}}

Updated August 17, 2020

\begin{itemize}
\item ~
  \hypertarget{why-does-standing-six-feet-away-from-others-help}{%
  \paragraph{Why does standing six feet away from others
  help?}\label{why-does-standing-six-feet-away-from-others-help}}

  \begin{itemize}
  \tightlist
  \item
    The coronavirus spreads primarily through droplets from your mouth
    and nose, especially when you cough or sneeze. The C.D.C., one of
    the organizations using that measure,
    \href{https://www.nytimes3xbfgragh.onion/2020/04/14/health/coronavirus-six-feet.html?action=click\&pgtype=Article\&state=default\&region=MAIN_CONTENT_3\&context=storylines_faq}{bases
    its recommendation of six feet} on the idea that most large droplets
    that people expel when they cough or sneeze will fall to the ground
    within six feet. But six feet has never been a magic number that
    guarantees complete protection. Sneezes, for instance, can launch
    droplets a lot farther than six feet,
    \href{https://jamanetwork.com/journals/jama/fullarticle/2763852}{according
    to a recent study}. It's a rule of thumb: You should be safest
    standing six feet apart outside, especially when it's windy. But
    keep a mask on at all times, even when you think you're far enough
    apart.
  \end{itemize}
\item ~
  \hypertarget{i-have-antibodies-am-i-now-immune}{%
  \paragraph{I have antibodies. Am I now
  immune?}\label{i-have-antibodies-am-i-now-immune}}

  \begin{itemize}
  \tightlist
  \item
    As of right
    now,\href{https://www.nytimes3xbfgragh.onion/2020/07/22/health/covid-antibodies-herd-immunity.html?action=click\&pgtype=Article\&state=default\&region=MAIN_CONTENT_3\&context=storylines_faq}{that
    seems likely, for at least several months.} There have been
    frightening accounts of people suffering what seems to be a second
    bout of Covid-19. But experts say these patients may have a
    drawn-out course of infection, with the virus taking a slow toll
    weeks to months after initial exposure. People infected with the
    coronavirus typically
    \href{https://www.nature.com/articles/s41586-020-2456-9}{produce}
    immune molecules called antibodies, which are
    \href{https://www.nytimes3xbfgragh.onion/2020/05/07/health/coronavirus-antibody-prevalence.html?action=click\&pgtype=Article\&state=default\&region=MAIN_CONTENT_3\&context=storylines_faq}{protective
    proteins made in response to an
    infection}\href{https://www.nytimes3xbfgragh.onion/2020/05/07/health/coronavirus-antibody-prevalence.html?action=click\&pgtype=Article\&state=default\&region=MAIN_CONTENT_3\&context=storylines_faq}{.
    These antibodies may} last in the body
    \href{https://www.nature.com/articles/s41591-020-0965-6}{only two to
    three months}, which may seem worrisome, but that's perfectly normal
    after an acute infection subsides, said Dr. Michael Mina, an
    immunologist at Harvard University. It may be possible to get the
    coronavirus again, but it's highly unlikely that it would be
    possible in a short window of time from initial infection or make
    people sicker the second time.
  \end{itemize}
\item ~
  \hypertarget{im-a-small-business-owner-can-i-get-relief}{%
  \paragraph{I'm a small-business owner. Can I get
  relief?}\label{im-a-small-business-owner-can-i-get-relief}}

  \begin{itemize}
  \tightlist
  \item
    The
    \href{https://www.nytimes3xbfgragh.onion/article/small-business-loans-stimulus-grants-freelancers-coronavirus.html?action=click\&pgtype=Article\&state=default\&region=MAIN_CONTENT_3\&context=storylines_faq}{stimulus
    bills enacted in March} offer help for the millions of American
    small businesses. Those eligible for aid are businesses and
    nonprofit organizations with fewer than 500 workers, including sole
    proprietorships, independent contractors and freelancers. Some
    larger companies in some industries are also eligible. The help
    being offered, which is being managed by the Small Business
    Administration, includes the Paycheck Protection Program and the
    Economic Injury Disaster Loan program. But lots of folks have
    \href{https://www.nytimes3xbfgragh.onion/interactive/2020/05/07/business/small-business-loans-coronavirus.html?action=click\&pgtype=Article\&state=default\&region=MAIN_CONTENT_3\&context=storylines_faq}{not
    yet seen payouts.} Even those who have received help are confused:
    The rules are draconian, and some are stuck sitting on
    \href{https://www.nytimes3xbfgragh.onion/2020/05/02/business/economy/loans-coronavirus-small-business.html?action=click\&pgtype=Article\&state=default\&region=MAIN_CONTENT_3\&context=storylines_faq}{money
    they don't know how to use.} Many small-business owners are getting
    less than they expected or
    \href{https://www.nytimes3xbfgragh.onion/2020/06/10/business/Small-business-loans-ppp.html?action=click\&pgtype=Article\&state=default\&region=MAIN_CONTENT_3\&context=storylines_faq}{not
    hearing anything at all.}
  \end{itemize}
\item ~
  \hypertarget{what-are-my-rights-if-i-am-worried-about-going-back-to-work}{%
  \paragraph{What are my rights if I am worried about going back to
  work?}\label{what-are-my-rights-if-i-am-worried-about-going-back-to-work}}

  \begin{itemize}
  \tightlist
  \item
    Employers have to provide
    \href{https://www.osha.gov/SLTC/covid-19/standards.html}{a safe
    workplace} with policies that protect everyone equally.
    \href{https://www.nytimes3xbfgragh.onion/article/coronavirus-money-unemployment.html?action=click\&pgtype=Article\&state=default\&region=MAIN_CONTENT_3\&context=storylines_faq}{And
    if one of your co-workers tests positive for the coronavirus, the
    C.D.C.} has said that
    \href{https://www.cdc.gov/coronavirus/2019-ncov/community/guidance-business-response.html}{employers
    should tell their employees} -\/- without giving you the sick
    employee's name -\/- that they may have been exposed to the virus.
  \end{itemize}
\item ~
  \hypertarget{what-is-school-going-to-look-like-in-september}{%
  \paragraph{What is school going to look like in
  September?}\label{what-is-school-going-to-look-like-in-september}}

  \begin{itemize}
  \tightlist
  \item
    It is unlikely that many schools will return to a normal schedule
    this fall, requiring the grind of
    \href{https://www.nytimes3xbfgragh.onion/2020/06/05/us/coronavirus-education-lost-learning.html?action=click\&pgtype=Article\&state=default\&region=MAIN_CONTENT_3\&context=storylines_faq}{online
    learning},
    \href{https://www.nytimes3xbfgragh.onion/2020/05/29/us/coronavirus-child-care-centers.html?action=click\&pgtype=Article\&state=default\&region=MAIN_CONTENT_3\&context=storylines_faq}{makeshift
    child care} and
    \href{https://www.nytimes3xbfgragh.onion/2020/06/03/business/economy/coronavirus-working-women.html?action=click\&pgtype=Article\&state=default\&region=MAIN_CONTENT_3\&context=storylines_faq}{stunted
    workdays} to continue. California's two largest public school
    districts --- Los Angeles and San Diego --- said on July 13, that
    \href{https://www.nytimes3xbfgragh.onion/2020/07/13/us/lausd-san-diego-school-reopening.html?action=click\&pgtype=Article\&state=default\&region=MAIN_CONTENT_3\&context=storylines_faq}{instruction
    will be remote-only in the fall}, citing concerns that surging
    coronavirus infections in their areas pose too dire a risk for
    students and teachers. Together, the two districts enroll some
    825,000 students. They are the largest in the country so far to
    abandon plans for even a partial physical return to classrooms when
    they reopen in August. For other districts, the solution won't be an
    all-or-nothing approach.
    \href{https://bioethics.jhu.edu/research-and-outreach/projects/eschool-initiative/school-policy-tracker/}{Many
    systems}, including the nation's largest, New York City, are
    devising
    \href{https://www.nytimes3xbfgragh.onion/2020/06/26/us/coronavirus-schools-reopen-fall.html?action=click\&pgtype=Article\&state=default\&region=MAIN_CONTENT_3\&context=storylines_faq}{hybrid
    plans} that involve spending some days in classrooms and other days
    online. There's no national policy on this yet, so check with your
    municipal school system regularly to see what is happening in your
    community.
  \end{itemize}
\end{itemize}

Beijing's retaliatory tariffs on American lobster nearly crippled
exports from wholesalers like Mr. Young. Maine lobster exports to China
fell by 48.24 percent in 2019.

Canadian resellers have stepped in, buying shellfish from Maine
wholesalers, albeit at a lower price, before
\href{https://www.nytimes3xbfgragh.onion/2017/11/12/business/trump-trade-lobster-canada.html}{sending
it to international markets} such as China and Europe. Mr. Trump has
criticized Europe for charging a higher tariff on American lobsters than
those from Canada, but that difference stems from a trade agreement the
European Union and Canada signed in 2016, which lowered European tariffs
on Canadian products.

Mr. Young's excess lobsters often go to the Canadian freezer plants at
the end of the season, but this year, with low sales and the
coronavirus, his buyer's plants are already full, he said.

``Last month we were trading an old dollar for a new one,'' Mr. Young
said, noting that he did not expect to receive federal aid anytime soon.
``If the tourists aren't here and we can't ship the other product to
Canada because they're full, it's going to be a different year as we try
and find a home for some of this stuff.''

Image

The shop where Mr. Hutchings sells lobsters directly to customers. Mr.
Hutchings estimated he was getting roughly half as much per pound for
the more costly hard shell lobsters than he was at this time last
year.Credit...Tristan Spinski for The New York Times

A good season for Mr. Young means roughly 20 boats from a constellation
of nearby towns like Searsport, Stockton Springs and Northport are
selling their catch to him. So far this year, he has just two boats,
leaving a glimmer of hope that fewer vessels on the water will translate
to a smaller lobster yield and higher prices.

Mr. Hutchings's 40-foot, Canadian-built, diesel-powered lobster boat,
Fundy Spray, is one of those two boats. And on Saturday, Mr. Hutchings
said he had decided to put the entirety of his 800 traps in the water
this season just in case those prices do turn.

Adjusting his camouflage ball cap adorned with ``Young's Lobster Pound''
atop his mop of white hair, Mr. Hutchings maneuvered his boat back
toward Lincolnville's harbor.

The wind picked up and the sun was out. The deck was covered in seaweed.
Several small crabs scurried among the ocean detritus along with the red
rubber bands that did not quite make it onto a lobster's claw. Over the
rhythmic churn of his boat's engine and the occasional chatter from the
marine radio, Mr. Hutchings muttered what could easily have been a Maine
mantra.

``If you're a fisherman, you have to make it work,'' Mr. Hutchings said.
``It's what you do.''

Ana Swanson contributed reporting from Washington.

Advertisement

\protect\hyperlink{after-bottom}{Continue reading the main story}

\hypertarget{site-index}{%
\subsection{Site Index}\label{site-index}}

\hypertarget{site-information-navigation}{%
\subsection{Site Information
Navigation}\label{site-information-navigation}}

\begin{itemize}
\tightlist
\item
  \href{https://help.nytimes3xbfgragh.onion/hc/en-us/articles/115014792127-Copyright-notice}{©~2020~The
  New York Times Company}
\end{itemize}

\begin{itemize}
\tightlist
\item
  \href{https://www.nytco.com/}{NYTCo}
\item
  \href{https://help.nytimes3xbfgragh.onion/hc/en-us/articles/115015385887-Contact-Us}{Contact
  Us}
\item
  \href{https://www.nytco.com/careers/}{Work with us}
\item
  \href{https://nytmediakit.com/}{Advertise}
\item
  \href{http://www.tbrandstudio.com/}{T Brand Studio}
\item
  \href{https://www.nytimes3xbfgragh.onion/privacy/cookie-policy\#how-do-i-manage-trackers}{Your
  Ad Choices}
\item
  \href{https://www.nytimes3xbfgragh.onion/privacy}{Privacy}
\item
  \href{https://help.nytimes3xbfgragh.onion/hc/en-us/articles/115014893428-Terms-of-service}{Terms
  of Service}
\item
  \href{https://help.nytimes3xbfgragh.onion/hc/en-us/articles/115014893968-Terms-of-sale}{Terms
  of Sale}
\item
  \href{https://spiderbites.nytimes3xbfgragh.onion}{Site Map}
\item
  \href{https://help.nytimes3xbfgragh.onion/hc/en-us}{Help}
\item
  \href{https://www.nytimes3xbfgragh.onion/subscription?campaignId=37WXW}{Subscriptions}
\end{itemize}
