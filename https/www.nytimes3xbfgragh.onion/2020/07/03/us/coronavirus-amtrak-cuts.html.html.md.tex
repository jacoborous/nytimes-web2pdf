Sections

SEARCH

\protect\hyperlink{site-content}{Skip to
content}\protect\hyperlink{site-index}{Skip to site index}

\href{https://www.nytimes3xbfgragh.onion/section/us}{U.S.}

\href{https://myaccount.nytimes3xbfgragh.onion/auth/login?response_type=cookie\&client_id=vi}{}

\href{https://www.nytimes3xbfgragh.onion/section/todayspaper}{Today's
Paper}

\href{/section/us}{U.S.}\textbar{}Critics Fear Amtrak Is Using Pandemic
to Cut Service That Won't Return

\url{https://nyti.ms/3eYgZhA}

\begin{itemize}
\item
\item
\item
\item
\item
\end{itemize}

\hypertarget{the-coronavirus-outbreak}{%
\subsubsection{\texorpdfstring{\href{https://www.nytimes3xbfgragh.onion/news-event/coronavirus?name=styln-coronavirus-national\&region=TOP_BANNER\&variant=undefined\&block=storyline_menu_recirc\&action=click\&pgtype=Article\&impression_id=48f69fa0-e3a6-11ea-92a9-439cc713b6c5}{The
Coronavirus
Outbreak}}{The Coronavirus Outbreak}}\label{the-coronavirus-outbreak}}

\begin{itemize}
\tightlist
\item
  live\href{https://www.nytimes3xbfgragh.onion/2020/08/21/world/covid-19-coronavirus.html?name=styln-coronavirus-national\&region=TOP_BANNER\&variant=undefined\&block=storyline_menu_recirc\&action=click\&pgtype=Article\&impression_id=48f69fa1-e3a6-11ea-92a9-439cc713b6c5}{Latest
  Updates}
\item
  \href{https://www.nytimes3xbfgragh.onion/interactive/2020/us/coronavirus-us-cases.html?name=styln-coronavirus-national\&region=TOP_BANNER\&variant=undefined\&block=storyline_menu_recirc\&action=click\&pgtype=Article\&impression_id=48f69fa2-e3a6-11ea-92a9-439cc713b6c5}{Maps
  and Cases}
\item
  \href{https://www.nytimes3xbfgragh.onion/interactive/2020/science/coronavirus-vaccine-tracker.html?name=styln-coronavirus-national\&region=TOP_BANNER\&variant=undefined\&block=storyline_menu_recirc\&action=click\&pgtype=Article\&impression_id=48f69fa3-e3a6-11ea-92a9-439cc713b6c5}{Vaccine
  Tracker}
\item
  \href{https://www.nytimes3xbfgragh.onion/2020/08/19/us/colleges-closing-covid.html?name=styln-coronavirus-national\&region=TOP_BANNER\&variant=undefined\&block=storyline_menu_recirc\&action=click\&pgtype=Article\&impression_id=48f69fa4-e3a6-11ea-92a9-439cc713b6c5}{Colleges
  Closing}
\item
  \href{https://www.nytimes3xbfgragh.onion/live/2020/08/20/business/stock-market-today-coronavirus?name=styln-coronavirus-national\&region=TOP_BANNER\&variant=undefined\&block=storyline_menu_recirc\&action=click\&pgtype=Article\&impression_id=48f69fa5-e3a6-11ea-92a9-439cc713b6c5}{Economy}
\end{itemize}

Advertisement

\protect\hyperlink{after-top}{Continue reading the main story}

Supported by

\protect\hyperlink{after-sponsor}{Continue reading the main story}

\hypertarget{critics-fear-amtrak-is-using-pandemic-to-cut-service-that-wont-return}{%
\section{Critics Fear Amtrak Is Using Pandemic to Cut Service That Won't
Return}\label{critics-fear-amtrak-is-using-pandemic-to-cut-service-that-wont-return}}

Members of Congress are angry that the rail agency is asking for more
money while planning to lay off staff and reduce services. Smaller
communities that rely on rail service could be badly hurt.

\includegraphics{https://static01.graylady3jvrrxbe.onion/images/2020/07/02/us/politics/02dc-virus-amtrak/merlin_171986655_8ce0666b-a7a6-47b3-afaa-2f89e68321d3-articleLarge.jpg?quality=75\&auto=webp\&disable=upscale}

By Pranshu Verma

\begin{itemize}
\item
  July 3, 2020
\item
  \begin{itemize}
  \item
  \item
  \item
  \item
  \item
  \end{itemize}
\end{itemize}

WASHINGTON --- Amtrak has long wanted to cut back on long-distance train
routes that span America's heartland, but political pressure from
Congress made it next to impossible.

Then came the coronavirus.

Since March, the pandemic has
\href{https://www.nytimes3xbfgragh.onion/interactive/2020/us/coronavirus-us-cases.html}{killed
over 128,000 Americans} and exacted a higher financial toll on the
transportation industry than the attacks of Sept. 11, 2001. Amtrak has
not been spared, with an internal analysis showing a 95 percent drop in
ridership, and revenue projected to fall by 50 percent in 2021.

In an effort to stay afloat, the rail agency announced last month it
would cut up to 20 percent of its work force by October. It also will
suspend daily service on long-distance train routes that service over
220 communities across the country, according to industry experts who
analyzed the effect of proposed cuts to national train routes when the
Trump administration tried to gut the service in 2019.

The moves have been met with intense skepticism, and even anger.

Amtrak received letters from 16 senators last week asking why it needed
to enact such steep cuts since it had already received \$1 billion in
emergency aid. The agency had also requested nearly \$1.5 billion in
additional funding on top of its standard \$2 billion budget request for
2021.

``We are deeply concerned by the downsizing plan,'' a bipartisan
coalition of seven senators, led by Steve Daines, Republican of Montana,
said in its letter to Amtrak. ``These cuts would not only dramatically
reduce the utility of the nation's passenger rail network, but would
also ignore congressional intent to expedite economic recovery following
the pandemic.''

Critics argue Amtrak's cutbacks are not a surprise, and fall in line
with the agency's recent desires to prioritize profitability and the
reduction of long-distance routes. The rail agency receives federal
funds but is independently run.

``I fear that the Covid-19 pandemic is convenient reasoning to do that
which Amtrak, over the past several years, has determined to do,'' said
John Robert Smith, a former chairman of Amtrak's board, ``and that is to
dismantle the national system.''

\hypertarget{latest-updates-the-coronavirus-outbreak}{%
\section{\texorpdfstring{\href{https://www.nytimes3xbfgragh.onion/2020/08/21/world/covid-19-coronavirus.html?action=click\&pgtype=Article\&state=default\&region=MAIN_CONTENT_1\&context=storylines_live_updates}{Latest
Updates: The Coronavirus
Outbreak}}{Latest Updates: The Coronavirus Outbreak}}\label{latest-updates-the-coronavirus-outbreak}}

Updated 2020-08-21T11:05:09.310Z

\begin{itemize}
\tightlist
\item
  \href{https://www.nytimes3xbfgragh.onion/2020/08/21/world/covid-19-coronavirus.html?action=click\&pgtype=Article\&state=default\&region=MAIN_CONTENT_1\&context=storylines_live_updates\#link-4690b6aa}{Shutdowns,
  warnings and scoldings follow gatherings on college campuses.}
\item
  \href{https://www.nytimes3xbfgragh.onion/2020/08/21/world/covid-19-coronavirus.html?action=click\&pgtype=Article\&state=default\&region=MAIN_CONTENT_1\&context=storylines_live_updates\#link-324af071}{As
  he accepts the Democratic nomination, Biden knocks Trump's pandemic
  response.}
\item
  \href{https://www.nytimes3xbfgragh.onion/2020/08/21/world/covid-19-coronavirus.html?action=click\&pgtype=Article\&state=default\&region=MAIN_CONTENT_1\&context=storylines_live_updates\#link-35890b73}{Hundreds
  of doctors in Kenya go on strike over their pay and protective gear.}
\end{itemize}

\href{https://www.nytimes3xbfgragh.onion/2020/08/21/world/covid-19-coronavirus.html?action=click\&pgtype=Article\&state=default\&region=MAIN_CONTENT_1\&context=storylines_live_updates}{See
more updates}

More live coverage:
\href{https://www.nytimes3xbfgragh.onion/live/2020/08/20/business/stock-market-today-coronavirus?action=click\&pgtype=Article\&state=default\&region=MAIN_CONTENT_1\&context=storylines_live_updates}{Markets}

A spokesman for Amtrak, Jason Abrams, said in a statement that the
agency was cutting back service ``to better align our service with
demand'' during the pandemic.

``We are fully committed to returning this service when demand returns
and with adequate funding,'' he said, adding: ``Amtrak is adjusting its
work force to better match this reduced demand. As a first step, we are
offering employees a voluntary incentive package.''

Supporters of continued rail service say Amtrak's decision will be
particularly damaging to small communities where the train provides an
economic lifeline and may be the only transit option available to
residents.

They also point out that ridership on long-distance routes in May
dropped by 78 percent compared with the same period last year, while
ridership on shorter routes fell more than 90 percent.

``A lot of people can't fly, they just can't get on the plane,'' said
Sean Jeans-Gail, the vice president at the Rail Passengers Association.
``Amtrak is the only way they can get around. Those users of the system
have a right to be very concerned.''

On Monday, the rail agency
\href{https://www.usatoday.com/story/travel/2020/06/29/amtrak-reduce-new-york-florida-trains-starting-july-6/3254400001/}{will
reduce} service from New York to Miami to four days a week rather than
each day. Up to 20 percent of the work force, or nearly 4,000 employees,
could be let go by October. Most of the network's 15 long-distance train
routes will see daily service cut to three days a week starting in the
fall. Even Amtrak's busiest route in the Northeast Corridor will have
fewer trains serving passengers every day.

Amtrak is prepared to cut its operating costs by \$500 million in its
2021 fiscal year, which starts in October, anticipating ridership will
not return to prepandemic levels because of ``changed behaviors, such as
increased telework and reduced discretionary income,'' according to a
letter William J. Flynn, Amtrak's new chief executive, wrote to Congress
in May.

But the agency recognizes that the future is hard to predict.

``Generating estimates of future demand is incredibly challenging, given
the unprecedented nature of our current circumstances and the unknowable
trajectory of the pandemic,'' Mr. Flynn wrote. ``Most demand predictions
anticipate a second wave of Covid-19 impacts in the fall, and that
further complicates any forecast.''

Many in Congress are skeptical, and are asking the rail agency to
provide stronger proof that such drastic action is needed even with
significant emergency funding from Congress.

``We would like to ensure that reductions in frequencies for
long-distance routes do not unnecessarily extend beyond the Covid-19
crisis,'' a group of five Republican senators, led by Roger Wicker of
Mississippi, wrote last week in a letter obtained by The New York Times.
``What data is Amtrak using to justify reductions in service for all of
fiscal year 2021?''

Since Amtrak was created in 1971 by Congress to be the nation's rail
operator, it has lost money. Federal subsidies and payments from states
have allowed the network to remain operational, and it has become an
integral mode of transport in corridors like the Northeast and Southern
California.

\href{https://www.nytimes3xbfgragh.onion/news-event/coronavirus?action=click\&pgtype=Article\&state=default\&region=MAIN_CONTENT_3\&context=storylines_faq}{}

\hypertarget{the-coronavirus-outbreak-}{%
\subsubsection{The Coronavirus Outbreak
›}\label{the-coronavirus-outbreak-}}

\hypertarget{frequently-asked-questions}{%
\paragraph{Frequently Asked
Questions}\label{frequently-asked-questions}}

Updated August 17, 2020

\begin{itemize}
\item ~
  \hypertarget{why-does-standing-six-feet-away-from-others-help}{%
  \paragraph{Why does standing six feet away from others
  help?}\label{why-does-standing-six-feet-away-from-others-help}}

  \begin{itemize}
  \tightlist
  \item
    The coronavirus spreads primarily through droplets from your mouth
    and nose, especially when you cough or sneeze. The C.D.C., one of
    the organizations using that measure,
    \href{https://www.nytimes3xbfgragh.onion/2020/04/14/health/coronavirus-six-feet.html?action=click\&pgtype=Article\&state=default\&region=MAIN_CONTENT_3\&context=storylines_faq}{bases
    its recommendation of six feet} on the idea that most large droplets
    that people expel when they cough or sneeze will fall to the ground
    within six feet. But six feet has never been a magic number that
    guarantees complete protection. Sneezes, for instance, can launch
    droplets a lot farther than six feet,
    \href{https://jamanetwork.com/journals/jama/fullarticle/2763852}{according
    to a recent study}. It's a rule of thumb: You should be safest
    standing six feet apart outside, especially when it's windy. But
    keep a mask on at all times, even when you think you're far enough
    apart.
  \end{itemize}
\item ~
  \hypertarget{i-have-antibodies-am-i-now-immune}{%
  \paragraph{I have antibodies. Am I now
  immune?}\label{i-have-antibodies-am-i-now-immune}}

  \begin{itemize}
  \tightlist
  \item
    As of right
    now,\href{https://www.nytimes3xbfgragh.onion/2020/07/22/health/covid-antibodies-herd-immunity.html?action=click\&pgtype=Article\&state=default\&region=MAIN_CONTENT_3\&context=storylines_faq}{that
    seems likely, for at least several months.} There have been
    frightening accounts of people suffering what seems to be a second
    bout of Covid-19. But experts say these patients may have a
    drawn-out course of infection, with the virus taking a slow toll
    weeks to months after initial exposure. People infected with the
    coronavirus typically
    \href{https://www.nature.com/articles/s41586-020-2456-9}{produce}
    immune molecules called antibodies, which are
    \href{https://www.nytimes3xbfgragh.onion/2020/05/07/health/coronavirus-antibody-prevalence.html?action=click\&pgtype=Article\&state=default\&region=MAIN_CONTENT_3\&context=storylines_faq}{protective
    proteins made in response to an
    infection}\href{https://www.nytimes3xbfgragh.onion/2020/05/07/health/coronavirus-antibody-prevalence.html?action=click\&pgtype=Article\&state=default\&region=MAIN_CONTENT_3\&context=storylines_faq}{.
    These antibodies may} last in the body
    \href{https://www.nature.com/articles/s41591-020-0965-6}{only two to
    three months}, which may seem worrisome, but that's perfectly normal
    after an acute infection subsides, said Dr. Michael Mina, an
    immunologist at Harvard University. It may be possible to get the
    coronavirus again, but it's highly unlikely that it would be
    possible in a short window of time from initial infection or make
    people sicker the second time.
  \end{itemize}
\item ~
  \hypertarget{im-a-small-business-owner-can-i-get-relief}{%
  \paragraph{I'm a small-business owner. Can I get
  relief?}\label{im-a-small-business-owner-can-i-get-relief}}

  \begin{itemize}
  \tightlist
  \item
    The
    \href{https://www.nytimes3xbfgragh.onion/article/small-business-loans-stimulus-grants-freelancers-coronavirus.html?action=click\&pgtype=Article\&state=default\&region=MAIN_CONTENT_3\&context=storylines_faq}{stimulus
    bills enacted in March} offer help for the millions of American
    small businesses. Those eligible for aid are businesses and
    nonprofit organizations with fewer than 500 workers, including sole
    proprietorships, independent contractors and freelancers. Some
    larger companies in some industries are also eligible. The help
    being offered, which is being managed by the Small Business
    Administration, includes the Paycheck Protection Program and the
    Economic Injury Disaster Loan program. But lots of folks have
    \href{https://www.nytimes3xbfgragh.onion/interactive/2020/05/07/business/small-business-loans-coronavirus.html?action=click\&pgtype=Article\&state=default\&region=MAIN_CONTENT_3\&context=storylines_faq}{not
    yet seen payouts.} Even those who have received help are confused:
    The rules are draconian, and some are stuck sitting on
    \href{https://www.nytimes3xbfgragh.onion/2020/05/02/business/economy/loans-coronavirus-small-business.html?action=click\&pgtype=Article\&state=default\&region=MAIN_CONTENT_3\&context=storylines_faq}{money
    they don't know how to use.} Many small-business owners are getting
    less than they expected or
    \href{https://www.nytimes3xbfgragh.onion/2020/06/10/business/Small-business-loans-ppp.html?action=click\&pgtype=Article\&state=default\&region=MAIN_CONTENT_3\&context=storylines_faq}{not
    hearing anything at all.}
  \end{itemize}
\item ~
  \hypertarget{what-are-my-rights-if-i-am-worried-about-going-back-to-work}{%
  \paragraph{What are my rights if I am worried about going back to
  work?}\label{what-are-my-rights-if-i-am-worried-about-going-back-to-work}}

  \begin{itemize}
  \tightlist
  \item
    Employers have to provide
    \href{https://www.osha.gov/SLTC/covid-19/standards.html}{a safe
    workplace} with policies that protect everyone equally.
    \href{https://www.nytimes3xbfgragh.onion/article/coronavirus-money-unemployment.html?action=click\&pgtype=Article\&state=default\&region=MAIN_CONTENT_3\&context=storylines_faq}{And
    if one of your co-workers tests positive for the coronavirus, the
    C.D.C.} has said that
    \href{https://www.cdc.gov/coronavirus/2019-ncov/community/guidance-business-response.html}{employers
    should tell their employees} -\/- without giving you the sick
    employee's name -\/- that they may have been exposed to the virus.
  \end{itemize}
\item ~
  \hypertarget{what-is-school-going-to-look-like-in-september}{%
  \paragraph{What is school going to look like in
  September?}\label{what-is-school-going-to-look-like-in-september}}

  \begin{itemize}
  \tightlist
  \item
    It is unlikely that many schools will return to a normal schedule
    this fall, requiring the grind of
    \href{https://www.nytimes3xbfgragh.onion/2020/06/05/us/coronavirus-education-lost-learning.html?action=click\&pgtype=Article\&state=default\&region=MAIN_CONTENT_3\&context=storylines_faq}{online
    learning},
    \href{https://www.nytimes3xbfgragh.onion/2020/05/29/us/coronavirus-child-care-centers.html?action=click\&pgtype=Article\&state=default\&region=MAIN_CONTENT_3\&context=storylines_faq}{makeshift
    child care} and
    \href{https://www.nytimes3xbfgragh.onion/2020/06/03/business/economy/coronavirus-working-women.html?action=click\&pgtype=Article\&state=default\&region=MAIN_CONTENT_3\&context=storylines_faq}{stunted
    workdays} to continue. California's two largest public school
    districts --- Los Angeles and San Diego --- said on July 13, that
    \href{https://www.nytimes3xbfgragh.onion/2020/07/13/us/lausd-san-diego-school-reopening.html?action=click\&pgtype=Article\&state=default\&region=MAIN_CONTENT_3\&context=storylines_faq}{instruction
    will be remote-only in the fall}, citing concerns that surging
    coronavirus infections in their areas pose too dire a risk for
    students and teachers. Together, the two districts enroll some
    825,000 students. They are the largest in the country so far to
    abandon plans for even a partial physical return to classrooms when
    they reopen in August. For other districts, the solution won't be an
    all-or-nothing approach.
    \href{https://bioethics.jhu.edu/research-and-outreach/projects/eschool-initiative/school-policy-tracker/}{Many
    systems}, including the nation's largest, New York City, are
    devising
    \href{https://www.nytimes3xbfgragh.onion/2020/06/26/us/coronavirus-schools-reopen-fall.html?action=click\&pgtype=Article\&state=default\&region=MAIN_CONTENT_3\&context=storylines_faq}{hybrid
    plans} that involve spending some days in classrooms and other days
    online. There's no national policy on this yet, so check with your
    municipal school system regularly to see what is happening in your
    community.
  \end{itemize}
\end{itemize}

In 2017, Amtrak's board hired Richard Anderson, a former chief executive
of Delta Air Lines, to take the reins of the rail network and make it
profitable and more reliable. His three-year contract recently expired,
but he will remain on staff for the remainder of 2020 as a senior
adviser to Mr. Flynn.

Mr. Anderson took his charge from the board seriously, and embarked on a
mission to trim the work force and improve shorter-haul, more trafficked
routes in corridors like the Northeast, while scaling back support for
longer-haul and less popular routes that run across parts of America's
Southern, mountain and Western regions.

``We should be looking at breaking up some of those long-distance
trains,'' Mr. Anderson said last year at a Senate hearing, ``and
figuring out how we serve the American consumer to provide high-quality
service in short-haul markets.''

In 2018, he decided to close an Amtrak reservation call center in
California, and told 500 employees they could keep their jobs if they
moved to Philadelphia --- where the positions were being relocated ---
in 60 days. Around 100 took the offer, according to the Transportation
Communications Union.

The same year, Mr. Anderson tried to significantly cut long-distance
service along a 2,200 mile route which stretches from Chicago to Los
Angeles, and replace the portion from Kansas to New Mexico with bus
service. Amtrak abandoned its plans after receiving fierce blowback from
members of Congress.

During a congressional hearing last year, Senator Jerry Moran,
Republican of Kansas, said ``the idea that Amtrak would think about
replacing passenger service with bus service for 400 miles'' was
something he could not ``get over because it tells me your attitude
toward that line or maybe toward long-distance nonprofitable passenger
service.''

Mr. Flynn has yet to testify before Congress about his vision for the
rail network, but he said this week
\href{https://www.washingtonpost.com/washington-post-live-amtrak-ceo-william-flynn/}{in
an interview with The Washington Post} that service cuts to Amtrak's
long-distance routes would remain in effect for the winter and would be
re-evaluated on a route-by-route basis next spring and summer.

Critics also warn that cutting service on long-distance train routes
makes them a less attractive option, with passengers having to wait days
for trips or endure long layovers between connecting routes.

Others point to
\href{https://www.govinfo.gov/content/pkg/CHRG-106shrg85968/pdf/CHRG-106shrg85968.pdf}{Amtrak's
statements} from as far back as 2000, in which its leadership said
attempts to reduce long-distance service ``ended up costing the company
more in lost revenue than we were able to take out in the way of
expenses'' because some fixed costs could not be cut even after reducing
service from daily to three days a week.

``We've done it before,'' Mr. Smith said. ``It doesn't pay, it doesn't
save the money and it drives off revenue.''

Advertisement

\protect\hyperlink{after-bottom}{Continue reading the main story}

\hypertarget{site-index}{%
\subsection{Site Index}\label{site-index}}

\hypertarget{site-information-navigation}{%
\subsection{Site Information
Navigation}\label{site-information-navigation}}

\begin{itemize}
\tightlist
\item
  \href{https://help.nytimes3xbfgragh.onion/hc/en-us/articles/115014792127-Copyright-notice}{©~2020~The
  New York Times Company}
\end{itemize}

\begin{itemize}
\tightlist
\item
  \href{https://www.nytco.com/}{NYTCo}
\item
  \href{https://help.nytimes3xbfgragh.onion/hc/en-us/articles/115015385887-Contact-Us}{Contact
  Us}
\item
  \href{https://www.nytco.com/careers/}{Work with us}
\item
  \href{https://nytmediakit.com/}{Advertise}
\item
  \href{http://www.tbrandstudio.com/}{T Brand Studio}
\item
  \href{https://www.nytimes3xbfgragh.onion/privacy/cookie-policy\#how-do-i-manage-trackers}{Your
  Ad Choices}
\item
  \href{https://www.nytimes3xbfgragh.onion/privacy}{Privacy}
\item
  \href{https://help.nytimes3xbfgragh.onion/hc/en-us/articles/115014893428-Terms-of-service}{Terms
  of Service}
\item
  \href{https://help.nytimes3xbfgragh.onion/hc/en-us/articles/115014893968-Terms-of-sale}{Terms
  of Sale}
\item
  \href{https://spiderbites.nytimes3xbfgragh.onion}{Site Map}
\item
  \href{https://help.nytimes3xbfgragh.onion/hc/en-us}{Help}
\item
  \href{https://www.nytimes3xbfgragh.onion/subscription?campaignId=37WXW}{Subscriptions}
\end{itemize}
