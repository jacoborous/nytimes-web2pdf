Sections

SEARCH

\protect\hyperlink{site-content}{Skip to
content}\protect\hyperlink{site-index}{Skip to site index}

\href{https://www.nytimes3xbfgragh.onion/section/us}{U.S.}

\href{https://myaccount.nytimes3xbfgragh.onion/auth/login?response_type=cookie\&client_id=vi}{}

\href{https://www.nytimes3xbfgragh.onion/section/todayspaper}{Today's
Paper}

\href{/section/us}{U.S.}\textbar{}Someone Tore Down a Union Army Statue
in Saratoga Springs. Why?

\url{https://nyti.ms/2OE1iAG}

\begin{itemize}
\item
\item
\item
\item
\item
\end{itemize}

\href{https://www.nytimes3xbfgragh.onion/news-event/george-floyd-protests-minneapolis-new-york-los-angeles?action=click\&pgtype=Article\&state=default\&region=TOP_BANNER\&context=storylines_menu}{Race
and America}

\begin{itemize}
\tightlist
\item
  \href{https://www.nytimes3xbfgragh.onion/2020/07/26/us/protests-portland-seattle-trump.html?action=click\&pgtype=Article\&state=default\&region=TOP_BANNER\&context=storylines_menu}{Protesters
  Return to Other Cities}
\item
  \href{https://www.nytimes3xbfgragh.onion/2020/07/24/us/portland-oregon-protests-white-race.html?action=click\&pgtype=Article\&state=default\&region=TOP_BANNER\&context=storylines_menu}{Portland
  at the Center}
\item
  \href{https://www.nytimes3xbfgragh.onion/2020/07/23/podcasts/the-daily/portland-protests.html?action=click\&pgtype=Article\&state=default\&region=TOP_BANNER\&context=storylines_menu}{Podcast:
  Showdown in Portland}
\item
  \href{https://www.nytimes3xbfgragh.onion/interactive/2020/07/16/us/black-lives-matter-protests-louisville-breonna-taylor.html?action=click\&pgtype=Article\&state=default\&region=TOP_BANNER\&context=storylines_menu}{45
  Days in Louisville}
\end{itemize}

Advertisement

\protect\hyperlink{after-top}{Continue reading the main story}

Supported by

\protect\hyperlink{after-sponsor}{Continue reading the main story}

\hypertarget{someone-tore-down-a-union-army-statue-in-saratoga-springs-why}{%
\section{Someone Tore Down a Union Army Statue in Saratoga Springs.
Why?}\label{someone-tore-down-a-union-army-statue-in-saratoga-springs-why}}

The monument in a New York park was dedicated to men who volunteered to
fight for the Union during the Civil War. It is unclear who toppled it,
and for what reason.

\includegraphics{https://static01.graylady3jvrrxbe.onion/images/2020/07/17/multimedia/17xp-unrest-unionstatue-pix1/merlin_174686838_33fb59aa-6743-4838-92e1-c86f3a067104-articleLarge.jpg?quality=75\&auto=webp\&disable=upscale}

By \href{https://www.nytimes3xbfgragh.onion/by/jacey-fortin}{Jacey
Fortin} and
\href{https://www.nytimes3xbfgragh.onion/by/bryan-pietsch}{Bryan
Pietsch}

\begin{itemize}
\item
  July 18, 2020
\item
  \begin{itemize}
  \item
  \item
  \item
  \item
  \item
  \end{itemize}
\end{itemize}

Early on Thursday morning, the police in Saratoga Springs, N.Y., found a
cast iron and zinc statue torn down from its stone pedestal in Congress
Park and scattered in pieces on the grass.

The statue is one of dozens that have been torn down across the United
States in recent weeks amid
\href{https://www.nytimes3xbfgragh.onion/interactive/2020/07/03/us/george-floyd-protests-crowd-size.html}{widespread
protests} against racism and police brutality.
\href{https://www.nytimes3xbfgragh.onion/2020/06/16/us/protests-statues-reckoning.html}{Many
of those toppled} have been monuments to Confederate soldiers.

But the statue that stood in Congress Park was dedicated to volunteers
who fought for the Union during the Civil War. In the days since it was
destroyed, residents of Saratoga Springs, a mostly white college town
about 30 miles north of Albany, have been calling the mayor's office and
posting on social media to express outrage and disappointment.

``The statue was memorializing those who fought against the Confederacy
and against slavery, so I think Saratogans were very proud that we had
that in our park,'' said David Snyder, executive assistant to Meg Kelly,
the mayor of Saratoga Springs.

The police are still investigating the episode and have yet to publicly
identify any suspects, leaving residents to wonder who toppled the
monument, and why.

``We're very confused,'' Mr. Snyder said. ``Was this in any way tied to
a Black Lives Matter protest in which they thought it was a Confederate
statue that needed to come down? Was it a reactionary or pro-Confederate
group that wanted a Union statue to come down? Or was it random?''

\includegraphics{https://static01.graylady3jvrrxbe.onion/images/2020/07/17/multimedia/17xp-unrest-unionstatue-pix2/17xp-unrest-unionstatue-pix2-articleLarge.jpg?quality=75\&auto=webp\&disable=upscale}

Lexis Figuereo, 33, a resident of Saratoga Springs and an organizer with
\href{https://www.facebookcorewwwi.onion/UntitledAndFree/}{All of Us}, a
group associated with the Black Lives Matter movement, said the
activists he knew had nothing to do with the toppling of the monument.

``If anything, something like this would be done by somebody who had no
idea what they were doing,'' he said. ``Or a person who was racist,
because this was a Union statue.''

It is possible that the toppling was a random act of destruction,
divorced from the historical significance of the monument. Like many
public areas, Congress Park has had
\href{https://cbs6albany.com/news/local/man-arrested-in-connection-to-congress-park-vandalism}{fountains}
and statues vandalized in recent years. Just days before the Union
monument was torn down, a memorial in the park to
\href{https://www.saratoga.com/aboutsaratoga/history/katrina-trask/}{Katrina
Trask}, a philanthropist who supported artists in the community, was
\href{https://wnyt.com/saratoga-county-ny-news/vandals-deface-new-500000-entrance-to-congress-park/5791939/}{defaced
with red spray paint}.

Mr. Figuereo speculated that the destruction of the statue could have
been done by people trying to make local activists look bad. But he also
wondered why it had captured so much attention.

``Certain people care more about property than about people,'' he said.
``It's sad.''

He added that activists in the area were more concerned with organizing
\href{https://dailygazette.com/galleries/2020/07/01/all-us-protestors-march-streets-saratoga-springs}{demonstrations},
pushing for more police transparency and accountability, and
\href{https://www.timesunion.com/news/article/A-death-in-Saratoga-but-no-internal-probe-13174766.php}{raising
awareness about the case of Darryl Mount Jr.}, a 21-year-old man who
died of his injuries after being pursued by Saratoga Springs police
officers in 2013.

``You can replace a statue,'' Mr. Figuereo said. ``You cannot replace
Darryl's life.''

The toppled statue in Saratoga Springs was first erected in 1875 and
placed on Broadway, a street that borders Congress Park, a downtown
green space that is home to the
\href{https://www.saratogahistory.org/}{Saratoga Springs History Museum}
and a bronze sculpture called
``\href{https://timesmachine.nytimes3xbfgragh.onion/timesmachine/1915/06/27/301807212.html?pageNumber=20}{The
Spirit of Life}.''

According to the Saratoga Springs Department of Public Works, Union Army
veterans from New York's 77th Infantry Regiment --- which the monument
commemorates --- donated \$3,000 to install the statue.

Image

The monument seen in the early 1900s in its original location on
Broadway, at the entrance to Congress Park.Credit...Library of Congress

``The 77th was made up of men from Saratoga Springs, Wilton,
Schuylerville, and other surrounding communities who volunteered, not
those who were drafted,'' said James D. Parillo, the executive director
of the Saratoga Springs History Museum. He added that museum staff
members were ``sad and disappointed'' about the monument's destruction.

The statue was moved to its current location inside the park, not far
from the ``Spirit of Life'' sculpture, in 1921 to prevent damage from
the traffic on Broadway. The sculpture, which portrayed a soldier but
did not represent any particular person, was positioned atop a stone
pedestal.

Samantha Bosshart, the executive director of the
\href{https://www.saratogapreservation.org/}{Saratoga Springs
Preservation Foundation}, said she gave a tour of the park on the day
before the statue fell. She recalled pointing out the statue to families
on the tour and explaining the history of the monument.

``To wake up the next day and see it gone --- smashed --- was
disappointing and sad and unfortunate,'' Ms. Bosshart said.

The episode in Saratoga Springs was not unique; in recent weeks, several
monuments to people who opposed slavery have been dismantled. A statue
of the Black abolitionist
\href{https://www.nytimes3xbfgragh.onion/2020/07/03/arts/frederick-douglass-yale.html?module=inline}{Frederick
Douglass} was detached from its base in a park in Rochester, N.Y., this
month and found some 50 feet away, dumped near a river gorge. And last
month in Madison, Wis., protesters toppled a statue of
\href{https://www.jsonline.com/story/news/local/wisconsin/2020/06/24/hans-christian-hegs-abolitionist-statue-toppled-madison-what-know/3248692001/}{Hans
Christian Heg}, a white abolitionist who fought against the Confederacy.

Michael Veitch, the business manager for the city's Department of Public
Works, said Saratoga Springs officials were working to repair or replace
the Union memorial in Congress Park.

``This is a great caring community that loves its public spaces, so a
number of individuals and organizations have already reached out to make
donations in support of the work necessary to rectify the vandalism,''
he said.

Advertisement

\protect\hyperlink{after-bottom}{Continue reading the main story}

\hypertarget{site-index}{%
\subsection{Site Index}\label{site-index}}

\hypertarget{site-information-navigation}{%
\subsection{Site Information
Navigation}\label{site-information-navigation}}

\begin{itemize}
\tightlist
\item
  \href{https://help.nytimes3xbfgragh.onion/hc/en-us/articles/115014792127-Copyright-notice}{©~2020~The
  New York Times Company}
\end{itemize}

\begin{itemize}
\tightlist
\item
  \href{https://www.nytco.com/}{NYTCo}
\item
  \href{https://help.nytimes3xbfgragh.onion/hc/en-us/articles/115015385887-Contact-Us}{Contact
  Us}
\item
  \href{https://www.nytco.com/careers/}{Work with us}
\item
  \href{https://nytmediakit.com/}{Advertise}
\item
  \href{http://www.tbrandstudio.com/}{T Brand Studio}
\item
  \href{https://www.nytimes3xbfgragh.onion/privacy/cookie-policy\#how-do-i-manage-trackers}{Your
  Ad Choices}
\item
  \href{https://www.nytimes3xbfgragh.onion/privacy}{Privacy}
\item
  \href{https://help.nytimes3xbfgragh.onion/hc/en-us/articles/115014893428-Terms-of-service}{Terms
  of Service}
\item
  \href{https://help.nytimes3xbfgragh.onion/hc/en-us/articles/115014893968-Terms-of-sale}{Terms
  of Sale}
\item
  \href{https://spiderbites.nytimes3xbfgragh.onion}{Site Map}
\item
  \href{https://help.nytimes3xbfgragh.onion/hc/en-us}{Help}
\item
  \href{https://www.nytimes3xbfgragh.onion/subscription?campaignId=37WXW}{Subscriptions}
\end{itemize}
