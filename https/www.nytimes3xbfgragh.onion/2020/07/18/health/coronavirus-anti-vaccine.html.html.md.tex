Sections

SEARCH

\protect\hyperlink{site-content}{Skip to
content}\protect\hyperlink{site-index}{Skip to site index}

\href{https://www.nytimes3xbfgragh.onion/section/health}{Health}

\href{https://myaccount.nytimes3xbfgragh.onion/auth/login?response_type=cookie\&client_id=vi}{}

\href{https://www.nytimes3xbfgragh.onion/section/todayspaper}{Today's
Paper}

\href{/section/health}{Health}\textbar{}Mistrust of a Coronavirus
Vaccine Could Imperil Widespread Immunity

\url{https://nyti.ms/2OAxMM1}

\begin{itemize}
\item
\item
\item
\item
\item
\item
\end{itemize}

\href{https://www.nytimes3xbfgragh.onion/news-event/coronavirus?action=click\&pgtype=Article\&state=default\&region=TOP_BANNER\&context=storylines_menu}{The
Coronavirus Outbreak}

\begin{itemize}
\tightlist
\item
  live\href{https://www.nytimes3xbfgragh.onion/2020/08/04/world/coronavirus-covid-19.html?action=click\&pgtype=Article\&state=default\&region=TOP_BANNER\&context=storylines_menu}{Latest
  Updates}
\item
  \href{https://www.nytimes3xbfgragh.onion/interactive/2020/us/coronavirus-us-cases.html?action=click\&pgtype=Article\&state=default\&region=TOP_BANNER\&context=storylines_menu}{Maps
  and Cases}
\item
  \href{https://www.nytimes3xbfgragh.onion/interactive/2020/science/coronavirus-vaccine-tracker.html?action=click\&pgtype=Article\&state=default\&region=TOP_BANNER\&context=storylines_menu}{Vaccine
  Tracker}
\item
  \href{https://www.nytimes3xbfgragh.onion/2020/08/02/us/covid-college-reopening.html?action=click\&pgtype=Article\&state=default\&region=TOP_BANNER\&context=storylines_menu}{College
  Reopening}
\item
  \href{https://www.nytimes3xbfgragh.onion/live/2020/08/03/business/stock-market-today-coronavirus?action=click\&pgtype=Article\&state=default\&region=TOP_BANNER\&context=storylines_menu}{Economy}
\end{itemize}

Advertisement

\protect\hyperlink{after-top}{Continue reading the main story}

Supported by

\protect\hyperlink{after-sponsor}{Continue reading the main story}

\hypertarget{mistrust-of-a-coronavirus-vaccine-could-imperil-widespread-immunity}{%
\section{Mistrust of a Coronavirus Vaccine Could Imperil Widespread
Immunity}\label{mistrust-of-a-coronavirus-vaccine-could-imperil-widespread-immunity}}

Billions are being poured into developing a shot, but the rapid
timetable and President Trump's cheerleading are creating a whole new
group of vaccine-hesitant patients.

\includegraphics{https://static01.graylady3jvrrxbe.onion/images/2020/07/19/science/00VIRUS-VAX-DOUBTS1/merlin_173852484_5fad07c2-7dbe-45c0-9dc8-fd4823271c2b-articleLarge.jpg?quality=75\&auto=webp\&disable=upscale}

By \href{https://www.nytimes3xbfgragh.onion/by/jan-hoffman}{Jan Hoffman}

\begin{itemize}
\item
  Published July 18, 2020Updated July 22, 2020
\item
  \begin{itemize}
  \item
  \item
  \item
  \item
  \item
  \item
  \end{itemize}
\end{itemize}

Almost daily, President Trump and leaders worldwide say they are racing
to develop a
\href{https://www.nytimes3xbfgragh.onion/2020/07/22/upshot/vaccine-coronavirus-government-purchase.html}{coronavirus
vaccine}, in perhaps the most urgent mission in the history of medical
science. But the repeated assurances of near-miraculous speed are
exacerbating a problem that has largely been overlooked and one that
public health experts say must be addressed now: persuading people to
actually get the shot.

A growing number of polls find so many people saying they would not get
a coronavirus vaccine that its potential to shut down the pandemic could
be in jeopardy. Distrust of it is particularly pronounced in
African-American communities, which have been disproportionately
devastated by the virus. But even many staunch supporters of
immunization say they are wary of this vaccine.

``The bottom line is I have absolutely no faith in the F.D.A. and in the
Trump administration,'' said Joanne Barnes, a retired fourth-grade
teacher from Fairbanks, Alaska, who said she was otherwise always
scrupulously up-to-date on getting her shots, including those for
shingles, flu and pneumonia. ``I just feel like there's a rush to get a
vaccine out, so I'm very hesitant.''

\href{https://www.nytimes3xbfgragh.onion/2019/09/23/health/anti-vaccination-movement-us.html}{Mistrust
of vaccines has been on the rise in the U.S.} in recent years, a
sentiment that resists categorization by political party, educational
background or socio-economic demographics. It has been fanned by a
handful of celebrities. But now, anti-vaccine groups are attracting a
new type of clientele altogether.

Jackie Schlegel, founder of Texans for Vaccine Choice, which presses for
school vaccine exemptions, said that her group's membership had
skyrocketed since April. ``Our phones are ringing off the hook with
people who are saying, `I've gotten every vaccine, but I'm not getting
this one,'' she said. ```How do I opt out?''' She said she often has to
assure callers, ```They're not coming to your home to force-vax you.'''

\includegraphics{https://static01.graylady3jvrrxbe.onion/images/2020/07/17/science/00VIRUS-VAX-DOUBTS4a/00VIRUS-VAX-DOUBTS4a-articleLarge.jpg?quality=75\&auto=webp\&disable=upscale}

\includegraphics{https://static01.graylady3jvrrxbe.onion/images/2017/01/29/podcasts/the-daily-album-art/the-daily-album-art-articleInline-v2.jpg?quality=75\&auto=webp\&disable=upscale}

\hypertarget{listen-to-the-daily-the-vaccine-trust-problem}{%
\subsubsection{Listen to `The Daily': The Vaccine Trust
Problem}\label{listen-to-the-daily-the-vaccine-trust-problem}}

Why developing a coronavirus vaccine may be easier than persuading
people to get it.

transcript

Back to The Daily

bars

0:00/29:14

-29:14

transcript

\hypertarget{listen-to-the-daily-the-vaccine-trust-problem-1}{%
\subsection{Listen to `The Daily': The Vaccine Trust
Problem}\label{listen-to-the-daily-the-vaccine-trust-problem-1}}

\hypertarget{hosted-by-michael-barbaro-produced-by-luke-vander-ploeg-and-annie-brown-and-edited-by-lisa-chow}{%
\subsubsection{Hosted by Michael Barbaro, produced by Luke Vander Ploeg
and Annie Brown, and edited by Lisa
Chow}\label{hosted-by-michael-barbaro-produced-by-luke-vander-ploeg-and-annie-brown-and-edited-by-lisa-chow}}

\hypertarget{why-developing-a-coronavirus-vaccine-may-be-easier-than-persuading-people-to-get-it}{%
\paragraph{Why developing a coronavirus vaccine may be easier than
persuading people to get
it.}\label{why-developing-a-coronavirus-vaccine-may-be-easier-than-persuading-people-to-get-it}}

\begin{itemize}
\item
  michael barbaro\\
  From The New York Times, I'm Michael Barbaro. This is ``The Daily.''

  Today: Public health officials are vowing to develop a coronavirus
  vaccine in record time. My colleague, health reporter Jan Hoffman, on
  how that speed could backfire.

  It's Tuesday, July 21st.
\item
  archived recording\\
  Thank you, very much, Mr. Chairman. Thank you to all of our witnesses
  for joining us here today. And, of course, thank your staff for
  setting up the technology so we can hold this hearing safely.
\end{itemize}

jan hoffman

So late last month, Dr. Anthony Fauci and Dr. Robert Redfield at the
C.D.C. sat down in front of a group of senators to answer their many
questions about what was going on with the coronavirus pandemic.

\begin{itemize}
\item
  archived recording (elizabeth warren)\\
  Dr. Fauci, based on what you're seeing now, how many Covid-19 deaths
  and infections should America expect before this is all over?
\item
  archived recording (dr. anthony fauci)\\
  I can't make an accurate prediction, but it is going to be very
  disturbing. I will guarantee you that.
\end{itemize}

jan hoffman

The big news that day was Dr. Anthony Fauci saying that he expected
cases to rise.

\begin{itemize}
\tightlist
\item
  archived recording (dr. anthony fauci)\\
  I would not be surprised if we go up to 100,000 a day ---
\end{itemize}

jan hoffman

To 100,000 a day.

\begin{itemize}
\tightlist
\item
  archived recording (dr. anthony fauci)\\
  --- if this does not turn around.
\end{itemize}

michael barbaro

Right. That was a big headline. I remember that.

jan hoffman

That shocked everyone. But what was also rumbling through, and was a
consistent theme in the questioning by the senators, was their concern
that Americans were afraid of the very speed at which this vaccine was
being developed.

{[}music{]}

\begin{itemize}
\tightlist
\item
  archived recording\\
  Dr. Fauci, I want to ask you about the concern that we have with
  certain parts of the country where you have public mistrust of
  vaccines, in general.
\end{itemize}

jan hoffman

And they were asking whether Americans would, in fact, be willing to get
it.

\begin{itemize}
\tightlist
\item
  archived recording\\
  My fear is that we may get to the place where --- we will get to that
  place where we have that successful vaccine. But we still have the
  concern for many, and a mistrust. And whether it's vaccine hesitation
  or vaccine confidence --- I don't know what the buzz word is --- but
  I'm worried that we don't have a plan for how to deal with that.
\end{itemize}

jan hoffman

It was not one party or the other. Both Republican and Democratic
senators kept firing away at Dr. Robert Redfield and Dr. Fauci.

\begin{itemize}
\item
  archived recording 1\\
  We know this is in our future, and we are not ready for it.
\item
  archived recording 2\\
  And this could cause problems down the road if we get to a vaccine,
  but people don't want to get the vaccine. So ---
\end{itemize}

jan hoffman

Saying, what are you going to do? How are you going to prepare
Americans?

\begin{itemize}
\tightlist
\item
  archived recording\\
  And that plan has to combat misinformation and vaccine hesitancy.
\end{itemize}

jan hoffman

We are sensing that they are afraid of this thing. They are saying they
won't take it.

\begin{itemize}
\item
  archived recording\\
  Dr. Redfield, do you agree a plan like that is needed?
\item
  archived recording (dr. robert redfield)\\
  Senator, I think it's very important that we have an integrated plan
  for this vaccine.
\end{itemize}

jan hoffman

And both of the gentlemen seemed somewhat disconcerted.

{[}music{]}

michael barbaro

And yet, how grounded are these fears that these senators are expressing
during this hearing?

jan hoffman

They are incredibly substantial. There was a survey done in late May by
the Associated Press and a research institute out of the University of
Chicago that showed that fully 50 percent of Americans were either
hesitant or absolutely would not take the vaccine.

michael barbaro

Wow.

jan hoffman

Which is really concerning.

michael barbaro

50 percent.

jan hoffman

50 percent.

michael barbaro

And in my mind, skepticism of vaccines in the United States has been
around for a really long time. And it's somewhat meaningful, but it's
not widespread. It's not 50 percent. It's kind of a niche. So that's not
what you're describing here --- a niche.

jan hoffman

No. This is a chasm. This was exponentially far greater than anything
we'd ever seen before.

michael barbaro

So how do we get to that enormous widespread figure? Because we have
talked a bit on this show about the origins of vaccine skepticism. And
my recollection is that it starts with questions around autism.

jan hoffman

Actually, it starts with questions around the invention of the smallpox
vaccine in the 18th century. Even then, there were vaccine skeptics.
Benjamin Franklin was himself a vaccine skeptic. He later recanted and
saw the light. So it has come in waves over the centuries.

Probably, what's most prominent in the modern memory is a study that Dr.
Andrew Wakefield published in the British Journal The Lancet in 1998,
where he associated autism and the measles, mumps, rubella vaccine,
which is given to children just around the time that they're about a
year and a half. And he asserted wrongly --- completely wrongly --- that
the vaccine caused autism. That has been completely refuted. And yet, it
still took hold in the hearts and minds of many, many parents. It has
become the basis for political movements.

For example, it's a very big movement in Texas with a politically
powerful group called Texans for Vaccine Choice. They have, in fact,
hijacked the language of the Abortion Rights Movement --- this is my
body. The government does not have the right to order me to put
something into it. It's my body, my choice. There are people who resent
big pharma. And they believe vaccines are totally a construct of big
pharma to make money. When in fact, actually, it's probably the reason
that most companies don't make vaccines, because they don't make a lot
of money out of it.

There is the crunchy granola --- to use a term of art --- parenting
movement, which basically says, nothing but the natural comes into my
child. Therefore, not a vaccine.

Certainly, vaccines skepticism has been shown to be more pronounced in
African-American and Latino communities, particularly because of the
revelations in the mid-70s of the Tuskegee experiments, in which the
American public health institutions knew that something like 300 Alabama
sharecroppers had been infected with syphilis. And although they had the
cure for it --- penicillin --- they refused to cure them, and instead
wanted to watch the disease progress so they could learn more about the
disease. When that horror broke, that reinforced nascent vaccine
skepticism in the African-American community, and the perception that
they were essentially being used as cannon fodder for privileged white
people.

{[}music{]}

So if you think you have someone in mind who you think is the archetype
of someone who opposes vaccines, you absolutely do not. It crosses
racial lines. It crosses socioeconomic backgrounds, educational
backgrounds. It crosses political affiliation.

michael barbaro

And Jan, how does Donald Trump and his arrival on the national political
scene --- how does that play into this?

jan hoffman

Since about 2012, he's been tweeting very skeptical comments about what
he thinks are the size of the doses. He frequently would say, this is
enough for a horse. And then he comes on the stage while he's a
candidate ---

\begin{itemize}
\tightlist
\item
  archived recording (donald trump)\\
  There's people that work for me --- just the other day, two years old,
  two and a half years old, a child --- a beautiful child went to have
  the vaccine. And came back, and a week later, got a tremendous fever.
  Got very, very sick. Now is autistic.
\end{itemize}

jan hoffman

And he says bluntly, during a major debate, that he doesn't believe in
the schedule. And he thinks kids are getting too many vaccines.

\begin{itemize}
\item
  archived recording (donald trump)\\
  I only say, it's not --- I'm in favor of vaccines. Do them over a
  longer period of time. Same amount, but just in little sections.
\item
  archived recording\\
  Dr. Carson.
\item
  archived recording (donald trump)\\
  And I think you're going to have --- I think you're going to see a big
  impact on autism.
\end{itemize}

jan hoffman

He has boasted before that he never himself would get a flu vaccine. He
said he slowed down his son Barron's vaccine schedule. So he became the
flag bearer for this growing movement that had so many myriad voices in
it from so many different perspectives.

michael barbaro

So all of this vaccine baggage --- for lack of a better phrase --- all
of this skepticism, it predates the pandemic. But I guess I still don't
quite understand how we get to that really alarming 50 percent figure of
Americans who are reluctant to use an eventual coronavirus vaccine. So
help me bridge that.

jan hoffman

We have a pandemic that, as the weeks go by, people are dying. Cases are
taking up. Our lives as we know it have changed completely. We don't
even have a new normal yet. We are making it up as we go along. And all
along, the word vaccine is being held out as a holy grail.

{[}music{]}

A vaccine will save us. A vaccine will restore us. A vaccine will bring
us life that we knew.

michael barbaro

Right.

jan hoffman

It is topic number one. You cannot turn around without hearing the V
word. It is front and center wherever we go. And that is the overlay on
top of this insurgent, multi-dimensional questioning of the value of a
vaccine.

michael barbaro

We'll be way right back.

Jan, when did you begin to realize that there was something about this
pandemic that was influencing how people thought about vaccines --- the
V word?

jan hoffman

I began to speak with doctors, pediatricians. And I asked them, if we
come up with a coronavirus vaccine, what will you tell your patients?
And I was struck over and over and over again by the long, loud silence
on the other end of the phone.

{[}music{]}

And I thought oh, my god, what are we hearing here? I began to watch
social media, and I saw the amping up of vaccine conspiracy theories.
Then I heard more and more from people who were beginning to say, you
know, I get all my vaccines, I'm up-to-date --- I will not take this
one. These are pro-science, pro-vaccine people who are cringing and
wanting to avoid this vaccine. And I thought, we have a problem.

michael barbaro

And what do you start to learn that would explain that level of
skepticism?

jan hoffman

There are a lot of different reasons. But the first profound roadblock
to it are many people's objection to President Trump himself. People
worry that he may have secret deals with certain pharma companies, and
may stand to --- either his friends will profit or he will profit.

And so, unfortunately, people are holding the product itself at arm's
length and looking at it through the lens of a political situation. In
fact, a major figure from the Trump administration called me just two
days ago to talk about what the government was going to try to do about
vaccine hesitancy. And he said, it's unfortunate that people are
wrapping their feelings of President Trump around the vaccine itself.

michael barbaro

Is what you're saying that some number of people, who would normally be
inclined to take a vaccine but do not trust President Trump, are now
thinking to themselves, well, if I don't trust President Trump, then
perhaps I shouldn't trust a vaccine that emerges from a process he
oversees. And just want to make sure I'm connecting the dots here.

jan hoffman

Those dots are beautifully connected. Because I've seen comments that go
along the lines of, I'll take a vaccine authorized by a President Biden.
I'll take a vaccine authorized by Angela Merkel. It's Trump's
association with it that is giving a certain quadrant of these skeptics
grave misgivings.

michael barbaro

But is that a reasonable form of skepticism? I mean, presidents have
lots of powers, but they don't have the power to mix a drug in a lab.
They don't dictate what a vaccine looks like. So is that rational?

jan hoffman

I'm trying to answer this politely because that presupposes that vaccine
skepticism is inherently rational. And, to some extent, I think it's
understandable. Whether it's rational and logical is another question
entirely. But remember, the president nominates the head of the F.D.A.,
who approves the vaccines. The president assigned the head of Operation
Warp Speed, which is overseeing the public-private partnership. The
president doesn't mix things in a test tube, but the president certainly
has a great deal of power to authorize oversight of this vaccine.

michael barbaro

What else is driving this skepticism?

jan hoffman

I think even a greater factor than the administration itself is the
speed with which it's being produced. Most vaccines take about a decade
to produce. Millions and even billions of dollars are poured into
research for them to prove nothing. We don't have an H.I.V. vaccine,
which has been in research for 20, 30 years. There's no vaccine against
breast cancer, which has been under research for arguably, even longer.
And so people are thinking, well, how can you have a vaccine that is
safe and effective come to market in six months? It boggles the mind.

And so, for someone who is a vaccine hesitant, who is a vaccine skeptic,
or even is just a pro-vaccine person, they are so apprehensive about the
speed at which this is being produced that they are willing to say,
``Let someone else go first in line. Not me.''

michael barbaro

Is there actually any evidence that Operation Warp Speed --- the project
underway now --- will bypass traditional safety measures? The normal
process of multiple clinical trials, lots of humans being tested, lots
of assessments of side effects, adverse effects. Do we know that?

jan hoffman

It seems, so far, that nothing in the due diligence processes is being
bypassed. It's only that it's being accelerated. But the same level of
scrutiny seems to be underway. That's what we know so far.

michael barbaro

So this is quite fascinating and pretty alarming. The only remedy for
this pandemic is a vaccine. And so the faster you get a vaccine, the
faster the pandemic comes to an end. But from what you're saying, the
faster the vaccine is produced, the more skeptical people are going to
be of the vaccine and its safety. And so, speed here, instead of being a
virtue, may actually be an undermining force and undermining of the
original goal of the vaccine.

jan hoffman

And I think that's the tragedy. Because there's urgency. We need a
vaccine. The world is crying out for it. To stop this thing. To shut it
down. Scientists are responding, and saying we're working as quickly as
we can. And yet, thoughtful people are saying, wait, does speed equate
with haste?

michael barbaro

So that's how you get to a figure like 50 percent. You take a lot of
generalized anxiety around the safety of vaccines. You overlay this
administration and its approach to science. And then you add what the
government is promising is the fastest vaccine in history. And you get a
much more amped up version of existing skepticism.

jan hoffman

Let me ask you a practical question. And you don't have to answer
because I'm switching caps here. But if you polled your colleagues and
friends, what do you think, roughly, would be the percentage who would
answer the following question in the affirmative or negative: Would you
take a coronavirus vaccine if it were offered sometime this year?

michael barbaro

I'd like to think that it's 3/4 off the bat? But I don't know. You're
asking me a question I haven't asked those friends and acquaintances and
family. I guess I now should.

jan hoffman

Well, I think it's important. Because what happens when you engage
somebody in a conversation about vaccines is you both begin to think
more deeply about, what does confidence mean to you? What do you need to
know to feel safe in sticking out your arm? What questions would you
want answered? And as you begin to enumerate those questions, as you
begin to express your concerns, you are essentially creating a
sketchbook for the kind of answers that any manufacturer or the
government needs to have in hand to make the public feel confident that
they are getting a safe and effective vaccine.

michael barbaro

But I guess what I would have to say, now that I've had a minute or so
to reflect on this, is that all the previous science --- the vast
majority of the previous science --- about vaccines tells us that the
process is safe. And that any kind of trade-off is worth it, given the
public health value of people being protected against a highly
transmissible disease.

jan hoffman

There's lots of ways to answer that question. I want you to think about
the cultural moment we're in.

{[}music{]}

We are in a time when nationalism is surging around the world. America
first. My family first. Myself first. The notion of a vaccine, writ
large, means, I protect my community. I do what I can to protect my
neighborhood, my country, people who travel across the world. It is one
way to express altruism --- is you say, I care about you. I will protect
you so I cannot get myself sick, and I will not get you sick. But we are
not at a cultural moment that looks like that.

We do not care as much about our community, about our neighbors as we
used to. The uptake for flu vaccine in adults 18 and older is only about
45 percent a year. And yet, if you ask a public health specialist what
is the safest way to protect an older person from flu, a baby from flu,
someone going through cancer treatment from flu, you say get everyone
vaccinated for flu, even if they are not. Because that stops
transmission. And yet, we only have about 45 percent uptake.

Dr. Fauci has said at minimum, we need 75 percent of people to take a
coronavirus vaccine, and he would prefer to see 85 percent. Right now,
50 percent of people are saying they don't want the vaccine. That means
--- even in the calculus of my mediocre math background --- we are not
anywhere close to what we need to causing across-the-board immunity and
shutting down this pandemic.

michael barbaro

So with all this in mind, what is the plan for making Americans feel as
comfortable as possible with the safety of this eventual vaccine? It
seems crucially important to ending this pandemic. And like something
that people in public health, in the federal government would be taking
very, very seriously and have a plan for.

jan hoffman

During the Senate subcommittee hearing when Dr. Redfield was asked
repeatedly about this ---

\begin{itemize}
\item
  archived recording (dr. robert redfield)\\
  C.D.C. is working on the issues that you said that I think are so
  important in building vaccine confidence in this country.
\item
  archived recording\\
  Can you tell me when C.D.C. will be giving us their plans, and C.D.C.
  would be writing the comprehensive plan?
\item
  archived recording (dr. robert redfield)\\
  We're developing a plan as we speak. And again, to keep building on
  ---
\end{itemize}

jan hoffman

He said that the Centers for Disease Control and Prevention have been
working on a plan and discussing this for 10 to 12 weeks.

\begin{itemize}
\item
  archived recording (dr. robert redfield)\\
  --- vaccine, prioritization of this vaccine, monitoring for safety of
  this vaccine ---
\item
  archived recording\\
  But you can't tell if it'll a couple weeks, a couple months, the end
  of the year? Do you have any estimate on when we'll see that plan?
\item
  archived recording (dr. robert redfield)\\
  Well, it's currently in development within the group. And I'd
  anticipate that we'll see that plan in the near weeks ahead, Senator.
\end{itemize}

jan hoffman

When I asked them to explain what, in fact, they were working on, they
refused to answer. So I wish I could tell you. I have no idea.

{[}music{]}

michael barbaro

Jan, what happens if we get this wrong? If the vaccine comes out and a
huge number of Americans say, ``Not me, you first. I'm not ready for
this.''

jan hoffman

That's probably, the greatest concern of all. Because if a huge number
of Americans say, ``not me, you first,'' or if they say, ``Wait a
minute, it's not working. They had the vaccine for six months, but now
they're getting sick with Covid again,'' what public health experts are
worried about is that this will undermine the very foundation upon which
our vaccine infrastructure is built. Which is that vaccines work. That
you need to get them. And you need to trust them. And really undermine
faith in public health. In the belief that there is a superstructure
that has the greater good in mind.

michael barbaro

So the stakes here are only the future, literally, of public health.

jan hoffman

Yep.

michael barbaro

Thank you, Jan. We really appreciate it.

jan hoffman

Thanks very much for letting me talk about it.

michael barbaro

On Monday, scientists at Oxford University reported that their
experimental vaccine for the coronavirus prompted a protective immune
response in hundreds of people who received a dose during an early
clinical trial. So far, the vaccine has produced only minor side
effects, like fever, chills and muscle pain. The clinical trial involved
about 1,000 people. Larger trials involving about 10,000 people are
underway. And an even larger trial involving about 30,000 people is set
to start soon in the U.S. We'll be right back.

{[}music{]}

Here's what else you need to know today. A major teachers' union has
sued the governor of Florida over an emergency order that would fully
reopen schools there next month, amid a surge of infections. The
American Federation of Teachers and its Florida affiliate accused
Governor Ron DeSantis of violating a state law that requires schools to
be safe and secure. The lawsuit, apparently the first of its kind, asks
that local education and health officials, not the governor, have
control over reopenings. And signals that teachers may take a range of
actions to protest what they see as a hasty return to the classroom. And
---

\begin{itemize}
\tightlist
\item
  archived recording\\
  Since we last convened and specifically, on Friday, July 17, 2020, the
  Honorable John Robert Lewis, representative of the 5th Congressional
  District of Georgia, our hero, our colleague, our brother, our friend,
  received and answered his final summons from God Almighty. And at that
  moment, transitioned from labor to reward.
\end{itemize}

michael barbaro

On Monday, members of the House of Representatives unanimously passed a
resolution honoring their former colleague, John Lewis, who brought the
moral authority of his time as a civil rights leader to his three-decade
career in Congress.

\begin{itemize}
\item
  archived recording 1\\
  The clerk will report the resolution.
\item
  archived recording 2\\
  House Resolution 1054.
\end{itemize}

michael barbaro

Lewis' death seemed to unify a body long defined by its divisions. And
when the moment came for the House clerk to read the resolution, she was
briefly overcome with emotion.

\begin{itemize}
\tightlist
\item
  archived recording\\
  Resolve that the House has heard with profound sorrow {[}PAUSES{]} at
  the death of the Honorable John Lewis, a representative from the state
  of Georgia. Resolved that a committee of such members of the House as
  the Speaker may designate, together with such members of the Senate as
  may be joined, be appointed to attend the funeral.
\end{itemize}

michael barbaro

That's it for ``The Daily.'' I'm Michael Barbaro. See you tomorrow.

The fastidious process to develop a safe, effective
\href{https://www.nytimes3xbfgragh.onion/interactive/2020/04/30/opinion/coronavirus-covid-vaccine.html}{vaccine
typically takes a decade}; some have taken far longer. But the
administration of Mr. Trump,
\href{https://www.nytimes3xbfgragh.onion/2020/03/09/health/trump-vaccines.html}{himself
once an outspoken vaccine skeptic}, has been saying recently that a
\href{https://www.businessinsider.com/trump-expects-covid-19-vaccine-fall-timeline-coronavirus-2020-7}{coronavirus
vaccine could be ready this fall.} While it has removed certain
conventional barriers, such as funding, many experts still believe that
the proposed timeline could be unduly optimistic.

But whenever a coronavirus vaccine is approved, the
\href{https://www.nytimes3xbfgragh.onion/2020/07/09/us/coronavirus-vaccine.html}{assumption
has been that initial demand would far outstrip supply}. The need to
establish a bedrock of confidence in it has largely gone overlooked and
unaddressed.

Earlier this month, a
\href{https://www.centerforhealthsecurity.org/our-work/pubs_archive/pubs-pdfs/2020/200709-The-Publics-Role-in-COVID-19-Vaccination.pdf}{nationwide
task force of 23 epidemiologists} and vaccine behavior specialists
released a detailed report --- which itself got little attention ---
saying that such work was urgent. Operation Warp Speed, the \$10 billion
public-private partnership that is driving much of the vaccine research,
they wrote, ``rests upon the compelling yet unfounded presupposition
that `if we build it, they will come.'''

\hypertarget{latest-updates-global-coronavirus-outbreak}{%
\section{\texorpdfstring{\href{https://www.nytimes3xbfgragh.onion/2020/08/04/world/coronavirus-covid-19.html?action=click\&pgtype=Article\&state=default\&region=MAIN_CONTENT_1\&context=storylines_live_updates}{Latest
Updates: Global Coronavirus
Outbreak}}{Latest Updates: Global Coronavirus Outbreak}}\label{latest-updates-global-coronavirus-outbreak}}

Updated 2020-08-04T10:15:32.518Z

\begin{itemize}
\tightlist
\item
  \href{https://www.nytimes3xbfgragh.onion/2020/08/04/world/coronavirus-covid-19.html?action=click\&pgtype=Article\&state=default\&region=MAIN_CONTENT_1\&context=storylines_live_updates\#link-6b644638}{`Long
  days, long nights': Washington prepares for a prolonged fight over
  virus relief.}
\item
  \href{https://www.nytimes3xbfgragh.onion/2020/08/04/world/coronavirus-covid-19.html?action=click\&pgtype=Article\&state=default\&region=MAIN_CONTENT_1\&context=storylines_live_updates\#link-7af9fca0}{Israel's
  rocky reopening of its schools may be a lesson for the U.S.}
\item
  \href{https://www.nytimes3xbfgragh.onion/2020/08/04/world/coronavirus-covid-19.html?action=click\&pgtype=Article\&state=default\&region=MAIN_CONTENT_1\&context=storylines_live_updates\#link-33bf9168}{Hurricane
  Isaias arrives in North Carolina as officials along the East Coast
  scramble.}
\end{itemize}

\href{https://www.nytimes3xbfgragh.onion/2020/08/04/world/coronavirus-covid-19.html?action=click\&pgtype=Article\&state=default\&region=MAIN_CONTENT_1\&context=storylines_live_updates}{See
more updates}

More live coverage:
\href{https://www.nytimes3xbfgragh.onion/live/2020/08/04/business/stock-market-today-coronavirus?action=click\&pgtype=Article\&state=default\&region=MAIN_CONTENT_1\&context=storylines_live_updates}{Markets}

In fact, wrote the group, led by researchers at the
\href{https://www.centerforhealthsecurity.org/}{Johns Hopkins Center for
Health Security} and the Texas State University anthropology department:
``If poorly designed and executed, a Covid-19 vaccination campaign in
the U.S. could undermine the increasingly tenuous belief in vaccines and
the public health authorities that recommend them --- especially among
people most at risk of Covid-19 impacts.''

The researchers noted that although billions of federal dollars were
pouring into biomedical research for a vaccine, there seemed to be
virtually no funding set aside for social scientists to investigate
hesitancy around vaccines. Focus groups to help pinpoint the most
effective messaging to counter opposition, the authors said, should get
underway immediately.

Image

Undated photo from the National Archives showing the Tuskegee syphilis
experiment: From 1932 to 1972, doctors intentionally did not treat Black
men for the disease, so they could study the progress of
symptoms.Credit...National Archives

The current political and cultural turbulence, abetted by the Trump
administration's frequent disregard for scientific expertise, is only
amplifying the diverse underpinnings of vaccine-skeptic positions. They
include the terrible legacy of federal medical experiments on
African-Americans and other disadvantaged groups; a distrust of Big
Pharma; resistance to government mandates like school immunization
requirements; adherence to homeopathy and other ``natural'' medicines;
and a clutch of apocalyptic beliefs and conspiracy theories particularly
around Covid-19, sometimes perpetuated by celebrities, most recently
Kanye West.

``It's so many of our children that are being vaccinated and
paralyzed,''
\href{https://www.forbes.com/sites/randalllane/2020/07/08/kanye-west-says-hes-done-with-trump-opens-up-about-white-house-bid-damaging-biden-and-everything-in-between/\#6774579e47aa}{he
told Forbes} this month. ``So when they say the way we're going to fix
Covid is with a vaccine, I'm extremely cautious. That's the mark of the
beast.''

A poll in May by
\href{https://apnews.com/dacdc8bc428dd4df6511bfa259cfec44}{The
Associated Press-NORC Center for Public Affairs Research} found that
only about half of Americans said they would be willing to get a
coronavirus vaccine. One in five said they would refuse and 31 percent
were uncertain.
\href{https://www.newsweek.com/will-black-americans-fear-vaccine-more-covid-19-opinion-1516087}{A
poll in late June by researchers at the University of Miami} found that
22 percent of white and Latino respondents and 42 percent of Black
respondents said they agreed with this statement: ``The coronavirus is
being used to force a dangerous and unnecessary vaccine on Americans.''

``The trust issues are just tremendous in the Black community,'' said
Edith Perry, a member of the
\href{https://sph.umd.edu/center/che/maryland-community-research-advisory-board-md-crab}{Maryland
Community Research Advisory Board}, which seeks to ensure that the
benefits of health research encompass Black and Latino communities.

The solution, she said, is not just to employ the conventional strategy
of meeting with Black church congregations, especially if the government
and vaccine producers want to reach millennials.

``The pharmaceutical industry would have to convince some of the young
people in Black Lives Matter to get on board,'' Mrs. Perry said. ``Throw
up your hands and say: `I apologize. I know we did it wrong and I need
your help to get it right.' Because we need a vaccine and we need Black
and Hispanic participation.''

The chatter at The Shop Spa, a large barbershop with a Black and Latino
clientele in Hyattsville, Md., underscores the challenges. Mike Brown,
the manager, whose staff members have been trained to talk up wellness
with clients, referred to the
\href{https://www.nytimes3xbfgragh.onion/1997/05/12/us/families-emerge-as-silent-victims-of-tuskegee-syphilis-experiment.html}{notorious
Tuskegee experiments}, and said, ``I hope they don't sabotage us
again.''

His clients and their families are still leery of drug companies, he
said. ``It's hard to trust that they're looking out for our
well-being,'' he continued. ``Me, I'm very skeptical about that shot. I
have my popcorn and my soda and I'm just watching it very carefully.''

Image

Mike Brown, manager of The Shop Spa in Hyattsville, Md., trains his
staff to extoll the benefits of wellness, but says he's wary of a
coronavirus shot.Credit...Michael A. McCoy for The New York Times

The new report on vaccine confidence includes input from epidemiologists
and experts in health inequities and communication. The overarching
recommendation is that public health agencies should listen to community
concerns early in the process, rather than issuing them directives from
on high after the fact. They should seek out trusted community leaders
to convey people's uncertainties around research transparency, access,
allocation and cost. Those representatives could, in turn, become
respected purveyors of updates, to combat what the World Health
Organization calls the ``infodemic'' of vaccine misinformation.

The strongest recommendations were about communities of color. The
authors urged that vaccines be provided for free and made available at
easy access neighborhood locales: churches, pharmacies, barbershops,
schools. Noting that the vaccine would be emerging at a time when
protests about systemic racism, not least in health care, have been
erupting, the researchers cautioned that if accessibility was perceived
to be unfair, the vaccine could become a flash point of continuing
unrest. And that perception could heighten mistrust of the vaccine.

\href{https://www.nytimes3xbfgragh.onion/2020/06/30/us/politics/fauci-coronavirus.html}{At
a recent Senate hearing}, Dr. Robert Redfield, director of the Centers
for Disease Control and Prevention, was asked repeatedly about plans to
address surging vaccine hesitation. He replied that discussions had been
underway for ``10 to 12 weeks.'' A spokesman for the C.D.C. declined to
elaborate after being asked repeatedly by The New York Times to do so.

\href{https://www.nytimes3xbfgragh.onion/news-event/coronavirus?action=click\&pgtype=Article\&state=default\&region=MAIN_CONTENT_3\&context=storylines_faq}{}

\hypertarget{the-coronavirus-outbreak-}{%
\subsubsection{The Coronavirus Outbreak
›}\label{the-coronavirus-outbreak-}}

\hypertarget{frequently-asked-questions}{%
\paragraph{Frequently Asked
Questions}\label{frequently-asked-questions}}

Updated August 3, 2020

\begin{itemize}
\item ~
  \hypertarget{im-a-small-business-owner-can-i-get-relief}{%
  \paragraph{I'm a small-business owner. Can I get
  relief?}\label{im-a-small-business-owner-can-i-get-relief}}

  \begin{itemize}
  \tightlist
  \item
    The
    \href{https://www.nytimes3xbfgragh.onion/article/small-business-loans-stimulus-grants-freelancers-coronavirus.html?action=click\&pgtype=Article\&state=default\&region=MAIN_CONTENT_3\&context=storylines_faq}{stimulus
    bills enacted in March} offer help for the millions of American
    small businesses. Those eligible for aid are businesses and
    nonprofit organizations with fewer than 500 workers, including sole
    proprietorships, independent contractors and freelancers. Some
    larger companies in some industries are also eligible. The help
    being offered, which is being managed by the Small Business
    Administration, includes the Paycheck Protection Program and the
    Economic Injury Disaster Loan program. But lots of folks have
    \href{https://www.nytimes3xbfgragh.onion/interactive/2020/05/07/business/small-business-loans-coronavirus.html?action=click\&pgtype=Article\&state=default\&region=MAIN_CONTENT_3\&context=storylines_faq}{not
    yet seen payouts.} Even those who have received help are confused:
    The rules are draconian, and some are stuck sitting on
    \href{https://www.nytimes3xbfgragh.onion/2020/05/02/business/economy/loans-coronavirus-small-business.html?action=click\&pgtype=Article\&state=default\&region=MAIN_CONTENT_3\&context=storylines_faq}{money
    they don't know how to use.} Many small-business owners are getting
    less than they expected or
    \href{https://www.nytimes3xbfgragh.onion/2020/06/10/business/Small-business-loans-ppp.html?action=click\&pgtype=Article\&state=default\&region=MAIN_CONTENT_3\&context=storylines_faq}{not
    hearing anything at all.}
  \end{itemize}
\item ~
  \hypertarget{what-are-my-rights-if-i-am-worried-about-going-back-to-work}{%
  \paragraph{What are my rights if I am worried about going back to
  work?}\label{what-are-my-rights-if-i-am-worried-about-going-back-to-work}}

  \begin{itemize}
  \tightlist
  \item
    Employers have to provide
    \href{https://www.osha.gov/SLTC/covid-19/standards.html}{a safe
    workplace} with policies that protect everyone equally.
    \href{https://www.nytimes3xbfgragh.onion/article/coronavirus-money-unemployment.html?action=click\&pgtype=Article\&state=default\&region=MAIN_CONTENT_3\&context=storylines_faq}{And
    if one of your co-workers tests positive for the coronavirus, the
    C.D.C.} has said that
    \href{https://www.cdc.gov/coronavirus/2019-ncov/community/guidance-business-response.html}{employers
    should tell their employees} -\/- without giving you the sick
    employee's name -\/- that they may have been exposed to the virus.
  \end{itemize}
\item ~
  \hypertarget{should-i-refinance-my-mortgage}{%
  \paragraph{Should I refinance my
  mortgage?}\label{should-i-refinance-my-mortgage}}

  \begin{itemize}
  \tightlist
  \item
    \href{https://www.nytimes3xbfgragh.onion/article/coronavirus-money-unemployment.html?action=click\&pgtype=Article\&state=default\&region=MAIN_CONTENT_3\&context=storylines_faq}{It
    could be a good idea,} because mortgage rates have
    \href{https://www.nytimes3xbfgragh.onion/2020/07/16/business/mortgage-rates-below-3-percent.html?action=click\&pgtype=Article\&state=default\&region=MAIN_CONTENT_3\&context=storylines_faq}{never
    been lower.} Refinancing requests have pushed mortgage applications
    to some of the highest levels since 2008, so be prepared to get in
    line. But defaults are also up, so if you're thinking about buying a
    home, be aware that some lenders have tightened their standards.
  \end{itemize}
\item ~
  \hypertarget{what-is-school-going-to-look-like-in-september}{%
  \paragraph{What is school going to look like in
  September?}\label{what-is-school-going-to-look-like-in-september}}

  \begin{itemize}
  \tightlist
  \item
    It is unlikely that many schools will return to a normal schedule
    this fall, requiring the grind of
    \href{https://www.nytimes3xbfgragh.onion/2020/06/05/us/coronavirus-education-lost-learning.html?action=click\&pgtype=Article\&state=default\&region=MAIN_CONTENT_3\&context=storylines_faq}{online
    learning},
    \href{https://www.nytimes3xbfgragh.onion/2020/05/29/us/coronavirus-child-care-centers.html?action=click\&pgtype=Article\&state=default\&region=MAIN_CONTENT_3\&context=storylines_faq}{makeshift
    child care} and
    \href{https://www.nytimes3xbfgragh.onion/2020/06/03/business/economy/coronavirus-working-women.html?action=click\&pgtype=Article\&state=default\&region=MAIN_CONTENT_3\&context=storylines_faq}{stunted
    workdays} to continue. California's two largest public school
    districts --- Los Angeles and San Diego --- said on July 13, that
    \href{https://www.nytimes3xbfgragh.onion/2020/07/13/us/lausd-san-diego-school-reopening.html?action=click\&pgtype=Article\&state=default\&region=MAIN_CONTENT_3\&context=storylines_faq}{instruction
    will be remote-only in the fall}, citing concerns that surging
    coronavirus infections in their areas pose too dire a risk for
    students and teachers. Together, the two districts enroll some
    825,000 students. They are the largest in the country so far to
    abandon plans for even a partial physical return to classrooms when
    they reopen in August. For other districts, the solution won't be an
    all-or-nothing approach.
    \href{https://bioethics.jhu.edu/research-and-outreach/projects/eschool-initiative/school-policy-tracker/}{Many
    systems}, including the nation's largest, New York City, are
    devising
    \href{https://www.nytimes3xbfgragh.onion/2020/06/26/us/coronavirus-schools-reopen-fall.html?action=click\&pgtype=Article\&state=default\&region=MAIN_CONTENT_3\&context=storylines_faq}{hybrid
    plans} that involve spending some days in classrooms and other days
    online. There's no national policy on this yet, so check with your
    municipal school system regularly to see what is happening in your
    community.
  \end{itemize}
\item ~
  \hypertarget{is-the-coronavirus-airborne}{%
  \paragraph{Is the coronavirus
  airborne?}\label{is-the-coronavirus-airborne}}

  \begin{itemize}
  \tightlist
  \item
    The coronavirus
    \href{https://www.nytimes3xbfgragh.onion/2020/07/04/health/239-experts-with-one-big-claim-the-coronavirus-is-airborne.html?action=click\&pgtype=Article\&state=default\&region=MAIN_CONTENT_3\&context=storylines_faq}{can
    stay aloft for hours in tiny droplets in stagnant air}, infecting
    people as they inhale, mounting scientific evidence suggests. This
    risk is highest in crowded indoor spaces with poor ventilation, and
    may help explain super-spreading events reported in meatpacking
    plants, churches and restaurants.
    \href{https://www.nytimes3xbfgragh.onion/2020/07/06/health/coronavirus-airborne-aerosols.html?action=click\&pgtype=Article\&state=default\&region=MAIN_CONTENT_3\&context=storylines_faq}{It's
    unclear how often the virus is spread} via these tiny droplets, or
    aerosols, compared with larger droplets that are expelled when a
    sick person coughs or sneezes, or transmitted through contact with
    contaminated surfaces, said Linsey Marr, an aerosol expert at
    Virginia Tech. Aerosols are released even when a person without
    symptoms exhales, talks or sings, according to Dr. Marr and more
    than 200 other experts, who
    \href{https://academic.oup.com/cid/article/doi/10.1093/cid/ciaa939/5867798}{have
    outlined the evidence in an open letter to the World Health
    Organization}.
  \end{itemize}
\end{itemize}

\href{https://www.txstate.edu/anthropology/people/faculty/brunson.html}{Emily
Brunson}, a medical anthropologist at Texas State University, said that
the myriad number of reasons people may be skeptical of this vaccine,
combined with the vast, unsparing reach of Covid-19 itself, meant that
creating a campaign for the vaccine's acceptance would be far more
difficult than one for a more narrowly defined group --- shingles
vaccine for older people, HPV vaccine for preteens. The researchers said
that a national promotional strategy should be in the planning stages as
soon as possible.

Over all, the worry that is consistently invoked by those hesitant about
this vaccine is haste. When health authorities repeatedly tout the
rapidity of development --- an idea underscored by the name Operation
Warp Speed --- they inadvertently aggravate the public's safety
concerns.

Image

Dr. Robert Redfield, the C.D.C. director, appeared before a Senate
committee on July 2 to talk about manufacturing a coronavirus
vaccine.Credit...Pool photo by Graeme Jennings

``If you're smart, you're worried we won't have a vaccine, and if you're
smart, you're worried that maybe we've moved so fast that we'll accept a
level of risk that we might not ordinarily accept,'' said
\href{https://sph.umd.edu/department/fmsc/bio/18882}{Sandra Crouse
Quinn}, a professor of public health at the University of Maryland.

Health communication experts say that those trying to persuade the
vaccine-hesitant to be immunized should not dismiss them as
``anti-vaxxers,'' which has become an insult and shuts down
conversations.

``You always have to listen to their concerns,'' said Dr. Quinn, the
senior associate director of the Maryland Center for Health Equity, who
studies issues around health care trust in communities of color.

Last week, a nonprofit public health initiative, the
\href{https://publicgoodprojects.org/}{Public Good Projects,} introduced
\href{https://stronger.org/}{Stronger}, a campaign to combat vaccine
misinformation, with a plethora of tips, including lists of established
scientists to follow on Twitter.

One path toward increasing the acceptance of the vaccine, Dr. Quinn
said, is to appeal to people's innate altruism: ``that getting a
vaccine, when it's available, is not just about you. It's about
protecting your grandmother who has diabetes and Uncle Sean, who is
immune-compromised,'' she said.

And when people respond by listing their objections to the vaccine, ask
them, she said, ``If that's what you think, then how do you protect your
community?''

Advertisement

\protect\hyperlink{after-bottom}{Continue reading the main story}

\hypertarget{site-index}{%
\subsection{Site Index}\label{site-index}}

\hypertarget{site-information-navigation}{%
\subsection{Site Information
Navigation}\label{site-information-navigation}}

\begin{itemize}
\tightlist
\item
  \href{https://help.nytimes3xbfgragh.onion/hc/en-us/articles/115014792127-Copyright-notice}{©~2020~The
  New York Times Company}
\end{itemize}

\begin{itemize}
\tightlist
\item
  \href{https://www.nytco.com/}{NYTCo}
\item
  \href{https://help.nytimes3xbfgragh.onion/hc/en-us/articles/115015385887-Contact-Us}{Contact
  Us}
\item
  \href{https://www.nytco.com/careers/}{Work with us}
\item
  \href{https://nytmediakit.com/}{Advertise}
\item
  \href{http://www.tbrandstudio.com/}{T Brand Studio}
\item
  \href{https://www.nytimes3xbfgragh.onion/privacy/cookie-policy\#how-do-i-manage-trackers}{Your
  Ad Choices}
\item
  \href{https://www.nytimes3xbfgragh.onion/privacy}{Privacy}
\item
  \href{https://help.nytimes3xbfgragh.onion/hc/en-us/articles/115014893428-Terms-of-service}{Terms
  of Service}
\item
  \href{https://help.nytimes3xbfgragh.onion/hc/en-us/articles/115014893968-Terms-of-sale}{Terms
  of Sale}
\item
  \href{https://spiderbites.nytimes3xbfgragh.onion}{Site Map}
\item
  \href{https://help.nytimes3xbfgragh.onion/hc/en-us}{Help}
\item
  \href{https://www.nytimes3xbfgragh.onion/subscription?campaignId=37WXW}{Subscriptions}
\end{itemize}
