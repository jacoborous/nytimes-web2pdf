Sections

SEARCH

\protect\hyperlink{site-content}{Skip to
content}\protect\hyperlink{site-index}{Skip to site index}

\href{https://www.nytimes3xbfgragh.onion/section/opinion/sunday}{Sunday
Review}

\href{https://myaccount.nytimes3xbfgragh.onion/auth/login?response_type=cookie\&client_id=vi}{}

\href{https://www.nytimes3xbfgragh.onion/section/todayspaper}{Today's
Paper}

\href{/section/opinion/sunday}{Sunday Review}\textbar{}The Real White
Fragility

\url{https://nyti.ms/3jmcDUa}

\begin{itemize}
\item
\item
\item
\item
\item
\item
\end{itemize}

Advertisement

\protect\hyperlink{after-top}{Continue reading the main story}

\href{/section/opinion}{Opinion}

Supported by

\protect\hyperlink{after-sponsor}{Continue reading the main story}

\hypertarget{the-real-white-fragility}{%
\section{The Real White Fragility}\label{the-real-white-fragility}}

Does the white upper class feel exhausted and oppressed by meritocracy?

\href{https://www.nytimes3xbfgragh.onion/by/ross-douthat}{\includegraphics{https://static01.graylady3jvrrxbe.onion/images/2018/04/03/opinion/ross-douthat/ross-douthat-thumbLarge.png}}

By \href{https://www.nytimes3xbfgragh.onion/by/ross-douthat}{Ross
Douthat}

Opinion Columnist

\begin{itemize}
\item
  July 18, 2020
\item
  \begin{itemize}
  \item
  \item
  \item
  \item
  \item
  \item
  \end{itemize}
\end{itemize}

\includegraphics{https://static01.graylady3jvrrxbe.onion/images/2020/07/19/opinion/19Douthat/19Douthat-articleLarge.jpg?quality=75\&auto=webp\&disable=upscale}

In 2001, when I was still attending college, David Brooks wrote an essay
for The Atlantic called
``\href{https://www.theatlantic.com/magazine/archive/2001/04/the-organization-kid/302164/}{The
Organization Kid},'' in which he spent a lot of time with young Ivy
Leaguers and came away struck by their basic existential contentment.
Instead of campus rebels, they were résumé builders and accomplishment
collectors and apple polishers, distinguished by their serenity, their
faux-adult professionalism, their politesse.

I thought at the time that Brooks made my cohort out to be more decent
than we really were, mistaking the mask we wore for encounters with,
say, an Atlantic journalist for our truer,
\href{https://www.nationalreview.com/magazine/2010/10/18/film-what-makes-preppy-run/}{darker},
more ambitious selves. But he was entirely correct that most of my peers
believed that meritocracy was fair and just and \emph{worked} ---
because after all it seemed to work for us.

I graduated the year after ``The Organization Kid'' ran, wrote a lot
about college in my 20s, and then drifted to other interests and
obsessions. To the extent that I followed the college admissions racket
thereafter, it seemed to become more competitive, more ruthless, more
itself --- and to extend its rigors ever earlier into childhood.

But a few years ago we moved back to the college town where I grew up,
which gave me a close vantage point on young-meritocratic life again.
Some of the striving culture that Brooks described remains very much in
place. But talking to students and professors, the most striking
difference is the disappearance of serenity, the evaporation of
contentment, the spread of anxiety and mental illness --- with the
reputed scale of antidepressant use a particular stark marker of this
change.

I don't think this alteration just reflects a darkening vision of the
wider world, a fear of climate change or Donald Trump. It also reflects
a transformation within the meritocracy itself --- a sense in which,
since 2001, the system has consistently been asking more of ladder
climbers and delivering less as its reward.

The scholar Peter Turchin of the University of Connecticut, whose work
on the cycles of American history may have predicted this year's unrest,
has a phrase that describes part of this dynamic: the
``\href{http://peterturchin.com/age-of-discord/}{overproduction of
elites}.'' In the context of college admissions that means exactly what
it sounds like: We've had a surplus of smart young Americans pursuing
admission to a narrow list of elite colleges whose enrollment doesn't
expand with population, even as foreign students increasingly compete
for the same stagnant share of slots.

Then, having run this gantlet, our meritocrats graduate into a big-city
ecosystem where the price of adult goods like schools and housing has
been bid up dramatically, while important cultural industries ---
especially academia and journalism --- supply fewer jobs even in good
economic times. And they live half in these crowded, over-competitive
worlds and half on the internet, which has extended the competition for
status
\href{https://www.mercatus.org/bridge/commentary/looking-glass-politics}{almost
infinitely} and weakened some of the normal ways that local prestige
might compensate for disappointing income.

These stresses have exposed the thinness of meritocracy as a culture, a
Hogwarts with SATs instead of magic, a secular substitute for older
forms of community, tradition or religion. For instance, it was the
frequent boast of Obama-era liberalism that it had restored certain
bourgeois virtues --- delayed childbearing, stable marriages --- without
requiring anything so anachronistic as Christianity or courtship
rituals. But if your bourgeois order is built on a cycle of competition
and reward, and the competition gets fiercer while the rewards diminish,
then instead of young people
\href{https://www.theatlantic.com/magazine/archive/2012/09/boys-on-the-side/309062/}{hooking
up safely} on the way to a lucrative job and a dual-income marriage with
2.1 kids, you'll get young people set adrift, unable to pair off,
postponing marriage permanently while they wait for a stability that
never comes.

Which brings us to the subject invoked in this column's title --- the
increasing appeal, to these unhappy young people and to their parents
and educators as well, of an emergent ideology that accuses many of them
of embodying white privilege, and of being ``fragile,'' in the words of
the now-famous anti-racism consultant
\href{https://www.nytimes3xbfgragh.onion/2020/07/15/magazine/white-fragility-robin-diangelo.html}{Robin
DiAngelo}, if they object or disagree.

Part of this ideology's appeal is clearly about meaning and morality:
The new anti-racism has a confessional,
\href{https://www.nytimes3xbfgragh.onion/2020/07/07/opinion/protestant-progressive-reformation.html}{religious
energy} that the secular meritocracy has always lacked. But there is
also something important about its more radical and even ridiculous
elements --- like the weird business that increasingly shows up in
official documents, from the
\href{https://nypost.com/2019/05/20/richard-carranza-held-doe-white-supremacy-culture-training/}{New
York Public Schools} or the
\href{https://twitter.com/ByronYork/status/1283372233730203651}{Smithsonian},
describing
\href{https://www.nytimes3xbfgragh.onion/2020/07/15/magazine/white-fragility-robin-diangelo.html}{things}
like ``perfectionism'' or ``worship of the written word'' or ``emphasis
on the scientific method'' or ``delayed gratification'' as features of a
toxic whiteness.

Imagine yourself as a relatively privileged white person exhausted by
meritocracy --- an overworked student or a fretful parent or a school
administrator constantly besieged by both. (Given the demographics of
this paper's readership, this may not require much imagination.)

Wouldn't it come as a relief, in some way, if it turned out that the
whole ``exhausting `Alice in Wonderland' Red Queen Race of full-time
meritocratic achievement,'' in the
\href{https://americanmind.org/features/the-death-of-virtue-and-the-rise-of-expertise/waking-from-meritocracy/}{words}
of a pseudonymous critic, was nothing more than a manifestation of the
very white supremacy that you, as a good liberal, are obliged to
dismantle and oppose? If all the testing, all the ``delayed
gratification'' and ``perfectionism,'' was, after all, just itself a
form of racism, and in easing up, chilling out, just \emph{relaxing} a
little bit, you can improve your life and your kid's life and, happily,
strike an anti-racist blow as well?

And wouldn't it be especially appealing if --- and here I'm afraid I'm
going to be very cynical --- in the course of relaxing the demands of
whiteness you could, just coincidentally, make your own family's
position a little bit more secure?

For instance: Once you dismiss the SAT as just a tool of white
supremacy, then it gets easier for elite schools to justify excluding
the Asian-American students whose standardized-test scores keep climbing
while white scores stay relatively flat. Or again: If you induce
inner-city charter schools to
\href{https://www.kipp.org/words-into-action/}{disavow} their previous
stress on hard work and discipline and meritocratic ambition, because
those are racist, too --- well, then their minority graduates might
become less competitive with your own kids in the college-admissions
race as well.

Not that anyone is consciously thinking like this. What I'm describing
is a subtle and subconscious current, deep down in the progressive
stream.

But deep currents can run strong. And if the avowed intention of the
moment is to challenge ``white fragility'' and yet lots of white people
seem strangely enthusiastic about the challenge, it's worth considering
that maybe a different kind of fragility is in play: The stress and
unhappiness felt by meritocracy's strivers, who may be open to a
revolution that seems to promise more stability and less exhaustion, and
asks them only to denounce the ``whiteness'' of a system that's made
even its most successful participants feel fragile and existentially
depressed.

\emph{The Times is committed to publishing}
\href{https://www.nytimes3xbfgragh.onion/2019/01/31/opinion/letters/letters-to-editor-new-york-times-women.html}{\emph{a
diversity of letters}} \emph{to the editor. We'd like to hear what you
think about this or any of our articles. Here are some}
\href{https://help.nytimes3xbfgragh.onion/hc/en-us/articles/115014925288-How-to-submit-a-letter-to-the-editor}{\emph{tips}}\emph{.
And here's our email:}
\href{mailto:letters@NYTimes.com}{\emph{letters@NYTimes.com}}\emph{.}

\emph{Follow The New York Times Opinion section on}
\href{https://www.facebookcorewwwi.onion/nytopinion}{\emph{Facebook}}\emph{,}
\href{http://twitter.com/NYTOpinion}{\emph{Twitter (@NYTOpinion)}}
\emph{and}
\href{https://www.instagram.com/nytopinion/}{\emph{Instagram}}\emph{,
join the Facebook political discussion group,}
\href{https://www.facebookcorewwwi.onion/groups/votingwhilefemale/}{\emph{Voting
While Female}}\emph{.}

Advertisement

\protect\hyperlink{after-bottom}{Continue reading the main story}

\hypertarget{site-index}{%
\subsection{Site Index}\label{site-index}}

\hypertarget{site-information-navigation}{%
\subsection{Site Information
Navigation}\label{site-information-navigation}}

\begin{itemize}
\tightlist
\item
  \href{https://help.nytimes3xbfgragh.onion/hc/en-us/articles/115014792127-Copyright-notice}{©~2020~The
  New York Times Company}
\end{itemize}

\begin{itemize}
\tightlist
\item
  \href{https://www.nytco.com/}{NYTCo}
\item
  \href{https://help.nytimes3xbfgragh.onion/hc/en-us/articles/115015385887-Contact-Us}{Contact
  Us}
\item
  \href{https://www.nytco.com/careers/}{Work with us}
\item
  \href{https://nytmediakit.com/}{Advertise}
\item
  \href{http://www.tbrandstudio.com/}{T Brand Studio}
\item
  \href{https://www.nytimes3xbfgragh.onion/privacy/cookie-policy\#how-do-i-manage-trackers}{Your
  Ad Choices}
\item
  \href{https://www.nytimes3xbfgragh.onion/privacy}{Privacy}
\item
  \href{https://help.nytimes3xbfgragh.onion/hc/en-us/articles/115014893428-Terms-of-service}{Terms
  of Service}
\item
  \href{https://help.nytimes3xbfgragh.onion/hc/en-us/articles/115014893968-Terms-of-sale}{Terms
  of Sale}
\item
  \href{https://spiderbites.nytimes3xbfgragh.onion}{Site Map}
\item
  \href{https://help.nytimes3xbfgragh.onion/hc/en-us}{Help}
\item
  \href{https://www.nytimes3xbfgragh.onion/subscription?campaignId=37WXW}{Subscriptions}
\end{itemize}
