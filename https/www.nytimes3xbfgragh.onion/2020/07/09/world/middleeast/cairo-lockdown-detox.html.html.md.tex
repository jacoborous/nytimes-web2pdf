Sections

SEARCH

\protect\hyperlink{site-content}{Skip to
content}\protect\hyperlink{site-index}{Skip to site index}

\href{https://www.nytimes3xbfgragh.onion/section/world/middleeast}{Middle
East}

\href{https://myaccount.nytimes3xbfgragh.onion/auth/login?response_type=cookie\&client_id=vi}{}

\href{https://www.nytimes3xbfgragh.onion/section/todayspaper}{Today's
Paper}

\href{/section/world/middleeast}{Middle East}\textbar{}Cairo Badly
Needed a Detox. Lockdown Supplied One, at a Steep Price.

\url{https://nyti.ms/32117Ho}

\begin{itemize}
\item
\item
\item
\item
\item
\end{itemize}

\href{https://www.nytimes3xbfgragh.onion/news-event/coronavirus?action=click\&pgtype=Article\&state=default\&region=TOP_BANNER\&context=storylines_menu}{The
Coronavirus Outbreak}

\begin{itemize}
\tightlist
\item
  live\href{https://www.nytimes3xbfgragh.onion/2020/08/01/world/coronavirus-covid-19.html?action=click\&pgtype=Article\&state=default\&region=TOP_BANNER\&context=storylines_menu}{Latest
  Updates}
\item
  \href{https://www.nytimes3xbfgragh.onion/interactive/2020/us/coronavirus-us-cases.html?action=click\&pgtype=Article\&state=default\&region=TOP_BANNER\&context=storylines_menu}{Maps
  and Cases}
\item
  \href{https://www.nytimes3xbfgragh.onion/interactive/2020/science/coronavirus-vaccine-tracker.html?action=click\&pgtype=Article\&state=default\&region=TOP_BANNER\&context=storylines_menu}{Vaccine
  Tracker}
\item
  \href{https://www.nytimes3xbfgragh.onion/interactive/2020/07/29/us/schools-reopening-coronavirus.html?action=click\&pgtype=Article\&state=default\&region=TOP_BANNER\&context=storylines_menu}{What
  School May Look Like}
\item
  \href{https://www.nytimes3xbfgragh.onion/live/2020/07/31/business/stock-market-today-coronavirus?action=click\&pgtype=Article\&state=default\&region=TOP_BANNER\&context=storylines_menu}{Economy}
\end{itemize}

Advertisement

\protect\hyperlink{after-top}{Continue reading the main story}

Supported by

\protect\hyperlink{after-sponsor}{Continue reading the main story}

Egypt Dispatch

\hypertarget{cairo-badly-needed-a-detox-lockdown-supplied-one-at-a-steep-price}{%
\section{Cairo Badly Needed a Detox. Lockdown Supplied One, at a Steep
Price.}\label{cairo-badly-needed-a-detox-lockdown-supplied-one-at-a-steep-price}}

Three months of lockdown slowed its pulse, stripped its grit and exposed
a new side to a weary city. But without the noise, bustle and grind, was
it really Cairo?

\includegraphics{https://static01.graylady3jvrrxbe.onion/images/2020/07/06/world/00cairo-dispatch1/merlin_174278163_69dce3c2-fb1e-4853-b995-058c120abf84-articleLarge.jpg?quality=75\&auto=webp\&disable=upscale}

\href{https://www.nytimes3xbfgragh.onion/by/declan-walsh}{\includegraphics{https://static01.graylady3jvrrxbe.onion/images/2018/10/15/multimedia/author-declan-walsh/author-declan-walsh-thumbLarge-v3.png}}

By \href{https://www.nytimes3xbfgragh.onion/by/declan-walsh}{Declan
Walsh}

\begin{itemize}
\item
  Published July 9, 2020Updated July 12, 2020
\item
  \begin{itemize}
  \item
  \item
  \item
  \item
  \item
  \end{itemize}
\end{itemize}

CAIRO --- If ever a city needed a good detox, it was Cairo.

Centuries of turbulent history, topped with recent decades of chaotic
urban development, have left the ancient metropolis in poor physical
shape. Its complexion is parched and blemished. Traffic clogs its
throbbing arteries. It has signs of major stress.

The coronavirus obliged. Three months of lockdown, including an 11-hour
nightly curfew, imposed a rejuvenating deep cleanse on Cairo. Roads once
choked with honking cars ran free. The air, free of fumes, seemed to
sparkle. Silence flooded the streets.

At my apartment near the Nile, a bedroom that was barely usable because
of the morning din became an oasis of calm. In the evenings, my family
gathered on the balcony to witness sunsets that were more saturated than
ever. The pollution app on my phone glowed an unfamiliar green.

Of course, it came at a jarring price. On dawn runs down deserted
streets, I passed anxious-looking people wearing masks, crowded around
the entrance to a hospital.

And then it was over.

Toward the end of June, the government announced it was allowing
mosques, restaurants and coffee houses to reopen. On the last night of
curfew, I scrambled into the streets to capture its delicate pleasures
one final time. Hundreds of Egyptians, it turned out, had the same idea.

They clustered at dusk on a bridge, watching the squadron of kites that
fluttered in the hot breeze shooting down the Nile. Young men in skinny
jeans tugged on strings. Veiled women chased after dating couples,
trying to sell them roses.

\includegraphics{https://static01.graylady3jvrrxbe.onion/images/2020/07/06/world/00cairo-dispatch5/merlin_174181782_03d568fd-090b-47e9-ac93-9208ce0314be-articleLarge.jpg?quality=75\&auto=webp\&disable=upscale}

Inevitably, the fun tweaked Egypt's rulers, always wary of unsanctioned
public gatherings. A senior lawmaker warned that the crowded skies posed
a threat to national security because Egypt's enemies could fit the
kites with surveillance cameras.

But on the bridge, nobody cared for such talk, preferring to wallow in
this odd moment, between serenity and anxiety, when their city's
famously frenetic pulse had been slowed by a virus.

I chatted with two brothers who held aloft a giant kite emblazoned with
photographs of themselves, preening, and the soccer star Mohamed Salah,
who was beaming. Nearby, Samiha Meneim, 62, perched on a rickety plastic
chair, surrounded by 15 family members as well as half-eaten platters of
koshary, Egypt's national dish of spiced lentils, rice and macaroni.

\hypertarget{latest-updates-global-coronavirus-outbreak}{%
\section{\texorpdfstring{\href{https://www.nytimes3xbfgragh.onion/2020/08/01/world/coronavirus-covid-19.html?action=click\&pgtype=Article\&state=default\&region=MAIN_CONTENT_1\&context=storylines_live_updates}{Latest
Updates: Global Coronavirus
Outbreak}}{Latest Updates: Global Coronavirus Outbreak}}\label{latest-updates-global-coronavirus-outbreak}}

Updated 2020-08-01T19:54:00.494Z

\begin{itemize}
\tightlist
\item
  \href{https://www.nytimes3xbfgragh.onion/2020/08/01/world/coronavirus-covid-19.html?action=click\&pgtype=Article\&state=default\&region=MAIN_CONTENT_1\&context=storylines_live_updates\#link-3ac56579}{Top
  officials work to break impasse over jobless benefit.}
\item
  \href{https://www.nytimes3xbfgragh.onion/2020/08/01/world/coronavirus-covid-19.html?action=click\&pgtype=Article\&state=default\&region=MAIN_CONTENT_1\&context=storylines_live_updates\#link-8796723}{The
  virus picks up dangerous speed in the Midwest, and in areas that had
  seen success.}
\item
  \href{https://www.nytimes3xbfgragh.onion/2020/08/01/world/coronavirus-covid-19.html?action=click\&pgtype=Article\&state=default\&region=MAIN_CONTENT_1\&context=storylines_live_updates\#link-25930521}{Thousands
  in Berlin protest Germany's coronavirus measures.}
\end{itemize}

\href{https://www.nytimes3xbfgragh.onion/2020/08/01/world/coronavirus-covid-19.html?action=click\&pgtype=Article\&state=default\&region=MAIN_CONTENT_1\&context=storylines_live_updates}{See
more updates}

More live coverage:
\href{https://www.nytimes3xbfgragh.onion/live/2020/07/31/business/stock-market-today-coronavirus?action=click\&pgtype=Article\&state=default\&region=MAIN_CONTENT_1\&context=storylines_live_updates}{Markets}

The picnic was a mercy dash after months cooped up in their cramped,
low-rent neighborhood. ``We had to get out,'' said Ms. Meneim, a retired
nurse in a black cloak who continued her treatment for breast cancer
throughout the lockdown.

She saw the coronavirus as a message from God. ``He wants us to look at
life differently,'' she said.

For much of Egypt's history, its fate has been shaped by the Nile. The
bridge of kites led to Roda Island, described in
``\href{https://www.britannica.com/topic/The-Thousand-and-One-Nights}{One
Thousand and One Nights}'' as a place of heavenly gardens, now a tight
sprawl of dust-smeared apartment blocks. At its southern tip, though,
there survives a Nilometer built in the ninth century ---~a structure
that measured the river's seasonal flooding and thus predicted the
annual harvest.

Image

An aerial view of the Nile.Credit...Khaled Desouki/Agence France-Presse
--- Getty Images

Now, disease was dictating the pace of life. As night fell and the
curfew officially began, I crossed into downtown Cairo, a jumble of old
palaces, crumbling elegance and gaudy shop fronts where, in normal
times, the traffic is so crazy that guidebooks offer tourists solemn
advice on how to survive.

``Look for locals and join a group,'' advises my edition of National
Geographic Traveler. ``They cross all together, one lane at a time.''

That night it would have taken a miracle to get knocked down. The strays
were in charge ---~skinny cats that strutted down empty boulevards, for
once unbothered, and a pair of lordly street dogs that snoozed atop an
SUV.

The Metro Cinema, with its dust-encrusted Art Deco facade, opened in
1940 with ``Gone With the Wind.'' Now it had the eerie air of an
abandoned film set, advertising the Egyptian movies it had been showing
in March: ``Peep Show,'' and ``The Thief of Baghdad.''

In the late 19th century, Egypt's ruler, Khedive Ismail, modeled this
area on the airy elegance of Haussmann's Paris, but for decades the
graceful buildings have gradually crumbled into disrepair. Now, in the
desolation of curfew, they seemed to stand proud again, as did the
statues lining the way.

Image

The Metro Cinema in Cairo's historical center.Credit...Declan Walsh/The
New York Times

The giant bronze lions that guard Qasr el Nil, the city's most scenic
bridge, looked more relaxed than ever with no foe in sight.

The mix of eerie desolation and faded splendor had a touch of magic, and
for an instant I thought of
\href{https://youtu.be/_CaAODL1XjM?t=5602}{the Egyptian version of the
movie ``Night at the Museum,''} in which the bronze lions come to life
under darkness.

But I was not entirely alone.

Teenagers clustered conspiratorially in doorways. Food-delivery riders
clustered around their motorcycles outside a restaurant. Business was
brisk.

``If it keeps going like this,'' remarked Mahmoud Abdel Fattah, leaning
over his handlebars, ``I'll be as rich as Naguib Sawiris'' ---~an
outspoken billionaire who has been a vociferous critic of the lockdown
measures.

Still, Mr. Fattah noted wryly that, at 28 cents per delivery, that
fortune could take a while. ``Maybe after one million pizzas,'' he
quipped.

For all their ebullience, the deliverymen also had a downbeat air. Sure,
they could zoom to any address in minutes. But Cairo without the grit,
the grind, the bustle of people --- was it really Cairo at all?

\href{https://www.nytimes3xbfgragh.onion/news-event/coronavirus?action=click\&pgtype=Article\&state=default\&region=MAIN_CONTENT_3\&context=storylines_faq}{}

\hypertarget{the-coronavirus-outbreak-}{%
\subsubsection{The Coronavirus Outbreak
›}\label{the-coronavirus-outbreak-}}

\hypertarget{frequently-asked-questions}{%
\paragraph{Frequently Asked
Questions}\label{frequently-asked-questions}}

Updated July 27, 2020

\begin{itemize}
\item ~
  \hypertarget{should-i-refinance-my-mortgage}{%
  \paragraph{Should I refinance my
  mortgage?}\label{should-i-refinance-my-mortgage}}

  \begin{itemize}
  \tightlist
  \item
    \href{https://www.nytimes3xbfgragh.onion/article/coronavirus-money-unemployment.html?action=click\&pgtype=Article\&state=default\&region=MAIN_CONTENT_3\&context=storylines_faq}{It
    could be a good idea,} because mortgage rates have
    \href{https://www.nytimes3xbfgragh.onion/2020/07/16/business/mortgage-rates-below-3-percent.html?action=click\&pgtype=Article\&state=default\&region=MAIN_CONTENT_3\&context=storylines_faq}{never
    been lower.} Refinancing requests have pushed mortgage applications
    to some of the highest levels since 2008, so be prepared to get in
    line. But defaults are also up, so if you're thinking about buying a
    home, be aware that some lenders have tightened their standards.
  \end{itemize}
\item ~
  \hypertarget{what-is-school-going-to-look-like-in-september}{%
  \paragraph{What is school going to look like in
  September?}\label{what-is-school-going-to-look-like-in-september}}

  \begin{itemize}
  \tightlist
  \item
    It is unlikely that many schools will return to a normal schedule
    this fall, requiring the grind of
    \href{https://www.nytimes3xbfgragh.onion/2020/06/05/us/coronavirus-education-lost-learning.html?action=click\&pgtype=Article\&state=default\&region=MAIN_CONTENT_3\&context=storylines_faq}{online
    learning},
    \href{https://www.nytimes3xbfgragh.onion/2020/05/29/us/coronavirus-child-care-centers.html?action=click\&pgtype=Article\&state=default\&region=MAIN_CONTENT_3\&context=storylines_faq}{makeshift
    child care} and
    \href{https://www.nytimes3xbfgragh.onion/2020/06/03/business/economy/coronavirus-working-women.html?action=click\&pgtype=Article\&state=default\&region=MAIN_CONTENT_3\&context=storylines_faq}{stunted
    workdays} to continue. California's two largest public school
    districts --- Los Angeles and San Diego --- said on July 13, that
    \href{https://www.nytimes3xbfgragh.onion/2020/07/13/us/lausd-san-diego-school-reopening.html?action=click\&pgtype=Article\&state=default\&region=MAIN_CONTENT_3\&context=storylines_faq}{instruction
    will be remote-only in the fall}, citing concerns that surging
    coronavirus infections in their areas pose too dire a risk for
    students and teachers. Together, the two districts enroll some
    825,000 students. They are the largest in the country so far to
    abandon plans for even a partial physical return to classrooms when
    they reopen in August. For other districts, the solution won't be an
    all-or-nothing approach.
    \href{https://bioethics.jhu.edu/research-and-outreach/projects/eschool-initiative/school-policy-tracker/}{Many
    systems}, including the nation's largest, New York City, are
    devising
    \href{https://www.nytimes3xbfgragh.onion/2020/06/26/us/coronavirus-schools-reopen-fall.html?action=click\&pgtype=Article\&state=default\&region=MAIN_CONTENT_3\&context=storylines_faq}{hybrid
    plans} that involve spending some days in classrooms and other days
    online. There's no national policy on this yet, so check with your
    municipal school system regularly to see what is happening in your
    community.
  \end{itemize}
\item ~
  \hypertarget{is-the-coronavirus-airborne}{%
  \paragraph{Is the coronavirus
  airborne?}\label{is-the-coronavirus-airborne}}

  \begin{itemize}
  \tightlist
  \item
    The coronavirus
    \href{https://www.nytimes3xbfgragh.onion/2020/07/04/health/239-experts-with-one-big-claim-the-coronavirus-is-airborne.html?action=click\&pgtype=Article\&state=default\&region=MAIN_CONTENT_3\&context=storylines_faq}{can
    stay aloft for hours in tiny droplets in stagnant air}, infecting
    people as they inhale, mounting scientific evidence suggests. This
    risk is highest in crowded indoor spaces with poor ventilation, and
    may help explain super-spreading events reported in meatpacking
    plants, churches and restaurants.
    \href{https://www.nytimes3xbfgragh.onion/2020/07/06/health/coronavirus-airborne-aerosols.html?action=click\&pgtype=Article\&state=default\&region=MAIN_CONTENT_3\&context=storylines_faq}{It's
    unclear how often the virus is spread} via these tiny droplets, or
    aerosols, compared with larger droplets that are expelled when a
    sick person coughs or sneezes, or transmitted through contact with
    contaminated surfaces, said Linsey Marr, an aerosol expert at
    Virginia Tech. Aerosols are released even when a person without
    symptoms exhales, talks or sings, according to Dr. Marr and more
    than 200 other experts, who
    \href{https://academic.oup.com/cid/article/doi/10.1093/cid/ciaa939/5867798}{have
    outlined the evidence in an open letter to the World Health
    Organization}.
  \end{itemize}
\item ~
  \hypertarget{what-are-the-symptoms-of-coronavirus}{%
  \paragraph{What are the symptoms of
  coronavirus?}\label{what-are-the-symptoms-of-coronavirus}}

  \begin{itemize}
  \tightlist
  \item
    Common symptoms
    \href{https://www.nytimes3xbfgragh.onion/article/symptoms-coronavirus.html?action=click\&pgtype=Article\&state=default\&region=MAIN_CONTENT_3\&context=storylines_faq}{include
    fever, a dry cough, fatigue and difficulty breathing or shortness of
    breath.} Some of these symptoms overlap with those of the flu,
    making detection difficult, but runny noses and stuffy sinuses are
    less common.
    \href{https://www.nytimes3xbfgragh.onion/2020/04/27/health/coronavirus-symptoms-cdc.html?action=click\&pgtype=Article\&state=default\&region=MAIN_CONTENT_3\&context=storylines_faq}{The
    C.D.C. has also} added chills, muscle pain, sore throat, headache
    and a new loss of the sense of taste or smell as symptoms to look
    out for. Most people fall ill five to seven days after exposure, but
    symptoms may appear in as few as two days or as many as 14 days.
  \end{itemize}
\item ~
  \hypertarget{does-asymptomatic-transmission-of-covid-19-happen}{%
  \paragraph{Does asymptomatic transmission of Covid-19
  happen?}\label{does-asymptomatic-transmission-of-covid-19-happen}}

  \begin{itemize}
  \tightlist
  \item
    So far, the evidence seems to show it does. A widely cited
    \href{https://www.nature.com/articles/s41591-020-0869-5}{paper}
    published in April suggests that people are most infectious about
    two days before the onset of coronavirus symptoms and estimated that
    44 percent of new infections were a result of transmission from
    people who were not yet showing symptoms. Recently, a top expert at
    the World Health Organization stated that transmission of the
    coronavirus by people who did not have symptoms was ``very rare,''
    \href{https://www.nytimes3xbfgragh.onion/2020/06/09/world/coronavirus-updates.html?action=click\&pgtype=Article\&state=default\&region=MAIN_CONTENT_3\&context=storylines_faq\#link-1f302e21}{but
    she later walked back that statement.}
  \end{itemize}
\end{itemize}

Plagues are nothing new in Cairo. On a visit to Cairo in the 14th
century, when the city's 500,000 inhabitants made it the world's largest
city outside China, the explorer Ibn Battuta noted that an outbreak of
the bubonic plague was killing as many as 20,000 people a day. Cholera
hit repeatedly in the 19th century.

This time, the human cost is amplified by Egypt's soaring population,
which in February crossed
\href{https://www.nytimes3xbfgragh.onion/2020/02/11/world/middleeast/egypt-population-100-million.html}{the
100 million threshold}, an unnerving milestone in a densely packed
country.

Outside the city center, the lockdown has been loosely observed
---~social distancing is little more than an admirable idea in the
city's cramped slums.

Image

Newlyweds posing on a bridge connecting the Egyptian capital, Cairo,
with the city of Giza.Credit...Khaled Desouki/Agence France-Presse ---
Getty Images

Many Egyptians wear masks over their chins or spurn them entirely, much
to the chagrin of President Abdel Fattah el-Sisi, a fitness enthusiast
who has urged Egyptians to stay safe, keep fit and lose weight during
the lockdown. ``Remember to do sport, it increases immunity levels,'' he
said in May.

Otherwise, it has been business as usual for Mr. el-Sisi during the
lockdown --- with the arrest of rights activists, belly dancers and even
young women who post dance videos to social media. The virus, though,
cannot be banished so easily.

Egypt has
\href{https://www.nytimes3xbfgragh.onion/interactive/2020/world/coronavirus-maps.html}{over
77,000 known cases}, and confirmed infections have grown by about 1,400
cases a day for the past month. Egypt has registered more than 3,400
deaths, the highest toll in the Arab world. In an ominous portent, Mr.
el-Sisi last week
\href{http://english.ahram.org.eg/NewsContent/1/64/373091/Egypt/Politics-/Egypts-Sisi-tours-Armed-Forces-medical-isolation-f.aspx}{opened
a 4,000-bed field hospital} to treat coronavirus patients.

And the economic toll is only now becoming apparent. Millions of workers
have lost income, and families are cutting back on meat and other items
that are now unaffordable. The International Monetary Fund has lent \$8
billion to get Egypt through the crisis. More may be needed.

The day after the lockdown was lifted, I walked the same route again.
The sense of magic had evaporated.

Police officers patrolled the bridge where the kites had flown. The
familiar rumble of traffic snarled the downtown, where some restaurants
had opened. But others remained shut ---~it's not worth it yet, the
manager of Abou Tarek, the city's most celebrated koshary emporium, told
me ~--- and there was talk that some restrictions could become
permanent.

Rules obliging restaurants and coffee houses to shut at 10 p.m. will
remain after the virus, a cabinet spokesman said --- an announcement
that was consistent with Mr. el-Sisi's desire to ``civilize'' Egyptians,
but that met with muted indignation in a city famous for its vibrant
all-night socializing.

Egyptian rulers have announced similar detox measures in the past, only
to quickly backtrack in the face of popular resistance. For now, what's
certain is that Cairenes are staying home, caught between their desire
to get back to normal and their fear of what may be coming next --- much
like everywhere else.

Nada Rashwan contributed reporting.

Image

The Great Sphinx and the pyramids were eerily bereft of~ tourists in
May.~Credit...Sima Diab for The New York Times

Advertisement

\protect\hyperlink{after-bottom}{Continue reading the main story}

\hypertarget{site-index}{%
\subsection{Site Index}\label{site-index}}

\hypertarget{site-information-navigation}{%
\subsection{Site Information
Navigation}\label{site-information-navigation}}

\begin{itemize}
\tightlist
\item
  \href{https://help.nytimes3xbfgragh.onion/hc/en-us/articles/115014792127-Copyright-notice}{©~2020~The
  New York Times Company}
\end{itemize}

\begin{itemize}
\tightlist
\item
  \href{https://www.nytco.com/}{NYTCo}
\item
  \href{https://help.nytimes3xbfgragh.onion/hc/en-us/articles/115015385887-Contact-Us}{Contact
  Us}
\item
  \href{https://www.nytco.com/careers/}{Work with us}
\item
  \href{https://nytmediakit.com/}{Advertise}
\item
  \href{http://www.tbrandstudio.com/}{T Brand Studio}
\item
  \href{https://www.nytimes3xbfgragh.onion/privacy/cookie-policy\#how-do-i-manage-trackers}{Your
  Ad Choices}
\item
  \href{https://www.nytimes3xbfgragh.onion/privacy}{Privacy}
\item
  \href{https://help.nytimes3xbfgragh.onion/hc/en-us/articles/115014893428-Terms-of-service}{Terms
  of Service}
\item
  \href{https://help.nytimes3xbfgragh.onion/hc/en-us/articles/115014893968-Terms-of-sale}{Terms
  of Sale}
\item
  \href{https://spiderbites.nytimes3xbfgragh.onion}{Site Map}
\item
  \href{https://help.nytimes3xbfgragh.onion/hc/en-us}{Help}
\item
  \href{https://www.nytimes3xbfgragh.onion/subscription?campaignId=37WXW}{Subscriptions}
\end{itemize}
