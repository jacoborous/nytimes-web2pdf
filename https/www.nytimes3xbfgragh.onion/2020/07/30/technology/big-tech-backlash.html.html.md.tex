Sections

SEARCH

\protect\hyperlink{site-content}{Skip to
content}\protect\hyperlink{site-index}{Skip to site index}

\href{https://www.nytimes3xbfgragh.onion/section/technology}{Technology}

\href{https://myaccount.nytimes3xbfgragh.onion/auth/login?response_type=cookie\&client_id=vi}{}

\href{https://www.nytimes3xbfgragh.onion/section/todayspaper}{Today's
Paper}

\href{/section/technology}{Technology}\textbar{}Big Tech's Backlash Is
Just Starting

\href{https://nyti.ms/2Pc0F1w}{https://nyti.ms/2Pc0F1w}

\begin{itemize}
\item
\item
\item
\item
\item
\end{itemize}

Advertisement

\protect\hyperlink{after-top}{Continue reading the main story}

Supported by

\protect\hyperlink{after-sponsor}{Continue reading the main story}

on tech

\hypertarget{big-techs-backlash-is-just-starting}{%
\section{Big Tech's Backlash Is Just
Starting}\label{big-techs-backlash-is-just-starting}}

The congressional antitrust hearing showed that concerns about the tech
stars aren't going away.

\includegraphics{https://static01.graylady3jvrrxbe.onion/images/2020/07/30/business/30ontech/30ontech-articleLarge-v6.gif?quality=75\&auto=webp\&disable=upscale}

\href{https://www.nytimes3xbfgragh.onion/by/shira-ovide}{\includegraphics{https://static01.graylady3jvrrxbe.onion/images/2020/03/18/reader-center/author-shira-ovide/author-shira-ovide-thumbLarge-v2.png}}

By \href{https://www.nytimes3xbfgragh.onion/by/shira-ovide}{Shira Ovide}

\begin{itemize}
\item
  July 30, 2020
\item
  \begin{itemize}
  \item
  \item
  \item
  \item
  \item
  \end{itemize}
\end{itemize}

\emph{This article is part of the On Tech newsletter. You can}
\href{https://www.nytimes3xbfgragh.onion/newsletters/signup/OT}{\emph{sign
up here}} \emph{to receive it weekdays.}

Wednesday's\href{https://www.nytimes3xbfgragh.onion/2020/07/29/technology/big-tech-hearing-apple-amazon-facebook-google.html}{five-plus-hour
congressional probing} of the bosses of America's tech giants did not
reveal a singular ``gotcha'' moment or smoking gun email. We've heard
many of these examples of Big Tech abuse before.

But the power of this hearing and others like it was the cumulative
repetition of tales of abusive behavior, and evidence of the harm this
has had on people's lives.

The spectacle also showed that the impact of congressional
investigations is the digging that happens when the C-SPAN cameras are
turned off.

Worries about America's tech stars have swirled for years. It's clear
now that this isn't going away. In world capitals, courtrooms and among
the public, we are wrestling with what it means for tech giants to have
enormous influence on our lives, elections, economy and minds.

And while what happens to the future of Google, Amazon, Apple and
Facebook is anyone's guess, it was clear from Wednesday's hearing that
Congress was pointing the way for other branches of government to pick
up the digging from here.

We saw on Wednesday old
\href{https://judiciary.house.gov/uploadedfiles/0002.pdf}{emails} and
\href{https://judiciary.house.gov/uploadedfiles/0006336700063372.pdf}{texts}
from Mark Zuckerberg, worried about Facebook
\href{https://www.nytimes3xbfgragh.onion/live/2020/07/29/technology/tech-ceos-hearing-testimony/lawmakers-said-documents-show-facebook-tried-to-neutralize-a-competitive-threat}{losing
ground to Instagram} and suggesting that buying competing apps is an
effective way to take out the competition. The big deal here: Trying to
reduce competition by purchasing a rival is a violation of antitrust
law. (Zuckerberg said that Instagram's success wasn't assured when
Facebook bought it.)

Representatives said that their interviews with former Amazon employees
backed up
\href{https://www.wsj.com/articles/amazon-scooped-up-data-from-its-own-sellers-to-launch-competing-products-11587650015}{news
reports} that the company used private data from its merchants to make
its own version of their products.

The subcommittee discussed their conversations with companies that
claimed Google funneled web searches to services it owned rather than to
rivals like Yelp. Through company documents and questioning, members of
Congress picked apart Apple's stance that it treats all app developers
the same.

My colleague Kevin Roose
\href{https://www.nytimes3xbfgragh.onion/2020/07/30/technology/big-tech-ceos.html}{wrote}
that the tech bosses seemed to be ``taken off guard by the rigor and
depth of the questions they faced.''

The
\href{https://www.nytimes3xbfgragh.onion/2020/06/25/technology/barr-google-investigation.html}{Department
of Justice} and the Federal Trade Commission are also investigating
whether these companies abuse their power, and I bet they watched
closely. The U.S. government's
\href{https://www.nytimes3xbfgragh.onion/live/2020/07/29/technology/tech-ceos-hearing-testimony/todays-hearing-has-echoes-of-bill-gates-22-years-ago}{antitrust
case against Microsoft} more than 20 years ago was built, in part, on
the emails of Bill Gates and other Microsoft executives discussing how
they planned to kill upstart competitors.

Here's one more sign that the backlash against Big Tech has only just
begun: The shouty tech critics in Congress and the tech bosses all
seemed to agree that these four companies have a meaningful impact on
many people's lives.

The tech bosses focused on the good that comes from their companies'
size, reach and influence. A New York bakery finds customers by buying
advertisements on Google. Merchants can thrive by selling their products
or apps on Amazon or Apple.

The representatives pointed out examples of the dark side of Big Tech's
size, reach and influence. In the pin drop moment of the hearing, a
House member played an audio recording of a book seller saying her
family was struggling because of a change Amazon apparently made that
dried up her sales there.

The subcommittee chairman said these tech powers can pick the winners
and the losers. That might be stretching it. But both sides demonstrated
that these four companies have a profound say in who wins or loses.

Lawmakers of all political stripes seemed uncomfortable with the
knowledge that four companies have this much influence. Beyond the legal
antitrust questions at issue, it's this feeling of discomfort that makes
it hard to imagine that nothing will change for these tech superpowers.

\begin{center}\rule{0.5\linewidth}{\linethickness}\end{center}

\hypertarget{another-theme-anti-conservative-bias}{%
\subsection{Another theme: anti-conservative
bias}\label{another-theme-anti-conservative-bias}}

Wednesday's hearing was really two hearings. The Democrats mostly asked
the four tech chief executives about ways their companies wielded their
power and influence. Republican members
\href{https://www.nytimes3xbfgragh.onion/live/2020/07/29/technology/tech-ceos-hearing-testimony/republicans-focused-on-bias-concerns-about-platforms}{largely
asked} about persistent concerns that Google and Facebook in particular
censor right-leaning viewpoints or treat conservative figures unfairly.

Some Republican politicians' complaints about political bias aren't
backed by credible evidence. Regardless, suspicion of bias is a thorny
problem for these companies.

In a 2018 Pew Research
\href{https://www.pewresearch.org/fact-tank/2020/05/29/fast-facts-about-americans-views-of-social-media-companies-as-trump-twitter-dispute-grows/}{survey},
Americans who described themselves as Republicans or Republican-leaning
overwhelmingly said that they believed that tech companies censor online
information for partisan reasons. (A smaller, but still majority, share
of Democrats said that they believed this, too.) Since then, polling has
shown a growing mistrust of tech companies, particularly among
conservatives.

This doesn't seem to have hurt the tech companies' businesses. In fact,
some Republican members on Wednesday argued that even though people
don't trust Big Tech, they have no choice but to continue using these
services because these companies have so much influence. It was an
effective way to connect bias concerns to investigations into tech
company market power. (Yes, I
\href{https://www.nytimes3xbfgragh.onion/2020/07/29/technology/congress-big-tech.html}{said}
earlier this week not to pay attention to bias claims. But maybe pay
attention a little?)

Even if allegations of bias don't cause the companies to lose customers,
the loss of faith among a large share of Americans should worry them.

It's also a problem if the tech companies overcorrect. Facebook
employees and critics have said fears of being accused of bias have made
the company reluctant to crack down on people, including President
Trump, who spread dangerous or inflammatory messages online. It's a fine
line to walk.

\begin{center}\rule{0.5\linewidth}{\linethickness}\end{center}

\hypertarget{before-we-go-}{%
\subsection{Before we go \ldots{}}\label{before-we-go-}}

\begin{itemize}
\item
  \textbf{The really important stuff from the Big Tech hearing: House
  plants and bookshelves.} My colleague Mike Isaac
  \href{https://www.nytimes3xbfgragh.onion/live/2020/07/29/technology/tech-ceos-hearing-testimony/tech-executives-looked-like-they-work-in-well-tech-offices}{rated
  the tech bosses' choices of backgrounds} for their webcast testimony.
  Mike gave Amazon's Jeff Bezos, who sat in front of wooden shelves with
  a sprinkling of books and tchotchkes, a score of 8 out of 10 for his
  ``cool Pacific Northwest dad office vibes.''
\item
  \textbf{Example infinity of technology as a flawed virus
  surveillance:} A Wall Street Journal technology columnist reviewed
  smart watches, internet-connected thermometers and other gizmos that
  say our heart rate readings or other bodily data can provide
  \href{https://www.wsj.com/articles/could-you-have-covid-19-soon-your-smartwatch-or-smart-ring-might-tell-you-11595949072}{early
  warnings of coronavirus infections}. Spoiler alert: Some of this stuff
  holds promise but needs further research, and we still need more
  laboratory virus testing.
\item
  \textbf{If you feel like screaming when you watch TV:} Rolling Stone
  \href{https://www.rollingstone.com/tv/tv-features/streaming-wars-user-experience-sepinwall-1031729/}{has
  a hilarious and smart rage fest} on why the video streaming services
  can be so infuriating to use.
\end{itemize}

\hypertarget{hugs-to-this}{%
\subsubsection{Hugs to this}\label{hugs-to-this}}

Best wishes forever to this
\href{https://www.tiktok.com/@carrot.castle/video/6847954959464434949}{tiny
rabbit peeking out of a canvas bag}.

\begin{center}\rule{0.5\linewidth}{\linethickness}\end{center}

\emph{We want to hear from you. Tell us what you think of this
newsletter and what else you'd like us to explore. You can reach us at}
\href{mailto:ontech@NYTimes.com?subject=On\%20Tech\%20Feedback}{\emph{ontech@NYTimes.com.}}
**

\emph{If you don't already get this newsletter in your inbox,}
\href{https://www.nytimes3xbfgragh.onion/newsletters/signup/OT}{\emph{please
sign up here}}\emph{.}

Advertisement

\protect\hyperlink{after-bottom}{Continue reading the main story}

\hypertarget{site-index}{%
\subsection{Site Index}\label{site-index}}

\hypertarget{site-information-navigation}{%
\subsection{Site Information
Navigation}\label{site-information-navigation}}

\begin{itemize}
\tightlist
\item
  \href{https://help.nytimes3xbfgragh.onion/hc/en-us/articles/115014792127-Copyright-notice}{©~2020~The
  New York Times Company}
\end{itemize}

\begin{itemize}
\tightlist
\item
  \href{https://www.nytco.com/}{NYTCo}
\item
  \href{https://help.nytimes3xbfgragh.onion/hc/en-us/articles/115015385887-Contact-Us}{Contact
  Us}
\item
  \href{https://www.nytco.com/careers/}{Work with us}
\item
  \href{https://nytmediakit.com/}{Advertise}
\item
  \href{http://www.tbrandstudio.com/}{T Brand Studio}
\item
  \href{https://www.nytimes3xbfgragh.onion/privacy/cookie-policy\#how-do-i-manage-trackers}{Your
  Ad Choices}
\item
  \href{https://www.nytimes3xbfgragh.onion/privacy}{Privacy}
\item
  \href{https://help.nytimes3xbfgragh.onion/hc/en-us/articles/115014893428-Terms-of-service}{Terms
  of Service}
\item
  \href{https://help.nytimes3xbfgragh.onion/hc/en-us/articles/115014893968-Terms-of-sale}{Terms
  of Sale}
\item
  \href{https://spiderbites.nytimes3xbfgragh.onion}{Site Map}
\item
  \href{https://help.nytimes3xbfgragh.onion/hc/en-us}{Help}
\item
  \href{https://www.nytimes3xbfgragh.onion/subscription?campaignId=37WXW}{Subscriptions}
\end{itemize}
