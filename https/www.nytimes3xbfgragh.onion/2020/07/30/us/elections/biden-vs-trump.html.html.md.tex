Sections

SEARCH

\protect\hyperlink{site-content}{Skip to
content}\protect\hyperlink{site-index}{Skip to site index}

\href{https://www.nytimes3xbfgragh.onion/news-event/2020-election}{Elections}

\href{https://myaccount.nytimes3xbfgragh.onion/auth/login?response_type=cookie\&client_id=vi}{}

\href{https://www.nytimes3xbfgragh.onion/section/todayspaper}{Today's
Paper}

\href{/news-event/2020-election}{Elections}\textbar{}Trump Defends
`Delay the Election' Tweet, Even Though He Can't Do It

\href{https://nyti.ms/3hOWlkN}{https://nyti.ms/3hOWlkN}

\begin{itemize}
\item
\item
\item
\item
\item
\item
\end{itemize}

\begin{itemize}
\item
  \href{https://www.nytimes3xbfgragh.onion/2020/07/31/us/elections/biden-vs-trump.html?action=click\&pgtype=Article\&state=default\&region=TOP_BANNER\&context=storylines_menu}{Election
  Updates}
\item
  \href{https://www.nytimes3xbfgragh.onion/article/biden-vice-president-2020.html?action=click\&pgtype=Article\&state=default\&region=TOP_BANNER\&context=storylines_menu}{Biden's
  V.P. Search}
\item
  \href{https://www.nytimes3xbfgragh.onion/interactive/2020/07/24/us/politics/trump-biden-campaign-donors.html?action=click\&pgtype=Article\&state=default\&region=TOP_BANNER\&context=storylines_menu}{Map
  of Donations}
\item
  \href{https://www.nytimes3xbfgragh.onion/interactive/2020/us/elections/delegate-count-primary-results.html?action=click\&pgtype=Article\&state=default\&region=TOP_BANNER\&context=storylines_menu}{Delegate
  Count}
\item
  \href{https://www.nytimes3xbfgragh.onion/interactive/2019/us/politics/2020-presidential-candidates.html?action=click\&pgtype=Article\&state=default\&region=TOP_BANNER\&context=storylines_menu}{The
  Candidates}
\item
  \href{https://www.nytimes3xbfgragh.onion/newsletters/politics?action=click\&pgtype=Article\&state=default\&region=TOP_BANNER\&context=storylines_menu}{Politics
  Newsletter}
\end{itemize}

Advertisement

\protect\hyperlink{after-top}{Continue reading the main story}

Supported by

\protect\hyperlink{after-sponsor}{Continue reading the main story}

\hypertarget{trump-defends-delay-the-election-tweet-even-though-he-cant-do-it}{%
\section{Trump Defends `Delay the Election' Tweet, Even Though He Can't
Do
It}\label{trump-defends-delay-the-election-tweet-even-though-he-cant-do-it}}

Trailing badly in the polls, President Trump baselessly suggested that
the November election would be fraudulent. Former President Barack Obama
called for sweeping changes to expand voting rights.

Published July 30, 2020Updated July 31, 2020

\begin{itemize}
\item
\item
\item
\item
\item
\item
\end{itemize}

\emph{This briefing has ended. Follow our latest coverage of the}
\href{https://www.nytimes3xbfgragh.onion/2020/07/31/us/elections/biden-vs-trump.html}{\emph{Biden
vs. Trump 2020 election here}}\emph{.}

\hypertarget{heres-what-you-need-to-know}{%
\subsubsection{Here's what you need to
know:}\label{heres-what-you-need-to-know}}

\begin{itemize}
\tightlist
\item
  \protect\hyperlink{link-301a32eb}{At White House briefing, Trump
  stands by `delay the election' tweet, ignoring G.O.P. pushback.}
\item
  \protect\hyperlink{link-7c96adc}{Obama calls for sweeping changes to
  expand voter rights.}
\item
  \protect\hyperlink{link-2636310}{The G.O.P. break with Trump on moving
  the election is a rare party split.}
\item
  \protect\hyperlink{link-7d171f8e}{Trump can't postpone the election.
  But the courts will help shape how Americans vote this fall.}
\item
  \protect\hyperlink{link-4a528e86}{Congress votes down Ocasio-Cortez's
  amendment to ban military recruitment on Twitch.}
\item
  \protect\hyperlink{link-5d840c86}{Trump's speakerphone conversation
  with a senator in a Washington restaurant is caught on tape.}
\item
  \protect\hyperlink{link-79cd5851}{Supreme Court won't allow electronic
  signatures for Idaho ballot initiative.}
\end{itemize}

\includegraphics{https://static01.graylady3jvrrxbe.onion/images/2020/07/30/business/30elections-briefing-trump1/merlin_175067439_ab0ab0be-6b55-49cb-a733-09e64b3f07f0-articleLarge.jpg?quality=75\&auto=webp\&disable=upscale}

\hypertarget{at-white-house-briefing-trump-stands-by-delay-the-election-tweet-ignoring-gop-pushback}{%
\subsection{At White House briefing, Trump stands by `delay the
election' tweet, ignoring G.O.P.
pushback.}\label{at-white-house-briefing-trump-stands-by-delay-the-election-tweet-ignoring-gop-pushback}}

President Trump refused to back down on Thursday afternoon from his
suggestion earlier in the day that the election be delayed, something he
has no authority to order and that top Republicans quickly rejected.

In an appearance at the White House that was ostensibly about the
coronavirus, Mr. Trump ignored a day of pushback from some of his
closest allies, who said the election would be held in November as
scheduled, and again attacked the process of mail-in voting, a service
he has used before.

``You're sending out hundreds of millions of universal mail-in ballots.
Hundreds of millions,'' Mr. Trump said. ``Where are they going? Who are
they being sent to? You don't have to know anything about politics.''

Mr. Trump sought to back up his claims by pointing to delayed vote
counts and rejected absentee ballots in some states' primaries, but his
\href{https://www.nytimes3xbfgragh.onion/article/mail-in-voting-explained.html}{claim
that mail voting leads to inaccurate counts or fraud is false}.

He also vastly overstated the number of ballots that would be needed ---
only around 138 million Americans voted in 2016 --- and continued trying
to sow doubt about the election process. ``I don't want to see a crooked
election,'' Mr. Trump said, referring to allowing Americans to use
mail-in voting. ``This will be the most rigged election in history if
that happens.''

The president appeared to brush off rebukes by Republican leaders and
allies, who spent the day pointing out that Mr. Trump had no authority
to move the election.

``Never in the history of the federal elections have we not held an
election, and we should go forward,'' said Representative Kevin
McCarthy, the House minority leader.

Senator Mitch McConnell, the majority leader, dismissed Mr. Trump's
suggestion
\href{https://twitter.com/MaxWinitz/status/1288875891985129480?s=20}{in
an interview with WNKY television} in Bowling Green, Ky.

``Never in the history of the country, through wars, depressions and the
Civil War, have we ever not had a federally scheduled election on time,
and we'll find a way to do that again this Nov. 3,'' Mr. McConnell said.

Even for Mr. Trump,
\href{https://twitter.com/realDonaldTrump/status/1288818160389558273?s=20}{suggesting
a delay in the election} is an extraordinary breach of democratic norms
that will increase the chances that he and his core supporters don't
accept the legitimacy of the election should he lose to former Vice
President Joseph R. Biden Jr.

Mr. Biden, speaking to Democratic National Committee members and
convention delegates on a conference call Thursday, did not address Mr.
Trump's tweet. At the White House, Mr. Trump still attacked him, falsely
claiming that Mr. Biden was against fracking. (He is not.)

Later that evening, however, Mr. Biden brought up the issue at a virtual
fund-raiser. ``By the way, as these numbers have gotten worse, what did
he do today?'' Mr. Biden said. ``He called for not having the election
on Nov. 3. He wants to postpone the election. Well that's for two
reasons. One, he believes it,'' Mr. Biden said, and secondly, he
suggested, because Mr. Trump wanted to detract from the funeral that day
of John Lewis, a former congressman and civil rights icon.

Several of the president's critics suggested on Thursday that he was
trying to distract from dire news about the economy. Mr. Trump posted to
Twitter minutes after the Commerce Department announced that the
nation's gross domestic product, the broadest measure of goods and
services produced,
\href{https://www.nytimes3xbfgragh.onion/live/2020/07/30/business/stock-market-today-coronavirus?action=click\&module=Top\%20Stories\&pgtype=Homepage}{fell
9.5 percent during the three months ending June 30, the largest
quarterly drop on record}.

``With Universal Mail-In Voting (not Absentee Voting, which is good),
2020 will be the most INACCURATE \& FRAUDULENT Election in history,''
Mr. Trump wrote. ``It will be a great embarrassment to the USA. Delay
the Election until people can properly, securely and safely vote???''

Rather than back off in the face of widespread criticism, Mr. Trump
pinned the tweet atop his Twitter profile.

Mr. Trump has no authority to unilaterally change the date of the
election,
\href{https://www.nytimes3xbfgragh.onion/2020/03/14/us/politics/election-postponed-canceled.html}{which
is set by federal law}. His suggestion comes as polls show him trailing
far behind Mr. Biden in surveys of nearly all of the key battleground
states.

``Only Congress can change the date of our elections, and under no
circumstances will we consider doing so to accommodate the president's
inept and haphazard response to the coronavirus pandemic,'' said
Representative Zoe Lofgren, Democrat of California and the chair of the
Administration Committee, which oversees elections.

So far, no major Republican figures have publicly agreed with Mr.
Trump's proposal, though they have avoided criticizing the president.

Even Fox News, a loyal Trump ally that the president watches for hours
inside the White House, interpreted his proposal as a sign the president
is flailing.

``It is a fragrant and flagrant expression of his current weakness,''
Fox News politics editor Chris Stirewalt said
\href{https://video.foxnews.com/v/6176638142001\#sp=show-clips}{during a
Fox News broadcast Thursday morning}. ``A person who is in a strong
position would never, never suggest anything like that. So Trump may be
making a tactical error here by further telegraphing his weak position
in the polls and his weak position for re-election.''

It is unclear how seriously Mr. Trump believes there ought to be
discussion about changing the date of the election. He often floats
extraordinary proposals only to back off from them after they have
dominated cable news cycles.

Mr. Trump's sustained attacks on mail voting, combined with more robust
Democratic efforts to encourage voters to request and submit ballots by
mail, have led to a significant Democratic advantage in absentee ballot
requests during the primaries. And amid the pandemic, states that
shifted their balloting largely to the mail have seen far larger voter
turnout than states that held their contests primarily in person.

\hypertarget{obama-calls-for-sweeping-changes-to-expand-voter-rights}{%
\subsection{Obama calls for sweeping changes to expand voter
rights.}\label{obama-calls-for-sweeping-changes-to-expand-voter-rights}}

Image

Mourners watch the broadcast of President Barack Obama speaking at John
Lewis's funeral outside Ebenezer Baptist Church in
Atlanta.Credit...Melissa Golden for The New York Times

While delivering a eulogy at former Representative John Lewis's funeral
in Atlanta on Thursday, former President Barack Obama called for a
sweeping expansion of voting rights.

\hypertarget{latest-updates-2020-election}{%
\section{\texorpdfstring{\href{https://www.nytimes3xbfgragh.onion/2020/07/31/us/elections/biden-vs-trump.html?action=click\&pgtype=Article\&state=default\&region=MAIN_CONTENT_1\&context=storylines_live_updates}{Latest
Updates: 2020
Election}}{Latest Updates: 2020 Election}}\label{latest-updates-2020-election}}

Updated 2020-08-01T01:26:45.732Z

\begin{itemize}
\tightlist
\item
  \href{https://www.nytimes3xbfgragh.onion/2020/07/31/us/elections/biden-vs-trump.html?action=click\&pgtype=Article\&state=default\&region=MAIN_CONTENT_1\&context=storylines_live_updates\#link-29fdff45}{Kamala
  Harris, a top vice-presidential contender, confronts double
  standards.}
\item
  \href{https://www.nytimes3xbfgragh.onion/2020/07/31/us/elections/biden-vs-trump.html?action=click\&pgtype=Article\&state=default\&region=MAIN_CONTENT_1\&context=storylines_live_updates\#link-13ec3d9c}{Karen
  Bass and Susan Rice are rising on Biden's vice-presidential
  shortlist.}
\item
  \href{https://www.nytimes3xbfgragh.onion/2020/07/31/us/elections/biden-vs-trump.html?action=click\&pgtype=Article\&state=default\&region=MAIN_CONTENT_1\&context=storylines_live_updates\#link-49e9a016}{Trump
  says Russian bounties to kill U.S. troops `never took place.'}
\end{itemize}

\href{https://www.nytimes3xbfgragh.onion/2020/07/31/us/elections/biden-vs-trump.html?action=click\&pgtype=Article\&state=default\&region=MAIN_CONTENT_1\&context=storylines_live_updates}{See
more updates}

Without saying Mr. Trump's name, Mr. Obama lacerated his and the
Republican Party's record on voting rights, which were a defining cause
for Mr. Lewis. The overarching message of his speech was that each
generation has a responsibility to carry on the activism of the last,
and that the injustices Mr. Lewis fought throughout his life were far
from vanquished.

The legacy of Jim Crow continues today, Mr. Obama said, with police
violence, voter suppression and entrenched racism.

``Even as we sit here,'' he said, ``there are those in power who are
doing their darnedest to discourage people from voting by closing
polling locations, and targeting minorities and students with
restrictive ID laws, and attacking our voting rights with surgical
precision.''

In one of his most specific forays yet into the policy debates Democrats
and Republicans are having this year, Mr. Obama called for automatic
voter registration, restoring voting rights to people released from
prison, expanding early voting, adding polling places, making Election
Day a federal holiday, ending partisan gerrymandering and fully
enfranchising residents of the District of Columbia and Puerto Rico.

``And if all this takes eliminating the filibuster, another Jim Crow
relic, in order to secure the God-given rights of every American,'' he
said, ``then that's what we should do.''

Since Mr. Lewis, a civil rights icon,
\href{https://www.nytimes3xbfgragh.onion/2020/07/17/us/john-lewis-dead.html}{died
on July 17}, many of his supporters have called on Congress
\href{https://www.nytimes3xbfgragh.onion/2020/07/21/us/john-lewis-voting-rights-act.html}{to
pass legislation} updating the Voting Rights Act of 1965. Mr. Lewis's
activism was instrumental in the passage of the original law, but the
Supreme Court substantially weakened it in 2013, allowing states to
enact a flood of restrictions that disproportionately affect Black
people.

The Democratic-controlled House passed such legislation last year, but
Senator Mitch McConnell of Kentucky, the majority leader, has refused to
allow the Republican-controlled Senate to take it up.

Mr. Obama suggested that if politicians wanted to honor Mr. Lewis's
legacy, they ought to go further than releasing ``a statement calling
him a hero'' --- an implicit swipe at Republicans like Mr. McConnell.

``Want to honor John?'' he said. ``Let's honor him by revitalizing the
law that he was willing to die for.''

Mr. Obama did not mince words about what he described as threats not
only to specific policies that Mr. Lewis supported, but to democracy
itself.

``Democracy isn't automatic. It has to be nurtured, it has to be tended
to, we have to work at it,'' he said. ``If we want our children to grow
up in a democracy --- not just with elections, but a true democracy, a
representative democracy, a bighearted, tolerant, vibrant, inclusive
America --- then we're going to have to be more like John.''

\hypertarget{the-gop-break-with-trump-on-moving-the-election-is-a-rare-party-split}{%
\subsection{The G.O.P. break with Trump on moving the election is a rare
party
split.}\label{the-gop-break-with-trump-on-moving-the-election-is-a-rare-party-split}}

Since Mr. Trump won the 2016 presidential election, fellow Republicans
have rarely crossed him, and nearly all of those who did were
subsequently drummed out of the party.

But when Mr. Trump proposed delaying the Nov. 3 election, something he
cannot do, Republicans sought to thread a needle by not attacking him
personally while making it clear they had no interest in his idea.

``For now, the Nov. 3 elections should be held,'' said Terry Lathan, the
chairwoman of the Alabama Republican Party. ``The president was asking a
question on this topic, not making a statement. He is right to be
concerned about the chatter of an all mail-in federally mandated
election that the Democrats are pushing.''

(Democrats are not pushing federal mandates for voting by mail. They
have enacted some and proposed more federal funding to help states
conduct elections during the pandemic. Decisions on how to conduct
elections are made by the states.)

Michael Whatley, the chairman of the North Carolina Republican Party,
who also said he hadn't seen the president's tweet, predicted the
elections in North Carolina would be held safely as scheduled.

``The election will absolutely be held on Nov. 3,'' Mr. Whatley said.
``We look forward to delivering the vote for President Trump and
candidates up and down the ballot.''

Other heads of state Republican parties also exhibited a tactic honed by
congressional Republicans: Saying they didn't know what the president
had tweeted.

``I haven't seen anything because I was on `Fox and Friends,''' said
Kelli Ward, the chairwoman of the Arizona Republican Party.

Andrew Hitt, the chairman of the Wisconsin Republican Party, would not
say if he thought the election should be held Nov. 3. ``I'm not going to
go on the record right now,'' he said in a brief telephone interview. Of
Mr. Trump's tweet, he said: ``I haven't seen it.''

Carly Fiorina, who clashed with Mr. Trump during the 2016 Republican
presidential primary and
\href{https://www.theatlantic.com/politics/archive/2020/06/carly-fiorina-vote-biden/613474/}{said
last month that she would vote for Mr. Biden}, called the missive
``classic Trump.''

``He is desperately trying to distract from his failure to lead, the
devastating consequences of this pandemic and terrible economic news,''
she said in an interview Thursday. ``It is certainly irresponsible and I
hope every elected official, Democrat and Republican alike, will stand
up and say it's irresponsible.''

Juliana Bergeron, a Republican National Committee member from New
Hampshire, offered a rare and mild rebuke of Mr. Trump for focusing
attention on the legitimacy of voting by mail.

``I do not believe it should be delayed nor do I believe it will be,''
Ms. Bergeron said. ``For one thing, it is not up to the president. I
have faith in our system and think we should spend more time getting out
the vote, even if it is through absentee ballots, than worrying about
fraud.''

And Rick Santorum, the former Pennsylvania senator who ran against Mr.
Trump for the Republican presidential nomination in 2016 and then became
a Trump-supportive cable television commentator, said Thursday that he
disagreed with any proposal to change the date of the election but
declined to say more.

``I'm on the golf course,'' Mr. Santorum said. ``I gotta go.''

\hypertarget{trump-cant-postpone-the-election-but-the-courts-will-help-shape-how-americans-vote-this-fall}{%
\subsection{Trump can't postpone the election. But the courts will help
shape how Americans vote this
fall.}\label{trump-cant-postpone-the-election-but-the-courts-will-help-shape-how-americans-vote-this-fall}}

Image

Election workers counting mail-in absentee ballots from the New York's
primary. President Trump ramped up his attacks on mail-in ballots on
Thursday, floating a ``delay'' in the election.Credit...Victor J. Blue
for The New York Times

Moving the date of the general election
\href{https://www.nytimes3xbfgragh.onion/2020/07/30/us/politics/trump-postpone-election.html}{would
require legislation enacted by Congress}, signed by the president, and
would be subject to challenge in the courts. To call that unlikely would
be an understatement.

Even if all of that happened, there would not be much flexibility in
choosing an alternate election date: The Constitution mandates the new
Congress to be sworn in on Jan. 3, and the new president's term to begin
on Jan. 20.

But while the date of the presidential election is set by federal law,
the procedures for voting are generally controlled at the state level.
For election lawyers, Mr. Trump's tweet on Thursday underscored the
stakes of the voting litigation playing out in courts across the country
that will help shape how Americans vote this fall.

In interviews with several nonpartisan lawyers, all of them agreed that
the president had no authority to delay the election. But Mr. Trump's
mere suggestion of changing the election date added yet another
complication to an increasingly contentious and litigious election year.

As the coronavirus pandemic forced many states to shift how they held
their primary elections, lawsuits backed by Republicans, Democrats and
nonpartisan voting rights groups flooded district and circuit courts.
They battled over changes like expanded mail-in voting, consolidated
polling locations and extended polling hours.

Decisions over how the general election will be conducted need to be
sorted out now, said Sylvia Albert, the director of Voting and Elections
at Common Cause, a voting rights group.

``The key is not waiting for November,'' Ms. Albert said. ``The point of
those lawsuits is to establish the policies and procedures that are
going to be used in November so there isn't going to be confusion on the
day of the election.''

Dale Ho, the director of the Voting Rights Project at the American Civil
Liberties Union, said his team of 11 election lawyers had filed 16
lawsuits during the pandemic, about four times as many as in a typical
election year. ``We're not finished,'' Mr. Ho said.

The two major political parties are also heavily involved. The
Republican National Committee announced early this year that it was
doubling its legal budget for the election to \$20 million. The
Democratic National Committee began placing voting protection directors
in states more than a year earlier than it had in the past.

R.N.C. officials said they were involved in about 40 voting litigation
cases \href{https://protectthevote.com/}{spread across 17 states}. They
have largely focused on protecting what they call ``safeguards'' on
absentee ballots, such as keeping signature matching requirements or
witness requirements.

``Normally, we would be more projecting our preparations toward Election
Day and the immediate days prior to Election Day,'' said Justin Reimer,
the chief counsel at the R.N.C. ``Because of all of the existing Covid
litigation, we have spent more of our time on those cases right now.''

\hypertarget{congress-votes-down-ocasio-cortezs-amendment-to-ban-military-recruitment-on-twitch}{%
\subsection{Congress votes down Ocasio-Cortez's amendment to ban
military recruitment on
Twitch.}\label{congress-votes-down-ocasio-cortezs-amendment-to-ban-military-recruitment-on-twitch}}

Image

Representative Alexandria Ocasio-Cortez at a hearing on Capitol Hill in
2019.Credit...Jacquelyn Martin/Associated Press

The House of Representatives on Thursday voted down Representative
Alexandria Ocasio-Cortez's
\href{https://www.vice.com/en_us/article/889vbv/aoc-introduces-measure-to-stop-the-military-from-recruiting-on-twitch}{measure
to ban military recruitment on the livestreaming site Twitch}. The
proposed amendment, according to a draft filed on July 22, would have
blocked the military from being able to ``maintain a presence on
Twitch.com or any video game, e-sports, or livestreaming platform.''

``War is not a game,'' the congresswoman
\href{https://twitter.com/AOC/status/1288869264640835585}{tweeted} on
Thursday. ``Twitch is a popular platform for children FAR under the age
of military recruitment rules. We should not conflate military service
with ``shoot-em-up'' style games and contests. The Marines pulled out of
Twitch for a reason. It's time to follow their lead.''

The U.S. Army and Navy have faced backlash recently for
\href{https://www.theverge.com/2020/7/17/21328130/us-army-twitch-esports-gaming-recruitment-fake-prize-giveaway}{targeting
children as young as 12 with ads for fake giveaways of video game
equipment}. Children who sought to enter the giveaways were directed to
fill out recruitment forms.

Since 2018, the Army and Navy have both operated official e-sports
channels on Twitch as part of a recruitment effort targeting young
people. Service members who perform on the channels stream themselves
playing violent video games including Call of Duty and Fortnite to an
audience of thousands of young people. On July 22, the military paused
its presence on the platform after accusations that it was violating
free speech by banning users who mentioned war crimes. First Amendment
advocacy groups
\href{https://www.nytimes3xbfgragh.onion/2020/07/22/style/army-gamers-war-crimes-first-amendment.html}{criticized
the military calling the bans unconstitutional}.

Ms. Ocasio-Cortez introduced her amendment in response to these
controversies and outlined her case on the House floor on Thursday.
``Children should not be targeted in general for many marketing purposes
in addition to recruitment,'' she
\href{https://twitter.com/Slasher/status/1288896196073918465}{said}.
``Right now, currently, children on platforms such as Twitch are
bombarded with banner ads with recruitment sign up forms that can be
submitted by children as young as 12.''

Representative Ilhan Omar of Minnesota voiced support for the amendment
on Twitter. ``Video games should not be recruitment tools for the
military,'' she
\href{https://twitter.com/IlhanMN/status/1288882866538131459}{said}.
``And it's particularly concerning that users were blocked from
platforms by the US military for asking about war crimes, raising
serious first amendment concerns. I am proud to work with @AOC on
this.''

\hypertarget{trumps-speakerphone-conversation-with-a-senator-in-a-washington-restaurant-is-caught-on-tape}{%
\subsection{Trump's speakerphone conversation with a senator in a
Washington restaurant is caught on
tape.}\label{trumps-speakerphone-conversation-with-a-senator-in-a-washington-restaurant-is-caught-on-tape}}

\includegraphics{https://static01.graylady3jvrrxbe.onion/images/2020/07/30/us/politics/coverimage/coverimage-videoSixteenByNineJumbo1600.png}

Mr. Trump called Senator James Inhofe, Republican of Oklahoma, on
Wednesday night for a conversation that Mr. Inhofe put on speakerphone
to hear better as he sat in a Washington restaurant, The Times's Maggie
Haberman reports.

The conversation, overheard and recorded by someone in the room, ranged
from a discussion about Anthony Tata, the retired Army brigadier general
whose nomination for a top Pentagon policy position
\href{https://www.nytimes3xbfgragh.onion/2020/07/30/us/politics/trump-inhofe-tata-pentagon.html}{has
become complicated}, to Mr. Trump's desire to preserve the name of
Robert E. Lee, a Confederate general, on a military base.

``We're going to keep the name of Robert E. Lee?'' Mr. Trump asked Mr.
Inhofe, 85, who sat at Trattoria Alberto, a Capitol Hill Italian
restaurant that is a favorite haunt of Washington Republicans, as he
took the call. Mr. Inhofe put the phone to his ear but put Mr. Trump on
speakerphone, and the president's voice was audible by people sitting at
other tables.

Mr. Inhofe, the chairman of the Senate Armed Services Committee,
replied, ``Just trust me, I'll make it happen.''

Mr. Trump went on, ``I had about 95,000 positive retweets on that.
That's a lot,'' appearing to refer to
\href{https://twitter.com/realDonaldTrump/status/1286669631072108546}{a
Twitter post} last Friday in which he said that Mr. Inhofe had assured
him that he won't change the names of ``Military Bases and Forts'' and
that the senator ``is not a believer in `Cancel Culture.'''

Mr. Trump, in the Wednesday night phone call, could be heard criticizing
``cancel culture'' and then told Mr. Inhofe that people ``want to be
able to go back to life,'' and then appeared to dismiss the focus on the
cultural shift taking place across the country with an expletive.

Earlier in the conversation, Mr. Trump and Mr. Inhofe discussed the
possibility of someone ``resigning'' and putting them into another
appointment. That appeared to be about Mr. Tata, whose nomination for
the Pentagon job has become immersed in criticism over his inflammatory
Twitter posts about Muslims, his description of Mr. Obama as a
``terrorist leader,'' and his embrace of conspiracy theories.

Mr. Inhofe could also be heard discussing ``divorces'' and personal
issues that could become a focus of news coverage. That again appeared
to be a reference to Mr. Tata.

Mr. Inhofe announced on Thursday morning that a hearing scheduled for
later in the day to advance Mr. Tata's nomination would be delayed.

An aide to Mr. Inhofe declined to comment on their conversation. Aides
to Mr. Trump did not immediately respond to requests for comment.

\hypertarget{supreme-court-wont-allow-electronic-signatures-for-idaho-ballot-initiative}{%
\subsection{Supreme Court won't allow electronic signatures for Idaho
ballot
initiative.}\label{supreme-court-wont-allow-electronic-signatures-for-idaho-ballot-initiative}}

The Supreme Court on Thursday
\href{https://www.supremecourt.gov/opinions/19pdf/20a18_f2qg.pdf}{blocked
a trial judge's order} that would have allowed a group proposing a
ballot initiative in Idaho to collect electronic signatures in light of
the coronavirus pandemic. Idaho law ordinarily requires hard-copy
signatures.

The group, Reclaim Idaho, seeks to place a measure on the November
ballot that would increase spending on education.

The Supreme Court's brief order was unsigned, and the vote count was
impossible to determine. But Chief Justice John G. Roberts Jr., joined
by Justices Samuel A. Alito Jr., Neil M. Gorsuch and Brett M. Kavanaugh,
issued a concurring opinion setting out his reasoning.

A stay of a trial judge's injunction was warranted, the chief justice
wrote, because the Supreme Court was likely to throw out the injunction,
which he said had placed an unacceptable burden on state election
administrators. ``In addition to preparing for elections with a record
number of absentee ballot requests,'' the chief justice wrote, ``the
county clerks must now also learn, under extraordinary time pressures,
how to verify digital signatures through an entirely new system mandated
by the district court.''

Justice Sonia Sotomayor, joined by Justice Ruth Bader Ginsburg,
dissented, saying that the Supreme Court should have waited for an
appeals court to rule before intervening in the case. This was, she
added, part of an unfortunate pattern.

``By jumping ahead of the court of appeals,'' she wrote, quoting from an
earlier opinion, ``this court once again forgets that it is `a court of
review, not of first view' and undermines the public's expectation that
its highest court will act only after considered deliberation.''

\hypertarget{herman-cain-dies-after-contracting-the-coronavirus}{%
\subsection{Herman Cain dies after contracting the
coronavirus.}\label{herman-cain-dies-after-contracting-the-coronavirus}}

Image

Herman Cain ran for the Republican presidential nomination in
2012.Credit...Monica Lopossay for The New York Times

\href{https://www.nytimes3xbfgragh.onion/2020/07/30/us/politics/herman-cain-dead.html}{Herman
Cain}, the former pizza executive whose insurgent campaign for president
in 2012 catapulted him to fame as an unlikely hero of Tea Party
conservatives, died on Thursday after contracting the coronavirus, a
former staff member confirmed. He was 74.

Mr. Cain attended Mr. Trump's rally in Tulsa, Okla., last month and was
\href{https://twitter.com/THEHermanCain/status/1274489632886075398?s=20}{photographed
not wearing a mask}, though he said in a video on his website about the
rally that he had
\href{https://hermancain.com/trump-tulsa-rally-i-was-there/}{worn one}
at times. He tested positive for coronavirus and was hospitalized in the
Atlanta area shortly after.

``People were concerned, because of the media, about whether or not this
was going to turn into another uptick in number of cases of Covid-19,''
Mr. Cain said in the video. On Twitter, the night of the event, he
encouraged attendees to
\href{https://twitter.com/THEHermanCain/status/1274438874253496320}{``ignore
the outrage'' and ``shaming.''}

Mr. Cain said that all rally participants, including him, had their
temperatures checked and that some people had worn masks. Sanitizer
stations were scattered throughout the arena.

``Whether or not it's going to work or not, we don't know, but the
chances are even though it was a crowded room of people, if they took
precautions, probably not going to be a big uptick,'' he said.

While it is not clear where Mr. Cain contracted the virus, public health
officials, who had urged the Trump campaign to call off the rally
because of a surge of cases in the state, have said that
\href{https://apnews.com/ad96548245e186382225818d8dc416eb}{the event was
a likely source of an uptick in cases} reported in Tulsa County.

Even after being hospitalized, he tweeted approvingly of masks not being
required at Mr. Trump's Mount Rushmore speech, which occurred nearly two
weeks after the Tulsa event. ``Masks will not be mandatory for the
event, which will be attended by President Trump,'' he wrote in a
since-deleted tweet. ``PEOPLE ARE FED UP!''

Mr. Cain, who was an official surrogate for Trump's 2020 re-election
campaign, also
\href{https://www.westernjournal.com/herman-cain-tulsa-trump-rally-crowd-huge-enthusiastic/}{wrote
an op-ed after the rally defending the event}, writing, ``The media
worked very hard to scare people out of attending the Trump campaign
rally last Saturday night in Tulsa.''

Mr. Cain's political ambitions were
\href{https://www.nytimes3xbfgragh.onion/2019/04/05/business/herman-cain-federal-reserve.html}{derailed
after women stepped forward and accused him of sexual misconduct}. But
his political celebrity endured after the election as he brought his
folksy, irreverent style --- best captured by his so-called ``9-9-9''
tax plan that would have set corporate, personal income and sales taxes
to 9 percent --- to Fox News and conservative political conferences
across the country.

He caught the eye of a similarly styled political novice, Donald J.
Trump, who
later\href{https://www.nytimes3xbfgragh.onion/2019/04/05/business/herman-cain-federal-reserve.html}{considered
nominating} Mr. Cain to a seat on the Federal Reserve's board of
governors. Mr. Trump did not follow through, though Mr. Cain remained a
political ally and served on the Trump campaign's Black Voices for Trump
coalition.

\hypertarget{bloombergs-gun-control-group-will-release-new-ads-geared-toward-latinos-in-swing-states}{%
\subsection{Bloomberg's gun control group will release new ads geared
toward Latinos in swing
states.}\label{bloombergs-gun-control-group-will-release-new-ads-geared-toward-latinos-in-swing-states}}

Nearly a year after the El Paso shooting, the
\href{https://www.nytimes3xbfgragh.onion/2020/02/06/us/politics/el-paso-shooting-federal-hate-crimes.html}{deadliest
anti-Latino attack in modern American history}, Michael R. Bloomberg's
gun control organization is releasing new
\href{https://youtu.be/UG1oqokLTwU}{advertisements} directed toward
Latino voters in battleground states including Texas, Arizona and
Florida.

The group, Everytown for Gun Safety, said it planned to spend more than
\$2 million on digital, radio and television advertising in both English
and \href{https://youtu.be/Rf-zrBP9b70}{Spanish}, starting in Texas.

The group's advertising criticizes Mr. Trump and Senator Mitch
McConnell, the majority leader, for not doing more to enact stricter gun
control laws since the El Paso attack, accusing them of ``empty words
and empty promises.'' The police said that the suspect in the shooting
told them he had been targeting Mexicans and that he had left a
manifesto saying he was carrying out the attack in ``response to the
Hispanic invasion of Texas.''

``Victory for gun sense candidates up and down the ballot is impossible
without the support of Latino voters,'' said Charlie Kelly, a senior
adviser to the group, Everytown for Gun Safety. ``We're making it a
priority to reach the Latino community this cycle with gun safety
messages that we know will resonate.''

The group also released new polling showing that Latino voters are more
likely to support gun control measures enthusiastically than they were
before the El Paso shooting. More than two-thirds of Latino voters said
they would not vote for a candidate who does not support background
checks on all gun sales and indicated the issue is equally important as
health care, job creation, racial equity and protecting children at the
border, according to the poll, which was conducted by Equis Research, a
Washington-based firm that focuses on Latino voters.

\hypertarget{another-florida-poll-shows-biden-in-front}{%
\subsection{Another Florida poll shows Biden in
front.}\label{another-florida-poll-shows-biden-in-front}}

A survey released on Thursday by Mason-Dixon Polling \& Strategy found
Mr. Biden ahead of Mr. Trump in Florida by a margin of 50 to 46 percent,
just within the margin of error of four percentage points, with 4
percent of respondents undecided.

It is the latest in a string of polls showing Mr. Biden in front of Mr.
Trump in the president's adopted home state, the nation's largest
battleground.

The Mason-Dixon poll, which surveyed 625 registered voters by phone from
July 20-23, shows Mr. Biden beating Mr. Trump among independents, women,
Black and Hispanic people, and younger voters, while Mr. Trump leads
among men, white people and older voters.

An earlier poll by
\href{https://poll.qu.edu/florida/release-detail?ReleaseID=3668}{Quinnipiac
University} showed the former vice president leading Mr. Trump by 13
percentage points, significantly outside the margin of error --- a
worrisome sign for Mr. Trump in a state he won four years ago.

In that July 23 poll, 51 percent of respondents supported Mr. Biden,
compared with 38 percent who supported Mr. Trump; the margin of error
was plus or minus 3.2 percentage points.

\hypertarget{larry-household-is-indicted-on-racketeering-charges-and-removed-as-ohios-house-speaker}{%
\subsection{Larry Household is indicted on racketeering charges and
removed as Ohio's House
speaker.}\label{larry-household-is-indicted-on-racketeering-charges-and-removed-as-ohios-house-speaker}}

Image

Larry Householder was indicted for what the U.S. Attorney for the
Southern District of Ohio called ``likely the largest bribery,
money-laundering scheme ever perpetrated against the people of the State
of Ohio.''Credit...John Minchillo/Associated Press

A federal grand jury has indicted Larry Householder, the speaker of the
Ohio House of Representatives, as well as four others on racketeering
charges in what federal officials described as a \$60 million scheme to
bail out a foundering energy company.

Shortly after the indictment was announced on Thursday, the State House
voted 90-0 to remove Mr. Householder, a Republican, as speaker.

In a criminal complaint last week, the F.B.I.
\href{https://www.nytimes3xbfgragh.onion/2020/07/21/us/larry-householder-ohio-speaker-arrested.html}{described}
a wide-ranging conspiracy in which an energy company, FirstEnergy Corp.,
and some of its subsidiaries, helped finance Mr. Householder's 2019
election as House speaker. According to the complaint, the company
bankrolled an effort led by Mr. Householder to pass the \$1.3 billion
bill that subsidized two of its troubled nuclear power plants, and then
also financed a campaign to defeat a referendum to repeal the measure.

Mr. Householder and the four others who were arrested --- including a
former head of the state's Republican Party --- face up to 20 years in
prison.

When the criminal complaint was unsealed, the United States Attorney for
the Southern District of Ohio, David M. DeVillers, said that the
conspiracy was ``likely the largest bribery, money-laundering scheme
ever perpetrated against the people of the State of Ohio.''

Gov. Mike DeWine, a Republican, called for the bailout bill to be
repealed and then replaced in a process involving greater public
vetting.

``The most important thing is that the public have confidence in the
process,'' Mr. DeWine said in a news conference last week.

Mr. DeWine has also called on Mr. Householder to resign, though Mr.
Householder previously
\href{https://www.cleveland.com/open/2020/07/house-speaker-larry-householder-says-he-wont-resign-despite-arrest.html}{told
reporters} that he was not planning to leave office.

\hypertarget{an-arrest-has-been-made-in-the-suspected-arson-at-the-arizona-democrats-headquarters}{%
\subsection{An arrest has been made in the suspected arson at the
Arizona Democrats'
headquarters.}\label{an-arrest-has-been-made-in-the-suspected-arson-at-the-arizona-democrats-headquarters}}

Image

A fire at the Arizona Democratic headquarters last week caused damage
but no injuries.Credit...Matt York/Associated Press

A man accused of setting a fire to the Arizona Democratic Party
headquarters this month was arrested on Wednesday, the authorities said.

The man, Matthew Egler, 29, was arrested on a charge of arson of an
occupied structure,
\href{https://www.phoenix.gov/newsroom/police/1428}{the Phoenix Police
Department said}. Mr. Egler, the police said, was a former volunteer at
the office but had been recently barred from volunteer service.

It was not immediately clear whether he had a lawyer, or why he had been
barred from volunteering.

The fire took place after midnight on July 24 at the party's offices in
downtown Phoenix, causing damage but no injuries.

The Phoenix police said that Mr. Egler had posted information on social
media that linked him to the fire. Investigators were also able to
connect a car seen in \href{https://youtu.be/MwKENYmA6fs}{a surveillance
video} from that night to a relative of Mr. Egler. The video showed a
man arriving in the car and breaking glass to get into the building.

``We are deeply saddened and shocked by today's news, but appreciate the
swift action by law enforcement to ensure that the suspect is in
custody,'' the Arizona Democratic
Party\href{https://azdem.org/joint-statement-arizona-democratic-party-and-maricopa-county-democratic-party-react-to-new-announcement-in-investigation/}{said}
in a statement.

Reporting was contributed by Maggie Astor, Emily Badger, Luke
Broadwater, Alexander Burns, Emily Cochrane, Nate Cohn, Johnny Diaz,
Reid J. Epstein, Sydney Ember, Robert Gebeloff, Katie Glueck, Shane
Goldmacher, Maggie Haberman, Annie Karni,Adam Liptak, Patricia Mazzei,
Giulia McDonnell Nieto del Rio, Jennifer Medina, Jeremy W. Peters, Katie
Rogers, Matt Stevens and Glenn Thrush.

\hypertarget{our-2020-election-guide}{%
\section{Our 2020 Election Guide}\label{our-2020-election-guide}}

Updated July 31, 2020

\begin{itemize}
\item
  \begin{center}\rule{0.5\linewidth}{\linethickness}\end{center}

  \hypertarget{the-latest}{%
  \subsection{The Latest}\label{the-latest}}

  \begin{itemize}
  \tightlist
  \item
    President Trump's assault on the Postal Service is intersecting with
    his attacks on mail-in voting.
    \href{https://www.nytimes3xbfgragh.onion/2020/07/31/us/politics/trump-usps-mail-delays.html?action=click\&pgtype=Article\&state=default\&region=BELOW_MAIN_CONTENT\&context=storylines_guide}{Voting
    rights groups say it is a recipe for disaster.}
  \end{itemize}
\item
  \begin{center}\rule{0.5\linewidth}{\linethickness}\end{center}

  \hypertarget{bidens-vp-search}{%
  \subsection{Biden's V.P. Search}\label{bidens-vp-search}}

  \begin{itemize}
  \tightlist
  \item
    \href{https://www.nytimes3xbfgragh.onion/article/biden-vice-president-2020.html?action=click\&pgtype=Article\&state=default\&region=BELOW_MAIN_CONTENT\&context=storylines_guide}{Here
    are 13 women} who have been under consideration to be Joe Biden's
    running mate, and why each might be chosen --- and might not be.
  \end{itemize}
\item
  \begin{center}\rule{0.5\linewidth}{\linethickness}\end{center}

  \hypertarget{keep-up-with-our-coverage}{%
  \subsection{Keep Up With Our
  Coverage}\label{keep-up-with-our-coverage}}

  \begin{itemize}
  \tightlist
  \item
    Get an
    \href{https://www.nytimes3xbfgragh.onion/newsletters/politics?action=click\&pgtype=Article\&state=default\&region=BELOW_MAIN_CONTENT\&context=storylines_guide}{email}
    recapping the day's news
  \end{itemize}

  \begin{itemize}
  \tightlist
  \item
    Download our mobile app on
    \href{https://apps.apple.com/us/app/nytimes/id284862083?ls=1\&mat_click_id=5c79ae7455014fd1bd66b5610c05b8f2-20191112-16948\&referrer=mat_click_id\%3D5c79ae7455014fd1bd66b5610c05b8f2-20191112-16948\%26link_click_id\%3D722930677036718082}{iOS}
    and
    \href{http://a.localytics.com/android?id=com.nytimes.android\&referrer=utm_source\%3Dother_nyt_mobile_web\%26utm_medium\%3DWeb\%2520page\%26utm_term\%3DGeneral\%2520Mobile\%2520Page\%26utm_campaign\%3DNYT\%2520Mobile\%2520General\%2520Page}{Android}
    and turn on Breaking News and Politics alerts
  \end{itemize}
\end{itemize}

Advertisement

\protect\hyperlink{after-bottom}{Continue reading the main story}

\hypertarget{site-index}{%
\subsection{Site Index}\label{site-index}}

\hypertarget{site-information-navigation}{%
\subsection{Site Information
Navigation}\label{site-information-navigation}}

\begin{itemize}
\tightlist
\item
  \href{https://help.nytimes3xbfgragh.onion/hc/en-us/articles/115014792127-Copyright-notice}{©~2020~The
  New York Times Company}
\end{itemize}

\begin{itemize}
\tightlist
\item
  \href{https://www.nytco.com/}{NYTCo}
\item
  \href{https://help.nytimes3xbfgragh.onion/hc/en-us/articles/115015385887-Contact-Us}{Contact
  Us}
\item
  \href{https://www.nytco.com/careers/}{Work with us}
\item
  \href{https://nytmediakit.com/}{Advertise}
\item
  \href{http://www.tbrandstudio.com/}{T Brand Studio}
\item
  \href{https://www.nytimes3xbfgragh.onion/privacy/cookie-policy\#how-do-i-manage-trackers}{Your
  Ad Choices}
\item
  \href{https://www.nytimes3xbfgragh.onion/privacy}{Privacy}
\item
  \href{https://help.nytimes3xbfgragh.onion/hc/en-us/articles/115014893428-Terms-of-service}{Terms
  of Service}
\item
  \href{https://help.nytimes3xbfgragh.onion/hc/en-us/articles/115014893968-Terms-of-sale}{Terms
  of Sale}
\item
  \href{https://spiderbites.nytimes3xbfgragh.onion}{Site Map}
\item
  \href{https://help.nytimes3xbfgragh.onion/hc/en-us}{Help}
\item
  \href{https://www.nytimes3xbfgragh.onion/subscription?campaignId=37WXW}{Subscriptions}
\end{itemize}
