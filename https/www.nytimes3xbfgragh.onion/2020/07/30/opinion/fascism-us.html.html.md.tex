Sections

SEARCH

\protect\hyperlink{site-content}{Skip to
content}\protect\hyperlink{site-index}{Skip to site index}

\href{https://myaccount.nytimes3xbfgragh.onion/auth/login?response_type=cookie\&client_id=vi}{}

\href{https://www.nytimes3xbfgragh.onion/section/todayspaper}{Today's
Paper}

\href{/section/opinion}{Opinion}\textbar{}Fascism: A Concern

\url{https://nyti.ms/2BJlONg}

\begin{itemize}
\item
\item
\item
\item
\item
\end{itemize}

Advertisement

\protect\hyperlink{after-top}{Continue reading the main story}

\href{/section/opinion}{Opinion}

Supported by

\protect\hyperlink{after-sponsor}{Continue reading the main story}

\hypertarget{fascism-a-concern}{%
\section{Fascism: A Concern}\label{fascism-a-concern}}

The word is being used more and more to describe American politics. Is
it an urgent diagnosis or a distraction?

\href{https://www.nytimes3xbfgragh.onion/by/spencer-bokat-lindell}{\includegraphics{https://static01.graylady3jvrrxbe.onion/images/2019/08/23/opinion/Bokat-Lindell-headshot/Bokat-Lindell-headshot-thumbLarge.png}}

By
\href{https://www.nytimes3xbfgragh.onion/by/spencer-bokat-lindell}{Spencer
Bokat-Lindell}

Mr. Bokat-Lindell is a staff editor.

\begin{itemize}
\item
  July 30, 2020
\item
  \begin{itemize}
  \item
  \item
  \item
  \item
  \item
  \end{itemize}
\end{itemize}

\includegraphics{https://static01.graylady3jvrrxbe.onion/images/2020/07/30/opinion/30debatableillo/30debatableillo-articleLarge.jpg?quality=75\&auto=webp\&disable=upscale}

\emph{This article is part of the Debatable newsletter. You can}
\href{https://www.nytimes3xbfgragh.onion/newsletters/debatable}{\emph{sign
up here}} \emph{to receive it on Tuesdays and Thursdays.}

In a tweet on Thursday morning, President Trump floated the very bad
idea of
\href{https://www.nytimes3xbfgragh.onion/2020/07/30/us/elections/biden-vs-trump.html}{delaying
the presidential election}. (He does not have the legal authority to do
so, though that doesn't mean there are no reasons for concern ---
\href{https://www.nytimes3xbfgragh.onion/2020/06/11/opinion/trump-2020-election.html}{more
on those here}.) Within hours, the president's statement was being
condemned, by
\href{https://twitter.com/SteveSchmidtSES/status/1288857374506323974}{conservatives}
and progressives alike, as fascism.

\begin{quote}
1) Surprise!\\
2) Fascism\\
3) He doesn't have the power to do this. \url{https://t.co/oRRhPEusHm}

--- Mehdi Hasan (@mehdirhasan)
\href{https://twitter.com/mehdirhasan/status/1288828096590893057?ref_src=twsrc\%5Etfw}{July
30, 2020}
\end{quote}

It's a word that's been
\href{https://www.youtube.com/watch?v=0jb1BMZflXQ}{appearing} with
\href{https://trends.google.com/trends/explore?q=fascism\&geo=US}{increasing
frequency} recently, including in
\href{https://www.nytimes3xbfgragh.onion/2020/06/10/books/fascism-debate-donald-trump.html}{The
Times}. But what does fascism actually mean? To what extent can American
politics, present and past, be described as fascist? And is it even a
useful word anymore? Here's what people are saying.

\hypertarget{how-fascism-works}{%
\subsection{How fascism works}\label{how-fascism-works}}

\includegraphics{https://static01.graylady3jvrrxbe.onion/images/2018/10/15/autossell/15op-fascism2/15op-fascism2-videoSixteenByNineJumbo1600.jpg}

The word fascism has become so freighted with meaning that it can be
difficult to define; today, it is often used as a shallow epithet for
any politics one strongly dislikes. As a historical term, however,
fascism refers to the current of far-right, anti-democratic
ultranationalism that coursed through Europe in the interwar period.
Although primarily associated with Adolf Hitler, fascism first gained
form as a paramilitary and political movement under Benito Mussolini in
1919. The name of Mussolini's party derived from
\href{http://www.classics.upenn.edu/myth/php/tools/dictionary.php?method=did\&regexp=1082\&setcard=0\&link=0\&media=0}{``fasces,''}
the Latin word for a bundle of wooden rods containing an ax that
symbolized power in ancient Rome, and which Mussolini used to represent
the Italian people bound by the authority of the state.

A fascist government, as Ruth Ben-Ghiat, a historian of authoritarianism
at New York University,
\href{https://www.npr.org/2020/07/27/895737977/what-is-fascism}{explains},
has only one party, led by a dictator who through violence has shut down
all opposition, including from the judiciary, the press and so-called
enemies of the state.

\href{https://www.theatlantic.com/politics/archive/2016/08/american-authoritarianism-under-donald-trump/495263/}{\emph{{[}Read
More: ``Donald Trump and Benito Mussolini''{]}}}

But what makes fascism distinct from other forms of authoritarianism?
Here are a few signature characteristics according to Jason Stanley, a
philosophy professor at Yale and the author of
\href{https://www.nytimes3xbfgragh.onion/2018/09/11/books/review/jason-stanley-how-fascism-works.html}{``How
Fascism Works.''}

\begin{itemize}
\item
  \textbf{The mythic past:} Fascism appeals to an imaginary and glorious
  past destroyed by the forces of liberalism, cosmopolitanism and
  globalism. The fantasy of a uniform past can take on multiple
  dimensions --- racial, cultural, religious --- but it is invariably
  patriarchal. The enshrinement of traditional gender roles lends moral
  authority to the strongman to impose his will on the present.
\item
  \textbf{``Us'' vs. ``them'':} Through appeals to the mythic past,
  fascism establishes a hierarchy of human worth: e.g., law-abiding over
  criminal, hard-working over lazy, racially pure over impure,
  heterosexual over homosexual, abled over disabled. Those deemed worthy
  are considered the nation's true people, or in German, the ``Volk.''
  Those deemed unworthy are singled out as threats to the Volk, ``straw
  men and women ready to be cast into the roles of rapists, murderers,
  terrorists.''
\item
  \textbf{Unreality:} False distinctions between worthy and unworthy
  populations are enforced through propaganda and anti-intellectualism
  that corrode shared reality, degrade language and create fertile
  ground for conspiracy theories to flourish. Crucially, as Hannah
  Arendt
  \href{https://books.google.com/books?id=5872U7QQl8oC\&printsec=frontcover\&source=gbs_ge_summary_r\&cad=0\#v=snippet\&q=It\%20was\%20always\%20a\%20too\%20little\%20noted\%20hallmark\%20of\%20fascist\%20propaganda\&f=false}{wrote},
  the hallmark of fascist propaganda is not just that it promotes lies,
  which is characteristic of propaganda in general, but that it promotes
  lies in service of policy that seeks to make them true.
\item
  \textbf{Atomization:} While fascist movements emphasize certain
  collective identities, they also tend to promote a social Darwinist
  ethic, according to which the individual must struggle against others
  for power and resources in free-market competition. Class divisions
  must therefore be minimized through the dismantling of labor movements
  and unions, possessing as they do the potential to promote solidarity
  across differences that fascism depends on exploiting. That fascism is
  most effective in times of severe economic inequality is another
  reason it targets labor unions.
\end{itemize}

\hypertarget{is-america-slipping-into-fascism}{%
\subsection{Is America slipping into
fascism?}\label{is-america-slipping-into-fascism}}

Critics of President Trump have described him as
\href{https://www.washingtonpost.com/opinions/this-is-how-fascism-comes-to-america/2016/05/17/c4e32c58-1c47-11e6-8c7b-6931e66333e7_story.html?hpid=hp_no-name_opinion-card-b\%3Ahomepage\%2Fstory}{promoting
fascism} since before he won the 2016 election. But the accusations have
gained new force in recent months with the deployment of federal law
enforcement in
\href{https://www.nytimes3xbfgragh.onion/2020/06/02/us/politics/trump-walk-lafayette-square.html}{Washington,
D.C.};
\href{https://www.nytimes3xbfgragh.onion/2020/07/17/us/portland-protests.html}{Portland,
Ore.;} and
\href{https://www.nytimes3xbfgragh.onion/2020/07/20/us/politics/trump-chicago-portland-federal-agents.html}{potentially
elsewhere} to disperse protests, sometimes brutalizing
\href{https://www.nytimes3xbfgragh.onion/video/us/100000007243995/portland-protests-federal-government.html}{protesters},
\href{https://www.businessinsider.com/portland-journalist-recounts-being-shot-in-the-face-by-police-2020-7}{journalists}
and
\href{https://www.nytimes3xbfgragh.onion/2020/07/23/us/portland-protest-tear-gas-mayor.html?action=click\&module=Top\%20Stories\&pgtype=Homepage}{politicians}
in the process.

America, of course, does not have a one-party government, and it is
still holding elections (though fears about their
\href{https://www.nytimes3xbfgragh.onion/2020/07/25/us/politics/2020-election-voter-fraud-interference.html}{future
legitimacy} abound), so it cannot credibly be called a fascist state.
But do recent events bear the mark of fascist tendencies? The Times
columnist Michelle Goldberg thinks so. ``This is a classic way that
violence happens in authoritarian regimes, whether it's Franco's Spain
or whether it's the Russian Empire,'' the historian Timothy Snyder
\href{https://www.nytimes3xbfgragh.onion/2020/07/20/opinion/portland-protests-trump.html}{told}
her. ``The people who are getting used to committing violence on the
border are then brought in to commit violence against people in the
interior.''

In The New York Post, Norman Podhoretz
\href{https://nypost.com/2020/07/23/its-not-fascism-to-protect-federal-property-from-riots-revolutionaries/}{describes}
such declarations as nothing more than ``elite hysterics'': Presidents
are perfectly within their rights to use federal forces to protect
federal property, as many have done before. Federal forces were sent
into Los Angeles in 1992,
\href{https://www.latimes.com/politics/story/2020-06-01/insurrection-act-allow-trump-send-troops-to-state}{at
the request} of California's governor, to control the Rodney King
uprisings, into Washington, Chicago and Baltimore in 1968 after Martin
Luther King Jr.'s assassination and into Chicago in 1877 during the
Great Railroad Strike. As the historian Heather Ann Thompson
\href{https://www.nytimes3xbfgragh.onion/2020/07/23/upshot/trump-portland.html}{told}
The Times, ``The idea of bringing in troops or law enforcement in its
many forms to quell civilian protest is as American as apple pie --- it
is foundational to this nation.''

It is on the shores of American history that discussion about domestic
fascism tends to come to grief. For if one accepts Dr. Stanley's
description, most of the country's politics to date could be said to
evince elements of fascism, as the historian Samuel Moyn
\href{https://www.nybooks.com/daily/2020/05/19/the-trouble-with-comparisons/}{writes}
in The New York Review of Books. And perhaps, he says, it does. When the
Nazis went about designing a legal regime to racialize citizenship and
prevent miscegenation, they looked to American race law for a model, as
the historian James Q. Whitman
\href{https://press.princeton.edu/books/hardcover/9780691172422/hitlers-american-model}{has
documented}: ``In `Mein Kampf,' Hitler praised America as nothing less
than `the one state' that had made progress toward the creation of a
healthy racist order of the kind the Nuremberg Laws were intended to
establish.''

\href{https://www.newyorker.com/magazine/2018/04/30/how-american-racism-influenced-hitler}{\emph{{[}Read
More: ``How American Racism Influenced Hitler''{]}}}

Much has also been made of recent incidents of unidentified federal
agents pulling protesters into unmarked vehicles. Yet as Brandon
Soderberg and Baynard Woods
\href{https://www.theguardian.com/commentisfree/2020/jul/29/think-the-federal-cops-in-portland-are-scary-police-use-these-tactics-all-the-time}{report}
for The Guardian, local police departments have used this
``quasi-fascist tactic'' for years. The plainclothes officers who were
seen in
\href{https://www.nytimes3xbfgragh.onion/2020/07/28/nyregion/nypd-protester-van.html}{a
widely shared video} pulling a New York City protester into an unmarked
van on Tuesday, for example, did so under the authority not of Donald
Trump but of Mayor
\href{https://gothamist.com/news/de-blasio-cops-dragging-protesters-into-unmarked-vans-is-the-kind-of-thing-we-dont-want-to-see}{Bill
de Blasio}.

Still, Dr. Thompson said of Mr. Trump, ``There is a way in which he is
taking this to the next level.'' Clark Neily, the vice president for
criminal justice at the Cato Institute,
\href{https://www.nytimes3xbfgragh.onion/2020/07/23/upshot/trump-portland.html}{pointed
out} that the Trump administration seems to be using federal agents as a
``run-of-the-mill domestic policing force,'' including in cities where
no violent protest has occurred. Unlike in 1968 or 1992, local officials
have not asked for federal intervention. And since then, the number of
federal agencies at the president's disposal has grown. (The Department
of Homeland Security was established only in 2002, and Immigration and
Customs Enforcement only in 2003.)

Ultimately, the semblance of fascism is still very different from the
fact of it. But the journalist Masha Gessen, like Dr. Stanley himself,
believes that the former is reason enough to worry. After all, fascists
have historically come to power through elections. ``Trump is now
performing his idea of power as he imagines it,'' Mx. Gessen
\href{https://www.newyorker.com/news/our-columnists/donald-trumps-fascist-performance}{wrote}
in The New Yorker last month. ``In his intuition, power is autocratic;
it affirms the superiority of one nation and one race; it asserts total
domination; and it mercilessly suppresses all opposition. Whether or not
he is capable of grasping the concept, Trump is performing fascism.''

\hypertarget{the-cost-of-calling-fascism}{%
\subsection{The cost of calling
`fascism'}\label{the-cost-of-calling-fascism}}

The appeal of reading history into the present is plain enough. But what
cost does it incur to understanding? The act of comparison can obscure
distinctions even as it illuminates similarities. Dr. Moyn
\href{https://www.nybooks.com/daily/2020/05/19/the-trouble-with-comparisons/}{argues}
that by comparing the current moment in America to fascism, one relieves
oneself of the responsibility to analyze what is truly new about it.
``For all its other virtues,'' he writes, ``comparison in general does
not do well with the novelty that Trump certainly represents, for all of
his preconditions and sources.'' Nor do analogies to fascism spare much
room to appreciate the ways in which the country's present is continuous
with its past. The historian David A. Bell tweeted:

Might there also be a political cost to invoking fascism? Perhaps, Dr.
Moyn says. But in Mr. Trump's case, the problem with such analogies may
be that they're not so much harmful as useless. ``Occluding what led to
the rise of Trump (who posed as a victims' candidate) and
`Trump-washing' the American political elite before him who led to so
much suffering are less serious mistakes than delaying and distorting a
collective resolve about what steps would lead us out of the present
morass,'' he
\href{https://www.nybooks.com/daily/2020/05/19/the-trouble-with-comparisons/}{writes}.
``Charging fascism does nothing on its own. Only building an alternative
to the present does, which requires imagining it first.''

\emph{Do you have a point of view we missed? Email us at}
\href{mailto:debatable@NYTimes.com}{\emph{debatable@NYTimes.com}}\emph{.
Please note your name, age and location in your response, which may be
included in the next newsletter.}

\begin{center}\rule{0.5\linewidth}{\linethickness}\end{center}

\hypertarget{more-on-the-f-word}{%
\subsubsection{MORE ON THE F WORD}\label{more-on-the-f-word}}

\href{https://www.nytimes3xbfgragh.onion/2020/06/10/books/fascism-debate-donald-trump.html}{``The
Debate Over the Word `Fascism' Takes a New Turn''} \emph{{[}The New York
Times{]}}

\href{https://www.businessinsider.com/is-trump-fascist-jason-stanley-says-it-is-wrong-question-2020-7}{Is
Trump a fascist? That may be the wrong question.} \emph{{[}Business
Insider{]}}

\href{https://newrepublic.com/article/154042/failure-define-fascism-today}{``The
Failure to Define Fascism Today''} \emph{{[}The New Republic{]}}

\href{https://slate.com/news-and-politics/2017/04/defining-fascism-isnt-as-important-as-subjecting-all-political-movements-to-moral-scrutiny.html}{``Defining
fascism isn't as important as subjecting all political movements to
moral scrutiny''}\emph{{[}Slate{]}}

\href{https://www.nybooks.com/daily/2020/01/07/why-historical-analogy-matters/}{``Why
Historical Analogy Matters''} \emph{{[}The New York Review of Books{]}}

\href{https://www.nytimes3xbfgragh.onion/2020/05/19/opinion/coronavirus-trump-orban.html}{``Donald
Trump Doesn't Want Authority''} \emph{{[}The New York Times{]}}

\begin{center}\rule{0.5\linewidth}{\linethickness}\end{center}

\hypertarget{what-youre-saying}{%
\subsubsection{WHAT YOU'RE SAYING}\label{what-youre-saying}}

\emph{Here's what one reader had to say about the last debate:}
\href{https://www.nytimes3xbfgragh.onion/2020/07/28/opinion/trump-china.html}{\emph{What
Would a Cold War With China Look Like?}}

Dave from Pennsylvania: ``I'm less concerned about a new Cold War than I
am about a new civil war at home. \ldots{} We have a deeply divided
country where one half of that divide is heavily armed and feels
threatened by loss of traditional power and a rise in ethnic diversity.
I fear overt conflict is coming. Who will stop it?''

Advertisement

\protect\hyperlink{after-bottom}{Continue reading the main story}

\hypertarget{site-index}{%
\subsection{Site Index}\label{site-index}}

\hypertarget{site-information-navigation}{%
\subsection{Site Information
Navigation}\label{site-information-navigation}}

\begin{itemize}
\tightlist
\item
  \href{https://help.nytimes3xbfgragh.onion/hc/en-us/articles/115014792127-Copyright-notice}{©~2020~The
  New York Times Company}
\end{itemize}

\begin{itemize}
\tightlist
\item
  \href{https://www.nytco.com/}{NYTCo}
\item
  \href{https://help.nytimes3xbfgragh.onion/hc/en-us/articles/115015385887-Contact-Us}{Contact
  Us}
\item
  \href{https://www.nytco.com/careers/}{Work with us}
\item
  \href{https://nytmediakit.com/}{Advertise}
\item
  \href{http://www.tbrandstudio.com/}{T Brand Studio}
\item
  \href{https://www.nytimes3xbfgragh.onion/privacy/cookie-policy\#how-do-i-manage-trackers}{Your
  Ad Choices}
\item
  \href{https://www.nytimes3xbfgragh.onion/privacy}{Privacy}
\item
  \href{https://help.nytimes3xbfgragh.onion/hc/en-us/articles/115014893428-Terms-of-service}{Terms
  of Service}
\item
  \href{https://help.nytimes3xbfgragh.onion/hc/en-us/articles/115014893968-Terms-of-sale}{Terms
  of Sale}
\item
  \href{https://spiderbites.nytimes3xbfgragh.onion}{Site Map}
\item
  \href{https://help.nytimes3xbfgragh.onion/hc/en-us}{Help}
\item
  \href{https://www.nytimes3xbfgragh.onion/subscription?campaignId=37WXW}{Subscriptions}
\end{itemize}
