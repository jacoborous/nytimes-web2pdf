Sections

SEARCH

\protect\hyperlink{site-content}{Skip to
content}\protect\hyperlink{site-index}{Skip to site index}

\href{https://www.nytimes3xbfgragh.onion/section/style}{Style}

\href{https://myaccount.nytimes3xbfgragh.onion/auth/login?response_type=cookie\&client_id=vi}{}

\href{https://www.nytimes3xbfgragh.onion/section/todayspaper}{Today's
Paper}

\href{/section/style}{Style}\textbar{}Coin Shortage? It May Be Time to
Use Your State Quarters

\href{https://nyti.ms/3k40wvg}{https://nyti.ms/3k40wvg}

\begin{itemize}
\item
\item
\item
\item
\item
\end{itemize}

\href{https://www.nytimes3xbfgragh.onion/spotlight/at-home?action=click\&pgtype=Article\&state=default\&region=TOP_BANNER\&context=at_home_menu}{At
Home}

\begin{itemize}
\tightlist
\item
  \href{https://www.nytimes3xbfgragh.onion/2020/07/28/books/time-for-a-literary-road-trip.html?action=click\&pgtype=Article\&state=default\&region=TOP_BANNER\&context=at_home_menu}{Take:
  A Literary Road Trip}
\item
  \href{https://www.nytimes3xbfgragh.onion/2020/07/29/magazine/bored-with-your-home-cooking-some-smoky-eggplant-will-fix-that.html?action=click\&pgtype=Article\&state=default\&region=TOP_BANNER\&context=at_home_menu}{Cook:
  Smoky Eggplant}
\item
  \href{https://www.nytimes3xbfgragh.onion/2020/07/27/travel/moose-michigan-isle-royale.html?action=click\&pgtype=Article\&state=default\&region=TOP_BANNER\&context=at_home_menu}{Look
  Out: For Moose}
\item
  \href{https://www.nytimes3xbfgragh.onion/interactive/2020/at-home/even-more-reporters-editors-diaries-lists-recommendations.html?action=click\&pgtype=Article\&state=default\&region=TOP_BANNER\&context=at_home_menu}{Explore:
  Reporters' Obsessions}
\end{itemize}

Advertisement

\protect\hyperlink{after-top}{Continue reading the main story}

Supported by

\protect\hyperlink{after-sponsor}{Continue reading the main story}

\hypertarget{coin-shortage-it-may-be-time-to-use-your-state-quarters}{%
\section{Coin Shortage? It May Be Time to Use Your State
Quarters}\label{coin-shortage-it-may-be-time-to-use-your-state-quarters}}

In the midst of reduced coin circulation, the U.S. Mint is winding down
its production of novelty quarters. Should they be saved, or spent?

\includegraphics{https://static01.graylady3jvrrxbe.onion/images/2020/08/02/fashion/30STATEQUARTERS1/oakImage-1595600335888-articleLarge.jpg?quality=75\&auto=webp\&disable=upscale}

\href{https://www.nytimes3xbfgragh.onion/by/lora-kelley}{\includegraphics{https://static01.graylady3jvrrxbe.onion/images/2020/07/01/opinion/lora-kelley-author/lora-kelley-author-thumbLarge.png}}

By \href{https://www.nytimes3xbfgragh.onion/by/lora-kelley}{Lora Kelley}

\begin{itemize}
\item
  July 30, 2020
\item
  \begin{itemize}
  \item
  \item
  \item
  \item
  \item
  \end{itemize}
\end{itemize}

Growing up in San Diego in the early 2000s, Kelsey Fehlberg proudly
displayed her state quarters in a map with inserts for each coin. Then,
she said, sometime in middle school, ``I started to be too cool for
it.''

Her collection collected dust in her parents' closet for 15 years, until
she was home for Christmas in 2019. The quarters meant something new to
her as an apartment-dwelling adult reliant on coin laundry. ``It was
just a total treat because my building is \$1.75 per wash and \$1.75 per
dry,'' she said.

Ms. Fehlberg, now 30 and a graphic designer in Chicago, is one of the
millions of Americans who diligently amassed state quarters in the
aughts. When the 50 State Quarters Program ended in 2008, it wasn't long
before another coin campaign began: the America the Beautiful quarter
series, featuring images of national parks and wildlife.

Now, after more than 20 years of cranking out quarters emblazoned with
Americana, the United States Mint is starting to wind down its
production of the themed coins. In early 2021, the America the Beautiful
Quarters Program will come to an end, with a
\href{https://www.usmint.gov/coins/coin-medal-programs/america-the-beautiful-quarters/tuskegee-airmen-national-historic-site}{coin}
honoring the Tuskegee Airmen National Historic Site.

When it began in the late 1990s, the 50 State Quarter Program was an
unusual choice for the traditional U.S. Mint, known for stamping the
faces of presidents and other male figures in American history on
currency. Led by Philip Diehl, then the Mint's director, the program was
widely considered a success: It raised billions in seigniorage, or
excess revenues, for the Mint, Mr. Diehl said.

But he also said his successors should have probably stopped making
themed quarters after the program's original 10-year run. Producing
novelty coins is ``like a performance on the stage,'' Mr. Diehl said.
``You always need to leave them wanting more.''

As it turns out, Americans have recently been left wanting more of all
coins: The coronavirus pandemic, with its curtailing of physical retail
and opportunities for cash payments, has
\href{https://www.nytimes3xbfgragh.onion/2020/06/25/business/economy/coin-shortage-coronavirus.html}{triggered
a national coin shortage}. In a news release last week, the Mint wrote:
``We ask that the American public start spending their coins. The coin
supply problem can be solved with each of us doing our part.''

As for whether Americans should spend their collections of state
quarters specifically, a spokesman for the U.S. Mint wrote in an email:
``We cannot offer an opinion on this. The decision to spend state
quarters is a choice only the individual collectors can make.''

Americans who have been sitting on their state quarter collections for
years may now wish to dust them off, either for entertainment or for
commerce. Jesse Kraft, the assistant curator of the American Numismatic
Society (that's ``coin heads'' in academic parlance), said that in
quarantine, some collectors may be inspired to ``dig out boxes of coins
they haven't looked at in years, sort their collections and even expand
their collecting interests.''

Austin Riddle, a rising high school senior in Alabama, has taken this
time to mine the coins for content. He created a TikTok series in 14
parts titled ``Ranking State Quarters,'' in which he provides commentary
on every state quarter to a soundtrack of songs like the Super Smash
Bros. theme. The videos amassed a total of 600,000 views in late March.

``I thought quarters were kind of boring so I thought it would be funny
to rank them,'' he wrote in a direct message on Instagram in April. His
justifications are brief and tend to accentuate the positive. No. 38,
Utah: ``I like the golden spike.'' No. 47, Kansas: ``It's just a
buffalo, but the sunflower is nice.'' No. 2, Ohio: ``This astronaut is
everything!''

Mr. Riddle joins a comedic tradition of dunking on state quarters. In
the early 2000s, Conan O'Brien had a continuing bit in which he compared
collecting the coins to being a die-hard Dungeons and Dragons player and
spoke of them in an exaggerated nerd drawl. But he also seemed genuinely
jazzed to present over-the-top fakes (``Florida: home of the
crabs-infested Donald Duck costume'').

Starting around 2005, a coin designer named Daniel Carr created a parody
series of state quarters, called ``State Carrters,'' which he forged in
his home mint and sold at craft fairs around the American West. Mr.
Carr, 62, said he got the idea after submitting the winning designs for
New York's and Rhode Island's state quarters. For the Rhode Island
quarter, he was proud to receive Numismatics Magazine's award for best
trade coin of 2001, an honor he likened to winning ``the Emmy Award of
coins.''

Others have felt inspired by the visuals of the coins. In the fall of
2019, Charlotte Clark, 21, started painting intricate details onto state
quarters and posting her creations on TikTok while living in her native
Britain. She said the quarters inspired her to learn about the U.S.
states she had never visited, like Wyoming. She now lives in New York.

Chanel Glenn, a 30-year-old marketing producer in Atlanta, is a lifelong
state quarter collector who said she is attracted to the unique,
artistic designs. She is constantly on the lookout for interesting
coins, she said: ``Any time I see a quarter and it looks like one I
don't have, I keep it. Period.''

Ms. Glenn recalled how, when she was a kid, the release of a new quarter
felt like an exclusive drop. She said her friends are not into
collecting, but she thinks her 5-year-old son will probably get into
quarters when he's older.

Some numismatics experts also suggested that younger generations may
reinvest in coin collection. ``It's actually entirely normal for
collectors to sort of take a break in their collecting in their younger
years, only to come back to it years later, with more money and more
fervor than they ever spent on it in the first place,'' said Sarah
Miller, a senior numismatist at Heritage Auction House.

Because the Mint produced billions of novelty coins over the last two
decades, Ms. Miller said that the vast majority of state quarters are
not worth more than 25 cents. But she added that she has seen quarters
in nearly pristine condition sell for hundreds or even thousands of
dollars at auction.

Mr. Kraft was more skeptical about former collectors renewing their
interest in coins. ``Every coin is a piece of propaganda,'' he said,
making a point that Ms. Fehlberg thought could be a barrier to younger
people's interest in collecting them.

``I just can't imagine Gen Z kids getting to have the kind of
uncomplicated, straightforward excitement about America that I got to
have as a kid,'' she said. Collecting objects for the sake of it may not
align well with young people's cultural values, either.

``The other generations below me are more cognizant of climate change
and how much hoarding and trying to gather resources contribute to
that,'' said Matt James, 29, a New Yorker who grew up collecting state
quarters.

Being a collector is a facet of one's identity, and many young Americans
may also not want to be associated with signs of the past. Even Mr.
Carr, a coin devotee, came to that conclusion, suggesting that the Mint
should ``move away from this stodgy old president theme.''

``People are bored of presidents,'' he said.

Advertisement

\protect\hyperlink{after-bottom}{Continue reading the main story}

\hypertarget{site-index}{%
\subsection{Site Index}\label{site-index}}

\hypertarget{site-information-navigation}{%
\subsection{Site Information
Navigation}\label{site-information-navigation}}

\begin{itemize}
\tightlist
\item
  \href{https://help.nytimes3xbfgragh.onion/hc/en-us/articles/115014792127-Copyright-notice}{©~2020~The
  New York Times Company}
\end{itemize}

\begin{itemize}
\tightlist
\item
  \href{https://www.nytco.com/}{NYTCo}
\item
  \href{https://help.nytimes3xbfgragh.onion/hc/en-us/articles/115015385887-Contact-Us}{Contact
  Us}
\item
  \href{https://www.nytco.com/careers/}{Work with us}
\item
  \href{https://nytmediakit.com/}{Advertise}
\item
  \href{http://www.tbrandstudio.com/}{T Brand Studio}
\item
  \href{https://www.nytimes3xbfgragh.onion/privacy/cookie-policy\#how-do-i-manage-trackers}{Your
  Ad Choices}
\item
  \href{https://www.nytimes3xbfgragh.onion/privacy}{Privacy}
\item
  \href{https://help.nytimes3xbfgragh.onion/hc/en-us/articles/115014893428-Terms-of-service}{Terms
  of Service}
\item
  \href{https://help.nytimes3xbfgragh.onion/hc/en-us/articles/115014893968-Terms-of-sale}{Terms
  of Sale}
\item
  \href{https://spiderbites.nytimes3xbfgragh.onion}{Site Map}
\item
  \href{https://help.nytimes3xbfgragh.onion/hc/en-us}{Help}
\item
  \href{https://www.nytimes3xbfgragh.onion/subscription?campaignId=37WXW}{Subscriptions}
\end{itemize}
