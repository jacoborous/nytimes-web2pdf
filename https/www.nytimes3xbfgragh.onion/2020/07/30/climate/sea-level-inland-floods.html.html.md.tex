Sections

SEARCH

\protect\hyperlink{site-content}{Skip to
content}\protect\hyperlink{site-index}{Skip to site index}

\href{https://www.nytimes3xbfgragh.onion/section/climate}{Climate}

\href{https://myaccount.nytimes3xbfgragh.onion/auth/login?response_type=cookie\&client_id=vi}{}

\href{https://www.nytimes3xbfgragh.onion/section/todayspaper}{Today's
Paper}

\href{/section/climate}{Climate}\textbar{}Rising Seas Could Menace
Millions Beyond Shorelines, Study Finds

\url{https://nyti.ms/318qlSi}

\begin{itemize}
\item
\item
\item
\item
\item
\item
\end{itemize}

\href{https://www.nytimes3xbfgragh.onion/section/climate?action=click\&pgtype=Article\&state=default\&region=TOP_BANNER\&context=storylines_menu}{Climate
and Environment}

\begin{itemize}
\tightlist
\item
  \href{https://www.nytimes3xbfgragh.onion/2020/07/30/climate/sea-level-inland-floods.html?action=click\&pgtype=Article\&state=default\&region=TOP_BANNER\&context=storylines_menu}{Rising
  Seas}
\item
  \href{https://www.nytimes3xbfgragh.onion/interactive/2020/climate/trump-environment-rollbacks.html?action=click\&pgtype=Article\&state=default\&region=TOP_BANNER\&context=storylines_menu}{Trump's
  Changes}
\item
  \href{https://www.nytimes3xbfgragh.onion/interactive/2020/04/19/climate/climate-crash-course-1.html?action=click\&pgtype=Article\&state=default\&region=TOP_BANNER\&context=storylines_menu}{Climate
  101}
\item
  \href{https://www.nytimes3xbfgragh.onion/interactive/2018/08/30/climate/how-much-hotter-is-your-hometown.html?action=click\&pgtype=Article\&state=default\&region=TOP_BANNER\&context=storylines_menu}{Is
  Your Hometown Hotter?}
\item
  \href{https://www.nytimes3xbfgragh.onion/newsletters/climate-change?action=click\&pgtype=Article\&state=default\&region=TOP_BANNER\&context=storylines_menu}{Newsletter}
\end{itemize}

Advertisement

\protect\hyperlink{after-top}{Continue reading the main story}

Supported by

\protect\hyperlink{after-sponsor}{Continue reading the main story}

\hypertarget{rising-seas-could-menace-millions-beyond-shorelines-study-finds}{%
\section{Rising Seas Could Menace Millions Beyond Shorelines, Study
Finds}\label{rising-seas-could-menace-millions-beyond-shorelines-study-finds}}

As climate change raises sea levels, storm surges and high tides will
push farther inland, a team of researchers says.

\includegraphics{https://static01.graylady3jvrrxbe.onion/images/2020/07/31/climate/30CLI-FLOODS3-print/merlin_174795138_88627dcf-7fa9-4208-9a13-90a7ce4846c2-articleLarge.jpg?quality=75\&auto=webp\&disable=upscale}

\href{https://www.nytimes3xbfgragh.onion/by/brad-plumer}{\includegraphics{https://static01.graylady3jvrrxbe.onion/images/2018/02/20/multimedia/author-brad-plumer/author-brad-plumer-thumbLarge.jpg}}

By \href{https://www.nytimes3xbfgragh.onion/by/brad-plumer}{Brad Plumer}

\begin{itemize}
\item
  July 30, 2020
\item
  \begin{itemize}
  \item
  \item
  \item
  \item
  \item
  \item
  \end{itemize}
\end{itemize}

As global warming pushes up ocean levels around the world, scientists
have long warned that many low-lying coastal areas will become
permanently submerged.

But \href{https://www.nature.com/articles/s41598-020-67736-6}{a new
study published Thursday} finds that much of the economic harm from
sea-level rise this century is likely to come from an additional threat
that will arrive even faster: As oceans rise, powerful coastal storms,
crashing waves and extreme high tides will be able to reach farther
inland, putting tens of millions more people and trillions of dollars in
assets worldwide at risk of periodic flooding.

The study, published in the journal Scientific Reports, calculated that
up to 171 million people living today face at least some risk of coastal
flooding from extreme high tides or storm surges, created when strong
winds from hurricanes or other storms pile up ocean water and push it
onshore. While many people are currently protected by sea walls or other
defenses, such as those in the Netherlands, not everyone is.

If the world's nations keep emitting greenhouse gases, and sea levels
rise just 1 to 2 more feet, the amount of coastal land at risk of
flooding would increase by roughly one-third, the research said. In
2050, up to 204 million people currently living along the coasts would
face flooding risks. By 2100, that rises to as many as 253 million
people under a moderate emissions scenario known as RCP4.5. (The actual
number of people at risk may vary, since the researchers did not try to
predict future coastal population changes.)

``Even though average sea levels rise relatively slowly, we found that
these other flooding risks like high tides, storm surge and breaking
waves will become much more frequent and more intense,'' said Ebru
Kirezci, a doctoral candidate at the University of Melbourne in
Australia and lead author of the study. ``Those are important to
consider.''

\href{https://www.nytimes3xbfgragh.onion/section/climate?action=click\&pgtype=Article\&state=default\&region=MAIN_CONTENT_1\&context=storylines_keepup}{}

\hypertarget{climate-and-environment-}{%
\subsubsection{Climate and Environment
›}\label{climate-and-environment-}}

\hypertarget{keep-up-on-the-latest-climate-news}{%
\paragraph{Keep Up on the Latest Climate
News}\label{keep-up-on-the-latest-climate-news}}

Updated July 30, 2020

Here's what you need to know about the latest climate change news this
week:

\begin{itemize}
\item
  \begin{itemize}
  \tightlist
  \item
    \href{https://www.nytimes3xbfgragh.onion/2020/07/30/climate/bangladesh-floods.html?action=click\&pgtype=Article\&state=default\&region=MAIN_CONTENT_1\&context=storylines_keepup}{Floods
    in}\href{https://www.nytimes3xbfgragh.onion/2020/07/30/climate/bangladesh-floods.html?action=click\&pgtype=Article\&state=default\&region=MAIN_CONTENT_1\&context=storylines_keepup}{Bangladesh}
    are punishing the people least responsible for climate change.
  \item
    As climate change raises sea levels,
    \href{https://www.nytimes3xbfgragh.onion/2020/07/30/climate/sea-level-inland-floods.html?action=click\&pgtype=Article\&state=default\&region=MAIN_CONTENT_1\&context=storylines_keepup}{storm
    surges and high tides} are likely to push farther inland.
  \item
    The E.P.A. inspector general plans to investigate whether a rollback
    of fuel efficiency standards
    \href{https://www.nytimes3xbfgragh.onion/2020/07/27/climate/trump-fuel-efficiency-rule.html?action=click\&pgtype=Article\&state=default\&region=MAIN_CONTENT_1\&context=storylines_keepup}{violated
    government rules}.
  \end{itemize}
\end{itemize}

Areas at particular risk include North Carolina, Virginia and Maryland
in the United States, northern France and northern Germany, the
southeastern coast of China, Bangladesh, and the Indian states of West
Bengal and Gujarat.

This flooding could cause serious economic damage. The study found that
people currently living in areas at risk from a 3-foot rise in sea
levels owned \$14 trillion in assets in 2011, an amount equal to 20
percent of global G.D.P. that year.

The authors acknowledge that theirs is a highly imperfect estimate of
the potential costs of sea-level rise. For one, they don't factor in the
likelihood that communities will take action to protect themselves, such
as elevating their homes, building sea walls or retreating inland.

The study also did not account for any valuable infrastructure, such as
roads or factories, that sits in harm's way. A fuller economic
accounting would require further research, Ms. Kireczi said.

There are already signs that periodic flooding is wreaking havoc along
coastlines. A
\href{https://www.nytimes3xbfgragh.onion/2020/07/14/climate/coastal-flooding-noaa.html}{July
analysis} from the National Oceanic and Atmospheric Administration found
that high-tide flooding in cities along the Atlantic and Gulf Coast has
increased fivefold since 2000, a shift that is damaging homes,
imperiling drinking-water supplies and inundating roads.

The new study tries to improve projections of future coastal flooding
risk by combining existing models of sea-level rise, tides, waves, storm
surges and coastal topography, while checking those models against data
gathered from tidal gauges around the world. Past research, Ms. Kirezci
said, had not looked in such detail at
\href{http://glossary.ametsoc.org/wiki/Wave_set-up/set-down}{factors
like breaking waves} that can temporarily lift local sea levels.

``Trying to model extreme sea levels and storm surge is an extremely
complicated problem and there are still lots of uncertainties,'' said
Michael Oppenheimer, a climate scientist at Princeton University who was
not involved in the study. But, he said, it was critical for scientists
to develop good estimates, because if cities like Boston or New York
hope to build costly new storm surge barriers or other defenses, they'll
need to plan decades before higher sea levels arrive.

The new study found that only one-third of future coastal flooding risk
came from rising sea levels that would permanently submerge low-lying
areas. Two-thirds of the risk came from a likely increase in extreme
high tides, storm surges and breaking waves. In many coastal areas, the
type of rare flooding that historically occurred once every 100 years,
on average, could occur every 10 years or less by the end of the
century.

Scientists say the world's nations can greatly
\href{https://www.nytimes3xbfgragh.onion/2019/09/25/climate/climate-change-oceans-united-nations.html}{reduce
future flooding risks} by cutting emissions rapidly, especially since
that could lower the odds of
\href{https://www.nytimes3xbfgragh.onion/interactive/2017/05/18/climate/antarctica-ice-melt-climate-change.html}{rapid
ice-sheet collapse in Antarctica} that would push up ocean levels even
higher than forecast later in the century.

But, Dr. Oppenheimer added, the world has now warmed so much that
significant sea-level rise by 2050 is assured no matter what happens
with emissions. ``That means we also need to start preparing to adapt
now,'' he said.

Advertisement

\protect\hyperlink{after-bottom}{Continue reading the main story}

\hypertarget{site-index}{%
\subsection{Site Index}\label{site-index}}

\hypertarget{site-information-navigation}{%
\subsection{Site Information
Navigation}\label{site-information-navigation}}

\begin{itemize}
\tightlist
\item
  \href{https://help.nytimes3xbfgragh.onion/hc/en-us/articles/115014792127-Copyright-notice}{©~2020~The
  New York Times Company}
\end{itemize}

\begin{itemize}
\tightlist
\item
  \href{https://www.nytco.com/}{NYTCo}
\item
  \href{https://help.nytimes3xbfgragh.onion/hc/en-us/articles/115015385887-Contact-Us}{Contact
  Us}
\item
  \href{https://www.nytco.com/careers/}{Work with us}
\item
  \href{https://nytmediakit.com/}{Advertise}
\item
  \href{http://www.tbrandstudio.com/}{T Brand Studio}
\item
  \href{https://www.nytimes3xbfgragh.onion/privacy/cookie-policy\#how-do-i-manage-trackers}{Your
  Ad Choices}
\item
  \href{https://www.nytimes3xbfgragh.onion/privacy}{Privacy}
\item
  \href{https://help.nytimes3xbfgragh.onion/hc/en-us/articles/115014893428-Terms-of-service}{Terms
  of Service}
\item
  \href{https://help.nytimes3xbfgragh.onion/hc/en-us/articles/115014893968-Terms-of-sale}{Terms
  of Sale}
\item
  \href{https://spiderbites.nytimes3xbfgragh.onion}{Site Map}
\item
  \href{https://help.nytimes3xbfgragh.onion/hc/en-us}{Help}
\item
  \href{https://www.nytimes3xbfgragh.onion/subscription?campaignId=37WXW}{Subscriptions}
\end{itemize}
