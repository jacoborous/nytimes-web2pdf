Sections

SEARCH

\protect\hyperlink{site-content}{Skip to
content}\protect\hyperlink{site-index}{Skip to site index}

\href{https://www.nytimes3xbfgragh.onion/section/world/asia}{Asia
Pacific}

\href{https://myaccount.nytimes3xbfgragh.onion/auth/login?response_type=cookie\&client_id=vi}{}

\href{https://www.nytimes3xbfgragh.onion/section/todayspaper}{Today's
Paper}

\href{/section/world/asia}{Asia Pacific}\textbar{}A Rarely Seen Weapon
Destroys a Helicopter in Afghanistan

\url{https://nyti.ms/39O4vaA}

\begin{itemize}
\item
\item
\item
\item
\item
\end{itemize}

Advertisement

\protect\hyperlink{after-top}{Continue reading the main story}

Supported by

\protect\hyperlink{after-sponsor}{Continue reading the main story}

\hypertarget{a-rarely-seen-weapon-destroys-a-helicopter-in-afghanistan}{%
\section{A Rarely Seen Weapon Destroys a Helicopter in
Afghanistan}\label{a-rarely-seen-weapon-destroys-a-helicopter-in-afghanistan}}

Another Afghan helicopter was hit in January by an anti-tank guided
missile in southern Afghanistan, in a swath of territory long contested
by the Taliban.

\includegraphics{https://static01.graylady3jvrrxbe.onion/images/2020/08/02/us/politics/02dc-missile-print/merlin_169431351_f193efcd-892c-48e5-af70-d8f474fcb776-articleLarge.jpg?quality=75\&auto=webp\&disable=upscale}

\href{https://www.nytimes3xbfgragh.onion/by/thomas-gibbons-neff}{\includegraphics{https://static01.graylady3jvrrxbe.onion/images/2018/07/12/multimedia/author-thomas-gibbons-neff/author-thomas-gibbons-neff-thumbLarge.png}}\href{https://www.nytimes3xbfgragh.onion/by/mujib-mashal}{\includegraphics{https://static01.graylady3jvrrxbe.onion/images/2018/10/15/multimedia/author-mujib-mashal/author-mujib-mashal-thumbLarge.png}}

By
\href{https://www.nytimes3xbfgragh.onion/by/thomas-gibbons-neff}{Thomas
Gibbons-Neff} and
\href{https://www.nytimes3xbfgragh.onion/by/mujib-mashal}{Mujib Mashal}

\begin{itemize}
\item
  July 30, 2020
\item
  \begin{itemize}
  \item
  \item
  \item
  \item
  \item
  \end{itemize}
\end{itemize}

An Afghan helicopter was attacked in the country's south this week by
what United States and Afghan officials say was a missile rarely seen in
the hands of the Taliban, raising new concerns for a beleaguered Afghan
military and questions about who supplied the weapon.

On Monday, a Black Hawk helicopter was returning from a medical
evacuation mission in Helmand Province and was preparing to land. It is
unclear if the helicopter had touched down or was hovering just feet off
the ground when it was struck by an anti-tank guided missile, American
and Afghan officials said. At least two of the crew members aboard were
wounded, one critically.

It was the second attack of its kind this year. In January, another
Afghan helicopter was hit by an anti-tank guided missile in the same
area near the Kajaki Dam, a swath of territory long contested by the
Taliban, the officials said. Initial reports at the time were
inconclusive about what had struck the helicopter.

American and Afghan officials claim the weapons used in both strikes
were most likely supplied by Iran, but they offered no evidence to
support the assertion. The accusation would be alarming if true, as the
influx of anti-tank guided missiles could not only give the Taliban a
tactical advantage over the Afghan military but also suggest Tehran was
trying to undermine the American mission as it is poised to wind down.
Iran has denied supplying weapons to the Taliban.

Anti-tank guided missiles, which come in many variants, are common in
Syria, Iraq and Yemen, having been captured from military bases and
supplied by countries such as the United States, Russia and Iran. But
the weapons have been rare in Afghanistan, U.S. military officials said.

In the 1980s, the C.I.A.-backed program that funneled arms and supplies
to Afghan insurgents fighting the Soviet Union provided a cache of
anti-tank guided missiles. And in 2008, the Taliban captured at least
one missile and its launcher
\href{https://www.dailystar.com.lb//News/Middle-East/2008/Oct-25/76596-france-plays-down-capture-of-anti-tank-missiles-by-taliban.ashx}{from
the French}.

In 2017, Osprey Flight Solutions, a private company that assesses
threats to commercial aviation in conflict zones,
\href{https://s3-eu-west-1.amazonaws.com/osprey-system-alerts/alert-472-2017-11-10.pdf}{tracked
a shipment of the weapons} into Afghanistan from Pakistan.

``Existing evidence suggests that acquisition and use of portable
anti-tank missiles by armed groups in Afghanistan is limited, especially
in comparison to places like Syria,'' Matthew Schroeder, a senior
researcher for the Small Arms Survey, which tracks the prevalence of
anti-tank guided missiles and other weapons in war zones, said on
Thursday.

Anti-tank guided missiles require training and multiple people to
effectively fire them; for the most part, they are unwieldy. But they
are capable of accurately hitting a target from kilometers away --- well
outside the range of small-arms fire --- making them dangerous to
vehicles, outposts and stationary aircraft. That makes their potential
emergence in Afghanistan especially troubling for the Afghan military,
which fights its battles mostly from checkpoints.

Shooting at helicopters that are on or near the ground, such as the two
incidents in Helmand Province this year, is a tactic that has been used
often by insurgent groups during the conflict in Syria.

Afghanistan's defense ministry said in a statement soon after the attack
this week that the Black Hawk had crashed ``due to technical issues
while it was attempting to land.'' In the days since, security officials
admitted privately that the aircraft was attacked.

One senior Afghan security official said it was near certain that the
helicopter had been hit by an anti-tank missile, but an investigating
team was sent to the Kajaki district on Tuesday to explore further. A
second senior official said he was unaware of this kind of weapon being
deployed against aircraft in Afghanistan beyond the two incidents in
Helmand Province this year.

``Based on what I heard from locals, the helicopter was shot by the
Taliban,'' said Attaullah Afghan, the head of the provincial council in
Helmand. ``The Taliban have got new weapons that they can use against
helicopters when it's on the ground --- a kind of rocket attached to
long wire used against tanks and helicopters. A similar weapon was used
against another aircraft that had landed in Kajaki.''

While Iranian officials have acknowledged their diplomatic channels with
the Taliban, they have repeatedly rejected accusations in recent years
of providing material support to the group. They say they support the
Afghan government in resisting the Taliban's quest for a return of their
Islamic Emirate, which was hostile to neighboring Iran.

``What is important is that we believe in preserving the current
constitution and the political system, we support the Islamic Republic
of Afghanistan and the government,'' Abbas Araghchi, Iran's deputy
foreign minister, said in a recent interview with the Afghan channel
ToloNews. ``Unlike other countries, we haven't come to give weapons or
money to the Taliban.''

In January, after a U.S. drone strike in Iraq killed
\href{https://www.nytimes3xbfgragh.onion/2020/01/03/world/middleeast/suleimani-dead.html}{Maj.
Gen. Qassim Suleimani}, a top Iranian military officer, many Afghan
officials, including the country's president, Ashraf Ghani, were worried
that Iran would use its reach in Afghanistan's messy battlefield to
retaliate against the Americans and intensify the Afghan conflict.
Around the time of Mr. Ghani's inauguration in March, a series of rocket
attacks similar to those launched by Iranian-backed militias in Iraq
seemed to amplify the officials' concerns. One hit an area around the
presidential palace.

Images from the attack on Monday, verified by a U.S. military officer
familiar with the incident, show the burning U.S.-supplied Black Hawk
along with a bundle of guiding wire, a distinct feature on some types of
anti-tank guided missiles.

For the duration of the war, U.S. military intelligence officers have
repeatedly made claims of weapons and supplies flowing from Pakistan,
Iran, Russia and other Central Asian countries to the Taliban, but often
with little proof. American officials have closely tracked the
appearance of surface-to-air missiles and other threats to aircraft, as
any type of foreign involvement with such types of weapons would be
contentious and substantially increase the risk to American and Afghan
forces.

Since anti-tank guided missiles are not designed to specifically target
aircraft, their introduction to the conflict is less likely to draw
significant condemnation from the Americans, the U.S. military officer
said, though it would certainly be an escalation. The United States
provided such weapons to Syrian opposition fighters in 2014 and portable
surface-to-air missiles to Islamist fighters in the 1980s.

About 60,000 Afghan security forces have been killed since 2014, when
U.S. forces began drawing down. And since the beginning of the year,
despite a
\href{https://www.nytimes3xbfgragh.onion/2020/02/29/world/asia/us-taliban-deal.html}{peace
agreement} between the United States and the Taliban in February, Afghan
troops and civilians continue to suffer heavy losses.

Speaking at an event in Kabul on Tuesday, Mr. Ghani said 3,560 Afghan
forces had been killed and nearly 6,800 others wounded since the deal
between the United States and the Taliban. The casualties are possibly
higher, some Afghan officials suggested, with many doubting that the
number included the losses of pro-government militias who bear the brunt
of the fighting. And from Jan. 1 to June 30, 1,282 civilians were killed
and 2,176 were wounded, according to a United Nations report released on
Monday.

On Tuesday, after weeks of deadly attacks on Afghan forces, the Taliban
announced a three-day cease-fire for the Muslim festival of Eid al-Adha.
The announcement came soon after Mr. Ghani said a prisoner swap that had
faced opposition from his government would be completed and that direct
negotiations with the Taliban would start in a week.

But the violence continued right up to the time of the cease-fire, with
a car bomb detonating at a crowded roundabout in Pul e Alam, a city
about 40 miles south of Kabul. Officials said the target was a security
convoy, but the 15 people killed and 30 wounded were a mix of civilians
and military.

Under the deal between the United States and the Taliban, which
initiated the phased withdrawal of American troops, direct peace
negotiations between the Afghan sides were conditioned on swapping 5,000
Taliban prisoners with 1,000 Afghan security forces held by the
insurgents.

Taimoor Shah and Farooq Jan Mangal contributed reporting.

Advertisement

\protect\hyperlink{after-bottom}{Continue reading the main story}

\hypertarget{site-index}{%
\subsection{Site Index}\label{site-index}}

\hypertarget{site-information-navigation}{%
\subsection{Site Information
Navigation}\label{site-information-navigation}}

\begin{itemize}
\tightlist
\item
  \href{https://help.nytimes3xbfgragh.onion/hc/en-us/articles/115014792127-Copyright-notice}{©~2020~The
  New York Times Company}
\end{itemize}

\begin{itemize}
\tightlist
\item
  \href{https://www.nytco.com/}{NYTCo}
\item
  \href{https://help.nytimes3xbfgragh.onion/hc/en-us/articles/115015385887-Contact-Us}{Contact
  Us}
\item
  \href{https://www.nytco.com/careers/}{Work with us}
\item
  \href{https://nytmediakit.com/}{Advertise}
\item
  \href{http://www.tbrandstudio.com/}{T Brand Studio}
\item
  \href{https://www.nytimes3xbfgragh.onion/privacy/cookie-policy\#how-do-i-manage-trackers}{Your
  Ad Choices}
\item
  \href{https://www.nytimes3xbfgragh.onion/privacy}{Privacy}
\item
  \href{https://help.nytimes3xbfgragh.onion/hc/en-us/articles/115014893428-Terms-of-service}{Terms
  of Service}
\item
  \href{https://help.nytimes3xbfgragh.onion/hc/en-us/articles/115014893968-Terms-of-sale}{Terms
  of Sale}
\item
  \href{https://spiderbites.nytimes3xbfgragh.onion}{Site Map}
\item
  \href{https://help.nytimes3xbfgragh.onion/hc/en-us}{Help}
\item
  \href{https://www.nytimes3xbfgragh.onion/subscription?campaignId=37WXW}{Subscriptions}
\end{itemize}
