Sections

SEARCH

\protect\hyperlink{site-content}{Skip to
content}\protect\hyperlink{site-index}{Skip to site index}

\href{https://www.nytimes3xbfgragh.onion/section/movies}{Movies}

\href{https://myaccount.nytimes3xbfgragh.onion/auth/login?response_type=cookie\&client_id=vi}{}

\href{https://www.nytimes3xbfgragh.onion/section/todayspaper}{Today's
Paper}

\href{/section/movies}{Movies}\textbar{}His Film Is a Punk Classic, but
the Credits Now Roll Without Him

\url{https://nyti.ms/2BIEBbD}

\begin{itemize}
\item
\item
\item
\item
\item
\end{itemize}

Advertisement

\protect\hyperlink{after-top}{Continue reading the main story}

Supported by

\protect\hyperlink{after-sponsor}{Continue reading the main story}

\hypertarget{his-film-is-a-punk-classic-but-the-credits-now-roll-without-him}{%
\section{His Film Is a Punk Classic, but the Credits Now Roll Without
Him}\label{his-film-is-a-punk-classic-but-the-credits-now-roll-without-him}}

Amos Poe lost control of his documentary about the music scene that
spawned artists like Blondie and Talking Heads after a dispute with Ivan
Kral, the guitarist who made the movie with him.

\includegraphics{https://static01.graylady3jvrrxbe.onion/images/2020/08/02/arts/02blank-gen1/merlin_174878031_129baf00-0832-4fa7-a5bd-1302fbf9ac6d-articleLarge.jpg?quality=75\&auto=webp\&disable=upscale}

\href{https://www.nytimes3xbfgragh.onion/by/cara-buckley}{\includegraphics{https://static01.graylady3jvrrxbe.onion/images/2018/02/16/multimedia/author-cara-buckley/author-cara-buckley-thumbLarge-v2.png}}

By \href{https://www.nytimes3xbfgragh.onion/by/cara-buckley}{Cara
Buckley}

\begin{itemize}
\item
  July 30, 2020
\item
  \begin{itemize}
  \item
  \item
  \item
  \item
  \item
  \end{itemize}
\end{itemize}

The film, raw and grainy and shot in black and white, is 54 minutes
long. It opens with Patti Smith in silhouette, haloed by her raggedy
hair, and the snarling opening lyrics to her anthemic song
\href{https://www.youtube.com/watch?v=j7sodwiQJ6c}{``Gloria.''}

\emph{Jesus died for somebody's sins but not mine.}

Made by Amos Poe, and his good friend
\href{https://www.nytimes3xbfgragh.onion/2020/02/05/arts/music/ivan-kral-dead.html}{Ivan
Kral}, a guitarist for Smith, the film compiled footage of Richard Hell,
Smith, Blondie, Talking Heads and the Ramones into a feature called
``The Blank Generation,'' named after
\href{https://www.youtube.com/watch?v=JsK8fHPjav0}{one of
Hell's}caterwauling songs.

It premiered in 1976 at CBGB, where much of it was filmed, to a built-in
appreciative crowd, and later secured midnight screenings in cities like
Cincinnati, San Francisco and Toronto. Though it never rose to cult
status, the movie is nonetheless a classic in the punk pantheon, a
signature No Wave film that captured a fleeting time when an eye-popping
number of future rock stars were lean and hungry unknowns.

So it made sense last fall, when ``The Blank Generation'' screened at
the Roxy Cinema, a jewel box of a theater just off Canal Street in New
York, that the cinema's curator, Illyse Singer, invited Poe, whom she
calls ``the godfather of indie cinema in New York,'' to the event.

\includegraphics{https://static01.graylady3jvrrxbe.onion/images/2020/08/02/arts/02blank-gen6/merlin_174136563_d4d2d5cd-690c-4740-99b0-6954a3f079c6-articleLarge.jpg?quality=75\&auto=webp\&disable=upscale}

But as the film rolled, Poe realized that something was wrong. New
segments had been added. Others scrapped. The ending --- of Lenny Kaye,
Smith's longtime bandmate, grinning into the camera as the door to CBGB
swings shut behind him --- had been swapped for a mini-documentary about
his partner, Kral, followed by the words, ``directed by Cindy Hudson''
--- Kral's widow.

The opening placard displaying Poe and Kral's names were gone. In fact,
Poe's name wasn't anywhere on the film.

In that moment, Amos Poe realized he had completely lost control of a
film that, beyond its role as a chronicle of music history, was very
much the pivot point for his entire life.

``I'm trying to be grown up about it,'' said Poe, who is 70. ``But
they're trying to rewrite history.''

\hypertarget{hell-and-heaven-on-the-bowery}{%
\subsection{Hell and Heaven on the
Bowery}\label{hell-and-heaven-on-the-bowery}}

The place stank of vomit and stale beer. Of course it did, it was
\href{https://www.cbgb.com/}{CBGB}, but Amos Poe didn't know what that
meant back in 1974, when he made his first visit.

Poe, a few years from becoming a pioneering filmmaker in downtown's No
Wave Cinema, an underground movement of guerrilla-style films, had been
invited by a clerk at Poe's favorite cinephile shop to check out his
band down on the Bowery.

So one night, Poe found himself venturing past panhandling winos into a
club so fetid and grimy his nerves jumped on end. Two dozen souls
languished inside, most of them belligerent and drunk. Eventually Poe's
buddy, the clerk, shuffled onstage. His name was Richard Hell, his band
was called Television, and after they started playing, one of the
drunks, annoyed at the disturbance, spat at Hell. And to Poe's
astonishment, Hell spat right back.

Poe was entranced.

Image

There was just as much graffiti inside CBGB as on the well-worn exterior
of the club, known for its fetid atmosphere and pioneering role in the
advent of punk.~Credit...Jack Vartoogian/Getty Images

It was around that time that he met Kral, a film buff and rock and
roller whose family had fled Soviet Czechoslovakia. Poe had immigrated
from Israel in 1958 and worshiped Godard and Bresson, and he and Kral
hit it off. Poe was no edgy punk, though, and Kral was his entry point
into the music scene.

``Ivan made Amos cool,'' Kral's first wife, Lynette Kral, said.

Kral had already been filming his musician friends, largely because he
feared deportation back to Czechoslovakia and wanted his memories
preserved. Poe said the pair gathered footage of Bowie, Queen and Roxy
Music into a short picture called
\href{https://www.youtube.com/watch?v=Y9xF04Hvnoc}{``Night Lunch,''} and
as glam music gave way to something more aggressive, they kept shooting,
at CBGB and Max's Kansas City, until they had enough footage to rent the
editing suite on Broadway where, fueled by amphetamines and hashish, Poe
and Kral cut ``The Blank Generation'' in 24 hours.

Image

Debbie Harry and Chris Stein of Blondie at CBGB. Harry appeared in ``The
Blank Generation'' and two other films Poe made.Credit...Roberta
Bayley/Redferns, via Getty Images

The music was added separately using the bands' own recordings or demos,
and it was out of sync, which Poe said was on purpose --- a homage to
experimental film. Not everyone got the point.

At midnight screenings in various cities, half the audience kept walking
out and demanding their money back. But the people who did stay loved
it, which for Poe meant a ton: Nobody was in the middle.

\hypertarget{riffing-on-shoots-and-onstage}{%
\subsection{Riffing on Shoots and
Onstage}\label{riffing-on-shoots-and-onstage}}

The film was something of a buoy at a time when Poe's personal life was
falling apart. Among the issues, he'd lost a job as a building
superintendent. But ``The Blank Generation'' inspired him to keep going.

Poe jumped into writing and directing his first movie,
``\href{https://www.youtube.com/watch?v=vUlzGT7jNvo}{Unmade Beds,}'' a
do-it-yourself picture starring his friends Duncan Hannah, Eric Mitchell
and Debbie Harry, and followed it a with
``\href{https://www.youtube.com/watch?v=FFrn54j06IA}{The Foreigner''}
and later ``Subway Riders,'' all D.I.Y. features shot on the decrepit
streets of New York. Along with fellow filmmakers like Mitchell, James
Nares, Vivienne Dick,
\href{https://www.nytimes3xbfgragh.onion/2019/06/07/movies/the-dead-dont-die-cast.html}{Jim
Jarmusch} and Abel Ferrara, Poe became a notable in the No Wave scene,
and seemed poised to make it big.

``Amos was really inspiring to me as a guerrilla-style filmmaker,'' said
Jarmusch, one of Poe's longtime friends. ``When I first saw `Unmade
Beds' and particularly `The Foreigner,' it really inspired me that I
could make a film too.''

By then, Kral was already living his own dream, playing guitar with the
Patti Smith Group. Kral had been enthralled by Smith from the moment he
had caught one of her searing poetry readings, and in 1974 he bested
some 50 other guitarists for a spot in her band.

He played on the group's first four albums --- ``Horses,'' ``Easter,''
``Radio Ethiopia'' and ``Wave'' --- and wrote songs with Smith including
``Dancing Barefoot.''

``He just became one of us,'' Smith said in
\href{https://www.youtube.com/watch?v=5z39lYLN6lg}{a Czech biopic} about
Kral.

Image

Ivan Kral, left, in Amsterdam with Patti Smith, with whom he performed
for years.~Credit...Rob Verhorst/Redferns, via Getty Images

Kral told the filmmakers: ``At that time, and to this day, there is no
woman that could compare to Patti Smith.''

But in 1979, as the band's popularity grew a year after its breakout hit
``Because the Night,'' Smith abruptly broke up the group on tour in
Italy. Kral was heartbroken.

He played for a spell with Iggy Pop and in other bands, and later
nurtured a solo career in Czechoslovakia, but would never regain the
career high he'd had with Smith.

The Poe-Kral friendship persevered despite some downturns in their own
careers. In 1995, when Smith reunited the band, she did not include Kral
for reasons that never became publicly clear. The exclusion crushed
Kral, and something else niggled him. U2 released their version of
\href{https://www.youtube.com/watch?v=FSSAmMwYK4s}{``Dancing Barefoot''}
as a B-side in 1989, and Kral suspected that he might be owed money.
According to people familiar with the matter, sometime after the band
regrouped without him, Kral sued Smith, to her great distress. The case
ended up settling. (Smith's representatives did not respond to queries.)

``He ruined all chances of ever being invited for a reunion,'' Lynette
Kral said.

Yet Kral still wanted to be close to Smith, and in 2006, he enlisted Poe
in that effort, asking him to get him on the guest list for the
\href{https://www.nytimes3xbfgragh.onion/2006/10/16/arts/music/16cbgb.html}{final
concert at CBGB}. Smith was headlining and Poe, unaware of the lawsuit,
asked the favor of Smith. He remembers her furiously saying no.

Poe went through his own woes as his career as a filmmaker fizzled. In
the late 1980s he had signed on to direct a movie he'd written,
``\href{https://www.youtube.com/watch?v=YY0bqQhVoHM}{Rocket
Gibraltar}.'' It starred Burt Lancaster and was supposed to be his
breakout picture, but he was fired from directing because of cost
overruns. In the mid-90s he declared bankruptcy.

\hypertarget{a-partnership-dissolves}{%
\subsection{A Partnership Dissolves}\label{a-partnership-dissolves}}

As the years passed, interest in ``The Blank Generation'' grew in tandem
with nostalgia for New York's grittier past. The film was periodically
screened. Television productions licensed footage. Rolling Stone
anointed it
\href{https://www.rollingstone.com/movies/movie-lists/25-greatest-punk-rock-movies-of-all-time-103577/the-blank-generation-1976-104139/}{one
of the greatest} punk rock films.

To the chagrin of Kral, Poe was often credited as its sole filmmaker.
And then there was the money. Poe said he and Kral each had a print of
the film that they licensed out, and that they had agreed to equally
split the earnings. But they did not know what the other was earning,
and Kral grew angry because Poe wasn't sharing his cut, even after Kral
demanded an accounting.

Poe says that Kral wasn't sharing his accounting of his earnings either,
though he concedes that by around 2011 he might have owed his friend up
to \$8,000 in all.

But Poe says he couldn't pay. He was consistently broke and was also, by
his own admission, ``hitting bottom on behaviors I wasn't very proud
of,'' namely ``the whole sex, drugs, rock 'n' roll ideology.''

For Kral, the tipping point came in 2011 when a documentary about No
Wave Cinema called
\href{https://www.youtube.com/watch?v=entV87ujz58}{``Blank City''} used
licensed footage from ``The Blank Generation'' and prominently featured
Poe as a No Wave auteur. The film made no mention of Kral, aside from
Poe referencing him briefly.

After Kral didn't see a penny from the film, he sued Poe in a Michigan
court, claiming, among other grievances, that Poe owed him more than
\$75,000 and that Poe had been wrongly recognized as the sole filmmaker
despite having only edited the film.

``Ivan would've preferred to work with Amos on friendly terms,'' said
Cindy Hudson, Kral's widow. ``But Amos didn't want to split any of the
income that he had over all those years. He did not have any money to
pay anything.''

Image

A title image in ``The Blank Generation,'' with one of the words
purposely missing. It fills in later.~Credit...Blank Generation LLC

Poe showed up for a deposition in Michigan in 2011, but says that he
couldn't afford a lawyer, and that he believed he and Kral could work it
all out over the phone. ``I had a real bad attitude; I couldn't get over
my own attitude,'' Poe said. ``I didn't look at it just like a business
thing. I looked at it as a personal betrayal.''

When the case went to trial, Poe skipped the court date, and the judge
found in Kral's favor, ruling that Poe owed Kral \$6,500 in profits from
``The Blank Generation'' plus nearly \$43,000 in lawyers' fees and other
costs. After Poe didn't pay, the judge ordered that Poe's copyright
interest in the ``The Blank Generation'' be seized and sold to Kral.
Early in 2012, the judge ordered that ownership of four of Poe's films
also be sold to Kral (the fee was \$10 apiece): ``Unmade Beds,'' ``The
Foreigner,'' ``Subway Riders'' and ``Empire II.'' They are now listed on
Kral's \href{https://theblankgeneration.com/}{website}, which credits
Kral as the director of ``The Blank Generation'' and Poe as co-editor.

That autumn, the judge awarded another \$107,000 in legal fees to Kral
and issued an order blocking his former partner from presenting
scheduled screenings of ``The Foreigner'' and ``Empire II'' at the New
Museum later that year.

Poe, who worked as an associate professor of film at New York University
and then Brooklyn College, said he couldn't afford the judgment and
didn't pay. In 2018, facing liens and garnished wages related to the
lawsuit, he filed for bankruptcy again. Poe also said he offered Kral
\$35,000 to buy his movies back --- money a friend agreed to loan him
--- but that Kral turned him down.

Poe's old friends were shocked to learn recently that he had lost
ownership of ''The Blank Generation.''

``What a farce that anyone else should claim his inspirational film,''
Debbie Harry, who had appeared in three of Poe's films, wrote in an
email.

\href{https://www.nytimes3xbfgragh.onion/2017/08/25/style/mudd-club-doorman-bowie-basquiat.html}{Richard
Boch}, a former doorman at the Mudd Club and a friend of Poe's, wondered
why there had been so much legal wrangling over such a low-earning film.
``We're not talking about a million-dollar property,'' Boch said.
``We're talking about an underground film that shows every now and then
at an underground film fest or some hipster boutique cinema. Somehow
Amos got screwed out of his legacy here.''

Things likely appear very different in the Kral camp. He died this past
February, and his widow, Hudson, stopped cooperating with this article
after an initial interview.

Poe fears that more changes will be made to ``The Blank Generation.''
Hudson had said in the initial interview that Kral had wanted to remove
the segments that weren't performance footage.

But she has stayed mum in response to questions about whether more edits
might be forthcoming, or whether Poe's name could be restored to the
credits, or whether she would consider letting Poe regain a stake in his
films.

Her lawyer, Susan Kornfield, said in an email that the revamped version
was a new, derivative work and that Poe is not named because, under
copyright law, he is not an author.

It was the revamped version that was screened, and marketed, as ``The
Blank Generation,'' the 1976 classic, in New York last fall.

As a legal matter, Gordon Platt, Poe's lawyer, agreed that the Kral
estate can make the changes it wants to ``The Blank Generation,'' be it
the original or the revamped version, since it now holds the copyrights.

As a practical matter, of course, it's not quite that simple. Poe still
stews inside about skipping that court date all those years ago.

``If I hadn't been as emotional at the time, it probably wouldn't have
been the same,'' Poe said. ``I would've said, `OK, let me deal with it,'
like people do.''

Advertisement

\protect\hyperlink{after-bottom}{Continue reading the main story}

\hypertarget{site-index}{%
\subsection{Site Index}\label{site-index}}

\hypertarget{site-information-navigation}{%
\subsection{Site Information
Navigation}\label{site-information-navigation}}

\begin{itemize}
\tightlist
\item
  \href{https://help.nytimes3xbfgragh.onion/hc/en-us/articles/115014792127-Copyright-notice}{©~2020~The
  New York Times Company}
\end{itemize}

\begin{itemize}
\tightlist
\item
  \href{https://www.nytco.com/}{NYTCo}
\item
  \href{https://help.nytimes3xbfgragh.onion/hc/en-us/articles/115015385887-Contact-Us}{Contact
  Us}
\item
  \href{https://www.nytco.com/careers/}{Work with us}
\item
  \href{https://nytmediakit.com/}{Advertise}
\item
  \href{http://www.tbrandstudio.com/}{T Brand Studio}
\item
  \href{https://www.nytimes3xbfgragh.onion/privacy/cookie-policy\#how-do-i-manage-trackers}{Your
  Ad Choices}
\item
  \href{https://www.nytimes3xbfgragh.onion/privacy}{Privacy}
\item
  \href{https://help.nytimes3xbfgragh.onion/hc/en-us/articles/115014893428-Terms-of-service}{Terms
  of Service}
\item
  \href{https://help.nytimes3xbfgragh.onion/hc/en-us/articles/115014893968-Terms-of-sale}{Terms
  of Sale}
\item
  \href{https://spiderbites.nytimes3xbfgragh.onion}{Site Map}
\item
  \href{https://help.nytimes3xbfgragh.onion/hc/en-us}{Help}
\item
  \href{https://www.nytimes3xbfgragh.onion/subscription?campaignId=37WXW}{Subscriptions}
\end{itemize}
