Sections

SEARCH

\protect\hyperlink{site-content}{Skip to
content}\protect\hyperlink{site-index}{Skip to site index}

\href{https://www.nytimes3xbfgragh.onion/section/books}{Books}

\href{https://myaccount.nytimes3xbfgragh.onion/auth/login?response_type=cookie\&client_id=vi}{}

\href{https://www.nytimes3xbfgragh.onion/section/todayspaper}{Today's
Paper}

\href{/section/books}{Books}\textbar{}In `Pew,' a Mysterious Stranger
Tests a Small Town's Tolerance

\url{https://nyti.ms/32dzhrl}

\begin{itemize}
\item
\item
\item
\item
\item
\item
\end{itemize}

\href{https://www.nytimes3xbfgragh.onion/spotlight/at-home?action=click\&pgtype=Article\&state=default\&region=TOP_BANNER\&context=at_home_menu}{At
Home}

\begin{itemize}
\tightlist
\item
  \href{https://www.nytimes3xbfgragh.onion/2020/07/28/books/time-for-a-literary-road-trip.html?action=click\&pgtype=Article\&state=default\&region=TOP_BANNER\&context=at_home_menu}{Take:
  A Literary Road Trip}
\item
  \href{https://www.nytimes3xbfgragh.onion/2020/07/29/magazine/bored-with-your-home-cooking-some-smoky-eggplant-will-fix-that.html?action=click\&pgtype=Article\&state=default\&region=TOP_BANNER\&context=at_home_menu}{Cook:
  Smoky Eggplant}
\item
  \href{https://www.nytimes3xbfgragh.onion/2020/07/27/travel/moose-michigan-isle-royale.html?action=click\&pgtype=Article\&state=default\&region=TOP_BANNER\&context=at_home_menu}{Look
  Out: For Moose}
\item
  \href{https://www.nytimes3xbfgragh.onion/interactive/2020/at-home/even-more-reporters-editors-diaries-lists-recommendations.html?action=click\&pgtype=Article\&state=default\&region=TOP_BANNER\&context=at_home_menu}{Explore:
  Reporters' Obsessions}
\end{itemize}

Advertisement

\protect\hyperlink{after-top}{Continue reading the main story}

Supported by

\protect\hyperlink{after-sponsor}{Continue reading the main story}

\href{/column/books-of-the-times}{Books of The Times}

\hypertarget{in-pew-a-mysterious-stranger-tests-a-small-towns-tolerance}{%
\section{In `Pew,' a Mysterious Stranger Tests a Small Town's
Tolerance}\label{in-pew-a-mysterious-stranger-tests-a-small-towns-tolerance}}

By \href{https://www.nytimes3xbfgragh.onion/by/dwight-garner}{Dwight
Garner}

\begin{itemize}
\item
  Published July 13, 2020Updated July 15, 2020
\item
  \begin{itemize}
  \item
  \item
  \item
  \item
  \item
  \item
  \end{itemize}
\end{itemize}

\includegraphics{https://static01.graylady3jvrrxbe.onion/images/2020/07/15/books/14BOOKLACEY1/14BOOKLACEY1-articleLarge.png?quality=75\&auto=webp\&disable=upscale}

Buy Book ▾

\begin{itemize}
\tightlist
\item
  \href{https://www.amazon.com/gp/search?index=books\&tag=NYTBSREV-20\&field-keywords=Pew+Catherine+Lacey}{Amazon}
\item
  \href{https://du-gae-books-dot-nyt-du-prd.appspot.com/buy?title=Pew\&author=Catherine+Lacey}{Apple
  Books}
\item
  \href{https://www.anrdoezrs.net/click-7990613-11819508?url=https\%3A\%2F\%2Fwww.barnesandnoble.com\%2Fw\%2F\%3Fean\%3D9780374230920}{Barnes
  and Noble}
\item
  \href{https://www.anrdoezrs.net/click-7990613-35140?url=https\%3A\%2F\%2Fwww.booksamillion.com\%2Fp\%2FPew\%2FCatherine\%2BLacey\%2F9780374230920}{Books-A-Million}
\item
  \href{https://bookshop.org/a/3546/9780374230920}{Bookshop}
\item
  \href{https://www.indiebound.org/book/9780374230920?aff=NYT}{Indiebound}
\end{itemize}

When you purchase an independently reviewed book through our site, we
earn an affiliate commission.

Does it matter what anybody looks like? Philip Roth wrote that the
body's surface is ``as serious a thing as there is in life.'' Susan
Sontag, in an early diary entry, commented that physical beauty was
``enormously, almost morbidly, important to me.''

Michel Houellebecq wrote that the criteria for physical love --- youth,
beauty, strength --- ``are exactly the same as those of Nazism.''
Barbara Kingsolver wrote that ``beautiful people liked to claim looks
didn't matter, while throwing that currency around like novice bank
robbers.''

Joyce Johnson, in her excellent memoir ``Minor Characters,'' spoke for
many when she broached a small but real tragedy --- that her outsides,
somehow, did not reflect her insides.

In Catherine Lacey's strange, estranging and heavy-handed third novel,
``Pew,'' there is a lot of earnest talk about whether these three cubic
feet of bone and blood and meat, to quote a Loudon Wainwright III song,
mean all that much.

``Why did they cause so much trouble for us?'' Lacey's narrator asks.
``Why did we use them against one another? Why did we think the content
of a body meant anything? Why did we draw our conclusions with our
bodies when the body is so inconclusive, so mercurial?''

This narrator --- genderless, racially ambiguous and seemingly mute ---
has a body that makes people nervous. When the narrator is found
sleeping on a church pew, a reverend does the narrator the grave
disservice of naming the stranger Pew.

Image

Catherine Lacey, whose new novel is ``Pew.''Credit....

Pew has few memories, has walked a long way, prefers social distancing
and seems vaguely lobotomized. Has Pew fallen to earth, like the
extraterrestrial in the David Bowie movie? Has Pew suffered a kick from
a horse? Has Pew, like Kurt Vonnegut's Billy Pilgrim, come unstuck in
time?

The action takes place in an unnamed town in the American South. Pew is
taken in by Hilda and Steven and their children; they're a God-fearing
Christian family that hopes to get Pew ``appropriate'' treatment. They
send Pew to counselors, thinking they might have a trauma victim on
their hands.

\emph{{[} This book is one of our most anticipated titles of July.}
\href{https://www.nytimes3xbfgragh.onion/2020/06/24/books/new-july-books.html}{\emph{See
the full list}}\emph{. {]}}

They hire doctors to examine Pew. Everyone wants to know what's between
those legs. It's the tabloid talk of the town, the way Brooke Shields's
virginity was in the 1980s. Pew rebuffs medical explorations. The family
puts Pew in an attic room. Sometimes they lock the door.

The town is looking forward to the Forgiveness Festival, around which
there is excitement and dread. This novel owes a debt to Ursula K. Le
Guin's short story ``The Ones Who Walk Away from Omelas'' (the novel's
epigraph comes from it), which features a similar festival. The reader
will also summon Shirley Jackson's short story ``The Lottery'' to mind,
and the reader is not entirely wrong to do so.

Pew has the ability to peer, if only slightly, into other people's
souls. ``I don't know how it is I can sometimes see all these things in
people --- see these silent things in people --- and though it has been
helpful, I think, at times, so often it feels like an affliction.''

Pew is a blank pawn inserted into the gender wars. Modern-day
\href{https://www.nytimes3xbfgragh.onion/2020/04/10/arts/television/mrs-america-cate-blanchett.html}{Phyllis
Schlaflys} say things to her like: ``Now, you might know that some
people these days like to think a person gets to decide whether they are
a boy or a girl, but we believe, our church believes and Jesus believed
that God decides if you're a boy or a girl.''

This is a novel that takes itself very seriously. The reader who has
kept pace with Lacey's fiction will be willing, mostly, to take it
seriously, too. Born in Mississippi, the author lives in Chicago. She is
the author of two good previous novels,
\href{https://www.nytimes3xbfgragh.onion/2014/07/23/books/catherine-laceys-nobody-is-ever-missing.html}{``Nobody
Is Ever Missing''} and (better)
\href{https://www.nytimes3xbfgragh.onion/2017/05/30/books/review-answers-catherine-lacey.html}{``The
Answers,''} and a book of plangent short stories, ``Certain American
States.''

What works in this novel is its Kafkaesque sense, through Pew, of
free-floating anxiety and mortification of a sort that is impossible to
define and thus impossible to soothe. Pew will not be characterized,
interpreted, diagnosed or annotated. Pew seems to drift, like the
planchette on a Ouija board.

Pew's muteness draws out other people's stories, in the manner of the
fiction of Rachel Cusk, among others. Some of these are confessional and
quite dark, yet few resonate.

Lacey has a mastery of the lives and lingo of the Have a Nice Day crowd,
the kind of people whose defensive optimism keeps them from learning
about anyone. She stacks the deck so heavily against these hair-sprayed
grotesques that they're brittle, however; they crack like dry spaghetti.

This novel walks a high wire between pretentiousness and a kind of cool,
disembodied unease. For me, it fell too often into the goo pit.
``Sometimes I think I might be writing a letter to sleep'' is a
not-atypical comment by Pew. Hilda's tightly held hair, Pew says, ``made
me feel the pressure and presence of every person who had never been
born.''

Pew feels as if Pew is lying perpetually in a canoe, able only to see
the sky above. The reader may feel stuck looking in the other direction,
as if his or her face has been inserted into the equivalent of one of
those holes at the ends of massage tables, where all one can see is
floor tile and dust mites.

Pew is aloof, recessive. People project onto Pew. Some think Pew is an
archangel. Others think, with Pew's brown skin, that Pew should be a
busboy or a dish washer. Pew would prefer not to be called anything; to
name, in this novel, is to take colonial possession. Names are sorting
errors.

Will the town come for Pew with pitchforks and torches? Will this novel
find wind and hoist sail? Lacey is such a talented writer that she casts
a certain spell, even when that spell is distant and difficult to tune
in.

Advertisement

\protect\hyperlink{after-bottom}{Continue reading the main story}

\hypertarget{site-index}{%
\subsection{Site Index}\label{site-index}}

\hypertarget{site-information-navigation}{%
\subsection{Site Information
Navigation}\label{site-information-navigation}}

\begin{itemize}
\tightlist
\item
  \href{https://help.nytimes3xbfgragh.onion/hc/en-us/articles/115014792127-Copyright-notice}{©~2020~The
  New York Times Company}
\end{itemize}

\begin{itemize}
\tightlist
\item
  \href{https://www.nytco.com/}{NYTCo}
\item
  \href{https://help.nytimes3xbfgragh.onion/hc/en-us/articles/115015385887-Contact-Us}{Contact
  Us}
\item
  \href{https://www.nytco.com/careers/}{Work with us}
\item
  \href{https://nytmediakit.com/}{Advertise}
\item
  \href{http://www.tbrandstudio.com/}{T Brand Studio}
\item
  \href{https://www.nytimes3xbfgragh.onion/privacy/cookie-policy\#how-do-i-manage-trackers}{Your
  Ad Choices}
\item
  \href{https://www.nytimes3xbfgragh.onion/privacy}{Privacy}
\item
  \href{https://help.nytimes3xbfgragh.onion/hc/en-us/articles/115014893428-Terms-of-service}{Terms
  of Service}
\item
  \href{https://help.nytimes3xbfgragh.onion/hc/en-us/articles/115014893968-Terms-of-sale}{Terms
  of Sale}
\item
  \href{https://spiderbites.nytimes3xbfgragh.onion}{Site Map}
\item
  \href{https://help.nytimes3xbfgragh.onion/hc/en-us}{Help}
\item
  \href{https://www.nytimes3xbfgragh.onion/subscription?campaignId=37WXW}{Subscriptions}
\end{itemize}
