Sections

SEARCH

\protect\hyperlink{site-content}{Skip to
content}\protect\hyperlink{site-index}{Skip to site index}

\href{https://www.nytimes3xbfgragh.onion/section/us}{U.S.}

\href{https://myaccount.nytimes3xbfgragh.onion/auth/login?response_type=cookie\&client_id=vi}{}

\href{https://www.nytimes3xbfgragh.onion/section/todayspaper}{Today's
Paper}

\href{/section/us}{U.S.}\textbar{}Los Angeles and San Diego Schools to
Go Online-Only in the Fall

\url{https://nyti.ms/2DEtLE9}

\begin{itemize}
\item
\item
\item
\item
\item
\end{itemize}

\href{https://www.nytimes3xbfgragh.onion/news-event/coronavirus?action=click\&pgtype=Article\&state=default\&region=TOP_BANNER\&context=storylines_menu}{The
Coronavirus Outbreak}

\begin{itemize}
\tightlist
\item
  live\href{https://www.nytimes3xbfgragh.onion/2020/08/04/world/coronavirus-cases.html?action=click\&pgtype=Article\&state=default\&region=TOP_BANNER\&context=storylines_menu}{Latest
  Updates}
\item
  \href{https://www.nytimes3xbfgragh.onion/interactive/2020/us/coronavirus-us-cases.html?action=click\&pgtype=Article\&state=default\&region=TOP_BANNER\&context=storylines_menu}{Maps
  and Cases}
\item
  \href{https://www.nytimes3xbfgragh.onion/interactive/2020/science/coronavirus-vaccine-tracker.html?action=click\&pgtype=Article\&state=default\&region=TOP_BANNER\&context=storylines_menu}{Vaccine
  Tracker}
\item
  \href{https://www.nytimes3xbfgragh.onion/2020/08/02/us/covid-college-reopening.html?action=click\&pgtype=Article\&state=default\&region=TOP_BANNER\&context=storylines_menu}{College
  Reopening}
\item
  \href{https://www.nytimes3xbfgragh.onion/live/2020/08/04/business/stock-market-today-coronavirus?action=click\&pgtype=Article\&state=default\&region=TOP_BANNER\&context=storylines_menu}{Economy}
\end{itemize}

Advertisement

\protect\hyperlink{after-top}{Continue reading the main story}

Supported by

\protect\hyperlink{after-sponsor}{Continue reading the main story}

\hypertarget{los-angeles-and-san-diego-schools-to-go-online-only-in-the-fall}{%
\section{Los Angeles and San Diego Schools to Go Online-Only in the
Fall}\label{los-angeles-and-san-diego-schools-to-go-online-only-in-the-fall}}

California's two largest districts made the joint call amid a White
House push to get children back into classrooms.

\includegraphics{https://static01.graylady3jvrrxbe.onion/images/2020/07/13/us/13VIRUS-CALSCHOOLS-la/merlin_170436477_002e2b06-42e2-411c-be04-8f1546659df1-articleLarge.jpg?quality=75\&auto=webp\&disable=upscale}

\href{https://www.nytimes3xbfgragh.onion/by/shawn-hubler}{\includegraphics{https://static01.graylady3jvrrxbe.onion/images/2020/06/05/reader-center/author-shawn-hubler/author-shawn-hubler-thumbLarge.png}}\href{https://www.nytimes3xbfgragh.onion/by/dana-goldstein}{\includegraphics{https://static01.graylady3jvrrxbe.onion/images/2018/06/12/multimedia/author-dana-goldstein/author-dana-goldstein-thumbLarge.png}}

By \href{https://www.nytimes3xbfgragh.onion/by/shawn-hubler}{Shawn
Hubler} and
\href{https://www.nytimes3xbfgragh.onion/by/dana-goldstein}{Dana
Goldstein}

\begin{itemize}
\item
  Published July 13, 2020Updated July 24, 2020
\item
  \begin{itemize}
  \item
  \item
  \item
  \item
  \item
  \end{itemize}
\end{itemize}

SACRAMENTO ---
\href{https://www.nytimes3xbfgragh.onion/2020/07/24/us/ca-schools-reopening.html}{California's}
two largest
\href{https://www.nytimes3xbfgragh.onion/2020/07/24/us/ca-schools-reopening.html}{public
school} districts said on Monday that instruction would be online-only
in the fall, in the latest sign that school administrators are
increasingly unwilling to risk crowding students back into classrooms
until the coronavirus is fully under control.

The school districts in Los Angeles and San Diego, which together enroll
some 825,000 students, are the largest in the country to abandon plans
for even a partial physical return to classrooms when they reopen in
August.

The decision came as Gov. Gavin Newsom
\href{https://twitter.com/GavinNewsom/status/1282752861835649024?s=20}{announced
some of the most sweeping rollbacks yet} of California's plans to
reopen. Indoor operations for restaurants, bars, wineries, movie
theaters and zoos were shut down statewide on Monday, and churches,
gyms, hair salons, malls and other businesses were shuttered for
four-fifths of the population.

``There's a public health imperative to keep schools from becoming a
petri dish,'' said Austin Beutner, the Los Angeles school district's
superintendent.

The California decisions are the latest blow to President Trump's push
to fully
\href{https://www.nytimes3xbfgragh.onion/interactive/2020/07/31/us/coronavirus-school-reopening-risk.html}{reopen
schools} across the country this fall in order to get the economy moving
by enabling parents to return to workplaces. Districts, parents and
teachers have struggled to maintain the education of tens of millions of
K-12 students while keeping them and their teachers healthy and safe.

At the White House, Mr. Trump denounced the decision in Los Angeles,
arguing that schools should resume because children wanted to attend.

``Schools should be opened,'' Mr. Trump said. ``You're losing a lot of
lives by keeping things closed.'' It was not clear what he meant, since
public health experts say the virus spreads quickly in poorly
ventilated, closed areas, the condition of many American schools.

Across the country, school districts are taking a patchwork approach to
reopening.

\href{https://www.nytimes3xbfgragh.onion/2020/07/08/nyregion/nyc-schools-reopening-plan.html}{New
York City}, the nation's largest school district, announced last week
that it would provide several days per week of in-person learning, with
students working online from home the rest of the time.
\href{https://www.seattleschools.org/district/calendars/news/what_s_new/coronavirus_update}{Seattle}
has also announced a hybrid model that is
\href{https://www.nytimes3xbfgragh.onion/2020/06/26/us/coronavirus-schools-reopen-fall.html}{emerging
as popular nationwide}, among both large and small districts. Chicago,
the nation's third-biggest system, has
\href{https://chicago.chalkbeat.org/2020/7/9/21319042/six-things-to-watch-as-chicago-weighs-reopening-school-buildings-this-fall}{not
yet announced} its reopening plan.

But in cities where the virus has continued to rage, efforts at
compromise solutions have increasingly proven unworkable --- a
shattering realization for families that have strained for months to
cobble normalcy out of a situation that is pitting their children's
development and education against parental livelihoods and household
health.

\hypertarget{latest-updates-global-coronavirus-outbreak}{%
\section{\texorpdfstring{\href{https://www.nytimes3xbfgragh.onion/2020/08/04/world/coronavirus-cases.html?action=click\&pgtype=Article\&state=default\&region=MAIN_CONTENT_1\&context=storylines_live_updates}{Latest
Updates: Global Coronavirus
Outbreak}}{Latest Updates: Global Coronavirus Outbreak}}\label{latest-updates-global-coronavirus-outbreak}}

Updated 2020-08-04T21:57:55.984Z

\begin{itemize}
\tightlist
\item
  \href{https://www.nytimes3xbfgragh.onion/2020/08/04/world/coronavirus-cases.html?action=click\&pgtype=Article\&state=default\&region=MAIN_CONTENT_1\&context=storylines_live_updates\#link-2daa96b5}{As
  talks drag on, McConnell signals openness to jobless aid extension
  that Republicans have opposed.}
\item
  \href{https://www.nytimes3xbfgragh.onion/2020/08/04/world/coronavirus-cases.html?action=click\&pgtype=Article\&state=default\&region=MAIN_CONTENT_1\&context=storylines_live_updates\#link-1228a480}{Novavax
  sees encouraging results from two studies of its experimental
  vaccine.}
\item
  \href{https://www.nytimes3xbfgragh.onion/2020/08/04/world/coronavirus-cases.html?action=click\&pgtype=Article\&state=default\&region=MAIN_CONTENT_1\&context=storylines_live_updates\#link-4825b93}{Public
  and private schools in Maryland and elsewhere are divided over
  in-person instruction.}
\end{itemize}

\href{https://www.nytimes3xbfgragh.onion/2020/08/04/world/coronavirus-cases.html?action=click\&pgtype=Article\&state=default\&region=MAIN_CONTENT_1\&context=storylines_live_updates}{See
more updates}

More live coverage:
\href{https://www.nytimes3xbfgragh.onion/live/2020/08/04/business/stock-market-today-coronavirus?action=click\&pgtype=Article\&state=default\&region=MAIN_CONTENT_1\&context=storylines_live_updates}{Markets}

Mahogany Taylor, a 39-year-old mother of two and the president of the
San Diego Unified Council of PTAs, said the loss of in-person
instruction was particularly destructive for elementary school students
--- many of whom cannot type --- and for low-income students, who often
lack internet access, and who make up nearly 60 percent of San Diego
Unified's students.

At the same time, Ms. Taylor said, a districtwide survey showed that 40
percent of parents already were planning to insist on remote
instruction. ``We obviously believe that school is the best place for
kids,'' she said, ``but we also want them to be safe.''

All across the nation, school officials are trying to balance safety
against learning losses.
\href{https://www.nytimes3xbfgragh.onion/2020/06/05/us/coronavirus-education-lost-learning.html}{Initial
research showed} that during the first round of
\href{https://www.nytimes3xbfgragh.onion/2020/07/29/health/covid-school-reopening.html}{school
closures}, American children were set back, on average, by seven months
in their reading and math learning, with children from low-income
families, and students of color, experiencing even bigger losses.

Still, district leaders in Los Angeles and San Diego said, California
was not in a position to reopen schools.

``Those countries that have managed to safely reopen schools have done
so with declining infection rates and on-demand testing available,'' the
statement said. ``California has neither. The skyrocketing infection
rates of the past few weeks make it clear the pandemic is not under
control.''

Mr. Beutner, whose district is the nation's second largest, said in an
interview that schools ``can't just tap our heels together'' like
Dorothy in ``The Wizard of Oz'' and ``pretend it's appropriate to bring
people back'' despite ``skyrocketing'' rates of new infections.

California's death toll from the coronavirus rose to more than 7,000
over the weekend, with 7.4 percent of test results
\href{https://update.covid19.ca.gov/}{coming back positive} over the
past two weeks, even as testing has ramped up to more than 100,000 tests
a day. The state's
\href{https://www.cdph.ca.gov/Programs/CID/DCDC/Pages/COVID-19/CountyMonitoringDataStep2.aspx}{watch
list of counties where the virus has surged}, which has flagged Los
Angeles and San Diego Counties, includes 30 of its 58 counties.

For the time being, Mr. Beutner said, the Los Angeles district will
maintain the online instruction it has been providing since its 700,000
students and 75,000 employees were sent home in mid-March. He said the
decision would be revisited when local infection rates have been
sufficiently lowered and public health authorities have put into place
adequate testing and contact tracing systems.

``It's disappointing,'' he said. ``But at the end of the day, we've got
to make sure everyone's safe.''

\includegraphics{https://static01.graylady3jvrrxbe.onion/images/2020/07/13/us/13VIRUS-CALSCHOOLS-sd/merlin_171588729_6a460be4-7db4-416c-baec-8b0dd671a8fd-articleLarge.jpg?quality=75\&auto=webp\&disable=upscale}

Many parents, students and teachers are still waiting to learn whether
their districts will open this fall.

On Monday night, the Atlanta Public Schools Board of Education is
expected to adopt a plan for full-time remote learning for at least the
first nine weeks of the school year.

Nashville originally planned to open five days a week, but
\href{https://www.tennessean.com/story/news/education/2020/07/09/metro-schools-academic-year-start-online-nashville-students/5383315002/}{rolled
that back} on July 9, citing the rising number of local coronavirus
cases.

Miami-Dade County Public Schools is currently asking parents to choose
between full-time remote learning and a ``schoolhouse model,'' which
would be in-person two to five days a week and online the rest of the
time, depending on the number of students enrolled in a building and the
amount of space available for social distancing.

Schools in New York will only reopen if the state can keep the virus
under control, Gov. Andrew M. Cuomo said on Monday.

The governor said he would allow reopenings only in regions of the state
that have daily infection rates under 5 percent over a two-week average.
Regions with infection rates over 9 percent over a one-week average will
not be allowed to open schools or will automatically have their schools
shuttered.

\href{https://www.nytimes3xbfgragh.onion/news-event/coronavirus?action=click\&pgtype=Article\&state=default\&region=MAIN_CONTENT_3\&context=storylines_faq}{}

\hypertarget{the-coronavirus-outbreak-}{%
\subsubsection{The Coronavirus Outbreak
›}\label{the-coronavirus-outbreak-}}

\hypertarget{frequently-asked-questions}{%
\paragraph{Frequently Asked
Questions}\label{frequently-asked-questions}}

Updated August 4, 2020

\begin{itemize}
\item ~
  \hypertarget{i-have-antibodies-am-i-now-immune}{%
  \paragraph{I have antibodies. Am I now
  immune?}\label{i-have-antibodies-am-i-now-immune}}

  \begin{itemize}
  \tightlist
  \item
    As of right
    now,\href{https://www.nytimes3xbfgragh.onion/2020/07/22/health/covid-antibodies-herd-immunity.html?action=click\&pgtype=Article\&state=default\&region=MAIN_CONTENT_3\&context=storylines_faq}{that
    seems likely, for at least several months.} There have been
    frightening accounts of people suffering what seems to be a second
    bout of Covid-19. But experts say these patients may have a
    drawn-out course of infection, with the virus taking a slow toll
    weeks to months after initial exposure. People infected with the
    coronavirus typically
    \href{https://www.nature.com/articles/s41586-020-2456-9}{produce}
    immune molecules called antibodies, which are
    \href{https://www.nytimes3xbfgragh.onion/2020/05/07/health/coronavirus-antibody-prevalence.html?action=click\&pgtype=Article\&state=default\&region=MAIN_CONTENT_3\&context=storylines_faq}{protective
    proteins made in response to an
    infection}\href{https://www.nytimes3xbfgragh.onion/2020/05/07/health/coronavirus-antibody-prevalence.html?action=click\&pgtype=Article\&state=default\&region=MAIN_CONTENT_3\&context=storylines_faq}{.
    These antibodies may} last in the body
    \href{https://www.nature.com/articles/s41591-020-0965-6}{only two to
    three months}, which may seem worrisome, but that's perfectly normal
    after an acute infection subsides, said Dr. Michael Mina, an
    immunologist at Harvard University. It may be possible to get the
    coronavirus again, but it's highly unlikely that it would be
    possible in a short window of time from initial infection or make
    people sicker the second time.
  \end{itemize}
\item ~
  \hypertarget{im-a-small-business-owner-can-i-get-relief}{%
  \paragraph{I'm a small-business owner. Can I get
  relief?}\label{im-a-small-business-owner-can-i-get-relief}}

  \begin{itemize}
  \tightlist
  \item
    The
    \href{https://www.nytimes3xbfgragh.onion/article/small-business-loans-stimulus-grants-freelancers-coronavirus.html?action=click\&pgtype=Article\&state=default\&region=MAIN_CONTENT_3\&context=storylines_faq}{stimulus
    bills enacted in March} offer help for the millions of American
    small businesses. Those eligible for aid are businesses and
    nonprofit organizations with fewer than 500 workers, including sole
    proprietorships, independent contractors and freelancers. Some
    larger companies in some industries are also eligible. The help
    being offered, which is being managed by the Small Business
    Administration, includes the Paycheck Protection Program and the
    Economic Injury Disaster Loan program. But lots of folks have
    \href{https://www.nytimes3xbfgragh.onion/interactive/2020/05/07/business/small-business-loans-coronavirus.html?action=click\&pgtype=Article\&state=default\&region=MAIN_CONTENT_3\&context=storylines_faq}{not
    yet seen payouts.} Even those who have received help are confused:
    The rules are draconian, and some are stuck sitting on
    \href{https://www.nytimes3xbfgragh.onion/2020/05/02/business/economy/loans-coronavirus-small-business.html?action=click\&pgtype=Article\&state=default\&region=MAIN_CONTENT_3\&context=storylines_faq}{money
    they don't know how to use.} Many small-business owners are getting
    less than they expected or
    \href{https://www.nytimes3xbfgragh.onion/2020/06/10/business/Small-business-loans-ppp.html?action=click\&pgtype=Article\&state=default\&region=MAIN_CONTENT_3\&context=storylines_faq}{not
    hearing anything at all.}
  \end{itemize}
\item ~
  \hypertarget{what-are-my-rights-if-i-am-worried-about-going-back-to-work}{%
  \paragraph{What are my rights if I am worried about going back to
  work?}\label{what-are-my-rights-if-i-am-worried-about-going-back-to-work}}

  \begin{itemize}
  \tightlist
  \item
    Employers have to provide
    \href{https://www.osha.gov/SLTC/covid-19/standards.html}{a safe
    workplace} with policies that protect everyone equally.
    \href{https://www.nytimes3xbfgragh.onion/article/coronavirus-money-unemployment.html?action=click\&pgtype=Article\&state=default\&region=MAIN_CONTENT_3\&context=storylines_faq}{And
    if one of your co-workers tests positive for the coronavirus, the
    C.D.C.} has said that
    \href{https://www.cdc.gov/coronavirus/2019-ncov/community/guidance-business-response.html}{employers
    should tell their employees} -\/- without giving you the sick
    employee's name -\/- that they may have been exposed to the virus.
  \end{itemize}
\item ~
  \hypertarget{should-i-refinance-my-mortgage}{%
  \paragraph{Should I refinance my
  mortgage?}\label{should-i-refinance-my-mortgage}}

  \begin{itemize}
  \tightlist
  \item
    \href{https://www.nytimes3xbfgragh.onion/article/coronavirus-money-unemployment.html?action=click\&pgtype=Article\&state=default\&region=MAIN_CONTENT_3\&context=storylines_faq}{It
    could be a good idea,} because mortgage rates have
    \href{https://www.nytimes3xbfgragh.onion/2020/07/16/business/mortgage-rates-below-3-percent.html?action=click\&pgtype=Article\&state=default\&region=MAIN_CONTENT_3\&context=storylines_faq}{never
    been lower.} Refinancing requests have pushed mortgage applications
    to some of the highest levels since 2008, so be prepared to get in
    line. But defaults are also up, so if you're thinking about buying a
    home, be aware that some lenders have tightened their standards.
  \end{itemize}
\item ~
  \hypertarget{what-is-school-going-to-look-like-in-september}{%
  \paragraph{What is school going to look like in
  September?}\label{what-is-school-going-to-look-like-in-september}}

  \begin{itemize}
  \tightlist
  \item
    It is unlikely that many schools will return to a normal schedule
    this fall, requiring the grind of
    \href{https://www.nytimes3xbfgragh.onion/2020/06/05/us/coronavirus-education-lost-learning.html?action=click\&pgtype=Article\&state=default\&region=MAIN_CONTENT_3\&context=storylines_faq}{online
    learning},
    \href{https://www.nytimes3xbfgragh.onion/2020/05/29/us/coronavirus-child-care-centers.html?action=click\&pgtype=Article\&state=default\&region=MAIN_CONTENT_3\&context=storylines_faq}{makeshift
    child care} and
    \href{https://www.nytimes3xbfgragh.onion/2020/06/03/business/economy/coronavirus-working-women.html?action=click\&pgtype=Article\&state=default\&region=MAIN_CONTENT_3\&context=storylines_faq}{stunted
    workdays} to continue. California's two largest public school
    districts --- Los Angeles and San Diego --- said on July 13, that
    \href{https://www.nytimes3xbfgragh.onion/2020/07/13/us/lausd-san-diego-school-reopening.html?action=click\&pgtype=Article\&state=default\&region=MAIN_CONTENT_3\&context=storylines_faq}{instruction
    will be remote-only in the fall}, citing concerns that surging
    coronavirus infections in their areas pose too dire a risk for
    students and teachers. Together, the two districts enroll some
    825,000 students. They are the largest in the country so far to
    abandon plans for even a partial physical return to classrooms when
    they reopen in August. For other districts, the solution won't be an
    all-or-nothing approach.
    \href{https://bioethics.jhu.edu/research-and-outreach/projects/eschool-initiative/school-policy-tracker/}{Many
    systems}, including the nation's largest, New York City, are
    devising
    \href{https://www.nytimes3xbfgragh.onion/2020/06/26/us/coronavirus-schools-reopen-fall.html?action=click\&pgtype=Article\&state=default\&region=MAIN_CONTENT_3\&context=storylines_faq}{hybrid
    plans} that involve spending some days in classrooms and other days
    online. There's no national policy on this yet, so check with your
    municipal school system regularly to see what is happening in your
    community.
  \end{itemize}
\end{itemize}

New York City, which has maintained an average infection rate of 1 to 2
percent, is on track to partially reopen in September.

All the plans, district leaders say, are subject to change at a moment's
notice, as public health guidance shifts or as governors make statewide
decisions.

Indeed, with the pandemic still raging across much of the country, it
has become clear that improving the quality of online learning will be
at least as important in the coming months as dealing with the logistics
of reopening physical schools.

Several other large California districts, including
\href{https://www.sfchronicle.com/bayarea/article/Santa-Clara-area-school-district-delays-return-to-15403000.php}{Santa
Clara}, \href{https://www.ousd.org/covid-19updates}{Oakland} and
\href{https://www.sbcusd.com/news/what_s_new/July22020_message_from_interim_superintendent}{San
Bernardino}, have already announced that they will stick, at least for
the foreseeable future, with full-time remote instruction, and the
state's politically powerful teachers' unions also have come out against
a return to in-person classes.

The Los Angeles teachers' union called last week for campuses to remain
closed and for learning to be fully remote when the district resumes
classes on Aug. 18, saying Mr. Trump's reopening push was part of a
``dangerous, anti-science agenda.'' In an
\href{https://www.utla.net/news/poll-results-83-utla-members-say-lausd-schools-should-not-physically-reopen-august-18}{informal
survey} of 18,000 United Teachers Los Angeles members that was released
on Friday, 83 percent agreed that campuses should not physically reopen.

And the state's largest teachers' union wrote Mr. Newsom --- a Democrat
elected with their support --- a
\href{http://image.cta-mailings.org/lib/fe8a1574766d017b7c/m/2/2167fb86-b25b-4ce3-9bc7-4248b105a80d.pdf?fbclid=IwAR2QqpANyH9HwsSJJjE1-1NyK_r8bxIcrqucygKKV1ehQ-i_JYCwt3kksZg}{sharply
worded letter} last week expressing concern ``that politics are being
played with the lives of children and the educators who serve them.''

``It is clear that communities and school districts have not come close
to meeting the threshold for a safe return to in-person learning, even
under a hybrid model,'' the 310,000-member California Teachers
Association wrote.

Some \$13.5 billion went to K-12 education from the federal relief
package passed in March by Congress. But education groups and school
districts estimate that
\href{https://www.nytimes3xbfgragh.onion/2020/07/09/us/schools-reopening-trump.html}{schools
will need much more money to safely reopen}, and with the economic
impact of the pandemic having depleted many local and state budgets, it
is unclear where it will come from. The Trump administration has
alternately threatened to cut funds to school districts that fail to
fully reopen and reward districts that do.

As recently as late last week, leaders in San Diego Unified were
promoting their plan to reopen five days a week, in person, for all
students whose families chose that option. But the district had also
warned that the health, sanitation and educational costs of reopening
physical classrooms safely were
\href{https://www.nytimes3xbfgragh.onion/2020/07/09/us/schools-reopening-trump.html}{so
steep} --- a minimum of \$90 million for the coming school year --- that
they would not be able to do so without a significant infusion of
federal dollars.

At the same time, the district's teachers' union was arguing that
reopening during an alarming increase in coronavirus cases was unwise,
and would
\href{https://www.nytimes3xbfgragh.onion/2020/07/11/us/virus-teachers-classrooms.html}{put
teachers' health at risk}.

The superintendent, Cindy Marten, had been working with education
leaders across the country to lobby the Senate to pass a second stimulus
package for schools.

Ms. Marten said the district had not given up on the possibility of
reopening physically if infection rates get down to a safe and
manageable level, and even moved forward over the weekend with plans to
buy \$11 million worth of masks and other protective equipment. But the
state's current infection levels, she said, ``should make it clear to
everyone that the virus is not under control.''

``School districts need to be able to walk and chew gum at the same
time,'' Ms. Marten said. ``We must both plan for a physical reopening
while taking measures to keep our communities safe.''

Shawn Hubler reported from Sacramento and Dana Goldstein from New York.
Eliza Shapiro contributed reporting from New York and Katie Rogers from
Washington.

Advertisement

\protect\hyperlink{after-bottom}{Continue reading the main story}

\hypertarget{site-index}{%
\subsection{Site Index}\label{site-index}}

\hypertarget{site-information-navigation}{%
\subsection{Site Information
Navigation}\label{site-information-navigation}}

\begin{itemize}
\tightlist
\item
  \href{https://help.nytimes3xbfgragh.onion/hc/en-us/articles/115014792127-Copyright-notice}{©~2020~The
  New York Times Company}
\end{itemize}

\begin{itemize}
\tightlist
\item
  \href{https://www.nytco.com/}{NYTCo}
\item
  \href{https://help.nytimes3xbfgragh.onion/hc/en-us/articles/115015385887-Contact-Us}{Contact
  Us}
\item
  \href{https://www.nytco.com/careers/}{Work with us}
\item
  \href{https://nytmediakit.com/}{Advertise}
\item
  \href{http://www.tbrandstudio.com/}{T Brand Studio}
\item
  \href{https://www.nytimes3xbfgragh.onion/privacy/cookie-policy\#how-do-i-manage-trackers}{Your
  Ad Choices}
\item
  \href{https://www.nytimes3xbfgragh.onion/privacy}{Privacy}
\item
  \href{https://help.nytimes3xbfgragh.onion/hc/en-us/articles/115014893428-Terms-of-service}{Terms
  of Service}
\item
  \href{https://help.nytimes3xbfgragh.onion/hc/en-us/articles/115014893968-Terms-of-sale}{Terms
  of Sale}
\item
  \href{https://spiderbites.nytimes3xbfgragh.onion}{Site Map}
\item
  \href{https://help.nytimes3xbfgragh.onion/hc/en-us}{Help}
\item
  \href{https://www.nytimes3xbfgragh.onion/subscription?campaignId=37WXW}{Subscriptions}
\end{itemize}
