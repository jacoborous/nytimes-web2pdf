Sections

SEARCH

\protect\hyperlink{site-content}{Skip to
content}\protect\hyperlink{site-index}{Skip to site index}

\href{https://www.nytimes3xbfgragh.onion/section/arts/music}{Music}

\href{https://myaccount.nytimes3xbfgragh.onion/auth/login?response_type=cookie\&client_id=vi}{}

\href{https://www.nytimes3xbfgragh.onion/section/todayspaper}{Today's
Paper}

\href{/section/arts/music}{Music}\textbar{}Opera Goes On in Salzburg,
With Lots and Lots of Testing

\url{https://nyti.ms/316JvaW}

\begin{itemize}
\item
\item
\item
\item
\item
\end{itemize}

\href{https://www.nytimes3xbfgragh.onion/news-event/coronavirus?action=click\&pgtype=Article\&state=default\&region=TOP_BANNER\&context=storylines_menu}{The
Coronavirus Outbreak}

\begin{itemize}
\tightlist
\item
  live\href{https://www.nytimes3xbfgragh.onion/2020/08/01/world/coronavirus-covid-19.html?action=click\&pgtype=Article\&state=default\&region=TOP_BANNER\&context=storylines_menu}{Latest
  Updates}
\item
  \href{https://www.nytimes3xbfgragh.onion/interactive/2020/us/coronavirus-us-cases.html?action=click\&pgtype=Article\&state=default\&region=TOP_BANNER\&context=storylines_menu}{Maps
  and Cases}
\item
  \href{https://www.nytimes3xbfgragh.onion/interactive/2020/science/coronavirus-vaccine-tracker.html?action=click\&pgtype=Article\&state=default\&region=TOP_BANNER\&context=storylines_menu}{Vaccine
  Tracker}
\item
  \href{https://www.nytimes3xbfgragh.onion/interactive/2020/07/29/us/schools-reopening-coronavirus.html?action=click\&pgtype=Article\&state=default\&region=TOP_BANNER\&context=storylines_menu}{What
  School May Look Like}
\item
  \href{https://www.nytimes3xbfgragh.onion/live/2020/07/31/business/stock-market-today-coronavirus?action=click\&pgtype=Article\&state=default\&region=TOP_BANNER\&context=storylines_menu}{Economy}
\end{itemize}

Advertisement

\protect\hyperlink{after-top}{Continue reading the main story}

Supported by

\protect\hyperlink{after-sponsor}{Continue reading the main story}

\hypertarget{opera-goes-on-in-salzburg-with-lots-and-lots-of-testing}{%
\section{Opera Goes On in Salzburg, With Lots and Lots of
Testing}\label{opera-goes-on-in-salzburg-with-lots-and-lots-of-testing}}

The Salzburg Festival is unfolding its abbreviated centennial season
with an elaborate coronavirus protection plan.

\includegraphics{https://static01.graylady3jvrrxbe.onion/images/2020/08/02/arts/02Salzburg-preview-1/merlin_174868050_10b64ef7-0f15-43b1-8024-0197fd429525-articleLarge.jpg?quality=75\&auto=webp\&disable=upscale}

By Ben Miller

\begin{itemize}
\item
  July 31, 2020
\item
  \begin{itemize}
  \item
  \item
  \item
  \item
  \item
  \end{itemize}
\end{itemize}

SALZBURG, Austria --- A poster advertising this year's Salzburg Festival
bears a quotation from one of the festival's founders, the poet and
dramatist Hugo von Hofmannsthal: ``Wo der Wille erwacht, dort ist schon
fast etwas erreicht.''

Roughly translated: ``Where there's a will, there's a way.''

Plenty of will --- along with political and financial resources few
other classical music organizations could possibly deploy --- is evident
here this summer. For its 100th anniversary season, Salzburg, bucking
the coronavirus-prompted trend of canceling cultural events or
presenting them only with onstage social distancing, is going ahead with
performances featuring casts interacting closely and full orchestras in
the pit.

Which is not to say the festival has been unaffected by the pandemic. A
sprawling, 44-day anniversary program has been mostly postponed until
next year. It has been replaced with a reduced, 30-day schedule, through
Aug. 30, of concerts, plays and two (instead of seven) staged operas:
Strauss's ``Elektra'' and a production of Mozart's ``Così Fan Tutte''
that was planned over the past few months, almost unheard-of short
notice for opera on this level.

While the 90 performances will take place in a country and region where
infections have ebbed and audiences of up to 1,000 --- about half the
capacity of Salzburg's main theater --- have been permitted, the threat
remains present. On July 27,
\href{https://uk.reuters.com/article/uk-health-coronavirus-austria/austrian-resort-town-st-wolfgang-grapples-with-coronavirus-outbreak-idUKKCN24S13J}{health
officials announced} an outbreak of the virus in St. Wolfgang, a
lakeside resort town about 20 miles from Salzburg.

Markus Hinterhäuser, the festival's artistic director, said in an
interview that he felt the ``sword of Damocles'' hanging over the
artists and staff. But in Austria, he added, ``we have measures for
cultural institutions --- which are 200 percent necessary --- that
respect the health of the people working and the audience.''

\includegraphics{https://static01.graylady3jvrrxbe.onion/images/2020/08/02/arts/02Salzburg-preview-2/merlin_174762588_043bc117-121a-4e93-ae33-87f9fd599122-articleLarge.jpg?quality=75\&auto=webp\&disable=upscale}

The otolaryngologist (and trained baritone) Joseph Schlömicher-Thier
helped form the festival's coronavirus plan, which he described in an
interview as ``Fortress Festspielhaus,'' a reference to the festival's
main theater. It puts in place stricter measures than the Austrian
government has mandated.

The festival's theaters will each be capped at about half their
capacities; audiences will sit in a staggered, chessboard-like formation
and will be asked to wear masks as they enter and leave, but can remove
them during performances. Intermissions will be eliminated, and
attendees will provide their contact information with the purchase of
each nontransferable ticket, so that they can be informed if it turns
out they attended a performance with an infected person.

Artists and staff have been divided into three groups, depending on
their ability to socially distance. Singers, orchestra musicians and
others who need to interact with one another closely are in the ``red''
group and are tested weekly, whether they have symptoms or not.

Other workers are divided into ``orange'' (those, like hair and makeup
artists and festival executives, who must closely interact with the red
group but can otherwise socially distance and wear masks) and ``yellow''
(those who can always socially distance and wear masks). Red and orange
employees must keep logs of their health and contacts. Visitors ---
including journalists --- must provide evidence of a recent negative
test before having even distanced contact with members of the red group.

\hypertarget{latest-updates-global-coronavirus-outbreak}{%
\section{\texorpdfstring{\href{https://www.nytimes3xbfgragh.onion/2020/08/01/world/coronavirus-covid-19.html?action=click\&pgtype=Article\&state=default\&region=MAIN_CONTENT_1\&context=storylines_live_updates}{Latest
Updates: Global Coronavirus
Outbreak}}{Latest Updates: Global Coronavirus Outbreak}}\label{latest-updates-global-coronavirus-outbreak}}

Updated 2020-08-01T21:19:55.782Z

\begin{itemize}
\tightlist
\item
  \href{https://www.nytimes3xbfgragh.onion/2020/08/01/world/coronavirus-covid-19.html?action=click\&pgtype=Article\&state=default\&region=MAIN_CONTENT_1\&context=storylines_live_updates\#link-3ac56579}{Top
  officials work to break impasse over jobless benefit.}
\item
  \href{https://www.nytimes3xbfgragh.onion/2020/08/01/world/coronavirus-covid-19.html?action=click\&pgtype=Article\&state=default\&region=MAIN_CONTENT_1\&context=storylines_live_updates\#link-8796723}{The
  virus picks up dangerous speed in the Midwest, and in areas that had
  seen success.}
\item
  \href{https://www.nytimes3xbfgragh.onion/2020/08/01/world/coronavirus-covid-19.html?action=click\&pgtype=Article\&state=default\&region=MAIN_CONTENT_1\&context=storylines_live_updates\#link-25930521}{Thousands
  in Berlin protest Germany's coronavirus measures.}
\end{itemize}

\href{https://www.nytimes3xbfgragh.onion/2020/08/01/world/coronavirus-covid-19.html?action=click\&pgtype=Article\&state=default\&region=MAIN_CONTENT_1\&context=storylines_live_updates}{See
more updates}

More live coverage:
\href{https://www.nytimes3xbfgragh.onion/live/2020/07/31/business/stock-market-today-coronavirus?action=click\&pgtype=Article\&state=default\&region=MAIN_CONTENT_1\&context=storylines_live_updates}{Markets}

The festival has contracted with a private laboratory so that it can
test large numbers of people quickly. For example, the singers,
choristers and orchestra players for ``Elektra,'' a company of around
200, can all be tested at the same time and get results within a few
hours.

``Fortress Festspielhaus'' depends both on Austria's generally strong
public health system and on the festival's deep pockets and connections.
The festival's president, Helga Rabl-Stadler, served as a prominent
member of parliament for Austria's conservative People's Party, now in
power, before joining Salzburg in 1995. (She originally planned to
retire after this summer, but has committed to stay on for an extra year
to see through the postponed centennial celebrations.)

Image

Markus Hinterhäuser, the festival's artistic director.Credit...Lydia
Gorges

Image

Helga Rabl-Stadler, the festival's president.Credit...Lydia Gorges

``Helga here has been fantastic,'' said Franz Welser-Möst, who is
conducting ``Elektra.'' ``You have to be proactive.''

And Austria may be among the only countries where classical music is a
powerful lobby. Daniel Froschauer, a member of the board of directors of
the Vienna Philharmonic and one of its violinists, described in an
interview calling the country's prime minister directly several times
during the lockdown to plan the orchestra's return to public
performance.

``He said he'd put it on his priority list,'' Mr. Froschauer said. ``On
top.''

The Philharmonic, the festival's traditional house band, will play in
the pit for the two operas and in a series of concerts. Its members have
been tested regularly since returning to public performances for small
audiences in Vienna in early June.

``Of course, if there are clusters in Salzburg, the whole thing will be
canceled,'' Mr. Froschauer said. ``The highest value we have is our
health, and we don't want to risk that.''

Although Austria had a strict lockdown from mid-March to mid-April,
Salzburg's narrow streets --- and its shops, cafes and restaurants ---
are now bustling with maskless patrons. But inside the festival's
facilities, said Dr. Schlömicher-Thier, ``our employees are acting as
though it's still March.''

``My fear is that there could be a problem with the people outside ---
tourists,'' he added. ``I tell everyone to be careful where they go.''

While one festival employee tested positive earlier this month, the
administration argues that the episode reveals the strength of the
system it has put in place. ``That girl was in a faraway office and had
no contact with any artists,'' said Ms. Rabl-Stadler, referring to the
infected employee, who is currently self-isolating with mild symptoms.
Ms. Rabl-Stadler added that all of the employee's contacts were tested
within hours and were found not to have the virus. The festival said it
would inform the press and public about every confirmed infection of an
employee, artist or ticket holder.

This ambitious plan comes with a substantial price, and just as
Salzburg's potential revenue has been slashed. The festival is giving
back 24.5 million euros (roughly \$28.2 million) it had sold in tickets
to its original program, and has only 7.5 million euros' worth of new
tickets to sell. While Ms. Rabl-Stadler said that sponsors and donors
had been generous this year, it would be nearly impossible to have
another festival under the same conditions.

``Every concert is a deficit,'' she said. ``For opera, you can't even
talk.''

If coronavirus-related cancellations have been a headache for the opera
world's administrators, they have dealt a devastating financial and
psychological blow to artists. The cast, director and conductor of
``Così Fan Tutte,'' which is being presented in a one-act version of
just over two hours, are aware of their privileged position at a time
when few can perform.

Image

Elsa Dreisig and Ms. Crebassa in ``Così,'' which was planned over just a
few months with the director Christof Loy and the conductor Joana
Mallwitz.Credit...Monika Rittershaus/Salzburger Festspiele

``It's huge luck,'' the soprano Elsa Dreisig said in an interview. Ms.
Dreisig, who will be singing the role of Fiordiligi for the first time,
had her scheduled debut in the part canceled at the Berlin State Opera
in March. Now, she said, ``I feel like I am part of the resistance.''

The mezzo-soprano Marianne Crebassa takes on the role of Dorabella after
two of her scheduled ``Così'' productions were canceled --- including in
Berlin, where she was going to appear opposite Ms. Dreisig.

\href{https://www.nytimes3xbfgragh.onion/news-event/coronavirus?action=click\&pgtype=Article\&state=default\&region=MAIN_CONTENT_3\&context=storylines_faq}{}

\hypertarget{the-coronavirus-outbreak-}{%
\subsubsection{The Coronavirus Outbreak
›}\label{the-coronavirus-outbreak-}}

\hypertarget{frequently-asked-questions}{%
\paragraph{Frequently Asked
Questions}\label{frequently-asked-questions}}

Updated July 27, 2020

\begin{itemize}
\item ~
  \hypertarget{should-i-refinance-my-mortgage}{%
  \paragraph{Should I refinance my
  mortgage?}\label{should-i-refinance-my-mortgage}}

  \begin{itemize}
  \tightlist
  \item
    \href{https://www.nytimes3xbfgragh.onion/article/coronavirus-money-unemployment.html?action=click\&pgtype=Article\&state=default\&region=MAIN_CONTENT_3\&context=storylines_faq}{It
    could be a good idea,} because mortgage rates have
    \href{https://www.nytimes3xbfgragh.onion/2020/07/16/business/mortgage-rates-below-3-percent.html?action=click\&pgtype=Article\&state=default\&region=MAIN_CONTENT_3\&context=storylines_faq}{never
    been lower.} Refinancing requests have pushed mortgage applications
    to some of the highest levels since 2008, so be prepared to get in
    line. But defaults are also up, so if you're thinking about buying a
    home, be aware that some lenders have tightened their standards.
  \end{itemize}
\item ~
  \hypertarget{what-is-school-going-to-look-like-in-september}{%
  \paragraph{What is school going to look like in
  September?}\label{what-is-school-going-to-look-like-in-september}}

  \begin{itemize}
  \tightlist
  \item
    It is unlikely that many schools will return to a normal schedule
    this fall, requiring the grind of
    \href{https://www.nytimes3xbfgragh.onion/2020/06/05/us/coronavirus-education-lost-learning.html?action=click\&pgtype=Article\&state=default\&region=MAIN_CONTENT_3\&context=storylines_faq}{online
    learning},
    \href{https://www.nytimes3xbfgragh.onion/2020/05/29/us/coronavirus-child-care-centers.html?action=click\&pgtype=Article\&state=default\&region=MAIN_CONTENT_3\&context=storylines_faq}{makeshift
    child care} and
    \href{https://www.nytimes3xbfgragh.onion/2020/06/03/business/economy/coronavirus-working-women.html?action=click\&pgtype=Article\&state=default\&region=MAIN_CONTENT_3\&context=storylines_faq}{stunted
    workdays} to continue. California's two largest public school
    districts --- Los Angeles and San Diego --- said on July 13, that
    \href{https://www.nytimes3xbfgragh.onion/2020/07/13/us/lausd-san-diego-school-reopening.html?action=click\&pgtype=Article\&state=default\&region=MAIN_CONTENT_3\&context=storylines_faq}{instruction
    will be remote-only in the fall}, citing concerns that surging
    coronavirus infections in their areas pose too dire a risk for
    students and teachers. Together, the two districts enroll some
    825,000 students. They are the largest in the country so far to
    abandon plans for even a partial physical return to classrooms when
    they reopen in August. For other districts, the solution won't be an
    all-or-nothing approach.
    \href{https://bioethics.jhu.edu/research-and-outreach/projects/eschool-initiative/school-policy-tracker/}{Many
    systems}, including the nation's largest, New York City, are
    devising
    \href{https://www.nytimes3xbfgragh.onion/2020/06/26/us/coronavirus-schools-reopen-fall.html?action=click\&pgtype=Article\&state=default\&region=MAIN_CONTENT_3\&context=storylines_faq}{hybrid
    plans} that involve spending some days in classrooms and other days
    online. There's no national policy on this yet, so check with your
    municipal school system regularly to see what is happening in your
    community.
  \end{itemize}
\item ~
  \hypertarget{is-the-coronavirus-airborne}{%
  \paragraph{Is the coronavirus
  airborne?}\label{is-the-coronavirus-airborne}}

  \begin{itemize}
  \tightlist
  \item
    The coronavirus
    \href{https://www.nytimes3xbfgragh.onion/2020/07/04/health/239-experts-with-one-big-claim-the-coronavirus-is-airborne.html?action=click\&pgtype=Article\&state=default\&region=MAIN_CONTENT_3\&context=storylines_faq}{can
    stay aloft for hours in tiny droplets in stagnant air}, infecting
    people as they inhale, mounting scientific evidence suggests. This
    risk is highest in crowded indoor spaces with poor ventilation, and
    may help explain super-spreading events reported in meatpacking
    plants, churches and restaurants.
    \href{https://www.nytimes3xbfgragh.onion/2020/07/06/health/coronavirus-airborne-aerosols.html?action=click\&pgtype=Article\&state=default\&region=MAIN_CONTENT_3\&context=storylines_faq}{It's
    unclear how often the virus is spread} via these tiny droplets, or
    aerosols, compared with larger droplets that are expelled when a
    sick person coughs or sneezes, or transmitted through contact with
    contaminated surfaces, said Linsey Marr, an aerosol expert at
    Virginia Tech. Aerosols are released even when a person without
    symptoms exhales, talks or sings, according to Dr. Marr and more
    than 200 other experts, who
    \href{https://academic.oup.com/cid/article/doi/10.1093/cid/ciaa939/5867798}{have
    outlined the evidence in an open letter to the World Health
    Organization}.
  \end{itemize}
\item ~
  \hypertarget{what-are-the-symptoms-of-coronavirus}{%
  \paragraph{What are the symptoms of
  coronavirus?}\label{what-are-the-symptoms-of-coronavirus}}

  \begin{itemize}
  \tightlist
  \item
    Common symptoms
    \href{https://www.nytimes3xbfgragh.onion/article/symptoms-coronavirus.html?action=click\&pgtype=Article\&state=default\&region=MAIN_CONTENT_3\&context=storylines_faq}{include
    fever, a dry cough, fatigue and difficulty breathing or shortness of
    breath.} Some of these symptoms overlap with those of the flu,
    making detection difficult, but runny noses and stuffy sinuses are
    less common.
    \href{https://www.nytimes3xbfgragh.onion/2020/04/27/health/coronavirus-symptoms-cdc.html?action=click\&pgtype=Article\&state=default\&region=MAIN_CONTENT_3\&context=storylines_faq}{The
    C.D.C. has also} added chills, muscle pain, sore throat, headache
    and a new loss of the sense of taste or smell as symptoms to look
    out for. Most people fall ill five to seven days after exposure, but
    symptoms may appear in as few as two days or as many as 14 days.
  \end{itemize}
\item ~
  \hypertarget{does-asymptomatic-transmission-of-covid-19-happen}{%
  \paragraph{Does asymptomatic transmission of Covid-19
  happen?}\label{does-asymptomatic-transmission-of-covid-19-happen}}

  \begin{itemize}
  \tightlist
  \item
    So far, the evidence seems to show it does. A widely cited
    \href{https://www.nature.com/articles/s41591-020-0869-5}{paper}
    published in April suggests that people are most infectious about
    two days before the onset of coronavirus symptoms and estimated that
    44 percent of new infections were a result of transmission from
    people who were not yet showing symptoms. Recently, a top expert at
    the World Health Organization stated that transmission of the
    coronavirus by people who did not have symptoms was ``very rare,''
    \href{https://www.nytimes3xbfgragh.onion/2020/06/09/world/coronavirus-updates.html?action=click\&pgtype=Article\&state=default\&region=MAIN_CONTENT_3\&context=storylines_faq\#link-1f302e21}{but
    she later walked back that statement.}
  \end{itemize}
\end{itemize}

``I know I am taking a risk by being onstage with people,'' Ms. Crebassa
said. ``But I really believe it's important for everyone to start
finding solutions. It is nice to watch streaming concerts, but after a
while I got a bit depressed. This is not what we are supposed to do.''

And streaming, she pointed out, doesn't pay the bills. Free streams, she
said, were ``a beautiful gesture and a gift for the audience, but I
don't believe it can last. I am in a privileged position, but if I
accept being paid less because of the virus, or free streaming, what
does that do for all artists?''

For Joana Mallwitz, a rising conductor who is music director of the
opera house in Nuremberg, rehearsing Mozart was a welcome change from
months of coronavirus contingency planning.

``Here, at the moment,'' she said, ``the joy outweighs everything
else.''

Not that bitter ``Così'' is always a joyful story. Fiordiligi and
Dorabella are sisters whose boyfriends pretend to go off to war;
instead, they return in disguise to try and seduce each other's girl.
The women are on the verge of marrying the ``wrong'' men before the game
is revealed.

Christof Loy, who has directed often at Salzburg and was originally
scheduled for ``Boris Godunov'' this summer, said ``Così'' is closer to
him than almost any other opera. Pointing to the subtitle --- ``The
School for Lovers'' --- he said in an interview that ``there is a very
long history of misunderstanding this piece.'' Many artists in the 19th
and 20th centuries, he added, ``treated it like a stupid comedy. It did
not give justice to the deep humanity of the piece; it portrayed the
women as too stupid and too cheap.''

Image

The production takes place on a simple white set in contemporary dress.
From left: Ms. Desandre, Mr. Schuen, Ms. Dreisig, Mr. Volkov, Ms.
Crebassa and Johannes Martin Kränzle.Credit...Monika
Rittershaus/Salzburger Festspiele

Mr. Loy translates the characters' thoughts and backgrounds into their
distance from one another onstage. ``The piece is about how distant you
are,'' he said. ``Very far or very close. Not so much in between.'' As
is typical of his often visually minimalist work, the production takes
place on a simple white set --- though one that can open to offer a
single view of a midsummer evening --- and in contemporary dress.

Mr. Loy and Ms. Mallwitz began their collaboration in June with a series
of long phone calls. A new production at Salzburg would normally emerge
from years of planning in which conductor and director could discuss
their conceptions of the work. But the breakneck planning of this
staging --- not to mention the cuts to the score --- has not stinted
depth: While Fiordiligi is often presented as a prudish caricature, Ms.
Dreisig said that Mr. Loy and Ms. Mallwitz were committed to presenting
every character in his or her full humanity.

``It's a comedy,'' Ms. Dreisig said, ``but everything has to be done
with an ethic of respecting the character you are singing, and
respecting the music. It's a true collaboration.''

One recent evening, at the first rehearsal of the work's finale, Ms.
Mallwitz led the cast through the score. Then the music stands were put
away, and Mr. Loy began to walk the performers through the staging. They
stopped and started as they discussed back stories and psychological
motivations. Except for the masks being worn by dramaturgs and
assistants, the pandemic seemed to have receded from the room.

The lovers circled each other uneasily, singing, ``Repeat the joyful
music, renew the lovely song, and we will sit here in the greatest
joy.'' Mr. Hinterhäuser, watching from the side of the room, looked,
above his mask, as though he might cry.

``It's maybe the fastest-ever-realized opera production in the world,''
he had said the previous day. ``Vital, fresh, unexpected. At the end,
with all the risks and all the thin ice that we are on, it is important
to make this try.''

Matthew Anderson contributed reporting.

Advertisement

\protect\hyperlink{after-bottom}{Continue reading the main story}

\hypertarget{site-index}{%
\subsection{Site Index}\label{site-index}}

\hypertarget{site-information-navigation}{%
\subsection{Site Information
Navigation}\label{site-information-navigation}}

\begin{itemize}
\tightlist
\item
  \href{https://help.nytimes3xbfgragh.onion/hc/en-us/articles/115014792127-Copyright-notice}{©~2020~The
  New York Times Company}
\end{itemize}

\begin{itemize}
\tightlist
\item
  \href{https://www.nytco.com/}{NYTCo}
\item
  \href{https://help.nytimes3xbfgragh.onion/hc/en-us/articles/115015385887-Contact-Us}{Contact
  Us}
\item
  \href{https://www.nytco.com/careers/}{Work with us}
\item
  \href{https://nytmediakit.com/}{Advertise}
\item
  \href{http://www.tbrandstudio.com/}{T Brand Studio}
\item
  \href{https://www.nytimes3xbfgragh.onion/privacy/cookie-policy\#how-do-i-manage-trackers}{Your
  Ad Choices}
\item
  \href{https://www.nytimes3xbfgragh.onion/privacy}{Privacy}
\item
  \href{https://help.nytimes3xbfgragh.onion/hc/en-us/articles/115014893428-Terms-of-service}{Terms
  of Service}
\item
  \href{https://help.nytimes3xbfgragh.onion/hc/en-us/articles/115014893968-Terms-of-sale}{Terms
  of Sale}
\item
  \href{https://spiderbites.nytimes3xbfgragh.onion}{Site Map}
\item
  \href{https://help.nytimes3xbfgragh.onion/hc/en-us}{Help}
\item
  \href{https://www.nytimes3xbfgragh.onion/subscription?campaignId=37WXW}{Subscriptions}
\end{itemize}
