Sections

SEARCH

\protect\hyperlink{site-content}{Skip to
content}\protect\hyperlink{site-index}{Skip to site index}

\href{https://www.nytimes3xbfgragh.onion/section/books/review}{Book
Review}

\href{https://myaccount.nytimes3xbfgragh.onion/auth/login?response_type=cookie\&client_id=vi}{}

\href{https://www.nytimes3xbfgragh.onion/section/todayspaper}{Today's
Paper}

\href{/section/books/review}{Book Review}\textbar{}New Books Take You
Through the Microscope to the World of Pathogens

\url{https://nyti.ms/3hXt1sK}

\begin{itemize}
\item
\item
\item
\item
\item
\end{itemize}

Advertisement

\protect\hyperlink{after-top}{Continue reading the main story}

Supported by

\protect\hyperlink{after-sponsor}{Continue reading the main story}

Shortlist

\hypertarget{new-books-take-you-through-the-microscope-to-the-world-of-pathogens}{%
\section{New Books Take You Through the Microscope to the World of
Pathogens}\label{new-books-take-you-through-the-microscope-to-the-world-of-pathogens}}

\includegraphics{https://static01.graylady3jvrrxbe.onion/images/2020/08/02/books/review/02Shortlist-Microscope/02Shortlist-Microscope-articleLarge.jpg?quality=75\&auto=webp\&disable=upscale}

By Rob Dunn

\begin{itemize}
\item
  July 31, 2020
\item
  \begin{itemize}
  \item
  \item
  \item
  \item
  \item
  \end{itemize}
\end{itemize}

\textbf{BIOGRAPHY OF RESISTANCE}\\
\textbf{The Epic Battle Between People and Pathogens}\\
By Muhammad H. Zaman\\
304 pp. Harper Wave. \$28.99.

\includegraphics{https://static01.graylady3jvrrxbe.onion/images/2020/05/27/books/review/Shortlist_Dunn3/Shortlist_Dunn3-articleLarge.jpg?quality=75\&auto=webp\&disable=upscale}

In ``Biography of Resistance,'' Zaman considers antibiotics as the major
weapon for killing bacterial pathogens --- and the ways in which this
war has backfired. Bacteria and fungi have been producing antibiotics
for many hundreds of millions of years. Relatively recently, humans
figured out that they could co-opt some of these to control pathogenic
bacteria. Doing so saved millions of lives. But it also increased the
commonness of resistant bacteria, that is, bacteria that are impervious
to antibiotics and, as a result, difficult to kill. ``Biography of
Resistance'' profiles these bacteria, but also the people who study
them. It is a useful, engaging opus.

There are now resistant strains of the bacteria that cause tuberculosis
and other disease, but also resistant malaria protists, bedbugs, head
lice, crop pests and even garden weeds. Zaman tells the stories of
researchers working to understand the evolution, and to a lesser extent
ecology, of bacterial resistance and when and why it emerges. Resistance
is ancient (resistant bacteria can be found deep in caves beyond the
reach of human influence), but it has taken on new forms and dynamics in
light of the ways in which we have wielded antibiotics.

Zaman's book includes histories of key moments in microbiology,
reminders of how fast our perspectives on the microscopic world have
evolved. When Anton van Leeuwenhoek first discovered microbial life in
the 1600s, he imagined it to be wondrous and mostly beneficial. Once
Louis Pasteur discovered that microbes could both make us sick and make
beer, he came to see some species as dangerous but others as beneficial.
Then once we developed antibiotics, scientists began to talk more often
about a ``war on germs,'' in which germs were understood to be faceless,
dangerous creatures all around us. With these increasingly resistant
strains, the germs now seem to be taking this war seriously.

\textbf{CLEAN}\\
\textbf{The New Science of Skin}\\
By James Hamblin\\
278 pp. Riverhead. \$28.

Image

Zaman considers the impact of microbes globally. Meanwhile, Hamblin's
new book, ``Clean,'' is an ode to the invisible world laid out between
his toes and in his armpits. Hamblin focuses on the skin, including that
of his own body. Just as Pasteur and others revealed that some
microscopic species could be dangerous, and long before antibiotics were
discovered in the early 20th century, it became clear that lives could
be saved through simple interventions that helped reduce the abundance
of those dangerous species.

Hand washing has saved hundreds of millions of lives, as has the
availability of drinking water that is free of pathogens (conversely,
the lack of access to such drinking water endangers millions still
today). The goal of these interventions is not to make hands sterile
(one can't) and to make water sterile (almost none is), but instead to
control problem pathogens. But the cosmetics industry and other
purveyors of solutions and creams came to recognize, as Hamblin
documents, an opportunity to sell products and lifestyles that not only
removed all germs but, just to be on the safe side, offered total and
complete cleanliness as a goal. This, Hamblin concludes, actually just
made us rashy and sick in new kinds of ways. And so begins the odyssey
upon which Hamblin embarked.

While trying to understand his own skin, Hamblin stopped bathing, though
he did still wash his hands and drink clean water. He does not use his
personal experiment as evidence, so much as a way to drive the
narrative. The writing is fun, interesting and credible, that of a
science journalist trying to make sense of the biology of bodies and how
they work in daily life. If Zaman's book is about war, Hamblin's is more
about finding ways to make peace, not with pathogens but instead with
our own bodies and the majority of species on and in them, species on
which, he comes to see, we depend for survival and well-being. That
peace can include ``products,'' just more carefully chosen ones.
Tellingly, the cover of Hamblin's book bears a fancy soap dispenser that
appears to be dispensing dirt.

\textbf{THE SENSITIVES}\\
\textbf{The Rise of Environmental Illness and the Search for America's
Last Pure Place}\\
By Oliver Broudy\\
339 pp. Simon \& Schuster. \$27.

Image

Just as some biological environments are more healthful than others, so
too are some chemical environments better for us than others.
Understanding just which combinations of chemicals are good and which
are bad and at what concentrations is challenging. Fortunately, many
thousands of toxicologists, sociologists, epidemiologists and biologists
have made it their lives' work to try to get to some answers. And,
unfortunately, many people don't trust them.

In ``The Sensitives,'' Broudy zeros in on a group of people who have
decided, on their own, that all of the chemistry associated with modern,
industrial life is making them sick. These individuals are ``sensitive''
to everything from plastic to perfume, which they have deemed toxic, and
have abandoned science and the medical system to find places where they
feel well --- something akin to the paleodiet but for chemistry.

``The Sensitives'' is at its best when Broudy is chronicling the very
real challenges of his subjects. Sensitives are united by the belief
that they are suffering from ``Environmental Illness.'' E.I. is not
recognized as a disease by any major medical organization. Those who
self-diagnose with the disease suffer a grab bag of debilitating
symptoms and are united by struggles to find clinicians able to help
them. Broudy's book is moved along by a kind of medical travel narrative
as Broudy searches with one sensitive, James, for another, Brian, who
has found a haven from what he believes to be the toxins of the world.
Broudy's writing inspires real empathy for the individuals he
chronicles, individuals who can't seem to get well or get help.

Where the book fails is in its implied conclusions. Broudy leaves the
reader feeling as though in dealing with E.I. or any set of mysterious
symptoms, science and self-diagnosis are just the same. At a moment when
our collective well-being depends upon the public's trust in experts, in
their knowledge about pathogens and the civilization-saving value of
vaccines, this is a very dangerous sentiment.

Advertisement

\protect\hyperlink{after-bottom}{Continue reading the main story}

\hypertarget{site-index}{%
\subsection{Site Index}\label{site-index}}

\hypertarget{site-information-navigation}{%
\subsection{Site Information
Navigation}\label{site-information-navigation}}

\begin{itemize}
\tightlist
\item
  \href{https://help.nytimes3xbfgragh.onion/hc/en-us/articles/115014792127-Copyright-notice}{©~2020~The
  New York Times Company}
\end{itemize}

\begin{itemize}
\tightlist
\item
  \href{https://www.nytco.com/}{NYTCo}
\item
  \href{https://help.nytimes3xbfgragh.onion/hc/en-us/articles/115015385887-Contact-Us}{Contact
  Us}
\item
  \href{https://www.nytco.com/careers/}{Work with us}
\item
  \href{https://nytmediakit.com/}{Advertise}
\item
  \href{http://www.tbrandstudio.com/}{T Brand Studio}
\item
  \href{https://www.nytimes3xbfgragh.onion/privacy/cookie-policy\#how-do-i-manage-trackers}{Your
  Ad Choices}
\item
  \href{https://www.nytimes3xbfgragh.onion/privacy}{Privacy}
\item
  \href{https://help.nytimes3xbfgragh.onion/hc/en-us/articles/115014893428-Terms-of-service}{Terms
  of Service}
\item
  \href{https://help.nytimes3xbfgragh.onion/hc/en-us/articles/115014893968-Terms-of-sale}{Terms
  of Sale}
\item
  \href{https://spiderbites.nytimes3xbfgragh.onion}{Site Map}
\item
  \href{https://help.nytimes3xbfgragh.onion/hc/en-us}{Help}
\item
  \href{https://www.nytimes3xbfgragh.onion/subscription?campaignId=37WXW}{Subscriptions}
\end{itemize}
