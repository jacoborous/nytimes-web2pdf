Sections

SEARCH

\protect\hyperlink{site-content}{Skip to
content}\protect\hyperlink{site-index}{Skip to site index}

\href{https://myaccount.nytimes3xbfgragh.onion/auth/login?response_type=cookie\&client_id=vi}{}

\href{https://www.nytimes3xbfgragh.onion/section/todayspaper}{Today's
Paper}

In a 19th-Century Cart House, a Designer Creates a Room of Her Own

\url{https://nyti.ms/3fde181}

\begin{itemize}
\item
\item
\item
\item
\item
\end{itemize}

Advertisement

\protect\hyperlink{after-top}{Continue reading the main story}

Supported by

\protect\hyperlink{after-sponsor}{Continue reading the main story}

\hypertarget{in-a-19th-century-cart-house-a-designer-creates-a-room-of-her-own}{%
\section{In a 19th-Century Cart House, a Designer Creates a Room of Her
Own}\label{in-a-19th-century-cart-house-a-designer-creates-a-room-of-her-own}}

Harriet Anstruther has transformed a former outbuilding on her farm in
the British countryside into a maximalist retreat.

\includegraphics{https://static01.graylady3jvrrxbe.onion/images/2020/07/16/t-magazine/16tmag-bothy-slide-P1ID/16tmag-bothy-slide-P1ID-articleLarge.jpg?quality=75\&auto=webp\&disable=upscale}

By \href{https://www.nytimes3xbfgragh.onion/by/aimee-farrell}{Aimee
Farrell}

\begin{itemize}
\item
  July 31, 2020
\item
  \begin{itemize}
  \item
  \item
  \item
  \item
  \item
  \end{itemize}
\end{itemize}

When the British designer
\href{http://www.harrietanstruther.com/}{Harriet Anstruther} first moved
into the 500-year-old wattle-and-daub West Sussex farmhouse she shares
with her husband, the photographer
\href{http://www.henrybourne.com/}{Henry Bourne}, in 1998, there wasn't
much of a garden. ``It was a nest of brambles and snakes,'' she recalls
of the square half-acre walled plot that separates the timber-frame home
from the property's roughly 100 acres of pasture, woodland and
wildflower meadows. ``It was cold and wet,'' she says, ``but
unbelievably idyllic.''

Today, the garden's appeal is easier to see. Over the past two decades,
Anstruther has lovingly transformed this onetime thicket of weeds into a
sanctuary of richly layered, ever-evolving plantings. At the rear of the
house, in an area that was once a dumping ground for old mattresses and
farm machinery, she has established a bountiful kitchen garden, in
raised beds made from railway sleepers, that yields harvests of carrots,
lettuces, peas, potatoes, fennel, beans, herbs and berries (the nearby
woods are also a generous source of wild damsons and elderflowers).
Pushing up against the knapped **** flint walls of the south-facing
front garden, there are fruit trees bearing greengages, damsons,
mulberries, apples and pears. And Bourne has gradually coaxed a lawn
into existence, which now flanks the home at the front and back. But
perhaps most glorious of all are the roses. Ascending the garden's gates
and a pyramid-topped iron gazebo, and rambling amid deep borders of
foxgloves and clematis, are more than 100 rose bushes of some 50
varieties, their fragrant abundance typifying the cottage-garden charm
that Anstruther has nurtured here. ``It's a little oasis,'' she says,
``Just like an interior, a garden can encapsulate your small corner of
the world.''

Image

In the study, deep blue walls offset antique wood furniture and Persian
rugs.Credit...Henry Bourne

Image

Built-in bookshelves house a collection of literature by Vita
Sackville-West, Rex Whistler and Virginia Woolf passed down from
Anstruther's great-aunt.Credit...Henry Bourne

But if the home's green spaces and main house are a shared retreat for
the couple **** --- Anstruther and Bourne, who previously spent most of
the year in London, moved here full-time last summer --- the designer's
latest project is a hideaway all her own. From the back of the property,
a gravel path meanders past a towering walnut tree, as old as the house,
before arriving at a low wrought-iron gate. Beyond it, covered partly in
wisteria and climbing roses, is a compact red brick two-story
outbuilding, its former barn door, now filled in with blackened
weatherboards and inset with a casement window, creating an arched
backdrop for a small terrace. The box-shaped, roughly 450-square-foot
structure was once a cart house used for storing agricultural machinery
and as lodging for a farmworker, who would have entered the upper floor
via an external staircase. ``But we've always called it a
\href{https://www.nytimes3xbfgragh.onion/2019/01/21/travel/in-search-of-britains-bothies.html}{bothy},''
Anstruther explains, employing the Scottish term typically associated
with the small, shared shelters found in the remote outer reaches of the
Highlands but also used to denote outlying buildings on English estates.
Simple but cozy, its interior recalls the cocooning comfort of her
childhood home nearby in Sussex.

Image

In the guest bedroom, a Victorian bed frame from a Sussex antiques fair
contrasts with contemporary artworks, including a porcelain sculpture of
a blouse by the ceramist Kaori Tatebayashi.Credit...Henry Bourne

Image

In the bathroom, Anstruther added a blue wainscot but kept the walls
white. ``Things can look grubby in a bathroom if it's too dark,'' she
says.Credit...Henry Bourne

Thought to date to the 1830s (it appears in a 19th-century watercolor
painting of the property that a friend spotted in a local junk shop),
the cottage had until recently served as a rental property. Not long
after she first arrived at the farm, Anstruther modernized the interior
and, crucially, added an internal staircase. But after its long-term
tenants moved out last fall, she decided to hold onto the space and
embark on a second renovation --- this time with her own preferences in
mind. She installed a modern kitchen with slate-topped counters in the
lean-to downstairs and a white ceramic, claw-footed roll-top tub in the
modestly sized top-floor bathroom. And while there is now a snug bedroom
with an oak-beam ceiling upstairs for guests, almost the entire ground
floor (the kitchen sits at the western end) **** serves as a study for
Anstruther, one where she can focus on her writing and research
surrounded by the objects that inspire her. ``I've always wanted a room
of my own,'' she says. ``It's the first time I've had such a private
place.''

Image

Foxgloves and roses proliferate on the garden's eastern
wall.Credit...Henry Bourne

Image

A path cuts from the back garden and through a grazing meadow into the
woodland that surrounds the farmhouse.Credit...Henry Bourne

Though Anstruther's namesake design practice is known for its
meticulously considered and elegantly restrained interiors, in the cart
house, she has indulged her quirkier side. The simple wooden floors are
painted matte black and covered in a patchwork of Persian rugs and each
wall and surface is arranged with personal artifacts, including
19th-century Scottish ceramics, paintings of horses and a taxidermied
barn owl encased in a glass dome. She began the process of redecorating
by first assembling in the cottage her most treasured and personal
pieces of furniture and art, such as a tasseled burgundy silk-damask
curtain discovered at a French brocante, a gently battered chaise longue
**** picked up at the nearby Petworth Antiques Market, and oil paintings
and antique walnut and fruitwood furniture passed down from her late
father --- that had been scattered between the couple's former London
home, their offices and the farmhouse. It's these cherished objects that
lend the interior its eclectic and intensely homey **** feel --- and
inspired its deeply saturated Prussian blue walls. ``I wanted something
all-encompassing that would reflect my Scottish heritage,'' Anstruther
says.

\includegraphics{https://static01.graylady3jvrrxbe.onion/images/2020/07/16/t-magazine/16tmag-bothy-slide-6R7J/16tmag-bothy-slide-6R7J-articleLarge.jpg?quality=75\&auto=webp\&disable=upscale}

Indeed, every corner of the bothy represents some part of her history.
In the bedroom, a wool blanket woven in her family's Campbell clan
tartan covers a high-backed grandfather chair topped with one of the
Victorian-style needlepoint cushions, depicting dogs and countryside
scenes, that Anstruther frequently stitches ``I have hundreds of them
--- largely thanks to the fact that you can't sew and smoke at the same
time,'' she jokes And at the center of the study, in front of a small
wood-burning cast-iron stove, is a restored circular, ebony-edged
fruitwood library table that once belonged to one of Anstruther's
ancestors, Princess Louise. The sixth child of Queen Victoria, the
princess was rumored to have had an affair with the queen's sculptor,
Joseph Edgar Boehm, and to have been discovered, on occasion, in
flagrante on this very table. ``When my father gave it to me, he told me
I mustn't mend the wobble,'' Anstruther says with a laugh. But most
often she sits at her father's writing bureau, an elegant, wall-facing
**** rosewood **** Napoleon III cylinder desk, dating to the 1860s, that
stands in the southeastern corner of the room. Hanging on the wall
beside it, next to a cluster of small oil and watercolor paintings, is a
pastel portrait of Anstruther's great-aunt, Joan Campbell, as a
flaxen-haired young girl. Campbell, who was friends with members of the
Bloomsbury Group at the turn of the 20th century, also bequeathed many
of her books, including early editions of works by Vita Sackville-West
and Virginia Woolf, to her great-niece. Today, they line a set of
bespoke wooden shelves tucked beneath the stairs on the neighboring
wall. It's a study, the designer reflects, that is the opposite of a
blank canvas. ``This one little room captures the **** past, the present
and the future,'' she says. ``It's full of things that mean something to
me. It's my castle.''

Image

A gravel path leads to the back of the cart house, whose original
second-floor barn doors now incorporate a bedroom window.Credit...Henry
Bourne

Image

Roses grow up the wall of the property's hay barn.Credit...Henry Bourne

Anstruther, who is currently working on the interior of a cafe and shop
for a major London museum, has long seen her role as a designer as
threading together the different layers of history and daily life that
must coexist within a home or building: **** In the bothy's bedroom,
above an ornate, Victorian gilt and wrought-iron bedstead, hangs what
looks like a child's blouse from a similar era but is, in fact, a
contemporary piece rendered in porcelain by the London-based ceramist
\href{http://www.kaoriceramics.com/}{Kaori Tatebayashi}. And downstairs
in the study, an 18th-century hat-maker's mannequin, made from
papier-mâché with a profile shaped like a Regency bonnet, sits on a
console table beneath an adolescent self-portrait sketched in pencil by
her now grown-up daughter, Celeste. ``The pleasure I get from just
looking at all these things is immeasurable,'' Anstruther says. ``It
sounds so sentimental, but it feels like the culmination of my life so
far. It's who I am.''

Advertisement

\protect\hyperlink{after-bottom}{Continue reading the main story}

\hypertarget{site-index}{%
\subsection{Site Index}\label{site-index}}

\hypertarget{site-information-navigation}{%
\subsection{Site Information
Navigation}\label{site-information-navigation}}

\begin{itemize}
\tightlist
\item
  \href{https://help.nytimes3xbfgragh.onion/hc/en-us/articles/115014792127-Copyright-notice}{©~2020~The
  New York Times Company}
\end{itemize}

\begin{itemize}
\tightlist
\item
  \href{https://www.nytco.com/}{NYTCo}
\item
  \href{https://help.nytimes3xbfgragh.onion/hc/en-us/articles/115015385887-Contact-Us}{Contact
  Us}
\item
  \href{https://www.nytco.com/careers/}{Work with us}
\item
  \href{https://nytmediakit.com/}{Advertise}
\item
  \href{http://www.tbrandstudio.com/}{T Brand Studio}
\item
  \href{https://www.nytimes3xbfgragh.onion/privacy/cookie-policy\#how-do-i-manage-trackers}{Your
  Ad Choices}
\item
  \href{https://www.nytimes3xbfgragh.onion/privacy}{Privacy}
\item
  \href{https://help.nytimes3xbfgragh.onion/hc/en-us/articles/115014893428-Terms-of-service}{Terms
  of Service}
\item
  \href{https://help.nytimes3xbfgragh.onion/hc/en-us/articles/115014893968-Terms-of-sale}{Terms
  of Sale}
\item
  \href{https://spiderbites.nytimes3xbfgragh.onion}{Site Map}
\item
  \href{https://help.nytimes3xbfgragh.onion/hc/en-us}{Help}
\item
  \href{https://www.nytimes3xbfgragh.onion/subscription?campaignId=37WXW}{Subscriptions}
\end{itemize}
