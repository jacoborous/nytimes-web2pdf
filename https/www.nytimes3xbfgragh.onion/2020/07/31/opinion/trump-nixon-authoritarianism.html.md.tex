Sections

SEARCH

\protect\hyperlink{site-content}{Skip to
content}\protect\hyperlink{site-index}{Skip to site index}

\href{https://myaccount.nytimes3xbfgragh.onion/auth/login?response_type=cookie\&client_id=vi}{}

\href{https://www.nytimes3xbfgragh.onion/section/todayspaper}{Today's
Paper}

\href{/section/opinion}{Opinion}\textbar{}Trump Has Been Comparing
Himself to Nixon. That's Hooey.

\url{https://nyti.ms/337u0SS}

\begin{itemize}
\item
\item
\item
\item
\item
\item
\end{itemize}

Advertisement

\protect\hyperlink{after-top}{Continue reading the main story}

\href{/section/opinion}{Opinion}

Supported by

\protect\hyperlink{after-sponsor}{Continue reading the main story}

\hypertarget{trump-has-been-comparing-himself-to-nixon-thats-hooey}{%
\section{Trump Has Been Comparing Himself to Nixon. That's
Hooey.}\label{trump-has-been-comparing-himself-to-nixon-thats-hooey}}

The former president could only dream of wielding the police powers Mr.
Trump has seized for himself.

By John W. Dean

Mr. Dean was White House counsel under Richard Nixon.

\begin{itemize}
\item
  July 31, 2020
\item
  \begin{itemize}
  \item
  \item
  \item
  \item
  \item
  \item
  \end{itemize}
\end{itemize}

\includegraphics{https://static01.graylady3jvrrxbe.onion/images/2020/07/30/opinion/30JohnDean2/30JohnDean2-articleLarge.jpg?quality=75\&auto=webp\&disable=upscale}

President Trump has been comparing himself to Richard Nixon, tweeting
\href{https://twitter.com/realDonaldTrump/status/1267227396341669889}{``LAW
\& ORDER,''} and claiming he
\href{https://www.youtube.com/watch?v=mY12Llo847I}{learned a lot from
Nixon}. Others have been comparing
\href{https://www.nytimes3xbfgragh.onion/2020/06/02/opinion/floyd-trump-nixon-coronavirus.html}{Mr.
Trump's handling of civil disorder to Nixon's.} No one will ever tag me
a Nixon apologist, but in Nixon's defense these claims are hooey.

I worked for our last authoritarian president, Richard Nixon --- a man
who experienced violent protests and demonstrations throughout his
political career. In 1968, he ran as the ``law and order'' candidate,
for it was
\href{https://www.theguardian.com/us-news/2020/jun/16/trump-nixon-1968-law-and-order-america}{a
time of tumult}: assassinations of the Rev. Dr. Martin Luther King Jr.
and Senator Robert Kennedy. Riots ripped Baltimore, Boston, Chicago, Los
Angeles, Memphis, Washington and other major cities. Civil rights and
antiwar protests closed down campuses large and small. There were
nightly news reports of endless death from the killing fields of
Vietnam, including the Tet offensive and the My Lai massacre.

Nixon was running on credentials established long before the 1968
presidential contest. As vice president, Nixon and his wife
\href{https://history.state.gov/historicaldocuments/frus1958-60v05/comp4}{traveled
though South America}, where they frequently were confronted by
protesters. Nixon used those protest situations to brandish his
I-am-fearless image by walking among the protesters to make clear that
he was not intimidated, nor would they influence American policy.

On becoming president in 1969, Nixon inherited a global anti-Vietnam War
protest movement that had contributed to the decision of his
predecessor, Lyndon Johnson, not to seek re-election.

From his first day in office, Nixon faced
\href{https://learning.blogs.nytimes3xbfgragh.onion/2011/11/15/nov-15-1969-anti-vietnam-war-demonstration-held/}{huge
demonstrations,} which he instructed his White House counsel to monitor
closely. When I was appointed to that post 18 months into his
presidency, I discovered that all of the key intelligence agencies
reported domestic and related foreign intelligence about disruptive
protests, demonstrations and civil unrest occurring throughout the
country to the counsel's office, where we digested and shared it with
the president and senior staff.

For some thousand days I had an exceptional overview of what was being
done by Nixon and his aides to deal with often violent unrest,
particularly that provoked by those strongly opposed to the war in
Vietnam. Nixon's behavior was
\href{https://www.nytimes3xbfgragh.onion/2020/06/04/opinion/trump-nixon.html}{vastly
different from Mr. Trump's}.

\includegraphics{https://static01.graylady3jvrrxbe.onion/images/2020/07/30/opinion/30JohnDean1/30JohnDean1-articleLarge.jpg?quality=75\&auto=webp\&disable=upscale}

Never once did I hear anyone in the Nixon White House or Justice
Department suggest using United States military forces, or any federal
officers outside the military, to quell civil unrest or disorder. Nor
have I found any evidence of such activity after the fact, when digging
through the historical record.

It is well known that on unique occasions presidents had used federal
forces for limited purposes before Nixon, as in 1877 when President
Rutherford B. Hayes used federal troops to end
\href{http://ohiohistorycentral.org/w/Great_Railroad_Strike_of_1877}{the
railroad strike}; and in 1894 when President Grover Cleveland dispatched
troops to end the
\href{http://www.encyclopedia.chicagohistory.org/pages/1029.html}{Pullman
railroad strike}.

Presidents have also sent federal forces to uphold court orders, as in
1957 when President Eisenhower sent troops to
\href{http://crdl.usg.edu/export/html/dde/ddetimeline/crdl_dde_ddetimeline_130.html}{Little
Rock}, Ark., and in 1962 when President Kennedy sent federal forces to
\href{http://crdl.usg.edu/events/ole_miss_integration/?Welcome}{Oxford,
Miss.}, in both cases to enforce court orders to desegregate schools.

Mr. Trump, assisted by Attorney General Bill Barr, has assembled
\href{https://www.nytimes3xbfgragh.onion/2020/07/20/us/politics/trump-chicago-portland-federal-agents.html}{a
mongrel federal law enforcement} operation from the F.B.I., the
Department of Homeland Security, the Drug Enforcement Administration,
U.S. Marshals, and other federal agencies to proceed to cities
throughout the country: Portland, Ore.; Chicago; Memphis; Oakland,
Calif.

Neither governors nor mayors have requested these attack weapon-wielding
federal soldiers, a few hundred men with minimal identification dressed
up in battlefield camouflage. This unprecedented action is way beyond
Nixon's authoritarianism. And it raises serious questions.

Most conspicuously, for Donald Trump it creates optics he believes he
can
\href{https://www.nytimes3xbfgragh.onion/2020/07/21/us/politics/trump-portland-federal-agents.html}{exploit
in his re-election campaign.} Indeed, Nixon successfully used images of
disorder in 1968, and falsely charged demonstrators in 1972 as working
for his opponent when he was running for re-election. But Mr. Trump is
provoking disorder by using federal forces, which is quite different.

The reason Attorney General Barr is backing this action is that he
believes the president should, in fact, be able to do most anything he
wishes, whenever he wants. Mr. Barr is using 200 federal officers here
and there today, so tomorrow he can dispatch 2,000 or 20,000. He is
making the unprecedented precedented.

Richard Nixon closeted his authoritarianism behind closed doors, and
only because \href{http://nixontapes.org/}{he taped himself}do we have a
good understanding of it. Donald Trump, however, has paraded his
authoritarianism in the Rose Garden and at rallies. He wants to be seen
as a demagogue.

Nixon did not have an authoritarian Republican Party to support his
imperial presidency and was forced to prematurely resign. Mr. Trump has
a G.O.P. that seeks to expand his authoritarian presidency. Militarizing
federal forces to perform state and local police functions is merely
another norm-shattering example.

Mr. Trump's latest threat is that he will not leave the presidency if he
loses. He is making Nixon's authoritarian behavior look tame.

\emph{The Times is committed to publishing}
\href{https://www.nytimes3xbfgragh.onion/2019/01/31/opinion/letters/letters-to-editor-new-york-times-women.html}{\emph{a
diversity of letters}} \emph{to the editor. We'd like to hear what you
think about this or any of our articles. Here are some}
\href{https://help.nytimes3xbfgragh.onion/hc/en-us/articles/115014925288-How-to-submit-a-letter-to-the-editor}{\emph{tips}}\emph{.
And here's our email:}
\href{mailto:letters@NYTimes.com}{\emph{letters@NYTimes.com}}\emph{.}

\emph{Follow The New York Times Opinion section on}
\href{https://www.facebookcorewwwi.onion/nytopinion}{\emph{Facebook}}\emph{,}
\href{http://twitter.com/NYTOpinion}{\emph{Twitter (@NYTopinion)}}
\emph{and}
\href{https://www.instagram.com/nytopinion/}{\emph{Instagram}}\emph{.}

John W. Dean was White House counsel under President Nixon, and author
of the forthcoming book ``Authoritarian Nightmare: Trump and His
Followers.''

Advertisement

\protect\hyperlink{after-bottom}{Continue reading the main story}

\hypertarget{site-index}{%
\subsection{Site Index}\label{site-index}}

\hypertarget{site-information-navigation}{%
\subsection{Site Information
Navigation}\label{site-information-navigation}}

\begin{itemize}
\tightlist
\item
  \href{https://help.nytimes3xbfgragh.onion/hc/en-us/articles/115014792127-Copyright-notice}{©~2020~The
  New York Times Company}
\end{itemize}

\begin{itemize}
\tightlist
\item
  \href{https://www.nytco.com/}{NYTCo}
\item
  \href{https://help.nytimes3xbfgragh.onion/hc/en-us/articles/115015385887-Contact-Us}{Contact
  Us}
\item
  \href{https://www.nytco.com/careers/}{Work with us}
\item
  \href{https://nytmediakit.com/}{Advertise}
\item
  \href{http://www.tbrandstudio.com/}{T Brand Studio}
\item
  \href{https://www.nytimes3xbfgragh.onion/privacy/cookie-policy\#how-do-i-manage-trackers}{Your
  Ad Choices}
\item
  \href{https://www.nytimes3xbfgragh.onion/privacy}{Privacy}
\item
  \href{https://help.nytimes3xbfgragh.onion/hc/en-us/articles/115014893428-Terms-of-service}{Terms
  of Service}
\item
  \href{https://help.nytimes3xbfgragh.onion/hc/en-us/articles/115014893968-Terms-of-sale}{Terms
  of Sale}
\item
  \href{https://spiderbites.nytimes3xbfgragh.onion}{Site Map}
\item
  \href{https://help.nytimes3xbfgragh.onion/hc/en-us}{Help}
\item
  \href{https://www.nytimes3xbfgragh.onion/subscription?campaignId=37WXW}{Subscriptions}
\end{itemize}
