Sections

SEARCH

\protect\hyperlink{site-content}{Skip to
content}\protect\hyperlink{site-index}{Skip to site index}

\href{https://www.nytimes3xbfgragh.onion/section/sports/baseball}{Baseball}

\href{https://myaccount.nytimes3xbfgragh.onion/auth/login?response_type=cookie\&client_id=vi}{}

\href{https://www.nytimes3xbfgragh.onion/section/todayspaper}{Today's
Paper}

\href{/section/sports/baseball}{Baseball}\textbar{}Baseball in Empty
Stadiums Is Weird. How Will It Affect Outcomes?

\url{https://nyti.ms/3fnVu9r}

\begin{itemize}
\item
\item
\item
\item
\item
\end{itemize}

Advertisement

\protect\hyperlink{after-top}{Continue reading the main story}

Supported by

\protect\hyperlink{after-sponsor}{Continue reading the main story}

\hypertarget{baseball-in-empty-stadiums-is-weird-how-will-it-affect-outcomes}{%
\section{Baseball in Empty Stadiums Is Weird. How Will It Affect
Outcomes?}\label{baseball-in-empty-stadiums-is-weird-how-will-it-affect-outcomes}}

Players, coaches and analysts of all stripes are watching this season's
games to see what effect --- if any --- the absence of fans has on the
games themselves.

\includegraphics{https://static01.graylady3jvrrxbe.onion/images/2020/08/02/sports/00mlb-empty-2/merlin_175098081_c81382ff-4bfe-42b6-bb09-61c8e7567704-articleLarge.jpg?quality=75\&auto=webp\&disable=upscale}

By Robert O'Connell

\begin{itemize}
\item
  July 31, 2020
\item
  \begin{itemize}
  \item
  \item
  \item
  \item
  \item
  \end{itemize}
\end{itemize}

The Nationals and the Yankees were tied, 2-2, in the eighth inning last
Sunday when Sean Doolittle came on in relief for Washington. In another
year, the moment would have been a boisterous one: some 35,000 fans
getting to their feet and amping up the volume, urging the defending
champions on in a tight spot against a marquee opponent.

But because it is 2020, the seats were empty, and the only sound came
tinny and distant from the stadium public-address system: An audio
engineer had adjusted the dial of piped-in crowd noise. Doolittle's
fastball was flat, his offspeed pitches fluttered, and he gave up a walk
and a pair of hits as the Yankees took a 3-2 lead they would not
relinquish.

``My execution and fastball location wasn't as crisp as I would've
liked,'' Doolittle
\href{https://www.nbcwashington.com/news/sports/nbcsports/nationals-bullpen-provides-that-old-familiar-feeling-for-a-day/2372928/}{told
reporters} afterward. ``What this season is going to come down to is
which team, with pitchers, can make the adjustments the quickest and get
into midseason form.''

One of the biggest adjustments for major leaguers during this 60-game
season will be playing in empty, cavernous stadiums, at least for the
time being. While baseball has attempted to fill the void with cardboard
fans, artificial noise and
\href{https://www.youtube.com/watch?v=q_FQcKH4xL4}{even virtual
``crowds'' on broadcasts}, there is no denying that games are being held
in an atmosphere that is far from normal.

But baseball is inseparable from curiosity even during a pandemic, so
players, coaches and analysts of all stripes find themselves wondering
what this year's unwelcome circumstances can reveal about the sport
itself: Are younger or older players more suited to the subdued
atmosphere? Is every team's home-field advantage equal? And how do live
fans really affect what happens on the field?

``I think it's going to affect things in weird ways that we can't even
fully anticipate right now,'' Russell Carleton, a psychologist and
analyst who has consulted with the Cleveland Indians and the Mets, said
of 2020's empty stadiums. ``And it's going to vary from guy to guy.''

A little over a week into the schedule, it's too early to draw any hard
conclusions about the on-field effects of the lack of fans. But while
acknowledging the fragile nature of this season --- underscored by
\href{https://www.nytimes3xbfgragh.onion/2020/07/27/sports/baseball/marlins-game-canceled.html}{an
early coronavirus outbreak on the Miami Marlins} --- many observers are
viewing it as a unique opportunity to test theories and examine new data
about the sport.

\includegraphics{https://static01.graylady3jvrrxbe.onion/images/2020/07/28/sports/00mlb-empty-1/merlin_174931617_5d4afd98-b9a7-483f-a3cd-0b8090749f70-articleLarge.jpg?quality=75\&auto=webp\&disable=upscale}

Central to baseball's mythology are those players capable of thriving on
the big moments, of harnessing the energy of a crowd, either friendly or
hostile. For some, this quality, as much as fastball velocity or bat
speed, distinguishes true superstars from the rest. ``That's a real
physical effect that could lead to that extra half-mile an hour that
gets it past the batter for strike three,'' Carleton said.

Pitchers agree.

``There are plenty of pitchers that leave the bullpen throwing 89 to 90
miles an hour, but their first pitch in front of the fans, in front of
the opponent, is 95,'' the former Cy Young winner Orel Hershiser said.

Reds starter Trevor Bauer, a full-bore adopter of pitching analytics,
sees proof of the phenomenon in himself. ``I know that when the crowd
gets going with runners on, my adrenaline gets going, and I tend to have
better stuff,'' he said.

But this season, pressure --- that amorphous but oft-cited concept ---
has taken on a new quality, and it remains to be seen whether players
used to the energy of thousands of fans can provide their own.

``It's like you have two of your senses that aren't coinciding with one
another,'' Angels third baseman Anthony Rendon
\href{https://www.espn.com/mlb/story/_/id/29436493/angels-anthony-rendon-wants-music-dumb-piped-fan-noise}{said}
of playing with fake crowd noise. ``It's like you're looking at pizza,
but you're smelling a hamburger.''

People around the game have their hunches about who might be most
affected by the changes.

Cliff Floyd, an MLB Network analyst and a former outfielder for a 1998
Marlins team that lost 108 games, said that certain lousy squads that
don't typically draw large crowds at home might not feel much of a
contrast in the new environment.

\hypertarget{the-games-resume}{%
\subsubsection{The Games Resume}\label{the-games-resume}}

\hypertarget{sports-and-the-virus}{%
\paragraph{Sports and the Virus}\label{sports-and-the-virus}}

Updated July 31, 2020

Here's what's happening as the world of sports slowly comes back to
life:

\begin{itemize}
\item
  \begin{itemize}
  \tightlist
  \item
    The
    \href{https://www.nytimes3xbfgragh.onion/2020/07/30/sports/basketball/clippers-lakers.html?action=click\&pgtype=Article\&state=default\&region=MAIN_CONTENT_2\&context=storylines_keepup}{N.B.A.
    returned}, and the Lakers held on to beat the Clippers in a
    thriller. Zion Williamson played in the first game of the night for
    the Pelicans.
  \item
    Players, coaches and analysts are watching this season's baseball
    games
    \href{https://www.nytimes3xbfgragh.onion/2020/07/31/sports/baseball/baseball-empty-stadiums-effects.html?action=click\&pgtype=Article\&state=default\&region=MAIN_CONTENT_2\&context=storylines_keepup}{to
    see what effect} the absence of fans has.
  \item
    With no summer tournaments to play in, top high school basketball
    stars are
    \href{https://www.nytimes3xbfgragh.onion/2020/07/30/sports/ncaabasketball/college-basketball-recruiting.html?action=click\&pgtype=Article\&state=default\&region=MAIN_CONTENT_2\&context=storylines_keepup}{committing
    to colleges earlier}. Villanova is one of the beneficiaries.
  \end{itemize}
\end{itemize}

``Players in New York, you're playing in front of 30,000 every night ---
it might be different,'' Floyd said.

Matt Quatraro, the bench coach for the Tampa Bay Rays, who were second
to last in overall attendance last season but who play in a division
with the high-drawing Yankees and Boston Red Sox, said empty ballparks
could offer something of a lift.

``For a team like ours where we used to go into Yankee Stadium or Fenway
with 45,000 raucous fans,'' he said, ``maybe that helps calm some guys,
when they're not going to have to deal with that.''

Image

A Citi Field worker looked for balls in the stands after batting
practice.Credit...Ben Solomon for The New York Times

Others predict a generational divide in how players will react to the
new environment, between veterans sharpened primarily by on-field
competition and younger players brought up in the more sterile,
data-driven settings of cutting-edge baseball facilities.

Brian Kaplan works as a pitching coordinator at Cressey Sports
Performance in Florida, which during baseball's shutdown was the site of
scrimmages featuring stars like Giancarlo Stanton, Max Scherzer and
Justin Verlander. Kaplan noticed that the pitchers who were used to
bigger stages tended to have trouble summoning their midseason velocity.

``We had maybe 30, 40 people there, and they were like, `This is the
biggest crowd we're going to throw in front of this season,''' Kaplan
said of the players' reactions.

Even with the piped-in noise, broadcasts of early games have featured
distinctly quieter ballparks, amplifying some of the sport's subtler
sounds --- the rush of a fastball, the one-two tick of a foul tip. That
offers an input rarely afforded big leaguers in game situations, which
may give the attentive player an advantage in the taut negotiation of an
at-bat.

Even a sound as slight as the spin of a breaking ball, or the tone of a
foul ball coming off the bat, could give perceptive players a valuable
data point over the course of an at-bat or a game, said the former
pitcher and current MLB Network analyst Ryan Dempster.

``You have a chance to add another sense to your scouting report,'' he
said.

How much those slivers of advantage and disadvantage affect things will
be hard to quantify; a season of (maybe) 60 games makes for a meager
sample size. But in baseball as in other sports, the analytics community
has seized on this year as a chance to study a phenomenon that
historically belongs more to feeling than to data: home-field advantage.

Across M.L.B.,
\href{https://fivethirtyeight.com/features/a-home-playoff-game-is-a-big-advantage-unless-you-play-hockey/}{54
percent of games} are won by the home team, an edge popularly traced to
umpire bias, crowd influence, the simple comforts of home or some
combination of those. Opportunities to eliminate variables have been
scarce, until now.

``We're about to find out what happens when you do take the crowd out of
it,'' said Jonathan Judge, an analyst for Baseball Prospectus. ``Does
the crowd really seem to matter very much, or is it just going home to
your family every night?''

Voros McCracken, a pioneering analytics expert who also consults for an
American League organization, said he suspected that home-field
advantage in M.L.B. would be reduced this season.

``The players are human beings, and you get more revved up when people
are cheering for you,'' he said.

He'll also be keeping an eye on the pitch-framing statistics from
M.L.B.'s Statcast program, which monitor how well catchers are able to
``steal'' strikes on pitches outside of the zone, as well as walk and
strikeout levels. ``If the home team is getting fewer benefits in the
strike zone,'' McCracken said, ``that tells us something.''

This new data set arrives unhappily, of course. The Marlins outbreak has
concentrated the anxiety lingering around the league, and analysts cut
their thoughts on in-game specifics with general hopes for safety.

``I hope all our fans will look at the mental side of this and how
challenging it is for our guys,'' Floyd, the former Marlins outfielder,
said. ``That should not be lost.''

Advertisement

\protect\hyperlink{after-bottom}{Continue reading the main story}

\hypertarget{site-index}{%
\subsection{Site Index}\label{site-index}}

\hypertarget{site-information-navigation}{%
\subsection{Site Information
Navigation}\label{site-information-navigation}}

\begin{itemize}
\tightlist
\item
  \href{https://help.nytimes3xbfgragh.onion/hc/en-us/articles/115014792127-Copyright-notice}{©~2020~The
  New York Times Company}
\end{itemize}

\begin{itemize}
\tightlist
\item
  \href{https://www.nytco.com/}{NYTCo}
\item
  \href{https://help.nytimes3xbfgragh.onion/hc/en-us/articles/115015385887-Contact-Us}{Contact
  Us}
\item
  \href{https://www.nytco.com/careers/}{Work with us}
\item
  \href{https://nytmediakit.com/}{Advertise}
\item
  \href{http://www.tbrandstudio.com/}{T Brand Studio}
\item
  \href{https://www.nytimes3xbfgragh.onion/privacy/cookie-policy\#how-do-i-manage-trackers}{Your
  Ad Choices}
\item
  \href{https://www.nytimes3xbfgragh.onion/privacy}{Privacy}
\item
  \href{https://help.nytimes3xbfgragh.onion/hc/en-us/articles/115014893428-Terms-of-service}{Terms
  of Service}
\item
  \href{https://help.nytimes3xbfgragh.onion/hc/en-us/articles/115014893968-Terms-of-sale}{Terms
  of Sale}
\item
  \href{https://spiderbites.nytimes3xbfgragh.onion}{Site Map}
\item
  \href{https://help.nytimes3xbfgragh.onion/hc/en-us}{Help}
\item
  \href{https://www.nytimes3xbfgragh.onion/subscription?campaignId=37WXW}{Subscriptions}
\end{itemize}
