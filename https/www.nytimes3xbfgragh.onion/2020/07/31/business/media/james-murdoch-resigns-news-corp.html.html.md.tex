Sections

SEARCH

\protect\hyperlink{site-content}{Skip to
content}\protect\hyperlink{site-index}{Skip to site index}

\href{https://www.nytimes3xbfgragh.onion/section/business/media}{Media}

\href{https://myaccount.nytimes3xbfgragh.onion/auth/login?response_type=cookie\&client_id=vi}{}

\href{https://www.nytimes3xbfgragh.onion/section/todayspaper}{Today's
Paper}

\href{/section/business/media}{Media}\textbar{}James Murdoch Resigns
From News Corp, Ending Role in Family Empire

\url{https://nyti.ms/3k1XgQM}

\begin{itemize}
\item
\item
\item
\item
\item
\item
\end{itemize}

Advertisement

\protect\hyperlink{after-top}{Continue reading the main story}

Supported by

\protect\hyperlink{after-sponsor}{Continue reading the main story}

\hypertarget{james-murdoch-resigns-from-news-corp-ending-role-in-family-empire}{%
\section{James Murdoch Resigns From News Corp, Ending Role in Family
Empire}\label{james-murdoch-resigns-from-news-corp-ending-role-in-family-empire}}

While his elder brother, Lachlan Murdoch, rises in the family business,
James Murdoch has grown more distant from his father's companies.

\includegraphics{https://static01.graylady3jvrrxbe.onion/images/2020/08/01/business/31james-murdoch-print/31james-murdoch-01-articleLarge.jpg?quality=75\&auto=webp\&disable=upscale}

\href{https://www.nytimes3xbfgragh.onion/by/michael-m-grynbaum}{\includegraphics{https://static01.graylady3jvrrxbe.onion/images/2018/10/22/multimedia/author-michael-m-grynbaum/author-michael-m-grynbaum-thumbLarge.png}}\href{https://www.nytimes3xbfgragh.onion/by/edmund-lee}{\includegraphics{https://static01.graylady3jvrrxbe.onion/images/2018/07/10/multimedia/author-edmund-lee/author-edmund-lee-thumbLarge.png}}

By
\href{https://www.nytimes3xbfgragh.onion/by/michael-m-grynbaum}{Michael
M. Grynbaum} and
\href{https://www.nytimes3xbfgragh.onion/by/edmund-lee}{Edmund Lee}

\begin{itemize}
\item
  Published July 31, 2020Updated Aug. 1, 2020, 12:37 p.m. ET
\item
  \begin{itemize}
  \item
  \item
  \item
  \item
  \item
  \item
  \end{itemize}
\end{itemize}

James Murdoch wants the world to know he is out of the family business.

Once considered a potential successor to Rupert Murdoch, Mr. Murdoch on
Friday resigned from the board of the newspaper publisher News Corp,
severing his last corporate tie to his father's global media empire.

``My resignation is due to disagreements over certain editorial content
published by the Company's news outlets and certain other strategic
decisions,'' Mr. Murdoch, 47,
\href{https://int.graylady3jvrrxbe.onion/data/documenttools/james-murdoch-s-resignation-letter/5d16f07153370f9d/full.pdf}{wrote
in his resignation letter}, which News Corp disclosed in a filing
shortly after the close of business on Friday.

The two sides began discussing Mr. Murdoch's departure from the News
Corp board earlier this year, according to two people with knowledge of
the matter.

But his terse resignation note belied the behind-the-scenes drama that
has brought Mr. Murdoch to this point in his life and career. And it
widened the schism that has emerged between James and his 89-year-old
father and his older brother, Lachlan, once a dynastic triumvirate that
for years held sweeping influence over the world's cultural and
political affairs.

A political outlier in his conservative-leaning family, James Murdoch
has sought to reinvent himself as an independent investor with a focus
on causes more closely associated with liberals, like environmentalism,
which he and his wife,
\href{https://www.nytimes3xbfgragh.onion/2019/09/26/climate/kathryn-murdoch-climate-change-voting.html}{Kathryn
Murdoch}, have long championed.

He has also taken public stands against President Trump, who has counted
Fox News, a prime Murdoch asset, among his closest media allies.

Weeks ago, James and his wife jointly contributed more than \$1 million
to a fund-raising committee for former Vice President Joseph R. Biden
Jr., the presumptive Democratic nominee for president. And in February,
as
\href{https://www.nytimes3xbfgragh.onion/2020/01/21/world/australia/fires-size-climate.html}{wildfires
raged} across Australia --- his father's birthplace --- Mr. Murdoch
issued a rebuke of his own family's media properties,
\href{https://www.nytimes3xbfgragh.onion/2020/02/12/business/dealbook/james-murdoch-environment.html}{criticizing}
how Murdoch publications have covered climate change.

Such public gestures came after a period when James Murdoch's hopes of
succeeding his father at the helm of a worldwide empire had been all but
extinguished.

He had already departed the Fox Corporation, the family's television and
entertainment arm, which was mostly dismantled after his family
transferred many of its assets to the Walt Disney Company in
\href{https://www.nytimes3xbfgragh.onion/2018/07/27/business/media/disney-fox-merger-vote.html}{a
blockbuster sale} that was completed last year.

His last formal link to the family business was through News Corp, which
publishes influential broadsheets like The Wall Street Journal as well
as powerful tabloids, including The Sun of London and The New York Post.
The company also oversees several other papers in Britain and
publications in Australia.

The London-born, Harvard-educated Mr. Murdoch remains a beneficiary of
his family's trust, meaning he will continue to financially benefit from
the profits of Rupert Murdoch's news and information assets.

And although his resignation letter cited ``certain editorial content,''
Mr. Murdoch did not speak specifically about Fox News, the hugely
profitable cable channel where prime-time hosts like Sean Hannity and
Laura Ingraham openly cheerlead for Mr. Trump.

A spokeswoman for Mr. Murdoch declined to comment further on the reasons
for his departure, saying the letter ``speaks for itself.''

Rupert, who holds the title of executive chairman at News Corp, and
Lachlan Murdoch, the co-chairman, said in a joint statement on Friday:
``We're grateful to James for his many years of service to the company.
We wish him the very best in his future endeavors.''

James Murdoch's drift from his family began in earnest during the early
part of the Trump era, around the time Lachlan was consolidating power
and becoming seen more widely as their father's preferred successor.

There had been discussions about James Murdoch taking a powerful new
role at Disney after the completion of the Fox sale, but those talks
came to nothing. His 48-year-old brother was named the executive
chairman and chief executive officer of Fox Corporation, which includes
Fox News, Fox Business and the Fox sports networks.

James Murdoch was the chief executive of 21st Century Fox from 2015
until it was sold to Disney, and he netted \$2 billion from the sale. He
opened his own investment firm and named it Lupa Systems. (In Roman
mythology, Lupa is the wolf goddess who nurtured Romulus and Remus, the
twin brothers who became the founders of Rome.)

The firm specializes in early stage start-ups and has focused on
sustainability projects, extending efforts that Mr. Murdoch made at Sky,
the European satellite giant that was
\href{https://www.bbc.com/news/business-45654792}{formerly part of the
Murdoch empire}, and his financial support of the National Geographic
Society's endowment fund.

Mr. Murdoch has also taken a starkly different tack with his media
investments. In October, he
\href{https://www.nytimes3xbfgragh.onion/2019/10/10/business/media/james-murdoch-vice-media.html}{bought
a small stake in Vice Media}, the irreverent --- and decidedly liberal
--- news brand focused on youth and entertainment. He has been less
interested in traditional media businesses.

In August, Mr. Murdoch led a consortium of investors to buy a
controlling stake in Tribeca Enterprises, which owns the Tribeca Film
Festival as well as a production studio. He also put money into
\href{https://www.nytimes3xbfgragh.onion/2019/03/20/arts/awa-comic-books-publisher.html}{Artists,
Writers \& Artisans}, a new comics publisher founded by former Marvel
executives.

In 2011, Mr. Murdoch was a chief figure in the
\href{https://www.nytimes3xbfgragh.onion/topic/organization/british-phone-hacking-scandal-leveson-report}{phone
hacking scandal} that led to the closure of News of the World, one of
the Murdochs' flagship properties, and strained his relationship with
his father. At the time, Mr. Murdoch was in charge of the family's
holdings across Europe, including the British newspapers that were
behind the hacks.

Called before a Parliamentary committee investigating the matter, he was
confronted with an email that appeared to show his knowledge of the
hacking; Mr. Murdoch said he had not read the entire email chain. The
committee chided James and his father for ``willful blindness'' about
the company's behavior.

The scandal dinged Mr. Murdoch's credibility in London, and he soon
relocated to New York to help run his father's businesses there, where
he focused on the Fox television empire and made investments in digital
ad technology.

This latest twist in the Murdoch saga is likely to show up in the myriad
pop culture products that depict the family's corporate and personal
dramas. The 2019 film ``Bombshell'' portrayed the Murdoch brothers
pushing out Roger Ailes, the founder of Fox News, after revelations of
sexual harassment and abuse at the network. In Britain, a new BBC
documentary series, ``The Rise of the Murdoch Dynasty,'' has offered a
searing review of the family's exploits.

Perhaps best known is the HBO series
``\href{https://www.nytimes3xbfgragh.onion/2019/08/04/business/media/hbo-succession-business.html}{Succession},''
which chronicles a Murdoch-like media family led by an aging patriarch
who pits his children against one another, sometimes in cruel ways.
Asked in an email exchange last year if he was a fan of the show, James
Murdoch pleaded ignorance.

``I've never watched it,'' he wrote.

Jim Rutenberg contributed reporting.

Advertisement

\protect\hyperlink{after-bottom}{Continue reading the main story}

\hypertarget{site-index}{%
\subsection{Site Index}\label{site-index}}

\hypertarget{site-information-navigation}{%
\subsection{Site Information
Navigation}\label{site-information-navigation}}

\begin{itemize}
\tightlist
\item
  \href{https://help.nytimes3xbfgragh.onion/hc/en-us/articles/115014792127-Copyright-notice}{©~2020~The
  New York Times Company}
\end{itemize}

\begin{itemize}
\tightlist
\item
  \href{https://www.nytco.com/}{NYTCo}
\item
  \href{https://help.nytimes3xbfgragh.onion/hc/en-us/articles/115015385887-Contact-Us}{Contact
  Us}
\item
  \href{https://www.nytco.com/careers/}{Work with us}
\item
  \href{https://nytmediakit.com/}{Advertise}
\item
  \href{http://www.tbrandstudio.com/}{T Brand Studio}
\item
  \href{https://www.nytimes3xbfgragh.onion/privacy/cookie-policy\#how-do-i-manage-trackers}{Your
  Ad Choices}
\item
  \href{https://www.nytimes3xbfgragh.onion/privacy}{Privacy}
\item
  \href{https://help.nytimes3xbfgragh.onion/hc/en-us/articles/115014893428-Terms-of-service}{Terms
  of Service}
\item
  \href{https://help.nytimes3xbfgragh.onion/hc/en-us/articles/115014893968-Terms-of-sale}{Terms
  of Sale}
\item
  \href{https://spiderbites.nytimes3xbfgragh.onion}{Site Map}
\item
  \href{https://help.nytimes3xbfgragh.onion/hc/en-us}{Help}
\item
  \href{https://www.nytimes3xbfgragh.onion/subscription?campaignId=37WXW}{Subscriptions}
\end{itemize}
