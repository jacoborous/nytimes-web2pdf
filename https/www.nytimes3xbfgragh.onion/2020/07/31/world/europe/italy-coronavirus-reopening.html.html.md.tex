Sections

SEARCH

\protect\hyperlink{site-content}{Skip to
content}\protect\hyperlink{site-index}{Skip to site index}

\href{https://www.nytimes3xbfgragh.onion/section/world/europe}{Europe}

\href{https://myaccount.nytimes3xbfgragh.onion/auth/login?response_type=cookie\&client_id=vi}{}

\href{https://www.nytimes3xbfgragh.onion/section/todayspaper}{Today's
Paper}

\href{/section/world/europe}{Europe}\textbar{}How Italy Turned Around
Its Coronavirus Calamity

\url{https://nyti.ms/33bBlkb}

\begin{itemize}
\item
\item
\item
\item
\item
\item
\end{itemize}

\href{https://www.nytimes3xbfgragh.onion/news-event/coronavirus?action=click\&pgtype=Article\&state=default\&region=TOP_BANNER\&context=storylines_menu}{The
Coronavirus Outbreak}

\begin{itemize}
\tightlist
\item
  live\href{https://www.nytimes3xbfgragh.onion/2020/08/01/world/coronavirus-covid-19.html?action=click\&pgtype=Article\&state=default\&region=TOP_BANNER\&context=storylines_menu}{Latest
  Updates}
\item
  \href{https://www.nytimes3xbfgragh.onion/interactive/2020/us/coronavirus-us-cases.html?action=click\&pgtype=Article\&state=default\&region=TOP_BANNER\&context=storylines_menu}{Maps
  and Cases}
\item
  \href{https://www.nytimes3xbfgragh.onion/interactive/2020/science/coronavirus-vaccine-tracker.html?action=click\&pgtype=Article\&state=default\&region=TOP_BANNER\&context=storylines_menu}{Vaccine
  Tracker}
\item
  \href{https://www.nytimes3xbfgragh.onion/interactive/2020/07/29/us/schools-reopening-coronavirus.html?action=click\&pgtype=Article\&state=default\&region=TOP_BANNER\&context=storylines_menu}{What
  School May Look Like}
\item
  \href{https://www.nytimes3xbfgragh.onion/live/2020/07/31/business/stock-market-today-coronavirus?action=click\&pgtype=Article\&state=default\&region=TOP_BANNER\&context=storylines_menu}{Economy}
\end{itemize}

Advertisement

\protect\hyperlink{after-top}{Continue reading the main story}

Supported by

\protect\hyperlink{after-sponsor}{Continue reading the main story}

\hypertarget{how-italy-turned-around-its-coronavirus-calamity}{%
\section{How Italy Turned Around Its Coronavirus
Calamity}\label{how-italy-turned-around-its-coronavirus-calamity}}

After a stumbling start, the country has gone from being a global pariah
to a model --- however imperfect --- of viral containment that holds
lessons for its neighbors and for the United States.

\includegraphics{https://static01.graylady3jvrrxbe.onion/images/2020/07/30/world/31virus-italy/merlin_173911632_5410458a-d14a-451c-b210-c7578df59244-articleLarge.jpg?quality=75\&auto=webp\&disable=upscale}

\href{https://www.nytimes3xbfgragh.onion/by/jason-horowitz}{\includegraphics{https://static01.graylady3jvrrxbe.onion/images/2018/10/10/multimedia/author-jason-horowitz/author-jason-horowitz-thumbLarge.png}}

By \href{https://www.nytimes3xbfgragh.onion/by/jason-horowitz}{Jason
Horowitz}

\begin{itemize}
\item
  July 31, 2020
\item
  \begin{itemize}
  \item
  \item
  \item
  \item
  \item
  \item
  \end{itemize}
\end{itemize}

ROME --- When the coronavirus erupted in the West, Italy was
\href{https://www.nytimes3xbfgragh.onion/interactive/2020/03/27/world/europe/coronavirus-italy-bergamo.html}{the
nightmarish epicenter}, a place to avoid at all costs and a shorthand in
the United States and much of Europe for uncontrolled contagion.

``You look at what's going on with Italy,'' President Trump told
reporters on March 17. ``We don't want to be in a position like that.''
Joseph R. Biden Jr., the presumptive Democratic nominee, used Italy's
overwhelmed hospitals as evidence for his opposition to Medicare for All
at a presidential debate.
``\href{https://edition.cnn.com/politics/live-news/2020-democratic-debate-live-updates/h_501d1e381370480bd021916a86029534}{It
is not working in Italy right now,'' he said}.

Fast forward a few months, and the United States has suffered tens of
thousands more deaths than any country in the world. European states
that once looked smugly at Italy are facing new flare-ups. Some are
imposing fresh restrictions and weighing whether to lock down again.

Prime Minister Boris Johnson of Britain on Friday announced a delay to a
planned easing of measures in England as the infection rate there rose.
Even Germany, lauded for its efficient response and rigorous contact
tracing, has warned that lax behavior is prompting a surge in cases.

And Italy? Its hospitals are basically empty of Covid-19 patients. Daily
deaths attributed to the virus in Lombardy, the northern region that
bore the brunt of the pandemic, hover around zero. The number of new
daily cases has plummeted to ``one of the lowest in Europe and the
world,'' said Giovanni Rezza, director of the infective illness
department at the National Institute of Health. ``We have been very
prudent.''

And lucky. Today, despite a tiny uptick in cases this week, Italians are
cautiously optimistic that they have the virus in check --- even as
Italy's leading health experts warn that complacency remains the jet
fuel of the pandemic. They are aware that the picture can change at any
moment.

\includegraphics{https://static01.graylady3jvrrxbe.onion/images/2020/07/30/world/31virus-italy2/merlin_173670762_9f840673-acff-40dc-8c08-694c8c07ece9-articleLarge.jpg?quality=75\&auto=webp\&disable=upscale}

How Italy has gone from being a
\href{https://www.nytimes3xbfgragh.onion/2020/03/21/world/europe/italy-coronavirus-center-lessons.html}{global
pariah} to a model --- however imperfect --- of viral containment holds
fresh lessons for the rest of the world, including the United States,
where the virus, never under control, now rages across the country.

After a stumbling start, Italy has consolidated, or at least maintained,
the rewards of a tough nationwide lockdown through a mix of vigilance
and painfully gained medical expertise.

Its government has been guided by scientific and technical committees.
Local doctors, hospitals and health officials collect more than 20
indicators on the virus daily and send them to regional authorities, who
then forward them to the National Institute of Health.

\hypertarget{latest-updates-global-coronavirus-outbreak}{%
\section{\texorpdfstring{\href{https://www.nytimes3xbfgragh.onion/2020/08/01/world/coronavirus-covid-19.html?action=click\&pgtype=Article\&state=default\&region=MAIN_CONTENT_1\&context=storylines_live_updates}{Latest
Updates: Global Coronavirus
Outbreak}}{Latest Updates: Global Coronavirus Outbreak}}\label{latest-updates-global-coronavirus-outbreak}}

Updated 2020-08-01T19:08:55.687Z

\begin{itemize}
\tightlist
\item
  \href{https://www.nytimes3xbfgragh.onion/2020/08/01/world/coronavirus-covid-19.html?action=click\&pgtype=Article\&state=default\&region=MAIN_CONTENT_1\&context=storylines_live_updates\#link-3ac56579}{Top
  officials work to break impasse over jobless benefit.}
\item
  \href{https://www.nytimes3xbfgragh.onion/2020/08/01/world/coronavirus-covid-19.html?action=click\&pgtype=Article\&state=default\&region=MAIN_CONTENT_1\&context=storylines_live_updates\#link-8796723}{The
  virus picks up dangerous speed in the Midwest, and in areas that had
  seen success.}
\item
  \href{https://www.nytimes3xbfgragh.onion/2020/08/01/world/coronavirus-covid-19.html?action=click\&pgtype=Article\&state=default\&region=MAIN_CONTENT_1\&context=storylines_live_updates\#link-25930521}{Thousands
  in Berlin protest Germany's coronavirus measures.}
\end{itemize}

\href{https://www.nytimes3xbfgragh.onion/2020/08/01/world/coronavirus-covid-19.html?action=click\&pgtype=Article\&state=default\&region=MAIN_CONTENT_1\&context=storylines_live_updates}{See
more updates}

More live coverage:
\href{https://www.nytimes3xbfgragh.onion/live/2020/07/31/business/stock-market-today-coronavirus?action=click\&pgtype=Article\&state=default\&region=MAIN_CONTENT_1\&context=storylines_live_updates}{Markets}

The result is a weekly X-ray of the country's health upon which policy
decisions are based. That is a long way from the state of panic, and
near collapse, that hit Italy in March.

This week, Parliament voted to extend the government's emergency powers
through Oct. 15 after Prime Minister Giuseppe Conte argued the nation
could not let its guard down ``because the virus is still circulating.''

Those powers allow the government to keep restrictions in place and
respond quickly --- including with lockdowns --- to any new clusters.
The government has already imposed travel restrictions on more than a
dozen countries to Italy, as the importation of the virus from countries
is now the government's greatest fear.

``There are a lot of situations in France, Spain, the Balkans, which
means that the virus is not off at all,'' said Ranieri Guerra, assistant
director general for strategic initiatives at the World Health
Organization and an Italian doctor. ``It can come back at any time.''

Image

Prime Minister Giuseppe Conte of Italy told Parliament on Tuesday that
the nation could not let its guard down ``because the virus is still
circulating.''Credit...Fabio Frustaci/EPA, via Shutterstock

There is no doubt that the privations of the lockdown were economically
costly. For three months, businesses and restaurants were ordered
closed, movement was highly restricted --- even between regions, towns
and streets --- and tourism ground to a halt. Italy is expected to lose
about 10 percent of its gross domestic product this year.

But at a certain point, as the virus threatened to spread
uncontrollably, Italian officials decided to put lives ahead of the
economy. ``The health of the Italian people comes and will always come
first,'' Mr. Conte said at the time.

Italian officials now hope that the worst of the cure came in one large
dose --- the painful lockdown --- and that the country is now safe to
resume normal life, albeit with limits. They argue that the only way to
start up the economy is to keep tamping down the virus, even now.

The strategy of closing down completely invited criticism that the
government's excessive caution was paralyzing the economy. But it may
prove to be more advantageous than trying to reopen the economy while
the virus still rages, as is happening in countries like
\href{https://www.nytimes3xbfgragh.onion/2020/03/13/us/coronavirus-deaths-estimate.html}{the
United States},
\href{https://www.nytimes3xbfgragh.onion/article/brazil-coronavirus-cases.html}{Brazil}
and
\href{https://www.nytimes3xbfgragh.onion/2020/06/05/world/americas/coronavirus-mexico-reopening.html}{Mexico}.

That does not mean that calls for continued vigilance, as elsewhere in
the world, have been immune to mockery, resistance and exasperation. In
that, Italy is no different.

Masks often are missing or lowered in trains or buses, where they are
mandatory. Young people are going out and
\href{https://www.nytimes3xbfgragh.onion/2020/05/29/world/europe/italy-young-people-coronavirus.html}{doing
the things young people do} --- and risk in that way spreading the virus
to more susceptible parts of the population. Adults started
\href{https://www.nytimes3xbfgragh.onion/2020/05/27/world/europe/italy-beaches-coronavirus-reopening.html}{gathering
at the beach} and for birthday barbecues. There is still no clear plan
for a return to school in September.

There is also a burgeoning, and politically motivated, anti-mask
contingent led by nationalist Matteo Salvini, who on July 27 declared
that replacing handshakes and hugs with elbow bumps was ``the end of the
human species.''

At his rallies, Mr. Salvini, the leader of the populist League party,
still shakes hands and wears his mask like a chin guard. In July, during
a news conference, he accused the Italian government of ``importing''
infected immigrants to create new clusters and extend the state of
emergency.

Image

Matteo Salvini, the hard-right nationalist politician, has said that
replacing handshakes and hugs with elbow bumps was ``the end of the
human species.''Credit...Remo Casilli/Reuters

This week, Mr. Salvini joined other mask skeptics --- nicknamed the
``negationists'' by critics --- for a protest in the Senate library,
along with special guests such as the Italian crooner Andrea Bocelli,
who said he did not believe the pandemic was so serious because ``I know
a lot of people and I don't know anyone who ended up in an I.C.U.''

But the country's leading health experts say that the lack of severe
cases is indicative of a decrease in the volume of infections, as only a
small percentage of the infected get very sick. And so far, Italy's
malcontents have not been numerous or powerful enough to undermine what
has been a hard-won trajectory of success in confronting the virus after
a calamitous start.

Italy's initial isolation by European neighbors at the outset of the
crisis, when masks and ventilators were hardly pouring in from across
the borders, may actually have helped, Mr. Guerra, the W.H.O. expert,
said.

\href{https://www.nytimes3xbfgragh.onion/news-event/coronavirus?action=click\&pgtype=Article\&state=default\&region=MAIN_CONTENT_3\&context=storylines_faq}{}

\hypertarget{the-coronavirus-outbreak-}{%
\subsubsection{The Coronavirus Outbreak
›}\label{the-coronavirus-outbreak-}}

\hypertarget{frequently-asked-questions}{%
\paragraph{Frequently Asked
Questions}\label{frequently-asked-questions}}

Updated July 27, 2020

\begin{itemize}
\item ~
  \hypertarget{should-i-refinance-my-mortgage}{%
  \paragraph{Should I refinance my
  mortgage?}\label{should-i-refinance-my-mortgage}}

  \begin{itemize}
  \tightlist
  \item
    \href{https://www.nytimes3xbfgragh.onion/article/coronavirus-money-unemployment.html?action=click\&pgtype=Article\&state=default\&region=MAIN_CONTENT_3\&context=storylines_faq}{It
    could be a good idea,} because mortgage rates have
    \href{https://www.nytimes3xbfgragh.onion/2020/07/16/business/mortgage-rates-below-3-percent.html?action=click\&pgtype=Article\&state=default\&region=MAIN_CONTENT_3\&context=storylines_faq}{never
    been lower.} Refinancing requests have pushed mortgage applications
    to some of the highest levels since 2008, so be prepared to get in
    line. But defaults are also up, so if you're thinking about buying a
    home, be aware that some lenders have tightened their standards.
  \end{itemize}
\item ~
  \hypertarget{what-is-school-going-to-look-like-in-september}{%
  \paragraph{What is school going to look like in
  September?}\label{what-is-school-going-to-look-like-in-september}}

  \begin{itemize}
  \tightlist
  \item
    It is unlikely that many schools will return to a normal schedule
    this fall, requiring the grind of
    \href{https://www.nytimes3xbfgragh.onion/2020/06/05/us/coronavirus-education-lost-learning.html?action=click\&pgtype=Article\&state=default\&region=MAIN_CONTENT_3\&context=storylines_faq}{online
    learning},
    \href{https://www.nytimes3xbfgragh.onion/2020/05/29/us/coronavirus-child-care-centers.html?action=click\&pgtype=Article\&state=default\&region=MAIN_CONTENT_3\&context=storylines_faq}{makeshift
    child care} and
    \href{https://www.nytimes3xbfgragh.onion/2020/06/03/business/economy/coronavirus-working-women.html?action=click\&pgtype=Article\&state=default\&region=MAIN_CONTENT_3\&context=storylines_faq}{stunted
    workdays} to continue. California's two largest public school
    districts --- Los Angeles and San Diego --- said on July 13, that
    \href{https://www.nytimes3xbfgragh.onion/2020/07/13/us/lausd-san-diego-school-reopening.html?action=click\&pgtype=Article\&state=default\&region=MAIN_CONTENT_3\&context=storylines_faq}{instruction
    will be remote-only in the fall}, citing concerns that surging
    coronavirus infections in their areas pose too dire a risk for
    students and teachers. Together, the two districts enroll some
    825,000 students. They are the largest in the country so far to
    abandon plans for even a partial physical return to classrooms when
    they reopen in August. For other districts, the solution won't be an
    all-or-nothing approach.
    \href{https://bioethics.jhu.edu/research-and-outreach/projects/eschool-initiative/school-policy-tracker/}{Many
    systems}, including the nation's largest, New York City, are
    devising
    \href{https://www.nytimes3xbfgragh.onion/2020/06/26/us/coronavirus-schools-reopen-fall.html?action=click\&pgtype=Article\&state=default\&region=MAIN_CONTENT_3\&context=storylines_faq}{hybrid
    plans} that involve spending some days in classrooms and other days
    online. There's no national policy on this yet, so check with your
    municipal school system regularly to see what is happening in your
    community.
  \end{itemize}
\item ~
  \hypertarget{is-the-coronavirus-airborne}{%
  \paragraph{Is the coronavirus
  airborne?}\label{is-the-coronavirus-airborne}}

  \begin{itemize}
  \tightlist
  \item
    The coronavirus
    \href{https://www.nytimes3xbfgragh.onion/2020/07/04/health/239-experts-with-one-big-claim-the-coronavirus-is-airborne.html?action=click\&pgtype=Article\&state=default\&region=MAIN_CONTENT_3\&context=storylines_faq}{can
    stay aloft for hours in tiny droplets in stagnant air}, infecting
    people as they inhale, mounting scientific evidence suggests. This
    risk is highest in crowded indoor spaces with poor ventilation, and
    may help explain super-spreading events reported in meatpacking
    plants, churches and restaurants.
    \href{https://www.nytimes3xbfgragh.onion/2020/07/06/health/coronavirus-airborne-aerosols.html?action=click\&pgtype=Article\&state=default\&region=MAIN_CONTENT_3\&context=storylines_faq}{It's
    unclear how often the virus is spread} via these tiny droplets, or
    aerosols, compared with larger droplets that are expelled when a
    sick person coughs or sneezes, or transmitted through contact with
    contaminated surfaces, said Linsey Marr, an aerosol expert at
    Virginia Tech. Aerosols are released even when a person without
    symptoms exhales, talks or sings, according to Dr. Marr and more
    than 200 other experts, who
    \href{https://academic.oup.com/cid/article/doi/10.1093/cid/ciaa939/5867798}{have
    outlined the evidence in an open letter to the World Health
    Organization}.
  \end{itemize}
\item ~
  \hypertarget{what-are-the-symptoms-of-coronavirus}{%
  \paragraph{What are the symptoms of
  coronavirus?}\label{what-are-the-symptoms-of-coronavirus}}

  \begin{itemize}
  \tightlist
  \item
    Common symptoms
    \href{https://www.nytimes3xbfgragh.onion/article/symptoms-coronavirus.html?action=click\&pgtype=Article\&state=default\&region=MAIN_CONTENT_3\&context=storylines_faq}{include
    fever, a dry cough, fatigue and difficulty breathing or shortness of
    breath.} Some of these symptoms overlap with those of the flu,
    making detection difficult, but runny noses and stuffy sinuses are
    less common.
    \href{https://www.nytimes3xbfgragh.onion/2020/04/27/health/coronavirus-symptoms-cdc.html?action=click\&pgtype=Article\&state=default\&region=MAIN_CONTENT_3\&context=storylines_faq}{The
    C.D.C. has also} added chills, muscle pain, sore throat, headache
    and a new loss of the sense of taste or smell as symptoms to look
    out for. Most people fall ill five to seven days after exposure, but
    symptoms may appear in as few as two days or as many as 14 days.
  \end{itemize}
\item ~
  \hypertarget{does-asymptomatic-transmission-of-covid-19-happen}{%
  \paragraph{Does asymptomatic transmission of Covid-19
  happen?}\label{does-asymptomatic-transmission-of-covid-19-happen}}

  \begin{itemize}
  \tightlist
  \item
    So far, the evidence seems to show it does. A widely cited
    \href{https://www.nature.com/articles/s41591-020-0869-5}{paper}
    published in April suggests that people are most infectious about
    two days before the onset of coronavirus symptoms and estimated that
    44 percent of new infections were a result of transmission from
    people who were not yet showing symptoms. Recently, a top expert at
    the World Health Organization stated that transmission of the
    coronavirus by people who did not have symptoms was ``very rare,''
    \href{https://www.nytimes3xbfgragh.onion/2020/06/09/world/coronavirus-updates.html?action=click\&pgtype=Article\&state=default\&region=MAIN_CONTENT_3\&context=storylines_faq\#link-1f302e21}{but
    she later walked back that statement.}
  \end{itemize}
\end{itemize}

``There was competition initially, there was no collaboration,'' Mr.
Guerra said. ``And everyone acknowledged Italy was left alone at that
time.'' As a result, he said, ``what they had to do at that time because
we were left alone turned out to be more effective than other
countries.''

Italy first
\href{https://www.nytimes3xbfgragh.onion/2020/02/23/world/europe/italy-coronavirus.html}{quarantined
towns} and then
\href{https://www.nytimes3xbfgragh.onion/2020/03/07/world/europe/coronavirus-italy.html}{the
Lombardy region} in the north and
\href{https://www.nytimes3xbfgragh.onion/2020/03/09/world/europe/italy-lockdown-coronavirus.html}{then
the entire peninsula} and its islands, despite the near absence of the
virus in much of central and southern Italy. That not only prevented
workers in the industrial north from returning home in the much more
vulnerable south, but it also fostered and forced a unified national
response.

Image

The Piazza Duomo in Milan during Italy's lockdown in early
April.Credit...Alessandro Grassani for The New York Times

During the lockdown, movement was strictly limited, between regions and
towns and even city blocks, and people had to fill in
``auto-certification'' forms to prove that they needed to go outside for
work, health or ``other necessities.'' Masks and social distancing
regulations were enforced by some regional authorities with steep fines.
Generally, if grudgingly,
\href{https://www.nytimes3xbfgragh.onion/2020/03/10/world/europe/italy-coronavirus-movement-restrictions.html}{the
rules were followed}.

As searing
\href{https://www.nytimes3xbfgragh.onion/2020/03/16/world/europe/italy-coronavirus-funerals.html}{scenes
of human suffering}, empty streets and the
\href{https://www.nytimes3xbfgragh.onion/2020/03/04/world/europe/coronavirus-italy-elderly.html}{heavy
toll on an elderly generation} of northern Italians spread, the
transmission rate of the virus quickly decreased, and the curve
flattened, as opposed to other European countries,
\href{https://www.nytimes3xbfgragh.onion/2020/07/07/business/sweden-economy-coronavirus.html}{such
as Sweden}, which pursued an alternative to locking down.

That the initial outbreak was localized in the overwhelmed hospitals
created enormous stress, but it also enabled doctors and nurses to
expedite contact tracing.

Then the country reopened, gradually, expanding liberties at two-week
intervals to respond to the virus's incubation period.

The lockdown eventually had a secondary effect of decreasing the volume
of virus circulating in society, and thus reducing the probability of
coming in contact with someone who had it. At the end of the lockdown,
the virus circulation had steeply fallen off and in some central and
southern regions, there were hardly any chains of transmission at all.

``It's always a matter of probability with these pathogens,'' said Mr.
Guerra, adding that new early alarm systems such as the monitoring of
wastewater for traces of virus had lowered the probability of infection
even more.

Image

Italy reopened gradually, expanding liberties at two-week intervals.
Travel between the country's regions was not allowed until early
June.Credit...Claudio Furlan/LaPresse, via Associated Press

Some Italian doctors say they believe that the virus is now behaving
differently in Italy. Matteo Bassetti, an infectious-disease doctor in
the northwestern city of Genoa, said that during the height of the
crisis, his hospital was inundated with 500 Covid-19 cases at one time.
Now, he said, his intensive care unit, with 50 beds, has no coronavirus
patients, and the 60-bed Covid-19 unit built specially for the crisis is
empty.

He said he thought that the virus had weakened --- an unproven view, he
acknowledged, that has nonetheless found an eager audience in Mr.
Salvini and other politicians opposed to extending the state of
emergency.

Most health experts said that the virus still loomed, and as the
government considers a new decree to reopen night clubs, festivals and
cruise ship travel, many of them have implored the country not to let
down its guard.

``Even if the situation is better than in other countries, we should
continue to be very prudent,'' said Dr. Rezza of the National Institute
of Health, adding that he thought the question of what Italy had done
right was better posed ``at the end of the epidemic.''

``We cannot exclude that we will have outbreaks in Italy in the next few
days,'' he said. ``Maybe it's just a matter of time.''

Emma Bubola contributed reporting from Milan.

Advertisement

\protect\hyperlink{after-bottom}{Continue reading the main story}

\hypertarget{site-index}{%
\subsection{Site Index}\label{site-index}}

\hypertarget{site-information-navigation}{%
\subsection{Site Information
Navigation}\label{site-information-navigation}}

\begin{itemize}
\tightlist
\item
  \href{https://help.nytimes3xbfgragh.onion/hc/en-us/articles/115014792127-Copyright-notice}{©~2020~The
  New York Times Company}
\end{itemize}

\begin{itemize}
\tightlist
\item
  \href{https://www.nytco.com/}{NYTCo}
\item
  \href{https://help.nytimes3xbfgragh.onion/hc/en-us/articles/115015385887-Contact-Us}{Contact
  Us}
\item
  \href{https://www.nytco.com/careers/}{Work with us}
\item
  \href{https://nytmediakit.com/}{Advertise}
\item
  \href{http://www.tbrandstudio.com/}{T Brand Studio}
\item
  \href{https://www.nytimes3xbfgragh.onion/privacy/cookie-policy\#how-do-i-manage-trackers}{Your
  Ad Choices}
\item
  \href{https://www.nytimes3xbfgragh.onion/privacy}{Privacy}
\item
  \href{https://help.nytimes3xbfgragh.onion/hc/en-us/articles/115014893428-Terms-of-service}{Terms
  of Service}
\item
  \href{https://help.nytimes3xbfgragh.onion/hc/en-us/articles/115014893968-Terms-of-sale}{Terms
  of Sale}
\item
  \href{https://spiderbites.nytimes3xbfgragh.onion}{Site Map}
\item
  \href{https://help.nytimes3xbfgragh.onion/hc/en-us}{Help}
\item
  \href{https://www.nytimes3xbfgragh.onion/subscription?campaignId=37WXW}{Subscriptions}
\end{itemize}
