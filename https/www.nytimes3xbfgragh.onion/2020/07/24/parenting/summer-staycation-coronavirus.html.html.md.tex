Sections

SEARCH

\protect\hyperlink{site-content}{Skip to
content}\protect\hyperlink{site-index}{Skip to site index}

\href{https://www.nytimes3xbfgragh.onion/section/parenting}{Parenting}

\href{https://myaccount.nytimes3xbfgragh.onion/auth/login?response_type=cookie\&client_id=vi}{}

\href{https://www.nytimes3xbfgragh.onion/section/todayspaper}{Today's
Paper}

\href{/section/parenting}{Parenting}\textbar{}Making the Most of a
Pandemic Staycation

\url{https://nyti.ms/3hDHERP}

\begin{itemize}
\item
\item
\item
\item
\item
\end{itemize}

\href{https://www.nytimes3xbfgragh.onion/spotlight/at-home?action=click\&pgtype=Article\&state=default\&region=TOP_BANNER\&context=at_home_menu}{At
Home}

\begin{itemize}
\tightlist
\item
  \href{https://www.nytimes3xbfgragh.onion/2020/07/28/books/time-for-a-literary-road-trip.html?action=click\&pgtype=Article\&state=default\&region=TOP_BANNER\&context=at_home_menu}{Take:
  A Literary Road Trip}
\item
  \href{https://www.nytimes3xbfgragh.onion/2020/07/29/magazine/bored-with-your-home-cooking-some-smoky-eggplant-will-fix-that.html?action=click\&pgtype=Article\&state=default\&region=TOP_BANNER\&context=at_home_menu}{Cook:
  Smoky Eggplant}
\item
  \href{https://www.nytimes3xbfgragh.onion/2020/07/27/travel/moose-michigan-isle-royale.html?action=click\&pgtype=Article\&state=default\&region=TOP_BANNER\&context=at_home_menu}{Look
  Out: For Moose}
\item
  \href{https://www.nytimes3xbfgragh.onion/interactive/2020/at-home/even-more-reporters-editors-diaries-lists-recommendations.html?action=click\&pgtype=Article\&state=default\&region=TOP_BANNER\&context=at_home_menu}{Explore:
  Reporters' Obsessions}
\end{itemize}

Advertisement

\protect\hyperlink{after-top}{Continue reading the main story}

Supported by

\protect\hyperlink{after-sponsor}{Continue reading the main story}

\hypertarget{making-the-most-of-a-pandemic-staycation}{%
\section{Making the Most of a Pandemic
Staycation}\label{making-the-most-of-a-pandemic-staycation}}

Families who stayed put in cities during the pandemic are rediscovering
new ways to appreciate their neighborhoods and even their homes.

\includegraphics{https://static01.graylady3jvrrxbe.onion/images/2020/07/24/multimedia/24-par-vacation-city/24-par-vacation-city-articleLarge.jpg?quality=75\&auto=webp\&disable=upscale}

\href{https://www.nytimes3xbfgragh.onion/by/christina-caron}{\includegraphics{https://static01.graylady3jvrrxbe.onion/images/2018/07/18/multimedia/author-christina-caron/author-christina-caron-thumbLarge.png}}

By
\href{https://www.nytimes3xbfgragh.onion/by/christina-caron}{Christina
Caron}

\begin{itemize}
\item
  Published July 24, 2020Updated July 26, 2020
\item
  \begin{itemize}
  \item
  \item
  \item
  \item
  \item
  \end{itemize}
\end{itemize}

It started as a simple text in June from a neighborhood friend: ``Walk
to Little Red Lighthouse this weekend?''

``That would be fun!'' I responded, with a level of excitement that now
accompanies nearly every offer of social interaction.

``Would Sunday work?'' he asked.

``It's not like we have anywhere else to be!'' I replied.

Suddenly, I realized that this would be the first time we had left our
neighborhood in more than three months. During the pandemic we had rigid
rules about travel and social distancing. Some of our friends had rented
cars to take scenic trips outside the city, posting photos of majestic
waterfalls or the luscious berries they handpicked at an organic farm.
But we never ventured beyond a half-mile radius from our New York
apartment. Certainly there were worse problems: We were lucky to be
healthy and employed. But after a while I started to feel increasingly
anxious as each day became similar to the last.

A trip to the
\href{https://www.nycgovparks.org/parks/fort-washington-park/highlights/11044}{Little
Red Lighthouse}, one of the surviving lighthouses in New York City,
would be a cheap and safe way to have a new adventure. It sits
underneath the George Washington Bridge in Fort Washington Park, a
two-mile walk from where our family lives. We had never seen it in
person.

The lighthouse was the main character in one of our 3-year-old
daughter's favorite picture books, ``The Little Red Lighthouse and the
Great Gray Bridge,'' written by Hildegarde H. Swift and published in
1942. The book shows how something as small as a lighthouse can hold
power and purpose, even in the shadow of the impressive bridge that
towers above it.

Even if you don't feel comfortable traveling far during the pandemic,
it's still important for families --- and parents --- to intentionally
carve out moments to do something out of the ordinary, experts say.

In other words, don't just tell yourself, ``OK, we're stuck in the city
and we're going to just drift our way through the summer and hope we
don't kill each other before the summer is over,'' said Claudia W.
Allen, Ph.D., a clinical psychologist and the director of the Family
Stress Clinic at the University of Virginia School of Medicine.

For New Yorkers, that means asking yourself, ``This is our New York
summer. What do we want to make sure we do?'' Dr. Allen said.

``Because while this feels interminable, in reality this will not
last,'' she added. ``In a weird way, this is precious time.''

On the day of our micro staycation, six of us began walking along the
Hudson River Greenway with our masks and strollers: Myself, my wife, our
daughter, our friends and their 3-year-old son. We were thrilled to be
going somewhere beyond the confines of our immediate surroundings, even
though it was humid and 92 degrees.

When we finally arrived, the lighthouse was a humble sight: It appeared
weatherworn. The door to the lighthouse had rusted, the red paint
flaking off to reveal unsightly blotches. It is remarkable how it has
endured. Were it not for fans of the picture book who rallied to save
it, the lighthouse would have been auctioned off decades ago.

My wife and I sat near the water and watched our daughter play. It felt
almost transformative to have a moment to ourselves.

``Remember how we used to enjoy each other's company?'' she asked, only
half-joking.

We settled down on a patch of grass to have a picnic, making sure to
spread out about six feet apart from our friends.

For many of us, the summer of 2020 will not be known for road trips,
amusement parks, lakeside retreats or anything remotely aspirational.
But families across the country are finding small yet meaningful ways to
escape, have fun again and experience something new.

Summer memories don't need to come with souvenirs that you can stick on
a fridge, said Kiki Blazevski-Charpentier, 37, of Queens. She and her
husband both work on construction projects in New York City brought
about by the pandemic. They have no plans to leave for the summer, as
\href{https://www.nytimes3xbfgragh.onion/interactive/2020/05/16/nyregion/nyc-coronavirus-moving-leaving.html}{other
New Yorkers} have done.

Blazevski-Charpentier said she and her husband and two children will
often escape to their small balcony, which is furnished with tables and
potted plants. In past summers, she said, they would only spend time on
the balcony about two times a week because they were busy with other
activities. Now it's at least twice a day. A cardboard painter's canvas
hangs from the brick wall, attached with black duct tape. In the
evening, after their art projects are done, the kids use the balcony to
play with water guns.

``We're finding the moments we didn't have before,'' she said.

She began documenting this in a series of photos she calls the ``balcony
bunker diaries.''

``We've seen pigeons lay eggs, nest, hatch and make a complete mess,''
she added. ``This is totally thrilling for a young kid. You cannot spend
two weeks at a zoo watching and waiting for eggs to be laid.''

Iris Tarou, 38, and her husband, who live in San Francisco, had looked
into vacation rentals to get away but Tarou said they were ``way late to
the game.'' Everything was booked within a three- to five-hour drive of
San Francisco through September, she added.

So Tarou asked her friends if they knew of anyone who was renting their
home and was connected with someone who had gotten a last-minute
cancellation in Novato, Calif., a picturesque city in San Francisco's
North Bay. They snapped it up.

``We stayed there for five nights and didn't leave their house. It was
like being at a private resort,'' she said.

The family has even improvised other escapes inside their apartment,
like dressing up for ``fancy dinner night'' and pretending to eat at an
upscale restaurant.

``While these past few months have been some of the hardest of my life,
they've also led to many amazing and fun times as a family, which we
wouldn't have had otherwise,'' Tarou said.

To create the sensation that you're getting away, it's important to
distinguish between your days off and your work days, Dr. Allen advised,
and then do something special during your free time, even if it's as
small as making a nice breakfast. One family she knew used a map to
color in each new street they had visited in their neighborhood, which
not only provided a visual record of their journeys but also offered a
sense of accomplishment as they highlighted street after street.

Paulina Perera-Riveroll, 47, her husband and their 12-year-old daughter
have weeknight picnics in Central Park with other families, while
staying socially distanced from one another.

At home, they've been doing a lot of cooking --- with mixed results.

``I made potato chips for the first time,'' she said. ``Never again.''

In Jersey City, a family of five found solace at a friend's empty
apartment, setting up a tent in the living room to go ``camping'' with
their three children, ages 6, 5 and 2.

``They were extremely excited,'' said their dad, Warren Bennett, 39.
``They didn't really mind that it was inside.''

But parents who want to create memorable summers for their kids
shouldn't forget themselves in the process, said Michele Weiner-Davis, a
marriage therapist in Boulder, Colo., who has written six books about
how couples can strengthen their relationships.

``I think people are often at the end of their rope trying to figure out
new, interesting, creative ways to keep their kids entertained and also
not rob them of a normal childhood,'' she said.

Speak your expectations out loud, she suggested, and set up a schedule.
Decide when you'll have alone time, which times you'll be together as a
family and when you'll have couple time.

And for parents of little kids, try not to worry if your summer
``vacation'' isn't what you thought it would be. It almost doesn't
matter what you choose to do. For the most part children really like
spending time with their parents, she said.

``So don't underestimate the importance of just showing up,'' she added.

Our delight in that lighthouse trip led us to have another socially
distanced picnic in a nearby park two weeks later. But soon after we
ate, the skies began to darken. As the rain came down in heavy sheets,
we raced along the tree-lined paths as our daughter laughed and
screeched in her stroller, only partially covered by its canopy. When we
finally got home, our clothes were drooping like soggy sacks, water
pooling at our feet. The rain suddenly stopped. We paused in the hallway
of our building to snap a selfie. From our kitchen window, a giant
rainbow appeared.

Much like the lighthouse excursion, I often think of that rainy day as
one of our best summer adventures. I didn't realize that my daughter
still thinks about it, too. This week, as another thunderstorm lit up
the night sky, our 3-year-old turned to me and said, ``Let's get caught
in the rain. Let's do that again.''

Advertisement

\protect\hyperlink{after-bottom}{Continue reading the main story}

\hypertarget{site-index}{%
\subsection{Site Index}\label{site-index}}

\hypertarget{site-information-navigation}{%
\subsection{Site Information
Navigation}\label{site-information-navigation}}

\begin{itemize}
\tightlist
\item
  \href{https://help.nytimes3xbfgragh.onion/hc/en-us/articles/115014792127-Copyright-notice}{©~2020~The
  New York Times Company}
\end{itemize}

\begin{itemize}
\tightlist
\item
  \href{https://www.nytco.com/}{NYTCo}
\item
  \href{https://help.nytimes3xbfgragh.onion/hc/en-us/articles/115015385887-Contact-Us}{Contact
  Us}
\item
  \href{https://www.nytco.com/careers/}{Work with us}
\item
  \href{https://nytmediakit.com/}{Advertise}
\item
  \href{http://www.tbrandstudio.com/}{T Brand Studio}
\item
  \href{https://www.nytimes3xbfgragh.onion/privacy/cookie-policy\#how-do-i-manage-trackers}{Your
  Ad Choices}
\item
  \href{https://www.nytimes3xbfgragh.onion/privacy}{Privacy}
\item
  \href{https://help.nytimes3xbfgragh.onion/hc/en-us/articles/115014893428-Terms-of-service}{Terms
  of Service}
\item
  \href{https://help.nytimes3xbfgragh.onion/hc/en-us/articles/115014893968-Terms-of-sale}{Terms
  of Sale}
\item
  \href{https://spiderbites.nytimes3xbfgragh.onion}{Site Map}
\item
  \href{https://help.nytimes3xbfgragh.onion/hc/en-us}{Help}
\item
  \href{https://www.nytimes3xbfgragh.onion/subscription?campaignId=37WXW}{Subscriptions}
\end{itemize}
