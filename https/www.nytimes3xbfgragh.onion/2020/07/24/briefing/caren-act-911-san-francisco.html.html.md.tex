Sections

SEARCH

\protect\hyperlink{site-content}{Skip to
content}\protect\hyperlink{site-index}{Skip to site index}

\href{https://www.nytimes3xbfgragh.onion/interactive/2018/briefing/global-morning-briefing-newsletter-signup.html}{Briefing}

\href{https://myaccount.nytimes3xbfgragh.onion/auth/login?response_type=cookie\&client_id=vi}{}

\href{https://www.nytimes3xbfgragh.onion/section/todayspaper}{Today's
Paper}

\href{/interactive/2018/briefing/global-morning-briefing-newsletter-signup.html}{Briefing}\textbar{}Will
It Take a Clever Acronym to Stop Racially Motivated 911 Calls?

\url{https://nyti.ms/3fXY3zM}

\begin{itemize}
\item
\item
\item
\item
\item
\end{itemize}

Advertisement

\protect\hyperlink{after-top}{Continue reading the main story}

Supported by

\protect\hyperlink{after-sponsor}{Continue reading the main story}

\hypertarget{will-it-take-a-clever-acronym-to-stop-racially-motivated-911-calls}{%
\section{Will It Take a Clever Acronym to Stop Racially Motivated 911
Calls?}\label{will-it-take-a-clever-acronym-to-stop-racially-motivated-911-calls}}

A San Francisco politician thinks a new bill --- named the ``CAREN Act''
--- could be what it takes to get people's attention, and limit
potential harm.

\includegraphics{https://static01.graylady3jvrrxbe.onion/images/2020/07/17/us/politics/00carenact1/merlin_174335079_c1ee386b-c5b9-4f6e-b7be-1d9d15246dd8-articleLarge.jpg?quality=75\&auto=webp\&disable=upscale}

By Evan Nicole Brown

\begin{itemize}
\item
  July 24, 2020
\item
  \begin{itemize}
  \item
  \item
  \item
  \item
  \item
  \end{itemize}
\end{itemize}

The stories showed up in Shamann Walton's social media feeds with
alarming regularity: An eight-year-old ****
\href{https://www.theguardian.com/us-news/2018/jun/25/permit-patty-eight-year-old-selling-water-san-francisco-video}{selling
water} to raise money to go to Disneyland. A family
\href{https://www.newsweek.com/lake-merritt-bbq-barbecue-video-oakland-racist-charcoal-east-bay-black-family-919355}{barbecuing}
food on a charcoal grill near Lake Merritt. A man
\href{https://www.nytimes3xbfgragh.onion/2020/06/14/nyregion/central-park-amy-cooper-christian-racism.html}{bird
watching} in Central Park. A woman sitting, leisurely, on a
\href{https://ny.eater.com/2020/6/2/21278346/french-restaurant-maison-vivienne-under-fire-calls-to-police}{public
bench} in Manhattan. All of these activities --- and several more like
them --- had been reported to law enforcement, often by strangers
passing by.

In elementary school, these 911-dialers might have been labeled a
tattletale. Today, a more widely used term is
``\href{https://theconversation.com/how-karen-went-from-a-popular-baby-name-to-a-stand-in-for-white-entitlement-139644}{Karen}'':
It's internet shorthand for a white woman who embodies entitled
\emph{``Can-I-speak-to-the-manager?''} energy, cloaking prejudice in
feigned innocence and concern.

Mr. Walton, who is a Democrat on the San Francisco board of supervisors,
is well aware that this behavior can lead to real consequences for Black
Americans: They are disproportionately likely to be killed by the
police, according to analysis by the
\href{https://www.washingtonpost.com/graphics/investigations/police-shootings-database/}{news
media} and
\href{https://www.nature.com/articles/d41586-019-02601-9}{academic
research}.

Enter Karen. Or CAREN. Mr. Walton recently proposed a new ordinance,
named the ``CAREN Act,'' to discourage and penalize people for making
racially motivated 911 calls without reasonable suspicion of a crime.

Most municipal bills are a little less catchy, burdened by phrases like
``planning code.'' But Mr. Walton's proposal speaks to the meme-ified
trope those who are online can readily identify.

He wouldn't comment specifically on the ordinance's wink of a name ---
which stands for Caution Against Racially Exploitative Non-Emergencies
--- or talk about how many drafts he went through before landing on this
acronym. He did note the dangers represented by the ``Karen'' meme, and
this ordinance certainly gathered more national attention than other
board of supervisors business, whether that was the intention or not.

It's one of several examples across the country of local political
leaders trying to limit racially motivated calls to the police. Last
year the city commission in Grand Rapids, Mich.,
\href{https://www.mlive.com/news/grand-rapids/2020/07/grand-rapids-already-has-a-caren-act-outlawing-racially-biased-911-calls.html}{unanimously
approved} an ordinance against discrimination in city housing and
employment programs. The statute also bans racial profiling in 911
calls. Similarly,
\href{https://www.opb.org/news/article/oregon-senate-passes-bill-prevents-racially-motivated-911-calls/}{in
Oregon} last year, the Senate passed a bill that allows people to sue if
they have had the police called on them as a result of discrimination.

Though critics say this type of legislation runs the risk of deterring
people from calling law enforcement even in the face of real danger,
supporters of the bills say they are designed to make crime reporting
more accurate and fair --- which ultimately saves officers' time and the
city money.

Beyond the CAREN Act, Mr. Walton's office is introducing several other
pieces of legislation focused on equity, including a charter for
independent oversight of the city's Sheriff's Department. The
non-emergency calls ordinance has nine sponsors, which means it will
almost certainly be endorsed by the board and make its way to Mayor
London Breed, which could happen within the next two months.

The Times recently spoke with Mr. Walton about his hopes for the piece
of legislation and how it could help reduce the number of frivolous 911
calls and citizen-officer interactions that result in brutality and
death. The interview has been lightly edited and condensed.

\includegraphics{https://static01.graylady3jvrrxbe.onion/images/2020/07/17/us/politics/00carenact2/merlin_173788140_b5795eba-974a-41a4-bbf7-d16cb0063d1b-articleLarge.jpg?quality=75\&auto=webp\&disable=upscale}

\textbf{What motivated you to propose the CAREN Act?}

People have to understand that if you call a police officer on a Black
person or a person of color it could lead to harm and possibly death. So
we need to make sure that if you're going to contact police its because
you really are being threatened, but we should not be calling police
because someone is writing ``Black Lives Matter'' on their own home, we
should not be calling police because a Black person is watching birds
and you don't feel they belong and shouldn't be there, or because
someone is barbecuing in a park and that's bothering you.

\textbf{In terms of getting this to gain traction, did you feel that it
was helpful to play off the popularity of the term ``Karen'' that we're
seeing all over social media? I'm curious about that tongue-in-cheek
approach to naming the issue.}

``Caution Against Racially Exploitative Non-Emergencies" calls to law
enforcement is just that. It's not directed toward any person or any
human being, we just came up with the acronym that worked for the type
of law that we think needs to be passed.

\textbf{Do you think people are making more unnecessary emergency calls
than they previously have?}

I've experienced people calling the police on me for things that were
not worthy at points in my life. It's not new, its just being caught on
camera and people are embarrassed about the things that they've done.
And I don't even think it's the embarrassment, I really think it's the
fact that people are losing their jobs because they're trying to
weaponize police officers against the Black community and people of
color.

It's a phenomenon that's been going on for a while. We can go back to
Emmett Till to see how false reporting can lead to death for Black
people.

\textbf{What legislative progress do you hope to make in this moment?}

We've proposed a resolution with the District Attorney here in San
Francisco that says our civil service commission should never hire
anyone in law enforcement from another city that has a lot of excessive
force complaints and racial profiling complaints. The mayor and I and
all my colleagues are redirecting resources from the police department
and investing in the Black community so we can address some of the
systemic issues that have led to negative outcomes.

\textbf{What do you hope to see come from this bill?}

In San Francisco the penalty is: if you contact law enforcement and
there's some harm brought to somebody they can file civilly and reap the
benefits for at least \$1,000 --- and it could be more than that based
on the type of damage and what's awarded. But we are also focusing on
some type of fine for folks who make those phone calls arbitrarily.

My ultimate goal is to make sure we have ordinances like this on the
books across the country, and to make sure that people don't do this
because, again, this is not a joke, it's not a game, people have
literally been killed by police officers because of arbitrary calls to
law enforcement.

Advertisement

\protect\hyperlink{after-bottom}{Continue reading the main story}

\hypertarget{site-index}{%
\subsection{Site Index}\label{site-index}}

\hypertarget{site-information-navigation}{%
\subsection{Site Information
Navigation}\label{site-information-navigation}}

\begin{itemize}
\tightlist
\item
  \href{https://help.nytimes3xbfgragh.onion/hc/en-us/articles/115014792127-Copyright-notice}{©~2020~The
  New York Times Company}
\end{itemize}

\begin{itemize}
\tightlist
\item
  \href{https://www.nytco.com/}{NYTCo}
\item
  \href{https://help.nytimes3xbfgragh.onion/hc/en-us/articles/115015385887-Contact-Us}{Contact
  Us}
\item
  \href{https://www.nytco.com/careers/}{Work with us}
\item
  \href{https://nytmediakit.com/}{Advertise}
\item
  \href{http://www.tbrandstudio.com/}{T Brand Studio}
\item
  \href{https://www.nytimes3xbfgragh.onion/privacy/cookie-policy\#how-do-i-manage-trackers}{Your
  Ad Choices}
\item
  \href{https://www.nytimes3xbfgragh.onion/privacy}{Privacy}
\item
  \href{https://help.nytimes3xbfgragh.onion/hc/en-us/articles/115014893428-Terms-of-service}{Terms
  of Service}
\item
  \href{https://help.nytimes3xbfgragh.onion/hc/en-us/articles/115014893968-Terms-of-sale}{Terms
  of Sale}
\item
  \href{https://spiderbites.nytimes3xbfgragh.onion}{Site Map}
\item
  \href{https://help.nytimes3xbfgragh.onion/hc/en-us}{Help}
\item
  \href{https://www.nytimes3xbfgragh.onion/subscription?campaignId=37WXW}{Subscriptions}
\end{itemize}
