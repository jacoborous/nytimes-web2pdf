Sections

SEARCH

\protect\hyperlink{site-content}{Skip to
content}\protect\hyperlink{site-index}{Skip to site index}

\href{https://www.nytimes3xbfgragh.onion/section/business}{Business}

\href{https://myaccount.nytimes3xbfgragh.onion/auth/login?response_type=cookie\&client_id=vi}{}

\href{https://www.nytimes3xbfgragh.onion/section/todayspaper}{Today's
Paper}

\href{/section/business}{Business}\textbar{}Goldman Sachs and Malaysia
Reach \$3.9 Billion Settlement in 1MDB Scandal

\url{https://nyti.ms/2OTaUHG}

\begin{itemize}
\item
\item
\item
\item
\item
\end{itemize}

Advertisement

\protect\hyperlink{after-top}{Continue reading the main story}

Supported by

\protect\hyperlink{after-sponsor}{Continue reading the main story}

\hypertarget{goldman-sachs-and-malaysia-reach-39-billion-settlement-in-1mdb-scandal}{%
\section{Goldman Sachs and Malaysia Reach \$3.9 Billion Settlement in
1MDB
Scandal}\label{goldman-sachs-and-malaysia-reach-39-billion-settlement-in-1mdb-scandal}}

The deal settles charges in Malaysia against the Wall Street bank for
its role in helping to raise hundreds of millions for a sovereign wealth
fund that was used as a personal piggy bank.

\includegraphics{https://static01.graylady3jvrrxbe.onion/images/2020/07/24/business/241MDB01/merlin_174583317_78935c32-fb4e-45a3-93b7-4836ee852210-articleLarge.jpg?quality=75\&auto=webp\&disable=upscale}

\href{https://www.nytimes3xbfgragh.onion/by/alexandra-stevenson}{\includegraphics{https://static01.graylady3jvrrxbe.onion/images/2018/02/20/multimedia/author-alexandra-stevenson/author-alexandra-stevenson-thumbLarge.jpg}}\href{https://www.nytimes3xbfgragh.onion/by/matthew-goldstein}{\includegraphics{https://static01.graylady3jvrrxbe.onion/images/2018/11/06/multimedia/author-matthew-goldstein/author-matthew-goldstein-thumbLarge.png}}

By
\href{https://www.nytimes3xbfgragh.onion/by/alexandra-stevenson}{Alexandra
Stevenson} and
\href{https://www.nytimes3xbfgragh.onion/by/matthew-goldstein}{Matthew
Goldstein}

\begin{itemize}
\item
  July 24, 2020
\item
  \begin{itemize}
  \item
  \item
  \item
  \item
  \item
  \end{itemize}
\end{itemize}

Goldman Sachs has agreed to a \$3.9 billion settlement with Malaysia as
it begins to put behind it a kleptocracy scandal that changed the course
of politics in the country.

Malaysian prosecutors filed charges in 2018 against several Goldman
units for their role in helping to raise billions of dollars for a
sovereign wealth fund, known as 1MDB, that officials were later found to
be using as a personal piggy bank. The scandal led to the
\href{https://www.nytimes3xbfgragh.onion/2018/05/15/world/asia/malaysia-najib-razak-fall.html}{ouster
of Malaysia's prime minister}, Najib Razak, and a far-reaching foreign
bribery and corruption investigation by U.S. prosecutors against the
bank and the purported mastermind of the scheme, the Malaysian financier
Jho Low.

Malaysia said on Friday that the Wall Street bank would pay \$2.5
billion to resolve the case. Goldman also pledged to cover any shortfall
from the sale of \$1.4 billion in assets that have been
\href{https://www.nytimes3xbfgragh.onion/2018/12/09/arts/jho-low-1mdb-assets-piano.html}{seized
by government authorities}, including a \$250 million yacht,
\href{https://www.nytimes3xbfgragh.onion/2020/07/03/business/viceroy-beverly-hills-1MDB-fraud.html}{several
hotels in the United States}, a \$35 million Bombardier jet and an Oscar
that once belonged to Marlon Brando.

``This settlement represents assets that rightfully belong to the
Malaysian people,'' the country's Ministry of Finance said in a
statement on Friday evening.

``We are confident that we are securing more money from Goldman Sachs
compared to previous attempts, which were far below expectations,'' it
said. The Malaysian government had
\href{https://www.nytimes3xbfgragh.onion/2018/12/17/business/goldman-sachs-malaysia-criminal-charges-1mdb.html}{previously
said it would seek criminal fines} in excess of \$2.7 billion and had
charged more than a dozen executives at the bank with fraud. Under the
settlement, the criminal charges against Goldman and those executives
were dismissed.

Goldman said in a statement the deal with Malaysia ``is an important
step towards putting the 1MDB matter behind us.''

But the bank still must still resolve the investigations by prosecutors
and bank regulators in the United States. Those negotiations began in
earnest in late 2018 and have proceeded in and fits and starts --- and
in recent months been delayed by the coronavirus pandemic.

Goldman, one of the most powerful firms on Wall Street, could have to
plead guilty to charges in the United States and pay another
multibillion-dollar fine. Federal authorities have also been talking to
Goldman about appointing a monitor to review its compliance procedures.

In
\href{https://www.nytimes3xbfgragh.onion/2020/06/11/business/goldman-sachs-1mdb-malaysia.html}{recent
weeks, Goldman has lobbied the Justice Department} to limit the
penalties it must face. The bank has asked prosecutors in Washington to
consider the penalties it pays to Malaysia when calculating its fine. It
has also sought to avoid a guilty plea by one of the bank's
subsidiaries, which would be a black eye for a firm that has never had
to admit guilt in a federal investigation. The U.S. attorney's office in
Brooklyn, which is overseeing the case, declined to comment.

Talks between prosecutors and the bank are continuing, according to a
person familiar with the matter. The person added that a framework for a
deal had been reached but not finalized.

Prosecutors in the United States
\href{https://www.justice.gov/opa/pr/us-seeks-recover-approximately-540-million-obtained-corruption-involving-malaysian-sovereign}{contend
that as much as \$4.5 billion} was pilfered from the sovereign wealth
fund --- officially known as 1Malaysia Development Berhad --- into the
bank accounts of Mr. Low, Mr. Najib, his family and his friends. Goldman
Sachs helped the fund raise \$6.5 billion in 2012 and 2013 through a
series of bond sales, \$2.5 billion of which authorities say was then
diverted to senior officials.

In 2018, prosecutors in the United States charged Mr. Low with money
laundering and other crimes in connection with the looting of the fund.
Mr. Low has never appeared in federal court in Brooklyn and is believed
to be hiding in China.

Mr. Low has denied he did anything wrong, but last year he agreed to
relinquish any claim to more than \$900 million in seized assets. Many
of those assets --- including the rights to the movie ``The Wolf of Wall
Street'' --- have been sold and the proceeds returned to Malaysia.

Federal prosecutors also secured a guilty plea from Tim Leissner, a
former top Goldman partner in Asia, who said he and others at Goldman
had conspired to circumvent the bank's internal controls to work with
Mr. Low to bribe officials in Malaysia to secure the lucrative bond work
for the bank.

Mr. Leissner, the husband of the fashion designer and model Kimora Lee
Simmons, personally profited from the scheme and agreed to forfeit up to
\$43.7 million. He is cooperating with the investigation. Federal
prosecutors also charged another former Goldman banker, Roger Ng, who
has pleaded not guilty and is set to go on trial next year.

Goldman Sachs, which received \$600 million in payments for its bond
work, has consistently denied any wrongdoing. But top officials have
repeatedly apologized for actions of Mr. Leissner, who they contended
acted without the bank's approval.

The settlement ``will help enable the Malaysian government to move
forward with additional recovery efforts and to execute on its economic
priorities,'' the bank
\href{https://www.goldmansachs.com/media-relations/press-releases/current/statement-24-july-2020.html}{said
in a statement}. ``There are important lessons to be learned from this
situation, and we must be self-critical to ensure that we only improve
from the experience.''

The investigation by federal prosecutors has been a major blow to
Goldman's reputation. It revealed that some within the bank were wary of
Mr. Low, a flamboyant businessman who had befriended many Hollywood
celebrities and was known for staging wild and extravagant parties in
Las Vegas. Some in the bank's compliance department were concerned about
Mr. Low because it was unclear how he had amassed his wealth.

Yet Mr. Low still had a way of getting close to officials at the bank.

In December 2012, Lloyd C. Blankfein, Goldman's chairman and chief
executive at the time,
\href{https://www.nytimes3xbfgragh.onion/2018/11/22/business/goldman-blankfein-1mdb-malaysia.html}{met
privately with Mr. Low} at Goldman's offices in New York, just weeks
before the bank arranged the third bond deal for the Malaysian fund. Two
people familiar with the meeting previously described it as a one-on-one
encounter; Goldman has said at least one more person attended and there
was no discussion of 1MDB.

Advertisement

\protect\hyperlink{after-bottom}{Continue reading the main story}

\hypertarget{site-index}{%
\subsection{Site Index}\label{site-index}}

\hypertarget{site-information-navigation}{%
\subsection{Site Information
Navigation}\label{site-information-navigation}}

\begin{itemize}
\tightlist
\item
  \href{https://help.nytimes3xbfgragh.onion/hc/en-us/articles/115014792127-Copyright-notice}{©~2020~The
  New York Times Company}
\end{itemize}

\begin{itemize}
\tightlist
\item
  \href{https://www.nytco.com/}{NYTCo}
\item
  \href{https://help.nytimes3xbfgragh.onion/hc/en-us/articles/115015385887-Contact-Us}{Contact
  Us}
\item
  \href{https://www.nytco.com/careers/}{Work with us}
\item
  \href{https://nytmediakit.com/}{Advertise}
\item
  \href{http://www.tbrandstudio.com/}{T Brand Studio}
\item
  \href{https://www.nytimes3xbfgragh.onion/privacy/cookie-policy\#how-do-i-manage-trackers}{Your
  Ad Choices}
\item
  \href{https://www.nytimes3xbfgragh.onion/privacy}{Privacy}
\item
  \href{https://help.nytimes3xbfgragh.onion/hc/en-us/articles/115014893428-Terms-of-service}{Terms
  of Service}
\item
  \href{https://help.nytimes3xbfgragh.onion/hc/en-us/articles/115014893968-Terms-of-sale}{Terms
  of Sale}
\item
  \href{https://spiderbites.nytimes3xbfgragh.onion}{Site Map}
\item
  \href{https://help.nytimes3xbfgragh.onion/hc/en-us}{Help}
\item
  \href{https://www.nytimes3xbfgragh.onion/subscription?campaignId=37WXW}{Subscriptions}
\end{itemize}
