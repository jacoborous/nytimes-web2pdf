Sections

SEARCH

\protect\hyperlink{site-content}{Skip to
content}\protect\hyperlink{site-index}{Skip to site index}

\href{https://www.nytimes3xbfgragh.onion/section/world/africa}{Africa}

\href{https://myaccount.nytimes3xbfgragh.onion/auth/login?response_type=cookie\&client_id=vi}{}

\href{https://www.nytimes3xbfgragh.onion/section/todayspaper}{Today's
Paper}

\href{/section/world/africa}{Africa}\textbar{}Benjamin Mkapa,
Ex-President of Tanzania, Dies at 81

\url{https://nyti.ms/30I7ZHp}

\begin{itemize}
\item
\item
\item
\item
\item
\end{itemize}

Advertisement

\protect\hyperlink{after-top}{Continue reading the main story}

Supported by

\protect\hyperlink{after-sponsor}{Continue reading the main story}

\hypertarget{benjamin-mkapa-ex-president-of-tanzania-dies-at-81}{%
\section{Benjamin Mkapa, Ex-President of Tanzania, Dies at
81}\label{benjamin-mkapa-ex-president-of-tanzania-dies-at-81}}

Mr. Mkapa served two terms, from 1995 to 2005. The third president of
his country since independence, he was the first democratically elected
leader after its transition to multiparty politics.

\includegraphics{https://static01.graylady3jvrrxbe.onion/images/2020/07/26/obituaries/26Tanzania-obit1/24Tanzania01-articleLarge.jpg?quality=75\&auto=webp\&disable=upscale}

By \href{https://www.nytimes3xbfgragh.onion/by/abdi-latif-dahir}{Abdi
Latif Dahir}

\begin{itemize}
\item
  Published July 24, 2020Updated July 25, 2020
\item
  \begin{itemize}
  \item
  \item
  \item
  \item
  \item
  \end{itemize}
\end{itemize}

NAIROBI, Kenya --- Benjamin Mkapa, the third president of Tanzania and
the leader of the country during a crucial period of democratic
transition, died on Friday at a hospital in the port city of Dar es
Salaam. He was 81.

His death was
\href{https://twitter.com/MsigwaGerson/status/1286429773590343680}{announced
by President John Magufuli}. Mr. Magufuli did not give a cause of death,
but he
\href{https://twitter.com/MsigwaGerson/status/1286444641743757321}{declared
seven days of mourning} during which flags would fly at half-staff
across the country.

``I will remember him for his great love for the nation, his piety, hard
work and his efforts in building the economy,'' Mr. Magufuli said in
\href{https://twitter.com/MagufuliJP/status/1286433004513038337}{a
message posted on Twitter}. ``Certainly, the nation has lost a strong
pillar.''

Mr. Mkapa was the president of Tanzania from November 1995 to December
2005 and was
\href{https://www.nytimes3xbfgragh.onion/1995/10/29/world/a-joyful-but-anxious-vote-in-tanzania.html}{the
first leader elected after the return of multiparty politics} in 1992.
During his tenure, he played a central role in helping the country
transition from a socialist system of development --- popularly known as
ujamaa --- into a free-market economy.

He overhauled the largely ineffective public sector, privatized
state-owned corporations, widened the tax collection base, secured
international debt relief and helped incentivize the growth of the
private sector.

``We recognize that the government has no business on the eve of the
21st century to be in business,''
\href{https://commons.wmu.se/cgi/viewcontent.cgi?article=1315\&context=all_dissertations\#page=52}{he
said a few weeks after coming to power}.

``By withdrawing from direct production activities, the government will
have more resources with which to strengthen essential social services,
economic infrastructure, human capital development and performance of
the traditional roles of state in particular the maintenance of law and
order,'' he added.

While Mr. Mkapa promised to wage war on graft during his campaign, his
10-year reign was dogged by allegations of corruption and mismanagement
of public funds. This was especially true of the mining sector, where
Mr. Mkapa was
\href{https://allafrica.com/stories/201905130115.html}{accused of
overseeing contracts that benefited him} and his family that were signed
during his administration.

Mr. Mkapa
\href{https://www.thecitizen.co.tz/news/Former-President-Benjamin-Mkapa-bares-it-all-his-memoir--/1840340-5347096-3201j8/index.html}{denied
the allegations}, but successive governments, all of them controlled by
the same party, have been accused of an unwillingness to investigate Mr.
Mkapa and his inner circle, while
\href{https://www.thecitizen.co.tz/news/-Tabloid-banned-for-2-years-for-linking-former-presidents-/1840340-3972404-w40es0/index.html}{newspapers
that mounted investigations were banned}. Mr. Mkapa was also criticized
for \href{http://news.bbc.co.uk/2/hi/africa/3719712.stm}{spending
millions on a presidential jet} even as the country struggled to reduce
poverty levels.

Before becoming president, Mr. Mkapa worked in various capacities in
government for almost three decades. In the 1970s, he was employed as
the presidential information officer for
\href{https://www.nytimes3xbfgragh.onion/1999/10/15/world/julius-nyerere-of-tanzania-dies-preached-african-socialism-to-the-world.html}{Julius
Nyerere}, the founding father of modern Tanzania.

Mr. Mkapa later became a lawmaker, and he served in cabinet positions
including minister of foreign affairs, minister of information and
culture, and minister of science, technology and higher education. He
also served as ambassador to countries including Canada, India, Nigeria
and the United States.

Earlier in his career, Mr. Mkapa was a prominent journalist. He was the
executive editor of the English-language newspaper The Daily News and,
in 1976, the founding editor of the Tanzania News Agency.

Benjamin William Mkapa was born on Nov. 12, 1938, in the town of Masasi,
in the Mtwara region of southern Tanzania. After attending schools in
Tanzania from 1945 to 1956, he completed a major in English language at
Makerere University in Uganda in 1962. He also attended Columbia
University in New York and received a master's degree in international
affairs there in 1963.

Mr. Mkapa is survived by his wife, Anna Mkapa, and two children.

After he left office, Mr. Mkapa worked with a number of global
organizations responding to political, economic and social crises. He
was on the board of the International Crisis Group, participated in a
United Nations panel on trade and development and was the chairman of a
team sent by Ban Ki-moon, then the U.N. secretary general, to monitor a
2010 referendum on independence for South Sudan.

Mr. Mkapa also set up \href{https://mkapafoundation.or.tz/}{a foundation
in his name} that worked closely with the Clinton Foundation to improve
maternal health and child care services, and to prevent the spread of
H.I.V. and AIDS in Tanzania.

Tributes to Mr. Mkapa poured in from across the world on Friday. In
neighboring Kenya, leaders remembered his contribution to mediation
efforts during
\href{https://www.nytimes3xbfgragh.onion/2008/02/06/world/africa/06kenya.html}{the
turbulent postelection violence in 2008}.

President Uhuru Kenyatta of Kenya sent a message of condolence,
\href{https://twitter.com/StateHouseKenya/status/1286535803238309890}{remembering
Mr. Mkapa} as ``an outstanding East African who worked tirelessly for
the integration, peace and progress of the region.''

Advertisement

\protect\hyperlink{after-bottom}{Continue reading the main story}

\hypertarget{site-index}{%
\subsection{Site Index}\label{site-index}}

\hypertarget{site-information-navigation}{%
\subsection{Site Information
Navigation}\label{site-information-navigation}}

\begin{itemize}
\tightlist
\item
  \href{https://help.nytimes3xbfgragh.onion/hc/en-us/articles/115014792127-Copyright-notice}{©~2020~The
  New York Times Company}
\end{itemize}

\begin{itemize}
\tightlist
\item
  \href{https://www.nytco.com/}{NYTCo}
\item
  \href{https://help.nytimes3xbfgragh.onion/hc/en-us/articles/115015385887-Contact-Us}{Contact
  Us}
\item
  \href{https://www.nytco.com/careers/}{Work with us}
\item
  \href{https://nytmediakit.com/}{Advertise}
\item
  \href{http://www.tbrandstudio.com/}{T Brand Studio}
\item
  \href{https://www.nytimes3xbfgragh.onion/privacy/cookie-policy\#how-do-i-manage-trackers}{Your
  Ad Choices}
\item
  \href{https://www.nytimes3xbfgragh.onion/privacy}{Privacy}
\item
  \href{https://help.nytimes3xbfgragh.onion/hc/en-us/articles/115014893428-Terms-of-service}{Terms
  of Service}
\item
  \href{https://help.nytimes3xbfgragh.onion/hc/en-us/articles/115014893968-Terms-of-sale}{Terms
  of Sale}
\item
  \href{https://spiderbites.nytimes3xbfgragh.onion}{Site Map}
\item
  \href{https://help.nytimes3xbfgragh.onion/hc/en-us}{Help}
\item
  \href{https://www.nytimes3xbfgragh.onion/subscription?campaignId=37WXW}{Subscriptions}
\end{itemize}
