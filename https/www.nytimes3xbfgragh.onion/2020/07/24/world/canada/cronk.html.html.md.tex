Sections

SEARCH

\protect\hyperlink{site-content}{Skip to
content}\protect\hyperlink{site-index}{Skip to site index}

\href{https://www.nytimes3xbfgragh.onion/section/world/canada}{Canada}

\href{https://myaccount.nytimes3xbfgragh.onion/auth/login?response_type=cookie\&client_id=vi}{}

\href{https://www.nytimes3xbfgragh.onion/section/todayspaper}{Today's
Paper}

\href{/section/world/canada}{Canada}\textbar{}Cracking Open a Bottle of
Calgary's Past

\url{https://nyti.ms/32SdCWg}

\begin{itemize}
\item
\item
\item
\item
\item
\end{itemize}

Advertisement

\protect\hyperlink{after-top}{Continue reading the main story}

Supported by

\protect\hyperlink{after-sponsor}{Continue reading the main story}

CANADA LETTER

\hypertarget{cracking-open-a-bottle-of-calgarys-past}{%
\section{Cracking Open a Bottle of Calgary's
Past}\label{cracking-open-a-bottle-of-calgarys-past}}

One man's social media post of archived 19th-century newspaper ads has
revived interest in a long forgotten drink.

\href{https://www.nytimes3xbfgragh.onion/by/ian-austen}{\includegraphics{https://static01.graylady3jvrrxbe.onion/images/2019/07/18/reader-center/author-ian-austen/author-ian-austen-thumbLarge.png}}

By \href{https://www.nytimes3xbfgragh.onion/by/ian-austen}{Ian Austen}

\begin{itemize}
\item
  July 24, 2020
\item
  \begin{itemize}
  \item
  \item
  \item
  \item
  \item
  \end{itemize}
\end{itemize}

For part of each week, I help put together The Times's
\href{https://www.nytimes3xbfgragh.onion/news-event/coronavirus}{live
briefing on the pandemic}, which continues to
\href{https://www.nytimes3xbfgragh.onion/interactive/2020/07/23/us/coronavirus-hotspots-countries.html}{ravage
the United States} and cause
\href{https://www.cbc.ca/news/politics/tam-young-people-covid19-virus-fatigue-1.5662206}{worrying
outbreaks} in Canada, where things are otherwise significantly better.

But the pandemic isn't all grim destruction. It's inspiring new
interests and resurrecting some too. That's currently the case in
Calgary, where one man's fascination with a remnant of the city's past
has captured its imagination.

It began late last month when Paul Fairie's plans for a walk were
thwarted by rain. Stuck inside, Mr. Fairie, a researcher and political
science instructor at the University of Calgary, revisited one of his
hobbies: reading old newspapers online.

The front page of the Calgary Weekly Herald from
\href{http://peel.library.ualberta.ca/newspapers/CWH/1883/09/28/1/Ar00103.html}{September
28, 1883} started off its local news section with a cryptic entry of one
simple word: ``Cronk.''

Image

Cronk ads in the Calgary Weekly Herald in 1883Credit...University of
Alberta Libraries

The mystery didn't stop there. Sprinkled throughout the front page ---
like the repetitive ads that clutter 21st-century web pages --- there
were several more entries, including ``Dr. Cronk,'' ``Cronk is Good,''
``Buy Cronk,'' ``Cronk is the drink'' and ``Cronk is made at the Star
Bakery.''

Dr. Fairie had come across very brief ads inserted into articles many
times before. But these one-word ads intrigued him.

``What really drew my eye was the one that was just `Cronk','' he told
me. ``I was like, This really isn't enough detail at all.''

Dr. Fairie doesn't just read old newspapers, he also frequently
\href{https://twitter.com/paulisci}{posts oddities from them on
Twitter.} They have included a
\href{https://twitter.com/paulisci/status/918220378685759488?s=20}{salad
recipe that uses four doughnuts} as its principal ingredient.

So
\href{https://twitter.com/emollick/status/1282802428652466180?s=20}{onto
Twitter went Cronk}.

``I thought people would like it and that would be the end of it in an
hour,'' Dr. Fairie said. He had badly underestimated the power of Cronk.

First, Dr. Fairie started receiving bits of Cronk lore from antique
bottle collectors, an avid group that was well-versed in Cronk packaging
if not the beverage itself.

Someone sent Dr. Fairie a recipe for
\href{https://twitter.com/JNerissa/status/1275321511646126080?s=20}{``Dr.
Cronk's Sarsaparilla Beer''}through Twitter. He didn't make it.
Requiring 100 gallons of water, the recipe was for Cronk on an
industrial scale rather than for anyone's kitchen.

\includegraphics{https://static01.graylady3jvrrxbe.onion/images/2020/07/24/world/24canadaletter-fairie/24canadaletter-fairie-articleLarge.jpg?quality=75\&auto=webp\&disable=upscale}

But as the Cronk discussion grew this month, Dr. Fairie made it his
pandemic project as he began piecing together all he could find out
about the long-forgotten drink.

He put all of that into \href{https://youtu.be/4anHiXHyPVo}{a 16 minute
video} that some have praised as the finest YouTube offering by a
Canadian about a forgotten 19th-century beverage.

Here's the brief version: Cronk appears to have been created in 1839 by
Warren Cronk, a soda water maker from Albany, N.Y. Within nine years, he
had franchised his creation across much of North America. By the time
the enigmatic ads appeared in the Calgary Weekly Herald, Dr. Fairie said
that the company was most likely headed by another Warren Cronk,
probably the founder's son. The second Warren Cronk was also in the soda
water trade and lived mainly in my hometown, Windsor, Ontario, although
he sometimes bounced across the river to Detroit.

It's unclear if either of them had even the slightest claim to the title
of doctor.

You'll have to watch Dr. Fairie's video to learn about Cronk's role in
the
\href{https://www.neh.gov/humanities/2011/septemberoctober/statement/the-circus-you-never-knew}{Hippodrome
War of 1853.}There's not enough space here.

Calgary did indeed have a Star Bakery for about a decade and it marked
its loaves with stars. From what Dr. Fairie can gather, Frank Claxton,
the owner, was also an unsuccessful municipal political candidate who
eventually left town.

The original age of Cronk appears to have ended around 1910, more than a
century before the new one dawned.

Dr. Fairie's digital news clippings have led to the creation of a
\href{https://thebigsteak.com/products/cronk-is-good-t-shirt?variant=31874543878259}{Cronk}T-shirt,
which is being sold to raise money for a charity. Perhaps more
important, the drink is being resurrected. A microbrewery in Calgary's
Inglewood neighborhood is brewing 800 liters of it, using the somewhat
vague recipe Dr. Fairie received.

Blake Belding, the head brewer at \href{https://www.coldgarden.ca}{Cold
Garden Brewery,}told me that in his quest to revive the drink he will
have to use a substitute for sassafras and sarsaparilla. After buying
several pounds of the items, he found that
\href{https://www.mcgill.ca/oss/article/did-you-know/root-root-beer-sassafras}{food
safety laws now ban} them as possible carcinogens. A neighborhood spice
shop helped him concoct alternatives.

It will take another two to three weeks for the Cronk to fully ferment
to eliminate any danger that its bottles will explode, Mr. Belding said.

His preliminary taste test results are not wholly encouraging.

``It's not bad," he said. ``I don't know if it's something I would crush
a 12-pack of on the weekend.''

And what does Dr. Fairie hope to find when he cracks open his first
bottle of Cronk next month?

``I'm hoping it's either a 10 out of 10 or a zero out of 10,'' he said.
``The worst case scenario would be that it would be kind of boring.''

\begin{center}\rule{0.5\linewidth}{\linethickness}\end{center}

Cronk.

\begin{center}\rule{0.5\linewidth}{\linethickness}\end{center}

\hypertarget{trans-canada}{%
\subsection{Trans Canada}\label{trans-canada}}

Image

The Quebec government is looking into why Nathalie Bondil was let
go.Credit...D Dipasupil/FilmMagic, via Getty Images

\begin{itemize}
\item
  To the surprise of many in the art world, the longtime director of the
  Montreal Museum of Fine Arts was
  \href{https://www.nytimes3xbfgragh.onion/2020/07/22/arts/design/montreal-museum-nathalie-bondil.html}{dismissed
  by its board}. The debate over why Nathalie Bondil was let go has led
  to such confusion and rancor that the government has stepped in to
  investigate.
\item
  The Federal Court has ruled that a treaty with the United States that
  allows Canada to turn away asylum seekers coming from the United
  States if they entered there from a third country
  \href{https://www.nytimes3xbfgragh.onion/2020/07/22/world/canada/asylum-Safe-Third-Country-Agreement.html}{violates
  the constitution}, my colleague Dan Bilefsky wrote.
\item
  The decision by the N.H.L. to play out its season in Edmonton and
  Toronto is a return to the past for the New York Rangers.
  \href{https://www.nytimes3xbfgragh.onion/2020/07/19/sports/hockey/coronavirus-nhl-canada-season.html}{Gerald
  Eskenazi, a sports reporter at The Times for 41 years, writes}that not
  only was everyone on the team Canadian at one time, but the team used
  to hold its training camps in Kitchener, Ontario.
\item
  In June, Rev. Junia Joplin revealed a secret truth to her Baptist
  congregation in Mississauga, Ontario: She is a transgender woman.
  Christine Hauser reports that the congregation
  \href{https://www.nytimes3xbfgragh.onion/2020/07/23/world/canada/junia-joplin-transgender-lorne-park-baptist.html}{responded
  this week by firing her}.
\item
  The all-terrain buses that carry visitors to the Columbia Icefields in
  Jasper National Park are among Alberta's top attractions. Last
  Saturday,
  \href{https://www.nytimes3xbfgragh.onion/2020/07/18/world/canada/bus-crash-glacier-jasper-alberta.html}{one
  of them became deadly.}
\end{itemize}

\begin{center}\rule{0.5\linewidth}{\linethickness}\end{center}

\emph{A native of Windsor, Ontario, Ian Austen was educated in Toronto,
lives in Ottawa and has reported about Canada for The New York Times for
the past 16 years. Follow him on Twitter at @ianrausten.}

\begin{center}\rule{0.5\linewidth}{\linethickness}\end{center}

\hypertarget{how-are-we-doing}{%
\subsubsection{\texorpdfstring{\textbf{How are we
doing?}}{How are we doing?}}\label{how-are-we-doing}}

We're eager to have your thoughts about this newsletter and events in
Canada in general. Please send them to
\href{mailto:nytcanada@NYTimes.com?\%20subject=Canada\%20Letter\%20Newsletter\%20Feedback}{nytcanada@NYTimes.com}.

\hypertarget{like-this-email}{%
\subsubsection{\texorpdfstring{\textbf{Like this
email?}}{Like this email?}}\label{like-this-email}}

Forward it to your friends, and let them know they can sign up
\href{https://www.nytimes3xbfgragh.onion/newsletters/canada-letter?smid=nytemail\&smvar=canadaletter\&te=1\&nl=canada-today\&emc=edit_cnda_20190622}{here}.

Advertisement

\protect\hyperlink{after-bottom}{Continue reading the main story}

\hypertarget{site-index}{%
\subsection{Site Index}\label{site-index}}

\hypertarget{site-information-navigation}{%
\subsection{Site Information
Navigation}\label{site-information-navigation}}

\begin{itemize}
\tightlist
\item
  \href{https://help.nytimes3xbfgragh.onion/hc/en-us/articles/115014792127-Copyright-notice}{©~2020~The
  New York Times Company}
\end{itemize}

\begin{itemize}
\tightlist
\item
  \href{https://www.nytco.com/}{NYTCo}
\item
  \href{https://help.nytimes3xbfgragh.onion/hc/en-us/articles/115015385887-Contact-Us}{Contact
  Us}
\item
  \href{https://www.nytco.com/careers/}{Work with us}
\item
  \href{https://nytmediakit.com/}{Advertise}
\item
  \href{http://www.tbrandstudio.com/}{T Brand Studio}
\item
  \href{https://www.nytimes3xbfgragh.onion/privacy/cookie-policy\#how-do-i-manage-trackers}{Your
  Ad Choices}
\item
  \href{https://www.nytimes3xbfgragh.onion/privacy}{Privacy}
\item
  \href{https://help.nytimes3xbfgragh.onion/hc/en-us/articles/115014893428-Terms-of-service}{Terms
  of Service}
\item
  \href{https://help.nytimes3xbfgragh.onion/hc/en-us/articles/115014893968-Terms-of-sale}{Terms
  of Sale}
\item
  \href{https://spiderbites.nytimes3xbfgragh.onion}{Site Map}
\item
  \href{https://help.nytimes3xbfgragh.onion/hc/en-us}{Help}
\item
  \href{https://www.nytimes3xbfgragh.onion/subscription?campaignId=37WXW}{Subscriptions}
\end{itemize}
