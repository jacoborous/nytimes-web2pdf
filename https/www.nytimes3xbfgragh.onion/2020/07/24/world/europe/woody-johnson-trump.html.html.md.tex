Sections

SEARCH

\protect\hyperlink{site-content}{Skip to
content}\protect\hyperlink{site-index}{Skip to site index}

\href{https://www.nytimes3xbfgragh.onion/section/world/europe}{Europe}

\href{https://myaccount.nytimes3xbfgragh.onion/auth/login?response_type=cookie\&client_id=vi}{}

\href{https://www.nytimes3xbfgragh.onion/section/todayspaper}{Today's
Paper}

\href{/section/world/europe}{Europe}\textbar{}Woody Johnson Was a Loyal
Trump Supporter in 2016. As an Ambassador, He May Be Too Loyal.

\url{https://nyti.ms/2WTpVhj}

\begin{itemize}
\item
\item
\item
\item
\item
\end{itemize}

Advertisement

\protect\hyperlink{after-top}{Continue reading the main story}

Supported by

\protect\hyperlink{after-sponsor}{Continue reading the main story}

\hypertarget{woody-johnson-was-a-loyal-trump-supporter-in-2016-as-an-ambassador-he-may-be-too-loyal}{%
\section{Woody Johnson Was a Loyal Trump Supporter in 2016. As an
Ambassador, He May Be Too
Loyal.}\label{woody-johnson-was-a-loyal-trump-supporter-in-2016-as-an-ambassador-he-may-be-too-loyal}}

In the view of some American diplomats, Mr. Johnson's willingness to
carry out President Trump's request to seek the British government's
help in moving the British Open to his golf resort in Scotland was only
the latest example of the envoy's poor judgment.

\includegraphics{https://static01.graylady3jvrrxbe.onion/images/2020/07/24/us/politics/24dc-woody1/merlin_174791733_5a127cfe-929e-4b62-ad11-8555ab18c576-articleLarge.jpg?quality=75\&auto=webp\&disable=upscale}

\href{https://www.nytimes3xbfgragh.onion/by/mark-landler}{\includegraphics{https://static01.graylady3jvrrxbe.onion/images/2019/10/22/reader-center/author-mark-landler/author-mark-landler-thumbLarge-v3.png}}\href{https://www.nytimes3xbfgragh.onion/by/lara-jakes}{\includegraphics{https://static01.graylady3jvrrxbe.onion/images/2019/07/25/reader-center/author-lara-jakes/author-lara-jakes-thumbLarge.png}}\href{https://www.nytimes3xbfgragh.onion/by/maggie-haberman}{\includegraphics{https://static01.graylady3jvrrxbe.onion/images/2018/07/12/multimedia/author-maggie-haberman/author-maggie-haberman-thumbLarge.png}}

By \href{https://www.nytimes3xbfgragh.onion/by/mark-landler}{Mark
Landler}, \href{https://www.nytimes3xbfgragh.onion/by/lara-jakes}{Lara
Jakes} and
\href{https://www.nytimes3xbfgragh.onion/by/maggie-haberman}{Maggie
Haberman}

\begin{itemize}
\item
  July 24, 2020
\item
  \begin{itemize}
  \item
  \item
  \item
  \item
  \item
  \end{itemize}
\end{itemize}

LONDON --- Playing host at a small dinner on Tuesday night in honor of
Secretary of State Mike Pompeo, the American ambassador to Britain,
\href{https://www.nytimes3xbfgragh.onion/2020/07/25/sports/football/woody-johnson-trump-jets.html}{Robert
Wood Johnson} IV, told his guests that the wine was from President
Trump's vineyard in Virginia. He was serving it, he joked, even though
it might be ethically improper.

The next day, Mr. Johnson was not making any more jokes about ethics.
\href{https://twitter.com/USAmbUK/status/1286029006500966402?s=20}{On
Twitter}, he insisted he had ``followed the ethical rules and
requirements of my office at all times'' after
\href{https://www.nytimes3xbfgragh.onion/2020/07/21/world/europe/trump-british-open.html}{The
New York Times reported} that at the president's request, he had raised
with a British official the idea of steering the British Open golf
tournament to Mr. Trump's Turnberry golf resort in Scotland.

In the ranks of the American diplomatic corps, Mr. Johnson's enthusiasm
for pleasing Mr. Trump has raised questions about whether Mr. Johnson
--- a 73-year-old pharmaceutical heir, N.F.L. team owner and longtime
friend of the president's --- has put promoting his boss over his
diplomatic duties.

It has also deepened the misgivings of the London embassy's staff about
his judgment, given his reputation for off-color jokes and remarks to
subordinates that some said have crossed the line into sexism or racism.
His behavior has eroded morale among career diplomats and has surfaced
in a State Department inspector general's look at the embassy, the
results of which are in a report filed in February but not yet released.

Among the seven guests who attended the dinner for Mr. Pompeo were the
British foreign secretary, Dominic Raab, and the chancellor of the
Exchequer, Rishi Sunak, as well as three American banking executives. A
spokesman for the embassy said that Mr. Johnson paid for the wine
himself.

Still, said Lewis A. Lukens, who was Mr. Johnson's deputy until November
2018, ``Instances like this reinforce the image of an ambassador out of
touch with government ethics requirements and more interested in serving
the president's personal interests than representing the country
overseas.''

The president has denied that he asked Mr. Johnson to lobby the British
government to award the Open to his Turnberry resort, and the ambassador
has declined to address his role. But Mr. Lukens, drawing on notes he
made at the time, provided a detailed timeline of his repeated efforts
to persuade Mr. Johnson not to carry out Mr. Trump's directive.

Mr. Johnson, Mr. Lukens said, returned to London on Jan. 30, 2018, after
a one-on-one meeting with Mr. Trump in the Oval Office the day before.
The ambassador, who is known by his nickname
\href{https://www.nytimes3xbfgragh.onion/2020/07/25/sports/football/woody-johnson-trump-jets.html}{Woody},
told Mr. Lukens about the president's request and asked him to suggest a
British official he could contact about it.

``I advised Woody that he should not do this, that it would violate
ethical guidelines at the State Department,'' Mr. Lukens recalled.

That seemed to put the matter to rest, he said. But on Feb. 21, Mr.
Johnson again raised the president's request, asking his deputy: ```Who
should I talk to in the British government?''' Mr. Lukens recalled. ``I
said, `No one.'''

He then warned the ambassador, Mr. Lukens said, that beyond the ethical
and legal red flags raised by the conversation, it would be hugely
embarrassing to the embassy and the State Department if it ever leaked
out.

Mr. Johnson, he said, conceded the risks and appeared frustrated to be
put in a compromised position. He told Mr. Lukens he might call the
president's son Eric Trump, who oversees the family's golf resorts, to
see whether he could handle the contact with the British government. It
is not clear whether Mr. Johnson did that.

But a few days later, he held a meeting with Britain's secretary of
state for Scotland, David Mundell. Shortly after returning to the
embassy, an American political official who was in the room with Mr.
Johnson reported to Mr. Lukens that the ambassador had broached the idea
of holding the Open at Turnberry.

``If the president denies it,'' Mr. Lukens said, ``he's not being
truthful.''

A White House spokeswoman declined to comment on the new information, as
did Mr. Johnson. But the ambassador was far less equivocal about the
accusations that he made offensive comments to his staff.

``These false claims of insensitive remarks about race and gender are
totally inconsistent with my longstanding record and values,''
\href{https://twitter.com/USAmbUK/status/1286029006500966402?s=20}{he
said Wednesday on Twitter}.

But according to a half-dozen current and former embassy employees, Mr.
Johnson regularly made his female and Black staff members uncomfortable
with comments about their appearances or race. One Black female diplomat
told colleagues that Mr. Johnson disparaged her efforts to schedule
events for Black History Month, accusations that were
\href{https://www.cnn.com/2020/07/22/politics/woody-johnson-oig-report/index.html}{first
reported by CNN}.

Mr. Johnson, the diplomat said, once asked if he had to speak to an
audience that was ``just a bunch of Black people.'' He told the
diplomat, who later left the Foreign Service, that she was
``marginalizing'' herself. On the occasion of Martin Luther King's
Birthday, he asked what made the Rev. Dr. Martin Luther King Jr.
deserving of a holiday.

In June 2018, Mr. Johnson lashed out in anger after a visit by the
education secretary, Betsy DeVos, because his staff organized a
reception at his residence, Winfield House, that included teenagers,
some of whom were racial minorities who had won funding to make
recruiting visits to American universities. Mr. Johnson complained he
had been blindsided; some officials suspected he was uncomfortable with
the guests.

The ambassador's weekly senior staff meeting, which brought together the
heads of the embassy's departments as well as the C.I.A., the Department
of Homeland Security and other agencies with outposts in London, could
be particularly tense, according to four current and former staff
members. Mr. Johnson, they said, liked to open the proceedings with
lighthearted comments that often fell flat.

At one meeting, said a person who attended, Mr. Johnson singled out a
Black gunnery sergeant, who headed the Marine security detachment at the
embassy and was wearing black pants and a black polo shirt. ``Black ---
I like it,'' he said, drawing no reaction from the Marine but winces
from others in the room.

Two people close to Mr. Johnson said his comment stemmed from the fact
that the new uniforms of the N.F.L. team he owns, the New York Jets,
rolled out last year, were black. Others viewed the episode as evidence
of Mr. Johnson's awkwardness rather than racist intent, part of a
pattern of tin-eared attempts at humor.

Mr. Johnson regularly commented on the appearances of female staff
members, once pointing out to colleagues that he had seen one of the
women working out in the embassy's gym that morning. Though he actively
recruited women for jobs in the embassy, his actions had the effect of
excluding them, **** according to several diplomats.

Soon after arriving in London, Mr. Johnson joined an exclusive men's
club, White's, and began holding business lunches there. Because White's
does not allow women, he could not bring the embassy's political affairs
counselor, Jennifer Gavito, and invited her male deputy instead. After
female employees raised it with the State Department, Mr. Lukens urged
Mr. Johnson to use the club only for social occasions.

Whatever his troubles, Mr. Johnson's relationship with the president
appears to remain intact. And so far in this election cycle, he has
given \$1.2 million to the Republican National Committee and the Trump
Victory fund, as well as another \$1 million to America First Action, a
super PAC supporting Mr. Trump's re-election.

But as the owner of the Jets, his alleged behavior could cause him
trouble in the N.F.L., where he is among those with problematic records
on race and other culturally charged issues. After
\href{https://www.cnn.com/2020/07/22/politics/woody-johnson-oig-report/index.html}{CNN
reported} on his comments to the embassy staff in London,
\href{https://twitter.com/Prez/status/1285983250628767744?s=20}{Jamal
Adams, an All-Pro safety on the Jets, tweeted}, ``We need the RIGHT
people at the top. Wrong is wrong!''

It has also become a divisive issue inside the State Department. A
senior official there described the complaints about Mr. Johnson's
comments as an alarming ``trifecta'' --- racist, misogynistic and
anti-Semitic --- that showed him to be the most culturally insensitive
ambassador in recent memory.

The State Department rallied to Mr. Johnson's defense after the
accusations about his personal behavior were first reported. It
described him as ``a valued member of the team who has led Mission U.K.
honorably and professionally.''

But tensions in the embassy have caused some career diplomats to leave
and one --- Mr. Lukens --- to be forced out. That happened in November
2018 after Mr. Johnson heard he had given a speech at a British
university in which he made a mildly positive reference to former
President Barack Obama.

A popular diplomat whose father had been an ambassador, Mr. Lukens
served as travel director for Hillary Clinton when she was secretary of
state and as acting ambassador in London before Mr. Johnson arrived in
November 2017.

In dismissing Mr. Lukens, Mr. Johnson told him he was a ``traitor.''

Mark Landler reported from London, Lara Jakes from Washington and Maggie
Haberman from New York.

Advertisement

\protect\hyperlink{after-bottom}{Continue reading the main story}

\hypertarget{site-index}{%
\subsection{Site Index}\label{site-index}}

\hypertarget{site-information-navigation}{%
\subsection{Site Information
Navigation}\label{site-information-navigation}}

\begin{itemize}
\tightlist
\item
  \href{https://help.nytimes3xbfgragh.onion/hc/en-us/articles/115014792127-Copyright-notice}{©~2020~The
  New York Times Company}
\end{itemize}

\begin{itemize}
\tightlist
\item
  \href{https://www.nytco.com/}{NYTCo}
\item
  \href{https://help.nytimes3xbfgragh.onion/hc/en-us/articles/115015385887-Contact-Us}{Contact
  Us}
\item
  \href{https://www.nytco.com/careers/}{Work with us}
\item
  \href{https://nytmediakit.com/}{Advertise}
\item
  \href{http://www.tbrandstudio.com/}{T Brand Studio}
\item
  \href{https://www.nytimes3xbfgragh.onion/privacy/cookie-policy\#how-do-i-manage-trackers}{Your
  Ad Choices}
\item
  \href{https://www.nytimes3xbfgragh.onion/privacy}{Privacy}
\item
  \href{https://help.nytimes3xbfgragh.onion/hc/en-us/articles/115014893428-Terms-of-service}{Terms
  of Service}
\item
  \href{https://help.nytimes3xbfgragh.onion/hc/en-us/articles/115014893968-Terms-of-sale}{Terms
  of Sale}
\item
  \href{https://spiderbites.nytimes3xbfgragh.onion}{Site Map}
\item
  \href{https://help.nytimes3xbfgragh.onion/hc/en-us}{Help}
\item
  \href{https://www.nytimes3xbfgragh.onion/subscription?campaignId=37WXW}{Subscriptions}
\end{itemize}
