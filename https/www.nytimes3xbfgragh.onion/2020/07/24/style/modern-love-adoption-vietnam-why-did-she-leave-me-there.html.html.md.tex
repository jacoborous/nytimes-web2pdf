Sections

SEARCH

\protect\hyperlink{site-content}{Skip to
content}\protect\hyperlink{site-index}{Skip to site index}

\href{https://www.nytimes3xbfgragh.onion/section/style}{Style}

\href{https://myaccount.nytimes3xbfgragh.onion/auth/login?response_type=cookie\&client_id=vi}{}

\href{https://www.nytimes3xbfgragh.onion/section/todayspaper}{Today's
Paper}

\href{/section/style}{Style}\textbar{}Why Did She Leave Me There?

\url{https://nyti.ms/2OTxRe0}

\begin{itemize}
\item
\item
\item
\item
\item
\end{itemize}

Advertisement

\protect\hyperlink{after-top}{Continue reading the main story}

Supported by

\protect\hyperlink{after-sponsor}{Continue reading the main story}

Modern Love

\hypertarget{why-did-she-leave-me-there}{%
\section{Why Did She Leave Me There?}\label{why-did-she-leave-me-there}}

A young man returns to the Vietnamese orphanage he had spent 25 years
trying to forget.

\includegraphics{https://static01.graylady3jvrrxbe.onion/images/2020/07/26/fashion/26MODERN-ORPHANAGE/26MODERN-ORPHANAGE-articleLarge.jpg?quality=75\&auto=webp\&disable=upscale}

By Kacey Vu Shap

\begin{itemize}
\item
  July 24, 2020
\item
  \begin{itemize}
  \item
  \item
  \item
  \item
  \item
  \end{itemize}
\end{itemize}

The gate to the orphanage was smaller than I remembered. Nearly 25 years
had passed since I lived there. I wondered if coming back was a good
idea.

My best friends Phu, Francis and Will had planned this trip to Vietnam
and invited me to come. I met them 15 years earlier when I was in high
school. They were in college and had started a support group for young,
gay Asian men.

At the time, I was rocking blonde highlights, blue contacts and
Abercrombie \& Fitch T-shirts with baggy ripped jeans. Phu and Francis
thought I was trying too hard, but they let me tag along. Now they had
become my overprotective brothers, nagging me about everything.

My friends knew I had lived in an orphanage near Ho Chi Minh City and
suggested, as part of our trip, that we try to visit. More than that,
they wanted to do a Kickstarter campaign to raise money for the orphans
who now lived there.

I thought that was crazy. We were set to leave in a week. And I had no
desire to return to a place I had spent most of my life trying to
forget.

``This is a once-in-a-lifetime opportunity,'' Francis said. ``You won't
be alone. We'll be with you.''

They were able to talk me into it, and within days we had raised more
than \$5,000 to buy clothes, toys and other essentials.

\emph{{[}}\href{https://www.nytimes3xbfgragh.onion/newsletters/love-letter}{\emph{Sign
up for Love Letter, our weekly email about Modern Love, weddings and
relationships.}}\emph{{]}}

The next week, as we unloaded the donations from a rented truck and made
our way inside the orphanage, we were greeted by tiny faces and bony,
outstretched hands. How strange that I once had been among those scared,
excited faces vying for attention from a stranger like me.

Back then, the big building was dilapidated, the white paint covered in
dirt, the walls battered. It had since been fixed up and expanded, but
one thing remained: the distinct odor of baby powder, sweat, urine,
decay, hopelessness and despair.

Even though I was 5 when I arrived, I was too small to quarter with the
older children but too big to be with the babies, so they put me with
those with deformities, missing limbs or mental illnesses. The memories
came rushing back as my friends and I walked inside. My eyes began to
swell, my heart pounded and my anxiety kicked in. Before long, I was
fleeing to the gated entrance, my friends calling after me.

As a child, whenever I told people I was adopted, I used to say I came
premade: I simply appeared on one summer night at the Baltimore airport
to be greeted by my mom, dad and sister, who were bearing candies,
flowers and kisses. It was easier to sanitize my story by speaking only
of my life as Kacey, who was loved and wanted, than to tell people of my
life as Vu, who was abandoned and undesired.

I never knew my birth mother, who died when I was 2 in the delivery room
along with my brother. I hardly knew my birth father, a migrant worker
who was never around. When I was 5, my older sister drowned in a river
near my grandmother's home. I watched from 10 feet away as she thrashed
and then disappeared in the murky water.

I had pleaded with her to play in the river with the other children,
despite my grandmother forbidding us from going when she wasn't around.
I wished it had been me who drowned that day.

Then it was just my grandmother and I living together in a poor farming
village in southern Vietnam. If my grandmother were a cat, I was her
tail, because wherever she went, I followed. I loved being near her in
the kitchen. Exotic spices mingling with seasoned meat and fresh herbs
would cocoon us in their delicious embrace as I peppered my grandmother
with questions about our favorite subject: my mother.

``Grandma, you have my eyes, my nose and my cheeks,'' I said. ``Do you
think my mother also looked like me?''

``Of course, silly! Who do you think gave you and your mother such
handsome features?'' She beamed her toothless grin. Then she stopped
chopping vegetables and said, ``Can I tell you a big secret? Your mother
was my favorite of all my children. She always tried to make everyone
laugh. I want you to be good, like your mother. Yes?''

``OK,'' I said.

After my sister died, I learned that my father had died too, and it
wasn't long before my grandmother told me to pack my things for a trip.
I was delighted, having never been on a trip before.

Eventually we arrived at a big white building full of children. After
touring the place, my grandmother seemed reluctant to leave. Finally,
she bent down and said, ``I am going home, but you are staying here.''

I stood there, frozen.

My grandmother cupped my cheeks with her leathery hands and directed my
face toward hers, her normally fierce eyes filled with sadness. She took
a floral-patterned handkerchief from her neck and wrapped it around
mine. It was her favorite, infused with her familiar scent. Then she
stood up and walked away without looking back.

I tried to follow, but strong hands gripped me. I screamed for my
grandmother, begging her to take me home. After she left, I waited for
days at the gated entrance, hoping for her return.

Some months later, a Jewish couple in Northern Virginia was in the final
stages of an adoption that fell through. Devastated, on the verge of
giving up, they received my photo from the adoption agency and decided
they wanted me as their child, a difficult process that took two years.
I was entirely unaware of my adoption until the day I was taken to the
airport. I would later learn that of the hundreds of children at the
orphanage, only a handful made it to America. Most were babies. I was 7.

A quarter-century has passed since my grandmother left me that day. I
still carry her handkerchief safely tucked away with me wherever I go,
but her scent has since faded. There are so many things I have wanted to
share with her of my American life: my loving parents, friends, dog, Los
Angeles apartment and freshly minted Ph.D. in social psychology. There
are also so many questions I have wanted to ask her.

Whenever I told people I was adopted, I didn't tell them about the day I
was abandoned, or of my fear that my friends and family would discover
that I had been worthless enough to deserve that.

Now, my friends had seen it. They knew. When they came out and found me
by the gate, they asked why I had left so abruptly.

``I knew that once you saw my orphanage,'' I said, talking fast, ``you'd
think less of me and wouldn't want to be my friends anymore.''

``Seriously?'' Phu said. ``We traveled across the globe, covered in
mosquito bites, soaked in sweat, and you're worried we might think less
of you? We've been subjected to worse. There is Kacey who is always
late, Kacey with a big head and Kacey who chases after emotionally
unavailable men. If all that didn't scare us off, nothing will.''

My friends surrounded me, wrapping me in their warm embrace.

``You're family,'' Francis said. ``We love you. Besides, being friends
with you is like catching herpes. It's very contagious, easily
treatable, but impossible to get rid of. And we've been treating it for
over 15 years now.''

Then Will said, ``And maybe your grandmother did love you. Maybe letting
you go was her final act of love, so you could have a chance at a better
life.''

It's something I had long wondered. Had she left me because I was a
burden, or to spare me from a brutal life of poverty?

My friends then told me that while I was outside, they had been able to
find my grandmother's last known address in the orphanage's records.
There was a chance she still might be there, only 30 minutes away.

If my grandmother were still living there, I could have my answer. I
thought about that, and also about the love and support of my friends,
family and others who had made this possible.

``No,'' I said. ``I don't need to know her address. We can go now.'' For
once, I could choose not to be defined by my abandonment.

With that, we left the orphanage and spilled out into Ho Chi Minh City,
where the sweet scent of sizzling pork mingled with the laughter of
children chasing each other, as if the streets were one giant
playground.

\includegraphics{https://static01.graylady3jvrrxbe.onion/images/2020/07/24/style/24MODERN-ORPHANAGE-VIDEO-STILL/Screen-Shot-2020-07-20-at-4-videoSixteenByNineJumbo1600.png}

Kacey Vu Shap is a researcher, writer and social entrepreneur in Los
Angeles.

Modern Love can be reached at
\href{mailto:modernlove@NYTimes.com}{\nolinkurl{modernlove@NYTimes.com}}.

Want more from Modern Love? Watch the
\href{https://www.nytimes3xbfgragh.onion/2019/09/12/style/modern-love-tv-show-trailer.html}{TV
series}; sign up for the
\href{https://www.nytimes3xbfgragh.onion/newsletters/love-letter}{newsletter};
or listen to the
\href{https://www.nytimes3xbfgragh.onion/column/modern-love-podcast}{podcast}
on
\href{https://itunes.apple.com/us/podcast/modern-love/id1065559535?mt=2\&version=meter+at+0\&module=meter-Links\&pgtype=article\&contentId=\&mediaId=\&referrer=\&priority=true\&action=click\&contentCollection=meter-links-click}{iTunes},
\href{https://open.spotify.com/show/03Er7mSPq9IEewOgbPD3vO}{Spotify} or
\href{https://play.google.com/music/listen?u=0\#/ps/Iktqjbkz7bychbnofblw32dik64}{Google
Play}. We also have swag at
\href{https://store.nytimes3xbfgragh.onion/collections/modern-love}{the
NYT Store} and a book,
``\href{https://www.penguinrandomhouse.com/books/623036/modern-love-revised-and-updated-by-edited-by-daniel-jones-with-contributions-by-andrew-rannells-ayelet-waldman-amy-krouse-rosenthal-veronica-chambers-and-more/}{Modern
Love: True Stories of Love, Loss, and Redemption}.''

Advertisement

\protect\hyperlink{after-bottom}{Continue reading the main story}

\hypertarget{site-index}{%
\subsection{Site Index}\label{site-index}}

\hypertarget{site-information-navigation}{%
\subsection{Site Information
Navigation}\label{site-information-navigation}}

\begin{itemize}
\tightlist
\item
  \href{https://help.nytimes3xbfgragh.onion/hc/en-us/articles/115014792127-Copyright-notice}{©~2020~The
  New York Times Company}
\end{itemize}

\begin{itemize}
\tightlist
\item
  \href{https://www.nytco.com/}{NYTCo}
\item
  \href{https://help.nytimes3xbfgragh.onion/hc/en-us/articles/115015385887-Contact-Us}{Contact
  Us}
\item
  \href{https://www.nytco.com/careers/}{Work with us}
\item
  \href{https://nytmediakit.com/}{Advertise}
\item
  \href{http://www.tbrandstudio.com/}{T Brand Studio}
\item
  \href{https://www.nytimes3xbfgragh.onion/privacy/cookie-policy\#how-do-i-manage-trackers}{Your
  Ad Choices}
\item
  \href{https://www.nytimes3xbfgragh.onion/privacy}{Privacy}
\item
  \href{https://help.nytimes3xbfgragh.onion/hc/en-us/articles/115014893428-Terms-of-service}{Terms
  of Service}
\item
  \href{https://help.nytimes3xbfgragh.onion/hc/en-us/articles/115014893968-Terms-of-sale}{Terms
  of Sale}
\item
  \href{https://spiderbites.nytimes3xbfgragh.onion}{Site Map}
\item
  \href{https://help.nytimes3xbfgragh.onion/hc/en-us}{Help}
\item
  \href{https://www.nytimes3xbfgragh.onion/subscription?campaignId=37WXW}{Subscriptions}
\end{itemize}
