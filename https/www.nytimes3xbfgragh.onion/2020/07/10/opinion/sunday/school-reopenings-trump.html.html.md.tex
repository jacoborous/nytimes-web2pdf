Sections

SEARCH

\protect\hyperlink{site-content}{Skip to
content}\protect\hyperlink{site-index}{Skip to site index}

\href{https://www.nytimes3xbfgragh.onion/section/opinion/sunday}{Sunday
Review}

\href{https://myaccount.nytimes3xbfgragh.onion/auth/login?response_type=cookie\&client_id=vi}{}

\href{https://www.nytimes3xbfgragh.onion/section/todayspaper}{Today's
Paper}

\href{/section/opinion/sunday}{Sunday Review}\textbar{}Trump Threatens
to Turn Pandemic Schooling Into a Culture War

\url{https://nyti.ms/2ZZj6vf}

\begin{itemize}
\item
\item
\item
\item
\item
\item
\end{itemize}

Advertisement

\protect\hyperlink{after-top}{Continue reading the main story}

\href{/section/opinion}{Opinion}

Supported by

\protect\hyperlink{after-sponsor}{Continue reading the main story}

\hypertarget{trump-threatens-to-turn-pandemic-schooling-into-a-culture-war}{%
\section{Trump Threatens to Turn Pandemic Schooling Into a Culture
War}\label{trump-threatens-to-turn-pandemic-schooling-into-a-culture-war}}

The president might sabotage parents' best hopes for getting their kids
back to school.

\href{https://www.nytimes3xbfgragh.onion/by/michelle-goldberg}{\includegraphics{https://static01.graylady3jvrrxbe.onion/images/2018/04/02/opinion/michelle-goldberg/michelle-goldberg-thumbLarge.png}}

By
\href{https://www.nytimes3xbfgragh.onion/by/michelle-goldberg}{Michelle
Goldberg}

Opinion Columnist

\begin{itemize}
\item
  July 10, 2020
\item
  \begin{itemize}
  \item
  \item
  \item
  \item
  \item
  \item
  \end{itemize}
\end{itemize}

\includegraphics{https://static01.graylady3jvrrxbe.onion/images/2020/07/12/opinion/10goldberg1/10goldberg1-articleLarge.jpg?quality=75\&auto=webp\&disable=upscale}

Two weeks ago, I asked Randi Weingarten, president of the American
Federation of Teachers, what a functioning Department of Education would
be doing to prepare the country to
\href{https://www.nytimes3xbfgragh.onion/2020/07/10/us/politics/trump-schools-reopening.html}{reopen
schools} in the fall.

``A functioning Department of Education would have been getting groups
of superintendents and principals and unions and others together from
the middle of March,'' she told me. It would have created a
clearinghouse of best practices for maintaining grab-and-go lunch
programs and online education. By mid-April it would have convened
experts to figure out how to reopen schools safely, and offered grants
to schools trying different models.

``None of that has happened,'' said Weingarten. ``Zero.''

Instead, Donald
\href{https://www.nytimes3xbfgragh.onion/2020/07/10/us/politics/trump-schools-reopening.html}{Trump}
has approached the extraordinarily complex challenge of educating
children during a pandemic just as he's approached most other matters of
governing: with bullying, bluster and propaganda.

While doing nothing to curb the wildfire spread of the coronavirus, he
has demanded that schools reopen and threatened to cut off funding for
those that don't. On Wednesday, he tweeted that the guidelines for
reopening schools from his own Centers for Disease Control and
Prevention were ``very tough \& expensive,'' adding, ``I will be meeting
with them!!!'' Mike Pence then suggested that the
\href{https://www.usatoday.com/story/news/politics/2020/07/08/pence-cdc-changing-coronavirus-school-guidelines-after-trump-attack/5398493002/}{guidelines
would be revised}. On Thursday the agency's director, Dr. Robert
Redfield,
\href{https://www.cnn.com/2020/07/09/politics/cdc-guidelines-school-reopenings/index.html}{said
they wouldn't be}, but later, seeming to give into pressure, said the
guidelines should be seen as
\href{https://news.yahoo.com/cdc-softens-guidelines-for-reopening-schools-after-trump-calls-them-impractical-and-expensive-181630718.html}{recommendations,
not requirements}.

Also on Thursday, Secretary of Education Betsy DeVos gestured toward a
plan of coronavirus-inspired school choice that would punish public
schools that don't fully reopen. Without offering details, she said
families could take the federal money spent at these schools and use it
elsewhere. She's long wanted to give public money to private schools;
perhaps she thinks this coronavirus has given her the chance.

When I spoke to Weingarten again on Thursday, she wasn't worried that
Trump and DeVos would be able to follow through on their threats; they
can't redirect the funds without Congress. But with their crude attempts
at coercion, they've politicized school reopening just as Trump
politicized mask-wearing and hydroxychloroquine.

As a result, the administration has made reopening schools more
difficult. ``The threats are empty, but the distrust they have caused is
not,'' Weingarten said.

At the end of June, the American Federation of Teachers
\href{https://www.aft.org/sites/default/files/june_2020_member_poll_slides_07072020.pdf}{surveyed
its members} and found a broad willingness to return to the classroom.
Two-thirds of respondents said school buildings should reopen in some
capacity, and 76 percent said they'd be comfortable being in school with
the proper safeguards. But after Trump began ranting about schools,
Weingarten started hearing from teachers who were scared that reopening
would be done rashly.

So as Trump tries to turn school reopenings into part of his culture
war, Weingarten fears ``a huge brain drain of people not willing to be
in schools anymore.''

To be clear: As a parent, I want schools to open full-time at least as
much as Trump does. On Wednesday, New York City announced its plan to
send kids back to school part time, and it is a calamity. To accommodate
C.D.C. guidelines calling for six feet of distance between desks,
students will be able to go to school only one to three days a week. It
is not yet clear if schools will be able to ensure that siblings will
attend on the same days. Working parents could end up needing full-time
child care indefinitely, and there are, as yet, no plans to provide it
publicly.

Similar hybrid schedules are being adopted all over the country --- and
grim as they are, they might turn out to be too optimistic, because they
depend on the virus being somewhat contained. Palm Beach, Fla.,
\href{https://www.palmbeachpost.com/news/20200708/pbc-school-campuses-will-remain-closed-to-students-board-members-decide}{just
announced} that schools there won't open at all. Other districts in
hard-hit areas will likely follow suit.

So far, the results of so-called ``remote learning'' --- a term I
dislike, since it presumes that learning is happening --- have been
terrible for students, especially disadvantaged ones. The fallout for
many parents' financial prospects and mental health is catastrophic. And
part-time schooling is likely to significantly amplify educational
inequalities that are already enormous. As those who can afford it hire
private teachers and tutors, we are rapidly heading toward a system of
neo-governesses in which basic schooling becomes a luxury good
unattainable for many people outside the 1 percent.

This is almost certainly not why Trump is eager to have school resume.
Rather, school closures and staggered schedules are a crushing weight on
the economy. To millions of parents, they're an intimate daily reminder
that the president's incompetence has ruined our lives. But to open
schools in a reasonable way, the government needs to do two things:
control the pandemic, as most other developed countries have done, and
give schools money to adapt. This administration has so far failed to do
either.

And now the president's interference with the C.D.C. has made things
worse.

Here's the thing: The C.D.C. guidelines might indeed be too stringent,
at least for elementary schools. There is some evidence that little kids
are less susceptible to Covid-19, and may be less likely to spread it;
in countries where schools have reopened, few clusters have been linked
to elementary schools. (There have been outbreaks in middle schools and
high schools,
\href{https://www.wsj.com/articles/israel-shuts-some-schools-as-coronavirus-cases-jump-after-reopening-11591203323}{most
notably in Israel}.)

In a recent statement on school reopenings, the American Academy of
Pediatrics says that three feet of distance between desks might be
sufficient, particularly if students wear masks. (Admittedly, getting
little kids to keep masks on is challenging.) ``Schools should weigh the
benefits of strict adherence to a 6-feet spacing rule between students
with the potential downside if remote learning is the only
alternative,'' it said.

The hybrid model that many large school districts are adopting is meant
to limit the number of people whom teachers and students are exposed to.
But Elliot Haspel, author of ``Crawling Behind: America's Childcare
Crisis and How to Fix It,'' points out that if kids disperse to various
kinds of child care when they aren't in school, they could end up being
exposed to more people than they would be in a regular classroom.

``It's a nightmare,'' he told me. ``I think we're going to have
significantly more harm to children and to families pursuing a staggered
schedule approach, particularly to elementary school students.''

But Trump's interference means that now no departure from the current
C.D.C. guidelines will be seen as credible outside of MAGAland. ``The
recklessness has made people distrust anything that they say because
they have downplayed the virus from the beginning,'' said Weingarten.

Last month,
\href{https://www.npr.org/2020/06/28/884351948/how-one-maryland-nursing-home-avoided-covid-19}{NPR
reported} on a mostly Black nursing home in Maryland that didn't lose
any residents to Covid because its director listened to what Trump said
about the virus and assumed the opposite was true. ``When I heard
President Trump say we only had 15 cases and by the end of the week that
it would be zero, I knew that it was time to act,'' the director said.

This is a president with negative credibility. The more Trump demands
that schools open, the more people who've paid close attention to him
will fear they all must remain closed.

\emph{The Times is committed to publishing}
\href{https://www.nytimes3xbfgragh.onion/2019/01/31/opinion/letters/letters-to-editor-new-york-times-women.html}{\emph{a
diversity of letters}} \emph{to the editor. We'd like to hear what you
think about this or any of our articles. Here are some}
\href{https://help.nytimes3xbfgragh.onion/hc/en-us/articles/115014925288-How-to-submit-a-letter-to-the-editor}{\emph{tips}}\emph{.
And here's our email:}
\href{mailto:letters@NYTimes.com}{\emph{letters@NYTimes.com}}\emph{.}

\emph{Follow The New York Times Opinion section on}
\href{https://www.facebookcorewwwi.onion/nytopinion}{\emph{Facebook}}\emph{,}
\href{http://twitter.com/NYTOpinion}{\emph{Twitter (@NYTopinion)}}
\emph{and}
\href{https://www.instagram.com/nytopinion/}{\emph{Instagram}}\emph{.}

Advertisement

\protect\hyperlink{after-bottom}{Continue reading the main story}

\hypertarget{site-index}{%
\subsection{Site Index}\label{site-index}}

\hypertarget{site-information-navigation}{%
\subsection{Site Information
Navigation}\label{site-information-navigation}}

\begin{itemize}
\tightlist
\item
  \href{https://help.nytimes3xbfgragh.onion/hc/en-us/articles/115014792127-Copyright-notice}{©~2020~The
  New York Times Company}
\end{itemize}

\begin{itemize}
\tightlist
\item
  \href{https://www.nytco.com/}{NYTCo}
\item
  \href{https://help.nytimes3xbfgragh.onion/hc/en-us/articles/115015385887-Contact-Us}{Contact
  Us}
\item
  \href{https://www.nytco.com/careers/}{Work with us}
\item
  \href{https://nytmediakit.com/}{Advertise}
\item
  \href{http://www.tbrandstudio.com/}{T Brand Studio}
\item
  \href{https://www.nytimes3xbfgragh.onion/privacy/cookie-policy\#how-do-i-manage-trackers}{Your
  Ad Choices}
\item
  \href{https://www.nytimes3xbfgragh.onion/privacy}{Privacy}
\item
  \href{https://help.nytimes3xbfgragh.onion/hc/en-us/articles/115014893428-Terms-of-service}{Terms
  of Service}
\item
  \href{https://help.nytimes3xbfgragh.onion/hc/en-us/articles/115014893968-Terms-of-sale}{Terms
  of Sale}
\item
  \href{https://spiderbites.nytimes3xbfgragh.onion}{Site Map}
\item
  \href{https://help.nytimes3xbfgragh.onion/hc/en-us}{Help}
\item
  \href{https://www.nytimes3xbfgragh.onion/subscription?campaignId=37WXW}{Subscriptions}
\end{itemize}
