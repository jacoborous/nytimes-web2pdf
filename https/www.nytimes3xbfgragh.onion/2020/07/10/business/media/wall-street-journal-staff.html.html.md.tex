Sections

SEARCH

\protect\hyperlink{site-content}{Skip to
content}\protect\hyperlink{site-index}{Skip to site index}

\href{https://www.nytimes3xbfgragh.onion/section/business/media}{Media}

\href{https://myaccount.nytimes3xbfgragh.onion/auth/login?response_type=cookie\&client_id=vi}{}

\href{https://www.nytimes3xbfgragh.onion/section/todayspaper}{Today's
Paper}

\href{/section/business/media}{Media}\textbar{}Wall Street Journal Staff
Members Push for Big Changes in News Coverage

\url{https://nyti.ms/30cJMJd}

\begin{itemize}
\item
\item
\item
\item
\item
\end{itemize}

\href{https://www.nytimes3xbfgragh.onion/news-event/george-floyd-protests-minneapolis-new-york-los-angeles?action=click\&pgtype=Article\&state=default\&region=TOP_BANNER\&context=storylines_menu}{Race
and America}

\begin{itemize}
\tightlist
\item
  \href{https://www.nytimes3xbfgragh.onion/2020/07/26/us/protests-portland-seattle-trump.html?action=click\&pgtype=Article\&state=default\&region=TOP_BANNER\&context=storylines_menu}{Protesters
  Return to Other Cities}
\item
  \href{https://www.nytimes3xbfgragh.onion/2020/07/24/us/portland-oregon-protests-white-race.html?action=click\&pgtype=Article\&state=default\&region=TOP_BANNER\&context=storylines_menu}{Portland
  at the Center}
\item
  \href{https://www.nytimes3xbfgragh.onion/2020/07/23/podcasts/the-daily/portland-protests.html?action=click\&pgtype=Article\&state=default\&region=TOP_BANNER\&context=storylines_menu}{Podcast:
  Showdown in Portland}
\item
  \href{https://www.nytimes3xbfgragh.onion/interactive/2020/07/16/us/black-lives-matter-protests-louisville-breonna-taylor.html?action=click\&pgtype=Article\&state=default\&region=TOP_BANNER\&context=storylines_menu}{45
  Days in Louisville}
\end{itemize}

Advertisement

\protect\hyperlink{after-top}{Continue reading the main story}

Supported by

\protect\hyperlink{after-sponsor}{Continue reading the main story}

\hypertarget{wall-street-journal-staff-members-push-for-big-changes-in-news-coverage}{%
\section{Wall Street Journal Staff Members Push for Big Changes in News
Coverage}\label{wall-street-journal-staff-members-push-for-big-changes-in-news-coverage}}

A letter from a group of Journal reporters and editors calls for ``more
muscular reporting about race and social inequities,'' as well as
skepticism toward business and government leaders.

\includegraphics{https://static01.graylady3jvrrxbe.onion/images/2020/07/11/business/10unrest-wsj-print/10unrest-wsj1-articleLarge.jpg?quality=75\&auto=webp\&disable=upscale}

By \href{https://www.nytimes3xbfgragh.onion/by/marc-tracy}{Marc Tracy}
and \href{https://www.nytimes3xbfgragh.onion/by/ben-smith}{Ben Smith}

\begin{itemize}
\item
  July 10, 2020
\item
  \begin{itemize}
  \item
  \item
  \item
  \item
  \item
  \end{itemize}
\end{itemize}

Staff members of The Wall Street Journal have been pressing newsroom
leaders to make fundamental changes in how the newspaper covers race,
policing, and its primary focus, the business world, along with other
matters.

In a June 23 letter to the editor in chief, Matt Murray, a group
identifying itself only as ``members of the WSJ newsroom'' said the
paper must ``encourage more muscular reporting about race and social
inequities,'' and laid out detailed proposals for revising its news
coverage.

``In part because WSJ's coverage has focused historically on industries
and leadership ranks dominated by white men, many of our newsroom
practices are inadequate for the present moment,'' the letter said.

Among its proposals: Mr. Murray should appoint journalists to cover
``race, ethnicity and inequality''; name two standards editors
specializing in diversity; conduct a study of the race, ethnicity and
gender breakdown of the subjects of The Journal's ``most prominent and
resource-intensive stories''; and bring more diversity to the newsroom
and leadership positions.

Speaking more broadly, the letter questioned whether The Journal put too
much stock in business leaders and government officials.

\includegraphics{https://static01.graylady3jvrrxbe.onion/images/2020/07/10/business/10unrest-wsj2/merlin_174395925_14859f4a-1150-4325-84a3-f7fb9b1b9510-articleLarge.jpg?quality=75\&auto=webp\&disable=upscale}

``Reporters frequently meet resistance when trying to reflect the
accounts and voices of workers, residents or customers, with some
editors voicing heightened skepticism of those sources' credibility
compared with executives, government officials or other entities,'' the
letter said. ``We should apply the same healthy skepticism toward
everyone we cover.''

On Friday, Kamilah M. Thomas, chief people officer with Dow Jones, the
publisher of The Journal, sent an internal email announcing the recent
creation of a new position of senior vice president of inclusion and
people management as well as other initiatives that, she said, are part
of ``a comprehensive review of diversity, equity and inclusion across
our business.''

The Journal is one of many media organizations, including The New York
Times, The Philadelphia Inquirer, The Los Angeles Times and Condé Nast,
where staff members have questioned leadership at a time of widespread
protests against racism and police brutality prompted by the killing in
May of
\href{https://www.nytimes3xbfgragh.onion/2020/07/08/us/george-floyd-body-camera-transcripts.html}{George
Floyd}, a Black man in Minneapolis who died after a white police officer
pressed a knee to his neck.

Confrontations between staff members and newsroom leaders have been rare
at the 131-year-old publication, which became
\href{https://www.nytimes3xbfgragh.onion/2007/05/04/business/media/04murdoch.html}{part
of Rupert Murdoch's media empire} in 2007. It has one of the country's
largest newsrooms, employing about 1,300.

The June 23 letter was sent to Mr. Murray, who
\href{https://www.nytimes3xbfgragh.onion/2018/06/05/business/media/wall-street-journal-editor-gerry-baker-matt-murray.html}{succeeded
Gerard Baker} as editor in chief two years ago, through the news
committee of the employee union and came from discussions on a private
channel on the interoffice communications app Slack, according to two
people with knowledge of how it came about. It was at least the third
instance of formal communication in recent weeks between the staff and
Journal leaders.

On June 12,
\href{https://www.wsj.com/articles/americas-newsrooms-face-a-reckoning-on-race-after-floyd-protests-11592256570}{more
than 150 journalists sent a letter} to Journal leaders saying the
paper's coverage of race was ``problematic'' and that its staff was not
diverse enough, The Journal reported in an article on newsroom revolts
across the country.

The week before that, the union representing Journal reporters and
editors
\href{https://www.nytimes3xbfgragh.onion/2020/06/09/business/wall-street-journal-gerard-baker-editor.html}{sent
a letter requesting that Mr. Baker}, who stayed on in the news
department as a columnist, be reassigned to the opinion section, which
is operated separately from the newsroom. Faulting columns Mr. Baker had
written on race, that letter said his work had violated newsroom
standards. Mr. Baker was moved to the opinion staff the day after the
letter was sent.

One of the proposals in the June 23 letter concerned changes to The
Journal's stylebook. ``Review the terminology used across WSJ content,
including editorial, to refer to various identity groups and compare
with latest industry standards,'' it suggested.

The following week, The Journal announced that it would capitalize
``Black'' when referring to members of the African diaspora. Several
other news organizations have made the same decision in recent weeks,
including \href{https://apnews.com/71386b46dbff8190e71493a763e8f45a}{The
Associated Press} and
\href{https://www.nytimes3xbfgragh.onion/2020/07/05/insider/capitalized-black.html}{The
Times}.

On Thursday, Mr. Murray announced in an email to the staff that Brent W.
Jones, an associate managing editor, who is Black, had been promoted to
the top echelon of newsroom leadership to fill a newly created role,
editor of culture, training and outreach.

In the note, which was obtained by The Times, Mr. Murray said Mr. Jones
was ``passionate about improving newsroom culture, diversity and
inclusion, talent development, training --- and the social value and
importance of fair, high-quality news and information.''

Mr. Murray added, ``His voice and experience, which have been especially
helpful to me as I focus more of my time on diversity and outreach, will
enliven and advance our continuing efforts.''

Advertisement

\protect\hyperlink{after-bottom}{Continue reading the main story}

\hypertarget{site-index}{%
\subsection{Site Index}\label{site-index}}

\hypertarget{site-information-navigation}{%
\subsection{Site Information
Navigation}\label{site-information-navigation}}

\begin{itemize}
\tightlist
\item
  \href{https://help.nytimes3xbfgragh.onion/hc/en-us/articles/115014792127-Copyright-notice}{©~2020~The
  New York Times Company}
\end{itemize}

\begin{itemize}
\tightlist
\item
  \href{https://www.nytco.com/}{NYTCo}
\item
  \href{https://help.nytimes3xbfgragh.onion/hc/en-us/articles/115015385887-Contact-Us}{Contact
  Us}
\item
  \href{https://www.nytco.com/careers/}{Work with us}
\item
  \href{https://nytmediakit.com/}{Advertise}
\item
  \href{http://www.tbrandstudio.com/}{T Brand Studio}
\item
  \href{https://www.nytimes3xbfgragh.onion/privacy/cookie-policy\#how-do-i-manage-trackers}{Your
  Ad Choices}
\item
  \href{https://www.nytimes3xbfgragh.onion/privacy}{Privacy}
\item
  \href{https://help.nytimes3xbfgragh.onion/hc/en-us/articles/115014893428-Terms-of-service}{Terms
  of Service}
\item
  \href{https://help.nytimes3xbfgragh.onion/hc/en-us/articles/115014893968-Terms-of-sale}{Terms
  of Sale}
\item
  \href{https://spiderbites.nytimes3xbfgragh.onion}{Site Map}
\item
  \href{https://help.nytimes3xbfgragh.onion/hc/en-us}{Help}
\item
  \href{https://www.nytimes3xbfgragh.onion/subscription?campaignId=37WXW}{Subscriptions}
\end{itemize}
