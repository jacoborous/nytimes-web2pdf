Sections

SEARCH

\protect\hyperlink{site-content}{Skip to
content}\protect\hyperlink{site-index}{Skip to site index}

\href{https://www.nytimes3xbfgragh.onion/section/food}{Food}

\href{https://myaccount.nytimes3xbfgragh.onion/auth/login?response_type=cookie\&client_id=vi}{}

\href{https://www.nytimes3xbfgragh.onion/section/todayspaper}{Today's
Paper}

\href{/section/food}{Food}\textbar{}When the Bake Sale Goes Global,
Millions Are Raised to Fight Injustice

\url{https://nyti.ms/32GRPR6}

\begin{itemize}
\item
\item
\item
\item
\item
\item
\end{itemize}

\href{https://www.nytimes3xbfgragh.onion/news-event/george-floyd-protests-minneapolis-new-york-los-angeles?action=click\&pgtype=Article\&state=default\&region=TOP_BANNER\&context=storylines_menu}{Race
and America}

\begin{itemize}
\tightlist
\item
  \href{https://www.nytimes3xbfgragh.onion/2020/07/26/us/protests-portland-seattle-trump.html?action=click\&pgtype=Article\&state=default\&region=TOP_BANNER\&context=storylines_menu}{Protesters
  Return to Other Cities}
\item
  \href{https://www.nytimes3xbfgragh.onion/2020/07/24/us/portland-oregon-protests-white-race.html?action=click\&pgtype=Article\&state=default\&region=TOP_BANNER\&context=storylines_menu}{Portland
  at the Center}
\item
  \href{https://www.nytimes3xbfgragh.onion/2020/07/23/podcasts/the-daily/portland-protests.html?action=click\&pgtype=Article\&state=default\&region=TOP_BANNER\&context=storylines_menu}{Podcast:
  Showdown in Portland}
\item
  \href{https://www.nytimes3xbfgragh.onion/interactive/2020/07/16/us/black-lives-matter-protests-louisville-breonna-taylor.html?action=click\&pgtype=Article\&state=default\&region=TOP_BANNER\&context=storylines_menu}{45
  Days in Louisville}
\end{itemize}

Advertisement

\protect\hyperlink{after-top}{Continue reading the main story}

Supported by

\protect\hyperlink{after-sponsor}{Continue reading the main story}

\hypertarget{when-the-bake-sale-goes-global-millions-are-raised-to-fight-injustice}{%
\section{When the Bake Sale Goes Global, Millions Are Raised to Fight
Injustice}\label{when-the-bake-sale-goes-global-millions-are-raised-to-fight-injustice}}

Online sales have become blockbuster events as long-sidelined pastry
chefs lead a charge toward activism.

\includegraphics{https://static01.graylady3jvrrxbe.onion/images/2020/07/22/dining/21Bakers6/merlin_174686868_658f0abc-2f7c-4d0c-bd55-fd22e255d49d-articleLarge.jpg?quality=75\&auto=webp\&disable=upscale}

\href{https://www.nytimes3xbfgragh.onion/by/julia-moskin}{\includegraphics{https://static01.graylady3jvrrxbe.onion/images/2018/09/25/multimedia/author-julia-moskin/author-julia-moskin-thumbLarge.png}}

By \href{https://www.nytimes3xbfgragh.onion/by/julia-moskin}{Julia
Moskin}

\begin{itemize}
\item
  July 21, 2020
\item
  \begin{itemize}
  \item
  \item
  \item
  \item
  \item
  \item
  \end{itemize}
\end{itemize}

Like a lot of 13-year-olds, Daniella Senior loved to bake, and thought
she might become a pastry chef.

Unlike most of them, she already had six employees.

Ms. Senior started out on her own, collecting and filling orders for
miniature sweets while growing up in Santo Domingo, in the Dominican
Republic. ``I was getting up at 4 a.m. every day before school to
bake,'' she said. Her mother, who had lent her \$200 in seed money, made
her bring on professional help.

\includegraphics{https://static01.graylady3jvrrxbe.onion/images/2020/07/22/dining/21Bakers1/merlin_174737976_2b40d2b5-a16a-4f89-92ed-c41e3cf2b13e-articleLarge.jpg?quality=75\&auto=webp\&disable=upscale}

Ms. Senior went on to attend the Culinary Institute of America, was
mentored by the chef
\href{https://www.nytimes3xbfgragh.onion/2017/10/30/dining/jose-andres-puerto-rico.html}{José
Andrés}, and is now an owner of five
\href{https://lacosechadc.com/merchants/serenata/}{bars} and
\href{https://www.coladashop.com/}{restaurants} in the Washington, D.C.,
area and a member of the board of \href{https://womenchefs.org/}{Women
Chefs and Restaurateurs}.

As the coronavirus swept through the Northeast this spring, closing
thousands of restaurants, Ms. Senior, 31, went back to baking. With the
Washington pastry chef
\href{https://www.instagram.com/smallorchids/?hl=en}{Paola Velez} (who
also has roots in the Dominican Republic), she repurposed her kitchens
and remaining employees as doughnut producers for a bake sale they
called \href{https://www.instagram.com/donadonadc/}{Doña Dona}.

Image

Ms. Senior, left, and Paola Velez, a pastry chef, started holding bake
sales to benefit immigrant workers in May.Credit...Andrew Seavey

They created doughnuts with Dominican flourishes of tamarind, pineapple,
guava and meringue, and sold them online, offering curbside pickups once
a week. In May, the effort raised \$6,000, enough to pay the bakers and
donate a thousand dollars to \href{https://www.ayuda.com/}{Ayuda}, a
national nonprofit group that provides help to low-income immigrants.

But in June, when the
\href{https://www.nytimes3xbfgragh.onion/2020/05/31/us/george-floyd-investigation.html}{killing
of George Floyd}and the debate over racial justice seized the nation's
attention, Ms. Velez, 29, said she saw that the scale of a traditional
bake sale was inadequate to the cause. ``It will make us feel good, but
it won't do anything to make real change,'' she said. ``We had to go
big.''

She went big. As of last week,
\href{https://www.bakersagainstracism.com/}{Bakers Against Racism}, a
global online bake sale she started with two other chefs, had raised
almost \$1.9 million for Black Lives Matter chapters and hundreds of
other groups working for racial justice.

Image

The bakers Willa Pelini, Paola Velez and Rob Rubba started Bakers
Against Racism in June; an online bake sale raised \$1.9 million for
racial and social justice causes.Credit...Jared Soares for The New York
Times

It's a given in the hospitality business that chefs show up for their
communities in big and small ways: feeding emergency medical workers,
cooking at charity benefits, donating dinners and sponsoring Little
League teams. Since Mr. Floyd's death on May 25, as protests against
systemic racism arose across the country, many chefs have emptied their
walk-in refrigerators to feed protesters and medical workers, and formed
organizations like \href{https://www.nouswithoutyou.la/}{No Us Without
You}, a Los Angeles group dedicated to food security for undocumented
restaurant cooks.

But it is pastry chefs and bakers who have been leading the industry
into activism, transforming bake sales into blockbuster political
fund-raisers for a variety of causes. And in a part of the cooking world
long dominated at the top by white women (and white men before them),
the voices of Latinx, Black and Asian women are rising --- and raising
real money for the fight against racism.

Image

Cheryl Day making biscuits at her bakery in Savannah, Ga. She is a
co-founder of Southern Restaurants for Racial Justice.Credit...Melissa
Golden for The New York Times

\href{https://www.instagram.com/srrj_coalition/?hl=en}{Southern
Restaurants for Racial Justice}, a new group started by three pastry
chefs --- \href{https://www.instagram.com/lisamariedonovan/?hl=en}{Lisa
Marie Donovan} in Nashville,
\href{http://www.littletartatl.com/who-we-are/}{Sarah O'Brien} in
Atlanta and \href{https://www.instagram.com/cherylday/?hl=en}{Cheryl
Day} in Savannah, Ga. --- raised \$100,000 for
\href{https://colorofchange.org/}{Color Of Change}, a racial-justice
advocacy group, with a Father's Day bake sale.

On June 19 --- Juneteenth --- more than 50 Los Angeles-area chefs and
bakers contributed to Pies For Justice, which raised \$36,000 before
selling out just five minutes after going live online. And in a global
online bake sale the next day, more than 2,000 people contributed baked
goods to Bakers Against Racism, and raised \$1.9 million in donations.

Why bakers and pastry chefs? Many of them say that in the restaurant
world, pastry is still dismissed as women's work. Those who succeed ---
especially if they are not white --- are used to fighting to be heard;
for them, baking is a language of protest.

Image

Mallory Cayon, a pastry chef in Los Angeles who participated in Pies For
Justice. Although pastry is still dismissed as ``women's work'' in some
professional kitchens, she says that respect for the field is
rising.Credit...Rozette Rago for The New York Times

\href{https://www.instagram.com/malkc/?hl=en}{Mallory Cayon}, who helped
create the cult-favorite brunch recipes at
\href{https://www.sundayinbrooklyn.com/}{Sunday in Brooklyn}, in
Williamsburg, has remained head pastry chef as the
\href{https://www.sundayhg.com/}{Sunday Hospitality} group grew to four
restaurants: two in Brooklyn and two in Los Angeles. She said that for
the first time since she entered the profession, she is in a workplace
where the pastry operation (mostly staffed by women) is on equal footing
with the ``savory'' side of the kitchen (mostly men).

``It starts in culinary school, because you look around and all the
bakers are girls,'' said Ms. Cayon, 30, who said she was surprised at
the time that gender imbalance in the field remained so persistent, long
after most workplaces had become more inclusive. ``The men who do it are
deemed less masculine.''

\href{https://www.instagram.com/dough_eung/?hl=en}{Dianna Daohueng}, the
culinary director at \href{https://www.blackseedbagels.com/}{Black Seed
Bagels} in New York City, said that working your way up in the
restaurant business as a woman, as a person of color or as a
first-generation American --- or, in her case, all three --- means
confronting prejudice every day.

``Just being a minority in the kitchen and in life turns you into a
natural activist,'' said Ms. Daohueng, 38, whose parents immigrated from
Thailand before she was born.

Ms. Day, of \href{https://backinthedaybakery.com/index.html\#/}{Back in
the Day Bakery} in Savannah, took a different path to protest baking.
She was raised in Los Angeles, but spent summers in Tuscaloosa, Ala.,
learning to bake from her grandmother. ``That was my culinary school,''
she said.

She also came to see how baking skills have defined Black women's lives,
especially in the South. ``My great-grandmother was both enslaved and a
pastry cook who was famous for her biscuits and cakes,'' she said.
``There is power in that.'' (Ms. Day, 59, has just completed a cookbook
based on her Southern lineage, to be published by Artisan next year.)

Her grandmother taught her the biscuit recipe she still uses, and
\href{https://cooking.nytimes3xbfgragh.onion/recipes/1021241-extra-flaky-pie-crust}{to
add apple cider vinegar to the crust} for
\href{https://cooking.nytimes3xbfgragh.onion/recipes/1021240-berry-hand-pies}{her
signature hand pies}.

Image

Ms. Day's baking is partly inspired by her great-grandmother, who was
enslaved and a pastry cook famed for her cakes and biscuits. ``There is
power in that.''Credit...Melissa Golden for The New York Times

Bake sales for civil-rights causes have a long history among
African-Americans. But the current surge of public protest baking
started during the run-up to the 2016 presidential election.

Social media posts about ``stress baking'' and ``anger baking'' turned
up a few years earlier. But
\href{https://twitter.com/tangerinejones}{Tangerine Jones}, a Black
artist in Brooklyn, began what she tagged ``rage baking'' in 2015 to
channel her anger about the overt racism she saw in the Trump campaign's
messaging. She gave her baked goods away to friends and neighbors, and
went on to use the term to identify herself on Instagram and Twitter.
(In February, an anthology titled ``Rage Baking,'' edited by two white
women who
\href{https://medium.com/@tangerinejones/the-privilege-of-rage-e5b2cb53d238}{did
not credit} Ms. Jones, was
\href{https://www.nytimes3xbfgragh.onion/2020/02/21/dining/rage-baking-book-tangerine-jones.html}{widely
criticized} for appropriating her idea and language.)

After the election, leading pastry chefs began to speak out. In 2017,
\href{https://www.instagram.com/natashapickowicz/?hl=en}{Natasha
Pickowicz} organized a high-profile ticketed bake sale in New York, to
raise money for Planned Parenthood. It became an annual event, with
sales rising from \$8,000 in 2017 to \$100,000 in 2019. That same year,
Los Angeles-area chefs led by the baker
\href{https://www.huckleberrycafe.com/our-story/}{Zoe Nathan} formed
\href{https://www.andgatherforgood.com/new-page}{Gather For Good},
holding frequent outdoor bake sales to benefit the American Civil
Liberties Union and other free-speech advocates.

Last year's marquee event was an all-cake sale for Planned Parenthood,
with ornate, outspoken creations decorated with female reproductive
organs, coat hangers and slogans like ``Keep The Government Out of My
Vagina!''

``The election made people brave enough to talk about immigration
rights, environmental rights and racial justice,'' said Stephanie Chen
of \href{http://www.sugarbearbakes.com/}{Sugarbear Bakes}, who
contributed a ``Mind Your Own Uterus'' cake. (She is also a founder of
Gather for Good.)

\href{https://www.instagram.com/sugarbearbakes/?hl=en}{Ms. Chen}, 36,
was a global advertising executive for the Apple iPhone before leaving
the technology industry to bake full time. From that marketing
perspective, she noted that bake sales --- traditional, friendly,
sugarcoated --- function not only as fund-raisers, but also as dialogue
openers with people who otherwise might not engage with the movement.

``It's a way of bringing information that isn't about protest or
violence,'' she said.

The migration of bake sales to social media --- especially Instagram,
where beauty shots of pastries and bread loaves draw enormous attention
--- has transformed them into even more powerful tools.

Image

Mr. Rubba,~a chef and baker, created the graphics for Bakers Against
Racism.Credit...Rob Rubba

On June 4, Bakers Against Racism went public on Instagram. Ms. Velez had
pulled in two other Washington-based founders: the pastry chef
\href{https://www.instagram.com/badwolf_88/?hl=en}{Willa Pelini}, and
the chef and baker \href{https://www.instagram.com/robrubba/?hl=en}{Rob
Rubba}, who is also a graphic artist. (``Cute but disruptive'' is how
Ms. Velez described the group's visual identity.)

They tweaked the bake-sale model in a way that ultimately allowed it to
go viral --- by not collecting any of the money that was raised.

The organizers created the name and hashtag, shared images and language
that bakers could use on social media in a Google Doc, and suggested
organizations to donate to, though each baker was allowed to decide
where to direct the funds. Individual bakers did the rest, connecting
with their local communities for orders and deliveries of everything
from brown-sugar
\href{https://www.selvacentralgoods.com/baked-goods}{pan dulce} baked in
Seattle to calamansi lime crinkle cookies (made in Chicago by the
Filipina-American baker
\href{https://www.instagram.com/cameliabakes/?hl=en}{Camelia Camara}) to
zucchini bread (Mr. Rubba's grandmother's recipe).

``We didn't want it to have to be slick and sponsored,'' Mr. Rubba said.
``The bakers and the buyers are equal participants in this movement.''

More than 2,500 bakery owners, pastry chefs and home bakers
participated, including clusters that materialized in Berlin, Paris and
London and as far afield as Australia, Tanzania and Turkey.

Image

Mr. Rubba made his grandmother's zucchini bread for the global Bakers
Against Racism bake sale in June, which included the creations of more
than 2,500 professional and home bakers.~Credit...Deb Rubba

``The bigger it got, the more afraid I was of taking this huge stand,''
said Ms. Velez, who noted that chefs --- especially in Washington ---
are often advised to stay out of politics to preserve a broad customer
base. ``But the backlash never came.''

Though bake sales have been successful, the outlook for bakers is not
rosy.

``We are always the first department to get cut,'' Ms. Pelini said.
Restaurant owners know that they can easily resort to serving ice cream
or cookie plates instead of labor-intensive desserts.

Almost every chef interviewed for this article had lost a job or closed
a bakery, at least temporarily, during the pandemic. The kitchen
assistants who worked for them, many of them low-income immigrants, are
often ineligible for unemployment and lack access to health care.

Most of the chefs say they are baking because it is currently the only
practical action they can take against chaos and injustice.

``I don't know policy, I am not a lawyer who can get people out of
prison, but I can make cookies,'' Ms. Pelini said. ``And maybe if I sell
someone cookies, it can open a conversation about why we are making
them.''

Recipes:
\textbf{\href{https://cooking.nytimes3xbfgragh.onion/recipes/1021241-extra-flaky-pie-crust}{Extra-Flaky
Pie Crust}} \textbar{}
\textbf{\href{https://cooking.nytimes3xbfgragh.onion/recipes/1021240-berry-hand-pies}{Berry
Hand Pies}}

\emph{Follow} \href{https://twitter.com/nytfood}{\emph{NYT Food on
Twitter}} \emph{and}
\href{https://www.instagram.com/nytcooking/}{\emph{NYT Cooking on
Instagram}}\emph{,}
\href{https://www.facebookcorewwwi.onion/nytcooking/}{\emph{Facebook}}\emph{,}
\href{https://www.youtube.com/nytcooking}{\emph{YouTube}} \emph{and}
\href{https://www.pinterest.com/nytcooking/}{\emph{Pinterest}}\emph{.}
\href{https://www.nytimes3xbfgragh.onion/newsletters/cooking}{\emph{Get
regular updates from NYT Cooking, with recipe suggestions, cooking tips
and shopping advice}}\emph{.}

Advertisement

\protect\hyperlink{after-bottom}{Continue reading the main story}

\hypertarget{site-index}{%
\subsection{Site Index}\label{site-index}}

\hypertarget{site-information-navigation}{%
\subsection{Site Information
Navigation}\label{site-information-navigation}}

\begin{itemize}
\tightlist
\item
  \href{https://help.nytimes3xbfgragh.onion/hc/en-us/articles/115014792127-Copyright-notice}{©~2020~The
  New York Times Company}
\end{itemize}

\begin{itemize}
\tightlist
\item
  \href{https://www.nytco.com/}{NYTCo}
\item
  \href{https://help.nytimes3xbfgragh.onion/hc/en-us/articles/115015385887-Contact-Us}{Contact
  Us}
\item
  \href{https://www.nytco.com/careers/}{Work with us}
\item
  \href{https://nytmediakit.com/}{Advertise}
\item
  \href{http://www.tbrandstudio.com/}{T Brand Studio}
\item
  \href{https://www.nytimes3xbfgragh.onion/privacy/cookie-policy\#how-do-i-manage-trackers}{Your
  Ad Choices}
\item
  \href{https://www.nytimes3xbfgragh.onion/privacy}{Privacy}
\item
  \href{https://help.nytimes3xbfgragh.onion/hc/en-us/articles/115014893428-Terms-of-service}{Terms
  of Service}
\item
  \href{https://help.nytimes3xbfgragh.onion/hc/en-us/articles/115014893968-Terms-of-sale}{Terms
  of Sale}
\item
  \href{https://spiderbites.nytimes3xbfgragh.onion}{Site Map}
\item
  \href{https://help.nytimes3xbfgragh.onion/hc/en-us}{Help}
\item
  \href{https://www.nytimes3xbfgragh.onion/subscription?campaignId=37WXW}{Subscriptions}
\end{itemize}
