Sections

SEARCH

\protect\hyperlink{site-content}{Skip to
content}\protect\hyperlink{site-index}{Skip to site index}

\href{https://www.nytimes3xbfgragh.onion/section/technology}{Technology}

\href{https://myaccount.nytimes3xbfgragh.onion/auth/login?response_type=cookie\&client_id=vi}{}

\href{https://www.nytimes3xbfgragh.onion/section/todayspaper}{Today's
Paper}

\href{/section/technology}{Technology}\textbar{}Beware the `But China'
Excuses

\href{https://nyti.ms/2CXCAIB}{https://nyti.ms/2CXCAIB}

\begin{itemize}
\item
\item
\item
\item
\item
\end{itemize}

Advertisement

\protect\hyperlink{after-top}{Continue reading the main story}

Supported by

\protect\hyperlink{after-sponsor}{Continue reading the main story}

on tech

\hypertarget{beware-the-but-china-excuses}{%
\section{Beware the `But China'
Excuses}\label{beware-the-but-china-excuses}}

Just because U.S. politicians and tech giants are blaming China, doesn't
mean we should be scared.

\includegraphics{https://static01.graylady3jvrrxbe.onion/images/2020/07/27/business/27ontech/21ontech-articleLarge-v5.jpg?quality=75\&auto=webp\&disable=upscale}

\href{https://www.nytimes3xbfgragh.onion/by/shira-ovide}{\includegraphics{https://static01.graylady3jvrrxbe.onion/images/2020/03/18/reader-center/author-shira-ovide/author-shira-ovide-thumbLarge-v2.png}}

By \href{https://www.nytimes3xbfgragh.onion/by/shira-ovide}{Shira Ovide}

\begin{itemize}
\item
  July 21, 2020
\item
  \begin{itemize}
  \item
  \item
  \item
  \item
  \item
  \end{itemize}
\end{itemize}

\emph{This article is part of the On Tech newsletter. You can}
\href{https://www.nytimes3xbfgragh.onion/newsletters/signup/OT}{\emph{sign
up here}} \emph{to receive it weekdays.}

Technology is part of the
\href{https://www.nytimes3xbfgragh.onion/2020/07/14/world/asia/cold-war-china-us.html}{tug
of war} between the
\href{https://www.nytimes3xbfgragh.onion/2020/07/22/world/asia/us-china-houston-consulate.html}{United
States and China}. But let me offer some advice: When you hear an
American technology executive mention China, put on your hmmmm face.
Ditto when you hear a U.S. government official talk about China in the
context of technology.

U.S. tech companies love to suggest that anything that hurts them
somehow opens the door to China's technology dominance. And American
politicians sometimes appear to fan fears of Chinese technology for
selfish reasons.

There are legitimate concerns about China's shaping global technology
norms,
\href{https://www.nytimes3xbfgragh.onion/2020/01/15/business/china-technology-transfer.html}{seeking
to swipe America's tech secrets},
\href{https://www.washingtonpost.com/national-security/us-china-covid-19-vaccine-research/2020/07/21/8b6ca0c0-cb58-11ea-91f1-28aca4d833a0_story.html}{sponsoring
criminal hackers} or using tech for political aims. But how can the
American public fairly evaluate technology policy when powerful people
and companies use ``but China \ldots'' as a catchall boogeyman?

I get a pain behind my eyes when a U.S. tech boss brings up China.
``Breakup strengthens Chinese companies,'' read
\href{https://www.nytimes3xbfgragh.onion/2018/04/11/us/mark-zuckerberg-senate-hearing-notes.html}{notes
from Facebook's Mark Zuckerberg} for a congressional hearing two years
ago. We'll likely hear lines like this when he and three other American
tech C.E.O.s
\href{https://www.nytimes3xbfgragh.onion/2020/07/01/technology/amazon-apple-alphabet-facebook-congress-antitrust.html}{testify
at a congressional hearing} next week about their companies' power.

The implication --- often
\href{https://techcrunch.com/2019/07/17/facebook-or-china/}{repeated} by
Zuckerberg and
\href{https://www.wired.com/story/big-tech-breaking-will-only-help-china/}{other}
tech
\href{https://www.cnn.com/2019/06/14/tech/sundar-pichai-google-antitrust/index.html}{executives}
--- is that if U.S. lawmakers put speed bumps in front of America's
digital stars, then \ldots{} something something \ldots{} China will
rule technology.

If it feels like I skipped some steps, that's what the tech companies
are doing. If they brought up worries about China giving a leg up to
homegrown tech companies through financial assistance or
\href{https://www.nytimes3xbfgragh.onion/2020/04/22/us/politics/coronavirus-china-disinformation.html}{spreading
government propaganda online}, I might have some sympathy. But what
they're doing instead is fanning China fear in a disingenuous attempt to
distract us.

Americans can want our companies to thrive AND want our corporations to
have guardrails. Protecting Americans from potential technology abuses
doesn't destroy U.S. innovation or strengthen any Chinese company.

U.S. government officials have a ``but China'' problem, too. President
Trump's campaign ran ads on Facebook in recent days
\href{https://www.cnn.com/2020/07/19/politics/trump-campaign-ads-facebook-tiktok}{that
accused TikTok}, the video app from the Chinese internet company
ByteDance, of spying on Americans by harvesting information from
people's smartphones. (The researchers who identified the data
collection mentioned in the ads said that
\href{https://www.mysk.blog/2020/03/10/popular-iphone-and-ipad-apps-snooping-on-the-pasteboard/}{many
other smartphone apps do the same thing}.)

Look, I understand in principle why U.S. officials are
\href{https://www.nytimes3xbfgragh.onion/2020/07/15/technology/tiktok-washington-lobbyist.html}{concerned
about TikTok's corporate owner}, and likewise about the role of China's
technology giant Huawei as an
\href{https://www.nytimes3xbfgragh.onion/2020/07/14/business/huawei-uk-5g.html}{essential
cog in the internet}. It's also fair for politicians
\href{https://www.nytimes3xbfgragh.onion/2020/07/16/us/politics/barr-china.html}{and
Attorney General William P. Barr} to question whether American tech
companies are hurting the country when they comply with Chinese laws and
norms.

But politicians, like American tech bosses, engage in fearmongering
about Chinese tech so often that it's hard to know when to believe them.

If politicians and policymakers wanted to do more to support U.S.
technology, they could consider investing more money domestically in
critical tech,
\href{https://www.nytimes3xbfgragh.onion/2020/05/14/technology/trump-tsmc-us-chip-facility.html}{including
computer chips} and artificial intelligence. They could help more
foreign students stay in the United States to
\href{https://www.theatlantic.com/business/archive/2018/06/trump-immigrant-entrepreneurs/561989/}{start
the next great American companies}. They could craft effective public
health measures to tame a pandemic that threatens the competitiveness of
U.S. companies.

Or they can say ``but China'' and let us fill in the blanks in this
catchall fear tactic. When you hear U.S. tech executives or politicians
blurt out China, remind yourself to consider what they really mean.

\emph{If you don't already get this newsletter in your inbox,}
\href{https://www.nytimes3xbfgragh.onion/newsletters/signup/OT}{\emph{please
sign up here}}\emph{.}

\begin{center}\rule{0.5\linewidth}{\linethickness}\end{center}

\hypertarget{three-suggestions-to-bridge-the-internet-divide}{%
\subsection{Three suggestions to bridge the internet
divide}\label{three-suggestions-to-bridge-the-internet-divide}}

One of the inequalities that this pandemic has exposed is just
\href{https://www.nytimes3xbfgragh.onion/2020/06/05/us/coronavirus-education-lost-learning.html}{how
devastating it can be} for Americans who can't access or afford reliable
internet service for remote work, school and other activities.

In
\href{https://www.nytimes3xbfgragh.onion/2020/07/18/opinion/sunday/broadband-internet-access-civil-rights.html}{a
recent editorial}, The New York Times wrote that it can't be acceptable
for millions of Americans
\href{https://www.nytimes3xbfgragh.onion/2020/07/18/opinion/sunday/broadband-internet-access-civil-rights.html}{to
go without an essential service of modern life}, or be forced to
\href{https://www.nytimes3xbfgragh.onion/2020/05/05/technology/parking-lots-wifi-coronavirus.html}{sit
in parking lots} to piggyback on reliable internet service.

I asked Greg Bensinger, a
\href{https://www.nytimes3xbfgragh.onion/interactive/2018/opinion/editorialboard.html}{member
of the Opinion section's editorial board}, to follow up with three
policies that he believed would help improve the country's internet
service. Here's what he said:

\textbf{1) We need a better count of who lacks internet access:} The
first challenge is correctly tallying how many Americans don't have
access to fast internet service. The Federal Communications Commission's
estimate of 21 million relied on self-reporting from internet service
providers, but they can count an entire census block as being covered
\href{https://www.nytimes3xbfgragh.onion/2020/05/05/technology/rural-america-digital-divide-coronavirus.html}{if
just one address has fast internet access}.

Independent estimates place the number at 42 million or more without
good service.

It's important to get it right because these numbers dictate federal
spending on extending broadband service. The F.C.C. should instead use
accounting methods that are
\href{https://broadbandnow.com/research/fcc-underestimates-unserved-by-50-percent}{independently
verifiable} and dive into census blocks to see the breadth of coverage.

\textbf{2) Revamp a subsidy for low-income Americans:} In cities that
generally have good internet networks, monthly subscription fees are a
burden, as are costs for laptops and tablets to access the internet. The
F.C.C. has programs for subsidizing telephone service for low-income
Americans --- particularly the one known as Lifeline --- that could be
revamped to include subsidies for monthly internet costs.

\textbf{3) Get the government involved upfront in paying for internet
infrastructure:} The federal government should directly fund an internet
build out. As it stands today, internet service providers generally seek
subsidies from the government after they have built internet networks.
That doesn't give companies an incentive to reach areas where profits
may be harder to come by because the funds aren't guaranteed upfront.

\begin{center}\rule{0.5\linewidth}{\linethickness}\end{center}

\hypertarget{before-we-go-}{%
\subsection{Before we go \ldots{}}\label{before-we-go-}}

\begin{itemize}
\item
  \textbf{Privacy loopholes in virus-fighting software:} Google and
  Apple have collaborated on smartphone technology that they pitched as
  a privacy-preserving way for health authorities to identify people who
  might have been exposed to the coronavirus. And yet, as my colleague
  Natasha Singer wrote, on Google's Android phones, the technology
  requires people to turn on their location settings, which
  \href{https://www.nytimes3xbfgragh.onion/2020/07/20/technology/google-covid-tracker-app.html}{could
  let Google follow those people as they roam around}.

  And in South Korea, among the world's leaders in digital public
  health, a mobile app that helps enforce coronavirus quarantines was
  found to have
  \href{https://www.nytimes3xbfgragh.onion/2020/07/21/technology/korea-coronavirus-app-security.html}{serious
  security flaws that made people's private information vulnerable to
  hackers}. The security holes were fixed, my colleagues reported.
\item
  \textbf{Virtual musical concerts might be pretty good now?} In the
  beginning of the pandemic, concerts streamed live on Instagram were
  pretty awkward. Musicians and fans have started to figure out how and
  where to hold relatively effective live performances online. My
  colleagues
  \href{https://www.nytimes3xbfgragh.onion/2020/07/21/arts/music/concerts-livestreams.html}{have
  a run down on lingering questions} --- like will people pay big bucks
  for this? --- and suggestions for the
  \href{https://www.nytimes3xbfgragh.onion/2020/07/21/arts/music/best-quarantine-concerts-livestream.html}{10
  best quarantine-era virtual concerts}.
\item
  \textbf{Internet stars face blowback from their product backlash:}
  Some online influencers and celebrities in Sri Lanka
  \href{https://www.buzzfeednews.com/article/meghara/unilever-sri-lankan-influencers-skin-whiteners}{told
  BuzzFeed News} that they faced professional backlash for publicly
  opposing skin lightening creams that they believed perpetuated the
  racist idea that light skin was more desirable.
\end{itemize}

\hypertarget{hugs-to-this}{%
\subsubsection{Hugs to this}\label{hugs-to-this}}

\href{https://twitter.com/DannyDeraney/status/1284883502844874752}{A man
and his dog are each eating a burrito}. You're welcome.

\begin{center}\rule{0.5\linewidth}{\linethickness}\end{center}

\emph{We want to hear from you. Tell us what you think of this
newsletter and what else you'd like us to explore. You can reach us at}
\href{mailto:ontech@NYTimes.com?subject=On\%20Tech\%20Feedback}{\emph{ontech@NYTimes.com.}}
**

\emph{If you don't already get this newsletter in your inbox,}
\href{https://www.nytimes3xbfgragh.onion/newsletters/signup/OT}{\emph{please
sign up here}}\emph{.}

Advertisement

\protect\hyperlink{after-bottom}{Continue reading the main story}

\hypertarget{site-index}{%
\subsection{Site Index}\label{site-index}}

\hypertarget{site-information-navigation}{%
\subsection{Site Information
Navigation}\label{site-information-navigation}}

\begin{itemize}
\tightlist
\item
  \href{https://help.nytimes3xbfgragh.onion/hc/en-us/articles/115014792127-Copyright-notice}{©~2020~The
  New York Times Company}
\end{itemize}

\begin{itemize}
\tightlist
\item
  \href{https://www.nytco.com/}{NYTCo}
\item
  \href{https://help.nytimes3xbfgragh.onion/hc/en-us/articles/115015385887-Contact-Us}{Contact
  Us}
\item
  \href{https://www.nytco.com/careers/}{Work with us}
\item
  \href{https://nytmediakit.com/}{Advertise}
\item
  \href{http://www.tbrandstudio.com/}{T Brand Studio}
\item
  \href{https://www.nytimes3xbfgragh.onion/privacy/cookie-policy\#how-do-i-manage-trackers}{Your
  Ad Choices}
\item
  \href{https://www.nytimes3xbfgragh.onion/privacy}{Privacy}
\item
  \href{https://help.nytimes3xbfgragh.onion/hc/en-us/articles/115014893428-Terms-of-service}{Terms
  of Service}
\item
  \href{https://help.nytimes3xbfgragh.onion/hc/en-us/articles/115014893968-Terms-of-sale}{Terms
  of Sale}
\item
  \href{https://spiderbites.nytimes3xbfgragh.onion}{Site Map}
\item
  \href{https://help.nytimes3xbfgragh.onion/hc/en-us}{Help}
\item
  \href{https://www.nytimes3xbfgragh.onion/subscription?campaignId=37WXW}{Subscriptions}
\end{itemize}
