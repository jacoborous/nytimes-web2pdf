Sections

SEARCH

\protect\hyperlink{site-content}{Skip to
content}\protect\hyperlink{site-index}{Skip to site index}

\href{https://www.nytimes3xbfgragh.onion/section/climate}{Climate}

\href{https://myaccount.nytimes3xbfgragh.onion/auth/login?response_type=cookie\&client_id=vi}{}

\href{https://www.nytimes3xbfgragh.onion/section/todayspaper}{Today's
Paper}

\href{/section/climate}{Climate}\textbar{}New Emails Show How Energy
Industry Moved Fast to Undo Curbs

\url{https://nyti.ms/39fLNbl}

\begin{itemize}
\item
\item
\item
\item
\item
\item
\end{itemize}

\href{https://www.nytimes3xbfgragh.onion/section/climate?action=click\&pgtype=Article\&state=default\&region=TOP_BANNER\&context=storylines_menu}{Climate
and Environment}

\begin{itemize}
\tightlist
\item
  \href{https://www.nytimes3xbfgragh.onion/2020/07/30/climate/sea-level-inland-floods.html?action=click\&pgtype=Article\&state=default\&region=TOP_BANNER\&context=storylines_menu}{Rising
  Seas}
\item
  \href{https://www.nytimes3xbfgragh.onion/interactive/2020/climate/trump-environment-rollbacks.html?action=click\&pgtype=Article\&state=default\&region=TOP_BANNER\&context=storylines_menu}{Trump's
  Changes}
\item
  \href{https://www.nytimes3xbfgragh.onion/interactive/2020/04/19/climate/climate-crash-course-1.html?action=click\&pgtype=Article\&state=default\&region=TOP_BANNER\&context=storylines_menu}{Climate
  101}
\item
  \href{https://www.nytimes3xbfgragh.onion/interactive/2018/08/30/climate/how-much-hotter-is-your-hometown.html?action=click\&pgtype=Article\&state=default\&region=TOP_BANNER\&context=storylines_menu}{Is
  Your Hometown Hotter?}
\item
  \href{https://www.nytimes3xbfgragh.onion/newsletters/climate-change?action=click\&pgtype=Article\&state=default\&region=TOP_BANNER\&context=storylines_menu}{Newsletter}
\end{itemize}

Advertisement

\protect\hyperlink{after-top}{Continue reading the main story}

Supported by

\protect\hyperlink{after-sponsor}{Continue reading the main story}

\hypertarget{new-emails-show-how-energy-industry-moved-fast-to-undo-curbs}{%
\section{New Emails Show How Energy Industry Moved Fast to Undo
Curbs}\label{new-emails-show-how-energy-industry-moved-fast-to-undo-curbs}}

The messages, made public in a lawsuit, suggest the E.P.A. rescinded a
requirement on methane at the behest of an executive just weeks after
President Trump took office.

\includegraphics{https://static01.graylady3jvrrxbe.onion/images/2020/07/20/climate/00CLI-METHANELAWSUIT1/00CLI-METHANELAWSUIT1-articleLarge.jpg?quality=75\&auto=webp\&disable=upscale}

\href{https://www.nytimes3xbfgragh.onion/by/lisa-friedman}{\includegraphics{https://static01.graylady3jvrrxbe.onion/images/2018/07/18/multimedia/author-lisa-friedman/author-lisa-friedman-thumbLarge.png}}

By \href{https://www.nytimes3xbfgragh.onion/by/lisa-friedman}{Lisa
Friedman}

\begin{itemize}
\item
  Published July 21, 2020Updated July 24, 2020
\item
  \begin{itemize}
  \item
  \item
  \item
  \item
  \item
  \item
  \end{itemize}
\end{itemize}

WASHINGTON --- Not long after President Trump's inauguration, the head
of a fossil fuels industry group requested a call with the president's
transition team. The subject: Barack Obama's requirement that oil and
gas companies begin collecting data on their releases of methane.

That outreach, by Kathleen Sgamma, president of the Western Energy
Alliance, appeared to quickly yield the desired results.

``Looks like this will be easier than we thought,'' David Kreutzer, an
economist who was helping to organize the new president's Environmental
Protection Agency, wrote of canceling the methane reporting requirement
in an email to another member of the transition team on Feb. 10, 2017.

Three weeks after that email, the E.P.A. officially withdrew the
reporting requirement --- and effectively blocked the compilation of
data that would allow for new regulations to control methane, a powerful
climate-warming gas.

The
\href{https://int.graylady3jvrrxbe.onion/data/documenttools/methane-ic-remails/a16ae11accccfb2c/full.pdf}{emails}
are included in hundreds of pages of E.P.A. staff correspondence and
interviews recently made public in a lawsuit that 15 states have brought
against the agency over the regulation of methane. Led by Massachusetts
and New York, the states say the documents prove that fossil fuel
industry players, working with allies in the early days of Mr. Trump's
E.P.A., engineered the repeal of the methane reporting requirements with
no internal analysis, then created the rationale for the decision after
the fact.

That repeal, the states assert, illegally delayed the development of
additional regulations to reduce methane emissions that the
administration did not want.

If the states succeed, a judge could, as early as this summer, order the
federal government to impose restrictions on thousands of oil and gas
wells, storage facilities and pipelines across the United States. Just
last week, a federal court, restoring an Obama-era regulation, struck
down a Bureau of Land Management effort to weaken restrictions on
methane gas releases from drilling on public lands.

In that case, Judge Yvonne Gonzalez Rogers ruled that the Trump
administration, in its ``haste'' and ``zeal,'' failed to properly
justify its rollback.

``In the early days they did very little justification,'' said Richard
Revesz, a professor of environmental law at New York University and
director of the Institute for Policy Integrity, the university's
nonpartisan think tank. So far, only about 10 percent of the Trump
administration's deregulatory efforts have held up in court, according
to the institute, compared to an average of 70 percent for other
administrations, both Republican and Democratic.

``They justify their policies on analytically flimsy or sometimes
nonexistent grounds, thinking, I guess, that they will get away with
it,'' Mr. Revesz said. ``But time and again, the courts say no.''

\href{https://www.nytimes3xbfgragh.onion/section/climate?action=click\&pgtype=Article\&state=default\&region=MAIN_CONTENT_1\&context=storylines_keepup}{}

\hypertarget{climate-and-environment-}{%
\subsubsection{Climate and Environment
›}\label{climate-and-environment-}}

\hypertarget{keep-up-on-the-latest-climate-news}{%
\paragraph{Keep Up on the Latest Climate
News}\label{keep-up-on-the-latest-climate-news}}

Updated July 30, 2020

Here's what you need to know about the latest climate change news this
week:

\begin{itemize}
\item
  \begin{itemize}
  \tightlist
  \item
    \href{https://www.nytimes3xbfgragh.onion/2020/07/30/climate/bangladesh-floods.html?action=click\&pgtype=Article\&state=default\&region=MAIN_CONTENT_1\&context=storylines_keepup}{Floods
    in}\href{https://www.nytimes3xbfgragh.onion/2020/07/30/climate/bangladesh-floods.html?action=click\&pgtype=Article\&state=default\&region=MAIN_CONTENT_1\&context=storylines_keepup}{Bangladesh}
    are punishing the people least responsible for climate change.
  \item
    As climate change raises sea levels,
    \href{https://www.nytimes3xbfgragh.onion/2020/07/30/climate/sea-level-inland-floods.html?action=click\&pgtype=Article\&state=default\&region=MAIN_CONTENT_1\&context=storylines_keepup}{storm
    surges and high tides} are likely to push farther inland.
  \item
    The E.P.A. inspector general plans to investigate whether a rollback
    of fuel efficiency standards
    \href{https://www.nytimes3xbfgragh.onion/2020/07/27/climate/trump-fuel-efficiency-rule.html?action=click\&pgtype=Article\&state=default\&region=MAIN_CONTENT_1\&context=storylines_keepup}{violated
    government rules}.
  \end{itemize}
\end{itemize}

Methane, which leaks from oil and gas wells, accounts for about 10
percent of greenhouse gas emissions from human activity in the United
States,
\href{https://www.epa.gov/ghgemissions/overview-greenhouse-gases}{according
to E.P.A. data}. But it is about 30 times more potent over the course of
a century than carbon dioxide in altering the Earth's climate and is
responsible for about a quarter of man-made global warming.

Several attorneys general have filed a
\href{https://www.mass.gov/doc/methane-summary-judgment/download}{motion
for summary judgment} with the United States Circuit Court for the
District of Columbia asking it to compel the E.P.A. to set new
standards.

``From the start, we've known the Trump Administration has been more
interested in greasing the skids for the fossil fuel industry to make
billions than protecting the health of our communities, and now we've
got the receipts,'' said Maura Healey, the attorney general of
Massachusetts.

\includegraphics{https://static01.graylady3jvrrxbe.onion/images/2020/07/20/climate/00CLI-METHANELAWSUIT2/merlin_160903677_ee5304ca-0945-47cf-a9fb-fd8bf79eb3a7-articleLarge.jpg?quality=75\&auto=webp\&disable=upscale}

James Hewitt, an E.P.A. spokesman, declined to comment on the substance
of the lawsuit's allegations, saying in a statement the agency intended
to file its response by Aug. 14.

Ms. Sgamma and Mr. Kreutzer said that, because it was clear that the
Trump administration would have a different policy on methane from the
Obama administration, career staff members at the E.P.A. agreed that
continuing with a requirement to collect data on releases of methane no
longer made sense.

``It would have been a waste of time to submit data that weren't going
to be used,'' Ms. Sgamma said of her email. ``I merely called this to
their attention.''

Mr. Kreutzer,
\href{https://www.heritage.org/environment/report/the-state-climate-science-no-justification-extreme-policies}{who
rejects the scientific consensus on climate change} and is now a senior
economist at the Institute for Energy Research, a research organization
that supports fossil fuels, said he did not recall ``any push back or
controversy'' over the decision. ``It would have been the worst sort of
bureaucratic indifference to impose a costly burden to collect
information that no longer had a purpose,'' he said.

Mr. Obama's 2015 methane regulation was part of a groundbreaking series
of federal climate regulations. With a goal of cutting methane emissions
in half by 2025, the rule, which was completed the following year,
required companies to install technology to detect and fix methane leaks
in all new and heavily modified facilities.

Under the Clean Air Act, when the E.P.A. moves to regulate pollution
from new sources, the agency must also develop pollution standards for
existing sources. In preparation for developing a second regulation for
existing facilities, the agency in late 2016 required companies to
report information about their emissions, their equipment and their
methane controls. The so-called information collection request became
known as the I.C.R.

When Mr. Trump took office, the government became more receptive to the
concerns of the oil and gas industry. In addition to eliminating the
reporting request, the E.P.A. also moved to
\href{https://www.nytimes3xbfgragh.onion/2018/09/10/climate/methane-emissions-epa.html}{weaken
the Obama administration's methane rules} for new facilities. A final
version of those regulations are expected out this summer.

According to the emails made public in the lawsuit, Ms. Sgamma reached
out to Mr. Kreutzer on Feb. 2. ``I know you're under water right now,''
she wrote, but she hoped they might find time to talk about the
reporting requirement that she said was creating ``confusion'' for
companies.

The two spoke and on Feb. 10. Ms. Sgamma followed up with another email
outlining ``key rationales'' for eliminating the reporting requirement,
or to allow companies more time.

``It seems unlikely that the new E.P.A. will approach this `existing'
source regulation in the same way,'' Ms. Sgamma wrote. If the agency is
not likely to regulate current sources of methane, she added, ``then it
does not make sense for every operator in the country to go through this
burdensome information request.''

That day, Mr. Kreutzer called Sarah Dunham, then the acting
administrator of the E.P.A.'s air office, and contacted David W.
Schnare, a longtime opponent of climate science and another member of
the E.P.A. transition team.

In an afternoon email to Ms. Dunham with the subject line ``Re: Quashing
the I.C.R.,'' Mr. Kreutzer asked her to draft a request to withdraw the
methane information collection.

On March 2, the E.P.A. administrator, Scott Pruitt, formally announced
the immediate withdrawal of the information request.

``Today's action will reduce burdens on businesses while we take a
closer look at the need for additional information from this industry,''
he said.

The agency later said the decision brought the E.P.A. into compliance
with Mr. Trump's executive order
to\href{https://www.nytimes3xbfgragh.onion/2017/03/28/climate/trump-executive-order-climate-change.html}{roll
back his predecessor's climate change regulations} --- but that
executive order was issued March 28, almost a month after the agency had
stopped collecting data.

Image

Scott Pruitt during a 2017 confirmation hearing before a Senate
subcommittee to be E.P.A. administrator.Credit...Gabriella Demczuk for
The New York Times

Meantime, the emails show, the morning of Mr. Pruitt's announcement, top
E.P.A. political staff had not yet prepared a legal justification for
the decision.

``The Administrator wants this turned into a Notice for Federal Register
publication and he wants it over there today for publication tomorrow,''
Mr. Schnare wrote to career staff just before 9 a.m. on March 2, adding,
``It can be literally three sentences long.''

Regulations can be legally revoked only after lengthy analyses and
public comment periods; information collection requests can simply be
canceled. Still, depositions show, senior E.P.A. officials in charge of
the program said there was no internal review and that they only became
aware of the basis for revoking the request shortly before Mr. Pruitt's
announcement.

In an interview, Mr. Schnare said plans to end the information request
and the rest of Mr. Obama's climate policies long predated Ms. Sgamma's
letter.

``We were already working on this,'' he said. ``She wasn't the first
person to tell us, `Hey, this is a problem.'''

The emails appear to tell a different story. Late in the day of Mr.
Pruitt's announcement, Ms. Sgamma emailed Mr. Kreutzer, ``From the
bottom of my heart, thank you.''

Mr. Kreutzer responded immediately, ``Thank you for bringing it to our
attention.''

``With all the commotion of the transition, the very sensible proposal
to cancel the I.C.R. fell through the cracks,'' he added. ``Kudos to you
for being alert!''

Advertisement

\protect\hyperlink{after-bottom}{Continue reading the main story}

\hypertarget{site-index}{%
\subsection{Site Index}\label{site-index}}

\hypertarget{site-information-navigation}{%
\subsection{Site Information
Navigation}\label{site-information-navigation}}

\begin{itemize}
\tightlist
\item
  \href{https://help.nytimes3xbfgragh.onion/hc/en-us/articles/115014792127-Copyright-notice}{©~2020~The
  New York Times Company}
\end{itemize}

\begin{itemize}
\tightlist
\item
  \href{https://www.nytco.com/}{NYTCo}
\item
  \href{https://help.nytimes3xbfgragh.onion/hc/en-us/articles/115015385887-Contact-Us}{Contact
  Us}
\item
  \href{https://www.nytco.com/careers/}{Work with us}
\item
  \href{https://nytmediakit.com/}{Advertise}
\item
  \href{http://www.tbrandstudio.com/}{T Brand Studio}
\item
  \href{https://www.nytimes3xbfgragh.onion/privacy/cookie-policy\#how-do-i-manage-trackers}{Your
  Ad Choices}
\item
  \href{https://www.nytimes3xbfgragh.onion/privacy}{Privacy}
\item
  \href{https://help.nytimes3xbfgragh.onion/hc/en-us/articles/115014893428-Terms-of-service}{Terms
  of Service}
\item
  \href{https://help.nytimes3xbfgragh.onion/hc/en-us/articles/115014893968-Terms-of-sale}{Terms
  of Sale}
\item
  \href{https://spiderbites.nytimes3xbfgragh.onion}{Site Map}
\item
  \href{https://help.nytimes3xbfgragh.onion/hc/en-us}{Help}
\item
  \href{https://www.nytimes3xbfgragh.onion/subscription?campaignId=37WXW}{Subscriptions}
\end{itemize}
