Sections

SEARCH

\protect\hyperlink{site-content}{Skip to
content}\protect\hyperlink{site-index}{Skip to site index}

\href{https://www.nytimes3xbfgragh.onion/section/climate}{Climate}

\href{https://myaccount.nytimes3xbfgragh.onion/auth/login?response_type=cookie\&client_id=vi}{}

\href{https://www.nytimes3xbfgragh.onion/section/todayspaper}{Today's
Paper}

\href{/section/climate}{Climate}\textbar{}Big Tech Has a Big Climate
Problem. Now, It's Being Forced to Clean Up.

\url{https://nyti.ms/39hBcg8}

\begin{itemize}
\item
\item
\item
\item
\item
\end{itemize}

\href{https://www.nytimes3xbfgragh.onion/section/climate?action=click\&pgtype=Article\&state=default\&region=TOP_BANNER\&context=storylines_menu}{Climate
and Environment}

\begin{itemize}
\tightlist
\item
  \href{https://www.nytimes3xbfgragh.onion/2020/07/30/climate/sea-level-inland-floods.html?action=click\&pgtype=Article\&state=default\&region=TOP_BANNER\&context=storylines_menu}{Rising
  Seas}
\item
  \href{https://www.nytimes3xbfgragh.onion/interactive/2020/climate/trump-environment-rollbacks.html?action=click\&pgtype=Article\&state=default\&region=TOP_BANNER\&context=storylines_menu}{Trump's
  Changes}
\item
  \href{https://www.nytimes3xbfgragh.onion/interactive/2020/04/19/climate/climate-crash-course-1.html?action=click\&pgtype=Article\&state=default\&region=TOP_BANNER\&context=storylines_menu}{Climate
  101}
\item
  \href{https://www.nytimes3xbfgragh.onion/interactive/2018/08/30/climate/how-much-hotter-is-your-hometown.html?action=click\&pgtype=Article\&state=default\&region=TOP_BANNER\&context=storylines_menu}{Is
  Your Hometown Hotter?}
\item
  \href{https://www.nytimes3xbfgragh.onion/newsletters/climate-change?action=click\&pgtype=Article\&state=default\&region=TOP_BANNER\&context=storylines_menu}{Newsletter}
\end{itemize}

Advertisement

\protect\hyperlink{after-top}{Continue reading the main story}

Supported by

\protect\hyperlink{after-sponsor}{Continue reading the main story}

\hypertarget{big-tech-has-a-big-climate-problem-now-its-being-forced-to-clean-up}{%
\section{Big Tech Has a Big Climate Problem. Now, It's Being Forced to
Clean
Up.}\label{big-tech-has-a-big-climate-problem-now-its-being-forced-to-clean-up}}

Apple said on Tuesday its devices would be carbon-neutral by 2030,
making it the latest tech giant to ramp up voluntary climate targets.

\includegraphics{https://static01.graylady3jvrrxbe.onion/images/2020/07/21/climate/21CLI-APPLE1/21CLI-APPLE1-articleLarge.jpg?quality=75\&auto=webp\&disable=upscale}

By \href{https://www.nytimes3xbfgragh.onion/by/somini-sengupta}{Somini
Sengupta} and Veronica Penney

Ms. Sengupta is a former technology reporter at The Times who now covers
climate change. Ms. Penney is a member of the 2020 Times Fellowship
class.

\begin{itemize}
\item
  July 21, 2020
\item
  \begin{itemize}
  \item
  \item
  \item
  \item
  \item
  \end{itemize}
\end{itemize}

The titans of the tech industry like to think of themselves as solvers
of big world problems, and, lately, they're tripping over themselves to
show that they are working to solve a problem for which they, too, are
culpable: climate change.

Apple on Tuesday became the latest tech giant to promise to do more to
reduce the emissions of planet-warming greenhouse gases, announcing
\href{https://www.apple.com/newsroom/2020/07/apple-commits-to-be-100-percent-carbon-neutral-for-its-supply-chain-and-products-by-2030/}{in
a statement} that, by 2030, ``every Apple device sold will have net-zero
climate impact.''

Apple said it aimed to reduce emissions by 75 percent in its
manufacturing chain, including by recycling more of the components that
go into each device and nudging its suppliers to use renewable energy.
As for the remaining 25 percent of emissions, the company said it
planned to balance them by funding reforestation projects. The company
also said it planned to improve energy efficiency in its operations.

Forests absorb carbon dioxide, and reforestation has become a popular
way for companies to offset the greenhouse gas emissions that they
produce, including from factories.

Climate advocates describe these offset efforts as inadequate because
they allow emissions to grow at a time when the scientific consensus
demands that emissions be cut in half by 2030 in order to avoid the
worst effects of climate change --- and be reduced to zero by 2050.

Separately on Tuesday, Microsoft announced that it would
\href{https://blogs.microsoft.com/on-the-issues/2020/07/21/carbon-negative-transform-to-net-zero/}{require
its suppliers to report their emissions}, as a first step toward making
reductions.

Like other corporate pledges, both are entirely voluntary.

``It feels like there's a virtuous follow-the-leader thing happening
here,'' said Simon Nicholson, co-director for the Institute for Carbon
Removal Law and Policy at American University.

\href{https://www.nytimes3xbfgragh.onion/section/climate?action=click\&pgtype=Article\&state=default\&region=MAIN_CONTENT_1\&context=storylines_keepup}{}

\hypertarget{climate-and-environment-}{%
\subsubsection{Climate and Environment
›}\label{climate-and-environment-}}

\hypertarget{keep-up-on-the-latest-climate-news}{%
\paragraph{Keep Up on the Latest Climate
News}\label{keep-up-on-the-latest-climate-news}}

Updated July 30, 2020

Here's what you need to know about the latest climate change news this
week:

\begin{itemize}
\item
  \begin{itemize}
  \tightlist
  \item
    \href{https://www.nytimes3xbfgragh.onion/2020/07/30/climate/bangladesh-floods.html?action=click\&pgtype=Article\&state=default\&region=MAIN_CONTENT_1\&context=storylines_keepup}{Floods
    in}\href{https://www.nytimes3xbfgragh.onion/2020/07/30/climate/bangladesh-floods.html?action=click\&pgtype=Article\&state=default\&region=MAIN_CONTENT_1\&context=storylines_keepup}{Bangladesh}
    are punishing the people least responsible for climate change.
  \item
    As climate change raises sea levels,
    \href{https://www.nytimes3xbfgragh.onion/2020/07/30/climate/sea-level-inland-floods.html?action=click\&pgtype=Article\&state=default\&region=MAIN_CONTENT_1\&context=storylines_keepup}{storm
    surges and high tides} are likely to push farther inland.
  \item
    The E.P.A. inspector general plans to investigate whether a rollback
    of fuel efficiency standards
    \href{https://www.nytimes3xbfgragh.onion/2020/07/27/climate/trump-fuel-efficiency-rule.html?action=click\&pgtype=Article\&state=default\&region=MAIN_CONTENT_1\&context=storylines_keepup}{violated
    government rules}.
  \end{itemize}
\end{itemize}

He noted the limitations of the pledge, though. ``What Apple has
signaled here is the beginning of a strategy on the carbon-removal
side,'' Dr. Nicholson said. ``Holding carbon in forests for a year or
two isn't going to cut it. It needs to be held in forests for the long
term, which means centuries.''

Big Tech's role in global warming varies from company to company.
Amazon, Facebook and Google all use enormous amounts of energy and water
for their data centers. Amazon relies on gas-guzzling trucks and
packages that themselves have a huge environmental footprint; even
recycling uses a lot of energy. And makers of devices --- like Amazon,
Apple, Google and Microsoft --- produce greenhouse gas emissions through
their supply chains, which involve contractors that do the actual
manufacturing in different parts of the world.

The pressure on companies to do something about their climate footprint
comes both from within the ranks of their employees and from advocacy
groups on the outside.

Not only are they under scrutiny for the emissions they produce.
Internet companies, like Facebook, have been criticized for allowing the
spread of disinformation about climate science. Greenpeace took aim at
Google, Microsoft and Amazon for using their artificial intelligence and
cloud computing services to help oil producers find and extract oil and
gas deposits, which
\href{https://www.greenpeace.org/usa/reports/oil-in-the-cloud/}{Greenpeace
said is ``significantly undermining''} the tech companies' other climate
commitments.

One by one, the giants of Silicon Valley have been compelled to address
their own role in the climate crisis.

\includegraphics{https://static01.graylady3jvrrxbe.onion/images/2020/07/21/climate/21CLI-APPLE2/merlin_171578934_417ad51b-3def-48b0-9c2e-0be8c52c4cd4-articleLarge.jpg?quality=75\&auto=webp\&disable=upscale}

\href{https://onezero.medium.com/google-says-it-will-not-build-custom-a-i-for-oil-and-gas-extraction-72d1f71f42c8}{Google
said in May} it would no longer build customized artificial intelligence
technology or machine learning algorithms for the oil and gas sector. It
has also pledged to include recycled material in its devices, including
its popular Chromebook computers, by 2022.

\href{https://www.nytimes3xbfgragh.onion/2019/09/19/technology/amazon-carbon-neutral.html?searchResultPosition=1}{Amazon
announced last September its bid to be carbon-neutral} by 2040, while
its chief executive, Jeff Bezos, committed \$10 billion to fund climate
science and advocacy.

Amazon's move came after sustained calls from its own employees to
reduce emissions to zero by 2030, a full 10 years earlier than the
company's current target. Its employees also pressed Mr. Bezos to stop
offering custom cloud-computing services to the oil and gas industry and
to suspend campaign donations to politicians who deny climate science.

Amazon continues to do business with fossil fuel companies, but Mr.
Bezos said the company would take a ``hard look'' at its political
donations. Amazon said it would reduce its climate change impact by,
among other things, buying a fleet of 100,000 electric delivery trucks.
But, like Apple, Amazon's pledge to be net-zero by 2040 relies on
reforestation projects to offset its continuing emissions. The company
has also said it plans to improve energy efficiency in its operations.

Microsoft this year said it would draw down more emissions than it adds
and also somehow
\href{https://www.nytimes3xbfgragh.onion/aponline/2020/01/16/business/bc-us-microsoft-climate-pledge.html}{remove
all the emissions} the company has ever produced. It promised to invest
\$1 billion in what it called climate innovations, but it left untouched
its partnerships with oil and gas companies.

Facebook announced that it would use 100 percent renewable energy in its
facilities and reduce water use in its data centers, though it has said
little about what it will do to stop the spread of climate
disinformation on its platform.

Apple's net-zero pledge is notable in that it seeks to address the main
source of its greenhouse gas emissions: from the manufacturing of its
phones, tablets and computers by its contractor companies.

Apple's statement on Tuesday underscored the need for businesses like it
to pivot away from fossil fuels for the sake of its own bottom line.
``We have a generational opportunity,'' said Lisa Jackson, a company
vice president responsible for environmental issue, ``to help build a
greener and more just economy, one where we develop whole new industries
in the pursuit of giving the next generation a planet worth calling
home.''

Elizabeth Jardim, who works on corporate issues at Greenpeace USA,
cautiously welcomed the pledge as a step up from the company's previous
commitments, noting in an email that ``as with all `carbon neutral' or
`carbon negative' goals, it is critical to see detailed plans for how
the company will pursue deep decarbonization rather than a reliance on
offsetting or weak nature-based carbon removal initiatives.''

Ms. Jardim also urged large, profitable companies to throw their weight
behind policies like the Green New Deal.

On Twitter, Edward Maibach, of the George Mason University Center for
Climate Change Communication,
\href{https://twitter.com/MaibachEd/status/1285571592832131074}{called
the Apple pledge} ``a big step in the right direction, if they make good
on it. Next, they should lobby governments worldwide to increase their
commitments to the Paris Climate Agreement.''

Advertisement

\protect\hyperlink{after-bottom}{Continue reading the main story}

\hypertarget{site-index}{%
\subsection{Site Index}\label{site-index}}

\hypertarget{site-information-navigation}{%
\subsection{Site Information
Navigation}\label{site-information-navigation}}

\begin{itemize}
\tightlist
\item
  \href{https://help.nytimes3xbfgragh.onion/hc/en-us/articles/115014792127-Copyright-notice}{©~2020~The
  New York Times Company}
\end{itemize}

\begin{itemize}
\tightlist
\item
  \href{https://www.nytco.com/}{NYTCo}
\item
  \href{https://help.nytimes3xbfgragh.onion/hc/en-us/articles/115015385887-Contact-Us}{Contact
  Us}
\item
  \href{https://www.nytco.com/careers/}{Work with us}
\item
  \href{https://nytmediakit.com/}{Advertise}
\item
  \href{http://www.tbrandstudio.com/}{T Brand Studio}
\item
  \href{https://www.nytimes3xbfgragh.onion/privacy/cookie-policy\#how-do-i-manage-trackers}{Your
  Ad Choices}
\item
  \href{https://www.nytimes3xbfgragh.onion/privacy}{Privacy}
\item
  \href{https://help.nytimes3xbfgragh.onion/hc/en-us/articles/115014893428-Terms-of-service}{Terms
  of Service}
\item
  \href{https://help.nytimes3xbfgragh.onion/hc/en-us/articles/115014893968-Terms-of-sale}{Terms
  of Sale}
\item
  \href{https://spiderbites.nytimes3xbfgragh.onion}{Site Map}
\item
  \href{https://help.nytimes3xbfgragh.onion/hc/en-us}{Help}
\item
  \href{https://www.nytimes3xbfgragh.onion/subscription?campaignId=37WXW}{Subscriptions}
\end{itemize}
