Sections

SEARCH

\protect\hyperlink{site-content}{Skip to
content}\protect\hyperlink{site-index}{Skip to site index}

\href{https://www.nytimes3xbfgragh.onion/section/arts/television}{Television}

\href{https://myaccount.nytimes3xbfgragh.onion/auth/login?response_type=cookie\&client_id=vi}{}

\href{https://www.nytimes3xbfgragh.onion/section/todayspaper}{Today's
Paper}

\href{/section/arts/television}{Television}\textbar{}Trump's Briefings,
`The Apprentice' and the Perils of the Second Season

\url{https://nyti.ms/3jjS0aW}

\begin{itemize}
\item
\item
\item
\item
\item
\end{itemize}

\begin{itemize}
\item
  \href{https://www.nytimes3xbfgragh.onion/2020/07/31/us/elections/biden-vs-trump.html?action=click\&pgtype=Article\&state=default\&region=TOP_BANNER\&context=storylines_menu}{Election
  Updates}
\item
  \href{https://www.nytimes3xbfgragh.onion/article/biden-vice-president-2020.html?action=click\&pgtype=Article\&state=default\&region=TOP_BANNER\&context=storylines_menu}{Biden's
  V.P. Search}
\item
  \href{https://www.nytimes3xbfgragh.onion/interactive/2020/07/24/us/politics/trump-biden-campaign-donors.html?action=click\&pgtype=Article\&state=default\&region=TOP_BANNER\&context=storylines_menu}{Map
  of Donations}
\item
  \href{https://www.nytimes3xbfgragh.onion/interactive/2020/us/elections/delegate-count-primary-results.html?action=click\&pgtype=Article\&state=default\&region=TOP_BANNER\&context=storylines_menu}{Delegate
  Count}
\item
  \href{https://www.nytimes3xbfgragh.onion/interactive/2019/us/politics/2020-presidential-candidates.html?action=click\&pgtype=Article\&state=default\&region=TOP_BANNER\&context=storylines_menu}{The
  Candidates}
\item
  \href{https://www.nytimes3xbfgragh.onion/newsletters/politics?action=click\&pgtype=Article\&state=default\&region=TOP_BANNER\&context=storylines_menu}{Politics
  Newsletter}
\end{itemize}

Advertisement

\protect\hyperlink{after-top}{Continue reading the main story}

Supported by

\protect\hyperlink{after-sponsor}{Continue reading the main story}

Critic's Notebook

\hypertarget{trumps-briefings-the-apprentice-and-the-perils-of-the-second-season}{%
\section{Trump's Briefings, `The Apprentice' and the Perils of the
Second
Season}\label{trumps-briefings-the-apprentice-and-the-perils-of-the-second-season}}

President Trump seems to think there's no such thing as too much of him
on TV. He's been wrong about that before.

\includegraphics{https://static01.graylady3jvrrxbe.onion/images/2020/07/21/arts/21TRUMPTV1/merlin_174568761_fee73ca3-ec44-43d9-a380-5909f58a1e30-articleLarge.jpg?quality=75\&auto=webp\&disable=upscale}

\href{https://www.nytimes3xbfgragh.onion/by/james-poniewozik}{\includegraphics{https://static01.graylady3jvrrxbe.onion/images/2018/02/16/multimedia/author-james-poniewozik/author-james-poniewozik-thumbLarge.jpg}}

By \href{https://www.nytimes3xbfgragh.onion/by/james-poniewozik}{James
Poniewozik}

\begin{itemize}
\item
  July 21, 2020
\item
  \begin{itemize}
  \item
  \item
  \item
  \item
  \item
  \end{itemize}
\end{itemize}

The pinnacle of Donald J. Trump's TV career lasted one night, and he has
never stopped trying to relive it.

The finale of the first season of ``The Apprentice'' in 2004 was the
top-rated show on TV. Afterward the host, finally a mass-media star
after decades of courting fame, believed that giving people twice as
much of him would be twice as good.

NBC agreed, scheduling the show for two cycles the following year (and
then a spinoff with Martha Stewart). The ``Apprentice'' that returned
was more Trump-centric, the host more brash, loud and insulting, his
boardroom firings more dramatic and stunt-filled. Mr. Trump himself took
to the talk- and comedy-show circuit like a starlet in Oscar season,
appearing in ads and on red carpets delivering his trademark ``You're
fired'' finger-point and sneer. He was everywhere.

It didn't work. The ratings declined, first gradually, then
precipitously. While competitors like ``American Idol'' topped the
charts for years, ``The Apprentice'' declined until Mr. Trump was left
hosting a gimmick version with C-list celebrities. For years after, he
would cling to that one glorious stat from 2004 like an Electoral
College map, to claim that his reality show was still the biggest thing
on TV.

The host, of course, rebooted himself, parlaying his network celebrity
into a second life as a political commentator on Fox News, then
candidate, then president. But his reality-TV experience is worth
keeping in mind as he plans to revive his evening coronavirus briefings,
in the apparent belief that rebooting
\href{https://www.nytimes3xbfgragh.onion/2020/03/25/business/media/trump-coronavirus-briefings-ratings.html}{last
spring's ratings hit} will
\href{https://www.nytimes3xbfgragh.onion/2020/07/22/us/politics/trump-polls-2020.html}{reboot
his poll numbers}.

\hypertarget{latest-updates-2020-election}{%
\section{\texorpdfstring{\href{https://www.nytimes3xbfgragh.onion/2020/07/31/us/elections/biden-vs-trump.html?action=click\&pgtype=Article\&state=default\&region=MAIN_CONTENT_1\&context=storylines_live_updates}{Latest
Updates: 2020
Election}}{Latest Updates: 2020 Election}}\label{latest-updates-2020-election}}

Updated 2020-08-01T01:26:45.732Z

\begin{itemize}
\tightlist
\item
  \href{https://www.nytimes3xbfgragh.onion/2020/07/31/us/elections/biden-vs-trump.html?action=click\&pgtype=Article\&state=default\&region=MAIN_CONTENT_1\&context=storylines_live_updates\#link-29fdff45}{Kamala
  Harris, a top vice-presidential contender, confronts double
  standards.}
\item
  \href{https://www.nytimes3xbfgragh.onion/2020/07/31/us/elections/biden-vs-trump.html?action=click\&pgtype=Article\&state=default\&region=MAIN_CONTENT_1\&context=storylines_live_updates\#link-13ec3d9c}{Karen
  Bass and Susan Rice are rising on Biden's vice-presidential
  shortlist.}
\item
  \href{https://www.nytimes3xbfgragh.onion/2020/07/31/us/elections/biden-vs-trump.html?action=click\&pgtype=Article\&state=default\&region=MAIN_CONTENT_1\&context=storylines_live_updates\#link-49e9a016}{Trump
  says Russian bounties to kill U.S. troops `never took place.'}
\end{itemize}

\href{https://www.nytimes3xbfgragh.onion/2020/07/31/us/elections/biden-vs-trump.html?action=click\&pgtype=Article\&state=default\&region=MAIN_CONTENT_1\&context=storylines_live_updates}{See
more updates}

NBC's mistake with ``The Apprentice'' was partly an eternal TV pitfall:
milking the prize cow until it runs dry. Donald Trump, it turned out,
was no more immune to overexposure than Regis Philbin and ``Who Wants to
Be a Millionaire.'' (``Idol,'' on the other hand, aired just one season
a year, and aimed to make stars of its contestants, not just its hosts.)

But it was also an error distinctive to Mr. Trump, who was both the star
and a producer of ``The Apprentice'': Since his 1980s tabloid days, he
never believed there was such a thing as bad publicity, at least for
him. Or as the ``Pod Save America'' host and former Obama strategist Dan
Pfeiffer put it in
\href{https://twitter.com/danpfeiffer/status/1285241783937429505}{a
Tweet on Tuesday}: ``Trump always thinks more Trump is the solution when
it is always the cause of the problem.''

Sure, attention is an asset, in politics as in reality TV. Mr. Trump's
willingness to feed the news beast in 2016 earned him billions in free
media and effectively made him the election's protagonist.

And
\href{https://www.nytimes3xbfgragh.onion/2020/03/30/arts/television/trump-coronavirus-briefing.html}{as
I wrote} during Mr. Trump's first run of briefings in the spring, they
offered him an opportunity he hadn't had since he started ``The
Apprentice'': a regular TV platform in which he could speak to a mass
audience beyond his loyalist base. For a moment, they allowed him to
create the visual impression that he was acting on the pandemic, by
going out and speaking on it. For a moment, his approval ratings --- and
TV ratings --- went up.

But what you do with the attention turns out to matter, at least when
the stakes are hundreds of thousands of lives, not a game-show prize. It
matters if you suggest that
\href{https://www.nytimes3xbfgragh.onion/2020/04/24/us/politics/trump-inject-disinfectant-bleach-coronavirus.html}{household
disinfectants} could be a medical treatment. It matters if you go to war
with your own medical experts. It matters if you minimize, on Page 1, a
terrible reality that everyone can read about for themselves in the
obituaries.

Judging by the president's decision, he doesn't see this as the problem.
Instead the problem is not \emph{enough} him on TV, giving the people
what worked for him before --- zinging, blustering, pointing fingers and
fighting.

His plan to return to prime time was not accompanied by an announced
shift in public-health policy. The thinking simply seems to be: People
want to see the president doing something. And to Donald Trump, going on
TV is the doing-somethingest thing of all.

Thus we saw him on Sunday, sitting for an excruciating interview with
\href{https://www.nytimes3xbfgragh.onion/2020/07/19/us/politics/trump-fox-interview-coronavirus-race.html}{Chris
Wallace of Fox News}, doubling down on blatant disinformation --- like
his claim that the United States has the lowest Covid-19 mortality rate
in the world --- in the face of ruthless fact-checking. As in his
``You're fired'' days, he fell back on his trusty catchphrase, calling
Mr. Wallace ``fake news'' as if the words could dispel the interviewer
from the boardroom.

At one point, Mr. Wallace brought up the president's past criticisms of
him, asking if he understood that it was a journalist's duty to
interview the president's rivals, too. A more blunt way of putting it
would be: Does the president think it's Fox's job to help him win the
election?

He seems to think so. He
\href{https://twitter.com/realDonaldTrump/status/1263537553073942528}{tweeted
a complaint in May} that Fox was ``doing nothing to help Republicans,
and me, get re-elected.'' But in a broader sense, he has suggested that
TV itself owes him payback for all he's given it. TV networks, he has
said, will
\href{https://www.hollywoodreporter.com/news/donald-trump-says-media-needs-him-win-election-1070632}{miss
the ratings} he brings if he is voted out of the White House.

He may be right, but he also assumes that TV viewers think like TV
networks. He acts as if Americans would suffer anything rather than the
boredom he imagines they would endure without him. Thus his preferred
epithet for his opponent, Joseph R. Biden Jr. --- ``Sleepy'' --- which
may not be the killer burn he imagines to a populace tired of staying up
all night in anxiety.

And yet Mr. Trump is, if you trust the current polls, currently losing
to a challenger who is running a quasi-incumbent-like media strategy,
avoiding making big splashes and letting his rival do the work for him.
Mr. Trump seems resentful of this --- ``Let him come out of his
basement,'' he told Mr. Wallace --- or maybe just incredulous. Why would
any sane person not get as much media attention as possible?

Mr. Trump seems to believe that Americans are yearning for a TV star
more than they are yearning for a leader --- or, at least, that they do
not recognize a difference between the two.

Criticize his approach, of course, and there is a ready answer: The
``Too much is never enough'' strategy worked for him in 2016. It worked
in 2004, too, in the first season of ``The Apprentice.''

It always works until it stops working. Until someone decides that too
much, in fact, is enough already.

\hypertarget{our-2020-election-guide}{%
\section{Our 2020 Election Guide}\label{our-2020-election-guide}}

Updated July 31, 2020

\begin{itemize}
\item
  \begin{center}\rule{0.5\linewidth}{\linethickness}\end{center}

  \hypertarget{the-latest}{%
  \subsection{The Latest}\label{the-latest}}

  \begin{itemize}
  \tightlist
  \item
    President Trump's assault on the Postal Service is intersecting with
    his attacks on mail-in voting.
    \href{https://www.nytimes3xbfgragh.onion/2020/07/31/us/politics/trump-usps-mail-delays.html?action=click\&pgtype=Article\&state=default\&region=BELOW_MAIN_CONTENT\&context=storylines_guide}{Voting
    rights groups say it is a recipe for disaster.}
  \end{itemize}
\item
  \begin{center}\rule{0.5\linewidth}{\linethickness}\end{center}

  \hypertarget{bidens-vp-search}{%
  \subsection{Biden's V.P. Search}\label{bidens-vp-search}}

  \begin{itemize}
  \tightlist
  \item
    \href{https://www.nytimes3xbfgragh.onion/article/biden-vice-president-2020.html?action=click\&pgtype=Article\&state=default\&region=BELOW_MAIN_CONTENT\&context=storylines_guide}{Here
    are 13 women} who have been under consideration to be Joe Biden's
    running mate, and why each might be chosen --- and might not be.
  \end{itemize}
\item
  \begin{center}\rule{0.5\linewidth}{\linethickness}\end{center}

  \hypertarget{keep-up-with-our-coverage}{%
  \subsection{Keep Up With Our
  Coverage}\label{keep-up-with-our-coverage}}

  \begin{itemize}
  \tightlist
  \item
    Get an
    \href{https://www.nytimes3xbfgragh.onion/newsletters/politics?action=click\&pgtype=Article\&state=default\&region=BELOW_MAIN_CONTENT\&context=storylines_guide}{email}
    recapping the day's news
  \end{itemize}

  \begin{itemize}
  \tightlist
  \item
    Download our mobile app on
    \href{https://apps.apple.com/us/app/nytimes/id284862083?ls=1\&mat_click_id=5c79ae7455014fd1bd66b5610c05b8f2-20191112-16948\&referrer=mat_click_id\%3D5c79ae7455014fd1bd66b5610c05b8f2-20191112-16948\%26link_click_id\%3D722930677036718082}{iOS}
    and
    \href{http://a.localytics.com/android?id=com.nytimes.android\&referrer=utm_source\%3Dother_nyt_mobile_web\%26utm_medium\%3DWeb\%2520page\%26utm_term\%3DGeneral\%2520Mobile\%2520Page\%26utm_campaign\%3DNYT\%2520Mobile\%2520General\%2520Page}{Android}
    and turn on Breaking News and Politics alerts
  \end{itemize}
\end{itemize}

Advertisement

\protect\hyperlink{after-bottom}{Continue reading the main story}

\hypertarget{site-index}{%
\subsection{Site Index}\label{site-index}}

\hypertarget{site-information-navigation}{%
\subsection{Site Information
Navigation}\label{site-information-navigation}}

\begin{itemize}
\tightlist
\item
  \href{https://help.nytimes3xbfgragh.onion/hc/en-us/articles/115014792127-Copyright-notice}{©~2020~The
  New York Times Company}
\end{itemize}

\begin{itemize}
\tightlist
\item
  \href{https://www.nytco.com/}{NYTCo}
\item
  \href{https://help.nytimes3xbfgragh.onion/hc/en-us/articles/115015385887-Contact-Us}{Contact
  Us}
\item
  \href{https://www.nytco.com/careers/}{Work with us}
\item
  \href{https://nytmediakit.com/}{Advertise}
\item
  \href{http://www.tbrandstudio.com/}{T Brand Studio}
\item
  \href{https://www.nytimes3xbfgragh.onion/privacy/cookie-policy\#how-do-i-manage-trackers}{Your
  Ad Choices}
\item
  \href{https://www.nytimes3xbfgragh.onion/privacy}{Privacy}
\item
  \href{https://help.nytimes3xbfgragh.onion/hc/en-us/articles/115014893428-Terms-of-service}{Terms
  of Service}
\item
  \href{https://help.nytimes3xbfgragh.onion/hc/en-us/articles/115014893968-Terms-of-sale}{Terms
  of Sale}
\item
  \href{https://spiderbites.nytimes3xbfgragh.onion}{Site Map}
\item
  \href{https://help.nytimes3xbfgragh.onion/hc/en-us}{Help}
\item
  \href{https://www.nytimes3xbfgragh.onion/subscription?campaignId=37WXW}{Subscriptions}
\end{itemize}
