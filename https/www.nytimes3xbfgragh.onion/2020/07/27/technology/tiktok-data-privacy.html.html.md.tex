Sections

SEARCH

\protect\hyperlink{site-content}{Skip to
content}\protect\hyperlink{site-index}{Skip to site index}

\href{https://www.nytimes3xbfgragh.onion/section/technology}{Technology}

\href{https://myaccount.nytimes3xbfgragh.onion/auth/login?response_type=cookie\&client_id=vi}{}

\href{https://www.nytimes3xbfgragh.onion/section/todayspaper}{Today's
Paper}

\href{/section/technology}{Technology}\textbar{}What to Do About TikTok

\href{https://nyti.ms/3hSIVof}{https://nyti.ms/3hSIVof}

\begin{itemize}
\item
\item
\item
\item
\item
\end{itemize}

Advertisement

\protect\hyperlink{after-top}{Continue reading the main story}

Supported by

\protect\hyperlink{after-sponsor}{Continue reading the main story}

on tech

\hypertarget{what-to-do-about-tiktok}{%
\section{What to Do About TikTok}\label{what-to-do-about-tiktok}}

Instead of banning the app, U.S. officials could force it to be more
transparent.

\includegraphics{https://static01.graylady3jvrrxbe.onion/images/2020/07/27/business/27ontech-bluebg/27ontech-bluebg-articleLarge.jpg?quality=75\&auto=webp\&disable=upscale}

\href{https://www.nytimes3xbfgragh.onion/by/shira-ovide}{\includegraphics{https://static01.graylady3jvrrxbe.onion/images/2020/03/18/reader-center/author-shira-ovide/author-shira-ovide-thumbLarge-v2.png}}

By \href{https://www.nytimes3xbfgragh.onion/by/shira-ovide}{Shira Ovide}

\begin{itemize}
\item
  Published July 27, 2020Updated Aug. 3, 2020
\item
  \begin{itemize}
  \item
  \item
  \item
  \item
  \item
  \end{itemize}
\end{itemize}

\emph{This article is part of the On Tech newsletter. You can}
\href{https://www.nytimes3xbfgragh.onion/newsletters/signup/OT}{\emph{sign
up here}} \emph{to receive it weekdays.}

Don't ban TikTok. Restrain it. And then apply those same restraints to
the American internet powers, too.

That's what Kevin Roose, a New York Times technology columnist,
\href{https://www.nytimes3xbfgragh.onion/2020/07/26/technology/tiktok-china-ban-model.html}{wrote
in his latest column} about
\href{https://www.nytimes3xbfgragh.onion/2020/08/03/technology/trump-tiktok-microsoft.html}{TikTok},
the app owned by an internet giant based in Beijing. Some
\href{https://www.nytimes3xbfgragh.onion/2020/07/15/technology/tiktok-washington-lobbyist.html}{U.S.
officials worry} that the app could let China's government spy on
Americans or spread propaganda.

I talked with Kevin about his proposed fix to make TikTok --- and
American internet companies, too --- more open and less data-hogging,
and how to sniff out the valid concerns about the video app from the
less legitimate ones.

\textbf{Shira: Let's start with TikTok. What are the reasonable concerns
about it?}

\textbf{Kevin:} Because TikTok is owned by a Chinese company, ByteDance,
it could be compelled to give the data it collects on people watching
videos to the Chinese government and abide by its censorship laws.

And let's be real, TikTok has done things in the past that contributed
to the sense of suspicion ---
\href{https://www.bloomberg.com/news/articles/2019-11-28/tiktok-restores-account-of-user-who-criticized-china-on-uighurs}{temporarily
removing a viral video} that criticized the Chinese government's
treatment of Uighur Muslims, for example.

\textbf{Well, if TikTok is potentially dangerous because it's Chinese,
then isn't the solution to ban it or}
\textbf{\href{https://www.nytimes3xbfgragh.onion/2020/07/23/business/dealbook/tiktok-bytedance-investors-trump.html}{sell
it to non-Chinese owners}?}

An American-owned TikTok could still legally sell user data to a
third-party data broker, who could then sell it to the Chinese
government.

What you really need is a federal privacy law that applies to all
internet platforms operating in the United States, no matter whether
they're Chinese or American. If a big worry is data security, then this
is a useful moment to impose more rules for TikTok and other companies
on how they're collecting and using information about us.

TikTok also plays an important role in American technology. It's
Facebook's only real competitor, and the
\href{https://www.nytimes3xbfgragh.onion/2019/10/19/style/high-school-tiktok-clubs.html}{creative
culture} on the app
\href{https://www.nytimes3xbfgragh.onion/interactive/2019/10/10/arts/TIK-TOK.html}{would
be a shame to lose}.

\textbf{Yes to}
\textbf{\href{https://www.nytimes3xbfgragh.onion/2020/07/15/technology/just-collect-less-data-period.html}{more
data regulation}! What else?}

Another thing that makes TikTok powerful --- and potentially dangerous
--- is that the videos served to us are based on opaque algorithms that
we can't see or inspect. The U.S. government could demand more
transparency as a condition of allowing TikTok to continue operating.

Ideally, this should also apply to Facebook, Instagram and YouTube.
These algorithms
\href{https://www.nytimes3xbfgragh.onion/2020/04/16/technology/rabbit-hole-podcast-kevin-roose.html}{shape
our culture, politics and personal beliefs}, and we know basically
nothing about how they work.

\textbf{Why doesn't the United States already have national laws about
algorithm transparency and data privacy?}

Great question! Many of our elected officials are --- how to put this
delicately --- \emph{undereducated} about technology. Tech companies
have also been lobbying against some of these issues when they've
arisen.

\textbf{Why all this talk about TikTok now? What changed?}

Right, we've been getting technology hardware from China for many years
with few complaints from regulators. I think what's new here is the
Trump administration's desire to appear tough on China.

\textbf{Isn't TikTok useful for American internet powers? At a
congressional antitrust}
\textbf{\href{https://www.nytimes3xbfgragh.onion/2020/07/01/technology/amazon-apple-alphabet-facebook-congress-antitrust.html}{hearing}}
\textbf{this week, I bet Facebook will cite TikTok as}
\textbf{\href{https://www.bloomberg.com/news/articles/2020-07-27/zuckerberg-to-tell-congress-facebook-s-success-is-patriotic}{evidence
of healthy competition}. And Facebook can say,}
\textbf{\href{https://www.nytimes3xbfgragh.onion/2020/07/21/technology/us-china-technology.html}{LOOK
OVER THERE}} \textbf{--- SCARY!}

Yes, it's a great foil. And Facebook's new TikTok competitor, Instagram
Reels, is
\href{https://www.usatoday.com/story/tech/2020/07/19/facebook-tiktok-competitor-instagram-reels-launch/5467934002/}{about
to launch}, which makes all of this much more interesting.

\begin{center}\rule{0.5\linewidth}{\linethickness}\end{center}

\hypertarget{tip-of-the-week}{%
\subsubsection{Tip of the Week}\label{tip-of-the-week}}

\hypertarget{the-nerd-gadget-i-wish-everyone-owned}{%
\subsection{The nerd gadget I wish everyone
owned}\label{the-nerd-gadget-i-wish-everyone-owned}}

\href{https://www.nytimes3xbfgragh.onion/by/brian-x-chen}{\emph{Brian X.
Chen}}\emph{, a personal technology writer for The Times, wants us all
to consider an alternative to cloud computing services like iCloud and
Dropbox.}

Let me tell you why it's worth considering the odd sounding N.A.S.

It stands for network-attached storage --- awful jargon for what is
essentially a mini computer data center in your home. Setting one up
isn't easy, but it gives us a more private, potentially less expensive
way to save and access our digital files, photos and videos from
anywhere.

A N.A.S. is a device containing one or more hard drives that you plug
into your home internet service. It creates something like a personal
cloud service --- similar to Google Drive or Dropbox, but you don't have
to pay a subscription fee to those companies. And because all the data
is stored on your own equipment, no company can see the information
you've saved there.

I have a Synology N.A.S. with a pair of one-terabyte hard drives that I
use instead of Apple's iCloud to back up the data on my Mac. When my
smartphone and tablets begin running out of storage space, I move large
video files and photos to the N.A.S. and delete the files from my
devices.

If I've lost you by now, I get it. N.A.S. servers are designed for
people with above-average levels of tech proficiency. And they're not
cheap to set up. A decent N.A.S. server, including hard drives, will
cost upward of \$500.

But it's worth considering if you have the interest and time to study
up.
\href{https://www.techradar.com/news/the-10-best-nas-devices-reviewed}{Read
Techradar's guide on picking a N.A.S}. Then check out tutorials from
Synology on using a N.A.S. to back up your
\href{https://www.synology.com/en-global/knowledgebase/DSM/tutorial/Backup/How_to_back_up_data_from_Windows_computer_to_Synology_NAS_using_Windows_or_3rd_party_applications}{Windows}
or\href{https://www.synology.com/en-us/knowledgebase/DSM/tutorial/Backup/How_to_back_up_files_from_Mac_to_Synology_NAS_with_Time_Machine}{Mac
computer}. You can also research other things you might want to do with
a personal cloud, like
\href{https://www.nytimes3xbfgragh.onion/2020/01/15/technology/personaltech/streaming-media-home-server.html}{streaming
movies} or
\href{https://www.synology.com/en-us/knowledgebase/DSM/help/VPNCenter/vpn_setup}{creating
a virtual private network} to protect your information when using a
hotel's insecure Wi-Fi connection, for example.

Setting up a N.A.S. can be difficult and frustrating, but the potential
payoff is huge.

\begin{center}\rule{0.5\linewidth}{\linethickness}\end{center}

\hypertarget{before-we-go-}{%
\subsection{Before we go \ldots{}}\label{before-we-go-}}

\begin{itemize}
\item
  \textbf{It's hard to be above the fray in the nation's capital:} The
  Amazon chief executive Jeff Bezos wanted to make a splash in
  Washington as a statesman tackling big policy topics like climate
  change and as the steward of The Washington Post. But Bezos has been
  \href{https://www.nytimes3xbfgragh.onion/2020/07/27/business/jeff-bezos-amazon-congress.html}{dragged
  into mucky political realities}, including testifying later this week
  at the congressional hearing investigating the power of big tech
  companies, my colleagues David McCabe and Karen Weise write. (In
  Tuesday's newsletter, I'll have a conversation with Karen about
  Bezos.)
\item
  \textbf{The information war is playing out in Facebook fact-checking:}
  My colleague Anton Troianovski digs into a small group that Facebook
  hired to slow Russian propaganda and other online misinformation in
  Ukraine. Critics say the
  \href{https://www.nytimes3xbfgragh.onion/2020/07/26/world/europe/ukraine-facebook-fake-news.html}{fact
  checkers' work veers into political activism}, showing how finding
  unbiased fact checkers can be tough in a country at war.
\item
  \textbf{Creative uses of TikTok carry on:} The Los Angeles Times
  \href{https://www.latimes.com/california/story/2020-07-26/papas-on-el-tiktok-how-latino-dads-are-using-the-app-to-connect-with-their-children?_amp=true\&__twitter_impression=true\&s=09}{writes}
  about middle-aged Latino dads who make TikTok videos of themselves
  playing around with their kids in ways that sometimes explode cultural
  stereotypes of hypermasculine Latino fathers.
\end{itemize}

\hypertarget{hugs-to-this}{%
\subsubsection{Hugs to this}\label{hugs-to-this}}

This 20-year-old college student made a
\href{https://www.tiktok.com/@thejulianbass/video/6844906456471457030?lang=en}{stunning
TikTok video} that shows him transforming into superheros using special
effects he created himself. And he's
\href{https://www.kqed.org/arts/13882973/a-student-gets-supersized-attention-after-superhero-video-goes-viral?utm_medium=Email\&utm_source=ExactTarget\&utm_campaign=summerlearning20\&mc_key=003i000001Xhtw9AAB}{getting
attention for it from Hollywood}.

\begin{center}\rule{0.5\linewidth}{\linethickness}\end{center}

\emph{We want to hear from you. Tell us what you think of this
newsletter and what else you'd like us to explore. You can reach us at}
\href{mailto:ontech@NYTimes.com?subject=On\%20Tech\%20Feedback}{\emph{ontech@NYTimes.com.}}
**

\emph{If you don't already get this newsletter in your inbox,}
\href{https://www.nytimes3xbfgragh.onion/newsletters/signup/OT}{\emph{please
sign up here}}\emph{.}

Advertisement

\protect\hyperlink{after-bottom}{Continue reading the main story}

\hypertarget{site-index}{%
\subsection{Site Index}\label{site-index}}

\hypertarget{site-information-navigation}{%
\subsection{Site Information
Navigation}\label{site-information-navigation}}

\begin{itemize}
\tightlist
\item
  \href{https://help.nytimes3xbfgragh.onion/hc/en-us/articles/115014792127-Copyright-notice}{©~2020~The
  New York Times Company}
\end{itemize}

\begin{itemize}
\tightlist
\item
  \href{https://www.nytco.com/}{NYTCo}
\item
  \href{https://help.nytimes3xbfgragh.onion/hc/en-us/articles/115015385887-Contact-Us}{Contact
  Us}
\item
  \href{https://www.nytco.com/careers/}{Work with us}
\item
  \href{https://nytmediakit.com/}{Advertise}
\item
  \href{http://www.tbrandstudio.com/}{T Brand Studio}
\item
  \href{https://www.nytimes3xbfgragh.onion/privacy/cookie-policy\#how-do-i-manage-trackers}{Your
  Ad Choices}
\item
  \href{https://www.nytimes3xbfgragh.onion/privacy}{Privacy}
\item
  \href{https://help.nytimes3xbfgragh.onion/hc/en-us/articles/115014893428-Terms-of-service}{Terms
  of Service}
\item
  \href{https://help.nytimes3xbfgragh.onion/hc/en-us/articles/115014893968-Terms-of-sale}{Terms
  of Sale}
\item
  \href{https://spiderbites.nytimes3xbfgragh.onion}{Site Map}
\item
  \href{https://help.nytimes3xbfgragh.onion/hc/en-us}{Help}
\item
  \href{https://www.nytimes3xbfgragh.onion/subscription?campaignId=37WXW}{Subscriptions}
\end{itemize}
