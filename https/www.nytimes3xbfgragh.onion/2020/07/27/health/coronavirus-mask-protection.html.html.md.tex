Sections

SEARCH

\protect\hyperlink{site-content}{Skip to
content}\protect\hyperlink{site-index}{Skip to site index}

\href{https://www.nytimes3xbfgragh.onion/section/health}{Health}

\href{https://myaccount.nytimes3xbfgragh.onion/auth/login?response_type=cookie\&client_id=vi}{}

\href{https://www.nytimes3xbfgragh.onion/section/todayspaper}{Today's
Paper}

\href{/section/health}{Health}\textbar{}Masks May Reduce Viral Dose,
Some Experts Say

\url{https://nyti.ms/3f4vPlK}

\begin{itemize}
\item
\item
\item
\item
\item
\item
\end{itemize}

\hypertarget{the-coronavirus-outbreak}{%
\subsubsection{\texorpdfstring{\href{https://www.nytimes3xbfgragh.onion/news-event/coronavirus?name=styln-coronavirus-national\&region=TOP_BANNER\&variant=undefined\&block=storyline_menu_recirc\&action=click\&pgtype=Article\&impression_id=e741d190-e3a9-11ea-9f52-e335c4ed056e}{The
Coronavirus
Outbreak}}{The Coronavirus Outbreak}}\label{the-coronavirus-outbreak}}

\begin{itemize}
\tightlist
\item
  live\href{https://www.nytimes3xbfgragh.onion/2020/08/21/world/covid-19-coronavirus.html?name=styln-coronavirus-national\&region=TOP_BANNER\&variant=undefined\&block=storyline_menu_recirc\&action=click\&pgtype=Article\&impression_id=e741d191-e3a9-11ea-9f52-e335c4ed056e}{Latest
  Updates}
\item
  \href{https://www.nytimes3xbfgragh.onion/interactive/2020/us/coronavirus-us-cases.html?name=styln-coronavirus-national\&region=TOP_BANNER\&variant=undefined\&block=storyline_menu_recirc\&action=click\&pgtype=Article\&impression_id=e741f8a0-e3a9-11ea-9f52-e335c4ed056e}{Maps
  and Cases}
\item
  \href{https://www.nytimes3xbfgragh.onion/interactive/2020/science/coronavirus-vaccine-tracker.html?name=styln-coronavirus-national\&region=TOP_BANNER\&variant=undefined\&block=storyline_menu_recirc\&action=click\&pgtype=Article\&impression_id=e741f8a1-e3a9-11ea-9f52-e335c4ed056e}{Vaccine
  Tracker}
\item
  \href{https://www.nytimes3xbfgragh.onion/2020/08/19/us/colleges-closing-covid.html?name=styln-coronavirus-national\&region=TOP_BANNER\&variant=undefined\&block=storyline_menu_recirc\&action=click\&pgtype=Article\&impression_id=e741f8a2-e3a9-11ea-9f52-e335c4ed056e}{Colleges
  Closing}
\item
  \href{https://www.nytimes3xbfgragh.onion/live/2020/08/21/business/stock-market-today-coronavirus?name=styln-coronavirus-national\&region=TOP_BANNER\&variant=undefined\&block=storyline_menu_recirc\&action=click\&pgtype=Article\&impression_id=e741f8a3-e3a9-11ea-9f52-e335c4ed056e}{Economy}
\end{itemize}

Advertisement

\protect\hyperlink{after-top}{Continue reading the main story}

Supported by

\protect\hyperlink{after-sponsor}{Continue reading the main story}

\hypertarget{masks-may-reduce-viral-dose-some-experts-say}{%
\section{Masks May Reduce Viral Dose, Some Experts
Say}\label{masks-may-reduce-viral-dose-some-experts-say}}

People wearing face coverings will take in fewer coronavirus particles,
evidence suggests, making disease less severe.

\includegraphics{https://static01.graylady3jvrrxbe.onion/images/2020/07/27/science/27VIRUS-MASKS/merlin_174293325_b383aca7-5ae6-4923-a029-03dd665d16ac-articleLarge.jpg?quality=75\&auto=webp\&disable=upscale}

By
\href{https://www.nytimes3xbfgragh.onion/by/katherine-j--wu}{Katherine
J. Wu}

\begin{itemize}
\item
  Published July 27, 2020Updated July 29, 2020
\item
  \begin{itemize}
  \item
  \item
  \item
  \item
  \item
  \item
  \end{itemize}
\end{itemize}

\href{https://www.nytimes3xbfgragh.onion/es/2020/07/29/espanol/ciencia-y-tecnologia/proteccion-cubrebocas-coronavirus.html}{Leer
en español}

Researchers have long known that masks can
\href{https://journals.plos.org/plospathogens/article?id=10.1371/journal.ppat.1003205}{prevent
people from spreading airway germs} to others --- findings that have
driven much of the conversation around these crucial accessories during
the coronavirus pandemic.

But now, as cases continue to rise across the country, experts are
pointing to an array of evidence suggesting that masks
\href{https://pubmed.ncbi.nlm.nih.gov/23498357/}{also protect the people
wearing them},~lessening the severity of symptoms, or in some instances,
staving off infection entirely.

Different kinds of masks ``block virus to a different degree, but they
all block the virus from getting in,'' said Dr. Monica Gandhi, an
infectious disease physician at the University of California, San
Francisco. If any virus particles do breach these barriers, she said,
the disease might still be milder.

Dr. Gandhi and her colleagues make this argument in a
\href{https://ucsf.app.box.com/s/blvolkp5z0mydzd82rjks4wyleagt036}{new
paper} slated to be published in the Journal of General Internal
Medicine. Drawing from animal experiments and observations of various
events during the pandemic, they contend that people wearing face
coverings will take in fewer coronavirus particles, making it easier for
their immune systems to bring any interlopers to heel.

Dr. Tsion Firew, an emergency physician at Columbia University who
wasn't involved in the work, cautioned that the links between masking
and milder disease haven't yet been proved as cause and effect. Even so,
the new paper ``reiterates what we say about masks,'' she said. ``It's
not just a selfless act.''

Ideas about the importance of viral dose in the development of disease
have cropped up in the medical literature since
\href{https://academic.oup.com/aje/article-abstract/27/3/493/99616}{at
least the 1930s}, when two researchers formally noted that mice exposed
to larger quantities of germs were more likely to die. More recently,
scientists have gone as far as to
\href{https://pubmed.ncbi.nlm.nih.gov/25416753/}{puff different amounts
of a flu virus} up the noses of human volunteers. The more virus in this
nasal plume, they found, the likelier the participants were to get
infected and experience symptoms.

That sort of experiment can't be done ethically for the new coronavirus,
given how dangerous it is. But earlier this year, a team of researchers
in China tried something similar in hamsters: They housed
coronavirus-infected and healthy animals in adjoining cages, some of
which were separated by buffers made of surgical masks. Many of the
healthy hamsters behind the partitions never got infected. And the
unlucky animals who did
\href{https://academic.oup.com/cid/article/doi/10.1093/cid/ciaa644/5848814}{got
less sick} than their ``maskless'' neighbors.

\hypertarget{latest-updates-the-coronavirus-outbreak}{%
\section{\texorpdfstring{\href{https://www.nytimes3xbfgragh.onion/2020/08/21/world/covid-19-coronavirus.html?action=click\&pgtype=Article\&state=default\&region=MAIN_CONTENT_1\&context=storylines_live_updates}{Latest
Updates: The Coronavirus
Outbreak}}{Latest Updates: The Coronavirus Outbreak}}\label{latest-updates-the-coronavirus-outbreak}}

Updated 2020-08-21T12:14:35.854Z

\begin{itemize}
\tightlist
\item
  \href{https://www.nytimes3xbfgragh.onion/2020/08/21/world/covid-19-coronavirus.html?action=click\&pgtype=Article\&state=default\&region=MAIN_CONTENT_1\&context=storylines_live_updates\#link-6a60a19d}{`Be
  adults': Universities in the U.S. are warning students about
  gatherings as they return to campus.}
\item
  \href{https://www.nytimes3xbfgragh.onion/2020/08/21/world/covid-19-coronavirus.html?action=click\&pgtype=Article\&state=default\&region=MAIN_CONTENT_1\&context=storylines_live_updates\#link-324af071}{As
  he accepts the Democratic nomination, Biden knocks Trump's pandemic
  response.}
\item
  \href{https://www.nytimes3xbfgragh.onion/2020/08/21/world/covid-19-coronavirus.html?action=click\&pgtype=Article\&state=default\&region=MAIN_CONTENT_1\&context=storylines_live_updates\#link-191d44be}{South
  Korea threatens to detain people who obstruct virus-control efforts.}
\end{itemize}

\href{https://www.nytimes3xbfgragh.onion/2020/08/21/world/covid-19-coronavirus.html?action=click\&pgtype=Article\&state=default\&region=MAIN_CONTENT_1\&context=storylines_live_updates}{See
more updates}

More live coverage:
\href{https://www.nytimes3xbfgragh.onion/live/2020/08/21/business/stock-market-today-coronavirus?action=click\&pgtype=Article\&state=default\&region=MAIN_CONTENT_1\&context=storylines_live_updates}{Markets}

Some indirect data has been accumulating from people as well.
Researchers have tentatively estimated that about 40 percent of
coronavirus infections
\href{https://www.cdc.gov/coronavirus/2019-ncov/hcp/planning-scenarios.html}{do
not produce any symptoms}. But when some people
\href{https://www.oregonlive.com/coronavirus/2020/06/big-coronavirus-outbreak-at-newport-seafood-plants-is-contained-health-authorities-say.html}{wear
masks}, the proportion of asymptomatic cases
\href{https://apnews.com/4b9d38f206db9ce5267a5898ac24f238}{seems to
skyrocket}, reportedly surpassing 90 percent during one outbreak at a
seafood plant in Oregon. Wearing a face covering doesn't make people
impervious to infection, but these trends of asymptomatic cases could
suggest that masks lead to milder disease, potentially reducing
hospitalizations and deaths.

Particularly compelling, Dr. Gandhi said, is the data from cruise ships,
which pack big groups of people into close quarters.
\href{https://www.ncbi.nlm.nih.gov/pmc/articles/PMC7078829/}{More than
80 percent} of those infected aboard Japan's Diamond Princess in
February --- before masking had become common practice ---~came down
with symptoms, she noted. But on another vessel that left Argentina in
March, and on which all passengers were issued surgical masks after
someone onboard came down with a fever, the level of symptomatic cases
was \href{https://thorax.bmj.com/content/75/8/693}{below 20 percent}.

Some independent experts say the paper is a welcome update, given the
pervasive idea that wearing a mask is a mostly altruistic act.

``It's been a real deficiency in the messaging about masking to say that
it only protects the other,'' said Charles Haas, an environmental
engineer and expert in risk assessment at Drexel University. ``From the
get go, that never made sense scientifically.''

In other settings, too, from
\href{https://jamanetwork.com/journals/jama/fullarticle/2768533}{hospitals}
to
\href{https://www.nytimes3xbfgragh.onion/2020/07/14/health/coronavirus-hair-salon-masks.html}{hair
salons}, face coverings may have driven down rates of overall infection,
perhaps preventing disastrous outbreaks. And countries like Japan,
Taiwan and South Korea, where outbreaks quickly sparked a wave of
widespread masking, managed to rein in the number of coronavirus-related
hospitalizations and deaths early on.

Even in the United States, the slow upward tick in mask-wearing has
coincided with what appears to be a
\href{https://www.nytimes3xbfgragh.onion/2020/07/03/health/coronavirus-mortality-testing.html}{more
modest death rate}, compared to the
\href{https://www.nytimes3xbfgragh.onion/interactive/2020/us/coronavirus-us-cases.html}{surge
that occurred} after the virus first made landfall in North America.
These trends have also likely been influenced by increased testing, a
downward shift in the average age of people contracting the virus and
improvements in coronavirus treatments. Still, masks probably aren't
hurting things, Dr. Gandhi said.

The idea that face coverings can curb disease severity, although not yet
proven, ``makes complete sense,'' said Linsey Marr, an expert in virus
transmission at Virginia Tech. ``It's another good argument for wearing
masks.''

\href{https://www.nytimes3xbfgragh.onion/news-event/coronavirus?action=click\&pgtype=Article\&state=default\&region=MAIN_CONTENT_3\&context=storylines_faq}{}

\hypertarget{the-coronavirus-outbreak-}{%
\subsubsection{The Coronavirus Outbreak
›}\label{the-coronavirus-outbreak-}}

\hypertarget{frequently-asked-questions}{%
\paragraph{Frequently Asked
Questions}\label{frequently-asked-questions}}

Updated August 17, 2020

\begin{itemize}
\item ~
  \hypertarget{why-does-standing-six-feet-away-from-others-help}{%
  \paragraph{Why does standing six feet away from others
  help?}\label{why-does-standing-six-feet-away-from-others-help}}

  \begin{itemize}
  \tightlist
  \item
    The coronavirus spreads primarily through droplets from your mouth
    and nose, especially when you cough or sneeze. The C.D.C., one of
    the organizations using that measure,
    \href{https://www.nytimes3xbfgragh.onion/2020/04/14/health/coronavirus-six-feet.html?action=click\&pgtype=Article\&state=default\&region=MAIN_CONTENT_3\&context=storylines_faq}{bases
    its recommendation of six feet} on the idea that most large droplets
    that people expel when they cough or sneeze will fall to the ground
    within six feet. But six feet has never been a magic number that
    guarantees complete protection. Sneezes, for instance, can launch
    droplets a lot farther than six feet,
    \href{https://jamanetwork.com/journals/jama/fullarticle/2763852}{according
    to a recent study}. It's a rule of thumb: You should be safest
    standing six feet apart outside, especially when it's windy. But
    keep a mask on at all times, even when you think you're far enough
    apart.
  \end{itemize}
\item ~
  \hypertarget{i-have-antibodies-am-i-now-immune}{%
  \paragraph{I have antibodies. Am I now
  immune?}\label{i-have-antibodies-am-i-now-immune}}

  \begin{itemize}
  \tightlist
  \item
    As of right
    now,\href{https://www.nytimes3xbfgragh.onion/2020/07/22/health/covid-antibodies-herd-immunity.html?action=click\&pgtype=Article\&state=default\&region=MAIN_CONTENT_3\&context=storylines_faq}{that
    seems likely, for at least several months.} There have been
    frightening accounts of people suffering what seems to be a second
    bout of Covid-19. But experts say these patients may have a
    drawn-out course of infection, with the virus taking a slow toll
    weeks to months after initial exposure. People infected with the
    coronavirus typically
    \href{https://www.nature.com/articles/s41586-020-2456-9}{produce}
    immune molecules called antibodies, which are
    \href{https://www.nytimes3xbfgragh.onion/2020/05/07/health/coronavirus-antibody-prevalence.html?action=click\&pgtype=Article\&state=default\&region=MAIN_CONTENT_3\&context=storylines_faq}{protective
    proteins made in response to an
    infection}\href{https://www.nytimes3xbfgragh.onion/2020/05/07/health/coronavirus-antibody-prevalence.html?action=click\&pgtype=Article\&state=default\&region=MAIN_CONTENT_3\&context=storylines_faq}{.
    These antibodies may} last in the body
    \href{https://www.nature.com/articles/s41591-020-0965-6}{only two to
    three months}, which may seem worrisome, but that's perfectly normal
    after an acute infection subsides, said Dr. Michael Mina, an
    immunologist at Harvard University. It may be possible to get the
    coronavirus again, but it's highly unlikely that it would be
    possible in a short window of time from initial infection or make
    people sicker the second time.
  \end{itemize}
\item ~
  \hypertarget{im-a-small-business-owner-can-i-get-relief}{%
  \paragraph{I'm a small-business owner. Can I get
  relief?}\label{im-a-small-business-owner-can-i-get-relief}}

  \begin{itemize}
  \tightlist
  \item
    The
    \href{https://www.nytimes3xbfgragh.onion/article/small-business-loans-stimulus-grants-freelancers-coronavirus.html?action=click\&pgtype=Article\&state=default\&region=MAIN_CONTENT_3\&context=storylines_faq}{stimulus
    bills enacted in March} offer help for the millions of American
    small businesses. Those eligible for aid are businesses and
    nonprofit organizations with fewer than 500 workers, including sole
    proprietorships, independent contractors and freelancers. Some
    larger companies in some industries are also eligible. The help
    being offered, which is being managed by the Small Business
    Administration, includes the Paycheck Protection Program and the
    Economic Injury Disaster Loan program. But lots of folks have
    \href{https://www.nytimes3xbfgragh.onion/interactive/2020/05/07/business/small-business-loans-coronavirus.html?action=click\&pgtype=Article\&state=default\&region=MAIN_CONTENT_3\&context=storylines_faq}{not
    yet seen payouts.} Even those who have received help are confused:
    The rules are draconian, and some are stuck sitting on
    \href{https://www.nytimes3xbfgragh.onion/2020/05/02/business/economy/loans-coronavirus-small-business.html?action=click\&pgtype=Article\&state=default\&region=MAIN_CONTENT_3\&context=storylines_faq}{money
    they don't know how to use.} Many small-business owners are getting
    less than they expected or
    \href{https://www.nytimes3xbfgragh.onion/2020/06/10/business/Small-business-loans-ppp.html?action=click\&pgtype=Article\&state=default\&region=MAIN_CONTENT_3\&context=storylines_faq}{not
    hearing anything at all.}
  \end{itemize}
\item ~
  \hypertarget{what-are-my-rights-if-i-am-worried-about-going-back-to-work}{%
  \paragraph{What are my rights if I am worried about going back to
  work?}\label{what-are-my-rights-if-i-am-worried-about-going-back-to-work}}

  \begin{itemize}
  \tightlist
  \item
    Employers have to provide
    \href{https://www.osha.gov/SLTC/covid-19/standards.html}{a safe
    workplace} with policies that protect everyone equally.
    \href{https://www.nytimes3xbfgragh.onion/article/coronavirus-money-unemployment.html?action=click\&pgtype=Article\&state=default\&region=MAIN_CONTENT_3\&context=storylines_faq}{And
    if one of your co-workers tests positive for the coronavirus, the
    C.D.C.} has said that
    \href{https://www.cdc.gov/coronavirus/2019-ncov/community/guidance-business-response.html}{employers
    should tell their employees} -\/- without giving you the sick
    employee's name -\/- that they may have been exposed to the virus.
  \end{itemize}
\item ~
  \hypertarget{what-is-school-going-to-look-like-in-september}{%
  \paragraph{What is school going to look like in
  September?}\label{what-is-school-going-to-look-like-in-september}}

  \begin{itemize}
  \tightlist
  \item
    It is unlikely that many schools will return to a normal schedule
    this fall, requiring the grind of
    \href{https://www.nytimes3xbfgragh.onion/2020/06/05/us/coronavirus-education-lost-learning.html?action=click\&pgtype=Article\&state=default\&region=MAIN_CONTENT_3\&context=storylines_faq}{online
    learning},
    \href{https://www.nytimes3xbfgragh.onion/2020/05/29/us/coronavirus-child-care-centers.html?action=click\&pgtype=Article\&state=default\&region=MAIN_CONTENT_3\&context=storylines_faq}{makeshift
    child care} and
    \href{https://www.nytimes3xbfgragh.onion/2020/06/03/business/economy/coronavirus-working-women.html?action=click\&pgtype=Article\&state=default\&region=MAIN_CONTENT_3\&context=storylines_faq}{stunted
    workdays} to continue. California's two largest public school
    districts --- Los Angeles and San Diego --- said on July 13, that
    \href{https://www.nytimes3xbfgragh.onion/2020/07/13/us/lausd-san-diego-school-reopening.html?action=click\&pgtype=Article\&state=default\&region=MAIN_CONTENT_3\&context=storylines_faq}{instruction
    will be remote-only in the fall}, citing concerns that surging
    coronavirus infections in their areas pose too dire a risk for
    students and teachers. Together, the two districts enroll some
    825,000 students. They are the largest in the country so far to
    abandon plans for even a partial physical return to classrooms when
    they reopen in August. For other districts, the solution won't be an
    all-or-nothing approach.
    \href{https://bioethics.jhu.edu/research-and-outreach/projects/eschool-initiative/school-policy-tracker/}{Many
    systems}, including the nation's largest, New York City, are
    devising
    \href{https://www.nytimes3xbfgragh.onion/2020/06/26/us/coronavirus-schools-reopen-fall.html?action=click\&pgtype=Article\&state=default\&region=MAIN_CONTENT_3\&context=storylines_faq}{hybrid
    plans} that involve spending some days in classrooms and other days
    online. There's no national policy on this yet, so check with your
    municipal school system regularly to see what is happening in your
    community.
  \end{itemize}
\end{itemize}

Dr. Marr and other researchers are still sussing out exactly how much
inbound or outbound virus different types of masks block. But based on a
\href{https://journals.plos.org/plosone/article?id=10.1371/journal.pone.0002618}{wealth}
of \href{https://www.nature.com/articles/s41591-020-0843-2}{past
evidence} and
\href{https://www.thelancet.com/journals/lancet/article/PIIS0140-6736(20)31142-9/fulltext\#\%20}{recent
observations}, the amount that's filtered out is probably high ---
perhaps 50 percent or more of the larger aerosols being sent in both
directions, Dr. Marr said. Certain coverings, like N95 respirators, will
do better than others, but even looser-fitting cloths can waylay some
viral particles.

Still, some experts are not ready to embrace all ideas about two-way
protection.

What's outlined in Dr. Gandhi's paper ``is still just a theory, and
needs more research,'' said Nancy Leung, an epidemiologist at the
University of Hong Kong. While there's good evidence that masks reduce
the spread of viruses within a population, it's much harder to nail down
how face coverings influence symptoms, Dr. Leung said, in part ``because
of the difficulty in conducting those studies.''

Dr. Gandhi acknowledged these limitations. But with no end to the
pandemic in sight, the need for masks is only growing, she said,
especially as researchers continue to document the virus's ability to
spread silently. Even people who don't have symptoms can spray the virus
into their environment when they sneeze, cough, sing, speak or even
breathe. And those who fall ill may be at their
\href{https://www.nature.com/articles/s41591-020-0869-5}{most
contagious} in the days before the first signs of sickness appear.

To tame this pandemic, people should act as if they've been infected,
``even if you feel right as rain,'' Dr. Gandhi said.

Masks alone aren't a substitute for other public health measures like
physical distancing and good hygiene. But unlike sustained lockdowns
that keep people apart, shielding our faces is easier and more
sustainable, Dr. Gandhi said.

Safeguarding yourself and others from this deadly disease, she added,
``is as simple as covering up the two holes in your face that shed the
virus.''

\textbf{\emph{{[}}\href{http://on.fb.me/1paTQ1h}{\emph{Like the Science
Times page on Facebook.}}} ****** \emph{\textbar{} Sign up for the}
\textbf{\href{http://nyti.ms/1MbHaRU}{\emph{Science Times
newsletter.}}\emph{{]}}}

Advertisement

\protect\hyperlink{after-bottom}{Continue reading the main story}

\hypertarget{site-index}{%
\subsection{Site Index}\label{site-index}}

\hypertarget{site-information-navigation}{%
\subsection{Site Information
Navigation}\label{site-information-navigation}}

\begin{itemize}
\tightlist
\item
  \href{https://help.nytimes3xbfgragh.onion/hc/en-us/articles/115014792127-Copyright-notice}{©~2020~The
  New York Times Company}
\end{itemize}

\begin{itemize}
\tightlist
\item
  \href{https://www.nytco.com/}{NYTCo}
\item
  \href{https://help.nytimes3xbfgragh.onion/hc/en-us/articles/115015385887-Contact-Us}{Contact
  Us}
\item
  \href{https://www.nytco.com/careers/}{Work with us}
\item
  \href{https://nytmediakit.com/}{Advertise}
\item
  \href{http://www.tbrandstudio.com/}{T Brand Studio}
\item
  \href{https://www.nytimes3xbfgragh.onion/privacy/cookie-policy\#how-do-i-manage-trackers}{Your
  Ad Choices}
\item
  \href{https://www.nytimes3xbfgragh.onion/privacy}{Privacy}
\item
  \href{https://help.nytimes3xbfgragh.onion/hc/en-us/articles/115014893428-Terms-of-service}{Terms
  of Service}
\item
  \href{https://help.nytimes3xbfgragh.onion/hc/en-us/articles/115014893968-Terms-of-sale}{Terms
  of Sale}
\item
  \href{https://spiderbites.nytimes3xbfgragh.onion}{Site Map}
\item
  \href{https://help.nytimes3xbfgragh.onion/hc/en-us}{Help}
\item
  \href{https://www.nytimes3xbfgragh.onion/subscription?campaignId=37WXW}{Subscriptions}
\end{itemize}
