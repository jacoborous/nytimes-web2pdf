Sections

SEARCH

\protect\hyperlink{site-content}{Skip to
content}\protect\hyperlink{site-index}{Skip to site index}

\href{https://myaccount.nytimes3xbfgragh.onion/auth/login?response_type=cookie\&client_id=vi}{}

\href{https://www.nytimes3xbfgragh.onion/section/todayspaper}{Today's
Paper}

\href{/section/opinion}{Opinion}\textbar{}I Played in the N.F.L. It
Needs Way More Than a Black Anthem.

\url{https://nyti.ms/322HChs}

\begin{itemize}
\item
\item
\item
\item
\item
\item
\end{itemize}

Advertisement

\protect\hyperlink{after-top}{Continue reading the main story}

\href{/section/opinion}{Opinion}

Supported by

\protect\hyperlink{after-sponsor}{Continue reading the main story}

\hypertarget{i-played-in-the-nfl-it-needs-way-more-than-a-black-anthem}{%
\section{I Played in the N.F.L. It Needs Way More Than a Black
Anthem.}\label{i-played-in-the-nfl-it-needs-way-more-than-a-black-anthem}}

If the league wants to show its commitment to its players, it should
hire and promote more Black coaches and executives.

By Donté Stallworth

\begin{itemize}
\item
  July 8, 2020
\item
  \begin{itemize}
  \item
  \item
  \item
  \item
  \item
  \item
  \end{itemize}
\end{itemize}

\includegraphics{https://static01.graylady3jvrrxbe.onion/images/2020/07/08/opinion/08Stallworth3/08Stallworth3-articleLarge.jpg?quality=75\&auto=webp\&disable=upscale}

WASHINGTON --- In response to the Black Lives Matter protests, the
N.F.L. has decided to play ``Lift Every Voice and Sing,'' known as the
Black national anthem, in Week 1 of its coming season.

As a former N.F.L. player, my initial reaction was: Why?

Is this a sign that the N.F.L. is serious now, that it truly wants to
honor its commitment to promote racial equality in the league? Or is it
just a symbolic gesture, one meant to placate its players, without any
meaningful change?

Don't get me wrong, symbolism can be a powerful thing. ``Lift Every
Voice and Sing'' is a fixture of Black life, a celebration of our
tumultuous experience --- the struggle and triumph, the joy and pain of
being Americans.

The song was originally a poem, written by James Weldon Johnson, the
historian, author and civil rights activist. He was no stranger to
police brutality. In his book ``Black Manhattan,'' he describes Black
people running away from white mobs during the New York race riot in
1900, only to be violently beaten by the police officers, from whom they
had sought protection. An investigation into the police violence was
turned on its head and the police were treated as if they were the
victims of a crime.

Similar themes are playing out today: no accountability and no justice.

The N.F.L. has had plenty of opportunities to be on the right side of
history. It could have supported Colin Kaepernick and other players who
took a knee four years ago to protest police brutality and racial
inequities in the U.S. justice system. But the league failed to protect
them, when the players needed them most.

It was only last month that the league issued an apology of sorts,
admitting that ``it was wrong for not listening to N.F.L. players
earlier.'' This mea culpa took place only after demands by more than a
dozen of its young stars, including Patrick Mahomes, the Kansas City
quarterback who was named the Super Bowl's most valuable player last
season.

How could the N.F.L. be so blind? The author and historian George M.
Fredrickson wrote that ``societal racism did not require an ideology to
sustain it so long as it was taken for granted.''

The N.F.L. is not immune from this observation. An overwhelming majority
of owners in the N.F.L.'s history have been white men. Today, more than
two-thirds of the players are Black. But across 32 teams, there are only
\href{https://www.espn.com/nfl/story/_/id/29354971/nfl-quarterback-coach-summit-urges-owners-look-deeper-minority-hires}{three
Black head coaches} and two Black general managers. Over the past three
years, there have been 20 head coaching vacancies, but Black coaches
filled only two of them.

And then there are the politics. Almost a dozen owners of N.F.L. teams
have supported President Trump by contributing money or hosting
fund-raisers. This is the man whose words and Twitter account can attest
to his racism --- who has insulted N.F.L. players who took a knee,
suggesting that they shouldn't be in the country.

If the N.F.L. wants to send an unambiguous message that its concern is
genuine and not performative, it must start with this political
disconnect.

The recent pledge of the N.F.L. and its team owners to contribute \$250
million over 10 years to fight systematic racism is not enough. Nor is
honoring victims of police brutality with helmet decals and jerseys.
(Though this is, no doubt, a departure for a league that has routinely
sanctioned its players for minor uniform infractions, including fines of
\$7,000 for untucked shirts.)

The owners must make radical changes. First, they must immediately stop
raising money for President Trump. It is impossible to walk in opposite
directions at the same time, and supporting the president is the
antithesis of supporting the players.

Then, using their vast political connections, the owners must personally
lobby for issues that matter to the players' coalition, like legislation
to reform policing.

And they should clean up their own house. The N.F.L. must be committed
to hiring more Black head coaches and Black executives. It needs to
build a pipeline for junior coaches, who can be promoted to coordinator
and play-caller positions, jobs that are essential for promotion to head
coach.

There is other work to be done --- including by my former team in
Washington. Its founder, George Preston Marshall, an avowed
segregationist, was the last N.F.L. owner to integrate his team. His
statue was finally
\href{https://www.nfl.com/news/george-preston-marshall-statue-removed-at-rfk-stadium}{removed}
from the front of RFK Stadium, the team's former home, as was his name
from the stadium's Ring of Honor. But the team has not removed the
club's offensive name, despite decades of opposition from Indigenous
people.

Dan Snyder, the team's current owner,
\href{https://www.usatoday.com/story/sports/nfl/redskins/2013/05/09/washington-redskins-daniel-snyder/2148127/}{said
in a 2013 interview} that he would ``never change the name.'' ``It's
that simple,'' he said. ``NEVER --- you can use caps.''

Now the team says it will review whether to change the name. But what
additional information does it need? The debate itself deprives our
Indigenous brothers and sisters of their humanity, and their voices have
been ignored for too long.

\includegraphics{https://static01.graylady3jvrrxbe.onion/images/2020/07/06/opinion/06Stallworth1/merlin_26754960_d36c2c12-8e29-44bf-b653-fcb5289064de-articleLarge.jpg?quality=75\&auto=webp\&disable=upscale}

While I will enjoy hearing ``Lift Every Voice and Sing'' during the
N.F.L.'s opening week, I will remain skeptical. I want to believe that
league officials and team owners finally get it --- and I know a number
of them do. But as in football, good intentions don't win games,
performance does. Radical change is truly needed. We don't need any more
symbolic gestures. We need the N.F.L. to step up and change a
decades-old playbook that has long been out of step with the times.

Donté Stallworth played in the National Football League for 10 seasons.

\emph{The Times is committed to publishing}
\href{https://www.nytimes3xbfgragh.onion/2019/01/31/opinion/letters/letters-to-editor-new-york-times-women.html}{\emph{a
diversity of letters}} \emph{to the editor. We'd like to hear what you
think about this or any of our articles. Here are some}
\href{https://help.nytimes3xbfgragh.onion/hc/en-us/articles/115014925288-How-to-submit-a-letter-to-the-editor}{\emph{tips}}\emph{.
And here's our email:}
\href{mailto:letters@NYTimes.com}{\emph{letters@NYTimes.com}}\emph{.}

\emph{Follow The New York Times Opinion section on}
\href{https://www.facebookcorewwwi.onion/nytopinion}{\emph{Facebook}}\emph{,}
\href{http://twitter.com/NYTOpinion}{\emph{Twitter (@NYTopinion)}}
\emph{and}
\href{https://www.instagram.com/nytopinion/}{\emph{Instagram}}\emph{.}

Advertisement

\protect\hyperlink{after-bottom}{Continue reading the main story}

\hypertarget{site-index}{%
\subsection{Site Index}\label{site-index}}

\hypertarget{site-information-navigation}{%
\subsection{Site Information
Navigation}\label{site-information-navigation}}

\begin{itemize}
\tightlist
\item
  \href{https://help.nytimes3xbfgragh.onion/hc/en-us/articles/115014792127-Copyright-notice}{©~2020~The
  New York Times Company}
\end{itemize}

\begin{itemize}
\tightlist
\item
  \href{https://www.nytco.com/}{NYTCo}
\item
  \href{https://help.nytimes3xbfgragh.onion/hc/en-us/articles/115015385887-Contact-Us}{Contact
  Us}
\item
  \href{https://www.nytco.com/careers/}{Work with us}
\item
  \href{https://nytmediakit.com/}{Advertise}
\item
  \href{http://www.tbrandstudio.com/}{T Brand Studio}
\item
  \href{https://www.nytimes3xbfgragh.onion/privacy/cookie-policy\#how-do-i-manage-trackers}{Your
  Ad Choices}
\item
  \href{https://www.nytimes3xbfgragh.onion/privacy}{Privacy}
\item
  \href{https://help.nytimes3xbfgragh.onion/hc/en-us/articles/115014893428-Terms-of-service}{Terms
  of Service}
\item
  \href{https://help.nytimes3xbfgragh.onion/hc/en-us/articles/115014893968-Terms-of-sale}{Terms
  of Sale}
\item
  \href{https://spiderbites.nytimes3xbfgragh.onion}{Site Map}
\item
  \href{https://help.nytimes3xbfgragh.onion/hc/en-us}{Help}
\item
  \href{https://www.nytimes3xbfgragh.onion/subscription?campaignId=37WXW}{Subscriptions}
\end{itemize}
