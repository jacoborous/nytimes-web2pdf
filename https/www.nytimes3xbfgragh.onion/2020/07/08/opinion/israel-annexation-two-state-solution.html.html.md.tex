Sections

SEARCH

\protect\hyperlink{site-content}{Skip to
content}\protect\hyperlink{site-index}{Skip to site index}

\href{https://myaccount.nytimes3xbfgragh.onion/auth/login?response_type=cookie\&client_id=vi}{}

\href{https://www.nytimes3xbfgragh.onion/section/todayspaper}{Today's
Paper}

\href{/section/opinion}{Opinion}\textbar{}I No Longer Believe in a
Jewish State

\url{https://nyti.ms/2ZNyef5}

\begin{itemize}
\item
\item
\item
\item
\item
\item
\end{itemize}

Advertisement

\protect\hyperlink{after-top}{Continue reading the main story}

\href{/section/opinion}{Opinion}

Supported by

\protect\hyperlink{after-sponsor}{Continue reading the main story}

\hypertarget{i-no-longer-believe-in-a-jewish-state}{%
\section{I No Longer Believe in a Jewish
State}\label{i-no-longer-believe-in-a-jewish-state}}

For decades I argued for separation between Israelis and Palestinians.
Now, I can imagine a Jewish home in an equal state.

By Peter Beinart

Mr. Beinart is editor at large of Jewish Currents.

\begin{itemize}
\item
  July 8, 2020
\item
  \begin{itemize}
  \item
  \item
  \item
  \item
  \item
  \item
  \end{itemize}
\end{itemize}

\includegraphics{https://static01.graylady3jvrrxbe.onion/images/2020/07/08/opinion/08beinart1/merlin_173658111_7e9fd2f9-a29c-4f26-af65-e5e34af755f6-articleLarge.jpg?quality=75\&auto=webp\&disable=upscale}

I was 22 in 1993 when Yitzhak Rabin and Yasir Arafat shook hands on the
White House lawn to officially begin the peace process that many hoped
would create a Palestinian state alongside Israel. I've been arguing for
a two-state solution --- first in late-night bull sessions, then in
articles and speeches --- ever since.

I believed in Israel as a Jewish state because I grew up in a family
that had hopscotched from continent to continent as diaspora Jewish
communities crumbled. I saw Israel's impact on my grandfather and
father, who were never as happy or secure as when enveloped in a society
of Jews. And I knew that Israel was a source of comfort and pride to
millions of other Jews, some of whose families had experienced traumas
greater than my own.

One day in early adulthood, I walked through Jerusalem, reading street
names that catalog Jewish history, and felt that comfort and pride
myself. I knew Israel was wrong to deny Palestinians in the West Bank
citizenship, due process, free movement and the right to vote in the
country in which they lived. But the dream of a two-state solution that
would give Palestinians a country of their own let me hope that I could
remain a liberal and a supporter of Jewish statehood at the same time.

Events have now extinguished that hope.

\href{https://fmep.org/resource/settlement-report-october-11-2019/}{About
640,000} Jewish settlers now live in East Jerusalem and the West Bank,
and the Israeli and American governments have divested Palestinian
statehood of any real meaning.
\href{https://www.nytimes3xbfgragh.onion/2020/01/28/world/middleeast/peace-plan.html}{The
Trump administration's peace plan} envisions an archipelago of
Palestinian towns, scattered across
\href{https://www.nytimes3xbfgragh.onion/2020/01/28/world/middleeast/israel-west-bank-annex-sovereignty.html}{as
little as 70 percent} of the West Bank, under Israeli control. Even the
\href{https://www.nytimes3xbfgragh.onion/2019/02/27/opinion/israel-election-two-state-solution.html}{leaders}
of Israel's
\href{https://www.timesofisrael.com/joining-forces-with-gantz-yaalon-rules-out-support-for-two-state-solution/}{supposedly
center-left}
\href{https://www.haaretz.com/israel-news/elections/yair-lapid-outlines-four-demands-for-peace-with-palestinians-1.7000533}{parties}
don't support a viable, sovereign Palestinian state. The West Bank hosts
Israel's
\href{https://www.timesofisrael.com/us-ambassador-toasts-opening-of-new-israeli-medical-school-in-west-bank/}{newest
medical school}.

If Prime Minister Benjamin Netanyahu fulfills his pledge to impose
Israeli sovereignty in parts of the West Bank, he will just formalize a
decades-old reality: In practice, Israel annexed the West Bank long ago.

Israel has all but made its decision: one country that includes millions
of Palestinians who lack basic rights. Now liberal Zionists must make
our decision, too. It's time to abandon the traditional two-state
solution and embrace the goal of equal rights for Jews and Palestinians.
It's time to imagine a Jewish home that is not a Jewish state.

Equality could come in the form of one state that includes Israel, the
West Bank, the Gaza Strip and East Jerusalem, as writers such as
\href{https://www.foreignaffairs.com/articles/israel/2019-10-15/there-will-be-one-state-solution}{Yousef
Munayyer} and
\href{https://www.nytimes3xbfgragh.onion/1999/01/10/magazine/the-one-state-solution.html}{Edward
Said} have proposed; or it could be a
\href{https://www.alandforall.org/english/?d=ltr}{confederation} that
allows free movement between two deeply integrated countries. (I discuss
these options at greater length in an
\href{https://jewishcurrents.org/yavne-a-jewish-case-for-equality-in-israel-palestine/}{essay
in Jewish Currents}). The process of achieving equality would be long
and difficult, and would most likely meet resistance from both
Palestinian and Jewish hard-liners.

But it's not fanciful. The goal of equality is now more realistic than
the goal of separation. The reason is that changing the status quo
requires a vision powerful enough to create a mass movement. A
fragmented Palestinian state under Israeli control does not offer that
vision. Equality can. Increasingly, one equal state is not only the
\href{https://www.ft.com/content/5c7ed0b2-fe74-11e6-96f8-3700c5664d30}{preference}
of young Palestinians. It is the preference
of\href{https://sadat.umd.edu/sites/sadat.umd.edu/files/UMCIP\%20Questionnaire\%20Sep\%20to\%20Oct\%202018.pdf}{young
Americans, too}.

\includegraphics{https://static01.graylady3jvrrxbe.onion/images/2020/07/08/opinion/08beinart2/08beinart2-articleLarge.jpg?quality=75\&auto=webp\&disable=upscale}

Critics will say binational states don't work. But Israel is already a
binational state. Two peoples, roughly equal in number, live under the
ultimate control of one government. (Even in Gaza, Palestinians
\href{https://gisha.org/publication/1649}{can't} import milk, export
tomatoes or travel abroad without Israel's permission.) And the
\href{https://journals.sagepub.com/doi/10.1177/019251219301400203}{political
science} literature is
\href{https://www.jstor.org/stable/40646192?seq=1}{clear}: Divided
societies are most stable and most peaceful when governments represent
all their people.

That's the lesson of Northern Ireland. When Protestants and the British
government excluded Catholics, the Irish Republican Army
\href{https://www.polisci.upenn.edu/ppec/PPEC\%20People/Brendan\%20O'Leary/publications/Journal\%20Articles/Oleary_field_day_review_ira_mission.pdf}{killed
an estimated 1,750 people} between 1969 and 1994. When Catholics became
equal political partners, the violence largely stopped. It's the lesson
of South Africa, where Nelson Mandela endorsed armed struggle until
Blacks won the right to vote.

That lesson applies to Israel-Palestine, too. Yes, there are
Palestinians who have committed acts of terrorism. But so have the
members of many oppressed groups. History shows that when people gain
their freedom, violence declines. In the
\href{https://www.timesofisrael.com/islam-is-more-than-ready-for-peace-with-israel-says-rabbi-who-has-met-with-the-whole-strata-of-radicals/}{words}
of Michael Melchior, an Orthodox rabbi and former Israeli cabinet member
who has spent more than a decade forging relationships with leaders of
Hamas, ``I have yet to meet with somebody who is not willing to make
peace.''

Rabbi Melchior recently told me that he still supports a two-state
solution, but his point transcends any particular political arrangement:
It is that Palestinians will live peacefully alongside Jews when they
are granted basic rights.

What makes that hard for many Jews to grasp is the memory of the
Holocaust. As the Israeli scholar Yehuda Elkana, a Holocaust survivor,
\href{http://web.ceu.hu/yehuda_the_need_to_forget.pdf}{wrote} in 1988,
what ``motivates much of Israeli society in its relations with the
Palestinians is not personal frustration, but rather a profound
existential `Angst' fed by a particular interpretation of the lessons of
the Holocaust.'' This Holocaust lens leads many Jews to assume that
anything short of Jewish statehood would mean Jewish suicide.

But before the Holocaust, many leading Zionists did not believe that.
``The aspiration for a nation-state was not central in the Zionist
movement before the 1940s,'' writes the Hebrew University historian
Dmitry Shumsky in his book,
``\href{https://yalebooks.yale.edu/book/9780300230130/beyond-nation-state}{Beyond
the Nation-State}.'' A Jewish state has become the dominant form of
Zionism. But it is not the essence of Zionism. The essence of Zionism is
a Jewish home in the land of Israel, a thriving Jewish society that can
provide refuge and rejuvenation for Jews across the world.

That's what my grandfather and father loved --- not a Jewish state but a
Jewish society, a Jewish home.

Israel-Palestine can be a Jewish home that is also, equally, a
Palestinian home. And building that home can bring liberation not just
for Palestinians but for us, too.

Peter Beinart (\href{https://twitter.com/PeterBeinart}{@PeterBeinart})
is a professor of journalism and political science at the Newmark School
of Journalism at CUNY and editor at large of Jewish Currents.

\emph{The Times is committed to publishing}
\href{https://www.nytimes3xbfgragh.onion/2019/01/31/opinion/letters/letters-to-editor-new-york-times-women.html}{\emph{a
diversity of letters}} \emph{to the editor. We'd like to hear what you
think about this or any of our articles. Here are some}
\href{https://help.nytimes3xbfgragh.onion/hc/en-us/articles/115014925288-How-to-submit-a-letter-to-the-editor}{\emph{tips}}\emph{.
And here's our email:}
\href{mailto:letters@NYTimes.com}{\emph{letters@NYTimes.com}}\emph{.}

\emph{Follow The New York Times Opinion section on}
\href{https://www.facebookcorewwwi.onion/nytopinion}{\emph{Facebook}}\emph{,}
\href{http://twitter.com/NYTOpinion}{\emph{Twitter (@NYTopinion)}}
\emph{and}
\href{https://www.instagram.com/nytopinion/}{\emph{Instagram}}\emph{.}

Advertisement

\protect\hyperlink{after-bottom}{Continue reading the main story}

\hypertarget{site-index}{%
\subsection{Site Index}\label{site-index}}

\hypertarget{site-information-navigation}{%
\subsection{Site Information
Navigation}\label{site-information-navigation}}

\begin{itemize}
\tightlist
\item
  \href{https://help.nytimes3xbfgragh.onion/hc/en-us/articles/115014792127-Copyright-notice}{©~2020~The
  New York Times Company}
\end{itemize}

\begin{itemize}
\tightlist
\item
  \href{https://www.nytco.com/}{NYTCo}
\item
  \href{https://help.nytimes3xbfgragh.onion/hc/en-us/articles/115015385887-Contact-Us}{Contact
  Us}
\item
  \href{https://www.nytco.com/careers/}{Work with us}
\item
  \href{https://nytmediakit.com/}{Advertise}
\item
  \href{http://www.tbrandstudio.com/}{T Brand Studio}
\item
  \href{https://www.nytimes3xbfgragh.onion/privacy/cookie-policy\#how-do-i-manage-trackers}{Your
  Ad Choices}
\item
  \href{https://www.nytimes3xbfgragh.onion/privacy}{Privacy}
\item
  \href{https://help.nytimes3xbfgragh.onion/hc/en-us/articles/115014893428-Terms-of-service}{Terms
  of Service}
\item
  \href{https://help.nytimes3xbfgragh.onion/hc/en-us/articles/115014893968-Terms-of-sale}{Terms
  of Sale}
\item
  \href{https://spiderbites.nytimes3xbfgragh.onion}{Site Map}
\item
  \href{https://help.nytimes3xbfgragh.onion/hc/en-us}{Help}
\item
  \href{https://www.nytimes3xbfgragh.onion/subscription?campaignId=37WXW}{Subscriptions}
\end{itemize}
