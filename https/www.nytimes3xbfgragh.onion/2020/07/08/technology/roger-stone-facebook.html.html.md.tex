Sections

SEARCH

\protect\hyperlink{site-content}{Skip to
content}\protect\hyperlink{site-index}{Skip to site index}

\href{https://www.nytimes3xbfgragh.onion/section/technology}{Technology}

\href{https://myaccount.nytimes3xbfgragh.onion/auth/login?response_type=cookie\&client_id=vi}{}

\href{https://www.nytimes3xbfgragh.onion/section/todayspaper}{Today's
Paper}

\href{/section/technology}{Technology}\textbar{}Facebook Removes Roger
Stone for Ties to Fake Accounts

\url{https://nyti.ms/3iLE6Os}

\begin{itemize}
\item
\item
\item
\item
\item
\end{itemize}

Advertisement

\protect\hyperlink{after-top}{Continue reading the main story}

Supported by

\protect\hyperlink{after-sponsor}{Continue reading the main story}

\hypertarget{facebook-removes-roger-stone-for-ties-to-fake-accounts}{%
\section{Facebook Removes Roger Stone for Ties to Fake
Accounts}\label{facebook-removes-roger-stone-for-ties-to-fake-accounts}}

The social network said the fake accounts were active around the 2016
presidential election.

\includegraphics{https://static01.graylady3jvrrxbe.onion/images/2020/07/08/business/08fb-stone/merlin_173964846_1f395e5f-8bb2-4051-966a-c04b5516151c-articleLarge.jpg?quality=75\&auto=webp\&disable=upscale}

By \href{https://www.nytimes3xbfgragh.onion/by/davey-alba}{Davey Alba}

\begin{itemize}
\item
  Published July 8, 2020Updated July 10, 2020
\item
  \begin{itemize}
  \item
  \item
  \item
  \item
  \item
  \end{itemize}
\end{itemize}

\emph{{[}Read more on}
\href{https://www.nytimes3xbfgragh.onion/2020/07/10/us/politics/trump-roger-stone-clemency.html}{\emph{Roger
Stone's sentence being commuted by President Trump}}\emph{.{]}}

Facebook on Wednesday said it was removing the personal accounts of
\href{https://www.nytimes3xbfgragh.onion/2020/07/19/us/politics/roger-stone-mo-kelly-slur.html}{Roger
J. Stone Jr.}, President Trump's friend and ally, because they had ties
to numerous fake accounts that were active around the 2016 presidential
election.

The company made the announcement as part of its
\href{https://about.fb.com/news/2020/07/removing-political-coordinated-inauthentic-behavior/}{monthly
report} on removing disinformation. Mr. Stone's personal accounts on
Facebook and Instagram, which is owned by Facebook, were entwined with a
U.S.-based network of accounts that had links to the Proud Boys, a group
that promotes white supremacy, the company said. The social network
banned the Proud Boys group in 2018.

``We first started looking into this network as part of our
investigation into the Proud Boys' attempt to return to Facebook after
we had designated and banned them from the platform,'' Nathaniel
Gleicher, Facebook's head of cybersecurity policy, wrote in a company
blog post announcing Facebook's takedown. ``Our investigation linked
this network to Roger Stone and his associates.''

Mr. Stone, 67, is set to go to prison this month. In November,
\href{https://www.nytimes3xbfgragh.onion/2019/11/15/us/politics/roger-stone-trial-guilty.html}{a
jury convicted him} on seven felonies, including lying to federal
investigators, tampering with a witness and impeding a congressional
inquiry. The charges were brought by the special counsel, Robert S.
Mueller III, whose investigators scrutinized Mr. Stone's attempts during
the 2016 presidential election to communicate with WikiLeaks about the
release of Democratic emails that had been stolen by Russian operatives.

In a statement, Mr. Stone denied overseeing fake accounts on Facebook or
Instagram. ``This extraordinary active censorship for which Facebook and
Instagram give entirely fabricated reasons,'' he said, ``is part of a
larger effort to censor supporters of the president, Republicans and
conservatives on social media platforms. The claim that I have utilized
or control unauthorized or fake accounts on any platform is
categorically and provably false.''

This is not the first time that Mr. Stone has been kicked off a major
social media platform. In October 2017, Mr. Stone was suspended from
Twitter after insulting several CNN news anchors and contributors. In
July 2019, the federal judge overseeing the case brought by Mr. Mueller
ordered Mr. Stone off major social media platforms. The judge said Mr.
Stone had violated a gag order by using them to attack the special
counsel's investigation and officials tied to it.

Mr. Stone's accounts were part of the 54 Facebook accounts, 50 pages and
four Instagram accounts that Facebook said were associated with the
Proud Boys network. The network, the company said, was most active in
2016 and 2017, during the run-up to the United States presidential
election and immediately after. A few accounts were still active into
2020, posting primarily about Mr. Stone's court case and judgment
according to Graphika, a company that specializes in analyzing social
media,
\href{https://public-assets.graphika.com/reports/graphika_report_roger_stone_takedown.pdf}{which
released a report} about Facebook's Wednesday takedown.

Many of the accounts that Facebook removed used fake personas, stole
pictures of people around the internet and published posts promoting Mr.
Stone, according to Graphika's analysis. The accounts publicized his
books in 2016, and pushed for his legal defenses in 2019 and appeals for
a pardon in 2020.

\includegraphics{https://static01.graylady3jvrrxbe.onion/images/2020/07/08/technology/oakImage-1594241720548/oakImage-1594241720548-articleLarge.png?quality=75\&auto=webp\&disable=upscale}

The accounts also posted hostile criticism of Hillary Clinton,
especially in the lead-up to the 2016 election, Graphika said, and
engaged in coordinated harassment against a judge who had temporarily
blocked Mr. Trump's executive order barring citizens of seven
predominantly Muslim countries from entering the United States.

Facebook said it had identified the full scope of the network after
hundreds of pages of search warrants and affidavits were released in
response to a lawsuit filed by The New York Times and other news media
organizations.

The social network said it also took down 35 Facebook accounts, 14
pages, one group and 38 Instagram accounts involved in a domestic
disinformation campaign in Brazil, which were linked to ``some of the
employees of the offices'' of President Jair Bolsonaro of Brazil and two
of his sons, Congressman Eduardo Bolsonaro and Senator Flávio Bolsonaro.
It was not clear whether Brazil's president had any direct role in those
accounts.

``This shows that coordinated inauthentic behavior can turn up in many
places, even the offices of high-profile politicians,'' said Ben Nimmo,
director of investigations at Graphika. ``It also shows that there's a
whole community out there hunting for this kind of operation. It's a
brain race between the influence operations and the people who hunt
them, and every takedown teaches us a little more.''

Advertisement

\protect\hyperlink{after-bottom}{Continue reading the main story}

\hypertarget{site-index}{%
\subsection{Site Index}\label{site-index}}

\hypertarget{site-information-navigation}{%
\subsection{Site Information
Navigation}\label{site-information-navigation}}

\begin{itemize}
\tightlist
\item
  \href{https://help.nytimes3xbfgragh.onion/hc/en-us/articles/115014792127-Copyright-notice}{©~2020~The
  New York Times Company}
\end{itemize}

\begin{itemize}
\tightlist
\item
  \href{https://www.nytco.com/}{NYTCo}
\item
  \href{https://help.nytimes3xbfgragh.onion/hc/en-us/articles/115015385887-Contact-Us}{Contact
  Us}
\item
  \href{https://www.nytco.com/careers/}{Work with us}
\item
  \href{https://nytmediakit.com/}{Advertise}
\item
  \href{http://www.tbrandstudio.com/}{T Brand Studio}
\item
  \href{https://www.nytimes3xbfgragh.onion/privacy/cookie-policy\#how-do-i-manage-trackers}{Your
  Ad Choices}
\item
  \href{https://www.nytimes3xbfgragh.onion/privacy}{Privacy}
\item
  \href{https://help.nytimes3xbfgragh.onion/hc/en-us/articles/115014893428-Terms-of-service}{Terms
  of Service}
\item
  \href{https://help.nytimes3xbfgragh.onion/hc/en-us/articles/115014893968-Terms-of-sale}{Terms
  of Sale}
\item
  \href{https://spiderbites.nytimes3xbfgragh.onion}{Site Map}
\item
  \href{https://help.nytimes3xbfgragh.onion/hc/en-us}{Help}
\item
  \href{https://www.nytimes3xbfgragh.onion/subscription?campaignId=37WXW}{Subscriptions}
\end{itemize}
