Sections

SEARCH

\protect\hyperlink{site-content}{Skip to
content}\protect\hyperlink{site-index}{Skip to site index}

\href{https://www.nytimes3xbfgragh.onion/section/us}{U.S.}

\href{https://myaccount.nytimes3xbfgragh.onion/auth/login?response_type=cookie\&client_id=vi}{}

\href{https://www.nytimes3xbfgragh.onion/section/todayspaper}{Today's
Paper}

\href{/section/us}{U.S.}\textbar{}Pandemic Plunges Puerto Rico Into Yet
Another Dire Emergency

\url{https://nyti.ms/2ZUnqvU}

\begin{itemize}
\item
\item
\item
\item
\item
\item
\end{itemize}

\hypertarget{the-coronavirus-outbreak}{%
\subsubsection{\texorpdfstring{\href{https://www.nytimes3xbfgragh.onion/news-event/coronavirus?name=styln-coronavirus-national\&region=TOP_BANNER\&variant=undefined\&block=storyline_menu_recirc\&action=click\&pgtype=Article\&impression_id=3f940c70-e399-11ea-8313-f9307944f60a}{The
Coronavirus
Outbreak}}{The Coronavirus Outbreak}}\label{the-coronavirus-outbreak}}

\begin{itemize}
\tightlist
\item
  live\href{https://www.nytimes3xbfgragh.onion/2020/08/21/world/covid-19-coronavirus.html?name=styln-coronavirus-national\&region=TOP_BANNER\&variant=undefined\&block=storyline_menu_recirc\&action=click\&pgtype=Article\&impression_id=3f943380-e399-11ea-8313-f9307944f60a}{Latest
  Updates}
\item
  \href{https://www.nytimes3xbfgragh.onion/interactive/2020/us/coronavirus-us-cases.html?name=styln-coronavirus-national\&region=TOP_BANNER\&variant=undefined\&block=storyline_menu_recirc\&action=click\&pgtype=Article\&impression_id=3f943381-e399-11ea-8313-f9307944f60a}{Maps
  and Cases}
\item
  \href{https://www.nytimes3xbfgragh.onion/interactive/2020/science/coronavirus-vaccine-tracker.html?name=styln-coronavirus-national\&region=TOP_BANNER\&variant=undefined\&block=storyline_menu_recirc\&action=click\&pgtype=Article\&impression_id=3f943382-e399-11ea-8313-f9307944f60a}{Vaccine
  Tracker}
\item
  \href{https://www.nytimes3xbfgragh.onion/2020/08/19/us/colleges-closing-covid.html?name=styln-coronavirus-national\&region=TOP_BANNER\&variant=undefined\&block=storyline_menu_recirc\&action=click\&pgtype=Article\&impression_id=3f943383-e399-11ea-8313-f9307944f60a}{Colleges
  Closing}
\item
  \href{https://www.nytimes3xbfgragh.onion/live/2020/08/20/business/stock-market-today-coronavirus?name=styln-coronavirus-national\&region=TOP_BANNER\&variant=undefined\&block=storyline_menu_recirc\&action=click\&pgtype=Article\&impression_id=3f943384-e399-11ea-8313-f9307944f60a}{Economy}
\end{itemize}

Advertisement

\protect\hyperlink{after-top}{Continue reading the main story}

Supported by

\protect\hyperlink{after-sponsor}{Continue reading the main story}

\hypertarget{pandemic-plunges-puerto-rico-into-yet-another-dire-emergency}{%
\section{Pandemic Plunges Puerto Rico Into Yet Another Dire
Emergency}\label{pandemic-plunges-puerto-rico-into-yet-another-dire-emergency}}

The island has had to weather a hurricane, a political crisis and
earthquakes, but those crises did not lead to the widespread
unemployment caused by the response to the coronavirus pandemic.

\includegraphics{https://static01.graylady3jvrrxbe.onion/images/2020/07/06/us/00virus-puertorico01/00virus-puertorico01-articleLarge.jpg?quality=75\&auto=webp\&disable=upscale}

By Alejandra Rosa and
\href{https://www.nytimes3xbfgragh.onion/by/frances-robles}{Frances
Robles}

\begin{itemize}
\item
  Published July 8, 2020Updated Aug. 7, 2020
\item
  \begin{itemize}
  \item
  \item
  \item
  \item
  \item
  \item
  \end{itemize}
\end{itemize}

SAN JUAN, P.R. --- With hundreds of thousands of people suddenly out of
jobs in
\href{https://www.nytimes3xbfgragh.onion/2020/08/07/us/puerto-rico-earthquake.html}{Puerto
Rico}, Luciano Soto, a tour guide who has not worked in nearly four
months, wanted to be first in line at the Puerto Rico Convention Center,
now outfitted as an unemployment office.

He showed up at 8 p.m. one night a few weeks ago, with a lunchbox full
of snacks, prepared to spend the night, so that he could find out why
the unemployment benefits he had applied for months earlier had never
arrived. By 5 a.m., more than 400 others were also at the convention
center, and many furious people were turned away.

Mr. Soto finally got his money last week, after finding three of his
checks at the post office: The government had mailed them to the wrong
address.

``This is going to go on for a while,'' said Mr. Soto, 57, who is
worried that the cruise industry he depends on will not quickly recover.
``Would you take a cruise right now, even if someone gave it to you for
free?''

As the coronavirus pandemic sweeps the globe, shutting businesses,
killing the vulnerable and crippling economies, Puerto Rico has taken
one of the country's hardest economic hits.

Gov. Wanda Vázquez was the first governor in the nation to order
businesses to close and people to stay home. Experts say that her quick
action helped stave off an even worse medical crisis on the island. But
the pandemic has nonetheless plunged Puerto Rico into its fifth dire
emergency in three years, one that the government has struggled to
manage.

Thanks largely to hurricane reconstruction, Puerto Rico's economy had
been inching toward recovery after a devastating 2017 storm and the
bankruptcy of the island's government the same year. A civic uprising
paralyzed the island last summer and led to the ouster of Governor
Vázquez's predecessor. Then a series of earthquakes shook the south side
of the island in January, damaging homes and buildings, sending
thousands to live on the street, and closing schools across the island.

As of last week, despite guidance from the Centers for Disease Control
and Prevention that everyone should be washing their hands frequently
during the coronavirus pandemic, the governor announced that because of
a
\href{https://apnews.com/485aa49fc349decdbbef5e663d92dc74\#:~:text=SAN\%20JUAN\%2C\%20Puerto\%20Rico\%20(AP,territory\%20amid\%20a\%20coronavirus\%20pandemic.\&text=Gov.\%20Wanda\%20V\%C3\%A1zquez\%20said\%2021,29\%20by\%20the\%20moderate\%20drought.}{severe
drought}, parts of the island would have running water only every other
day for the foreseeable future. So far, the island has had 8,714
confirmed and likely cases of the virus, and 157 deaths.

Experts say this latest economic crisis has been even more difficult
than the one that followed Hurricane Maria. For one thing, aid has not
come pouring in from around the world, as it did after disasters of the
natural kind. And with Covid-19 creating serious problems in Florida and
other parts of the United States, unemployed Puerto Ricans, who fled to
the mainland in droves after Hurricane Maria, have nowhere to turn.

\hypertarget{latest-updates-the-coronavirus-outbreak}{%
\section{\texorpdfstring{\href{https://www.nytimes3xbfgragh.onion/2020/08/21/world/covid-19-coronavirus.html?action=click\&pgtype=Article\&state=default\&region=MAIN_CONTENT_1\&context=storylines_live_updates}{Latest
Updates: The Coronavirus
Outbreak}}{Latest Updates: The Coronavirus Outbreak}}\label{latest-updates-the-coronavirus-outbreak}}

Updated 2020-08-21T10:13:38.790Z

\begin{itemize}
\tightlist
\item
  \href{https://www.nytimes3xbfgragh.onion/2020/08/21/world/covid-19-coronavirus.html?action=click\&pgtype=Article\&state=default\&region=MAIN_CONTENT_1\&context=storylines_live_updates\#link-4690b6aa}{Shutdowns,
  warnings and scoldings follow gatherings on college campuses.}
\item
  \href{https://www.nytimes3xbfgragh.onion/2020/08/21/world/covid-19-coronavirus.html?action=click\&pgtype=Article\&state=default\&region=MAIN_CONTENT_1\&context=storylines_live_updates\#link-324af071}{As
  he accepts the Democratic nomination, Biden knocks Trump's pandemic
  response.}
\item
  \href{https://www.nytimes3xbfgragh.onion/2020/08/21/world/covid-19-coronavirus.html?action=click\&pgtype=Article\&state=default\&region=MAIN_CONTENT_1\&context=storylines_live_updates\#link-35890b73}{Hundreds
  of doctors in Kenya go on strike over their pay and protective gear.}
\end{itemize}

\href{https://www.nytimes3xbfgragh.onion/2020/08/21/world/covid-19-coronavirus.html?action=click\&pgtype=Article\&state=default\&region=MAIN_CONTENT_1\&context=storylines_live_updates}{See
more updates}

More live coverage:
\href{https://www.nytimes3xbfgragh.onion/live/2020/08/20/business/stock-market-today-coronavirus?action=click\&pgtype=Article\&state=default\&region=MAIN_CONTENT_1\&context=storylines_live_updates}{Markets}

As a result, on an island that already had the highest
\href{https://www.census.gov/quickfacts/fact/table/MS,PR/LFE305218}{poverty
rate} in the United States, at least 300,000 Puerto Ricans have filed
unemployment claims linked to the pandemic --- out of a civilian labor
force of 1.05 million --- and many others are ineligible for aid because
they are part of the island's large informal economy. Puerto Rico in
mid-June had the highest insured unemployment rate in the country, at 23
percent, according to the U.S. Department of Labor.

Thousands of people were left waiting for their checks as Puerto Rico's
understaffed bureaucracy struggled to keep up with the flood of claims.
A postal worker's video went viral when he showed unemployment checks
stuck in mail purgatory because they had been addressed to ``Same'':
Many applicants had written their home addresses on the form and then,
when asked for their mailing address, wrote ``same as above.''

The island's labor secretary, Briseida Torres, resigned in June, when
public fury was mounting over the long lines and delayed payouts,
including associated delays in receiving federal stimulus checks. Since
then, claims have started being paid, but workers without jobs have
remained deeply unsettled.

``I lost my job on June 1,'' said Marelys Figueroa, 27, who was a
marketing and recruitment officer at a university. ``I was working from
home, and I got an email. It said I was unemployed, effective
immediately. My income was the strongest of my household, and I have a
4-year-old child.''

She turned to a local food bank for help, vowing to go back to school
and start her own business.

\includegraphics{https://static01.graylady3jvrrxbe.onion/images/2020/07/06/us/00virus-puertorico03/merlin_173544108_b285d403-c28f-43cf-9825-7b9625314d0e-articleLarge.jpg?quality=75\&auto=webp\&disable=upscale}

Image

Comedores Sociales received 8,000 requests for help in the first two
months of the pandemic. Credit...Dennis M. Rivera Pichardo for The New
York Times

The food bank, Comedores Sociales, said that it received 8,000 requests
for help in the first two months of the pandemic. Paola Aponte Cotto, a
worker there, said she spends 20 hours a week on the phone, fielding
requests.

``We've received calls from people crying,'' Ms. Aponte said. ``The
callers who are most shocked are the ones who lost their jobs.''

Puerto Rico's new labor secretary, Carlos Rivera Santiago, acknowledged
that there were delays in getting unemployment checks to those who
needed them, but said the government had opened more offices and
improved online services to address the backlog. After weeks of chaotic
scenes like the one Mr. Soto experienced, the government implemented a
more orderly indoor process, with socially distanced seats where weary
applicants wait for their appointments.

More than 300,000 people are now getting benefits, Mr. Rivera Santiago
said, while total claims in two available assistance programs have
reached 500,000.

``It's a challenge, and one Puerto Rico is going to confront,'' he said.
``We have to reinvent ourselves, change the way we work. Remote work has
become the order of the day, and that was not very usual in Puerto
Rico.''

He stressed that the key to recovery will be injecting money into the
economy.

Puerto Rico is expected to receive \$13 billion in Covid-related federal
funds, according to the Financial Oversight and Management Board, the
agency that has managed Puerto Rico's finances since it defaulted on
\$72 billion in debt. The board has estimated that Puerto Rico's economy
will contract by 4 percent.

In March, Governor Vázquez announced a \$787 million stimulus package,
which included \$160 million in grants to small businesses and the
self-employed. The government also set aside some special Covid-related
federal grants to help heavily affected sectors like hospitals and
tourism. About \$350 million in federal funds went to the private sector
to help pay employees at businesses that were disrupted, according to a
Puerto Rico government report.

\href{https://www.nytimes3xbfgragh.onion/news-event/coronavirus?action=click\&pgtype=Article\&state=default\&region=MAIN_CONTENT_3\&context=storylines_faq}{}

\hypertarget{the-coronavirus-outbreak-}{%
\subsubsection{The Coronavirus Outbreak
›}\label{the-coronavirus-outbreak-}}

\hypertarget{frequently-asked-questions}{%
\paragraph{Frequently Asked
Questions}\label{frequently-asked-questions}}

Updated August 17, 2020

\begin{itemize}
\item ~
  \hypertarget{why-does-standing-six-feet-away-from-others-help}{%
  \paragraph{Why does standing six feet away from others
  help?}\label{why-does-standing-six-feet-away-from-others-help}}

  \begin{itemize}
  \tightlist
  \item
    The coronavirus spreads primarily through droplets from your mouth
    and nose, especially when you cough or sneeze. The C.D.C., one of
    the organizations using that measure,
    \href{https://www.nytimes3xbfgragh.onion/2020/04/14/health/coronavirus-six-feet.html?action=click\&pgtype=Article\&state=default\&region=MAIN_CONTENT_3\&context=storylines_faq}{bases
    its recommendation of six feet} on the idea that most large droplets
    that people expel when they cough or sneeze will fall to the ground
    within six feet. But six feet has never been a magic number that
    guarantees complete protection. Sneezes, for instance, can launch
    droplets a lot farther than six feet,
    \href{https://jamanetwork.com/journals/jama/fullarticle/2763852}{according
    to a recent study}. It's a rule of thumb: You should be safest
    standing six feet apart outside, especially when it's windy. But
    keep a mask on at all times, even when you think you're far enough
    apart.
  \end{itemize}
\item ~
  \hypertarget{i-have-antibodies-am-i-now-immune}{%
  \paragraph{I have antibodies. Am I now
  immune?}\label{i-have-antibodies-am-i-now-immune}}

  \begin{itemize}
  \tightlist
  \item
    As of right
    now,\href{https://www.nytimes3xbfgragh.onion/2020/07/22/health/covid-antibodies-herd-immunity.html?action=click\&pgtype=Article\&state=default\&region=MAIN_CONTENT_3\&context=storylines_faq}{that
    seems likely, for at least several months.} There have been
    frightening accounts of people suffering what seems to be a second
    bout of Covid-19. But experts say these patients may have a
    drawn-out course of infection, with the virus taking a slow toll
    weeks to months after initial exposure. People infected with the
    coronavirus typically
    \href{https://www.nature.com/articles/s41586-020-2456-9}{produce}
    immune molecules called antibodies, which are
    \href{https://www.nytimes3xbfgragh.onion/2020/05/07/health/coronavirus-antibody-prevalence.html?action=click\&pgtype=Article\&state=default\&region=MAIN_CONTENT_3\&context=storylines_faq}{protective
    proteins made in response to an
    infection}\href{https://www.nytimes3xbfgragh.onion/2020/05/07/health/coronavirus-antibody-prevalence.html?action=click\&pgtype=Article\&state=default\&region=MAIN_CONTENT_3\&context=storylines_faq}{.
    These antibodies may} last in the body
    \href{https://www.nature.com/articles/s41591-020-0965-6}{only two to
    three months}, which may seem worrisome, but that's perfectly normal
    after an acute infection subsides, said Dr. Michael Mina, an
    immunologist at Harvard University. It may be possible to get the
    coronavirus again, but it's highly unlikely that it would be
    possible in a short window of time from initial infection or make
    people sicker the second time.
  \end{itemize}
\item ~
  \hypertarget{im-a-small-business-owner-can-i-get-relief}{%
  \paragraph{I'm a small-business owner. Can I get
  relief?}\label{im-a-small-business-owner-can-i-get-relief}}

  \begin{itemize}
  \tightlist
  \item
    The
    \href{https://www.nytimes3xbfgragh.onion/article/small-business-loans-stimulus-grants-freelancers-coronavirus.html?action=click\&pgtype=Article\&state=default\&region=MAIN_CONTENT_3\&context=storylines_faq}{stimulus
    bills enacted in March} offer help for the millions of American
    small businesses. Those eligible for aid are businesses and
    nonprofit organizations with fewer than 500 workers, including sole
    proprietorships, independent contractors and freelancers. Some
    larger companies in some industries are also eligible. The help
    being offered, which is being managed by the Small Business
    Administration, includes the Paycheck Protection Program and the
    Economic Injury Disaster Loan program. But lots of folks have
    \href{https://www.nytimes3xbfgragh.onion/interactive/2020/05/07/business/small-business-loans-coronavirus.html?action=click\&pgtype=Article\&state=default\&region=MAIN_CONTENT_3\&context=storylines_faq}{not
    yet seen payouts.} Even those who have received help are confused:
    The rules are draconian, and some are stuck sitting on
    \href{https://www.nytimes3xbfgragh.onion/2020/05/02/business/economy/loans-coronavirus-small-business.html?action=click\&pgtype=Article\&state=default\&region=MAIN_CONTENT_3\&context=storylines_faq}{money
    they don't know how to use.} Many small-business owners are getting
    less than they expected or
    \href{https://www.nytimes3xbfgragh.onion/2020/06/10/business/Small-business-loans-ppp.html?action=click\&pgtype=Article\&state=default\&region=MAIN_CONTENT_3\&context=storylines_faq}{not
    hearing anything at all.}
  \end{itemize}
\item ~
  \hypertarget{what-are-my-rights-if-i-am-worried-about-going-back-to-work}{%
  \paragraph{What are my rights if I am worried about going back to
  work?}\label{what-are-my-rights-if-i-am-worried-about-going-back-to-work}}

  \begin{itemize}
  \tightlist
  \item
    Employers have to provide
    \href{https://www.osha.gov/SLTC/covid-19/standards.html}{a safe
    workplace} with policies that protect everyone equally.
    \href{https://www.nytimes3xbfgragh.onion/article/coronavirus-money-unemployment.html?action=click\&pgtype=Article\&state=default\&region=MAIN_CONTENT_3\&context=storylines_faq}{And
    if one of your co-workers tests positive for the coronavirus, the
    C.D.C.} has said that
    \href{https://www.cdc.gov/coronavirus/2019-ncov/community/guidance-business-response.html}{employers
    should tell their employees} -\/- without giving you the sick
    employee's name -\/- that they may have been exposed to the virus.
  \end{itemize}
\item ~
  \hypertarget{what-is-school-going-to-look-like-in-september}{%
  \paragraph{What is school going to look like in
  September?}\label{what-is-school-going-to-look-like-in-september}}

  \begin{itemize}
  \tightlist
  \item
    It is unlikely that many schools will return to a normal schedule
    this fall, requiring the grind of
    \href{https://www.nytimes3xbfgragh.onion/2020/06/05/us/coronavirus-education-lost-learning.html?action=click\&pgtype=Article\&state=default\&region=MAIN_CONTENT_3\&context=storylines_faq}{online
    learning},
    \href{https://www.nytimes3xbfgragh.onion/2020/05/29/us/coronavirus-child-care-centers.html?action=click\&pgtype=Article\&state=default\&region=MAIN_CONTENT_3\&context=storylines_faq}{makeshift
    child care} and
    \href{https://www.nytimes3xbfgragh.onion/2020/06/03/business/economy/coronavirus-working-women.html?action=click\&pgtype=Article\&state=default\&region=MAIN_CONTENT_3\&context=storylines_faq}{stunted
    workdays} to continue. California's two largest public school
    districts --- Los Angeles and San Diego --- said on July 13, that
    \href{https://www.nytimes3xbfgragh.onion/2020/07/13/us/lausd-san-diego-school-reopening.html?action=click\&pgtype=Article\&state=default\&region=MAIN_CONTENT_3\&context=storylines_faq}{instruction
    will be remote-only in the fall}, citing concerns that surging
    coronavirus infections in their areas pose too dire a risk for
    students and teachers. Together, the two districts enroll some
    825,000 students. They are the largest in the country so far to
    abandon plans for even a partial physical return to classrooms when
    they reopen in August. For other districts, the solution won't be an
    all-or-nothing approach.
    \href{https://bioethics.jhu.edu/research-and-outreach/projects/eschool-initiative/school-policy-tracker/}{Many
    systems}, including the nation's largest, New York City, are
    devising
    \href{https://www.nytimes3xbfgragh.onion/2020/06/26/us/coronavirus-schools-reopen-fall.html?action=click\&pgtype=Article\&state=default\&region=MAIN_CONTENT_3\&context=storylines_faq}{hybrid
    plans} that involve spending some days in classrooms and other days
    online. There's no national policy on this yet, so check with your
    municipal school system regularly to see what is happening in your
    community.
  \end{itemize}
\end{itemize}

José Caraballo-Cueto, an associate professor at the University of Puerto
Rico, said the rollout of the stimulus was problematic, because it gave
priority to things like hazard pay for front-line workers, who were
still employed.

Puerto Rico's official unemployment rate has not been reported since
February, but Mr. Caraballo-Cueto estimates that it is now close to 40
percent. It is unclear how many people have returned to work since the
governor
\href{https://www.nytimes3xbfgragh.onion/2020/05/08/travel/travel-reopenings-virus.html}{authorized
businesses} to reopen in mid-June.

``The claims were the biggest we ever saw, since the beginning of
recording unemployment in the 1980s,'' Mr. Caraballo-Cueto said. ``And
the government pretended to process all of those new claims with the
same number of employees they always had.''

Image

Puerto Rico saw weeks of chaotic scenes in unemployment
lines.Credit...Dennis M. Rivera Pichardo for The New York Times

Image

A government worker collected information from unemployment applicants
in San Juan.Credit...Dennis M. Rivera Pichardo for The New York Times

Maria Enchautegui, an economist at the Youth Development Institute of
Puerto Rico, a policy and research organization, said the greatest
challenge was the technical capacity of the island's Labor Department,
which was too limited to deal with the crush of applications. If the
promised aid does not quickly arrive, she said, the poverty rate among
workers in businesses affected by the recent lockdowns could climb to 77
percent.

``The effect on poverty could be quite bad,'' Ms. Enchautegui said.

Many Puerto Ricans have informal off-the-books jobs that do not qualify
them for unemployment benefits, making the situation on the island even
more complicated, said Amanda M. Rivera, the institute's executive
director.

The poverty rate did not climb after Hurricane Maria in 2017, she said,
despite experts' predictions, because so many people left the island.

``After the hurricane, there was a valve, a place people could go where
things were normal,'' Ms. Rivera said. ``Now that people don't have that
valve, that's a real concern.''

Roberto Rivera, 55, a driver for a guided tour company, slept in his car
one night a few weeks ago, his crutches by his side, so he would be the
first in line for a short-lived drive-through service center established
by the Labor Department to receive disaster unemployment assistance
applications.

The scene resembled gasoline lines that formed during Hurricane Maria,
but were much longer, as thousands of people traveled overnight from
different parts of Puerto Rico to arrive at the convention center before
sunrise, driving on dark highways where the streetlights have not been
turned on since government austerity cuts three years ago.

``I arrived here at 11 p.m. ---~I need the money, I have to pay rent,
buy food,'' Mr. Rivera said. ``I submitted everything a month ago, and I
haven't received anything. I've been surviving with the little I have
left.''

Advertisement

\protect\hyperlink{after-bottom}{Continue reading the main story}

\hypertarget{site-index}{%
\subsection{Site Index}\label{site-index}}

\hypertarget{site-information-navigation}{%
\subsection{Site Information
Navigation}\label{site-information-navigation}}

\begin{itemize}
\tightlist
\item
  \href{https://help.nytimes3xbfgragh.onion/hc/en-us/articles/115014792127-Copyright-notice}{©~2020~The
  New York Times Company}
\end{itemize}

\begin{itemize}
\tightlist
\item
  \href{https://www.nytco.com/}{NYTCo}
\item
  \href{https://help.nytimes3xbfgragh.onion/hc/en-us/articles/115015385887-Contact-Us}{Contact
  Us}
\item
  \href{https://www.nytco.com/careers/}{Work with us}
\item
  \href{https://nytmediakit.com/}{Advertise}
\item
  \href{http://www.tbrandstudio.com/}{T Brand Studio}
\item
  \href{https://www.nytimes3xbfgragh.onion/privacy/cookie-policy\#how-do-i-manage-trackers}{Your
  Ad Choices}
\item
  \href{https://www.nytimes3xbfgragh.onion/privacy}{Privacy}
\item
  \href{https://help.nytimes3xbfgragh.onion/hc/en-us/articles/115014893428-Terms-of-service}{Terms
  of Service}
\item
  \href{https://help.nytimes3xbfgragh.onion/hc/en-us/articles/115014893968-Terms-of-sale}{Terms
  of Sale}
\item
  \href{https://spiderbites.nytimes3xbfgragh.onion}{Site Map}
\item
  \href{https://help.nytimes3xbfgragh.onion/hc/en-us}{Help}
\item
  \href{https://www.nytimes3xbfgragh.onion/subscription?campaignId=37WXW}{Subscriptions}
\end{itemize}
