Sections

SEARCH

\protect\hyperlink{site-content}{Skip to
content}\protect\hyperlink{site-index}{Skip to site index}

\href{https://www.nytimes3xbfgragh.onion/section/nyregion}{New York}

\href{https://myaccount.nytimes3xbfgragh.onion/auth/login?response_type=cookie\&client_id=vi}{}

\href{https://www.nytimes3xbfgragh.onion/section/todayspaper}{Today's
Paper}

\href{/section/nyregion}{New York}\textbar{}`Anti-Feminist' Lawyer Is
Suspect in Killing of Son of Federal Judge in N.J.

\url{https://nyti.ms/2CWqeR1}

\begin{itemize}
\item
\item
\item
\item
\item
\item
\end{itemize}

Advertisement

\protect\hyperlink{after-top}{Continue reading the main story}

Supported by

\protect\hyperlink{after-sponsor}{Continue reading the main story}

\hypertarget{anti-feminist-lawyer-is-suspect-in-killing-of-son-of-federal-judge-in-nj}{%
\section{`Anti-Feminist' Lawyer Is Suspect in Killing of Son of Federal
Judge in
N.J.}\label{anti-feminist-lawyer-is-suspect-in-killing-of-son-of-federal-judge-in-nj}}

Roy Den Hollander had openly seethed against the judge, Esther Salas.
After the shooting at her home, he was found dead in an apparent
suicide.

\includegraphics{https://static01.graylady3jvrrxbe.onion/images/2020/08/20/world/20njjudge1/20njjudge1-articleLarge-v2.jpg?quality=75\&auto=webp\&disable=upscale}

By \href{https://www.nytimes3xbfgragh.onion/by/nicole-hong}{Nicole
Hong},
\href{https://www.nytimes3xbfgragh.onion/by/william-k-rashbaum}{William
K. Rashbaum} and
\href{https://www.nytimes3xbfgragh.onion/by/mihir-zaveri}{Mihir Zaveri}

\begin{itemize}
\item
  Published July 20, 2020Updated July 22, 2020
\item
  \begin{itemize}
  \item
  \item
  \item
  \item
  \item
  \item
  \end{itemize}
\end{itemize}

\href{https://www.nytimes3xbfgragh.onion/2020/07/22/nyregion/roy-hollander-esther-salas.html}{Roy
Den Hollander} was a self-described ``anti-feminist'' lawyer who flooded
the courts with seemingly frivolous lawsuits that sought to eliminate
women's studies programs and prohibit nightclubs from holding ``ladies'
nights.''

In one of his most recent cases, he openly seethed against a federal
judge in New Jersey,
\href{https://www.nytimes3xbfgragh.onion/2020/07/25/nyregion/roy-den-hollander-esther-salas-list.html}{Esther
Salas}, whom he described in a self-published, 1,700-page book as ``a
lazy and incompetent Latina judge appointed by Obama.''

Mr. Den Hollander left the case, in which he challenged the male-only
United States military draft, last summer, telling a lawyer who replaced
him that he had terminal cancer.

On Sunday afternoon, Mr. Den Hollander showed up at Judge Salas's home
in North Brunswick, N.J., and fired multiple gunshots,
\href{https://www.nytimes3xbfgragh.onion/2020/07/19/nyregion/shooting-nj-judge-esther-salas.html}{killing
the judge's son and seriously wounding her husband}, who is a criminal
defense lawyer, investigators said. The judge, who was in the basement
at the time, was not injured.

The New York State Police found Mr. Den Hollander's body near Liberty,
N.Y. --- about a two-hour drive from the judge's home --- after he shot
himself in an apparent suicide, officials said.

The startling sequence of events was a reminder of the dangers
encountered by judges, who typically do not receive special security
outside the courthouse unless they face specific threats. Judge Salas
worked in one of the busiest courthouses in the country, overseeing
dozens of cases at a time involving a wide range of defendants and
litigants.

The F.B.I. on Monday contacted New York State's chief judge, Janet M.
DiFiore, to notify her that Mr. Den Hollander had her name and photo in
his car, according to her spokesman, Lucian Chalfen. The agents did not
indicate whether Mr. Den Hollander had intended to target her as well,
he said.

Investigators were exploring whether Mr. Den Hollander had decided to
``take out'' some of his enemies, given his cancer diagnosis, before he
died, according to one law enforcement official.

Mr. Den Hollander, 72, identified with a broader movement of men who in
often abusive, misogynist and hateful language rail against
``feminazis.'' He wrote numerous online screeds, some of which exceeded
a thousand pages.

Mr. Den Hollander had a long history of filing lawsuits against programs
that he believed favored women. In 2008, he
\href{https://cityroom.blogs.nytimes3xbfgragh.onion/2008/08/18/lawyer-files-antifeminist-suit-against-columbia/\#:~:text=Roy\%20Den\%20Hollander\%20is\%20a,and\%20a\%20self\%2Ddescribed\%20antifeminist.\&text=On\%20Monday\%2C\%20he\%20filed\%20a,sees\%20as\%20discriminatory\%20toward\%20men}{told
The New York Times that his anger toward feminists} stemmed from his
bitter divorce from a woman he married in Russia.

He called women ``the real oppressors'' in a 2008 Fox News appearance
and wrote online about his grievances against female judges.

\includegraphics{https://static01.graylady3jvrrxbe.onion/images/2020/07/21/nyregion/21njjudge2/merlin_174767310_9a34c2b7-ff2c-4e7a-b772-e7a95c0e1490-articleLarge.jpg?quality=75\&auto=webp\&disable=upscale}

When investigators discovered Mr. Den Hollander's body, they found a
package nearby that was addressed to Judge Salas, according to a law
enforcement official. The package was empty.

On Sunday afternoon, the judge's husband was at home when he looked out
the window and thought he saw a FedEx deliveryman.

Carlos Salas, an older brother of Judge Salas, described an account of
the shooting that he said was provided to him by federal authorities.
After the doorbell rang, the couple's son opened the door and was shot.
When the judge's husband went over to see what happened, he was shot
multiple times.

The judge ran upstairs from the basement when she heard a scream and the
gunshots.

The judge's husband, Mark Anderl, 63, was in the hospital in stable
condition, Mr. Salas said. The couple's son and only child, Daniel
Anderl, 20, died from a gunshot wound to the heart.

Daniel Anderl was about to start his junior year at Catholic University
of America in Washington and was interested in pursuing a legal career
as his parents had.

``It's surreal,'' Mr. Salas said. ``He was a vibrant, young,
good-looking man. He had so much promise.''

The F.B.I. has been conducting the investigation with the U.S. marshals
alongside other federal and local authorities. A spokesman for FedEx
said in a statement that the company was ``fully cooperating with the
authorities in their investigation.''

Two law enforcement officials, cautioning that the investigation was in
its earliest stages, said federal authorities were examining whether Mr.
Den Hollander might be linked to the July 11 killing of another men's
rights lawyer, Marc Angelucci, in San Bernardino County, Calif.

Mr. Angelucci was shot at his front door by a gunman wearing a FedEx
uniform, one of the officials said.

Judge Salas, 51, is the
\href{https://www.nj.com/news/2011/06/nj_appoints_first_hispanic_wom.html}{first
Hispanic woman to serve as a federal judge in New Jersey}.
\href{https://obamawhitehouse.archives.gov/the-press-office/2010/12/01/president-obama-names-seven-united-states-district-court}{President
Barack Obama nominated} her to the United States District Court for New
Jersey in 2010. She had previously served as a magistrate judge and an
assistant federal public defender.

Judge Salas met her husband when he was a prosecutor in the Essex County
Prosecutor's Office,
\href{https://njmonthly.com/articles/politics-public-affairs/immigration-stories-esther-salas/}{according
to a 2018 profile of her in New Jersey Monthly}. After a decade as a
prosecutor, Mark Anderl became a criminal defense lawyer and now works
at his own law firm, Anderl \& Oakley P.C.

According to the federal docket, the only case that Mr. Den Hollander
had before Judge Salas was a class-action lawsuit filed in 2015. He
accused the Selective Service System, the independent government agency
that maintains a database of Americans eligible for a potential draft,
of violating women's equal protection rights by requiring only men to
register with the service.

In a 2018 ruling, Judge Salas allowed the case to proceed, a victory for
Mr. Den Hollander. But in his online writings, he criticized the judge
for not moving the case along fast enough.

Nicholas A. Gravante Jr., a partner at Boies Schiller Flexner, said Mr.
Den Hollander had called him in May 2019 and asked him to take over the
case. The two lawyers had overlapped as associates at the white-shoe law
firm Cravath Swaine \& Moore in the late 1980s.

Mr. Den Hollander said in the phone call that he could not continue in
the case because he had terminal cancer and suggested that he did not
have long to live, Mr. Gravante said. The case is ongoing.

Mr. Den Hollander has also sued various nightclubs, claiming they
violated the 14th Amendment by having ``ladies' nights'' discounts for
women. After the case was dismissed, Mr. Den Hollander petitioned the
Supreme Court, which declined to hear the case.

``Of course, the three females on the court probably voted against it,''
Mr. Den Hollander
\href{https://cityroom.blogs.nytimes3xbfgragh.onion/2011/01/13/one-mans-odd-fight-against-ladies-nights/}{told
The Times in 2011}. ``Fighting for the rights of men is not very popular
thing to do in America these days.''

In 2008, he accused Columbia University of trying to establish feminism
as a ``religion'' at the school through its women's studies program and
proposed creating a men's studies program that could ``train males to
recognize and handle the power females often use to manipulate them.''

Shortly before the 2016 presidential election, he
\href{https://assets.documentcloud.org/documents/3034119/16cv6624-1-Complaint.pdf}{filed
a lawsuit in Manhattan} accusing several prominent news reporters of
conspiring together in violation of federal racketeering law to
disseminate ``misleading news reports'' about President Trump.

Mr. Den Hollander's online writings identified with the men's rights
movement, which gained traction in the late 1980s and 1990s as a
response to the feminist critique of traditional masculinity.

The movement embraces ``a celebration of all things masculine and a near
infatuation with the traditional masculine role itself,'' the
sociologist, Michael Kimmel, wrote in his book ``Angry White Men.''

In a 150-page document posted on his website, which has since been taken
down, Mr. Den Hollander wrote extensive commentary about how to fight
feminists and their supporters, describing the judiciary as ``useless
for men.''

``The courts support the violation of the rights of men whenever it
benefits females,'' Mr. Den Hollander wrote. ``Men just don't count to
the courts.''

Mr. Den Hollander seemed to presage the violence that took place over
the weekend in a 2010 article he wrote for A Voice for Men, a men's
rights website.

``The future prospect of the Men's Movement raising enough money to
exercise some influence in America is unlikely,'' he wrote. ``But there
is one remaining source of power in which men still have a near monopoly
--- firearms.''

Mr. Den Hollander graduated from the George Washington University Law
School in 1985. He later received a degree from Columbia Business
School, according to a LinkedIn profile under his name.

In the epilogue of the online book he published in 2019, Mr. Den
Hollander alluded to his cancer diagnosis, describing a visit to a
surgeon and the need to hand off his cases to other lawyers. ``Death's
hand is on my left shoulder,'' he wrote, adding that ``nothing in this
life matters anymore.''

He said that he had enjoyed fighting against people who violated his
rights.

``The only problem with a life lived too long under Feminazi rule,'' he
said, ``is that a man ends up with so many enemies he can't even the
score with all of them.''

Kevin Armstrong, Jo Corona and Alan Feuer contributed reporting, and
Kitty Bennett contributed research.

Advertisement

\protect\hyperlink{after-bottom}{Continue reading the main story}

\hypertarget{site-index}{%
\subsection{Site Index}\label{site-index}}

\hypertarget{site-information-navigation}{%
\subsection{Site Information
Navigation}\label{site-information-navigation}}

\begin{itemize}
\tightlist
\item
  \href{https://help.nytimes3xbfgragh.onion/hc/en-us/articles/115014792127-Copyright-notice}{©~2020~The
  New York Times Company}
\end{itemize}

\begin{itemize}
\tightlist
\item
  \href{https://www.nytco.com/}{NYTCo}
\item
  \href{https://help.nytimes3xbfgragh.onion/hc/en-us/articles/115015385887-Contact-Us}{Contact
  Us}
\item
  \href{https://www.nytco.com/careers/}{Work with us}
\item
  \href{https://nytmediakit.com/}{Advertise}
\item
  \href{http://www.tbrandstudio.com/}{T Brand Studio}
\item
  \href{https://www.nytimes3xbfgragh.onion/privacy/cookie-policy\#how-do-i-manage-trackers}{Your
  Ad Choices}
\item
  \href{https://www.nytimes3xbfgragh.onion/privacy}{Privacy}
\item
  \href{https://help.nytimes3xbfgragh.onion/hc/en-us/articles/115014893428-Terms-of-service}{Terms
  of Service}
\item
  \href{https://help.nytimes3xbfgragh.onion/hc/en-us/articles/115014893968-Terms-of-sale}{Terms
  of Sale}
\item
  \href{https://spiderbites.nytimes3xbfgragh.onion}{Site Map}
\item
  \href{https://help.nytimes3xbfgragh.onion/hc/en-us}{Help}
\item
  \href{https://www.nytimes3xbfgragh.onion/subscription?campaignId=37WXW}{Subscriptions}
\end{itemize}
