The Japanese-American Sculptor Who, Despite Persecution, Made Her Mark

\url{https://nyti.ms/2E4N9dP}

\begin{itemize}
\item
\item
\item
\item
\item
\item
\end{itemize}

\includegraphics{https://static01.graylady3jvrrxbe.onion/images/2020/07/10/t-magazine/art/asawa-slide-M1IM/asawa-slide-M1IM-articleLarge.jpg?quality=75\&auto=webp\&disable=upscale}

Sections

\protect\hyperlink{site-content}{Skip to
content}\protect\hyperlink{site-index}{Skip to site index}

The Great ReadTrue Believers

\hypertarget{the-japanese-american-sculptor-who-despite-persecution-made-her-mark}{%
\section{The Japanese-American Sculptor Who, Despite Persecution, Made
Her
Mark}\label{the-japanese-american-sculptor-who-despite-persecution-made-her-mark}}

Seven years after her death, Ruth Asawa is finally being recognized as
an American master. What can we learn from this overdue reappraisal?

Ruth Asawa as a young artist in 1954, surrounded by several of her wire
sculptures, which she began making in the late 1940s.Credit...Nat
Farbman/The Life Picture Collection/Getty Images

Supported by

\protect\hyperlink{after-sponsor}{Continue reading the main story}

By
\href{https://www.nytimes3xbfgragh.onion/by/thessaly-la-force}{Thessaly
La Force}

\begin{itemize}
\item
  July 20, 2020
\item
  \begin{itemize}
  \item
  \item
  \item
  \item
  \item
  \item
  \end{itemize}
\end{itemize}

IN 2009, THE New York City auction house Christie's received an
unsolicited query: A woman named Addie Lanier had a painting by
\href{https://www.nytimes3xbfgragh.onion/1976/03/26/archives/josef-albers-artist-and-teacher-dies.html}{Josef
Albers}, the midcentury abstract artist who pioneered modern arts
education. Could Christie's help her sell it? It wasn't uncommon for a
major auction house like Christie's to get cold calls. News generated by
large sales can create curiosity and spark interest; people often
approach auction houses in the hope of confirming that they have been
sitting on priceless works of art. Jonathan Laib --- then a senior vice
president and senior specialist of postwar and contemporary art at
Christie's --- was excited to hear of an Albers.

The details surrounding the painting, from Albers's ``Homage to the
Square'' series, intrigued Laib. Like many artists, Albers was fond of
trades and frequently gave artworks away. Rarely, though, did he gift a
painting as substantial as this one. In some respects, the series was
his masterpiece; for 26 years, Albers repeatedly nested three to four
superimposed squares of varying hues, a cumulative expression of his
life's work in revealing how perception could be manipulated by the
arrangement of form and color. Lanier also possessed a signed note from
Albers, verifying the painting's authenticity. It was surprisingly
affectionate: ``Dear Ruthie, This is just for revenge, And it is yours
for the promise not to acknowledge receiving it. Love, A.'' Lanier
attested that her mother, a woman named
\href{https://www.nytimes3xbfgragh.onion/news/the-lives-they-lived/2013/12/21/ruth-asawa/}{Ruth
Asawa}, and Albers had been friends.

\href{https://www.nytimes3xbfgragh.onion/issue/t-magazine/2020/07/02/true-believers-art-issue}{\includegraphics{https://static01.graylady3jvrrxbe.onion/newsgraphics/2020/06/29/tmag-art-embeds-new/assets/images/art_issue_gif_special_editon.gif}}

\includegraphics{https://static01.graylady3jvrrxbe.onion/images/2020/07/10/t-magazine/art/asawa-slide-CLQU/asawa-slide-CLQU-articleLarge.jpg?quality=75\&auto=webp\&disable=upscale}

Laib began a correspondence with Lanier, learning more about her mother,
a San Francisco-based artist, who was then 83 years old and bedridden
with lupus, a chronic autoimmune disease. The family was in need of
money to provide the round-the-clock care that Asawa required. The
eventual sale of the Albers brought in over \$100,000, but it left Laib
wondering if there was more to Asawa's story. She had been a student of
Albers **** at
\href{https://www.nytimes3xbfgragh.onion/2015/03/19/arts/artsspecial/in-the-spirit-of-black-mountain-college-an-avant-garde-incubator.html}{Black
Mountain College} in the 1940s. Her own art incorporated many of the
principles Albers espoused: the use of negative space, beauty in
repetition and a deep awareness of the material at hand. Asawa often
worked with coiled lines of metal wire that she wove into undulating,
biomorphic shapes that hung from the wood rafters of her house in Noe
Valley. She had shown these pieces in a New York City gallery, Peridot,
where she was represented for six years beginning in 1954, placing works
with top collectors including the
\href{https://www.nytimes3xbfgragh.onion/topic/organization/museum-of-modern-art}{Museum
of Modern Art}, the architect
\href{https://www.nytimes3xbfgragh.onion/2017/03/23/t-magazine/philip-johnson-rockefeller-guest-house-manhattan.html}{Philip
Johnson} and Mary Rockefeller, the first wife of New York Governor
\href{https://www.nytimes3xbfgragh.onion/1964/02/16/archives/political-evolution-of-nelson-rockefeller-in-less-than-six-years-he.html}{Nelson
Rockefeller}. It was surprising to Laib that Asawa's name was not as
well known as those of her contemporaries, such as
\href{https://tmagazine.blogs.nytimes3xbfgragh.onion/2010/03/25/just-looking-eva-hesse/}{Eva
Hesse},
\href{https://www.nytimes3xbfgragh.onion/2016/10/03/t-magazine/art/louise-bourgeois-turning-inwards.html}{Louise
Bourgeois} and
\href{https://www.nytimes3xbfgragh.onion/2016/09/01/t-magazine/art/yayoi-kusama-glass-house.html}{Yayoi
Kusama}. Asawa's last show with Peridot was in 1958. Less than a decade
later, she had all but disappeared from the New York art world.

Today, Asawa has returned as a subject of rediscovery --- someone who
has finally been given the kind of international recognition that was
owed during her lifetime, and whose legacy reflects both her own
contributions as an artist as well as the singular path she forged for
herself as the child of immigrants, a woman and an Asian-American. This
past April, the United States Postal Service
\href{https://ruthasawa.com/usps-announces-2020-ruth-asawa-stamp/}{announced}
that 10 different works of Asawa's would be featured on a series of
postage stamps, out next month. Also in April, the first comprehensive
biography of Asawa,
``\href{https://www.chroniclebooks.com/products/everything-she-touched}{Everything
She Touched}'' by \href{https://www.marilynchase.com/}{Marilyn Chase},
was published by Chronicle Books. She is now routinely included in
comprehensive group shows alongside artists such as
\href{https://www.nytimes3xbfgragh.onion/1994/05/10/obituaries/anni-albers-94-textile-artist-and-the-widow-of-josef-albers.html}{Anni
Albers},
\href{https://www.nytimes3xbfgragh.onion/2017/05/03/arts/design/sheila-hicks-work-on-the-high-line.html}{Sheila
Hicks} and Bourgeois. Laib, who took the original call from Asawa's
daughter, eventually moved from Christie's to the David Zwirner gallery
and is responsible for several lauded solo shows of her work, resulting
in sales of her sculptures for well over a million dollars.

Image

The artist with one of her hanging looped-wire sculptures in 1957.
Photograph by Imogen Cunningham.Credit...Imogen Cunningham. © The Imogen
Cunningham Trust

In a culture of acknowledging those who were previously overlooked, when
artists and their earliest champions are finally getting their dues,
there is a satisfaction in witnessing the record be corrected. Yet a
purely revisionist approach **** ignores the ways in which Asawa's art
is still remarkably contemporary, how it is a clear articulation of
midcentury art's engagement with spatial abstraction. I have stood in a
gallery hung with Asawa's wire sculptures, where the movement of my own
body has caused them to sway, the shadows of the woven wire dancing
against the floor. For a moment, I was quietly transported elsewhere ---
to the deep sea, to a forest or maybe to someplace altogether unearthly.

In interviews, Asawa chose her words carefully. I suspect she would have
resisted ever being portrayed as a victim. But there are the plain facts
of her existence --- that she was incarcerated as a teenager in a
Japanese-American concentration camp; that she overcame incredible
prejudice and racism to be an artist. How much is different from today,
as people of Asian descent **** encounter
\href{https://www.nytimes3xbfgragh.onion/2020/06/24/us/asian-american-racism-coronavirus-kelly-yang.html}{new
levels of racism}, as the federal government continues to unjustly
\href{https://www.nytimes3xbfgragh.onion/topic/subject/immigration-detention}{detain
immigrants} based on where they are from? Asawa's biography is,
ultimately, one with a happy ending. But it is also a painful reminder
that the struggles she faced are not novel, and that history repeats
itself. What, exactly, can we learn from her life?

ASAWA WAS BORN in 1926 in Norwalk, Calif. Her father, Umakichi, had
worked as a tofu vendor, leaving Japan in 1902 to avoid conscription in
the Russo-Japanese War. Her mother, Haru, was a Japanese picture bride
--- one of the thousands of Japanese women who, at the beginning of the
last century, agreed, through the exchange of black-and-white portraits,
to marry a Japanese man living in the United States in the hopes of a
better life. By the time Ruth was born, the family was leasing an
80-acre farm in what would later become greater Los Angeles, unable to
own property as immigrants because of the California Alien Land Law of
1913. Eventually, the Asawa family would grow to include seven children.
Ruth was the fourth oldest.

Image

Asawa's mother, Haru (center), with her sister Ura (left) and their
mother in Japan.Credit...Courtesy of the Estate of Ruth Asawa. From
``Everything She Touched: The Life of Ruth Asawa,'' by Marilyn Chase,
published by Chronicle Books, 2020

Image

Asawa at age 13.Credit...Courtesy of the Estate of Ruth Asawa and David
Zwirner

Life on the farm was tough and unsparing, with long days and little time
for idleness. The family lived in a board-and-batten house, covered by a
paper ceiling and a tin roof, that Umakichi built himself. Asawa's
mother, as Chase describes in ``Everything She Touched,'' woke at around
3 a.m. each day to begin cooking the family's rice; her father rose an
hour later to check the gopher traps. Onions, broccoli and cauliflower
were harvested every winter, strawberries every spring, and tomatoes and
melons in the summer. They recycled the wooden crates down to the nails,
which Umakichi would re-flatten with a hammer. ``In my home we had
virtually no materials,'' Asawa said in a 1981 interview, ``just a set
of encyclopedia and a player piano. All the children wanted to play
music but we didn't have any money for lessons.'' Umakichi was a truck
farmer, and after the farm's produce had been packed, he would drive to
the Los Angeles farmer's market to sell it. In the economy of the 1930s,
a box of tomatoes cost a nickel, two dozen melons cost 10 cents and a
crate of cabbages cost 35. Umakichi was often ripped off by buyers,
which Asawa credited to the family's own naïveté. As a result, school
was a welcome haven for Asawa --- even then, she loved to draw --- but
the children were still expected to finish their chores.

In December 1941, when Asawa was 15 years old, the Japanese bombed Pearl
Harbor, prompting the United States to declare war on Japan. Afraid of
what could be construed as evidence against them, Ruth's father burned
the ikebana books one of her older siblings had brought back from a trip
to Japan. In February, President Franklin D. Roosevelt signed Executive
Order 9066, which would result in 120,000 men, women and children of
Japanese ancestry being evicted from across the West Coast and held in
American concentration camps scattered throughout the country. Umakichi
was arrested that same month: ``It was a Sunday, I guess, in February,''
Asawa recalled in an interview with the University of California,
Berkeley, in the mid-1970s, ``that we were working in the field and two
FBI men came. They went and found my father in the field and marched him
back into the house. He had lunch and then they took him away.''

Image

The identification card issued to Asawa by the War Relocation Authority,
the main government agency created to oversee the incarceration of
Japanese-Americans during World War II.Credit...National Portrait
Gallery, Smithsonian Institution. Gift of the children of Ruth Asawa.
Courtesy of the Estate of Ruth Asawa and David Zwirner

By April of that year, with Umakichi already imprisoned in New Mexico,
Asawa, her mother and her siblings --- with the exception of a younger
sister who had been living in Japan on an extended visit, where she
would remain throughout the war --- had been told to pack up their lives
and join the thousands of other Japanese-Americans at Santa Anita, one
of two local detention centers, where they were assigned to wait until
they received a permanent camp location further inland. There, they
lived in the stables of the converted racetrack. ``Hair from the
horse{[}'s{]} mane \& tail were stuck between cracks of the walls. The
heat of the summer accentuated the odor of recent tenants,'' remembered
Asawa. Family units were broken up. Privacy was limited. While
incarceration was an undignified experience, it also, paradoxically, set
in motion Asawa's career. She had more free time in Santa Anita than on
the family farm and was introduced to three Walt Disney artists --- Tom
Okamoto, Chris Ishii and James Tanaka --- who had begun teaching art.
With the paper, charcoal and ink donated by the same men who had worked
on ``Snow White'' and ``Pinocchio,'' Asawa began to take her own talent
more seriously.

Image

Asawa standing before trellised beans at the Rohwer War Relocation
Center in Arkansas in 1943.Credit...Courtesy of the Estate of Ruth Asawa
and David Zwirner

After five months, the Asawa family was ordered to pack up again,
heading now to the bayous of Arkansas and the Rohwer War Relocation
Center, where Asawa would live for the next year. She recalled of the
journey there: ``The Louisiana swamps were just as I imagined them to be
\ldots{} enchanting, beautiful, and weird.'' Cypress trees grew in the
bayous and creeks snaked through large swaths of farmland, which were
worked by sharecroppers, whose own poverty was often bleaker than that
of those in the camps. In freshly erected barracks where the soil turned
to black muck when it rained, Asawa and her family were imprisoned with
over 8,000 other Japanese-Americans. (I visited Rohwer earlier this year
and I was shocked by its flatness and disappointed that almost nothing
remained of the camps except a smokestack, a gymnasium that now sat on
private property and the beautiful cement tombstones that
Japanese-American prisoners made for themselves.)

Like nearly everyone around them, the Asawas had lost their way of life
and their security for the future. Still, there were small moments of
relief: The gardens the families planted thrived in the Arkansas soil.
They flew paper kites against the open blue sky. Asawa's mother got her
hair permed for the first time and socialized with the other women at
the camp, activities her hardscrabble farm life never allowed.

In the spring of 1943, Asawa became eligible for early release as a
high-school graduate (on the condition that she attend a college in the
country's interior, which was considered less of a national security
threat, and that she find a financial sponsor). One of her teachers
handed her a catalog for the Art Institute of Chicago. She couldn't
afford it and instead chose the Milwaukee State Teachers College, where
a semester only cost \$25 (roughly \$360 today). Leaving behind her
mother and her younger siblings, Asawa said goodbye to Rohwer and took a
train north.

FREE FROM THE prison of Rohwer, Asawa found Milwaukee was still a
disappointment in many respects. Her tuition was paid for by a Quaker
scholarship, but she earned her living expenses working as a live-in
maid for a local family. During her third year of study, with the modest
aim of becoming an art teacher, Asawa was told her race was a liability
--- as a Japanese-American, she would not be able to graduate with a
teaching certificate, and without that, she would be unable to be hired
as a teacher. Two of her friends from Milwaukee, both artists,
\href{https://www.nytimes3xbfgragh.onion/2015/01/11/arts/design/ray-johnson-defies-categories-20-years-after-his-death.html}{Ray
Johnson} and
\href{http://www.blackmountaincollege.org/elaine-schmitt-urbain/}{Elaine
Schmitt}, were planning to attend a summer course at a school called
Black Mountain College and urged Asawa to join them. After first
arriving for a summer session, Asawa finally enrolled as a full-time
student in the fall of 1946.

Image

Asawa at Black Mountain College in North Carolina, where she first
enrolled as a student in 1946, staying for three years.Credit...Hazel
Larsen Archer. © Estate of Hazel Larsen Archer, by permission of Erika
Archer-Zarow. From ``Everything She Touched: The Life of Ruth Asawa,''
by Marilyn Chase, published by Chronicle Books, 2020

Situated along rolling meadowlands of the Great Craggy Mountains of
North Carolina, Black Mountain College was a creative paradise whose
pedagogical practices would go on to influence America's liberal arts
education in every way imaginable --- even if it was a relatively
short-lived experiment, dissolving in 1957. It was radically
open-minded, a place for personal and creative discovery that couldn't
have been any more different from the Teachers College in Milwaukee.
Founded in 1933 by a pedagogue named John A. Rice, the school embraced a
holistic, interdisciplinary curriculum. Students weren't given grades
and could choose when to graduate. The art program was run by Josef
Albers, who had fled Hitler's Germany that same year with his wife and
fellow artist, Anni Albers. The couple had fallen in love at the Bauhaus
school, where Anni had been a precocious weaving student and Josef a
teacher. The
\href{https://www.nytimes3xbfgragh.onion/2019/02/04/t-magazine/bauhaus-school-architecture-history.html}{Bauhaus}
itself was a radical moment in German design, combining fine arts with
crafts and emphasizing a more democratic relationship between
practicality and aesthetics. Black Mountain, as a result, was a rare
amalgamation of European modernism, American individualism and of
Albers's old-world rigor. It also possessed an undeniably romantic
atmosphere, in which students and teachers were equals, eating, living
and socializing together. In a letter to
\href{https://www.nytimes3xbfgragh.onion/2011/10/23/arts/artsspecial/kandinsky-painting-reveals-a-mystery-beneath.html}{Wassily
Kandinsky} upon their arrival, the Albers wrote: ``Black Mountain is
wonderful, deep in the mountains, the same height as the Harz {[}a
mountain range in northern Germany{]} I think, but everything lush. The
woods are full of wild rhododendrons, as big as trees, we go out without
coats and sat **** in spring sunshine this morning.''

Image

Josef Albers teaching at Black Mountain College.Credit...Courtesy of the
Estate of Hazel Larsen Archer and the Black Mountain College Museum +
Arts Center. From ``Everything She Touched: The Life of Ruth Asawa,'' by
Marilyn Chase, published by Chronicle Books, 2020

Image

Asawa and a fellow student, Ora Williams, at Black Mountain College in
the summer of 1946.Credit...Mary Parks Washington, courtesy of the
Estate of Ruth Asawa. From ``Everything She Touched: The Life of Ruth
Asawa,'' by Marilyn Chase, published by Chronicle Books, 2020

It is a common assumption, given the tactile **** nature of Asawa's wire
sculptures, that she studied weaving with Anni Albers. Anni, in fact,
initially rejected Asawa, telling her it was impossible to teach a
summer student weaving in just six weeks. Instead, it was Josef Albers
who had an outsize influence on Asawa --- his economical drawing classes
**** helped discipline her mind and her hands. And the couple's courage
and tolerance --- Josef, wary of elitism, came from a working-class
family in a coal-mining town in West Germany; Anni was of Jewish descent
(a good friend and colleague of the couple's from the Bauhaus, Otti
Berger, was killed at Auschwitz) --- was a ballast for someone whose
life was so marked by prejudice. (After Black Mountain admitted its
first Black students in 1944, Albers suggested the school should admit
more Asian students as well.) In many respects, Black Mountain was a
place where sexuality, race and gender were treated with a startling
impartiality for the times. Asawa blossomed under Albers's tutelage. He
showed people how to see, she later explained. To think critically and
creatively. To use humble materials. For the first time in her life,
Asawa finally saw herself as an \emph{artist}.

Image

The artist forming a looped-wire sculpture in 1957. Photograph by Imogen
Cunningham.Credit...Imogen Cunningham; © The Imogen Cunningham Trust.
From ``Everything She Touched: The Life of Ruth Asawa,'' by Marilyn
Chase, published by Chronicle Books, 2020

THE SUCCESS ASAWA achieved in her lifetime was not unremarkable. Had she
remained with Peridot or some other New York City gallery, there is
reason to believe that she would have risen alongside her
contemporaries. Peridot, which opened in 1948, was a small, successful
Upper East Side gallery run by Lou Pollack, who, as Hilton Kramer
\href{https://www.nytimes3xbfgragh.onion/1970/12/06/archives/an-ambience-all-too-rare.html}{wrote}
in his obituary in 1970, was a ``sweet, soft‐spoken, courageous man''
(Pollack died suddenly while vacationing in Corsica, after which the
gallery changed name and ownership). Kramer described Pollack as having
``a firm sense of that other art world, light-years removed from the
hucksterism and fashion-mongering that make the headlines and collect
the (blue) chips, where the aesthetic transaction exists primarily as a
private pleasure and a spiritual need.'' Other artists on Peridot's
roster included
\href{https://www.nytimes3xbfgragh.onion/2017/05/16/t-magazine/art/philip-guston-venice.html}{Philip
Guston}, Bourgeois,
\href{https://www.nytimes3xbfgragh.onion/1988/02/27/obituaries/james-rosati-76-sculptor-noted-for-monumental-works-is-dead.html}{James
Rosati} and
\href{https://www.nytimes3xbfgragh.onion/1988/05/07/obituaries/costantino-nivola-76-a-sculptor-of-public-and-small-scale-works.html}{Costantino
Nivola}. The partnership between Asawa and Pollack eventually ended, in
part because Asawa found it too costly to ship her wire sculptures
across the country, especially as she had to assume the financial burden
for any damages. Peridot's ceilings were also only eight feet high ---
too short for her more ambitious and larger works.

But with hindsight, it is easy to see how Asawa was dismissed. Vogue
magazine featured her artwork in 1952 alongside fashion models, who
posed in front of the sculptures as if they were accessories. A positive
1955 review of two separate exhibitions by Asawa and
\href{https://www.nytimes3xbfgragh.onion/2020/02/25/arts/design/isamu-noguchi-midtown-installation.html}{Isamu
Noguchi} in Time magazine referred to Noguchi as a ``leading U.S.
sculptor'' and Asawa as ``a housewife.'' Orientalism, too, infused the
language around Asawa's work --- it wasn't uncommon for an article about
her to make reference to ``ancient'' traditions or her ``far Eastern
patience,'' ignoring the distinctly European influence of Albers as well
as Asawa's own American origins. Her sculptures, made of wire and by
hand, were also often labeled ``craft,'' a term that today may carry
more positive associations but was still limiting for a woman moving in
the same circles as Abstract Expressionists, postmodernists and
conceptualists.

Image

Asawa at work in her home studio, surrounded by her children, in 1957.
Photograph by Imogen Cunningham.Credit...Imogen Cunningham; © The Imogen
Cunningham Trust. From ``Everything She Touched: The Life of Ruth
Asawa,'' by Marilyn Chase, published by Chronicle Books, 2020

Asawa, who eventually became a mother of six, didn't neatly fit into the
categories that then defined the politics of feminism. An Asian-American
woman, married with children, **** was never going to be seen as defying
the patriarchy --- even if her own interracial marriage was illegal in
many states when she wed in 1949. As the artist
\href{https://www.nytimes3xbfgragh.onion/2019/11/19/t-magazine/suzanne-jackson-artist.html}{Suzanne
Jackson}, who became a friend of Asawa's while serving on the California
Arts Council in the 1970s, explained: ``For some of us, there was a kind
of cultural attitude expressing --- there were no bras to take off. No
pedestals to fall from. No privilege to abandon.'' Nevertheless, after
Asawa's children entered the California school system, beginning in
1968, she turned to activism and teaching. She garnered a prestigious
profile as an educator and advocate for San Francisco's public schools,
bringing her other Black Mountain mentor, Buckminster Fuller, and his
environmental thinking to classrooms. All of this meant time outside of
the studio.

Most crucially though, there was no lexicon to explain or understand
Asawa's own trajectory from a dusty farm of Norwalk to being
incarcerated during World War II to being in the same room as near
mythological figures such as
\href{https://www.nytimes3xbfgragh.onion/2015/06/03/t-magazine/robert-rauschenberg-endless-combinations.html}{Robert
Rauschenberg},
\href{https://www.nytimes3xbfgragh.onion/2017/02/07/t-magazine/art/merce-cunningham-exhibit-walker-art-center.html}{Merce
Cunningham} and
\href{https://www.nytimes3xbfgragh.onion/2016/11/15/t-magazine/art/willem-de-kooning-zao-wou-ki-two-men-show.html}{Willem
de Kooning}. For six weeks in 1948, while still a scholarship student at
Black Mountain, Asawa rejoined her parents near Los Angeles to help them
rebuild their lives as farmers after the war. Her own sense of
responsibility to her family contradicted the notion of the selfish
artist so espoused by her peers. And Asawa was not one to highlight her
own experience with injustice to score points. In what remains of her
several applications for a Guggenheim fellowship throughout the 1950s,
for which she was repeatedly rejected (she continued to apply through
the 1990s), Asawa never once mentions her own incarceration.

Image

Asawa with masks she liked to make of friends and visitors who came to
her house and studio in the Noe Valley neighborhood of San Francisco,
1988.Credit...© Terry Schmitt. From "Everything She Touched: The Life of
Ruth Asawa" by Marilyn Chase, published by Chronicle Books, 2020

GROWING UP IN the Bay Area, I was familiar with Asawa's work before I
knew her name --- my parents liked to take me to the Ghirardelli
chocolate factory in San Francisco, where her second public commission,
a bronze fountain featuring two mermaids, one of whom is breastfeeding
(1968), still stands. Whatever divide I had to mentally cross to
understand that this artist was the same one who deserved to stand
alongside others such as
\href{https://www.nytimes3xbfgragh.onion/topic/person/frida-kahlo}{Frida
Kahlo} and Bourgeois happened much later, when --- as art has the
capacity to do --- I was struck by the acute simplicity of an Asawa
sculpture (``\href{https://whitney.org/collection/works/2260}{S.270},''
1955) **** hanging in the window of the
\href{https://www.nytimes3xbfgragh.onion/topic/organization/whitney-museum-of-american-art}{Whitney
Museum of American Art} in 2015, the West Village cramped and alive
behind me, the patina of the sculpture's wire evocative of a time now
lost. It is unfortunate to me that women who enter the pantheon of great
artists are often close to dead or, like Bourgeois, old enough that they
seem to be eclipsed by their own careers --- so that their story of
genius is always one of overcoming, of wise, womanly perseverance. I am
reluctant to see Asawa as anything more than what she was: a remarkable
individual with a story that is so American in its triumph against
adversity that it's impossible to imagine it going another direction, as
it did with thousands of Japanese-Americans of her generation who were
promised a better life, as it did with her parents, who were forced to
start over, who never fully regained what they painstakingly built for
themselves as immigrants.

Image

A detail from Asawa's 1994 Japanese American Internment Memorial, in San
Jose, Calif.Credit...© Terry Schmitt. From ``Everything She Touched: The
Life of Ruth Asawa,'' by Marilyn Chase, published by Chronicle Books,
2020

Only much later in Asawa's life, when she was in her 60s, did she
confront her experience in the camps with a 1994 commissioned bronze
bas-relief memorial for the city of San Jose. In it, she fastidiously
depicts scenes of Santa Anita and Rohwer --- as well as those of the
imprisonment of the larger Japanese-American community and of their
struggle for justice afterward. Its literalness is uncharacteristic of
her more abstract work. But the piece was also a reflection **** of the
larger gestures she had begun to make as an educator and an activist,
actions that finally addressed, as directly as possible, not just her
own experience as a teenager but what happened, on a whole, to three
generations of Japanese-Americans. In a letter to a friend written not
long after the memorial's unveiling, she explained: ``I had to dig deep
into my past to find the common threads with other Japanese immigrants
who endured the struggle and am glad I was part of it.'' The exhumation
of her own experience was necessary. Even if Asawa always maintained
that art --- and its ability to offer us a way to think critically about
the world --- was what actually saved her. In a 1980 interview, she put
it as such:

\begin{quote}
Well I don't think that art by itself is important. I think that the
reason the arts are important is because it is the only thing that an
individual can do and maintain his individuality. I think that is very
important --- making your own decisions. If you count everything that we
fight for --- better schools, better health care, more social awareness
--- we are letting other people make the decisions for us. We are not
taking our lives into our own hands and making those decisions for
ourselves.
\end{quote}

Image

An undated photo of Asawa by Imogen Cunningham.Credit...Imogen
Cunningham. © The Imogen Cunningham Trust

This past December, I visited the Noe Valley house where Asawa died in
2013. Perched on a slanted hill, the house could be what Donald Judd's
former home in SoHo is today, a preservation of the artist's space that
is so complete that it is nearly a work of art itself. Judd's home
underwent a costly rehabilitation, and it requires a professional staff
to maintain. For now, Asawa's youngest son, Paul, and his wife, Sandra,
live there with their children. The front doors, six hulking slabs of
redwood, were hand-carved and burnished by Asawa with the help of her
husband and children. The cobblestones that pave the pathway through the
front garden were carried up by the family from a nearby beach. A ume
plum tree that Asawa planted still stands in a the verdant garden, now
overgrown with oxalis and nasturtium.

Image

Sculptures by Asawa hanging from the Douglas fir rafters in the living
room of her Noe Valley home in 1995.Credit...Laurence Cuneo, courtesy of
the Estate of Ruth Asawa and David Zwirner. Artwork © Estate of Ruth
Asawa

San Francisco is a city of heights and fog and light --- crossing a
street can sometimes feel like stepping from darkness into pure blue
sky. Standing in her living room, flooded by the midday sun, the city
unspooling below, I was able to conjure a black-and-white 1995
photograph that depicted how Asawa's most important sculptures were hung
in her home (many have since been placed in prominent museums and
collections). Her children told me anecdotes, collected over the years:
Asawa was fond of pointing at her sculptures, constellations in her own
universe, and remarking that ``this one is a seminal piece,'' ``that one
should go to a museum.'' ``Somehow,'' Addie Lanier told me, ``She knew
that the works would get there.'' The children are now middle-aged and
parents themselves; they **** had devoted years to caring for their
parents at the end of their lives. I sensed how overwhelmed they had
been by what had been left behind --- they told stories of uncovering
lesser artwork stuffed into **** basement crawl spaces, of painstakingly
cataloging scores of photographs of their mother by her friend the
photographer
\href{https://www.nytimes3xbfgragh.onion/1973/05/06/archives/imogen-cunningham-at-ninety-a-remarkable-empathy.html}{Imogen
Cunningham} that had never been published. It made me think of my own
parents, of the duty a child feels to her elders, of the abundance of
life and then the quiet that comes after death. Their mother, they said,
used to hang her feet off the side of her father's horse-drawn leveler,
creating undulating patterns in the dirt that would eventually be
repeated in her work. They were protective of their mother's legacy.
They understood what was at stake as custodians. They told me how, after
she had taken a class with Albers, their mother told him she didn't want
to paint what he wanted her to paint. She wanted to paint flowers
instead. ``Fine,'' Albers had replied. ``But make them Asawa flowers.''
The clarity of her own existence was obvious.

\hypertarget{true-believers-art-issue}{%
\subsubsection{\texorpdfstring{\href{https://www.nytimes3xbfgragh.onion/issue/t-magazine/2020/07/02/true-believers-art-issue}{True
Believers Art
Issue}}{True Believers Art Issue}}\label{true-believers-art-issue}}

Advertisement

\protect\hyperlink{after-bottom}{Continue reading the main story}

\hypertarget{site-index}{%
\subsection{Site Index}\label{site-index}}

\hypertarget{site-information-navigation}{%
\subsection{Site Information
Navigation}\label{site-information-navigation}}

\begin{itemize}
\tightlist
\item
  \href{https://help.nytimes3xbfgragh.onion/hc/en-us/articles/115014792127-Copyright-notice}{©~2020~The
  New York Times Company}
\end{itemize}

\begin{itemize}
\tightlist
\item
  \href{https://www.nytco.com/}{NYTCo}
\item
  \href{https://help.nytimes3xbfgragh.onion/hc/en-us/articles/115015385887-Contact-Us}{Contact
  Us}
\item
  \href{https://www.nytco.com/careers/}{Work with us}
\item
  \href{https://nytmediakit.com/}{Advertise}
\item
  \href{http://www.tbrandstudio.com/}{T Brand Studio}
\item
  \href{https://www.nytimes3xbfgragh.onion/privacy/cookie-policy\#how-do-i-manage-trackers}{Your
  Ad Choices}
\item
  \href{https://www.nytimes3xbfgragh.onion/privacy}{Privacy}
\item
  \href{https://help.nytimes3xbfgragh.onion/hc/en-us/articles/115014893428-Terms-of-service}{Terms
  of Service}
\item
  \href{https://help.nytimes3xbfgragh.onion/hc/en-us/articles/115014893968-Terms-of-sale}{Terms
  of Sale}
\item
  \href{https://spiderbites.nytimes3xbfgragh.onion}{Site Map}
\item
  \href{https://help.nytimes3xbfgragh.onion/hc/en-us}{Help}
\item
  \href{https://www.nytimes3xbfgragh.onion/subscription?campaignId=37WXW}{Subscriptions}
\end{itemize}
