Sections

SEARCH

\protect\hyperlink{site-content}{Skip to
content}\protect\hyperlink{site-index}{Skip to site index}

\href{https://www.nytimes3xbfgragh.onion/section/politics}{Politics}

\href{https://myaccount.nytimes3xbfgragh.onion/auth/login?response_type=cookie\&client_id=vi}{}

\href{https://www.nytimes3xbfgragh.onion/section/todayspaper}{Today's
Paper}

\href{/section/politics}{Politics}\textbar{}Georgia Democrats Choose
Nikema Williams to Fill John Lewis's Seat in Congress

\url{https://nyti.ms/39fB1C7}

\begin{itemize}
\item
\item
\item
\item
\item
\end{itemize}

Advertisement

\protect\hyperlink{after-top}{Continue reading the main story}

Supported by

\protect\hyperlink{after-sponsor}{Continue reading the main story}

\hypertarget{georgia-democrats-choose-nikema-williams-to-fill-john-lewiss-seat-in-congress}{%
\section{Georgia Democrats Choose Nikema Williams to Fill John Lewis's
Seat in
Congress}\label{georgia-democrats-choose-nikema-williams-to-fill-john-lewiss-seat-in-congress}}

Ms. Williams, a state senator, is considered the overwhelming favorite
in the general election this fall.

\includegraphics{https://static01.graylady3jvrrxbe.onion/images/2020/07/20/us/politics/20lewis-successor00/merlin_174772956_eb5349b7-cff0-4c56-b829-a16b5be2ff18-articleLarge.jpg?quality=75\&auto=webp\&disable=upscale}

By \href{https://www.nytimes3xbfgragh.onion/by/rick-rojas}{Rick Rojas}

\begin{itemize}
\item
  July 20, 2020
\item
  \begin{itemize}
  \item
  \item
  \item
  \item
  \item
  \end{itemize}
\end{itemize}

ATLANTA --- Nikema Williams, a state senator from Atlanta, emerged from
a conclave of Georgia Democratic officials on Monday as the appointed
successor to Representative John Lewis, taking his place on the November
ballot and very likely inheriting the seat he occupied in Congress for
17 terms.

The decision came less than 72 hours after Mr. Lewis's death, as much of
Atlanta remained in grief over the loss of the civil rights leader who
had represented the city for over 30 years, and as his colleagues in
Congress held a moment of silence on Monday in his memory.

Democratic officials said they had to rush because of a deadline imposed
by state law that required a name be put forward by Monday afternoon.
Party leaders said they did not want to risk losing a seat that they
have traditionally held in a tight grip, one representing a swath of
Atlanta and adjoining suburbs.

Ms. Williams, who is also the chairwoman of the state party, was one of
five finalists up for consideration, with that group winnowed from a
pool of dozens who submitted applications over the weekend. She was
selected by the party's executive committee of 44 members during a
virtual meeting on Monday. (Ms. Williams abstained from any votes
related to the nomination.)

``It's surreal that we're forced to endure this nomination process while
still grieving,'' she said in her speech before the vote, noting that
she considered Mr. Lewis to be a friend and mentor.

Ms. Williams and her supporters sought to cast her as the person best
suited to carrying on Mr. Lewis's legacy, saying that she would advance
to Congress with the zeal fueled by years of activism and practical
legislative experience. She noted another parallel: She and Mr. Lewis
both grew up in rural Alabama, where she was raised by her grandparents
in Smiths Station, a small town just over the Georgia state line.

``Nobody could possibly fill the shoes of Congressman Lewis, but his
leadership and fighting spirit is needed now more than ever in this
country,'' Ms. Williams said. ``We need someone who's not afraid to put
themselves on the line for their constituents in the same way
Congressman Lewis taught us to. I would be honored if you chose me to be
that person.''

The choice fell to the party leadership because Georgia's primary
elections were held last month. The seat will probably remain vacant
until the winner of the November election is inaugurated in January.
Gov. Brian Kemp, a Republican, has 10 days following Mr. Lewis's death
to call a special election; the governor has not announced a decision.

Critics within the Democratic Party, including Barrington D. Martin II,
who had mounted a failed primary challenge to Mr. Lewis in June,
assailed a closed process that allowed party leaders instead of voters
to decide the nominee. Some had
called\href{https://www.ajc.com/politics/politics-blog/georgia-democrats-to-decide-today-who-will-replace-john-lewis-on-the-ballot/EZ33XDYU2FA43IGL2NBOE4EI3Q/}{for
the committee to nominate a place holder} who would resign the seat upon
being sworn in, giving way to a special election in 2021, an idea that
was ultimately rejected.

``Everyone is on notice from here on out,'' Mr. Martin said in a post on
Twitter. ``It's disgraceful to think that Congressman Lewis wanted the
people to be heard and today they were not heard at all.''

Scott Hogan, the executive director of the state party, argued that they
were hamstrung by the law and were forced to rely upon a ``system that
falls short of a full districtwide election to ensure that we have a
strong Democratic nominee.''

``While this system was not perfect and we were forced to use what the
secretary of state and Georgia law demands,'' he added, ``we know that
we have the absolute best candidate in Nikema Williams who will fight
hard for Georgians.''

\includegraphics{https://static01.graylady3jvrrxbe.onion/images/2020/07/20/us/politics/20lewis-successor/merlin_166155114_3d489cff-d8ea-4e28-ad9b-9e5ca3d1191b-articleLarge.jpg?quality=75\&auto=webp\&disable=upscale}

The party put out a call for applicants on Saturday, just hours after it
was announced on Friday that Mr. Lewis died at the age of 80. Officials
said 131 people raised their hands to run to represent the district,
which covers parts of Atlanta and seeps into suburban DeKalb and Clayton
Counties.

Being the Democratic candidate in Georgia's Fifth District is tantamount
to a ticket to Congress: Mr. Lewis won with at least 70 percent of the
vote in all but one of his re-election bids; Hillary Clinton won with 85
percent of the vote against
\href{https://www.nytimes3xbfgragh.onion/interactive/2020/us/elections/donald-trump.html}{President
Trump} in the district during the 2016 presidential race.

The process was overseen by a committee that includes some of the
state's most prominent Democrats, including Keisha Lance Bottoms, the
Atlanta mayor who catapulted to national prominence in recent months
after her response to unrest in the city and clashes with Governor Kemp
over coronavirus precautions, and Stacey Abrams, who narrowly lost the
race for governor in 2018 to Mr. Kemp.

The other finalists selected by the committee were Park Cannon, a state
representative
\href{https://www.cnn.com/2016/03/29/us/park-cannon-georgia-house-feat/index.html}{who
became the youngest member of the Georgia General Assembly} when she was
sworn in at the age of 24 in 2016;
\href{https://saportareport.com/a-conversation-with-atlanta-city-councilman-andre-dickens/}{Andre
Dickens}, who serves on the Atlanta City Council;
\href{https://www.morehouse.edu/about/presidentsbio.html}{Robert M.
Franklin Jr.}, a scholar of theology and a former president of Morehouse
College in Atlanta; and James Woodall, the president of the Georgia
N.A.A.C.P. and, at 26,
\href{https://www.thecrisismagazine.com/single-post/2019/12/09/James-Woodall-Makes-History-as-Youngest-NAACP-State-Conference-President}{one
of the youngest leaders} in the organization.

Ms. Williams, 41, was elected in 2017 to a State Senate district that
overlaps in large part with the Fifth Congressional District that Mr.
Lewis represented.

She
\href{https://www.motherjones.com/politics/2018/11/georgia-state-senator-nikema-williams-arrest-1/}{gained
some national attention in 2018} when she was arrested in front of a
pack of cameras during demonstrations at the State Capitol over the
midterm elections. ``I was not yelling,'' she told reporters with her
hands behind her back and state troopers tightening zip ties around her
wrists. ``I was not chanting. I was standing peacefully next to my
constituents.''

Outside of the legislature, Ms. Williams was deputy political director
for the National Domestic Workers Alliance, an advocacy group
representing nannies, house cleaners and other care workers. Before
that, she had been the vice president of public policy for Planned
Parenthood Southeast.

Ms. Williams will face what is regarded as a long shot bid from the
Republican nominee, Angela Stanton-King, a reality-television
personality who was convicted of her role in a stolen-vehicle ring. Mr.
Trump pardoned her in February. In a tweet on Monday afternoon, Ms.
Stanton-King pointed out Ms. Williams' ties to Planned Parenthood and
underscored her opposition to abortion rights.

During the virtual meeting of Democratic leaders, many acknowledged a
measure of awkwardness as a series of candidates gave speeches, actively
campaigning for Mr. Lewis's job so soon after his death. But in his
speech, Mr. Franklin, the former Morehouse president, noted the gravity
of the moment.

``History has its eyes on you,'' he said, invoking a song from the
musical ``Hamilton.'' ``How do you follow the extraordinary legacy and
the impact of Congressman John Robert Lewis?''

Still, he added, ``You can't make a bad decision today.''

Reid J. Epstein contributed reporting from Timber Ridge, Va.

Advertisement

\protect\hyperlink{after-bottom}{Continue reading the main story}

\hypertarget{site-index}{%
\subsection{Site Index}\label{site-index}}

\hypertarget{site-information-navigation}{%
\subsection{Site Information
Navigation}\label{site-information-navigation}}

\begin{itemize}
\tightlist
\item
  \href{https://help.nytimes3xbfgragh.onion/hc/en-us/articles/115014792127-Copyright-notice}{©~2020~The
  New York Times Company}
\end{itemize}

\begin{itemize}
\tightlist
\item
  \href{https://www.nytco.com/}{NYTCo}
\item
  \href{https://help.nytimes3xbfgragh.onion/hc/en-us/articles/115015385887-Contact-Us}{Contact
  Us}
\item
  \href{https://www.nytco.com/careers/}{Work with us}
\item
  \href{https://nytmediakit.com/}{Advertise}
\item
  \href{http://www.tbrandstudio.com/}{T Brand Studio}
\item
  \href{https://www.nytimes3xbfgragh.onion/privacy/cookie-policy\#how-do-i-manage-trackers}{Your
  Ad Choices}
\item
  \href{https://www.nytimes3xbfgragh.onion/privacy}{Privacy}
\item
  \href{https://help.nytimes3xbfgragh.onion/hc/en-us/articles/115014893428-Terms-of-service}{Terms
  of Service}
\item
  \href{https://help.nytimes3xbfgragh.onion/hc/en-us/articles/115014893968-Terms-of-sale}{Terms
  of Sale}
\item
  \href{https://spiderbites.nytimes3xbfgragh.onion}{Site Map}
\item
  \href{https://help.nytimes3xbfgragh.onion/hc/en-us}{Help}
\item
  \href{https://www.nytimes3xbfgragh.onion/subscription?campaignId=37WXW}{Subscriptions}
\end{itemize}
