Sections

SEARCH

\protect\hyperlink{site-content}{Skip to
content}\protect\hyperlink{site-index}{Skip to site index}

\href{https://myaccount.nytimes3xbfgragh.onion/auth/login?response_type=cookie\&client_id=vi}{}

\href{https://www.nytimes3xbfgragh.onion/section/todayspaper}{Today's
Paper}

\href{/section/opinion}{Opinion}\textbar{}Would the Prophet Muhammad
Convert Hagia Sophia?

\begin{itemize}
\item
\item
\item
\item
\item
\end{itemize}

Advertisement

\protect\hyperlink{after-top}{Continue reading the main story}

\href{/section/opinion}{Opinion}

Supported by

\protect\hyperlink{after-sponsor}{Continue reading the main story}

Opinion

\hypertarget{would-the-prophet-muhammad-convert-hagia-sophia}{%
\section{Would the Prophet Muhammad Convert Hagia
Sophia?}\label{would-the-prophet-muhammad-convert-hagia-sophia}}

Turkey's decision to change the former cathedral into a mosque flies
against the pluralist instincts of Islam's founders.

\href{https://topics.nytimes3xbfgragh.onion/top/reference/timestopics/people/a/mustafa_akyol/index.html}{\includegraphics{https://static01.graylady3jvrrxbe.onion/images/2013/10/04/opinion/Akyol-contributor/Akyol-contributor-thumbLarge-v3.jpg}}

By
\href{https://topics.nytimes3xbfgragh.onion/top/reference/timestopics/people/a/mustafa_akyol/index.html}{Mustafa
Akyol}

Mr. Akyol is a contributing Op-Ed writer.

\begin{itemize}
\item
  July 20, 2020
\item
  \begin{itemize}
  \item
  \item
  \item
  \item
  \item
  \end{itemize}
\end{itemize}

\includegraphics{https://static01.graylady3jvrrxbe.onion/images/2020/07/20/opinion/20Akyol/20Akyol-articleLarge.jpg?quality=75\&auto=webp\&disable=upscale}

The recent decision by the Turkish government to reconvert the majestic
\href{https://www.nytimes3xbfgragh.onion/2020/07/24/world/europe/turkey-hagia-sophia-mosque-prayers.html}{Hagia
Sophia}, which was once the world's greatest cathedral, from a museum
back to a mosque has been bad news for Christians around the world. They
include Pope Francis,
\href{https://cruxnow.com/vatican/2020/07/pope-francis-pained-by-decision-to-turn-hagia-sophia-into-mosque/}{who
said he was ``pained''} by the move, and the spiritual leader of Eastern
Christianity, Ecumenical Patriarch Bartholomew,
\href{https://greekcitytimes.com/2020/06/25/ecumenical-patriarch-bartholomew-saddened-and-shaken-over-plans-to-convert-hagia-sophia-into-a-mosque/}{who
said he was ``saddened and shaken}.'' When contrasted with the joy of
Turkey's conservative Muslims, all this may seem like a new episode in
an old story: Islam vs. Christianity.

But some Muslims, including myself, are not fully comfortable with this
historic step, and for a good reason: forced conversion of shrines,
which has occurred too many times in human history in all directions,
can be questioned even from a purely Islamic point of view.

To see why, look closely into early Islam, which was born in seventh
century Arabia as a monotheist campaign against polytheism. The Prophet
Muhammad and his small group of believers saw the earlier monotheists
--- Jews and Christians --- as allies. So when those first Muslims were
persecuted in pagan Mecca, some found asylum in the Christian kingdom in
Ethiopia. Years later, when the Prophet ruled Medina, he welcomed a
group of Christians from the city of Najran to worship in his own
mosque. He also signed a
\href{https://www.amazon.com/Covenants-Prophet-Muhammad-Christians-World/dp/159731465X}{treaty}
with them, which read:

``There shall be no interference with the practice of their faith.
\ldots{} No bishop will be removed from his bishopric, no monk from his
monastery, no priest from his parish.''

This religious pluralism was also reflected in the Quran, when it said
God protects ``monasteries, churches, synagogues, and mosques in which
the name of God is much mentioned.'' (22:40) It is the only verse in the
Quran that mentions churches --- and only in a reverential tone.

To be sure, these theological affinities did not prevent political
conflicts. Nor did they prevent Muslims, right after the Prophet's
passing, from conquering Christian lands, from Syria to Spain. Yet
still, the early Muslim conquerors did something uncommon at the time:
They did not touch the shrines of the subjugated peoples.

The Prophet's spirit was best exemplified by his second successor, or
caliph, Umar ibn Al-Khattab, soon after his conquest of Jerusalem in the
year 637. The city, which had been ruled by Roman Christians for
centuries, had been taken by Muslims after a long and bloody siege.
Christians feared a massacre, but instead found \emph{aman}, or safety.
Caliph Umar, ``the servant of God'' and ``the commander of the
faithful,'' gave them security ``for their possessions, their churches
and crosses.'' He further
\href{https://www.amazon.com/dp/B004KKXO0O/ref=dp-kindle-redirect?_encoding=UTF8\&btkr=1}{assured}:

``Their churches shall not be taken for residence and shall not be
demolished \ldots{} nor shall their crosses be removed.''

The Christian historian Eutychius even
\href{https://books.google.com/books?id=nTjRzNwZEWAC\&pg=PA33\&lpg=PA33\&dq=Caliph+Umar+the+Patriarch+of+Jerusalem\&source=bl\&ots=Ysqo-GGhyo\&sig=ACfU3U1FWGR-RnwWq6jn9-7Kdrf7k7OZnw\&hl=en\&sa=X\&ved=2ahUKEwiE-fbJjs3qAhUzYTUKHbLUBFgQ6AEwEXoECAoQAQ\#v=onepage\&q=Caliph\%20Umar\%20the\%20Patriarch\%20of\%20Jerusalem\&f=false}{tells
us} that when Caliph Umar entered the city, the patriarch of Jerusalem,
Sophronius, invited him to pray at the holiest of all Christian shrines:
the Church of the Holy Sepulcher. Umar politely declined, saying that
Muslims might later take this as a reason to convert the church into a
mosque. He instead prayed at an empty area that Christians ignored but
Jews honored, then as now, as their holiest site, the Temple Mount,
where today the Western Wall, the last remnant of that ancient Jewish
temple, rises to the top of the Mount, on which the Mosque of Umar and
the Dome of the Rock were built.

In other words, Islam entered Jerusalem without really converting it.
Even ``four centuries after the Muslim conquest,'' as the Israeli
historian Oded Peri
\href{https://www.jstor.org/stable/3399441?seq=1\#metadata_info_tab_contents}{observes},
``the urban landscape of Jerusalem was still dominated by Christian
public and religious buildings.''

Yet Islam was becoming the religion of an empire, which, like all
empires, had to justify its appetite for hegemony. Soon, some jurists
found an excuse to overcome the Jerusalem model: There, Christians were
given full security, because they had ultimately agreed on a peaceful
surrender. The cities that resisted Muslim conquerors, however, were
fair game for plunder, enslavement, and conversion of their churches.

In the words of the Turkish scholar
\href{https://www.academia.edu/8126217/_Churches_and_Synagogues_in_Classical_Islamic_Law_Debates_on_Construction_Continuance_and_Repair_International_Conference_on_Religious_Tourism_and_Tolerance_9-12_May_2013_Konya_pp._353-360}{Necmeddin
Guney}, this legitimatization of conversion of churches came from not
the Quran nor the Prophetic example, but rather ``administrative
regulation.'' The jurists who made this case, he adds, ``were probably
trying to create a society that makes manifest the supremacy of Islam in
an age of religion wars.''

Another scholar, Fred Donner, an expert on early Islam,
\href{https://www.amazon.com/dp/1597404586/ref=dp-kindle-redirect?_encoding=UTF8\&btkr=1}{argues}
that this political drive even distorted records of the earlier state of
affairs. For example, later versions of the \emph{aman} given to the
Christians of Damascus allotted Muslims ``half of their homes and
churches.'' In the earlier version of the document, there was no such
clause.

When the Ottomans reached the gates of Constantinople in 1453, Islamic
attitudes had long been imperialized, and also toughened in the face of
endless conflicts with the Crusaders. Using a disputed license of the
Hanafi school of jurisprudence they followed, they converted Hagia
Sophia and a few other major churches. But they also did other things
that represent the better values of Islam: They gave full protection to
not only Greek but also Armenian Christians, rebuilt Istanbul as a
cosmopolitan city, and soon also welcomed the Spanish Jews who were
fleeing the Catholic Inquisition.

Today, centuries later, the question for Turkey is what aspect of this
complex Ottoman heritage is really more valuable.

For the religious conservatives who have rallied behind President Recep
Tayyip Erdogan in the past two decades, the main answer seems to be
imperial glory embodied in an absolute ruler.

For other Turks, however, the greatness of the Ottomans lies in their
pluralism, rooted at the very heart of Islam, and it would inspire
different moves today --- perhaps opening Hagia Sophia to both Muslim
\emph{and} Christian worship, as I have
\href{https://www.hurriyetdailynews.com/opinion/mustafa-akyol/hagia-sophia-could-be-a-mosquechurch-66074}{advised}
for years. Another would be reopening the
\href{https://www.nytimes3xbfgragh.onion/2019/02/06/world/europe/greece-tsipras-halki-seminary.html}{Halki
Seminary}, a Christian school of theology that opened in 1844 under
Ottoman auspices, went victim to secular nationalism in 1971, but is
still closed despite all the calls from advocates for religious freedom.

For the broader Muslim world, Hagia Sophia is a reminder that our
tradition includes both our everlasting faith and values, as well as a
legacy of imperialism. The latter is a bitter fact of history, like
Christian imperialism or nationalism, which have targeted our mosques
and even lives as well --- from Cordoba to Srebrenica. But today, we
should try to heal such wounds of the past, not open new ones.

So, if we Muslims really want to revive something from the past, let's
focus on the model initiated by the Prophet and implemented by Caliph
Umar. That means no shrines should be converted --- or reconverted. All
religious traditions should be respected. And the magnanimity of
tolerance should overcome the pettiness of supremacism.

Mustafa Akyol, a contributing Opinion writer, is a senior fellow on
Islam and modernity at the
\href{https://www.cato.org/people/mustafa-akyol}{Cato Institute} and the
author of the forthcoming book
``\href{https://us.macmillan.com/books/9781250256065}{Reopening Muslim
Minds:} A Return to Reason, Freedom, and Tolerance.''

\emph{The Times is committed to publishing}
\href{https://www.nytimes3xbfgragh.onion/2019/01/31/opinion/letters/letters-to-editor-new-york-times-women.html}{\emph{a
diversity of letters}} \emph{to the editor. We'd like to hear what you
think about this or any of our articles. Here are some}
\href{https://help.nytimes3xbfgragh.onion/hc/en-us/articles/115014925288-How-to-submit-a-letter-to-the-editor}{\emph{tips}}\emph{.
And here's our email:}
\href{mailto:letters@NYTimes.com}{\emph{letters@NYTimes.com}}\emph{.}

\emph{Follow The New York Times Opinion section on}
\href{https://www.facebookcorewwwi.onion/nytopinion}{\emph{Facebook}}\emph{,}
\href{http://twitter.com/NYTOpinion}{\emph{Twitter (@NYTopinion)}}
\emph{and}
\href{https://www.instagram.com/nytopinion/}{\emph{Instagram}}\emph{.}

Advertisement

\protect\hyperlink{after-bottom}{Continue reading the main story}

\hypertarget{site-index}{%
\subsection{Site Index}\label{site-index}}

\hypertarget{site-information-navigation}{%
\subsection{Site Information
Navigation}\label{site-information-navigation}}

\begin{itemize}
\tightlist
\item
  \href{https://help.nytimes3xbfgragh.onion/hc/en-us/articles/115014792127-Copyright-notice}{©~2020~The
  New York Times Company}
\end{itemize}

\begin{itemize}
\tightlist
\item
  \href{https://www.nytco.com/}{NYTCo}
\item
  \href{https://help.nytimes3xbfgragh.onion/hc/en-us/articles/115015385887-Contact-Us}{Contact
  Us}
\item
  \href{https://www.nytco.com/careers/}{Work with us}
\item
  \href{https://nytmediakit.com/}{Advertise}
\item
  \href{http://www.tbrandstudio.com/}{T Brand Studio}
\item
  \href{https://www.nytimes3xbfgragh.onion/privacy/cookie-policy\#how-do-i-manage-trackers}{Your
  Ad Choices}
\item
  \href{https://www.nytimes3xbfgragh.onion/privacy}{Privacy}
\item
  \href{https://help.nytimes3xbfgragh.onion/hc/en-us/articles/115014893428-Terms-of-service}{Terms
  of Service}
\item
  \href{https://help.nytimes3xbfgragh.onion/hc/en-us/articles/115014893968-Terms-of-sale}{Terms
  of Sale}
\item
  \href{https://spiderbites.nytimes3xbfgragh.onion}{Site Map}
\item
  \href{https://help.nytimes3xbfgragh.onion/hc/en-us}{Help}
\item
  \href{https://www.nytimes3xbfgragh.onion/subscription?campaignId=37WXW}{Subscriptions}
\end{itemize}
