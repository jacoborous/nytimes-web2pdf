\href{/section/opinion}{Opinion}\textbar{}The Black Sanitation Workers
Who Are Saying, `I Am a Man'

\url{https://nyti.ms/2BkBgiQ}

\begin{itemize}
\item
\item
\item
\item
\item
\item
\end{itemize}

\includegraphics{https://static01.graylady3jvrrxbe.onion/images/2020/07/21/opinion/20wilken1/merlin_174711816_2bcf34a4-509e-4448-9fde-a5ab0ef01777-articleLarge.jpg?quality=75\&auto=webp\&disable=upscale}

Sections

\protect\hyperlink{site-content}{Skip to
content}\protect\hyperlink{site-index}{Skip to site index}

\href{/section/opinion}{Opinion}

\hypertarget{the-black-sanitation-workers-who-are-saying-i-am-a-man}{%
\section{The Black Sanitation Workers Who Are Saying, `I Am a
Man'}\label{the-black-sanitation-workers-who-are-saying-i-am-a-man}}

A determined handful of men in New Orleans carry on the cause Dr. King
died defending in Memphis.

Anthony Perkins, one of the members of the new City Waste Union on
strike.Credit...William Widmer for The New York Times

Supported by

\protect\hyperlink{after-sponsor}{Continue reading the main story}

By Daytrian Wilken

Ms. Wilken is the spokesperson for the City Waste Union in New Orleans.

\begin{itemize}
\item
  July 20, 2020
\item
  \begin{itemize}
  \item
  \item
  \item
  \item
  \item
  \item
  \end{itemize}
\end{itemize}

``All labor has dignity,'' the Rev. Dr. Martin Luther King Jr. told
striking sanitation workers in Memphis more than 50 years ago.

``One day,'' he said, ``our society will come to respect the sanitation
worker, if it is to survive. For the person who picks up our garbage, in
the final analysis, is as significant as the physician. For if he
doesn't do his job, diseases are rampant.''

I never paid much attention to what sanitation workers did until a small
group of them went on strike in early May in my hometown, New Orleans.
They are called ``hoppers,'' because they spend all day hopping on and
off the backs of trucks, rounding up garbage containers, and using their
strength to dump them into the barrel that crushes the trash.

My Uncle Jonathan is one of them, and he asked me to help him and his
fellow Black workers organize their City Waste Union in the first weeks
of the strike. Their fight, which has now gone on for more than two
months, has shown me more clearly than ever before that Black people are
still shackled to a cycle of generational poverty and mistreatment.

Image

D'Artanian DeJean, a garbage hopper in New Orleans, said: ``Being a
sanitation worker is not easy. The first day I worked there was just as
hard as the last day.''Credit...William Widmer for The New York Times

Image

``I'm on strike for my dignity, for my family, and for my community,''
said Jonathan Edward, another hopper. ``This is what it looks like to
stand up like a man.''Credit...William Widmer for The New York Times

They often carry signs that say, ``I Am a Man,'' as they protest. It's
the iconic sign Memphis sanitation workers first carried in 1968, in
their bitter, 65-day strike, during which Dr. King was assassinated
after coming to support them. I am only 25, but it's obvious to me that
my uncle and his co-workers are still waging the same civil rights
battle 52 years later.

In 1968, a living wage and safer working conditions were among the
Memphis strikers' top
\href{https://kinginstitute.stanford.edu/encyclopedia/memphis-sanitation-workers-strike}{demands}
--- the same things New Orleans strikers are asking for in 2020. The men
in Memphis worked full time, but their pay was so low that they still
qualified for food stamps.

In New Orleans, before our strike, my uncle, for example, got paid
\$10.25 an hour, which isn't a living wage.

``I get up every day and go to work,'' said Darnell Harris, 34, another
hopper. ``But I can't take care of my family off what they paid me. I am
just tired of being stepped on. Me and all the guys, we're tired of
it.''

Our members are asking for \$15 an hour. ``In the 14 years I've been
working as a hopper,'' said Harold Peters, 43, ``I've never made much
more than \$100 a day. To actually see a decent income, you have to be
out there 60 or 70 hours a week.''

Image

Harold Peters, the oldest of the strikers, said he ``is out here to lead
by example.''Credit...William Widmer for The New York Times

Image

Jerry Simon, who has worked as a hopper for 11 years, said he is on
strike because ``I'm a man and I have to stand up for something. I want
my sons to look up to their father.''Credit...William Widmer for The New
York Times

In 1968, work-safety fears set off the Memphis strike, after two workers
were
\href{https://www.theroot.com/watch-the-tragic-deaths-of-robert-walker-and-echol-col-1822619781}{crushed
to death} in the barrel of their truck. Today in New Orleans, fears of
Covid-19, which hit the city so early and so hard, prompted our strike.
The men's longtime concern that their health and safety on the job are
not taken seriously turned urgent. That's why the hoppers are asking for
\$150 a week in hazard pay, and assurances of a steadier supply of
personal protective equipment.

One difference between the two strikes is that the New Orleans
sanitation workers today actually have less bargaining power than the
1968 Memphis strikers had.

\includegraphics{https://static01.graylady3jvrrxbe.onion/images/2020/07/21/opinion/20wilken7/merlin_136166961_d80ae6c9-a159-4a04-8279-cfff8a52e7bf-articleLarge.jpg?quality=75\&auto=webp\&disable=upscale}

The 1,300 Black men who stood up against the mayor and the city of
Memphis worked for the sanitation department and negotiated directly
with city leaders. But in 2020, outsourcing of garbage pickup means a
few private contracting companies manage many small groups of New
Orleans sanitation workers.

Only 14 Black men are on strike in New Orleans, but their experience
echoes those of many more hoppers in the city. And support from the
larger community has kept us going. A strike fund we set up on GoFundMe
has raised almost \$200,000. In addition, the National Labor Relations
Board is investigating some of our complaints.

But with the mix of private employers, one of which hired a public
relations firm to help during the strike, it is nearly impossible for a
large number of the workers doing the same jobs across the city to band
together and negotiate their working conditions with any one company or
with elected officials. That means Mayor LaToya Cantrell and the
sanitation department are insulated, remaining one or two steps removed
from dealing directly with the men on the front lines.

Image

Members of the City Waste Union in New Orleans have been on strike for
more than two months.Credit...William Widmer for The New York Times

Image

City Waste Union workers are asking for a living wage and safer working
conditions.Credit...William Widmer for The New York Times

In my uncle's case, the city contracts with Metro Service Group, a
Black-owned, New Orleans-based company, for part of its residential
sanitation pickup. Then, Metro subcontracts with an employment company
called PeopleReady, a division of TrueBlue, based in Washington State,
that oversees and pays my uncle and his co-workers.

So when we spoke out about how the men's pay was less than the \$11.19
living wage that the city
\href{https://www.nola.gov/economic-development/workforce-development/living-wage/}{requires},
the mayor
\href{https://www.nola.com/news/coronavirus/article_e96a4e14-995a-11ea-87ed-cf029941ee9a.html}{pointed}
to Metro for answers. And Metro
\href{https://www.nola.com/news/business/article_b37bb7ee-b64b-11ea-ac72-eb8ecb99b035.html}{pointed}
to PeopleReady. After more than two months, no one from the mayor's
office has spoken directly with the men.

At one point, Metro
\href{https://www.nola.com/news/coronavirus/article_336a7742-93d3-11ea-a344-1bdefd47e647.html}{subcontracted}
with another company to replace the strikers with prison inmates, who
were paid even less than the men on strike got paid. But after that
arrangement was made public, the subcontractor backed out.

Image

Darnell Harris has been working as a hopper for 12 years and is seeking
better working conditions.Credit...William Widmer for The New York Times

Image

``I'm out here for the garbage men in the past, and those that will come
in the future,'' Kendrick Anderson said.Credit...William Widmer for The
New York Times

As I understood it, one of the original goals of contracting out the
work years ago was to give more opportunity and power to Black and brown
private contractors in a majority-Black city. And a goal of the city's
living wage ordinance was to protect the people those companies hired. I
don't think anyone set out to take advantage of working-class Black men;
I just think it has turned into that.

``Instead of actually helping everybody,'' said Kendrick Anderson, 27, a
hopper, ``they just went along with that system they already have
going.''

In a city that makes
\href{https://www.wdsu.com/article/nungesser-new-orleans-region-to-see-a-billion-dollar-tax-revenue-loss-due-to-pandemic/32179410}{millions}
of dollars off Mardi Gras, the New Orleans Jazz \& Heritage Festival and
the Essence Festival, when you see City Council members swinging beads
and Mayor Cantrell second lining, our guys are riding behind them,
cleaning it all up. But these men feel invisible and uncared for.

Don't my uncle, the other hoppers and their families deserve the dignity
that Dr. King spoke of a half-century ago? Isn't it about time to do
right by these Black men, and meet their simple demands to be treated as
significant in their own city?

Daytrian Wilken is the spokesperson for the City Waste Union in New
Orleans. This was written in collaboration with Emily Yellin, who
produced the video series
``\href{https://www.theroot.com/c/1300-men-memphis-strike-68}{1,300 Men:
Memphis Strike '68''} on The Root.com.

\emph{The Times is committed to publishing}
\href{https://www.nytimes3xbfgragh.onion/2019/01/31/opinion/letters/letters-to-editor-new-york-times-women.html}{\emph{a
diversity of letters}} \emph{to the editor. We'd like to hear what you
think about this or any of our articles. Here are some}
\href{https://help.nytimes3xbfgragh.onion/hc/en-us/articles/115014925288-How-to-submit-a-letter-to-the-editor}{\emph{tips}}\emph{.
And here's our email:}
\href{mailto:letters@NYTimes.com}{\emph{letters@NYTimes.com}}\emph{.}

\emph{Follow The New York Times Opinion section on}
\href{https://www.facebookcorewwwi.onion/nytopinion}{\emph{Facebook}}\emph{,}
\href{http://twitter.com/NYTOpinion}{\emph{Twitter (@NYTopinion)}}
\emph{and}
\href{https://www.instagram.com/nytopinion/}{\emph{Instagram}}\emph{.}

Advertisement

\protect\hyperlink{after-bottom}{Continue reading the main story}

\hypertarget{site-index}{%
\subsection{Site Index}\label{site-index}}

\hypertarget{site-information-navigation}{%
\subsection{Site Information
Navigation}\label{site-information-navigation}}

\begin{itemize}
\tightlist
\item
  \href{https://help.nytimes3xbfgragh.onion/hc/en-us/articles/115014792127-Copyright-notice}{©~2020~The
  New York Times Company}
\end{itemize}

\begin{itemize}
\tightlist
\item
  \href{https://www.nytco.com/}{NYTCo}
\item
  \href{https://help.nytimes3xbfgragh.onion/hc/en-us/articles/115015385887-Contact-Us}{Contact
  Us}
\item
  \href{https://www.nytco.com/careers/}{Work with us}
\item
  \href{https://nytmediakit.com/}{Advertise}
\item
  \href{http://www.tbrandstudio.com/}{T Brand Studio}
\item
  \href{https://www.nytimes3xbfgragh.onion/privacy/cookie-policy\#how-do-i-manage-trackers}{Your
  Ad Choices}
\item
  \href{https://www.nytimes3xbfgragh.onion/privacy}{Privacy}
\item
  \href{https://help.nytimes3xbfgragh.onion/hc/en-us/articles/115014893428-Terms-of-service}{Terms
  of Service}
\item
  \href{https://help.nytimes3xbfgragh.onion/hc/en-us/articles/115014893968-Terms-of-sale}{Terms
  of Sale}
\item
  \href{https://spiderbites.nytimes3xbfgragh.onion}{Site Map}
\item
  \href{https://help.nytimes3xbfgragh.onion/hc/en-us}{Help}
\item
  \href{https://www.nytimes3xbfgragh.onion/subscription?campaignId=37WXW}{Subscriptions}
\end{itemize}
