Sections

SEARCH

\protect\hyperlink{site-content}{Skip to
content}\protect\hyperlink{site-index}{Skip to site index}

\href{https://www.nytimes3xbfgragh.onion/section/politics}{Politics}

\href{https://myaccount.nytimes3xbfgragh.onion/auth/login?response_type=cookie\&client_id=vi}{}

\href{https://www.nytimes3xbfgragh.onion/section/todayspaper}{Today's
Paper}

\href{/section/politics}{Politics}\textbar{}A Liberal Town Built Around
Confederate Generals Rethinks Its Identity

\url{https://nyti.ms/3g4WXlX}

\begin{itemize}
\item
\item
\item
\item
\item
\end{itemize}

Advertisement

\protect\hyperlink{after-top}{Continue reading the main story}

Supported by

\protect\hyperlink{after-sponsor}{Continue reading the main story}

\hypertarget{a-liberal-town-built-around-confederate-generals-rethinks-its-identity}{%
\section{A Liberal Town Built Around Confederate Generals Rethinks Its
Identity}\label{a-liberal-town-built-around-confederate-generals-rethinks-its-identity}}

In Lexington, Va., where Robert E. Lee and Stonewall Jackson are buried,
people are reassessing the town's ties to a legacy that symbolizes
slavery and oppression.

\includegraphics{https://static01.graylady3jvrrxbe.onion/images/2020/07/25/us/politics/00lexington-1/00lexington-1-articleLarge-v2.jpg?quality=75\&auto=webp\&disable=upscale}

\href{https://www.nytimes3xbfgragh.onion/by/reid-j-epstein}{\includegraphics{https://static01.graylady3jvrrxbe.onion/images/2019/06/25/reader-center/author-reid-epstein/9e877853d8234217b58e5762253aa771-thumbLarge.png}}

By \href{https://www.nytimes3xbfgragh.onion/by/reid-j-epstein}{Reid J.
Epstein}

\begin{itemize}
\item
  July 26, 2020
\item
  \begin{itemize}
  \item
  \item
  \item
  \item
  \item
  \end{itemize}
\end{itemize}

LEXINGTON, Va. --- It's a short drive in Lexington from a home on
Confederate Circle past the
\href{https://lexingtonvirginia.com/directory/attractions/stonewall-jackson-memorial-cemetery}{Stonewall
Jackson Memorial Cemetery} and over to the
\href{https://roberteleehotel.com/}{Robert E. Lee Hotel}, where locals
like to stop for a drink.

There may be tourists there looking for directions to the
\href{https://my.wlu.edu/lee-chapel-and-museum}{Lee Chapel}, or one of
the two Stonewall Jackson statues in town. They might see a Washington
and Lee University student paddling a canoe down the Maury River, named
for the Confederate oceanographer Matthew Fontaine Maury.

If medical treatment is needed, residents can head to the Stonewall
Jackson Hospital. For groceries, there's a Food Lion at Stonewall
Square, which isn't far from Rebel Ridge Road, just up the way from
Stonewall Street and Jackson Avenue.

For 150 years Lexington, a picturesque city nestled in Virginia's Blue
Ridge Mountains, has been known to the outside world as the final
resting place of Lee, the Confederacy's commanding general during the
Civil War, and Jackson, whom Lee referred to as his ``right arm.'' They
form the basis of a daily existence here that has long been tethered to
the iconography of the Civil War and its two most famous Confederate
generals, whose legacy has seeped into the town's culture like the July
humidity.

\includegraphics{https://static01.graylady3jvrrxbe.onion/images/2020/07/25/us/politics/00lexington-2/merlin_174933948_65318f03-7150-4934-a9ab-6e739372445f-articleLarge.jpg?quality=75\&auto=webp\&disable=upscale}

But Lexington is no longer a bastion of conservatism. It is a liberal
college town of about 7,000 people that voted
\href{https://www.nytimes3xbfgragh.onion/elections/2016/results/virginia}{60
percent for Hillary Clinton four years ago}, and in 2018 gave
\href{https://www.nytimes3xbfgragh.onion/elections/results/virginia-senate}{70
percent of its vote to the Democratic Senate candidate}, Tim Kaine.
Black Lives Matter signs dot the windows of downtown stores, and
residents haven't backed a Republican for president since Ronald Reagan.

These dueling sensibilities place Lexington at particularly delicate
intersection of the
\href{https://www.nytimes3xbfgragh.onion/2020/06/24/us/confederate-statues-photos.html}{national
debate over Confederate monuments and emblems}. As Americans protesting
racial injustice have torn down statues and memorials to Confederates,
the town finds itself reassessing its identity, divided between the
growing imperative to eradicate symbols of slavery and decades of
cultural and economic ties to the Confederates who fought to preserve
it.

``When you're surrounded by all of the symbols, it just is a way of
life,'' said Marylin Alexander, 67, the lone Black member of the City
Council. ``It was not until recently that there was a realization for me
that there was such an outcry from the community, that felt these
symbols and signs needed to come down or be changed.''

City Council meetings in July have been almost entirely devoted to the
question of the city-owned cemetery named for Jackson;
\href{https://www.youtube.com/watch?v=Ip6tg5VIbCE}{one session lasted
five hours}, ending with a unanimous after-midnight vote to remove signs
bearing Jackson's name. A second meeting began with pleas from residents
to put the signs back up. The council plans a session on Friday to
discuss new names, with a vote possible in September.

Image

Lexington Mayor Frank Friedman stands on Main Street in downtown
Lexington.Credit...Eze Amos for The New York Times

``I long for the days of people complaining about potholes and not
heritage,'' said Lexington's mayor, Frank Friedman.

Ms. Alexander said it had never occurred to her to propose taking
Jackson's name off the cemetery, believing that it would have no support
from white Lexingtonians. ``Most of my life I have come to realize that
these are things that have just been, this is the way it is and this is
the way it's always going to be,'' she said.

For decades, the names of Lexington's Confederate forebears have mostly
gone unchallenged. A 2011 City Council vote to forbid flying the
Confederate flag on municipal flagpoles drew a lawsuit,
\href{https://casetext.com/case/sons-of-confederate-veterans-v-city-of-lexington-2}{eventually
dismissed by a federal appeals court}, from the local chapter of the
Sons of Confederate Veterans; until this spring no one had proposed
removing Jackson's name from the cemetery, where a towering statue of
the general rises above his family plot.

At Washington and Lee, students' degrees still come
\href{https://www.law.com/2019/11/20/robert-e-lee-on-your-diploma-some-law-students-say-no-thanks/?slreturn=20200625074259}{with
portraits of its two namesake}s, and at the Virginia Military Institute,
where Jackson taught before the war, first-year students are required to
\href{https://www.vmi.edu/archives/civil-war-and-new-market/battle-of-new-market/new-market-ceremony-history/}{re-enact
the 1864 Battle of New Market} as Confederate soldiers.

Image

The Colonnade at the Washington and Lee University campus. This month,
79 percent of the faculty voted to strip Lee's name from the
school.Credit...Eze Amos for The New York Times

Still, attitudes have started to change in recent years.
\href{http://www.graceepiscopallexington.org/about-us/}{Grace Episcopal
Church downtown dropped Robert E. Lee from its name in 2017}, and last
year the local Boy Scout council
\href{https://www.whsv.com/content/news/Local-scouting-council-changes-name-from-Stonewall-Jackson-Area-Council-to-Virginia-Headwaters-Council-565449041.html}{changed
its name from the Stonewall Jackson Area Council to the Virginia
Headwaters Council}.

Bigger changes are now afoot in town, which has a Black population of
just under 9 percent. Carilion, the Roanoke, Va.-based health care
conglomerate that owns the Stonewall Jackson Hospital, said Thursday
that
\href{https://www.wsls.com/news/local/2020/07/24/carilion-to-rename-lexingtons-stonewall-jackson-hospital/}{it
would change the name to Rockbridge Community Hospital}. Francesco
Benincasa, whose family owns the Robert E. Lee Hotel, said Friday that
it would be renamed ``The Gin'' starting next month.

``It's a little hard to brand hospitality after generals,'' Mr.
Benincasa said in an interview.

Adama Kamara grew up in Lexington, attending preschool in a church named
for Stonewall Jackson.
\href{https://news.emory.edu/stories/2020/05/er_commencement_award_orator_kamara/campus.html}{A
2020 graduate of Emory University}, in Atlanta, she had never protested
the city's Confederate memorials, but when the City Council met on July
2 to debate the cemetery's name she called in via video conference.

``It's not just the history that's shameful, it's the way the people are
so committed to preserving it in this town,'' she told city officials.
``This preservation has caused me deep pain.''

Image

Adama Kamara outside her home in Lexington.~Credit...Eze Amos for The
New York Times

Almost instantly, Ms. Kamara, 22, began receiving supportive text
messages and emails from former classmates, teachers and longtime
friends in town, people with whom she'd never before discussed the
city's Confederate forefathers. She and other young people, Lexington
natives who'd gone away to college but returned during the coronavirus
pandemic, began organizing to protest the city's street names, statues
and the local public school curriculum, which they said focused too much
on lionizing local Confederate history at the expense of America's Black
experience.

``I don't think we have ever been given the space to say we as Black
people feel very uncomfortable about this,'' Ms. Kamara said. ``We have
been silently thinking these things and silently compartmentalized this,
but until we started hearing each other we had no idea that we all felt
this way.''

It did not take long for resistance to removing Jackson's name from the
cemetery to grow.

Representative Ben Cline, a Republican who represents Lexington in
Congress,
\href{https://www.facebookcorewwwi.onion/48631937877/posts/10158889404357878/?d=n}{wrote
on Facebook}: ``I suppose they'll rename it something like `Lexington
Cemetery: Now with Surprise Inside!' Or if they want to be more
accurate, something like `Future Democrat Voter Quarry.''' His office
did not respond to phone calls, emails or text messages seeking an
interview.

Heather Hopkins Barone, a local marketer, wrote to the City Council that
she had more than 2,000 names on a petition opposing the change.

``You cannot erase history because a few people are offended,'' she
wrote in the letter that she also shared
\href{https://www.facebookcorewwwi.onion/historicdowntownlexingtonvirginia/posts/1897047810425491}{on
a Facebook page}devoted to local affairs. ``The affect that it will also
have on the tourism industry and the Alumni will destroy this town.''

Tourism is the biggest component of the city's revenues after property
taxes, and the biggest economic drivers are the two universities, which
are inextricably linked to Lee and Jackson.

Image

Ellen Darlene Bane's house, adorned with three flags, including the
Confederate battle flag, sits across the street from a predominantly
Black church.Credit...Eze Amos for The New York Times

In a house two blocks from a downtown shopping strip that includes the
Red Hen ---
\href{https://www.nytimes3xbfgragh.onion/2018/06/23/us/politics/sarah-huckabee-sanders-restaurant.html}{a
restaurant briefly famous} for refusing to serve then-White House press
secretary Sarah Huckabee Sanders in 2018 --- Ellen Darlene Bane, 64,
flies three flags: The Confederate battle flag, a flag that combines the
Confederate emblem with the Virginia state seal and the yellow Gadsden
flag that's become associated with the Tea Party.

Ms. Bane, who lives across the street from a Black church, the Gospel
Way Church of God in Christ, said she began flying the flags six years
ago and has never received a complaint. She called the movement to
remove Lee and Jackson's names ``crap'' and predicted escalating racial
tensions in Lexington.

``Everybody's getting racist,'' she said. ``It's going to be the Blacks
against the whites.''

Lexington's universities are facing their own reckoning. At Washington
and Lee,
\href{https://www.washingtonpost.com/education/2020/07/06/faculty-resoundingly-votes-change-name-washington-lee/}{79
percent of the faculty voted on July 6} to strip Lee's name from the
school, prompting the board of trustees to announce
``\href{https://www.wlu.edu/the-w-l-story/leadership/board-of-trustees/messages-from-the-board/}{a
thoughtful and deliberative process}'' to examine Lee's legacy.

Image

Lucas Morel, chairman of the politics department at the Washington and
Lee University, in front of his house in Staunton, Va.Credit...Eze Amos
for The New York Times

One of the leading proponents of keeping the Lee name is Lucas E. Morel,
\href{https://my.wlu.edu/directory/profile?ID=x1338}{an Abraham Lincoln
scholar} who is chairman of the politics department. He argued that the
name honors Lee's contributions to the school ---
\href{https://www.wlu.edu/the-w-l-story/university-history/}{he led its
revival after the war} --- without making a judgment about his
leadership of the Confederate army.

``We can separate Lee's generalship of the Confederacy and his symbolism
as patron saint of the Lost Cause from his laudable contribution to the
university,'' Professor Morel said. ``To remove Lee's name is to say,
`Thank you for the gift of saving this college, but we don't appreciate
that contribution to such an extent that we think we should continue to
honor you.'''

At the Virginia Military Institute, until 2015 all first-year students
were required to salute the statue of Jackson when passing it. A public
university, the school has retained its conservative politics, well
after the
\href{https://www.nytimes3xbfgragh.onion/1996/09/22/us/defiant-vmi-to-admit-women-but-will-not-ease-rules-for-them.html}{Supreme
Court ordered it to admit women in 1996}.

But Virginia's state politics, which govern the school, have changed.
\href{https://www.nytimes3xbfgragh.onion/2019/11/05/us/politics/virginia-elections.html}{Democrats
control the state legislature}. Gov. Ralph Northam, a 1981 V.M.I.
graduate who
\href{https://www.nytimes3xbfgragh.onion/2020/06/03/us/robert-e-lee-statue-richmond.html}{is
working to take down state-owned Confederate monuments}, ``has
confidence that V.M.I.'s Board of Visitors will do the right thing,''
said his spokesman, Grant Neely.

Jennifer Carroll Foy, a member of the Virginia House of Delegates who in
2003 was among the first group of Black women to graduate from V.M.I.,
said the Jackson statue should be moved to a museum.

``We can't say in Virginia that we're open for business but we're closed
to diversity and inclusion,'' said Ms. Foy, who is now running for
governor. ``No child looks at a Confederate monument and feels
inspired.''

Image

A Statue of Stonewall Jackson above his family crypt at the Stonewall
Jackson Memorial Cemetery.Credit...Eze Amos for The New York Times

David Sigler, a City Council member who graduated from Washington and
Lee and works as the financial aid director at V.M.I., said renaming the
Stonewall Jackson Cemetery ought to be the first move to pivot the
town's identity away from its Confederate past.

``Our small business owners, they have products to sell, meals to
prepare, they want their tables filled in their restaurants,'' he said.
``I will feel bad if they lose one customer because we renamed the
cemetery. But I think we might gain two customers for every one we might
lose in the long run if we're not so one-dimensional.''

Advertisement

\protect\hyperlink{after-bottom}{Continue reading the main story}

\hypertarget{site-index}{%
\subsection{Site Index}\label{site-index}}

\hypertarget{site-information-navigation}{%
\subsection{Site Information
Navigation}\label{site-information-navigation}}

\begin{itemize}
\tightlist
\item
  \href{https://help.nytimes3xbfgragh.onion/hc/en-us/articles/115014792127-Copyright-notice}{©~2020~The
  New York Times Company}
\end{itemize}

\begin{itemize}
\tightlist
\item
  \href{https://www.nytco.com/}{NYTCo}
\item
  \href{https://help.nytimes3xbfgragh.onion/hc/en-us/articles/115015385887-Contact-Us}{Contact
  Us}
\item
  \href{https://www.nytco.com/careers/}{Work with us}
\item
  \href{https://nytmediakit.com/}{Advertise}
\item
  \href{http://www.tbrandstudio.com/}{T Brand Studio}
\item
  \href{https://www.nytimes3xbfgragh.onion/privacy/cookie-policy\#how-do-i-manage-trackers}{Your
  Ad Choices}
\item
  \href{https://www.nytimes3xbfgragh.onion/privacy}{Privacy}
\item
  \href{https://help.nytimes3xbfgragh.onion/hc/en-us/articles/115014893428-Terms-of-service}{Terms
  of Service}
\item
  \href{https://help.nytimes3xbfgragh.onion/hc/en-us/articles/115014893968-Terms-of-sale}{Terms
  of Sale}
\item
  \href{https://spiderbites.nytimes3xbfgragh.onion}{Site Map}
\item
  \href{https://help.nytimes3xbfgragh.onion/hc/en-us}{Help}
\item
  \href{https://www.nytimes3xbfgragh.onion/subscription?campaignId=37WXW}{Subscriptions}
\end{itemize}
