Sections

SEARCH

\protect\hyperlink{site-content}{Skip to
content}\protect\hyperlink{site-index}{Skip to site index}

\href{https://myaccount.nytimes3xbfgragh.onion/auth/login?response_type=cookie\&client_id=vi}{}

\href{https://www.nytimes3xbfgragh.onion/section/todayspaper}{Today's
Paper}

\href{/section/opinion}{Opinion}\textbar{}We're 20 Percent of America,
and We're Still Invisible

\url{https://nyti.ms/30R4EWn}

\begin{itemize}
\item
\item
\item
\item
\item
\end{itemize}

Advertisement

\protect\hyperlink{after-top}{Continue reading the main story}

\href{/section/opinion}{Opinion}

Supported by

\protect\hyperlink{after-sponsor}{Continue reading the main story}

disability

\hypertarget{were-20-percent-of-america-and-were-still-invisible}{%
\section{We're 20 Percent of America, and We're Still
Invisible}\label{were-20-percent-of-america-and-were-still-invisible}}

Disabled Americans are asking for true inclusion.

By Judith Heumann and John Wodatch

Ms. Heumann is a disability rights activist. Mr. Wodatch is a civil
rights lawyer.

\begin{itemize}
\item
  July 26, 2020
\item
  \begin{itemize}
  \item
  \item
  \item
  \item
  \item
  \end{itemize}
\end{itemize}

\includegraphics{https://static01.graylady3jvrrxbe.onion/images/2020/07/26/opinion/26disability1/merlin_174912816_ebfa6d9c-7196-46ba-85fb-8b63cdde0f0b-articleLarge.jpg?quality=75\&auto=webp\&disable=upscale}

\emph{This month as the 30th anniversary of the Americans With
Disabilities Act approached, we asked two prominent figures in the
disability rights movement, Judy Heumann and John Wodatch, where they
thought the United States stood in its quest to secure full rights for
people with disabilities.}

\emph{Mr. Wodatch is a former Department of Justice lawyer and the chief
author of the regulations of both the A.D.A. and Section 504 of the
Rehabilitation Act of 1973, an anti-discrimination law that was a
precursor to the A.D.A. He led the Justice Department office in charge
of enforcing the A.D.A. until 2010. Ms. Heumann, an international
disability rights activist, was the leader of the ``504 sit-in'' in San
Francisco in 1977, at 25 days the longest nonviolent occupation of a
federal building in American history. Ms. Heumann's role in that protest
has been documented in the recently released film
``}\href{https://www.nytimes3xbfgragh.onion/2020/03/24/movies/crip-camp-review.html}{\emph{Crip
Camp}}\emph{'' and her memoir,
``}\href{https://www.penguinrandomhouse.com/books/621090/being-heumann-by-judith-heumann/}{\emph{Being
Heumann}}\emph{.''}

\begin{center}\rule{0.5\linewidth}{\linethickness}\end{center}

On July 26, 1990, President George Bush signed the Americans With
Disabilities Act into law. Like the Civil Rights Act of 1964, the A.D.A.
was watershed legislation, the culmination of a decades-long campaign of
organized protest and activism. It, too, was a victory in the struggle
for equality for
\href{https://www.nytimes3xbfgragh.onion/2020/07/20/us/judy-heumann-alice-wong-haben-girma-disability-activists.html}{a
group of people} who had been systematically denied basic rights and
access to public spaces and services. On the 30th anniversary of the
law, it's only natural to want to celebrate. And we should.

Yet just as many of the injustices that the Civil Rights Act aimed to
eliminate are still very much with us, and still being resisted, the
full promise of the Americans With Disabilities Act has yet to be
realized. We are not yet where we need to be.

To begin to understand why, it's important to acknowledge where we
started. Our nation's disability history is daunting. Every single state
has at some point enforced legalized segregation of persons with
disabilities; disabled children were excluded from public schools;
people with only minor disabling conditions were routinely shut away for
life in custodial institutions; and states prohibited marriage between
disabled people and forced them to be sterilized.

Revelations about the brutal conditions at institutions like the
\href{https://www.nytimes3xbfgragh.onion/2020/02/21/nyregion/willowbrook-state-school-staten-island.html}{Willowbrook
State School on Staten Island} in the early 1970s shocked the public.
They led to a 1975 federal court settlement intended to move
Willowbrook's residents into their own homes in the community and
prompted similar actions against other institutions.

During our lifetimes (we are both in our 70s) we've seen children with
disabilities be denied education; we've been in cities that still had
``ugly laws'' that forbade disabled people to appear in public because
their appearance was considered offensive. We came of age in a society
rife with discrimination, with
\href{https://www.nytimes3xbfgragh.onion/2020/07/20/arts/disabilities-architecture-design.html}{few
accessible buildings}, almost no public accessible restrooms,
\href{https://www.nytimes3xbfgragh.onion/2020/07/23/business/disability-discrimination-jobs-sheltered-workshop.html}{limited
employment opportunities} for people with all types of disabilities, and
little usable public transportation.

Today, 30 years after the passage of the A.D.A., and after a series of
other disability rights laws ---
\href{https://www.nytimes3xbfgragh.onion/2020/07/22/us/504-sit-in-disability-rights.html}{Section
504}, the Fair Housing Act, the Individuals With Disabilities Education
Act among them --- this picture has changed radically. The arc of the
moral universe, as the Rev. Dr. Martin Luther King Jr. said, is indeed
bending toward justice. But he also said that arc is long.

One of the most profound outcomes of the passage of the A.D.A. has been
the gain of dignity and self-worth for disabled persons. The law not
only made our world more physically accessible, it confirmed our belief
in ourselves, our knowledge that we have the same rights as all others,
including the right to pursue and have access to a full life in its
broadest sense. It has also
\href{https://www.nytimes3xbfgragh.onion/2020/07/17/style/americans-with-disabilities-act.html}{empowered
a new generation} of disabled people. We are on our way to leaving
behind the days of shame --- when one of our greatest presidents felt he
had to hide his disability --- to the open and proud embrace of
disability and disability culture.

But laws can only do so much. To be fully realized, the people
themselves must do more than follow them by the letter. They must
embrace their spirit.

People with disabilities are the largest minority group in the United
States, but for the most part, we remain invisible. We represent about
20 percent of the population. We live in every state and in every
community; we are members of all social and racial and ethnic classes;
we are present in most families. But we are still often subject to the
same unthinking responses to emerging problems that ignore the needs,
issues or concerns of disabled persons. In most cases, we remain an
afterthought.

That invisibility persists at least partly because so few disabled
people are in leadership positions in government, business and
education. We are rarely in boardrooms,
\href{https://www.nytimes3xbfgragh.onion/2020/07/19/arts/after-oscarssowhite-disability-waits-for-its-moment.html}{featured
in TV shows or movies}, or occupying positions of political power (the
recent prominence of Senator Tammy Duckworth is a welcome exception).

\includegraphics{https://static01.graylady3jvrrxbe.onion/images/2020/07/26/opinion/26disability2/merlin_169907469_cc5320f5-ece0-4db8-b5c5-e5d5a70335e0-articleLarge.jpg?quality=75\&auto=webp\&disable=upscale}

But there are also deeper cultural factors at play. At screenings of the
Netflix documentary
``\href{https://www.nytimes3xbfgragh.onion/2020/03/24/movies/crip-camp-review.html}{Crip
Camp},'' at Sundance, audience members often asked why they had never
been told the story of Camp Jened for young people with disabilities in
the 1960s, and of the activism many of the campers pursued in the
disability rights movement as adults.

One theory is this: They didn't want to know. Historically, we have been
hidden away. Disabled people can make nondisabled people feel
vulnerable. We are a reminder of those fellow humans they may have
avoided or shunned in the past, and of the fact that so many of us
acquire disabilities as we get older.

This situation is thrown into sharper relief when we compare our
visibility to that of other identity groups. If you are unconvinced, try
this experiment: Randomly look at any 50 print advertisements. You will
no doubt find racial and ethnic diversity; you'll see women and men of
different sexual orientations; you will see gender fluidity and people
of all ages. What you won't see (or see very little of) are
representations of disabled persons.

This is just one expression of how the stories of our lives are excluded
from general public discourse. Even though it is common for disability
to overlap with identities across the spectrum of minority groups,
fighting discrimination on the basis of disability continues to take a
back seat in our national consciousness.

Certainly, part of the solution will require new laws and better
enforcement of the existing ones. We have a laundry list of changes that
need to be made: amending the nation's fair housing laws to create
accessible, affordable, permanent housing; federal regulations on the
\href{https://www.nytimes3xbfgragh.onion/2020/07/14/style/assistive-technology.html}{accessibility
of websites and information technology}; addressing the scandalous
unemployment of disabled persons (just 30 percent of disabled people of
working age are employed), expanding mental health services,
particularly for teenagers; getting people out of nursing homes and into
their own communities; ensuring that disabled people are part of, not
victims of, our responses to national disasters and emergencies,
including the Covid-19 pandemic.

Our laws are important and they have formed the bedrock for our future.
But the truth is, the A.D.A. was never intended to be the sole remedy
for all the oppression and injustice disabled people face. It is just
one tool. Our goal is to enact a broader, more nuanced approach,
extending beyond the legal abolishment of discriminatory practices.

Requirements like making playgrounds and movie theaters accessible,
providing sign language interpreters in emergency rooms or accessible
websites for registering for community programs have been life-changing.
But only when people with disabilities routinely work and play alongside
their fellow citizens will deeper change occur. The Individuals with
Disabilities Education Act and its predecessors have required inclusive
education since the 1970s and we have seen firsthand how the attitudinal
barriers long common in this country are disappearing in those students
who have been educated with disabled peers.

Having disabled persons in decision-making --- in product development,
design, governance --- in the digital world is also crucial. And the
``\href{https://www.nytimes3xbfgragh.onion/2020/07/17/style/americans-with-disabilities-act.html}{A.D.A.
Generation},'' an apt term coined by
\href{https://www.refinery29.com/en-us/2020/07/9923121/ada-american-disabilities-act-money-coronavirus}{Rebecca
Cokley} for disabled persons born after the A.D.A., will lead the way.
This generation is active, aware, and taking steps to call out and
challenge ableism when they encounter it.

But this generation cannot bring about change alone, nor should they.
When President Bush declared on the White House lawn 30 years ago, ``Let
the shameful walls of exclusion finally come tumbling down,'' he was
calling on us as a nation to recognize our responsibility to end
discrimination. If the moral arc of the universe is to continue to bend
toward justice, we must embrace disability as a critical part of
diversity, and truly welcome one another, in both letter and spirit, as
equal members of society.

Judith Heumann is a disability rights activist and the author of the
memoir ``Being Heumann.'' John Wodatch is a former Department of Justice
lawyer and the chief author of the regulations of both the A.D.A. and
Section 504 of the Rehabilitation Act of 1973.

\emph{Disability is a series of essays, art and opinion by and about
people living with disabilities.}

\emph{\textbf{Now in print:}}
\emph{``}\href{https://www.aboutusbook.com/}{\emph{About Us: Essays From
the Disability Series of The New York Times}}\emph{,'' edited by Peter
Catapano and Rosemarie Garland-Thomson, published by Liveright.}

\emph{The Times is committed to publishing}
\href{https://www.nytimes3xbfgragh.onion/2019/01/31/opinion/letters/letters-to-editor-new-york-times-women.html}{\emph{a
diversity of letters}} \emph{to the editor. We'd like to hear what you
think about this or any of our articles. Here are some}
\href{https://help.nytimes3xbfgragh.onion/hc/en-us/articles/115014925288-How-to-submit-a-letter-to-the-editor}{\emph{tips}}\emph{.
And here's our email:}
\href{mailto:letters@NYTimes.com}{\emph{letters@NYTimes.com}}\emph{.}

\emph{Follow The New York Times Opinion section on}
\href{https://www.facebookcorewwwi.onion/nytopinion}{\emph{Facebook}}\emph{,}
\href{http://twitter.com/NYTOpinion}{\emph{Twitter (@NYTopinion)}}
\emph{and}
\href{https://www.instagram.com/nytopinion/}{\emph{Instagram}}\emph{.}

Advertisement

\protect\hyperlink{after-bottom}{Continue reading the main story}

\hypertarget{site-index}{%
\subsection{Site Index}\label{site-index}}

\hypertarget{site-information-navigation}{%
\subsection{Site Information
Navigation}\label{site-information-navigation}}

\begin{itemize}
\tightlist
\item
  \href{https://help.nytimes3xbfgragh.onion/hc/en-us/articles/115014792127-Copyright-notice}{©~2020~The
  New York Times Company}
\end{itemize}

\begin{itemize}
\tightlist
\item
  \href{https://www.nytco.com/}{NYTCo}
\item
  \href{https://help.nytimes3xbfgragh.onion/hc/en-us/articles/115015385887-Contact-Us}{Contact
  Us}
\item
  \href{https://www.nytco.com/careers/}{Work with us}
\item
  \href{https://nytmediakit.com/}{Advertise}
\item
  \href{http://www.tbrandstudio.com/}{T Brand Studio}
\item
  \href{https://www.nytimes3xbfgragh.onion/privacy/cookie-policy\#how-do-i-manage-trackers}{Your
  Ad Choices}
\item
  \href{https://www.nytimes3xbfgragh.onion/privacy}{Privacy}
\item
  \href{https://help.nytimes3xbfgragh.onion/hc/en-us/articles/115014893428-Terms-of-service}{Terms
  of Service}
\item
  \href{https://help.nytimes3xbfgragh.onion/hc/en-us/articles/115014893968-Terms-of-sale}{Terms
  of Sale}
\item
  \href{https://spiderbites.nytimes3xbfgragh.onion}{Site Map}
\item
  \href{https://help.nytimes3xbfgragh.onion/hc/en-us}{Help}
\item
  \href{https://www.nytimes3xbfgragh.onion/subscription?campaignId=37WXW}{Subscriptions}
\end{itemize}
