Sections

SEARCH

\protect\hyperlink{site-content}{Skip to
content}\protect\hyperlink{site-index}{Skip to site index}

\href{https://www.nytimes3xbfgragh.onion/section/nyregion}{New York}

\href{https://myaccount.nytimes3xbfgragh.onion/auth/login?response_type=cookie\&client_id=vi}{}

\href{https://www.nytimes3xbfgragh.onion/section/todayspaper}{Today's
Paper}

\href{/section/nyregion}{New York}\textbar{}Inside the Violent and
Misogynistic World of Roy Den Hollander

\url{https://nyti.ms/3jS2E9k}

\begin{itemize}
\item
\item
\item
\item
\item
\item
\end{itemize}

Advertisement

\protect\hyperlink{after-top}{Continue reading the main story}

Supported by

\protect\hyperlink{after-sponsor}{Continue reading the main story}

\hypertarget{inside-the-violent-and-misogynistic-world-of-roy-den-hollander}{%
\section{Inside the Violent and Misogynistic World of Roy Den
Hollander}\label{inside-the-violent-and-misogynistic-world-of-roy-den-hollander}}

He was known for his hatred of women and his frivolous lawsuits. Then he
killed the son of a New Jersey federal judge before taking his own life,
officials said.

\includegraphics{https://static01.graylady3jvrrxbe.onion/images/2020/07/24/nyregion/00nj-judge1/00nj-judge1-articleLarge.jpg?quality=75\&auto=webp\&disable=upscale}

By \href{https://www.nytimes3xbfgragh.onion/by/nicole-hong}{Nicole
Hong}, \href{https://www.nytimes3xbfgragh.onion/by/mihir-zaveri}{Mihir
Zaveri} and
\href{https://www.nytimes3xbfgragh.onion/by/william-k-rashbaum}{William
K. Rashbaum}

\begin{itemize}
\item
  July 26, 2020
\item
  \begin{itemize}
  \item
  \item
  \item
  \item
  \item
  \item
  \end{itemize}
\end{itemize}

Roy Den Hollander sounded bitter and angry when he bumped into a former
rugby teammate in December at a library in Manhattan. He said he was so
sick from a rare cancer that he could die at any moment, wondering aloud
if he should sue his doctor for malpractice.

Things kept getting worse for Mr. Den Hollander, a self-described
``anti-feminist'' lawyer who was known for his misogynistic tirades and
the dozens of lawsuits he filed, many frivolous. A Manhattan judge
dismissed one of them in May, and a few weeks later, a federal judge in
New Jersey named Esther Salas canceled a scheduled hearing in a
different suit.

The delay followed years of resentment that he had harbored against
Judge Salas over his unfounded claim that she was moving the case too
slowly. That, in turn, built upon a lifetime of seething hatred toward
women: He accused his mother of preventing him from having a girlfriend,
and his ex-wife of marrying him only to obtain a green card.

Mr. Den Hollander's rage turned to violence this month when he showed up
at Judge Salas's home in New Jersey posing as a FedEx deliveryman and
opened fire, killing her 20-year-old son and wounding her husband,
investigators said. The judge, who was in the basement at the time, was
not injured.

Days before, Mr. Den Hollander, 72, had traveled by train to San
Bernardino County, Calif.,
\href{https://www.nytimes3xbfgragh.onion/2020/07/22/nyregion/roy-den-hollander-esther-salas.html}{where
he shot and killed a rival men's rights lawyer at his home}, the
authorities said.

Hours after the shooting in New Jersey, the police found Mr. Den
Hollander's body off a road in upstate New York with a single gunshot to
the head.

\includegraphics{https://static01.graylady3jvrrxbe.onion/images/2020/07/26/nyregion/26NJ-JUDGE-sur/26NJ-JUDGE-sur-articleLarge.jpg?quality=75\&auto=webp\&disable=upscale}

In his nearby rental car,
\href{https://www.nytimes3xbfgragh.onion/2020/07/25/nyregion/roy-den-hollander-esther-salas-list.html}{investigators
found a list naming more than a dozen possible targets}, according to
people briefed on the investigation. Aside from Judge Salas and the
rival lawyer, the list included the names of three other female judges
and two oncologists, at least one of whom had treated Mr. Den Hollander.

An examination of Mr. Den Hollander's life shows how he represented the
most violent elements of a male supremacist movement whose discourse
online has become increasingly threatening toward women.

He made his views clear in thousands of pages of writing. In his final
months, he uploaded the last version of his autobiography, a 1,698-page
manifesto that ended with an ominous epilogue about his determination to
fight ``feminazis'' until his last breath.

His beliefs swirled between the worlds of self-proclaimed anti-feminists
and men's rights activists. He ranted about what he perceived to be
\href{https://www.nytimes3xbfgragh.onion/2018/07/13/style/mens-rights-movement.html}{gender
discrimination against men} in family courts and other institutions, a
focus of men's rights activists, but also wrote blog posts calling for
women to be killed.

After a contentious divorce in 2001, Mr. Den Hollander began using the
court system to address his grievances, suing nightclubs for advertising
ladies' nights discounts and Columbia University for having a women's
studies program. When he lost in court, as he almost always did, he
would sometimes respond with lawsuits targeting the opposing lawyers
personally --- and even once sued a judge who had ruled against him.

But by the end of his life, he was a man alone, facing terminal cancer,
financial instability and a growing ostracization from the legal
community and advocates for men's rights, according to his writings and
interviews with a dozen people who spoke with him in the past four
decades.

``It really appeared to be a classical story of someone who felt
scorned, but then took it to a delusional, psychotic level in his
response to it,'' said Nicholas J. Mundy, the divorce lawyer who
represented Mr. Den Hollander's ex-wife.

When Mr. Den Hollander felt aggrieved, Mr. Mundy said: ``He stopped at
nothing to harass you and make your life miserable. He was like The
Terminator.''

Mr. Den Hollander's turn to violence appeared to be years in the making.
In his autobiography, he mused about killing his mother and about sexual
violence against a female judge in his divorce case.

During his divorce proceedings in 2001, his wife accused him of
threatening her with a gun.

Mr. Den Hollander had long harbored a grudge against Marc Angelucci, the
rival lawyer who served as the vice president of the National Coalition
for Men, a men's rights group. Last year, Mr. Angelucci
\href{https://www.nytimes3xbfgragh.onion/2019/02/24/us/military-draft-men-unconstitutional.html}{won
a major victory in a lawsuit he brought} challenging the male-only
military draft, an issue that Mr. Den Hollander believed belonged only
to him.

After Mr. Angelucci filed the lawsuit in 2013, Mr. Den Hollander called
the coalition's president and threatened violence. Mr. Den Hollander was
then kicked out of the group.

Mr. Den Hollander shot and killed Mr. Angelucci, 52, on July 11 at his
home in San Bernardino County, the authorities said.

Image

Judge Esther Salas, who presided over a case of Mr. Den Hollander's, was
nominated to the federal court in 2010.Credit...Rutgers Law School, via
Associated Press

Mr. Den Hollander had brought a lawsuit in New Jersey similar to Mr.
Angelucci's that was pending before Judge Salas. In his autobiography,
he complained about delays in the case and appeared to be jealous of Mr.
Angelucci's victory. Last month, Judge Salas had canceled a hearing in
the suit that had been set for June 25.

In both the New Jersey and California shootings, the gunman wore a
uniform resembling a FedEx driver, according to people briefed on the
investigation. A spokesman for FedEx has said the company is cooperating
with the inquiry.

In his autobiography, Mr. Den Hollander wrote about impersonating a
FedEx delivery man on the phone when he was stalking his ex-wife after
their divorce to figure out when she would be home.

Mr. Den Hollander grew up in Midland Park, N.J., a middle-class town
about 25 miles northwest of Manhattan, and had a loathing for his
mother, clinging to grudges against her. In his autobiography, he
claimed that she told him she wished he had never been born and that she
would not let him have a girlfriend or learn to play a musical
instrument.

Mr. Den Hollander described getting in trouble in the third grade after
trying to forcibly kiss two girls in his class, a pattern that would
continue throughout his life. He later wrote that he was kicked out of a
martial arts academy for ``flirting'' with women.

After graduating high school in 1965, he briefly attended the University
of Colorado and later took courses at Columbia University, but it was
not clear if he ever received a bachelor's degree. He graduated from the
George Washington University Law School in 1985 and received his M.B.A.
from Columbia Business School in 1997.

During this period, he avoided the Vietnam War draft and drifted between
jobs, working as a local news reporter and for political campaigns in
New York, according to his resume.

His many degrees and former employers created a veneer of
respectability. Later in life, he sometimes brokered introductions with
other lawyers by citing the fact that he once worked as an associate at
Cravath Swaine \& Moore, one of the most prestigious law firms in New
York. (A spokeswoman for Cravath did not respond to a request for
comment, but two lawyers confirmed working with him there in the late
1980s.)

He then moved to Moscow, a turning point in his life. In 1999, he was
hired to work in the Moscow office at Kroll Associates, a corporate
investigations firm.

Joe Serio, who helped him transition into the job as his replacement,
said Mr. Den Hollander, who was in his early 50s at the time, talked
openly about ``keeping women in their place'' and pursuing much younger
women. He was obsessed with his own appearance, Mr. Serio said, dyeing
his hair to look younger.

A spokeswoman for Kroll did not respond to a request for comment.

In 2000, Mr. Den Hollander married a Russian woman, Alina Shipilina, in
Moscow and returned to New York with her later that year.

The marriage quickly fell apart. He filed for divorce in 2001, accusing
his wife of being a prostitute and of duping him into marriage to obtain
a green card. She accused him of publishing her diary and naked photos
online, citing an incident in which he threatened her with a gun,
according to a complaint she filed that was posted on Mr. Den
Hollander's personal website.

Image

A roadblock in upstate New York, near where Mr. Den Hollander's body was
found.Credit...Jim Sabastian/Times Herald-Record, via Associated Press

At the time of the divorce, she was 25 and he was 53, according to the
records posted on the site. She did not respond to requests for comment.

The divorce consumed him. He failed in his effort to seek an annulment
to invalidate the marriage. Standing on the courthouse steps in New York
after a divorce hearing, his devastation turned to hatred, he wrote in
his autobiography. He said he wanted to bomb a feminist organization.

``Finally,'' he wrote. ``I knew my real enemies, the ones who plotted my
destruction from birth, the ones who smiled so sweetly through their
blood red lips --- dames.''

For years afterward, the subject of his ex-wife came up constantly,
according to four people who recall the conversations.

After his divorce, Mr. Den Hollander's legal crusade escalated, and he
started filing a flurry of federal and state lawsuits. He often
identified himself in court filings as an anti-feminist and a men's
rights lawyer, but some cases seemed to have no ideological purpose ---
like a lawsuit he filed in 2013 accusing the M.T.A. of overcharging for
MetroCards.

In 2007, he sued an upstairs neighbor in his Manhattan building,
complaining of excess noise. After a routine court hearing in the case,
Mr. Den Hollander chased the neighbor's lawyer, Paul Steinberg, down a
hallway and grabbed him from behind. They ended up scuffling on the
floor, Mr. Steinberg said.

That night, Mr. Steinberg filed a complaint with the court and cited Mr.
Den Hollander's misogynistic blog posts in arguing for extra security,
``particularly when there are factors (such as a female judge) which may
trigger unpredictable behavior by Mr. Den Hollander,'' according to a
copy of the complaint provided to The New York Times.

Another female lawyer who had faced off against Mr. Den Hollander said a
male colleague, after reading Mr. Den Hollander's violent blog posts,
once accompanied her to court in order to sit next to Mr. Den Hollander
and create a physical shield for her.

His unorthodox legal battles gained Mr. Den Hollander appearances on The
Colbert Report and Fox News, but his notoriety alienated him from
mainstream clients. In recent years, he took contract assignments
helping big law firms review documents. One job paid \$31 an hour.

He filed for bankruptcy in 2011 and frequently bemoaned his declining
income. At a hearing in 2018, he seemed embarrassed by his status,
telling the judge, ``I get by doing the lowest of lowest of legal work,
called document review.''

A legal services firm, Epiq Systems, fired Mr. Den Hollander in 2016
after he called another office worker an ``illegal,'' according to a
lawsuit he filed against the company. A spokeswoman for Epiq confirmed
that Mr. Den Hollander was terminated, without elaborating.

Later that year, he made calls for the Trump campaign as a volunteer,
according to his autobiography. He said he was drawn to Donald Trump's
views on immigration.

An official with the Trump campaign said, ``We don't know anything about
him, but the crimes in this case are horrific.''

In 2015, in his suit challenging the constitutionality of the male-only
military draft, Mr. Den Hollander represented a woman who wanted to
enlist. It was a legal cause supported by women's advocacy groups, but
Mr. Den Hollander had a different motivation. He wrote in his
autobiography that women should ``finally know not just the benefits but
also some of the real hell of manhood.''

When the case was assigned to Judge Salas, he wrote that he was
initially attracted to her and wanted to ask her out. Later on, he
called her ``a lazy and incompetent Latina.'' He claimed that she had
worked for organizations ``trying to convince America that whites,
especially white males, were barbarians.''

The lawsuit is still pending and on Wednesday was assigned to a
different federal judge in New Jersey.

In late 2018, Mr. Den Hollander received a diagnosis of mucosal
melanoma, a severe form of cancer, he wrote.

It seemed that one of Mr. Den Hollander's few remaining joys was playing
rugby, which he said he originally joined to ``keep myself in shape for
the girls.''

The rugby teammate who saw him in December, when he sounded bitter, said
he also chatted with Mr. Den Hollander in a short phone call during the
pandemic. Mr. Den Hollander did not elaborate on his health but said he
appreciated the call, recalled the teammate, who spoke on condition of
anonymity.

In one of Mr. Den Hollander's last court appearances, a federal judge in
Manhattan ruled against him at a hearing in February 2018. Mr. Den
Hollander became angry, and the judge urged him not to take the ruling
personally.

``It was a pleasure appearing before you, Your Honor,'' Mr. Den
Hollander told the judge, ``but it is always personal.''

Katherine Rosman and Alan Feuer contributed reporting. Kitty Bennett
contributed research.

Advertisement

\protect\hyperlink{after-bottom}{Continue reading the main story}

\hypertarget{site-index}{%
\subsection{Site Index}\label{site-index}}

\hypertarget{site-information-navigation}{%
\subsection{Site Information
Navigation}\label{site-information-navigation}}

\begin{itemize}
\tightlist
\item
  \href{https://help.nytimes3xbfgragh.onion/hc/en-us/articles/115014792127-Copyright-notice}{©~2020~The
  New York Times Company}
\end{itemize}

\begin{itemize}
\tightlist
\item
  \href{https://www.nytco.com/}{NYTCo}
\item
  \href{https://help.nytimes3xbfgragh.onion/hc/en-us/articles/115015385887-Contact-Us}{Contact
  Us}
\item
  \href{https://www.nytco.com/careers/}{Work with us}
\item
  \href{https://nytmediakit.com/}{Advertise}
\item
  \href{http://www.tbrandstudio.com/}{T Brand Studio}
\item
  \href{https://www.nytimes3xbfgragh.onion/privacy/cookie-policy\#how-do-i-manage-trackers}{Your
  Ad Choices}
\item
  \href{https://www.nytimes3xbfgragh.onion/privacy}{Privacy}
\item
  \href{https://help.nytimes3xbfgragh.onion/hc/en-us/articles/115014893428-Terms-of-service}{Terms
  of Service}
\item
  \href{https://help.nytimes3xbfgragh.onion/hc/en-us/articles/115014893968-Terms-of-sale}{Terms
  of Sale}
\item
  \href{https://spiderbites.nytimes3xbfgragh.onion}{Site Map}
\item
  \href{https://help.nytimes3xbfgragh.onion/hc/en-us}{Help}
\item
  \href{https://www.nytimes3xbfgragh.onion/subscription?campaignId=37WXW}{Subscriptions}
\end{itemize}
