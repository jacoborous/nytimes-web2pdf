Sections

SEARCH

\protect\hyperlink{site-content}{Skip to
content}\protect\hyperlink{site-index}{Skip to site index}

\href{https://www.nytimes3xbfgragh.onion/section/world/europe}{Europe}

\href{https://myaccount.nytimes3xbfgragh.onion/auth/login?response_type=cookie\&client_id=vi}{}

\href{https://www.nytimes3xbfgragh.onion/section/todayspaper}{Today's
Paper}

\href{/section/world/europe}{Europe}\textbar{}Fighting False News in
Ukraine, Facebook Fact Checkers Tread a Blurry Line

\url{https://nyti.ms/2Ee7lK2}

\begin{itemize}
\item
\item
\item
\item
\item
\end{itemize}

Advertisement

\protect\hyperlink{after-top}{Continue reading the main story}

Supported by

\protect\hyperlink{after-sponsor}{Continue reading the main story}

\hypertarget{fighting-false-news-in-ukraine-facebook-fact-checkers-tread-a-blurry-line}{%
\section{Fighting False News in Ukraine, Facebook Fact Checkers Tread a
Blurry
Line}\label{fighting-false-news-in-ukraine-facebook-fact-checkers-tread-a-blurry-line}}

Facebook hired a Ukrainian group battling Russian disinformation to flag
misleading posts. But critics say the fact checkers' work veers into
activism.

\includegraphics{https://static01.graylady3jvrrxbe.onion/images/2020/07/22/world/xxukraine-facebook1/merlin_117719744_4ac4bdb0-8ee6-4bb3-a0e8-05caaf05fe8a-articleLarge.jpg?quality=75\&auto=webp\&disable=upscale}

\href{https://www.nytimes3xbfgragh.onion/by/anton-troianovski}{\includegraphics{https://static01.graylady3jvrrxbe.onion/images/2019/09/24/reader-center/author-anton-troianovski/author-anton-troianovski-thumbLarge.png}}

By \href{https://www.nytimes3xbfgragh.onion/by/anton-troianovski}{Anton
Troianovski}

\begin{itemize}
\item
  July 26, 2020
\item
  \begin{itemize}
  \item
  \item
  \item
  \item
  \item
  \end{itemize}
\end{itemize}

MOSCOW --- To understand the complexity of policing online
disinformation, consider the small Ukrainian fact-checking group
StopFake.

Earlier this year, Facebook hired StopFake to help curb the flow of
Russian propaganda and other false news across its platform in Ukraine.

StopFake, like all of Facebook's outside fact checkers, signed
\href{https://ifcncodeofprinciples.poynter.org/know-more/the-commitments-of-the-code-of-principles}{a
pledge} to be nonpartisan and not to focus its checks ``on any one
side.'' But in recent weeks, StopFake has been battling accusations of
ties to the Ukrainian far right and of bias in its fact-checking. The
episode has raised thorny questions for Facebook over whom it allows to
separate truth from lies --- and who is considered a neutral fact
checker in a country at war.

``They are empowering these organizations and these people to be making
calls about what kind of information, what kind of opinions, what kind
of communications are illegitimate or legitimate,'' Matthew Schaaf, who
leads the Ukraine office of the American human rights group Freedom
House, said of Facebook and its fact checkers. ``The question that needs
to be asked is: Do these people deserve our trust?''

A Ukrainian news outlet, Zaborona, published an
\href{http://zaborona.com/ru/stopfake-i-faktcheking-v-facebook/}{article}
this month citing photographs of a prominent StopFake member meeting
with nationalist figures, including a white-power rock musician whose
lyrics deny the Holocaust. StopFake denied having any far-right ties or
bias,
\href{https://www.stopfake.org/en/the-stopfake-supervisory-board-position-about-the-escalating-information-attacks-directed-against-the-project-team/}{calling}
the Zaborona article part of a campaign of slanderous ``information
attacks.''

Zaborona's editor, Katerina Sergatskova, said she fled Ukraine on
Wednesday after receiving death threats. (StopFake has condemned the
threats.) On Facebook, some of her critics had claimed, without
evidence, that she was a Kremlin agent.

The episode underlines the high stakes facing American social media
companies as they try to respond to disinformation in the world's
geopolitical hot spots. After being criticized for failing to stop the
spread of disinformation during the 2016 presidential campaign in the
United States,
Facebook\href{https://www.nytimes3xbfgragh.onion/2016/12/15/technology/facebook-fake-news.html?ribbon-ad-idx=4\&rref=business/media\&module=Ribbon\&version=context\&region=Header\&action=click\&contentCollection=Media\&pgtype=oak}{sought
to avoid} becoming an arbiter of truth by creating a third-party
fact-checking program.

The program
\href{https://www.facebookcorewwwi.onion/journalismproject/fact-checking-expansion-and-investment-2020}{now
includes} more than 50 organizations that check facts in more than 40
languages, including global news agencies such as Agence France-Presse
and Reuters alongside smaller groups like StopFake.

Yevhen Fedchenko, StopFake's editor in chief, declined to comment for
this article. He has told other
\href{https://www.dw.com/ru/\%D1\%81\%D0\%BA\%D0\%B0\%D0\%BD\%D0\%B4\%D0\%B0\%D0\%BB-\%D0\%B8\%D0\%B7-\%D0\%B7\%D0\%B0-\%D1\%81\%D1\%82\%D0\%B0\%D1\%82\%D1\%8C\%D0\%B8-\%D0\%BE-\%D1\%81\%D0\%B2\%D1\%8F\%D0\%B7\%D1\%8F\%D1\%85-\%D1\%81-\%D1\%83\%D0\%BB\%D1\%8C\%D1\%82\%D1\%80\%D0\%B0\%D0\%BF\%D1\%80\%D0\%B0\%D0\%B2\%D1\%8B\%D0\%BC\%D0\%B8-\%D0\%BE-\%D1\%87\%D0\%B5\%D0\%BC-\%D1\%81\%D0\%BF\%D0\%BE\%D1\%80\%D1\%8F\%D1\%82-\%D1\%81\%D0\%BC\%D0\%B8-\%D1\%83\%D0\%BA\%D1\%80\%D0\%B0\%D0\%B8\%D0\%BD\%D1\%8B/a-54223470}{media
outlets} that he plans to file a lawsuit to defend StopFake's
reputation, and he wrote in an email that ``our legal team advised us
against talking to media until the hearing in the court.''

Facebook said in a written statement that all its fact checkers followed
a ``Code of Principles to promote fairness and nonpartisanship in
fact-checking.'' Baybars Örsek, the director of the group that
administers that \href{https://ifcncodeofprinciples.poynter.org/}{code
of principles}, said it was conducting an ``interim assessment'' of
StopFake in light of the Zaborona report.

He said his organization, the
\href{https://www.poynter.org/ifcn/}{International Fact-Checking
Network}, which was set up by the Poynter Institute for Media Studies in
St. Petersburg, Fla., takes reports of far-right ties seriously. He
acknowledged that nonpartisanship had long been a particularly difficult
thing to ascertain during an armed conflict like Ukraine's.

``They are working in a country where they are still practically in war
with Russia,'' Mr. Örsek said of StopFake. ``This is a question we also
struggle with as fact checkers: How do you do nonpartisan fact-checking
when you have tanks on the street?''

Many European countries struggle with far-right groups, but critics say
they are tolerated to an excessive degree in Ukraine because they share
a common enemy with the country's intellectual mainstream: Russia. The
notion that Ukraine has a far-right problem, in turn, is amplified and
distorted by Russian state propaganda, which often falsely refers to
Ukraine's pro-Western revolution in 2014 as a fascist coup.

\includegraphics{https://static01.graylady3jvrrxbe.onion/images/2020/07/22/world/xxukraine-facebook3/merlin_174152472_bdf1e6dc-2fab-49b6-80ca-bef4f761cb91-articleLarge.jpg?quality=75\&auto=webp\&disable=upscale}

Deadly fighting between Ukrainian forces and Russian-backed separatists
continues to simmer in the country's east. And propaganda has for years
been a key tool for the Kremlin in its effort to keep Ukraine in
Russia's orbit.

The debate over treatment of the far right came to a head after Zaborona
published its article describing what it said was evidence of StopFake's
bias. The evidence included
\href{https://twitter.com/ColborneMichael/status/1243560294233899008}{social
media photographs} showing Marko Suprun, who hosts StopFake's
English-language video program about Russian disinformation, meeting
with two Ukrainian nationalist musicians at a gathering in 2017.

The songs of one of the musicians, Arseniy Bilodub, include ``Heroes of
the White Race'' and, referring to the Holocaust, ``Six Million Words of
Lies.'' Anton Shekhovtsov, an external lecturer at the University of
Vienna who studies far-right movements in Europe, said in an interview
that he did not see StopFake itself as a far-right organization, ``but I
don't think that they are nonpartisan.''

StopFake countered that Zaborona was employing ``the fallacy of guilt by
association'' in presenting the photographs as evidence of far-right
connections on the part of Mr. Suprun. Mr. Suprun did not respond to
requests for comment.

``He has also been photographed alongside Rabbi Yakov Bleich, but this
does not make him a member of his synagogue,'' StopFake said in a
lengthy response to the Zaborona article posted online. Mr. Suprun, the
statement added, ``is not involved in the joint fact-checking project
StopFake has with Facebook.''

Ms. Sergatskova, Zaborona's editor, is originally from Russia and
received Ukrainian citizenship in 2015. A prominent Ukrainian journalist
on Facebook called her a ``lefty F.S.B. mold'' --- referring to the
Russian spy agency --- and other commenters posted her Kyiv home address
before she went into hiding.

Image

``Truth is a lie, freedom is slavery --- it's an Orwell kind of story,''
Katerina Sergatskova said. ``By all appearances we really, really
touched a nerve.''Credit...Serhiy Morgunov

\href{https://www.hrw.org/news/2020/07/14/ukraine-independent-journalist-threatened\#}{Human
Rights Watch} and the
\href{https://cpj.org/2020/07/ukrainian-journalist-katerina-sergatskova-in-hiding-amid-threats-doxing/}{Committee
to Protect Journalists} urged the Ukrainian authorities to investigate
the threats against Ms. Sergatskova. Ukrainian media organizations,
including StopFake,
\href{https://detector.media/community/article/178813/2020-07-15-gromadski-organizatsii-vimagayut-zakhistiti-katerinu-sergatskovu/}{signed
an open letter} condemning the threats. The Ukrainian police did not
respond to a request for comment.

Ms. Sergatskova said in a telephone interview after she went into hiding
that her record as an independent journalist has been distorted by
critics who saw her as playing into the Kremlin's hands.

``Truth is a lie, freedom is slavery --- it's an Orwell kind of story,''
Ms. Sergatskova said. ``By all appearances we really, really touched a
nerve.''

Ukrainian journalism students and faculty members
\href{https://www.nytimes3xbfgragh.onion/2017/02/26/world/europe/ukraine-kiev-fake-news.html}{launched
StopFake in 2014} to counter Russian disinformation, drawing praise from
Kyiv civil society and Western supporters of Ukraine. StopFake's
agreement this year to sign on as one of Facebook's two fact-checking
partners in Ukraine gave it newfound clout.

Facebook says it reduces a post's distribution in users' news feeds if a
third-party fact checker marks a post as false, but it does not take it
down. Maksym Skubenko, who heads Facebook's other Ukrainian
fact-checking partner, VoxCheck, said users typically saw posts and
articles marked as false within seconds of when his team enters a fact
check into Facebook's system.

StopFake's website shows that the organization has carried out some 200
fact checks of posts and articles for Facebook in
\href{https://www.stopfake.org/ru/category/factcheck_for_facebook_ru/}{Russian}
and
\href{https://www.stopfake.org/uk/category/factcheck_facebook_ua/}{Ukrainian}
since the group started working for the social network in April. Many of
the fact checks are apolitical and related to the coronavirus pandemic.
A smaller number address issues of Ukrainian national identity,
generally when the item being fact-checked fits into a pro-Russian
narrative.

Image

Ukrainian soldiers at a front-line position in Popasna, eastern Ukraine,
last year.Credit...Brendan Hoffman for The New York Times

In one case,
\href{https://www.stopfake.org/ru/fejk-dlya-90-ukraintsev-den-pobedy-yavlyaetsya-prazdnikom/}{StopFake
contested} Facebook users' claims that most Ukrainians celebrated the
May 9 Soviet-era Victory Day holiday marking the defeat of Nazi Germany.
In another, StopFake fact-checked
\href{https://golospravdy.eu/elena-markosyan-ukraina-eto-russkie-lyudi/}{an
interview} with a pro-Russian commentator that carried the headline
``Ukraine Is Russian People.'' When Facebook users try to share the
article featuring the interview, they see a pop-up box titled ``False
Information in This Post.''

If users then click ``Post Anyway,'' the article appears grayed-out on
their profile with the words ``False Information: Checked by independent
fact checkers,'' with a link to StopFake. The original article quotes
the commentator as saying that there are ``many Russian people'' in
Ukraine.
\href{https://www.stopfake.org/ru/fejk-ukraintsy-eto-russkie-lyudi/}{StopFake's
fact check} cites a poll in which 90 percent of Ukrainians describe
their ethnicity as Ukrainian.

The idea that Ukrainians are Russian is often repeated in Russian
disinformation, said Nina Jankowicz, a fellow at the Woodrow Wilson
International Center for Scholars in Washington who recently
\href{https://www.wilsoncenter.org/book/how-lose-information-war-russia-fake-news-and-future-conflict}{published
a book on Russian disinformation}. But it is ``also a view that many
Russians (and some Ukrainians themselves) subscribe to.''

That leads to an unanswered question for Facebook. As Ms. Jankowicz put
it: ``Should opinions be fact-checked?''

Mr. Skubenko, who heads Facebook's other Ukrainian fact-checking
partner, said he stayed away from issues of national identity.

``In some cases, it's impossible to check this in a fact-based way,'' he
said. ``Then you have to write that it's your personal opinion and not a
fact check.''

StopFake's journalists, like many Ukrainians, ``are trying to find this
compromise between liberal values and patriotic values,'' said Volodymyr
Yermolenko, a philosopher who edits Ukraine World magazine. In so doing,
he went on, StopFake seeks to break down a long history of Russian
propaganda in Ukraine.

``It's a deconstruction of myths, on top of fact-checking,'' he said.

Maria Varenikova contributed reporting from Kyiv, Ukraine, and Davey
Alba from New York.

Advertisement

\protect\hyperlink{after-bottom}{Continue reading the main story}

\hypertarget{site-index}{%
\subsection{Site Index}\label{site-index}}

\hypertarget{site-information-navigation}{%
\subsection{Site Information
Navigation}\label{site-information-navigation}}

\begin{itemize}
\tightlist
\item
  \href{https://help.nytimes3xbfgragh.onion/hc/en-us/articles/115014792127-Copyright-notice}{©~2020~The
  New York Times Company}
\end{itemize}

\begin{itemize}
\tightlist
\item
  \href{https://www.nytco.com/}{NYTCo}
\item
  \href{https://help.nytimes3xbfgragh.onion/hc/en-us/articles/115015385887-Contact-Us}{Contact
  Us}
\item
  \href{https://www.nytco.com/careers/}{Work with us}
\item
  \href{https://nytmediakit.com/}{Advertise}
\item
  \href{http://www.tbrandstudio.com/}{T Brand Studio}
\item
  \href{https://www.nytimes3xbfgragh.onion/privacy/cookie-policy\#how-do-i-manage-trackers}{Your
  Ad Choices}
\item
  \href{https://www.nytimes3xbfgragh.onion/privacy}{Privacy}
\item
  \href{https://help.nytimes3xbfgragh.onion/hc/en-us/articles/115014893428-Terms-of-service}{Terms
  of Service}
\item
  \href{https://help.nytimes3xbfgragh.onion/hc/en-us/articles/115014893968-Terms-of-sale}{Terms
  of Sale}
\item
  \href{https://spiderbites.nytimes3xbfgragh.onion}{Site Map}
\item
  \href{https://help.nytimes3xbfgragh.onion/hc/en-us}{Help}
\item
  \href{https://www.nytimes3xbfgragh.onion/subscription?campaignId=37WXW}{Subscriptions}
\end{itemize}
