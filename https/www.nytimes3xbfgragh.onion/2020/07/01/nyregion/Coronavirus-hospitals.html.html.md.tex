Sections

SEARCH

\protect\hyperlink{site-content}{Skip to
content}\protect\hyperlink{site-index}{Skip to site index}

\href{/section/nyregion}{New York}\textbar{}Why Surviving the Virus
Might Come Down to Which Hospital Admits You

\url{https://nyti.ms/31zP97w}

\begin{itemize}
\item
\item
\item
\item
\item
\item
\end{itemize}

\href{https://www.nytimes3xbfgragh.onion/news-event/coronavirus?action=click\&pgtype=Article\&state=default\&region=TOP_BANNER\&context=storylines_menu}{The
Coronavirus Outbreak}

\begin{itemize}
\tightlist
\item
  live\href{https://www.nytimes3xbfgragh.onion/2020/08/04/world/coronavirus-cases.html?action=click\&pgtype=Article\&state=default\&region=TOP_BANNER\&context=storylines_menu}{Latest
  Updates}
\item
  \href{https://www.nytimes3xbfgragh.onion/interactive/2020/us/coronavirus-us-cases.html?action=click\&pgtype=Article\&state=default\&region=TOP_BANNER\&context=storylines_menu}{Maps
  and Cases}
\item
  \href{https://www.nytimes3xbfgragh.onion/interactive/2020/science/coronavirus-vaccine-tracker.html?action=click\&pgtype=Article\&state=default\&region=TOP_BANNER\&context=storylines_menu}{Vaccine
  Tracker}
\item
  \href{https://www.nytimes3xbfgragh.onion/2020/08/02/us/covid-college-reopening.html?action=click\&pgtype=Article\&state=default\&region=TOP_BANNER\&context=storylines_menu}{College
  Reopening}
\item
  \href{https://www.nytimes3xbfgragh.onion/live/2020/08/04/business/stock-market-today-coronavirus?action=click\&pgtype=Article\&state=default\&region=TOP_BANNER\&context=storylines_menu}{Economy}
\end{itemize}

\includegraphics{https://static01.graylady3jvrrxbe.onion/images/2020/06/26/nyregion/nyvirus-inequality10/merlin_172404540_3a853b86-47f7-4f74-9df6-ec7d75262929-articleLarge.jpg?quality=75\&auto=webp\&disable=upscale}

\hypertarget{why-surviving-the-virus-might-come-down-to-which-hospital-admits-you}{%
\section{Why Surviving the Virus Might Come Down to Which Hospital
Admits
You}\label{why-surviving-the-virus-might-come-down-to-which-hospital-admits-you}}

In New York City's poor neighborhoods, some patients have languished in
understaffed hospitals, with substandard equipment. It was a different
story in Manhattan's private medical centers.

Credit...Erin Schaff/The New York Times

Supported by

\protect\hyperlink{after-sponsor}{Continue reading the main story}

\href{https://www.nytimes3xbfgragh.onion/by/brian-m-rosenthal}{\includegraphics{https://static01.graylady3jvrrxbe.onion/images/2018/02/20/multimedia/author-brian-m-rosenthal/author-brian-m-rosenthal-thumbLarge.jpg}}\href{https://www.nytimes3xbfgragh.onion/by/joseph-goldstein}{\includegraphics{https://static01.graylady3jvrrxbe.onion/images/2018/07/16/multimedia/author-joseph-goldstein/author-joseph-goldstein-thumbLarge.png}}\href{https://www.nytimes3xbfgragh.onion/by/sharon-otterman}{\includegraphics{https://static01.graylady3jvrrxbe.onion/images/2018/06/14/multimedia/author-sharon-otterman/author-sharon-otterman-thumbLarge.png}}\href{https://www.nytimes3xbfgragh.onion/by/sheri-fink}{\includegraphics{https://static01.graylady3jvrrxbe.onion/images/2018/08/24/multimedia/author-sheri-fink/author-sheri-fink-thumbLarge.png}}

By \href{https://www.nytimes3xbfgragh.onion/by/brian-m-rosenthal}{Brian
M. Rosenthal},
\href{https://www.nytimes3xbfgragh.onion/by/joseph-goldstein}{Joseph
Goldstein},
\href{https://www.nytimes3xbfgragh.onion/by/sharon-otterman}{Sharon
Otterman} and
\href{https://www.nytimes3xbfgragh.onion/by/sheri-fink}{Sheri Fink}

\begin{itemize}
\item
  Published July 1, 2020Updated July 31, 2020
\item
  \begin{itemize}
  \item
  \item
  \item
  \item
  \item
  \item
  \end{itemize}
\end{itemize}

In Queens, the borough with the most
\href{https://www.nytimes3xbfgragh.onion/2020/07/27/podcasts/the-daily/new-york-hospitals-covid.html}{coronavirus}
cases and the fewest hospital beds per capita, hundreds of patients
languished in understaffed wards, often unwatched by nurses or doctors.
Some died after removing oxygen masks to go to the bathroom.

In hospitals in impoverished neighborhoods around the boroughs, some
critically ill patients were put on ventilator machines lacking key
settings, and others pleaded for experimental drugs, only to be told
that there were none available.

It was another story at the private medical centers in Manhattan, which
have billions of dollars in endowments and cater largely to wealthy
people with insurance. Patients there got access to heart-lung bypass
machines and specialized drugs like remdesivir, even as those in the
city's community hospitals were denied more basic
\href{https://www.nytimes3xbfgragh.onion/2020/07/20/world/covid-19-treatment-synairgen-interferon-beta.html}{treatments}
like continuous dialysis.

In its first four months in New York, the coronavirus tore through
\href{https://www.nytimes3xbfgragh.onion/2020/05/18/nyregion/coronavirus-deaths-nyc.html}{low-income
neighborhoods}, infected
\href{https://www.nytimes3xbfgragh.onion/2020/04/09/nyregion/coronavirus-queens-corona-jackson-heights-elmhurst.html}{immigrants
and essential workers} unable to stay home and disproportionately killed
\href{https://www.nytimes3xbfgragh.onion/2020/04/08/nyregion/coronavirus-race-deaths.html}{Black
and Latino people}, especially those with underlying health conditions.

Now, evidence is emerging of another inequality affecting low-income
city residents: disparities in hospital care.

While the pandemic continues, it is not possible to determine exactly
how much the gaps in hospital care have hurt patients. Many factors
affect who recovers from the coronavirus and who does not. Hospitals
treat vastly different patient populations, and experts have hesitated
to criticize any hospital while workers valiantly fight the outbreak.

Still, mortality data from three dozen hospitals obtained by The New
York Times indicates that the likelihood of survival may depend in part
on where a patient is treated. At the peak of the pandemic in April, the
data suggests, patients at some community hospitals were three times
more likely to die as patients at medical centers in the wealthiest
parts of the city.

Underfunded hospitals in the neighborhoods hit the hardest often had
lower staffing, worse equipment and less access to drug trials and
advanced treatments at the height of the crisis than the private,
well-financed academic medical centers in wealthy parts of Manhattan,
according to interviews with workers at all 47 of the city's general
hospitals.

``If we had proper staffing and proper equipment, we could have saved
much more lives,'' said Dr. Alexander Andreev, a medical resident and
union representative at Brookdale University Hospital and Medical
Center, a struggling independent hospital in Brooklyn. ``Out of 10
deaths, I think at least two or three could have been saved.''

\includegraphics{https://static01.graylady3jvrrxbe.onion/images/2020/06/29/nyregion/nyvirus-inequality43/merlin_171555177_bc1496a8-ce3f-4213-8c3d-cd9afbe16202-articleLarge.jpg?quality=75\&auto=webp\&disable=upscale}

\includegraphics{https://static01.graylady3jvrrxbe.onion/images/2017/01/29/podcasts/the-daily-album-art/the-daily-album-art-articleInline-v2.jpg?quality=75\&auto=webp\&disable=upscale}

\hypertarget{listen-to-the-daily-the-mistakes-new-york-made}{%
\subsubsection{Listen to `The Daily': The Mistakes New York
Made}\label{listen-to-the-daily-the-mistakes-new-york-made}}

An investigation into hospitals during the peak of the city's
coronavirus outbreak exposed significant disparities in health care.

transcript

Back to The Daily

bars

0:00/33:28

-33:28

transcript

\hypertarget{listen-to-the-daily-the-mistakes-new-york-made-1}{%
\subsection{Listen to `The Daily': The Mistakes New York
Made}\label{listen-to-the-daily-the-mistakes-new-york-made-1}}

\hypertarget{hosted-by-michael-barbaro-produced-by-neena-pathak-austin-mitchell-and-andy-mills-and-edited-by-lisa-chow-and-lisa-tobin}{%
\subsubsection{Hosted by Michael Barbaro, produced by Neena Pathak,
Austin Mitchell and Andy Mills, and edited by Lisa Chow and Lisa
Tobin}\label{hosted-by-michael-barbaro-produced-by-neena-pathak-austin-mitchell-and-andy-mills-and-edited-by-lisa-chow-and-lisa-tobin}}

\hypertarget{an-investigation-into-hospitals-during-the-peak-of-the-citys-coronavirus-outbreak-exposed-significant-disparities-in-health-care}{%
\paragraph{An investigation into hospitals during the peak of the city's
coronavirus outbreak exposed significant disparities in health
care.}\label{an-investigation-into-hospitals-during-the-peak-of-the-citys-coronavirus-outbreak-exposed-significant-disparities-in-health-care}}

\begin{itemize}
\item
  michael barbaro\\
  From The New York Times, I'm Michael Barbaro. This is ``The Daily.''
\item
  {[}music{]}\\
  Today: A Times investigation finds that surviving the coronavirus in
  New York had a lot to do with which hospital a person went to. My
  colleague, investigative reporter Brian Rosenthal, on inequality and
  the pandemic. It's Monday, July 27.
\item
  archived recording (andrew cuomo)\\
  Thank you for being here today. This is an amazing accomplishment.

  Strategy, plan of action all along. Step one: flatten the curve. Step
  two: increase hospital capacity.

  That's what this is all about --- not overwhelming hospital capacity
  and, at the same time, increasing the hospital capacity that we have.
  So if it does exceed those numbers, which it will in most probability,
  that we have the additional capacity to deal with it.
\end{itemize}

michael barbaro

Brian, you have been part of a team investigating how the coronavirus
was handled in New York City. And I'm curious why you undertook this
project. My sense is that New York has done a fairly solid job
flattening the curve over the past few months. So what was your aim?

brian rosenthal

So New York was clearly the first big hotspot for the coronavirus in the
United States. And yes, we did succeed in flattening the curve. But we
also experienced a lot of tragedy along the way. A lot of death and a
lot of heartbreak. And now that the rest of the country is going through
different surges in the virus and different versions of what we went
through in March and April, I think it's really important to look at the
experience in New York --- the successes that we've had, but also the
mistakes that were made. And if you look at what happened in hospitals
in New York in a real close way, you'll see that there were a lot of
mistakes. And as a result, people died.

michael barbaro

And where does that story start in your reporting?

brian rosenthal

When the pandemic began in New York, a team of us on the metro desk
really were trying to follow what was happening. And we realized very
quickly that there was no one story about how this was playing out in
hospitals, because there are 47 different hospitals in New York City.
And each one was having its own experience. So a team of us divided
them. Some of us took the public hospitals. Some of us took more of the
private hospitals. And we started calling doctors, nurses, physician
assistants, all kinds of workers in each of those hospitals.

michael barbaro

And Brian, why does that distinction matter, public and private?

brian rosenthal

Well, the public hospitals are the hospitals that are run by the
government. And they cater mostly to residents who have Medicaid or
Medicare, or don't have any insurance at all. And the private hospitals
are kind of the more elite institutions that we might be familiar with
--- Mount Sinai, N.Y.U. Langone, Columbia, Cornell. And they cater
mostly to wealthier residents with health insurance through their
employer or purchased privately.

michael barbaro

And after you talked to doctors and nurses and staff from all these
different hospitals, both the public and the private, what did you
learn?

brian rosenthal

We found significant differences between the level of care available at
these wealthy private hospitals, mostly in Manhattan, and the public
hospitals and small independent hospitals scattered throughout the other
boroughs. There were differences in basically everything once you walk
in the door.

But the biggest differences were in staffing --- the level of nurses and
doctors and other types of staff that were available on a per patient
basis, as well as the equipment that was available. The age of the
equipment, the type of the equipment, and access to drug trials and
experimental treatments and advanced treatments that cost a lot of money
and may not necessarily always work, but give the patients a fighting
chance. Those are available much more in the private hospitals than the
public hospitals.

michael barbaro

Tell me about the staffing ratios.

{[}music{]}

brian rosenthal

Yeah, so the staffing ratio is very important in whether patients live
or die. Research has shown that. And there are some best practices that
have been established through the years.

If you look at an emergency room, for example, the best practice is that
there should be four patients for every one nurse. So that way, the
nurse is not having too many patients that they are trying to monitor.
And we were able to collect numbers showing the ratios in emergency
rooms at private hospitals versus public hospitals. And you could see
that the ratio is increased at every hospital. But at the private
hospitals, while the ratio went up to one nurse for six or seven
patients, it went up at the public hospitals to one nurse for 10 or 15
or even 20 patients.

michael barbaro

So about twice.

brian rosenthal

Yeah, and in the I.C.U.s, the general ratio is, because the patients are
so severe, it's two patients for every nurse. And again, those ratios
got stretched at every hospital in the city, but in private hospitals,
it would be stretched to three or four patients for every nurse. And in
the public hospitals, it was getting stretched to seven, eight, nine
patients for every nurse, which was obviously very dangerous.

michael barbaro

And what did the staff you talked to say were the consequences in some
of these public hospitals? What did that translate into during the
pandemic?

brian rosenthal

It meant that doctors and nurses have less time to spend with each
patient in public hospitals to see how they were doing, to talk with
them, to run tests, and, perhaps most importantly, just to monitor them.
Almost all of them were on ventilators and really needed to be
constantly monitored. One of the things that we've learned with the
coronavirus is that patients can detoriate very quickly. They can seem
like they're doing fine one minute, and the next minute, they could be
going into cardiac arrest. And at the understaffed public hospitals, we
even heard some cases of patients waking up from medically induced
comas, finding that there were no nurses around, and in their confusion,
actually removing their life supports and dying.

michael barbaro

Wow.

brian rosenthal

It was something that was a pattern, so much of a pattern that at
Elmhurst Hospital --- that overwhelmed hospital that received a lot of
attention --- this happens so often where somebody woke up, confused and
removed their life support because they needed to go to the bathroom.
And they collapsed and they were discovered either in the bathroom or
near the bathroom. Some of the doctors there actually developed a name
for it. They called them ``bathroom codes.'' And in those cases, the
patients were discovered, you know, half an hour later, 45 minutes later
by doctors and nurses who were devastated, because if there had been
staff there monitoring them, they would have been cared for.

michael barbaro

But instead, a nurse was doing the rounds for 15 or 20 other patients.

brian rosenthal

That's right. In every case that we heard about --- at least four cases
at Elmhurt Hospital --- the patients died.

michael barbaro

Mm-hmm. How else do the people you talk to in these hospitals tell you
that staffing impacted mortality?

brian rosenthal

Well, another example is something called ``proning,'' which is quite
simply flipping a patient on their stomach. And it was something that
very quickly, during the pandemic, doctors realized that if they did ---
if they flipped patients on their stomach --- it would help the patient
breathe and could be a useful tool in helping them recover. And so that
was something that was used a lot in New York in private hospitals, but
unfortunately, in public hospitals, there was not the staffing available
to do it.

michael barbaro

Why?

brian rosenthal

Well, it turns out that proning --- just flipping someone on their
stomach --- can actually be quite complicated if they have a bunch of
I.V. lines and tubes running through them. And it can require five or
six people to coordinate all the movements and make sure those lines are
still running while flipping the patient. So it seems very simple. And
the doctors knew that it would help. But in some of those public
hospitals, they were not able to do it because they did not have the
staff available.

michael barbaro

Mm-hmm.

brian rosenthal

One doctor at a small independent hospital told us that out of 10 of the
deaths that he witnessed, he thought two or three of the patients could
have been saved.

michael barbaro

If there had been better staffing.

brian rosenthal

Yeah, if the hospital had the resources of a private hospital.

michael barbaro

Wow. I mean, that's 20 to 30 percent

brian rosenthal

Yeah, I mean, it translates to thousands of people. And we actually
looked at the mortality rates at most of the 47 hospitals in the city.
And in some cases, the mortality rate was three times higher in the
public hospitals in the lower income areas. Some of that mortality
difference could be explained by differences in patient populations ---
you know, underlying health conditions of the patients. But the experts
and the doctors that we talked to said that the quality of care was
definitely a factor in those differences.

michael barbaro

Brian, as horrible as everything you're describing is, it feels like
there's a pretty logical solution to it. Which is taking Covid-19
patients from these overburdened, understaffed public hospitals, and
transferring them to the less burdened, better staffed private
hospitals.

brian rosenthal

You'd think that, yes. And Governor Cuomo even said at the peak of the
pandemic that that was going to happen.

\begin{itemize}
\tightlist
\item
  archived recording (andrew cuomo)\\
  How many beds would you need at the apex? Between 70 and 110,000.
  Right now, we have 53,000 statewide. We have only 36,000 downstate.
  Every hospital by mandate has to add a 50 percent increase. And they
  have all done that. We're setting up extra facilities, which ---
\end{itemize}

michael barbaro

But in the end, it didn't. And why not? Like, what prevents a patient at
Elmhurst Hospital in Queens from being transferred to N.Y.U. Langone,
which happens to be on the east side of Manhattan. It's not that far.

brian rosenthal

Well, Elmhurst is a public hospital. And for decades, they have not
really transferred patients to N.Y.U. Langone. They've transferred
patients to other hospitals within the public system, but they just
don't really work together with the private system.

michael barbaro

So there's no infrastructure set up to make such transfers. And
therefore, they're unlikely to happen.

brian rosenthal

Well, nothing physically prevents a patient from being transferred. But
first of all, the hospital, Elmhurst, may not want to transfer the
patient because there is revenue attached to every patient. Even a
public hospital cares about maximizing its revenue. So the doctor and
the nurse inside the hospital may want very much to transfer a patient
to Langone, but the administrator, the C.E.O. of the hospital, might not
want to do that for financial reasons. So there was a problem on that
end.

And then there's a problem on the other end, because N.Y.U. Langone is a
private hospital. And it wants to treat patients with private health
insurance because that's going to bring the biggest profit. And the
patient coming from Elmhurst, the public hospital, is going to be a
patient without private health insurance. So it's not a patient that
N.Y.U. Langone really wants, anyway. So on both ends, Elmhurst may not
want to transfer the patient, and N.Y.U. Langone might not want to take
the patient.

michael barbaro

So the incentives are not there for this very simple fix to work.

brian rosenthal

That's right. Because the incentive is profit.

michael barbaro

So at the end of the day, were there any transfers between the public
and private hospitals? Any meaningful number of transfers?

brian rosenthal

There were less than 50 ---

michael barbaro

Wow.

brian rosenthal

--- during the whole course of the pandemic, thousands of people in
hospitals. There were less than 50 transfers from public hospitals to
private hospitals.

michael barbaro

That is a genuinely shocking number.

brian rosenthal

Yeah, and again, the transfers were wanted by the doctors and the
nurses. But they didn't end up happening.

michael barbaro

I'm very rarely shocked.

brian rosenthal

Yeah. So that brings us to the other possible solution, which New York
explored and actually put a lot of money into, which was the overflow
hospitals --- makeshift hospitals set up around the city that could take
patients from these overburdened hospitals. But it turns out those
didn't work either.

{[}music{]}

michael barbaro

We'll be right back.

\begin{itemize}
\item
  archived recording 1\\
  Now as we all know, New York is the national epicenter of the
  coronavirus crisis. Now it is all hands on deck there.
\item
  archived recording\\
  And the death toll in New York City from the Covid-19 pandemic has
  climbed to 450, with 26,000 testing positive so far. This is the Naval
  ship Comfort due to arrive in the area on Monday from Virginia. And a
  field hospital ---
\end{itemize}

michael barbaro

Brian, I remember these overflow hospitals really well.

\begin{itemize}
\tightlist
\item
  archived recording (andrew cuomo)\\
  What we're doing here at the Javits Center is constructing four
  emergency hospitals.
\end{itemize}

michael barbaro

I remember Governor Cuomo walking through the Javits Center, this huge
convention center ---

\begin{itemize}
\tightlist
\item
  archived recording (andrew cuomo)\\
  This was never an anticipated use. But you do what you have to do.
  That's the New York way. That's the American way.
\end{itemize}

michael barbaro

--- on the west side of Manhattan, kind of showing off the hundreds of
beds. I remember there being little flowers on the sides of the tables
next to the cots. And I know these were set up in each borough. So what
happened that meant that they didn't actually do their job?

brian rosenthal

Well, let's take the example of the Billie Jean King Tennis Center.

\begin{itemize}
\tightlist
\item
  archived recording\\
  Part of the Billie Jean King National Tennis Center right now is being
  converted into a temporary hospital.
\end{itemize}

brian rosenthal

It's one of the biggest tennis centers in the world. It's where the U.S.
Open is held.

\begin{itemize}
\tightlist
\item
  archived recording\\
  Some patients from nearby Elmhurst Hospital are expected to be
  transferred to the National Tennis Center Hospital.
\end{itemize}

brian rosenthal

It was going to have 470 beds and hundreds of employees that were going
to be available to take patients, specifically from Elmhurst and Queens.

\begin{itemize}
\tightlist
\item
  archived recording\\
  This place will be a lifesaving place. It's going to help take the
  pressure off Elmhurst.
\end{itemize}

brian rosenthal

It was supposed to be a crucial facility. But the first problem that it
had was bureaucracy. There were paperwork requirements. There were all
kinds of orientations that the doctors needed to do, training on the
computer systems, training on the type of equipment that was going to be
there and the paperwork that had to be filled out. And you had doctors
in the middle of the peak of the pandemic, when people were dying,
spending time doing things that had nothing to do with patient care.

Another problem was that the hospital was suffering from a bit of an
identity crisis about which types of patients it was going to treat and
at different points of time, even within the week that it was being set
up.

\begin{itemize}
\tightlist
\item
  archived recording\\
  As of this morning, the complex was not likely to include Covid-19
  patients. The U.S. Open is ---
\end{itemize}

brian rosenthal

City officials were changing their mind about that question.

\begin{itemize}
\tightlist
\item
  archived recording\\
  This facility will be able to take people from Elmhurst, other
  coronavirus patients, bring them over here, relieve some of that
  pressure immediately.
\end{itemize}

brian rosenthal

And they were conveying different directives to other hospitals about
which types of patients they should be transferring to the Billie Jean
King Tennis Center. And they ended up crafting a series of rules that
were very restrictive about the types of patients that could go to
Billie Jean King. There were over 25 different exclusionary criteria,
which is basically disqualifying conditions that if the patient has,
they can't go to Billie Jean King. And one of them was just the fact
that the patient had a fever, which is a hallmark symptom of the
coronavirus.

But at the same time, there were also a series of rules about the types
of patients that they would not see because they were not severe enough.
They were patients that were quarantining with the virus in hotels, and,
in some cases, ended up dying in those hotels. And when employees at
Billie Jean King asked why they couldn't see and care for those
patients, they were told that those patients aren't severe enough to be
at Billie Jean King. So they couldn't see the really severe patients.
They also couldn't see the patients that were not severe. And as a
result, they didn't end up treating much of anybody.

michael barbaro

Hm. So did they see any patients?

brian rosenthal

Well, hold on, because there's another problem, and it relates to
ambulances. So in the peak of the pandemic, if you were at your house
and you called 911, the ambulance that arrived could not take you to
Billie Jean King directly.

michael barbaro

Why not?

brian rosenthal

Well, the city had decided that ambulances would have to first take
patients to a hospital, even if they're overburdened. And that hospital
would triage the patient and then figure out where to send them. So
Billie Jean King was really only taking transfers from other hospitals.
But even the transferring process was blocked by ambulance regulations.
Because there were situations where hospitals wanted to transfer
patients, but there was no ambulance available to transfer them. And
Billie Jean King had its own ambulances on site that could have gone to
the hospital and picked up the patient. But the regular hospitals had
exclusivity agreements with ambulance companies that said that nobody
could pick up their patients. They could only send patients out in their
own ambulances with these companies.

michael barbaro

And so that patient is just going to stay at Elmhurst and not get
transferred to Billie Jean King.

brian rosenthal

Until an ambulance from that company with the exclusive agreement is
available, yes. And that happened, so patients had to wait.

michael barbaro

OK, so back to that question. In the end, how many patients made it into
this Billie Jean King overflow hospital?

brian rosenthal

79.

michael barbaro

Geez.

brian rosenthal

That's 79 throughout the course of the month that the Billie Jean King
Center was open. At any one time, there were maybe 20 or 30 patients
there.

michael barbaro

So what were all the staff, the nurses, the doctors at Billie Jean King
Field Hospital, overflow hospital, what were they doing?

brian rosenthal

Well, in many cases, nothing. You know, I want to be clear, because the
doctors and nurses and other staffers that came to work at Billie Jean
King, they came, in many cases, from around the country. They were
experienced medical professionals. And they really wanted to help. And
they were extremely well paid as well. They were paid, the doctors in
many cases, over \$600 an hour.

michael barbaro

Wow.

brian rosenthal

So they showed up to work ready to help, eager to help, but no patients
came in the door. So I talked to some of them that said that in the peak
of the pandemic, they were just sitting around on their phone all day.
One of the workers at Billie Jean King who I talked with, who is a nurse
practitioner who came up from Baltimore, she said, ``I basically got
paid \$2,000 a day to sit on my phone and look at Facebook. We all felt
guilty. I felt really ashamed, to be honest.''

michael barbaro

Right, because like you said, they came to serve the public in New York.
In particular, a public that was trying to get into overburdened public
hospitals, and here they are, not able to do that because of exclusive
ambulance agreements and kind of bureaucratic nonsense.

brian rosenthal

That's right, yeah. The facility ended up closing in early May after the
peak of the pandemic. There was really no need for it. And ultimately,
for its work in treating 79 patients, so far the city has paid the
contractor about \$52 million. But the bill is actually still coming in.
The total bill might actually be over \$100 million.

michael barbaro

Brian, whenever we talk about inequality, it can feel like a very
out-of-reach set of solutions, right? Because almost by definition, it
is systemic deeply rooted issues. But in the case of hospitals in New
York, the solutions felt very practical and very simple, as you have
laid them out. You know, cancel those exclusive ambulance agreements.
Transfer patients from public to private hospitals. They all seemed
quite within reach.

brian rosenthal

Yeah, I think that's right. And I think it's also important to note that
even while the pandemic was going on, there were plenty of doctors and
other hospital workers who noticed these inequalities and were trying to
fix them. We talked with a number of doctors that actually rotated
between working in the private hospitals and working in the public
hospitals, and were trying to raise alarms, and even hospitals within
the private networks trying to push their bosses to do more to address
inequalities. But the reality was by that point, the inequalities were
so ingrained into the hospital system that there wasn't a whole lot that
they could do.

I think the story of what happened in hospitals in New York, in the
height of the coronavirus pandemic, is really a story about officials
and hospital executives and bureaucrats who accepted these inequalities
in the system long ago, and have obviously known about inequalities for
decades, but chose not to address them and found that they got exposed
in this pandemic.

michael barbaro

But of course, in that case, isn't it the role of government? Isn't it
the role of the mayor of New York City, the governor of New York, to not
accept those kinds of inequalities, and to do everything in their power
to slice through that kind of complacency in the midst of a public
health crisis?

brian rosenthal

Yeah, and I think if you talk to the governor or the mayor, if you had
them sitting here, they would say that they did as much as they could.
And they did certainly spend a lot of money setting up field hospitals
to help and set up a system to help with transfers.

But one thing that I think is very telling is when I called the
governor's office to ask why more patients were not transferred from
overburdened hospitals to private hospitals that had open beds, the
governor's office said that they accommodated every transfer that was
requested by the hospitals. And they felt like that was their job. So
they handled each request, but they were not willing to force hospitals
to transfer. They were not willing to take that more fundamental step in
changing the government's role. And I think it's because they themselves
kind of accepted the reality as it was, that there were going to be
inequalities between different types of hospitals and different types of
patients.

michael barbaro

Right, to say that they processed every request they got for transfers
is to say, like, I caught a couple of the raindrops in this giant storm,
but what about that flood down the street?

brian rosenthal

Right, it's not addressing the more fundamental problem.

michael barbaro

Brian, at the start of our conversation, you mentioned that peak
hospitalizations are now occurring throughout much of the rest of the
country. It's subsided in New York, but it's now happening in Texas.
It's happening in Florida. It's happening in Arizona.

brian rosenthal

Yes.

michael barbaro

I know that your investigation was into the hospitals in New York. But
do we expect that what you saw in New York --- these inequities, these
private-public hospital disparities --- that they are likely to play out
across the rest of the country?

brian rosenthal

There will definitely be disparities in every city in America. I think
the question is whether other cities have learned from New York and are
going to be willing to put in place systems and policies that can help
balance out those inequalities in a more real way than we saw in New
York. And I think that's still to be determined.

michael barbaro

Brian, thank you very much. We appreciate it.

brian rosenthal

Thank you.

{[}music{]}

michael barbaro

On Sunday, The Times reported that the total number of infections in
Florida has now surpassed that of New York, making the state the new
epicenter of the pandemic. Florida has nearly 424,000 reported cases,
compared with about 415,000 cases in New York. We'll be right back.

Here's what else you need to know today.

The Times reports that the presence of federal agents in Portland
galvanized thousands of people to join protests across the country over
the weekend, reviving nationwide protests that had largely dissipated.

\begin{itemize}
\tightlist
\item
  archived recording\\
  Black lives matter! {[}CAR HONKING{]} Black lives matter! {[}CAR
  HONKING{]}
\end{itemize}

michael barbaro

One of the most intense protests was in Seattle, where a demonstration
against police brutality turned violent, after some protesters lit a
detention center on fire, smashed windows and damaged a police building.

In response, police declared the protest a riot, fired flash grenades,
unleashed pepper spray and rushed into crowds, knocking people to the
ground.

That's it for ``The Daily.'' I'm Michael Barbaro. See you tomorrow.

Inequality did not arrive with the virus; the divide between the haves
and the have-nots has long been a part of the web of hospitals in the
city.

Manhattan is home to several of the world's most prestigious medical
centers, a constellation of academic institutions that attract wealthy
residents with private health insurance. The other boroughs are served
by a patchwork of satellite campuses, city-run public hospitals and
independent facilities, all of which treat more residents on Medicaid or
Medicare, or without insurance.

The pandemic exposed and amplified the inequities, especially during the
peak, according to doctors, nurses and other workers.

Overall, more than
17,500\href{https://www.nytimes3xbfgragh.onion/interactive/2020/nyregion/new-york-city-coronavirus-cases.html}{people
have been confirmed to have died} in New York City of Covid-19, the
illness caused by the coronavirus. More than 11,500 lived in ZIP codes
with median household incomes below the city median, according to city
data.

Deaths have slowed, but with the possibility of a second surge looming,
doctors are examining the disparities.

At the NewYork-Presbyterian Hospital, the city's largest private
hospital network, 20 doctors drafted a letter in April warning
leadership about inequalities, according to a copy obtained by The
Times. The doctors had found that the mortality rate at an understaffed
satellite was more than twice as high as at a flagship center, despite
not treating sicker patients.

At NYU Langone Health, another large network, 43 medical residents wrote
their own letter to the chief medical officer expressing concerns about
disparities in hospital care.

Both networks said in statements that they deliver the same level of
care at all their locations.

Image

At NYU Langone's flagship hospital in Manhattan, about 11 percent of
coronavirus patients have died, according to data obtained by The
Times.Credit...Jonah Markowitz for The New York Times

Gov. Andrew M. Cuomo and Mayor Bill de Blasio have spoken throughout the
pandemic of adding hospital beds across the city, transferring patients
and sending supplies and workers to community hospitals, implying that
they have ensured all New Yorkers with Covid-19 receive the same quality
care.

``We are one health care system,'' Mr. Cuomo said on March 31. The same
day, he described the coronavirus as ``the great equalizer.''

In interviews, doctors scoffed at that notion, noting, among other
issues, that government reinforcements
\href{https://www.nytimes3xbfgragh.onion/2020/04/08/nyregion/coronavirus-new-york-volunteers.html}{were
slowed by bureaucratic hurdles} and mostly arrived after the peak.

\hypertarget{latest-updates-global-coronavirus-outbreak}{%
\section{\texorpdfstring{\href{https://www.nytimes3xbfgragh.onion/2020/08/04/world/coronavirus-cases.html?action=click\&pgtype=Article\&state=default\&region=MAIN_CONTENT_1\&context=storylines_live_updates}{Latest
Updates: Global Coronavirus
Outbreak}}{Latest Updates: Global Coronavirus Outbreak}}\label{latest-updates-global-coronavirus-outbreak}}

Updated 2020-08-04T19:32:24.665Z

\begin{itemize}
\tightlist
\item
  \href{https://www.nytimes3xbfgragh.onion/2020/08/04/world/coronavirus-cases.html?action=click\&pgtype=Article\&state=default\&region=MAIN_CONTENT_1\&context=storylines_live_updates\#link-4825b93}{Public
  and private schools in Maryland and elsewhere are divided over
  in-person instruction.}
\item
  \href{https://www.nytimes3xbfgragh.onion/2020/08/04/world/coronavirus-cases.html?action=click\&pgtype=Article\&state=default\&region=MAIN_CONTENT_1\&context=storylines_live_updates\#link-4d1eafa8}{N.Y.C.'s
  health commissioner resigns after clashing with the mayor over the
  virus.}
\item
  \href{https://www.nytimes3xbfgragh.onion/2020/08/04/world/coronavirus-cases.html?action=click\&pgtype=Article\&state=default\&region=MAIN_CONTENT_1\&context=storylines_live_updates\#link-6b644638}{`Long
  days, long nights': Washington prepares for a prolonged fight over
  virus relief.}
\end{itemize}

\href{https://www.nytimes3xbfgragh.onion/2020/08/04/world/coronavirus-cases.html?action=click\&pgtype=Article\&state=default\&region=MAIN_CONTENT_1\&context=storylines_live_updates}{See
more updates}

More live coverage:
\href{https://www.nytimes3xbfgragh.onion/live/2020/08/04/business/stock-market-today-coronavirus?action=click\&pgtype=Article\&state=default\&region=MAIN_CONTENT_1\&context=storylines_live_updates}{Markets}

``There was no cavalry,'' said Dr. Ralph Madeb, surgery director at the
independent New York Community Hospital in Brooklyn. ``Everything we did
was on our own.''

In a statement, Dani Lever, the governor's communications director, said
Mr. Cuomo has repeatedly pointed out inequalities in health care. The
state worked during the peak to transfer patients so everybody could at
least access care, she said.

``The governor said Covid was the `great equalizer' in that it infected
anyone regardless of race, age, etc. --- not that everyone would receive
the same the level of health care,'' she said. ``The governor said we
are one health system in terms of ensuring that everyone who needed it
had access to a hospital.''

A spokeswoman for Mr. de Blasio, Avery Cohen, said the mayor agreed that
the pandemic had exposed inequalities, and the city had worked to
address disparities.

``From nearly tripling hospital capacity at the virus' peak, to creating
a massive testing apparatus from the ground-up, we have channeled every
possible resource into helping our most vulnerable and remain undeterred
in doing so,'' she said.

\begin{center}\rule{0.5\linewidth}{\linethickness}\end{center}

Image

Weill Cornell, part of the NewYork-Presbyterian system, is one of the
more prestigious health care facilities in New York
City.~Credit...Stephen Speranza for The New York Times

New York has never had a unified hospital system. It has several
different hospital systems, and in recent years, they have consolidated
and contracted, through mergers and more than a dozen hospital closures.

Today, most beds in the city are in hospitals in five private networks.
NewYork-Presbyterian, which has Weill Cornell Medical Center and
Columbia University Irving Medical Center; NYU Langone; the Mount Sinai
Health System; Northwell Health; and the Montefiore Medical Center.

These networks are wealthy nonprofits aided by decades of government
policies that have steered money to them. They also rake in revenue
because, on average, two-thirds of their patients are on Medicare or
have commercial insurance, through their employer or purchased
privately.

Collectively, they annually spend \$150 million on advertising and pay
their chief executives \$30 million, records show. They also pay their
doctors the most, and score the highest marks on industry ratings
regarding safety, mortality and patient satisfaction.

The city has 11 public hospitals. This is
\href{https://www.nytimes3xbfgragh.onion/2020/04/26/nyregion/coronavirus-new-york-university-hospital.html}{the
city's safety net}, along with the private networks' satellite campuses
and a shrinking number of smaller independent hospitals, which have been
financially struggling for years.

At the safety-net hospitals, only 10 percent of patients have private
insurance. The hospitals provide all the basic services but often have
to transfer patients for specialty care.

Most of the private networks are based at expansive campuses in
Manhattan, as are some of the public hospitals. (Montefiore is based in
the Bronx; many of Northwell's hospitals are outside of New York City.)

The hospital closures have largely been outside of Manhattan, including
three beloved safety-net hospitals in Queens in just a few months in
2008 and 2009.

There are now five hospital beds for every 1,000 residents in Manhattan,
while only 1.8 per 1,000 residents in Queens, 2.2 in Brooklyn and 2.4 in
the Bronx, according to government data.

Yet in a cruel twist, the coronavirus has mostly clobbered areas outside
of Manhattan.

Manhattan has only had 16 confirmed cases for every 1,000 residents,
while there have been 28 per 1,000 residents in Queens, 23 in Brooklyn
and 33 in the Bronx,
\href{https://www1.nyc.gov/site/doh/covid/covid-19-data.page}{the latest
count} shows.

Image

Over all, more than 17,500 people have been confirmed to have died in
New York City of Covid-19.~Credit...Erin Schaff/The New York Times

These areas have lower median incomes --- \$38,000 in the Bronx versus
\$82,000 in Manhattan --- and are filled with residents whose jobs have
put them at higher risk of infection.

``Certain hospitals are located in the heart of a pandemic that hit on
top of an epidemic of poverty and stress and pollution and segregation
and racism,'' said Dr. Carol Horowitz, director of the Institute for
Health Equity Research at Mount Sinai.

At the pandemic's peak, ambulances generally took patients to the
nearest hospital --- not the one with the most capacity. That
contributed to crushing surges in hospitals in areas with high infection
rates, overwhelming some hospitals and reducing their ability to care
for patients.

In Manhattan, where
\href{https://www.nytimes3xbfgragh.onion/interactive/2020/05/15/upshot/who-left-new-york-coronavirus.html}{many
residents fled the city}, hospitals also found patient release valves.
Mount Sinai sent hundreds to a
\href{https://www.nytimes3xbfgragh.onion/2020/04/15/nyregion/coronavirus-central-park-hospital-tent.html}{Central
Park tent hospital}. NewYork-Presbyterian sent 150 to the Hospital for
Special Surgery.

In all, the census at some emergency rooms actually declined.

At Lenox Hill Hospital, a private hospital on the Upper East Side, Dr.
Andrew Bauerschmidt said on April 8 --- near the city's peak in cases
--- that the hospital had more patients than usual, but not by much.

``Nothing dire is going on here, like the stories we've heard at other
hospitals,'' he said.

\begin{center}\rule{0.5\linewidth}{\linethickness}\end{center}

Image

People waited for hours in long lines outside of Elmhurst Hospital
Center in Queens to be tested for coronavirus in late
March.Credit...Dave Sanders for The New York Times

Image

Personnel at Mount Sinai Hospital in Manhattan experienced a more calm
atmosphere compared to facilities in the other boroughs.Credit...Hilary
Swift for The New York Times

After weeks battling the virus at the Elmhurst Hospital Center, a
\href{https://www.nytimes3xbfgragh.onion/2020/03/25/nyregion/nyc-coronavirus-hospitals.html}{public
hospital in Queens that was overwhelmed by Covid-19 deaths}, Dr. Ravi
Katari took a shift at the Mount Sinai Hospital.

When he arrived at the towering campus just east of Central Park, he was
surprised to see fewer patients and more workers than at Elmhurst, and a
sense of calm.

Dr. Katari was a fourth-year emergency medicine resident in a program
run by Mount Sinai that rotates residents through different hospitals,
to give them varied experiences.

In interviews, seven of these residents described vast disparities
during the pandemic --- especially in staffing levels.

At the height of the crisis, doctors and nurses at every hospital had to
care for more patients than normal. But at the safety-net hospitals,
which could not deploy large numbers of specialists or students, or
quickly hire workers, patient-to-staff ratios spiraled out of control.

In the emergency room, where best practices call for a maximum of four
patients per nurse, the ratio hit 23 to 1 at Queens Hospital Center and
15 to 1 at Jacobi Medical Center in the Bronx, both public hospitals,
and 20 to 1 at Kingsbrook Jewish Medical Center, an independent facility
in Brooklyn, workers said.

``We could not care for the number of the patients we had,'' said
Feyoneisha McGrath, a nurse at Kingsbrook. ``I worked 16 hours a day,
and then I got in my car and cried.''

Image

Nurses at Kingsbrook Jewish Medical Center said the ratio of nurses to
patients was~20 to 1 during the peak of the coronavirus
epidemic.~Credit...Demetrius Freeman for The New York Times

In intensive-care units, where patients require such close monitoring
that the standard ratio is just two patients per nurse, ratios
quadrupled at some hospitals, including at Interfaith Medical Center in
Brooklyn, an independent facility, and at NewYork-Presbyterian's
satellite in Queens, workers said.

The city's public hospital system disputed those ratios cited by
workers. It added that during the pandemic, it recruited thousands of
nurses and streamlined monitoring, including by opening doors to patient
rooms. The chief executive of Kingsbrook and Interfaith also disputed
the ratios at those hospitals.

\href{https://www.ncbi.nlm.nih.gov/pubmed/29110907}{Research}
\href{https://psnet.ahrq.gov/primer/nursing-and-patient-safety}{has}
\href{https://patientengagementhit.com/news/how-nurse-staffing-ratios-impact-patient-safety-access-to-care}{shown}
that staffing levels affect mortality, and that may be even more true
during this pandemic because many Covid-19 patients quickly deteriorate
without warning.

At Brookdale, the independent hospital, three doctors said that many
patients on ventilators had to remain for days or weeks in understaffed
wards because the intensive-care unit was full. Amid shortages in
sedatives, some patients awoke from comas alone and, in a reflexive
response, removed the tubes in their airways that were keeping them
breathing. Alarms rang, and staff rushed to reintubate the patients. But
they all eventually died, the doctors said.

A hospital spokesman, Khari Edwards, said, ``Protocols for sedation of
intubated patients are in place and are monitored by our quality
improvement processes.''

Similar episodes occurred at Kingsbrook, the Queens Hospital Center and
the Allen Hospital, a NewYork-Presbyterian hospital in Northern
Manhattan, according to workers.

Dr. Dawn Maldonado, a resident doctor at Elmhurst, described a worrisome
pattern of deaths on its understaffed general medicine floors. She said
at least four patients collapsed after removing their oxygen masks to
try to walk to the bathroom. Workers discovered their bodies later ---
in one case, as much as 45 minutes later --- in the bathroom or nearby.

``We'd call them bathroom codes,'' Dr. Maldonado said.

\begin{center}\rule{0.5\linewidth}{\linethickness}\end{center}

Image

Lincoln Medical and Mental Health Center in the Bronx had to
increasingly use portable ventilators sent by the state to treat
patients during the virus peak.~Credit...Todd Heisler/The New York Times

As the coronavirus raged, Lincoln Medical and Mental Health Center in
the Bronx kept running into problems with ventilators.

Lincoln, a public hospital, had a limited number, and it could not
acquire many more, so it had to increasingly use portable ventilators
sent by the state. The machines did not have some settings to adjust to
patients' breathing, including a high-pressure mode called ``airway
pressure release ventilation.''

\href{https://www.nytimes3xbfgragh.onion/news-event/coronavirus?action=click\&pgtype=Article\&state=default\&region=MAIN_CONTENT_3\&context=storylines_faq}{}

\hypertarget{the-coronavirus-outbreak-}{%
\subsubsection{The Coronavirus Outbreak
›}\label{the-coronavirus-outbreak-}}

\hypertarget{frequently-asked-questions}{%
\paragraph{Frequently Asked
Questions}\label{frequently-asked-questions}}

Updated August 4, 2020

\begin{itemize}
\item ~
  \hypertarget{i-have-antibodies-am-i-now-immune}{%
  \paragraph{I have antibodies. Am I now
  immune?}\label{i-have-antibodies-am-i-now-immune}}

  \begin{itemize}
  \tightlist
  \item
    As of right
    now,\href{https://www.nytimes3xbfgragh.onion/2020/07/22/health/covid-antibodies-herd-immunity.html?action=click\&pgtype=Article\&state=default\&region=MAIN_CONTENT_3\&context=storylines_faq}{that
    seems likely, for at least several months.} There have been
    frightening accounts of people suffering what seems to be a second
    bout of Covid-19. But experts say these patients may have a
    drawn-out course of infection, with the virus taking a slow toll
    weeks to months after initial exposure. People infected with the
    coronavirus typically
    \href{https://www.nature.com/articles/s41586-020-2456-9}{produce}
    immune molecules called antibodies, which are
    \href{https://www.nytimes3xbfgragh.onion/2020/05/07/health/coronavirus-antibody-prevalence.html?action=click\&pgtype=Article\&state=default\&region=MAIN_CONTENT_3\&context=storylines_faq}{protective
    proteins made in response to an
    infection}\href{https://www.nytimes3xbfgragh.onion/2020/05/07/health/coronavirus-antibody-prevalence.html?action=click\&pgtype=Article\&state=default\&region=MAIN_CONTENT_3\&context=storylines_faq}{.
    These antibodies may} last in the body
    \href{https://www.nature.com/articles/s41591-020-0965-6}{only two to
    three months}, which may seem worrisome, but that's perfectly normal
    after an acute infection subsides, said Dr. Michael Mina, an
    immunologist at Harvard University. It may be possible to get the
    coronavirus again, but it's highly unlikely that it would be
    possible in a short window of time from initial infection or make
    people sicker the second time.
  \end{itemize}
\item ~
  \hypertarget{im-a-small-business-owner-can-i-get-relief}{%
  \paragraph{I'm a small-business owner. Can I get
  relief?}\label{im-a-small-business-owner-can-i-get-relief}}

  \begin{itemize}
  \tightlist
  \item
    The
    \href{https://www.nytimes3xbfgragh.onion/article/small-business-loans-stimulus-grants-freelancers-coronavirus.html?action=click\&pgtype=Article\&state=default\&region=MAIN_CONTENT_3\&context=storylines_faq}{stimulus
    bills enacted in March} offer help for the millions of American
    small businesses. Those eligible for aid are businesses and
    nonprofit organizations with fewer than 500 workers, including sole
    proprietorships, independent contractors and freelancers. Some
    larger companies in some industries are also eligible. The help
    being offered, which is being managed by the Small Business
    Administration, includes the Paycheck Protection Program and the
    Economic Injury Disaster Loan program. But lots of folks have
    \href{https://www.nytimes3xbfgragh.onion/interactive/2020/05/07/business/small-business-loans-coronavirus.html?action=click\&pgtype=Article\&state=default\&region=MAIN_CONTENT_3\&context=storylines_faq}{not
    yet seen payouts.} Even those who have received help are confused:
    The rules are draconian, and some are stuck sitting on
    \href{https://www.nytimes3xbfgragh.onion/2020/05/02/business/economy/loans-coronavirus-small-business.html?action=click\&pgtype=Article\&state=default\&region=MAIN_CONTENT_3\&context=storylines_faq}{money
    they don't know how to use.} Many small-business owners are getting
    less than they expected or
    \href{https://www.nytimes3xbfgragh.onion/2020/06/10/business/Small-business-loans-ppp.html?action=click\&pgtype=Article\&state=default\&region=MAIN_CONTENT_3\&context=storylines_faq}{not
    hearing anything at all.}
  \end{itemize}
\item ~
  \hypertarget{what-are-my-rights-if-i-am-worried-about-going-back-to-work}{%
  \paragraph{What are my rights if I am worried about going back to
  work?}\label{what-are-my-rights-if-i-am-worried-about-going-back-to-work}}

  \begin{itemize}
  \tightlist
  \item
    Employers have to provide
    \href{https://www.osha.gov/SLTC/covid-19/standards.html}{a safe
    workplace} with policies that protect everyone equally.
    \href{https://www.nytimes3xbfgragh.onion/article/coronavirus-money-unemployment.html?action=click\&pgtype=Article\&state=default\&region=MAIN_CONTENT_3\&context=storylines_faq}{And
    if one of your co-workers tests positive for the coronavirus, the
    C.D.C.} has said that
    \href{https://www.cdc.gov/coronavirus/2019-ncov/community/guidance-business-response.html}{employers
    should tell their employees} -\/- without giving you the sick
    employee's name -\/- that they may have been exposed to the virus.
  \end{itemize}
\item ~
  \hypertarget{should-i-refinance-my-mortgage}{%
  \paragraph{Should I refinance my
  mortgage?}\label{should-i-refinance-my-mortgage}}

  \begin{itemize}
  \tightlist
  \item
    \href{https://www.nytimes3xbfgragh.onion/article/coronavirus-money-unemployment.html?action=click\&pgtype=Article\&state=default\&region=MAIN_CONTENT_3\&context=storylines_faq}{It
    could be a good idea,} because mortgage rates have
    \href{https://www.nytimes3xbfgragh.onion/2020/07/16/business/mortgage-rates-below-3-percent.html?action=click\&pgtype=Article\&state=default\&region=MAIN_CONTENT_3\&context=storylines_faq}{never
    been lower.} Refinancing requests have pushed mortgage applications
    to some of the highest levels since 2008, so be prepared to get in
    line. But defaults are also up, so if you're thinking about buying a
    home, be aware that some lenders have tightened their standards.
  \end{itemize}
\item ~
  \hypertarget{what-is-school-going-to-look-like-in-september}{%
  \paragraph{What is school going to look like in
  September?}\label{what-is-school-going-to-look-like-in-september}}

  \begin{itemize}
  \tightlist
  \item
    It is unlikely that many schools will return to a normal schedule
    this fall, requiring the grind of
    \href{https://www.nytimes3xbfgragh.onion/2020/06/05/us/coronavirus-education-lost-learning.html?action=click\&pgtype=Article\&state=default\&region=MAIN_CONTENT_3\&context=storylines_faq}{online
    learning},
    \href{https://www.nytimes3xbfgragh.onion/2020/05/29/us/coronavirus-child-care-centers.html?action=click\&pgtype=Article\&state=default\&region=MAIN_CONTENT_3\&context=storylines_faq}{makeshift
    child care} and
    \href{https://www.nytimes3xbfgragh.onion/2020/06/03/business/economy/coronavirus-working-women.html?action=click\&pgtype=Article\&state=default\&region=MAIN_CONTENT_3\&context=storylines_faq}{stunted
    workdays} to continue. California's two largest public school
    districts --- Los Angeles and San Diego --- said on July 13, that
    \href{https://www.nytimes3xbfgragh.onion/2020/07/13/us/lausd-san-diego-school-reopening.html?action=click\&pgtype=Article\&state=default\&region=MAIN_CONTENT_3\&context=storylines_faq}{instruction
    will be remote-only in the fall}, citing concerns that surging
    coronavirus infections in their areas pose too dire a risk for
    students and teachers. Together, the two districts enroll some
    825,000 students. They are the largest in the country so far to
    abandon plans for even a partial physical return to classrooms when
    they reopen in August. For other districts, the solution won't be an
    all-or-nothing approach.
    \href{https://bioethics.jhu.edu/research-and-outreach/projects/eschool-initiative/school-policy-tracker/}{Many
    systems}, including the nation's largest, New York City, are
    devising
    \href{https://www.nytimes3xbfgragh.onion/2020/06/26/us/coronavirus-schools-reopen-fall.html?action=click\&pgtype=Article\&state=default\&region=MAIN_CONTENT_3\&context=storylines_faq}{hybrid
    plans} that involve spending some days in classrooms and other days
    online. There's no national policy on this yet, so check with your
    municipal school system regularly to see what is happening in your
    community.
  \end{itemize}
\end{itemize}

Virtually every hospital had to use some old ventilators. But at
hospitals like Lincoln, almost all patients received emergency machines,
doctors said.

Safety-net hospitals also ran
\href{https://www.nytimes3xbfgragh.onion/2020/04/18/health/kidney-dialysis-coronavirus.html}{low
on dialysis machines}, for patients with kidney damage. Many independent
hospitals could not provide continuous dialysis even before the
pandemic. At the peak, some facilities like
\href{https://www.nytimes3xbfgragh.onion/video/us/100000007097093/coronavirus-st-johns-hospital-far-rockaway.html}{St.
John's Episcopal Hospital} in Queens had to restrict dialysis even
further, providing only a couple hours at a time or
\href{https://www.nytimes3xbfgragh.onion/2020/05/01/health/coronavirus-dialysis-death.html}{not
providing any to some patients}.

While many interventions for Covid-19 are routine, like supplying oxygen
through masks, safety-net hospital patients also have not had much
access to
\href{https://www.nytimes3xbfgragh.onion/2020/04/26/health/coronavirus-patient-ventilator.html}{advanced
treatments}, including a heart-lung bypass called extracorporeal
membrane oxygenation, or E.C.M.O.

For weeks, many independent hospitals did not have
\href{https://www.nytimes3xbfgragh.onion/2020/04/29/health/gilead-remdesivir-coronavirus.html}{remdesivir,
the experimental anti-viral drug} that has been used to treat Covid-19.

``We are not anybody's priority,'' said Dr. Josh Rosenberg of the
Brooklyn Hospital Center, a 175-year-old independent facility that took
longer than others to gain entry to a clinical trial that provided
access to the drug.

Image

For weeks, many hospitals, including~Brooklyn Hospital Center, did not
have remdesivir, the experimental antiviral drug that has been used to
treat Covid-19.Credit...Victor J. Blue for The New York Times

Historically, safety-net hospitals have not been chosen for many drug
trials.

Dr. Mangala Narasimhan, a regional director of critical care at
Northwell, said just participating in a trial affects outcomes,
regardless of whether the drug works.

``You're super attentive to those patients,'' she said. ``That is an
effect in itself.''

Some low-income patients have even missed the most basic of treatment
strategies, like being turned onto their stomach. The technique, called
proning, has helped many patients breathe, but because it requires
several workers to keep IV lines untangled, some safety-net hospitals
have been unable to provide it.

Many private centers have beds that automatically turn.

\begin{center}\rule{0.5\linewidth}{\linethickness}\end{center}

Image

A patient is transferred from NYU Langone Hospital.Credit...Hiroko
Masuike/The New York Times

Image

At Bellevue Hospital, a doctor gets ready to perform a coronavirus test
inside a makeshift hospital room.Credit...Brittainy Newman/The New York
Times

Near the corner of 1st Avenue and East 30th Street in Manhattan sit two
hospital campuses that illustrate the disparities on the most tragic of
measures: mortality rate.

One is NYU Langone's flagship hospital. So far, about 11 percent of its
coronavirus patients have died, according to data obtained by The Times.
The other is Bellevue Hospital Center, the city's most renowned public
hospital, where about 22 percent of virus patients have died.

In other parts of the city, the gaps are even wider.

Overall, about one in five coronavirus patients in New York City
hospitals has died, according to a Times data analysis. At some
prestigious medical centers, it has been as low as one in 10. At some
community hospitals outside Manhattan, it has been one in three, or
worse.

Many factors have affected those numbers, including the severity of the
patients' illnesses, the extent of their exposure to the virus, their
underlying conditions, how long they waited to go to the hospital and
whether their hospital transferred healthier patients, or sicker
patients.

Still, experts and doctors agreed that disparities in hospital care have
played a role, too.

``It's hard to look at the data and come to any other conclusion,'' said
Mary T. Bassett, who led the New York City Department of Health and
Mental Hygiene from 2014 until 2018 before joining Harvard University's
School of Public Health. ``We've known for a long time that these
institutions are under-resourced. The answer should be to give them more
resources.''

The data was obtained from a number of sources, including government
agencies and the individual hospital systems.

Many of the sharpest disparities have occurred between hospitals in the
same network.

At Mount Sinai, about 17 percent of patients at its flagship Manhattan
hospital have died, a much lower rate than at its campuses in Brooklyn
(34 percent) and Queens (33 percent).

Dr. David Reich, chief executive at the Mount Sinai Hospital and the
Queens site, said the satellites were located near nursing homes and
transferred out some of their healthy patients, making their numbers
look worse. But he added that he was not surprised that large hospitals
with more specialists had better mortality rates.

There have even been differences within the public system, where most
hospitals have had mortality rates far higher than Bellevue's.

At the Coney Island Hospital, 363 patients have died --- 41 percent of
those admitted.

In an interview, Dr. Mitchell H. Katz, the head of the public system,
strenuously objected to the use of raw mortality data, saying it was
meaningless if not adjusted for patient conditions. He agreed public
hospitals needed more resources, but he defended their performance in
the pandemic.

``I'm not going to say that the quality of care that people got at my 11
hospitals wasn't as good or better as what people got at other
hospitals,'' he said. ``Our hospitals worked heroically to keep people
alive.''

\begin{center}\rule{0.5\linewidth}{\linethickness}\end{center}

Image

At NYU Langone Health, one of the city's large hospital networks, 43
medical residents wrote a letter to the chief medical officer expressing
concerns about disparities in hospital care.Credit...Demetrius Freeman
for The New York Times

On April 17, NYU Langone employees received an email that quoted
President Trump praising the network's response to Covid-19: ``I've
heard that you guys at NYU Langone are doing really great things.''

The email, from Dr. Fritz François, the network's chief medical officer,
infuriated residents who had worked at both NYU Langone and Bellevue.
They believed that if the private network was doing great, it was
because of donors and government policies letting it get more funding.

``When given the ear of the arguably most powerful man in the world ---
who has control over essential allocation of resources and government
funding --- it is a physician's duty to take this opportunity and to
advocate for the resources that all patients need,'' they responded.

At the same time, another conversation was happening. It began in late
March, when doctors at the Lower Manhattan Hospital concluded their
mortality rate for Covid-19 patients was more than twice the rate at
Weill Cornell, a prestigious hospital in its same network,
NewYork-Presbyterian.

The doctors saw an alarming potential explanation. In a draft letter
dated April 11, they said their nurses cared for up to five critically
ill patients, while Weill Cornell nurses had two or three. They also
noted staffing shortages at the Allen Hospital and NewYork-Presbyterian
Queens.

``What hospital a patient goes to (or that E.M.S. takes them to) should
not be a choice that increases adverse outcomes, including mortality,''
the draft letter said.

Image

Collectively, the hospital networks in New York City annually spend
\$150 million on advertising and pay their chief executives \$30
million, records show. Credit...Erin Schaff/The New York Times

It is unclear if the doctors sent the letter. But in mid April, network
leaders sent more staff to the Lower Manhattan Hospital, and that gap
narrowed.

Another group of network doctors undertook a deeper study and found that
some of the gap was explained by differences in the ages of patients and
their health conditions. But even after controlling for those issues,
they found a disparity, and they vowed to study it further.

In a statement, the network denied that any nurses had to monitor five
critically ill patients. ``Short-term, raw data snapshots do not show an
accurate or full picture of the entire crisis,'' it said.

Still, one doctor who works at both hospitals said he was disturbed by
one episode during the peak at the Lower Manhattan Hospital.

The doctor, who spoke on condition of anonymity because he had been
warned against talking to reporters, recalled he had three patients who
needed to be intubated. When he called the intensive-care unit, he was
told there was only space for one.

One man was in his mid-40s, younger than the other two, who were both
over 70.

``Everyone looked bad, but he had the best chance,'' the doctor said.
``The others had to wait.''

The doctor said he did not know what happened to the patients after he
left work. Given the high mortality rate at the hospital, he said he was
reluctant to look it up.

``What good is it going to do me, to know what happened?'' he said.

Lindsey Rogers Cook, Elaine Chen, Michael Rothfeld and Nicole Hong
contributed reporting. Susan C. Beachy contributed research.

Advertisement

\protect\hyperlink{after-bottom}{Continue reading the main story}

\hypertarget{site-index}{%
\subsection{Site Index}\label{site-index}}

\hypertarget{site-information-navigation}{%
\subsection{Site Information
Navigation}\label{site-information-navigation}}

\begin{itemize}
\tightlist
\item
  \href{https://help.nytimes3xbfgragh.onion/hc/en-us/articles/115014792127-Copyright-notice}{©~2020~The
  New York Times Company}
\end{itemize}

\begin{itemize}
\tightlist
\item
  \href{https://www.nytco.com/}{NYTCo}
\item
  \href{https://help.nytimes3xbfgragh.onion/hc/en-us/articles/115015385887-Contact-Us}{Contact
  Us}
\item
  \href{https://www.nytco.com/careers/}{Work with us}
\item
  \href{https://nytmediakit.com/}{Advertise}
\item
  \href{http://www.tbrandstudio.com/}{T Brand Studio}
\item
  \href{https://www.nytimes3xbfgragh.onion/privacy/cookie-policy\#how-do-i-manage-trackers}{Your
  Ad Choices}
\item
  \href{https://www.nytimes3xbfgragh.onion/privacy}{Privacy}
\item
  \href{https://help.nytimes3xbfgragh.onion/hc/en-us/articles/115014893428-Terms-of-service}{Terms
  of Service}
\item
  \href{https://help.nytimes3xbfgragh.onion/hc/en-us/articles/115014893968-Terms-of-sale}{Terms
  of Sale}
\item
  \href{https://spiderbites.nytimes3xbfgragh.onion}{Site Map}
\item
  \href{https://help.nytimes3xbfgragh.onion/hc/en-us}{Help}
\item
  \href{https://www.nytimes3xbfgragh.onion/subscription?campaignId=37WXW}{Subscriptions}
\end{itemize}
