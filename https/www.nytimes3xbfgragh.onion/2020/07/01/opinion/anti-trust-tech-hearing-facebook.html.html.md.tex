Sections

SEARCH

\protect\hyperlink{site-content}{Skip to
content}\protect\hyperlink{site-index}{Skip to site index}

\href{https://myaccount.nytimes3xbfgragh.onion/auth/login?response_type=cookie\&client_id=vi}{}

\href{https://www.nytimes3xbfgragh.onion/section/todayspaper}{Today's
Paper}

\href{/section/opinion}{Opinion}\textbar{}Here Come the 4 Horsemen of
the Techopolypse

\url{https://nyti.ms/2YOzYFu}

\begin{itemize}
\item
\item
\item
\item
\item
\item
\end{itemize}

Advertisement

\protect\hyperlink{after-top}{Continue reading the main story}

\href{/section/opinion}{Opinion}

Supported by

\protect\hyperlink{after-sponsor}{Continue reading the main story}

\hypertarget{here-come-the-4-horsemen-of-the-techopolypse}{%
\section{Here Come the 4 Horsemen of the
Techopolypse}\label{here-come-the-4-horsemen-of-the-techopolypse}}

What will happen when the leaders of Apple, Google, Facebook and Amazon
appear before Congress? We are about to find out.

\includegraphics{https://static01.graylady3jvrrxbe.onion/images/2018/08/02/opinion/02swisher/02swisher-thumbLarge.png}

By Kara Swisher

Contributing opinion writer

\begin{itemize}
\item
  July 1, 2020
\item
  \begin{itemize}
  \item
  \item
  \item
  \item
  \item
  \item
  \end{itemize}
\end{itemize}

\includegraphics{https://static01.graylady3jvrrxbe.onion/images/2020/07/01/opinion/01swisher/merlin_163192443_4124fdf6-cdcc-4245-ab90-4d72478445bd-articleLarge.jpg?quality=75\&auto=webp\&disable=upscale}

You can call it Techpalooza.

The chief executives of four of the most powerful tech companies in the
world --- Apple, Facebook, Google and Amazon --- have agreed to appear
in late July before a congressional committee as part of an
investigation focused on antitrust.

Representative David Cicilline, a Democrat from Rhode Island who has
become one of the biggest critics of Big Tech's enormous power, told me
on Wednesday that Jeff Bezos of Amazon, Mark Zuckerberg of Facebook,
Sundar Pichai of Google and Tim Cook of Apple will testify in what could
be an all-day event.

Mr. Cicilline said that the yearlong congressional investigation has
included eight round-table discussions, 93 requests for information, 43
experts testifying and five hearings.

``It's the first major look at antitrust in this industry in 50 years
and a lot of people worldwide are watching how lawmakers deal with
tech,'' he said. ``But throughout, we know it is impossible to properly
conclude this without hearing from the decision makers themselves.''

He said that all of the chief executives agreed to appear voluntarily
and that logistics are still being worked out for what he hopes will be
an in-person hearing in Washington. But safety concerns over the
coronavirus may mean that the executives end up testifying remotely.

And while tech leaders have appeared before Congress in the past --- and
there is often less illumination than noise at these kinds of hearings
--- given the growing public alarm about the power of the tech giants,
this gathering of the four horsemen of the Techopolypse could be an epic
show. Winter may be coming for Silicon Valley.

Mr. Zuckerberg, Mr. Pichai and Mr. Bezos had indicated in letters
previously that they were open to appearing at a hearing, after being
invited by Mr. Cicilline's antitrust subcommittee, while Apple had said
that it was open to sending a top official.

It's clear that the chief executives wanted to appear together, not so
much for support --- frenemies is about as close as I would describe
them, and there is intense dislike between some of the companies --- but
in the hopes that a group appearance will keep any one of them from
being singled out for intense scrutiny. Some are suggesting that a
multiday interrogation, with each chief executive facing a small number
of experienced questioners, as well as real people they hurt, would be a
better way to grill the tech moguls.

Still, if the lawmakers do their job in the planned format and ask
pointed questions about the true impact of these companies' power on
competition, there could be some important moments.

This hearing will be part of a wider bipartisan inquiry into how the
tech giants dominate the digital industry and hurt rivals and consumers.

Along with fines, politicians and regulators are contemplating new laws
on privacy and competition, the repeal of a law that gives platforms
broad immunity for content on their sites, and, perhaps most
drastically, breaking them up.

And while each company has different problems --- such as a damaging
role in the spread of disinformation and hate speech (Hello, Facebook!)
--- the near monopolistic power of their services and what to do about
it is the focus of the House investigation.

``We have very serious concerns about the absence of competition,'' Mr.
Cicilline said. ``So we are interested in a wide range of things like
their acquisitions, bullying, market power, their favoring of their
products and services.''

While the House has been conducting its investigation, the Justice
Department and the Federal Trade Commission have also been looking into
competition in tech. So, too, have state attorneys general and also
international regulators, most notably the European Commission under its
antitrust head, Margrethe Vestager.

Too bad Ms. Vestager won't be asking these tech titans the questions ---
she is the bane of the tech industry's existence, along with Senator
Elizabeth Warren of Massachusetts. How the hearing will be conducted
will be critical.

No surprise that I prefer public grillings with a side of shame, but
more important will be how the companies portray themselves and how they
differentiate themselves. While it's convenient to apply the catchall
term ``Big Tech'' to them, they are not a monolith and some in this
group are further along in understanding that with great power comes
great responsibility --- and, more important, accountability.

Hopefully, that is what we are going to finally see at the hearing. Mr.
Cicilline said the House will issue a report of its findings later in
the year, along with recommendations.

He also said that he had not been an expert in antitrust issues before
he took over the helm of the antitrust subcommittee, but that ``the more
I have studied and learned, the more terrifying the power of their large
digital platforms is made clear, including the impact on innovation and
start-ups.''

At the heart of these inquiries, of course, is how can we continue to
innovate as power has become more concentrated than ever. I have done
innumerable interviews with start-ups and investors in which they talk
about the chilling effect of big companies on their business.

Ask yourself, how easy it is to start an ad-based search engine, a
social network, a major online retailer or an app platform when these
companies completely cover the field with their money and power and
might?

Answer: It's not easy, which is why I will try to grab a seat in the
front row.

\emph{The Times is committed to publishing}
\href{https://www.nytimes3xbfgragh.onion/2019/01/31/opinion/letters/letters-to-editor-new-york-times-women.html}{\emph{a
diversity of letters}} \emph{to the editor. We'd like to hear what you
think about this or any of our articles. Here are some}
\href{https://help.nytimes3xbfgragh.onion/hc/en-us/articles/115014925288-How-to-submit-a-letter-to-the-editor}{\emph{tips}}\emph{.
And here's our email:}
\href{mailto:letters@NYTimes.com}{\emph{letters@NYTimes.com}}\emph{.}

\emph{Follow The New York Times Opinion section on}
\href{https://www.facebookcorewwwi.onion/nytopinion}{\emph{Facebook}}\emph{,}
\href{http://twitter.com/NYTOpinion}{\emph{Twitter (@NYTopinion)}}
\emph{and}
\href{https://www.instagram.com/nytopinion/}{\emph{Instagram}}\emph{,
and sign up for the}
\href{http://www.nytimes3xbfgragh.onion/newsletters/opiniontoday/}{\emph{Opinion
Today newsletter}}\emph{.}

Advertisement

\protect\hyperlink{after-bottom}{Continue reading the main story}

\hypertarget{site-index}{%
\subsection{Site Index}\label{site-index}}

\hypertarget{site-information-navigation}{%
\subsection{Site Information
Navigation}\label{site-information-navigation}}

\begin{itemize}
\tightlist
\item
  \href{https://help.nytimes3xbfgragh.onion/hc/en-us/articles/115014792127-Copyright-notice}{©~2020~The
  New York Times Company}
\end{itemize}

\begin{itemize}
\tightlist
\item
  \href{https://www.nytco.com/}{NYTCo}
\item
  \href{https://help.nytimes3xbfgragh.onion/hc/en-us/articles/115015385887-Contact-Us}{Contact
  Us}
\item
  \href{https://www.nytco.com/careers/}{Work with us}
\item
  \href{https://nytmediakit.com/}{Advertise}
\item
  \href{http://www.tbrandstudio.com/}{T Brand Studio}
\item
  \href{https://www.nytimes3xbfgragh.onion/privacy/cookie-policy\#how-do-i-manage-trackers}{Your
  Ad Choices}
\item
  \href{https://www.nytimes3xbfgragh.onion/privacy}{Privacy}
\item
  \href{https://help.nytimes3xbfgragh.onion/hc/en-us/articles/115014893428-Terms-of-service}{Terms
  of Service}
\item
  \href{https://help.nytimes3xbfgragh.onion/hc/en-us/articles/115014893968-Terms-of-sale}{Terms
  of Sale}
\item
  \href{https://spiderbites.nytimes3xbfgragh.onion}{Site Map}
\item
  \href{https://help.nytimes3xbfgragh.onion/hc/en-us}{Help}
\item
  \href{https://www.nytimes3xbfgragh.onion/subscription?campaignId=37WXW}{Subscriptions}
\end{itemize}
