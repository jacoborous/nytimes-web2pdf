Sections

SEARCH

\protect\hyperlink{site-content}{Skip to
content}\protect\hyperlink{site-index}{Skip to site index}

\href{https://myaccount.nytimes3xbfgragh.onion/auth/login?response_type=cookie\&client_id=vi}{}

\href{https://www.nytimes3xbfgragh.onion/section/todayspaper}{Today's
Paper}

\href{/section/upshot}{The Upshot}\textbar{}Republican Leaders Want to
End Obamacare. Their Voters Are Expanding It.

\url{https://nyti.ms/2NOhpuZ}

\begin{itemize}
\item
\item
\item
\item
\item
\item
\end{itemize}

Advertisement

\protect\hyperlink{after-top}{Continue reading the main story}

Upshot

Supported by

\protect\hyperlink{after-sponsor}{Continue reading the main story}

\hypertarget{republican-leaders-want-to-end-obamacare-their-voters-are-expanding-it}{%
\section{Republican Leaders Want to End Obamacare. Their Voters Are
Expanding
It.}\label{republican-leaders-want-to-end-obamacare-their-voters-are-expanding-it}}

Oklahoma is the latest state where voters, in choosing to expand
Medicaid, have delivered a rebuke to their elected officials.

By \href{https://www.nytimes3xbfgragh.onion/by/sarah-kliff}{Sarah Kliff}

\begin{itemize}
\item
  July 1, 2020
\item
  \begin{itemize}
  \item
  \item
  \item
  \item
  \item
  \item
  \end{itemize}
\end{itemize}

\includegraphics{https://static01.graylady3jvrrxbe.onion/images/2020/07/01/upshot/01up-medicaid-okla/merlin_174094893_fd0d6641-6d6c-4cba-9d5e-4c7e7df01bb0-articleLarge.jpg?quality=75\&auto=webp\&disable=upscale}

Deeply conservative Oklahoma narrowly
\href{https://www.nytimes3xbfgragh.onion/interactive/2020/06/30/us/elections/results-oklahoma-primary-elections.html}{approved}
a ballot initiative Tuesday to expand Medicaid to nearly 200,000
low-income adults, the first state to do so in the midst of the
coronavirus pandemic.

The vote to expand the Affordable Care Act's reach once again put
voters, many of them conservative, at odds with Republican leaders, who
have worked to block it or invalidate it. Five states --- Maine, Utah,
Idaho, Nebraska, and now Oklahoma --- have used ballot initiatives to
expand Medicaid after their Republican governors refused to do so.

Oklahoma pushed the G.O.P. over a notable threshold: Most congressional
Republicans now represent Medicaid-expansion states. The vote also came
at a striking moment, less than a week after the Trump administration
\href{https://www.nytimes3xbfgragh.onion/2020/06/26/us/politics/obamacare-trump-administration-supreme-court.html}{asked}the
Supreme Court to overturn the entirety of Obamacare --- including
Medicaid expansion.

``What we saw last night was Medicaid expansion triumph over party and
ideology,'' said Jonathan Schleifer, executive director of the Fairness
Project, which has helped organize all the Medicaid votes. ``Oklahoma
voted for Medicaid expansion even as Trump is doubling down on repeal.''

Medicaid expansion could spread further into Republican-controlled
states this year, as they weigh how to cover the many unemployed
Americans expected to lose health insurance along with their jobs.
Missouri voters will decide on a ballot initiative at the state's August
primary. If it passes, it will expand Obamacare coverage to 217,000
low-income people.

Some Wyoming legislators recently took a fresh look at the program, too,
as they watched job losses mount. ``I've voted against it about 10
times, never voted for it,'' said the state's House speaker, Steve
Harshman, a Republican. ``Now I'm thinking of our work force. We're a
mineral and oil kind of state. That's a lot of able-bodied adults in a
lot of industries who will probably need some coverage.''

Mr. Harshman voted in May to have a legislative committee study the
topic, but he does not expect any action until the body's next session
begins in January.

Medicaid expansion has proved an especially resilient part of the health
care law, despite early challenges. The program, which provides coverage
to Americans earning less than 133 percent of the federal poverty line
(about \$16,970 for an individual), was initially meant to serve all 50
states.

But in
\href{https://www.nytimes3xbfgragh.onion/2012/07/25/health/policy/3-million-more-may-lack-insurance-due-to-ruling-study-says.html}{a
2012 ruling}, the Supreme Court declared that states could decline to
participate. The program began in 2014 with about half of the states,
mostly run by Democratic governors.

That figure has grown to
\href{https://www.kff.org/medicaid/issue-brief/status-of-state-medicaid-expansion-decisions-interactive-map/}{37
states and the District of Columbia}, as more Republican-controlled
states have signed on. Many academic studies have
\href{https://www.kff.org/medicaid/report/the-effects-of-medicaid-expansion-under-the-aca-updated-findings-from-a-literature-review/}{found}
that the program increases enrollees' access to medical care. A more
limited body of research
\href{https://www.nber.org/papers/w26081}{shows} that the program also
reduces mortality rates.

The program still faces threats, most significantly the Trump
administration lawsuit to overturn the health law. The Department of
Justice, alongside a coalition of 20 Republican-controlled states,
\href{https://www.nytimes3xbfgragh.onion/2020/06/26/us/politics/obamacare-trump-administration-supreme-court.html}{submitted
briefs}to the Supreme Court last week arguing that the recent repeal of
the individual mandate, which required all Americans to carry health
coverage or pay a fine, made the entire law unconstitutional.

President Trump has found strong support in Oklahoma; he took 65 percent
of the vote there in 2016 in a 36-point victory, and recently held a
campaign rally in Tulsa, his first since the start of the pandemic.

Still, voters there broke with him on this issue, albeit by the margin
of one percentage point. The ballot initiative drew 30,000 more voters
than the state's Senate primaries, suggesting that some Oklahomans came
out specifically to support the insurance expansion.

``Oklahoma is an awfully red state,'' said Adam Searing, an associate
professor at Georgetown University who has tracked the state's ballot
effort. ``It's very conservative, very rural. To have it pass there is
quite significant.''

Oklahoma's Republican leadership had opposed Medicaid expansion and
initially offered more limited alternatives. Gov. Kevin Stitt
\href{https://oklahoman.com/article/5653672/stitt-trump-administration-officials-unveil-medicaid-block-grant-plan}{outlined}
a program in January in which new low-income enrollees would pay modest
premiums and be required to work to gain coverage.

He went on to
\href{https://www.usnews.com/news/best-states/oklahoma/articles/2020-05-21/oklahoma-governor-vetoes-bill-to-fund-his-medicaid-plan}{veto
that}program, after the legislature secured its funding.

Oklahoma was also
\href{https://www.healthcaredive.com/news/oklahoma-1st-to-seek-waiver-to-block-grant-medicaid-despite-pandemic/576460/}{the
first state} to ask the Trump administration for permission to convert
its Medicaid program to ``block grant'' funding, an idea strongly pushed
by Mr. Trump's health appointees. The state would receive a lump-sum
payment from the federal government to run the program with additional
flexibility. Opponents of that proposal worry that such a funding
formula could struggle to keep up with increased enrollment in an
economic downturn.

Oklahoma submitted its application in April, and the Trump
administration had not issued a decision before the Tuesday vote.

Oklahoma's ballot initiative is notable in being the first to add the
Medicaid expansion to the state's Constitution. That will make it hard
for Governor Stitt and the Republican-controlled legislature to tinker
with or block the program, as other governors have sought to do in the
wake of successful ballot initiatives. Most notably, when Paul LePage
was governor of Maine,
\href{https://www.pressherald.com/2018/07/12/paul-lepage-says-hed-go-to-jail-before-he-expands-medicaid/}{he
declared} he would go to jail before implementing the state's Medicaid
ballot initiative. The situation was resolved when a Democratic governor
was elected and set up the coverage expansion.

In Oklahoma, ballot organizers can pursue either statutory or
constitutional initiatives. The latter have more staying power but also
require gathering
\href{https://ballotpedia.org/Laws_governing_the_initiative_process_in_Oklahoma}{twice}
as many signatures. Amber England, who led the ballot effort, felt the
additional work was worth it.

``If we're going to ask people to get clipboards and pens, and gather
signatures, we want to make the policy as strong as possible,'' she
said. ``It was important that we protect Oklahomans' access to health
with the Constitution. We didn't want politicians to be able to take
that right away.''

Missouri will be the next state to vote on Medicaid expansion, in its
Aug. 4 primary. The state is a party to the Trump administration's case
against Obamacare. Gov. Michael Parson, a Republican, has publicly
opposed that ballot initiative, which he argues is too costly in the
midst of an economic downturn. Missouri would need to cover 10 percent
of new Medicaid enrollees' bills, with the federal government paying the
other 90 percent.

``I don't think it's the time to be expanding anything in the state of
Missouri right now,'' Mr. Parson told a local television station in
early May. ``There's absolutely not going to be any extra money
whatsoever.''

Advertisement

\protect\hyperlink{after-bottom}{Continue reading the main story}

\hypertarget{site-index}{%
\subsection{Site Index}\label{site-index}}

\hypertarget{site-information-navigation}{%
\subsection{Site Information
Navigation}\label{site-information-navigation}}

\begin{itemize}
\tightlist
\item
  \href{https://help.nytimes3xbfgragh.onion/hc/en-us/articles/115014792127-Copyright-notice}{©~2020~The
  New York Times Company}
\end{itemize}

\begin{itemize}
\tightlist
\item
  \href{https://www.nytco.com/}{NYTCo}
\item
  \href{https://help.nytimes3xbfgragh.onion/hc/en-us/articles/115015385887-Contact-Us}{Contact
  Us}
\item
  \href{https://www.nytco.com/careers/}{Work with us}
\item
  \href{https://nytmediakit.com/}{Advertise}
\item
  \href{http://www.tbrandstudio.com/}{T Brand Studio}
\item
  \href{https://www.nytimes3xbfgragh.onion/privacy/cookie-policy\#how-do-i-manage-trackers}{Your
  Ad Choices}
\item
  \href{https://www.nytimes3xbfgragh.onion/privacy}{Privacy}
\item
  \href{https://help.nytimes3xbfgragh.onion/hc/en-us/articles/115014893428-Terms-of-service}{Terms
  of Service}
\item
  \href{https://help.nytimes3xbfgragh.onion/hc/en-us/articles/115014893968-Terms-of-sale}{Terms
  of Sale}
\item
  \href{https://spiderbites.nytimes3xbfgragh.onion}{Site Map}
\item
  \href{https://help.nytimes3xbfgragh.onion/hc/en-us}{Help}
\item
  \href{https://www.nytimes3xbfgragh.onion/subscription?campaignId=37WXW}{Subscriptions}
\end{itemize}
