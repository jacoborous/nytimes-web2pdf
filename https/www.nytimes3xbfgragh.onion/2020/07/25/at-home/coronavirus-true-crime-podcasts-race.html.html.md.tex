Sections

SEARCH

\protect\hyperlink{site-content}{Skip to
content}\protect\hyperlink{site-index}{Skip to site index}

\href{https://www.nytimes3xbfgragh.onion/spotlight/at-home}{At Home}

\href{https://myaccount.nytimes3xbfgragh.onion/auth/login?response_type=cookie\&client_id=vi}{}

\href{https://www.nytimes3xbfgragh.onion/section/todayspaper}{Today's
Paper}

\href{/spotlight/at-home}{At Home}\textbar{}True Crime Podcasts at the
Intersection of Race

\url{https://nyti.ms/2Ejfr4g}

\begin{itemize}
\item
\item
\item
\item
\item
\end{itemize}

Advertisement

\protect\hyperlink{after-top}{Continue reading the main story}

Supported by

\protect\hyperlink{after-sponsor}{Continue reading the main story}

\hypertarget{true-crime-podcasts-at-the-intersection-of-race}{%
\section{True Crime Podcasts at the Intersection of
Race}\label{true-crime-podcasts-at-the-intersection-of-race}}

True crime is the lifeblood of podcasting. Here's a list of shows that
make racial justice their focus.

\includegraphics{https://static01.graylady3jvrrxbe.onion/images/2020/07/26/podcasts/26ah-truecrime-podcast/26ah-truecrime-podcast-articleLarge.jpg?quality=75\&auto=webp\&disable=upscale}

By \href{https://www.nytimes3xbfgragh.onion/by/phoebe-lett}{Phoebe Lett}

\begin{itemize}
\item
  Published July 25, 2020Updated July 27, 2020
\item
  \begin{itemize}
  \item
  \item
  \item
  \item
  \item
  \end{itemize}
\end{itemize}

When people think of true crime podcasts, they inevitably think of
``Serial.'' But the genre is vast, and the podcasts listed here
represent that diversity: a mother who investigates the death of her own
son, a journalist recovering the lost history of a nearly forgotten race
massacre, two friends telling each other stories about their favorite
serial killers of color. They all have one thing in common: These
stories are also about how racism and inequality intersect with the
deeply flawed systems of criminal justice.

\hypertarget{somebody}{%
\subsection{\texorpdfstring{\href{https://theintercept.com/podcasts/somebody/}{Somebody}}{Somebody}}\label{somebody}}

Produced by the digital news outfit The Intercept,
``\href{https://theintercept.com/podcasts/somebody/}{Somebody}'' is an
investigation into the 2016 death of 22-year old Courtney Copeland in
Chicago. What makes the show stand out is its host, Shapearl Wells, who
is Copeland's mother. After the police refused to release any
information or look into the night Copeland was found outside a police
station with a fatal bullet wound, Wells decided to get to the bottom of
what happened herself. The result is a deeply personal story of a
mother's pursuit of justice, enhanced by the music and testimony of one
of Copeland's friends from high school, Chance the Rapper.

\hypertarget{black-wall-street-1921}{%
\subsection{\texorpdfstring{\href{https://www.blackwallstreet-1921.com/}{Black
Wall Street
1921}}{Black Wall Street 1921}}\label{black-wall-street-1921}}

The recent HBO series ``Watchmen'' renewed America's attention to what
is thought to be one of the worst incidents of racial violence in
American history: the race massacre that destroyed Greenwood, an
affluent district in Tulsa, Okla., known as ``Black Wall Street.'' In
this meticulously narrated podcast, the reporter Nia Clark uses new and
archival interviews to paint the fullest picture of the intertwined
economic and racial conditions that exploded into two days of mass death
and property destruction at the hands of white terrorists, and of what
happened in the aftermath.

\hypertarget{missing-and-murdered-finding-cleo}{%
\subsection{\texorpdfstring{\href{https://www.cbc.ca/radio/findingcleo}{Missing
and Murdered: Finding
Cleo}}{Missing and Murdered: Finding Cleo}}\label{missing-and-murdered-finding-cleo}}

There is an epidemic of violence against Native women in North America.
The
\href{https://www.justice.gov/archives/ovw/blog/protecting-native-american-and-alaska-native-women-violence-november-native-american}{U.S.
Department of Justice} found that Native American women are murdered at
a rate of more than 10 times the national average, and one in three
Native women will experience sexual violence at some point in her life.
A
\href{https://www.nytimes3xbfgragh.onion/2019/06/03/world/canada/canada-indigenous-genocide.html}{Canadian
national inquiry} last year called the country's crisis of missing and
murdered Indigenous women in recent decades ``a Canadian genocide.'' The
investigative reporter Connie Walker, who is Cree and from the Okanese
First Nation in Saskatchewan, tells the stories of some of these women
and girls in ``Missing and Murdered.'' In the first season, Walker
hosted an eight-part series on the 1989 unsolved murder of 24-year-old
Alberta Williams in British Columbia. The second season centers on the
unexplained disappearance of a Saskatchewan girl, Cleopatra Semaganis
Nicotine, who was, along with her five siblings, a victim of the forced
separation of Indigenous children from their families by social workers
in Canada, known as the
``\href{https://www.nytimes3xbfgragh.onion/2017/10/06/world/canada/indigenous-forced-adoption-sixties-scoop.html}{Sixties
Scoop}.'' But Cleo vanished, and her family has spent decades trying to
find her, believing she was raped and murdered. Walker's shoe-leather
reporting eventually does answer the question: What happened to Cleo?

\hypertarget{74-seconds}{%
\subsection{\texorpdfstring{\href{https://www.mprnews.org/story/2018/04/24/74-seconds-podcast-peabody-mpr-news}{74
Seconds}}{74 Seconds}}\label{74-seconds}}

In this Peabody Award-winning Minnesota Public Radio series, the title
refers to the time elapsed after the police officer Jeronimo Yanez
turned on his lights to pull over Philando Castile's white Oldsmobile in
2016 in a Minneapolis suburb, and the moment Yanez fired seven bullets
into the elementary-school cafeteria worker. The hosts Jon Collins,
Riham Feshir and Tracy Mumford start their 22-part story with some of
the events that led up to that fateful day and follow the case through
its verdict.

\hypertarget{fruit-loops}{%
\subsection{\texorpdfstring{\href{https://fruitloopspod.com/\#:~:text=Fruitloops\%20is\%20a\%20weekly\%20podcast,by\%20serial\%20killers\%20of\%20color.}{Fruit
Loops}}{Fruit Loops}}\label{fruit-loops}}

Murder doesn't necessarily lend itself to humor, but the 2016 podcast
``\href{https://www.nytimes3xbfgragh.onion/2018/05/19/style/my-favorite-murder-podcast-murderinos.html}{My
Favorite Murder}'' --- in which funny people tell each other the stories
of serial killers and horrific crimes --- put
``\href{https://www.nytimes3xbfgragh.onion/2018/02/16/arts/television/the-transgressive-appeal-of-the-comedy-murder-podcast.html}{comedy
murder podcasts}'' on the map. Still, it is hardly the only pod of its
kind. Enter Wendy and Beth Williams (both pseudonyms), two best friends
and true-crime lovers who noticed the dearth of diversity in the genre,
in terms of those shows' hosts and the subjects they choose. Their show
``Fruit Loops'' has many of the same beats as other buddy chat shows;
the difference is in their topic of choice. Wendy, a millennial who
identifies as Black and Latinx, and Beth, a white Gen X'er, swap stories
of serial killers of color and their victims, because, as they say,
contrary to popular belief, ``not all serial killers are white!'' Like
all true crime co-hosts, the pair chew on and reacts to the details of
each case, but from multiracial, multigenerational perspectives.

\emph{Join The New York
Times}\href{https://www.facebookcorewwwi.onion/groups/nytpodcastclub/}{\emph{Podcast
Club on Facebook}} \emph{for more suggestions and discussions about all
things audio.}

Advertisement

\protect\hyperlink{after-bottom}{Continue reading the main story}

\hypertarget{site-index}{%
\subsection{Site Index}\label{site-index}}

\hypertarget{site-information-navigation}{%
\subsection{Site Information
Navigation}\label{site-information-navigation}}

\begin{itemize}
\tightlist
\item
  \href{https://help.nytimes3xbfgragh.onion/hc/en-us/articles/115014792127-Copyright-notice}{©~2020~The
  New York Times Company}
\end{itemize}

\begin{itemize}
\tightlist
\item
  \href{https://www.nytco.com/}{NYTCo}
\item
  \href{https://help.nytimes3xbfgragh.onion/hc/en-us/articles/115015385887-Contact-Us}{Contact
  Us}
\item
  \href{https://www.nytco.com/careers/}{Work with us}
\item
  \href{https://nytmediakit.com/}{Advertise}
\item
  \href{http://www.tbrandstudio.com/}{T Brand Studio}
\item
  \href{https://www.nytimes3xbfgragh.onion/privacy/cookie-policy\#how-do-i-manage-trackers}{Your
  Ad Choices}
\item
  \href{https://www.nytimes3xbfgragh.onion/privacy}{Privacy}
\item
  \href{https://help.nytimes3xbfgragh.onion/hc/en-us/articles/115014893428-Terms-of-service}{Terms
  of Service}
\item
  \href{https://help.nytimes3xbfgragh.onion/hc/en-us/articles/115014893968-Terms-of-sale}{Terms
  of Sale}
\item
  \href{https://spiderbites.nytimes3xbfgragh.onion}{Site Map}
\item
  \href{https://help.nytimes3xbfgragh.onion/hc/en-us}{Help}
\item
  \href{https://www.nytimes3xbfgragh.onion/subscription?campaignId=37WXW}{Subscriptions}
\end{itemize}
