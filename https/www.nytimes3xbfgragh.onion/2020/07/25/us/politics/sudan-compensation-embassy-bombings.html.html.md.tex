Sections

SEARCH

\protect\hyperlink{site-content}{Skip to
content}\protect\hyperlink{site-index}{Skip to site index}

\href{https://www.nytimes3xbfgragh.onion/section/politics}{Politics}

\href{https://myaccount.nytimes3xbfgragh.onion/auth/login?response_type=cookie\&client_id=vi}{}

\href{https://www.nytimes3xbfgragh.onion/section/todayspaper}{Today's
Paper}

\href{/section/politics}{Politics}\textbar{}Compensation for Embassy
Bombing Victims Could Imperil Thaw With Sudan

\url{https://nyti.ms/2WW34Sa}

\begin{itemize}
\item
\item
\item
\item
\item
\end{itemize}

Advertisement

\protect\hyperlink{after-top}{Continue reading the main story}

Supported by

\protect\hyperlink{after-sponsor}{Continue reading the main story}

\hypertarget{compensation-for-embassy-bombing-victims-could-imperil-thaw-with-sudan}{%
\section{Compensation for Embassy Bombing Victims Could Imperil Thaw
With
Sudan}\label{compensation-for-embassy-bombing-victims-could-imperil-thaw-with-sudan}}

A settlement to compensate victims of the 1998 attacks in Kenya and
Tanzania would award more money to American embassy employees than the
Africans.

\includegraphics{https://static01.graylady3jvrrxbe.onion/images/2020/07/26/us/politics/25dc-diplo-sudan1/00dc-diplo-sudan1-articleLarge.jpg?quality=75\&auto=webp\&disable=upscale}

\href{https://www.nytimes3xbfgragh.onion/by/lara-jakes}{\includegraphics{https://static01.graylady3jvrrxbe.onion/images/2019/07/25/reader-center/author-lara-jakes/author-lara-jakes-thumbLarge.png}}

By \href{https://www.nytimes3xbfgragh.onion/by/lara-jakes}{Lara Jakes}

\begin{itemize}
\item
  Published July 25, 2020Updated July 30, 2020
\item
  \begin{itemize}
  \item
  \item
  \item
  \item
  \item
  \end{itemize}
\end{itemize}

WASHINGTON --- Riz Khaliq and Doreen Oport were both working at the
American Embassy in Nairobi on Aug. 7, 1998, when a truck bomb tore
through the compound. Both were bloodied. Mr. Khaliq was knocked
unconscious and Ms. Oport trapped beneath fallen rubble.

Decades later, both periodically pick shreds of glass and metal from
their bodies as embedded debris from the blast continues to work its way
to the skin.

Both are also naturalized U.S. citizens. There, the similarities end.

Because Mr. Khaliq was already an American citizen at the time of the
attack, he is eligible for at least \$3 million in compensation as part
of a tentative State Department settlement with the government of Sudan.

Ms. Oport was then a citizen of Kenya, so she stands to receive
\$400,000 from Sudan, which had harbored Qaeda militants who
\href{https://www.fbi.gov/history/famous-cases/east-african-embassy-bombings}{bombed
the U.S. embassies in Kenya and Tanzania} in near-simultaneous strikes.

Mr. Khaliq and Ms. Oport are among hundreds of victims and family
members at the center of a yearslong process
\href{https://www.nytimes3xbfgragh.onion/2017/11/16/world/africa/sudan-terrorism-sanctions.html}{to
remove Sudan from a U.S. government list} of state sponsors of
terrorism. Doing so will open the way for the East African country to
move toward economic stability, and potentially greater democracy, after
a generation of oppression.

Yet the payment disparity between victims who were Americans at the time
of the bombings and those who were not has delayed --- and could derail
--- the deal. It has divided Congress and created a rift between the
victims and their lawyers as the United States grapples with how to
correct unequal or discriminatory standards in its legal system.

``It's cold --- why would they even think of compensating the Kenyans at
a lesser percentage than the Americans?'' said Ms. Oport, who worked at
the U.S. Embassy in Nairobi, Kenya's capital, for 15 years before she
immigrated to the United States in 2002 and became a citizen in 2010.

``I can only say it's discrimination,'' she added.

She called the international employees at American embassies abroad the
``backbone'' of the missions by keeping operations running and recounted
returning to work a few days after the blasts to pick through the rubble
for documents that would have been lost. ``The recognition of equality
is very important,'' she said.

For Mr. Khaliq, the settlement serves a broader purpose. Not only would
it be the first time that Sudan's government has acknowledged
responsibility for the bombings, but it could also help raise nearly
half of the
\href{https://www.cia.gov/library/publications/the-world-factbook/geos/su.html}{country's
45 million citizens out of poverty} by making available international
assistance --- and, potentially, keep it from being a breeding ground
for terrorism.

\includegraphics{https://static01.graylady3jvrrxbe.onion/images/2020/07/26/us/politics/25dc-diplo-sudan3-sub/merlin_174759276_85976f4a-981a-4b3b-9036-174e01c4a385-articleLarge.jpg?quality=75\&auto=webp\&disable=upscale}

``Will this make up for all the pain and suffering, and all the pain
that I put my family through, with my PTSD?'' Mr. Khaliq asked. ``No, it
doesn't. But I also feel like it at least gets to some level of
resolution.''

Just as important is ``to help these countries understand that
supporting terror activity, or harboring terrorism, is not a nameless
crime,'' said Mr. Khaliq, who was in a meeting with the American
ambassador to Kenya, Prudence Bushnell, when the bomb exploded. ``And I
would hate to see the potential agreement fall apart or crumble, because
it's not exactly perfect.''

Of the 224 people who were killed in the 1998 bombings in Nairobi and
Dar es Salaam, 54 were embassy employees or contractors, including 12
Americans, according to the State Department. Thousands more were
injured, including 139 embassy employees and contractors.

Under its authoritarian president at the time,
\href{https://www.nytimes3xbfgragh.onion/2020/07/30/world/middleeast/darfur-sudan.html}{Omar
Hassan al-Bashir}, Sudan was added to the State Department's list of
nations abetting terrorism in 1993 for supporting Hezbollah and
Palestinian militant groups.
\href{https://www.nytimes3xbfgragh.onion/2020/05/18/us/supreme-court-sudan-terrorism.html}{American
courts ruled that Sudan} was a vital participant in the embassy bombings
that took place five years later, having provided passports,
unrestricted border travel and shelter to the Qaeda militants before
they attacked.

Only three other nations --- Iran, North Korea and Syria --- are
currently on the State Department list that restricts assistance from
the United States and, effectively, from the World Bank and the
International Monetary Fund.

But in 2016, Sudan
\href{https://www.reuters.com/article/us-saudi-iran-sudan/sudan-cuts-diplomatic-ties-with-iran-idUSKBN0UI17720160104}{cut
its diplomatic ties with Iran} and
\href{https://www.crisisgroup.org/africa/horn-africa/horn-africa-states-follow-gulf-yemen-war}{joined
Saudi Arabia in its fight against rebel Houthis} in nearby Yemen. (Both
moves hew to policies supported by the U.S. government, although many of
the Sudanese soldiers who were sent to fight in Yemen were children from
the impoverished
\href{https://www.nytimes3xbfgragh.onion/2020/07/30/world/middleeast/darfur-sudan.html}{Darfur
region}.) And in the final days of the Obama administration, after years
of negotiations, the United States
\href{https://www.france24.com/en/20170113-sudan-usa-sanctions-easing-terrorism}{began
easing sanctions} against Sudan to reward its
\href{https://www.dabangasudan.org/en/all-news/article/donald-booth-sanctions-relief-is-a-start-to-address-sudan-s-human-rights-issues}{government's
cooperation} on fighting terrorism and ending military attacks on its
people.

The effort was picked up by the Trump administration, and gathered
momentum after Mr. al-Bashir was ousted from power in April 2019 in a
coup. That gave the United States a new opening for normalizing
diplomatic relations with Sudan and to help stabilize the Horn of
Africa, one of the world's most
\href{https://agsiw.org/gulf-strategic-interests-reshaping-the-horn-of-africa/}{strategic
and volatile regions}.

Image

Months of street protests last year led to the ouster of President Omar
al-Bashir of Sudan.Credit...Bryan Denton for The New York Times

Saudi Arabia and Israel are among the American allies that are pressing
for a diplomatic thaw,
\href{https://jcpa.org/israel-comes-full-circle-with-sudan-analysis/}{mostly
to counter Iran}. Prime Minister Benjamin Netanyahu of Israel
\href{https://www.aljazeera.com/news/2020/02/netanyahu-israel-sudan-normalise-ties-200203182536972.html}{met
in February} with Sudan's de facto leader, Lt. Gen. Abdel Fattah
al-Burhan. Days later,
\href{https://www.reuters.com/article/us-israel-sudan/netanyahu-says-israeli-airliners-have-started-overflying-sudan-idUSKBN20A0NK}{Sudan
began allowing} Israeli commercial planes to fly in its airspace.

The opportunity for détente could be short-lived. So far this year,
Sudan's
\href{https://www.aljazeera.com/news/2020/06/sudan-protesters-return-streets-demand-reforms-200630154320885.html}{festering
political instability} has been fueled by the coronavirus outbreak and a
\href{https://www.bbc.com/news/world-africa-51800278}{recent
assassination attempt} against the transitional government's prime
minister, Abdalla Hamdok.

``The next few months may be one of our only opportunities to support
Sudan's progress,'' said Senator Chris Coons, Democrat of Delaware and
an Africa expert.

He called it a ``once-in-a generation opportunity to improve our
relationship with Sudan and support the new government's efforts to
transition toward democracy and a more inclusive society.''

The State Department insists that Sudan compensate the embassy bombing
victims before it is taken off the list of state sponsors of terrorism.
The tentative settlement would pay victims, or their surviving families,
a total of about \$335 million --- all but \$100 million would go to
those who were American citizens at the time of the attacks.

Before it will pay, however, Sudan has demanded that it receive immunity
from future lawsuits related to the bombings and other attacks while on
the terrorism list. That must be approved by Congress, which has stalled
over whether the embassies' international employees are being unjustly
undercompensated.

``I would not want foreign nationals who worked in our embassies to be
treated as second-class citizens,'' said Representative Bennie Thompson,
Democrat of Mississippi and the chairman of the Homeland Security
Committee.

He said the State Department appeared unwilling to try to do more for
those employees. ``Everybody should be treated in a fair and equitable
manner,'' Mr. Thompson said.

Image

Mr. Khaliq helping Prudence Bushnell, then the American ambassador to
Kenya, escape from the embassy in Nairobi after it was bombed in
1998.Credit...Sayyid Azim/Associated Press

Image

Ms. Oport's hands are marked with scars from the
attack.Credit...Nitashia Johnson for The New York Times

State Department officials said that given its financial fragility,
Sudan could not afford to pay more than \$335 million, and that the
United States was not required to compensate the international victims.
They said even the lower payments provided some recognition to those who
were killed or wounded simply because they worked for the American
government.

The amount given to each victim is loosely based on a
\href{https://archive.vn/20120707022256/http:/articles.cnn.com/2002-05-28/us/libya.lockerbie.settlement_1_libyan-offer-commercial-sanctions-families-of-terror-victims}{\$2.7
billion settlement} that the Libyan government, under Col. Muammar
el-Qaddafi, agreed to pay to the families of each person who died in the
1988 bombing of Pan Am Flight 103 over Lockerbie, Scotland. Libya
offered the payments --- \$10 million for each victim who was killed, no
matter the country of origin --- in 2002 as Colonel el-Qaddafi sought
removal
\href{https://abcnews.go.com/International/story?id=1965753\&page=1}{from
the State Department's terrorism list}, which took effect in 2006.

In settling the cases of the 1998 embassy bombings, Sudan would pay \$10
million to families of Americans who died; U.S. victims who were injured
but survived would receive at least \$3 million. By contrast, the
families of slain foreign citizens who worked at the embassies would be
entitled to \$800,000, and the victims who were wounded would receive
\$400,000.

State Department officials said that the Lockerbie settlement was
reached in a private lawsuit that the families of the Flight 103 victims
--- not the United States government --- brought against Libya. They
also said the difference in compensation mirrored the differences in
salaries and other employment benefits between American diplomats and
international employees at U.S. embassies, which is largely based on
local standards of living in host nations.

In a June letter to the Senate Foreign Relations Committee, the spouses
of three Tanzanian guards who died confronting the truck bomb at the
U.S. Embassy in Dar es Salaam called the settlement with Sudan ``a very
small portion of what it owes.'' But they also said the compensation was
a ``very significant sum in Tanzania and will do much good for our
families and our community.''

``It would be unfortunate if we lost the opportunity to recover anything
from Sudan simply because some individuals are demanding more
compensation,'' wrote the spouses, Hanuni Shamte Ndange, Judith Mwila,
and Shabani Mtulya.

Many of the victims of the embassy bombings, and their families, are
already being compensated from a separate fund that the Justice
Department administers.

Image

Of the 224 people who were killed in the 1998 bombings in Nairobi and
Dar-es-Salaam, 54 were embassy employees or contractors, including 12
Americans. Thousands more were injured, including 139 embassy employees
and contractors.Credit...Agence France-Presse --- Getty Images

By the end of the year, it is estimated that they will have received a
combined \$605 million from the U.S. Victims of State Sponsored
Terrorism Fund, which pays Americans and international citizens alike
after a judge decides compensation. The fund is supported by payments
made by banks and companies that have violated terrorism-related U.S.
sanctions against Iran.

Congressional officials said lawmakers were considering extending the
payouts to the embassy bombing victims for additional years as a way to
break the impasse over the disparate compensation levels in the State
Department's settlement with Sudan. That way, the officials said, the
international employees and their survivors would be assured more money.

In the years after the deaths of her father and brother at the U.S.
Embassy in Nairobi, Edith Bartley pushed Congress to ensure survivors'
benefits for all employees of diplomatic missions targeted in terrorist
attacks, going back to 1983. She also had lobbied lawmakers before they
enacted the victims' compensation fund that is run by the Justice
Department.

She said it was what her father --- Julian L. Bartley, a career U.S.
diplomat and the first African-American consul general in Kenya ---
would have wanted her to do.

But she predicted ``grave consequences'' to national security if the
settlement with Sudan fell apart.

``Sudan wants to turn themselves around. They've taken great steps to do
that,'' Ms. Bartley said. ``That should not be taken lightly.''

Advertisement

\protect\hyperlink{after-bottom}{Continue reading the main story}

\hypertarget{site-index}{%
\subsection{Site Index}\label{site-index}}

\hypertarget{site-information-navigation}{%
\subsection{Site Information
Navigation}\label{site-information-navigation}}

\begin{itemize}
\tightlist
\item
  \href{https://help.nytimes3xbfgragh.onion/hc/en-us/articles/115014792127-Copyright-notice}{©~2020~The
  New York Times Company}
\end{itemize}

\begin{itemize}
\tightlist
\item
  \href{https://www.nytco.com/}{NYTCo}
\item
  \href{https://help.nytimes3xbfgragh.onion/hc/en-us/articles/115015385887-Contact-Us}{Contact
  Us}
\item
  \href{https://www.nytco.com/careers/}{Work with us}
\item
  \href{https://nytmediakit.com/}{Advertise}
\item
  \href{http://www.tbrandstudio.com/}{T Brand Studio}
\item
  \href{https://www.nytimes3xbfgragh.onion/privacy/cookie-policy\#how-do-i-manage-trackers}{Your
  Ad Choices}
\item
  \href{https://www.nytimes3xbfgragh.onion/privacy}{Privacy}
\item
  \href{https://help.nytimes3xbfgragh.onion/hc/en-us/articles/115014893428-Terms-of-service}{Terms
  of Service}
\item
  \href{https://help.nytimes3xbfgragh.onion/hc/en-us/articles/115014893968-Terms-of-sale}{Terms
  of Sale}
\item
  \href{https://spiderbites.nytimes3xbfgragh.onion}{Site Map}
\item
  \href{https://help.nytimes3xbfgragh.onion/hc/en-us}{Help}
\item
  \href{https://www.nytimes3xbfgragh.onion/subscription?campaignId=37WXW}{Subscriptions}
\end{itemize}
