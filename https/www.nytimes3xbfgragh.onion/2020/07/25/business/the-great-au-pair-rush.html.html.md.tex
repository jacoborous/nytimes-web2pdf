\href{/section/business}{Business}\textbar{}The Great Au Pair Rush

\url{https://nyti.ms/335TRe3}

\begin{itemize}
\item
\item
\item
\item
\item
\item
\end{itemize}

\href{https://www.nytimes3xbfgragh.onion/spotlight/at-home?action=click\&pgtype=Article\&state=default\&region=TOP_BANNER\&context=at_home_menu}{At
Home}

\begin{itemize}
\tightlist
\item
  \href{https://www.nytimes3xbfgragh.onion/2020/07/28/books/time-for-a-literary-road-trip.html?action=click\&pgtype=Article\&state=default\&region=TOP_BANNER\&context=at_home_menu}{Take:
  A Literary Road Trip}
\item
  \href{https://www.nytimes3xbfgragh.onion/2020/07/29/magazine/bored-with-your-home-cooking-some-smoky-eggplant-will-fix-that.html?action=click\&pgtype=Article\&state=default\&region=TOP_BANNER\&context=at_home_menu}{Cook:
  Smoky Eggplant}
\item
  \href{https://www.nytimes3xbfgragh.onion/2020/07/27/travel/moose-michigan-isle-royale.html?action=click\&pgtype=Article\&state=default\&region=TOP_BANNER\&context=at_home_menu}{Look
  Out: For Moose}
\item
  \href{https://www.nytimes3xbfgragh.onion/interactive/2020/at-home/even-more-reporters-editors-diaries-lists-recommendations.html?action=click\&pgtype=Article\&state=default\&region=TOP_BANNER\&context=at_home_menu}{Explore:
  Reporters' Obsessions}
\end{itemize}

\includegraphics{https://static01.graylady3jvrrxbe.onion/images/2020/07/26/business/26aupairsjp/merlin_174871512_b1de58df-100c-4682-881e-d26acdceb4b6-articleLarge.jpg?quality=75\&auto=webp\&disable=upscale}

Sections

\protect\hyperlink{site-content}{Skip to
content}\protect\hyperlink{site-index}{Skip to site index}

\hypertarget{the-great-au-pair-rush}{%
\section{The Great Au Pair Rush}\label{the-great-au-pair-rush}}

When the Trump administration shut the borders to many new au pairs,
those already in the country found they had something new: options.

Credit...Eva Redamonti

Supported by

\protect\hyperlink{after-sponsor}{Continue reading the main story}

By Jordan Salama

\begin{itemize}
\item
  July 25, 2020
\item
  \begin{itemize}
  \item
  \item
  \item
  \item
  \item
  \item
  \end{itemize}
\end{itemize}

When the au pair decided to change families, she feared she was taking a
major risk.

Since the fall, the Colombian woman in her mid-20s had been working in
New York as an au pair, one of about 20,000 young people --- mostly
women --- who come to the United States each year to live with families
and take care of their children. Her yearlong contract wasn't set to
expire until late 2020, but one morning in mid-June, an argument with
her host dad proved to be the breaking point of
\href{https://www.nytimes3xbfgragh.onion/2020/07/17/style/this-is-not-the-america-these-au-pairs-were-expecting.html}{a
tense home environment} in quarantine.

``I can't take these people anymore,'' the au pair texted me in Spanish.
``I want to get out of here today.'' She reported the situation to her
local coordinator and decided to leave, giving her two weeks to find a
new family or return to Colombia. She hadn't the slightest clue where
she would end up next.

But the woman's anxiety turned to surprise a few days later when she
checked her email --- she already had dozens of families across the
country asking for interviews. Normally, the demand for au pairs already
in the United States is not nearly as high, but something had changed:
On June 22, the Trump administration issued an executive
order\href{https://www.nytimes3xbfgragh.onion/2020/06/19/us/foreign-worker-visas-trump-coronavirus.html}{suspending
many foreign work visas}at least until the end of this year. The order
included the J-1 visa program, under which the au pair program, managed
by the State Department, is categorized.

While the coronavirus pandemic had already made
\href{https://www.nytimes3xbfgragh.onion/2020/05/13/us/politics/trump-coronavirus-border-restrictions.html?action=click\&module=RelatedLinks\&pgtype=Article}{international
travel difficult} for many, the
\href{https://www.nytimes3xbfgragh.onion/2020/06/12/us/politics/coronavirus-trump-immigration-policies.html?action=click\&module=RelatedLinks\&pgtype=Article}{visa
restrictions} confirmed that new au pairs preparing to come to the
United States wouldn't be able to enter the country. The American
families expecting them, often with working parents relying on the
program as their primary source of child care, have been left scrambling
to find replacements.

I spoke to nearly a dozen au pairs now in the country, and read the
testimonies of many more on social media. They asked that their names
not be used for this story, because they feared retaliation.

Many host parents have taken to unofficial forums on Facebook and other
sites as an additional way to search for potential matches. That has
created a frenzied social-media rush to woo the dwindling number of au
pairs in the country who are still available.

``Pretty much everyone is saying it's pretty unlikely that you'll get an
au pair,'' said Erin Burkhart, a high-school teacher and two-time host
mom in the Seattle area whose most recent au pair was set to join her
family this summer from Germany. ``The search process itself is a
full-time job. Right now I will email everyone, I will reach out to
everyone. I've had about 15 video chats in the last week.''

On the other end, while au pairs entering the program might speak with
only two or three families in the initial interview process, in-country
candidates are now hearing from 10, 20, sometimes closer to 50
prospective families. Even male au pairs, who often find it harder to
match, are having an easy time. ``Because they know they don't have
options, they are accepting males for their families too,'' said an au
pair from Brazil. ``It's not a big deal anymore.''

``Now we feel powerful,'' the Colombian au pair said. ``For once, we
have a choice.''

\includegraphics{https://static01.graylady3jvrrxbe.onion/images/2020/07/22/business/00AUPAIR-01/merlin_174741990_2881ba44-bbc7-47c0-b2b1-7f9abeb4eb72-articleLarge.jpg?quality=75\&auto=webp\&disable=upscale}

\hypertarget{beach-houses-and-skydiving-trips}{%
\subsection{Beach Houses and Skydiving
Trips}\label{beach-houses-and-skydiving-trips}}

Though administered by the State Department, the au pair program is
operated by a network of private agencies (Cultural Care, Au Pair Care
and Au Pair in America are a few big ones) that are in charge of vetting
and matching au pairs with host families before they even set foot in
the United States. On the ground, au pairs and host families deal more
directly with local child care consultants, or L.C.C.s --- regional
counselors for the agencies who oversee day-to-day issues that arise in
households.

If an in-country au pair wants to rematch, or switch families later on,
her request must first be approved by the L.C.C. and the match
ultimately approved by the agency.

But many introductory conversations are often carried out via unofficial
channels --- Facebook, WhatsApp and personal referrals between au pairs
and families --- to streamline the process. In recent weeks, these
unofficial networks have become inundated.

Many in-country au pairs are now telling interested hosts that they are
only willing to match in exchange for certain assurances, such as a
personal car or payment upward of \$400 a week. The minimum stipend for
au pairs is \$195.75 a week for a maximum of 45 hours of work, which is
set by the State Department.

Host families have taken note of the new dynamic, too: Perusing some
Facebook groups in mid-June, I found posts announcing benefits like
unlimited public transportation passes, new cars, access to beach houses
and skydiving trips, and double the pay. ``We're offering a 2000 USD
sign-on bonus,'' one parent wrote.

Not all host families are advertising perks, though, and not all au
pairs are seeking them out. Coming from difficult working conditions
with her first host family --- including verbal abuse, additional chores
like housecleaning and dog-grooming, and long hours for no extra pay ---
the Colombian au pair's top priority was finding a family that would be
the best fit.

Many host families feel similarly that the match must be right.
``Offering benefits is fine, but people should not lose sight of the
spirit of the program, which is cultural exchange and having an au pair
join your family,'' Ms. Burkhart said. ``You're going to eat dinner with
this person regularly, spend holidays and vacations together for a year.
It's important to find a good fit.''

The current shortage of in-country au pairs caused by the one-two punch
of quarantine and visa restrictions has further highlighted the lack of
affordable child care in America, to the point where young foreigners
expecting a year or two of cultural exchange have become lifelines,
often unintentionally, for two-earner couples hoping to keep both their
jobs.

\hypertarget{scrambling-for-child-care}{%
\subsection{Scrambling for Child Care}\label{scrambling-for-child-care}}

When the order was officially announced on June 22, au pairs from around
the world, preparing to leave home for a year or longer in the United
States, saw their dreams crushed.

``I was honestly heartbroken,'' said Kristina Kobzeva, 23, from
Kazakhstan. ``My mom told me that I can't wait so much time until next
year, that I'll have to quit the program and get married if the borders
won't be reopened this year for au pairs.''

Au pairs pay fees to participate in the program, navigating a complex
web of foreign recruiters, satellite offices and U.S. agencies that vary
on a case-by-case basis. Including expenses associated with the J-1 visa
application, the total out-of-pocket enrollment cost for au pairs
usually hovers between \$1,000 and \$2,000, much of which is often
nonrefundable. ``I worked at least three months nonstop, two jobs, in
order to save the money for the program,'' Ms. Kobzeva added. ``Now I'm
literally in the middle of nowhere with no idea what to do.''

Enrollment for American host families is more straightforward: Between
agency program fees and required au pair expenses (such as weekly
stipends, travel and food, and up to \$500 toward a mandatory education
requirement), the total minimum cost of the program is around \$20,000 a
year, regardless of the number of children in the family. If a family
pays only the minimum, it's affordable when compared with traditional
child care options.

Image

Dawn Gile, a lawyer and host mom in Maryland, said her daughters had
``been so enriched'' by having an au pair.Credit...Ting Shen for The New
York Times

When the match is a good one, families and au pairs can come away with
long-lasting relationships. ``Child care is one aspect of it, but we've
really appreciated the cultural exchange component,'' said Dawn Gile, a
lawyer and host mom in Maryland. ``We were going to travel to Europe to
go visit our former au pairs. We keep in touch with them, our girls had
this exposure to foreign languages, culture, food --- they've been so
enriched by the au pair program.''

But the primary motivation, by far, for most families to host an au pair
is the flexible and affordable child care. Now, as the coronavirus
threatens to keep schools and day cares closed, and as traditional
babysitting becomes complicated in a socially distanced world, live-in
child care is even more appealing. That's especially the case for
essential workers ---
\href{https://www.nytimes3xbfgragh.onion/2020/03/16/us/coronavirus-doctors-nurses.html}{physicians
and other health professionals, in particular} --- who rely on au pair
support to maintain long hours during the pandemic.

Nearly a month after the initial rules were issued, the State Department
announced that some au pairs --- namely, those caring for the children
of medical professionals involved in the fight against Covid-19, or
children with medical or other special needs --- would be granted an
exception to the visa restrictions rule and be allowed to enter the
country.

Military families, often on the move, are also among those most affected
by the rule. ``It's frustrating in a lot of ways because military
spouses try so hard to maintain a career despite the impact of their
spouse's service,'' said Ms. Gile, whose husband is in the military and
who also serves as president of the Military Spouse JD Network. Now that
her next au pair is barred from entering the country, Ms. Gile fears the
lack of child care will affect her ability to keep working. ``This is
just another setback in trying to maintain a career,'' she said.

``There are a lot of parents who, because of this, will have to quit
their jobs,'' Ms. Burkhart added.

Rachel Block, a former World Bank economist and experienced host mom,
put it more bluntly: ``The main substitute is women working less and
having to
\href{https://www.nytimes3xbfgragh.onion/2020/06/03/business/economy/coronavirus-working-women.html}{pull
back from the work force.}''

\hypertarget{you-are-not-part-of-the-family}{%
\subsection{`You Are Not Part of the
Family'}\label{you-are-not-part-of-the-family}}

There are fears that the rush of perks offered by families might cloud
au pairs' ability to select kind and properly qualified hosts. While
many au pairs are treated with respect, many aren't, as recent
\href{https://www.politico.com/magazine/story/2017/03/au-pair-program-abuse-state-department-214956}{investigations}
and
\href{https://www.nytimes3xbfgragh.onion/2020/01/08/us/au-pair-massachusetts-ruling.html}{court
cases} have shown.

``Very soon, au pairs realize that while you can have a great, amazing
relationship with a family, they are your boss, and you are their
employee,'' added the Colombian au pair. ``You are not part of the
family.''

Au pairs have reported working far more than 45 hours per week, and
being berated by host parents; some have seen their food restricted, or
their activities monitored by surveillance cameras. Afraid of being sent
home early, many suffer in silence.

A Brazilian au pair in New Jersey who said she was verbally abused daily
by her host's children and was ``basically a maid,'' was afraid to ask
for a switch. ``There's a lot of stories about girls getting kicked out
of the house when they ask for a rematch,'' she said.

When she reported the situation to her local agency counselor, she was
told in an email to work things out or she would likely be sent home.
Only after the host mom approved the rematch a month later, the au pair
said, did the agency agree to facilitate a change.

These experiences are far from uncommon. A
\href{https://cdmigrante.org/wp-content/uploads/2018/08/Shortchanged.pdf}{2018
investigation} by several labor-rights groups argued that the J-1 au
pair program is a work program with little real opportunity for cultural
exchange, and that au pairs should be protected as domestic workers.

``This is an employment relationship. The law has upheld that to be the
case,'' said Rocío Ávila, a senior lawyer with the National Domestic
Workers Alliance, one of the co-authors of the report. In December 2019,
a Massachusetts court ruled that minimum-wage laws applied to au pairs
in that state. As a result, in Massachusetts, the weekly cost of a
full-time au pair rose from the roughly \$200 minimum stipend (\$4.35
per hour for 45 hours) to more than \$500 in weekly wages, after
deductions for meals and lodging.

``I think it's the sponsor's role to set the expectations of both the au
pair and the host family,'' said Jean Quinn, the director of Au Pair in
America, an agency. ``What we want are both families and au pairs to
come with the right expectations. It doesn't do anybody any good if
that's not the case,'' she added. ``I think we do a very good job at
making it clear that this has to work for both sides in order for it to
be a successful placement.''

Also key to au pair protections, labor advocates and some host parents
like Ms. Block have argued, are more government regulation,
know-your-rights education for incoming au pairs, and a more streamlined
system for complaints, independent of private agencies.

\hypertarget{the-right-fit}{%
\subsection{The Right Fit}\label{the-right-fit}}

As the matching frenzy continued, the Colombian au pair narrowed her
dozens of options to just a handful of families. After her fourth day of
nonstop interviews, she was triumphant. ``I have a family!'' she
announced, smiling from ear to ear.

Ms. Burkhart was one of a few host parents who could say the same. ``We
just signed our au pair tonight :)'' she wrote in an email.

The au pair was glad, in the end, that she hadn't let herself be wooed
by promises of cars or beach houses or more money, which could have been
deceiving --- because even for a complicated program that involves so
many different actors, everything ultimately comes down to the quality
of the match. ``My sense is that this is a family that's really going to
care about me,'' she said. And because she was here now, she could work
with that.

Jordan Salama
(\href{https://twitter.com/JordanSalama19}{@jordansalama19}) is a writer
whose essays and stories have appeared, most recently, in The New York
Times, National Geographic and Smithsonian.

Advertisement

\protect\hyperlink{after-bottom}{Continue reading the main story}

\hypertarget{site-index}{%
\subsection{Site Index}\label{site-index}}

\hypertarget{site-information-navigation}{%
\subsection{Site Information
Navigation}\label{site-information-navigation}}

\begin{itemize}
\tightlist
\item
  \href{https://help.nytimes3xbfgragh.onion/hc/en-us/articles/115014792127-Copyright-notice}{©~2020~The
  New York Times Company}
\end{itemize}

\begin{itemize}
\tightlist
\item
  \href{https://www.nytco.com/}{NYTCo}
\item
  \href{https://help.nytimes3xbfgragh.onion/hc/en-us/articles/115015385887-Contact-Us}{Contact
  Us}
\item
  \href{https://www.nytco.com/careers/}{Work with us}
\item
  \href{https://nytmediakit.com/}{Advertise}
\item
  \href{http://www.tbrandstudio.com/}{T Brand Studio}
\item
  \href{https://www.nytimes3xbfgragh.onion/privacy/cookie-policy\#how-do-i-manage-trackers}{Your
  Ad Choices}
\item
  \href{https://www.nytimes3xbfgragh.onion/privacy}{Privacy}
\item
  \href{https://help.nytimes3xbfgragh.onion/hc/en-us/articles/115014893428-Terms-of-service}{Terms
  of Service}
\item
  \href{https://help.nytimes3xbfgragh.onion/hc/en-us/articles/115014893968-Terms-of-sale}{Terms
  of Sale}
\item
  \href{https://spiderbites.nytimes3xbfgragh.onion}{Site Map}
\item
  \href{https://help.nytimes3xbfgragh.onion/hc/en-us}{Help}
\item
  \href{https://www.nytimes3xbfgragh.onion/subscription?campaignId=37WXW}{Subscriptions}
\end{itemize}
