\href{/section/science}{Science}\textbar{}Why the `Super Weird' Moons of
Mars Fascinate Scientists

\url{https://nyti.ms/32RN0EF}

\begin{itemize}
\item
\item
\item
\item
\item
\end{itemize}

\hypertarget{missions-to-mars}{%
\subsubsection{\texorpdfstring{\href{https://www.nytimes3xbfgragh.onion/news-event/summer-of-mars?name=styln-mars\&region=TOP_BANNER\&variant=undefined\&block=storyline_menu_recirc\&action=click\&pgtype=Article\&impression_id=67539610-e388-11ea-881f-496d10ec9bb8}{Missions
to Mars}}{Missions to Mars}}\label{missions-to-mars}}

\begin{itemize}
\tightlist
\item
  \href{https://www.nytimes3xbfgragh.onion/2020/07/30/science/nasa-mars-launch.html?name=styln-mars\&region=TOP_BANNER\&variant=undefined\&block=storyline_menu_recirc\&action=click\&pgtype=Article\&impression_id=6753bd20-e388-11ea-881f-496d10ec9bb8}{NASA
  Launch Highlights}
\item
  \href{https://www.nytimes3xbfgragh.onion/interactive/2020/science/mars-perseverance-tianwen-hope.html?name=styln-mars\&region=TOP_BANNER\&variant=undefined\&block=storyline_menu_recirc\&action=click\&pgtype=Article\&impression_id=6753bd21-e388-11ea-881f-496d10ec9bb8}{Meet
  the Spacecraft}
\item
  \href{https://www.nytimes3xbfgragh.onion/2020/07/28/science/nasa-jezero-perseverance.html?name=styln-mars\&region=TOP_BANNER\&variant=undefined\&block=storyline_menu_recirc\&action=click\&pgtype=Article\&impression_id=6753bd22-e388-11ea-881f-496d10ec9bb8}{NASA's
  Destination}
\item
  \href{https://www.nytimes3xbfgragh.onion/2020/07/28/science/mars-nasa-science.html?name=styln-mars\&region=TOP_BANNER\&variant=undefined\&block=storyline_menu_recirc\&action=click\&pgtype=Article\&impression_id=6753bd23-e388-11ea-881f-496d10ec9bb8}{Too
  Much Mars?}
\end{itemize}

\includegraphics{https://static01.graylady3jvrrxbe.onion/images/2020/07/28/science/28SCI-MARSMOONS/merlin_174901806_c697045b-15f4-489c-8196-173bf399e4d2-articleLarge.jpg?quality=75\&auto=webp\&disable=upscale}

Sections

\protect\hyperlink{site-content}{Skip to
content}\protect\hyperlink{site-index}{Skip to site index}

\hypertarget{why-the-super-weird-moons-of-mars-fascinate-scientists}{%
\section{Why the `Super Weird' Moons of Mars Fascinate
Scientists}\label{why-the-super-weird-moons-of-mars-fascinate-scientists}}

What's the big deal about little Phobos and tinier Deimos?

A close-up view of Phobos, the larger of Mars's two moons. It is 17
miles across.Credit...NASA/JPL/University of Arizona

Supported by

\protect\hyperlink{after-sponsor}{Continue reading the main story}

By Robin George Andrews

\begin{itemize}
\item
  July 25, 2020
\item
  \begin{itemize}
  \item
  \item
  \item
  \item
  \item
  \end{itemize}
\end{itemize}

Mars is the darling of many planetary scientists, who continue to visit
it through increasingly advanced robotic explorers. But don't forget
that our planetary neighbor is adorned with two moons: puny
\href{https://solarsystem.nasa.gov/moons/mars-moons/phobos/in-depth/}{Phobos},
a lumpy mass 17 miles across; and diminutive
\href{https://solarsystem.nasa.gov/moons/mars-moons/deimos/in-depth/}{Deimos},
just 9 miles long. Their names in ancient Greek may mean ``fear'' and
``dread', but the aesthetics of these Lilliputian space potatoes inspire
anything but.

They don't look anywhere near as interesting as the
\href{https://www.nytimes3xbfgragh.onion/2019/06/26/science/io-volcanic-moon.html}{volcanic}
or
\href{https://www.nytimes3xbfgragh.onion/2019/12/09/science/enceladus-stripes-moon.html}{icy-ocean
moons} of Jupiter and Saturn, nor is their
\href{https://www.nytimes3xbfgragh.onion/2019/07/12/science/nasa-moon-apollo-artemis.html}{desolation}
as extreme or diverse as Earth's moon. But that hasn't stopped
generations of planetary scientists from being eager to get a closer
look at the ramshackle duo.

The Soviet Union and, later, Russia have tried three times to reach
Phobos, but
\href{https://www.planetary.org/explore/space-topics/space-missions/missions-to-mars.html\#phobos2}{software
errors} and
\href{https://solarsystem.nasa.gov/missions/phobos-grunt/in-depth/}{launch
disasters} have doomed every attempt. Scientists in the U.S. have tried
and, so far, failed to convince the powers-that-be at NASA that a
\href{https://www.lpi.usra.edu/meetings/LPSC99/pdf/1155.pdf}{mission} to
the two moons will be worthwhile. The next great hope is Japan, which is
aiming to launch a heist mission to Phobos in 2024 that will try to
steal some of its rocks.

What's all the fuss about? For many, the desire to visit Phobos and
Deimos was galvanized by their deeply mysterious nature. ``They're super
weird, confusing and interesting,'' said
\href{https://science.jpl.nasa.gov/people/Fraeman/}{Abigail Fraeman}, a
planetary scientist studying Mars, Phobos and Deimos at NASA's Jet
Propulsion Laboratory.

We don't know where the moons came from because they look like asteroids
foreign to the red planet but behave like byproducts of Mars' early,
impact-laden history. And if that Japanese mission manages to grab some
samples and decode the chemistry of the mangled moons, we might be able
to discover their origins. In doing so, we won't just gain
\href{https://www.nytimes3xbfgragh.onion/2020/07/24/science/mars-life-water.html}{a
better understanding of Mars' ancient past}. We will also be able to
peer back in time to the chaos that shaped the early solar system.

\includegraphics{https://static01.graylady3jvrrxbe.onion/images/2020/07/28/science/28SCI-MARSMOONS2/28SCI-MARSMOONS2-articleLarge.jpg?quality=75\&auto=webp\&disable=upscale}

Much of what we know about Phobos and Deimos comes from serendipitous
observations. Rovers on the surface as well as mechanical eyes orbiting
Mars are sometimes in the right place to angle their cameras and
\href{https://www.jpl.nasa.gov/news/news.php?feature=7674}{snap}
\href{https://mars.nasa.gov/resources/22392/curiosity-observes-phobos-eclipse-sol-2359/}{shots}
of the two moons. Phobos, being larger and closer to Mars, can be seen
in greater detail: a misshapen mess scarred by a large crater and
multiple grooves that look like they were made by a cosmic cat's claws.

Remote observations of their surfaces haven't revealed any standout
mineral features or textures that could definitively detail the moons'
overall compositions and, ultimately, their origins, said
\href{https://science.jpl.nasa.gov/people/Kerber/}{Laura Kerber}, the
Mars Odyssey spacecraft's deputy project scientist at NASA's Jet
Propulsion Laboratory.

``They check all of the boxes that are consistent with them being these
captured asteroids,'' said Dr. Fraeman --- rubbly patchworks that
drifted too close to Mars long ago and became trapped in the planet's
orbit.

But both moons orbit the equator in a neat-and-tidy circular fashion,
which suggests they coalesced from a disk of debris that danced around a
young Mars. It's difficult to capture an asteroid and have it ``wind up
in this beautiful, symmetric, circular orbit,'' said
\href{https://science.jpl.nasa.gov/people/Plaut/}{Jeffrey Plaut}, the
project scientist for the Mars Odyssey mission.

Mars, having a tenth of Earth's mass, has a relatively weak
gravitational pull, so it seems improbable that it would be able to
capture asteroids zipping by, said
\href{https://planetb.sci.isas.jaxa.jp/aqua/pages/people.html}{Tomohiro
Usui}, a robotic planetary exploration expert at the Japan Aerospace
Exploration Agency. But if they formed from a debris disk launched up
from Mars after a colossal impact, Deimos should be orbiting closer to
Mars than it is today.

Reconciling their appearances with their orbits is difficult.

``They just shouldn't exist,'' said Dr. Fraeman. ``They don't make any
sense.''

Cast as siblings, both moons may not even have the same origin story.

\includegraphics{https://static01.graylady3jvrrxbe.onion/images/2020/07/28/autossell/28SCI-MARSMOONS-cover/PIA17352-superJumbo.jpg}

In fewer than
\href{https://sci.esa.int/web/mars-express/-/31031-phobos}{100 million
years}, said
\href{https://www.seti.org/our-scientists/matija-cuk}{Matija Ćuk}, a
research scientist at the SETI Institute in Mountain View, Calif.,
Phobos will get so close to Mars that its gravity will tear the moon
apart, transforming into a beautiful system of rings.

It won't be the first time, some scientists say.
\href{https://www.nature.com/articles/ngeo2916}{Recent}
\href{https://iopscience.iop.org/article/10.3847/2041-8213/ab974f}{calculations}
suggest that Phobos was once 20 times more massive. But, as one
hypothesis goes, it drifted toward Mars and shattered into ring
material, much of it raining onto Mars. The remaining ring material
clumped together into a new, smaller Phobos. This cycle has repeated
several times over billions of years, with Phobos shrinking with every
completed cycle.

Although made of ancient matter, the Phobos we see today may have been
assembled just 200 million years ago. If it were confirmed that Phobos
is a haphazardly clumped-together mass, it would be a revelation,
suggesting planets with rings
\href{https://webbtelescope.org/resource-gallery/articles/pagecontent/filter-articles/why-do-planets-have-rings?filterUUID=a776e097-0c60-421c-baec-1d8ad049bfb0}{are
the norm} for our solar system.

Tiny Deimos, on the other hand, may have remained intact for 3.5 billion
years. It is on track to eventually escape Mars entirely, Dr. Usui says,
to wander elsewhere in the void.

NASA's Perseverance rover, launching
\href{https://spaceflightnow.com/2020/06/24/launch-of-nasas-perseverance-mars-rover-delayed-to-july-22/}{July
30}, is the first stage of a protracted series of missions to
\href{https://www.scientificamerican.com/article/rocks-rockets-and-robots-the-plan-to-bring-mars-down-to-earth/}{bring
pristine samples of Mars back to Earth} for study. Mars's atmosphere,
ancient volcanism and flowing water has eroded away many of the planet's
earliest rocks. But if Phobos is made of Martian flotsam from a massive
impact shortly after the planet's birth, then the moon has retained the
earliest memories of Mars.

Image

Phobos, seen by the Viking spacecraft during a 1978
flyby.Credit...Viking Project/JPL/NASA

``Many theories suggest the terrestrial planets, including the Earth,
formed dry, and water was delivered by icy meteorites that were
scattered inward,'' said Dr. Usui. ``If the moons are captured
asteroids, they are evidence of this process in action and their
composition shows what materials were delivered to the early Earth.''

Meteors crashing into Mars could have coated Phobos in a fine layer of
Martian dust sourced from all over the planet. This matter may be both
very young and extremely old, and grabbing some of it will help
scientists unpack ``how Mars may have progressed from a habitable world
to an uninhabitable one,'' said Dr. Usui.

With one successful (albeit
\href{https://gizmodo.com/everything-that-could-go-wrong-for-hayabusa-did-and-ye-1730940605}{problem-riddled})
asteroid sample-return mission under its belt and
\href{https://www.nytimes3xbfgragh.onion/2020/03/23/science/japan-hayabusa2-asteroid-ryugu.html}{another}
speeding back to Earth with more asteroid matter, the Japanese space
agency has a ``fairly good track record'' at snatching up space rocks,
said Dr. Fraeman. Hopes are understandably high for Japan's Martian
Moons eXploration, or MMX
\href{http://mmx.isas.jaxa.jp/en/mission/}{mission}, which will arrive
at Mars in 2025.

Should it be successful, these huge planetary science questions may
finally have clear answers. We will come to see not just Mars, but also
Earth, in a new light.

``To me, that's really cool,'' said Dr. Fraeman.

Advertisement

\protect\hyperlink{after-bottom}{Continue reading the main story}

\hypertarget{site-index}{%
\subsection{Site Index}\label{site-index}}

\hypertarget{site-information-navigation}{%
\subsection{Site Information
Navigation}\label{site-information-navigation}}

\begin{itemize}
\tightlist
\item
  \href{https://help.nytimes3xbfgragh.onion/hc/en-us/articles/115014792127-Copyright-notice}{©~2020~The
  New York Times Company}
\end{itemize}

\begin{itemize}
\tightlist
\item
  \href{https://www.nytco.com/}{NYTCo}
\item
  \href{https://help.nytimes3xbfgragh.onion/hc/en-us/articles/115015385887-Contact-Us}{Contact
  Us}
\item
  \href{https://www.nytco.com/careers/}{Work with us}
\item
  \href{https://nytmediakit.com/}{Advertise}
\item
  \href{http://www.tbrandstudio.com/}{T Brand Studio}
\item
  \href{https://www.nytimes3xbfgragh.onion/privacy/cookie-policy\#how-do-i-manage-trackers}{Your
  Ad Choices}
\item
  \href{https://www.nytimes3xbfgragh.onion/privacy}{Privacy}
\item
  \href{https://help.nytimes3xbfgragh.onion/hc/en-us/articles/115014893428-Terms-of-service}{Terms
  of Service}
\item
  \href{https://help.nytimes3xbfgragh.onion/hc/en-us/articles/115014893968-Terms-of-sale}{Terms
  of Sale}
\item
  \href{https://spiderbites.nytimes3xbfgragh.onion}{Site Map}
\item
  \href{https://help.nytimes3xbfgragh.onion/hc/en-us}{Help}
\item
  \href{https://www.nytimes3xbfgragh.onion/subscription?campaignId=37WXW}{Subscriptions}
\end{itemize}
