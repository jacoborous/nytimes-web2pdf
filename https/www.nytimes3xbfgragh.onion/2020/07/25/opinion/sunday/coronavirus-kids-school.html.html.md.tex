Sections

SEARCH

\protect\hyperlink{site-content}{Skip to
content}\protect\hyperlink{site-index}{Skip to site index}

\href{https://www.nytimes3xbfgragh.onion/section/opinion/sunday}{Sunday
Review}

\href{https://myaccount.nytimes3xbfgragh.onion/auth/login?response_type=cookie\&client_id=vi}{}

\href{https://www.nytimes3xbfgragh.onion/section/todayspaper}{Today's
Paper}

\href{/section/opinion/sunday}{Sunday Review}\textbar{}`Online School Is
Not the End of the World'

\url{https://nyti.ms/2Bvaq7z}

\begin{itemize}
\item
\item
\item
\item
\item
\end{itemize}

Advertisement

\protect\hyperlink{after-top}{Continue reading the main story}

\href{/section/opinion}{Opinion}

Supported by

\protect\hyperlink{after-sponsor}{Continue reading the main story}

\hypertarget{online-school-is-not-the-end-of-the-world}{%
\section{`Online School Is Not the End of the
World'}\label{online-school-is-not-the-end-of-the-world}}

Students tell us how they feel about the prospect of returning to the
classroom.

\href{https://www.nytimes3xbfgragh.onion/by/lora-kelley}{\includegraphics{https://static01.graylady3jvrrxbe.onion/images/2020/07/01/opinion/lora-kelley-author/lora-kelley-author-thumbLarge.png}}

By \href{https://www.nytimes3xbfgragh.onion/by/lora-kelley}{Lora Kelley}

Ms. Kelley is an editorial assistant.

\begin{itemize}
\item
  July 25, 2020
\item
  \begin{itemize}
  \item
  \item
  \item
  \item
  \item
  \end{itemize}
\end{itemize}

\includegraphics{https://static01.graylady3jvrrxbe.onion/images/2020/07/26/opinion/sunday/25opreaders-kids/25opreaders-kids-articleLarge.jpg?quality=75\&auto=webp\&disable=upscale}

We've heard from the adults about reopening schools: Teachers have
presented safety concerns, politicians have presented political
concerns, health experts have presented health concerns, parents have
presented parenting concerns, and so on.

But what about the people who will actually be sitting in classrooms (or
not) come September?

We wanted to hear directly from the students, so we invited readers to
ask their kids how they felt about going back to school, and if they do
go back, what precautions they think schools should be taking.

Kids don't get to set their own bedtimes; there's a good reason they
aren't determining state reopening timelines. But they do have a lot at
stake, and a lot to say. More than 600 parents wrote in on behalf of
their children, ages 5 to 17. Overwhelmingly, we heard that students
were nervous and confused and in many cases, afraid for themselves,
their parents and their teachers.

Many were uninspired by the idea of more remote classes. Online learning
``sucks,'' said Anthony, 15, ``but I'd rather do that than die!''

And yet they also missed their friends and the library and rituals of
the classroom. Eugene Koesmadjie, 11, summed it up like this: ``What I
feel about going to school is that I think it is wonderful.''

Below is a selection of comments from students across the country,
presented in their own words. They have been condensed and edited for
clarity.

\begin{center}\rule{0.5\linewidth}{\linethickness}\end{center}

\hypertarget{after-one-virus-the-world-has-totally-turned-upside-down}{%
\subsection{`After one virus the world has totally turned upside
down'}\label{after-one-virus-the-world-has-totally-turned-upside-down}}

I personally do not think that it is safe to go back to school. Although
I love school because it is the best place to enhance my mind in
knowledge, I know that after one virus the world has totally turned
upside down. Times are really dangerous now because this new virus is
killing thousands of people in seconds, even though they don't feel it
until it's too late. Although there are mixed feeling about this topic,
I truly believe that if we continue the distance learning even during
the school year, we can gradually cease the flow of the virus.

Online learning is honestly not so bad because it is a way to help us
learn while protecting our lives, and during this grave crisis this is
the best way to learn. As the schools are starting to reopen, I feel
that the teachers and principals should continue doing the distance
learning because although this virus is not spreading so much, the
pandemic is not over. --- \emph{Jediael Chintha, 12, Hanover Park, Ill.}

\hypertarget{the-teachers-should-make-a-game-out-of-what-we-are-learning}{%
\subsection{`The teachers should make a game out of what we are
learning'}\label{the-teachers-should-make-a-game-out-of-what-we-are-learning}}

I'll feel sad if the library is closed because of bacteria and germs.
All of my teachers have said, ``You have a great mind for reading.''
Reading is what I do in my free time. Another thing I do is writing
stories for fun with my friends.

To make school safer I will want smaller classes. There should be
different times for lunch. When going outside, there should be only one
or two classes outside at the same time. At the end of the day the
teachers should sanitize everything. Teachers and children should wear
masks except when we eat. Also, we should only do two days of distant
learning and three days of real school. Students should bring their own
pencil sharpeners, tissues and sanitizer to school to keep things safer.

Teachers should record what they are teaching. Then they can give the
videos to parents if their children don't understand the lesson. The
teachers should make a game out of what we are learning. It can be a fun
review so students can learn and have a good time. Students and teachers
should have one on one meetings. One on one meetings would help improve
their grade if they do it right. Teachers can do their best with games
for online learning and sharing screens so it can be fun during this
hard time. --- \emph{Mia Greene, 9, Philadelphia}

\hypertarget{i-realize-that-i-am-extremely-lucky-to-have-a-computer-to-learn}{%
\subsection{`I realize that I am extremely lucky to have a computer to
learn'}\label{i-realize-that-i-am-extremely-lucky-to-have-a-computer-to-learn}}

While I wish we could go back to school feeling safe and sure of
ourselves, I don't think it's a very realistic idea. It is impractical
to have teachers be the ``mask police'' in addition to their teaching
responsibilities.

If we were to go back to school, however, I would suggest staggering the
number of students on campus each week. Presumably, water fountains
would be removed and doors would be kept propped open to help students
avoid contact with door handles. Because classrooms would be unable to
accommodate as many students, public and outdoor venues such as parks,
beaches and amphitheaters could be used to teach students, too.

Online learning has been difficult for me. It is much more challenging
to connect with teachers and peers through a screen. But overall, I
realize that I am extremely lucky to have a computer to learn with and a
quiet space to work in this troubling time. --- \emph{Anya Shah, 13, Los
Angeles}

\hypertarget{it-is-inevitable-that-i-will-bring-the-virus-home-to-my-family}{%
\subsection{`It is inevitable that I will bring the virus home to my
family'}\label{it-is-inevitable-that-i-will-bring-the-virus-home-to-my-family}}

I don't feel comfortable returning to in-person classes. Living in
N.Y.C., between relying on overcrowded public transit lines and school
buildings that are undersized for the number of kids they serve, it is
inevitable that I will bring the virus home to my family. I don't see
how they can cut the education budget and still hire people to do
testing, extra teachers and extra cleaning staff.

I'm typically a pretty antisocial person and even I find online learning
very isolating. It's very hard to meet people online, because typically
meeting people in person happens through some shared experience like
being at the same table or working on a project together. Those
opportunities are no longer available.

I wish they could fund schools better in the first place. I have thought
about this a lot but there is no silver bullet. --- \emph{Wilson Prieve,
15, Brooklyn}

\hypertarget{while-i-am-not-super-worried-about-dying-from-the-virus-i-do-not-want-to-get-it}{%
\subsection{`While I am not super worried about dying from the virus, I
do not want to get
it'}\label{while-i-am-not-super-worried-about-dying-from-the-virus-i-do-not-want-to-get-it}}

I am not super comfortable returning to class. This is mainly because I
do not have confidence that everyone in my area has been in quarantine.
While I am not super worried about dying from the virus, I do not want
to get it. Also, while I will not go to school without wearing a mask, I
find masks hard to breathe in and am not looking forward to long school
days in them. However, I acknowledge that health care workers do it
every day and really appreciate it.

I think that if everyone in each school isolated for three weeks before
school started, and social distanced from all strangers, then the risk
of having an outbreak in a school would be pretty low. Masks and
distancing in the school should still take place just in case. Americans
are not good at giving up their freedom for the greater good. Citizens
of the United States will continue to get sick with Covid-19 until there
is a vaccine. That is just the sad truth.

Online school is not the end of the world. It has turned out to be
mediocre. Some teachers adapt well, while others do the bare minimum.
But I can't say that isn't true with in-person school. Some teachers and
classes are better than others. Overall, I would say that online school
is comparable to in person. The main difference is that after-school
activities and electives are not the same.--- \emph{Oliver Stockman, 16,
Swarthmore, Pa.}

\hypertarget{i-will-only-get-to-attend-school-once-a-week}{%
\subsection{`I will only get to attend school once a
week'}\label{i-will-only-get-to-attend-school-once-a-week}}

Being a senior was something that I've looked forward to since my
freshman year. Taking the hardest classes, having a spot in the senior
parking lot and being able to get the best out of high school all in one
year. Now, many of the typical ``back to school'' activities like
homecoming, senior fun day and painting cars are no longer able to take
place. My school has over 2,000 students in attendance, with about 500
kids per grade. This large number of students means that I will only get
to attend school once a week in person, and with a group of people
grouped by last name.

The portion of my junior year that I did complete through e-learning was
not the worst experience. Teachers made assignments on Google classroom
(our learning platform) and students had to complete the assignment by
11:59 p.m. on the day it was due. To me, this made life much easier, as
I got to wake up much later and actually get sleep, as well as have time
to eat. This made for a much easier school year. Although my senior year
experience is affected, I feel as though I will be less stressed overall
and be safe at the same time, which is all that one can ask for through
a pandemic. --- \emph{Riya Goel, 16, West Orange, N.J.}

\emph{The Times is committed to publishing}
\href{https://www.nytimes3xbfgragh.onion/2019/01/31/opinion/letters/letters-to-editor-new-york-times-women.html}{\emph{a
diversity of letters}} \emph{to the editor. We'd like to hear what you
think about this or any of our articles. Here are some}
\href{https://help.nytimes3xbfgragh.onion/hc/en-us/articles/115014925288-How-to-submit-a-letter-to-the-editor}{\emph{tips}}\emph{.
And here's our email:}
\href{mailto:letters@NYTimes.com}{\emph{letters@NYTimes.com}}\emph{.}

\emph{Follow The New York Times Opinion section on}
\href{https://www.facebookcorewwwi.onion/nytopinion}{\emph{Facebook}}\emph{,}
\href{http://twitter.com/NYTOpinion}{\emph{Twitter (@NYTopinion)}}
\emph{and}
\href{https://www.instagram.com/nytopinion/}{\emph{Instagram}}\emph{.}

Advertisement

\protect\hyperlink{after-bottom}{Continue reading the main story}

\hypertarget{site-index}{%
\subsection{Site Index}\label{site-index}}

\hypertarget{site-information-navigation}{%
\subsection{Site Information
Navigation}\label{site-information-navigation}}

\begin{itemize}
\tightlist
\item
  \href{https://help.nytimes3xbfgragh.onion/hc/en-us/articles/115014792127-Copyright-notice}{©~2020~The
  New York Times Company}
\end{itemize}

\begin{itemize}
\tightlist
\item
  \href{https://www.nytco.com/}{NYTCo}
\item
  \href{https://help.nytimes3xbfgragh.onion/hc/en-us/articles/115015385887-Contact-Us}{Contact
  Us}
\item
  \href{https://www.nytco.com/careers/}{Work with us}
\item
  \href{https://nytmediakit.com/}{Advertise}
\item
  \href{http://www.tbrandstudio.com/}{T Brand Studio}
\item
  \href{https://www.nytimes3xbfgragh.onion/privacy/cookie-policy\#how-do-i-manage-trackers}{Your
  Ad Choices}
\item
  \href{https://www.nytimes3xbfgragh.onion/privacy}{Privacy}
\item
  \href{https://help.nytimes3xbfgragh.onion/hc/en-us/articles/115014893428-Terms-of-service}{Terms
  of Service}
\item
  \href{https://help.nytimes3xbfgragh.onion/hc/en-us/articles/115014893968-Terms-of-sale}{Terms
  of Sale}
\item
  \href{https://spiderbites.nytimes3xbfgragh.onion}{Site Map}
\item
  \href{https://help.nytimes3xbfgragh.onion/hc/en-us}{Help}
\item
  \href{https://www.nytimes3xbfgragh.onion/subscription?campaignId=37WXW}{Subscriptions}
\end{itemize}
