Sections

SEARCH

\protect\hyperlink{site-content}{Skip to
content}\protect\hyperlink{site-index}{Skip to site index}

\href{https://www.nytimes3xbfgragh.onion/section/world/middleeast}{Middle
East}

\href{https://myaccount.nytimes3xbfgragh.onion/auth/login?response_type=cookie\&client_id=vi}{}

\href{https://www.nytimes3xbfgragh.onion/section/todayspaper}{Today's
Paper}

\href{/section/world/middleeast}{Middle East}\textbar{}Laid Off and
Locked Up: Virus Traps Domestic Workers in Arab States

\url{https://nyti.ms/2Z1vUSA}

\begin{itemize}
\item
\item
\item
\item
\item
\end{itemize}

\href{https://www.nytimes3xbfgragh.onion/news-event/coronavirus?action=click\&pgtype=Article\&state=default\&region=TOP_BANNER\&context=storylines_menu}{The
Coronavirus Outbreak}

\begin{itemize}
\tightlist
\item
  live\href{https://www.nytimes3xbfgragh.onion/2020/08/04/world/coronavirus-cases.html?action=click\&pgtype=Article\&state=default\&region=TOP_BANNER\&context=storylines_menu}{Latest
  Updates}
\item
  \href{https://www.nytimes3xbfgragh.onion/interactive/2020/us/coronavirus-us-cases.html?action=click\&pgtype=Article\&state=default\&region=TOP_BANNER\&context=storylines_menu}{Maps
  and Cases}
\item
  \href{https://www.nytimes3xbfgragh.onion/interactive/2020/science/coronavirus-vaccine-tracker.html?action=click\&pgtype=Article\&state=default\&region=TOP_BANNER\&context=storylines_menu}{Vaccine
  Tracker}
\item
  \href{https://www.nytimes3xbfgragh.onion/2020/08/02/us/covid-college-reopening.html?action=click\&pgtype=Article\&state=default\&region=TOP_BANNER\&context=storylines_menu}{College
  Reopening}
\item
  \href{https://www.nytimes3xbfgragh.onion/live/2020/08/04/business/stock-market-today-coronavirus?action=click\&pgtype=Article\&state=default\&region=TOP_BANNER\&context=storylines_menu}{Economy}
\end{itemize}

Advertisement

\protect\hyperlink{after-top}{Continue reading the main story}

Supported by

\protect\hyperlink{after-sponsor}{Continue reading the main story}

\hypertarget{laid-off-and-locked-up-virus-traps-domestic-workers-in-arab-states}{%
\section{Laid Off and Locked Up: Virus Traps Domestic Workers in Arab
States}\label{laid-off-and-locked-up-virus-traps-domestic-workers-in-arab-states}}

The pandemic and economic crises have caused many workers to lose their
jobs. Some have been detained, abused, deprived of wages and stranded
far from home with nowhere to turn for help.

\includegraphics{https://static01.graylady3jvrrxbe.onion/images/2020/07/05/world/05mideast-labor/merlin_173189445_5f11297e-9cb1-49af-87c6-5f53754aecf6-articleLarge.jpg?quality=75\&auto=webp\&disable=upscale}

By \href{https://www.nytimes3xbfgragh.onion/by/ben-hubbard}{Ben Hubbard}
and Louise Donovan

\begin{itemize}
\item
  July 6, 2020
\item
  \begin{itemize}
  \item
  \item
  \item
  \item
  \item
  \end{itemize}
\end{itemize}

BEIRUT, Lebanon --- When the nine African women lost their jobs as
domestic workers in Saudi Arabia because of the coronavirus lockdown,
the agency that had recruited them stuffed them in a bare room with a
few thin mattresses and locked the door.

Some have been there since March. One is now six months pregnant but
receiving no maternity care. Another tore her clothes off in a fit of
distress, so the agency chained her to a wall.

The women receive food once a day, they said, but don't know when they
will get out, much less be able to return to their countries.

``Everybody is fearing,'' one of the women, Apisaki, from Kenya, said
via WhatsApp. ``The environment here is not good. No one will listen to
our voice.''

Families in many Arab countries rely on millions of low-paid workers
from Asia and Africa to drive their cars, clean their homes and care for
their children and elderly relatives under conditions that rights groups
have long said allow exploitation and abuse.

Now, the pandemic and associated economic downturns have exacerbated
these dangers. Many families will not let their housekeepers leave the
house, fearing they will bring back the virus, while requiring them to
work more since entire families are staying home, workers' advocates
say.

Other workers have been laid off, deprived of wages and left stranded
far from home with nowhere to turn for help.

\includegraphics{https://static01.graylady3jvrrxbe.onion/images/2020/07/05/world/05mideast-labor4/merlin_170989203_3b6ca8d4-2ff2-4429-9f7e-bd3396d72b62-articleLarge.jpg?quality=75\&auto=webp\&disable=upscale}

In Lebanon, employers have deposited scores of Ethiopian women in front
of their country's consulate in Beirut because they could no longer pay
them as
\href{https://www.nytimes3xbfgragh.onion/2020/05/10/world/middleeast/lebanon-economic-crisis.html}{the
economy imploded}.

Persian Gulf countries alone had nearly four million domestic laborers
in 2016, more than half of them women, according
to\href{http://abudhabidialogue.org/sites/default/files/document-library/2018_Future\%20of\%20Domestic\%20Work\%20Study.pdf}{a
study for the Abu Dhabi Dialogue}, which focuses on migrant labor in the
region. Experts say the real number has risen since and is probably much
higher.

Hundreds of thousands of foreign housekeepers and nannies work in other
Arab countries, including Lebanon and Jordan, giving the Arab world the
most female domestic migrant workers of any region, according to the
International Labor Organization.

\hypertarget{latest-updates-global-coronavirus-outbreak}{%
\section{\texorpdfstring{\href{https://www.nytimes3xbfgragh.onion/2020/08/04/world/coronavirus-cases.html?action=click\&pgtype=Article\&state=default\&region=MAIN_CONTENT_1\&context=storylines_live_updates}{Latest
Updates: Global Coronavirus
Outbreak}}{Latest Updates: Global Coronavirus Outbreak}}\label{latest-updates-global-coronavirus-outbreak}}

Updated 2020-08-04T21:34:02.738Z

\begin{itemize}
\tightlist
\item
  \href{https://www.nytimes3xbfgragh.onion/2020/08/04/world/coronavirus-cases.html?action=click\&pgtype=Article\&state=default\&region=MAIN_CONTENT_1\&context=storylines_live_updates\#link-2daa96b5}{As
  talks drag on, McConnell signals openness to jobless aid extension
  that Republicans have opposed.}
\item
  \href{https://www.nytimes3xbfgragh.onion/2020/08/04/world/coronavirus-cases.html?action=click\&pgtype=Article\&state=default\&region=MAIN_CONTENT_1\&context=storylines_live_updates\#link-1228a480}{Novavax
  sees encouraging results from two studies of its experimental
  vaccine.}
\item
  \href{https://www.nytimes3xbfgragh.onion/2020/08/04/world/coronavirus-cases.html?action=click\&pgtype=Article\&state=default\&region=MAIN_CONTENT_1\&context=storylines_live_updates\#link-4825b93}{Public
  and private schools in Maryland and elsewhere are divided over
  in-person instruction.}
\end{itemize}

\href{https://www.nytimes3xbfgragh.onion/2020/08/04/world/coronavirus-cases.html?action=click\&pgtype=Article\&state=default\&region=MAIN_CONTENT_1\&context=storylines_live_updates}{See
more updates}

More live coverage:
\href{https://www.nytimes3xbfgragh.onion/live/2020/08/04/business/stock-market-today-coronavirus?action=click\&pgtype=Article\&state=default\&region=MAIN_CONTENT_1\&context=storylines_live_updates}{Markets}

Most come to the Middle East through recruitment agencies and are
employed under a sponsorship system that links their residency status to
their jobs, giving their employers tremendous power. In many cases, they
cannot quit without losing their residency, or move to new jobs or leave
the country without an employer's permission.

And in practice, many employers confiscate workers' passports and
deprive them of time off, rights groups say. Some prevent them from
using cellphones or the internet. Physical and sexual abuse are common.

The combination of their gender, the sponsorship system and their
isolation makes female domestic workers especially vulnerable, said Vani
Saraswathi, associate editor
of\href{https://www.migrant-rights.org/}{Migrant-Rights.org}, an
advocacy group.

``You have this person who controls your every movement, and you are in
their house 24-7,'' she said, ``so imagine the kind of power that gives
them.''

Alarm among such workers rose as Covid-19, the disease caused by the
coronavirus, spread across the Middle East and shook the economies many
migrants depend on.

``Even in cases of extreme abuse, workers are hesitant to leave their
employers, as they fear being made completely homeless,'' Ms. Saraswathi
said.

Image

Protesting the mistreatment of domestic workers in Beirut last year.
Rights groups have long said that working conditions for these migrants
leave them vulnerable to exploitation and abuse.Credit...Wael
Hamzeh/EPA, via Shutterstock

Dozens of Kenyan women in Saudi Arabia have complained of ``not enough
food, no rest, violence, even being threatened, trapped and monitored,''
said Ruth Khakame, chairwoman of the National Domestic Workers Council
of\href{http://www.kudheiha.org/}{Kudheiha}, a Kenyan union. ``You're
being denied from using your phone. So you're struggling, you're alone
and you've nowhere to turn.''

Fear of contagion has upset relations between many domestic workers and
their employers. Some who used to get breaks when their employers left
for the office now have to serve and clean up after entire families
stuck at home all day. Other families distrust their workers as
potential vectors for the virus.

``From the beginning, my employers were not friendly,'' said Justine
Mukisa, 33, a Ugandan working in Oman. But during the pandemic, her
salary of about \$180 per month has been cut in half, her workload has
increased and her employers have grown hostile.

``Before coronavirus, we sometimes played with the children,'' she said.
``Now this is not allowed. My employers do not want me to touch their
food or sit near them.''

In recent years, a number of countries have passed regulations regarding
domestic workers, granting them one day off each week, annual or
biannual leave and an end-of-service benefit based on length of
employment.

Qatar has capped the workday at 10 hours, the
\href{https://www.nytimes3xbfgragh.onion/2020/07/14/science/mars-united-arab-emirates.html}{United
Arab Emirates} and Kuwait at 12 hours and Saudi Arabia at 15 hours.
Kuwait has a monthly minimum wage of about \$195 for domestic workers.
Kenyans in Saudi Arabia are
\href{https://www.standardmedia.co.ke/business/article/2001310333/pact-with-saudi-arabia-raises-kenyan-workers-minimum-pay}{supposed
to earn} at least \$375 per month plus benefits, and the Philippines has
set a \$400 minimum wage for its citizens across countries.

As the coronavirus has spread, Bahrain, Kuwait and the Emirates have
facilitated visa renewals to stranded migrants to help them avoid fines
and detention if their residency status lapses. Qatar and Saudi Arabia
have announced free treatment for migrant workers who get Covid-19.

But workers' advocates say that enforcement of regulations is often
spotty and that those who face abuse have little recourse.

``The way these countries have perfected this system of disposable labor
lends itself to a high level of exploitation,'' said Mustafa Qadri, the
executive director of\href{https://www.equidemresearch.org/}{Equidem}, a
labor rights organization based in Britain.

Image

Ethiopian women dismissed by their employers gathered last month with
their belongings outside the Ethiopian Consulate in
Lebanon.Credit...Joseph Eid/Agence France-Presse --- Getty Images

Those who get the virus can be easily discarded by their employers.

Two months ago, Hanico Quinlat, a Filipino domestic worker in Saudi
Arabia, came down with a severe headache and moved into her agency's
hostel to recover. When she tested positive for Covid-19, the agency
supervisor locked her in a room alone, giving her only painkillers and
vitamin C to treat her illness.

``When they give us food, they throw it into the room,'' Ms. Quinlat
said by telephone from the room where she was being held. ``We are
people, not animals.''

Among the most vulnerable workers are women who have fled their
employers or entered countries on tourist visas, hoping to freelance.

\href{https://www.nytimes3xbfgragh.onion/news-event/coronavirus?action=click\&pgtype=Article\&state=default\&region=MAIN_CONTENT_3\&context=storylines_faq}{}

\hypertarget{the-coronavirus-outbreak-}{%
\subsubsection{The Coronavirus Outbreak
›}\label{the-coronavirus-outbreak-}}

\hypertarget{frequently-asked-questions}{%
\paragraph{Frequently Asked
Questions}\label{frequently-asked-questions}}

Updated August 4, 2020

\begin{itemize}
\item ~
  \hypertarget{i-have-antibodies-am-i-now-immune}{%
  \paragraph{I have antibodies. Am I now
  immune?}\label{i-have-antibodies-am-i-now-immune}}

  \begin{itemize}
  \tightlist
  \item
    As of right
    now,\href{https://www.nytimes3xbfgragh.onion/2020/07/22/health/covid-antibodies-herd-immunity.html?action=click\&pgtype=Article\&state=default\&region=MAIN_CONTENT_3\&context=storylines_faq}{that
    seems likely, for at least several months.} There have been
    frightening accounts of people suffering what seems to be a second
    bout of Covid-19. But experts say these patients may have a
    drawn-out course of infection, with the virus taking a slow toll
    weeks to months after initial exposure. People infected with the
    coronavirus typically
    \href{https://www.nature.com/articles/s41586-020-2456-9}{produce}
    immune molecules called antibodies, which are
    \href{https://www.nytimes3xbfgragh.onion/2020/05/07/health/coronavirus-antibody-prevalence.html?action=click\&pgtype=Article\&state=default\&region=MAIN_CONTENT_3\&context=storylines_faq}{protective
    proteins made in response to an
    infection}\href{https://www.nytimes3xbfgragh.onion/2020/05/07/health/coronavirus-antibody-prevalence.html?action=click\&pgtype=Article\&state=default\&region=MAIN_CONTENT_3\&context=storylines_faq}{.
    These antibodies may} last in the body
    \href{https://www.nature.com/articles/s41591-020-0965-6}{only two to
    three months}, which may seem worrisome, but that's perfectly normal
    after an acute infection subsides, said Dr. Michael Mina, an
    immunologist at Harvard University. It may be possible to get the
    coronavirus again, but it's highly unlikely that it would be
    possible in a short window of time from initial infection or make
    people sicker the second time.
  \end{itemize}
\item ~
  \hypertarget{im-a-small-business-owner-can-i-get-relief}{%
  \paragraph{I'm a small-business owner. Can I get
  relief?}\label{im-a-small-business-owner-can-i-get-relief}}

  \begin{itemize}
  \tightlist
  \item
    The
    \href{https://www.nytimes3xbfgragh.onion/article/small-business-loans-stimulus-grants-freelancers-coronavirus.html?action=click\&pgtype=Article\&state=default\&region=MAIN_CONTENT_3\&context=storylines_faq}{stimulus
    bills enacted in March} offer help for the millions of American
    small businesses. Those eligible for aid are businesses and
    nonprofit organizations with fewer than 500 workers, including sole
    proprietorships, independent contractors and freelancers. Some
    larger companies in some industries are also eligible. The help
    being offered, which is being managed by the Small Business
    Administration, includes the Paycheck Protection Program and the
    Economic Injury Disaster Loan program. But lots of folks have
    \href{https://www.nytimes3xbfgragh.onion/interactive/2020/05/07/business/small-business-loans-coronavirus.html?action=click\&pgtype=Article\&state=default\&region=MAIN_CONTENT_3\&context=storylines_faq}{not
    yet seen payouts.} Even those who have received help are confused:
    The rules are draconian, and some are stuck sitting on
    \href{https://www.nytimes3xbfgragh.onion/2020/05/02/business/economy/loans-coronavirus-small-business.html?action=click\&pgtype=Article\&state=default\&region=MAIN_CONTENT_3\&context=storylines_faq}{money
    they don't know how to use.} Many small-business owners are getting
    less than they expected or
    \href{https://www.nytimes3xbfgragh.onion/2020/06/10/business/Small-business-loans-ppp.html?action=click\&pgtype=Article\&state=default\&region=MAIN_CONTENT_3\&context=storylines_faq}{not
    hearing anything at all.}
  \end{itemize}
\item ~
  \hypertarget{what-are-my-rights-if-i-am-worried-about-going-back-to-work}{%
  \paragraph{What are my rights if I am worried about going back to
  work?}\label{what-are-my-rights-if-i-am-worried-about-going-back-to-work}}

  \begin{itemize}
  \tightlist
  \item
    Employers have to provide
    \href{https://www.osha.gov/SLTC/covid-19/standards.html}{a safe
    workplace} with policies that protect everyone equally.
    \href{https://www.nytimes3xbfgragh.onion/article/coronavirus-money-unemployment.html?action=click\&pgtype=Article\&state=default\&region=MAIN_CONTENT_3\&context=storylines_faq}{And
    if one of your co-workers tests positive for the coronavirus, the
    C.D.C.} has said that
    \href{https://www.cdc.gov/coronavirus/2019-ncov/community/guidance-business-response.html}{employers
    should tell their employees} -\/- without giving you the sick
    employee's name -\/- that they may have been exposed to the virus.
  \end{itemize}
\item ~
  \hypertarget{should-i-refinance-my-mortgage}{%
  \paragraph{Should I refinance my
  mortgage?}\label{should-i-refinance-my-mortgage}}

  \begin{itemize}
  \tightlist
  \item
    \href{https://www.nytimes3xbfgragh.onion/article/coronavirus-money-unemployment.html?action=click\&pgtype=Article\&state=default\&region=MAIN_CONTENT_3\&context=storylines_faq}{It
    could be a good idea,} because mortgage rates have
    \href{https://www.nytimes3xbfgragh.onion/2020/07/16/business/mortgage-rates-below-3-percent.html?action=click\&pgtype=Article\&state=default\&region=MAIN_CONTENT_3\&context=storylines_faq}{never
    been lower.} Refinancing requests have pushed mortgage applications
    to some of the highest levels since 2008, so be prepared to get in
    line. But defaults are also up, so if you're thinking about buying a
    home, be aware that some lenders have tightened their standards.
  \end{itemize}
\item ~
  \hypertarget{what-is-school-going-to-look-like-in-september}{%
  \paragraph{What is school going to look like in
  September?}\label{what-is-school-going-to-look-like-in-september}}

  \begin{itemize}
  \tightlist
  \item
    It is unlikely that many schools will return to a normal schedule
    this fall, requiring the grind of
    \href{https://www.nytimes3xbfgragh.onion/2020/06/05/us/coronavirus-education-lost-learning.html?action=click\&pgtype=Article\&state=default\&region=MAIN_CONTENT_3\&context=storylines_faq}{online
    learning},
    \href{https://www.nytimes3xbfgragh.onion/2020/05/29/us/coronavirus-child-care-centers.html?action=click\&pgtype=Article\&state=default\&region=MAIN_CONTENT_3\&context=storylines_faq}{makeshift
    child care} and
    \href{https://www.nytimes3xbfgragh.onion/2020/06/03/business/economy/coronavirus-working-women.html?action=click\&pgtype=Article\&state=default\&region=MAIN_CONTENT_3\&context=storylines_faq}{stunted
    workdays} to continue. California's two largest public school
    districts --- Los Angeles and San Diego --- said on July 13, that
    \href{https://www.nytimes3xbfgragh.onion/2020/07/13/us/lausd-san-diego-school-reopening.html?action=click\&pgtype=Article\&state=default\&region=MAIN_CONTENT_3\&context=storylines_faq}{instruction
    will be remote-only in the fall}, citing concerns that surging
    coronavirus infections in their areas pose too dire a risk for
    students and teachers. Together, the two districts enroll some
    825,000 students. They are the largest in the country so far to
    abandon plans for even a partial physical return to classrooms when
    they reopen in August. For other districts, the solution won't be an
    all-or-nothing approach.
    \href{https://bioethics.jhu.edu/research-and-outreach/projects/eschool-initiative/school-policy-tracker/}{Many
    systems}, including the nation's largest, New York City, are
    devising
    \href{https://www.nytimes3xbfgragh.onion/2020/06/26/us/coronavirus-schools-reopen-fall.html?action=click\&pgtype=Article\&state=default\&region=MAIN_CONTENT_3\&context=storylines_faq}{hybrid
    plans} that involve spending some days in classrooms and other days
    online. There's no national policy on this yet, so check with your
    municipal school system regularly to see what is happening in your
    community.
  \end{itemize}
\end{itemize}

Kelleh Njoki, 25, arrived in Dubai from Kenya as a tourist in February
seeking work, but soon discovered she was pregnant. She is now sleeping
in a crowded private dorm and cannot afford maternity care or a \$400
repatriation flight.

``I'm seven months pregnant; how am I going to have my baby here?'' she
said in a phone interview. ``I'm stuck. I'm confused. I really need
help.''

For Apisaki, the Kenyan locked up with eight other women in the Saudi
capital, Riyadh, the trouble started when she left her job last month
after not being paid for months and returned to her recruiting agency,
she said.

She was soon held with others from Kenya and Uganda who also had no work
and no way to get home because of the lockdown --- and because the
agency had taken their passports.

The New York Times verified the details of the women's confinement
through interviews with two women in the room, including Apisaki, who is
being identified only by her middle name for her safety, and videos she
shared showing their conditions.

The women are in a single room whose only sunlight comes from a small
window that was recently taped shut. They share a toilet, wash clothes
in the sink and cook one meal per day when the agency drops off food.
The pregnant woman hasn't seen a doctor in months, Apisaki said, and the
woman who tore off her clothes spent weeks lying naked on the tile, her
arm chained to the wall.

New arrivals are not tested for the coronavirus, potentially endangering
the others in a country recording thousands of new cases daily.

Image

Throughout the Middle East, millions of poorly paid Southeast Asians and
Africans clean their employers' homes, serve as drivers and care for
their children and their elderly.Credit...Hussein Malla/Associated Press

For the Kenyan women, the agency that recruited them in Kenya is
responsible for helping them return home. But the agency that recruited
Apisaki is no longer answering its phone or responding to messages, she
said.

Last week, the Kenyan Embassy in Riyadh
\href{https://twitter.com/KenyaRiyadh/status/1277911187724828673}{announced}
a possible repatriation flight to Nairobi but said travelers had to
prove they did not have Covid-19, buy a \$525 ticket and quarantine once
home.

But Apisaki can't get tested or fly if she can't leave the room, and her
efforts to reach the embassy have failed, she said.

The women said they had been locked up by their Saudi agency,
\href{https://www.facebookcorewwwi.onion/\%D9\%85\%D9\%83\%D8\%AA\%D8\%A8-\%D8\%A7\%D9\%84\%D9\%85\%D8\%AD\%D9\%8A\%D8\%B7-\%D9\%84\%D9\%84\%D8\%A7\%D8\%B3\%D8\%AA\%D9\%82\%D8\%AF\%D8\%A7\%D9\%85-103491741091774/?ref=nf\&hc_ref=ARSq4vc8q4S5QfycNQXadsjq_HLexcYyzaiFQZAKq_cXIDFfzK1uEecA1F1fnZyolcM}{Almuhait
Recruitment}. It did not respond to requests for comment on Sunday.

In an emailed response to questions, Peter Ogego, the Kenyan ambassador
to Saudi Arabia, said that he was alarmed by the ``serious allegations''
of the women's detention and that he would work with the Saudi
government ``to bring justice to the victims and address the loopholes
in the law and any underlying causes.''

But he said it was Saudi Arabia's job to ensure the safety of foreigners
working there and questioned Apisaki's inability to reach embassy
officials.

``Much of our work is daily spent largely addressing such allegations,''
he wrote.

After The New York Times contacted Almuhait Recruitment about the
women's situation on Sunday, Apisaki told an associate outside of Saudi
Arabia that several of the women, including the pregnant woman, had been
taken to a hospital for medical checkups and Covid-19 tests.

``They can't hold women without any right,'' Apisaki said. ``I don't get
sun to my body, no space to stretch my legs, to walk or exercise. These
are the things that make my brain crazy.''

Ben Hubbard reported from Beirut, Lebanon, and Louise Donovan from
London. Hwaida Saad contributed reporting from Beirut. This article is a
collaboration between The New York Times and
\href{https://fullerproject.org/}{The Fuller Project}, a journalism
nonprofit that reports on global issues impacting women.

Advertisement

\protect\hyperlink{after-bottom}{Continue reading the main story}

\hypertarget{site-index}{%
\subsection{Site Index}\label{site-index}}

\hypertarget{site-information-navigation}{%
\subsection{Site Information
Navigation}\label{site-information-navigation}}

\begin{itemize}
\tightlist
\item
  \href{https://help.nytimes3xbfgragh.onion/hc/en-us/articles/115014792127-Copyright-notice}{©~2020~The
  New York Times Company}
\end{itemize}

\begin{itemize}
\tightlist
\item
  \href{https://www.nytco.com/}{NYTCo}
\item
  \href{https://help.nytimes3xbfgragh.onion/hc/en-us/articles/115015385887-Contact-Us}{Contact
  Us}
\item
  \href{https://www.nytco.com/careers/}{Work with us}
\item
  \href{https://nytmediakit.com/}{Advertise}
\item
  \href{http://www.tbrandstudio.com/}{T Brand Studio}
\item
  \href{https://www.nytimes3xbfgragh.onion/privacy/cookie-policy\#how-do-i-manage-trackers}{Your
  Ad Choices}
\item
  \href{https://www.nytimes3xbfgragh.onion/privacy}{Privacy}
\item
  \href{https://help.nytimes3xbfgragh.onion/hc/en-us/articles/115014893428-Terms-of-service}{Terms
  of Service}
\item
  \href{https://help.nytimes3xbfgragh.onion/hc/en-us/articles/115014893968-Terms-of-sale}{Terms
  of Sale}
\item
  \href{https://spiderbites.nytimes3xbfgragh.onion}{Site Map}
\item
  \href{https://help.nytimes3xbfgragh.onion/hc/en-us}{Help}
\item
  \href{https://www.nytimes3xbfgragh.onion/subscription?campaignId=37WXW}{Subscriptions}
\end{itemize}
