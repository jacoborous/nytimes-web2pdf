Sections

SEARCH

\protect\hyperlink{site-content}{Skip to
content}\protect\hyperlink{site-index}{Skip to site index}

\href{https://www.nytimes3xbfgragh.onion/section/us}{U.S.}

\href{https://myaccount.nytimes3xbfgragh.onion/auth/login?response_type=cookie\&client_id=vi}{}

\href{https://www.nytimes3xbfgragh.onion/section/todayspaper}{Today's
Paper}

\href{/section/us}{U.S.}\textbar{}Footage of Police Body Cameras Offers
Devastating Account of Floyd Killing

\url{https://nyti.ms/2B37P4x}

\begin{itemize}
\item
\item
\item
\item
\item
\end{itemize}

\href{https://www.nytimes3xbfgragh.onion/news-event/george-floyd-protests-minneapolis-new-york-los-angeles?action=click\&pgtype=Article\&state=default\&region=TOP_BANNER\&context=storylines_menu}{Race
and America}

\begin{itemize}
\tightlist
\item
  \href{https://www.nytimes3xbfgragh.onion/2020/07/26/us/protests-portland-seattle-trump.html?action=click\&pgtype=Article\&state=default\&region=TOP_BANNER\&context=storylines_menu}{Protesters
  Return to Other Cities}
\item
  \href{https://www.nytimes3xbfgragh.onion/2020/07/24/us/portland-oregon-protests-white-race.html?action=click\&pgtype=Article\&state=default\&region=TOP_BANNER\&context=storylines_menu}{Portland
  at the Center}
\item
  \href{https://www.nytimes3xbfgragh.onion/2020/07/23/podcasts/the-daily/portland-protests.html?action=click\&pgtype=Article\&state=default\&region=TOP_BANNER\&context=storylines_menu}{Podcast:
  Showdown in Portland}
\item
  \href{https://www.nytimes3xbfgragh.onion/interactive/2020/07/16/us/black-lives-matter-protests-louisville-breonna-taylor.html?action=click\&pgtype=Article\&state=default\&region=TOP_BANNER\&context=storylines_menu}{45
  Days in Louisville}
\end{itemize}

Advertisement

\protect\hyperlink{after-top}{Continue reading the main story}

Supported by

\protect\hyperlink{after-sponsor}{Continue reading the main story}

\hypertarget{footage-of-police-body-cameras-offers-devastating-account-of-floyd-killing}{%
\section{Footage of Police Body Cameras Offers Devastating Account of
Floyd
Killing}\label{footage-of-police-body-cameras-offers-devastating-account-of-floyd-killing}}

The footage was made available by appointment at a court in Minneapolis,
and shows officers seemingly more concerned with controlling George
Floyd's body than saving his life.

\includegraphics{https://static01.graylady3jvrrxbe.onion/images/2020/07/15/us/15minneapolis/merlin_173568411_4163a78b-c2a9-475e-b479-ff7bbbf55163-articleLarge.jpg?quality=75\&auto=webp\&disable=upscale}

By \href{https://www.nytimes3xbfgragh.onion/by/tim-arango}{Tim Arango},
Matt Furber and
\href{https://www.nytimes3xbfgragh.onion/by/nicholas-bogel-burroughs}{Nicholas
Bogel-Burroughs}

\begin{itemize}
\item
  Published July 15, 2020Updated July 29, 2020
\item
  \begin{itemize}
  \item
  \item
  \item
  \item
  \item
  \end{itemize}
\end{itemize}

MINNEAPOLIS --- Almost from the moment George Floyd encountered the
police on May 25, with a gun pointed at him, he appeared terrified and
emotionally distraught, according to police camera footage that was
newly made available for viewing Wednesday at a courthouse in downtown
Minneapolis.

Mr. Floyd was visibly shaken, with his head down, and crying, as if he
were in the throes of a panic attack, as he put his hands on the
steering wheel in response to a frantic order from an officer.

He told the officers over and over that he was claustrophobic, as two
officers struggled to push him to the back seat of a police vehicle.
Throughout the video, he never appeared to present a physical threat to
the officers, and even after he was handcuffed and searched for weapons,
the officers seemed to be more concerned with controlling his body than
saving his life, the footage showed.

The video offers the fullest portrait yet of the tragic events around
Mr. Floyd's killing. It begins with officers driving to the scene, after
a convenience store clerk called 911 and said a man had used a
counterfeit \$20 bill, and it ends showing officers on the street
discussing what happened, after Mr. Floyd is driven away in an
ambulance. At one point, in footage not previously seen, the officers
are shown dragging Mr. Floyd to the ground after he resisted being put
in the squad car.

Once he was on the ground, as Mr. Floyd again said he couldn't breathe,
and asked for water, and begged for his life, Derek Chauvin, the senior
officer on the scene, said, in a nonchalant,
\href{https://www.nytimes3xbfgragh.onion/2020/07/08/us/george-floyd-body-camera-transcripts.html}{almost
mocking, tone}, ``takes a heck of a lot of oxygen to say that.''

The footage provides more detail into the action of Mr. Chauvin, who has
been charged with second-degree murder and second-degree manslaughter
for keeping his knee on Mr. Floyd's neck for more than eight minutes
while he gasped for life. He was later pronounced dead at the hospital.

As the minutes ticked by, and Mr. Floyd became quieter and his body went
limp, one officer checked his pulse and said he couldn't find one.

Mr. Chauvin's response, uttered with no emotion, was, ``uh huh.''

Just before, after being told that Mr. Floyd appeared to be passing out,
Mr. Chauvin appears to express more concern for his fellow officers than
the man dying under his knee.

``You guys all right, though?'' he said.

``My knee might be a little scratched, but I'll survive,'' responded
another officer, Thomas Lane.

The footage was made available for viewing Wednesday to the public and
media by appointment at the Hennepin County Government Center in
downtown Minneapolis --- in a conference room with a dozen laptop
stations --- but was not allowed to be copied or recorded.

A coalition of media organizations, including The New York Times, has
petitioned the court to obtain the footage, which would allow for
release to the public. Judge Peter Cahill, who is overseeing the case,
will hold a hearing on the matter on Tuesday.

Mr. Floyd's family on Wednesday filed a lawsuit against four of the
officers at the scene and against the City of Minneapolis, arguing that
the police had violated the Fourth Amendment in killing Mr. Floyd and
that the city had failed to properly dismiss problem officers and train
recruits about the dangers of neck restraints.

\includegraphics{https://static01.graylady3jvrrxbe.onion/images/2020/07/15/us/15minneapolis02/merlin_174598062_cbf3f55d-28d9-41a7-823f-4e05c03d4b58-videoSixteenByNine3000.jpg}

``It was not just the knee of Derek Chauvin on George Floyd's neck for 8
minutes and 46 seconds,'' Ben Crump, a lawyer representing Mr. Floyd's
family, said at a news conference. ``But it was the knee of the entire
Minneapolis Police Department on the neck of George Floyd that killed
him.''

In the lawsuit, Mr. Crump and a team of other prominent lawyers argue
that the Police Department's policies had allowed for officers to use
``neck restraint'' techniques that could be deadly even when they were
not in life-or-death situations. It also said that training materials
given to officers in 2014, including Mr. Chauvin and another officer
charged in Mr. Floyd's killing, show an officer placing a knee on the
neck of a person who is being arrested and is handcuffed in a prone
position, as Mr. Floyd was.

The lawyers said in the lawsuit that the policies and training, approved
or condoned by the mayor, City Council and police chief, ``were the
moving force behind and caused'' Mr. Floyd's death.

Erik Nilsson, the Minneapolis city attorney, said the city would review
and respond to the lawsuit. A spokesman for the Police Department did
not respond to an inquiry about the lawsuit's claims.

\href{https://www.nytimes3xbfgragh.onion/2020/07/08/us/george-floyd-body-camera-transcripts.html}{Transcripts}
of the body camera footage, from two of the four police officers charged
in the killing of Mr. Floyd, were released last week as part of a motion
on behalf of one of the junior officers, Mr. Lane, to have the case
against him dismissed.

Mr. Lane, 37, was a rookie officer, and one of the first officers on the
scene. His lawyer, Earl Gray, has sought to shift the blame to Mr.
Chauvin, a senior officer who trained new recruits to the force, arguing
that Mr. Lane was following the lead of Mr. Chauvin.

According to the transcripts and an interview Mr. Lane gave to
investigators, Mr. Lane suspected that Mr. Floyd was having a medical
emergency and asked Mr. Chauvin if they should turn Mr. Floyd on his
side as he was facedown and gasping for breath. Mr. Lane also rode along
in the ambulance to the hospital with Mr. Floyd, administering chest
compressions in an attempt to revive him.

Mr. Chauvin, a 19-year veteran of the Minneapolis Police Department,
faces the most severe criminal charges, and three other officers are
charged with aiding and abetting second-degree murder. All four were
fired shortly after Mr. Floyd's death. Their trial is scheduled to begin
March 8.

Once an ambulance arrived --- late, because paramedics had first gone to
the wrong location --- Mr. Lane went inside and administered chest
compressions on Mr. Floyd, whose face appeared bloodied.

But even in the ambulance, at first, there appeared to be little sense
of urgency, according to the newly seen footage, with minutes passing
before anyone tended to Mr. Floyd.

Later, they strapped a mechanical chest compression device on a nearly
naked Mr. Floyd, which kept pumping as Mr. Floyd's body was rising and
falling.

Back at the scene, Mr. Chauvin, who had arrived later than Mr. Lane and
another junior officer, J. Alexander Kueng, stood erect, his lips
pursed, with his hands on his hips as Mr. Kueng, who called his
superior, ``sir,'' showed him what he believed was the fake \$20 bill.

Tim Arango and Matt Furber reported from Minneapolis, and Nicholas
Bogel-Burroughs from New York.

Advertisement

\protect\hyperlink{after-bottom}{Continue reading the main story}

\hypertarget{site-index}{%
\subsection{Site Index}\label{site-index}}

\hypertarget{site-information-navigation}{%
\subsection{Site Information
Navigation}\label{site-information-navigation}}

\begin{itemize}
\tightlist
\item
  \href{https://help.nytimes3xbfgragh.onion/hc/en-us/articles/115014792127-Copyright-notice}{©~2020~The
  New York Times Company}
\end{itemize}

\begin{itemize}
\tightlist
\item
  \href{https://www.nytco.com/}{NYTCo}
\item
  \href{https://help.nytimes3xbfgragh.onion/hc/en-us/articles/115015385887-Contact-Us}{Contact
  Us}
\item
  \href{https://www.nytco.com/careers/}{Work with us}
\item
  \href{https://nytmediakit.com/}{Advertise}
\item
  \href{http://www.tbrandstudio.com/}{T Brand Studio}
\item
  \href{https://www.nytimes3xbfgragh.onion/privacy/cookie-policy\#how-do-i-manage-trackers}{Your
  Ad Choices}
\item
  \href{https://www.nytimes3xbfgragh.onion/privacy}{Privacy}
\item
  \href{https://help.nytimes3xbfgragh.onion/hc/en-us/articles/115014893428-Terms-of-service}{Terms
  of Service}
\item
  \href{https://help.nytimes3xbfgragh.onion/hc/en-us/articles/115014893968-Terms-of-sale}{Terms
  of Sale}
\item
  \href{https://spiderbites.nytimes3xbfgragh.onion}{Site Map}
\item
  \href{https://help.nytimes3xbfgragh.onion/hc/en-us}{Help}
\item
  \href{https://www.nytimes3xbfgragh.onion/subscription?campaignId=37WXW}{Subscriptions}
\end{itemize}
