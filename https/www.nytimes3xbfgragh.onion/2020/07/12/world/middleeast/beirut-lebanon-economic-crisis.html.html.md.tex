Sections

SEARCH

\protect\hyperlink{site-content}{Skip to
content}\protect\hyperlink{site-index}{Skip to site index}

\href{https://www.nytimes3xbfgragh.onion/section/world/middleeast}{Middle
East}

\href{https://myaccount.nytimes3xbfgragh.onion/auth/login?response_type=cookie\&client_id=vi}{}

\href{https://www.nytimes3xbfgragh.onion/section/todayspaper}{Today's
Paper}

\href{/section/world/middleeast}{Middle East}\textbar{}Bartering Child's
Dress for Food: Life in Lebanon's Economic Crisis

\url{https://nyti.ms/2ZYMeTi}

\begin{itemize}
\item
\item
\item
\item
\item
\end{itemize}

Advertisement

\protect\hyperlink{after-top}{Continue reading the main story}

Supported by

\protect\hyperlink{after-sponsor}{Continue reading the main story}

Lebanon Dispatch

\hypertarget{bartering-childs-dress-for-food-life-in-lebanons-economic-crisis}{%
\section{Bartering Child's Dress for Food: Life in Lebanon's Economic
Crisis}\label{bartering-childs-dress-for-food-life-in-lebanons-economic-crisis}}

A TV chef abandons unaffordable beef. Blackouts make for sweltering
summer nights. Changing money feels like a drug deal: The financial
meltdown means daily pain and a blow to a country's pride.

\includegraphics{https://static01.graylady3jvrrxbe.onion/images/2020/07/12/world/11Lebanon-Dispatch/11Lebanon-Dispatch-articleLarge.jpg?quality=75\&auto=webp\&disable=upscale}

By \href{https://www.nytimes3xbfgragh.onion/by/ben-hubbard}{Ben Hubbard}
and \href{https://www.nytimes3xbfgragh.onion/by/hwaida-saad}{Hwaida
Saad}

\begin{itemize}
\item
  July 12, 2020
\item
  \begin{itemize}
  \item
  \item
  \item
  \item
  \item
  \end{itemize}
\end{itemize}

BEIRUT, Lebanon --- For three decades, Chef Antoine El Hajj has appeared
on television five days a week to help cooks across Lebanon improve
their grasp of the culinary arts.

Two months ago, as an economic crisis caused Lebanon's currency to
collapse and prices to soar, he realized that many of his viewers could
no longer afford staples he had long relied on in his recipes, like
beef.

``There used to be a middle class in Lebanon, but now the rich are rich,
the middle class has become poor and the poor have become destitute,''
Mr. El Hajj, 65, said in an interview this past week before going on the
air.

He has since cut beef from his menus and fills his segments with tips on
how to keep dishes tasty with less oil, fewer eggs and cheaper
vegetables.

\includegraphics{https://static01.graylady3jvrrxbe.onion/images/2020/07/12/world/11Lebanon-Dispatch-02/11Lebanon-Dispatch-02-articleLarge.jpg?quality=75\&auto=webp\&disable=upscale}

Lebanon's crisis, the result of years of
\href{https://www.nytimes3xbfgragh.onion/2019/12/03/world/middleeast/lebanon-protests-corruption.html}{government
corruption} and financial mismanagement, has caused unemployment and
poverty rates to skyrocket, businesses to shutter and salaries to lose
their value as inflation soars.

\href{https://www.nytimes3xbfgragh.onion/2019/11/15/world/middleeast/lebanon-protests-economy.html}{Mass
protests} against the political elite erupted across the country last
fall, and sometimes turned violent. The demonstrations tapered off when
the country
\href{https://www.nytimes3xbfgragh.onion/2020/05/03/world/middleeast/beirut-lebanon-nightlife-coronavirus.html}{shut
down because of the coronavirus} but have recently
\href{https://www.nytimes3xbfgragh.onion/2020/06/11/world/middleeast/lebanon-protests.html}{picked
up again} as the lockdown has added to the economic distress.

The effects of the economic meltdown are increasingly infiltrating the
daily lives of many Lebanese. Power cuts darken streets, banks refuse to
hand over depositors' cash and families struggle to buy imported
essentials like diapers and laundry detergent.

The government has long failed to provide sufficient electricity. But
blackouts have grown so long that the din of traffic in
\href{https://www.nytimes3xbfgragh.onion/2020/08/04/world/middleeast/beirut-explosion-blast.html}{Beirut},
where about one-third of Lebanon's 5.4 million people live, has been
replaced by the roar of overworked generators.

Their exhaust fouls the air, and many residential buildings are shutting
them off to rest at night, depriving residents of air-conditioning
during the sweatiest stretch of the Mediterranean summer.

For two days recently, Rafik Hariri University Hospital, the main
facility treating Beirut's Covid-19 cases, suddenly went from one hour
without power per day to 20 hours without power, according to its
director, Dr. Firass Abiad.

So the hospital, which now lacks power six hours a day, has closed some
operating rooms and delayed surgeries.

``It feels like you are continuously firefighting with no end in
sight,'' Dr. Abiad said.

Image

The capital city, Beirut, during a power outage of more than six hours
last~ week.Credit...Diego Ibarra Sanchez for The New York Times

After dark, Beirut's once-raucous nightlife has given way to an eerie
desolation. Bars have few patrons, main streets are dark and traffic
lights at major intersections are out, leaving drivers to navigate on
their own, flashing their high beams and hoping for the best as they
plow through.

The swift collapse has struck a blow to the pride of many Lebanese, who
often have claimed to have the Middle East's best cuisine and have seen
themselves as more sophisticated than others in the region. Now, many
wonder how far their standard of living will fall.

``Beirut is a survival city. People always find ways to eat and drink
and make music and do activism. But now, the air is very thick,'' said
Carmen Geha, an assistant professor of public administration at the
American University of Beirut. ``Now, even upper-middle-class people
can't afford to eat outside the house. It's like you take your salary
and divide it by nine.''

The Lebanese pound, or lira, has lost about 85 percent of its value on
the black market since last fall, getting its own satirical Twitter feed
where it reacts to its own decline.

``I'm the cheapest but I'm not a piece of junk,'' the account
\href{https://twitter.com/LebaneseLira/status/1278727340768534529}{said
early this month} amid reports that it was trading at 9,500 to the U.S.
dollar, far from the official bank rate of 1,500.

Much of the financial distress comes from chaos in the banking system.
The central bank ran what
\href{https://www.nytimes3xbfgragh.onion/2019/12/02/opinion/lebanon-protests.html}{critics
have called a Ponzi scheme}, enticing commercial banks to make large
deposits of U.S. dollars with high interest rates that could be covered
only by bringing in more large depositors with even higher interest
rates.

But that system ground to a halt last year when new investors stopped
coming, leaving the country's banks far short of thedollars they owed
their depositors.

The banks have reacted by mostly refusing to give out dollars, which the
Lebanese had long used interchangeably with local currency in daily
life.

A former Lebanese banker, Dan Azzi, has taken credit for coining a term
now widely used for these theoretical dollars that exist only in
Lebanese banks: \href{https://www.lollar.club/}{Lollars}.

The result has been financial pandemonium, and pain.

Image

Rising levels of hunger amid a major currency slide have brought
protesters back onto the streets after a tapering off during the
coronavirus lockdown.Credit...Diego Ibarra Sanchez for The New York
Times

The government has sought to control the black market exchange, where
changing money on better terms than the official rate can feel like
buying drugs, requiring quick meetings in alleyways with money-changers
who use fake names and fear arrest.

The effects of the crisis on the country's poor have been acute, as was
made clear by four recent suicides in one two-day period, all linked to
the economic crisis. A man who shot himself on one of Beirut's
best-known boulevards left behind a handwritten sign reading ``I am not
an infidel,'' a line from a well-known song whose next lyric is ``but
hunger is an infidel.''

Membership of a Facebook group called
\href{https://www.facebookcorewwwi.onion/groups/697518051074366/permalink/722296395263198/}{Lebanon
Barters} has swelled, its members offering everything from poker chips
to hookahs in exchange for food. Their posts read like tragic poetry.

``New weights, never used, to trade for a package of diapers, size 6,
and a bottle of oil,'' read
\href{https://www.facebookcorewwwi.onion/groups/697518051074366/permalink/722718038554367/}{a
post} with a photo of dumbbells still in the box. ``People need them.''

Another post featured a
\href{https://www.facebookcorewwwi.onion/groups/697518051074366/permalink/722296395263198/}{lime-green
dress} that Fatima al-Hussein, a mother of six from northern Lebanon,
had bought as a gift for her daughter. She was looking to trade it for
sugar, milk and detergent.

In a phone interview, Ms. al-Hussein said her husband makes 200,000
Lebanese pounds per week as a manual laborer, an amount that used to be
worth \$130.

Now it is worth less than \$30, leaving her family struggling to afford
essentials.

She said she decided to trade the dress after she had to start feeding
her children bread dipped in water. But so far, she had found no takers.

When her neighbors cook, she closes her doors and windows.

``I don't want my children to smell the food,'' she said.

Before going on air last week, Mr. El Hajj, the television chef, said
that what counted as affordable recipes was a moving target.

Image

Protesters blocked roads in Beirut on July 2.Credit...Diego Ibarra
Sanchez for The New York Times

``Beef got expensive so we moved to chicken, and now people are telling
me that chicken is expensive, too,'' he said.

As he prepared the dishes for the day's show, he fielded calls from
viewers struggling with preserving food amid power cuts. How do you make
jam from plums or cherries? How do you keep meat fresh when you can't
count on the freezer?

He laid out the options. Fruits and vegetables could be canned, pickled
or dried. Ground meat could be preserved in fat as confit.

``Everything has a solution,'' he said after the show, and added, ``What
is important for me with my program is to help people to continue to
eat.''

Advertisement

\protect\hyperlink{after-bottom}{Continue reading the main story}

\hypertarget{site-index}{%
\subsection{Site Index}\label{site-index}}

\hypertarget{site-information-navigation}{%
\subsection{Site Information
Navigation}\label{site-information-navigation}}

\begin{itemize}
\tightlist
\item
  \href{https://help.nytimes3xbfgragh.onion/hc/en-us/articles/115014792127-Copyright-notice}{©~2020~The
  New York Times Company}
\end{itemize}

\begin{itemize}
\tightlist
\item
  \href{https://www.nytco.com/}{NYTCo}
\item
  \href{https://help.nytimes3xbfgragh.onion/hc/en-us/articles/115015385887-Contact-Us}{Contact
  Us}
\item
  \href{https://www.nytco.com/careers/}{Work with us}
\item
  \href{https://nytmediakit.com/}{Advertise}
\item
  \href{http://www.tbrandstudio.com/}{T Brand Studio}
\item
  \href{https://www.nytimes3xbfgragh.onion/privacy/cookie-policy\#how-do-i-manage-trackers}{Your
  Ad Choices}
\item
  \href{https://www.nytimes3xbfgragh.onion/privacy}{Privacy}
\item
  \href{https://help.nytimes3xbfgragh.onion/hc/en-us/articles/115014893428-Terms-of-service}{Terms
  of Service}
\item
  \href{https://help.nytimes3xbfgragh.onion/hc/en-us/articles/115014893968-Terms-of-sale}{Terms
  of Sale}
\item
  \href{https://spiderbites.nytimes3xbfgragh.onion}{Site Map}
\item
  \href{https://help.nytimes3xbfgragh.onion/hc/en-us}{Help}
\item
  \href{https://www.nytimes3xbfgragh.onion/subscription?campaignId=37WXW}{Subscriptions}
\end{itemize}
