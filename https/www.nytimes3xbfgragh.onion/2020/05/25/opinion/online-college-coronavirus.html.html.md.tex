Sections

SEARCH

\protect\hyperlink{site-content}{Skip to
content}\protect\hyperlink{site-index}{Skip to site index}

\href{https://myaccount.nytimes3xbfgragh.onion/auth/login?response_type=cookie\&client_id=vi}{}

\href{https://www.nytimes3xbfgragh.onion/section/todayspaper}{Today's
Paper}

\href{/section/opinion}{Opinion}\textbar{}The Future of College Is
Online, and It's Cheaper

\url{https://nyti.ms/3ejLhL2}

\begin{itemize}
\item
\item
\item
\item
\item
\item
\end{itemize}

Advertisement

\protect\hyperlink{after-top}{Continue reading the main story}

\href{/section/opinion}{Opinion}

Supported by

\protect\hyperlink{after-sponsor}{Continue reading the main story}

\hypertarget{the-future-of-college-is-online-and-its-cheaper}{%
\section{The Future of College Is Online, and It's
Cheaper}\label{the-future-of-college-is-online-and-its-cheaper}}

The coronavirus forced a shift to virtual classes, but their
continuation could be beneficial even after the pandemic ends.

By Hans Taparia

Mr. Taparia is a clinical associate professor at the New York University
Stern School of Business.

\begin{itemize}
\item
  May 25, 2020
\item
  \begin{itemize}
  \item
  \item
  \item
  \item
  \item
  \item
  \end{itemize}
\end{itemize}

\includegraphics{https://static01.graylady3jvrrxbe.onion/images/2020/05/26/opinion/25taparia/25taparia-articleLarge.jpg?quality=75\&auto=webp\&disable=upscale}

Forty years ago, going to college in America was a reliable pathway for
upward mobility. Today, it has become yet another 21st-century symbol of
privilege for the wealthy. Through this period,
\href{https://nces.ed.gov/programs/digest/d13/tables/dt13_330.10.asp}{tuition
rates soared 260 percent,} double the rate of inflation. In 2019, the
average
\href{https://www.collegedata.com/en/pay-your-way/college-sticker-shock/how-much-does-college-cost/whats-the-price-tag-for-a-college-education/}{cost
of attending} a four-year private college was over \$200,000. For a
four-year public college, it was over \$100,000. To sustain these
prices,
\href{http://www.equality-of-opportunity.org/assets/documents/coll_mrc_paper.pdf}{more
students are now admitted} from the top 1 percent of the income scale
than the entire bottom 40 percent at the top 80 colleges. Universities
have \href{http://graphics.wsj.com/international-students/}{also opened
the floodgates} to wealthy international students, willing to pay full
tuition for the American brand.

Covid-19 is about to ravage that business model. Mass unemployment is
looming large and is likely to put college out of reach for many. With
America now the epicenter of the pandemic and bungling its response,
many students are looking to
\href{https://www.nytimes3xbfgragh.onion/2020/05/01/us/coronavirus-college-enrollment.html}{defer}enrollment.
Foreign students are questioning whether to register at all, with
greater uncertainty around visas and work prospects. The ``Trump
Effect'' had already begun to cause
\href{https://www.insidehighered.com/admissions/article/2019/11/18/international-enrollments-declined-undergraduate-graduate-and}{declining
foreign student}enrollment over the past three years.

The mightiest of institutions are bracing for the worst. Harvard, home
to the country's largest endowment, recently
\href{https://www.harvard.edu/president/news/2020/economic-impact-covid-19}{announced}
drastic steps to manage the fallout, including salary cuts for its
leadership, hiring freezes and cuts in discretionary spending. Most
other universities have been forced to make similar decisions, and are
nervous that if they continue with online teaching this fall, students
will demand at least a partial remission of tuition.

Up until now, online education has been relegated to the equivalent of a
hobby at most universities. With the pandemic, it has become a backup
plan. But if universities embrace this moment strategically, online
education could expand access exponentially and drop its cost by
magnitudes --- all while shoring up revenues for universities in a way
that is more recession-proof, policy-proof and pandemic-proof.

To be clear, the scramble to move online over just a few days this March
did not go well. Faculty members were forced to revamp lesson plans
overnight. ``Zoom-bombers'' took advantage of lax privacy protocols.
Students fled home, with many in faraway time zones prolonging jet lag
just to continue synchronous learning. Not surprisingly, the experience
for both students and faculty has left much to be desired. According to
one
\href{https://oneclass.com/blog/featured/177356-7525-of-college-students-unhappy-with-quality-of-elearning-during-covid-19.en.html}{survey},
more than 75 percent of students do not feel they received a quality
learning experience after classrooms closed.

But what surveys miss are the numerous spirited efforts to break new
ground, as only a crisis can be the impetus for.

One professor at New York University's Tisch School of the Arts
\href{https://www.ny1.com/nyc/all-boroughs/coronavirus-blog/2020/05/06/nyu-vr-drama-classes}{taught}
a drama course that allows students to ``act'' with each other in
virtual reality using Oculus Quest headsets. A music professor at
Stanford
\href{https://news.stanford.edu/2020/04/06/stanford-faculty-students-connect-digital-classrooms/}{train}ed
his students on software that allows musicians in different locations to
perform together using internet streaming. Professors are pioneering new
methods and ed-tech companies are developing platforms at a pace not
seen before, providing a glimpse into the untapped potential of online
education. Not to be forgotten, of course, is the fact that just a few
years ago, a transition to online learning at the current scale would
have been unimaginable.

Before the pandemic, most universities never truly embraced online
education, at least not strategically. For years, universities have
allowed professors to offer some courses online, making them accessible
through aggregators such as edX or Coursera. But rarely do universities
offer their most popular and prestigious degrees remotely. It is still
not possible to get an M.B.A. at Stanford, a biology degree at M.I.T. or
a computer science degree at Brown online.

On one hand, universities don't want to be seen as limiting access to
education, so they have dabbled in the space. But to fully embrace it
might render much of the faculty redundant, reduce the exclusivity of
those degrees, and threaten the very existence of the physical campus,
for which vast resources have been allocated over centuries.

For good reason, many educators have been skeptical of online learning.
They have questioned how discussion-based courses, which require more
intimate settings, would be coordinated. They wonder how lab work might
be administered. Of course, no one doubts that the student experience
would not be as holistic. But universities don't need to abandon
in-person teaching for students who see the value in it.

They simply need to create ``parallel'' online degrees for all their
core degree programs. By doing so, universities could expand their reach
by thousands, creating the economies of scale to drop their costs by
tens of thousands.

There are a few, but instructive, examples of prestigious universities
that have already shown the way. Georgia Tech, a top engineering school,
launched an online \href{https://www.omscs.gatech.edu/}{masters in
computer science} in 2014. The degree costs just \$7,000 (one-sixth the
cost of its in-person program), and the school now has nearly
\href{https://www.omscs.gatech.edu/prospective-students/numbers}{10,000
students enrolled}, making it the
\href{https://www.insidehighered.com/digital-learning/article/2018/03/20/analysis-shows-georgia-techs-online-masters-computer-science}{largest}
computer science program in the country. Notably, the online degree has
not cannibalized its on-campus revenue stream. Instead, it has opened up
a prestigious degree program to a different population, mostly midcareer
applicants looking for a meaningful skills upgrade.

Similarly, in 2015, the University of Illinois launched an
\href{https://onlinemba.illinois.edu/imba-experience/student-experience/}{online
M.B.A}. for \$22,000, a fraction of the cost of most business schools.
In order to provide a forum for networking and experiential learning,
critical to the business school experience, the university created
micro-immersions, where students can connect with other students and
work on live projects at companies at a regional level.

To do this would require a major reorientation of university resources
and activities. Classrooms would need to be fitted with new technology
so that lectures could be simultaneously delivered to students on campus
as well as across the world. Professors would need to undergo training
on how to effectively teach to a blended classroom. Universities would
also be well served to build competencies in content production. Today,
almost all theory-based content, whether in chemistry, computer science
or finance, can be produced in advance and effectively delivered
asynchronously. By tapping their best-rated professors to be the stars
of those productions, universities could actually raise the pedagogical
standard.

There are already strong examples of this. Most biology professors, for
instance, would find themselves hard pressed to match the pedagogical
quality, production values and inspirational nature of Eric Lander's
\href{https://www.edx.org/course/introduction-to-biology-the-secret-of-life-3}{online
Introduction to Biology course} at M.I.T. That free course currently has
over 134,000 students enrolled this semester.

Once universities have developed a library of content, they can choose
to draw from it for asynchronous delivery for years, both for their
on-campus and online programs. Students may not mind. It would, after
all, open up professor capacity for a larger number of live
interactions. Three-hour lectures, which were never good for anyone,
would become a thing of the past. Instead, a typical day might be broken
up into one-hour sessions with a focus on problem-solving, Q. and A. or
discussion.

Many universities are sounding bold about reopening in-person
instruction this fall. The current business model requires them to, or
face financial ruin. But a hasty decision driven by the financial
imperative could prove lethal, and do little to help them weather a
storm. The pandemic provides universities an opportunity to reimagine
education around the pillars of access and affordability with the myriad
tools and techniques now at their disposal. It could make them true
pathways of upward mobility again.

Hans Taparia is a clinical associate professor at the New York
University Stern School of Business.

\emph{The Times is committed to publishing}
\href{https://www.nytimes3xbfgragh.onion/2019/01/31/opinion/letters/letters-to-editor-new-york-times-women.html}{\emph{a
diversity of letters}} \emph{to the editor. We'd like to hear what you
think about this or any of our articles. Here are some}
\href{https://help.nytimes3xbfgragh.onion/hc/en-us/articles/115014925288-How-to-submit-a-letter-to-the-editor}{\emph{tips}}\emph{.
And here's our email:}
\href{mailto:letters@NYTimes.com}{\emph{letters@NYTimes.com}}\emph{.}

\emph{Follow The New York Times Opinion section on}
\href{https://www.facebookcorewwwi.onion/nytopinion}{\emph{Facebook}}\emph{,}
\href{http://twitter.com/NYTOpinion}{\emph{Twitter (@NYTopinion)}}
\emph{and}
\href{https://www.instagram.com/nytopinion/}{\emph{Instagram}}\emph{.}

Advertisement

\protect\hyperlink{after-bottom}{Continue reading the main story}

\hypertarget{site-index}{%
\subsection{Site Index}\label{site-index}}

\hypertarget{site-information-navigation}{%
\subsection{Site Information
Navigation}\label{site-information-navigation}}

\begin{itemize}
\tightlist
\item
  \href{https://help.nytimes3xbfgragh.onion/hc/en-us/articles/115014792127-Copyright-notice}{©~2020~The
  New York Times Company}
\end{itemize}

\begin{itemize}
\tightlist
\item
  \href{https://www.nytco.com/}{NYTCo}
\item
  \href{https://help.nytimes3xbfgragh.onion/hc/en-us/articles/115015385887-Contact-Us}{Contact
  Us}
\item
  \href{https://www.nytco.com/careers/}{Work with us}
\item
  \href{https://nytmediakit.com/}{Advertise}
\item
  \href{http://www.tbrandstudio.com/}{T Brand Studio}
\item
  \href{https://www.nytimes3xbfgragh.onion/privacy/cookie-policy\#how-do-i-manage-trackers}{Your
  Ad Choices}
\item
  \href{https://www.nytimes3xbfgragh.onion/privacy}{Privacy}
\item
  \href{https://help.nytimes3xbfgragh.onion/hc/en-us/articles/115014893428-Terms-of-service}{Terms
  of Service}
\item
  \href{https://help.nytimes3xbfgragh.onion/hc/en-us/articles/115014893968-Terms-of-sale}{Terms
  of Sale}
\item
  \href{https://spiderbites.nytimes3xbfgragh.onion}{Site Map}
\item
  \href{https://help.nytimes3xbfgragh.onion/hc/en-us}{Help}
\item
  \href{https://www.nytimes3xbfgragh.onion/subscription?campaignId=37WXW}{Subscriptions}
\end{itemize}
