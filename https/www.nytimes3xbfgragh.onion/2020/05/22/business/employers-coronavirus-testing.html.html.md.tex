Sections

SEARCH

\protect\hyperlink{site-content}{Skip to
content}\protect\hyperlink{site-index}{Skip to site index}

\href{https://www.nytimes3xbfgragh.onion/section/business}{Business}

\href{https://myaccount.nytimes3xbfgragh.onion/auth/login?response_type=cookie\&client_id=vi}{}

\href{https://www.nytimes3xbfgragh.onion/section/todayspaper}{Today's
Paper}

\href{/section/business}{Business}\textbar{}What Role Should Employers
Play in Testing Workers?

\url{https://nyti.ms/2LRfSDp}

\begin{itemize}
\item
\item
\item
\item
\item
\end{itemize}

\hypertarget{the-coronavirus-outbreak}{%
\subsubsection{\texorpdfstring{\href{https://www.nytimes3xbfgragh.onion/news-event/coronavirus?name=styln-coronavirus-markets\&region=TOP_BANNER\&variant=undefined\&block=storyline_menu_recirc\&action=click\&pgtype=Article\&impression_id=ebda9af0-e3a6-11ea-945d-0babcf13dc84}{The
Coronavirus
Outbreak}}{The Coronavirus Outbreak}}\label{the-coronavirus-outbreak}}

\begin{itemize}
\tightlist
\item
  live\href{https://www.nytimes3xbfgragh.onion/2020/08/21/world/covid-19-coronavirus.html?name=styln-coronavirus-markets\&region=TOP_BANNER\&variant=undefined\&block=storyline_menu_recirc\&action=click\&pgtype=Article\&impression_id=ebda9af1-e3a6-11ea-945d-0babcf13dc84}{Latest
  Updates}
\item
  \href{https://www.nytimes3xbfgragh.onion/interactive/2020/us/coronavirus-us-cases.html?name=styln-coronavirus-markets\&region=TOP_BANNER\&variant=undefined\&block=storyline_menu_recirc\&action=click\&pgtype=Article\&impression_id=ebda9af2-e3a6-11ea-945d-0babcf13dc84}{Maps
  and Cases}
\item
  \href{https://www.nytimes3xbfgragh.onion/interactive/2020/science/coronavirus-vaccine-tracker.html?name=styln-coronavirus-markets\&region=TOP_BANNER\&variant=undefined\&block=storyline_menu_recirc\&action=click\&pgtype=Article\&impression_id=ebdac200-e3a6-11ea-945d-0babcf13dc84}{Vaccine
  Tracker}
\item
  \href{https://www.nytimes3xbfgragh.onion/2020/08/19/us/colleges-closing-covid.html?name=styln-coronavirus-markets\&region=TOP_BANNER\&variant=undefined\&block=storyline_menu_recirc\&action=click\&pgtype=Article\&impression_id=ebdac201-e3a6-11ea-945d-0babcf13dc84}{Colleges
  Closing}
\item
  \href{https://www.nytimes3xbfgragh.onion/live/2020/08/20/business/stock-market-today-coronavirus?name=styln-coronavirus-markets\&region=TOP_BANNER\&variant=undefined\&block=storyline_menu_recirc\&action=click\&pgtype=Article\&impression_id=ebdac202-e3a6-11ea-945d-0babcf13dc84}{Economy}
\end{itemize}

Advertisement

\protect\hyperlink{after-top}{Continue reading the main story}

Supported by

\protect\hyperlink{after-sponsor}{Continue reading the main story}

\hypertarget{what-role-should-employers-play-in-testing-workers}{%
\section{What Role Should Employers Play in Testing
Workers?}\label{what-role-should-employers-play-in-testing-workers}}

Amazon and other companies are planning to test workers for the
coronavirus. But there is little federal guidance, and some fear it
could lead to a false sense of security.

\includegraphics{https://static01.graylady3jvrrxbe.onion/images/2020/05/23/world/23-VIRUS-EMPLOYERS-02-PRINT/merlin_172725237_d9270118-d91b-4c94-90c7-c1d915afe799-articleLarge.jpg?quality=75\&auto=webp\&disable=upscale}

By \href{https://www.nytimes3xbfgragh.onion/by/steve-eder}{Steve Eder},
\href{https://nytimes3xbfgragh.onion/by/ellen-gabler/}{Ellen Gabler},
\href{https://www.nytimes3xbfgragh.onion/by/sarah-kliff}{Sarah Kliff}
and \href{https://www.nytimes3xbfgragh.onion/by/heather-murphy}{Heather
Murphy}

\begin{itemize}
\item
  May 22, 2020
\item
  \begin{itemize}
  \item
  \item
  \item
  \item
  \item
  \end{itemize}
\end{itemize}

As the country reopens, employers are looking into how to safely bring
back their workers. One recurring question: Should they be tested for
the new coronavirus?

Some businesses are moving ahead. In Indianapolis, the family-owned
Shapiro's Delicatessen tested about 25 employees in its parking lot this
month.

Amazon plans to spend as much as \$1 billion this year to regularly test
its work force, while laying the groundwork to build its own lab near
the Cincinnati airport.

Las Vegas casinos are testing thousands of employees as they prepare to
return to work, collecting nasal samples in convention halls.

And Major League Baseball, eager to begin its season, is proposing a
detailed regimen that involves testing players and critical staff
members multiple times a week.

While public health experts and government officials have emphasized
that widespread testing will be critical to reopening, there is little
clear guidance from state and federal agencies on the role employers
should play in detecting and tracking the coronavirus. As a result,
businesses are largely on their own in sorting out whether to test ---
and how to do it --- to reassure employees and customers. For now, many
companies are just waiting.

``It is a really hard conversation because people want absolutes: `If I
do this, will it guarantee I'll have a safe workplace?' None of the
testing is going to provide that right now,'' said John Constantine, the
chief executive of ARCPoint Franchise Group, a nationwide lab network
offering virus testing to employers. He added that if done smartly,
testing could reduce health risks. ``Even if it's not perfect, some
testing is better than no testing.''

Despite rapid advancements in testing, there are still limitations.
Diagnostic tests, for example, only detect infections during a certain
period. And while blood tests administered after an infection can find
antibodies that might offer some immunity, they should not be used alone
to make decisions about when people can return to work, the Association
of Public Health Laboratories and Council of State and Territorial
Epidemiologists
\href{https://www.aphlblog.org/antibody-testing-important-covid-19-response-data-needed-expand-role/}{warned
this month}.

Some companies have been sensitive about announcing such plans because
of the shortages that left many patients and health care workers unable
to be tested in hard-hit areas. While capacity has dramatically
increased in recent weeks, it is unclear whether labs can keep up with
demand if employers nationwide repeatedly test workers.

Some public health officials say broad-based testing may be unnecessary,
and might have unintended consequences.

\hypertarget{latest-updates-the-coronavirus-outbreak-and-the-economy}{%
\section{\texorpdfstring{\href{https://www.nytimes3xbfgragh.onion/live/2020/08/20/business/stock-market-today-coronavirus?action=click\&pgtype=Article\&state=default\&region=MAIN_CONTENT_1\&context=storylines_live_updates}{Latest
Updates: The Coronavirus Outbreak and the
Economy}}{Latest Updates: The Coronavirus Outbreak and the Economy}}\label{latest-updates-the-coronavirus-outbreak-and-the-economy}}

\href{https://www.nytimes3xbfgragh.onion/live/2020/08/20/business/stock-market-today-coronavirus?action=click\&pgtype=Article\&state=default\&region=MAIN_CONTENT_1\&context=storylines_live_updates\#american-airlines-to-stop-flights-to-15-cities-after-government-aid-ends}{21h
ago}

\href{https://www.nytimes3xbfgragh.onion/live/2020/08/20/business/stock-market-today-coronavirus?action=click\&pgtype=Article\&state=default\&region=MAIN_CONTENT_1\&context=storylines_live_updates\#american-airlines-to-stop-flights-to-15-cities-after-government-aid-ends}{American
Airlines to stop flights to 15 cities after government aid ends.}

\href{https://www.nytimes3xbfgragh.onion/live/2020/08/20/business/stock-market-today-coronavirus?action=click\&pgtype=Article\&state=default\&region=MAIN_CONTENT_1\&context=storylines_live_updates\#without-school-plays-and-assemblies-a-technicians-livelihood-withers}{22h
ago}

\href{https://www.nytimes3xbfgragh.onion/live/2020/08/20/business/stock-market-today-coronavirus?action=click\&pgtype=Article\&state=default\&region=MAIN_CONTENT_1\&context=storylines_live_updates\#without-school-plays-and-assemblies-a-technicians-livelihood-withers}{Without
school plays and assemblies, a technician's livelihood withers.}

\href{https://www.nytimes3xbfgragh.onion/live/2020/08/20/business/stock-market-today-coronavirus?action=click\&pgtype=Article\&state=default\&region=MAIN_CONTENT_1\&context=storylines_live_updates\#finding-a-job-after-a-long-search-but-settling-for-less-pay}{22h
ago}

\href{https://www.nytimes3xbfgragh.onion/live/2020/08/20/business/stock-market-today-coronavirus?action=click\&pgtype=Article\&state=default\&region=MAIN_CONTENT_1\&context=storylines_live_updates\#finding-a-job-after-a-long-search-but-settling-for-less-pay}{Finding
a job after a long search, but settling for less pay.}

\href{https://www.nytimes3xbfgragh.onion/live/2020/08/20/business/stock-market-today-coronavirus?action=click\&pgtype=Article\&state=default\&region=MAIN_CONTENT_1\&context=storylines_live_updates}{See
more updates}

More live coverage:
\href{https://www.nytimes3xbfgragh.onion/2020/08/21/world/covid-19-coronavirus.html?action=click\&pgtype=Article\&state=default\&region=MAIN_CONTENT_1\&context=storylines_live_updates}{Global}

``We don't want people to get a false sense of security,'' said Karen
Landers, a district medical officer with the Alabama Department of
Public Health, which is not recommending that employers test all workers
as they come back. ``You might have a negative now and later be
exposed.''

The
\href{https://www.nytimes3xbfgragh.onion/2020/05/22/health/Coronavirus-touching-surfaces.html}{Centers
for Disease Control and Prevention} has not offered guidelines on the
issue, though
it\href{https://www.nytimes3xbfgragh.onion/2020/05/15/us/cdc-coronavirus-checklists-decision-trees.html}{released
checklists this month} to help various industries decide when to reopen.
A checklist for workplaces asks if monitoring is in place, noting that
businesses should implement procedures to check employees daily on
arrival for signs of illness.

Many employers have already adopted protective measures like checking
temperatures, disinfecting surfaces after every shift, requiring masks
and social distancing.

\includegraphics{https://static01.graylady3jvrrxbe.onion/images/2020/05/23/world/23-VIRUS-EMPLOYERS-01-PRINT/merlin_172725846_267373bb-2c08-4654-928a-edb9ee2eb7fc-articleLarge.jpg?quality=75\&auto=webp\&disable=upscale}

CVS Health, which has about 300,000 employees, has been taking
employees' temperatures in its pharmacies and retail stores since April.
Anyone with a temperature of 100 degrees or higher is to be sent home.
That protocol is being rolled out to the company's distribution centers
and to several corporate offices as they open up. CVS may consider
testing employees as capacity increases, said Michael DeAngelis, a
company spokesman.

Walmart, with 2.2 million workers, has taken similar steps. Its chief
executive, Doug McMillon, on an earnings call this past Tuesday, said
the retail giant was assessing how to proceed with diagnostic and
antibody tests on employees.

``There are a lot of moving parts there,'' he said.

For employers trying to decide whether to test staff, their decisions
may depend on how much contact workers have with one another, and how
prevalent the virus is in the surrounding community.

In a very low-risk area, ``you can probably get away without it,'' said
Dr. Ashish Jha, a professor of global health at Harvard University.
``But if you're a company in a major city or suburb, and you have a
bunch of people coming into an indoor space together, it will be
critical.''

Dense work places, such as factories and meatpacking plants, will
probably need to test workers more often than corporate offices with
less in-person interaction, he said

Labs and marketers are pitching employers on ``return-to-work'' programs
that include everything from at-home and onsite testing, with medical
staff available, to questionnaires that help screen for which employees
might benefit from receiving a diagnostic test.

Across the country, testing has increased to about 400,000 people a day,
more than double since last month, according to
\href{https://covidtracking.com/}{the Covid Tracking Project}. Some
experts say that number is still inadequate. So far, nearly 13 million
tests have been completed, according to the
\href{https://www.nytimes3xbfgragh.onion/2020/05/22/health/Coronavirus-touching-surfaces.html}{C.D.C.},
accounting for less than 4 percent of the population in the United
States. It is unclear how many of the tests are diagnostic and how many
are for antibodies.

Two of the county's largest commercial laboratories, Quest Diagnostics
and LabCorp, have said they expect to continue building capacity for
performing both tests.

LabCorp can now handle about 75,000 diagnostic tests a day and plans to
double that number by June. Quest has similar capacity. As for antibody
tests, Quest can run 200,000 daily, while LabCorp expects to be able to
process 300,000 by June.

``We're going to just do everything we can, every which way we can, to
be able to build capacity to do as many tests as fast as possible,''
Adam Schechter, LabCorp's chief executive, said in an interview.

One nascent strategy circulating among public health experts is running
``pooled'' coronavirus tests, in which a workplace could combine
multiple saliva or nasal swabs into one larger sample representing
dozens of employees.

This technique --- which traces back to World War II, when soldiers were
tested
\href{https://www.semanticscholar.org/paper/The-Detection-of-Defective-Members-of-Large-Dorfman/3226748e9905a4ad5174c7a3fad2fccf1abfa105}{en
masse for syphilis} --- would allow companies to see whether there is
coronavirus circulating among workers. A positive result would lead to
further individual testing within a group.

``You can use one test to rule out a big group of people, and that makes
it cost-effective,'' said Natalie Dean, an assistant professor of
biostatistics at the University of Florida.

A health clinic in Germany is already using pooled testing to monitor
workers, and a virology lab at Stanford University is exploring how to
use the technique locally.

Some employees are clamoring for tests, believing it will keep them
safe, but others are skeptical.

In Atlanta, Morehouse School of Medicine decided to require periodic
testing for its 1,100 employees, no matter if they work on the medical,
marketing or administrative side of the operation. And all employees
will be required to complete a test before returning to work.

Dr. Valerie Montgomery Rice, president and dean of the medical school,
said she decided on that approach after getting feedback that many
employees wanted mandatory testing. Some pointed out that staff members
were required to be tested for tuberculosis, another infectious disease.

The frequency of testing will be determined along the way --- maybe
every two weeks, maybe every month, Dr. Rice said. Results will be sent
to workers and the school's employee health office. Last Friday, after
running 500 tests, the first positive result came in. The employee, who
did not have any symptoms, will be allowed to work remotely or take sick
leave.

In Las Vegas,
\href{https://m.culinaryunion226.org/news/press/las-vegas-gaming-industry-to-offer-covid-19-testing-for-employees-prior-to-return-to-work}{the
culinary union} had requested testing for the virus and antibodies as
part of casinos' plans to protect workers and guests. On Tuesday, gaming
and resort operators, along with the union, announced plans for
widespread testing. The local health authority, the Southern Nevada
Health District, said it supported systematic testing of employees,
especially those who have frequent contact with the public.

Some casino operators --- Las Vegas Sands, Station Casinos and Wynn
Resorts --- have already begun testing. ``It is to protect, first and
foremost, employees,'' said Ron Reese, a spokesman for Las Vegas Sands,
who said that the practice was expected to continue and that staff could
bring family members in for testing. ``It sends a signal that we are
doing all we can to assure people when they come to Las Vegas, that a
company like ours is doing everything possible to make it the safest
environment.''

Image

Brian Shapiro, owner of Shapiro's Delicatessen in Indianapolis, had
about 25 employees tested this month.Credit...Luke Sharrett for The New
York Times

At Shapiro's, the Indianapolis deli, the owner, Brian Shapiro, said some
employees were apprehensive when he announced plans to test them before
reopening the dining room. But soon, workers at the 115-year-old
establishment --- which has stayed up and running to provide carryout
corned beef and pastrami sandwiches --- realized how much they depended
on one another to remain healthy. ``They became more of a team,'' said
Mr. Shapiro, 61.

``I want to do the best I possibly can to keep myself, my employees, my
customers --- to keep them from getting sick.''

Reporting was contributed by Michael Corkery, Sopan Deb, Benjamin
Mueller, Katie Thomas, Karen Weise and James Wagner.

Advertisement

\protect\hyperlink{after-bottom}{Continue reading the main story}

\hypertarget{site-index}{%
\subsection{Site Index}\label{site-index}}

\hypertarget{site-information-navigation}{%
\subsection{Site Information
Navigation}\label{site-information-navigation}}

\begin{itemize}
\tightlist
\item
  \href{https://help.nytimes3xbfgragh.onion/hc/en-us/articles/115014792127-Copyright-notice}{©~2020~The
  New York Times Company}
\end{itemize}

\begin{itemize}
\tightlist
\item
  \href{https://www.nytco.com/}{NYTCo}
\item
  \href{https://help.nytimes3xbfgragh.onion/hc/en-us/articles/115015385887-Contact-Us}{Contact
  Us}
\item
  \href{https://www.nytco.com/careers/}{Work with us}
\item
  \href{https://nytmediakit.com/}{Advertise}
\item
  \href{http://www.tbrandstudio.com/}{T Brand Studio}
\item
  \href{https://www.nytimes3xbfgragh.onion/privacy/cookie-policy\#how-do-i-manage-trackers}{Your
  Ad Choices}
\item
  \href{https://www.nytimes3xbfgragh.onion/privacy}{Privacy}
\item
  \href{https://help.nytimes3xbfgragh.onion/hc/en-us/articles/115014893428-Terms-of-service}{Terms
  of Service}
\item
  \href{https://help.nytimes3xbfgragh.onion/hc/en-us/articles/115014893968-Terms-of-sale}{Terms
  of Sale}
\item
  \href{https://spiderbites.nytimes3xbfgragh.onion}{Site Map}
\item
  \href{https://help.nytimes3xbfgragh.onion/hc/en-us}{Help}
\item
  \href{https://www.nytimes3xbfgragh.onion/subscription?campaignId=37WXW}{Subscriptions}
\end{itemize}
