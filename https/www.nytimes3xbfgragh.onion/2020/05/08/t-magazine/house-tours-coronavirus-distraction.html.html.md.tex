Sections

SEARCH

\protect\hyperlink{site-content}{Skip to
content}\protect\hyperlink{site-index}{Skip to site index}

\href{https://myaccount.nytimes3xbfgragh.onion/auth/login?response_type=cookie\&client_id=vi}{}

\href{https://www.nytimes3xbfgragh.onion/section/todayspaper}{Today's
Paper}

10 Eclectic Homes to Get Lost In

\url{https://nyti.ms/2zoiIfY}

\begin{itemize}
\item
\item
\item
\item
\item
\end{itemize}

Advertisement

\protect\hyperlink{after-top}{Continue reading the main story}

Supported by

\protect\hyperlink{after-sponsor}{Continue reading the main story}

\hypertarget{10-eclectic-homes-to-get-lost-in}{%
\section{10 Eclectic Homes to Get Lost
In}\label{10-eclectic-homes-to-get-lost-in}}

Rooms that are not just different than those you're self-isolating in
but unlike most others, too.

Published May 8, 2020Updated May 9, 2020

\begin{itemize}
\item
\item
\item
\item
\item
\end{itemize}

Months into collective efforts to
\href{https://www.nytimes3xbfgragh.onion/article/flatten-curve-coronavirus.html}{flatten
the curve}, those of us lucky enough to be
\href{https://www.nytimes3xbfgragh.onion/2020/03/10/technology/working-from-home.html}{working
remotely} may be feeling overly familiar with our own homes, rooms once
traveled through quite casually on our way to somewhere else and that
must now function as our entire world. Inevitably, they fall short, and
we start to imagine ourselves --- worry-free --- in different, better
versions.
\href{https://www.nytimes3xbfgragh.onion/2020/04/09/technology/zoom-security.html}{Zoom
calls} and Instagram Live clips taunt us with small slivers of
alternatives --- intimacy has, in so many ways, been curtailed, and yet
we're getting glimpses into spaces we've never seen before. Perhaps you,
too, have been distracted by the poster, stack of books or unusually
colored wall behind the person on the other end of the platform and
wondered about what is just out of view and what it might reveal about
its owner's true essence.

To satisfy that voyeuristic impulse, we recommend browsing some of the
most eclectic homes featured in T. These spaces are imbued not simply
with good taste (which over time can become stale) but with a point of
view, and therefore really do seem like worlds unto themselves. Take the
immaculately preserved Corona, Queens,
\href{https://www.nytimes3xbfgragh.onion/2020/02/20/t-magazine/louis-armstrong-home-queens.html}{abode}
of the musician Louis Armstrong (with its mirrored bathroom and
space-age-inspired kitchen), a
\href{https://www.nytimes3xbfgragh.onion/2019/11/05/t-magazine/portland-house-allie-furlotti-osmose-design.html}{surrealist
house} in Portland with LED lights lining the staircase or an erotic
film studio turned
\href{https://www.nytimes3xbfgragh.onion/2018/09/14/t-magazine/los-angeles-dream-house-flamingo-estate.html}{California
dream house}. Each one is its own sort of sanctuary, and a virtual tour
or two is sure to get you, however briefly, out of your house and head.

\includegraphics{https://static01.graylady3jvrrxbe.onion/images/2019/07/16/t-magazine/16tmag-cary-01/16tmag-cary-01-videoSixteenByNineJumbo1600.png}

\hypertarget{an-eccentric-upstate-home-that-some-people-confuse-for-a-restaurant}{%
\subsubsection{\texorpdfstring{\textbf{\href{https://www.nytimes3xbfgragh.onion/2019/07/17/t-magazine/art/cary-leibowitz-simon-lince-upstate-home.html}{An
Eccentric Upstate Home That Some People Confuse for a
Restaurant}}}{An Eccentric Upstate Home That Some People Confuse for a Restaurant}}\label{an-eccentric-upstate-home-that-some-people-confuse-for-a-restaurant}}

In 2005, the creative director Simon Lince and the artist Cary Leibowitz
(who sometimes works under the memorable sobriquet Candy Ass) purchased
a
\href{https://www.nytimes3xbfgragh.onion/2017/10/23/t-magazine/columbia-county-gay-utopia-new-york.html}{farmhouse
in Ghent}, in upstate New York. Its main purpose was to act as a weekend
retreat (Leibowitz owns a
\href{https://www.nytimes3xbfgragh.onion/video/t-magazine/design/100000005955125/house-tour-cary-leibowitz.html?playlistId=video/house-tours}{no
less distinctive townhouse} in Harlem), but it's also an effective
warehouse for their sprawling collection of antiques and art objects ---
including a marquee of aluminum letters spelling out ``Linceowitz,'' a
combination of their two names, below which, in 2016, the couple
married.

\begin{center}\rule{0.5\linewidth}{\linethickness}\end{center}

\includegraphics{https://static01.graylady3jvrrxbe.onion/images/2020/02/20/t-magazine/20tmag-louisarmstrong-slide-EEFM/20tmag-louisarmstrong-slide-EEFM-articleLarge.jpg?quality=75\&auto=webp\&disable=upscale}

\hypertarget{louis-armstrong-the-king-of-queens}{%
\subsubsection{\texorpdfstring{\textbf{\href{https://www.nytimes3xbfgragh.onion/2020/02/20/t-magazine/louis-armstrong-home-queens.html}{Louis
Armstrong, the King of
Queens}}}{Louis Armstrong, the King of Queens}}\label{louis-armstrong-the-king-of-queens}}

Nearly 50 years after Louis Armstrong's death, his Corona, Queens, home
--- now the site of the
\href{https://www.louisarmstronghouse.org/}{Louis Armstrong House
Museum} --- remains remarkably well preserved, right down to the
half-full bottle of Jack Daniel's that remains in his liquor cabinet. As
a result, it's not only a stunning tribute to the musician (housing an
archive of his books and his array of trumpets) but also to the
influence of midcentury design.

\begin{center}\rule{0.5\linewidth}{\linethickness}\end{center}

\includegraphics{https://static01.graylady3jvrrxbe.onion/images/2019/03/04/t-magazine/04tmag-isaac-mizrahi/04tmag-isaac-mizrahi-videoSixteenByNineJumbo1600.png}

\hypertarget{inside-isaac-mizrahis-art-filled-greenwich-village-apartment}{%
\subsubsection{\texorpdfstring{\textbf{\href{https://www.nytimes3xbfgragh.onion/2019/03/06/t-magazine/isaac-mizrahi-home-tour.html}{Inside
Isaac Mizrahi's Art-Filled Greenwich Village
Apartment}}}{Inside Isaac Mizrahi's Art-Filled Greenwich Village Apartment}}\label{inside-isaac-mizrahis-art-filled-greenwich-village-apartment}}

After purchasing three contiguous apartments on the 14th floor of a
prewar Greenwich Village apartment building, the designer
\href{https://www.nytimes3xbfgragh.onion/topic/person/isaac-mizrahi}{Isaac
Mizrahi} set about connecting them, creating a roomy, art-filled nest
where he can, he says, ``adjust knickknacks'' and ``watch `Jeopardy!' on
DVR'' --- his favorite pastime. The views of Lower Manhattan are pretty
nice, too.

\begin{center}\rule{0.5\linewidth}{\linethickness}\end{center}

\includegraphics{https://static01.graylady3jvrrxbe.onion/images/2020/03/23/t-magazine/23tmag-wallacavage/23tmag-wallacavage-videoSixteenByNine3000.jpg}

\hypertarget{an-artists-gothic-townhouse-filled-with-octopus-shaped-lamps}{%
\subsubsection{\texorpdfstring{\textbf{\href{https://www.nytimes3xbfgragh.onion/2020/03/25/t-magazine/adam-wallacavage-home.html}{An
Artist's Gothic Townhouse, Filled With Octopus-Shaped
Lamps}}}{An Artist's Gothic Townhouse, Filled With Octopus-Shaped Lamps}}\label{an-artists-gothic-townhouse-filled-with-octopus-shaped-lamps}}

Home and work blend seamlessly together within the lighting designer and
sculptor Adam Wallacavage's Philadelphia brownstone: The same year that
he purchased the three-story house, in 2000, he made his first octopus
chandelier, on a whim, for his Jules Verne-themed living room. The piece
--- and subsequent iterations of it, many of which adorn other rooms in
the house --- quickly became a signature of his practice. Throughout,
taxidermy and Art Nouveau details abound, but the sea remains a
preoccupation: As Wallacavage says, ``You never know what could wash
up.''

\begin{center}\rule{0.5\linewidth}{\linethickness}\end{center}

Image

For the living room of Allie Furlotti and Adam Kostiv's Portland home,
the firm Osmose Design made a custom enameled birch D.J. station to pair
with a B\&B Italia sofa, a Glas Italia coffee table and a Lievore
Altherr Molina for Arper Colina lounge chair upholstered in Tibetan
lamb's wool.Credit...Dave Lauridsen

\hypertarget{outside-its-another-portland-house-inside-its-something-else}{%
\subsubsection{\texorpdfstring{\textbf{\href{https://www.nytimes3xbfgragh.onion/2019/11/05/t-magazine/portland-house-allie-furlotti-osmose-design.html}{Outside,
It's Another Portland House. Inside, It's Something
Else.}}}{Outside, It's Another Portland House. Inside, It's Something Else.}}\label{outside-its-another-portland-house-inside-its-something-else}}

In a tidy, polite neighborhood in Portland, Ore., the home of the
comedian and philanthropist Allie Furlotti and the multimedia artist
Adam Kostiv is, Nick Marino writes, ``a Trojan horse of subversive
design.'' From the exterior, it's unobtrusive; inside, though, it's
lined with unexpected materials (labradorite --- ``Carrara marble gone
goth'' --- and vinyl) and dotted with surrealist details, a result of a
thoughtful, offbeat redesign by Andee Hess of
\href{http://www.osmosedesign.com/}{Osmose Design}.

\begin{center}\rule{0.5\linewidth}{\linethickness}\end{center}

Image

In a hallway to the bathroom of Guillermo Santomà`s Casa Horta, a Quinta
chair by Mario Botta and an Arnold Circus stool by Martino Gamper
modified by Santomà.Credit...Ricardo Labougle

Image

Santomà managed to fit a small pool in the backyard.Credit...Ricardo
Labougle

\hypertarget{in-barcelona-a-house-that-was-remade-by-taking-it-apart}{%
\subsubsection{\texorpdfstring{\textbf{\href{https://www.nytimes3xbfgragh.onion/2018/08/27/t-magazine/architect-guillermo-santoma-casa-horta-barcelona.html}{In
Barcelona, a House That Was Remade by Taking It
Apart}}}{In Barcelona, a House That Was Remade by Taking It Apart}}\label{in-barcelona-a-house-that-was-remade-by-taking-it-apart}}

The furniture designer \href{http://guillermosantoma.com/}{Guillermo
Santomà}'s Casa Horta in Barcelona is another house that plays tricks on
the visitor: What at first appears to be a modest single-story home
actually comprises three stories, with rooms that seem to melt into one
another thanks to shared color schemes, holes in walls and grated floors
that light shines through. The bones of the house date back to the early
1900s, but after purchasing it roughly six years ago, Santomà and a
small crew started remodeling, which, for the most part, meant tearing
down existing walls and creating space. In a way, the result is an
extension of his design practice: ``That it grows, that it connects,
that it gets mixed up,'' he says, ``you don't quite know where one thing
ends and another begins.''

\begin{center}\rule{0.5\linewidth}{\linethickness}\end{center}

Image

Kevin Nakashima has never moved from his family home, the first that his
father, George Nakashima, built on the property in 1946. In the office
sits the handcrafted walnut furniture for which his father became
famous, and a lifetime's accumulation of art, books, papers and family
mementos, including artworks by Ben Shahn.Credit...Chris Mottalini

\hypertarget{how-two-children-are-keeping-their-fathers-design-legacy-alive}{%
\subsubsection{\texorpdfstring{\textbf{\href{https://www.nytimes3xbfgragh.onion/2020/03/16/t-magazine/george-nakashima-legacy.html}{How
Two Children Are Keeping Their Father's Design Legacy
Alive}}}{How Two Children Are Keeping Their Father's Design Legacy Alive}}\label{how-two-children-are-keeping-their-fathers-design-legacy-alive}}

Throughout the middle of the 20th century, the furniture designer George
Nakashima developed a parcel of land in eastern Pennsylvania into an
8.8-acre compound that housed his design studio and his family ---
including his daughter,
\href{https://www.nytimes3xbfgragh.onion/2013/09/05/garden/mira-nakashima-a-daughter-knocks-on-wood.html}{Mira},
and son, Kevin, both of whom still live in the homes their father
created. These days, the structures stand as a monument to Nakashima's
legacy and to the folk-art principles he upheld: utilitarian, deeply
personal and beautiful.

\begin{center}\rule{0.5\linewidth}{\linethickness}\end{center}

\includegraphics{https://static01.graylady3jvrrxbe.onion/images/2018/09/14/t-magazine/14tmag-flamingo-estate/14tmag-flamingo-estate-videoSixteenByNine3000.jpg}

\hypertarget{how-a-love-of-jane-fonda-and-the-color-pink-created-a-california-dream-house}{%
\subsubsection{\texorpdfstring{\textbf{\href{https://www.nytimes3xbfgragh.onion/2018/09/14/t-magazine/los-angeles-dream-house-flamingo-estate.html}{How
a Love of Jane Fonda and the Color Pink Created a California Dream
House}}}{How a Love of Jane Fonda and the Color Pink Created a California Dream House}}\label{how-a-love-of-jane-fonda-and-the-color-pink-created-a-california-dream-house}}

You don't even have to step through the front door to be charmed by the
creative director
\href{https://www.nytimes3xbfgragh.onion/2019/03/14/t-magazine/owl-bureau-los-angeles-book-store.html}{Richard
Christiansen}'s aptly named Flamingo Estate (the exterior of the Los
Angeles house is painted a soft pink). Christiansen himself made an
offer on the home before ever setting foot inside. It was only afterward
that he began to uncover its unexpected history as the headquarters of
an erotic film studio.

\begin{center}\rule{0.5\linewidth}{\linethickness}\end{center}

Image

The courtyard of House N (2008), in Oita. The architect Sou Fujimoto has
become known for his adventurous designs.Credit...Benjamin Hosking

Image

Omotesando Branches, a 2014 multiuse building with an apartment, offices
and ground-floor retail space, in the Shibuya district of
Tokyo.Credit...Benjamin Hosking

\hypertarget{the-architect-making-conceptual-art-out-of-buildings}{%
\subsubsection{\texorpdfstring{\textbf{\href{https://www.nytimes3xbfgragh.onion/2020/03/02/t-magazine/sou-fujimoto.html}{The
Architect Making Conceptual Art Out of
Buildings}}}{The Architect Making Conceptual Art Out of Buildings}}\label{the-architect-making-conceptual-art-out-of-buildings}}

The architect \href{http://www.sou-fujimoto.net/}{Sou Fujimoto} spent
the six years following his graduation from the University of Tokyo
doing, well, nothing. During that period,
\href{https://www.nytimes3xbfgragh.onion/by/nikil-saval}{Nikil Saval}
writes, he honed ``his idea of what architecture ought to be.'' And it
proved fruitful: He has since gone on to render that idea ---
interrogating sets of oppositions like city and nature, inside and
outside, isolation and exposure --- in a series of clever, conceptual
buildings and residences. The architect's work, in Tokyo and beyond,
stands out for its ``perfect fusion of conceptual daring and
architectural function.''

\begin{center}\rule{0.5\linewidth}{\linethickness}\end{center}

Image

Graciela Iturbide on the rooftop terrace of the studio that her son
Mauricio Rocha co-designed for her in Mexico City.Credit...Ben Sklar

\hypertarget{the-sunlit-studio-a-son-built-for-his-photographer-mother}{%
\subsubsection{\texorpdfstring{\textbf{\href{https://www.nytimes3xbfgragh.onion/2018/05/18/t-magazine/design/graciela-iturbide-mauricio-rocha-studio.html}{The
Sunlit Studio a Son Built for His Photographer
Mother}}}{The Sunlit Studio a Son Built for His Photographer Mother}}\label{the-sunlit-studio-a-son-built-for-his-photographer-mother}}

The architect Mauricio Rocha, of the firm
\href{http://www.tallerdearquitectura.com.mx/}{Taller / Mauricio Rocha +
Gabriela Carrillo}, has designed university buildings and museums ---
but it's the studio he created for his mother, the renowned photographer
\href{https://www.nytimes3xbfgragh.onion/2019/01/08/lens/graciela-iturbide-mexico-photos.html}{Graciela
Iturbide}, that is among his most significant works. It emerged out of
hundreds of possible designs and two years of work, and represents a
personal creative exchange as much as a professional one: ``We share
ideas about the way we see the world,'' said Rocha.

Advertisement

\protect\hyperlink{after-bottom}{Continue reading the main story}

\hypertarget{site-index}{%
\subsection{Site Index}\label{site-index}}

\hypertarget{site-information-navigation}{%
\subsection{Site Information
Navigation}\label{site-information-navigation}}

\begin{itemize}
\tightlist
\item
  \href{https://help.nytimes3xbfgragh.onion/hc/en-us/articles/115014792127-Copyright-notice}{©~2020~The
  New York Times Company}
\end{itemize}

\begin{itemize}
\tightlist
\item
  \href{https://www.nytco.com/}{NYTCo}
\item
  \href{https://help.nytimes3xbfgragh.onion/hc/en-us/articles/115015385887-Contact-Us}{Contact
  Us}
\item
  \href{https://www.nytco.com/careers/}{Work with us}
\item
  \href{https://nytmediakit.com/}{Advertise}
\item
  \href{http://www.tbrandstudio.com/}{T Brand Studio}
\item
  \href{https://www.nytimes3xbfgragh.onion/privacy/cookie-policy\#how-do-i-manage-trackers}{Your
  Ad Choices}
\item
  \href{https://www.nytimes3xbfgragh.onion/privacy}{Privacy}
\item
  \href{https://help.nytimes3xbfgragh.onion/hc/en-us/articles/115014893428-Terms-of-service}{Terms
  of Service}
\item
  \href{https://help.nytimes3xbfgragh.onion/hc/en-us/articles/115014893968-Terms-of-sale}{Terms
  of Sale}
\item
  \href{https://spiderbites.nytimes3xbfgragh.onion}{Site Map}
\item
  \href{https://help.nytimes3xbfgragh.onion/hc/en-us}{Help}
\item
  \href{https://www.nytimes3xbfgragh.onion/subscription?campaignId=37WXW}{Subscriptions}
\end{itemize}
