Sections

SEARCH

\protect\hyperlink{site-content}{Skip to
content}\protect\hyperlink{site-index}{Skip to site index}

\href{https://www.nytimes3xbfgragh.onion/section/nyregion}{New York}

\href{https://myaccount.nytimes3xbfgragh.onion/auth/login?response_type=cookie\&client_id=vi}{}

\href{https://www.nytimes3xbfgragh.onion/section/todayspaper}{Today's
Paper}

\href{/section/nyregion}{New York}\textbar{}Three Children Have Died in
New York of Illness Linked to Virus

\url{https://nyti.ms/3clLxby}

\begin{itemize}
\item
\item
\item
\item
\item
\end{itemize}

\hypertarget{the-coronavirus-outbreak}{%
\subsubsection{\texorpdfstring{\href{https://www.nytimes3xbfgragh.onion/news-event/coronavirus?name=styln-coronavirus-national\&region=TOP_BANNER\&variant=undefined\&block=storyline_menu_recirc\&action=click\&pgtype=Article\&impression_id=5d883de0-e39a-11ea-a48e-b9252519409f}{The
Coronavirus
Outbreak}}{The Coronavirus Outbreak}}\label{the-coronavirus-outbreak}}

\begin{itemize}
\tightlist
\item
  live\href{https://www.nytimes3xbfgragh.onion/2020/08/21/world/covid-19-coronavirus.html?name=styln-coronavirus-national\&region=TOP_BANNER\&variant=undefined\&block=storyline_menu_recirc\&action=click\&pgtype=Article\&impression_id=5d883de1-e39a-11ea-a48e-b9252519409f}{Latest
  Updates}
\item
  \href{https://www.nytimes3xbfgragh.onion/interactive/2020/us/coronavirus-us-cases.html?name=styln-coronavirus-national\&region=TOP_BANNER\&variant=undefined\&block=storyline_menu_recirc\&action=click\&pgtype=Article\&impression_id=5d883de2-e39a-11ea-a48e-b9252519409f}{Maps
  and Cases}
\item
  \href{https://www.nytimes3xbfgragh.onion/interactive/2020/science/coronavirus-vaccine-tracker.html?name=styln-coronavirus-national\&region=TOP_BANNER\&variant=undefined\&block=storyline_menu_recirc\&action=click\&pgtype=Article\&impression_id=5d8864f0-e39a-11ea-a48e-b9252519409f}{Vaccine
  Tracker}
\item
  \href{https://www.nytimes3xbfgragh.onion/2020/08/19/us/colleges-closing-covid.html?name=styln-coronavirus-national\&region=TOP_BANNER\&variant=undefined\&block=storyline_menu_recirc\&action=click\&pgtype=Article\&impression_id=5d8864f1-e39a-11ea-a48e-b9252519409f}{Colleges
  Closing}
\item
  \href{https://www.nytimes3xbfgragh.onion/live/2020/08/20/business/stock-market-today-coronavirus?name=styln-coronavirus-national\&region=TOP_BANNER\&variant=undefined\&block=storyline_menu_recirc\&action=click\&pgtype=Article\&impression_id=5d8864f2-e39a-11ea-a48e-b9252519409f}{Economy}
\end{itemize}

Advertisement

\protect\hyperlink{after-top}{Continue reading the main story}

Supported by

\protect\hyperlink{after-sponsor}{Continue reading the main story}

\hypertarget{three-children-have-died-in-new-york-of-illness-linked-to-virus}{%
\section{Three Children Have Died in New York of Illness Linked to
Virus}\label{three-children-have-died-in-new-york-of-illness-linked-to-virus}}

Gov. Andrew M. Cuomo said New York was working with the C.D.C. to
investigate a mysterious illness linked to the coronavirus that causes
life-threatening inflammation in children.

\begin{itemize}
\item
  Published May 9, 2020Updated May 10, 2020
\item
  \begin{itemize}
  \item
  \item
  \item
  \item
  \item
  \end{itemize}
\end{itemize}

\hypertarget{heres-what-you-need-to-know}{%
\subsubsection{Here's what you need to
know:}\label{heres-what-you-need-to-know}}

\begin{itemize}
\tightlist
\item
  \protect\hyperlink{link-3f08228}{Three children have died of a
  mysterious syndrome linked to the coronavirus.}
\item
  \protect\hyperlink{link-6dde2e2b}{New cases and deaths continue to
  drop in New Jersey.}
\item
  \protect\hyperlink{link-1f0c70d2}{Upstate art museums are preparing a
  social-distanced return for visitors.}
\item
  \protect\hyperlink{link-da157b4}{Charity organizations face financial
  disaster amid pandemic, report finds.}
\item
  \protect\hyperlink{link-cf07561}{The M.T.A. will provide buses to
  protect the homeless from bad weather this weekend.}
\item
  \protect\hyperlink{link-7e6fe594}{These are the things that New
  Yorkers achingly miss.}
\item
  \protect\hyperlink{link-2783e401}{Tell us about the moments that have
  brought you hope, strength, humor and relief.}
\end{itemize}

Cases and deaths in New York State

0

5,000

10,000 cases

March

April

May

June

July

Aug.

New cases

7-day average

Total cases

432,523

Deaths

32,451

Includes confirmed and probable cases where available

See maps of the coronavirus outbreak in New York »

\includegraphics{https://static01.graylady3jvrrxbe.onion/images/2020/05/09/business/09video-cuomo/09video-cuomo-videoSixteenByNine3000.jpg}

\hypertarget{three-children-have-died-of-a-mysterious-syndrome-linked-to-the-coronavirus}{%
\subsection{Three children have died of a mysterious syndrome linked to
the
coronavirus.}\label{three-children-have-died-of-a-mysterious-syndrome-linked-to-the-coronavirus}}

Three young children have died in New York of a mysterious, toxic-shock
inflammation syndrome with links to the coronavirus, Gov. Andrew M.
Cuomo said on Saturday.

``The illness has taken the lives of three young New Yorkers,'' Mr.
Cuomo said during his daily briefing in Manhattan. ``This is new. This
is developing.''

As of Saturday, more than 73 children in New York have been sickened by
\href{https://www.nytimes3xbfgragh.onion/article/kawasaki-disease-coronavirus-children.html}{the
rare illness,} which has some similarities to
\href{https://www.cdc.gov/kawasaki/index.html}{Kawasaki disease} and was
publicly identified for the first time earlier this week.

Governor Cuomo said many of these children, some as young as toddlers,
did not show respiratory symptoms commonly associated with the
coronavirus when they were brought to area hospitals, but all of them
tested positive either for Covid-19, the disease caused by the
coronavirus, or for its antibodies.

``So it is still very much a situation that is developing, but it is a
serious situation,'' he added.

The state will be working with the New York Genome Center and
Rockefeller University to determine what is causing the illness, which
Governor Cuomo described on Saturday as ``truly disturbing.''

When the coronavirus pandemic began ravaging the New York area two
months ago, the state found solace in the initial evidence that children
would be largely unaffected. That sense of relief was shattered this
week when a 5-year-old died in New York City of the newly discovered
disease, which doctors described as a ``pediatric multisystem
inflammatory syndrome.'' The inflammation of the blood vessels, Mr.
Cuomo said, causes ``problems with their heart.''

Mr. Cuomo did not elaborate on the deaths of the two additional
children.

``We were laboring under the impression that young people were not
affected by Covid-19, and that was actually good news,'' Mr. Cuomo said.
``We still have a lot to learn about this virus.''

Mr. Cuomo has asked parents to be vigilant in looking for symptoms such
as prolonged fever, severe abdominal pain, change in skin color, racing
heart and chest pain.

Before the announcement of the deaths attributed to the new illness,
fewer than four children under age 10 had died of the virus in New York,
according to the
\href{https://covid19tracker.health.ny.gov/views/NYS-COVID19-Tracker/NYSDOHCOVID-19Tracker-Fatalities?\%3Aembed=yes\&\%3Atoolbar=no\&\%3Atabs=n}{most
recent breakdown from the state}. Mr. Cuomo said the state was working
with the Centers for Disease Control to determine if the confounding
illness had been affecting children infected with the virus before this
week.

``It is very possible that this has been going on for several weeks and
it hasn't been diagnosed as related to Covid,'' he said.

Mr. Cuomo also announced 226 more deaths due to the coronavirus, 10 more
than the number reported a day earlier.

``That number has been infuriatingly constant,'' he said. ``We would
like to see that number dropping at a faster rate that it is currently
dropping.''

Despite the setbacks, New York continued to make inroads in its fight
against the coronavirus, Mr. Cuomo said.

\hypertarget{latest-updates-the-coronavirus-outbreak}{%
\section{\texorpdfstring{\href{https://www.nytimes3xbfgragh.onion/2020/08/21/world/covid-19-coronavirus.html?action=click\&pgtype=Article\&state=default\&region=MAIN_CONTENT_1\&context=storylines_live_updates}{Latest
Updates: The Coronavirus
Outbreak}}{Latest Updates: The Coronavirus Outbreak}}\label{latest-updates-the-coronavirus-outbreak}}

Updated 2020-08-21T10:13:38.790Z

\begin{itemize}
\tightlist
\item
  \href{https://www.nytimes3xbfgragh.onion/2020/08/21/world/covid-19-coronavirus.html?action=click\&pgtype=Article\&state=default\&region=MAIN_CONTENT_1\&context=storylines_live_updates\#link-4690b6aa}{Shutdowns,
  warnings and scoldings follow gatherings on college campuses.}
\item
  \href{https://www.nytimes3xbfgragh.onion/2020/08/21/world/covid-19-coronavirus.html?action=click\&pgtype=Article\&state=default\&region=MAIN_CONTENT_1\&context=storylines_live_updates\#link-324af071}{As
  he accepts the Democratic nomination, Biden knocks Trump's pandemic
  response.}
\item
  \href{https://www.nytimes3xbfgragh.onion/2020/08/21/world/covid-19-coronavirus.html?action=click\&pgtype=Article\&state=default\&region=MAIN_CONTENT_1\&context=storylines_live_updates\#link-35890b73}{Hundreds
  of doctors in Kenya go on strike over their pay and protective gear.}
\end{itemize}

\href{https://www.nytimes3xbfgragh.onion/2020/08/21/world/covid-19-coronavirus.html?action=click\&pgtype=Article\&state=default\&region=MAIN_CONTENT_1\&context=storylines_live_updates}{See
more updates}

More live coverage:
\href{https://www.nytimes3xbfgragh.onion/live/2020/08/20/business/stock-market-today-coronavirus?action=click\&pgtype=Article\&state=default\&region=MAIN_CONTENT_1\&context=storylines_live_updates}{Markets}

New hospitalizations for Covid-19 patients remained relatively flat,
with 572 new patients being treated at city hospitals for the
coronavirus. On Friday, 604 people were hospitalized, and that number
hovered in the 600s this week.

\hypertarget{new-cases-and-deaths-continue-to-drop-in-new-jersey}{%
\subsection{New cases and deaths continue to drop in New
Jersey.}\label{new-cases-and-deaths-continue-to-drop-in-new-jersey}}

The number of new coronavirus cases and the number of people
hospitalized with the illness in New Jersey continued to drop, Gov.
Philip D. Murphy said Saturday.

Mr. Murphy reported 1,759 new cases, a drop of more than 200 from the
day before; that brought the total number of cases in the state to
137,085, as of Friday night, he said. He also announced 166 new deaths
in the state.

``Our battle here is not a battle to just bring down numbers,'' Mr.
Murphy said. ``It's a battle to save lives.''

The picture remained bleak at nursing homes. There have been more than
26,000 cases and 4,825 deaths, Mr. Murphy reported on Saturday,
accounting for more than half of the total number of deaths in the
state.

On Friday, the state reported the first death of a child under 18 years
old, but the governor said there was no evidence that the death of the
child was caused by the mysterious syndrome that has killed three
children in New York. Mr. Murphy said the child, who was 4, had an
underlying condition, but would not offer any more details because of
privacy concerns.

Mr. Murphy also announced there would be two ``convalescent plasma''
collection sites set up in the northern part of the state. Convalescent
plasma is the term used for plasma that is removed from the blood of a
person who has recovered from a disease, then transfused into a patient
still battling it.

An American Red Cross collection site will open in Fairfield and another
at University Hospital in Newark on Monday, May 11.

To donate plasma, a person must have recovered from the coronavirus and
be symptom free, officials said.

\hypertarget{upstate-art-museums-are-preparing-a-social-distanced-return-for-visitors}{%
\subsection{Upstate art museums are preparing a social-distanced return
for
visitors.}\label{upstate-art-museums-are-preparing-a-social-distanced-return-for-visitors}}

\includegraphics{https://static01.graylady3jvrrxbe.onion/images/2020/05/09/arts/08virus-rollout-briefing/merlin_172283457_50af5170-f7ca-4380-9c1a-dd87c8015c14-articleLarge.jpg?quality=75\&auto=webp\&disable=upscale}

About an hour's drive from New York City, the
\href{https://www.nytimes3xbfgragh.onion/2019/06/13/arts/design/dia-beacon.html}{Dia:Beacon}
art museum has been sitting empty for nearly two months. Mostly empty.

Landscapers have shown up to mulch the garden and a couple of staff
members have been fixing worn floorboards, all in preparation for some
elusive date when visitors will trickle back into the museum's bright,
airy rooms.

This uncertain future became a bit more conceivable this week when Gov.
Andrew M. Cuomo
\href{https://www.nytimes3xbfgragh.onion/2020/05/04/nyregion/coronavirus-reopen-cuomo-ny.html}{outlined
his phased plan} for reopening during the pandemic. The plan is to allow
upstate areas to transition back to normal life before the downstate
regions do, but only after they reach certain public health benchmarks.

New York has classified arts institutions in the fourth and final phase
of businesses that will be allowed to reopen, after restaurants, hotels
and retail stores.

Still, the directors of community theaters, museums and art centers in
the \href{https://esd.ny.gov/regions/mid-hudson}{Mid-Hudson region} and
beyond were relieved: As they had hoped, an institution like the
Herkimer County Historical Society, which typically hosts about five
visitors at a time in the summer, will be able to open up sooner than,
say, the Metropolitan Museum of Art.

But they realize it will mean working for several weeks to transform
their institutions so that visitors will feel safe.

``We're going to try to create a contact-free experience from the moment
a visitor steps onto our property,'' said Paul S. D'Ambrosio, the
president of the Fenimore Art Museum, a renovated 1930s Georgian Revival
mansion in Cooperstown, N.Y., which is among the regions that could open
soonest.

To put visitors at ease, Dia plans to institute a timed-ticket system to
limit the number of people in the building, and is installing hands-free
faucets in the restrooms. Upon the reopening, gallery attendants would
be tasked with regulating the number of people in each room.

\hypertarget{charity-organizations-face-financial-disaster-amid-pandemic-report-finds}{%
\subsection{Charity organizations face financial disaster amid pandemic,
report
finds.}\label{charity-organizations-face-financial-disaster-amid-pandemic-report-finds}}

Image

A community volunteer for City Harvest waits to distribute food at the
Melrose Mobile Market site in the Bronx.Credit...Desiree Rios for The
New York Times

As New York's stay-at-home order has all but decimated the city's
economy over the last few weeks, millions of vulnerable residents have
turned to charity organizations for shelter, food and other necessities.
But a new report suggests the aid many New Yorkers have come to rely on
during the pandemic may not be sustainable for very long.

\href{https://www.nytimes3xbfgragh.onion/news-event/coronavirus?action=click\&pgtype=Article\&state=default\&region=MAIN_CONTENT_3\&context=storylines_faq}{}

\hypertarget{the-coronavirus-outbreak-}{%
\subsubsection{The Coronavirus Outbreak
›}\label{the-coronavirus-outbreak-}}

\hypertarget{frequently-asked-questions}{%
\paragraph{Frequently Asked
Questions}\label{frequently-asked-questions}}

Updated August 17, 2020

\begin{itemize}
\item ~
  \hypertarget{why-does-standing-six-feet-away-from-others-help}{%
  \paragraph{Why does standing six feet away from others
  help?}\label{why-does-standing-six-feet-away-from-others-help}}

  \begin{itemize}
  \tightlist
  \item
    The coronavirus spreads primarily through droplets from your mouth
    and nose, especially when you cough or sneeze. The C.D.C., one of
    the organizations using that measure,
    \href{https://www.nytimes3xbfgragh.onion/2020/04/14/health/coronavirus-six-feet.html?action=click\&pgtype=Article\&state=default\&region=MAIN_CONTENT_3\&context=storylines_faq}{bases
    its recommendation of six feet} on the idea that most large droplets
    that people expel when they cough or sneeze will fall to the ground
    within six feet. But six feet has never been a magic number that
    guarantees complete protection. Sneezes, for instance, can launch
    droplets a lot farther than six feet,
    \href{https://jamanetwork.com/journals/jama/fullarticle/2763852}{according
    to a recent study}. It's a rule of thumb: You should be safest
    standing six feet apart outside, especially when it's windy. But
    keep a mask on at all times, even when you think you're far enough
    apart.
  \end{itemize}
\item ~
  \hypertarget{i-have-antibodies-am-i-now-immune}{%
  \paragraph{I have antibodies. Am I now
  immune?}\label{i-have-antibodies-am-i-now-immune}}

  \begin{itemize}
  \tightlist
  \item
    As of right
    now,\href{https://www.nytimes3xbfgragh.onion/2020/07/22/health/covid-antibodies-herd-immunity.html?action=click\&pgtype=Article\&state=default\&region=MAIN_CONTENT_3\&context=storylines_faq}{that
    seems likely, for at least several months.} There have been
    frightening accounts of people suffering what seems to be a second
    bout of Covid-19. But experts say these patients may have a
    drawn-out course of infection, with the virus taking a slow toll
    weeks to months after initial exposure. People infected with the
    coronavirus typically
    \href{https://www.nature.com/articles/s41586-020-2456-9}{produce}
    immune molecules called antibodies, which are
    \href{https://www.nytimes3xbfgragh.onion/2020/05/07/health/coronavirus-antibody-prevalence.html?action=click\&pgtype=Article\&state=default\&region=MAIN_CONTENT_3\&context=storylines_faq}{protective
    proteins made in response to an
    infection}\href{https://www.nytimes3xbfgragh.onion/2020/05/07/health/coronavirus-antibody-prevalence.html?action=click\&pgtype=Article\&state=default\&region=MAIN_CONTENT_3\&context=storylines_faq}{.
    These antibodies may} last in the body
    \href{https://www.nature.com/articles/s41591-020-0965-6}{only two to
    three months}, which may seem worrisome, but that's perfectly normal
    after an acute infection subsides, said Dr. Michael Mina, an
    immunologist at Harvard University. It may be possible to get the
    coronavirus again, but it's highly unlikely that it would be
    possible in a short window of time from initial infection or make
    people sicker the second time.
  \end{itemize}
\item ~
  \hypertarget{im-a-small-business-owner-can-i-get-relief}{%
  \paragraph{I'm a small-business owner. Can I get
  relief?}\label{im-a-small-business-owner-can-i-get-relief}}

  \begin{itemize}
  \tightlist
  \item
    The
    \href{https://www.nytimes3xbfgragh.onion/article/small-business-loans-stimulus-grants-freelancers-coronavirus.html?action=click\&pgtype=Article\&state=default\&region=MAIN_CONTENT_3\&context=storylines_faq}{stimulus
    bills enacted in March} offer help for the millions of American
    small businesses. Those eligible for aid are businesses and
    nonprofit organizations with fewer than 500 workers, including sole
    proprietorships, independent contractors and freelancers. Some
    larger companies in some industries are also eligible. The help
    being offered, which is being managed by the Small Business
    Administration, includes the Paycheck Protection Program and the
    Economic Injury Disaster Loan program. But lots of folks have
    \href{https://www.nytimes3xbfgragh.onion/interactive/2020/05/07/business/small-business-loans-coronavirus.html?action=click\&pgtype=Article\&state=default\&region=MAIN_CONTENT_3\&context=storylines_faq}{not
    yet seen payouts.} Even those who have received help are confused:
    The rules are draconian, and some are stuck sitting on
    \href{https://www.nytimes3xbfgragh.onion/2020/05/02/business/economy/loans-coronavirus-small-business.html?action=click\&pgtype=Article\&state=default\&region=MAIN_CONTENT_3\&context=storylines_faq}{money
    they don't know how to use.} Many small-business owners are getting
    less than they expected or
    \href{https://www.nytimes3xbfgragh.onion/2020/06/10/business/Small-business-loans-ppp.html?action=click\&pgtype=Article\&state=default\&region=MAIN_CONTENT_3\&context=storylines_faq}{not
    hearing anything at all.}
  \end{itemize}
\item ~
  \hypertarget{what-are-my-rights-if-i-am-worried-about-going-back-to-work}{%
  \paragraph{What are my rights if I am worried about going back to
  work?}\label{what-are-my-rights-if-i-am-worried-about-going-back-to-work}}

  \begin{itemize}
  \tightlist
  \item
    Employers have to provide
    \href{https://www.osha.gov/SLTC/covid-19/standards.html}{a safe
    workplace} with policies that protect everyone equally.
    \href{https://www.nytimes3xbfgragh.onion/article/coronavirus-money-unemployment.html?action=click\&pgtype=Article\&state=default\&region=MAIN_CONTENT_3\&context=storylines_faq}{And
    if one of your co-workers tests positive for the coronavirus, the
    C.D.C.} has said that
    \href{https://www.cdc.gov/coronavirus/2019-ncov/community/guidance-business-response.html}{employers
    should tell their employees} -\/- without giving you the sick
    employee's name -\/- that they may have been exposed to the virus.
  \end{itemize}
\item ~
  \hypertarget{what-is-school-going-to-look-like-in-september}{%
  \paragraph{What is school going to look like in
  September?}\label{what-is-school-going-to-look-like-in-september}}

  \begin{itemize}
  \tightlist
  \item
    It is unlikely that many schools will return to a normal schedule
    this fall, requiring the grind of
    \href{https://www.nytimes3xbfgragh.onion/2020/06/05/us/coronavirus-education-lost-learning.html?action=click\&pgtype=Article\&state=default\&region=MAIN_CONTENT_3\&context=storylines_faq}{online
    learning},
    \href{https://www.nytimes3xbfgragh.onion/2020/05/29/us/coronavirus-child-care-centers.html?action=click\&pgtype=Article\&state=default\&region=MAIN_CONTENT_3\&context=storylines_faq}{makeshift
    child care} and
    \href{https://www.nytimes3xbfgragh.onion/2020/06/03/business/economy/coronavirus-working-women.html?action=click\&pgtype=Article\&state=default\&region=MAIN_CONTENT_3\&context=storylines_faq}{stunted
    workdays} to continue. California's two largest public school
    districts --- Los Angeles and San Diego --- said on July 13, that
    \href{https://www.nytimes3xbfgragh.onion/2020/07/13/us/lausd-san-diego-school-reopening.html?action=click\&pgtype=Article\&state=default\&region=MAIN_CONTENT_3\&context=storylines_faq}{instruction
    will be remote-only in the fall}, citing concerns that surging
    coronavirus infections in their areas pose too dire a risk for
    students and teachers. Together, the two districts enroll some
    825,000 students. They are the largest in the country so far to
    abandon plans for even a partial physical return to classrooms when
    they reopen in August. For other districts, the solution won't be an
    all-or-nothing approach.
    \href{https://bioethics.jhu.edu/research-and-outreach/projects/eschool-initiative/school-policy-tracker/}{Many
    systems}, including the nation's largest, New York City, are
    devising
    \href{https://www.nytimes3xbfgragh.onion/2020/06/26/us/coronavirus-schools-reopen-fall.html?action=click\&pgtype=Article\&state=default\&region=MAIN_CONTENT_3\&context=storylines_faq}{hybrid
    plans} that involve spending some days in classrooms and other days
    online. There's no national policy on this yet, so check with your
    municipal school system regularly to see what is happening in your
    community.
  \end{itemize}
\end{itemize}

\href{https://nycfuture.org/research/essential-yet-vulnerable}{The
report, just released by the Center for an Urban Future, a research
institute}, concluded that many go-to charity organizations are facing a
crippling combination of increasing overhead costs and diminishing
revenues.

The report warns that many human service nonprofits, like The Catholic
Charities of the Archdiocese of New York and Good Shepherd Services, may
find it difficult to keep their doors open if city and state governments
don't commit to future funding. It cited multiple organizations that had
already predicted extra costs or revenue losses exceeding \$1 million.

The message from city hall and Albany regarding funding has been
ominous. Mr. de Blasio recently announced that the city will need to
make more than ``\$2 billion in very tough budget cuts'' to balance a
city budget battered by the health crisis. Gov. Andrew M. Cuomo has
publicly stated, ``We can't spend what we don't have.''

``When all the dust settles, it's the provider community that's going to
be holding the bill for having fully accommodated all of the decisions
both the city and state have made,'' Bill Baccaglini, president and CEO
of New York Foundling, a child welfare organization, said in the report.
``We're hoping that everybody, at the end of the day, makes us whole.''

The organizations' inability to hold spring and summer fund-raisers,
which bring in millions of dollars a year, is particularly compounding
to the problem, according to the report.

\hypertarget{the-mta-will-provide-buses-to-protect-the-homeless-from-bad-weather-this-weekend}{%
\subsection{The M.T.A. will provide buses to protect the homeless from
bad weather this
weekend.}\label{the-mta-will-provide-buses-to-protect-the-homeless-from-bad-weather-this-weekend}}

Image

An M.T.A. bus provided shelter to homeless people outside the Stillwell
Avenue station in Coney Island, Brooklyn, early Saturday
morning.Credit...Jonah Markowitz for The New York Times

With
\href{https://www.nytimes3xbfgragh.onion/2020/05/07/us/northeast-polar-vortex-snow.html}{cold
and rainy weather expected in the Northeast over the next few days}, New
York City's transit agency announced on Friday night that it would
provide stationary buses outside some end-of-the-line subway stations
this weekend as shelter for homeless people.

The M.T.A., which operates the city's subway and bus system,
\href{https://www.nytimes3xbfgragh.onion/2020/05/06/nyregion/nyc-subway-close-coronavirus.html}{began
shutting down the subway system overnight} on Wednesday, forcing those
who otherwise would have ridden throughout the night to accept shelter
offered by city employees or find their own.

The M.T.A. is providing 40 buses at 30 stations, and the vehicles will
be controlled by the Police Department after they are dropped off, the
transit agency said.

In a statement announcing the move, transit officials reiterated that
the M.T.A. is ``not a social services agency'' and stressed that the
buses were a short-term solution. They called on the city, which
requested the buses, ``to step up and take responsibility for providing
safe shelter for those individuals experiencing homelessness.''

\hypertarget{these-are-the-things-that-new-yorkers-achingly-miss}{%
\subsection{These are the things that New Yorkers achingly
miss.}\label{these-are-the-things-that-new-yorkers-achingly-miss}}

Image

Empty food vendors outside the Metropolitan Museum of Art this
week.Credit...Andrew Seng for The New York Times

To hop on the train, any train, earbuds intact, alone in the crowd on
the way somewhere else. To walk out of the Metropolitan Museum of Art,
exhausted as if from a march. The sweet-potato fries and a beer at Tubby
Hook Tavern in Inwood; the coffee-cart guy on West 40th Street who
remembers you take it black.

Sunday Mass and the bakery after. Seeing old friends in the synagogue.
Play dates. The High Line. Hugs.

\href{https://www.nytimes3xbfgragh.onion/2020/05/09/nyregion/new-yorkers-missing-nyc-coronavirus.html}{Ask
New Yorkers what they miss most, nearly two months into isolation}. To
hear their answers is to witness a perfect version of the city built
from the ground up, a place refracted through a lens of loss, where the
best parts are huge and the annoyances become all but invisible.

The cheap seats in the outfield, the shouting to be heard at happy hour.
Meeting cousins with a soccer ball in Brooklyn Bridge Park. The din of
the theater as you scan the Playbill before the lights go down.

``I miss my gym equipment,'' said Barbara James of Brooklyn.

``The lamb over rice from the food cart by my office, at Seventh and
49th,'' said Chris Meredith of East Harlem.

``Just everything,'' sighed a police officer sitting behind the wheel of
his vehicle in Williamsburg, Brooklyn, last week. ``I miss everything.''

\hypertarget{tell-us-about-the-moments-that-have-brought-you-hope-strength-humor-and-relief}{%
\subsection{Tell us about the moments that have brought you hope,
strength, humor and
relief.}\label{tell-us-about-the-moments-that-have-brought-you-hope-strength-humor-and-relief}}

The coronavirus outbreak has brought much of life in New York to a halt
and there is no clear end in sight. But there are also moments that
offer a sliver of strength, hope, humor or some other type of relief: a
joke from a stranger on line at the supermarket; a favor from a friend
down the block; a great meal ordered from a restaurant we want to
survive; trivia night via Zoom with the bar down the street.

We'd like to hear about your moments, the ones that are helping you
through these dark times. A reporter or editor may contact you. Your
information will not be published without your consent.

Reporting was contributed by Maria Cramer, Michael Gold, Julia Jacobs,
Andy Newman, Sarah Maslin Nir, Joel Petterson, Andrea Salcedo, Edgar
Sandoval, Matt Stevens and Michael Wilson.

Advertisement

\protect\hyperlink{after-bottom}{Continue reading the main story}

\hypertarget{site-index}{%
\subsection{Site Index}\label{site-index}}

\hypertarget{site-information-navigation}{%
\subsection{Site Information
Navigation}\label{site-information-navigation}}

\begin{itemize}
\tightlist
\item
  \href{https://help.nytimes3xbfgragh.onion/hc/en-us/articles/115014792127-Copyright-notice}{©~2020~The
  New York Times Company}
\end{itemize}

\begin{itemize}
\tightlist
\item
  \href{https://www.nytco.com/}{NYTCo}
\item
  \href{https://help.nytimes3xbfgragh.onion/hc/en-us/articles/115015385887-Contact-Us}{Contact
  Us}
\item
  \href{https://www.nytco.com/careers/}{Work with us}
\item
  \href{https://nytmediakit.com/}{Advertise}
\item
  \href{http://www.tbrandstudio.com/}{T Brand Studio}
\item
  \href{https://www.nytimes3xbfgragh.onion/privacy/cookie-policy\#how-do-i-manage-trackers}{Your
  Ad Choices}
\item
  \href{https://www.nytimes3xbfgragh.onion/privacy}{Privacy}
\item
  \href{https://help.nytimes3xbfgragh.onion/hc/en-us/articles/115014893428-Terms-of-service}{Terms
  of Service}
\item
  \href{https://help.nytimes3xbfgragh.onion/hc/en-us/articles/115014893968-Terms-of-sale}{Terms
  of Sale}
\item
  \href{https://spiderbites.nytimes3xbfgragh.onion}{Site Map}
\item
  \href{https://help.nytimes3xbfgragh.onion/hc/en-us}{Help}
\item
  \href{https://www.nytimes3xbfgragh.onion/subscription?campaignId=37WXW}{Subscriptions}
\end{itemize}
