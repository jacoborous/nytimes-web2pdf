Sections

SEARCH

\protect\hyperlink{site-content}{Skip to
content}\protect\hyperlink{site-index}{Skip to site index}

\href{https://myaccount.nytimes3xbfgragh.onion/auth/login?response_type=cookie\&client_id=vi}{}

\href{https://www.nytimes3xbfgragh.onion/section/todayspaper}{Today's
Paper}

\href{/section/opinion}{Opinion}\textbar{}The First Invasion of America

\url{https://nyti.ms/2zaAJir}

\begin{itemize}
\item
\item
\item
\item
\item
\item
\end{itemize}

Advertisement

\protect\hyperlink{after-top}{Continue reading the main story}

\href{/section/opinion}{Opinion}

Supported by

\protect\hyperlink{after-sponsor}{Continue reading the main story}

\hypertarget{the-first-invasion-of-america}{%
\section{The First Invasion of
America}\label{the-first-invasion-of-america}}

And the cultural earthquake it's unleashing.

\href{https://www.nytimes3xbfgragh.onion/by/david-brooks}{\includegraphics{https://static01.graylady3jvrrxbe.onion/images/2018/04/03/opinion/david-brooks/david-brooks-thumbLarge-v2.png}}

By \href{https://www.nytimes3xbfgragh.onion/by/david-brooks}{David
Brooks}

Opinion Columnist

\begin{itemize}
\item
  May 21, 2020
\item
  \begin{itemize}
  \item
  \item
  \item
  \item
  \item
  \item
  \end{itemize}
\end{itemize}

\includegraphics{https://static01.graylady3jvrrxbe.onion/images/2020/05/22/opinion/22brooks_Sub/merlin_163978530_a1596a81-d8c3-4c4a-bd8b-b6339d71799a-articleLarge.jpg?quality=75\&auto=webp\&disable=upscale}

I was an American history major in college, back in the 1980s.

I'll be honest with you. I thrilled to the way the American story was
told back then. To immigrate to America was to join the luckiest and
greatest nation in history. ``Nothing in all history had ever succeeded
like America, and every American knew it,'' Henry Steele Commager wrote
in his 1950 book, ``The American Mind.''

To be born American was to be born to a glorious destiny. We were the
nation of the future, the vanguard of justice, the last best hope of
mankind. ``Have the elder races halted?'' Walt Whitman asked, ``Do they
droop and end their lesson, wearied over there beyond the seas? We take
up the task eternal.''

To be born American was to be born boldly individual, daring and
self-sufficient. ``Trust thyself: Every heart vibrates to that iron
string,'' Ralph Waldo Emerson wrote in an essay called, very Americanly,
``Self-Reliance.''

To be born American was to bow down to no one, to say: \emph{I'm no
better than anyone else, but nobody's better than me.} Tocqueville wrote
about the equality of condition he found in America; no one putting on
airs over anyone else. In 1981, Samuel Huntington wrote that American
creed was built around a suspicion of authority and a fervent rejection
of hierarchy: ``The essence of egalitarianism is rejection of the idea
that one person has the right to exercise power over another.''

I found it all so energizing. Being an American was not just a
citizenship. It was a vocation, a call to serve a grand national
mission.

Today, of course, we understand what was wrong with that version of
American history. It didn't include everybody. It left out the full
horrors of slavery and genocide.

But here's what has struck me forcefully, especially during the
pandemic: That whole version of the American creed was all based on an
assumption of existential security. Americans had the luxury of thinking
and living the way they did because they had two whopping great oceans
on either side. The United States was immune to foreign invasion, the
corruptions of the old world. It was often spared the plagues that swept
over so many other parts of the globe.

We could be individualistic, anti-authority, daring and self-sufficient
because on an elemental level we felt so damn safe.

University of Maryland scholar Michele Gelfand has spent her career
comparing national cultures. Some nations grow up relatively spared from
foreign invasion and the frequent devastation of infectious disease.
Gelfand finds that these are loose nations: individualistic, creative
but also disordered, uncoordinated and reckless.

Other nations have not been so lucky. Harsh necessity has made them
tight nations. Hardship has taught them to pull together, to be more
conformist, but also better at building social order and self-control.

Gelfand wrote a book called ``Rule Makers, Rule Breakers.'' We Americans
have been rule-breakers, the classic loose nation.

But what happens to a loose nation when the sense of existential
security disappears? Over the first two decades of the 21st century,
America has lost its sense of safety, the calm confidence that the
future is ours, that our institutions are sound or even minimally
competent.

And if there was any shred of existential safety left, surely the
pandemic has taken it away --- around 100,000 dead so far, an economy
ravaged. We've had threats before, a few foreign incursions like in
1812, even pandemics when America was less just than it is today. But
we've never had them smack in the middle of a crisis of confidence, a
crisis of authority, plus social and spiritual crises all at once.

So in that sense, this is the first invasion of America. This is the
first time that a menace has crossed our borders, upended the daily
lives of every American and rocked our ancient sense of safety. Welcome
to life in the rest of the world.

Aside from a few protesters and a depraved president, most of us have
understood we need to suspend the old individualistic American creed. In
the midst of a complex epidemiological disaster, to be anti-authority is
to be ignorant. In the midst of a contagion, to act as if you are
self-sufficient is just selfish.

But something more profound is going on. We are undergoing a more
permanent shift in national consciousness, a reconstruction of meanings,
symbols, values and narratives. If the old American creed grew up in an
atmosphere of assumed security and liberty, the new one is growing up in
an atmosphere of vulnerability and precariousness.

In this atmosphere,
\href{https://www.nytimes3xbfgragh.onion/2020/04/20/opinion/marco-rubio-coronavirus-economy.html}{economic
resilience will be more valued than maximized efficiency}. We'll spend
more time minimizing downside risks than maximizing upside gains. The
local and the rooted will be valued more than the distantly networked.
We'll value community over individualism, embeddedness over autonomy.

Something lovely is being lost. America's old idea of itself unleashed a
torrent of energy. But the American identity that grows up in the shadow
of the plague can have the humanity of shared vulnerability, the
humility that comes with an understanding of the precariousness of life
and a fierce solidarity that emerges during a long struggle against an
invading force.

\emph{The Times is committed to publishing}
\href{https://www.nytimes3xbfgragh.onion/2019/01/31/opinion/letters/letters-to-editor-new-york-times-women.html}{\emph{a
diversity of letters}} \emph{to the editor. We'd like to hear what you
think about this or any of our articles. Here are some}
\href{https://help.nytimes3xbfgragh.onion/hc/en-us/articles/115014925288-How-to-submit-a-letter-to-the-editor}{\emph{tips}}\emph{.
And here's our email:}
\href{mailto:letters@NYTimes.com}{\emph{letters@NYTimes.com}}\emph{.}

\emph{Follow The New York Times Opinion section on}
\href{https://www.facebookcorewwwi.onion/nytopinion}{\emph{Facebook}}\emph{,}
\href{http://twitter.com/NYTOpinion}{\emph{Twitter (@NYTopinion)}}
\emph{and}
\href{https://www.instagram.com/nytopinion/}{\emph{Instagram}}\emph{.}

Advertisement

\protect\hyperlink{after-bottom}{Continue reading the main story}

\hypertarget{site-index}{%
\subsection{Site Index}\label{site-index}}

\hypertarget{site-information-navigation}{%
\subsection{Site Information
Navigation}\label{site-information-navigation}}

\begin{itemize}
\tightlist
\item
  \href{https://help.nytimes3xbfgragh.onion/hc/en-us/articles/115014792127-Copyright-notice}{©~2020~The
  New York Times Company}
\end{itemize}

\begin{itemize}
\tightlist
\item
  \href{https://www.nytco.com/}{NYTCo}
\item
  \href{https://help.nytimes3xbfgragh.onion/hc/en-us/articles/115015385887-Contact-Us}{Contact
  Us}
\item
  \href{https://www.nytco.com/careers/}{Work with us}
\item
  \href{https://nytmediakit.com/}{Advertise}
\item
  \href{http://www.tbrandstudio.com/}{T Brand Studio}
\item
  \href{https://www.nytimes3xbfgragh.onion/privacy/cookie-policy\#how-do-i-manage-trackers}{Your
  Ad Choices}
\item
  \href{https://www.nytimes3xbfgragh.onion/privacy}{Privacy}
\item
  \href{https://help.nytimes3xbfgragh.onion/hc/en-us/articles/115014893428-Terms-of-service}{Terms
  of Service}
\item
  \href{https://help.nytimes3xbfgragh.onion/hc/en-us/articles/115014893968-Terms-of-sale}{Terms
  of Sale}
\item
  \href{https://spiderbites.nytimes3xbfgragh.onion}{Site Map}
\item
  \href{https://help.nytimes3xbfgragh.onion/hc/en-us}{Help}
\item
  \href{https://www.nytimes3xbfgragh.onion/subscription?campaignId=37WXW}{Subscriptions}
\end{itemize}
