Sections

SEARCH

\protect\hyperlink{site-content}{Skip to
content}\protect\hyperlink{site-index}{Skip to site index}

\href{https://www.nytimes3xbfgragh.onion/section/us}{U.S.}

\href{https://myaccount.nytimes3xbfgragh.onion/auth/login?response_type=cookie\&client_id=vi}{}

\href{https://www.nytimes3xbfgragh.onion/section/todayspaper}{Today's
Paper}

\href{/section/us}{U.S.}\textbar{}An `Avalanche of Evictions' Could Be
Bearing Down on America's Renters

\url{https://nyti.ms/3d7b2xU}

\begin{itemize}
\item
\item
\item
\item
\item
\item
\end{itemize}

\hypertarget{the-coronavirus-outbreak}{%
\subsubsection{\texorpdfstring{\href{https://www.nytimes3xbfgragh.onion/news-event/coronavirus?name=styln-coronavirus-national\&region=TOP_BANNER\&variant=undefined\&block=storyline_menu_recirc\&action=click\&pgtype=Article\&impression_id=acc5c330-e397-11ea-922b-e97d4ae673ca}{The
Coronavirus
Outbreak}}{The Coronavirus Outbreak}}\label{the-coronavirus-outbreak}}

\begin{itemize}
\tightlist
\item
  live\href{https://www.nytimes3xbfgragh.onion/2020/08/21/world/covid-19-coronavirus.html?name=styln-coronavirus-national\&region=TOP_BANNER\&variant=undefined\&block=storyline_menu_recirc\&action=click\&pgtype=Article\&impression_id=acc5c331-e397-11ea-922b-e97d4ae673ca}{Latest
  Updates}
\item
  \href{https://www.nytimes3xbfgragh.onion/interactive/2020/us/coronavirus-us-cases.html?name=styln-coronavirus-national\&region=TOP_BANNER\&variant=undefined\&block=storyline_menu_recirc\&action=click\&pgtype=Article\&impression_id=acc5c332-e397-11ea-922b-e97d4ae673ca}{Maps
  and Cases}
\item
  \href{https://www.nytimes3xbfgragh.onion/interactive/2020/science/coronavirus-vaccine-tracker.html?name=styln-coronavirus-national\&region=TOP_BANNER\&variant=undefined\&block=storyline_menu_recirc\&action=click\&pgtype=Article\&impression_id=acc5c333-e397-11ea-922b-e97d4ae673ca}{Vaccine
  Tracker}
\item
  \href{https://www.nytimes3xbfgragh.onion/2020/08/19/us/colleges-closing-covid.html?name=styln-coronavirus-national\&region=TOP_BANNER\&variant=undefined\&block=storyline_menu_recirc\&action=click\&pgtype=Article\&impression_id=acc5ea40-e397-11ea-922b-e97d4ae673ca}{Colleges
  Closing}
\item
  \href{https://www.nytimes3xbfgragh.onion/live/2020/08/20/business/stock-market-today-coronavirus?name=styln-coronavirus-national\&region=TOP_BANNER\&variant=undefined\&block=storyline_menu_recirc\&action=click\&pgtype=Article\&impression_id=acc5ea41-e397-11ea-922b-e97d4ae673ca}{Economy}
\end{itemize}

Advertisement

\protect\hyperlink{after-top}{Continue reading the main story}

Supported by

\protect\hyperlink{after-sponsor}{Continue reading the main story}

\hypertarget{an-avalanche-of-evictions-could-be-bearing-down-on-americas-renters}{%
\section{An `Avalanche of Evictions' Could Be Bearing Down on America's
Renters}\label{an-avalanche-of-evictions-could-be-bearing-down-on-americas-renters}}

The economic downturn is shaping up to be particularly devastating for
renters, who are more likely to be lower-income and work hourly jobs cut
during the pandemic.

\includegraphics{https://static01.graylady3jvrrxbe.onion/images/2020/05/19/us/00virus-evictions1/merlin_172646622_562afea8-f3e7-4c57-b816-a16df04fc5da-articleLarge.jpg?quality=75\&auto=webp\&disable=upscale}

\href{https://www.nytimes3xbfgragh.onion/by/sarah-mervosh}{\includegraphics{https://static01.graylady3jvrrxbe.onion/images/2018/07/18/multimedia/author-sarah-mervosh/author-sarah-mervosh-thumbLarge-v3.png}}

By \href{https://www.nytimes3xbfgragh.onion/by/sarah-mervosh}{Sarah
Mervosh}

\begin{itemize}
\item
  Published May 27, 2020Updated May 30, 2020
\item
  \begin{itemize}
  \item
  \item
  \item
  \item
  \item
  \item
  \end{itemize}
\end{itemize}

EUCLID, Ohio --- The United States, already wrestling with an economic
collapse not seen in a generation, is facing a wave of
\href{https://www.nytimes3xbfgragh.onion/2020/07/23/business/evictions-moratorium-cares-act.html}{evictions}
as government relief payments and legal protections run out for millions
of out-of-work Americans who have little financial cushion and few
choices when looking for new housing.

The hardest hit are tenants who had low incomes and little savings even
before the pandemic, and whose housing costs ate up more of their
paychecks. They were also more likely to work in industries where job
losses have been particularly severe.

Temporary government assistance has helped, as have government orders
that put evictions on hold in many cities. But evictions will soon be
allowed in about half of the states, according to Emily A. Benfer, a
housing expert and associate professor at Columbia Law School who is
tracking eviction policies.

``I think we will enter into a severe renter crisis and very quickly,''
Professor Benfer said. Without a new round of government intervention,
she added, ``we will have an avalanche of evictions across the
country.''

That means more and more families may soon experience the dreaded
eviction notice on the front door, the stomach-turning knock from
sheriff's deputies, the possessions piled up on the sidewalk. They will
face displacement at a time when people are still being urged to
\href{https://www.nytimes3xbfgragh.onion/interactive/2020/us/states-reopen-map-coronavirus.html}{stay
at home} to keep themselves and their communities safe, with the death
toll from the virus now
\href{https://www.nytimes3xbfgragh.onion/2020/05/27/us/coronavirus-live-updates.html?}{having
passed 100,000 in the United States}.

That fear of eviction has been eating away at Sandy Naffah ever since
she lost her income as the virus led to economic shutdowns. Ms. Naffah,
who had been juggling two part-time jobs --- teaching elementary school
students how to read and working as a beauty consultant at a mall ---
quickly fell behind on the \$800 she pays in rent each month for a
one-bedroom apartment in Euclid, Ohio, a suburb of Cleveland.

She is now staring down a precarious future, desperately hoping that a
one-off federal stimulus check and unemployment benefits --- both of
which she said she had yet to receive --- will keep her afloat and stave
off eviction.

``It's a ticking clock,'' said Ms. Naffah, who is in her 50s. ``I can't
continue to go on this way, otherwise I will be out on the street.''

\hypertarget{latest-updates-the-coronavirus-outbreak}{%
\section{\texorpdfstring{\href{https://www.nytimes3xbfgragh.onion/2020/08/21/world/covid-19-coronavirus.html?action=click\&pgtype=Article\&state=default\&region=MAIN_CONTENT_1\&context=storylines_live_updates}{Latest
Updates: The Coronavirus
Outbreak}}{Latest Updates: The Coronavirus Outbreak}}\label{latest-updates-the-coronavirus-outbreak}}

Updated 2020-08-21T10:13:38.790Z

\begin{itemize}
\tightlist
\item
  \href{https://www.nytimes3xbfgragh.onion/2020/08/21/world/covid-19-coronavirus.html?action=click\&pgtype=Article\&state=default\&region=MAIN_CONTENT_1\&context=storylines_live_updates\#link-4690b6aa}{Shutdowns,
  warnings and scoldings follow gatherings on college campuses.}
\item
  \href{https://www.nytimes3xbfgragh.onion/2020/08/21/world/covid-19-coronavirus.html?action=click\&pgtype=Article\&state=default\&region=MAIN_CONTENT_1\&context=storylines_live_updates\#link-324af071}{As
  he accepts the Democratic nomination, Biden knocks Trump's pandemic
  response.}
\item
  \href{https://www.nytimes3xbfgragh.onion/2020/08/21/world/covid-19-coronavirus.html?action=click\&pgtype=Article\&state=default\&region=MAIN_CONTENT_1\&context=storylines_live_updates\#link-35890b73}{Hundreds
  of doctors in Kenya go on strike over their pay and protective gear.}
\end{itemize}

\href{https://www.nytimes3xbfgragh.onion/2020/08/21/world/covid-19-coronavirus.html?action=click\&pgtype=Article\&state=default\&region=MAIN_CONTENT_1\&context=storylines_live_updates}{See
more updates}

More live coverage:
\href{https://www.nytimes3xbfgragh.onion/live/2020/08/20/business/stock-market-today-coronavirus?action=click\&pgtype=Article\&state=default\&region=MAIN_CONTENT_1\&context=storylines_live_updates}{Markets}

In many places, the threat has already begun. The Texas Supreme Court
recently ruled that
\href{https://www.texastribune.org/2020/05/14/texas-evictions-debt-collections-resume-may-moratoriums-lifted/}{evictions
could begin} again in the nation's second-largest state. In the Oklahoma
City area, sheriffs
\href{https://twitter.com/OkCountySheriff/status/1262825685036535809?s=20}{apologetically
announced} that they planned to start enforcing eviction notices this
week. And a handful of states, like Ohio, had few statewide protections
in place to begin with, leaving residents particularly vulnerable as
eviction cases stacked up or ticked forward during the pandemic.

\includegraphics{https://static01.graylady3jvrrxbe.onion/images/2020/05/19/us/00virus-evictions2/merlin_172646736_d99f1e09-7583-4892-a418-baa1547bccd5-articleLarge.jpg?quality=75\&auto=webp\&disable=upscale}

Christie Wilson, 37, was among them. After fleeing a dangerous
relationship, she said, she spent several months sleeping in her car
last year before a veterans program helped her pay for a two-bedroom
apartment in Decatur, Ga. She had recently become responsible for the
\$1,143-a-month rent herself, she said, and had lined up a job at a
warehouse.

But after two days on the job, she said, she was laid off as the
coronavirus outbreak intensified in March.

A few weeks later, she found an eviction notice on her door. She now
fears losing her apartment, where, in the fragile stability of recent
months, she has enjoyed small luxuries, like listening to gospel music
on her patio in the mornings and spending Mother's Day in her own home
with her teenage son.

The real estate company managing her apartment said that it had followed
protocol in filing for eviction, and that employees were working with
Ms. Wilson to waive fees and help connect her to nonprofit groups. If
she has to move out, she worries she would end up in a homeless shelter,
where preliminary testing has shown
\href{https://www.wbur.org/commonhealth/2020/05/15/boston-homeless-coronavirus-testing}{high
rates of infection}.

``There would be no six-feet distance --- we'd be sleeping on top of
each other,'' said Ms. Wilson, who is racing to pay back more than
\$2,000 in back rent before Georgia courts reopen next month.

Though about 90 percent of renters made full or partial rent payments by
late May,
\href{https://www.nmhc.org/research-insight/nmhc-rent-payment-tracker/}{down
only 2 percent} from last year, lawyers and landlords alike fear that
the trend will not last.
\href{https://www.nytimes3xbfgragh.onion/2020/05/21/business/economy/coronavirus-unemployment-claims.html?action=click\&module=RelatedLinks\&pgtype=Article}{More
than 38 million people} have filed jobless claims since March, including
a high proportion of people living in households making less than
\$40,000 a year. In a
\href{https://www.census.gov/data-tools/demo/hhp/\#/table?measures=HIR}{survey
released this month by the Census Bureau}, nearly a quarter of
respondents said they missed their last rent or mortgage payment or had
little to no confidence that they would be able to pay on time next
month.

The devastation has drawn comparisons to the Great Recession, when
millions of people lost their homes during a foreclosure crisis. But
this time, renters are likely to be on the front lines.

``We sort of expect this to be more of a renter crisis than a
homeownership crisis,'' said Elora Lee Raymond, an assistant professor
at the Georgia Institute of Technology who focuses on affordable housing
and real estate.

Even before the current joblessness crisis, eviction was troublingly
common in American life. Researchers estimate that about 3.7 million
eviction cases were filed in 2016, a year when the unemployment rate was
4.7 percent.

\href{https://www.nytimes3xbfgragh.onion/news-event/coronavirus?action=click\&pgtype=Article\&state=default\&region=MAIN_CONTENT_3\&context=storylines_faq}{}

\hypertarget{the-coronavirus-outbreak-}{%
\subsubsection{The Coronavirus Outbreak
›}\label{the-coronavirus-outbreak-}}

\hypertarget{frequently-asked-questions}{%
\paragraph{Frequently Asked
Questions}\label{frequently-asked-questions}}

Updated August 17, 2020

\begin{itemize}
\item ~
  \hypertarget{why-does-standing-six-feet-away-from-others-help}{%
  \paragraph{Why does standing six feet away from others
  help?}\label{why-does-standing-six-feet-away-from-others-help}}

  \begin{itemize}
  \tightlist
  \item
    The coronavirus spreads primarily through droplets from your mouth
    and nose, especially when you cough or sneeze. The C.D.C., one of
    the organizations using that measure,
    \href{https://www.nytimes3xbfgragh.onion/2020/04/14/health/coronavirus-six-feet.html?action=click\&pgtype=Article\&state=default\&region=MAIN_CONTENT_3\&context=storylines_faq}{bases
    its recommendation of six feet} on the idea that most large droplets
    that people expel when they cough or sneeze will fall to the ground
    within six feet. But six feet has never been a magic number that
    guarantees complete protection. Sneezes, for instance, can launch
    droplets a lot farther than six feet,
    \href{https://jamanetwork.com/journals/jama/fullarticle/2763852}{according
    to a recent study}. It's a rule of thumb: You should be safest
    standing six feet apart outside, especially when it's windy. But
    keep a mask on at all times, even when you think you're far enough
    apart.
  \end{itemize}
\item ~
  \hypertarget{i-have-antibodies-am-i-now-immune}{%
  \paragraph{I have antibodies. Am I now
  immune?}\label{i-have-antibodies-am-i-now-immune}}

  \begin{itemize}
  \tightlist
  \item
    As of right
    now,\href{https://www.nytimes3xbfgragh.onion/2020/07/22/health/covid-antibodies-herd-immunity.html?action=click\&pgtype=Article\&state=default\&region=MAIN_CONTENT_3\&context=storylines_faq}{that
    seems likely, for at least several months.} There have been
    frightening accounts of people suffering what seems to be a second
    bout of Covid-19. But experts say these patients may have a
    drawn-out course of infection, with the virus taking a slow toll
    weeks to months after initial exposure. People infected with the
    coronavirus typically
    \href{https://www.nature.com/articles/s41586-020-2456-9}{produce}
    immune molecules called antibodies, which are
    \href{https://www.nytimes3xbfgragh.onion/2020/05/07/health/coronavirus-antibody-prevalence.html?action=click\&pgtype=Article\&state=default\&region=MAIN_CONTENT_3\&context=storylines_faq}{protective
    proteins made in response to an
    infection}\href{https://www.nytimes3xbfgragh.onion/2020/05/07/health/coronavirus-antibody-prevalence.html?action=click\&pgtype=Article\&state=default\&region=MAIN_CONTENT_3\&context=storylines_faq}{.
    These antibodies may} last in the body
    \href{https://www.nature.com/articles/s41591-020-0965-6}{only two to
    three months}, which may seem worrisome, but that's perfectly normal
    after an acute infection subsides, said Dr. Michael Mina, an
    immunologist at Harvard University. It may be possible to get the
    coronavirus again, but it's highly unlikely that it would be
    possible in a short window of time from initial infection or make
    people sicker the second time.
  \end{itemize}
\item ~
  \hypertarget{im-a-small-business-owner-can-i-get-relief}{%
  \paragraph{I'm a small-business owner. Can I get
  relief?}\label{im-a-small-business-owner-can-i-get-relief}}

  \begin{itemize}
  \tightlist
  \item
    The
    \href{https://www.nytimes3xbfgragh.onion/article/small-business-loans-stimulus-grants-freelancers-coronavirus.html?action=click\&pgtype=Article\&state=default\&region=MAIN_CONTENT_3\&context=storylines_faq}{stimulus
    bills enacted in March} offer help for the millions of American
    small businesses. Those eligible for aid are businesses and
    nonprofit organizations with fewer than 500 workers, including sole
    proprietorships, independent contractors and freelancers. Some
    larger companies in some industries are also eligible. The help
    being offered, which is being managed by the Small Business
    Administration, includes the Paycheck Protection Program and the
    Economic Injury Disaster Loan program. But lots of folks have
    \href{https://www.nytimes3xbfgragh.onion/interactive/2020/05/07/business/small-business-loans-coronavirus.html?action=click\&pgtype=Article\&state=default\&region=MAIN_CONTENT_3\&context=storylines_faq}{not
    yet seen payouts.} Even those who have received help are confused:
    The rules are draconian, and some are stuck sitting on
    \href{https://www.nytimes3xbfgragh.onion/2020/05/02/business/economy/loans-coronavirus-small-business.html?action=click\&pgtype=Article\&state=default\&region=MAIN_CONTENT_3\&context=storylines_faq}{money
    they don't know how to use.} Many small-business owners are getting
    less than they expected or
    \href{https://www.nytimes3xbfgragh.onion/2020/06/10/business/Small-business-loans-ppp.html?action=click\&pgtype=Article\&state=default\&region=MAIN_CONTENT_3\&context=storylines_faq}{not
    hearing anything at all.}
  \end{itemize}
\item ~
  \hypertarget{what-are-my-rights-if-i-am-worried-about-going-back-to-work}{%
  \paragraph{What are my rights if I am worried about going back to
  work?}\label{what-are-my-rights-if-i-am-worried-about-going-back-to-work}}

  \begin{itemize}
  \tightlist
  \item
    Employers have to provide
    \href{https://www.osha.gov/SLTC/covid-19/standards.html}{a safe
    workplace} with policies that protect everyone equally.
    \href{https://www.nytimes3xbfgragh.onion/article/coronavirus-money-unemployment.html?action=click\&pgtype=Article\&state=default\&region=MAIN_CONTENT_3\&context=storylines_faq}{And
    if one of your co-workers tests positive for the coronavirus, the
    C.D.C.} has said that
    \href{https://www.cdc.gov/coronavirus/2019-ncov/community/guidance-business-response.html}{employers
    should tell their employees} -\/- without giving you the sick
    employee's name -\/- that they may have been exposed to the virus.
  \end{itemize}
\item ~
  \hypertarget{what-is-school-going-to-look-like-in-september}{%
  \paragraph{What is school going to look like in
  September?}\label{what-is-school-going-to-look-like-in-september}}

  \begin{itemize}
  \tightlist
  \item
    It is unlikely that many schools will return to a normal schedule
    this fall, requiring the grind of
    \href{https://www.nytimes3xbfgragh.onion/2020/06/05/us/coronavirus-education-lost-learning.html?action=click\&pgtype=Article\&state=default\&region=MAIN_CONTENT_3\&context=storylines_faq}{online
    learning},
    \href{https://www.nytimes3xbfgragh.onion/2020/05/29/us/coronavirus-child-care-centers.html?action=click\&pgtype=Article\&state=default\&region=MAIN_CONTENT_3\&context=storylines_faq}{makeshift
    child care} and
    \href{https://www.nytimes3xbfgragh.onion/2020/06/03/business/economy/coronavirus-working-women.html?action=click\&pgtype=Article\&state=default\&region=MAIN_CONTENT_3\&context=storylines_faq}{stunted
    workdays} to continue. California's two largest public school
    districts --- Los Angeles and San Diego --- said on July 13, that
    \href{https://www.nytimes3xbfgragh.onion/2020/07/13/us/lausd-san-diego-school-reopening.html?action=click\&pgtype=Article\&state=default\&region=MAIN_CONTENT_3\&context=storylines_faq}{instruction
    will be remote-only in the fall}, citing concerns that surging
    coronavirus infections in their areas pose too dire a risk for
    students and teachers. Together, the two districts enroll some
    825,000 students. They are the largest in the country so far to
    abandon plans for even a partial physical return to classrooms when
    they reopen in August. For other districts, the solution won't be an
    all-or-nothing approach.
    \href{https://bioethics.jhu.edu/research-and-outreach/projects/eschool-initiative/school-policy-tracker/}{Many
    systems}, including the nation's largest, New York City, are
    devising
    \href{https://www.nytimes3xbfgragh.onion/2020/06/26/us/coronavirus-schools-reopen-fall.html?action=click\&pgtype=Article\&state=default\&region=MAIN_CONTENT_3\&context=storylines_faq}{hybrid
    plans} that involve spending some days in classrooms and other days
    online. There's no national policy on this yet, so check with your
    municipal school system regularly to see what is happening in your
    community.
  \end{itemize}
\end{itemize}

``Now we have 14.7 percent,'' said Matthew Desmond, a sociologist at
Princeton and the author of the book ``Evicted,'' who is leading an
effort at the university's Eviction Lab to track cases nationally.
Without intervention, he said, ``I don't see how we wouldn't have a wave
of evictions.''

A
\href{https://www.nytimes3xbfgragh.onion/2020/05/15/us/politics/house-simulus-vote.html}{\$3
trillion coronavirus relief bill} backed by House Democrats includes a
proposal to dedicate \$100 billion for rental assistance, a measure that
could bring broad relief, but Republicans have criticized the package as
too costly, and it is unlikely to pass in its current form.

And some argue that the federal government has already intervened
effectively, in the form of the stimulus checks and a
\href{https://www.nytimes3xbfgragh.onion/interactive/2020/04/23/business/economy/unemployment-benefits-stimulus-coronavirus.html}{\$600
weekly boost to unemployment} payments.

Many low-wage workers are making more money on unemployment than they
were when they were working, said Ken Rosen, an economist at the
University of California, Berkeley. ``It's happening, not through the
housing system, but through the unemployment compensation system,'' he
said.

But there is a looming question about what happens next. ``People may be
paying their rents, but at what cost?'' said Tara Raghuveer, the
director of KC Tenants, an advocacy group in Kansas City, Mo. ``I know
several people who are taking out title loans. They are paying their
rent on their credit card.''

Image

Emma Witbolsfeugen stood on the shoulder of I-70 in Independence, Mo.,
to rally for rent forgiveness.Credit...Christopher Smith for The New
York Times

Many landlords say they are working with their tenants, waiving late
fees and advocating that the government cover missed rent. ``We are in
uncharted waters,'' said Tom Bannon, chief executive of the California
Apartment Association, who added that most landlords were not eager to
evict residents when there was little guarantee of a replacement.

Still, landlords have bills to pay, too. When tenants cannot pay their
rent, landlords with mortgages remain responsible to the banks, who
answer to investors. ``I call it the responsibility chain,'' Mr. Bannon
said. ``There is this link, and if there is a break in the link, the
ripple effect is pretty significant.''

Among the first to face eviction have been those who were already
struggling before the pandemic.

Stephen Jenkins, 64, was let go from his assembly job in January, making
it difficult to pay his \$900 monthly rent in Springfield, Ohio. By
March, he said, his savings had run out, and he asked his landlord if he
could pay late after his Social Security check came through.

His landlord, who declined to comment, filed for eviction.

In the weeks since, Mr. Jenkins said, his wife lost her hostess job at
Bob Evans when restaurants shut down. They have not been able to move
out as few realtors are showing homes because of the virus.

The stress is giving him health problems, and he is anxiously counting
down the days until his eviction hearing, now scheduled for Wednesday.

``I haven't slept through a night since March,'' he said. ``I wake up at
three or four in the morning worried about what's going to happen
tomorrow.''

Advertisement

\protect\hyperlink{after-bottom}{Continue reading the main story}

\hypertarget{site-index}{%
\subsection{Site Index}\label{site-index}}

\hypertarget{site-information-navigation}{%
\subsection{Site Information
Navigation}\label{site-information-navigation}}

\begin{itemize}
\tightlist
\item
  \href{https://help.nytimes3xbfgragh.onion/hc/en-us/articles/115014792127-Copyright-notice}{©~2020~The
  New York Times Company}
\end{itemize}

\begin{itemize}
\tightlist
\item
  \href{https://www.nytco.com/}{NYTCo}
\item
  \href{https://help.nytimes3xbfgragh.onion/hc/en-us/articles/115015385887-Contact-Us}{Contact
  Us}
\item
  \href{https://www.nytco.com/careers/}{Work with us}
\item
  \href{https://nytmediakit.com/}{Advertise}
\item
  \href{http://www.tbrandstudio.com/}{T Brand Studio}
\item
  \href{https://www.nytimes3xbfgragh.onion/privacy/cookie-policy\#how-do-i-manage-trackers}{Your
  Ad Choices}
\item
  \href{https://www.nytimes3xbfgragh.onion/privacy}{Privacy}
\item
  \href{https://help.nytimes3xbfgragh.onion/hc/en-us/articles/115014893428-Terms-of-service}{Terms
  of Service}
\item
  \href{https://help.nytimes3xbfgragh.onion/hc/en-us/articles/115014893968-Terms-of-sale}{Terms
  of Sale}
\item
  \href{https://spiderbites.nytimes3xbfgragh.onion}{Site Map}
\item
  \href{https://help.nytimes3xbfgragh.onion/hc/en-us}{Help}
\item
  \href{https://www.nytimes3xbfgragh.onion/subscription?campaignId=37WXW}{Subscriptions}
\end{itemize}
