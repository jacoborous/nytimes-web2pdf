Sections

SEARCH

\protect\hyperlink{site-content}{Skip to
content}\protect\hyperlink{site-index}{Skip to site index}

\href{https://myaccount.nytimes3xbfgragh.onion/auth/login?response_type=cookie\&client_id=vi}{}

\href{https://www.nytimes3xbfgragh.onion/section/todayspaper}{Today's
Paper}

\href{/section/opinion}{Opinion}\textbar{}The Michigan Dam Failures Are
a Warning

\begin{itemize}
\item
\item
\item
\item
\item
\end{itemize}

Advertisement

\protect\hyperlink{after-top}{Continue reading the main story}

\href{/section/opinion}{Opinion}

Supported by

\protect\hyperlink{after-sponsor}{Continue reading the main story}

\hypertarget{the-michigan-dam-failures-are-a-warning}{%
\section{The Michigan Dam Failures Are a
Warning}\label{the-michigan-dam-failures-are-a-warning}}

Many need repairs. Let's fix them before climate-related flooding gets
worse.

By Upmanu Lall and Paulina Concha Larrauri

Dr. Lall is director of the Columbia Water Center at Columbia
University, where Ms. Concha Larrauri is a researcher.

\begin{itemize}
\item
  May 27, 2020
\item
  \begin{itemize}
  \item
  \item
  \item
  \item
  \item
  \end{itemize}
\end{itemize}

\includegraphics{https://static01.graylady3jvrrxbe.onion/images/2020/05/28/opinion/27lall/27lall-articleLarge.jpg?quality=75\&auto=webp\&disable=upscale}

Two dams down, a few thousand more to go.

Luckily, no one died last week when rain-swollen
\href{https://www.nytimes3xbfgragh.onion/interactive/2020/06/29/climate/hidden-flood-risk-maps.html}{flooding}
breached two dams in Central Michigan. But thousands were evacuated,
homes and businesses were inundated, and
\href{https://www.nytimes3xbfgragh.onion/reuters/2020/05/21/us/21reuters-usa-flood-michigan.html}{floodwaters
spilled} into a chemical plant and Superfund site.

Appropriately, President Trump signed an emergency declaration. But once
again, there has been little serious discussion since about what to do
to address the looming national hazard of aging dams like those that
failed in Michigan.

In November 2019, The Associated Press
\href{https://apnews.com/f5f09a300d394900a1a88362238dbf77}{reported}
that 19 dams in Michigan, including the first of the dams to breach,
were in unsatisfactory condition and presented high hazards, meaning
their failure can cause loss of life. The events of last week should not
have come as a surprise, and it is only a matter of time before a
catastrophic dam collapse will occur somewhere in the United States. The
combination of aging and poorly maintained dams and extreme,
climate-caused flooding presents potentially deadly risks for people
downstream.

We won't be able to say we weren't warned. The federal government
offered a stark message in its
\href{https://nca2018.globalchange.gov/chapter/3/}{national climate
assessment} in 2018, cautioning that aging and deteriorating dams and
levees ``represent an increasing hazard when exposed to extreme or, in
some cases, even moderate rainfall.'' The report noted that heavy
rainfalls led to widespread dam or levee failures in 2005, 2015, 2016
and 2017. ``The national exposure to this risk,'' the report said, ``has
not yet been fully assessed.''

But here is what we do know. A majority of the roughly 90,000 dams in
the United States are older than their nominal design life of 50 years,
the point when they become increasingly more difficult and expensive to
keep safe, assuming they've been properly maintained in the first place.
The \href{https://nid.sec.usace.army.mil/ords/f?p=105:1::::::}{National
Inventory of Dams} includes about 25,000 dams considered high or
significant hazards if they failed.

We recently wrote a
\href{http://water.columbia.edu/files/2020/05/GRI_Report.pdf}{report}
assessing the risks of climate-induced dam failures. We found that much
critical infrastructure --- other dams, electricity-generating plants,
highways, bridges, toxic Superfund sites, water treatment and wastewater
treatment plants --- lie in a path of potential destruction below aging
dams. The Michigan dam failures are an example of a cascading failure
--- the breach of the Edenville Dam, which was rated by the state in
2018
\href{https://www.nytimes3xbfgragh.onion/aponline/2020/05/20/us/ap-us-midwest-flooding.html}{as
being in unsatisfactory condition}, led to floodwaters overflowing the
downstream Sanford Dam, which was rated as being in only fair condition.

A presidential disaster declaration certainly makes sense after a dam
fails. But wouldn't it be better to prioritize which ones to fix or
remove before disaster strikes? We recommend an approach that assesses
the potential for climate extremes, the probability of a dam failure,
and resulting direct and indirect financial losses. This would enable
regulators to expeditiously screen and identify the subset of dams that
need urgent attention and investment.

There is no doubt that climate change is increasing the frequency of
extreme rainfalls and the risk of floodwaters overtopping dams, the main
reason a dam fails. But while climate change may not be so easily
fixable, making sure dams can withstand flooding is, and it is much
cheaper than the consequences.

The Oroville Dam in the foothills of California's Sierra Nevada
illustrates the point. In 2005, three environmental groups urged the
repair of an emergency spillway on the dam, which at 770 feet is
\href{https://www.worldatlas.com/articles/tallest-dams-in-the-united-states.html}{the
tallest} in the United States. At the time, the work would have cost
roughly
\href{https://www.washingtonpost.com/news/post-nation/wp/2017/02/13/officials-were-warned-the-oroville-dam-emergency-spillway-wasnt-safe-they-didnt-listen/}{\$100
million}, according to one of those groups. In 2017, the spillway
failed. Some 185,000 people were evacuated downstream because of the
potential of catastrophic flooding. The cost of repairs following the
near catastrophe was
\href{https://www.latimes.com/local/california/la-me-oroville-cost-20180905-story.html}{\$1.1
billion}.

In Michigan, Gov. Gretchen Whitmer said experts characterized the
flooding that led to the recent dam failures as a 500-year event ---
something that would have a one in 500 chance of occurring in any given
year. If we consider dams in the eight-state Great Lakes region older
than 60 years (most have a design life of 50 years) that are in counties
with a population larger than 500,000, 317 dams are classified as having
a high potential for hazard in a failure. The chances of one or more of
these dams experiencing a 500- or 1,000-year flooding event in a year
would be 47 percent and 27 percent --- which strikes us as pretty high.

The Great Lakes region exhibits approximately 10-year cycles of rainfall
and is currently near
\href{https://www.glerl.noaa.gov/data/wlevels/levels.html\#observations}{record
high levels}. Extreme rainfalls are happening
\href{https://michigan-weather-center.org/weather-extremes-pt-2}{much
more frequently} in the region than in the past 100 years. What is being
done to prepare for potential flooding and dam failures?

The state and the federal government have multiple offices that assess
dam safety. What we lack is an overall strategy to fix the problem and
the requisite financial resources. Rehabilitating dams with high hazard
potential \href{https://fas.org/sgp/crs/homesec/IF10606.pdf}{will cost}
an estimated \$3 billion for federal impoundments and another \$19
billion for nonfederal ones --- a cost that vastly exceeds current
spending.

We need a real plan and real money, and we need them soon. The
coronavirus pandemic, which we are spending billions to battle, should
at least remind us that a little bit of prevention can avert an enormous
amount of anguish.

\href{http://www.columbia.edu/~ula2/}{Upmanu Lall} is the chairman of
the department of earth and environmental engineering and the director
of the Columbia Water Center at Columbia University.
\href{https://www.earth.columbia.edu/users/profile/paulina-concha-larrauri}{Paulina
Concha Larrauri} is a researcher at the water center.

\emph{The Times is committed to publishing}
\href{https://www.nytimes3xbfgragh.onion/2019/01/31/opinion/letters/letters-to-editor-new-york-times-women.html}{\emph{a
diversity of letters}} \emph{to the editor. We'd like to hear what you
think about this or any of our articles. Here are some}
\href{https://help.nytimes3xbfgragh.onion/hc/en-us/articles/115014925288-How-to-submit-a-letter-to-the-editor}{\emph{tips}}\emph{.
And here's our email:}
\href{mailto:letters@NYTimes.com}{\emph{letters@NYTimes.com}}\emph{.}

\emph{Follow The New York Times Opinion section on}
\href{https://www.facebookcorewwwi.onion/nytopinion}{\emph{Facebook}}\emph{,}
\href{http://twitter.com/NYTOpinion}{\emph{Twitter (@NYTopinion)}}
\emph{and}
\href{https://www.instagram.com/nytopinion/}{\emph{Instagram}}\emph{.}

Advertisement

\protect\hyperlink{after-bottom}{Continue reading the main story}

\hypertarget{site-index}{%
\subsection{Site Index}\label{site-index}}

\hypertarget{site-information-navigation}{%
\subsection{Site Information
Navigation}\label{site-information-navigation}}

\begin{itemize}
\tightlist
\item
  \href{https://help.nytimes3xbfgragh.onion/hc/en-us/articles/115014792127-Copyright-notice}{©~2020~The
  New York Times Company}
\end{itemize}

\begin{itemize}
\tightlist
\item
  \href{https://www.nytco.com/}{NYTCo}
\item
  \href{https://help.nytimes3xbfgragh.onion/hc/en-us/articles/115015385887-Contact-Us}{Contact
  Us}
\item
  \href{https://www.nytco.com/careers/}{Work with us}
\item
  \href{https://nytmediakit.com/}{Advertise}
\item
  \href{http://www.tbrandstudio.com/}{T Brand Studio}
\item
  \href{https://www.nytimes3xbfgragh.onion/privacy/cookie-policy\#how-do-i-manage-trackers}{Your
  Ad Choices}
\item
  \href{https://www.nytimes3xbfgragh.onion/privacy}{Privacy}
\item
  \href{https://help.nytimes3xbfgragh.onion/hc/en-us/articles/115014893428-Terms-of-service}{Terms
  of Service}
\item
  \href{https://help.nytimes3xbfgragh.onion/hc/en-us/articles/115014893968-Terms-of-sale}{Terms
  of Sale}
\item
  \href{https://spiderbites.nytimes3xbfgragh.onion}{Site Map}
\item
  \href{https://help.nytimes3xbfgragh.onion/hc/en-us}{Help}
\item
  \href{https://www.nytimes3xbfgragh.onion/subscription?campaignId=37WXW}{Subscriptions}
\end{itemize}
