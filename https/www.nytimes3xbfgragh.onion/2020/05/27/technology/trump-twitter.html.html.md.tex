Sections

SEARCH

\protect\hyperlink{site-content}{Skip to
content}\protect\hyperlink{site-index}{Skip to site index}

\href{https://www.nytimes3xbfgragh.onion/section/technology}{Technology}

\href{https://myaccount.nytimes3xbfgragh.onion/auth/login?response_type=cookie\&client_id=vi}{}

\href{https://www.nytimes3xbfgragh.onion/section/todayspaper}{Today's
Paper}

\href{/section/technology}{Technology}\textbar{}Twitter Comes Under
Attack From Trump's Supporters

\url{https://nyti.ms/2TGVTeU}

\begin{itemize}
\item
\item
\item
\item
\item
\end{itemize}

Advertisement

\protect\hyperlink{after-top}{Continue reading the main story}

Supported by

\protect\hyperlink{after-sponsor}{Continue reading the main story}

\hypertarget{twitter-comes-under-attack-from-trumps-supporters}{%
\section{Twitter Comes Under Attack From Trump's
Supporters}\label{twitter-comes-under-attack-from-trumps-supporters}}

After the social media company labeled two of the president's tweets as
inaccurate on Tuesday, his adherents pounced.

\includegraphics{https://static01.graylady3jvrrxbe.onion/images/2020/05/27/business/27twitter1/merlin_161161425_6221ae46-90e5-4209-ad8c-732266056eb0-articleLarge.jpg?quality=75\&auto=webp\&disable=upscale}

By \href{https://www.nytimes3xbfgragh.onion/by/kate-conger}{Kate Conger}
and \href{https://www.nytimes3xbfgragh.onion/by/davey-alba}{Davey Alba}

\begin{itemize}
\item
  Published May 27, 2020Updated May 29, 2020
\item
  \begin{itemize}
  \item
  \item
  \item
  \item
  \item
  \end{itemize}
\end{itemize}

OAKLAND, Calif. --- Not long after
\href{https://www.nytimes3xbfgragh.onion/2020/05/28/us/politics/trump-order-social-media.html}{Twitter}
added a warning label to two of
\href{https://www.nytimes3xbfgragh.onion/2020/05/28/us/politics/trump-order-social-media.html}{President
Trump's} tweets on Tuesday, his supporters swung into action.

On Twitter, Mr. Trump's adherents targeted one of the company's
executives for old tweets in which he had criticized the president and
other Republicans. On Capitol Hill, lawmakers including Senator Marco
Rubio, Republican of Florida, and Senator Josh Hawley, Republican of
Missouri, said they would move to regulate
\href{https://www.nytimes3xbfgragh.onion/2020/05/29/technology/trump-twitter.html}{Twitter}.

In right-wing media, pundits such as Trish Regan and websites like the
Gateway Pundit decried the decision and accused Twitter of bias. The
furor quickly spread through dozens of Facebook groups, Reddit forums
and YouTube videos.

The activity was payback for Twitter, Mr. Trump's favorite social media
platform, after the company took action on the president's tweets for
the first time. While for years
\href{https://www.nytimes3xbfgragh.onion/2020/05/29/technology/trump-twitter.html}{Twitter}
had been hands-off on Mr. Trump's posts, which have often included
falsehoods and threats, it added fact-checking labels to two of the
president's messages related to mail-in ballots on Tuesday to signal
that they were inaccurate.

That sparked a vitriolic reaction from Mr. Trump, who said on Twitter
that the company was interfering with the presidential election and
stifling free speech. His supporters --- a mixture of mainstream
Republicans, far-right personalities and online acolytes --- then
quickly turned to a well-worn playbook of vilifying those whom they saw
as slighting him.

In recent months, the
\href{https://www.nytimes3xbfgragh.onion/2020/03/28/technology/coronavirus-fauci-trump-conspiracy-target.html}{targets
of their ire have included Dr. Anthony Fauci}, the infectious disease
expert who has
\href{https://www.nytimes3xbfgragh.onion/2020/03/23/us/politics/coronavirus-trump-fauci.html}{corrected
overly rosy} pronouncements about
\href{https://www.nytimes3xbfgragh.onion/interactive/2020/world/coronavirus-maps.html}{the
coronavirus}, and
\href{https://www.nytimes3xbfgragh.onion/2020/04/17/technology/bill-gates-virus-conspiracy-theories.html}{Bill
Gates, the tech billionaire-turned-philanthropist} who has indirectly
criticized the Trump administration's handling of the pandemic.

But this time, the right-wing machinery training its sights on a
publicly traded company --- Twitter --- and its roughly 5,000 employees,
took on an added menace, disinformation researchers said.

The tactics are something that Mr.
\href{https://www.nytimes3xbfgragh.onion/2020/06/21/us/politics/trump-rally-supporters.html}{Trump's
supporters} ``return to again and again,'' said Melissa Ryan, chief
executive of Card Strategies, a consulting firm that researches
disinformation. ``Where it gets worrisome for the tech companies is, of
course, the Trump administration has the power to make their life very
difficult.''

On Wednesday, Mr. Trump continued his tirade. In two tweets, he accused
social media companies of working to ``totally silence conservatives
voices.'' He added, ``We will strongly regulate, or close them down,
before we can ever allow this to happen.''

The right-wing backlash against Twitter built even as some researchers
questioned how effective the labels on Mr. Trump's tweets would be. Some
said they were unlikely to sway public opinion about the reliability of
Mr. Trump's statements; others criticized the ``get the facts'' language
that Twitter had added to the posts as vague. Several pointed out that
Twitter had not gone as far as removing the posts.

Studies have found that fact-checking false claims on social media can
help readers, but that
\href{https://www.dartmouth.edu/~nyhan/fake-news-solutions.pdf}{specific
labels like ``disputed'' or ``rated false''} were more effective in
raising public understanding.

Even labeling a claim ``false'' on social media reduced its perceived
accuracy by only about 13 percentage points, said Katie Clayton, a
researcher who worked on a 2019 study at Dartmouth College that examined
fact-check labels on news headlines on Facebook.

Still, she said, Twitter's actions were ``a step in the right
direction.''

A Twitter spokesman said the online harassment that one of its
executives was experiencing was ``disappointing.'' The San Francisco
company, whose employees curated its fact-checks, added that it would
continue labeling tweets that contained misinformation about elections
or coronavirus. It said it might expand those policies to include labels
for misinformation about additional topics.

In total, the backlash against Twitter has spread to more than 100
Facebook group and pages, thousands of tweets and several Reddit forums
in which Mr. Trump's followers have claimed that Twitter suppresses
conservative speech, according to a New York Times analysis. In those
online threads, Trump supporters said Twitter employees were biased
liberals and urged Mr. Trump to fast-track regulations to limit the
company.

``I wonder if Jack Dorsey grew up dreaming: `one day I will connect the
world with an easy to use app, giving every individual a voice, then I
will censor, throttle, and ban the people with whom I disagree,''' wrote
one Twitter user, referring to the social network's chief executive.

Fans of Mr. Trump also rapidly turned on Yoel Roth, a Twitter executive
who combats bots, election interference and fake accounts. The campaign
against Mr. Roth started late Tuesday when Liz Wheeler, a TV host on One
America News Network, a cable network that has championed the Trump
administration's agenda,
\href{https://twitter.com/Liz_Wheeler/status/1265463081997484032}{unearthed
and reposted tweets} in which Mr. Roth had referred to Mr. Trump as a
``racist tangerine.'' Others added a tweet from Mr. Roth that called
Senate majority leader Mitch McConnell, a Republican from Kentucky, a
``bag of farts.''

Far-right media outlets like
\href{https://www.breitbart.com/tech/2020/05/26/twitter-fact-checker-claimed-trump-actual-nazis-mocked-flyover-states/}{Breitbart}
and
\href{https://www.thegatewaypundit.com/2020/05/report-twitters-head-site-integrity-responsible-election-security-misinformation-says-nazis-white-house/}{The
Gateway Pundit} immediately seized on the messages as proof that Twitter
was biased against conservatives. By Wednesday morning, Mr. Roth's old
tweets had reached the White House. On Fox News, Kellyanne Conway
referenced him by name and called on supporters to ``wake him up.''

``I think I want to raise the name of somebody at Twitter,'' Ms. Conway
said in the interview, spelling out Mr. Roth's Twitter handle so that
viewers could find his social media profile. ``Somebody in San
Francisco, go wake him up and tell him he's about to get a lot more
followers.''

Mentions of Mr. Roth on Twitter spiked to 180 mentions in a five-minute
span on Tuesday evening, and to 478 mentions on Wednesday morning,
according to The Times analysis. Mr. Roth declined to comment.

The decision to fact-check Mr. Trump was not Mr. Roth's, a Twitter
spokesman said. It was instead made by executives focused on legal and
policy issues after Mr. Trump's tweets were reported to the company
through a portal used by election-related nonprofits and those who
administer elections in states.

In Washington, the confrontation between Mr. Trump and Twitter
reinvigorated calls among Republican lawmakers to change
\href{https://www.nytimes3xbfgragh.onion/2020/02/04/technology/section-230-lobby.html}{Section
230} of the Communications Decency Act, which protects technology
companies from most liability for content posted by their users.

``The law still protects social media companies like Twitter because
they are considered forums not publishers,'' Mr. Rubio
\href{https://twitter.com/marcorubio/status/1265442093641732096}{tweeted}on
Tuesday. ``But if they have now decided to exercise an editorial role
like a publisher then they should no longer be shielded from liability
\& treated as publishers under the law.''

In a letter on Wednesday to Mr. Dorsey, Mr. Hawley said Twitter enjoyed
a
``\href{https://twitter.com/HawleyMO/status/1265444047365312515}{special
immunity worth billions}'' and called on lawmakers to put an end to the
``sweetheart deal.''

Ms. Ryan, the disinformation researcher, said Twitter is in ``uncharted
territory.''

``You can predict pretty easily how Trump is going to respond,'' she
said. But with Twitter, ``once the company enforces a policy, is it
going to succumb to the blowback? Or is it going to stay the course? I
think that's the important thing to watch moving forward.''

Kate Conger reported from Oakland, Calif., and Davey Alba from New York.
Ben Decker contributed reporting.

Advertisement

\protect\hyperlink{after-bottom}{Continue reading the main story}

\hypertarget{site-index}{%
\subsection{Site Index}\label{site-index}}

\hypertarget{site-information-navigation}{%
\subsection{Site Information
Navigation}\label{site-information-navigation}}

\begin{itemize}
\tightlist
\item
  \href{https://help.nytimes3xbfgragh.onion/hc/en-us/articles/115014792127-Copyright-notice}{©~2020~The
  New York Times Company}
\end{itemize}

\begin{itemize}
\tightlist
\item
  \href{https://www.nytco.com/}{NYTCo}
\item
  \href{https://help.nytimes3xbfgragh.onion/hc/en-us/articles/115015385887-Contact-Us}{Contact
  Us}
\item
  \href{https://www.nytco.com/careers/}{Work with us}
\item
  \href{https://nytmediakit.com/}{Advertise}
\item
  \href{http://www.tbrandstudio.com/}{T Brand Studio}
\item
  \href{https://www.nytimes3xbfgragh.onion/privacy/cookie-policy\#how-do-i-manage-trackers}{Your
  Ad Choices}
\item
  \href{https://www.nytimes3xbfgragh.onion/privacy}{Privacy}
\item
  \href{https://help.nytimes3xbfgragh.onion/hc/en-us/articles/115014893428-Terms-of-service}{Terms
  of Service}
\item
  \href{https://help.nytimes3xbfgragh.onion/hc/en-us/articles/115014893968-Terms-of-sale}{Terms
  of Sale}
\item
  \href{https://spiderbites.nytimes3xbfgragh.onion}{Site Map}
\item
  \href{https://help.nytimes3xbfgragh.onion/hc/en-us}{Help}
\item
  \href{https://www.nytimes3xbfgragh.onion/subscription?campaignId=37WXW}{Subscriptions}
\end{itemize}
