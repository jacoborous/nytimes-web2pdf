Sections

SEARCH

\protect\hyperlink{site-content}{Skip to
content}\protect\hyperlink{site-index}{Skip to site index}

\href{https://www.nytimes3xbfgragh.onion/section/politics}{Politics}

\href{https://myaccount.nytimes3xbfgragh.onion/auth/login?response_type=cookie\&client_id=vi}{}

\href{https://www.nytimes3xbfgragh.onion/section/todayspaper}{Today's
Paper}

\href{/section/politics}{Politics}\textbar{}As Trump Pushes to Reopen,
Government Sees Virus Toll Nearly Doubling

\url{https://nyti.ms/3c5O8X0}

\begin{itemize}
\item
\item
\item
\item
\item
\end{itemize}

\href{https://www.nytimes3xbfgragh.onion/news-event/coronavirus?action=click\&pgtype=Article\&state=default\&region=TOP_BANNER\&context=storylines_menu}{The
Coronavirus Outbreak}

\begin{itemize}
\tightlist
\item
  live\href{https://www.nytimes3xbfgragh.onion/2020/08/04/world/coronavirus-cases.html?action=click\&pgtype=Article\&state=default\&region=TOP_BANNER\&context=storylines_menu}{Latest
  Updates}
\item
  \href{https://www.nytimes3xbfgragh.onion/interactive/2020/us/coronavirus-us-cases.html?action=click\&pgtype=Article\&state=default\&region=TOP_BANNER\&context=storylines_menu}{Maps
  and Cases}
\item
  \href{https://www.nytimes3xbfgragh.onion/interactive/2020/science/coronavirus-vaccine-tracker.html?action=click\&pgtype=Article\&state=default\&region=TOP_BANNER\&context=storylines_menu}{Vaccine
  Tracker}
\item
  \href{https://www.nytimes3xbfgragh.onion/2020/08/02/us/covid-college-reopening.html?action=click\&pgtype=Article\&state=default\&region=TOP_BANNER\&context=storylines_menu}{College
  Reopening}
\item
  \href{https://www.nytimes3xbfgragh.onion/live/2020/08/04/business/stock-market-today-coronavirus?action=click\&pgtype=Article\&state=default\&region=TOP_BANNER\&context=storylines_menu}{Economy}
\end{itemize}

Advertisement

\protect\hyperlink{after-top}{Continue reading the main story}

Supported by

\protect\hyperlink{after-sponsor}{Continue reading the main story}

\hypertarget{as-trump-pushes-to-reopen-government-sees-virus-toll-nearly-doubling}{%
\section{As Trump Pushes to Reopen, Government Sees Virus Toll Nearly
Doubling}\label{as-trump-pushes-to-reopen-government-sees-virus-toll-nearly-doubling}}

An internal Trump administration model projects a near-doubling of daily
coronavirus deaths by June 1 as the nation begins to reopen, as well as
a rapid rise in daily infections.

\includegraphics{https://static01.graylady3jvrrxbe.onion/images/2020/05/04/us/politics/04dc-virus-deaths1/merlin_172026258_71f4d54f-baf6-4968-a447-31d88fc5546e-articleLarge.jpg?quality=75\&auto=webp\&disable=upscale}

\href{https://www.nytimes3xbfgragh.onion/by/sheryl-gay-stolberg}{\includegraphics{https://static01.graylady3jvrrxbe.onion/images/2018/11/26/multimedia/author-sheryl-gay-stolberg/author-sheryl-gay-stolberg-thumbLarge.png}}\href{https://www.nytimes3xbfgragh.onion/by/eileen-sullivan}{\includegraphics{https://static01.graylady3jvrrxbe.onion/images/2019/12/13/reader-center/author-eileen-sullivan/author-eileen-sullivan-thumbLarge.png}}

By
\href{https://www.nytimes3xbfgragh.onion/by/sheryl-gay-stolberg}{Sheryl
Gay Stolberg} and
\href{https://www.nytimes3xbfgragh.onion/by/eileen-sullivan}{Eileen
Sullivan}

\begin{itemize}
\item
  May 4, 2020
\item
  \begin{itemize}
  \item
  \item
  \item
  \item
  \item
  \end{itemize}
\end{itemize}

WASHINGTON --- As President Trump presses states to reopen their
economies, his administration is privately projecting a steady rise in
coronavirus infections and deaths over the next several weeks, reaching
about 3,000 daily deaths on June 1 --- nearly double the current level.

The projections, based on data collected by various agencies, including
the Centers for Disease Control and Prevention, and laid out in
\href{https://int.graylady3jvrrxbe.onion/data/documenthelper/6926-mayhhsbriefing/af7319f4a55fd0ce5dc9/optimized/full.pdf\#page=1}{an
internal document obtained Monday by The New York Times}, forecast about
200,000 new cases each day by the end of May, up from about 30,000 cases
now. There are currently about 1,750 deaths per day, the data shows.

They are not the only ones forecasting more carnage.
\href{https://covid19.healthdata.org/united-states-of-america}{Another
model,} closely watched by the White House, raised its fatality
projections on Monday to more than 134,000 American deaths from
Covid-19, the disease caused by the coronavirus, by early August. The
Institute for Health Metrics and Evaluation at the University of
Washington more than doubled its
\href{http://www.healthdata.org/news-release/ihme-hold-media-briefing-4-pm-eastern-today-details-below}{previous
projection} of about 60,000 total deaths, an increase that it said
partly reflects ``changes in mobility and social distancing policies.''

The numbers underscore a sobering reality: While the United States has
been hunkered down for the past seven weeks, the prognosis has not
markedly improved. As states reopen --- many without meeting White House
guidelines that call for a steady decline in coronavirus cases or in the
number of people testing positive over a 14-day period --- the cost of
the shift is likely to be tallied in funerals.

``There remains a large number of counties whose burden continues to
grow,'' the C.D.C. warned, alongside a map that offered a detailed view
of the growth of the pandemic.

The projections amplify the primary fear of public health experts: that
a reopening of the economy will put the nation right back where it was
in mid-March, when cases were rising so rapidly in some parts of the
country that patients were dying on gurneys in hospital hallways amid
overloaded health systems.

Under the White House's reopening plan, called
\href{https://www.whitehouse.gov/openingamerica/}{``Opening Up America
Again,''} states considering relaxing stay-at-home policies are supposed
to show a ``downward trajectory'' either in the number of new infections
or positive tests as a percent of total tests over 14 days, and a
``robust testing program'' for at-risk health care workers.

But some of the states moving the quickest are not honoring all of those
guidelines.

In fact, the Trump administration has steered clear of enacting a
national policy to prevent its own projections from coming to pass. On a
conference call with the nation's governors on Monday, Vice President
Mike Pence cheered on state-level coronavirus testing, and he again
promised this week to ship out more tests to all 50 states.

But a recording of the call, obtained by The Times, made clear that the
White House was taking its cues from state governments. Mr. Pence's
upbeat assessment also included some public relations advice for the
governors.

\hypertarget{latest-updates-global-coronavirus-outbreak}{%
\section{\texorpdfstring{\href{https://www.nytimes3xbfgragh.onion/2020/08/04/world/coronavirus-cases.html?action=click\&pgtype=Article\&state=default\&region=MAIN_CONTENT_1\&context=storylines_live_updates}{Latest
Updates: Global Coronavirus
Outbreak}}{Latest Updates: Global Coronavirus Outbreak}}\label{latest-updates-global-coronavirus-outbreak}}

Updated 2020-08-04T21:57:55.984Z

\begin{itemize}
\tightlist
\item
  \href{https://www.nytimes3xbfgragh.onion/2020/08/04/world/coronavirus-cases.html?action=click\&pgtype=Article\&state=default\&region=MAIN_CONTENT_1\&context=storylines_live_updates\#link-2daa96b5}{As
  talks drag on, McConnell signals openness to jobless aid extension
  that Republicans have opposed.}
\item
  \href{https://www.nytimes3xbfgragh.onion/2020/08/04/world/coronavirus-cases.html?action=click\&pgtype=Article\&state=default\&region=MAIN_CONTENT_1\&context=storylines_live_updates\#link-1228a480}{Novavax
  sees encouraging results from two studies of its experimental
  vaccine.}
\item
  \href{https://www.nytimes3xbfgragh.onion/2020/08/04/world/coronavirus-cases.html?action=click\&pgtype=Article\&state=default\&region=MAIN_CONTENT_1\&context=storylines_live_updates\#link-4825b93}{Public
  and private schools in Maryland and elsewhere are divided over
  in-person instruction.}
\end{itemize}

\href{https://www.nytimes3xbfgragh.onion/2020/08/04/world/coronavirus-cases.html?action=click\&pgtype=Article\&state=default\&region=MAIN_CONTENT_1\&context=storylines_live_updates}{See
more updates}

More live coverage:
\href{https://www.nytimes3xbfgragh.onion/live/2020/08/04/business/stock-market-today-coronavirus?action=click\&pgtype=Article\&state=default\&region=MAIN_CONTENT_1\&context=storylines_live_updates}{Markets}

``It's important that as we see progress being made, and declining
hospitalizations and emergency room admissions and positive rates going
down, that all of these governors are also aware as they're increasing
testing, the number of cases that are going to be reported are going
up,'' the vice president said on the call. ``But it's all going to be a
matter of making sure that the public sees the whole picture. But it's
all progress.''

While the Trump White House is emphasizing testing, experts say a whole
range of additional policies are needed to contain the fast-moving
virus: isolation of those infected, contact tracing to locate people who
interacted with a coronavirus-positive person and quarantines for those
people.

In New York, where the number of overall cases is declining, a
cautious-sounding Gov. Andrew M. Cuomo said Monday that the state would
monitor four ``core factors'' to determine if a region is ready to
reopen: the number of new infections; the capacity of the health care
system; the testing capacity; and the capacity for ``contact tracing''
to identify people exposed to those who test positive.

``While we continue to reduce the spread of the Covid-19 virus, we can
begin to focus on reopening, but we have to be careful and use the
information we've learned so we don't erase the strides we've already
made,'' Mr. Cuomo said. ``Reopening is not going to happen statewide all
at once.''

Nationally, 27 states had loosened social distancing restrictions in
some way as of Monday, and others had announced changes that will take
effect in the coming weeks, according to an
\href{https://www.kff.org/coronavirus-policy-watch/lifting-social-distancing-measures-in-america-state-actions-metrics/}{analysis
by the Kaiser Family Foundation}. But only 20 of those states meet the
caseload or testing criteria set out by the Trump administration.

The remaining seven --- Indiana, Iowa, Kansas, Minnesota, Mississippi,
Nebraska and Wyoming --- are still showing a rise in daily infections
and positive tests, but have moved toward reopening anyway.

``It is true that there are parts of the country that are doing better
and can begin to look at ways to ease the requirements, but there are
large swaths of the country that are not, and the growth that is
projected is based mostly on these other parts of the country,''
Jennifer Kates, the foundation's director of global health and H.I.V.
policy and an author of the analysis, said in an interview.

The administration's forecast, she said, ``says we are far from out of
the woods on this, and it's quite concerning.''

Before reopening, Ms. Kates said, governors must consider other factors
beyond caseload and testing: ``Do we have enough I.C.U. beds? How is our
hospital capacity? How is our contact tracing?'' Based on its own
metrics, which urge states to increase the number of tests conducted and
the share of their populations tested each week, her analysis concluded
that just nine of the 27 states could consider relaxing social
distancing requirements.

If anything, the administration's projections are too optimistic,
forecasting experts said Monday. In the projections, the number of
actual deaths for one of the last days in April turned out to be
slightly lower than what the model showed. But for much of April and
parts of May, actual deaths were some 10 times higher than the model
predicted.

``The model is overly optimistic and not particularly useful in guiding
decisions about the disease's trajectory,'' said Dr. Donald Burke, a
professor of epidemiology at the University of Pittsburgh Graduate
School of Public Health.

\href{https://www.nytimes3xbfgragh.onion/news-event/coronavirus?action=click\&pgtype=Article\&state=default\&region=MAIN_CONTENT_3\&context=storylines_faq}{}

\hypertarget{the-coronavirus-outbreak-}{%
\subsubsection{The Coronavirus Outbreak
›}\label{the-coronavirus-outbreak-}}

\hypertarget{frequently-asked-questions}{%
\paragraph{Frequently Asked
Questions}\label{frequently-asked-questions}}

Updated August 4, 2020

\begin{itemize}
\item ~
  \hypertarget{i-have-antibodies-am-i-now-immune}{%
  \paragraph{I have antibodies. Am I now
  immune?}\label{i-have-antibodies-am-i-now-immune}}

  \begin{itemize}
  \tightlist
  \item
    As of right
    now,\href{https://www.nytimes3xbfgragh.onion/2020/07/22/health/covid-antibodies-herd-immunity.html?action=click\&pgtype=Article\&state=default\&region=MAIN_CONTENT_3\&context=storylines_faq}{that
    seems likely, for at least several months.} There have been
    frightening accounts of people suffering what seems to be a second
    bout of Covid-19. But experts say these patients may have a
    drawn-out course of infection, with the virus taking a slow toll
    weeks to months after initial exposure. People infected with the
    coronavirus typically
    \href{https://www.nature.com/articles/s41586-020-2456-9}{produce}
    immune molecules called antibodies, which are
    \href{https://www.nytimes3xbfgragh.onion/2020/05/07/health/coronavirus-antibody-prevalence.html?action=click\&pgtype=Article\&state=default\&region=MAIN_CONTENT_3\&context=storylines_faq}{protective
    proteins made in response to an
    infection}\href{https://www.nytimes3xbfgragh.onion/2020/05/07/health/coronavirus-antibody-prevalence.html?action=click\&pgtype=Article\&state=default\&region=MAIN_CONTENT_3\&context=storylines_faq}{.
    These antibodies may} last in the body
    \href{https://www.nature.com/articles/s41591-020-0965-6}{only two to
    three months}, which may seem worrisome, but that's perfectly normal
    after an acute infection subsides, said Dr. Michael Mina, an
    immunologist at Harvard University. It may be possible to get the
    coronavirus again, but it's highly unlikely that it would be
    possible in a short window of time from initial infection or make
    people sicker the second time.
  \end{itemize}
\item ~
  \hypertarget{im-a-small-business-owner-can-i-get-relief}{%
  \paragraph{I'm a small-business owner. Can I get
  relief?}\label{im-a-small-business-owner-can-i-get-relief}}

  \begin{itemize}
  \tightlist
  \item
    The
    \href{https://www.nytimes3xbfgragh.onion/article/small-business-loans-stimulus-grants-freelancers-coronavirus.html?action=click\&pgtype=Article\&state=default\&region=MAIN_CONTENT_3\&context=storylines_faq}{stimulus
    bills enacted in March} offer help for the millions of American
    small businesses. Those eligible for aid are businesses and
    nonprofit organizations with fewer than 500 workers, including sole
    proprietorships, independent contractors and freelancers. Some
    larger companies in some industries are also eligible. The help
    being offered, which is being managed by the Small Business
    Administration, includes the Paycheck Protection Program and the
    Economic Injury Disaster Loan program. But lots of folks have
    \href{https://www.nytimes3xbfgragh.onion/interactive/2020/05/07/business/small-business-loans-coronavirus.html?action=click\&pgtype=Article\&state=default\&region=MAIN_CONTENT_3\&context=storylines_faq}{not
    yet seen payouts.} Even those who have received help are confused:
    The rules are draconian, and some are stuck sitting on
    \href{https://www.nytimes3xbfgragh.onion/2020/05/02/business/economy/loans-coronavirus-small-business.html?action=click\&pgtype=Article\&state=default\&region=MAIN_CONTENT_3\&context=storylines_faq}{money
    they don't know how to use.} Many small-business owners are getting
    less than they expected or
    \href{https://www.nytimes3xbfgragh.onion/2020/06/10/business/Small-business-loans-ppp.html?action=click\&pgtype=Article\&state=default\&region=MAIN_CONTENT_3\&context=storylines_faq}{not
    hearing anything at all.}
  \end{itemize}
\item ~
  \hypertarget{what-are-my-rights-if-i-am-worried-about-going-back-to-work}{%
  \paragraph{What are my rights if I am worried about going back to
  work?}\label{what-are-my-rights-if-i-am-worried-about-going-back-to-work}}

  \begin{itemize}
  \tightlist
  \item
    Employers have to provide
    \href{https://www.osha.gov/SLTC/covid-19/standards.html}{a safe
    workplace} with policies that protect everyone equally.
    \href{https://www.nytimes3xbfgragh.onion/article/coronavirus-money-unemployment.html?action=click\&pgtype=Article\&state=default\&region=MAIN_CONTENT_3\&context=storylines_faq}{And
    if one of your co-workers tests positive for the coronavirus, the
    C.D.C.} has said that
    \href{https://www.cdc.gov/coronavirus/2019-ncov/community/guidance-business-response.html}{employers
    should tell their employees} -\/- without giving you the sick
    employee's name -\/- that they may have been exposed to the virus.
  \end{itemize}
\item ~
  \hypertarget{should-i-refinance-my-mortgage}{%
  \paragraph{Should I refinance my
  mortgage?}\label{should-i-refinance-my-mortgage}}

  \begin{itemize}
  \tightlist
  \item
    \href{https://www.nytimes3xbfgragh.onion/article/coronavirus-money-unemployment.html?action=click\&pgtype=Article\&state=default\&region=MAIN_CONTENT_3\&context=storylines_faq}{It
    could be a good idea,} because mortgage rates have
    \href{https://www.nytimes3xbfgragh.onion/2020/07/16/business/mortgage-rates-below-3-percent.html?action=click\&pgtype=Article\&state=default\&region=MAIN_CONTENT_3\&context=storylines_faq}{never
    been lower.} Refinancing requests have pushed mortgage applications
    to some of the highest levels since 2008, so be prepared to get in
    line. But defaults are also up, so if you're thinking about buying a
    home, be aware that some lenders have tightened their standards.
  \end{itemize}
\item ~
  \hypertarget{what-is-school-going-to-look-like-in-september}{%
  \paragraph{What is school going to look like in
  September?}\label{what-is-school-going-to-look-like-in-september}}

  \begin{itemize}
  \tightlist
  \item
    It is unlikely that many schools will return to a normal schedule
    this fall, requiring the grind of
    \href{https://www.nytimes3xbfgragh.onion/2020/06/05/us/coronavirus-education-lost-learning.html?action=click\&pgtype=Article\&state=default\&region=MAIN_CONTENT_3\&context=storylines_faq}{online
    learning},
    \href{https://www.nytimes3xbfgragh.onion/2020/05/29/us/coronavirus-child-care-centers.html?action=click\&pgtype=Article\&state=default\&region=MAIN_CONTENT_3\&context=storylines_faq}{makeshift
    child care} and
    \href{https://www.nytimes3xbfgragh.onion/2020/06/03/business/economy/coronavirus-working-women.html?action=click\&pgtype=Article\&state=default\&region=MAIN_CONTENT_3\&context=storylines_faq}{stunted
    workdays} to continue. California's two largest public school
    districts --- Los Angeles and San Diego --- said on July 13, that
    \href{https://www.nytimes3xbfgragh.onion/2020/07/13/us/lausd-san-diego-school-reopening.html?action=click\&pgtype=Article\&state=default\&region=MAIN_CONTENT_3\&context=storylines_faq}{instruction
    will be remote-only in the fall}, citing concerns that surging
    coronavirus infections in their areas pose too dire a risk for
    students and teachers. Together, the two districts enroll some
    825,000 students. They are the largest in the country so far to
    abandon plans for even a partial physical return to classrooms when
    they reopen in August. For other districts, the solution won't be an
    all-or-nothing approach.
    \href{https://bioethics.jhu.edu/research-and-outreach/projects/eschool-initiative/school-policy-tracker/}{Many
    systems}, including the nation's largest, New York City, are
    devising
    \href{https://www.nytimes3xbfgragh.onion/2020/06/26/us/coronavirus-schools-reopen-fall.html?action=click\&pgtype=Article\&state=default\&region=MAIN_CONTENT_3\&context=storylines_faq}{hybrid
    plans} that involve spending some days in classrooms and other days
    online. There's no national policy on this yet, so check with your
    municipal school system regularly to see what is happening in your
    community.
  \end{itemize}
\end{itemize}

Dr. George Rutherford, a professor of epidemiology at the University of
California, San Francisco, noted that the government's model has already
come in below reported deaths from Covid-19, and that death toll is not
counting deaths not officially recorded. ``Remember,'' he said, ``these
are reported deaths; the true number is likely higher.''

In the absence of a national policy to slow the virus, state officials
have been left to answer a wrenching question: How many deaths are
acceptable?

The White House distanced itself from the projections, saying the
document, dated May 2, was not produced by or presented to the
president's coronavirus task force, which does its own modeling. ``The
data is not reflective of any of the modeling done by the task force or
data that the task force has analyzed,'' Judd Deere, Mr. Trump's deputy
press secretary, told reporters on Monday.

On Sunday, the president offered his own projections, saying that deaths
in the United States could reach 100,000, twice as many as he had
forecast only two weeks ago. But that figure falls short of what his own
administration is now predicting to be the total death toll by the end
of May --- much less in the months that follow. It follows a pattern for
Mr. Trump, who has frequently understated the effect of the disease.

``We're going to lose anywhere from 75, 80 to 100,000 people,'' he said
in a virtual town hall on Fox News. ``That's a horrible thing. We
shouldn't lose one person over this.''

\includegraphics{https://static01.graylady3jvrrxbe.onion/images/2020/05/04/us/politics/04dc-virus-deaths2/merlin_172153812_3293d8a7-cceb-4c43-bea1-0b0ae68c2af4-articleLarge.jpg?quality=75\&auto=webp\&disable=upscale}

Public health experts and epidemiologists say they were not surprised by
the administration's numbers. Many do not expect the virus to slow down
until 60 to 70 percent of the population is infected, creating what
experts call ``herd immunity.''

Dr. Michael T. Osterholm, the director of the Center for Infectious
Disease Research and Policy at the University of Minnesota,
\href{https://www.cidrap.umn.edu/sites/default/files/public/downloads/cidrap-covid19-viewpoint-part1_0.pdf}{published
an analysis} last week describing three possible pandemic wave scenarios
through the end of 2021: a series of repetitive smaller waves that
gradually diminish over time; a sharp rise in cases in the fall and one
or more subsequent smaller waves; and a ``slow burn'' of continuing
transmission, without a clear wave pattern.

``I'm not saying we don't have to deal with our economy and lost jobs in
society,'' Dr. Osterholm said in an interview on Monday. ``We can't
spend 18 to 24 months in lockdown, but at the same time, this virus is
going to keep transmitting, and we have to not let our hospitals get
overrun.''

Yet many states are still operating under stay-at-home orders. Public
health officials said their goal was to ``bend the curve'' --- to slow
and ultimately reverse the rising trajectory of infections --- by
shutting down schools and businesses. But after nearly two months of a
near total shutdown, the curve does not appear to have bent as far as
they had hoped.

``While mitigation didn't fail, I think it's fair to say that it didn't
work as well as we expected,'' Scott Gottlieb, Mr. Trump's former
commissioner of food and drugs, said Sunday on the CBS program
``\href{https://www.cbsnews.com/news/transcript-scott-gottlieb-discusses-coronavirus-on-face-the-nation-may-3-2020/}{Face
the Nation}.'' ``We expected that we would start seeing more significant
declines in new cases and deaths around the nation at this point. And
we're just not seeing that.''

Jonathan Martin contributed reporting from Washington, and Benedict
Carey from New York.

Advertisement

\protect\hyperlink{after-bottom}{Continue reading the main story}

\hypertarget{site-index}{%
\subsection{Site Index}\label{site-index}}

\hypertarget{site-information-navigation}{%
\subsection{Site Information
Navigation}\label{site-information-navigation}}

\begin{itemize}
\tightlist
\item
  \href{https://help.nytimes3xbfgragh.onion/hc/en-us/articles/115014792127-Copyright-notice}{©~2020~The
  New York Times Company}
\end{itemize}

\begin{itemize}
\tightlist
\item
  \href{https://www.nytco.com/}{NYTCo}
\item
  \href{https://help.nytimes3xbfgragh.onion/hc/en-us/articles/115015385887-Contact-Us}{Contact
  Us}
\item
  \href{https://www.nytco.com/careers/}{Work with us}
\item
  \href{https://nytmediakit.com/}{Advertise}
\item
  \href{http://www.tbrandstudio.com/}{T Brand Studio}
\item
  \href{https://www.nytimes3xbfgragh.onion/privacy/cookie-policy\#how-do-i-manage-trackers}{Your
  Ad Choices}
\item
  \href{https://www.nytimes3xbfgragh.onion/privacy}{Privacy}
\item
  \href{https://help.nytimes3xbfgragh.onion/hc/en-us/articles/115014893428-Terms-of-service}{Terms
  of Service}
\item
  \href{https://help.nytimes3xbfgragh.onion/hc/en-us/articles/115014893968-Terms-of-sale}{Terms
  of Sale}
\item
  \href{https://spiderbites.nytimes3xbfgragh.onion}{Site Map}
\item
  \href{https://help.nytimes3xbfgragh.onion/hc/en-us}{Help}
\item
  \href{https://www.nytimes3xbfgragh.onion/subscription?campaignId=37WXW}{Subscriptions}
\end{itemize}
