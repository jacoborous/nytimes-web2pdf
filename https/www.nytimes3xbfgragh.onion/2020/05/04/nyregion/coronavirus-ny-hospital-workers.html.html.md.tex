\href{/section/nyregion}{New York}\textbar{}3 Hospital Workers Gave Out
Masks. Weeks Later, They All Were Dead.

\url{https://nyti.ms/2SwMk1G}

\begin{itemize}
\item
\item
\item
\item
\item
\item
\end{itemize}

\href{https://www.nytimes3xbfgragh.onion/news-event/coronavirus?action=click\&pgtype=Article\&state=default\&region=TOP_BANNER\&context=storylines_menu}{The
Coronavirus Outbreak}

\begin{itemize}
\tightlist
\item
  live\href{https://www.nytimes3xbfgragh.onion/2020/08/04/world/coronavirus-cases.html?action=click\&pgtype=Article\&state=default\&region=TOP_BANNER\&context=storylines_menu}{Latest
  Updates}
\item
  \href{https://www.nytimes3xbfgragh.onion/interactive/2020/us/coronavirus-us-cases.html?action=click\&pgtype=Article\&state=default\&region=TOP_BANNER\&context=storylines_menu}{Maps
  and Cases}
\item
  \href{https://www.nytimes3xbfgragh.onion/interactive/2020/science/coronavirus-vaccine-tracker.html?action=click\&pgtype=Article\&state=default\&region=TOP_BANNER\&context=storylines_menu}{Vaccine
  Tracker}
\item
  \href{https://www.nytimes3xbfgragh.onion/2020/08/02/us/covid-college-reopening.html?action=click\&pgtype=Article\&state=default\&region=TOP_BANNER\&context=storylines_menu}{College
  Reopening}
\item
  \href{https://www.nytimes3xbfgragh.onion/live/2020/08/04/business/stock-market-today-coronavirus?action=click\&pgtype=Article\&state=default\&region=TOP_BANNER\&context=storylines_menu}{Economy}
\end{itemize}

\includegraphics{https://static01.graylady3jvrrxbe.onion/images/2020/05/04/nyregion/04nyvirus-workers01a/04nyvirus-workers01a-articleLarge.jpg?quality=75\&auto=webp\&disable=upscale}

Sections

\protect\hyperlink{site-content}{Skip to
content}\protect\hyperlink{site-index}{Skip to site index}

\hypertarget{3-hospital-workers-gave-out-masks-weeks-later-they-all-were-dead}{%
\section{3 Hospital Workers Gave Out Masks. Weeks Later, They All Were
Dead.}\label{3-hospital-workers-gave-out-masks-weeks-later-they-all-were-dead}}

The coronavirus has taken a steep toll on the often-invisible army of
employees who keep New York hospitals running.

Rosalyn Washington's husband, Gary, 56, died from Covid-19 the day
before their wedding anniversary.Credit...Sara Naomi Lewkowicz for The
New York Times

Supported by

\protect\hyperlink{after-sponsor}{Continue reading the main story}

By \href{https://www.nytimes3xbfgragh.onion/by/nicole-hong}{Nicole Hong}

\begin{itemize}
\item
  Published May 4, 2020Updated May 6, 2020
\item
  \begin{itemize}
  \item
  \item
  \item
  \item
  \item
  \item
  \end{itemize}
\end{itemize}

They did not treat patients, but Wayne Edwards, Derik Braswell and
Priscilla Carrow held some of the most vital jobs at
\href{https://www.nytimes3xbfgragh.onion/2020/05/20/nyregion/hospitals-coronavirus-cases-decline.html}{Elmhurst
Hospital Center} in Queens.

As the
\href{https://www.nytimes3xbfgragh.onion/2020/05/06/us/politics/coronavirus-masks-tests-ppe.html}{coronavirus}
tore through the surrounding neighborhood, their department managed the
\href{https://www.nytimes3xbfgragh.onion/2020/05/06/us/politics/coronavirus-masks-tests-ppe.html}{masks},
gloves and other protective gear inside
\href{https://www.nytimes3xbfgragh.onion/2020/03/25/nyregion/nyc-coronavirus-hospitals.html}{Elmhurst,
a public hospital at the center} of the city's outbreak. They ordered
the inventory, replenished the stockroom and handed out supplies,
keeping a close count as
\href{https://www.nytimes3xbfgragh.onion/2020/03/19/health/coronavirus-masks-shortage.html}{the
number of available masks began to dwindle}.

By April 12, they were all dead.

Image

Wayne Edwards worked at Elmhurst Hospital Center for about 40 years. His
kindness was ``like a bottomless pit,'' a friend said.Credit...via
Genevieve Allen

\href{https://www.nytimes3xbfgragh.onion/2020/03/30/nyregion/ny-coronavirus-doctors-sick.html}{The
pandemic has taken an undisputed toll} on
\href{https://www.nytimes3xbfgragh.onion/2020/04/27/nyregion/new-york-city-doctor-suicide-coronavirus.html}{doctors},
\href{https://www.nytimes3xbfgragh.onion/2020/03/26/nyregion/nurse-dies-coronavirus-mount-sinai.html}{nurses}
and other
\href{https://www.nytimes3xbfgragh.onion/2020/04/15/nyregion/coronavirus-woodhull-madhvi-aya-dead.html}{front-line
health care workers.} But it has also ravaged the often-invisible army
of nonmedical workers in hospitals, many of whom have fallen ill or died
with little public recognition of their roles.

The victims included the security guards watching over emergency rooms.
They were the chefs who cooked food for patients and other hospital
workers. They assigned hospital beds and checked patients' medical
records. They greeted visitors and answered phones. They mopped the
hallways and took out the garbage.

``\href{https://www.nytimes3xbfgragh.onion/interactive/2020/04/10/nyregion/nyc-7pm-cheer-thank-you-coronavirus.html}{You
know how people clap for health workers at 7 o'clock}? It's mainly for
the nurses and doctors. I get it. But people are not seeing the other
parts of the hospital,'' said Eneida Becote, whose husband died last
month after working for two decades as a patient transporter. ``I feel
like those other employees are not focused upon as much.''

Her husband, Edward Becote, made about \$45,000 a year moving patients
around the Brooklyn Hospital Center on stretchers and wheelchairs. He
was among at least 32 nonmedical hospital workers in New York City who
have died during the pandemic, according to an analysis by The New York
Times.

Image

Edward Becote, 51, was so popular among his colleagues that he was known
as the ``mayor'' of the Brooklyn Hospital Center.Credit...via Eneida
Becote

These workers make some of the lowest wages in hospitals, and they are
more likely
\href{https://www.nytimes3xbfgragh.onion/2020/04/08/nyregion/coronavirus-race-deaths.html}{than
medical staff members to be black or Latino}. In New York City's public
hospitals, 79 percent of the workers who assist doctors and nurses are
black or Hispanic, compared with 44 percent of the medical staff,
according to the most recent city data.

In the early weeks of the pandemic,
\href{https://www.nytimes3xbfgragh.onion/2020/03/19/us/hospitals-coronavirus-ppe-shortage.html}{when
even emergency room nurses had to reuse N95 masks for days at a time},
nonmedical workers were often given less protective gear than their
colleagues who treated patients --- or none at all --- according to
union leaders and hospital employees.

\hypertarget{latest-updates-global-coronavirus-outbreak}{%
\section{\texorpdfstring{\href{https://www.nytimes3xbfgragh.onion/2020/08/04/world/coronavirus-cases.html?action=click\&pgtype=Article\&state=default\&region=MAIN_CONTENT_1\&context=storylines_live_updates}{Latest
Updates: Global Coronavirus
Outbreak}}{Latest Updates: Global Coronavirus Outbreak}}\label{latest-updates-global-coronavirus-outbreak}}

Updated 2020-08-04T21:41:55.934Z

\begin{itemize}
\tightlist
\item
  \href{https://www.nytimes3xbfgragh.onion/2020/08/04/world/coronavirus-cases.html?action=click\&pgtype=Article\&state=default\&region=MAIN_CONTENT_1\&context=storylines_live_updates\#link-2daa96b5}{As
  talks drag on, McConnell signals openness to jobless aid extension
  that Republicans have opposed.}
\item
  \href{https://www.nytimes3xbfgragh.onion/2020/08/04/world/coronavirus-cases.html?action=click\&pgtype=Article\&state=default\&region=MAIN_CONTENT_1\&context=storylines_live_updates\#link-1228a480}{Novavax
  sees encouraging results from two studies of its experimental
  vaccine.}
\item
  \href{https://www.nytimes3xbfgragh.onion/2020/08/04/world/coronavirus-cases.html?action=click\&pgtype=Article\&state=default\&region=MAIN_CONTENT_1\&context=storylines_live_updates\#link-4825b93}{Public
  and private schools in Maryland and elsewhere are divided over
  in-person instruction.}
\end{itemize}

\href{https://www.nytimes3xbfgragh.onion/2020/08/04/world/coronavirus-cases.html?action=click\&pgtype=Article\&state=default\&region=MAIN_CONTENT_1\&context=storylines_live_updates}{See
more updates}

More live coverage:
\href{https://www.nytimes3xbfgragh.onion/live/2020/08/04/business/stock-market-today-coronavirus?action=click\&pgtype=Article\&state=default\&region=MAIN_CONTENT_1\&context=storylines_live_updates}{Markets}

``If you work in a hospital, you are exposed to the same kind of virus
as the doctors and nurses,'' said Carmen Charles, president of the union
that represents 8,500 nonmedical staff members at New York City
hospitals. ``I understand management wanting to ration the supplies, but
at what cost?''

Ms. Charles, who leads Local 420, part of the umbrella union for city
workers, said some of her members had been denied the N95 masks that
were reserved for doctors and nurses. At least 11 members have died, she
said.

A spokesman for Health and Hospitals, the city's public hospital system,
acknowledged that it saved N95 masks for clinical employees who treated
Covid-19 patients and other employees in ``hot zones,'' such as the
emergency department. Early in the pandemic, the spokesman said, most
government guidance on masks focused on clinical employees. He said the
agency offered surgical masks to its nonclinical workers.

Elmhurst did not require every employee to wear at least a surgical mask
until April 15, the same day Gov. Andrew M. Cuomo announced an order
mandating New Yorkers to wear face coverings in public, according to
emails viewed by The Times.

In March, Ms. Carrow, Mr. Edwards and Mr. Braswell handed out supplies
in the materials management department, in the hospital's subbasement.
Their deaths have shaken other nonclinical employees at Elmhurst who
hoped that their distance from patients offered some protection against
contracting the virus.

Image

Priscilla Carrow was a union steward and community leader in Queens. Her
activism centered on issues including housing and fair
wages.Credit...via Marci Rosenblum

Ms. Carrow, 65, died on March 30 after working at Elmhurst for 25 years.
Mr. Edwards, 61, died two days later, after a friend found him on the
floor of his apartment, gasping for air. Both of them had expected to
retire within the next year.

Mr. Braswell, Mr. Edwards's supervisor, died on April 12. He was 57.

Image

Derik Braswell enjoyed fishing trips. A colleague remembered him as ``a
gentle giant.''Credit...via LinkedIn

``As you start mending your heart for one, then the next one came,''
said Gary Johnson, who previously worked in their department and
discovered Mr. Edwards in his apartment. ``You wonder when the pain
stops.''

Hospitals generally have not released the names of employees who have
died, leading workers to collect the names through word of mouth and
organize their own memorials. The Times compiled its list through
obituaries and interviews with hospital employees and relatives.

\href{https://www.nytimes3xbfgragh.onion/2020/03/26/nyregion/coronavirus-brooklyn-hospital.html}{At
the Brooklyn Hospital Center} in Fort Greene, at least five employees
have died in recent weeks, according to interviews.

Rafael Cargill handled medical records at the hospital, including
sometimes retrieving them from floors with virus patients, said his
sister, Lillian Cargill. She said that he was concerned about a
colleague who showed up to work despite testing positive for the virus,
and that he had not received any protective gear when he developed a dry
cough in late March.

Image

Rafael Cargill worked for more than two decades at the Brooklyn Hospital
Center, where he met his wife. He was ``light on his feet and would
dance the night away,'' his obituary said.Credit...via Lillian Cargill

Mr. Cargill, 60, died at home on March 30.

``We ran over there and had to stand outside,'' his sister said. The
paramedics ``wouldn't allow us to go in. They came out and said they
couldn't save him.''

Kim C. Flodin, a spokeswoman for
\href{https://www.nytimes3xbfgragh.onion/2020/04/12/nyregion/coronavirus-births-mothers.html}{the
Brooklyn Hospital Center}, said the hospital was following state and
federal protocols to ``protect our staff, clinical and nonclinical, from
the transmission of this virus.''

It is difficult to pinpoint how any hospital employee contracted the
virus; many commute by public transit and live with family members who
are also unable to work from home. But anyone working in hospitals
inundated with patients was potentially exposed.

In a lawsuit filed on April 20, the largest nurses' union in New York
accused the state's Department of Health of enacting policies that
turned hospitals into ``petri dishes where the virus can fester and then
spread to other health care workers.''

\href{https://www.nytimes3xbfgragh.onion/news-event/coronavirus?action=click\&pgtype=Article\&state=default\&region=MAIN_CONTENT_3\&context=storylines_faq}{}

\hypertarget{the-coronavirus-outbreak-}{%
\subsubsection{The Coronavirus Outbreak
›}\label{the-coronavirus-outbreak-}}

\hypertarget{frequently-asked-questions}{%
\paragraph{Frequently Asked
Questions}\label{frequently-asked-questions}}

Updated August 4, 2020

\begin{itemize}
\item ~
  \hypertarget{i-have-antibodies-am-i-now-immune}{%
  \paragraph{I have antibodies. Am I now
  immune?}\label{i-have-antibodies-am-i-now-immune}}

  \begin{itemize}
  \tightlist
  \item
    As of right
    now,\href{https://www.nytimes3xbfgragh.onion/2020/07/22/health/covid-antibodies-herd-immunity.html?action=click\&pgtype=Article\&state=default\&region=MAIN_CONTENT_3\&context=storylines_faq}{that
    seems likely, for at least several months.} There have been
    frightening accounts of people suffering what seems to be a second
    bout of Covid-19. But experts say these patients may have a
    drawn-out course of infection, with the virus taking a slow toll
    weeks to months after initial exposure. People infected with the
    coronavirus typically
    \href{https://www.nature.com/articles/s41586-020-2456-9}{produce}
    immune molecules called antibodies, which are
    \href{https://www.nytimes3xbfgragh.onion/2020/05/07/health/coronavirus-antibody-prevalence.html?action=click\&pgtype=Article\&state=default\&region=MAIN_CONTENT_3\&context=storylines_faq}{protective
    proteins made in response to an
    infection}\href{https://www.nytimes3xbfgragh.onion/2020/05/07/health/coronavirus-antibody-prevalence.html?action=click\&pgtype=Article\&state=default\&region=MAIN_CONTENT_3\&context=storylines_faq}{.
    These antibodies may} last in the body
    \href{https://www.nature.com/articles/s41591-020-0965-6}{only two to
    three months}, which may seem worrisome, but that's perfectly normal
    after an acute infection subsides, said Dr. Michael Mina, an
    immunologist at Harvard University. It may be possible to get the
    coronavirus again, but it's highly unlikely that it would be
    possible in a short window of time from initial infection or make
    people sicker the second time.
  \end{itemize}
\item ~
  \hypertarget{im-a-small-business-owner-can-i-get-relief}{%
  \paragraph{I'm a small-business owner. Can I get
  relief?}\label{im-a-small-business-owner-can-i-get-relief}}

  \begin{itemize}
  \tightlist
  \item
    The
    \href{https://www.nytimes3xbfgragh.onion/article/small-business-loans-stimulus-grants-freelancers-coronavirus.html?action=click\&pgtype=Article\&state=default\&region=MAIN_CONTENT_3\&context=storylines_faq}{stimulus
    bills enacted in March} offer help for the millions of American
    small businesses. Those eligible for aid are businesses and
    nonprofit organizations with fewer than 500 workers, including sole
    proprietorships, independent contractors and freelancers. Some
    larger companies in some industries are also eligible. The help
    being offered, which is being managed by the Small Business
    Administration, includes the Paycheck Protection Program and the
    Economic Injury Disaster Loan program. But lots of folks have
    \href{https://www.nytimes3xbfgragh.onion/interactive/2020/05/07/business/small-business-loans-coronavirus.html?action=click\&pgtype=Article\&state=default\&region=MAIN_CONTENT_3\&context=storylines_faq}{not
    yet seen payouts.} Even those who have received help are confused:
    The rules are draconian, and some are stuck sitting on
    \href{https://www.nytimes3xbfgragh.onion/2020/05/02/business/economy/loans-coronavirus-small-business.html?action=click\&pgtype=Article\&state=default\&region=MAIN_CONTENT_3\&context=storylines_faq}{money
    they don't know how to use.} Many small-business owners are getting
    less than they expected or
    \href{https://www.nytimes3xbfgragh.onion/2020/06/10/business/Small-business-loans-ppp.html?action=click\&pgtype=Article\&state=default\&region=MAIN_CONTENT_3\&context=storylines_faq}{not
    hearing anything at all.}
  \end{itemize}
\item ~
  \hypertarget{what-are-my-rights-if-i-am-worried-about-going-back-to-work}{%
  \paragraph{What are my rights if I am worried about going back to
  work?}\label{what-are-my-rights-if-i-am-worried-about-going-back-to-work}}

  \begin{itemize}
  \tightlist
  \item
    Employers have to provide
    \href{https://www.osha.gov/SLTC/covid-19/standards.html}{a safe
    workplace} with policies that protect everyone equally.
    \href{https://www.nytimes3xbfgragh.onion/article/coronavirus-money-unemployment.html?action=click\&pgtype=Article\&state=default\&region=MAIN_CONTENT_3\&context=storylines_faq}{And
    if one of your co-workers tests positive for the coronavirus, the
    C.D.C.} has said that
    \href{https://www.cdc.gov/coronavirus/2019-ncov/community/guidance-business-response.html}{employers
    should tell their employees} -\/- without giving you the sick
    employee's name -\/- that they may have been exposed to the virus.
  \end{itemize}
\item ~
  \hypertarget{should-i-refinance-my-mortgage}{%
  \paragraph{Should I refinance my
  mortgage?}\label{should-i-refinance-my-mortgage}}

  \begin{itemize}
  \tightlist
  \item
    \href{https://www.nytimes3xbfgragh.onion/article/coronavirus-money-unemployment.html?action=click\&pgtype=Article\&state=default\&region=MAIN_CONTENT_3\&context=storylines_faq}{It
    could be a good idea,} because mortgage rates have
    \href{https://www.nytimes3xbfgragh.onion/2020/07/16/business/mortgage-rates-below-3-percent.html?action=click\&pgtype=Article\&state=default\&region=MAIN_CONTENT_3\&context=storylines_faq}{never
    been lower.} Refinancing requests have pushed mortgage applications
    to some of the highest levels since 2008, so be prepared to get in
    line. But defaults are also up, so if you're thinking about buying a
    home, be aware that some lenders have tightened their standards.
  \end{itemize}
\item ~
  \hypertarget{what-is-school-going-to-look-like-in-september}{%
  \paragraph{What is school going to look like in
  September?}\label{what-is-school-going-to-look-like-in-september}}

  \begin{itemize}
  \tightlist
  \item
    It is unlikely that many schools will return to a normal schedule
    this fall, requiring the grind of
    \href{https://www.nytimes3xbfgragh.onion/2020/06/05/us/coronavirus-education-lost-learning.html?action=click\&pgtype=Article\&state=default\&region=MAIN_CONTENT_3\&context=storylines_faq}{online
    learning},
    \href{https://www.nytimes3xbfgragh.onion/2020/05/29/us/coronavirus-child-care-centers.html?action=click\&pgtype=Article\&state=default\&region=MAIN_CONTENT_3\&context=storylines_faq}{makeshift
    child care} and
    \href{https://www.nytimes3xbfgragh.onion/2020/06/03/business/economy/coronavirus-working-women.html?action=click\&pgtype=Article\&state=default\&region=MAIN_CONTENT_3\&context=storylines_faq}{stunted
    workdays} to continue. California's two largest public school
    districts --- Los Angeles and San Diego --- said on July 13, that
    \href{https://www.nytimes3xbfgragh.onion/2020/07/13/us/lausd-san-diego-school-reopening.html?action=click\&pgtype=Article\&state=default\&region=MAIN_CONTENT_3\&context=storylines_faq}{instruction
    will be remote-only in the fall}, citing concerns that surging
    coronavirus infections in their areas pose too dire a risk for
    students and teachers. Together, the two districts enroll some
    825,000 students. They are the largest in the country so far to
    abandon plans for even a partial physical return to classrooms when
    they reopen in August. For other districts, the solution won't be an
    all-or-nothing approach.
    \href{https://bioethics.jhu.edu/research-and-outreach/projects/eschool-initiative/school-policy-tracker/}{Many
    systems}, including the nation's largest, New York City, are
    devising
    \href{https://www.nytimes3xbfgragh.onion/2020/06/26/us/coronavirus-schools-reopen-fall.html?action=click\&pgtype=Article\&state=default\&region=MAIN_CONTENT_3\&context=storylines_faq}{hybrid
    plans} that involve spending some days in classrooms and other days
    online. There's no national policy on this yet, so check with your
    municipal school system regularly to see what is happening in your
    community.
  \end{itemize}
\end{itemize}

Nurses and other health care workers were denied testing even though
they exhibited symptoms, the lawsuit said.

In response, a Health Department spokesman said the state was taking
every step to ensure health workers have the support and supplies they
need.

That support may have come too late for Adiel Montgomery, who worked as
a security guard in the emergency department of Kingsbrook Jewish
Medical Center in Brooklyn.

He spoke out in March about lacking the protective suits that he saw
doctors wearing around Covid-19 patients, according to a colleague who
spoke on the condition of anonymity out of fear of retribution. After
Mr. Montgomery and other security guards complained, the colleague said,
more protective gear arrived.

Image

Adiel Montgomery grew up in Brooklyn. He enjoyed collecting sneakers,
his brother said.~Credit...via Norman Johnson

In late March, Mr. Montgomery lost his sense of taste and smell and
experienced flulike symptoms, according to his brother.

Mr. Montgomery, 39, was hospitalized at Kingsbrook a week later with
chest pains. While waiting hours for the results of his blood work, he
began coughing up blood.

He died on April 5. The hospital told his family that he died of a heart
attack, but his family believes he had the virus.

``I just feel that being an employee of the same hospital, he was
neglected,'' his mother, Griselda Bubb-Johnson, said. ``I feel they
should have done more.''

After the deaths of at least five Kingsbrook employees, nurses there
\href{https://gothamist.com/news/we-will-all-die-kingsbrook-hospital-nurses-demand-protective-gear-after-their-colleagues-succumb-covid-19}{protested
outside the hospital about the shortage of protective gear}. A
spokeswoman for Kingsbrook did not respond to multiple requests for
comment.

Many hospital employees worked as long as they could after they felt
sick, driven by financial necessity and a desire to help their
overstretched colleagues.

The last day Gary Washington reported to work at NewYork-Presbyterian
Allen Hospital in northern Manhattan was March 29. His body was aching,
and a colleague saw him lying down in the cafeteria.

Rosalyn Washington, his wife, thought he was growing too old to keep
working as a housekeeping employee there. He cleaned the rooms of virus
patients after they were discharged, and his brother thought he should
stop going to work, she said.

So many housekeepers called out sick that the hospital began bringing in
temporary workers, one of his colleagues said. But Mr. Washington was
the family's primary breadwinner.

``He was not going to quit his job and not take care of his family,''
Mrs. Washington said.

Image

Gary Washington was active in his fraternity, Iota Phi Theta, and
volunteered at his church's food pantry.Credit...via Rosalyn Washington

Mr. Washington, 56, died from the virus on April 8, the day before his
wedding anniversary.

Before his death, he texted his wife from his hospital bed: ``I can't
explain how much I truly love you. I didn't want to tell you how I cried
like a baby thinking about how good you've been to me.''

His wife had his urn engraved with Boop P Doop, the pet name they called
each other.

``I had 25 years with this man. I'm so empty. Now I'm getting calls
about widows' benefits,'' she said, her voice breaking. ``He's trying to
take care of me still.''

Robin Stein contributed reporting. Susan C. Beachy contributed research.

Advertisement

\protect\hyperlink{after-bottom}{Continue reading the main story}

\hypertarget{site-index}{%
\subsection{Site Index}\label{site-index}}

\hypertarget{site-information-navigation}{%
\subsection{Site Information
Navigation}\label{site-information-navigation}}

\begin{itemize}
\tightlist
\item
  \href{https://help.nytimes3xbfgragh.onion/hc/en-us/articles/115014792127-Copyright-notice}{©~2020~The
  New York Times Company}
\end{itemize}

\begin{itemize}
\tightlist
\item
  \href{https://www.nytco.com/}{NYTCo}
\item
  \href{https://help.nytimes3xbfgragh.onion/hc/en-us/articles/115015385887-Contact-Us}{Contact
  Us}
\item
  \href{https://www.nytco.com/careers/}{Work with us}
\item
  \href{https://nytmediakit.com/}{Advertise}
\item
  \href{http://www.tbrandstudio.com/}{T Brand Studio}
\item
  \href{https://www.nytimes3xbfgragh.onion/privacy/cookie-policy\#how-do-i-manage-trackers}{Your
  Ad Choices}
\item
  \href{https://www.nytimes3xbfgragh.onion/privacy}{Privacy}
\item
  \href{https://help.nytimes3xbfgragh.onion/hc/en-us/articles/115014893428-Terms-of-service}{Terms
  of Service}
\item
  \href{https://help.nytimes3xbfgragh.onion/hc/en-us/articles/115014893968-Terms-of-sale}{Terms
  of Sale}
\item
  \href{https://spiderbites.nytimes3xbfgragh.onion}{Site Map}
\item
  \href{https://help.nytimes3xbfgragh.onion/hc/en-us}{Help}
\item
  \href{https://www.nytimes3xbfgragh.onion/subscription?campaignId=37WXW}{Subscriptions}
\end{itemize}
