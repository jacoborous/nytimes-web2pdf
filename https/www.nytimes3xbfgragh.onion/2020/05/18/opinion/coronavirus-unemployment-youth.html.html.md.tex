Sections

SEARCH

\protect\hyperlink{site-content}{Skip to
content}\protect\hyperlink{site-index}{Skip to site index}

\href{https://myaccount.nytimes3xbfgragh.onion/auth/login?response_type=cookie\&client_id=vi}{}

\href{https://www.nytimes3xbfgragh.onion/section/todayspaper}{Today's
Paper}

\href{/section/opinion}{Opinion}\textbar{}7.7 Million Young People Are
Unemployed. We Need a New `Tree Army.'

\url{https://nyti.ms/2WDR6Nb}

\begin{itemize}
\item
\item
\item
\item
\item
\end{itemize}

Advertisement

\protect\hyperlink{after-top}{Continue reading the main story}

\href{/section/opinion}{Opinion}

Supported by

\protect\hyperlink{after-sponsor}{Continue reading the main story}

\hypertarget{77-million-young-people-are-unemployed-we-need-a-new-tree-army}{%
\section{7.7 Million Young People Are Unemployed. We Need a New `Tree
Army.'}\label{77-million-young-people-are-unemployed-we-need-a-new-tree-army}}

The Depression-era Civilian Conservation Corps helped build America at a
time of national crisis. Let's do it again.

By Collin O'Mara

Mr. O'Mara is the president and C.E.O. of the National Wildlife
Federation.

\begin{itemize}
\item
  May 18, 2020
\item
  \begin{itemize}
  \item
  \item
  \item
  \item
  \item
  \end{itemize}
\end{itemize}

\includegraphics{https://static01.graylady3jvrrxbe.onion/images/2020/05/18/opinion/18omara1/18omara1-articleLarge.jpg?quality=75\&auto=webp\&disable=upscale}

Nearly
\href{https://int.graylady3jvrrxbe.onion/data/documenthelper/6953-unemployment-under-30/8b646ef429cecb7d77a7/optimized/full.pdf\#page=1}{7.7
million} American workers younger than 30 are now unemployed and
\href{https://int.graylady3jvrrxbe.onion/data/documenthelper/6954-3-million-left-workforce/8b646ef429cecb7d77a7/optimized/full.pdf\#page=1}{three
million} dropped out of the labor force in the past month. Combined
that's nearly one in three young workers, by far
the\href{https://int.graylady3jvrrxbe.onion/data/documenthelper/6955-highestunemplyment/d1c48c3099970861ff19/optimized/full.pdf\#page=1}{highest
rate} since the country started tracking unemployment by age in 1948.

Nearly \href{https://www.bls.gov/news.release/empsit.nr0.htm}{40
percent} worked in the devastated retail and food service sectors. And
as the most recently hired, young workers are typically the first let go
and often the last rehired, especially those of color.

As our country's leaders consider a range of solutions to address this
crisis, there's one fix that will put millions of young Americans
directly to work: a 21st-century version of the Civilian Conservation
Corps.

In 1933, when President Franklin Roosevelt created the C.C.C., he was
facing, as we are today, the possibility of a lost generation of young
people. The conservation-minded president's idea was to hire young
unemployed men for projects in forestry, soil conservation and
recreation. By 1942, the
\href{http://archiveswest.orbiscascade.org/ark:/80444/xv31307}{3.4
million participants} in ``Roosevelt's Tree Army'' had planted more than
three billion trees, built hundreds of parks and wildlife refuges and
completed thousands of miles of trails and roads.

While the corps was not perfect --- only men were hired, work camps were
segregated, and some projects caused ecological damage --- the C.C.C.
was the most expansive and successful youth employment program in
American history. It also played a crucial role in forging the Greatest
Generation, which defeated fascism and built the strongest economy in
the world. Today, there's plenty to do for a revitalized conservation
corps that would put young Americans back to work.

We've amassed a staggering backlog of restoration needs for our nation's
lands and waters, and face escalating vulnerabilities to fires, floods,
hurricanes and droughts. Our national parks, wildlife refuges and other
public lands have \href{https://fas.org/sgp/crs/misc/R43997.pdf}{\$20
billion in deferred maintenance} --- and states have tens of billions of
dollars more. Eighty million acres of national forests need
rehabilitation.
\href{https://www.abandonedmines.gov/extent_of_the_problem}{Half a
million} abandoned coal and hard-rock mines and
\href{http://iogcc.ok.gov/Websites/iogcc/images/Publications/2019\%2012\%2031\%20Idle\%20and\%20Orphan\%20oil\%20and\%20gas\%20wells\%20-\%20state\%20and\%20provincial\%20regulatory\%20strategies\%20(2019).pdf}{thousands}
of orphaned oil and gas wells need reclamation. More than
\href{https://www.nwf.org/-/media/Documents/PDFs/NWF-Reports/2018/Reversing-Americas-Wildlife-Crisis_2018.ashx}{12,000
species} of at-risk wildlife, fish and plants need conservation.

Smart investments in natural solutions could create millions of
immediate jobs for the demographic groups and regions acutely affected
by the downturn. One study found that restoration jobs support up to
\href{https://curs.unc.edu/files/2014/01/RestorationEconomy.pdf}{33 jobs
per \$1 million of investment}, which can stimulate economic growth and
employment in other industries. Those that would stand to benefit
include outdoor recreation, agriculture, forestry and ranching, which
have been hit hard by the pandemic.

These projects would expand recreational opportunities, increase our
resilience to extreme weather and use nature by planting trees to
sequester the carbon dioxide emissions that are warming the planet.

We already have federal, state, local and tribal plans and projects that
have been vetted and are ready to go, and pending bipartisan
legislation, like the
\href{https://www.congress.gov/bill/116th-congress/senate-bill/3422}{Great
America Outdoors Act} and the
\href{https://www.congress.gov/bill/116th-congress/house-bill/3742/}{Recovering
America's Wildlife Act}, which would provide financing for some of this
work. We also have the infrastructure of AmeriCorps'
\href{https://www.nationalservice.gov/programs/americorps/americorps-programs/americorps-nccc}{National
Civilian Community Corps} and other programs that are part of the
Corporation for National and Community Service. They can be scaled up
and modernized, as proposed by Senators Chris Coons of Delaware and
Martin Heinrich of New Mexico, both Democrats, and others.

To maximize these benefits, three lessons from the C.C.C. should guide
our recovery today.

First, all communities must benefit, including youth of color and
Indigenous young people. Second, states, local governments and tribes
must be full partners. Third, high-quality educational opportunities and
apprenticeships must be included to ensure that the program's
participants are fully prepared for private-sector employment
opportunities.

How much would all this cost? That depends on the scope of our
ambitions. But clearly, there is plenty of work to be done and too many
young people without jobs or prospects who are available to do it.

We can prevent a youth unemployment crisis from hobbling the next
generation, strengthen local economies and bolster community resilience,
but we must act now to put millions of young people to work restoring
America's natural treasures.

\href{https://www.nwf.org/about-us/leadership/collin-omara}{Collin
O'Mara} (@Collin\_OMara) is president and C.E.O. of the National
Wildlife Federation and a former participant in the AmeriCorps VISTA
program*.*

\emph{The Times is committed to publishing}
\href{https://www.nytimes3xbfgragh.onion/2019/01/31/opinion/letters/letters-to-editor-new-york-times-women.html}{\emph{a
diversity of letters}} \emph{to the editor. We'd like to hear what you
think about this or any of our articles. Here are some}
\href{https://help.nytimes3xbfgragh.onion/hc/en-us/articles/115014925288-How-to-submit-a-letter-to-the-editor}{\emph{tips}}\emph{.
And here's our email:}
\href{mailto:letters@NYTimes.com}{\emph{letters@NYTimes.com}}\emph{.}

\emph{Follow The New York Times Opinion section on}
\href{https://www.facebookcorewwwi.onion/nytopinion}{\emph{Facebook}}\emph{,}
\href{http://twitter.com/NYTOpinion}{\emph{Twitter (@NYTopinion)}}
\emph{and}
\href{https://www.instagram.com/nytopinion/}{\emph{Instagram}}\emph{.}

Advertisement

\protect\hyperlink{after-bottom}{Continue reading the main story}

\hypertarget{site-index}{%
\subsection{Site Index}\label{site-index}}

\hypertarget{site-information-navigation}{%
\subsection{Site Information
Navigation}\label{site-information-navigation}}

\begin{itemize}
\tightlist
\item
  \href{https://help.nytimes3xbfgragh.onion/hc/en-us/articles/115014792127-Copyright-notice}{©~2020~The
  New York Times Company}
\end{itemize}

\begin{itemize}
\tightlist
\item
  \href{https://www.nytco.com/}{NYTCo}
\item
  \href{https://help.nytimes3xbfgragh.onion/hc/en-us/articles/115015385887-Contact-Us}{Contact
  Us}
\item
  \href{https://www.nytco.com/careers/}{Work with us}
\item
  \href{https://nytmediakit.com/}{Advertise}
\item
  \href{http://www.tbrandstudio.com/}{T Brand Studio}
\item
  \href{https://www.nytimes3xbfgragh.onion/privacy/cookie-policy\#how-do-i-manage-trackers}{Your
  Ad Choices}
\item
  \href{https://www.nytimes3xbfgragh.onion/privacy}{Privacy}
\item
  \href{https://help.nytimes3xbfgragh.onion/hc/en-us/articles/115014893428-Terms-of-service}{Terms
  of Service}
\item
  \href{https://help.nytimes3xbfgragh.onion/hc/en-us/articles/115014893968-Terms-of-sale}{Terms
  of Sale}
\item
  \href{https://spiderbites.nytimes3xbfgragh.onion}{Site Map}
\item
  \href{https://help.nytimes3xbfgragh.onion/hc/en-us}{Help}
\item
  \href{https://www.nytimes3xbfgragh.onion/subscription?campaignId=37WXW}{Subscriptions}
\end{itemize}
