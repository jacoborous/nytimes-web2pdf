Sections

SEARCH

\protect\hyperlink{site-content}{Skip to
content}\protect\hyperlink{site-index}{Skip to site index}

\href{https://www.nytimes3xbfgragh.onion/section/world/asia}{Asia
Pacific}

\href{https://myaccount.nytimes3xbfgragh.onion/auth/login?response_type=cookie\&client_id=vi}{}

\href{https://www.nytimes3xbfgragh.onion/section/todayspaper}{Today's
Paper}

\href{/section/world/asia}{Asia Pacific}\textbar{}Yu Lihua, 90, Dies;
Writer Spoke to `Rootless' Chinese Émigrés

\url{https://nyti.ms/3g2N3BL}

\begin{itemize}
\item
\item
\item
\item
\item
\end{itemize}

\href{https://www.nytimes3xbfgragh.onion/news-event/coronavirus?action=click\&pgtype=Article\&state=default\&region=TOP_BANNER\&context=storylines_menu}{The
Coronavirus Outbreak}

\begin{itemize}
\tightlist
\item
  live\href{https://www.nytimes3xbfgragh.onion/2020/08/04/world/coronavirus-covid-19.html?action=click\&pgtype=Article\&state=default\&region=TOP_BANNER\&context=storylines_menu}{Latest
  Updates}
\item
  \href{https://www.nytimes3xbfgragh.onion/interactive/2020/us/coronavirus-us-cases.html?action=click\&pgtype=Article\&state=default\&region=TOP_BANNER\&context=storylines_menu}{Maps
  and Cases}
\item
  \href{https://www.nytimes3xbfgragh.onion/interactive/2020/science/coronavirus-vaccine-tracker.html?action=click\&pgtype=Article\&state=default\&region=TOP_BANNER\&context=storylines_menu}{Vaccine
  Tracker}
\item
  \href{https://www.nytimes3xbfgragh.onion/2020/08/02/us/covid-college-reopening.html?action=click\&pgtype=Article\&state=default\&region=TOP_BANNER\&context=storylines_menu}{College
  Reopening}
\item
  \href{https://www.nytimes3xbfgragh.onion/live/2020/08/03/business/stock-market-today-coronavirus?action=click\&pgtype=Article\&state=default\&region=TOP_BANNER\&context=storylines_menu}{Economy}
\end{itemize}

Advertisement

\protect\hyperlink{after-top}{Continue reading the main story}

Supported by

\protect\hyperlink{after-sponsor}{Continue reading the main story}

Those we've lost

\hypertarget{yu-lihua-90-dies-writer-spoke-to-rootless-chinese-uxe9migruxe9s}{%
\section{Yu Lihua, 90, Dies; Writer Spoke to `Rootless' Chinese
Émigrés}\label{yu-lihua-90-dies-writer-spoke-to-rootless-chinese-uxe9migruxe9s}}

In her fiction she depicted ``the struggle of Chinese immigrants in
American society'' --- not the ``Oriental exoticism'' preferred by many
publishers in the '60s.

\includegraphics{https://static01.graylady3jvrrxbe.onion/images/2020/05/20/obituaries/12Yu/12Yu-articleLarge.jpg?quality=75\&auto=webp\&disable=upscale}

\href{https://www.nytimes3xbfgragh.onion/by/amy-qin}{\includegraphics{https://static01.graylady3jvrrxbe.onion/images/2018/10/03/multimedia/author-amy-qin/author-amy-qin-thumbLarge.png}}

By \href{https://www.nytimes3xbfgragh.onion/by/amy-qin}{Amy Qin}

\begin{itemize}
\item
  May 19, 2020
\item
  \begin{itemize}
  \item
  \item
  \item
  \item
  \item
  \end{itemize}
\end{itemize}

\href{https://cn.nytimes3xbfgragh.onion/obits/20200520/yu-lihua-90-dies-writer-spoke-to-rootless-chinese-emigres/}{阅读简体中文版}\href{https://cn.nytimes3xbfgragh.onion/obits/20200520/yu-lihua-90-dies-writer-spoke-to-rootless-chinese-emigres/zh-hant}{閱讀繁體中文版}

\emph{This obituary is part of a series about people who have died in
the coronavirus pandemic. Read about others}
\href{https://www.nytimes3xbfgragh.onion/series/people-who-have-died-of-the-coronavirus}{\emph{here}}\emph{.}

Yu Lihua, a writer whose nuanced portraits of overseas Chinese students
and intellectuals in America captured the cultural displacement and
identity crisis felt by many in the Chinese diaspora, died on April 30
at her home in Gaithersburg, Md. She was 90.

The cause was respiratory failure brought on by Covid-19, said her
daughter Lena Sun, a reporter for The Washington Post who has been
covering the coronavirus pandemic since January.

Ms. Yu produced more than two dozen novels and short story collections
over five decades, drawing on her experience as a Chinese émigré in
postwar America. She was celebrated in the diaspora for giving voice to
what she called the ``rootless generation'' --- émigrés who had left for
a better life but remained nostalgic for their homeland.

Her 1967 breakout novel, ``Again the Palm Trees,'' for example, tells
the story of a Chinese man who graduates from a Taiwan university and
goes to the United States for graduate school, where he struggles with
loneliness and disillusionment. But when he goes back to Taiwan to
rediscover his ``Chineseness,'' his sense of alienation is only
intensified by his family's glorification of life in the West,
particularly in America.

It was a theme that resonated among Taiwan-educated Chinese émigrés at
the time. Many had already been uprooted once before, compelled to flee
to Taiwan in 1949 after Mao Zedong's Communists defeated the
Nationalists in the Chinese Civil War.

Having experienced both the highs and lows of immigrant life in America,
Ms. Yu remained wary of what she saw as a tendency among Chinese to
worship the West blindly. When students from mainland China began
arriving in the United States in waves after the end of the Cultural
Revolution in the late 1970s, she wrote an open letter to them that was
published in The People's Daily, the flagship newspaper of the Chinese
Communist Party.

``Come here,'' she wrote, ``bring the wisdom of China that is by no
means inferior, bring our unique diligence and resilience, and do not
forget to bring self-respect for yourself and your nation. Stand up and
come with your head held high.''

Lee-hwa Yu was born on Nov. 28, 1929, in Shanghai, though she gave 1931
as her birth year from an early age. As an adult she mostly used the
first name Lihua.

The second of eight children, she grew up in the eastern city of Ningbo.
Later, as China became mired in the second Sino-Japanese War (1937-45),
the family moved around the country, and Ms. Yu attended school
irregularly. In 1947, her father, Yu Sheng-feng, moved the family to
Taiwan to take a job as a senior manager at a state-run sugar company
there. Her mother, Liu Hsing Ch'ing, was a homemaker.

Image

Yu Lihua in 1964. She insisted that her American-born children learn
Chinese. In an open letter to students arriving in the United States
from China, she wrote,~``Bring the wisdom of China that is by no means
inferior, bring our unique diligence and resilience.''

After graduating from National Taiwan University in 1953 with a degree
in history, Ms. Yu moved to California and attended journalism school at
the University of California, Los Angeles. In 1956, the year she
graduated, she won the prestigious
\href{https://timesmachine.nytimes3xbfgragh.onion/timesmachine/1956/06/01/86601140.pdf?pdf_redirect=true\&ip=0}{Samuel
Goldwyn Creative Writing Award} for her English-language short story
``Sorrow at the End of the Yangtze River,'' about a young woman's
journey to find her lost father.

Ms. Yu's later attempts to publish stories in English, however, were
rejected by American publishers. ``They were only interested in stories
that fit the pattern of Oriental exoticism --- the feet-binding of women
and the addiction of opium-smoking men,'' she once recalled in an
interview. ``I didn't want to write that stuff, I wanted to write about
the struggle of Chinese immigrants in American society.'' She went on to
write mostly in Chinese for Chinese-language publishers.

Ms. Yu taught Chinese language and literature at what is now the
University at Albany, the State University of New York, and was
instrumental in starting exchange programs that brought many Chinese
students to the campus. She retired from teaching in 1993.

In 2006, she was awarded an honorary doctorate from Middlebury College
in Vermont. The citation called her ``one of the five most influential
Chinese-born women writers in the postwar era and the progenitor of the
Chinese students' overseas genre.''

Her first marriage, to Chih-Ree Sun, ended in divorce. In 1982 she
married Vincent O'Leary, president of SUNY Albany. He
\href{https://archive.nytimes3xbfgragh.onion/query.nytimes3xbfgragh.onion/gst/fullpage-980DEEDF1E3BF931A35756C0A9679D8B63.html}{died}
in 2011.

In addition to her daughter Ms. Sun, her survivors include a son, Eugene
Sun; another daughter, Anna Sun; two stepdaughters, Beth O'Leary and
Cathy Goldwyn; a sister, Meihua Yu; four brothers, Jack, Ben, Henry and
Eddie Yu; 10 grandchildren and step-grandchildren; and two
great-grandchildren.

Over the decades, Ms. Yu embraced the culture of her adopted home. She
translated stories by Edith Wharton and Katherine Anne Porter into
Chinese and developed a special passion for football and Broadway
theater. But her devotion to China never faltered. She was adamant that
her American-born children learn Chinese.

Though her work was sometimes politicized, and even briefly banned in
Taiwan, Ms. Yu continued to visit mainland China. On one visit, in 1975,
she was reunited with her sister, Meihua, who had stayed behind on the
mainland when the family moved to Taiwan.

In a
\href{https://xw.qq.com/partner/standard/20200503A0LZLN/20200503A0LZLN00?ADTAG=standard\&pgv_ref=standard}{2013
interview}, Ms. Yu explained her relationship with her homeland by
referring to the traditional Chinese idiom ``fallen leaves return to
their roots.''

``In the United States, my leaves may fall but they won't return to
their roots,'' she said. ``My roots are in China.''

\href{https://www.nytimes3xbfgragh.onion/interactive/2020/obituaries/people-died-coronavirus-obituaries.html?action=click\&pgtype=Article\&state=default\&region=BELOW_MAIN_CONTENT\&context=covid_obits_promo}{}

\hypertarget{those-weve-lost}{%
\section{Those We've Lost}\label{those-weve-lost}}

The coronavirus pandemic has taken an incalculable death toll. This
series is designed to put names and faces to the numbers.

Read more

\includegraphics{https://static01.graylady3jvrrxbe.onion/images/2020/07/30/obituaries/30Pedro/30Pedro-square640.jpg}

\hypertarget{bernaldina-josuxe9-pedro}{%
\section{Bernaldina José Pedro}\label{bernaldina-josuxe9-pedro}}

d. Boa Vista, Brazil

Leader among the Indigenous Macuxi

\includegraphics{https://static01.graylady3jvrrxbe.onion/images/2020/07/31/obituaries/31Swing/merlin_175167783_8913bc90-0d64-43f3-a655-1bb1bf1601c9-square640.jpg}

\hypertarget{john-eric-swing}{%
\section{John Eric Swing}\label{john-eric-swing}}

d. Fountain Valley, Calif.

Champion of Filipino-Americans

\includegraphics{https://static01.graylady3jvrrxbe.onion/images/2020/07/27/obituaries/27Victor/merlin_175001436_38b11f8e-227a-4e2c-9821-7618af9b2524-square640.jpg}

\hypertarget{victor-victor}{%
\section{Victor Victor}\label{victor-victor}}

d. Santo Domingo, Dominican Republic

Beloved musician of the Dominican Republic

\includegraphics{https://static01.graylady3jvrrxbe.onion/images/2020/07/31/obituaries/31Negron/merlin_175160169_516322ae-fd23-4969-b6b2-193ced371105-square640.jpg}

\hypertarget{dr-eddie-negruxf3n}{%
\section{Dr. Eddie Negrón}\label{dr-eddie-negruxf3n}}

d. Fort Walton Beach, Fla.

Internist on Florida's Emerald Coast

\includegraphics{https://static01.graylady3jvrrxbe.onion/images/2020/07/30/obituaries/30Dobson/merlin_175115928_f6b9271c-8f05-4fe1-a38a-5ca4a58f8935-square640.jpg}

\hypertarget{dobby-dobson}{%
\section{Dobby Dobson}\label{dobby-dobson}}

d. Coral Springs, Fla.

Jamaican singer and songwriter

\includegraphics{https://static01.graylady3jvrrxbe.onion/images/2020/08/01/obituaries/28Gonzalez/merlin_175002771_beb57888-3951-409a-ae13-03a94b2e962e-square640.jpg}

\hypertarget{waldemar-gonzalez}{%
\section{Waldemar Gonzalez}\label{waldemar-gonzalez}}

d. White Plains, N.Y.

Teacher and social worker

Advertisement

\protect\hyperlink{after-bottom}{Continue reading the main story}

\hypertarget{site-index}{%
\subsection{Site Index}\label{site-index}}

\hypertarget{site-information-navigation}{%
\subsection{Site Information
Navigation}\label{site-information-navigation}}

\begin{itemize}
\tightlist
\item
  \href{https://help.nytimes3xbfgragh.onion/hc/en-us/articles/115014792127-Copyright-notice}{©~2020~The
  New York Times Company}
\end{itemize}

\begin{itemize}
\tightlist
\item
  \href{https://www.nytco.com/}{NYTCo}
\item
  \href{https://help.nytimes3xbfgragh.onion/hc/en-us/articles/115015385887-Contact-Us}{Contact
  Us}
\item
  \href{https://www.nytco.com/careers/}{Work with us}
\item
  \href{https://nytmediakit.com/}{Advertise}
\item
  \href{http://www.tbrandstudio.com/}{T Brand Studio}
\item
  \href{https://www.nytimes3xbfgragh.onion/privacy/cookie-policy\#how-do-i-manage-trackers}{Your
  Ad Choices}
\item
  \href{https://www.nytimes3xbfgragh.onion/privacy}{Privacy}
\item
  \href{https://help.nytimes3xbfgragh.onion/hc/en-us/articles/115014893428-Terms-of-service}{Terms
  of Service}
\item
  \href{https://help.nytimes3xbfgragh.onion/hc/en-us/articles/115014893968-Terms-of-sale}{Terms
  of Sale}
\item
  \href{https://spiderbites.nytimes3xbfgragh.onion}{Site Map}
\item
  \href{https://help.nytimes3xbfgragh.onion/hc/en-us}{Help}
\item
  \href{https://www.nytimes3xbfgragh.onion/subscription?campaignId=37WXW}{Subscriptions}
\end{itemize}
