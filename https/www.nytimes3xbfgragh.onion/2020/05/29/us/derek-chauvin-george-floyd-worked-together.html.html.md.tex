Sections

SEARCH

\protect\hyperlink{site-content}{Skip to
content}\protect\hyperlink{site-index}{Skip to site index}

\href{https://www.nytimes3xbfgragh.onion/section/us}{U.S.}

\href{https://myaccount.nytimes3xbfgragh.onion/auth/login?response_type=cookie\&client_id=vi}{}

\href{https://www.nytimes3xbfgragh.onion/section/todayspaper}{Today's
Paper}

\href{/section/us}{U.S.}\textbar{}What Happened in the Chaotic Moments
Before George Floyd Died

\url{https://nyti.ms/2ZNMhTo}

\begin{itemize}
\item
\item
\item
\item
\item
\end{itemize}

\href{https://www.nytimes3xbfgragh.onion/news-event/george-floyd-protests-minneapolis-new-york-los-angeles?action=click\&pgtype=Article\&state=default\&region=TOP_BANNER\&context=storylines_menu}{Race
and America}

\begin{itemize}
\tightlist
\item
  \href{https://www.nytimes3xbfgragh.onion/2020/07/26/us/protests-portland-seattle-trump.html?action=click\&pgtype=Article\&state=default\&region=TOP_BANNER\&context=storylines_menu}{Protesters
  Return to Other Cities}
\item
  \href{https://www.nytimes3xbfgragh.onion/2020/07/24/us/portland-oregon-protests-white-race.html?action=click\&pgtype=Article\&state=default\&region=TOP_BANNER\&context=storylines_menu}{Portland
  at the Center}
\item
  \href{https://www.nytimes3xbfgragh.onion/2020/07/23/podcasts/the-daily/portland-protests.html?action=click\&pgtype=Article\&state=default\&region=TOP_BANNER\&context=storylines_menu}{Podcast:
  Showdown in Portland}
\item
  \href{https://www.nytimes3xbfgragh.onion/interactive/2020/07/16/us/black-lives-matter-protests-louisville-breonna-taylor.html?action=click\&pgtype=Article\&state=default\&region=TOP_BANNER\&context=storylines_menu}{45
  Days in Louisville}
\end{itemize}

Advertisement

\protect\hyperlink{after-top}{Continue reading the main story}

Supported by

\protect\hyperlink{after-sponsor}{Continue reading the main story}

\hypertarget{what-happened-in-the-chaotic-moments-before-george-floyd-died}{%
\section{What Happened in the Chaotic Moments Before George Floyd
Died}\label{what-happened-in-the-chaotic-moments-before-george-floyd-died}}

The episode began with a report of a \$20 counterfeit bill. It ended in
a fatal encounter with the police, which the authorities have described
in detail for the first time.

\includegraphics{https://static01.graylady3jvrrxbe.onion/images/2020/05/29/merlin_172936683_3e64975d-5e1e-4dc2-bbb7-74f8325567ea/merlin_172936683_3e64975d-5e1e-4dc2-bbb7-74f8325567ea-articleLarge.jpg?quality=75\&auto=webp\&disable=upscale}

By Matt Furber,
\href{https://www.nytimes3xbfgragh.onion/by/audra-d-s-burch}{Audra D. S.
Burch} and
\href{https://www.nytimes3xbfgragh.onion/by/frances-robles}{Frances
Robles}

\begin{itemize}
\item
  Published May 29, 2020Updated June 10, 2020
\item
  \begin{itemize}
  \item
  \item
  \item
  \item
  \item
  \end{itemize}
\end{itemize}

MINNEAPOLIS --- One was a veteran of the Minneapolis Police Department
who moonlighted as a security guard. The other provided security at a
Salvation Army store, and spent some of his evenings at local clubs,
working as a bouncer.

In the year before their fatal encounter, George Floyd, 46, and the
officer now charged with his death,
\href{https://www.nytimes3xbfgragh.onion/2020/07/18/us/derek-chauvin-george-floyd.html}{Derek
Chauvin}, 44, worked at the same Minneapolis Latin nightclub, both part
of the team responsible for keeping rowdy customers under control.

Their paths crossed for the last time in the waning light of a Memorial
Day evening, outside a corner store known as the best place in town to
find menthol cigarettes. Within an hour, Mr. Floyd was dead, his last
pleas and gasps captured in a horrifically graphic video.

In a move that has since
\href{https://www.nytimes3xbfgragh.onion/2020/05/29/us/minneapolis-protests-george-floyd-death.html}{prompted
protests in cities across the country}, Mr. Chauvin knelt down on Mr.
Floyd behind a police vehicle outside the store. For eight minutes and
46 seconds,
\href{https://www.nytimes3xbfgragh.onion/2020/05/29/us/derek-chauvin-criminal-complaint.html}{according
to a criminal complaint} filed on Friday by the Hennepin County District
Attorney, the police officer pressed his knee into Mr. Floyd's neck in
silence, staring toward the ground as his captive gasped repeatedly that
he could not breathe.

Bystanders waved their cellphones, cursed and pleaded for help, and
still, for two minutes and 53 seconds after Mr. Floyd had stopped
protesting and became unresponsive, the officer continued to kneel.

{[}\href{https://www.nytimes3xbfgragh.onion/2020/05/29/us/derek-chauvin-criminal-complaint.html}{\emph{Read
the criminal complaint against Derek Chauvin}}\emph{.}{]}

The case has become part of a now-familiar history of police violence in
recent years in which African-American men have died in encounters that
were shockingly mundane in their origins --- Eric Garner, who died after
a 2014 arrest in New York for selling cigarettes without tax stamps;
Michael Brown,
\href{https://www.nytimes3xbfgragh.onion/interactive/2014/08/13/us/ferguson-missouri-town-under-siege-after-police-shooting.html}{who
died in an encounter with the police} the same year in Ferguson, Mo.,
after walking in the street instead of using the sidewalk.

Mr. Floyd's case began with a report of a counterfeit \$20 bill that a
storekeeper said he tried to pass to buy cigarettes.

``He died for nothing --- something about a fake bill --- that was
nothing,'' said Jason Polk, 53, a city bus driver and one of a number of
South Minneapolis residents who have expressed outrage over the case.

\href{https://www.nytimes3xbfgragh.onion/2020/05/29/us/tim-walz-minnesota-governor.html}{Gov.
Tim Walz} called the fatal arrest, and the nights of violent protests
that have come after it, ``one of our darkest chapters.''

``Thank God a young person had a camera to video it,'' the governor
said.

\includegraphics{https://static01.graylady3jvrrxbe.onion/images/2020/05/29/us/29MINN-Ticktok-walz/merlin_172958580_1d5e1e2e-2062-4939-bf1d-2b6bec2bf4bb-articleLarge.jpg?quality=75\&auto=webp\&disable=upscale}

With Mr. Chauvin in custody and formally charged with third-degree
murder and second-degree manslaughter, prosecutors must now try to
understand what happened in the chaotic moments before Mr. Floyd was
taken to the Hennepin County Medical Center and pronounced dead at 9:25
p.m.

Accounts from witnesses, cellphone and surveillance video and charging
documents released on Friday tell much of the story of how the
``forgery-in-progress'' arrest unfolded.

Mr. Floyd had been a star football and basketball player in high school,
moving to Minneapolis about five years ago. When he returned to Houston
for his mother's funeral two years ago, he told a cousin that
Minneapolis had come to feel like home. ``He was such a happy guy, he
loved to be around people, loved to dance and he loved Minneapolis,''
said Jovanni Thunstrom, who owned the Conga Latin Bistro where Mr. Floyd
worked security on salsa nights. ``He walked in every day with a smile
on his face.''

It was another club, El Nuevo Rodeo, where both Mr. Floyd and Mr.
Chauvin worked. Maya Santamaria, who sold the club in January, said she
doubted that the two men interacted.

Mr. Floyd worked the occasional weeknight, she said, while Mr. Chauvin
worked security on weekends over the past 17 years. Sometimes during the
club's boisterous ``urban nights,'' she said, when it draws a primarily
African-American clientele, Mr. Chauvin was sometimes overly aggressive
with customers, sometimes using pepper spray, she said.

``I did have words with him on various occasions, when I thought he was
not reacting appropriately based on the situation at hand,'' she said.
``It was like, zero strikes and you're out.''

Mr. Floyd's younger brother, Rodney Floyd, 36, said he was the center of
any room he walked into. ``Always smiling, always somebody you could
talk to and know that you would not be judged.''

The fatal encounter began just before 8 p.m., when Mr. Floyd entered Cup
Foods, a community store run by four brothers, and a store clerk claimed
that he had paid for cigarettes with a counterfeit \$20 bill. The police
got a call from the store at 8:01 p.m.

``Um, someone comes our store and give us fake bills, and we realize it
before he left the store,'' the caller said, according to a transcript
released by the authorities, ``and we ran back outside, they was sitting
on their car.''

The store clerk demanded the cigarettes back. ``But he doesn't want to
do that, and he's sitting on his car `cause he is awfully drunk and he's
not in control of himself,'' the clerk said, according to a transcript
of the call to police. ``He is not acting right.''

The dispatcher pressed for a description, and the caller described the
man as tall, bald, about 6 feet tall.

``Is he white, black, Native, Hispanic, Asian?''

``Something like that,'' the caller replied.

``Which one? White, black, Native, Hispanic, Asian?''

``No, he's a black guy,'' the caller said.

Image

Protesters gathered outside the Cup Foods on Chicago Avenue on Wednesday
in Minneapolis, near the site of the death of George
Floyd.Credit...Stephen Maturen/Getty Images

Not long after, Angel Stately, a regular customer and former employee,
arrived at the store looking for menthol cigarettes. The police were
already outside. Ms. Stately said the clerk, a teenager, was feeling
bad; he had called the police, he told her, only because it was
protocol.

The clerk held up a folded bill and showed it to her. The bill was an
obvious fake, she said. ``The ink was still running,'' she said.

Ms. Stately said she saw an officer approach Mr. Floyd, with his hand at
his gun at his hip.

The charging documents say that officers found Mr. Floyd in a parked
blue car with two passengers. Soon, additional police units arrived and
the officers tried to get Mr. Floyd into a police vehicle. But he
struggled.

``Mr. Floyd did not voluntarily get in the car and struggled with the
officers, intentionally falling down, saying he was not going in the
car, and refusing to stand still,'' according to the charging document.

Even before he was placed on the ground under Mr. Chauvin's knee,
according to the prosecutors' account, while standing outside the car,
Mr. Floyd began saying repeatedly that he could not breathe.

Mr. Chauvin tried to place him in the police car with Officer J.A.
Kueng's help.

At 8:19, Mr. Chauvin pulled Mr. Floyd out of the passenger side of the
squad car. Mr. Floyd hit the ground, face down, handcuffs still on. Mr.
Kueng held Mr. Floyd's back while Officer Thomas Lane held his legs.

Mr. Chauvin lodged his left knee in ``the area of Mr. Floyd's head and
neck,'' the documents said, and Mr. Floyd continued to protest: ``I
can't breathe,'' he said repeatedly.

\includegraphics{https://static01.graylady3jvrrxbe.onion/images/2020/05/26/video/minneapolis-video/minneapolis-video-videoSixteenByNineJumbo1600-v4.jpg}

He called for his mother. He said, ``Please.''

One of the officers dismissed his pleas.

``You are talking fine,'' one officer said, according to the charging
documents.

At least one officer was worried: Mr. Lane asked if the officers should
roll Mr. Floyd over on his side.

``No, staying put where we got him,'' Mr. Chauvin replied.

``I am worried about excited delirium or whatever,'' Mr. Lane said.

``That's why we have him on his stomach,'' Mr. Chauvin responded.

At 8:24 p.m., Mr. Floyd stopped moving.

Mr. Kueng checked Mr. Floyd's right wrist for a pulse. ``I couldn't find
one,'' he said.

Still, none of the officers moved.

At 8:27 p.m., eight minutes and 46 seconds after he had lowered himself
onto Mr. Floyd's neck, Mr. Chauvin finally released his knee.

The medical examiner's office listed the time of death as 9:25 p.m.

Matt Furber reported from Minneapolis, Audra D.S. Burch from Hollywood,
Fla., and Frances Robles from Key West, Fla. Manny Fernandez contributed
reporting from Houston. Susan Beachy contributed research.

Advertisement

\protect\hyperlink{after-bottom}{Continue reading the main story}

\hypertarget{site-index}{%
\subsection{Site Index}\label{site-index}}

\hypertarget{site-information-navigation}{%
\subsection{Site Information
Navigation}\label{site-information-navigation}}

\begin{itemize}
\tightlist
\item
  \href{https://help.nytimes3xbfgragh.onion/hc/en-us/articles/115014792127-Copyright-notice}{©~2020~The
  New York Times Company}
\end{itemize}

\begin{itemize}
\tightlist
\item
  \href{https://www.nytco.com/}{NYTCo}
\item
  \href{https://help.nytimes3xbfgragh.onion/hc/en-us/articles/115015385887-Contact-Us}{Contact
  Us}
\item
  \href{https://www.nytco.com/careers/}{Work with us}
\item
  \href{https://nytmediakit.com/}{Advertise}
\item
  \href{http://www.tbrandstudio.com/}{T Brand Studio}
\item
  \href{https://www.nytimes3xbfgragh.onion/privacy/cookie-policy\#how-do-i-manage-trackers}{Your
  Ad Choices}
\item
  \href{https://www.nytimes3xbfgragh.onion/privacy}{Privacy}
\item
  \href{https://help.nytimes3xbfgragh.onion/hc/en-us/articles/115014893428-Terms-of-service}{Terms
  of Service}
\item
  \href{https://help.nytimes3xbfgragh.onion/hc/en-us/articles/115014893968-Terms-of-sale}{Terms
  of Sale}
\item
  \href{https://spiderbites.nytimes3xbfgragh.onion}{Site Map}
\item
  \href{https://help.nytimes3xbfgragh.onion/hc/en-us}{Help}
\item
  \href{https://www.nytimes3xbfgragh.onion/subscription?campaignId=37WXW}{Subscriptions}
\end{itemize}
