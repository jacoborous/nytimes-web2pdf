Sections

SEARCH

\protect\hyperlink{site-content}{Skip to
content}\protect\hyperlink{site-index}{Skip to site index}

\href{https://www.nytimes3xbfgragh.onion/section/health}{Health}

\href{https://myaccount.nytimes3xbfgragh.onion/auth/login?response_type=cookie\&client_id=vi}{}

\href{https://www.nytimes3xbfgragh.onion/section/todayspaper}{Today's
Paper}

\href{/section/health}{Health}\textbar{}Blaming China for Pandemic,
Trump Says U.S. Will Leave the W.H.O.

\url{https://nyti.ms/3eDhE7D}

\begin{itemize}
\item
\item
\item
\item
\item
\end{itemize}

\hypertarget{the-coronavirus-outbreak}{%
\subsubsection{\texorpdfstring{\href{https://www.nytimes3xbfgragh.onion/news-event/coronavirus?name=styln-coronavirus-national\&region=TOP_BANNER\&variant=undefined\&block=storyline_menu_recirc\&action=click\&pgtype=Article\&impression_id=90041c20-e387-11ea-ab24-c151c34c6c9f}{The
Coronavirus
Outbreak}}{The Coronavirus Outbreak}}\label{the-coronavirus-outbreak}}

\begin{itemize}
\tightlist
\item
  live\href{https://www.nytimes3xbfgragh.onion/2020/08/20/world/coronavirus-covid.html?name=styln-coronavirus-national\&region=TOP_BANNER\&variant=undefined\&block=storyline_menu_recirc\&action=click\&pgtype=Article\&impression_id=90041c21-e387-11ea-ab24-c151c34c6c9f}{Latest
  Updates}
\item
  \href{https://www.nytimes3xbfgragh.onion/interactive/2020/us/coronavirus-us-cases.html?name=styln-coronavirus-national\&region=TOP_BANNER\&variant=undefined\&block=storyline_menu_recirc\&action=click\&pgtype=Article\&impression_id=90041c22-e387-11ea-ab24-c151c34c6c9f}{Maps
  and Cases}
\item
  \href{https://www.nytimes3xbfgragh.onion/interactive/2020/science/coronavirus-vaccine-tracker.html?name=styln-coronavirus-national\&region=TOP_BANNER\&variant=undefined\&block=storyline_menu_recirc\&action=click\&pgtype=Article\&impression_id=90041c23-e387-11ea-ab24-c151c34c6c9f}{Vaccine
  Tracker}
\item
  \href{https://www.nytimes3xbfgragh.onion/2020/08/19/us/colleges-closing-covid.html?name=styln-coronavirus-national\&region=TOP_BANNER\&variant=undefined\&block=storyline_menu_recirc\&action=click\&pgtype=Article\&impression_id=90041c24-e387-11ea-ab24-c151c34c6c9f}{Colleges
  Closing}
\item
  \href{https://www.nytimes3xbfgragh.onion/live/2020/08/20/business/stock-market-today-coronavirus?name=styln-coronavirus-national\&region=TOP_BANNER\&variant=undefined\&block=storyline_menu_recirc\&action=click\&pgtype=Article\&impression_id=90041c25-e387-11ea-ab24-c151c34c6c9f}{Economy}
\end{itemize}

Advertisement

\protect\hyperlink{after-top}{Continue reading the main story}

Supported by

\protect\hyperlink{after-sponsor}{Continue reading the main story}

\hypertarget{blaming-china-for-pandemic-trump-says-us-will-leave-the-who}{%
\section{Blaming China for Pandemic, Trump Says U.S. Will Leave the
W.H.O.}\label{blaming-china-for-pandemic-trump-says-us-will-leave-the-who}}

America's decades-long relationship with the organization has been
instrumental in improving health around the world.

\includegraphics{https://static01.graylady3jvrrxbe.onion/images/2020/05/29/science/29VIRUS-WHO1/merlin_172963686_1a683fa1-2aa6-4ef8-a484-a240e7e82997-articleLarge.jpg?quality=75\&auto=webp\&disable=upscale}

By
\href{https://www.nytimes3xbfgragh.onion/by/donald-g-mcneil-jr}{Donald
G. McNeil Jr.} and
\href{https://www.nytimes3xbfgragh.onion/by/andrew-jacobs}{Andrew
Jacobs}

\begin{itemize}
\item
  May 29, 2020
\item
  \begin{itemize}
  \item
  \item
  \item
  \item
  \item
  \end{itemize}
\end{itemize}

After spending weeks accusing the
\href{https://www.nytimes3xbfgragh.onion/2020/05/29/us/politics/trump-hong-kong-china-WHO.html}{World
Health Organization} of helping the Chinese government cover up the
early days of the coronavirus epidemic in China,
\href{https://www.nytimes3xbfgragh.onion/2020/05/29/us/politics/trump-hong-kong-china-WHO.html}{President
Trump} said on Friday that the United States would terminate its
relationship with the agency.

``The world is now suffering as a result of the malfeasance of the
Chinese government,'' Mr. Trump said in a speech in the Rose Garden.
``Countless lives have been taken, and profound economic hardship has
been inflicted all around the globe.''

In his 10-minute address, Mr. Trump took no responsibility for the
deaths of 100,000 Americans from the virus, instead saying China had
``instigated a global pandemic.''

There is no evidence that the W.H.O. or the government in Beijing hid
the extent of the epidemic in China, and public health experts generally
view Mr. Trump's charges as a way to deflect attention from his
administration's own bungled attempts to respond to the virus's spread
in the United States.

A spokeswoman for the W.H.O. in Geneva, where word of Mr. Trump's
announcement first landed at 9 p.m., said the agency would not have a
response until Saturday.

In April, when he was asked about Mr. Trump's accusation that the W.H.O.
was ``China-centric,'' Tedros Adhanom Ghebreyesus, the organization's
director-general said: ``It is wrong to be any `country-centric.' I am
sure we are not China-centric. The truth is, if we are going to be
blamed, it is right to blame us for being U.S.-centric.''

Public health experts in the United States reacted to Mr. Trump's
announcement with alarm.

``We helped create the W.H.O.,'' said Dr. Thomas Frieden, the former
director of the Centers for Disease Control and Prevention, which has
worked with the organization since its creation in 1948.

``We're part of it --- it is part of the world,'' Dr. Frieden said.
``Turning our back on the W.H.O. makes us and the world less safe.''

The Infectious Diseases Society of America ``stands strongly against
President Trump's decision,'' said Dr. Thomas M. File, its president.
``We will not succeed against this pandemic, or any future outbreak,
unless we stand together, share information and coordinate actions.''

\includegraphics{https://static01.graylady3jvrrxbe.onion/images/2020/05/29/science/29VIRUS-WHO2/merlin_167602749_b1a22139-9559-4b25-a375-656b29e59a8e-articleLarge.jpg?quality=75\&auto=webp\&disable=upscale}

It is not clear whether the president can simply withdraw the United
States from the World Health Organization without Congressional
approval.

\hypertarget{latest-updates-the-coronavirus-outbreak}{%
\section{\texorpdfstring{\href{https://www.nytimes3xbfgragh.onion/2020/08/20/world/coronavirus-covid.html?action=click\&pgtype=Article\&state=default\&region=MAIN_CONTENT_1\&context=storylines_live_updates}{Latest
Updates: The Coronavirus
Outbreak}}{Latest Updates: The Coronavirus Outbreak}}\label{latest-updates-the-coronavirus-outbreak}}

Updated 2020-08-21T07:46:15.883Z

\begin{itemize}
\tightlist
\item
  \href{https://www.nytimes3xbfgragh.onion/2020/08/20/world/coronavirus-covid.html?action=click\&pgtype=Article\&state=default\&region=MAIN_CONTENT_1\&context=storylines_live_updates\#link-68774d88}{Shutdowns,
  warnings and scoldings follow alarming incidents on college campuses.}
\item
  \href{https://www.nytimes3xbfgragh.onion/2020/08/20/world/coronavirus-covid.html?action=click\&pgtype=Article\&state=default\&region=MAIN_CONTENT_1\&context=storylines_live_updates\#link-26b58724}{Biden
  knocks Trump's pandemic response, and outlines a national strategy.}
\item
  \href{https://www.nytimes3xbfgragh.onion/2020/08/20/world/coronavirus-covid.html?action=click\&pgtype=Article\&state=default\&region=MAIN_CONTENT_1\&context=storylines_live_updates\#link-4e542da3}{U.S.
  health agencies announce moves to confront the flu season and
  plummeting child vaccination rates.}
\end{itemize}

\href{https://www.nytimes3xbfgragh.onion/2020/08/20/world/coronavirus-covid.html?action=click\&pgtype=Article\&state=default\&region=MAIN_CONTENT_1\&context=storylines_live_updates}{See
more updates}

More live coverage:
\href{https://www.nytimes3xbfgragh.onion/live/2020/08/20/business/stock-market-today-coronavirus?action=click\&pgtype=Article\&state=default\&region=MAIN_CONTENT_1\&context=storylines_live_updates}{Markets}

``The president can't unilaterally withdraw us,'' said Lawrence O.
Gostin, director of the World Health Organization Collaborating Center
on National \& Global Health Law at The Georgetown University Law
Center.

``It's a nonstarter,'' he added. ``This is literally a whim of one man,
without any consultation with Congress, in the middle of the greatest
health emergency of our lifetime.''

The reaction among Democrats in Congress was swift and negative.

Representative Ami Bera, Democrat of California and the chairman of the
House Foreign Affairs Subcommittee on Asia, the Pacific, and
Nonproliferation, called Mr. Trump's announcement ``shameful and
irresponsible.''

The W.H.O. ``is not a perfect organization,'' he said on Twitter, ``but
leaving will make the United States and the world less safe. President
Trump is ceding American global leadership and handing it over on a
golden platter to China.''

Representative Nita Lowey, Democrat of New York and the chairwoman of
the House Appropriations Committee, said: ``The president wants to blame
everyone else --- the W.H.O., Twitter, the media --- when his own
shortcomings as a leader are contributing to harm and further dividing
us here at home and among global partners.''

The administration's response to the emergency has been fumbling and
inadequate, many public health experts say, especially when compared to
China's.

The coronavirus has been the leading cause of death in the United States
since mid-April, killing roughly 100,000 citizens to date. By
comparison, only 4,600 Chinese citizens have died of the infection.

About 20,000 Americans are infected each day, while China virtually
ended its outbreak by April. On most days China records zero to five new
infections, usually in travelers from abroad.

The W.H.O. was founded in 1948 as part of the postwar creation of the
United Nations and is the world's premier global health organization.
Mr. Trump supported and generously funded the organization as it fought
an Ebola outbreak in Africa for three years, but abruptly turned on the
W.H.O. a few weeks ago, when he began accusing the organization of doing
too little to warn the world of the spread of the coronavirus.

In fact, the agency issued its first
\href{https://twitter.com/WHO/status/1213523866703814656?s=20}{alarm on
Jan. 4}, just five days after the local health department of Wuhan ---
at the time a city few non-Chinese had even heard of --- announced a
cluster of 27 cases of an unusual pneumonia at a local seafood market.

The W.H.O. followed up with
\href{https://www.who.int/csr/don/05-january-2020-pneumonia-of-unkown-cause-china/en/}{a
detailed report the next day}. On Jan. 20 and 21, a W.H.O. field team
visited China and
\href{https://www.who.int/china/news/detail/22-01-2020-field-visit-wuhan-china-jan-2020}{reported
that} there could be human-to-human transmission of the new
pneumonia-causing virus.

Almost simultaneously, China's leading epidemiologist, who had just
completed his own investigation on behalf of the Beijing government,
\href{https://abcnews.go.com/Health/human-human-transmission-coronavirus-reported-china/story?id=68403105}{confirmed
during a Jan. 20 interview} on state television that transmission to
doctors was occurring in Wuhan, although he said on
\href{https://www.cnn.com/videos/world/2020/05/19/chinas-dr-fauci-zhong-nanshan-coronavirus-intv-culver-pkg-2-intl-hnk-vpx.cnn}{a
recent interview with CNN} that local officials had lied about it and
even tried to mislead him.

Within three days, Beijing had shut off all travel out of Wuhan. Mr.
Trump did not
\href{https://www.nytimes3xbfgragh.onion/2020/01/31/business/china-travel-coronavirus.html}{order
any restrictions on travel from China} until Jan. 31.

The United States has been by far the W.H.O.'s largest donor since its
inception. The budget for the W.H.O.
\href{http://open.who.int/2018-19/budget-and-financing/gpw-overview}{is
about \$6 billion}, which comes from member countries around the world.
In 2019, the last year for which figures were available, the United
States contributed about \$553 million.

\href{https://www.nytimes3xbfgragh.onion/news-event/coronavirus?action=click\&pgtype=Article\&state=default\&region=MAIN_CONTENT_3\&context=storylines_faq}{}

\hypertarget{the-coronavirus-outbreak-}{%
\subsubsection{The Coronavirus Outbreak
›}\label{the-coronavirus-outbreak-}}

\hypertarget{frequently-asked-questions}{%
\paragraph{Frequently Asked
Questions}\label{frequently-asked-questions}}

Updated August 17, 2020

\begin{itemize}
\item ~
  \hypertarget{why-does-standing-six-feet-away-from-others-help}{%
  \paragraph{Why does standing six feet away from others
  help?}\label{why-does-standing-six-feet-away-from-others-help}}

  \begin{itemize}
  \tightlist
  \item
    The coronavirus spreads primarily through droplets from your mouth
    and nose, especially when you cough or sneeze. The C.D.C., one of
    the organizations using that measure,
    \href{https://www.nytimes3xbfgragh.onion/2020/04/14/health/coronavirus-six-feet.html?action=click\&pgtype=Article\&state=default\&region=MAIN_CONTENT_3\&context=storylines_faq}{bases
    its recommendation of six feet} on the idea that most large droplets
    that people expel when they cough or sneeze will fall to the ground
    within six feet. But six feet has never been a magic number that
    guarantees complete protection. Sneezes, for instance, can launch
    droplets a lot farther than six feet,
    \href{https://jamanetwork.com/journals/jama/fullarticle/2763852}{according
    to a recent study}. It's a rule of thumb: You should be safest
    standing six feet apart outside, especially when it's windy. But
    keep a mask on at all times, even when you think you're far enough
    apart.
  \end{itemize}
\item ~
  \hypertarget{i-have-antibodies-am-i-now-immune}{%
  \paragraph{I have antibodies. Am I now
  immune?}\label{i-have-antibodies-am-i-now-immune}}

  \begin{itemize}
  \tightlist
  \item
    As of right
    now,\href{https://www.nytimes3xbfgragh.onion/2020/07/22/health/covid-antibodies-herd-immunity.html?action=click\&pgtype=Article\&state=default\&region=MAIN_CONTENT_3\&context=storylines_faq}{that
    seems likely, for at least several months.} There have been
    frightening accounts of people suffering what seems to be a second
    bout of Covid-19. But experts say these patients may have a
    drawn-out course of infection, with the virus taking a slow toll
    weeks to months after initial exposure. People infected with the
    coronavirus typically
    \href{https://www.nature.com/articles/s41586-020-2456-9}{produce}
    immune molecules called antibodies, which are
    \href{https://www.nytimes3xbfgragh.onion/2020/05/07/health/coronavirus-antibody-prevalence.html?action=click\&pgtype=Article\&state=default\&region=MAIN_CONTENT_3\&context=storylines_faq}{protective
    proteins made in response to an
    infection}\href{https://www.nytimes3xbfgragh.onion/2020/05/07/health/coronavirus-antibody-prevalence.html?action=click\&pgtype=Article\&state=default\&region=MAIN_CONTENT_3\&context=storylines_faq}{.
    These antibodies may} last in the body
    \href{https://www.nature.com/articles/s41591-020-0965-6}{only two to
    three months}, which may seem worrisome, but that's perfectly normal
    after an acute infection subsides, said Dr. Michael Mina, an
    immunologist at Harvard University. It may be possible to get the
    coronavirus again, but it's highly unlikely that it would be
    possible in a short window of time from initial infection or make
    people sicker the second time.
  \end{itemize}
\item ~
  \hypertarget{im-a-small-business-owner-can-i-get-relief}{%
  \paragraph{I'm a small-business owner. Can I get
  relief?}\label{im-a-small-business-owner-can-i-get-relief}}

  \begin{itemize}
  \tightlist
  \item
    The
    \href{https://www.nytimes3xbfgragh.onion/article/small-business-loans-stimulus-grants-freelancers-coronavirus.html?action=click\&pgtype=Article\&state=default\&region=MAIN_CONTENT_3\&context=storylines_faq}{stimulus
    bills enacted in March} offer help for the millions of American
    small businesses. Those eligible for aid are businesses and
    nonprofit organizations with fewer than 500 workers, including sole
    proprietorships, independent contractors and freelancers. Some
    larger companies in some industries are also eligible. The help
    being offered, which is being managed by the Small Business
    Administration, includes the Paycheck Protection Program and the
    Economic Injury Disaster Loan program. But lots of folks have
    \href{https://www.nytimes3xbfgragh.onion/interactive/2020/05/07/business/small-business-loans-coronavirus.html?action=click\&pgtype=Article\&state=default\&region=MAIN_CONTENT_3\&context=storylines_faq}{not
    yet seen payouts.} Even those who have received help are confused:
    The rules are draconian, and some are stuck sitting on
    \href{https://www.nytimes3xbfgragh.onion/2020/05/02/business/economy/loans-coronavirus-small-business.html?action=click\&pgtype=Article\&state=default\&region=MAIN_CONTENT_3\&context=storylines_faq}{money
    they don't know how to use.} Many small-business owners are getting
    less than they expected or
    \href{https://www.nytimes3xbfgragh.onion/2020/06/10/business/Small-business-loans-ppp.html?action=click\&pgtype=Article\&state=default\&region=MAIN_CONTENT_3\&context=storylines_faq}{not
    hearing anything at all.}
  \end{itemize}
\item ~
  \hypertarget{what-are-my-rights-if-i-am-worried-about-going-back-to-work}{%
  \paragraph{What are my rights if I am worried about going back to
  work?}\label{what-are-my-rights-if-i-am-worried-about-going-back-to-work}}

  \begin{itemize}
  \tightlist
  \item
    Employers have to provide
    \href{https://www.osha.gov/SLTC/covid-19/standards.html}{a safe
    workplace} with policies that protect everyone equally.
    \href{https://www.nytimes3xbfgragh.onion/article/coronavirus-money-unemployment.html?action=click\&pgtype=Article\&state=default\&region=MAIN_CONTENT_3\&context=storylines_faq}{And
    if one of your co-workers tests positive for the coronavirus, the
    C.D.C.} has said that
    \href{https://www.cdc.gov/coronavirus/2019-ncov/community/guidance-business-response.html}{employers
    should tell their employees} -\/- without giving you the sick
    employee's name -\/- that they may have been exposed to the virus.
  \end{itemize}
\item ~
  \hypertarget{what-is-school-going-to-look-like-in-september}{%
  \paragraph{What is school going to look like in
  September?}\label{what-is-school-going-to-look-like-in-september}}

  \begin{itemize}
  \tightlist
  \item
    It is unlikely that many schools will return to a normal schedule
    this fall, requiring the grind of
    \href{https://www.nytimes3xbfgragh.onion/2020/06/05/us/coronavirus-education-lost-learning.html?action=click\&pgtype=Article\&state=default\&region=MAIN_CONTENT_3\&context=storylines_faq}{online
    learning},
    \href{https://www.nytimes3xbfgragh.onion/2020/05/29/us/coronavirus-child-care-centers.html?action=click\&pgtype=Article\&state=default\&region=MAIN_CONTENT_3\&context=storylines_faq}{makeshift
    child care} and
    \href{https://www.nytimes3xbfgragh.onion/2020/06/03/business/economy/coronavirus-working-women.html?action=click\&pgtype=Article\&state=default\&region=MAIN_CONTENT_3\&context=storylines_faq}{stunted
    workdays} to continue. California's two largest public school
    districts --- Los Angeles and San Diego --- said on July 13, that
    \href{https://www.nytimes3xbfgragh.onion/2020/07/13/us/lausd-san-diego-school-reopening.html?action=click\&pgtype=Article\&state=default\&region=MAIN_CONTENT_3\&context=storylines_faq}{instruction
    will be remote-only in the fall}, citing concerns that surging
    coronavirus infections in their areas pose too dire a risk for
    students and teachers. Together, the two districts enroll some
    825,000 students. They are the largest in the country so far to
    abandon plans for even a partial physical return to classrooms when
    they reopen in August. For other districts, the solution won't be an
    all-or-nothing approach.
    \href{https://bioethics.jhu.edu/research-and-outreach/projects/eschool-initiative/school-policy-tracker/}{Many
    systems}, including the nation's largest, New York City, are
    devising
    \href{https://www.nytimes3xbfgragh.onion/2020/06/26/us/coronavirus-schools-reopen-fall.html?action=click\&pgtype=Article\&state=default\&region=MAIN_CONTENT_3\&context=storylines_faq}{hybrid
    plans} that involve spending some days in classrooms and other days
    online. There's no national policy on this yet, so check with your
    municipal school system regularly to see what is happening in your
    community.
  \end{itemize}
\end{itemize}

The American government and private donors, including the Bill and
Melinda Gates Foundation, Bloomberg Philanthropies and Rotary
International, have wielded enormous influence on W.H.O. policies.

For example, although the war on smallpox that was begun in the 1960s
was at first largely a Soviet initiative, the W.H.O. chose American
doctors, including Dr. William Foege and Dr. Donald A. Henderson, to
lead the global campaign.

The agency also chose American-made vaccines over Soviet ones for the
war on polio.

For many years, the American government, working on behalf of the
Western pharmaceutical industry, pressured the W.H.O. not to publicly
fight for lower drug prices that might threaten the patent monopolies of
American companies.

That changed in the early 2000s, when many American companies began
sub-licensing their patents and technology to generics makers in India
and elsewhere.

No American has ever been director-general of the W.H.O., but that is
because of a decades-old understanding that the World Bank and the
United Nations Children's Fund would always be run by Americans, while
the leadership of some other U.N. agencies, including the W.H.O., would
be taken in turn by other nations.

Image

The W.H.O.'s headquarters in Geneva in January.Credit...Salvatore Di
Nolfi/EPA, via Shutterstock

The C.D.C. and many other branches of the American government have
worked with the W.H.O. for decades. Along with the C.D.C., doctors from
the American military and even the Army's 82nd Airborne Division worked
in cooperation with the W.H.O. to fight the 2014 West African Ebola
epidemic --- partly in an effort to keep it from reaching the United
States.

The W.H.O. provides essential diplomatic cover when American government
agencies work in foreign countries. All countries that belong to the
U.N. are also de facto members of the ruling body of the W.H.O.

Asked for comment Friday, a C.D.C. official said the agency did not know
what impact the announcement would have on its 72-year-old working
relationship with the W.H.O., and referred all further questions to the
White House.

As he began facing harsh questions about his handling of the disease
here, Mr. Trump swiftly diverted the blame to the W.H.O., threatening in
a letter earlier this month to pull funding if it did not ``commit to
major substantive improvements in the next 30 days.''

In fact, under Dr. Tedros, the agency has been in the middle of major
reforms for several years, focusing more of its attention on pandemics
and less on the causes championed by wealthy donors, including tobacco,
lung cancer and obesity.

Last month at the World Health Assembly --- the annual meeting of the
health ministers of all U.N. member nations that serves as the agency's
governing board --- other member states rebuffed Mr. Trump's demands.
They voted instead to conduct an ``impartial, independent'' examination
of the W.H.O.'s pandemic response.

Mr. Trump's Rose Garden address came as cities across the United States
were convulsing with protests over recent cases of police brutality
against black Americans.

He did not take questions after delivering his speech, even as assembled
reporters shouted for him to address protests in Minneapolis.

Advertisement

\protect\hyperlink{after-bottom}{Continue reading the main story}

\hypertarget{site-index}{%
\subsection{Site Index}\label{site-index}}

\hypertarget{site-information-navigation}{%
\subsection{Site Information
Navigation}\label{site-information-navigation}}

\begin{itemize}
\tightlist
\item
  \href{https://help.nytimes3xbfgragh.onion/hc/en-us/articles/115014792127-Copyright-notice}{©~2020~The
  New York Times Company}
\end{itemize}

\begin{itemize}
\tightlist
\item
  \href{https://www.nytco.com/}{NYTCo}
\item
  \href{https://help.nytimes3xbfgragh.onion/hc/en-us/articles/115015385887-Contact-Us}{Contact
  Us}
\item
  \href{https://www.nytco.com/careers/}{Work with us}
\item
  \href{https://nytmediakit.com/}{Advertise}
\item
  \href{http://www.tbrandstudio.com/}{T Brand Studio}
\item
  \href{https://www.nytimes3xbfgragh.onion/privacy/cookie-policy\#how-do-i-manage-trackers}{Your
  Ad Choices}
\item
  \href{https://www.nytimes3xbfgragh.onion/privacy}{Privacy}
\item
  \href{https://help.nytimes3xbfgragh.onion/hc/en-us/articles/115014893428-Terms-of-service}{Terms
  of Service}
\item
  \href{https://help.nytimes3xbfgragh.onion/hc/en-us/articles/115014893968-Terms-of-sale}{Terms
  of Sale}
\item
  \href{https://spiderbites.nytimes3xbfgragh.onion}{Site Map}
\item
  \href{https://help.nytimes3xbfgragh.onion/hc/en-us}{Help}
\item
  \href{https://www.nytimes3xbfgragh.onion/subscription?campaignId=37WXW}{Subscriptions}
\end{itemize}
