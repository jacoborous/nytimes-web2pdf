Sections

SEARCH

\protect\hyperlink{site-content}{Skip to
content}\protect\hyperlink{site-index}{Skip to site index}

\href{https://myaccount.nytimes3xbfgragh.onion/auth/login?response_type=cookie\&client_id=vi}{}

\href{https://www.nytimes3xbfgragh.onion/section/todayspaper}{Today's
Paper}

\href{/section/opinion}{Opinion}\textbar{}China and the Rhineland Moment

\url{https://nyti.ms/2TMlPG7}

\begin{itemize}
\item
\item
\item
\item
\item
\item
\end{itemize}

Advertisement

\protect\hyperlink{after-top}{Continue reading the main story}

\href{/section/opinion}{Opinion}

Supported by

\protect\hyperlink{after-sponsor}{Continue reading the main story}

\hypertarget{china-and-the-rhineland-moment}{%
\section{China and the Rhineland
Moment}\label{china-and-the-rhineland-moment}}

America and its allies must not simply accept Beijing's aggression.

\href{https://www.nytimes3xbfgragh.onion/by/bret-stephens}{\includegraphics{https://static01.graylady3jvrrxbe.onion/images/2017/08/27/insider/bretstephens/bretstephens-thumbLarge-v6.png}}

By \href{https://www.nytimes3xbfgragh.onion/by/bret-stephens}{Bret
Stephens}

Opinion Columnist

\begin{itemize}
\item
  May 29, 2020
\item
  \begin{itemize}
  \item
  \item
  \item
  \item
  \item
  \item
  \end{itemize}
\end{itemize}

\includegraphics{https://static01.graylady3jvrrxbe.onion/images/2020/06/01/opinion/01stephensWeb/merlin_172872576_a8cb27c5-3a49-44f2-b06a-0ae836735c35-articleLarge.jpg?quality=75\&auto=webp\&disable=upscale}

\href{https://cn.nytimes3xbfgragh.onion/opinion/20200604/china-hong-kong/}{阅读简体中文版}\href{https://cn.nytimes3xbfgragh.onion/opinion/20200604/china-hong-kong/zh-hant/}{閱讀繁體中文版}

Great struggles between great powers tend to have a tipping point. It's
the moment when the irreconcilability of differences becomes obvious to
nearly everyone.

In 1911 Germany sparked an international crisis when it sent a gunboat
into the Moroccan port of Agadir and, as Winston Churchill wrote in his
history of the First World War, ``all the alarm bells throughout Europe
began immediately to quiver.'' In 1936 Germany provoked another crisis
when it marched troops into the Rhineland, in flagrant breach of its
treaty obligations. In 1946, the Soviet Union made it obvious it had no
intention of honoring democratic principles in Central Europe, and
Churchill was left to warn that ``an iron curtain has descended across
the Continent.''

Analogies between these past episodes and China's decision this week
draft a new national security law on Hong Kong aren't perfect. Hong Kong
is a Chinese port, not a faraway foreign one. Hong Kong's people have
ferociously resisted Beijing's efforts to impose control, unlike the
Rhineland Germans who welcomed Berlin's. And the curtailment of freedom
that awaits Hong Kong is nothing like the totalitarian tyranny that
Joseph Stalin imposed on Warsaw, Budapest and other cities.

But the analogies aren't inapt, either. Beijing has spent the better
part of 20 years subverting its promises to preserve Hong Kong's
democratic institutions. Now it is moving to quash what remains of the
city's civic freedoms through a forthcoming law that allows the
government to punish speech as subversion and protest as sedition. The
concept of ``one country, two systems,'' was supposed to last at least
until 2047 under the terms of the 1984 Sino-British Joint Declaration.
Now China's rulers have been openly violating that treaty, much as
Germany openly violated the treaties of Locarno and Versailles.

And again, alarm bells quiver.

For years, Donald Trump's comments on China have swung between
\href{https://www.politico.eu/article/trump-seizes-a-new-cudgel-to-bash-china-taiwan/}{the
truculent} and the
\href{https://www.politico.com/news/2020/04/15/trump-china-coronavirus-188736}{obsequious}.
But beneath the president's mental foam, the administration has
undertaken a sober rethink of the U.S. strategic approach to China, the
outlines of which are described in a new
\href{https://www.whitehouse.gov/wp-content/uploads/2020/05/U.S.-Strategic-Approach-to-The-Peoples-Republic-of-China-Report-5.24v1.pdf}{interagency
document} quietly released by the White House last week.

Gone from this new vision are the platitudes about encouraging China's
``peaceful rise'' as a ``responsible stakeholder'' in a ``rules-based
order.'' Instead, Beijing is described, accurately, as a habitual and
aggressive violator of that order --- a domestic tyrant, international
bully and economic bandit that systematically robs companies of their
intellectual property, countries of their sovereign authorities, and its
own people of their natural rights.

``Beijing has repeatedly demonstrated that it does not offer compromises
in response to American displays of goodwill, and that its actions are
not constrained by its prior commitments,'' the report reads. ``We
acknowledge and respond in kind to Beijing's transactional approach with
timely incentives and costs, or credible threats thereof.''

A critic might note that this description of China's behavior sounds a
lot like Trump's. Sort of, except that the comparison trivializes the
scale of China's abuses and neglects the breadth and longevity of its
challenge. A Biden administration will be confronted with the same
unpleasant facts about a geopolitical adversary that seeks not only to
dominate its region but also dethrone liberal democracy as the dominant
political model of the 21st century.

All of which makes the Hong Kong crisis so consequential. Beijing almost
certainly chose this moment to strike because it calculated that a world
straining under the weight of a pandemic and a depression lacked the
will and attention to react. On Friday, Trump said he would strip Hong
Kong of its privileged commercial and legal ties to the U.S. But that
punishes the people of Hong Kong at least as much as it does their
rulers in Beijing.

What's a better course for the U.S.? A few ideas:

Sanction Chinese officials engaged in human-rights abuses in Hong Kong
under the
\href{https://www.congress.gov/bill/114th-congress/senate-bill/284/text/es}{Global
Magnitsky Act}. Upgrade relations with Taiwan and increase arms sales,
including top-shelf weapons' systems such as the F-35 and the Navy's
future frigate. Re-enter the Trans-Pacific Partnership agreement as a
counter to China's economic influence. (This won't happen in a Trump
administration, but should in a Biden one.) Publicly press all G-7
countries to stop doing business with telecom-giant Huawei as a
meaningful response to the Hong Kong law.

One other idea is now being explored by Britain, the former colonial
power. Give every Hong Kong person an opportunity to easily obtain a
U.K. residency card, even a passport. As Tom Tugendhat, the chair of
Parliament's Foreign Affairs Committee and founder of its China Research
Group, told me on Thursday, doing so would ``right a wrong done when the
U.K. removed the status in the 1980s. After a century of rule, Britain
has obligations.'' A future American president who believes in the value
of immigration could join that effort, in the same way we helped
Hungarian refugees and Vietnamese boat people.

If all this and more were announced now, it might persuade Beijing to
pull back from the brink. In the meantime, think of this as our
Rhineland moment with China --- and remember what happened the last time
the free world looked aggression in the eye, and blinked.

\emph{The Times is committed to publishing}
\href{https://www.nytimes3xbfgragh.onion/2019/01/31/opinion/letters/letters-to-editor-new-york-times-women.html}{\emph{a
diversity of letters}} \emph{to the editor. We'd like to hear what you
think about this or any of our articles. Here are some}
\href{https://help.nytimes3xbfgragh.onion/hc/en-us/articles/115014925288-How-to-submit-a-letter-to-the-editor}{\emph{tips}}\emph{.
And here's our email:}
\href{mailto:letters@NYTimes.com}{\emph{letters@NYTimes.com}}\emph{.}

\emph{Follow The New York Times Opinion section on}
\href{https://www.facebookcorewwwi.onion/nytopinion}{\emph{Facebook}}\emph{,}
\href{http://twitter.com/NYTOpinion}{\emph{Twitter (@NYTopinion)}}
\emph{and}
\href{https://www.instagram.com/nytopinion/}{\emph{Instagram}}\emph{.}

Advertisement

\protect\hyperlink{after-bottom}{Continue reading the main story}

\hypertarget{site-index}{%
\subsection{Site Index}\label{site-index}}

\hypertarget{site-information-navigation}{%
\subsection{Site Information
Navigation}\label{site-information-navigation}}

\begin{itemize}
\tightlist
\item
  \href{https://help.nytimes3xbfgragh.onion/hc/en-us/articles/115014792127-Copyright-notice}{©~2020~The
  New York Times Company}
\end{itemize}

\begin{itemize}
\tightlist
\item
  \href{https://www.nytco.com/}{NYTCo}
\item
  \href{https://help.nytimes3xbfgragh.onion/hc/en-us/articles/115015385887-Contact-Us}{Contact
  Us}
\item
  \href{https://www.nytco.com/careers/}{Work with us}
\item
  \href{https://nytmediakit.com/}{Advertise}
\item
  \href{http://www.tbrandstudio.com/}{T Brand Studio}
\item
  \href{https://www.nytimes3xbfgragh.onion/privacy/cookie-policy\#how-do-i-manage-trackers}{Your
  Ad Choices}
\item
  \href{https://www.nytimes3xbfgragh.onion/privacy}{Privacy}
\item
  \href{https://help.nytimes3xbfgragh.onion/hc/en-us/articles/115014893428-Terms-of-service}{Terms
  of Service}
\item
  \href{https://help.nytimes3xbfgragh.onion/hc/en-us/articles/115014893968-Terms-of-sale}{Terms
  of Sale}
\item
  \href{https://spiderbites.nytimes3xbfgragh.onion}{Site Map}
\item
  \href{https://help.nytimes3xbfgragh.onion/hc/en-us}{Help}
\item
  \href{https://www.nytimes3xbfgragh.onion/subscription?campaignId=37WXW}{Subscriptions}
\end{itemize}
