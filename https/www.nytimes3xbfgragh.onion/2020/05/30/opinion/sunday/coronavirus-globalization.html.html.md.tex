\href{/section/opinion/sunday}{Sunday Review}\textbar{}How We Broke the
World

\url{https://nyti.ms/2BcnC0x}

\begin{itemize}
\item
\item
\item
\item
\item
\item
\end{itemize}

\includegraphics{https://static01.graylady3jvrrxbe.onion/images/2020/05/31/opinion/sunday/31friedman-top/31friedman-top-articleLarge.jpg?quality=75\&auto=webp\&disable=upscale}

Sections

\protect\hyperlink{site-content}{Skip to
content}\protect\hyperlink{site-index}{Skip to site index}

\href{/section/opinion}{Opinion}

\hypertarget{how-we-broke-the-world}{%
\section{How We Broke the World}\label{how-we-broke-the-world}}

Greed and globalization set us up for disaster.

Credit...Tyler Comrie

Supported by

\protect\hyperlink{after-sponsor}{Continue reading the main story}

\href{https://www.nytimes3xbfgragh.onion/by/thomas-l-friedman}{\includegraphics{https://static01.graylady3jvrrxbe.onion/images/2018/04/02/opinion/thomas-l-friedman/thomas-l-friedman-thumbLarge.png}}

By \href{https://www.nytimes3xbfgragh.onion/by/thomas-l-friedman}{Thomas
L. Friedman}

Opinion Columnist

\begin{itemize}
\item
  May 30, 2020
\item
  \begin{itemize}
  \item
  \item
  \item
  \item
  \item
  \item
  \end{itemize}
\end{itemize}

If recent weeks have shown us anything, it's that the world is not just
flat. It's fragile.

And we're the ones who made it that way with our own hands. Just look
around. Over the past 20 years, we've been steadily removing man-made
and natural buffers, redundancies, regulations and norms that provide
resilience and protection when big systems --- be they ecological,
geopolitical or financial --- get stressed. We've been recklessly
removing these buffers out of an obsession with short-term efficiency
and growth, or without thinking at all.

At the same time, we've been behaving in extreme ways --- pushing
against, and breaching, common-sense political, financial and planetary
boundaries.

And, all the while, we've taken the world technologically from connected
to interconnected to interdependent --- by removing more friction and
installing more grease in global markets, telecommunications systems,
the internet and travel. In doing so, we've made globalization faster,
deeper, cheaper and tighter than ever before. Who knew that there were
regular direct flights from Wuhan, China, to America?

Put all three of these trends together and what you have is a world more
easily prone to shocks and extreme behaviors --- but with fewer buffers
to cushion those shocks --- and many more networked companies and people
to convey them globally.

This, of course, was revealed clearly in the latest world-spanning
crisis --- the
\href{https://www.nytimes3xbfgragh.onion/2020/06/01/health/coronavirus-mysteries.html}{coronavirus}pandemic.
But this trend of more frequent destabilizing crises has been building
over the past 20 years: 9/11, the Great Recession of 2008, Covid-19 and
climate change. Pandemics are no longer just biological --- they are now
geopolitical, financial and atmospheric, too. And we will suffer
increasing consequences unless **** we start behaving differently and
treating Mother Earth differently.

Note the pattern: Before each crisis I mentioned, we first experienced
what could be called a ``mild'' heart attack, alerting us that we had
gone to extremes and stripped away buffers that had protected us from
catastrophic failure. In each case, though, we did not take that warning
seriously enough --- and in each case the result was a full global
coronary.

``We created globalized networks because they could make us more
efficient and productive and our lives more convenient,'' explained
Gautam Mukunda, the author of ``Indispensable: When Leaders Really
Matter.'' ``But when you steadily remove their buffers, backup
capacities and surge protectors in pursuit of short-term efficiency or
just greed, you ensure that these systems are not only less resistant to
shocks, but that we spread those shocks everywhere.''

\includegraphics{https://static01.graylady3jvrrxbe.onion/images/2020/05/31/opinion/29friedman1/29friedman1-articleLarge.jpg?quality=75\&auto=webp\&disable=upscale}

\hypertarget{sept-11-2001}{%
\subsubsection{Sept. 11, 2001}\label{sept-11-2001}}

Let's start with 9/11. You could view Al Qaeda and its leader, Osama bin
Laden, as political pathogens that emerged out of the Middle East after
1979. ``Islam lost its brakes in 1979'' --- its resistance to extremism
was badly compromised --- said Mamoun Fandy, an expert on Arab politics.

That was the year that Saudi Arabia lurched backward, after Islamist
extremists took over the Grand Mosque in Mecca and an Islamic revolution
in Iran brought Ayatollah Ruhollah Khomeini to power. Those events set
up a competition between Shiite Iran and Sunni Saudi Arabia over who was
the real leader of the Muslim world. That battle coincided with a surge
in oil prices that gave both fundamentalist regimes the resources to
propagate their brands of puritanical Islam, through mosques and
schools, across the globe.

In doing so, they together weakened any emerging trends toward religious
and political pluralism --- and strengthened austere fundamentalism and
its violent fringes.

Remember: The Muslim world was probably at its most influential,
culturally, scientifically and economically, in the Middle Ages, when it
was a rich and diverse polyculture in Moorish Spain.

Diverse ecosystems, in nature and in politics, are always more resilient
than monocultures. Monocultures in agriculture are enormously
susceptible to disease --- one virus or germ can wipe out an entire
crop. Monocultures in politics are enormously susceptible to diseased
ideas.

Thanks to Iran and Saudi Arabia, the Arab-Muslim world became much more
of a monoculture after 1979. And the idea that violent Islamist jihadism
would be the engine of Islam's revival --- and that purging the region
of foreign influences, particularly American, was its necessary first
step --- gained much wider currency.

This ideological pathogen spread --- through mosques, cassette tapes and
then the internet --- to Pakistan, North Africa, Europe, India and
Indonesia.

The warning bell that this idea could destabilize even America rang ****
on Feb. 26, 1993, at 12:18 p.m., when a rental van packed with
explosives blew up in the parking garage below the 1 World Trade Center
building in Manhattan. The bomb failed to bring down the building as
intended, but it badly damaged the main structure, killing six people
and injuring more than 1,000.

The mastermind of the attack, Ramzi Ahmed Yousef, a Pakistani, later
told F.B.I. agents that his only regret was that the 110-story tower did
not collapse into its twin and kill thousands.

What happened next we all know: The direct hits on both twin towers on
Sept. 11, 2001, which set off a global economic and geopolitical crisis
that ended with the United States spending several trillion dollars
trying to immunize America against violent Islamic extremism --- via a
massive government-directed surveillance system, renditions and airport
metal detectors --- and by invading the Middle East.

The United States and its allies toppled the dictators in Iraq and
Afghanistan, hoping to stimulate more political pluralism, gender
pluralism and religious and educational pluralism --- antibodies to
fanaticism and authoritarianism. Unfortunately, we didn't really know
how to do this in such distant lands, and we botched it; the natural
pluralistic antibodies in the region also proved to be weak.

Either way --- as in biology, so, too, in geopolitics --- the virus of
Al Qaeda mutated, picking up new elements from its hosts in Iraq and
Afghanistan. As a result, violent Islamic extremism became even more
virulent, thanks to subtle changes in its genome that transformed it
into ISIS, or the Islamic State.

This emergence of ISIS, and parallel mutations in the Taliban, forced
the United States to remain in the area to just manage the outbreaks,
but nothing more.

Image

Traders on the floor of the New York Stock Exchange in
2008.Credit...Mario Tama/Getty Images

\hypertarget{the-great-recession}{%
\subsubsection{The Great Recession}\label{the-great-recession}}

The 2008 global banking crisis played out in similar ways. The warning
was delivered by a virus known by the initials LTCM --- Long-Term
Capital Management.

LTCM was a hedge fund set up in 1994 by the investment banker John
Meriweather, who assembled a team of mathematicians, industry veterans
and two Nobel Prize winners. The fund used mathematical models to
predict prices and tons of leverage to amplify its founding capital of
\$1.25 billion to make massive, and massively profitable, arbitrage
bets.

It all worked --- until it didn't.

``In August 1998,''
\href{https://www.businessinsider.com/the-fall-of-long-term-capital-management-2014-7}{recalled
Business Insider}, ``Russia defaulted on its debt. Three days later,
markets all over the world started sinking. Investors began pulling out
left and right. Swap spreads were at unbelievable levels. Everything was
plummeting. In one day, Long-Term lost \$553 million, 15 percent of its
capital. In one month it lost almost \$2 billion.''

Hedge funds lose money all the time, default and go extinct. But LTCM
was different.

The firm had leveraged its bets with so much capital from so many
different big global banks --- with no trading transparency, so none of
its counterparties had a picture of LTCM's total exposure --- that if it
were allowed to go bankrupt and default, it would have exacted huge
losses on dozens of investments houses and banks on Wall Street and
abroad.

More than \$1 trillion was at risk. It took a \$3.65 billion bailout
package from the Federal Reserve to create herd immunity from LTCM for
the Wall Street bulls.

The crisis was contained and the lesson was clear: Don't let anyone make
such big, and in some ways extreme, bets with such tremendous leverage
in a global banking system where there is no transparency as to how much
a single player has borrowed from many different sources.

A decade later, the lesson was forgotten, and we got the full financial
disaster of 2008.

This time we were all in the casino. There were four main financial
vehicles (that became financial pathogens) that interacted to create the
global crisis of 2008. They were called subprime mortgages, adjustable
rate mortgages (ARMs), commercial mortgage-backed securities (C.M.B.S.)
and collateralized debt obligations (C.D.O.s).

Banks and less-regulated financial institutions engaged in extremely
reckless subprime and adjustable rate mortgage lending, and then they
and others bundled these mortgages into mortgage-backed securities.
Meanwhile, rating agencies classified these bonds as much less risky
than they really were.

The whole system depended on housing prices endlessly rising. When the
housing bubble burst --- and many homeowners could not pay their
mortgages --- the financial contagion infected huge numbers of global
banks and insurance companies, not to mention millions of mom-and-pops.

We had breached the boundaries of financial common sense. With the
world's financial system more hyper-connected and leveraged than ever,
only huge bailouts by central banks prevented a full-on economic
pandemic and depression caused by failing commercial banks and stock
markets.

In 2010, we tried to immunize the banking system against a repeat with
the Dodd-Frank Wall Street Reform and Consumer Protection Act in America
and with the Basel III new capital and liquidity standards adopted by
banking systems around the world. But ever since then, and particularly
under the Trump administration, financial services companies have been
lobbying, often successfully, to weaken these buffers, threatening a new
financial contagion down the road.

This one could be even more dangerous because computerized trading now
makes up more than half of stock trading volume globally. These traders
use algorithms and computer networks that process data at a thousandth
or millionth of a second to buy and sell stocks, bonds or commodities.

Alas, there is no herd immunity to greed.

Image

Doctors and nurses with a Covid-19 patient at Elmhurst Hospital in
Queens.Credit...Erin Schaff/The New York Times

\hypertarget{covid-19}{%
\subsubsection{Covid-19}\label{covid-19}}

I don't think that I need to spend much time on the Covid-19 pandemic,
except to say that the warning sign was also there. It appeared in late
2002 in the Guangdong province of southern China. It was a viral
respiratory illness caused by a coronavirus --- SARS-CoV --- known for
short as SARS.

As the Centers for Disease Control and Prevention
\href{https://www.cdc.gov/sars/about/fs-sars.html}{website notes},
``Over the next few months, the illness spread to more than two dozen
countries in North America, South America, Europe, and Asia'' before it
was contained. More than 8,000 people worldwide became sick, including
close to 800 who died. The United States had eight confirmed cases of
infection and no deaths.

The coronavirus that caused SARS was hosted by bats and palm civets. It
jumped to humans because we had been pushing and pushing high-density
urban population centers more deeply into wilderness areas, destroying
that natural buffer and replacing it with monoculture crops and
concrete.

When you simultaneously accelerate development in ways that destroy more
and more natural habitats and then hunt for more wildlife there, ``the
natural balance of species collapses due to loss of top predators and
other iconic species, leading to an abundance of more generalized
species adapted to live in human-dominated habitats,'' Johan Rockstrom,
the chief scientist at Conservation International, explained to me.

These include rats, bats, palm civets and some primates, which together
host a majority of all known viruses that can be passed on to humans.
And when these animals are then hunted, trapped and taken to markets ---
in particular in China, Central Africa and Vietnam, where they are sold
for food, traditional medicine, potions and pets --- they endanger
humans, who did not evolve with these viruses.

SARS jumped from mainland China to Hong Kong in February 2003, when a
visiting professor, Dr. Liu Jianlun, who unknowingly had SARS, checked
into Room 911 at Hong Kong's Metropole Hotel.

Yup, Room 9-1-1. I am not making that up.

``By the time he checked out,'' The Washington Post
\href{https://www.washingtonpost.com/archive/politics/2003/05/20/sars-signals-missed-in-hong-kong/50ff4807-4862-4229-8bbd-ec5932b5c896/}{reported},
``Liu had spread a deadly virus directly to at least eight guests. They
would unknowingly take it with them to Singapore, Toronto, Hong Kong and
Hanoi, where the virus would continue to spread. Of more than 7,700
cases of severe acute respiratory syndrome tallied so far worldwide, the
World Health Organization estimates that more than 4,000 can be traced
to Liu's stay on the ninth floor of the Metropole Hotel.''

It is important to note, though, that SARS was contained by July 2003
before becoming a full-fledged pandemic --- thanks in large part to
rapid quarantines and tight global cooperation among public health
authorities in many countries. Collaborative multinational governance
proved to be a good buffer.

Alas, that was then. The latest coronavirus is aptly named SARS-CoV-2
--- with emphasis on the number 2. We don't yet know for sure where this
coronavirus that causes the disease Covid-19 came from, but it is widely
suspected to have jumped to a human from a wild animal, maybe a bat, in
Wuhan, China. Similar jumps are bound to happen more and more as we keep
stripping away nature's natural biodiversity and buffers.

``The more simplified and less diverse ecological systems become,
especially in huge and ever-expanding urban areas, the more we will
become the targets of these emerging pests, unbuffered by the vast array
of other species in a healthy ecosystem,'' explained Russ Mittermeier,
the head of
\href{https://www.globalwildlife.org/blog/coronaviruses-and-the-human-meat-market/}{Global
Wildlife Conservation} and one of the world's top experts on primates.

What we know for sure, though, is that some five months after this
coronavirus jumped into a human in Wuhan, more than 100,000 Americans
were dead and more than 40 million unemployed.

While the coronavirus arrived in the U.S. via both Europe and Asia, most
Americans probably don't realize just how easy it was for this pathogen
to get here. From December through March, when the pandemic was
launching, there were some 3,200 flights from China to major U.S.
cities, according to a
\href{https://abcnews.go.com/Health/disaster-motion-flights-coronavirus-ravaged-countries-landed-us/story?id=70025470}{study
by ABC News}. Among those were 50 direct flights from Wuhan. From Wuhan!
How many Americans had even heard of Wuhan?

The vastly expanded global network of planes, trains and ships, combined
with far too few buffers of global cooperation and governance, combined
with the fact that there are almost eight billion people on the planet
today (compared with 1.8 billion when the 1918 flu pandemic hit),
enabled this coronavirus to spread globally in the blink of an eye.

Image

Hurricane Harvey left Beaumont, Texas, under water in
2017.Credit...Alyssa Schukar for The New York Times

\hypertarget{climate-catastrophe}{%
\subsubsection{Climate Catastrophe}\label{climate-catastrophe}}

You have to be in total denial not to see all of this as one giant
flashing warning signal for our looming --- and potentially worst ---
global disaster, climate change.

I don't like the term climate change to describe what's coming. I much
prefer
``\href{https://www.thenewhumanitarian.org/news/2009/02/05}{global
weirding},'' because the weather getting weird is what is actually
happening. The frequency, intensity and cost of extreme weather events
all increase. The wets get wetter, the hots get hotter, the dry periods
get drier, the snows get heavier, the hurricanes get stronger.

Weather is too complex to attribute any single event to climate change,
but the fact that
\href{https://www.c2es.org/content/extreme-weather-and-climate-change/}{extreme
weather events} are becoming more frequent and more expensive ---
especially in a world of crowded cities like Houston and New Orleans ---
is indisputable.

The wise thing would be for us to get busy preserving all of the
ecological buffers that nature endowed us with, so we could manage what
are now the unavoidable effects of climate change and focus on avoiding
what would be unmanageable consequences.

Because, unlike biological pandemics like Covid-19, climate change does
not ``peak.'' Once we deforest the Amazon or melt the Greenland ice
sheet, it's gone --- and we will have to live with whatever extreme
weather that unleashes.

One tiny example:
\href{https://www.washingtonpost.com/national/michigan-dam-disaster-infrastructure/2020/05/22/26bc380a-9c34-11ea-ac72-3841fcc9b35f_story.html}{The
Washington Post noted} that the Edenville Dam that burst in Midland,
Mich., this month, forcing 11,000 people out of their homes after
unusually heavy spring rains, ``took some residents by surprise, but it
didn't come as such a shock to hydrologists and civil engineers, who
have warned that climate change and increased runoff from development is
putting more pressure on poorly maintained dams, many of them built ---
like those in Midland --- to generate power early in the 20th century.''

But unlike the Covid-19 pandemic, we have all the antibodies we need to
both live with and limit climate change. We can have herd immunity if we
just preserve and enhance the buffers that we know give us resilience.
That means reducing CO₂ emissions, protecting forests that store carbon
and filter water and the ecosystems and species diversity that keep them
healthy, protecting mangroves that buffer storm surges and, more
generally, coordinating global governmental responses that set goals and
limits and monitor performance.

As I look back over the last 20 years, what all four of these global
calamities have in common is that they are all ``black elephants,'' a
term coined by the environmentalist Adam Sweidan. A black elephant is a
cross between ``a black swan'' --- an unlikely, unexpected event with
enormous ramifications --- and the ``elephant in the room'' --- a
looming disaster that is visible to everyone, yet no one wants to
address.

In other words, this journey I have taken you on may sound rather
mechanistic and inevitable. It was not. It was all about different
choices, and different values, that humans and their leaders brought to
bear at different times in our globalizing age --- or didn't.

Technically speaking, globalization is inevitable. How we shape it is
not.

Or, as Nick Hanauer, the venture capitalist and political economist,
remarked to me the other day: ``Pathogens are inevitable, but that they
turn into pandemics is not.''

We decided to remove buffers in the name of efficiency; we decided to
let capitalism run wild and shrink our government's capacities when we
needed them most; we decided not to cooperate with one another in a
pandemic; we decided to deforest the Amazon; we decided to invade
pristine ecosystems and hunt their wildlife. Facebook decided not to
restrict any of President Trump's incendiary posts; Twitter did. And too
many Muslim clerics decided to let the past bury the future, not the
future bury the past.

That's the uber lesson here: As the world gets more deeply intertwined,
everyone's behavior --- the values that each of us bring to this
interdependent world --- matters more than ever. And, therefore, so does
the ``Golden Rule.'' It's never been more important.

Do unto others as you wish them to do unto you, because more people in
more places in more ways on more days can now do unto you and you unto
them like never before.

\emph{The Times is committed to publishing}
\href{https://www.nytimes3xbfgragh.onion/2019/01/31/opinion/letters/letters-to-editor-new-york-times-women.html}{\emph{a
diversity of letters}} \emph{to the editor. We'd like to hear what you
think about this or any of our articles. Here are some}
\href{https://help.nytimes3xbfgragh.onion/hc/en-us/articles/115014925288-How-to-submit-a-letter-to-the-editor}{\emph{tips}}\emph{.
And here's our email:}
\href{mailto:letters@NYTimes.com}{\emph{letters@NYTimes.com}}\emph{.}

\emph{Follow The New York Times Opinion section on}
\href{https://www.facebookcorewwwi.onion/nytopinion}{\emph{Facebook}}\emph{,}
\href{http://twitter.com/NYTOpinion}{\emph{Twitter (@NYTopinion)}}
\emph{and}
\href{https://www.instagram.com/nytopinion/}{\emph{Instagram}}\emph{.}

Advertisement

\protect\hyperlink{after-bottom}{Continue reading the main story}

\hypertarget{site-index}{%
\subsection{Site Index}\label{site-index}}

\hypertarget{site-information-navigation}{%
\subsection{Site Information
Navigation}\label{site-information-navigation}}

\begin{itemize}
\tightlist
\item
  \href{https://help.nytimes3xbfgragh.onion/hc/en-us/articles/115014792127-Copyright-notice}{©~2020~The
  New York Times Company}
\end{itemize}

\begin{itemize}
\tightlist
\item
  \href{https://www.nytco.com/}{NYTCo}
\item
  \href{https://help.nytimes3xbfgragh.onion/hc/en-us/articles/115015385887-Contact-Us}{Contact
  Us}
\item
  \href{https://www.nytco.com/careers/}{Work with us}
\item
  \href{https://nytmediakit.com/}{Advertise}
\item
  \href{http://www.tbrandstudio.com/}{T Brand Studio}
\item
  \href{https://www.nytimes3xbfgragh.onion/privacy/cookie-policy\#how-do-i-manage-trackers}{Your
  Ad Choices}
\item
  \href{https://www.nytimes3xbfgragh.onion/privacy}{Privacy}
\item
  \href{https://help.nytimes3xbfgragh.onion/hc/en-us/articles/115014893428-Terms-of-service}{Terms
  of Service}
\item
  \href{https://help.nytimes3xbfgragh.onion/hc/en-us/articles/115014893968-Terms-of-sale}{Terms
  of Sale}
\item
  \href{https://spiderbites.nytimes3xbfgragh.onion}{Site Map}
\item
  \href{https://help.nytimes3xbfgragh.onion/hc/en-us}{Help}
\item
  \href{https://www.nytimes3xbfgragh.onion/subscription?campaignId=37WXW}{Subscriptions}
\end{itemize}
