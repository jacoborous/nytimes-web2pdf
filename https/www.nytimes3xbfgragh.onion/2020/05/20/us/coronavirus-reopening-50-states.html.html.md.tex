Sections

SEARCH

\protect\hyperlink{site-content}{Skip to
content}\protect\hyperlink{site-index}{Skip to site index}

\href{https://www.nytimes3xbfgragh.onion/section/us}{U.S.}

\href{https://myaccount.nytimes3xbfgragh.onion/auth/login?response_type=cookie\&client_id=vi}{}

\href{https://www.nytimes3xbfgragh.onion/section/todayspaper}{Today's
Paper}

\href{/section/us}{U.S.}\textbar{}All 50 States Are Now Reopening. But
at What Cost?

\url{https://nyti.ms/36flRva}

\begin{itemize}
\item
\item
\item
\item
\item
\item
\end{itemize}

\hypertarget{the-coronavirus-outbreak}{%
\subsubsection{\texorpdfstring{\href{https://www.nytimes3xbfgragh.onion/news-event/coronavirus?name=styln-coronavirus-national\&region=TOP_BANNER\&variant=undefined\&block=storyline_menu_recirc\&action=click\&pgtype=Article\&impression_id=1b5d27a0-e39a-11ea-ad89-c1522ca3443c}{The
Coronavirus
Outbreak}}{The Coronavirus Outbreak}}\label{the-coronavirus-outbreak}}

\begin{itemize}
\tightlist
\item
  live\href{https://www.nytimes3xbfgragh.onion/2020/08/21/world/covid-19-coronavirus.html?name=styln-coronavirus-national\&region=TOP_BANNER\&variant=undefined\&block=storyline_menu_recirc\&action=click\&pgtype=Article\&impression_id=1b5d27a1-e39a-11ea-ad89-c1522ca3443c}{Latest
  Updates}
\item
  \href{https://www.nytimes3xbfgragh.onion/interactive/2020/us/coronavirus-us-cases.html?name=styln-coronavirus-national\&region=TOP_BANNER\&variant=undefined\&block=storyline_menu_recirc\&action=click\&pgtype=Article\&impression_id=1b5d27a2-e39a-11ea-ad89-c1522ca3443c}{Maps
  and Cases}
\item
  \href{https://www.nytimes3xbfgragh.onion/interactive/2020/science/coronavirus-vaccine-tracker.html?name=styln-coronavirus-national\&region=TOP_BANNER\&variant=undefined\&block=storyline_menu_recirc\&action=click\&pgtype=Article\&impression_id=1b5d27a3-e39a-11ea-ad89-c1522ca3443c}{Vaccine
  Tracker}
\item
  \href{https://www.nytimes3xbfgragh.onion/2020/08/19/us/colleges-closing-covid.html?name=styln-coronavirus-national\&region=TOP_BANNER\&variant=undefined\&block=storyline_menu_recirc\&action=click\&pgtype=Article\&impression_id=1b5d4eb0-e39a-11ea-ad89-c1522ca3443c}{Colleges
  Closing}
\item
  \href{https://www.nytimes3xbfgragh.onion/live/2020/08/20/business/stock-market-today-coronavirus?name=styln-coronavirus-national\&region=TOP_BANNER\&variant=undefined\&block=storyline_menu_recirc\&action=click\&pgtype=Article\&impression_id=1b5d4eb1-e39a-11ea-ad89-c1522ca3443c}{Economy}
\end{itemize}

Advertisement

\protect\hyperlink{after-top}{Continue reading the main story}

Supported by

\protect\hyperlink{after-sponsor}{Continue reading the main story}

\hypertarget{all-50-states-are-now-reopening-but-at-what-cost}{%
\section{All 50 States Are Now Reopening. But at What
Cost?}\label{all-50-states-are-now-reopening-but-at-what-cost}}

Governors face intensifying pressure to reopen their economies, but
experts warn it could mean thousands of new deaths.

\includegraphics{https://static01.graylady3jvrrxbe.onion/images/2020/05/20/us/20virus-states-CT1/merlin_172678425_8f9ab489-d087-4e0b-983d-d631f25089b3-articleLarge.jpg?quality=75\&auto=webp\&disable=upscale}

\href{https://www.nytimes3xbfgragh.onion/by/sarah-mervosh}{\includegraphics{https://static01.graylady3jvrrxbe.onion/images/2018/07/18/multimedia/author-sarah-mervosh/author-sarah-mervosh-thumbLarge-v3.png}}\href{https://www.nytimes3xbfgragh.onion/by/amy-harmon}{\includegraphics{https://static01.graylady3jvrrxbe.onion/images/2020/04/29/reader-center/author-amy-harmon/author-amy-harmon-thumbLarge-v2.png}}

By \href{https://www.nytimes3xbfgragh.onion/by/sarah-mervosh}{Sarah
Mervosh} and \href{https://www.nytimes3xbfgragh.onion/by/amy-harmon}{Amy
Harmon}

\begin{itemize}
\item
  Published May 20, 2020Updated May 26, 2020
\item
  \begin{itemize}
  \item
  \item
  \item
  \item
  \item
  \item
  \end{itemize}
\end{itemize}

In Connecticut, flags that had been lowered during the somber peak of
\href{https://www.nytimes3xbfgragh.onion/news-event/coronavirus}{the
coronavirus pandemic} were raised to full-staff on Wednesday to signal a
return to business.

In Kentucky, gift shops creaked open their doors.

And across Alaska, restaurants, bars and gyms, which have been open to
small numbers of customers for weeks, were getting ready to rev back up
to full capacity. ``It will all be open,'' Gov. Mike Dunleavy announced,
``just like it was prior to the virus.''

The United States has crossed an uneasy threshold with all 50 states
beginning to
\href{https://www.nytimes3xbfgragh.onion/2020/05/26/us/lake-of-the-ozarks-coronavirus.html}{reopen}
in some way, two months after the coronavirus thrust the country into
lockdown. But there are vast variations in how states are deciding to
open up, with some forging far ahead of others.

The increasing moves to lift restrictions on businesses --- or at least
open up outdoor spaces like beaches and state parks --- reflect the
immense political and societal pressures weighing on the nation's
governors, even as epidemiologists remain cautious and warn of a second
wave of cases.

With
\href{https://www.nytimes3xbfgragh.onion/2020/05/14/business/economy/coronavirus-unemployment-claims.html}{millions
of people out of work} and many Americans entering their third month
isolated at home, the push to take action rivals what states faced at
the beginning of the crisis, when governors were urged to shut down.

``You have 50 different governors doing 50 different things,'' said
Andrew Noymer, an associate professor of public health at the University
of California, Irvine. ``There will be states that open too soon or
states that are too conservative. It is hard to thread the needle.''

\includegraphics{https://static01.graylady3jvrrxbe.onion/images/2020/05/20/us/20VIRUS-STATES-NH/merlin_172680453_351a5778-5d9c-40b8-961c-5fda428cdfe7-articleLarge.jpg?quality=75\&auto=webp\&disable=upscale}

But if reopening has become a buzz word among politicians --- many
states have issued sweeping documents with color-coded plans to
``rebound'' and ``bounce back'' --- life remains far from normal in most
places across America. Even in Georgia, which opened many businesses
last month ahead of other states, restaurants are seeing only about 15
percent of normal traffic, according to
\href{https://www.opentable.com/state-of-industry}{data published by
OpenTable}, a restaurant reservation website.

The White House has said that states should have a ``downward
trajectory'' of cases over a 14-day period before reopening, but many
states reopened well short of meeting those benchmarks. **** Some
epidemiologists see warning signs of a rebound, especially in the South,
and because it can take as long as three weeks for a newly infected
person to become sick enough to go to the hospital, the impact of
reopening is unlikely to be detectable immediately.

\hypertarget{latest-updates-the-coronavirus-outbreak}{%
\section{\texorpdfstring{\href{https://www.nytimes3xbfgragh.onion/2020/08/21/world/covid-19-coronavirus.html?action=click\&pgtype=Article\&state=default\&region=MAIN_CONTENT_1\&context=storylines_live_updates}{Latest
Updates: The Coronavirus
Outbreak}}{Latest Updates: The Coronavirus Outbreak}}\label{latest-updates-the-coronavirus-outbreak}}

Updated 2020-08-21T10:13:38.790Z

\begin{itemize}
\tightlist
\item
  \href{https://www.nytimes3xbfgragh.onion/2020/08/21/world/covid-19-coronavirus.html?action=click\&pgtype=Article\&state=default\&region=MAIN_CONTENT_1\&context=storylines_live_updates\#link-4690b6aa}{Shutdowns,
  warnings and scoldings follow gatherings on college campuses.}
\item
  \href{https://www.nytimes3xbfgragh.onion/2020/08/21/world/covid-19-coronavirus.html?action=click\&pgtype=Article\&state=default\&region=MAIN_CONTENT_1\&context=storylines_live_updates\#link-324af071}{As
  he accepts the Democratic nomination, Biden knocks Trump's pandemic
  response.}
\item
  \href{https://www.nytimes3xbfgragh.onion/2020/08/21/world/covid-19-coronavirus.html?action=click\&pgtype=Article\&state=default\&region=MAIN_CONTENT_1\&context=storylines_live_updates\#link-35890b73}{Hundreds
  of doctors in Kenya go on strike over their pay and protective gear.}
\end{itemize}

\href{https://www.nytimes3xbfgragh.onion/2020/08/21/world/covid-19-coronavirus.html?action=click\&pgtype=Article\&state=default\&region=MAIN_CONTENT_1\&context=storylines_live_updates}{See
more updates}

More live coverage:
\href{https://www.nytimes3xbfgragh.onion/live/2020/08/20/business/stock-market-today-coronavirus?action=click\&pgtype=Article\&state=default\&region=MAIN_CONTENT_1\&context=storylines_live_updates}{Markets}

``We really are playing with fire here in a very broad sense," said
Charles Courtemanche, an economist at the University of Kentucky. In a
\href{https://www.healthaffairs.org/doi/full/10.1377/hlthaff.2020.00608}{recent
paper} for the journal Health Affairs, he estimated that the number of
confirmed cases in the United States, which reached
\href{https://www.nytimes3xbfgragh.onion/2020/04/28/us/coronavirus-updates.html\#link-20ff56ca}{a
million at the end of April}, would have been closer to 35 million
without the restaurant closures and stay-at-home orders that began in
mid-March. ``Just because it hasn't been a catastrophe yet in your
state, doesn't mean it doesn't have the potential to be,'' he said.

Ipakoi Grigoriadis was left to navigate the complexities while fielding
breakfast orders at her family's diner, Pop's Family Restaurant, in
Milford, Conn., which reopened to outdoor dining on Wednesday as
Connecticut lifted its stay-at-home order and allowed some businesses to
reopen.

``It is still a little scary considering we don't exactly know what this
is,'' said Ms. Grigoriadis, who said the restaurant was taking a number
of precautions. Employees were instructed to wear masks and gloves at
all times, she said, and patrons were expected to wear masks while at
the restaurant as well --- ``except when they are eating and drinking.''

Connecticut was among the last states to take the plunge toward
reopening, representing the more cautious approach that has defined much
of the Northeast. New York, which has seen by far the most cases and
deaths in the nation, is proceeding with a regional reopening that
excludes hard-hit New York City. In Washington, D.C., a stay-at-home
order is in effect until June and the surrounding region remains closed.

Image

Mystic Aquarium in Connecticut reopened to some on Wednesday and is
expected to reopen to the public on Friday.Credit...Christopher
Capozziello for The New York Times

Several states on the West Coast and certain Democratic-led states in
the Midwest have also moved slowly, taking a regional or step-by-step
approach.

By contrast, a number of states in the South opened earlier and more
fully. Businesses
\href{https://www.nytimes3xbfgragh.onion/2020/04/24/us/coronavirus-georgia-oklahoma-alaska-reopen.html}{have
been open with social distancing requirements} for nearly a month in
Georgia, where the
\href{https://www.nytimes3xbfgragh.onion/interactive/2020/us/georgia-coronavirus-cases.html}{number
of new cases} has remained more or less the same. Mississippi saw its
largest single-day increase in reported cases and deaths only after the
state began to reopen.

The variation illustrates the political and regional differences that
have come to define the state-by-state response to the coronavirus, as
governors navigate
\href{https://www.nytimes3xbfgragh.onion/2020/05/17/us/coronavirus-states-reopen.html}{a
pandemic that comes with no political playbook}.

Texas, the nation's second-largest state, with 29 million residents, had
among the shortest stay-at-home orders in the country when it reopened
many businesses on May 1, in a move that appealed to the state's
pro-business spirit. But weeks later, officials reported the
\href{https://www.texastribune.org/2020/05/17/coronavirus-updates-texas/}{highest
one-day total of new cases}, and some fear many businesses will still
not survive.

Of the more than 50,000 restaurants in Texas, 12 percent have gone out
of business because of the pandemic, said Emily Williams Knight, chief
executive of the Texas Restaurant Association. She said she expected
that up to 30 percent would ``not make it through the crisis.'' The
state's restaurant industry has already lost 700,000 jobs, she said, and
would most likely lose more.

``I think you see customers now having an emotional impact of driving up
and having a restaurant they've spent years at simply closed, with a
note saying, `Thank you for your patronage over the years,''' she said.

\href{https://www.nytimes3xbfgragh.onion/news-event/coronavirus?action=click\&pgtype=Article\&state=default\&region=MAIN_CONTENT_3\&context=storylines_faq}{}

\hypertarget{the-coronavirus-outbreak-}{%
\subsubsection{The Coronavirus Outbreak
›}\label{the-coronavirus-outbreak-}}

\hypertarget{frequently-asked-questions}{%
\paragraph{Frequently Asked
Questions}\label{frequently-asked-questions}}

Updated August 17, 2020

\begin{itemize}
\item ~
  \hypertarget{why-does-standing-six-feet-away-from-others-help}{%
  \paragraph{Why does standing six feet away from others
  help?}\label{why-does-standing-six-feet-away-from-others-help}}

  \begin{itemize}
  \tightlist
  \item
    The coronavirus spreads primarily through droplets from your mouth
    and nose, especially when you cough or sneeze. The C.D.C., one of
    the organizations using that measure,
    \href{https://www.nytimes3xbfgragh.onion/2020/04/14/health/coronavirus-six-feet.html?action=click\&pgtype=Article\&state=default\&region=MAIN_CONTENT_3\&context=storylines_faq}{bases
    its recommendation of six feet} on the idea that most large droplets
    that people expel when they cough or sneeze will fall to the ground
    within six feet. But six feet has never been a magic number that
    guarantees complete protection. Sneezes, for instance, can launch
    droplets a lot farther than six feet,
    \href{https://jamanetwork.com/journals/jama/fullarticle/2763852}{according
    to a recent study}. It's a rule of thumb: You should be safest
    standing six feet apart outside, especially when it's windy. But
    keep a mask on at all times, even when you think you're far enough
    apart.
  \end{itemize}
\item ~
  \hypertarget{i-have-antibodies-am-i-now-immune}{%
  \paragraph{I have antibodies. Am I now
  immune?}\label{i-have-antibodies-am-i-now-immune}}

  \begin{itemize}
  \tightlist
  \item
    As of right
    now,\href{https://www.nytimes3xbfgragh.onion/2020/07/22/health/covid-antibodies-herd-immunity.html?action=click\&pgtype=Article\&state=default\&region=MAIN_CONTENT_3\&context=storylines_faq}{that
    seems likely, for at least several months.} There have been
    frightening accounts of people suffering what seems to be a second
    bout of Covid-19. But experts say these patients may have a
    drawn-out course of infection, with the virus taking a slow toll
    weeks to months after initial exposure. People infected with the
    coronavirus typically
    \href{https://www.nature.com/articles/s41586-020-2456-9}{produce}
    immune molecules called antibodies, which are
    \href{https://www.nytimes3xbfgragh.onion/2020/05/07/health/coronavirus-antibody-prevalence.html?action=click\&pgtype=Article\&state=default\&region=MAIN_CONTENT_3\&context=storylines_faq}{protective
    proteins made in response to an
    infection}\href{https://www.nytimes3xbfgragh.onion/2020/05/07/health/coronavirus-antibody-prevalence.html?action=click\&pgtype=Article\&state=default\&region=MAIN_CONTENT_3\&context=storylines_faq}{.
    These antibodies may} last in the body
    \href{https://www.nature.com/articles/s41591-020-0965-6}{only two to
    three months}, which may seem worrisome, but that's perfectly normal
    after an acute infection subsides, said Dr. Michael Mina, an
    immunologist at Harvard University. It may be possible to get the
    coronavirus again, but it's highly unlikely that it would be
    possible in a short window of time from initial infection or make
    people sicker the second time.
  \end{itemize}
\item ~
  \hypertarget{im-a-small-business-owner-can-i-get-relief}{%
  \paragraph{I'm a small-business owner. Can I get
  relief?}\label{im-a-small-business-owner-can-i-get-relief}}

  \begin{itemize}
  \tightlist
  \item
    The
    \href{https://www.nytimes3xbfgragh.onion/article/small-business-loans-stimulus-grants-freelancers-coronavirus.html?action=click\&pgtype=Article\&state=default\&region=MAIN_CONTENT_3\&context=storylines_faq}{stimulus
    bills enacted in March} offer help for the millions of American
    small businesses. Those eligible for aid are businesses and
    nonprofit organizations with fewer than 500 workers, including sole
    proprietorships, independent contractors and freelancers. Some
    larger companies in some industries are also eligible. The help
    being offered, which is being managed by the Small Business
    Administration, includes the Paycheck Protection Program and the
    Economic Injury Disaster Loan program. But lots of folks have
    \href{https://www.nytimes3xbfgragh.onion/interactive/2020/05/07/business/small-business-loans-coronavirus.html?action=click\&pgtype=Article\&state=default\&region=MAIN_CONTENT_3\&context=storylines_faq}{not
    yet seen payouts.} Even those who have received help are confused:
    The rules are draconian, and some are stuck sitting on
    \href{https://www.nytimes3xbfgragh.onion/2020/05/02/business/economy/loans-coronavirus-small-business.html?action=click\&pgtype=Article\&state=default\&region=MAIN_CONTENT_3\&context=storylines_faq}{money
    they don't know how to use.} Many small-business owners are getting
    less than they expected or
    \href{https://www.nytimes3xbfgragh.onion/2020/06/10/business/Small-business-loans-ppp.html?action=click\&pgtype=Article\&state=default\&region=MAIN_CONTENT_3\&context=storylines_faq}{not
    hearing anything at all.}
  \end{itemize}
\item ~
  \hypertarget{what-are-my-rights-if-i-am-worried-about-going-back-to-work}{%
  \paragraph{What are my rights if I am worried about going back to
  work?}\label{what-are-my-rights-if-i-am-worried-about-going-back-to-work}}

  \begin{itemize}
  \tightlist
  \item
    Employers have to provide
    \href{https://www.osha.gov/SLTC/covid-19/standards.html}{a safe
    workplace} with policies that protect everyone equally.
    \href{https://www.nytimes3xbfgragh.onion/article/coronavirus-money-unemployment.html?action=click\&pgtype=Article\&state=default\&region=MAIN_CONTENT_3\&context=storylines_faq}{And
    if one of your co-workers tests positive for the coronavirus, the
    C.D.C.} has said that
    \href{https://www.cdc.gov/coronavirus/2019-ncov/community/guidance-business-response.html}{employers
    should tell their employees} -\/- without giving you the sick
    employee's name -\/- that they may have been exposed to the virus.
  \end{itemize}
\item ~
  \hypertarget{what-is-school-going-to-look-like-in-september}{%
  \paragraph{What is school going to look like in
  September?}\label{what-is-school-going-to-look-like-in-september}}

  \begin{itemize}
  \tightlist
  \item
    It is unlikely that many schools will return to a normal schedule
    this fall, requiring the grind of
    \href{https://www.nytimes3xbfgragh.onion/2020/06/05/us/coronavirus-education-lost-learning.html?action=click\&pgtype=Article\&state=default\&region=MAIN_CONTENT_3\&context=storylines_faq}{online
    learning},
    \href{https://www.nytimes3xbfgragh.onion/2020/05/29/us/coronavirus-child-care-centers.html?action=click\&pgtype=Article\&state=default\&region=MAIN_CONTENT_3\&context=storylines_faq}{makeshift
    child care} and
    \href{https://www.nytimes3xbfgragh.onion/2020/06/03/business/economy/coronavirus-working-women.html?action=click\&pgtype=Article\&state=default\&region=MAIN_CONTENT_3\&context=storylines_faq}{stunted
    workdays} to continue. California's two largest public school
    districts --- Los Angeles and San Diego --- said on July 13, that
    \href{https://www.nytimes3xbfgragh.onion/2020/07/13/us/lausd-san-diego-school-reopening.html?action=click\&pgtype=Article\&state=default\&region=MAIN_CONTENT_3\&context=storylines_faq}{instruction
    will be remote-only in the fall}, citing concerns that surging
    coronavirus infections in their areas pose too dire a risk for
    students and teachers. Together, the two districts enroll some
    825,000 students. They are the largest in the country so far to
    abandon plans for even a partial physical return to classrooms when
    they reopen in August. For other districts, the solution won't be an
    all-or-nothing approach.
    \href{https://bioethics.jhu.edu/research-and-outreach/projects/eschool-initiative/school-policy-tracker/}{Many
    systems}, including the nation's largest, New York City, are
    devising
    \href{https://www.nytimes3xbfgragh.onion/2020/06/26/us/coronavirus-schools-reopen-fall.html?action=click\&pgtype=Article\&state=default\&region=MAIN_CONTENT_3\&context=storylines_faq}{hybrid
    plans} that involve spending some days in classrooms and other days
    online. There's no national policy on this yet, so check with your
    municipal school system regularly to see what is happening in your
    community.
  \end{itemize}
\end{itemize}

While scores of restaurants flounder or even collapse, El Arroyo, a
popular Mexican restaurant at the western edge of downtown Austin, found
a ticket to survival with a savvy marketing move --- the sale of to-go
margaritas. The restaurant has since begun serving customers on its
patio and will return to indoor dining when Texas allows restaurants to
expand to 50 percent capacity for indoor sales from 25 percent, starting
Friday.

Image

A group of people removed their masks to eat lunch outside of Central
Market, a Texas grocery chain, this month.Credit...Ilana Panich-Linsman
for The New York Times

``We're full,'' said Shane Thompson, the manager. ``People are happy to
be out and about.''

In Louisville, Ky., a handful of customers browsed through the bourbon,
horse-themed jewelry and collectible items that stocked the Kentucky
Derby Museum gift shop on Wednesday, when the state lifted restrictions
on retail stores.

``We really didn't know what to expect,'' said Rachel Collier, a
spokeswoman for the museum, which remains closed even as the gift shop
opened back up. She said the museum's revenue had declined by 95 percent
during the pandemic, which also forced the Kentucky Derby to be
postponed for the first time in 75 years.

Even without the boon of tourism, she said, a small but steady stream of
customers drifted through the gift shop on Wednesday, many eager to buy
cups engraved with this year's canceled derby date. ``We are pleasantly
surprised,'' she said.

Researchers expect that reopening the United States could cause
thousands of additional deaths, while also saving several million jobs,
a balancing act that has swung more toward the economy in recent weeks.

A
\href{https://budgetmodel.wharton.upenn.edu/issues/2020/5/1/coronavirus-reopening-simulator}{forecast
from the University of Pennsylvania's Penn Wharton Budget Model}
estimated that the number of cumulative deaths from the virus in the
United States would rise to 157,000 from the current about 92,000 by the
end of July if states maintained restrictions. A partial or full
reopening could bring an additional 15,000 or 73,000 deaths,
respectively.

Researchers found that the biggest risk for negative health outcomes was
probably not state regulations, but people's own behavior. If Americans
get out of the habit of social distancing --- returning to their
pre-pandemic behavior by not wearing masks or staying six feet apart ---
the forecast predicted that deaths could rise by as many as 135,000.

``Everyone wants us to talk about policy, but in fact personal behavior
still matters a lot here,'' said Kent Smetters, the faculty director at
the Penn Wharton Budget Model.

Many are still hesitant. A
\href{https://apnews.com/3562b5a082a27221e532075de509a36c}{new poll by
The Associated Press-NORC Center for Public Affairs Research} found that
most Americans were somewhat concerned that lifting restrictions in
their area would lead to new infections, and at least half were very or
extremely concerned. About six in 10 people were in favor of people
remaining in their homes except for essential needs.

\emph{{[}Sign up}
\href{https://www.nytimes3xbfgragh.onion/newsletters/california-today}{\emph{for
California Today}}\emph{, our daily newsletter about the Golden
State.{]}}

Mary Lou Giles, a 73-year-old resident of El Dorado County, Calif., said
that she and her husband planned to shelter in place for another several
weeks, though businesses in her remote mountain county between
Sacramento and Lake Tahoe were allowed to reopen sooner than in other
parts of the state.

``I sincerely hope there will not be a surge in Covid cases as a result
of what I believe is a premature rush to reopen,'' she said. ``But I'm
not willing to gamble.''

Reporting was contributed by David Montgomery, Sarah Maslin Nir, Jill
Cowan and Mike Baker.

Advertisement

\protect\hyperlink{after-bottom}{Continue reading the main story}

\hypertarget{site-index}{%
\subsection{Site Index}\label{site-index}}

\hypertarget{site-information-navigation}{%
\subsection{Site Information
Navigation}\label{site-information-navigation}}

\begin{itemize}
\tightlist
\item
  \href{https://help.nytimes3xbfgragh.onion/hc/en-us/articles/115014792127-Copyright-notice}{©~2020~The
  New York Times Company}
\end{itemize}

\begin{itemize}
\tightlist
\item
  \href{https://www.nytco.com/}{NYTCo}
\item
  \href{https://help.nytimes3xbfgragh.onion/hc/en-us/articles/115015385887-Contact-Us}{Contact
  Us}
\item
  \href{https://www.nytco.com/careers/}{Work with us}
\item
  \href{https://nytmediakit.com/}{Advertise}
\item
  \href{http://www.tbrandstudio.com/}{T Brand Studio}
\item
  \href{https://www.nytimes3xbfgragh.onion/privacy/cookie-policy\#how-do-i-manage-trackers}{Your
  Ad Choices}
\item
  \href{https://www.nytimes3xbfgragh.onion/privacy}{Privacy}
\item
  \href{https://help.nytimes3xbfgragh.onion/hc/en-us/articles/115014893428-Terms-of-service}{Terms
  of Service}
\item
  \href{https://help.nytimes3xbfgragh.onion/hc/en-us/articles/115014893968-Terms-of-sale}{Terms
  of Sale}
\item
  \href{https://spiderbites.nytimes3xbfgragh.onion}{Site Map}
\item
  \href{https://help.nytimes3xbfgragh.onion/hc/en-us}{Help}
\item
  \href{https://www.nytimes3xbfgragh.onion/subscription?campaignId=37WXW}{Subscriptions}
\end{itemize}
