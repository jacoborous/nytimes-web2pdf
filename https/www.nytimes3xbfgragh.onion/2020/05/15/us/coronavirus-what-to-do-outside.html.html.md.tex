Sections

SEARCH

\protect\hyperlink{site-content}{Skip to
content}\protect\hyperlink{site-index}{Skip to site index}

\href{https://www.nytimes3xbfgragh.onion/section/us}{U.S.}

\href{https://myaccount.nytimes3xbfgragh.onion/auth/login?response_type=cookie\&client_id=vi}{}

\href{https://www.nytimes3xbfgragh.onion/section/todayspaper}{Today's
Paper}

\href{/section/us}{U.S.}\textbar{}What We Know About Your Chances of
Catching the Virus Outdoors

\url{https://nyti.ms/2WBgO4M}

\begin{itemize}
\item
\item
\item
\item
\item
\end{itemize}

\hypertarget{the-coronavirus-outbreak}{%
\subsubsection{\texorpdfstring{\href{https://www.nytimes3xbfgragh.onion/news-event/coronavirus?name=styln-coronavirus-national\&region=TOP_BANNER\&block=storyline_menu_recirc\&action=click\&pgtype=Article\&impression_id=38bfacf0-efbb-11ea-a941-d3ff0b5fd50c\&variant=undefined}{The
Coronavirus
Outbreak}}{The Coronavirus Outbreak}}\label{the-coronavirus-outbreak}}

\begin{itemize}
\tightlist
\item
  live\href{https://www.nytimes3xbfgragh.onion/2020/09/05/world/coronavirus-covid.html?name=styln-coronavirus-national\&region=TOP_BANNER\&block=storyline_menu_recirc\&action=click\&pgtype=Article\&impression_id=38bfd400-efbb-11ea-a941-d3ff0b5fd50c\&variant=undefined}{Latest
  Updates}
\item
  \href{https://www.nytimes3xbfgragh.onion/interactive/2020/us/coronavirus-us-cases.html?name=styln-coronavirus-national\&region=TOP_BANNER\&block=storyline_menu_recirc\&action=click\&pgtype=Article\&impression_id=38bfd401-efbb-11ea-a941-d3ff0b5fd50c\&variant=undefined}{Maps
  and Cases}
\item
  \href{https://www.nytimes3xbfgragh.onion/interactive/2020/science/coronavirus-vaccine-tracker.html?name=styln-coronavirus-national\&region=TOP_BANNER\&block=storyline_menu_recirc\&action=click\&pgtype=Article\&impression_id=38bfd402-efbb-11ea-a941-d3ff0b5fd50c\&variant=undefined}{Vaccine
  Tracker}
\item
  \href{https://www.nytimes3xbfgragh.onion/2020/09/02/your-money/eviction-moratorium-covid.html?name=styln-coronavirus-national\&region=TOP_BANNER\&block=storyline_menu_recirc\&action=click\&pgtype=Article\&impression_id=38bfd403-efbb-11ea-a941-d3ff0b5fd50c\&variant=undefined}{Eviction
  Moratorium}
\item
  \href{https://www.nytimes3xbfgragh.onion/interactive/2020/09/02/magazine/food-insecurity-hunger-us.html?name=styln-coronavirus-national\&region=TOP_BANNER\&block=storyline_menu_recirc\&action=click\&pgtype=Article\&impression_id=38bfd404-efbb-11ea-a941-d3ff0b5fd50c\&variant=undefined}{American
  Hunger}
\end{itemize}

Advertisement

\protect\hyperlink{after-top}{Continue reading the main story}

Supported by

\protect\hyperlink{after-sponsor}{Continue reading the main story}

\hypertarget{what-we-know-about-your-chances-of-catching-the-virus-outdoors}{%
\section{What We Know About Your Chances of Catching the Virus
Outdoors}\label{what-we-know-about-your-chances-of-catching-the-virus-outdoors}}

A stir-crazy nation wonders: Is it safe to stroll on the beach in a
deadly pandemic? How about a picnic in the park? Or coffee with a friend
at an outdoor table? The risk is in the details.

\includegraphics{https://static01.graylady3jvrrxbe.onion/images/2020/05/15/multimedia/15xp-virus-outdoors-image1/merlin_172525647_226f7542-292f-41ac-bd72-7290eea08d80-articleLarge.jpg?quality=75\&auto=webp\&disable=upscale}

By \href{https://www.nytimes3xbfgragh.onion/by/michael-levenson}{Michael
Levenson},
\href{https://www.nytimes3xbfgragh.onion/by/tara-parker-pope}{Tara
Parker-Pope} and
\href{https://www.nytimes3xbfgragh.onion/by/james-gorman}{James Gorman}

\begin{itemize}
\item
  Published May 15, 2020Updated July 3, 2020
\item
  \begin{itemize}
  \item
  \item
  \item
  \item
  \item
  \end{itemize}
\end{itemize}

\href{https://www.nytimes3xbfgragh.onion/es/2020/05/19/espanol/coronavirus-que-hacer-afuera.html}{Leer
en español}

The warmer weather across the country calls to mind carefree summers ---
picnics in the park, swimming at the beach, fireworks on the Fourth. But
nothing feels carefree now.

Even the simplest outdoor activities seem fraught with a thousand
questions and calculations about the coronavirus pandemic.

Is it safe to meet friends in the park, as long as they stay six feet
away, on the other side of a blanket? What about a burger and beer at an
outdoor restaurant? How risky is a trip to the beach or swimming pool
with the kids?

The good news: Interviews show a growing consensus among experts that,
if Americans are going to leave their homes, it's safer to be outside
than in the office or the mall. With fresh air and more space between
people, the risk goes down.

But experts also expressed particular caution about outdoor dining,
using locker rooms at pools and crowds in places like beaches. While
going outside can help people cope with quarantine fatigue, there is a
risk they will lower their guard or meet people who are not being safe.

``I think going outside is important for health,'' said Julia L. Marcus,
an epidemiologist and assistant professor at Harvard Medical School.
``We know that being outdoors is lower risk for coronavirus transmission
than being indoors. On a sunny, beautiful weekend, I think going outside
is indicated, but I also think there are things to do to reduce our
risk.''

\includegraphics{https://static01.graylady3jvrrxbe.onion/images/2020/05/15/multimedia/15xp-virus-outdoors-image2/merlin_172525632_14df6d12-6c52-42bf-b1ad-5ad6bee80b7d-articleLarge.jpg?quality=75\&auto=webp\&disable=upscale}

While many treasured outdoor rites of the season have been closed or
canceled, including Disney's amusement parks, the Coachella festival in
California and
\href{https://publictheater.org/programs/shakespeare-in-the-park/}{Free
Shakespeare in the Park} in New York, governors across the country have
been opening golf courses,
\href{https://www.nytimes3xbfgragh.onion/2020/05/21/us/appalachian-trail-coronavirus.html}{trails}and
beaches, hoping to restore some semblance of a normal summer for
restless Americans.

Some parks, including small urban ones like
\href{https://www.nps.gov/elis/index.htm}{Ellis Island} and vast ones
like Joshua Tree National Park, remain closed. But Yellowstone planned
to \href{https://www.nps.gov/yell/learn/news/20015.htm}{reopen on a
limited basis in late May}, and the Grand Canyon reopened its South Rim
entrance. In Los Angeles County, beaches
\href{https://twitter.com/SupJaniceHahn/status/1259875923785965568?s=20}{reopened},
but not for sunbathing. Only active pursuits like jogging, swimming and
surfing are allowed.

Even in the hard-hit New York region, some restrictions will be eased.
Connecticut planned to allow outdoor seating at restaurants and outdoor
exhibits at zoos on May 20. New Jersey, New York, Delaware and
Connecticut will open state beaches on Memorial Day weekend, restricting
them to 50 percent capacity. But New York City's beaches and pools will
remain closed.

\hypertarget{latest-updates-the-coronavirus-outbreak}{%
\section{\texorpdfstring{\href{https://www.nytimes3xbfgragh.onion/2020/09/04/world/covid-19-coronavirus.html?action=click\&pgtype=Article\&state=default\&region=MAIN_CONTENT_1\&context=storylines_live_updates}{Latest
Updates: The Coronavirus
Outbreak}}{Latest Updates: The Coronavirus Outbreak}}\label{latest-updates-the-coronavirus-outbreak}}

Updated 2020-09-05T12:05:40.998Z

\begin{itemize}
\tightlist
\item
  \href{https://www.nytimes3xbfgragh.onion/2020/09/04/world/covid-19-coronavirus.html?action=click\&pgtype=Article\&state=default\&region=MAIN_CONTENT_1\&context=storylines_live_updates\#link-1654f6ad}{Research
  connects vaping to a higher chance of catching the virus --- and
  suffering its worst effects.}
\item
  \href{https://www.nytimes3xbfgragh.onion/2020/09/04/world/covid-19-coronavirus.html?action=click\&pgtype=Article\&state=default\&region=MAIN_CONTENT_1\&context=storylines_live_updates\#link-52e4198a}{Another
  college football game won't be played as planned.}
\item
  \href{https://www.nytimes3xbfgragh.onion/2020/09/04/world/covid-19-coronavirus.html?action=click\&pgtype=Article\&state=default\&region=MAIN_CONTENT_1\&context=storylines_live_updates\#link-181cef0}{Pharmaceutical
  companies plan a joint pledge on safety standards as they move
  vaccines to the marketplace.}
\end{itemize}

\href{https://www.nytimes3xbfgragh.onion/2020/09/04/world/covid-19-coronavirus.html?action=click\&pgtype=Article\&state=default\&region=MAIN_CONTENT_1\&context=storylines_live_updates}{See
more updates}

More live coverage:
\href{https://www.nytimes3xbfgragh.onion/live/2020/09/04/business/stock-market-today-coronavirus?action=click\&pgtype=Article\&state=default\&region=MAIN_CONTENT_1\&context=storylines_live_updates}{Markets}

The different approaches have left many Americans bewildered about what
is safe behavior outside. Experts have a simple answer: Practice social
distancing and wear a mask when that is not possible.

Ideally, people should socialize only with people who live in their
homes, they say. If you decide to meet friends, you're increasing your
risk, but you can take precautions. It's important to keep gatherings
small. Don't share food, utensils or beverages; keep your hands clean;
and keep at least six feet from people who don't live in your home.

Image

A park enforcement patrol officer distributed free masks near Pier 45 at
Hudson River Park in Manhattan.Credit...Jeenah Moon for The New York
Times

``I think outdoors is so much better than indoors in almost all cases,''
said Linsey Marr, an engineering professor and aerosol scientist at
Virginia Tech. ``There's so much dilution that happens outdoors. As long
as you're staying at least six feet apart, I think the risk is very
low.''

Pandemic life is safer outdoors, in part, because even a light wind will
quickly dilute the virus. If a person nearby is sick, the wind will
scatter the virus, potentially exposing nearby people but in far smaller
quantities, which are less likely to be harmful.

``The virus load is important,'' said Eugene Chudnovsky, a physicist at
Lehman College and the City University of New York's Graduate Center.
``A single virus will not make anyone sick; it will be immediately
destroyed by the immune system. The belief is that one needs a few
hundred to a few thousand of SARS-CoV-2 viruses to overwhelm the immune
response.''

While the risk of outdoor transmission is low, it can happen. In one
study of more than
\href{https://www.medrxiv.org/content/10.1101/2020.04.04.20053058v1.full.pdf}{7,300
cases in China}, just one was connected to outdoor transmission. In that
case, a 27-year-old man had a conversation outdoors with a traveler who
had just returned from Wuhan. Seven days later, he had his first
symptoms of Covid-19.

``The risk is lower outdoors, but it's not zero,'' said Shan Soe-Lin, a
lecturer at the Yale Jackson Institute for Global Affairs. ``And I think
the risk is higher if you have two people who are stationary next to
each other for a long time, like on a beach blanket, rather than people
who are walking and passing each other.''

One
\href{https://www.nytimes3xbfgragh.onion/2020/05/14/health/coronavirus-infections.html}{recent
study} found that just talking can launch thousands of droplets that can
remain suspended in the air for eight to 14 minutes. But the risk of
inhaling those droplets is lower outdoors.

Image

Social distancing in parks should be the norm, experts say. Stay with
people from your own household if you can, and keep gatherings small.
Don't share food.Credit...Jeenah Moon for The New York Times

For many Americans who have spent anxious months at home, wide-open
parks and trails feel like the safest options these days.

Kate Wathall, a Los Angeles television producer and reporter, went for
her first hike in weeks, one day after local trails reopened. She drove
an hour to Trail Canyon Falls in Tujunga, avoiding more popular trails
in the city.

``It was like being back to normal life,'' she said. ``Obviously, it's
not. But it's a day where I forgot what was going on.''

In Memorial Park in Maplewood, N.J., Gabriella Gabriel, 22, was
exercising with her friend Candace Brodie, also 22, on mats a few feet
apart on the grass.

``People are spread out and there's no way for someone to be right on
top of me,'' Ms. Gabriel said. ``But in a pool or beach, everyone is so
condensed --- too close for comfort.''

Experts agreed that the risk of swimming in pools, lakes or the ocean is
not from the water, but from exposure to people in and near the water.

\href{https://www.nytimes3xbfgragh.onion/news-event/coronavirus?action=click\&pgtype=Article\&state=default\&region=MAIN_CONTENT_3\&context=storylines_faq}{}

\hypertarget{the-coronavirus-outbreak-}{%
\subsubsection{The Coronavirus Outbreak
›}\label{the-coronavirus-outbreak-}}

\hypertarget{frequently-asked-questions}{%
\paragraph{Frequently Asked
Questions}\label{frequently-asked-questions}}

Updated September 4, 2020

\begin{itemize}
\item ~
  \hypertarget{what-are-the-symptoms-of-coronavirus}{%
  \paragraph{What are the symptoms of
  coronavirus?}\label{what-are-the-symptoms-of-coronavirus}}

  \begin{itemize}
  \tightlist
  \item
    In the beginning, the coronavirus
    \href{https://www.nytimes3xbfgragh.onion/article/coronavirus-facts-history.html?action=click\&pgtype=Article\&state=default\&region=MAIN_CONTENT_3\&context=storylines_faq\#link-6817bab5}{seemed
    like it was primarily a respiratory illness}~--- many patients had
    fever and chills, were weak and tired, and coughed a lot, though
    some people don't show many symptoms at all. Those who seemed
    sickest had pneumonia or acute respiratory distress syndrome and
    received supplemental oxygen. By now, doctors have identified many
    more symptoms and syndromes. In April,
    \href{https://www.nytimes3xbfgragh.onion/2020/04/27/health/coronavirus-symptoms-cdc.html?action=click\&pgtype=Article\&state=default\&region=MAIN_CONTENT_3\&context=storylines_faq}{the
    C.D.C. added to the list of early signs}~sore throat, fever, chills
    and muscle aches. Gastrointestinal upset, such as diarrhea and
    nausea, has also been observed. Another telltale sign of infection
    may be a sudden, profound diminution of one's
    \href{https://www.nytimes3xbfgragh.onion/2020/03/22/health/coronavirus-symptoms-smell-taste.html?action=click\&pgtype=Article\&state=default\&region=MAIN_CONTENT_3\&context=storylines_faq}{sense
    of smell and taste.}~Teenagers and young adults in some cases have
    developed painful red and purple lesions on their fingers and toes
    --- nicknamed ``Covid toe'' --- but few other serious symptoms.
  \end{itemize}
\item ~
  \hypertarget{why-is-it-safer-to-spend-time-together-outside}{%
  \paragraph{Why is it safer to spend time together
  outside?}\label{why-is-it-safer-to-spend-time-together-outside}}

  \begin{itemize}
  \tightlist
  \item
    \href{https://www.nytimes3xbfgragh.onion/2020/05/15/us/coronavirus-what-to-do-outside.html?action=click\&pgtype=Article\&state=default\&region=MAIN_CONTENT_3\&context=storylines_faq}{Outdoor
    gatherings}~lower risk because wind disperses viral droplets, and
    sunlight can kill some of the virus. Open spaces prevent the virus
    from building up in concentrated amounts and being inhaled, which
    can happen when infected people exhale in a confined space for long
    stretches of time, said Dr. Julian W. Tang, a virologist at the
    University of Leicester.
  \end{itemize}
\item ~
  \hypertarget{why-does-standing-six-feet-away-from-others-help}{%
  \paragraph{Why does standing six feet away from others
  help?}\label{why-does-standing-six-feet-away-from-others-help}}

  \begin{itemize}
  \tightlist
  \item
    The coronavirus spreads primarily through droplets from your mouth
    and nose, especially when you cough or sneeze. The C.D.C., one of
    the organizations using that measure,
    \href{https://www.nytimes3xbfgragh.onion/2020/04/14/health/coronavirus-six-feet.html?action=click\&pgtype=Article\&state=default\&region=MAIN_CONTENT_3\&context=storylines_faq}{bases
    its recommendation of six feet}~on the idea that most large droplets
    that people expel when they cough or sneeze will fall to the ground
    within six feet. But six feet has never been a magic number that
    guarantees complete protection. Sneezes, for instance, can launch
    droplets a lot farther than six feet,
    \href{https://jamanetwork.com/journals/jama/fullarticle/2763852}{according
    to a recent study}. It's a rule of thumb: You should be safest
    standing six feet apart outside, especially when it's windy. But
    keep a mask on at all times, even when you think you're far enough
    apart.
  \end{itemize}
\item ~
  \hypertarget{i-have-antibodies-am-i-now-immune}{%
  \paragraph{I have antibodies. Am I now
  immune?}\label{i-have-antibodies-am-i-now-immune}}

  \begin{itemize}
  \tightlist
  \item
    As of right
    now,\href{https://www.nytimes3xbfgragh.onion/2020/07/22/health/covid-antibodies-herd-immunity.html?action=click\&pgtype=Article\&state=default\&region=MAIN_CONTENT_3\&context=storylines_faq}{~that
    seems likely, for at least several months.}~There have been
    frightening accounts of people suffering what seems to be a second
    bout of Covid-19. But experts say these patients may have a
    drawn-out course of infection, with the virus taking a slow toll
    weeks to months after initial exposure.~People infected with the
    coronavirus typically
    \href{https://www.nature.com/articles/s41586-020-2456-9}{produce}~immune
    molecules called antibodies, which are
    \href{https://www.nytimes3xbfgragh.onion/2020/05/07/health/coronavirus-antibody-prevalence.html?action=click\&pgtype=Article\&state=default\&region=MAIN_CONTENT_3\&context=storylines_faq}{protective
    proteins made in response to an
    infection}\href{https://www.nytimes3xbfgragh.onion/2020/05/07/health/coronavirus-antibody-prevalence.html?action=click\&pgtype=Article\&state=default\&region=MAIN_CONTENT_3\&context=storylines_faq}{.
    These antibodies may}~last in the body
    \href{https://www.nature.com/articles/s41591-020-0965-6}{only two to
    three months}, which may seem worrisome, but that's~perfectly normal
    after an acute infection subsides, said Dr. Michael Mina, an
    immunologist at Harvard University. It may be possible to get the
    coronavirus again, but it's highly unlikely that it would be
    possible in a short window of time from initial infection or make
    people sicker the second time.
  \end{itemize}
\item ~
  \hypertarget{what-are-my-rights-if-i-am-worried-about-going-back-to-work}{%
  \paragraph{What are my rights if I am worried about going back to
  work?}\label{what-are-my-rights-if-i-am-worried-about-going-back-to-work}}

  \begin{itemize}
  \tightlist
  \item
    Employers have to provide
    \href{https://www.osha.gov/SLTC/covid-19/standards.html}{a safe
    workplace}~with policies that protect everyone equally.
    \href{https://www.nytimes3xbfgragh.onion/article/coronavirus-money-unemployment.html?action=click\&pgtype=Article\&state=default\&region=MAIN_CONTENT_3\&context=storylines_faq}{And
    if one of your co-workers tests positive for the coronavirus, the
    C.D.C.}~has said that
    \href{https://www.cdc.gov/coronavirus/2019-ncov/community/guidance-business-response.html}{employers
    should tell their employees}~-\/- without giving you the sick
    employee's name -\/- that they may have been exposed to the virus.
  \end{itemize}
\end{itemize}

Although scientists don't have data on the novel coronavirus
specifically, other coronaviruses are not stable in water and are very
sensitive to chlorine, said Angela Rasmussen, a virologist at the
Columbia University Mailman School of Public Health.

``In my opinion, pool water, fresh water in a lake or river, or seawater
exposure would be extremely low transmission risk even without dilution
(which would reduce risk further),'' Dr. Rasmussen said in an email.
``Probably the biggest risk for summer water recreation is crowds --- a
crowded pool locker room, dock or beach, especially if coupled with
limited physical distancing or prolonged proximity to others. The most
concentrated sources of virus in such an environment will be the people
hanging out at the pool, not the pool itself.''

Experts say that a person walking, jogging or cycling too close for a
few seconds is not a big worry. But they recommend joggers wear a mask
or some other form of face covering if they're going to come close to
other people. If someone sets up a picnic blanket within your six-foot
perimeter and plans to stay a while, that's a bigger concern. Try to
avoid a confrontation. That only increases your risk of exposure. Such
conflicts could spike as more people head outside.

``If someone is too close to you and not wearing a mask and you don't
feel safe, instead of yelling at them, just say, `I need some space,
please,''' Dr. Marcus said.

For families with small children, navigating the outdoors can produce a
special anxiety.

Ms. Gabriel said her brother, who is 6, had wanted to go to the
playground, but her mother wouldn't allow it. She worries about the
virus lingering on slides and swings and about a mysterious
\href{https://www.nytimes3xbfgragh.onion/2020/05/13/health/coronavirus-children-kawasaki-pmis.html}{inflammatory
syndrome} linked to the virus that has been sickening and killing some
children.

``It's hard for a child to understand,'' Ms. Gabriel said. ``At least we
can stay six feet apart. You can't tell a little kid that.''

Image

Navigating protective precautions with children is challenging and
playgrounds are still disputed terrain.Credit...Jeenah Moon for The New
York Times

One challenge in dense cities is finding six feet to call your own on a
running path or in a bicycle lane. An open-air cafe may seem safe, until
people start walking by on the sidewalk without masks.

Some cities, including New York, Boston, Minneapolis and Oakland, have
\href{https://www.nytimes3xbfgragh.onion/2020/04/11/us/coronavirus-street-closures.html}{closed
streets} to traffic, giving people room to spread out. Others have
extended sidewalks to make more space for pedestrians and outdoor
seating.

Even outside, there is a risk of contracting the virus by touching a
contaminated surface --- a restaurant menu, park bench or lawn chair ---
and then touching your face. Studies have shown the
\href{https://www.nytimes3xbfgragh.onion/2020/03/17/health/coronavirus-surfaces-aerosols.html}{virus
can last three days on hard surfaces} like steel and plastic and about
24 hours on cardboard under laboratory conditions. The virus is also
more stable in heat and humidity than many other viruses are.

According to Dr. Chudnovsky, a sunny day is better than a cloudy day,
because there's more sunlight to kill the virus and more wind to dilute
it*.* If you want to take extreme precautions, position yourself upwind
from other people. ``This may be especially important at the beach,
where people tend to spend a long time at one localized place,'' he
said.

Experts said that although outdoor restaurant patrons can't wear masks
while eating, servers should. The main risk of exposure is if the guests
within a few feet at the table aren't from your household. Sitting and
talking for extended periods of time as well as sharing food and common
serving utensils are also potential sources of exposure if one of the
guests is infected and doesn't know it.

Image

Cocktail hour has been moved outdoors --- mixed with masks and social
distancing.Credit...Jeenah Moon for The New York Times

Another worry: Because it can take two weeks for symptoms to appear
after a person is infected, there is no way to know if you're going to
the beach or the park in the midst of an invisible local outbreak,
experts said. It's yet another reason to take precautions.

``If we now go back to the old normal and don't follow the social
distancing strategy anymore, it's like a ticking time bomb,'' said Peter
Jüni, an epidemiologist at the University of Toronto and St. Michael's
Hospital. ``You never know where it blows up and when.''

Advertisement

\protect\hyperlink{after-bottom}{Continue reading the main story}

\hypertarget{site-index}{%
\subsection{Site Index}\label{site-index}}

\hypertarget{site-information-navigation}{%
\subsection{Site Information
Navigation}\label{site-information-navigation}}

\begin{itemize}
\tightlist
\item
  \href{https://help.nytimes3xbfgragh.onion/hc/en-us/articles/115014792127-Copyright-notice}{©~2020~The
  New York Times Company}
\end{itemize}

\begin{itemize}
\tightlist
\item
  \href{https://www.nytco.com/}{NYTCo}
\item
  \href{https://help.nytimes3xbfgragh.onion/hc/en-us/articles/115015385887-Contact-Us}{Contact
  Us}
\item
  \href{https://www.nytco.com/careers/}{Work with us}
\item
  \href{https://nytmediakit.com/}{Advertise}
\item
  \href{http://www.tbrandstudio.com/}{T Brand Studio}
\item
  \href{https://www.nytimes3xbfgragh.onion/privacy/cookie-policy\#how-do-i-manage-trackers}{Your
  Ad Choices}
\item
  \href{https://www.nytimes3xbfgragh.onion/privacy}{Privacy}
\item
  \href{https://help.nytimes3xbfgragh.onion/hc/en-us/articles/115014893428-Terms-of-service}{Terms
  of Service}
\item
  \href{https://help.nytimes3xbfgragh.onion/hc/en-us/articles/115014893968-Terms-of-sale}{Terms
  of Sale}
\item
  \href{https://spiderbites.nytimes3xbfgragh.onion}{Site Map}
\item
  \href{https://help.nytimes3xbfgragh.onion/hc/en-us}{Help}
\item
  \href{https://www.nytimes3xbfgragh.onion/subscription?campaignId=37WXW}{Subscriptions}
\end{itemize}
