Sections

SEARCH

\protect\hyperlink{site-content}{Skip to
content}\protect\hyperlink{site-index}{Skip to site index}

\href{https://www.nytimes3xbfgragh.onion/spotlight/podcasts}{Podcasts}

\href{https://myaccount.nytimes3xbfgragh.onion/auth/login?response_type=cookie\&client_id=vi}{}

\href{https://www.nytimes3xbfgragh.onion/section/todayspaper}{Today's
Paper}

\href{/spotlight/podcasts}{Podcasts}\textbar{}`Whatever We Have, We Have
to Work With It'

\url{https://nyti.ms/2SFDAGD}

\begin{itemize}
\item
\item
\item
\item
\item
\item
\end{itemize}

Advertisement

\protect\hyperlink{after-top}{Continue reading the main story}

transcript

Back to Sugar Calling

bars

0:00/28:58

-28:58

transcript

\hypertarget{whatever-we-have-we-have-to-work-with-it}{%
\subsection{`Whatever We Have, We Have to Work With
It'}\label{whatever-we-have-we-have-to-work-with-it}}

\hypertarget{hosted-by-cheryl-strayed-produced-by-kelly-prime-and-edited-by-sara-sarasohn-editorial-oversight-by-wendy-dorr}{%
\subsubsection{Hosted by Cheryl Strayed, produced by Kelly Prime and
edited by Sara Sarasohn. Editorial oversight by Wendy
Dorr.}\label{hosted-by-cheryl-strayed-produced-by-kelly-prime-and-edited-by-sara-sarasohn-editorial-oversight-by-wendy-dorr}}

\hypertarget{cheryl-strayed-talks-with-the-writer-alice-walker-about-ancestors-solitude-and-the-time-it-takes-to-heal}{%
\paragraph{Cheryl Strayed talks with the writer Alice Walker about
ancestors, solitude and the time it takes to
heal.}\label{cheryl-strayed-talks-with-the-writer-alice-walker-about-ancestors-solitude-and-the-time-it-takes-to-heal}}

Wednesday, May 6th, 2020

\begin{itemize}
\item
  cheryl strayed\\
  Today, I'm going to call Alice Walker. She won the Pulitzer Prize in
  fiction for her novel, The Color Purple. She was the first black woman
  to win that prize. She also won the National Book Award that year.
  She's published many books, novels, poetry collections, essay
  collections. And she really for many decades now has been telling the
  truth about who we are and how we struggle and how we persist. Her
  most recent book is a collection of poetry called Taking the Arrow Out
  of the Heart. I've been reading it the past few days. It's terrific.
\item
  {[}music{]}\\
  So I don't think there's any better person to talk to right now than
  Alice Walker. I'm going to give her a call.
\item
  {[}phone ringing{]}
\item
  alice walker\\
  Hello.
\item
  cheryl strayed\\
  Hi, is this Alice?
\item
  alice walker\\
  Yes.
\item
  cheryl strayed\\
  Hi, this is Cheryl.
\item
  alice walker\\
  Hi.
\item
  cheryl strayed\\
  Hi, it's so nice to talk to you and such an honor. Where are you now?
\item
  alice walker\\
  Where am I? I'm in Mendocino in California in the middle of virtually
  nowhere, which is the perfect place to be. It's very sunny and warm
  today. And it's just, you know, it's just beautiful. I'm just so happy
  to be here.
\item
  cheryl strayed\\
  Have you been socially isolating? Are you there alone or with somebody
  else?
\item
  alice walker\\
  Well, I was in Mexico for most of the winter.
\item
  cheryl strayed\\
  Mhm.
\item
  alice walker\\
  And I isolated pretty much there until I came back north. And yeah,
  I've been --- you know, I see a few people. But we're more than arm's
  length --- {[}LAUGHS{]}
\item
  cheryl strayed\\
  Right.
\item
  alice walker\\
  --- apart. We wear our masks and have been very cautious and careful
  and loving of each other and respectful, you know? So we're all just
  learning how to move together, apart.
\item
  cheryl strayed\\
  Together, apart. What's the experience been for you? Does it feel
  scary or lonely? What is your experience of what's happened over this
  past month or so in our lives?
\item
  alice walker\\
  Well, it is upsetting. It has its terrifying aspects, especially if
  you watch a lot of media.
\item
  cheryl strayed\\
  Mm-hmm.
\item
  alice walker\\
  So I don't do a lot of that. If the virus is going to get me, it's
  just going to get me {[}LAUGHS{]} the way any other thing would while
  I'm busy doing something else, you know?
\item
  cheryl strayed\\
  Mhm.
\item
  alice walker\\
  And that's pretty much how I take it. You know, I work in my garden.
  And I read. And I write my blog. And I play with my dog a lot because
  that is so joyful.
\item
  cheryl strayed\\
  Right. So listen, I was reading the other day in The New York Times
  that people are having strange or interesting or unusual dreams during
  this time. Or maybe they're just remembering their dreams. And ---
\item
  alice walker\\
  Mhm.
\item
  cheryl strayed\\
  --- I remember reading that you pay attention to your dreams, as well.
  I'm wondering, have you had any strange pandemic-induced dreams?
\item
  alice walker\\
  Now that is so interesting because usually I do dream a lot. And I'm
  fascinated by my dreams, whatever is going on in them. But during this
  period, I'm not dreaming. I'm ---
\item
  cheryl strayed\\
  Huh.
\item
  alice walker\\
  --- deeply sleeping.
\item
  cheryl strayed\\
  I mean, you usually remember your dreams quite well. Do you have a
  theory about why you're not remembering them now?
\item
  alice walker\\
  I don't. I don't. I think it's partly just that I have relaxed into
  the present and relaxed, also, into exactly where I am ---
\item
  cheryl strayed\\
  Mhm.
\item
  alice walker\\
  --- because I think on some level I understand and have really known
  that we're imperiled, you know, all the time and in all kinds of ways.
  But I also have always made every effort to meet the glory of this
  existence in where it's so apparent, which is, usually for me, in the
  woods, in silence, in a kind of solitude.

  So you know, a lot of that is still just my way. It is me going
  through this peril and whatever other peril --- {[}LAUGHS{]} peril
  might appear ---
\item
  cheryl strayed\\
  {[}LAUGHS{]} Right.
\item
  alice walker\\
  --- you know, recognizing that with all of that, it's still fabulous.
  I mean, it is just a stunning thing that we have here in life on this
  planet.
\item
  cheryl strayed\\
  As you're talking, I'm reminded that a word that I associate with you
  is joy. When I think of so much of your work, it's so full of sorrow
  and hardship and really, very difficult things. But always, there is
  joy at its center. And I'm curious, how did you foster that within
  you? How did you nurture that within you as you lived through the
  various hard times of your life?
\item
  alice walker\\
  Well, I think I had to accept that hard times are with us. And they're
  with not just me but with most people and that they're not avoidable.
  I mean, you can shift and not have that particular thing by doing
  something and avoiding it. But then that will catch you, too. So you
  know, as the Buddha said to us, there is suffering. And the question
  is always, well, what do you do with it? And does it have a use? And I
  maintain that it does and that, therefore, you should learn how to do
  it and accept that that's what's happening. And that way, you have a
  chance to find out what's on the other side of that suffering because
  on the other side, there may well be the most amazing, joyful epiphany
  about reality.
\item
  cheryl strayed\\
  Yeah, I've been reading your book, Taking the Arrow Out of the Heart,
  which is such a beautiful collection of poems. And I just loved the
  title, I want to say, of that book because I do think that that
  describes so aptly how it is I've healed my wounds. And everyone I
  know who's healed their wounds have had to do that thing where they
  take the arrow out of their own heart.
\item
  alice walker\\
  Well, Pema Chodron, the Buddhist nun and teacher, was very helpful in
  that area because I had actually --- the way we encountered each other
  was I had a love affair that just ended terribly. And I was, you know,
  dying of pain and sorrow and suffering and loss and all those things.
  And I couldn't seem to shake any of it. And then somebody came for an
  interview. And she brought Pema's medicine about tonglen, this
  practice where you learn how to breathe in disaster and sorrow and
  pain and suffering and all those things, just let the darkness, the
  sadness, whatever, inside as much as you can. I mean, just fill
  yourself with it. But then you breathe out what you'd rather have
  yourself. And you breathe it out for everyone who is feeling as you're
  feeling. And this resonated with me as a practice because it's
  something that you can learn to do. It's something that will take time
  to do. And actually, this pandemic is a great opportunity for people
  to learn this practice. And I recommend it.
\item
  cheryl strayed\\
  So you breathe in the suffering and the difficulty or the ugliness.
  And you breathe out gratitude, beauty, appreciation.
\item
  alice walker\\
  A walk on the beach, yeah ---
\item
  cheryl strayed\\
  Uh-huh.
\item
  alice walker\\
  --- but for everybody. I mean, the wonderful thing is that it's for
  everybody. That's where we are now. There is no longer any point in
  trying to breathe out just what's good for you or to breathe in just
  your suffering and nobody else's. {[}LAUGHS{]}
\item
  cheryl strayed\\
  Right. Well, probably there was never a time to do that.
\item
  alice walker\\
  Well, oh, no, that's the American way.
\item
  cheryl strayed\\
  {[}LAUGHS{]} That's ---
\item
  alice walker\\
  I mean, you can breathe in what's happening on your block, but you
  don't really care what's happening down in the ghetto.
\item
  cheryl strayed\\
  Mhm.
\item
  alice walker\\
  So it's a different way of handling sorrow and distress. And I really
  prefer it because I like the idea that there's no longer any point in
  trying to save just oneself, you know? There's just no point.
\item
  cheryl strayed\\
  Yeah.
\item
  alice walker\\
  Mhm.
\item
  cheryl strayed\\
  I loved what you said about breathing in that suffering for all of us.
  And yet, your sentiment in taking the arrow out of the heart is about
  that we all really do have to find a way to treat the wound we
  ourselves have individually. I'm curious about, how have you done
  that? You've done the breathing. You've done that work of accepting
  your suffering and breathing it out.
\item
  alice walker\\
  Mhm.
\item
  cheryl strayed\\
  But it seems to me this is something we have to do for ourselves over
  and over and over again ---
\item
  alice walker\\
  That's right.
\item
  cheryl strayed\\
  --- to keep going.
\item
  alice walker\\
  That's right. And I mustn't leave out the big factor here, which is
  time. Time might be one of the biggest factor, probably the biggest
  factor, in healing. And that's one of the places we can get a little
  stuck because we don't want to take time. We want it done. I remember
  I had a friend who --- I was suffering terribly because of some love
  affair that had gone astray. And I said, well, how long did it take
  you to get over a broken heart or whatever? And she said two weeks.
  {[}LAUGHING{]}
\item
  cheryl strayed\\
  Oh, dear. That wasn't a very broken heart.
\item
  alice walker\\
  And I did say to her, I said, well, you know, really my heart's going
  to take a lot longer than that. And so it's like that, I think, that
  we have our sufferings. And in my life, so many of the people that I
  have loved have been assassinated. And at times, when these things
  happen, you can't imagine that you'll ever stand up again. I mean, the
  pain is so intense and the sense of loss. You know, when Martin Luther
  King was assassinated, I had a miscarriage. It was physical as well as
  mental and spiritual. And so there's no way of ever being able to say
  with any kind of truth when you will recover, when you will take that
  arrow out of your heart and stand up again either in the place that
  you were already standing or even a bit forward. You know, that's the
  test. I mean, can you get up where you fell and at least hold that? Or
  can you get up where you fell and then take a step forward?
\item
  cheryl strayed\\
  Mhm. So the first poem in your book, Taking the Arrow Out of the
  Heart, is called ``The Long Road Home.'' And it's about Muhammad Ali.
  Would you be so kind as to read it to me?
\item
  alice walker\\
  Sure, sure. Tell me why you like that one.
\item
  cheryl strayed\\
  I just, I love, really --- I so agree with you about this idea that
  our suffering can contain a gift.
\item
  alice walker\\
  Yes.
\item
  cheryl strayed\\
  And you write about that so beautifully and so powerfully across all
  of your books and all of the genres that you've written in. And so I
  saw aspects of that in this beautiful poem.
\item
  alice walker\\
  OK, well, I loved Muhammad Ali. And of course, he was himself with all
  his foibles. But ultimately, he was someone who could not be forced to
  murder other people. And he refused to be drafted into an immoral war.
  And so I hold him in very high esteem. ``The Long Road Home'' for
  Muhammad Ali. ``I am beginning to comprehend the mystery of the gift
  of suffering. It is true, as some have said, that it is a crucible in
  which the gold of one's spirit is rendered and shines. Ali, you
  represent all of us who stand the test of suffering most often alone,
  because who could understand who or what has brought us to our feet?
  Their knees worn out, ancestors stood us up from the awkward position
  they had to honor on the floor beneath the floor. I have been weeping
  all day thinking of this, the cloud of witness, the endless teaching,
  the long road home.''
\item
  cheryl strayed\\
  Beautiful. Tell me what you were thinking when you wrote this poem and
  what it means to you.
\item
  alice walker\\
  Well, his refusal to be drafted into the Vietnam War and the suffering
  that came to him because of that, because they put him in prison. And
  they took away his title. And they did all the things that they
  thought would break him as a human being. And he accepted --- you
  know, he found a place to stand. And he stood there. And this is also
  saying that this is our history, that we are standing on the shoulders
  of ancestors who could barely stand in the slave dungeons. But those
  ancestors are part of the reason that we stand, because they could not
  in that position. So it goes very deep in us now that we are
  surrounded by what I call the cloud of witness. And the cloud of
  witness for African-Americans are these ancestors who went through
  suffering that most Americans have never, ever thought about, still
  made sure that we got that we are supposed to honor other people. You
  know, we got that from them. And this is just such a miracle.
\item
  cheryl strayed\\
  Mhm. In your own life when you were a young woman, how did you get
  called to write? And how did you come to that consciousness about what
  you owe the ancestors or what we all owe them?
\item
  alice walker\\
  Oh, {[}DEEPLY EXHALES{]} I think, growing up, I was in a small
  community. People were close. And the music was good in church. And
  you just had this feeling, even before you understood it --- because
  we didn't study history exactly --- but we could get it that somehow
  this beautiful --- I mean, you can hear it in Aretha Franklin's voice.
  I mean, there's a good example. In Aretha Franklin's voice, she is
  carrying the ancestral medicine for the tribe.
\item
  cheryl strayed\\
  Hm, mhm.
\item
  alice walker\\
  And that is so important to know, you know? So yeah, I had my little
  Easter speeches. And I had adoring relatives and friends in the
  church. And unfortunately, or fortunately, I really got very tired of
  the church by the time I was 12. And I went off into nature.
\item
  cheryl strayed\\
  Mhm. And then when did you go off into writing?
\item
  alice walker\\
  And into writing? {[}LAUGHS{]} Probably around the same time because
  if all my family was in church, that meant I had privacy and I could
  write at home.
\item
  cheryl strayed\\
  Right. But when you first began seriously writing, did you experience
  it as a call? Or was it something more intellectual --- you loved
  books and you thought, I'll give it a try?
\item
  alice walker\\
  You know, Cheryl, I honestly feel like I was just like any plant that
  produces a bloom or a twig or whatever plants produce, that it was
  just so natural. In fact, my mother says that when I was crawling, she
  would find me at the back of the house writing in the dirt with a
  twig.
\item
  cheryl strayed\\
  {[}LAUGHS{]}
\item
  alice walker\\
  And I just --- it's just completely natural. It really is. It's just
  like, well, yeah, well, this is what one would do.
\item
  cheryl strayed\\
  Are you turning to your writing now in this time of disconnect? Are
  you finding yourself drawn to making that connection that we can make
  in writing? Or are you staying away from it?
\item
  alice walker\\
  Oh, I write all the time. And I have a blog, which I adore. I don't
  care if I have only five people who read it ---
\item
  cheryl strayed\\
  {[}LAUGHS{]}
\item
  alice walker\\
  --- because my sign is Aquarius. And we love the idea of just sending
  thoughts through the air. I get on there, and I offer what I have and
  send it off and think very little more about it. And like any other
  thing that's happening in nature --- you know, I was out this morning
  picking irises. And I was just marveling at how that green stem, where
  you would never see any purple or white or orange or anything,
  suddenly out of that comes this incredible flower. And I love that
  feeling of connectivity with what is just blooming naturally.
\item
  cheryl strayed\\
  Yeah, it is a powerful feeling. So I'm curious, how is it that --- you
  turn to nature, obviously, when you're seeking solace or beauty. What
  about wisdom?
\item
  alice walker\\
  Well, I read a lot. I've always been drawn to wonderful books, like
  the I Ching, which I adore, all those long Russian novels. And I mean,
  I don't know, just basically living and being open and interested. And
  wisdom just kind of accrues.
\item
  cheryl strayed\\
  Right, right. So I read an interview with you that your favorite book
  is Jane Eyre. I'm wondering if you could just tell me about that. What
  is it that you feel --- you said you felt such a deep kinship to that
  novel.
\item
  alice walker\\
  I love Jane Eyre because Charlotte Bronte was in the middle of nowhere
  up there, wherever they were --- {[}LAUGHS{]} cold.
\item
  cheryl strayed\\
  In England somewhere.
\item
  alice walker\\
  Right, right. And I'm trying to think of the actual village. But
  anyway, cold and dreary and no sun for most of the year, you know? And
  her father was oblivious to a lot. He never realized that his
  daughters were writing novels, for instance. But anyway, so there she
  was in this very strange place where she could not even admit being a
  woman writing a book. Her books had to be published under a man's
  name. And she persevered and created this incredibly enthralling,
  romantic, but at the same time, in a way because of Jane's character,
  stringent book which never lets women down. And this is a good thing
  to think about in some of these novels that people love. Often women
  are, in the end, let down. They're not permitted to truly test their
  mettle when it comes to what they will and will not accept. And I love
  Jane because she stands on her little feet there with this man that
  she falls in love with. And she realizes that they could live in sin
  because they couldn't get married because of the church.
\item
  cheryl strayed\\
  Yeah, they couldn't get married because he had --- right, he was
  already married.
\item
  alice walker\\
  He was married. And so the church, you know how the church is. So
  anyway ---
\item
  cheryl strayed\\
  Yeah.
\item
  alice walker\\
  --- they couldn't get married. And really, she loved him so
  passionately that another kind of woman would have said, you know,
  let's just go live in, I don't know, Egypt or somewhere. But no, I
  mean, Jane has real integrity.

  And I love that. I think women, especially --- you know, romance is
  all well and good. But when it makes you sell out your soul, it's not
  worth having. So that's one of the reasons I absolutely love it.
\item
  cheryl strayed\\
  When did you first read Jane Eyre? How old were you?
\item
  alice walker\\
  Oh, I was 13 or 14. {[}LAUGHING{]}
\item
  cheryl strayed\\
  {[}LAUGHS{]} OK.
\item
  alice walker\\
  But then I read it --- Cheryl, I read it a lot after that.
  {[}LAUGHING{]}
\item
  cheryl strayed\\
  Did you? {[}LAUGHS{]} Yeah, so you were swept up in that romance. You
  know, I think it's interesting, a couple of the times that you've
  mentioned struggle, it's been romantic heartbreak. And I'm wondering,
  are those the hardest things that you feel you've sort of survived and
  endured? Or what is the hardest thing you've lived through?
\item
  alice walker\\
  Well, yeah, I lived through my daughter's decision that I was not a
  good mother. And of course, I disagree. But there it is.
\item
  cheryl strayed\\
  Aw.
\item
  alice walker\\
  And we're fine now. So I'm just telling you that that was a really
  difficult passage.
\item
  cheryl strayed\\
  How did you get through that? How did you come to peace?
\item
  alice walker\\
  Well, I think we just did, you know? I think at some point after 10
  years and with a grandson whom I really adore, I think we came to our
  senses more or less.
\item
  cheryl strayed\\
  Mm-hmm.
\item
  alice walker\\
  And that's the other thing about time that I was referring to earlier.
  That part of getting the arrow out and finding the joy again is time.
  And what you do while you're waiting means a lot. And so these periods
  when there is so much pain are great for deepening oneself. And that's
  why study is a wonderful thing. I believe in study with my whole
  heart.
\item
  cheryl strayed\\
  Mm-hmm. Yeah, it's this concept of time I think has been really on my
  mind a lot. I have two teenage kids. And they are always inquiring
  like, OK, when can things go back to normal? And it's really hard
  because before this pandemic, we all sort of thought we knew what we
  were going to be doing this summer or what was going to happen next
  month. And it's very difficult to explain to them that we're in an era
  right now of deep uncertainty that shows us that we don't know what
  tomorrow will bring, which we never knew. But to talk to them about,
  like, even if this goes on a year, that's actually, relative to the
  rest of our lives, just a short period of time. It doesn't feel like
  it. And I know that that's always how discomfort or suffering or
  difficult times feel. They feel like your heart will never mend,
  right?
\item
  alice walker\\
  Mhm.
\item
  cheryl strayed\\
  And this pain will last forever. This estrangement will go on and on
  and on. And then something shifts, and it's a different time. I think
  that's such a wise thing to remember. But wow, it's hard to do that
  sometimes.
\item
  alice walker\\
  Yeah, it is. And I think with children, it's good to, I don't know,
  have them use this time to do something they never thought they'd have
  time to do. You might ask them, was there something that you thought
  you'd never get around to, that you'd never have time to do, you'd
  never have time to explore? This is a perfect time for that.
\item
  cheryl strayed\\
  Yeah, Alice, I'm really trying. {[}LAUGHS{]} I'm really trying to do
  that. But my teenagers are less cooperative with {[}LAUGHS{]} those
  instructions. Maybe if you tell them what to study, I can ---
  {[}LAUGHS{]}
\item
  alice walker\\
  No, no, no, Cheryl, trust me, no. This will probably go on a very long
  time ---
\item
  cheryl strayed\\
  Yeah.
\item
  alice walker\\
  --- in one form or another. Or it will recur in some other crazy way.
  And there will be plenty of time. So you just tell them the best
  counsel that you can offer. And then they will have to learn the rest.
\item
  cheryl strayed\\
  Yeah, I try to do that. I try to be a guide to my children and open a
  space for them, especially now that they're teenagers and they want
  that space. And here we are in this house all together all the time.
  And yeah, I'll keep trying. I'll keep trying, encouraging that.
\item
  alice walker\\
  Well, my heart goes out to all of you because I actually like, very
  much, solitude. {[}LAUGHING{]}
\item
  cheryl strayed\\
  {[}LAUGHING{]} You're content.
\item
  alice walker\\
  {[}LAUGHING{]} So I feel like, you know, well --- but I just think
  that whatever we have, we have to work with it. And we can, and we do.
\item
  cheryl strayed\\
  Yeah. So are you afraid that you're going to get the virus? Are you
  afraid of your own risks when it comes to this virus?
\item
  alice walker\\
  Well, I'm in a mode of accepting my fate and whatever that is, you
  know? I have no way, beyond trying to wear my gloves and my mask and
  keep people six feet away and all of that --- there's nothing more
  that I can do. And I've had a wonderful adventure. I mean, my life has
  been really just a huge adventure. I have no complaints about it
  really. I mean, whatever awful things that I had to basically help to
  change, we changed as much of those awful things as we could. I
  enjoyed that, even when it was dangerous. So I've had that. And you
  know, I, like everybody, in the middle of the night, I think, oh,
  dear, I hear it's not a pretty sight dying from this virus. But I then
  tell myself, well, dying doesn't take forever and that because I think
  that there is something on the other side of death, which is
  essentially more life. It may not be life that we recognize as life.
  {[}LAUGHS{]}
\item
  cheryl strayed\\
  What do you think is on the other side, what kind of life?
\item
  alice walker\\
  Well, nothing really dies, I mean, really. I mean, the personality
  here would not be around. But there is no such thing as just a
  non-existence.
\item
  cheryl strayed\\
  I guess we all become the dirt from which the next flower blooms,
  right?
\item
  alice walker\\
  Absolutely, yeah. So that's comforting.
\item
  cheryl strayed\\
  Yeah.
\item
  {[}music{]}\\
  Well, Alice, it's been really lovely to speak with you. Thank you so
  much for taking the time to chat with me today.
\item
  alice walker\\
  Take care.
\item
  cheryl strayed\\
  Thank you.
\item
  alice walker\\
  And your family --- to your family, say hello for me.
\item
  cheryl strayed\\
  All right. Bye, bye.
\item
  alice walker\\
  Bye.
\item
  {[}music{]}
\item
  cheryl strayed\\
  I'm Cheryl Strayed. This is ``Sugar Calling.'' Next week, Billy
  Collins.
\item
  {[}music{]}
\end{itemize}

\href{https://www.nytimes3xbfgragh.onion/column/sugar-calling}{\includegraphics{https://static01.graylady3jvrrxbe.onion/images/2020/04/29/podcasts/sugar-calling-album-art/sugar-calling-album-art-square320.jpg}Sugar
Calling}Subscribe:

\begin{itemize}
\tightlist
\item
  \href{https://itunes.apple.com/us/podcast/id1505881384}{Apple
  Podcasts}
\item
  \href{https://podcasts.google.com/?feed=aHR0cHM6Ly9yc3MuYXJ0MTkuY29tL3N1Z2FyLWNhbGxpbmc\&ved=0CAUQrrcFahcKEwjA8Kyn09voAhUAAAAAHQAAAAAQBQ}{Google
  Podcasts}
\end{itemize}

\hypertarget{whatever-we-have-we-have-to-work-with-it-1}{%
\section{`Whatever We Have, We Have to Work With
It'}\label{whatever-we-have-we-have-to-work-with-it-1}}

\hypertarget{cheryl-strayed-talks-with-the-writer-alice-walker-about-ancestors-solitude-and-the-time-it-takes-to-heal-1}{%
\subsection{Cheryl Strayed talks with the writer Alice Walker about
ancestors, solitude and the time it takes to
heal.}\label{cheryl-strayed-talks-with-the-writer-alice-walker-about-ancestors-solitude-and-the-time-it-takes-to-heal-1}}

Hosted by Cheryl Strayed, produced by Kelly Prime and edited by Sara
Sarasohn. Editorial oversight by Wendy Dorr.

Transcript

transcript

Back to Sugar Calling

bars

0:00/28:58

-0:00

transcript

\hypertarget{whatever-we-have-we-have-to-work-with-it-2}{%
\subsection{`Whatever We Have, We Have to Work With
It'}\label{whatever-we-have-we-have-to-work-with-it-2}}

\hypertarget{hosted-by-cheryl-strayed-produced-by-kelly-prime-and-edited-by-sara-sarasohn-editorial-oversight-by-wendy-dorr-1}{%
\subsubsection{Hosted by Cheryl Strayed, produced by Kelly Prime and
edited by Sara Sarasohn. Editorial oversight by Wendy
Dorr.}\label{hosted-by-cheryl-strayed-produced-by-kelly-prime-and-edited-by-sara-sarasohn-editorial-oversight-by-wendy-dorr-1}}

\hypertarget{cheryl-strayed-talks-with-the-writer-alice-walker-about-ancestors-solitude-and-the-time-it-takes-to-heal-2}{%
\paragraph{Cheryl Strayed talks with the writer Alice Walker about
ancestors, solitude and the time it takes to
heal.}\label{cheryl-strayed-talks-with-the-writer-alice-walker-about-ancestors-solitude-and-the-time-it-takes-to-heal-2}}

Wednesday, May 6th, 2020

\begin{itemize}
\item
  cheryl strayed\\
  Today, I'm going to call Alice Walker. She won the Pulitzer Prize in
  fiction for her novel, The Color Purple. She was the first black woman
  to win that prize. She also won the National Book Award that year.
  She's published many books, novels, poetry collections, essay
  collections. And she really for many decades now has been telling the
  truth about who we are and how we struggle and how we persist. Her
  most recent book is a collection of poetry called Taking the Arrow Out
  of the Heart. I've been reading it the past few days. It's terrific.
\item
  {[}music{]}\\
  So I don't think there's any better person to talk to right now than
  Alice Walker. I'm going to give her a call.
\item
  {[}phone ringing{]}
\item
  alice walker\\
  Hello.
\item
  cheryl strayed\\
  Hi, is this Alice?
\item
  alice walker\\
  Yes.
\item
  cheryl strayed\\
  Hi, this is Cheryl.
\item
  alice walker\\
  Hi.
\item
  cheryl strayed\\
  Hi, it's so nice to talk to you and such an honor. Where are you now?
\item
  alice walker\\
  Where am I? I'm in Mendocino in California in the middle of virtually
  nowhere, which is the perfect place to be. It's very sunny and warm
  today. And it's just, you know, it's just beautiful. I'm just so happy
  to be here.
\item
  cheryl strayed\\
  Have you been socially isolating? Are you there alone or with somebody
  else?
\item
  alice walker\\
  Well, I was in Mexico for most of the winter.
\item
  cheryl strayed\\
  Mhm.
\item
  alice walker\\
  And I isolated pretty much there until I came back north. And yeah,
  I've been --- you know, I see a few people. But we're more than arm's
  length --- {[}LAUGHS{]}
\item
  cheryl strayed\\
  Right.
\item
  alice walker\\
  --- apart. We wear our masks and have been very cautious and careful
  and loving of each other and respectful, you know? So we're all just
  learning how to move together, apart.
\item
  cheryl strayed\\
  Together, apart. What's the experience been for you? Does it feel
  scary or lonely? What is your experience of what's happened over this
  past month or so in our lives?
\item
  alice walker\\
  Well, it is upsetting. It has its terrifying aspects, especially if
  you watch a lot of media.
\item
  cheryl strayed\\
  Mm-hmm.
\item
  alice walker\\
  So I don't do a lot of that. If the virus is going to get me, it's
  just going to get me {[}LAUGHS{]} the way any other thing would while
  I'm busy doing something else, you know?
\item
  cheryl strayed\\
  Mhm.
\item
  alice walker\\
  And that's pretty much how I take it. You know, I work in my garden.
  And I read. And I write my blog. And I play with my dog a lot because
  that is so joyful.
\item
  cheryl strayed\\
  Right. So listen, I was reading the other day in The New York Times
  that people are having strange or interesting or unusual dreams during
  this time. Or maybe they're just remembering their dreams. And ---
\item
  alice walker\\
  Mhm.
\item
  cheryl strayed\\
  --- I remember reading that you pay attention to your dreams, as well.
  I'm wondering, have you had any strange pandemic-induced dreams?
\item
  alice walker\\
  Now that is so interesting because usually I do dream a lot. And I'm
  fascinated by my dreams, whatever is going on in them. But during this
  period, I'm not dreaming. I'm ---
\item
  cheryl strayed\\
  Huh.
\item
  alice walker\\
  --- deeply sleeping.
\item
  cheryl strayed\\
  I mean, you usually remember your dreams quite well. Do you have a
  theory about why you're not remembering them now?
\item
  alice walker\\
  I don't. I don't. I think it's partly just that I have relaxed into
  the present and relaxed, also, into exactly where I am ---
\item
  cheryl strayed\\
  Mhm.
\item
  alice walker\\
  --- because I think on some level I understand and have really known
  that we're imperiled, you know, all the time and in all kinds of ways.
  But I also have always made every effort to meet the glory of this
  existence in where it's so apparent, which is, usually for me, in the
  woods, in silence, in a kind of solitude.

  So you know, a lot of that is still just my way. It is me going
  through this peril and whatever other peril --- {[}LAUGHS{]} peril
  might appear ---
\item
  cheryl strayed\\
  {[}LAUGHS{]} Right.
\item
  alice walker\\
  --- you know, recognizing that with all of that, it's still fabulous.
  I mean, it is just a stunning thing that we have here in life on this
  planet.
\item
  cheryl strayed\\
  As you're talking, I'm reminded that a word that I associate with you
  is joy. When I think of so much of your work, it's so full of sorrow
  and hardship and really, very difficult things. But always, there is
  joy at its center. And I'm curious, how did you foster that within
  you? How did you nurture that within you as you lived through the
  various hard times of your life?
\item
  alice walker\\
  Well, I think I had to accept that hard times are with us. And they're
  with not just me but with most people and that they're not avoidable.
  I mean, you can shift and not have that particular thing by doing
  something and avoiding it. But then that will catch you, too. So you
  know, as the Buddha said to us, there is suffering. And the question
  is always, well, what do you do with it? And does it have a use? And I
  maintain that it does and that, therefore, you should learn how to do
  it and accept that that's what's happening. And that way, you have a
  chance to find out what's on the other side of that suffering because
  on the other side, there may well be the most amazing, joyful epiphany
  about reality.
\item
  cheryl strayed\\
  Yeah, I've been reading your book, Taking the Arrow Out of the Heart,
  which is such a beautiful collection of poems. And I just loved the
  title, I want to say, of that book because I do think that that
  describes so aptly how it is I've healed my wounds. And everyone I
  know who's healed their wounds have had to do that thing where they
  take the arrow out of their own heart.
\item
  alice walker\\
  Well, Pema Chodron, the Buddhist nun and teacher, was very helpful in
  that area because I had actually --- the way we encountered each other
  was I had a love affair that just ended terribly. And I was, you know,
  dying of pain and sorrow and suffering and loss and all those things.
  And I couldn't seem to shake any of it. And then somebody came for an
  interview. And she brought Pema's medicine about tonglen, this
  practice where you learn how to breathe in disaster and sorrow and
  pain and suffering and all those things, just let the darkness, the
  sadness, whatever, inside as much as you can. I mean, just fill
  yourself with it. But then you breathe out what you'd rather have
  yourself. And you breathe it out for everyone who is feeling as you're
  feeling. And this resonated with me as a practice because it's
  something that you can learn to do. It's something that will take time
  to do. And actually, this pandemic is a great opportunity for people
  to learn this practice. And I recommend it.
\item
  cheryl strayed\\
  So you breathe in the suffering and the difficulty or the ugliness.
  And you breathe out gratitude, beauty, appreciation.
\item
  alice walker\\
  A walk on the beach, yeah ---
\item
  cheryl strayed\\
  Uh-huh.
\item
  alice walker\\
  --- but for everybody. I mean, the wonderful thing is that it's for
  everybody. That's where we are now. There is no longer any point in
  trying to breathe out just what's good for you or to breathe in just
  your suffering and nobody else's. {[}LAUGHS{]}
\item
  cheryl strayed\\
  Right. Well, probably there was never a time to do that.
\item
  alice walker\\
  Well, oh, no, that's the American way.
\item
  cheryl strayed\\
  {[}LAUGHS{]} That's ---
\item
  alice walker\\
  I mean, you can breathe in what's happening on your block, but you
  don't really care what's happening down in the ghetto.
\item
  cheryl strayed\\
  Mhm.
\item
  alice walker\\
  So it's a different way of handling sorrow and distress. And I really
  prefer it because I like the idea that there's no longer any point in
  trying to save just oneself, you know? There's just no point.
\item
  cheryl strayed\\
  Yeah.
\item
  alice walker\\
  Mhm.
\item
  cheryl strayed\\
  I loved what you said about breathing in that suffering for all of us.
  And yet, your sentiment in taking the arrow out of the heart is about
  that we all really do have to find a way to treat the wound we
  ourselves have individually. I'm curious about, how have you done
  that? You've done the breathing. You've done that work of accepting
  your suffering and breathing it out.
\item
  alice walker\\
  Mhm.
\item
  cheryl strayed\\
  But it seems to me this is something we have to do for ourselves over
  and over and over again ---
\item
  alice walker\\
  That's right.
\item
  cheryl strayed\\
  --- to keep going.
\item
  alice walker\\
  That's right. And I mustn't leave out the big factor here, which is
  time. Time might be one of the biggest factor, probably the biggest
  factor, in healing. And that's one of the places we can get a little
  stuck because we don't want to take time. We want it done. I remember
  I had a friend who --- I was suffering terribly because of some love
  affair that had gone astray. And I said, well, how long did it take
  you to get over a broken heart or whatever? And she said two weeks.
  {[}LAUGHING{]}
\item
  cheryl strayed\\
  Oh, dear. That wasn't a very broken heart.
\item
  alice walker\\
  And I did say to her, I said, well, you know, really my heart's going
  to take a lot longer than that. And so it's like that, I think, that
  we have our sufferings. And in my life, so many of the people that I
  have loved have been assassinated. And at times, when these things
  happen, you can't imagine that you'll ever stand up again. I mean, the
  pain is so intense and the sense of loss. You know, when Martin Luther
  King was assassinated, I had a miscarriage. It was physical as well as
  mental and spiritual. And so there's no way of ever being able to say
  with any kind of truth when you will recover, when you will take that
  arrow out of your heart and stand up again either in the place that
  you were already standing or even a bit forward. You know, that's the
  test. I mean, can you get up where you fell and at least hold that? Or
  can you get up where you fell and then take a step forward?
\item
  cheryl strayed\\
  Mhm. So the first poem in your book, Taking the Arrow Out of the
  Heart, is called ``The Long Road Home.'' And it's about Muhammad Ali.
  Would you be so kind as to read it to me?
\item
  alice walker\\
  Sure, sure. Tell me why you like that one.
\item
  cheryl strayed\\
  I just, I love, really --- I so agree with you about this idea that
  our suffering can contain a gift.
\item
  alice walker\\
  Yes.
\item
  cheryl strayed\\
  And you write about that so beautifully and so powerfully across all
  of your books and all of the genres that you've written in. And so I
  saw aspects of that in this beautiful poem.
\item
  alice walker\\
  OK, well, I loved Muhammad Ali. And of course, he was himself with all
  his foibles. But ultimately, he was someone who could not be forced to
  murder other people. And he refused to be drafted into an immoral war.
  And so I hold him in very high esteem. ``The Long Road Home'' for
  Muhammad Ali. ``I am beginning to comprehend the mystery of the gift
  of suffering. It is true, as some have said, that it is a crucible in
  which the gold of one's spirit is rendered and shines. Ali, you
  represent all of us who stand the test of suffering most often alone,
  because who could understand who or what has brought us to our feet?
  Their knees worn out, ancestors stood us up from the awkward position
  they had to honor on the floor beneath the floor. I have been weeping
  all day thinking of this, the cloud of witness, the endless teaching,
  the long road home.''
\item
  cheryl strayed\\
  Beautiful. Tell me what you were thinking when you wrote this poem and
  what it means to you.
\item
  alice walker\\
  Well, his refusal to be drafted into the Vietnam War and the suffering
  that came to him because of that, because they put him in prison. And
  they took away his title. And they did all the things that they
  thought would break him as a human being. And he accepted --- you
  know, he found a place to stand. And he stood there. And this is also
  saying that this is our history, that we are standing on the shoulders
  of ancestors who could barely stand in the slave dungeons. But those
  ancestors are part of the reason that we stand, because they could not
  in that position. So it goes very deep in us now that we are
  surrounded by what I call the cloud of witness. And the cloud of
  witness for African-Americans are these ancestors who went through
  suffering that most Americans have never, ever thought about, still
  made sure that we got that we are supposed to honor other people. You
  know, we got that from them. And this is just such a miracle.
\item
  cheryl strayed\\
  Mhm. In your own life when you were a young woman, how did you get
  called to write? And how did you come to that consciousness about what
  you owe the ancestors or what we all owe them?
\item
  alice walker\\
  Oh, {[}DEEPLY EXHALES{]} I think, growing up, I was in a small
  community. People were close. And the music was good in church. And
  you just had this feeling, even before you understood it --- because
  we didn't study history exactly --- but we could get it that somehow
  this beautiful --- I mean, you can hear it in Aretha Franklin's voice.
  I mean, there's a good example. In Aretha Franklin's voice, she is
  carrying the ancestral medicine for the tribe.
\item
  cheryl strayed\\
  Hm, mhm.
\item
  alice walker\\
  And that is so important to know, you know? So yeah, I had my little
  Easter speeches. And I had adoring relatives and friends in the
  church. And unfortunately, or fortunately, I really got very tired of
  the church by the time I was 12. And I went off into nature.
\item
  cheryl strayed\\
  Mhm. And then when did you go off into writing?
\item
  alice walker\\
  And into writing? {[}LAUGHS{]} Probably around the same time because
  if all my family was in church, that meant I had privacy and I could
  write at home.
\item
  cheryl strayed\\
  Right. But when you first began seriously writing, did you experience
  it as a call? Or was it something more intellectual --- you loved
  books and you thought, I'll give it a try?
\item
  alice walker\\
  You know, Cheryl, I honestly feel like I was just like any plant that
  produces a bloom or a twig or whatever plants produce, that it was
  just so natural. In fact, my mother says that when I was crawling, she
  would find me at the back of the house writing in the dirt with a
  twig.
\item
  cheryl strayed\\
  {[}LAUGHS{]}
\item
  alice walker\\
  And I just --- it's just completely natural. It really is. It's just
  like, well, yeah, well, this is what one would do.
\item
  cheryl strayed\\
  Are you turning to your writing now in this time of disconnect? Are
  you finding yourself drawn to making that connection that we can make
  in writing? Or are you staying away from it?
\item
  alice walker\\
  Oh, I write all the time. And I have a blog, which I adore. I don't
  care if I have only five people who read it ---
\item
  cheryl strayed\\
  {[}LAUGHS{]}
\item
  alice walker\\
  --- because my sign is Aquarius. And we love the idea of just sending
  thoughts through the air. I get on there, and I offer what I have and
  send it off and think very little more about it. And like any other
  thing that's happening in nature --- you know, I was out this morning
  picking irises. And I was just marveling at how that green stem, where
  you would never see any purple or white or orange or anything,
  suddenly out of that comes this incredible flower. And I love that
  feeling of connectivity with what is just blooming naturally.
\item
  cheryl strayed\\
  Yeah, it is a powerful feeling. So I'm curious, how is it that --- you
  turn to nature, obviously, when you're seeking solace or beauty. What
  about wisdom?
\item
  alice walker\\
  Well, I read a lot. I've always been drawn to wonderful books, like
  the I Ching, which I adore, all those long Russian novels. And I mean,
  I don't know, just basically living and being open and interested. And
  wisdom just kind of accrues.
\item
  cheryl strayed\\
  Right, right. So I read an interview with you that your favorite book
  is Jane Eyre. I'm wondering if you could just tell me about that. What
  is it that you feel --- you said you felt such a deep kinship to that
  novel.
\item
  alice walker\\
  I love Jane Eyre because Charlotte Bronte was in the middle of nowhere
  up there, wherever they were --- {[}LAUGHS{]} cold.
\item
  cheryl strayed\\
  In England somewhere.
\item
  alice walker\\
  Right, right. And I'm trying to think of the actual village. But
  anyway, cold and dreary and no sun for most of the year, you know? And
  her father was oblivious to a lot. He never realized that his
  daughters were writing novels, for instance. But anyway, so there she
  was in this very strange place where she could not even admit being a
  woman writing a book. Her books had to be published under a man's
  name. And she persevered and created this incredibly enthralling,
  romantic, but at the same time, in a way because of Jane's character,
  stringent book which never lets women down. And this is a good thing
  to think about in some of these novels that people love. Often women
  are, in the end, let down. They're not permitted to truly test their
  mettle when it comes to what they will and will not accept. And I love
  Jane because she stands on her little feet there with this man that
  she falls in love with. And she realizes that they could live in sin
  because they couldn't get married because of the church.
\item
  cheryl strayed\\
  Yeah, they couldn't get married because he had --- right, he was
  already married.
\item
  alice walker\\
  He was married. And so the church, you know how the church is. So
  anyway ---
\item
  cheryl strayed\\
  Yeah.
\item
  alice walker\\
  --- they couldn't get married. And really, she loved him so
  passionately that another kind of woman would have said, you know,
  let's just go live in, I don't know, Egypt or somewhere. But no, I
  mean, Jane has real integrity.

  And I love that. I think women, especially --- you know, romance is
  all well and good. But when it makes you sell out your soul, it's not
  worth having. So that's one of the reasons I absolutely love it.
\item
  cheryl strayed\\
  When did you first read Jane Eyre? How old were you?
\item
  alice walker\\
  Oh, I was 13 or 14. {[}LAUGHING{]}
\item
  cheryl strayed\\
  {[}LAUGHS{]} OK.
\item
  alice walker\\
  But then I read it --- Cheryl, I read it a lot after that.
  {[}LAUGHING{]}
\item
  cheryl strayed\\
  Did you? {[}LAUGHS{]} Yeah, so you were swept up in that romance. You
  know, I think it's interesting, a couple of the times that you've
  mentioned struggle, it's been romantic heartbreak. And I'm wondering,
  are those the hardest things that you feel you've sort of survived and
  endured? Or what is the hardest thing you've lived through?
\item
  alice walker\\
  Well, yeah, I lived through my daughter's decision that I was not a
  good mother. And of course, I disagree. But there it is.
\item
  cheryl strayed\\
  Aw.
\item
  alice walker\\
  And we're fine now. So I'm just telling you that that was a really
  difficult passage.
\item
  cheryl strayed\\
  How did you get through that? How did you come to peace?
\item
  alice walker\\
  Well, I think we just did, you know? I think at some point after 10
  years and with a grandson whom I really adore, I think we came to our
  senses more or less.
\item
  cheryl strayed\\
  Mm-hmm.
\item
  alice walker\\
  And that's the other thing about time that I was referring to earlier.
  That part of getting the arrow out and finding the joy again is time.
  And what you do while you're waiting means a lot. And so these periods
  when there is so much pain are great for deepening oneself. And that's
  why study is a wonderful thing. I believe in study with my whole
  heart.
\item
  cheryl strayed\\
  Mm-hmm. Yeah, it's this concept of time I think has been really on my
  mind a lot. I have two teenage kids. And they are always inquiring
  like, OK, when can things go back to normal? And it's really hard
  because before this pandemic, we all sort of thought we knew what we
  were going to be doing this summer or what was going to happen next
  month. And it's very difficult to explain to them that we're in an era
  right now of deep uncertainty that shows us that we don't know what
  tomorrow will bring, which we never knew. But to talk to them about,
  like, even if this goes on a year, that's actually, relative to the
  rest of our lives, just a short period of time. It doesn't feel like
  it. And I know that that's always how discomfort or suffering or
  difficult times feel. They feel like your heart will never mend,
  right?
\item
  alice walker\\
  Mhm.
\item
  cheryl strayed\\
  And this pain will last forever. This estrangement will go on and on
  and on. And then something shifts, and it's a different time. I think
  that's such a wise thing to remember. But wow, it's hard to do that
  sometimes.
\item
  alice walker\\
  Yeah, it is. And I think with children, it's good to, I don't know,
  have them use this time to do something they never thought they'd have
  time to do. You might ask them, was there something that you thought
  you'd never get around to, that you'd never have time to do, you'd
  never have time to explore? This is a perfect time for that.
\item
  cheryl strayed\\
  Yeah, Alice, I'm really trying. {[}LAUGHS{]} I'm really trying to do
  that. But my teenagers are less cooperative with {[}LAUGHS{]} those
  instructions. Maybe if you tell them what to study, I can ---
  {[}LAUGHS{]}
\item
  alice walker\\
  No, no, no, Cheryl, trust me, no. This will probably go on a very long
  time ---
\item
  cheryl strayed\\
  Yeah.
\item
  alice walker\\
  --- in one form or another. Or it will recur in some other crazy way.
  And there will be plenty of time. So you just tell them the best
  counsel that you can offer. And then they will have to learn the rest.
\item
  cheryl strayed\\
  Yeah, I try to do that. I try to be a guide to my children and open a
  space for them, especially now that they're teenagers and they want
  that space. And here we are in this house all together all the time.
  And yeah, I'll keep trying. I'll keep trying, encouraging that.
\item
  alice walker\\
  Well, my heart goes out to all of you because I actually like, very
  much, solitude. {[}LAUGHING{]}
\item
  cheryl strayed\\
  {[}LAUGHING{]} You're content.
\item
  alice walker\\
  {[}LAUGHING{]} So I feel like, you know, well --- but I just think
  that whatever we have, we have to work with it. And we can, and we do.
\item
  cheryl strayed\\
  Yeah. So are you afraid that you're going to get the virus? Are you
  afraid of your own risks when it comes to this virus?
\item
  alice walker\\
  Well, I'm in a mode of accepting my fate and whatever that is, you
  know? I have no way, beyond trying to wear my gloves and my mask and
  keep people six feet away and all of that --- there's nothing more
  that I can do. And I've had a wonderful adventure. I mean, my life has
  been really just a huge adventure. I have no complaints about it
  really. I mean, whatever awful things that I had to basically help to
  change, we changed as much of those awful things as we could. I
  enjoyed that, even when it was dangerous. So I've had that. And you
  know, I, like everybody, in the middle of the night, I think, oh,
  dear, I hear it's not a pretty sight dying from this virus. But I then
  tell myself, well, dying doesn't take forever and that because I think
  that there is something on the other side of death, which is
  essentially more life. It may not be life that we recognize as life.
  {[}LAUGHS{]}
\item
  cheryl strayed\\
  What do you think is on the other side, what kind of life?
\item
  alice walker\\
  Well, nothing really dies, I mean, really. I mean, the personality
  here would not be around. But there is no such thing as just a
  non-existence.
\item
  cheryl strayed\\
  I guess we all become the dirt from which the next flower blooms,
  right?
\item
  alice walker\\
  Absolutely, yeah. So that's comforting.
\item
  cheryl strayed\\
  Yeah.
\item
  {[}music{]}\\
  Well, Alice, it's been really lovely to speak with you. Thank you so
  much for taking the time to chat with me today.
\item
  alice walker\\
  Take care.
\item
  cheryl strayed\\
  Thank you.
\item
  alice walker\\
  And your family --- to your family, say hello for me.
\item
  cheryl strayed\\
  All right. Bye, bye.
\item
  alice walker\\
  Bye.
\item
  {[}music{]}
\item
  cheryl strayed\\
  I'm Cheryl Strayed. This is ``Sugar Calling.'' Next week, Billy
  Collins.
\item
  {[}music{]}
\end{itemize}

Previous

More episodes ofSugar Calling

\href{https://www.nytimes3xbfgragh.onion/2020/05/20/podcasts/sugar-calling-joy-harjo-poetry-virus.html?action=click\&module=audio-series-bar\&region=header\&pgtype=Article}{\includegraphics{https://static01.graylady3jvrrxbe.onion/images/2020/05/22/podcasts/20sugar-hajo3/20sugar-hajo3-thumbLarge.jpg}}

May 20, 2020~~•~ 35:30`I Release You, Fear'

\href{https://www.nytimes3xbfgragh.onion/2020/05/13/podcasts/sugar-calling-billy-collins-poetry-virus.html?action=click\&module=audio-series-bar\&region=header\&pgtype=Article}{\includegraphics{https://static01.graylady3jvrrxbe.onion/images/2020/05/13/podcasts/13sugar-calling/13sugar-calling-thumbLarge.jpg}}

May 13, 2020`There's a Quiet All Over the World'

\href{https://www.nytimes3xbfgragh.onion/2020/05/06/podcasts/sugar-calling-alice-walker-quarantine-virus.html?action=click\&module=audio-series-bar\&region=header\&pgtype=Article}{\includegraphics{https://static01.graylady3jvrrxbe.onion/images/2020/05/06/podcasts/06sugarcalling/06sugarcalling-thumbLarge.jpg}}

May 6, 2020~~•~ 28:58`Whatever We Have, We Have to Work With It'

\href{https://www.nytimes3xbfgragh.onion/2020/04/29/podcasts/sugar-calling-judy-blume-quarantine-virus.html?action=click\&module=audio-series-bar\&region=header\&pgtype=Article}{\includegraphics{https://static01.graylady3jvrrxbe.onion/images/2020/04/29/podcasts/29sugarcalliing-blume-sub/29sugarcalliing-blume-sub-thumbLarge.jpg}}

April 29, 2020`This Terrible Thing Is Happening, but the World Goes On.'

\href{https://www.nytimes3xbfgragh.onion/2020/04/22/podcasts/sugar-calling-amy-tan-quarantine-virus.html?action=click\&module=audio-series-bar\&region=header\&pgtype=Article}{\includegraphics{https://static01.graylady3jvrrxbe.onion/images/2020/04/27/podcasts/22sugarcalling/22sugarcalling-thumbLarge.jpg}}

April 22, 2020~~•~ 39:19`You Don't Take Dictation. You Find the Truth.'

\href{https://www.nytimes3xbfgragh.onion/2020/04/15/podcasts/sugar-calling-pico-iyer-coronavirus.html?action=click\&module=audio-series-bar\&region=header\&pgtype=Article}{\includegraphics{https://static01.graylady3jvrrxbe.onion/images/2020/04/21/podcasts/15sugarcalling1/15sugarcalling1-thumbLarge.jpg}}

April 15, 2020~~•~ 35:45`Joyful Participation in a World of Sorrows'

\href{https://www.nytimes3xbfgragh.onion/2020/04/08/podcasts/sugar-calling-margaret-atwood-coronavirus.html?action=click\&module=audio-series-bar\&region=header\&pgtype=Article}{\includegraphics{https://static01.graylady3jvrrxbe.onion/images/2020/04/02/books/08sugarcalling1/08sugarcalling1-thumbLarge-v3.jpg}}

April 8, 2020~~•~ 34:32`Roll Up Your Sleeves, Girls'

\href{https://www.nytimes3xbfgragh.onion/2020/04/03/podcasts/sugar-calling-george-saunders-coronavirus.html?action=click\&module=audio-series-bar\&region=header\&pgtype=Article}{\includegraphics{https://static01.graylady3jvrrxbe.onion/images/2020/04/09/podcasts/03sugarcalling-image/merlin_171264408_4ac7fc67-d8cc-45b9-9ec6-bdd20672e694-thumbLarge.jpg}}

April 3, 2020~~•~ 41:16`Everything Is Always Keep Changing'

\href{https://www.nytimes3xbfgragh.onion/column/sugar-calling}{See All
Episodes ofSugar Calling}

Next

May 6, 2020

\begin{itemize}
\item
\item
\item
\item
\item
\item
\end{itemize}

\emph{\textbf{Listen and subscribe to our podcast from your mobile
device:}}
\textbf{\href{https://podcasts.apple.com/us/podcast/sugar-calling/id1505881384}{\emph{Via
Apple Podcasts}}} \emph{\textbf{\textbar{}}}
\textbf{\href{https://open.spotify.com/show/4U8hPiNGIBvTS9zLeiDCN7?si=gRyigD47SPWl-QWgNjgt2w}{\emph{Via
Spotify}}} \emph{\textbf{\textbar{}}}
\textbf{\href{https://www.stitcher.com/podcast/the-new-york-times/sugar-calling}{\emph{Via
Stitcher}}}

\hypertarget{thats-the-test-i-mean-can-you-get-up-where-you-fell-and-at-least-hold-that-or-can-you-get-up-where-you-fell-and-then-take-a-step-forward}{%
\subsection{`That's the test. I mean, can you get up where you fell and
at least hold that? Or can you get up where you fell and then take a
step
forward?'}\label{thats-the-test-i-mean-can-you-get-up-where-you-fell-and-at-least-hold-that-or-can-you-get-up-where-you-fell-and-then-take-a-step-forward}}

\emph{--- The author Alice Walker}

Today, Cheryl calls Alice Walker, the poet and novelist, at her home in
Mendocino, Calif. Alice tells Cheryl about her natural inclination for
writing: ``My mother says that when I was crawling, she would find me at
the back of the house, writing in the dirt with a twig.''

Cheryl asks Alice about remembering her dreams, and the two discuss
suffering and resilience --- via the boxing world champion Muhammad Ali.

\includegraphics{https://static01.graylady3jvrrxbe.onion/images/2020/05/06/podcasts/06sugarcalling/merlin_148381932_a062d891-1508-4959-8092-be7ae15b35c3-articleLarge.jpg?quality=75\&auto=webp\&disable=upscale}

\hypertarget{on-todays-episode}{%
\subsubsection{\texorpdfstring{\textbf{On today's
episode:}}{On today's episode:}}\label{on-todays-episode}}

\href{https://alicewalkersgarden.com/}{Alice Walker} is a poet, writer
and political activist. She is the author of over 30 novels, children's
books, and collections of poetry and short stories. Her 1982 novel ``The
Color Purple,''
\href{https://www.nytimes3xbfgragh.onion/2004/10/24/books/review/alice-walker-in-love-and-trouble.html}{made
her the first black woman to win the Pulitzer Prize in fiction}. That
novel also won the National Book Award, and was
\href{https://www.nytimes3xbfgragh.onion/2005/12/02/theater/reviews/one-womans-awakening-in-double-time.html}{adapted}
into two
\href{https://www.nytimes3xbfgragh.onion/2015/12/11/theater/review-the-color-purple-on-broadway-stripped-to-its-essence.html}{heart-clutching,
gospel-flavored} Broadway musicals.

\hypertarget{alice-walkers-quarantine-reading-list}{%
\subsubsection{\texorpdfstring{\textbf{Alice Walker's quarantine reading
list:}}{Alice Walker's quarantine reading list:}}\label{alice-walkers-quarantine-reading-list}}

\begin{itemize}
\item
  ``\href{https://us.macmillan.com/books/9781466890671}{The Coming},''
  by Daniel Black
\item
  ``\href{https://www.harpercollins.com/9780062915795/hitting-a-straight-lick-with-a-crooked-stick/}{Hitting
  a Straight Lick With a Crooked Stick},'' by Zora Neale Hurston
\end{itemize}

\begin{itemize}
\item
  ``\href{https://www.amazon.com/dp/B00U58TZH6/ref=dp-kindle-redirect?_encoding=UTF8\&btkr=1}{Signs
  Preceding the End of the World},'' by Yuri Herrera
\item
  ``\href{https://groveatlantic.com/book/small-fry/}{Small Fry},'' by
  Lisa Brennan-Jobs
\item
  ``\href{https://www.penguinrandomhouse.com/books/550171/the-water-dancer-by-ta-nehisi-coates/}{The
  Water Dancer},'' by Ta-Nehisi Coates
\end{itemize}

\begin{center}\rule{0.5\linewidth}{\linethickness}\end{center}

Cheryl Strayed is the author of ``Tiny Beautiful Things,'' ``Torch,''
``Brave Enough,'' and the New York Times best seller ``Wild.'' Her books
have been translated into more than 40 languages. She lives in Portland,
Ore.
\href{https://twitter.com/CherylStrayed?ref_src=twsrc\%5Egoogle\%7Ctwcamp\%5Eserp\%7Ctwgr\%5Eauthor}{@CherylStrayed}

``Sugar Calling'' is produced by Kelly Prime and edited by Sara
Sarasohn, with editorial oversight by Wendy Dorr. This episode was mixed
by Jamie Collazo. Our theme music is by Dan Powell.

Advertisement

\protect\hyperlink{after-bottom}{Continue reading the main story}

\hypertarget{site-index}{%
\subsection{Site Index}\label{site-index}}

\hypertarget{site-information-navigation}{%
\subsection{Site Information
Navigation}\label{site-information-navigation}}

\begin{itemize}
\tightlist
\item
  \href{https://help.nytimes3xbfgragh.onion/hc/en-us/articles/115014792127-Copyright-notice}{©~2020~The
  New York Times Company}
\end{itemize}

\begin{itemize}
\tightlist
\item
  \href{https://www.nytco.com/}{NYTCo}
\item
  \href{https://help.nytimes3xbfgragh.onion/hc/en-us/articles/115015385887-Contact-Us}{Contact
  Us}
\item
  \href{https://www.nytco.com/careers/}{Work with us}
\item
  \href{https://nytmediakit.com/}{Advertise}
\item
  \href{http://www.tbrandstudio.com/}{T Brand Studio}
\item
  \href{https://www.nytimes3xbfgragh.onion/privacy/cookie-policy\#how-do-i-manage-trackers}{Your
  Ad Choices}
\item
  \href{https://www.nytimes3xbfgragh.onion/privacy}{Privacy}
\item
  \href{https://help.nytimes3xbfgragh.onion/hc/en-us/articles/115014893428-Terms-of-service}{Terms
  of Service}
\item
  \href{https://help.nytimes3xbfgragh.onion/hc/en-us/articles/115014893968-Terms-of-sale}{Terms
  of Sale}
\item
  \href{https://spiderbites.nytimes3xbfgragh.onion}{Site Map}
\item
  \href{https://help.nytimes3xbfgragh.onion/hc/en-us}{Help}
\item
  \href{https://www.nytimes3xbfgragh.onion/subscription?campaignId=37WXW}{Subscriptions}
\end{itemize}
