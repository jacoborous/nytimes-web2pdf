Sections

SEARCH

\protect\hyperlink{site-content}{Skip to
content}\protect\hyperlink{site-index}{Skip to site index}

\href{https://myaccount.nytimes3xbfgragh.onion/auth/login?response_type=cookie\&client_id=vi}{}

\href{https://www.nytimes3xbfgragh.onion/section/todayspaper}{Today's
Paper}

\href{/section/opinion}{Opinion}\textbar{}Farewell, Office. You Were the
Last Boundary Between Work and Home.

\url{https://nyti.ms/2zOTfMV}

\begin{itemize}
\item
\item
\item
\item
\item
\item
\end{itemize}

Advertisement

\protect\hyperlink{after-top}{Continue reading the main story}

\href{/section/opinion}{Opinion}

Supported by

\protect\hyperlink{after-sponsor}{Continue reading the main story}

\hypertarget{farewell-office-you-were-the-last-boundary-between-work-and-home}{%
\section{Farewell, Office. You Were the Last Boundary Between Work and
Home.}\label{farewell-office-you-were-the-last-boundary-between-work-and-home}}

Your workplace shaped your identity in ways you never knew.

\href{https://www.nytimes3xbfgragh.onion/by/jennifer-senior}{\includegraphics{https://static01.graylady3jvrrxbe.onion/images/2018/10/26/opinion/jennifer-senior/jennifer-senior-thumbLarge.png}}

By \href{https://www.nytimes3xbfgragh.onion/by/jennifer-senior}{Jennifer
Senior}

Opinion columnist

\begin{itemize}
\item
  May 17, 2020
\item
  \begin{itemize}
  \item
  \item
  \item
  \item
  \item
  \item
  \end{itemize}
\end{itemize}

\includegraphics{https://static01.graylady3jvrrxbe.onion/images/2020/05/17/opinion/17senior/17senior-articleLarge.jpg?quality=75\&auto=webp\&disable=upscale}

The office, for the foreseeable future, is dead. Google and Facebook are
telling employees they can work remotely
\href{https://www.nytimes3xbfgragh.onion/2020/05/08/technology/coronavirus-work-from-home.html}{until
2021}. Twitter is allowing employees to work from home
``\href{https://www.forbes.com/sites/johnbbrandon/2020/05/12/this-is-huge-twitter-ceo-says-employees-can-work-from-home-forever/}{forever}.''
A number of big banks are contemplating never fully refilling their
office towers in Manhattan. Last week, my colleague Matthew Haag wrote a
\href{https://www.nytimes3xbfgragh.onion/2020/05/12/nyregion/coronavirus-work-from-home.html}{thoroughly
depressing story} in which the chief executive of Halstead Real Estate
asked him point blank: ``Looking forward, are people going to want to
crowd into offices?''

Call me crazy, but I'm still thinking: \emph{Yes}. Maybe not today,
maybe not tomorrow, but someday. The modern office may be the target of
bleak caricature --- the lighting is bad, the meetings are long, the
only recourse to boredom is filching a colleague's stapler and
\href{https://www.youtube.com/watch?v=glFrp-CmNVA}{embalming it in lemon
Jell-O} (if you work at Dunder Mifflin). But over the coming months, I
suspect that those of us who spent most of our careers in offices will
grow to miss them.

What will we miss about them, specifically? Camaraderie, for one thing.
Maybe it's obvious that offices are social hubs --- it's certainly an
idea that TV sitcoms and dramas have long grasped --- but the numbers
are still interesting. Two-thirds of all women who work outside the
home, for instance, say that ``the social aspect'' of their jobs is a
``major reason'' for showing up each day, according to
\href{https://www.uvu.edu/uwlp/docs/gallupworkandlifewelllived-oct.pdf}{a
comprehensive survey by Gallup}.

I'll admit I fall rather contentedly into this group. Until my mid-30s,
I was a serene creature of the cubicle. Not being religious, the office
was where I often found fellowship; not yet being married, it was where
I had a work spouse. For people in that liminal period of emergent
adulthood --- when they're still
\href{http://metta-spencer.blogspot.com/2006/01/old-schmoozers-and-machers.html}{schmoozers,
rather than machers}, to use the sociologist Robert Putnam's memorable
distinction --- the office can play a crucial and happy role.

And have I mentioned that offices are great places to find actual
spouses? A surprising number of marriages start in their fluorescent
halls. (Famous examples: Barack and Michelle, Bill and Melinda.) The
statistics on this phenomenon vary --- I've seen studies
\href{https://qz.com/1546677/around-40-of-us-couples-now-first-meet-online/}{ranging
from 11} to
\href{http://press.careerbuilder.com/2018-02-01-Office-Romance-Hits-10-Year-Low-According-to-CareerBuilders-Annual-Valentines-Day-Survey}{31
percent} --- but even the smallest number isn't trivial, and the most
outlandish examples can make for delightful trivia. Southwest Airlines
announced 21 years ago that more than 1,600 of its 26,900 employees were
married to each other. (Under the perhaps inevitable headline,
``\href{https://www.swamedia.com/releases/release-9a64e058fd5edca195eccf844b705447-Love-is-in-the-Air-at-Southwest-Airlines}{Love
is in the air}.'')

But the benefits to office life are more than just social. They are also
intellectual. Without offices, we miss out on the chance for
serendipitous encounters, and it's precisely those moments of felicitous
engagement that spark the best ideas.

Years ago, the productivity philosopher and author Adam Grant
\href{https://www.thecut.com/2015/01/what-we-give-up-when-we-become-entrepreneurs.html}{pointed
out to me} that
\href{https://www.cnn.com/2013/04/04/tech/post-it-note-history/index.html}{the
reason we have Post-it Notes} is because a chemist at 3M, Spencer
Silver, spent years trying fruitlessly to promote his low-tack adhesive
in and around the office --- until a churchgoing colleague, Art Fry,
finally saw one of his presentations and realized the sticky stuff would
be perfect for keeping his bookmarks affixed to his hymnals. Propinquity
made all the difference.

Another way to think about this: Working from home rather than the
office is sort of like shopping on Amazon rather than in a proper
bookstore. In a bookstore, you never know what you might find. You can't
even know what you don't know until you wander down the wrong aisle and
stumble across it.

But to me, the best arguments for the office have always been
psychological --- and never have they felt more urgent than at this
moment. I'll start with a subtle thing: Remote work leaves a terrible
feedback vacuum. Communication with colleagues is no longer casual but
effortful; no matter how hard you try, you're going to have less contact
--- particularly of the casual variety --- and with fewer people.

And what do we humans do in the absence of interaction? We invent
stories about what that silence means. They are often negative ones.
It's a formula for anxiety, misunderstanding, all-around messiness.

``You need time to develop informal patterns with colleagues, especially
if you don't know them well,'' Nancy Rothbard, a professor of management
at Wharton, told me. She added that power differences also complicate
things, and not in a way I found reassuring.
\href{https://www.ncbi.nlm.nih.gov/pubmed/17100492}{The literature}
suggests that if a boss delays in replying to an email, we underlings
assume he or she is off doing important things. But if \emph{we're} late
in replying, the boss assumes we're indolent or don't have much to say.
Great.

More broadly speaking, even without an office, there will still be
office politics. They're much easier to navigate if you can actually
\emph{see} your colleagues --- and therefore discern where the power
resides, how business gets done, and who the kind people are.

But perhaps the most profound effect of working in an office has to do
with our very sense of self. We live in an age where our identities
aren't merely assigned to us; they are realized and achieved, and places
are powerful triggers of them. How much do I feel like a columnist if
I'm wearing a 21-year-old Austin Powers T-shirt (``It's Cannes, baby!'')
and picking at my kid's leftovers as I type? I mean, somewhat, sure. But
I suspect I'd feel more like one if I got dolled up and walked into the
Times building each morning.

Rothbard, who's made a study of the borders between our professional and
domestic selves, told me she sees this confusion all the time. There are
``integrators,'' she said, who don't mind the dissolution of those
borders, and ``segmenters,'' who don't care for it. (``The pandemic,''
she said, ``is a segmenter's hell.'') It's hardly uncommon to have
multiple identities across multiple contexts, each of them authentic.
But remote work makes it awfully hard for segmenters to give full
expression to their professional selves, and when they do, it often
rattles those around them. ``Your kids may see you talking to your
employees in a different way and be like, `Who \emph{is} this person?'''
she told me.

But it's young people, I'd argue, who'll miss out most if the office
disappeared. Offices are often the very place where professional
identities are forged --- an especially valuable thing in an age of
declining religious engagement and deferred marriage and childbearing.
Yes, perhaps that's slightly ominous, just another depressing sign that
work has replaced religion as a source of meaning, as Derek Thompson
argued
\href{https://www.theatlantic.com/ideas/archive/2019/02/religion-workism-making-americans-miserable/583441/}{so
beautifully in The Atlantic} last year.

Unfortunately, technology has already collapsed the boundary between
work and home. The office, at least, was a solid membrane between the
two. And it may possibly be the last.

\emph{The Times is committed to publishing}
\href{https://www.nytimes3xbfgragh.onion/2019/01/31/opinion/letters/letters-to-editor-new-york-times-women.html}{\emph{a
diversity of letters}} \emph{to the editor. We'd like to hear what you
think about this or any of our articles. Here are some}
\href{https://help.nytimes3xbfgragh.onion/hc/en-us/articles/115014925288-How-to-submit-a-letter-to-the-editor}{\emph{tips}}\emph{.
And here's our email:}
\href{mailto:letters@NYTimes.com}{\emph{letters@NYTimes.com}}\emph{.}

\emph{Follow The New York Times Opinion section on}
\href{https://www.facebookcorewwwi.onion/nytopinion}{\emph{Facebook}}\emph{,}
\href{http://twitter.com/NYTOpinion}{\emph{Twitter (@NYTopinion)}}
\emph{and}
\href{https://www.instagram.com/nytopinion/}{\emph{Instagram}}\emph{.}

Advertisement

\protect\hyperlink{after-bottom}{Continue reading the main story}

\hypertarget{site-index}{%
\subsection{Site Index}\label{site-index}}

\hypertarget{site-information-navigation}{%
\subsection{Site Information
Navigation}\label{site-information-navigation}}

\begin{itemize}
\tightlist
\item
  \href{https://help.nytimes3xbfgragh.onion/hc/en-us/articles/115014792127-Copyright-notice}{©~2020~The
  New York Times Company}
\end{itemize}

\begin{itemize}
\tightlist
\item
  \href{https://www.nytco.com/}{NYTCo}
\item
  \href{https://help.nytimes3xbfgragh.onion/hc/en-us/articles/115015385887-Contact-Us}{Contact
  Us}
\item
  \href{https://www.nytco.com/careers/}{Work with us}
\item
  \href{https://nytmediakit.com/}{Advertise}
\item
  \href{http://www.tbrandstudio.com/}{T Brand Studio}
\item
  \href{https://www.nytimes3xbfgragh.onion/privacy/cookie-policy\#how-do-i-manage-trackers}{Your
  Ad Choices}
\item
  \href{https://www.nytimes3xbfgragh.onion/privacy}{Privacy}
\item
  \href{https://help.nytimes3xbfgragh.onion/hc/en-us/articles/115014893428-Terms-of-service}{Terms
  of Service}
\item
  \href{https://help.nytimes3xbfgragh.onion/hc/en-us/articles/115014893968-Terms-of-sale}{Terms
  of Sale}
\item
  \href{https://spiderbites.nytimes3xbfgragh.onion}{Site Map}
\item
  \href{https://help.nytimes3xbfgragh.onion/hc/en-us}{Help}
\item
  \href{https://www.nytimes3xbfgragh.onion/subscription?campaignId=37WXW}{Subscriptions}
\end{itemize}
