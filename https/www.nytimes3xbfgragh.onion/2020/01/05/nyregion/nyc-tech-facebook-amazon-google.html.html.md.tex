Sections

SEARCH

\protect\hyperlink{site-content}{Skip to
content}\protect\hyperlink{site-index}{Skip to site index}

\href{https://www.nytimes3xbfgragh.onion/section/nyregion}{New York}

\href{https://myaccount.nytimes3xbfgragh.onion/auth/login?response_type=cookie\&client_id=vi}{}

\href{https://www.nytimes3xbfgragh.onion/section/todayspaper}{Today's
Paper}

\href{/section/nyregion}{New York}\textbar{}Silicon Valley's Newest
Rival: The Banks of the Hudson

\url{https://nyti.ms/37G33Vt}

\begin{itemize}
\item
\item
\item
\item
\item
\item
\end{itemize}

Advertisement

\protect\hyperlink{after-top}{Continue reading the main story}

Supported by

\protect\hyperlink{after-sponsor}{Continue reading the main story}

\hypertarget{silicon-valleys-newest-rival-the-banks-of-the-hudson}{%
\section{Silicon Valley's Newest Rival: The Banks of the
Hudson}\label{silicon-valleys-newest-rival-the-banks-of-the-hudson}}

Amazon, Apple, Facebook and Google will soon have 20,000 workers in New
York City, many in offices on the West Side.

\includegraphics{https://static01.graylady3jvrrxbe.onion/images/2019/12/11/nyregion/nytech01/nytech01-articleLarge.jpg?quality=75\&auto=webp\&disable=upscale}

\href{https://www.nytimes3xbfgragh.onion/by/matthew-haag}{\includegraphics{https://static01.graylady3jvrrxbe.onion/images/2018/06/14/multimedia/author-matthew-haag/author-matthew-haag-thumbLarge.jpg}}

By \href{https://www.nytimes3xbfgragh.onion/by/matthew-haag}{Matthew
Haag}

\begin{itemize}
\item
  Published Jan. 5, 2020Updated Jan. 8, 2020
\item
  \begin{itemize}
  \item
  \item
  \item
  \item
  \item
  \item
  \end{itemize}
\end{itemize}

When Facebook was searching for another New York office, one big enough
to fit as many as 6,000 workers, more than double the number it
currently employs in the city, it had one major demand: It needed the
space urgently.

So after the company settled on Hudson Yards, the vast mini-city taking
shape on Manhattan's Far West Side, existing tenants were told to move
and a small army of construction workers quickly began to revamp the
building even before a lease had been signed.

Facebook's push to accommodate its booming operations is part of a rush
by the West Coast technology giants to expand in New York City. The
rapid growth is turning a broad swath of Manhattan into one of the
world's most vibrant tech corridors.

Four companies --- Amazon, Apple, Facebook and Google --- already have
big offices along the Hudson River, from Midtown to Lower Manhattan, or
have been hunting for new ones in recent months, often competing with
one another for the same space.

In all, the companies are expected to have roughly 20,000 workers in New
York by 2022.

Cities across the United States and around the world have long vied to
establish themselves as worthy rivals to Silicon Valley. New York City
is certainly not anywhere close to overtaking the Bay Area as the
nation's tech leader, but it is increasingly competing for tech
companies and talent.

New York's rise as a tech hub comes as industries that have long
dominated the city's economic landscape are transformed by technology,
and are themselves increasingly reliant on software engineers and other
highly skilled workers.

The growth in New York is occurring largely without major economic
incentives from the city and state governments. Officials are mindful of
the outcry last year over at least \$3 billion in public subsidies that
Amazon was offered to build a corporate campus in Queens.

The retail behemoth, stung by the backlash,
\href{https://www.nytimes3xbfgragh.onion/2019/02/14/nyregion/amazon-hq2-queens.html}{canceled
its plans abruptly in February}. It is continuing to add jobs in the
city, although at a slower pace.

Still,
\href{https://www.nytimes3xbfgragh.onion/2019/12/06/nyregion/amazon-hudson-yards.html}{Amazon's
announcement last month that it would lease space in Midtown} for 1,500
workers renewed a debate over whether incentives should be used to woo
huge tech companies to New York.

Opponents of the earlier deal, including Representative Alexandria
Ocasio-Cortez, Democrat of Queens, said Amazon's decision to expand in
Manhattan showed that New York was so attractive that tax breaks were
unnecessary.

Others responded that the Hudson Yards space the company was leasing
paled next to the campus proposed for Long Island City, Queens, and to
the 25,000 people Amazon had pledged to employ there.

Tech companies are choosing New York to tap into its deep and skilled
talent pool and to attract employees who prefer the city's diverse
economy over technology-dominated hubs on the West Coast. New York is
also closer to Europe, an important market.

``For a long time, if you lived in the broader tech sector, there was
inertia that brought you to Silicon Valley,'' said Julie Samuels,
executive director of Tech: NYC, a nonprofit industry group. ``So many
people wanted to live here and move here, but felt the jobs weren't
here. Now the jobs are here.''

Google has grown so quickly and is so squeezed for space that it is
temporarily leasing two buildings until a much larger development in
Manhattan near the Holland Tunnel, St. John's Terminal, is ready in
2022.

The big tech firms started in New York with small outposts. Google's
first New York employee, a sales worker, arrived in 2000, and worked out
of a Starbucks in Manhattan. It was the company's first office outside
California.

Tech industry offices were once mostly filled with sales and marketing
employees who needed to be closer to their customers and to industries
like fashion, finance, media and real estate that power the city's
economy.

Over the past five years, though, the makeup of the companies' combined
New York work force has come to resemble the West Coast version: a mix
of engineers and others involved in software development.

\hypertarget{the-new-tech-corridor}{%
\subsection{The New Tech Corridor}\label{the-new-tech-corridor}}

Four big technology companies will have a combined 20,000 workers in the
city by 2022, mostly concentrated along Manhattan's West Side. Squares
on the map show where Facebook, Google and Amazon have leased space, or
are planning to.

The New York Times

At Google's New York office, highly skilled workers now outnumber their
colleagues in sales and marketing. Of the
\href{https://www.amazon.jobs/en/search?offset=0\&result_limit=10\&sort=relevant\&job_type=Full-Time\&cities\%5B\%5D=New\%20York\%2C\%20New\%20York\%2C\%20USA\&distanceType=Mi\&radius=24km\&latitude=40.71455\&longitude=-74.00714\&loc_group_id=\&loc_query=New\%20York\%2C\%20New\%20York\%2C\%20United\%20States\&base_query=\&city=New\%20York\&country=USA\&region=New\%20York\&county=New\%20York\&query_options=\&}{nearly
800 job openings that Amazon has} in the city, more than half are for
developers, engineers and data scientists.

``Every line of business and every platform is represented quite
healthfully,'' said William Floyd, Google's head of external affairs in
New York, the company's largest office except for its Mountain View,
Calif., headquarters. ``Not everyone wants to be in California.''

Oren Michels, a tech adviser and investor who sold Mashery, a company
based in San Francisco, to Intel in 2013, said that New York City had
become a refuge for tech workers who did not want to be surrounded
solely by those working in the same industry.

``You have younger engineers and those sorts of people who frankly want
to live in New York City because it's a more interesting and fun place
to live,'' he said. ``San Francisco is turning into a company town and
the company is tech, both professionally and personally.''

Mr. Michels said that his family had bought a home in Manhattan in 2014
with a plan to split their time between San Francisco and New York. They
soon decided to live full time in New York, where Mr. Michels is on the
boards of four tech firms.

The number of tech jobs in New York City has surged 80 percent in the
past decade, to 142,600, from 79,400 in 2009,
\href{https://www.osc.state.ny.us/osdc/rpt1-2020.pdf}{according to the
New York State Comptroller's office}. (The business services industry,
which includes accountants and lawyers and is the largest private
sector, employed 762,000 people in 2018, according to the comptroller's
office.)

Since 2016, the number of job openings in the city's tech sector has
jumped 38 percent, an analysis for The Times by
\href{http://www.glassdoor.com/}{the jobs website Glassdoor} found. In
November, New York had the third-highest number of tech openings among
United States cities, 26,843, behind just San Francisco and Seattle.

\includegraphics{https://static01.graylady3jvrrxbe.onion/images/2020/01/05/nyregion/05nytech02/merlin_164749131_348982af-167f-45e0-8e30-3ed2a3c9b423-articleLarge.jpg?quality=75\&auto=webp\&disable=upscale}

It is not only the biggest tech firms that are growing in New York. From
2018 through the third quarter of 2019, investors pumped more than \$27
billion into start-ups in the New York City region, the second most in
that time for any area outside San Francisco, according to the
\href{https://www.pwc.com/us/en/moneytree-report/assets/moneytree-report-q3-2019.pdf}{MoneyTree
Report by PwC-CB Insights}. (Nearly \$100 billion was invested in
start-ups in the Silicon Valley area in that period.)

Industries like finance, retail and health care provide more jobs, but
the tech sector, with an average salary of \$153,000, has become one of
New York City's main economic drivers.

That has raised concerns about whether the industry is intensifying
income inequality and making New York unaffordable for more people.

The four big tech companies ``attract thousands of out-of-state
employees with advanced degrees and work experience, and drive
unprecedented influxes in luxury rentals, rent hikes, and the flipping
of buildings and private homes,'' said~Kiana~Davis,~a policy analyst at
the Urban Justice Center.

``It should go without saying,'' she added, ``that middle-income,
low-wage, poor and unemployed residents in these cities cannot access
the luxury housing market nor the rising rents and have been driven out
of their communities as a result.''

Jonathan Miller, president of Miller Samuel, a real estate appraisal
firm, said that the residential market in Manhattan had been strong in
areas where the tech firms had grown.

``I speak to brokerage groups twice a week, and the conversation is
always peppered with questions about the tech sector,'' Mr. Miller said.
``If you have 20,000 employees coming in who are high-wage earners, that
can have a pronounced impact.''

The major tech firms are expected to grow to the point that they are
among the largest private tenants in New York in the coming years,
rivaling longtime leaders like JPMorgan Chase.

Image

Big technology firms are expected to rival longtime leaders like JP
Morgan Chase as the largest private tenants in New York.Credit...Amr
Alfiky for The New York Times

Among companies in the technology, advertising, media and information
industries, Google and Facebook are now the largest tenants, beating out
legacy companies like Condé Nast, News Corp. and Warner Media, according
to an analysis performed for The Times by the real estate company
Cushman \& Wakefield.

Facebook employs 2,900 people in New York, and recently signed the lease
at Hudson Yards for 1.5 million square feet in three buildings. In
addition to providing space for 6,000 workers, the deal gives the
company an option to take over another several hundred thousand square
feet in the development.

Facebook executives initially set their sights on a marquee building on
Madison Avenue in the Flatiron district, not far from the company's
existing offices, according to a person familiar with Facebook's plans.

But then Facebook executives toured Hudson Yards and were impressed with
the amenities, including shops and restaurants, and with the short walk
to major subway lines.

A deal was struck in November, but with a requirement on Facebook's part
that about 300,000 square feet in two buildings, 30 and 55 Hudson Yards,
be ready very soon.

Workers were immediately brought in to begin preparing the space and to
move out existing tenants.

Two blocks east, Facebook is close to signing a lease for about 700,000
square feet in the
\href{https://www.nytimes3xbfgragh.onion/2017/08/17/nyregion/manhattans-farley-post-office-will-soon-be-a-grand-train-hall.html}{107-year-old
James A. Farley Building} across from Pennsylvania Station, according to
three people familiar with the deal. The property, also known as the
Farley Post Office, is being renovated by the Related Companies and
another developer,
\href{https://www.vno.com/office/property/the-farley-building/3313609/landing}{Vornado
Realty Trust}.

More than 2,500 employees could eventually work there.
(\href{https://www.wsj.com/articles/facebook-in-talks-for-new-york-office-in-deal-making-company-one-of-citys-largest-11575628203?mod=article_inline}{The
Wall Street Journal reported} earlier on the potential lease.)

Image

Facebook plans to put as many as 2,500 workers in the Farley Building,
left, opposite Pennsylvania Station.Credit...Hiroko Masuike/The New York
Times

``It's hard to predict future growth, but we believe New York is a
vibrant market with a tremendous pool of talent,'' a Facebook
spokeswoman, Jamila Reeves, said. She declined to comment on the
company's specific plans.

Just north of the Farley building,
Amazon\href{https://www.nytimes3xbfgragh.onion/2019/12/06/nyregion/amazon-hudson-yards.html}{said
recently} that it had signed a lease for 350,000 square feet in a
building on 10th Avenue near Hudson Yards, enough space for 1,500
employees. The social media company LinkedIn, whose New York offices are
not far away, in the Empire State Building, recently said it would
expand to four additional floors in the landmark property.

The tech titan whose intentions in New York are probably least known is
Apple.

Executives at the company, which has had an office in the Flatiron area,
have toured buildings in that neighborhood and in the Hudson Yards area
but a deal has not yet been signed. Apple has inquired about leasing
much less space than other big tech companies, roughly 50,000 square
feet.

Apple declined to comment.

For every West Coast company with a household name that has expanded in
New York, there are many large but lesser-known firms with headquarters
in the city.

One, Datadog, which provides cloud-based software for businesses, went
public in September and is valued at \$10.5 billion. The company has 480
employees in its New York offices, up from 125 three years ago.

Advertisement

\protect\hyperlink{after-bottom}{Continue reading the main story}

\hypertarget{site-index}{%
\subsection{Site Index}\label{site-index}}

\hypertarget{site-information-navigation}{%
\subsection{Site Information
Navigation}\label{site-information-navigation}}

\begin{itemize}
\tightlist
\item
  \href{https://help.nytimes3xbfgragh.onion/hc/en-us/articles/115014792127-Copyright-notice}{©~2020~The
  New York Times Company}
\end{itemize}

\begin{itemize}
\tightlist
\item
  \href{https://www.nytco.com/}{NYTCo}
\item
  \href{https://help.nytimes3xbfgragh.onion/hc/en-us/articles/115015385887-Contact-Us}{Contact
  Us}
\item
  \href{https://www.nytco.com/careers/}{Work with us}
\item
  \href{https://nytmediakit.com/}{Advertise}
\item
  \href{http://www.tbrandstudio.com/}{T Brand Studio}
\item
  \href{https://www.nytimes3xbfgragh.onion/privacy/cookie-policy\#how-do-i-manage-trackers}{Your
  Ad Choices}
\item
  \href{https://www.nytimes3xbfgragh.onion/privacy}{Privacy}
\item
  \href{https://help.nytimes3xbfgragh.onion/hc/en-us/articles/115014893428-Terms-of-service}{Terms
  of Service}
\item
  \href{https://help.nytimes3xbfgragh.onion/hc/en-us/articles/115014893968-Terms-of-sale}{Terms
  of Sale}
\item
  \href{https://spiderbites.nytimes3xbfgragh.onion}{Site Map}
\item
  \href{https://help.nytimes3xbfgragh.onion/hc/en-us}{Help}
\item
  \href{https://www.nytimes3xbfgragh.onion/subscription?campaignId=37WXW}{Subscriptions}
\end{itemize}
