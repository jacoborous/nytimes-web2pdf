Sections

SEARCH

\protect\hyperlink{site-content}{Skip to
content}\protect\hyperlink{site-index}{Skip to site index}

\href{https://www.nytimes3xbfgragh.onion/section/world/middleeast}{Middle
East}

\href{https://myaccount.nytimes3xbfgragh.onion/auth/login?response_type=cookie\&client_id=vi}{}

\href{https://www.nytimes3xbfgragh.onion/section/todayspaper}{Today's
Paper}

\href{/section/world/middleeast}{Middle East}\textbar{}U.S. Strike in
Iraq Kills Qassim Suleimani, Commander of Iranian Forces

\url{https://nyti.ms/36iPzyp}

\begin{itemize}
\item
\item
\item
\item
\item
\item
\end{itemize}

Advertisement

\protect\hyperlink{after-top}{Continue reading the main story}

Supported by

\protect\hyperlink{after-sponsor}{Continue reading the main story}

\hypertarget{us-strike-in-iraq-kills-qassim-suleimani-commander-of-iranian-forces}{%
\section{U.S. Strike in Iraq Kills Qassim Suleimani, Commander of
Iranian
Forces}\label{us-strike-in-iraq-kills-qassim-suleimani-commander-of-iranian-forces}}

Suleimani was planning attacks on Americans across the region, leading
to an airstrike in Baghdad, the Pentagon statement said. Iran's supreme
leader called for vengeance.

\includegraphics{https://static01.graylady3jvrrxbe.onion/images/2020/02/02/us/02iraq-airport-sub2/02iraq-airport-sub2-articleLarge-v2.jpg?quality=75\&auto=webp\&disable=upscale}

By \href{https://www.nytimes3xbfgragh.onion/by/michael-crowley}{Michael
Crowley}, Falih Hassan and
\href{https://www.nytimes3xbfgragh.onion/by/eric-schmitt}{Eric Schmitt}

\begin{itemize}
\item
  Published Jan. 2, 2020Updated July 9, 2020
\item
  \begin{itemize}
  \item
  \item
  \item
  \item
  \item
  \item
  \end{itemize}
\end{itemize}

WEST PALM BEACH, Fla. --- Iran's top security and intelligence commander
was killed early Friday in a drone strike at Baghdad International
Airport that was authorized by President Trump, American officials said.

The commander,
\href{https://www.nytimes3xbfgragh.onion/2020/01/03/world/middleeast/suleimani-dead.html}{Maj.
Gen. Qassim Suleimani}, who led the powerful
\href{https://www.nytimes3xbfgragh.onion/2020/01/03/world/middleeast/suleimani-dead.html}{Quds
Force of the Islamic Revolutionary Guards Corps}, was killed along with
several officials from Iraqi militias backed by Tehran when an American
MQ-9 Reaper drone fired missiles into a convoy that was leaving the
airport.

\href{https://www.nytimes3xbfgragh.onion/2020/07/09/world/middleeast/qassim-suleimani-killing-unlawful.html}{General
Suleimani} was the architect of nearly every significant operation by
Iranian intelligence and military forces over the past two decades, and
his death
\href{https://www.nytimes3xbfgragh.onion/2020/01/02/world/middleeast/qassem-soleimani-iraq-iran-attack.html}{was
a staggering blow for Iran} at a time of sweeping geopolitical conflict.

The strike was also a serious escalation of Mr. Trump's growing
confrontation with Tehran, one that began with the
\href{https://www.nytimes3xbfgragh.onion/2019/12/27/us/politics/american-rocket-attack-iraq.html}{death
of an American contractor} in Iraq in late December.

In Iran, the leadership convened an emergency security meeting. And the
country's supreme leader, Ayatollah Ali Khamenei, issued a statement
calling for three days of public mourning and then retaliation.

``His departure to God does not end his path or his mission,'' the
statement said, ``but a forceful revenge awaits the criminals who have
his blood and the blood of the other martyrs last night on their
hands.''

United States officials were braced for potential Iranian retaliatory
attacks, possibly including cyberattacks and terrorism, on American
interests and allies.

\href{https://www.nytimes3xbfgragh.onion/2020/01/03/world/middleeast/israel-soleimani.html}{Israel},
too, was preparing for Iranian strikes. Some of the country's most
popular tourist sites, including the ski resort at Hermon, were closed,
and the armed forces went on alert, officials said.

From the start of the Syrian civil war, General Suleimani was one of the
chief leaders of an effort to protect President Bashar al-Assad of Syria
--- an important Iranian ally --- that brought together disparate
militias, national security forces and regional powers, including Russia
in recent years.

But that was far from the only front he operated on. American officials
accuse General Suleimani of causing the deaths of hundreds of soldiers
during the Iraq war, when he provided Iraqi insurgents with advanced
bomb-making equipment and training. They also say he has masterminded
destabilizing Iranian activities that continue throughout the Middle
East and are aimed at the United States, Israel and Saudi Arabia.

``General Suleimani was actively developing plans to attack American
diplomats and service members in Iraq and throughout the region,'' the
Pentagon said in a statement. ``General Suleimani and his Quds Force
were responsible for the deaths of hundreds of American and coalition
service members and the wounding of thousands more.''

It did not elaborate on the specific intelligence that led them to carry
out General Suleimani's killing. The highly classified mission was set
in motion after
\href{https://www.nytimes3xbfgragh.onion/2019/12/27/us/politics/american-rocket-attack-iraq.html}{the
American contractor's death} on Dec. 27 during a rocket attack by an
Iranian-backed militia, a senior American official said.

In killing General Suleimani, Mr. Trump took an action that Presidents
George W. Bush and Barack Obama had rejected, fearing it would lead to
war between the United States and Iran.

\includegraphics{https://static01.graylady3jvrrxbe.onion/images/2020/01/02/us/02Iraq-quds-killed/02Iraq-quds-killed-articleLarge.jpg?quality=75\&auto=webp\&disable=upscale}

While many
\href{https://www.nytimes3xbfgragh.onion/2020/01/02/us/politics/us-iran-war.html}{Republicans
said that the president had been justified} in the attack, Mr. Trump's
most significant use of military force to date, critics of his Iran
policy called the strike a reckless unilateral escalation that could
have drastic and unforeseen consequences that could ripple violently
throughout the Middle East.

``Soleimani was an enemy of the United States. That's not a question,''
Senator Christopher S. Murphy, Democrat of Connecticut,
\href{https://twitter.com/ChrisMurphyCT/status/1212913952436445185?ref_src=twsrc\%5Egoogle\%7Ctwcamp\%5Eserp\%7Ctwgr\%5Etweet}{wrote
on Twitter}, using an alternate spelling of the Iranian's name. ``The
question is this --- as reports suggest, did America just assassinate,
without any congressional authorization, the second most powerful person
in Iran, knowingly setting off a potential massive regional war?''

Iran's foreign minister, Javad Zarif, called the killing of General
Suleimani an act of ``international terrorism'' and warned it was
``extremely dangerous \& a foolish escalation.''

``The US bears responsibility for all consequences of its rogue
adventurism,'' Mr. Zarif tweeted.

Speaking to reporters while on vacation at his Mar-a-Lago resort in Palm
Beach, Fla., on Tuesday night, hours after an assault on the American
Embassy in Baghdad that United States officials said was orchestrated by
Iran, Mr. Trump, who has repeatedly vowed to end American entanglements
in the Middle East, insisted that he did not want war.

``I don't think that would be a good idea for Iran. It wouldn't last
very long,'' Mr. Trump said. ``Do I want to? No. I want to have peace. I
like peace.''

After initial reports of the strike emerged on Thursday, Mr. Trump was
unusually cryptic, but he appeared to revel in the news when he
\href{https://twitter.com/realDonaldTrump/status/1212924762827046918}{posted
a tweet} that consisted only of the image of an American flag.

Within minutes, Twitter accounts associated with Iranian figures were
responding in kind, sending images of Iran's flag --- often accompanied
by dire threats of revenge.

The strikes followed a warning on Thursday afternoon from Defense
Secretary Mark T. Esper, who said the United States military would
pre-emptively strike Iranian-backed forces in Iraq and Syria if there
were signs the paramilitary groups were planning more attacks against
American bases and personnel in the region.

``If we get word of attacks, we will take pre-emptive action as well to
protect American forces, protect American lives,'' Mr. Esper said. ``The
game has changed.''

``This strike was aimed at deterring future Iranian attack plans,'' the
Pentagon statement said late Thursday. ``The United States will continue
to take all necessary action to protect our people and our interests
wherever they are around the world.''

In Iran, state television interrupted its programing to announce General
Suleimani's death, with the news anchor reciting the Islamic prayer for
the dead --- ``From God we came and to God we return'' --- beside a
picture of the general.

Hawkish Iran experts said the strike would be deeply painful for Iran's
leadership. ``This is devastating for the Iranian Revolutionary Guards
Corps, the regime and Khamenei's regional ambitions,'' said Mark
Dubowitz, the chief executive of the Foundation for Defense of
Democracies, referring to Ayatollah Khamenei.

``For 23 years, he has been the equivalent of the J.S.O.C. commander,
the C.I.A. director and Iran's real foreign minister,'' Mr. Dubowitz
said, using an acronym for the United States' Joint Special Operations
Command. ``He is irreplaceable and indispensable'' to Iran's military
establishment.

For those same reasons, other regional analysts warned, Iran is likely
to respond with an intensity of dangerous proportions.

``From Iran's perspective, it is hard to imagine a more deliberately
provocative act,'' said Robert Malley, the president and chief executive
of the International Crisis Group. ``And it is hard to imagine that Iran
will not retaliate in a highly aggressive manner.''

``Whether President Trump intended it or not, it is, for all practical
purposes, a declaration of war,'' added Mr. Malley, who served as White
House coordinator for the Middle East, North Africa and the gulf region
in the Obama administration.

Image

Iranians mourned the killing of General Suleimani in Tehran on
Friday.Credit...Abedin Taherkenareh/EPA, via Shutterstock

Some United States officials and Trump administration advisers offered a
less dire scenario, arguing that the show of force might convince Iran
that its acts of aggression against American interests and allies have
grown too dangerous, and that a president the Iranians may have come to
see as risk-averse is in fact willing to escalate.

One senior administration official said the president's senior advisers
had come to worry that Mr. Trump had sent too many signals --- including
when he
\href{https://www.nytimes3xbfgragh.onion/2019/09/21/us/politics/trump-iran-decision.html}{called
off a planned missile strike} in late June ---~that he did not want a
war with Iran.

Tracking Mr. Suleimani's location at any given time had long been a
priority for the American and Israeli spy services and militaries.
Current and former American commanders and intelligence officials said
that Thursday night's attack, specifically, drew upon a combination of
highly classified information from informants, electronic intercepts,
reconnaissance aircraft and other surveillance.

The strike killed five people, including the pro-Iranian chief of an
umbrella group for Iraqi militias, Iraqi television reported and militia
officials confirmed. The militia chief, Abu Mahdi al-Muhandis, was a
strongly pro-Iranian figure.

The public relations chief for the umbrella group, the Popular
Mobilization Forces in Iraq, Mohammed Ridha Jabri, was also killed.

American officials said that multiple missiles hit the convoy in a
strike carried out by the Joint Special Operations Command.

American military officials said they were aware of a potentially
violent response from Iran and its proxies, and were taking steps they
declined to specify to protect American personnel in the Middle East and
elsewhere around the world.

Two other people were killed in the strike, according to a general at
the Baghdad joint command, who spoke on the condition of anonymity
because he was not authorized to speak to the news media.

The Iraqi general said that General Suleimani and Mr. Ridha, the militia
public relations official, arrived by plane at Baghdad International
Airport from Syria.

A senior security official who was familiar with the operation's
details, and who spoke on condition of anonymity to discuss
intelligence, said that General Suleimani had been particularly troubled
by the wave of anti-Iran demonstration in Iraq and had flown in to urge
local militia forces to more forcibly curb the protests.

Two cars stopped at the bottom of the airplane steps and picked them up.
Mr. al-Muhandis was in one of the cars. As the cars left the airport,
they were struck, the general said.

The strike was the second attack at the airport within hours.

Terminals

Baghdad

International

Airport

Suleimani was in

a vehicle struck

by two missiles as

his convoy exited the airport.

airport st.

Terminals

Baghdad

International

Airport

Suleimani was in a vehicle struck by two missiles as his convoy exited
the airport.

airport st.

Terminals

Baghdad

International

Airport

Suleimani was in a vehicle struck by two missiles as his convoy exited
the airport.

airport st.

The New York Times; satellite image by Maxar via Bing.

An earlier attack, late Thursday, involved three rockets that did not
appear to have caused any injuries.

The strikes come days after American forces bombed three outposts of
Kataib Hezbollah, an Iranian-supported militia in Iraq and Syria, in
retaliation for the death of an American contractor in a rocket attack
last week near the Iraqi city of Kirkuk.

The United States said that Kataib Hezbollah fired 31 rockets into a
base in Kirkuk Province last week, killing an American contractor and
wounding several American and Iraqi servicemen.

The Americans responded by bombing three of the militia's sites near
Qaim in western Iraq, and two sites in Syria. Kataib Hezbollah denied
involvement in the attack in Kirkuk.

Pro-Iranian militia members then marched on the American Embassy on
Tuesday, effectively imprisoning its diplomats inside for more than 24
hours while thousands of militia members thronged outside. They burned
the embassy's reception area, planted militia flags on its roof and
scrawled graffiti on its walls.

No injuries or deaths were reported, and the militia members did not
enter the embassy building.

They withdrew late Wednesday afternoon.

Image

The drone strike occurred after protesters broke into the United States
Embassy compound in Baghdad on Tuesday.Credit...Khalid
Mohammed/Associated Press

The Pentagon statement Thursday night said that General Suleimani ``had
orchestrated attacks on coalition bases in Iraq over the last several
months,'' including the one that killed the American contractor last
week.

General Suleimani also ``approved the attacks on the U.S. Embassy in
Baghdad,'' the statement said.

Mr. Trump said on Tuesday that Iran would ``be held fully responsible''
for the attack on the embassy, in which protesters set fire to a
reception building on the embassy compound, which covers more than 100
acres. He also blamed Tehran for directing the unrest.

In the past several months, Iranian-supported militias have increased
rocket attacks on bases housing American troops. The Pentagon has
dispatched more than 14,000 troops to the region since May.

Caught in the middle is the Iraqi government, which is too weak to
establish any military authority over some of the more established
Iranian-supported Shiite militias.

On Thursday, Mr. Esper said the Iraqi government was not doing enough to
contain them. The Iraqis need to ``stop these attacks from happening and
get the Iranian influence out of the government,'' Mr. Esper said.

Representative Andy Kim, Democrat of New Jersey, who served as the
National Security Council's director for Iraq under Mr. Obama, said the
strike would most likely elicit ``a very serious backlash'' from a
number of Iraqi leaders for taking the action on Iraqi soil, as well as
from Shiite communities ``that already were protesting and upset in
recent days.''

``This is something that is going to make it very difficult for our
diplomatic presence there, our military presence there,'' Mr. Kim said
in an interview.

General Suleimani, who led the Islamic Revolutionary Guards Corps' Quds
Force, a special forces unit responsible for Iranian operations outside
Iran's borders, was long a figure of intense interest.

He was not only in charge of Iranian intelligence gathering and covert
military operations, he was regarded as one of Iran's most cunning and
autonomous military figures. He was also believed to be very close to
Ayatollah Khamenei, and was seen as a potential future leader of Iran.

The United States and Iran have long been involved in a shadow war in
battlegrounds across the Middle East --- including in Iraq, Yemen and
Syria. The tactics have generally involved using proxies to carry out
the fighting, providing a buffer from a direct confrontation between
Washington and Tehran that could draw America into yet other ground
conflict with no discernible endgame.

The potential for a regional conflagration was a basis of the Obama
administration's push for a 2015 agreement that froze Iran's nuclear
program in return for sanctions relief.

Mr. Trump withdrew from the deal in 2018, saying that Mr. Obama's
agreement had emboldened Iran, giving it economic breathing room to plow
hundreds of millions of dollars into a campaign of violence around the
region. Mr. Trump responded with a campaign of ``maximum pressure'' that
began with punishing new economic sanctions, which began a new era of
brinkmanship and uncertainty, with neither side knowing just how far the
other was willing to escalate violence and risk a wider war. In recent
days, it has spilled into the military arena.

General Suleimani once described himself to a senior Iraqi intelligence
official as the ``sole authority for Iranian actions in Iraq,'' the
official later told American officials in Baghdad.

In a speech denouncing Mr. Trump, General Suleimani was even less
discreet --- and openly mocking.

``We are near you, where you can't even imagine,'' he said. ``We are
ready. We are the man of this arena.''

Michael Crowley reported from West Palm Beach, Fla.; Falih Hassan from
Baghdad; and Eric Schmitt from Washington. Reporting was contributed by
Ronen Bergman from Tel Aviv, Israel; Alissa J. Rubin from Paris; Farnaz
Fassihi from New York; Thomas Gibbons-Neff, Helene Cooper, Mark
Mazzetti, Catie Edmondson and Edward Wong from Washington; and Tim
Arango from Los Angeles.

Advertisement

\protect\hyperlink{after-bottom}{Continue reading the main story}

\hypertarget{site-index}{%
\subsection{Site Index}\label{site-index}}

\hypertarget{site-information-navigation}{%
\subsection{Site Information
Navigation}\label{site-information-navigation}}

\begin{itemize}
\tightlist
\item
  \href{https://help.nytimes3xbfgragh.onion/hc/en-us/articles/115014792127-Copyright-notice}{©~2020~The
  New York Times Company}
\end{itemize}

\begin{itemize}
\tightlist
\item
  \href{https://www.nytco.com/}{NYTCo}
\item
  \href{https://help.nytimes3xbfgragh.onion/hc/en-us/articles/115015385887-Contact-Us}{Contact
  Us}
\item
  \href{https://www.nytco.com/careers/}{Work with us}
\item
  \href{https://nytmediakit.com/}{Advertise}
\item
  \href{http://www.tbrandstudio.com/}{T Brand Studio}
\item
  \href{https://www.nytimes3xbfgragh.onion/privacy/cookie-policy\#how-do-i-manage-trackers}{Your
  Ad Choices}
\item
  \href{https://www.nytimes3xbfgragh.onion/privacy}{Privacy}
\item
  \href{https://help.nytimes3xbfgragh.onion/hc/en-us/articles/115014893428-Terms-of-service}{Terms
  of Service}
\item
  \href{https://help.nytimes3xbfgragh.onion/hc/en-us/articles/115014893968-Terms-of-sale}{Terms
  of Sale}
\item
  \href{https://spiderbites.nytimes3xbfgragh.onion}{Site Map}
\item
  \href{https://help.nytimes3xbfgragh.onion/hc/en-us}{Help}
\item
  \href{https://www.nytimes3xbfgragh.onion/subscription?campaignId=37WXW}{Subscriptions}
\end{itemize}
