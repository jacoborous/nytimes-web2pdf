Sections

SEARCH

\protect\hyperlink{site-content}{Skip to
content}\protect\hyperlink{site-index}{Skip to site index}

\href{https://www.nytimes3xbfgragh.onion/section/health}{Health}

\href{https://myaccount.nytimes3xbfgragh.onion/auth/login?response_type=cookie\&client_id=vi}{}

\href{https://www.nytimes3xbfgragh.onion/section/todayspaper}{Today's
Paper}

\href{/section/health}{Health}\textbar{}The Flu Season May Yet Turn
Ugly, C.D.C. Warns

\url{https://nyti.ms/2T5Bvou}

\begin{itemize}
\item
\item
\item
\item
\item
\end{itemize}

Advertisement

\protect\hyperlink{after-top}{Continue reading the main story}

Supported by

\protect\hyperlink{after-sponsor}{Continue reading the main story}

Global health

\hypertarget{the-flu-season-may-yet-turn-ugly-cdc-warns}{%
\section{The Flu Season May Yet Turn Ugly, C.D.C.
Warns}\label{the-flu-season-may-yet-turn-ugly-cdc-warns}}

Almost as many people are falling ill as did two years ago, in what was
a particularly severe flu season. But this season's virus is unusual,
and it's too early to tell how dangerous.

\includegraphics{https://static01.graylady3jvrrxbe.onion/images/2020/01/08/science/08FLU/08FLU-articleLarge.jpg?quality=75\&auto=webp\&disable=upscale}

\href{https://www.nytimes3xbfgragh.onion/by/donald-g-mcneil-jr}{\includegraphics{https://static01.graylady3jvrrxbe.onion/images/2018/06/13/multimedia/author-donald-g-mcneil-jr/author-donald-g-mcneil-jr-thumbLarge-v4.png}}

By
\href{https://www.nytimes3xbfgragh.onion/by/donald-g-mcneil-jr}{Donald
G. McNeil Jr.}

\begin{itemize}
\item
  Published Jan. 8, 2020Updated Jan. 17, 2020
\item
  \begin{itemize}
  \item
  \item
  \item
  \item
  \item
  \end{itemize}
\end{itemize}

The United States may be headed into a bad flu season, according to
figures recently released by the Centers for Disease Control and
Prevention.

As of the last week of December,
\href{https://www.cdc.gov/flu/weekly/index.htm}{``widespread'' flu
activity was reported} by health departments in 46 states. More
ominously, a second measure --- the percentage of patients with flu
symptoms visiting medical clinics --- shot up almost to the peak reached
at the height of the 2017-18 flu season, which was
\href{https://www.nytimes3xbfgragh.onion/2018/01/26/health/flu-rates-deaths.html}{the
most severe in a decade}.

About\href{https://www.cdc.gov/flu/about/burden-averted/2017-2018.htm}{61,000
Americans died of flu that season}, the C.D.C. said. (The original
estimate of 79,000
\href{https://www.cdc.gov/flu/about/burden-averted/2017-2018.htm\#anchor_1574361280230}{was
revised downward} last year; the agency said the number changed as more
death certificate information became available.)

This year's flu vaccine may not be particularly effective against the
strain of the virus now widespread in the United States, experts said.
But even so, it's worth getting the shot: people who are vaccinated fare
better if struck by the flu than those who are not.

It is still too early to know how severe this season will be, said
\href{https://www.cdc.gov/flu/resource-center/partners/flu-fighter-lynnette-brammer.htm}{Lynnette
Brammer}, leader of the agency's domestic influenza surveillance team.

Although many people are coming down with flu, the two chief indicators
of severity --- hospitalizations and deaths --- are not yet elevated,
she noted.

Deaths from pneumonia and flu are actually lower than normal at this
time. But reports of hospitalization and death normally lag other
indicators by at least two weeks.

The current season
\href{https://time.com/5746409/early-flu-season-2019-2020/}{did begin
unusually early}. By late November, the flu had hit hard in the Deep
South, from Texas to Georgia. The virus then broke out in California and
the Rocky Mountain states, but was not widespread in the Northeast until
recently.

That pattern echoes what happened in Australia, where winter runs from
June through August.
\href{https://www.nytimes3xbfgragh.onion/2019/10/04/health/flu-australia-america.html}{Flu
came unusually early to the Southern Hemisphere} in 2019. In seasons
when Australia has a bad flu season, the Northern Hemisphere sometimes
does, too.

In another important way, however, the United States is not following
Australia's lead. The A(H3N2) strain of influenza was dominant there
last year, while most American cases this season have been caused by a
very different strain, called B Victoria. (B strains are named for the
cities where they were first isolated.)

B strain flus do not normally arrive until late in the season. But when
they do, ``they often impact children more than adults and older
adults,'' Ms. Brammer said.

\emph{{[}What parents need to know
about}\href{https://parenting.nytimes3xbfgragh.onion/childrens-health/flu-season-children}{\emph{this
flu season}}\emph{{]}}

The C.D.C. tracks the deaths of children individually, rather than
making estimates, as is done for adults. Those over 65 are usually the
group hardest hit by flu. Thus far this season, 27 children have died of
flu --- in 2017-18, 187 died --- but pediatric deaths don't normally
start peaking until mid-January.

On the rise now is the A(H1N1)pdm09 strain, which is a descendant of the
pandemic ``swine flu'' that first appeared in 2009 and then morphed into
a seasonal flu.

H1N1 strains are usually the first to appear. They usually cause fewer
hospitalizations and deaths per capita than B strains or A(H3N2).

Thus far, based on limited testing data, this season's flu
shot\href{https://www.cdc.gov/flu/season/faq-flu-season-2019-2020.htm}{does
not look like a good match} for the B Victoria flu and may not be very
effective, the C.D.C. said. But the shot does still appear to be well
matched for the A(H1N1)pdm09 strain.

C.D.C. flu data relies on reports from doctors' offices, clinics and
hospital emergency rooms about how many patients come in with flu
symptoms.

An even faster measurement of flu's spread comes from
\href{https://www.nytimes3xbfgragh.onion/2018/01/16/health/smart-thermometers-flu.html}{Kinsa
Health}, which collects daily readings of fevers from up to two million
users around the country who own its thermometers. The devices connect
to smartphones and instantly upload readings to the company's app.

\textbf{\emph{{[}}\href{http://on.fb.me/1paTQ1h}{\emph{Like the Science
Times page on Facebook.}}} ****** \emph{\textbar{} Sign up for the}
\textbf{\href{http://nyti.ms/1MbHaRU}{\emph{Science Times
newsletter.}}\emph{{]}}}

Kinsa readings indicate that flulike activity peaked on Dec. 24 at a
level just below the 2017-18 level --- confirming what the C.D.C. found
--- and has since dropped by almost a third, said Nita Nehru, a company
spokeswoman.

But even this week's lower figure ``is much higher than is typical of
this time of year,'' she added. It may bounce up again soon, now that
students have returned to school from holiday vacations.

The company assumes that fevers lasting three or more days indicate flu
rather than a common cold, said Inder Singh, the company's founder.

The C.D.C. has not endorsed Kinsa's methods, but the data does show flu
patterns at least a week or two ahead of reports from medical clinics.

Thus far, almost none of the hundreds of samples tested by the C.D.C.
have been resistant to Tamiflu or any other common antiflu drug. Those
medications do not cure the flu; they only reduce the severity of an
infection, and only if they are taken early.

Advertisement

\protect\hyperlink{after-bottom}{Continue reading the main story}

\hypertarget{site-index}{%
\subsection{Site Index}\label{site-index}}

\hypertarget{site-information-navigation}{%
\subsection{Site Information
Navigation}\label{site-information-navigation}}

\begin{itemize}
\tightlist
\item
  \href{https://help.nytimes3xbfgragh.onion/hc/en-us/articles/115014792127-Copyright-notice}{©~2020~The
  New York Times Company}
\end{itemize}

\begin{itemize}
\tightlist
\item
  \href{https://www.nytco.com/}{NYTCo}
\item
  \href{https://help.nytimes3xbfgragh.onion/hc/en-us/articles/115015385887-Contact-Us}{Contact
  Us}
\item
  \href{https://www.nytco.com/careers/}{Work with us}
\item
  \href{https://nytmediakit.com/}{Advertise}
\item
  \href{http://www.tbrandstudio.com/}{T Brand Studio}
\item
  \href{https://www.nytimes3xbfgragh.onion/privacy/cookie-policy\#how-do-i-manage-trackers}{Your
  Ad Choices}
\item
  \href{https://www.nytimes3xbfgragh.onion/privacy}{Privacy}
\item
  \href{https://help.nytimes3xbfgragh.onion/hc/en-us/articles/115014893428-Terms-of-service}{Terms
  of Service}
\item
  \href{https://help.nytimes3xbfgragh.onion/hc/en-us/articles/115014893968-Terms-of-sale}{Terms
  of Sale}
\item
  \href{https://spiderbites.nytimes3xbfgragh.onion}{Site Map}
\item
  \href{https://help.nytimes3xbfgragh.onion/hc/en-us}{Help}
\item
  \href{https://www.nytimes3xbfgragh.onion/subscription?campaignId=37WXW}{Subscriptions}
\end{itemize}
