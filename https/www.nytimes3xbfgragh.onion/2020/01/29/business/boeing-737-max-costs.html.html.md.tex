Sections

SEARCH

\protect\hyperlink{site-content}{Skip to
content}\protect\hyperlink{site-index}{Skip to site index}

\href{https://www.nytimes3xbfgragh.onion/section/business}{Business}

\href{https://myaccount.nytimes3xbfgragh.onion/auth/login?response_type=cookie\&client_id=vi}{}

\href{https://www.nytimes3xbfgragh.onion/section/todayspaper}{Today's
Paper}

\href{/section/business}{Business}\textbar{}Boeing Expects 737 Max Costs
Will Surpass \$18 Billion

\url{https://nyti.ms/2uJDuV2}

\begin{itemize}
\item
\item
\item
\item
\item
\item
\end{itemize}

Advertisement

\protect\hyperlink{after-top}{Continue reading the main story}

Supported by

\protect\hyperlink{after-sponsor}{Continue reading the main story}

\hypertarget{boeing-expects-737-max-costs-will-surpass-18-billion}{%
\section{Boeing Expects 737 Max Costs Will Surpass \$18
Billion}\label{boeing-expects-737-max-costs-will-surpass-18-billion}}

The grounding of the 737 Max jet continues to drag on the company's
results.

\includegraphics{https://static01.graylady3jvrrxbe.onion/images/2020/01/29/business/29boeing2/merlin_167603652_b458ecfc-2954-4d59-9b90-72649c6e5559-articleLarge.jpg?quality=75\&auto=webp\&disable=upscale}

\href{https://www.nytimes3xbfgragh.onion/by/david-gelles}{\includegraphics{https://static01.graylady3jvrrxbe.onion/images/2018/07/24/multimedia/author-david-gelles/author-david-gelles-thumbLarge.png}}

By \href{https://www.nytimes3xbfgragh.onion/by/david-gelles}{David
Gelles}

\begin{itemize}
\item
  Published Jan. 29, 2020Updated July 15, 2020
\item
  \begin{itemize}
  \item
  \item
  \item
  \item
  \item
  \item
  \end{itemize}
\end{itemize}

\href{https://www.nytimes3xbfgragh.onion/2020/01/22/business/trump-boeing-davos-737-max.html?searchResultPosition=2}{Boeing}
said on Wednesday that the costs associated with the grounding of the
\href{https://www.nytimes3xbfgragh.onion/2020/07/15/business/boeing-737-max-return.html}{737
Max} were likely to surpass \$18 billion, a significant increase over
earlier forecasts.

The new estimate, announced during Boeing's quarterly earnings report,
is the company's most recent approximation of just how expensive it will
be to return the Max to service, compensate airline customers and
restart the shuttered 737 factory.

Boeing continues to grapple with the fallout from the crashes of two Max
jets in 2018 and 2019, which killed 346 people, leading to the worldwide
grounding of the plane in March. In addition to the rising costs, the
company is contending with a new chief executive, the temporary shutdown
of the 737 factory and a range of challenges in other parts of the
business.

Boeing said on Wednesday that the costs associated with shutting down
and restarting the factory would amount to some **** \$4 billion. The
decision to
\href{https://www.nytimes3xbfgragh.onion/2019/12/16/business/boeing-737-max.html}{temporarily
halt production} of the Max was made only last month, and Boeing had not
previously given guidance on what the move would cost.

The company also said that the cost of compensating airlines that had
lost sales as a result of the grounding of the Max was now expected to
reach \$8.3 billion, up from a previous estimate of \$5.6 billion. That
figure represents a mixture of cash payments to airlines and discounts
on future sales.

And Boeing said that as a result of the grounding, which has lasted
nearly a year now, it expected the overall cost to produce the 737 Max
to rise to \$6.3 billion in the years ahead, up from an earlier estimate
of \$3.6 billion.

In total, the anticipated costs now equal more than \$18.6 billion, or
nearly 20 percent of Boeing's annual sales before the Max was grounded.

The Max crisis continued to weigh on the company's financial results.
Revenue for the quarter was \$17.9 billion, down 37 percent from a year
earlier, before the jet was grounded. For the full year, Boeing reported
revenues of \$76.6 billion, a 24 percent decline from the previous year.

Boeing also said it would incur a charge of \$410 million as a result of
its
\href{https://www.nytimes3xbfgragh.onion/2019/12/20/science/boeing-starliner-launch.html}{botched
rocket launch} late last year, when a space capsule it designed for the
National Aeronautics and Space Administration failed to reach the
correct orbit.

Wall Street analysts and traders viewed the results favorably, sending
Boeing shares up about 2 percent in midday trading.

``Taken together, the 737 items are a bit less than we feared,'' said
Jonathan Raviv, an analyst at Citi. ``But we acknowledge they're both
moving numbers lacking regulator clarity.''

This was the company's first quarterly earnings report with
\href{https://www.nytimes3xbfgragh.onion/2019/12/23/business/david-calhoun.html}{David
L. Calhoun} at the helm, after the ouster of the previous chief
executive,
\href{https://www.nytimes3xbfgragh.onion/2019/12/22/business/boeing-dennis-muilenburg-737-max.html?action=click\&module=RelatedLinks\&pgtype=Article}{Dennis
A. Muilenburg}.

Since taking over this month, Mr. Calhoun has tried to set himself apart
from Mr. Muilenburg, who was
\href{https://www.nytimes3xbfgragh.onion/2019/12/23/business/Boeing-ceo-muilenburg.html}{pushed
out} after alienating airline customers and the Federal Aviation
Administration with overly optimistic projections about when the Max
would return to service.

``We recognize we have a lot of work to do,'' Mr. Calhoun said in a
statement. ``We are focused on returning the 737 Max to service safely
and restoring the longstanding trust that the Boeing brand represents
with the flying public. We are committed to transparency and excellence
in everything we do.''

There is still no precise timeline for the return of the Max. Last week,
Boeing said it did not expect regulators to approve the plane to fly
\href{https://www.nytimes3xbfgragh.onion/2020/01/21/business/boeing-737-max.html}{until
June or July}, though that estimate was conservative. If regulators do
not find any additional problems with the plane, the Max could return to
service before then, though
\href{https://www.nytimes3xbfgragh.onion/2020/01/05/business/boeing-737-max.html}{new
issues} cropped up earlier in the process.

The company has enjoyed rare bits of good news in recent weeks. It
successfully completed the
\href{https://www.nytimes3xbfgragh.onion/reuters/2020/01/25/business/25reuters-boeing-777x-landing.html}{first
flight test} of the 777X, its new wide-body jet. And the
\href{https://www.nytimes3xbfgragh.onion/2020/01/15/business/economy/china-trade-deal.html}{trade
deal} that the White House struck with China included a commitment for
the sale of new American aircraft to Chinese customers.

Yet Boeing still faces enormous challenges. The grounding of the Max is
costing the company many billions of dollars, and costs are still
rising. The fatal crashes and a
\href{https://www.nytimes3xbfgragh.onion/2020/01/10/business/boeing-737-employees-messages.html}{cascade
of damning revelations} have badly damaged Boeing's reputation. Other
Boeing programs, including its work for NASA and the United States
military, are behind schedule.

On a conference call with the news media, Mr. Calhoun, who was chairman
of the company before becoming chief executive, said he was working to
reform Boeing's culture.

``I see things clearly that have to change,'' he said. ``My job is to
get on with it and make the changes that we always thought were
necessary.''

The Max is Boeing's most important product, representing hundreds of
billions of dollars in expected future sales. But just over a year after
it was introduced in 2017, a Max crashed off the coast of Indonesia,
after data from a faulty sensor triggered a new automated system. Less
than five months later, a second Max crashed in Ethiopia under similar
circumstances, leading to the worldwide grounding.

That has thrust Boeing into the biggest crisis in its history and led to
the idling of its 737 factory in Renton, Wash. Boeing has halted
deliveries of the Max during the grounding, and now has about 400
completed planes in storage. It will take well over a year to deliver
those to customers.

Boeing has developed a software update and has been working with
regulators to win approval to return the plane to service, but the
grounding is now likely to last at least a year.

Even when airlines do resume commercial flights with the Max, it may be
a challenge to persuade passengers to board the planes. Last month,
\href{https://www.nytimes3xbfgragh.onion/2019/12/24/business/boeing-737-max-survey.html}{Boeing's
own research} showed that 40 percent of travelers were unwilling to get
on the Max.

Advertisement

\protect\hyperlink{after-bottom}{Continue reading the main story}

\hypertarget{site-index}{%
\subsection{Site Index}\label{site-index}}

\hypertarget{site-information-navigation}{%
\subsection{Site Information
Navigation}\label{site-information-navigation}}

\begin{itemize}
\tightlist
\item
  \href{https://help.nytimes3xbfgragh.onion/hc/en-us/articles/115014792127-Copyright-notice}{©~2020~The
  New York Times Company}
\end{itemize}

\begin{itemize}
\tightlist
\item
  \href{https://www.nytco.com/}{NYTCo}
\item
  \href{https://help.nytimes3xbfgragh.onion/hc/en-us/articles/115015385887-Contact-Us}{Contact
  Us}
\item
  \href{https://www.nytco.com/careers/}{Work with us}
\item
  \href{https://nytmediakit.com/}{Advertise}
\item
  \href{http://www.tbrandstudio.com/}{T Brand Studio}
\item
  \href{https://www.nytimes3xbfgragh.onion/privacy/cookie-policy\#how-do-i-manage-trackers}{Your
  Ad Choices}
\item
  \href{https://www.nytimes3xbfgragh.onion/privacy}{Privacy}
\item
  \href{https://help.nytimes3xbfgragh.onion/hc/en-us/articles/115014893428-Terms-of-service}{Terms
  of Service}
\item
  \href{https://help.nytimes3xbfgragh.onion/hc/en-us/articles/115014893968-Terms-of-sale}{Terms
  of Sale}
\item
  \href{https://spiderbites.nytimes3xbfgragh.onion}{Site Map}
\item
  \href{https://help.nytimes3xbfgragh.onion/hc/en-us}{Help}
\item
  \href{https://www.nytimes3xbfgragh.onion/subscription?campaignId=37WXW}{Subscriptions}
\end{itemize}
