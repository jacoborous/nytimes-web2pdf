Sections

SEARCH

\protect\hyperlink{site-content}{Skip to
content}\protect\hyperlink{site-index}{Skip to site index}

\href{https://myaccount.nytimes3xbfgragh.onion/auth/login?response_type=cookie\&client_id=vi}{}

\href{https://www.nytimes3xbfgragh.onion/section/todayspaper}{Today's
Paper}

\href{/section/opinion}{Opinion}\textbar{}How to Avoid the Coronavirus?
Wash Your Hands

\url{https://nyti.ms/2vmSuIP}

\begin{itemize}
\item
\item
\item
\item
\item
\end{itemize}

Advertisement

\protect\hyperlink{after-top}{Continue reading the main story}

\href{/section/opinion}{Opinion}

Supported by

\protect\hyperlink{after-sponsor}{Continue reading the main story}

\hypertarget{how-to-avoid-the-coronavirus-wash-your-hands}{%
\section{How to Avoid the Coronavirus? Wash Your
Hands}\label{how-to-avoid-the-coronavirus-wash-your-hands}}

I covered the SARS outbreak as a reporter in China, and I saw that
common sense is the best defense against viral illness.

\href{https://www.nytimes3xbfgragh.onion/by/elisabeth-rosenthal}{\includegraphics{https://static01.graylady3jvrrxbe.onion/images/2019/09/25/opinion/elisabeth-rosenthal/elisabeth-rosenthal-thumbLarge.png}}

By
\href{https://www.nytimes3xbfgragh.onion/by/elisabeth-rosenthal}{Elisabeth
Rosenthal}

Ms. Rosenthal, a journalist and physician, is a contributing opinion
writer.

\begin{itemize}
\item
  Jan. 28, 2020
\item
  \begin{itemize}
  \item
  \item
  \item
  \item
  \item
  \end{itemize}
\end{itemize}

\includegraphics{https://static01.graylady3jvrrxbe.onion/images/2020/01/28/opinion/28rosenthal/28rosenthal-articleLarge.jpg?quality=75\&auto=webp\&disable=upscale}

\href{https://cn.nytimes3xbfgragh.onion/opinion/20200131/coronavirus-prevention-tips/}{阅读简体中文版}\href{https://cn.nytimes3xbfgragh.onion/opinion/20200131/coronavirus-prevention-tips/zh-hant/}{閱讀繁體中文版}

Americans are watching with alarm as a new
\href{https://www.nytimes3xbfgragh.onion/2020/01/29/business/china-coronavirus-economy.html}{coronavirus
spreads in China} and
\href{https://www.nytimes3xbfgragh.onion/interactive/2020/world/asia/china-wuhan-coronavirus-maps.html}{cases
pop up in the United States}. They are barraged with information about
what kinds of masks are best to prevent viral spread. Students are
\href{https://www.washingtonpost.com/world/coronavirus-china-live-updates/2020/01/25/0ca57a5e-3ed7-11ea-afe2-090eb37b60b1_story.html}{handing
out masks} in Seattle. Masks
\href{https://www.kbtx.com/content/news/Local-drug-stores-out-of-face-masks-after-potential-case-of-Coronavirus-detected-567279461.html}{have
run out} in Brazos County, Tex.

Hang on.

I've worked as an emergency room physician. And as a New York Times
correspondent in China, I covered the
\href{https://www.who.int/ith/diseases/sars/en/}{SARS outbreak} in 2002
and 2003 during which a novel coronavirus first detected in Guangdong
sickened more than 8,000 people and killed more than 800. My two
children attended elementary school in Beijing throughout the outbreak.

Here are my main takeaways from that experience for ordinary people on
the ground:

1. Wash your hands frequently.

2. Don't go to the office when you are sick. Don't send your kids to
school or day care when they are ill, either.

Notice I didn't say anything about masks. Having a mask with you as a
precaution makes sense if you are in the midst of an outbreak, as I was
when out reporting in the field during those months. But wearing it
constantly is another matter. I donned a mask when visiting hospitals
where SARS patients had been housed. I wore it in the markets where wild
animals that were the suspected source of the outbreak were being
butchered, blood droplets flying. I wore it in crowded enclosed spaces
that I couldn't avoid, like airplanes and trains, as I traveled to
cities involved in the outbreak, like Guangzhou and Hong Kong. You never
know if the guy coughing and sneezing two rows ahead of you is ill or
just has an allergy.

But outdoors, infections don't spread well through the air. Those photos
of people walking down streets in China wearing masks are dramatic but
uninformed. And remember if a mask has, perchance, intercepted viruses
that would have otherwise ended up in your body, then the mask is
contaminated. So, in theory, to be protected maybe you should use a new
one for each outing.

The simple masks are better than nothing, but not all that effective,
since they don't seal well. For anyone tempted to go out and buy the
gold standard, N95 respirators, note that they are uncomfortable.
Breathing is more work. It's hard to talk to people. On one long flight
at the height of the outbreak, on which my few fellow passengers were
mostly epidemiologists trying to solve the SARS puzzle, many of us
(including me) wore our masks for the first couple of hours on the
flight. Then the food and beverage carts came.

Though viruses spread through droplets in the air, a bigger worry to me
was always transmission via what doctors call ``fomites,'' infected
items. A virus gets on a surface --- a shoe or a doorknob or a tissue,
for example. You touch the surface and then next touch your face or rub
your nose. It's a great way to acquire illness. So after walking in the
animal markets, I removed my shoes carefully and did not take them into
the hotel room. And of course I washed my hands immediately.

Remember, by all indications SARS,
\href{https://www.washingtonpost.com/health/how-the-new-coronavirus-differs-from-sars-measles-and-ebola/2020/01/23/aac6bb06-3e1b-11ea-b90d-5652806c3b3a_story.html}{which
killed about 10 percent} of those infected, was a deadlier virus than
the new coronavirus circulating now. So keep things in perspective.

Faced with SARS, many foreigners chose to leave Beijing or at least to
send their children back to the United States. Our family stayed, kids
included. We wanted them with us and didn't want them to miss school,
especially during what would be their final year in China. But equally
important in making the decision was that the risk of getting SARS on an
airplane or in the airport seemed greater than being smart and careful
while staying put in Beijing.

And we were: I stopped taking my kids to indoor playgrounds or crowded
malls or delicious but densely packed neighborhood Beijing restaurants.
Out of an abundance of caution we canceled a family vacation to Cambodia
--- though my fear was less about catching SARS on the flight than that
one of the kids would have a fever from an ear infection upon our return
at a border screening, and would be stuck in a prolonged quarantine in
China. We instead took a vacation within China, where we carried masks
with us but didn't use them except on a short domestic flight.

In time, during the SARS outbreak, the government shut down theaters and
schools in Beijing, as it is doing now in many Chinese cities because
these viruses are more easily transmitted in such crowded places.

But there was also a lot of irrational behavior: Entering a village on
the way to a hike near the Great Wall, our car was stopped by locals who
had set up a roadblock to check the temperature of all passengers. They
used an oral thermometer that was only minimally cleaned after each use.
What a great way to spread a virus.

The International School of Beijing, where my children were students,
was one of the few in the capital --- perhaps the only one --- that
stayed open throughout the SARS outbreak, though the classes were
emptier, since so many kids had departed to their home countries. It was
a studied but brave move, since a parent at the school had gotten SARS
at the very beginning of the outbreak on a flight back from Hong Kong.
She recovered fine, but it was close to home and families were scared.

The school instituted a bunch of simple precautionary policies: a stern
note to parents reminding them not to send a child to school who was
sick and warning them that students would be screened for fevers with
ear thermometers at the school door. There was no sharing of food at
lunch. The teacher led the kids in frequent hand washing throughout the
day at classroom sinks, while singing a prolonged ``hand washing song''
to ensure they did more than a cursory pass under the faucet with water
only.

If a family left Beijing and came back, the child would have to stay at
home for an extended period before returning to class to make sure they
hadn't caught SARS elsewhere.

With those precautions in place, I observed something of a public health
miracle: Not only did no child get SARS, but it seemed no student was
sick with anything at all for months on end. No stomach bugs. No common
colds. Attendance was more or less perfect.

The World Health Organization
\href{https://www.history.com/this-day-in-history/world-health-organization-declares-sars-contained-worldwide}{declared
the SARS outbreak contained} in July 2003. But, oh, that those habits
persisted. The best first-line defenses against SARS or the new
coronavirus or most any virus at all are the ones that Grandma and
common sense taught us, after all.

\emph{The Times is committed to publishing}
\href{https://www.nytimes3xbfgragh.onion/2019/01/31/opinion/letters/letters-to-editor-new-york-times-women.html}{\emph{a
diversity of letters}} \emph{to the editor. We'd like to hear what you
think about this or any of our articles. Here are some}
\href{https://help.nytimes3xbfgragh.onion/hc/en-us/articles/115014925288-How-to-submit-a-letter-to-the-editor}{\emph{tips}}\emph{.
And here's our email:}
\href{mailto:letters@NYTimes.com}{\emph{letters@NYTimes.com}}\emph{.}

\emph{Follow The New York Times Opinion section on}
\href{https://www.facebookcorewwwi.onion/nytopinion}{\emph{Facebook}}\emph{,}
\href{http://twitter.com/NYTOpinion}{\emph{Twitter (@NYTopinion)}}
\emph{and}
\href{https://www.instagram.com/nytopinion/}{\emph{Instagram}}\emph{.}

Advertisement

\protect\hyperlink{after-bottom}{Continue reading the main story}

\hypertarget{site-index}{%
\subsection{Site Index}\label{site-index}}

\hypertarget{site-information-navigation}{%
\subsection{Site Information
Navigation}\label{site-information-navigation}}

\begin{itemize}
\tightlist
\item
  \href{https://help.nytimes3xbfgragh.onion/hc/en-us/articles/115014792127-Copyright-notice}{©~2020~The
  New York Times Company}
\end{itemize}

\begin{itemize}
\tightlist
\item
  \href{https://www.nytco.com/}{NYTCo}
\item
  \href{https://help.nytimes3xbfgragh.onion/hc/en-us/articles/115015385887-Contact-Us}{Contact
  Us}
\item
  \href{https://www.nytco.com/careers/}{Work with us}
\item
  \href{https://nytmediakit.com/}{Advertise}
\item
  \href{http://www.tbrandstudio.com/}{T Brand Studio}
\item
  \href{https://www.nytimes3xbfgragh.onion/privacy/cookie-policy\#how-do-i-manage-trackers}{Your
  Ad Choices}
\item
  \href{https://www.nytimes3xbfgragh.onion/privacy}{Privacy}
\item
  \href{https://help.nytimes3xbfgragh.onion/hc/en-us/articles/115014893428-Terms-of-service}{Terms
  of Service}
\item
  \href{https://help.nytimes3xbfgragh.onion/hc/en-us/articles/115014893968-Terms-of-sale}{Terms
  of Sale}
\item
  \href{https://spiderbites.nytimes3xbfgragh.onion}{Site Map}
\item
  \href{https://help.nytimes3xbfgragh.onion/hc/en-us}{Help}
\item
  \href{https://www.nytimes3xbfgragh.onion/subscription?campaignId=37WXW}{Subscriptions}
\end{itemize}
