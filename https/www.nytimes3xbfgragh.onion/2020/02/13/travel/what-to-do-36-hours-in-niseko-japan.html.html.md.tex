Sections

SEARCH

\protect\hyperlink{site-content}{Skip to
content}\protect\hyperlink{site-index}{Skip to site index}

\href{https://www.nytimes3xbfgragh.onion/section/travel}{Travel}

\href{https://myaccount.nytimes3xbfgragh.onion/auth/login?response_type=cookie\&client_id=vi}{}

\href{https://www.nytimes3xbfgragh.onion/section/todayspaper}{Today's
Paper}

\href{/section/travel}{Travel}\textbar{}36 Hours in Niseko

\url{https://nyti.ms/3byDdFx}

\begin{itemize}
\item
\item
\item
\item
\item
\item
\end{itemize}

Advertisement

\protect\hyperlink{after-top}{Continue reading the main story}

Supported by

\protect\hyperlink{after-sponsor}{Continue reading the main story}

\hypertarget{36-hours-in-niseko}{%
\section{36 Hours in Niseko}\label{36-hours-in-niseko}}

Sublime skiing, snowboarding and ``snow-surfing'' are only part of the
story in this Japanese resort. Culinary adventures abound, local whiskey
flows and hot springs are the perfect après-ski option.

\includegraphics{https://static01.graylady3jvrrxbe.onion/images/2020/02/13/travel/13Hours-Niseko1/13Hours-Niseko1-articleLarge.jpg?quality=75\&auto=webp\&disable=upscale}

By Janet O'Grady

\begin{itemize}
\item
  Feb. 13, 2020
\item
  \begin{itemize}
  \item
  \item
  \item
  \item
  \item
  \item
  \end{itemize}
\end{itemize}

Niseko --- where the powder is voluminous, après ski happens in onsens
(hot springs), and culinary adventures abound --- is a popular
destination for international travelers. Japow, the nickname for the
average 600 inches of powder that arrives from Siberia each winter, is
what puts the resort so firmly on the map with skiers and snowboarders.
On the north island of Hokkaido, Niseko includes four main separate, but
linked, resorts (collectively called \href{http://niseko.ne.jp}{Niseko
United}). Beyond the slopes, the island is renowned for world-class
seafood, produce and beef as well as beer, whiskey and even some sake.
Its excellent restaurants, from simple noodle bars and laid-back
izakayas (Japanese pubs) to fine dining at the Michelin-starred
\href{http://www.kamimura-niseko.com}{Kamimura}, spotlight the island's
bounty. Though often called the Aspen of Asia --- and, indeed, Niseko is
undergoing a similar luxury building boom --- this Japanese resort is
forging its own identity, from design to food to culture and wellness
scenes.

\hypertarget{36-hours-in-niseko-japan}{%
\subsection{36 Hours in Niseko, Japan}\label{36-hours-in-niseko-japan}}

\hypertarget{friday}{%
\subsection{Friday}\label{friday}}

\hypertarget{1-130-pm-artful-dining}{%
\subsubsection{1) 1:30 p.m. Artful
dining}\label{1-130-pm-artful-dining}}

Perched on the side of a cliff in the Hanazono area,
\href{http://Somoza.jp}{Somoza}, a serene gallery and artful restaurant,
is housed in a renovated wood kominka (historic farmhouse). A typical
thatched roof has been replaced with steel, but its traditional shape
still evokes a samurai helmet. Head downstairs to its main gallery where
an ongoing exhibition, ``Hokkaido Through the Ages,'' presents a history
lesson while showcasing artifacts from the founder and designer Shouya
Grigg's personal collection. There's pottery from the island's ancient
Jomon period (from about 10,000 B.C. to 300 B.C.); and a carved kabuto
sword stand with deer antlers and other works by the Ainu, people only
recently recognized by the government as Japan's indigenous inhabitants.
On the main level are handmade ceramics from Japanese artisans and Mr.
Grigg's minimalist black-and-white photographs of snow, trees, mountains
and abstract patterns inspired by nature. (Somoza will open an
additional adjacent art gallery this spring.)

Somoza serves some of the most creatively delicious food --- inspired
takes on Japanese cuisine with Italian influences --- in Niseko. Its
changing seasonal menus focus on local ingredients, such as duck breast
with Kutchan potato, crab and sea urchin consommé, or salmon smoked
on-site, all served on beautiful ceramics. Lunch is about 5,000 yen, or
about \$46. (Also, ask about the restaurant's matcha tea ceremony.) Sit
by its glass windows and gaze onto silver birch, oak and ezo red pine
trees. The restaurant offers free transportation from Hanazono for
lunch.

\hypertarget{2-5-pm-snow-surfing}{%
\subsubsection{2) 5 p.m. Snow-surfing}\label{2-5-pm-snow-surfing}}

West of downtown Hirafu village is
\href{http://www.Gentemstick.com}{Gentemstick}, the groovy shop of the
local legend Taro Tamai, the father of Japanese ``snow-surfing.'' This
Zen-like approach brings snowboarding back to its roots, with the rider
using body movements and techniques from surfing to follow the terrain
of the mountain. Lining the walls like sculptures are colorful surf
boards from Mr. Tamai's personal collection, and handcrafted snowboards
made with bamboo and other woods (88,000 to 162,000 yen). With good surf
breaks a short drive away, some locals even ski in the morning, then
surf in the afternoon. Purchase a beanie or T-shirt, then head upstairs
to the shop's cozy cafe, art gallery and yoga studio.

\includegraphics{https://static01.graylady3jvrrxbe.onion/images/2020/02/13/travel/13Hours-Niseko-A/merlin_167827839_86b183b8-9e79-4e21-b8f5-12d12a79433c-articleLarge.jpg?quality=75\&auto=webp\&disable=upscale}

\hypertarget{3-6-pm-strolling-through-hirafu}{%
\subsubsection{3) 6 p.m. Strolling through
Hirafu}\label{3-6-pm-strolling-through-hirafu}}

Hirafu village is Niseko's cultural heartbeat, brimming with bars and
restaurants, food trucks, coffee spots like the
\href{https://www.mountainkioskcoffee.com/}{Mountain Kiosk Coffee}
stand, ski shops, condos, boutique hotels and luxury chalets. Drop into
\href{http://www.powderart.net}{Powder Art Gallery} for contemporary
works by emerging artists from Tokyo, Europe and New York. Stop at
\href{http://nisekotaproom.com}{Niseko Taproom} for local craft beers
like Obihiro Kurouto or fruity Onuma Snowdrop I.P.A. Enjoy Hirafu's
organic funkiness before a new massive town center development, called
Aruku-zaka Street, comes in the next few years.

\hypertarget{4-730-pm-dinner-on-a-stick}{%
\subsubsection{4) 7:30 p.m. Dinner on a
stick}\label{4-730-pm-dinner-on-a-stick}}

Yakitori --- meat, seafood or vegetables on skewers --- is popular
throughout Japan, and \href{http://nisekobangbang.com}{Bang Bang} is a
Niseko institution. Reserve bar seats and watch chefs grill over a
special charcoal called binchotan, made from oak and valued for its
high, clean heat that enhances textures and flavors. From gizzards and
heart to neck and feet, almost every part of the chicken's here, with
crispy skin a standout. So are Hokkaido Wagyu beef skewers, Hokkaido
crabs and Akkeshi Kakiemon oysters (about 10,000 yen). If you can't get
a reservation, Bang 2 --- next door, its more casual eatery --- now
offers the same menu.

\hypertarget{5-930-pm-whiskey-nights}{%
\subsubsection{5) 9:30 p.m. Whiskey
nights}\label{5-930-pm-whiskey-nights}}

Down a dimly lit Hirafu side street, people stand in line to pass
through an old-fashioned red refrigerator door (an Instagram favorite)
plastered with stickers. Dubbed The Fridge, \href{http://gyubar.com}{Bar
Gyu+}, with its cozy speakeasy ambience and candlelit wooden tables, is
famous for its old and rare Japanese single-malts, a selection that
changes every season. Ask what's behind the bar for off-the-menu
pourings of sought-after whiskeys like Karuizawa or Hanyu, and expect to
pay almost 22,000 yen (\$201) for some shots. Sip as a D.J. spins vinyl,
mostly jazz tunes. Still want to dance? A short walk away is the fun new
\href{http://www.powderroomniseko.com/}{Powder Room}, an upscale club
with quality wines and cocktails that feels more Hong Kong than Niseko.

Image

Visitors ascend for their runs with Mountain Yotei in the
backgroundCredit...Andrew Faulk for The New York Times

\hypertarget{saturday}{%
\subsection{Saturday}\label{saturday}}

\hypertarget{6-8-am-fuel-up}{%
\subsubsection{6) 8 a.m. Fuel up}\label{6-8-am-fuel-up}}

In central Hirafu, \href{http://www.greenfarmcafe.com}{Green Farm Deli
\& Cafe} roasts its own coffee beans. Fuel up for the slopes over hearty
pork hash with poached eggs, or an egg wrap, all from local ingredients,
alongside your latte or cappuccino. Breakfast, about 2,000 yen.

\hypertarget{7-9-am-four-mountains}{%
\subsubsection{7) 9 a.m. Four mountains}\label{7-9-am-four-mountains}}

One pass offers access to Annupuri, Niseko Village, Grand Hirafu and
Hanazono; one day costs 8,000 yen if bought
\href{https://www.niseko.ne.jp/en/online-liftpass/}{online}. (Niseko
also participates in the global
\href{http://mountaincollective.com}{Mountain Collective} and
\href{http://ikonpass.com}{Ikon Pass}, season-long lift passes to top
worldwide resorts, including Niseko United.) Venture out with
\href{https://www.gosnowniseko.com/}{an instructor} to guide you around
the mountains for an overview, or to get tips for powder. Niseko's
slopes offer lots of variety, from beginners to advanced, with the
highest elevation only 3,937 feet. Intrepid skiers can go from one
resort to the next, beginning at Hanazono and ending at Annupuri. (A
free bus can also take you to each base.) At the top of the Niseko
gondola, ski to the sleek wooden
\href{http://www.niseko-village.com/en/thevillage/on-mountain.html}{Lookout
Cafe} and tuck into a bowl of simple seafood ramen (about 2,000 yen).
Trekking up and skiing down the crater of Mount Yotei --- a volcano,
resembling a smaller Mount Fuji, that looms over Niseko --- belongs on
your adrenaline bucket list (guide required; contact
\href{http://risingsunguides.com/}{Rising Sun Guides}).

Image

A skier descends Annupuri.Credit...Andrew Faulk for The New York Times

\hypertarget{8-230-pm-apruxe8s-ski-ritual}{%
\subsubsection{8) 2:30 p.m. Après ski
ritual}\label{8-230-pm-apruxe8s-ski-ritual}}

With Japan's volcanic landscape, there are ample onsens (geothermal hot
springs) in Niseko; taking to the waters at a public bathhouse is both
an essential ritual of Japanese culture and part of the ski experience.
Join the locals at
\href{http://www.niseko-annupurionsen.com}{Yugokorotei}, in a ryokan
(traditional Japanese inn) in Annupuri. Know the Japanese
\href{http://www.jnto.org.au/experience/onsen/onsen-etiquette}{etiquette},
such as bringing your own little ``modesty towel'' and soap to cleanse
thoroughly before dipping, naked (no bathing suits allowed), into a
communal outdoor pool. Like most onsen, this one is separated by gender.
The outdoor pool, under a wood pergola and surrounded by snow and
boulders, features mineral water pumped from the base of Mount Annupuri.
It may be cold outside, but you're relaxing your body and soul in about
130 degree Fahrenheit, mineral-rich bubbling water. Folding and placing
the towel on your head is a custom. Cost: 1,000 yen.

\hypertarget{9-6-pm-the-art-of-noodles}{%
\subsubsection{9) 6 p.m. The art of
noodles}\label{9-6-pm-the-art-of-noodles}}

Begin at \href{https://www.karabinaniseko.com/}{Karabina,} an izakaya
occupying a small wooden hut at Annupuri's base (note: this is the last
season the restaurant will be open at this location). Shoes off, walk up
a few steps to a cozy alcove, sit around a wood-burning stove and sip a
local sake. Then walk a stone's throw away down a pathway to another
rustic wood dwelling, \href{http://www.rakuichisoba.com}{Rakuichi}.
Helmed by Tatsuru Rai, known for his artisan, local buckwheat noodles,
this soba master was celebrated on Anthony Bourdain's ``No
Reservations,'' and ever since it's been a hard reservation to get. His
wife, Midori, greets diners as they don slippers and sit at a no-frills,
12-seat wooden counter with views of the master at work on a ball of
dough. Dinner comes kaiseki-style, a Japanese haute cuisine tasting menu
that changes with the seasons. Nine simple dishes with bright flavors
are presented on pretty Japanese ceramic plates and lacquer bowls. Order
a Hokkaido sake to accompany, for instance, sashimi of sea urchin, toro
tuna, smoked scallop and Mr. Rai's hand-cut soba noodles. Select either
cold with tempura vegetables, or hot with duck (an additional 800 yen).
Finish with dessert. (Dinner is about 12,000 yen.)

\hypertarget{10-930-pm-niseko-noir}{%
\subsubsection{10) 9:30 p.m. Niseko noir}\label{10-930-pm-niseko-noir}}

Stop at \href{http://toshiros-bar.com}{Toshiro's} for a cocktail created
by its bespectacled namesake proprietor and mixologist, celebrated for
concoctions like Penicillin, a mix of whiskey, ginger and citrus, with a
smoky spin (1,600 yen). Or try a ginger gimlet and a smoked old
fashioned with local whiskey. More than 400 bottles sit behind the bar;
select a tasting flight, from 3,600 to 45,000 yen.

Image

Food trucks serve hot meals near the base of the Hirafu ski
area.Credit...Andrew Faulk for The New York Times

\hypertarget{sunday}{%
\subsection{Sunday}\label{sunday}}

\hypertarget{11-10-am-shrines-and-temples}{%
\subsubsection{11) 10 a.m. Shrines and
temples}\label{11-10-am-shrines-and-temples}}

Grab a coffee at the Mountain Kiosk Coffee stand or one of the trucks in
Hirafu and head to the non-touristy town of Kutchan (about six miles
from Hirafu and 20 minutes by cab) where you'll find the
\href{http://visitniseko.tourismbuilder.com}{Daibutsuji Buddhist
temple}, featuring a prayer room with a gold-painted ceiling depicting a
dragon shielding a Buddhist elder from a tiger (no charge, book in
advance). Be mindful that temples and shrines are places of worship for
local residents, as well as places to protect sacred objects. Other area
shrines: Kutchan-jinja, where red, green and yellow flags line the
simple wood interior; and in Niseko Town stands the small
Kaributo-jinja.

\hypertarget{12-1-pm-swirl-and-other-milk-treats}{%
\subsubsection{12) 1 p.m. Swirl and other milk
treats}\label{12-1-pm-swirl-and-other-milk-treats}}

Hokkaido, Japan's top dairy-producing region, is recognized for offering
the best milk in the country.
\href{http://www.niseko-takahashi.jp/milkkobo}{Milk Kobo}, next to the
Takahashi working dairy farm in Niseko Village, is noted for
hand-milking its cows, and its popular cafe and shop makes desserts and
cheeses from the milk. Knock back a banana yogurt drink, which is
packaged in a cute little bottle with cows on its label. Don't miss the
cheese tarts and cream puffs.

\begin{center}\rule{0.5\linewidth}{\linethickness}\end{center}

\hypertarget{lodging}{%
\subsection{Lodging}\label{lodging}}

At the nine-room boutique \href{http://kimamaya.com}{Kimamaya} in
Hirafu, European alpine design meets Japanese aesthetics. Feel at home
sitting around the living-room fireplace, sipping a glass of Burgundy
from the owner's private vineyard. An adjacent restaurant,
\href{http://www.nisekobarn.com/}{The Barn,} inspired by Hokkaido farm
architecture, serves Western and Japanese breakfasts, and for dinner,
French-Japanese food using local ingredients; it's worth eating here
even if staying elsewhere. There's a small spa and two stone soaking
tubs. Rates: 22,400 to 55,440 yen for doubles; lofts are 41,440 to
94,080 yen.

Nestled in a picturesque forest, the stunning 15-room
\href{http://zaborin.com/en/}{Zaborin} fuses a traditional ryokan
experience with contemporary luxury. Guests relax in Japanese house
pajamas, and dinner is an 11-course kaiseki meal of beautifully
presented plates, with many foraged local ingredients. Rooms feature
indoor and outdoor onsens. Rates start at 75,000 yen.

If you chose the Airbnb route, try to find a property in the vibrant
Hirafu area. Here you can easily walk to restaurants and shops, and are
near public transportation. Expect to pay around \$155 and up for a one
bedroom.

\begin{center}\rule{0.5\linewidth}{\linethickness}\end{center}

\textbf{52 PLACES AND MUCH, MUCH MORE} \emph{Discover}
\href{https://www.nytimes3xbfgragh.onion/interactive/2020/travel/places-to-visit.html}{\emph{the
best places to go in 2020,}} \emph{and find more Travel coverage by
following us on}
\href{https://twitter.com/nytimestravel}{\emph{Twitter}} \emph{and}
\href{https://www.facebookcorewwwi.onion/nytimestravel/}{\emph{Facebook}}\emph{.
And}
\href{https://www.nytimes3xbfgragh.onion/newsletters/traveldispatch?action=click\&module=inline\&pgtype=Article}{\emph{sign
up for our}} **
\href{https://www.nytimes3xbfgragh.onion/newsletters/traveldispatch}{\emph{Travel
Dispatch newsletter}}\emph{: Each week you'll receive tips on traveling
smarter, stories on hot destinations and access to photos from all over
the world.}

Advertisement

\protect\hyperlink{after-bottom}{Continue reading the main story}

\hypertarget{site-index}{%
\subsection{Site Index}\label{site-index}}

\hypertarget{site-information-navigation}{%
\subsection{Site Information
Navigation}\label{site-information-navigation}}

\begin{itemize}
\tightlist
\item
  \href{https://help.nytimes3xbfgragh.onion/hc/en-us/articles/115014792127-Copyright-notice}{©~2020~The
  New York Times Company}
\end{itemize}

\begin{itemize}
\tightlist
\item
  \href{https://www.nytco.com/}{NYTCo}
\item
  \href{https://help.nytimes3xbfgragh.onion/hc/en-us/articles/115015385887-Contact-Us}{Contact
  Us}
\item
  \href{https://www.nytco.com/careers/}{Work with us}
\item
  \href{https://nytmediakit.com/}{Advertise}
\item
  \href{http://www.tbrandstudio.com/}{T Brand Studio}
\item
  \href{https://www.nytimes3xbfgragh.onion/privacy/cookie-policy\#how-do-i-manage-trackers}{Your
  Ad Choices}
\item
  \href{https://www.nytimes3xbfgragh.onion/privacy}{Privacy}
\item
  \href{https://help.nytimes3xbfgragh.onion/hc/en-us/articles/115014893428-Terms-of-service}{Terms
  of Service}
\item
  \href{https://help.nytimes3xbfgragh.onion/hc/en-us/articles/115014893968-Terms-of-sale}{Terms
  of Sale}
\item
  \href{https://spiderbites.nytimes3xbfgragh.onion}{Site Map}
\item
  \href{https://help.nytimes3xbfgragh.onion/hc/en-us}{Help}
\item
  \href{https://www.nytimes3xbfgragh.onion/subscription?campaignId=37WXW}{Subscriptions}
\end{itemize}
