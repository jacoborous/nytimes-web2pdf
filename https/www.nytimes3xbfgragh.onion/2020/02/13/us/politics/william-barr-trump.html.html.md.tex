Sections

SEARCH

\protect\hyperlink{site-content}{Skip to
content}\protect\hyperlink{site-index}{Skip to site index}

\href{https://www.nytimes3xbfgragh.onion/section/politics}{Politics}

\href{https://myaccount.nytimes3xbfgragh.onion/auth/login?response_type=cookie\&client_id=vi}{}

\href{https://www.nytimes3xbfgragh.onion/section/todayspaper}{Today's
Paper}

\href{/section/politics}{Politics}\textbar{}Barr Says Attacks From Trump
Make Work `Impossible'

\url{https://nyti.ms/2tVh4jH}

\begin{itemize}
\item
\item
\item
\item
\item
\item
\end{itemize}

Advertisement

\protect\hyperlink{after-top}{Continue reading the main story}

Supported by

\protect\hyperlink{after-sponsor}{Continue reading the main story}

\hypertarget{barr-says-attacks-from-trump-make-work-impossible}{%
\section{Barr Says Attacks From Trump Make Work
`Impossible'}\label{barr-says-attacks-from-trump-make-work-impossible}}

The attorney general said that the president's attacks on prosecutors'
handling of his friend Roger Stone's sentencing undermine the legal
system and the Justice Department.

\includegraphics{https://static01.graylady3jvrrxbe.onion/images/2020/02/15/us/politics/13dc-barr/13dc-barr-articleLarge-v2.jpg?quality=75\&auto=webp\&disable=upscale}

\href{https://www.nytimes3xbfgragh.onion/by/katie-benner}{\includegraphics{https://static01.graylady3jvrrxbe.onion/images/2018/02/16/multimedia/author-katie-benner/author-katie-benner-thumbLarge-v2.png}}

By \href{https://www.nytimes3xbfgragh.onion/by/katie-benner}{Katie
Benner}

\begin{itemize}
\item
  Published Feb. 13, 2020Updated Feb. 19, 2020
\item
  \begin{itemize}
  \item
  \item
  \item
  \item
  \item
  \item
  \end{itemize}
\end{itemize}

WASHINGTON --- Attorney General
\href{https://www.nytimes3xbfgragh.onion/2020/02/19/us/politics/trump-barr-justice-department.html}{William
P. Barr} delivered an extraordinary rebuke of
\href{https://www.nytimes3xbfgragh.onion/2020/02/19/us/politics/trump-barr-justice-department.html}{President
Trump} on Thursday, saying that his attacks on the Justice Department
had made it ``impossible for me to do my job'' and that ``I'm not going
to be bullied or influenced by anybody.''

Mr. Barr has been among the president's most loyal allies and denigrated
by Democrats as nothing more than his personal lawyer, but he publicly
challenged Mr. Trump in a way that no sitting cabinet member has.

``Whether it's Congress, newspaper editorial boards or the president,
I'm going to do what I think is right,'' Mr. Barr said in an
\href{https://abcnews.go.com/Politics/barr-blasts-trumps-tweets-stone-case-impossible-job/story?id=68963276\&cid=social_twitter_wnt}{interview
with ABC News}, echoing comments he made a year ago at his confirmation
hearing. ``I cannot do my job here at the department with a constant
background commentary that undercuts me.''

Mr. Barr's remarks were aimed at containing the fallout from the
department's botched
\href{https://www.nytimes3xbfgragh.onion/2020/02/12/us/politics/trump-stone.html}{handling
of its sentencing recommendation} for Mr. Trump's longtime friend Roger
J. Stone Jr., who was
\href{https://www.nytimes3xbfgragh.onion/2019/11/15/us/politics/roger-stone-trial-guilty.html}{convicted}
of seven felonies in a bid to obstruct a congressional investigation
that threatened the president. After career prosecutors initially
recommended a sentence of seven to nine years in prison, Mr. Trump spent
days attacking them, the department and the judge presiding over Mr.
Stone's case.

Such tweets ``make it impossible for me to do my job and to assure the
courts and the prosecutors in the department that we're doing our work
with integrity,'' Mr. Barr said.

He added, ``It's time to stop the tweeting about Department of Justice
criminal cases.''

The fallout from the Stone episode threatened to spin out of control
after the four prosecutors on the case withdrew from it and Mr. Trump
widened his attacks on law enforcement, thrusting Mr. Barr into a
full-blown crisis. Career prosecutors began to express worry that their
work could be used to
\href{https://www.nytimes3xbfgragh.onion/2020/02/12/us/politics/justice-department-roger-stone-sentencing.html}{settle
political scores} and doubts that he could protect them from political
interference.

The attorney general had been contemplating how to respond since he
became aware of Mr. Trump's attacks on the department, according to a
person familiar with his thinking. Speaking up could have put Mr. Barr
at risk of losing the backing of the president, but remaining silent
would have permitted Mr. Trump to continue attacking law enforcement and
all but invited open revolt among the some 115,000 employees of the
Justice Department.

Ultimately, Mr. Barr concluded that he had to speak out to preserve his
ability to do his job effectively, the person said.

Mr. Trump did not immediately respond on Twitter, but his press
secretary, Stephanie Grisham, played down the attorney general's
remarks. ``The president wasn't bothered by the comments at all, and he
has the right, just like every American citizen, to publicly offer his
opinions,'' she said, adding that Mr. Trump has confidence in his
attorney general.

Mr. Barr was hardly the first top adviser to the president to wish he
would stop tweeting, but he was the first to say it so publicly and
forcefully while still in office. His action instantly set off
speculation inside the administration about what it would mean for his
future.

The attorney general's office had let the president know some of what he
planned to say and is remaining in his job, a person familiar with the
events said. But as with other issues, Mr. Trump's view may depend on
how the news media, particularly Fox News, covers Mr. Barr's comments.

Some Fox personalities quickly drubbed Mr. Barr for crossing the
president. ``I am so disappointed in Bill Barr,'' Lou Dobbs, one of Mr.
Trump's favorite hosts, said on Fox Business, just a day after praising
the attorney general for ``doing the Lord's work'' by overruling the
career prosecutors.

Republicans in Congress rushed to voice support for Mr. Barr, urging the
president to heed his advice. ``If the attorney general says it's
getting in the way of doing his job, maybe the president should
listen,'' Senator Mitch McConnell, Republican of Kentucky and the
majority leader, said in an interview on Fox News.

Senator Lindsey Graham, Republican of South Carolina and the chairman of
the Senate Judiciary Committee, who is close to the president, said in a
statement that the attorney general was ``the right man at the right
time to reform the department and stand up for the rule of law.''

In the ABC interview, Mr. Barr declared his independence in what
amounted to an explicit challenge for a president who prizes loyalty
over almost anything.

``The thing I have most responsibility for are the issues that are
brought to me for decision,'' Mr. Barr said. ``And I will make those
decisions based on what I think is the right thing to do, and I'm not
going to be bullied or influenced by anybody.''

Mr. Trump has made it difficult for Mr. Barr to maintain the appearance
of independence, threatening the attorney general's credibility by
repeatedly calling for federal investigations of his own perceived
enemies. Mr. Trump suggested to the president of Ukraine
\href{https://www.nytimes3xbfgragh.onion/2019/09/25/us/politics/donald-trump-impeachment-probe.html}{on
a July call} that helped prompt impeachment that he work with Mr. Barr
and the president's personal lawyer Rudolph W. Giuliani to investigate
some of Mr. Trump's political opponents.

Justice Department officials have sought to distance Mr. Barr from the
episode, saying that he did not know the president named him on the call
and that he had no contact with Ukraine about any such cases.

Mr. Barr, whose expansive views on executive power are well established,
said in the ABC interview that presidents have the right to ask law
enforcement officials to scrutinize issues outside their personal
interests, like terrorism or bank fraud, but he drew a line at
interventions for personal benefit.

``If he were to say, `Go investigate somebody,' and you sense it's
because they're a political opponent, then an attorney general shouldn't
carry that out, wouldn't carry that out,'' Mr. Barr said.

But given Mr. Barr's remarkable deference to Mr. Trump's interests until
now, critics of the attorney general were loath to accept his comments
at face value, seeing them mainly as a face-saving way to deflect
responsibility for his own role in carrying out the president's
political wishes.

Joe Lockhart, a White House press secretary under President Bill
Clinton, said that it was ``impossible to believe'' that after all he
has done to advance Mr. Trump's political interests ``that now Barr is
genuinely upset.''

``The tell here will be Trump's reaction,'' Mr. Lockhart added. ``If he
doesn't lash out, we'll all know this was pure political theater because
everyone agrees Trump has no self-restraint.''

Mr. Barr said in the interview that he did not see the president's
comments on Tuesday about the Stone sentencing before he decided to
lower the recommendation --- he reads only tweets that aides show to
him, he said ---~and acknowledged that Mr. Trump's behavior boxed him
into a corner.

``Do you go forward with what you think is the right decision or do you
pull back because of the tweet?'' Mr. Barr said. ``That just sort of
illustrates how disruptive these tweets can be.''

Yet Mr. Barr shared Mr. Trump's dim view of the initial Stone sentencing
request, a senior administration official noted.

Officials on Tuesday blamed the original filing on a miscommunication
and said they had intended to correct it even before Mr. Trump assailed
it.

Mr. Barr detailed his account of the dramatic week. The United States
attorney in Washington, Timothy Shea, a longtime Barr adviser who
started the job only last week, briefed him on Monday about the
prosecutors' desire for the longer sentence, Mr. Barr said. He suggested
that the prosecutors instead lay out factors for Judge Amy Berman
Jackson to consider in sentencing Mr. Stone but defer to her on the
length of the sentence.

``He thought there was a way of satisfying everybody and providing more
flexibility,'' he said of Mr. Shea. Mr. Barr added, ``I was under the
impression that what was going to happen was very much what I had
suggested.''

Instead, the prosecutors stuck to their recommendation, surprising Mr.
Barr, he said, and angering the president.

Mr. Trump also accused the prosecutors of engaging in an ``illegal''
investigation of Mr. Stone. He incorrectly accused Judge Jackson, who
presided over multiple cases from the special counsel inquiry, of
placing his former campaign chairman Paul Manafort in solitary
confinement.

The chief judge of the Federal District Court in Washington, Beryl A.
Howell, issued a rare public response to the president's attacks on
Thursday, saying that ``public criticism or pressure is not a factor''
when judges make sentencing decisions.

Mr. Trump also attacked Robert S. Mueller III, the former F.B.I.
director and special counsel, putting Mr. Barr in an especially awkward
position. ``Even Bob Mueller lied to Congress!''
\href{https://twitter.com/realDonaldTrump/status/1227561237782855680?s=20}{the
president wrote on Twitter this week}, accusing him of a felony, which,
if true, the attorney general would presumably be obliged to prosecute.

Mr. Trump did not explain at the time what he meant, but on Geraldo
Rivera's radio show on Thursday, he said he was referring to Mr.
Mueller's denial during congressional testimony that he had applied to
replace James B. Comey as F.B.I. director before becoming special
counsel.

``He wanted to be the director again,'' Mr. Trump told Mr. Rivera. ``And
I told him, basically, `You've had enough time.' And then within a very
short period of time, he was appointed special prosecutor.''

Mr. Trump has made this claim before without offering evidence for it.
Mr. Mueller has said he agreed to see Mr. Trump during his search for a
new F.B.I. director as a courtesy to offer advice, not to seek the job,
a recollection confirmed by Stephen K. Bannon, the president's former
chief strategist.

Mr. Trump renewed his complaints on Thursday, claiming
\href{https://twitter.com/realDonaldTrump/status/1227939838751657984}{in
a tweet} that the Stone jury forewoman had ``significant bias,'' his
first vocal assault on a juror. He was responding to
\href{https://www.foxnews.com/politics/roger-stone-juror-justice-department-anti-trump-social-media}{a
Fox News report} that the forewoman was an anti-Trump Democratic
activist.

Mr. Barr's comments were remarkable in part for his decision to
criticize the president while still serving him. Other top advisers have
denounced Mr. Trump only after they left the administration. His former
chief of staff John F. Kelly
\href{https://www.nytimes3xbfgragh.onion/2020/02/13/us/politics/trump-roger-stone.html}{said
in a speech} on Wednesday that a military aide was right to raise
questions about whether the president was exploiting American policy for
personal gain in his call with the president of Ukraine.

Mr. Barr quickly became one of Mr. Trump's trusted advisers after taking
office last February, erasing tensions between the White House and the
Justice Department that had flourished under Mr. Barr's predecessor Jeff
Sessions, whom the president soured on after Mr. Sessions recused
himself from oversight of the Russia investigation.

Mr. Barr also quickly became one of Mr. Trump's most articulate and
nimble defenders. But detractors accused Mr. Barr of playing partisan
politics when he released his own summary of the Mueller report that
proved to underplay investigators' most serious findings against the
president and when he cleared Mr. Trump of obstruction of justice after
Mr. Mueller declined to make a determination.

The attorney general has since authorized a criminal investigation into
the roots of the Mueller inquiry and publicly challenged an inspector
general's finding that the F.B.I. had adequate reason to begin
investigating the Trump campaign's links to Russia.

Peter Baker and Michael D. Shear contributed reporting from Washington,
and Maggie Haberman from New York.

Advertisement

\protect\hyperlink{after-bottom}{Continue reading the main story}

\hypertarget{site-index}{%
\subsection{Site Index}\label{site-index}}

\hypertarget{site-information-navigation}{%
\subsection{Site Information
Navigation}\label{site-information-navigation}}

\begin{itemize}
\tightlist
\item
  \href{https://help.nytimes3xbfgragh.onion/hc/en-us/articles/115014792127-Copyright-notice}{©~2020~The
  New York Times Company}
\end{itemize}

\begin{itemize}
\tightlist
\item
  \href{https://www.nytco.com/}{NYTCo}
\item
  \href{https://help.nytimes3xbfgragh.onion/hc/en-us/articles/115015385887-Contact-Us}{Contact
  Us}
\item
  \href{https://www.nytco.com/careers/}{Work with us}
\item
  \href{https://nytmediakit.com/}{Advertise}
\item
  \href{http://www.tbrandstudio.com/}{T Brand Studio}
\item
  \href{https://www.nytimes3xbfgragh.onion/privacy/cookie-policy\#how-do-i-manage-trackers}{Your
  Ad Choices}
\item
  \href{https://www.nytimes3xbfgragh.onion/privacy}{Privacy}
\item
  \href{https://help.nytimes3xbfgragh.onion/hc/en-us/articles/115014893428-Terms-of-service}{Terms
  of Service}
\item
  \href{https://help.nytimes3xbfgragh.onion/hc/en-us/articles/115014893968-Terms-of-sale}{Terms
  of Sale}
\item
  \href{https://spiderbites.nytimes3xbfgragh.onion}{Site Map}
\item
  \href{https://help.nytimes3xbfgragh.onion/hc/en-us}{Help}
\item
  \href{https://www.nytimes3xbfgragh.onion/subscription?campaignId=37WXW}{Subscriptions}
\end{itemize}
