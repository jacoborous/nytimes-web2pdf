Sections

SEARCH

\protect\hyperlink{site-content}{Skip to
content}\protect\hyperlink{site-index}{Skip to site index}

\href{https://www.nytimes3xbfgragh.onion/section/science}{Science}

\href{https://myaccount.nytimes3xbfgragh.onion/auth/login?response_type=cookie\&client_id=vi}{}

\href{https://www.nytimes3xbfgragh.onion/section/todayspaper}{Today's
Paper}

\href{/section/science}{Science}\textbar{}The Further Adventures of
Betelgeuse, the Fainting Star

\url{https://nyti.ms/2StrbG7}

\begin{itemize}
\item
\item
\item
\item
\item
\end{itemize}

Advertisement

\protect\hyperlink{after-top}{Continue reading the main story}

Supported by

\protect\hyperlink{after-sponsor}{Continue reading the main story}

Out There

\hypertarget{the-further-adventures-of-betelgeuse-the-fainting-star}{%
\section{The Further Adventures of Betelgeuse, the Fainting
Star}\label{the-further-adventures-of-betelgeuse-the-fainting-star}}

The red supergiant is no closer to exploding, it seems. It also no
longer appears round.

\includegraphics{https://static01.graylady3jvrrxbe.onion/images/2020/02/14/science/14betelgeuse-video-image/14betelgeuse-video-image-videoSixteenByNineJumbo1600.jpg}

\href{https://www.nytimes3xbfgragh.onion/by/dennis-overbye}{\includegraphics{https://static01.graylady3jvrrxbe.onion/images/2018/07/30/multimedia/author-dennis-overbye/author-dennis-overbye-thumbLarge.png}}

By \href{https://www.nytimes3xbfgragh.onion/by/dennis-overbye}{Dennis
Overbye}

\begin{itemize}
\item
  Published Feb. 14, 2020Updated Feb. 17, 2020
\item
  \begin{itemize}
  \item
  \item
  \item
  \item
  \item
  \end{itemize}
\end{itemize}

Betelgeuse, the red supergiant star that marks the armpit of Orion the
Hunter, has been
\href{https://www.nytimes3xbfgragh.onion/2020/01/09/science/astronomy-supernova-betelgeuse.html}{dramatically
and mysteriously dimming} for the last six months.

Some astronomers and excitable members of the public have wondered if
the star is
\href{https://www.nytimes3xbfgragh.onion/interactive/2020/01/09/science/betelgeuse-supernova-fading.html}{about
to explode} as a supernova. Others have suggested more prosaic
explanations, involving long-term cycles of variability, sunspots or
dust.

Now new light, so to speak, has been shed on the mystery.

Recent high-resolution photographs of the star suggest that it is
changing shape, astronomers from the European Southern Observatory said
in a \href{https://www.eso.org/public/news/eso2003/?lang}{news release}
on Valentine's Day. Instead of appearing round, the star now appears
squashed into an oval.

A team led by Miguel Montargès of KU Leuven in Belgium used a special
camera on the Very Large Telescope of the European Southern Observatory
in Chile. The camera, called Sphere, for Spectro-Polarimetric
High-contrast Exoplanet Research instrument, was designed to take
pictures of worlds that orbit distant suns.

The result were high-resolution images of the surface of a star 700
light years from Earth. The portrait is not as detailed as the close-up
\href{https://www.nytimes3xbfgragh.onion/2020/01/29/science/daniel-inouye-solar-telescope-pictures.html}{images
of our own sun}obtained recently by the new solar telescope on
Haleakala, which revealed a surface resembling
\href{https://www.nytimes3xbfgragh.onion/2020/01/29/science/daniel-inouye-solar-telescope-pictures.html}{popcorn}.
Nonetheless, Betelgeuse, one of the great beacons of the sky, is quite
clearly going through some changes.

\href{https://www.nytimes3xbfgragh.onion/interactive/2020/02/14/science/betelgeuse-images-fading.html}{}

\includegraphics{https://static01.graylady3jvrrxbe.onion/images/2020/02/14/science/betelgeuse-images-fading-1581649698145/betelgeuse-images-fading-1581649698145-articleLarge.jpg}

\hypertarget{new-images-of-a-fading-betelgeuse}{%
\subsection{New Images of a Fading
Betelgeuse}\label{new-images-of-a-fading-betelgeuse}}

Glimpsing the surface of a shape-shifting star.

In January 2019, before all this began, the Betelgeuse that Dr.
Montargès viewed through the camera was ``a bright round disk,'' he said
in an email. A year later, all the brightness of the star had been
squashed into an oval occupying the northern half of the star.

Dr. Montargès declined to discuss any deeper details, pending a
peer-reviewed publication of his scientific conclusions.

``Well, what I mean is that in the visible we do not see anymore a
bright round disk,'' he said. ``It could be either a local cooling of
the surface that causes the star to look asymmetric, or a dust cloud
hiding part of the star.''

As supergiant stars like Betelgeuse evolve into supernova funeral pyres,
they typically go through unstable periods in which they shed layers of
gas dust into nearby space, shrouding themselves.

The possibility that dust might be responsible for Betelgeuse's dimming
was underscored by other infrared, or heat, images from the Very Large
Telescope. Those images showed huge, flame-like protuberances of dust
arcing out from the limb of Betelgeuse.

Edward Guinan, an astrophysicist at Villanova University who has been
following Betelgeuse, called the new images of a squashed star
``fantastic.'' But based on his own observations he took exception to
the idea that Betelgeuse was hiding behind a veil of dust.

``We think the star itself is doing this --- not dust,'' he said by
email.

Like our own sun, Betelgeuse transfers its thermonuclear energy by
convection from the center, where it is generated, to its surface.
Picture boiling oatmeal, with giant gobs of hot gas rising, radiating
away their heat and energy and then cooling, turning over and sinking
again.

Dr. Guinan said that the dimming of Betelgeuse was likely caused by the
sinking and cooling of one of these giant globs or convective cells.
Another, less likely explanation is a massive outbreak of starspots,
akin to the dark blemishes that appear in great numbers on our sun every
11 years.

But the show might already be over. Dr. Guinan reports that the dimming
of Betelgeuse has slowed and may have even stopped over the last week.

``We may be at/near the bottom of this `fainting' spell,'' he wrote.

\href{https://www.nytimes3xbfgragh.onion/interactive/2020/science/2020-astronomy-space-calendar.html}{}

\includegraphics{https://static01.graylady3jvrrxbe.onion/images/2019/12/04/science/04SUN1/04SUN1-articleLarge.png}

\hypertarget{sync-your-calendar-with-the-solar-system}{%
\subsection{Sync your calendar with the solar
system}\label{sync-your-calendar-with-the-solar-system}}

Never miss an eclipse, a meteor shower, a rocket launch or any other
astronomical and space event that's out of this world.

Advertisement

\protect\hyperlink{after-bottom}{Continue reading the main story}

\hypertarget{site-index}{%
\subsection{Site Index}\label{site-index}}

\hypertarget{site-information-navigation}{%
\subsection{Site Information
Navigation}\label{site-information-navigation}}

\begin{itemize}
\tightlist
\item
  \href{https://help.nytimes3xbfgragh.onion/hc/en-us/articles/115014792127-Copyright-notice}{©~2020~The
  New York Times Company}
\end{itemize}

\begin{itemize}
\tightlist
\item
  \href{https://www.nytco.com/}{NYTCo}
\item
  \href{https://help.nytimes3xbfgragh.onion/hc/en-us/articles/115015385887-Contact-Us}{Contact
  Us}
\item
  \href{https://www.nytco.com/careers/}{Work with us}
\item
  \href{https://nytmediakit.com/}{Advertise}
\item
  \href{http://www.tbrandstudio.com/}{T Brand Studio}
\item
  \href{https://www.nytimes3xbfgragh.onion/privacy/cookie-policy\#how-do-i-manage-trackers}{Your
  Ad Choices}
\item
  \href{https://www.nytimes3xbfgragh.onion/privacy}{Privacy}
\item
  \href{https://help.nytimes3xbfgragh.onion/hc/en-us/articles/115014893428-Terms-of-service}{Terms
  of Service}
\item
  \href{https://help.nytimes3xbfgragh.onion/hc/en-us/articles/115014893968-Terms-of-sale}{Terms
  of Sale}
\item
  \href{https://spiderbites.nytimes3xbfgragh.onion}{Site Map}
\item
  \href{https://help.nytimes3xbfgragh.onion/hc/en-us}{Help}
\item
  \href{https://www.nytimes3xbfgragh.onion/subscription?campaignId=37WXW}{Subscriptions}
\end{itemize}
