Sections

SEARCH

\protect\hyperlink{site-content}{Skip to
content}\protect\hyperlink{site-index}{Skip to site index}

\href{https://www.nytimes3xbfgragh.onion/section/health}{Health}

\href{https://myaccount.nytimes3xbfgragh.onion/auth/login?response_type=cookie\&client_id=vi}{}

\href{https://www.nytimes3xbfgragh.onion/section/todayspaper}{Today's
Paper}

\href{/section/health}{Health}\textbar{}Some Assisted-Living Residents
Don't Get Promised Care, Suit Charges

\url{https://nyti.ms/31Sw4vp}

\begin{itemize}
\item
\item
\item
\item
\item
\item
\end{itemize}

Advertisement

\protect\hyperlink{after-top}{Continue reading the main story}

Supported by

\protect\hyperlink{after-sponsor}{Continue reading the main story}

the new old age

\hypertarget{some-assisted-living-residents-dont-get-promised-care-suit-charges}{%
\section{Some Assisted-Living Residents Don't Get Promised Care, Suit
Charges}\label{some-assisted-living-residents-dont-get-promised-care-suit-charges}}

Court decisions in California may shed light on how large chains make
staffing decisions.

\includegraphics{https://static01.graylady3jvrrxbe.onion/images/2020/02/18/science/18SPANASSISTED/18SPANASSISTED-articleLarge.jpg?quality=75\&auto=webp\&disable=upscale}

By \href{https://www.nytimes3xbfgragh.onion/by/paula-span}{Paula Span}

\begin{itemize}
\item
  Feb. 14, 2020
\item
  \begin{itemize}
  \item
  \item
  \item
  \item
  \item
  \item
  \end{itemize}
\end{itemize}

The letter went out to about 1,900 Californians a few weeks ago from law
firms bringing a class-action suit against one of the country's largest
assisted-living chains.

If the recipients, or their family members, had lived in a community
operated by Sunrise Senior Living in recent years, ``we would like to
speak with you regarding your residency and experience,'' the letter
said.

It was the latest action in an ongoing campaign: Since 2013, a group of
law firms has systematically sued several major chains operating in the
state, employing an unusual strategy.

``It all boils down to the use of assessments, or lack thereof,'' said
Kathryn Stebner, the trial counsel in the case. ``These are à la carte
facilities --- the more needs you have, the more you have to pay. So,
they assess you.''

The plaintiffs' complaint, filed in 2017 and now before the U.S.
District Court for Central California, argues that when staff members
conduct such periodic assessments --- to determine whether a resident
needs help bathing or dressing, for example, or suffers from dementia
--- the facilities don't use the results to determine an adequate number
of staff members.

Instead, the plaintiffs argue, administrators make staffing decisions
financially, based on budgets and return on investment. When assessments
show increasing needs, the suit alleges, fees rise but staffing ratios
may not change.

``People pay more, but they're not getting more care,'' Ms. Stebner
said.

The suit claims that Sunrise is misrepresenting its practices and
deceiving customers, in violation of state business statutes, and lacks
enough trained staff members to deliver the care specified in resident
contracts and marketing materials.

``The business model is fraudulent, and it's putting people at risk,''
Ms. Stebner said.

Sunrise's practices are unlawful in another way, as well, the suit
charges. ``If you take an elder's property, knowing it could harm them,
that's financial elder abuse,'' Ms. Stebner said. ``In this case,
they're taking their money.''

In an emailed statement, Sunrise denounced ``baseless lawsuits like
these, in which the plaintiffs' lawyers file copycat allegations,'' a
reference to the firms having brought four previous suits using
essentially the same tactics. Sunrise called the claims ``categorically
false.''

The lawyers' scorecard to date: two settlements reached in 2016,
totaling \$13 million from Emeritus Corporation (since merged with
Brookdale Senior Living) and \$6.4 million from Atria Senior Living.

Suits against two other chains, Aegis Living and Oakmont Senior Living,
will proceed when and if courts certify them as class actions. The
Sunrise case, with a ``purported class'' of 13,000 current and former
residents in California, also awaits certification. (A smaller group
received the letters asking for information.)

But in the meantime, the plaintiffs have compelled the chain, with 268
facilities across the country and 52 in California, to turn over a trove
of documents showing how it determines staffing levels.

Sunrise argued that those were ``protected trade secrets.'' The judge
disagreed. Given that this is an industry whose practices often remain
opaque, that might constitute a victory in itself for the plaintiffs.

``It gets at internal systemic issues,'' said Eric Carlson, a directing
attorney at Justice in Aging, a legal advocacy group not involved in the
lawsuits. When facilities disclose information like how much time staff
members spend on tasks, ``it gets at what's happening behind closed
doors.''

Are assisted-living facilities understaffed? It's a common complaint
from residents and families, but one difficult to document.

``We don't have a very clear picture of what staff looks like in
assisted living,'' said Kali Thomas, a health services researcher at the
Brown University School of Public Health.

``We don't know what an individual assisted living's staff ratios are.
Many states don't even require them to track or report them.''

Offering a less institutional environment for seniors who require help
with the so-called activities of daily living --- but don't need
round-the-clock care in a nursing home --- the assisted-living industry
can house close to a million people in almost 30,000 facilities
nationwide.

It includes both small four-bed care homes and complexes with more than
100 residents, but chains dominate the field. And the industry has
staved off the kind of regulations that make it much easier to see
what's going on --- for better or worse --- in nursing homes.

Though Medicaid pays for a small but growing proportion of residents,
assisted living remains primarily a private-pay option. (The average
cost last year,
\href{https://www.genworth.com/aging-and-you/finances/cost-of-care.html}{in
one annual survey}: \$4,051 a month nationally, and \$4,500 in
California.)

The lack of federal dollars helps explain why assisted living is subject
not to federal oversight, like nursing homes, but to state regulations,
which vary wildly.

Colorado, for instance, requires that an assisted-living facility
employs one aide per 10 residents during the day, and one to 16 at
night. In Missouri, it's one to 15, and one to 25. Only 19 states
specify minimum staffing ratios at all.

Families can find it difficult to make informed decisions about assisted
living; there's no equivalent of the federal inspection findings and
quality rankings at Nursing Home Compare. State websites are inadequate
substitutes, a recent study found.

Yet the people moving into these complexes need more help than they did
years ago.

``They're older,'' Ms. Thomas pointed out. ``They're entering with more
chronic diseases.'' More than 40 percent
\href{https://www.healthaffairs.org/doi/full/10.1377/hlthaff.2013.1255}{have
moderate or severe dementia}, a study in the journal Health Affairs
reported.

The California lawsuits don't seek individual damages, and because the
classes involved contain thousands of individuals, their checks from
settlements so far have been paltry --- a few hundred dollars each.

But the suits also seek ``injunctive relief,'' a court-ordered
requirement that the defendants change their practices.

``They'd be told to be transparent,'' Ms. Stebner said. ``We want them
to use the assessments properly and to tell people what they're doing.
And to have sufficient staff.''

Veteran researchers sounded more skeptical about the lawsuits' impact.
Understaffing represents ``a complaint about long-term care in
general,'' said Dr. Philip Sloane, a geriatrician at the University of
North Carolina School of Medicine. ``Of course it's true. The question
is, what is realistic?''

``These places do set very high expectations,'' said Sheryl Zimmerman, a
health services researcher at the University of North Carolina School of
Social Work. ``Everyone's website sounds like Utopia.''

But she added, ``These group settings cannot individualize everything
for every person.'' Being forced to add staff members could make
assisted living even more expensive, unreachable for many older adults,
she said.

Dr. Sloane said facilities might respond to the suits by adding
disclaimers to their marketing: ``They'll probably address the promises,
rather than the care.''

After Florida increased staffing requirements for nursing home aides in
2001, Ms. Thomas recalled, she led a study showing that the homes
complied, then cut hours for their housekeeping and activities staffs.

Assisted-living providers targeted by lawsuits might respond similarly,
she noted.

Mr. Carlson saw it differently. It's tough to compel substantial changes
in long-term care facilities, he acknowledged. But, he said, ``You get
pressure from residents, from surveyors, from plaintiffs' counsel, and
it all pushes providers to be more accountable to residents.''

If the court certifies Sunrise residents as a class, a resolution of the
case probably lies two years away, Ms. Stebner estimated. And if the
plaintiffs win further settlements, lawyers in other states may start
taking notes.

Advertisement

\protect\hyperlink{after-bottom}{Continue reading the main story}

\hypertarget{site-index}{%
\subsection{Site Index}\label{site-index}}

\hypertarget{site-information-navigation}{%
\subsection{Site Information
Navigation}\label{site-information-navigation}}

\begin{itemize}
\tightlist
\item
  \href{https://help.nytimes3xbfgragh.onion/hc/en-us/articles/115014792127-Copyright-notice}{©~2020~The
  New York Times Company}
\end{itemize}

\begin{itemize}
\tightlist
\item
  \href{https://www.nytco.com/}{NYTCo}
\item
  \href{https://help.nytimes3xbfgragh.onion/hc/en-us/articles/115015385887-Contact-Us}{Contact
  Us}
\item
  \href{https://www.nytco.com/careers/}{Work with us}
\item
  \href{https://nytmediakit.com/}{Advertise}
\item
  \href{http://www.tbrandstudio.com/}{T Brand Studio}
\item
  \href{https://www.nytimes3xbfgragh.onion/privacy/cookie-policy\#how-do-i-manage-trackers}{Your
  Ad Choices}
\item
  \href{https://www.nytimes3xbfgragh.onion/privacy}{Privacy}
\item
  \href{https://help.nytimes3xbfgragh.onion/hc/en-us/articles/115014893428-Terms-of-service}{Terms
  of Service}
\item
  \href{https://help.nytimes3xbfgragh.onion/hc/en-us/articles/115014893968-Terms-of-sale}{Terms
  of Sale}
\item
  \href{https://spiderbites.nytimes3xbfgragh.onion}{Site Map}
\item
  \href{https://help.nytimes3xbfgragh.onion/hc/en-us}{Help}
\item
  \href{https://www.nytimes3xbfgragh.onion/subscription?campaignId=37WXW}{Subscriptions}
\end{itemize}
