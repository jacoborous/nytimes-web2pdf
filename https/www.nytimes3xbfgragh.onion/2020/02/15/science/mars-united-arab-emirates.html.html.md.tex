Sections

SEARCH

\protect\hyperlink{site-content}{Skip to
content}\protect\hyperlink{site-index}{Skip to site index}

\href{https://www.nytimes3xbfgragh.onion/section/science}{Science}

\href{https://myaccount.nytimes3xbfgragh.onion/auth/login?response_type=cookie\&client_id=vi}{}

\href{https://www.nytimes3xbfgragh.onion/section/todayspaper}{Today's
Paper}

\href{/section/science}{Science}\textbar{}From Dubai to Mars, With Stops
in Colorado and Japan

\url{https://nyti.ms/2UUQgeG}

\begin{itemize}
\item
\item
\item
\item
\item
\end{itemize}

Advertisement

\protect\hyperlink{after-top}{Continue reading the main story}

Supported by

\protect\hyperlink{after-sponsor}{Continue reading the main story}

\hypertarget{from-dubai-to-mars-with-stops-in-colorado-and-japan}{%
\section{From Dubai to Mars, With Stops in Colorado and
Japan}\label{from-dubai-to-mars-with-stops-in-colorado-and-japan}}

The United Arab Emirates used a novel approach to build the Hope
spacecraft, which launches for the red planet this summer.

\includegraphics{https://static01.graylady3jvrrxbe.onion/images/2020/02/16/science/16sci-hope1/merlin_168738882_153d71d7-9469-430c-a2f8-28a153fc3590-articleLarge.jpg?quality=75\&auto=webp\&disable=upscale}

\href{https://www.nytimes3xbfgragh.onion/by/kenneth-chang}{\includegraphics{https://static01.graylady3jvrrxbe.onion/images/2018/02/16/multimedia/author-kenneth-chang/author-kenneth-chang-thumbLarge.jpg}}

By \href{https://www.nytimes3xbfgragh.onion/by/kenneth-chang}{Kenneth
Chang}

\begin{itemize}
\item
  Published Feb. 15, 2020Updated July 19, 2020
\item
  \begin{itemize}
  \item
  \item
  \item
  \item
  \item
  \end{itemize}
\end{itemize}

BOULDER, Colo. --- In December, a spacecraft named Hope was motionless
in the middle of a large clean room on the campus of the University of
Colorado, mounted securely on a stand.

But engineers were tricking Hope --- a foil-wrapped box about the size
and weight of a Mini Cooper --- into thinking it was speeding at more
than 10,000 miles per hour as it pulled into orbit at Mars. It was a
simulation to make sure that the guidance, navigation and control
systems would respond correctly to a variety of less-than-perfect
circumstances when it arrives at Mars for real next year.

While this spacecraft was assembled on American soil, it will not be
exploring the red planet for NASA. Hope is instead an effort by the
\href{https://www.nytimes3xbfgragh.onion/2020/07/14/science/mars-united-arab-emirates.html}{United
Arab Emirates}, an oil-rich country smaller than the state of Maine and
one that has never sent anything out into the solar system.

Emirati engineers worked here, close to the ski slopes of the Rocky
Mountains and far from the sands of the Middle East, learning from their
American counterparts. It was part of the Emirates' planning for the
future when petroleum no longer flows as bountifully, to invest its
current wealth in new ``knowledge-based'' industries.

\includegraphics{https://static01.graylady3jvrrxbe.onion/images/2020/02/13/science/00SCI-HOPE/00SCI-HOPE-articleLarge.jpg?quality=75\&auto=webp\&disable=upscale}

``How do you develop highly skilled people that are able to take on
higher risks?'' said Sarah al-Amiri, the minister of state for advanced
sciences for the U.A.E., who also leads the science portion of the Mars
mission. ``That was the reason to go to space exploration.''

As a newcomer, the U.A.E. has taken a novel approach. It could have
tried to do everything itself, developing homegrown technology similar
to what India has done. That would have taken years longer.
Alternatively, it could have bought someone else's spacecraft design,
which would have been the quickest path.

Instead, the country has sought partners with long experience in sending
machines into space. This, its space team believed, would help avoid
many of the pitfalls of trying to pull off such missions for the first
time, while training future engineers who will be expected to step up to
bigger roles in the next mission. In the process, the country's leaders
hope to sow seeds for future companies.

``The government really wanted to create that ecosystem or at least help
in creating that ecosystem,'' said Omran Sharaf, the project manager for
the Emirates' Mars mission. ``Soon. They want to accelerate the process.
Don't start from scratch. Work with others. Take it to the next level
now.''

\hypertarget{were-going-to-learn-a-tremendous-amount}{%
\subsection{`We're going to learn a tremendous
amount'}\label{were-going-to-learn-a-tremendous-amount}}

Last year, for a nascent astronaut program, the Emirates bought a seat
on a Russian Soyuz rocket. They sent
\href{https://www.nytimes3xbfgragh.onion/2019/09/25/science/emirati-astronaut-uae-international-space-station.html}{Hazzaa
al-Mansoori for an eight-day stay at the International Space Station}.

Hope will be just one of a flotilla of robotic spacecraft scheduled to
launch this summer during a once-in-26-months alignment of Earth and
Mars that enables a relatively short trip of some 300 million miles and
seven months to the red planet.

The other three will be the products of established space powers: NASA,
China and a collaboration between Russia and the European Space Agency.

Image

Hazzaa al-Mansoori, the first Emirati astronaut, shortly after his
return to Earth from the space station in October last
year.Credit...Dmitri Lovetsky/Agence France-Presse --- Getty Images

Compared with those, Hope is modest in size and scope, with costs
fitting into what managers described as a ``tight budget.'' While the
other missions each aim to put a rover on the surface, the Emirati
spacecraft will make observations from orbit.

Still, it will be more than just a technical triumph.

``We were requested to send a spacecraft to Mars, but not send space
junk, basically,'' Ms. al-Amiri said. ``Send a spacecraft that not only
captures an image of the planet to declare you're there, but actually
provides you with valuable scientific data.''

In September 2014,
\href{https://www.nytimes3xbfgragh.onion/2014/09/25/world/asia/on-a-shoestring-india-sends-orbiter-to-mars.html}{India
celebrated putting a spacecraft in orbit around Mars} and boasted how
its price tag was a fraction of that for MAVEN, a NASA probe that
\href{https://www.nytimes3xbfgragh.onion/2014/09/22/science/space/nasa-craft-mars.html}{arrived
two days earlier}. Both are still there.

But the Indian spacecraft did not have scientific instruments sensitive
enough to make significant discoveries. By contrast, MAVEN has
determined how quickly the Martian atmosphere is being stripped away by
the solar wind: about four pounds a second. This information is an
important clue in the puzzle of understanding why early Mars, which was
warmer and wetter, turned into the cold, barren, almost airless place it
is today.

Hope's aim is to fill in a gap in MAVEN's findings by looking at the
dynamics closer to the ground that influence the rate of leaking.

Image

Indian space agency staff celebrating when their Mars orbiter arrived in
2014. The country's probe was a technical achievement on a budget, but
it did not have sensitive scientific tools.Credit...Manjunath
Kiran/Agence France-Presse --- Getty Images

``You need to understand the role that Mars plays in the loss of its
atmosphere,'' Ms. al-Amiri said.

When a planet-wide dust storm raged on Mars in the summer of 2018, MAVEN
observed that the amount of hydrogen in the upper atmosphere rose. The
three instruments on Hope --- an infrared spectrometer, an ultraviolet
spectrometer and a camera --- would be able to help explain how the dust
pushed the hydrogen upward.

In addition, from its high-altitude perch --- an elliptical orbit that
varies from 12,400 miles to 27,000 miles above the surface --- Hope will
give scientists a global view of Martian weather, noting changes in
temperature and other conditions during the course of a day.

``That's one of the fundamental new measurements we haven't seen
before,'' said Bruce M. Jakosky, a professor of geological sciences at
the University of Colorado who is MAVEN's principal investigator and a
member of the science team for the Emirati mission.

Previous orbiters have generally swooped much closer to the Martian
surface, usually in orbits devised to pass over a given location at the
same time of day each time. That was more useful for detecting slow
changes on the surface rather than in the air.

``I think the atmosphere has been understudied,'' said Philip R.
Christensen, a planetary sciences professor at Arizona State University,
which built the infrared spectrometer for Hope. That instrument will
capture data on the dust particles and ice clouds and track the movement
of water vapor and heat through the atmosphere.

The spacecraft is to spend at least two years in orbit, monitoring a
full cycle of Martian seasons.

``I think we're going to learn a tremendous amount,'' Dr. Christensen
said.

\hypertarget{brainstorming-their-way-to-mars}{%
\subsection{Brainstorming their way to
Mars}\label{brainstorming-their-way-to-mars}}

Hope will be a well-traveled vehicle even before it heads to space in
July.

Until Monday, it had never been anywhere near the United Arab Emirates.
That day, the finished spacecraft landed in Dubai, after a 7,800-mile
trip from Denver inside a Ukrainian Antonov cargo plane.

After another round of testing in Dubai, one of the seven city-states
that make up the U.A.E. federation, the spacecraft will take another
long plane trip, to Japan, for the rocket launch to leave Earth.

The Emirati Mars strategy replicates what the country did in the 2000s
when the Dubai government wanted to build its own earth observation
satellites. For that project, Dubai turned to a South Korean satellite
manufacturer.

The first product of the collaboration, DubaiSat-1, was built in South
Korea, with Emirati engineers spending months there, essentially
learning as apprentices. The Russians launched it in 2009. The 400-pound
satellite's camera has been used for urban planning, disaster relief and
environmental monitoring.

Its second satellite, DubaiSat-2, included a sharper camera and a faster
communications system. It was also built in South Korea, but the work
was split more as an equal partnership between the Emirati and South
Korean engineers. A third satellite, KhalifaSat, was the first to be
developed and built in the U.A.E.

At the time of the launch of DubaiSat-2 in November 2013, Emirati
leaders were brainstorming more ambitious space projects.

Ms. al-Amiri said Sheikh Mohammed bin Rashid al-Maktoum, the ruler of
Dubai and prime minister of the U.A.E., asked if it might be feasible to
send a spacecraft to Mars.

To date, only NASA, the European Space Agency, India and the Soviet
Union in the 1970s and 1980s have successfully sent probes to Mars.

Mr. Sharaf, who was then a deputy program manager for DubaiSat-2, said
after couple of weeks of study that a Mars mission seemed plausible.
``We think it's something we should look into more,'' he recalled
telling higher officials.

Image

Sheikh Mohammed bin Rashid al-Maktoum, Prime Minister of the Emirates
and ruler of Dubai, center. While the country is considered more
progressive than some of its neighbors, the collaboration presented some
controversy for the University of Colorado.Credit...Karim Sahib/Agence
France-Presse --- Getty Images

A few weeks later, Sheikh Mohammed visited the space center now named
after him. ``Basically what he told us was that `I want us to reach Mars
before the second of December 2021,' which is the 50th anniversary of
the establishment of the U.A.E.,'' Mr. Sharaf said. ``He really wanted
to inspire Emirati youth and accelerate them going into sciences.''

Mr. Sharaf said Sheikh Mohammed also wanted to offer inspiration for
youth in the wider Arab world. ``That's why he called the spacecraft
Hope,'' Mr. Sharaf said.

Emirati officials, including Mr. Sharaf and Ms. al-Amiri, started
reaching out to space organizations around the world, including the
Laboratory for Atmospheric and Space Physics, a research institute at
the University of Colorado that has been working on space missions for
more than half a century.

They visited Colorado in the spring of 2014, quizzing laboratory
officials about what kind of scientific investigation might be
worthwhile to pursue at Mars.

The laboratory submitted a winning proposal. Arizona State University
and the University of California, Berkeley were given roles in
developing and building the spacecraft's instruments.

The federal U.A.E. space agency, which is financing the mission, was
created in 2014. The space center in Dubai is in charge of its
construction and operation of the spacecraft. (It is as if California
established a space program first and the United States set up NASA
later.)

For the University of Colorado, a collaboration with the Emirates is not
free of controversy. While the U.A.E. is often considered more
progressive and open than many of its neighbors, it supported Saudi
Arabia's intervention in Yemen,
a\href{https://www.nytimes3xbfgragh.onion/2018/08/28/world/middleeast/un-yemen-war-crimes.html}{civil
war that has killed thousands of people}, before mostly
\href{https://www.nytimes3xbfgragh.onion/2019/07/11/world/middleeast/yemen-emirates-saudi-war.html}{pulling
out last year}. The country has also
\href{https://www.justice.gov/eoir/page/file/1181666/download}{jailed
political dissidents}.

Daniel N. Baker, the laboratory director, said that officials from the
university, Colorado and even NASA, supported the project. ``From my
point of view, the criteria that we apply is we like to have like-minded
people, people who are driven by excellence,'' Dr. Baker said.

\hypertarget{freshly-minted-planetary-scientists}{%
\subsection{Freshly minted planetary
scientists}\label{freshly-minted-planetary-scientists}}

Engineers from the U.A.E., some who worked in South Korea on the
satellites and some who were right out of college, started arriving in
Colorado. Two teams --- an Emirati group led by Mr. Sharaf, and one from
Colorado led by Peter Withnell --- worked side-by-side.

The mission is a stretch for the Colorado laboratory as well; it is the
largest spacecraft it has ever built. In the past, it had mostly built
scientific instruments for missions rather than the spacecraft itself.

Sending a spacecraft to Mars poses bigger challenges than putting a
satellite in Earth orbit. Radio communications now have to travel
millions of miles, not a few hundred, and are blocked periodically by
the sun or Mars. The spacecraft will have to take care of itself for
stretches of time.

The Emirati team is much younger --- 90 percent of them are under 35 ---
than its American counterparts. ``When I started I was 30,'' Mr. Sharaf
said.

Image

The probe during vacuum testing in Colorado. The Emirati team working on
Hope is young --- 90 percent under 35 --- and a third of them are
women.Credit...Mohammed Bin Rashid Space Center

A third of them are women, a high percentage in an engineering field
that has often been dominated by men. For the Emirati science team that
will be studying the Mars data, the percentage of women is even higher:
80 percent. Until an effort to recruit men, the science team was
entirely women.

Mr. Withnell, with a head of white hair and 25 years of experience at
the laboratory, said he was impressed by his younger teammates. ``The
enthusiasm, the drive, is palpable,'' he said. ``I would hire any of
these people in an instant.''

That includes Mohsen al-Awadhi who, six years ago, was working as a
maintenance engineer for Emirates, the Dubai-based airline. ``I knew
almost nothing about space,'' he recalled.

But when he saw a job posting at the space center, ``I just randomly
sent my C.V. to them about this job,'' he said. ``I got a job offer, and
I was not even intending to leave the airline.''

He was asked to move to Colorado in 2015 to work on Hope as a systems
engineer. He and his wife moved. ``For a deep space mission, this is the
first for the region, not just for the country,'' Mr. al-Awadhi said.
``It was like an honor.''

While working full-time on the Mars mission, he also completed a
master's degree in aerospace engineering.

Another challenge for a country like the U.A.E. undertaking a planetary
science mission: a lack of planetary scientists.

Project leaders decided to convert some engineers like Hoor al-Maazmi
into apprentice scientists.

When she was in college, Ms. al-Maazmi was interested in nuclear and
mechanical engineering, not space research. ``It wasn't that much of a
dream for me, because it wasn't really possible,'' she said.

Now, she is using computer models to predict what Hope might see when
data starts arriving next year, and she intends to pursue a doctoral
degree in planetary science.

Hope will not be the last Emirati planetary mission.

``It's not a one-off,'' Ms. al-Amiri said. ``It was never a one-off
project. It's one that is meant to develop a space sector, one that is
meant to also spill over, once the mission is successful, into other
sectors.''

Advertisement

\protect\hyperlink{after-bottom}{Continue reading the main story}

\hypertarget{site-index}{%
\subsection{Site Index}\label{site-index}}

\hypertarget{site-information-navigation}{%
\subsection{Site Information
Navigation}\label{site-information-navigation}}

\begin{itemize}
\tightlist
\item
  \href{https://help.nytimes3xbfgragh.onion/hc/en-us/articles/115014792127-Copyright-notice}{©~2020~The
  New York Times Company}
\end{itemize}

\begin{itemize}
\tightlist
\item
  \href{https://www.nytco.com/}{NYTCo}
\item
  \href{https://help.nytimes3xbfgragh.onion/hc/en-us/articles/115015385887-Contact-Us}{Contact
  Us}
\item
  \href{https://www.nytco.com/careers/}{Work with us}
\item
  \href{https://nytmediakit.com/}{Advertise}
\item
  \href{http://www.tbrandstudio.com/}{T Brand Studio}
\item
  \href{https://www.nytimes3xbfgragh.onion/privacy/cookie-policy\#how-do-i-manage-trackers}{Your
  Ad Choices}
\item
  \href{https://www.nytimes3xbfgragh.onion/privacy}{Privacy}
\item
  \href{https://help.nytimes3xbfgragh.onion/hc/en-us/articles/115014893428-Terms-of-service}{Terms
  of Service}
\item
  \href{https://help.nytimes3xbfgragh.onion/hc/en-us/articles/115014893968-Terms-of-sale}{Terms
  of Sale}
\item
  \href{https://spiderbites.nytimes3xbfgragh.onion}{Site Map}
\item
  \href{https://help.nytimes3xbfgragh.onion/hc/en-us}{Help}
\item
  \href{https://www.nytimes3xbfgragh.onion/subscription?campaignId=37WXW}{Subscriptions}
\end{itemize}
