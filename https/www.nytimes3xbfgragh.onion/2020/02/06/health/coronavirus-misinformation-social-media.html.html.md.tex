Sections

SEARCH

\protect\hyperlink{site-content}{Skip to
content}\protect\hyperlink{site-index}{Skip to site index}

\href{https://www.nytimes3xbfgragh.onion/section/health}{Health}

\href{https://myaccount.nytimes3xbfgragh.onion/auth/login?response_type=cookie\&client_id=vi}{}

\href{https://www.nytimes3xbfgragh.onion/section/todayspaper}{Today's
Paper}

\href{/section/health}{Health}\textbar{}W.H.O. Fights a Pandemic Besides
Coronavirus: An `Infodemic'

\url{https://nyti.ms/387S7QX}

\begin{itemize}
\item
\item
\item
\item
\item
\item
\end{itemize}

\href{https://www.nytimes3xbfgragh.onion/news-event/coronavirus?action=click\&pgtype=Article\&state=default\&region=TOP_BANNER\&context=storylines_menu}{The
Coronavirus Outbreak}

\begin{itemize}
\tightlist
\item
  live\href{https://www.nytimes3xbfgragh.onion/2020/08/01/world/coronavirus-covid-19.html?action=click\&pgtype=Article\&state=default\&region=TOP_BANNER\&context=storylines_menu}{Latest
  Updates}
\item
  \href{https://www.nytimes3xbfgragh.onion/interactive/2020/us/coronavirus-us-cases.html?action=click\&pgtype=Article\&state=default\&region=TOP_BANNER\&context=storylines_menu}{Maps
  and Cases}
\item
  \href{https://www.nytimes3xbfgragh.onion/interactive/2020/science/coronavirus-vaccine-tracker.html?action=click\&pgtype=Article\&state=default\&region=TOP_BANNER\&context=storylines_menu}{Vaccine
  Tracker}
\item
  \href{https://www.nytimes3xbfgragh.onion/interactive/2020/07/29/us/schools-reopening-coronavirus.html?action=click\&pgtype=Article\&state=default\&region=TOP_BANNER\&context=storylines_menu}{What
  School May Look Like}
\item
  \href{https://www.nytimes3xbfgragh.onion/live/2020/07/31/business/stock-market-today-coronavirus?action=click\&pgtype=Article\&state=default\&region=TOP_BANNER\&context=storylines_menu}{Economy}
\end{itemize}

Advertisement

\protect\hyperlink{after-top}{Continue reading the main story}

Supported by

\protect\hyperlink{after-sponsor}{Continue reading the main story}

\hypertarget{who-fights-a-pandemic-besides-coronavirus-an-infodemic}{%
\section{W.H.O. Fights a Pandemic Besides Coronavirus: An
`Infodemic'}\label{who-fights-a-pandemic-besides-coronavirus-an-infodemic}}

Working with the big tech companies, the U.N. health agency has made
strides in combating rumors and falsehoods on the internet about the new
infection.

\includegraphics{https://static01.graylady3jvrrxbe.onion/images/2020/02/06/science/06virus-misinformation03/merlin_167683917_8de71eee-646c-4de8-881b-876a476cef20-articleLarge.jpg?quality=75\&auto=webp\&disable=upscale}

By \href{https://www.nytimes3xbfgragh.onion/by/matt-richtel}{Matt
Richtel}

\begin{itemize}
\item
  Feb. 6, 2020
\item
  \begin{itemize}
  \item
  \item
  \item
  \item
  \item
  \item
  \end{itemize}
\end{itemize}

SAN FRANCISCO --- With the
\href{https://www.nytimes3xbfgragh.onion/2020/02/02/health/coronavirus-pandemic-china.html}{threat
of the coronavirus growing}, Aleksandra Kuzmanovic sat at her computer
in Geneva on Monday and sent out an important public health email. She
works for the World Health Organization and her aim was to assess and
stop a global spread --- not of the dangerous virus but of hazardous
false information.

She wanted to halt what her colleagues at the health agency are calling
an ``infodemic.''

She emailed a contact at Pinterest, the image-sharing social media giant
based here in San Francisco, to ask if the site could help the W.H.O.
combat the blazing spread of misinformation, lies and rumors about the
new virus.

Offer accepted. Starting Thursday afternoon, when Pinterest users search
for coronavirus, they get
\href{https://www.pinterest.ch/worldhealthorganization/2019-ncov-new-coronavirus/}{a
link} to a page of coronavirus ``mythbusters'' from the W.H.O.

Since the virus hit, Ms. Kuzmanovic said she and her colleagues have
been in regular contact with the world's biggest and most powerful
disseminators of information --- including Facebook, Twitter and Google,
as well as social media influencers around the globe.

Next week, Andrew Pattison, manager of digital solutions at the W.H.O.,
will travel to Menlo Park, Calif., to visit the headquarters of
Facebook, which has arranged for him to make a pitch for further
assistance to a gathering of 20 big tech companies, including Uber and
Airbnb. ``I'd love to see Airbnb give advice to people traveling about
coronavirus,'' he said.

He also plans to meet with Amazon in Seattle in hopes of having the
e-commerce titan provide accurate health information when people buy
such things as protective masks or respirators, or even books already
popping up about the crisis that could contain misinformation.

The efforts of the W.H.O., the health arm of the United Nations,
represent a new, far-reaching effort to reinvent what has largely been a
failed fight against misinformation.

Over the last two weeks, tech companies working with the W.H.O. have
been prominently posting links to W.H.O. content, making falsehoods
harder to find in searches or on news streams, and sometimes removing
content altogether.

The companies, including Google, Facebook and Twitter, would not provide
interviews for this article but did confirm that the work they are doing
with the W.H.O. was among their efforts to combat coronavirus
misinformation. The companies also are doing work independently of the
W.H.O. relationship to help eradicate coronavirus misinformation.

They are facing an array of wildly untrue claims, such as that the
coronavirus was created as a bioweapon or was funded by the Bill \&
Melinda Gates Foundation to further vaccine sales, or that it can be
cured by eating garlic or drinking a bleach concoction (which can itself
cause liver failure). These ideas, like a virus itself, can be easily
transmitted from person to person, carried by both the unwitting and the
devious and spreading almost invisibly through a vast virtual world.

\hypertarget{latest-updates-global-coronavirus-outbreak}{%
\section{\texorpdfstring{\href{https://www.nytimes3xbfgragh.onion/2020/08/01/world/coronavirus-covid-19.html?action=click\&pgtype=Article\&state=default\&region=MAIN_CONTENT_1\&context=storylines_live_updates}{Latest
Updates: Global Coronavirus
Outbreak}}{Latest Updates: Global Coronavirus Outbreak}}\label{latest-updates-global-coronavirus-outbreak}}

Updated 2020-08-02T07:42:09.613Z

\begin{itemize}
\tightlist
\item
  \href{https://www.nytimes3xbfgragh.onion/2020/08/01/world/coronavirus-covid-19.html?action=click\&pgtype=Article\&state=default\&region=MAIN_CONTENT_1\&context=storylines_live_updates\#link-34047410}{The
  U.S. reels as July cases more than double the total of any other
  month.}
\item
  \href{https://www.nytimes3xbfgragh.onion/2020/08/01/world/coronavirus-covid-19.html?action=click\&pgtype=Article\&state=default\&region=MAIN_CONTENT_1\&context=storylines_live_updates\#link-780ec966}{Top
  U.S. officials work to break an impasse over the federal jobless
  benefit.}
\item
  \href{https://www.nytimes3xbfgragh.onion/2020/08/01/world/coronavirus-covid-19.html?action=click\&pgtype=Article\&state=default\&region=MAIN_CONTENT_1\&context=storylines_live_updates\#link-2bc8948}{Its
  outbreak untamed, Melbourne goes into even greater lockdown.}
\end{itemize}

\href{https://www.nytimes3xbfgragh.onion/2020/08/01/world/coronavirus-covid-19.html?action=click\&pgtype=Article\&state=default\&region=MAIN_CONTENT_1\&context=storylines_live_updates}{See
more updates}

More live coverage:
\href{https://www.nytimes3xbfgragh.onion/live/2020/07/31/business/stock-market-today-coronavirus?action=click\&pgtype=Article\&state=default\&region=MAIN_CONTENT_1\&context=storylines_live_updates}{Markets}

The reality is that the
\href{https://www.nytimes3xbfgragh.onion/article/what-is-coronavirus.html?action=click\&pgtype=Article\&state=default\&module=styln-coronavirus\&variant=show\&region=MID_MAIN_CONTENT\&context=storyline_guide}{coronavirus
is a rapidly spreading respiratory infection} that originated in Wuhan,
China. Most of the cases, and nearly all of the deaths, have so far been
in China, though the germ has reached dozens of other countries in
recent weeks.

\textbf{\emph{{[}}\href{http://on.fb.me/1paTQ1h}{\emph{Like the Science
Times page on Facebook.}}} ****** \emph{\textbar{} Sign up for the}
\textbf{\href{http://nyti.ms/1MbHaRU}{\emph{Science Times
newsletter.}}\emph{{]}}}

Medical misinformation on the virus has been driven by ideologues who
distrust science and proven measures like vaccines, and by profiteers
who scare up internet traffic with zany tales and try to capitalize on
that traffic by selling ``cures'' or other health and wellness products.

``There are self-appointed experts, people working from anecdote, or
making up wild claims to get traffic or notoriety,'' said Mr. Pattison
of the W.H.O.

\includegraphics{https://static01.graylady3jvrrxbe.onion/images/2020/02/06/science/06virus-misinformation02/merlin_168427560_bd8f3656-79a6-4f56-acdd-a4fc61cf0774-articleLarge.jpg?quality=75\&auto=webp\&disable=upscale}

The groundwork for the coordination around the coronavirus was laid two
years ago, when Mr. Pattison went to the W.H.O. general director, Dr.
Tedros Adhanom Ghebreyesus, and suggested a full-blown effort to connect
with social media titans to combat health misinformation. Now about a
half-dozen W.H.O. staffers in Geneva are working on the issue, building
relationships with digital and social media sites. Over time, the
cooperative efforts have grown. For instance, last August, Pinterest
teamed up with the W.H.O. to link to accurate information about vaccines
when people search the service for that topic.

Ifeoma Ozoma, public policy and social impact manager at Pinterest, said
the company ``has been working with the World Health Organization over
the last year,'' with an aim to ``make sure people can find
authoritative information when it really counts.''

The W.H.O. seeks no money, nor pays any, in these relationships, Mr.
Pattison said. Rather, he explained, it is lending its credibility and
hoping to use ``their reach.''

The relationship has borne concrete results.

Google launched what it calls an ``SOS Alert,'' which directs people who
search for ``coronavirus'' to news and other information from the
W.H.O., including to the organization's Twitter account; that was
expanded Thursday to include information in not just English but also
French, Spanish, Chinese, Arabic and Russian. The W.H.O. has also worked
with the major Chinese-owned social media site WeChat to add a news feed
featuring correct information, translated into Chinese by the W.H.O.

The health agency has worked especially closely with Facebook. The
company has used human fact checkers to flag misinformation, which can
come to their attention through computer programs that identify
suspicious keywords and trends. Such posts can then be moved down in
news feeds, or, in rare cases, removed altogether.

For example, several weeks ago, Facebook removed a W.H.O. infographic
that had been modified to claim people should avoid having sex with
animals to prevent coronavirus. Facebook also is providing people who
search for information on coronavirus on Facebook and Instagram with
links to credible sources of information, including from the W.H.O. and
the Centers for Disease Control and Prevention.

Some of the tech companies have issued public statements of support for
the W.H.O. Kang Xing Jin, Facebook's head of health, said the social
platform is providing ``relevant and up-to-date information'' and
``working to limit the spread of misinformation and harmful content''
and is doing so ``based on guidance from the W.H.O.''

Despite the efforts, hundreds of thousands of people have consumed
dozens of documented falsehoods about the coronavirus on these platforms
and others, including Reddit and the Chinese-owned social media platform
TikTok, as well as numerous smaller websites. On TikTok, there are
several videos featuring the Gates conspiracy that had been viewed over
160,000 times and have since been taken down. (The New York Times is not
linking to this content to limit the spread of misinformation.)

The ground for such medical misinformation is fertile, experts said.
Sarah E. Kreps, a professor of government at Cornell University,
considers the people deliberately spreading distortions to be
practitioners of ``algorithmic capitalism,'' in which people scare up
traffic and sell against it.

\href{https://www.nytimes3xbfgragh.onion/news-event/coronavirus?action=click\&pgtype=Article\&state=default\&region=MAIN_CONTENT_3\&context=storylines_faq}{}

\hypertarget{the-coronavirus-outbreak-}{%
\subsubsection{The Coronavirus Outbreak
›}\label{the-coronavirus-outbreak-}}

\hypertarget{frequently-asked-questions}{%
\paragraph{Frequently Asked
Questions}\label{frequently-asked-questions}}

Updated July 27, 2020

\begin{itemize}
\item ~
  \hypertarget{should-i-refinance-my-mortgage}{%
  \paragraph{Should I refinance my
  mortgage?}\label{should-i-refinance-my-mortgage}}

  \begin{itemize}
  \tightlist
  \item
    \href{https://www.nytimes3xbfgragh.onion/article/coronavirus-money-unemployment.html?action=click\&pgtype=Article\&state=default\&region=MAIN_CONTENT_3\&context=storylines_faq}{It
    could be a good idea,} because mortgage rates have
    \href{https://www.nytimes3xbfgragh.onion/2020/07/16/business/mortgage-rates-below-3-percent.html?action=click\&pgtype=Article\&state=default\&region=MAIN_CONTENT_3\&context=storylines_faq}{never
    been lower.} Refinancing requests have pushed mortgage applications
    to some of the highest levels since 2008, so be prepared to get in
    line. But defaults are also up, so if you're thinking about buying a
    home, be aware that some lenders have tightened their standards.
  \end{itemize}
\item ~
  \hypertarget{what-is-school-going-to-look-like-in-september}{%
  \paragraph{What is school going to look like in
  September?}\label{what-is-school-going-to-look-like-in-september}}

  \begin{itemize}
  \tightlist
  \item
    It is unlikely that many schools will return to a normal schedule
    this fall, requiring the grind of
    \href{https://www.nytimes3xbfgragh.onion/2020/06/05/us/coronavirus-education-lost-learning.html?action=click\&pgtype=Article\&state=default\&region=MAIN_CONTENT_3\&context=storylines_faq}{online
    learning},
    \href{https://www.nytimes3xbfgragh.onion/2020/05/29/us/coronavirus-child-care-centers.html?action=click\&pgtype=Article\&state=default\&region=MAIN_CONTENT_3\&context=storylines_faq}{makeshift
    child care} and
    \href{https://www.nytimes3xbfgragh.onion/2020/06/03/business/economy/coronavirus-working-women.html?action=click\&pgtype=Article\&state=default\&region=MAIN_CONTENT_3\&context=storylines_faq}{stunted
    workdays} to continue. California's two largest public school
    districts --- Los Angeles and San Diego --- said on July 13, that
    \href{https://www.nytimes3xbfgragh.onion/2020/07/13/us/lausd-san-diego-school-reopening.html?action=click\&pgtype=Article\&state=default\&region=MAIN_CONTENT_3\&context=storylines_faq}{instruction
    will be remote-only in the fall}, citing concerns that surging
    coronavirus infections in their areas pose too dire a risk for
    students and teachers. Together, the two districts enroll some
    825,000 students. They are the largest in the country so far to
    abandon plans for even a partial physical return to classrooms when
    they reopen in August. For other districts, the solution won't be an
    all-or-nothing approach.
    \href{https://bioethics.jhu.edu/research-and-outreach/projects/eschool-initiative/school-policy-tracker/}{Many
    systems}, including the nation's largest, New York City, are
    devising
    \href{https://www.nytimes3xbfgragh.onion/2020/06/26/us/coronavirus-schools-reopen-fall.html?action=click\&pgtype=Article\&state=default\&region=MAIN_CONTENT_3\&context=storylines_faq}{hybrid
    plans} that involve spending some days in classrooms and other days
    online. There's no national policy on this yet, so check with your
    municipal school system regularly to see what is happening in your
    community.
  \end{itemize}
\item ~
  \hypertarget{is-the-coronavirus-airborne}{%
  \paragraph{Is the coronavirus
  airborne?}\label{is-the-coronavirus-airborne}}

  \begin{itemize}
  \tightlist
  \item
    The coronavirus
    \href{https://www.nytimes3xbfgragh.onion/2020/07/04/health/239-experts-with-one-big-claim-the-coronavirus-is-airborne.html?action=click\&pgtype=Article\&state=default\&region=MAIN_CONTENT_3\&context=storylines_faq}{can
    stay aloft for hours in tiny droplets in stagnant air}, infecting
    people as they inhale, mounting scientific evidence suggests. This
    risk is highest in crowded indoor spaces with poor ventilation, and
    may help explain super-spreading events reported in meatpacking
    plants, churches and restaurants.
    \href{https://www.nytimes3xbfgragh.onion/2020/07/06/health/coronavirus-airborne-aerosols.html?action=click\&pgtype=Article\&state=default\&region=MAIN_CONTENT_3\&context=storylines_faq}{It's
    unclear how often the virus is spread} via these tiny droplets, or
    aerosols, compared with larger droplets that are expelled when a
    sick person coughs or sneezes, or transmitted through contact with
    contaminated surfaces, said Linsey Marr, an aerosol expert at
    Virginia Tech. Aerosols are released even when a person without
    symptoms exhales, talks or sings, according to Dr. Marr and more
    than 200 other experts, who
    \href{https://academic.oup.com/cid/article/doi/10.1093/cid/ciaa939/5867798}{have
    outlined the evidence in an open letter to the World Health
    Organization}.
  \end{itemize}
\item ~
  \hypertarget{what-are-the-symptoms-of-coronavirus}{%
  \paragraph{What are the symptoms of
  coronavirus?}\label{what-are-the-symptoms-of-coronavirus}}

  \begin{itemize}
  \tightlist
  \item
    Common symptoms
    \href{https://www.nytimes3xbfgragh.onion/article/symptoms-coronavirus.html?action=click\&pgtype=Article\&state=default\&region=MAIN_CONTENT_3\&context=storylines_faq}{include
    fever, a dry cough, fatigue and difficulty breathing or shortness of
    breath.} Some of these symptoms overlap with those of the flu,
    making detection difficult, but runny noses and stuffy sinuses are
    less common.
    \href{https://www.nytimes3xbfgragh.onion/2020/04/27/health/coronavirus-symptoms-cdc.html?action=click\&pgtype=Article\&state=default\&region=MAIN_CONTENT_3\&context=storylines_faq}{The
    C.D.C. has also} added chills, muscle pain, sore throat, headache
    and a new loss of the sense of taste or smell as symptoms to look
    out for. Most people fall ill five to seven days after exposure, but
    symptoms may appear in as few as two days or as many as 14 days.
  \end{itemize}
\item ~
  \hypertarget{does-asymptomatic-transmission-of-covid-19-happen}{%
  \paragraph{Does asymptomatic transmission of Covid-19
  happen?}\label{does-asymptomatic-transmission-of-covid-19-happen}}

  \begin{itemize}
  \tightlist
  \item
    So far, the evidence seems to show it does. A widely cited
    \href{https://www.nature.com/articles/s41591-020-0869-5}{paper}
    published in April suggests that people are most infectious about
    two days before the onset of coronavirus symptoms and estimated that
    44 percent of new infections were a result of transmission from
    people who were not yet showing symptoms. Recently, a top expert at
    the World Health Organization stated that transmission of the
    coronavirus by people who did not have symptoms was ``very rare,''
    \href{https://www.nytimes3xbfgragh.onion/2020/06/09/world/coronavirus-updates.html?action=click\&pgtype=Article\&state=default\&region=MAIN_CONTENT_3\&context=storylines_faq\#link-1f302e21}{but
    she later walked back that statement.}
  \end{itemize}
\end{itemize}

Examples abound. Infowars, the far right website that purveys conspiracy
theories and fake news, and others are now banned on several leading
social media sites but are still advertising pseudoscientific remedies
directly through their own shops. An early distortion of the coronavirus
news appeared in an Infowars video on Jan. 22 --- claiming that the
virus could be part of some man-made plot to thin the population.

``The globalists and the deep state have declared war on humanity,'' a
host on the video said. ``They hate human life. This is why they kill
babies.''

Next to the box in which the video appears is an advertisement for an
immune gargle product that, the ad claims, ``is designed to support your
immune system like no other,'' and that is ``scientifically proven.''

However, the Mayo Clinic reports that the ingredient mentioned in the
product, colloidal silver, has not been proved safe or effective in
treating disease. And even the Infowars shop where the product is listed
reads at the bottom: ``\emph{This product is not intended to diagnose,
treat, cure or prevent any disease.''}

Renée DiResta, the research manager at the Stanford Internet
Observatory, where she studies the spread of false narratives online,
described the coronavirus distortion effort as a product of a
``conspiratorial ecosystem'' that draws on ``die-hard anti-vaxxers or
conspiracy theorists and people who have alternative health modalities
to push and an economic incentive.''

``They liken big pharma to drug pushers,'' she continued, ``and then
tell you how their mushroom or oil is their approach to healing.''

Experts said the cooperation between the W.H.O. and major websites is a
significant change in efforts to stop misinformation. Major internet
companies have been pilloried for their role as sources of
disinformation and for turning a blind eye to the spread of political
lies.

The effort led by the W.H.O. ``is very new,'' said Danny Rogers, who
teaches about disinformation and narrative warfare at the New York
University Center for Global Affairs and is also the chief technology
officer of the Global Disinformation Index, which tracks misinformation
activity online.

Mr. Rogers said dealing with the coronavirus may be easier to address
than political disinformation because it doesn't have partisan or clear
ideological strains.

``The coronavirus is not a voting or paying constituency,'' Mr. Rogers
said. ``We're all united against people getting sick. It's much easier
for platforms to pioneer this coordinated effort around public health
crisis.''

At the same time, he said, the coordination around coronavirus
underscores the reality that social media does have the power to take on
falsehoods.

``It proves that when platforms do choose to act they can be very
influential,'' he said. ``It undercuts the throw up your hands and say
we have no power or ability to control information.''

\emph{Ben Decker contributed reporting.}

Advertisement

\protect\hyperlink{after-bottom}{Continue reading the main story}

\hypertarget{site-index}{%
\subsection{Site Index}\label{site-index}}

\hypertarget{site-information-navigation}{%
\subsection{Site Information
Navigation}\label{site-information-navigation}}

\begin{itemize}
\tightlist
\item
  \href{https://help.nytimes3xbfgragh.onion/hc/en-us/articles/115014792127-Copyright-notice}{©~2020~The
  New York Times Company}
\end{itemize}

\begin{itemize}
\tightlist
\item
  \href{https://www.nytco.com/}{NYTCo}
\item
  \href{https://help.nytimes3xbfgragh.onion/hc/en-us/articles/115015385887-Contact-Us}{Contact
  Us}
\item
  \href{https://www.nytco.com/careers/}{Work with us}
\item
  \href{https://nytmediakit.com/}{Advertise}
\item
  \href{http://www.tbrandstudio.com/}{T Brand Studio}
\item
  \href{https://www.nytimes3xbfgragh.onion/privacy/cookie-policy\#how-do-i-manage-trackers}{Your
  Ad Choices}
\item
  \href{https://www.nytimes3xbfgragh.onion/privacy}{Privacy}
\item
  \href{https://help.nytimes3xbfgragh.onion/hc/en-us/articles/115014893428-Terms-of-service}{Terms
  of Service}
\item
  \href{https://help.nytimes3xbfgragh.onion/hc/en-us/articles/115014893968-Terms-of-sale}{Terms
  of Sale}
\item
  \href{https://spiderbites.nytimes3xbfgragh.onion}{Site Map}
\item
  \href{https://help.nytimes3xbfgragh.onion/hc/en-us}{Help}
\item
  \href{https://www.nytimes3xbfgragh.onion/subscription?campaignId=37WXW}{Subscriptions}
\end{itemize}
