Sections

SEARCH

\protect\hyperlink{site-content}{Skip to
content}\protect\hyperlink{site-index}{Skip to site index}

\href{https://www.nytimes3xbfgragh.onion/section/health}{Health}

\href{https://myaccount.nytimes3xbfgragh.onion/auth/login?response_type=cookie\&client_id=vi}{}

\href{https://www.nytimes3xbfgragh.onion/section/todayspaper}{Today's
Paper}

\href{/section/health}{Health}\textbar{}Another H.I.V. Vaccine Fails a
Trial, Disappointing Researchers

\url{https://nyti.ms/2OssuTj}

\begin{itemize}
\item
\item
\item
\item
\item
\end{itemize}

Advertisement

\protect\hyperlink{after-top}{Continue reading the main story}

Supported by

\protect\hyperlink{after-sponsor}{Continue reading the main story}

Global health

\hypertarget{another-hiv-vaccine-fails-a-trial-disappointing-researchers}{%
\section{Another H.I.V. Vaccine Fails a Trial, Disappointing
Researchers}\label{another-hiv-vaccine-fails-a-trial-disappointing-researchers}}

After more than 30 years of research, 1.7 million people are still
infected each year with the virus that causes AIDS.

\includegraphics{https://static01.graylady3jvrrxbe.onion/images/2020/02/04/science/04HIV/04HIV-articleLarge.jpg?quality=75\&auto=webp\&disable=upscale}

\href{https://www.nytimes3xbfgragh.onion/by/donald-g-mcneil-jr}{\includegraphics{https://static01.graylady3jvrrxbe.onion/images/2018/06/13/multimedia/author-donald-g-mcneil-jr/author-donald-g-mcneil-jr-thumbLarge-v4.png}}

By
\href{https://www.nytimes3xbfgragh.onion/by/donald-g-mcneil-jr}{Donald
G. McNeil Jr.}

\begin{itemize}
\item
  Published Feb. 4, 2020Updated Feb. 5, 2020
\item
  \begin{itemize}
  \item
  \item
  \item
  \item
  \item
  \end{itemize}
\end{itemize}

In another setback in the long quest to prevent H.I.V. infection, a
trial in South Africa has been shut down because an experimental vaccine
was not working, federal health officials announced on Monday.

The trial, which began in 2016, followed one in Thailand that ended in
2009. That vaccine
\href{https://www.nytimes3xbfgragh.onion/2009/10/21/health/research/21vaccine.html}{offered
only modest protection} against infection. Experts argued over how much,
but the vaccine was no more than 30 percent protective.

Nonetheless, it was the only vaccine that had appeared to work at all.

``We hoped this vaccine candidate would work --- regrettably, it does
not,'' said Dr. Anthony S. Fauci, director of the National Institute for
Allergy and Infectious Diseases, which conducted the trial.

``Research continues on other approaches to a safe and effective H.I.V.
vaccine, which I still believe can be achieved,'' he added.

A vaccine against H.I.V., the virus that causes AIDS, is sorely needed.
Even now, nearly 40 years after the start of the epidemic,
\href{https://www.unaids.org/en/resources/documents/2019/2019-UNAIDS-data}{1.7
million people are newly infected each year} --- most of them in Africa,
especially southern Africa, according to UNAIDS, the United Nations'
AIDS-fighting agency.

The trial --- known as HVTN 702 but nicknamed Uhambo, which means
``journey'' in Zulu --- included 5,407 young adult men and women in
South Africa.

Last month, a safety-monitoring panel looked at early results and found
that there were 123 infections among participants who got a placebo
injection and 129 among those who got the vaccine.

That clearly indicated that the vaccine was not protective, but did not
mean it was making participants more vulnerable to H.I.V., scientists
said. A difference of just six infections in so large a pool of
participants could have been due to chance.

The Uhambo vaccine had to be significantly changed from the one tested
in Thailand because South Africa has a different dominant strain of
H.I.V.

The vaccine used canarypox, a bird virus that can infect human cells but
cannot multiply in them, to deliver into the body a protein found on the
outer envelope of H.I.V. The immune system learns to recognize the
protein and to make protective antibodies to it.

``This is a big disappointment, and it means that canarypox is clearly
not a road to success,'' said Mitchell J. Warren, executive director of
AVAC, an H.I.V.-prevention advocacy group based in New York. ``But it
was a well-conducted trial --- it enrolled thousands of participants,
and they came back for their tests.''

``As they say in the pharmaceutical industry, `If you're going to fail,
fail fast,''' he added. ``That way, you don't waste financial and human
resources.''

Two other H.I.V. vaccine trials, Nos. 705 and 706, known as Imbokodo and
Mosaico, are still underway. Both use a common cold virus as the vector
and different surface proteins.

Imbokodo is enrolling women in five southern African countries, and
Mosaico is enrolling gay men and some transgender individuals in Europe
and North America.

\textbf{\emph{{[}}\href{http://on.fb.me/1paTQ1h}{\emph{Like the Science
Times page on Facebook.}}} ****** \emph{\textbar{} Sign up for the}
\textbf{\href{http://nyti.ms/1MbHaRU}{\emph{Science Times
newsletter.}}\emph{{]}}}

For ethical reasons, trial participants were also offered pre-exposure
prophylaxis or PrEP --- a daily pill that prevents H.I.V. infection
---~as well as condoms and advice about how to avoid becoming infected.

But 30 years of condom distribution and counseling have failed to curb
the raging H.I.V. epidemic in southern Africa. Several PrEP trials have
failed there, too, because people enroll in them but then do not take
the pills. So a vaccine remains an important goal.

Scientists are also testing other methods of stopping H.I.V.: injections
of cocktails of antibodies that can block it, injections of long-lasting
H.I.V. drugs, and implants and vaginal rings that release small amounts
of preventive drugs over time.

Advertisement

\protect\hyperlink{after-bottom}{Continue reading the main story}

\hypertarget{site-index}{%
\subsection{Site Index}\label{site-index}}

\hypertarget{site-information-navigation}{%
\subsection{Site Information
Navigation}\label{site-information-navigation}}

\begin{itemize}
\tightlist
\item
  \href{https://help.nytimes3xbfgragh.onion/hc/en-us/articles/115014792127-Copyright-notice}{©~2020~The
  New York Times Company}
\end{itemize}

\begin{itemize}
\tightlist
\item
  \href{https://www.nytco.com/}{NYTCo}
\item
  \href{https://help.nytimes3xbfgragh.onion/hc/en-us/articles/115015385887-Contact-Us}{Contact
  Us}
\item
  \href{https://www.nytco.com/careers/}{Work with us}
\item
  \href{https://nytmediakit.com/}{Advertise}
\item
  \href{http://www.tbrandstudio.com/}{T Brand Studio}
\item
  \href{https://www.nytimes3xbfgragh.onion/privacy/cookie-policy\#how-do-i-manage-trackers}{Your
  Ad Choices}
\item
  \href{https://www.nytimes3xbfgragh.onion/privacy}{Privacy}
\item
  \href{https://help.nytimes3xbfgragh.onion/hc/en-us/articles/115014893428-Terms-of-service}{Terms
  of Service}
\item
  \href{https://help.nytimes3xbfgragh.onion/hc/en-us/articles/115014893968-Terms-of-sale}{Terms
  of Sale}
\item
  \href{https://spiderbites.nytimes3xbfgragh.onion}{Site Map}
\item
  \href{https://help.nytimes3xbfgragh.onion/hc/en-us}{Help}
\item
  \href{https://www.nytimes3xbfgragh.onion/subscription?campaignId=37WXW}{Subscriptions}
\end{itemize}
