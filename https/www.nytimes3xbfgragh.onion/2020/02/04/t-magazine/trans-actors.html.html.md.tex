Sections

SEARCH

\protect\hyperlink{site-content}{Skip to
content}\protect\hyperlink{site-index}{Skip to site index}

\href{https://myaccount.nytimes3xbfgragh.onion/auth/login?response_type=cookie\&client_id=vi}{}

\href{https://www.nytimes3xbfgragh.onion/section/todayspaper}{Today's
Paper}

The Trans Actors Challenging Outmoded Ideas of Masculinity

\url{https://nyti.ms/2UrgoxA}

\begin{itemize}
\item
\item
\item
\item
\item
\end{itemize}

Advertisement

\protect\hyperlink{after-top}{Continue reading the main story}

Supported by

\protect\hyperlink{after-sponsor}{Continue reading the main story}

Social Studies

\hypertarget{the-trans-actors-challenging-outmoded-ideas-of-masculinity}{%
\section{The Trans Actors Challenging Outmoded Ideas of
Masculinity}\label{the-trans-actors-challenging-outmoded-ideas-of-masculinity}}

Despite years of progress, trans male representation in film and
television has remained all but nonexistent. Now, there's a new group of
rising stars.

\includegraphics{https://static01.graylady3jvrrxbe.onion/images/2020/01/30/t-magazine/oakImage-1580405768393/oakImage-1580405768393-articleLarge.jpg?quality=75\&auto=webp\&disable=upscale}

By David Ebershoff

\begin{itemize}
\item
  Feb. 4, 2020
\item
  \begin{itemize}
  \item
  \item
  \item
  \item
  \item
  \end{itemize}
\end{itemize}

\hypertarget{1-first-time-i-saw-me}{%
\subsubsection{1. First Time I Saw Me}\label{1-first-time-i-saw-me}}

Last August, at a premiere party at the NeueHouse on Sunset Boulevard in
Los Angeles, the actor \href{https://www.thebrianmichael.com/}{Brian
Michael Smith} was biting into a slider when he turned around and there
was
\href{https://www.nytimes3xbfgragh.onion/topic/person/oprah-winfrey}{Oprah
Winfrey}. Several years before, as a black transgender man struggling to
break into Hollywood, Smith saw no obvious trajectory to a meaningful
career. Even a college acting teacher said no one would cast him. ``I
saw zero representation of transmasculinity,'' he says, using an
umbrella term that means different things to different people but often
describes trans men and nonbinary people who identify more with
masculinity. ``It was very isolating to grow up and have these dreams. I
didn't see how I was going to be able to do it.''

This is how dreams are murdered, but instead of succumbing, Smith told
himself ``to Oprah this situation'' --- meaning create his own path. And
so he did. He now plays a trio of distinct trans characters on TV:
Toine, the gentle cop on OWN's
``\href{https://www.nytimes3xbfgragh.onion/watching/titles/queen-sugar}{Queen
Sugar},'' whose 2017 coming-out episode coincided with Smith's public
coming out; Pierce, a political strategist on Showtime's
\href{https://www.nytimes3xbfgragh.onion/2019/11/29/arts/television/l-word-generation-q.html}{sequel
to ``The L Word,''} which debuted in December; and a firefighter on
Fox's new
``\href{https://www.nytimes3xbfgragh.onion/2020/01/02/arts/television/50-shows-to-watch.html}{9-1-1:
Lone Star}.'' At the premiere, Smith saw his opportunity to thank the
woman whose name had become his own inspirational verb. He swallowed the
slider, extended his hand --- and you know what Oprah said? ``I know who
you are.''

\includegraphics{https://static01.graylady3jvrrxbe.onion/images/2020/02/04/t-magazine/04tmag-transmaleactors-slide-QH4U/04tmag-transmaleactors-slide-QH4U-articleLarge.jpg?quality=75\&auto=webp\&disable=upscale}

Smith, 37, tells me this story in a Los Angeles hotel lobby on a rare
day off. He keeps his head shaved, his beard trimmed and a polished
stone from a yoga retreat in his pocket. Something he talks about ---
something all the trans male and nonbinary actors I interviewed for this
story talk about --- is why visibility matters. ``It's necessary for
people to see themselves onscreen,'' says
\href{https://www.shaandasani.com/}{Shaan Dasani}, who appeared in two
2019 web comedies, ``\href{http://www.thesethems.com/}{These Thems}''
and ``Razor Tongue.'' ``It's necessary for people to see multiple
versions of masculinity.''

In 2014, Laverne Cox appeared on Time magazine's cover with the headline
``\href{https://time.com/135480/transgender-tipping-point/}{The
Transgender Tipping Point}.'' Since then, trans women have been working
in Hollywood in increasing numbers, but that tipping point is only
coming now for trans male and transmasculine actors and story lines.
``We've been invisible,'' says
\href{https://www.glaad.org/nick-adams-director-transgender-media-representation}{Nick
Adams}, the director of transgender representation at Glaad. He keeps an
unofficial tally of trans men in film and television, dating back to a
1987 episode of
``\href{https://www.nytimes3xbfgragh.onion/watching/recommendations/the-golden-girls?auto=true}{The
Golden Girls}.'' The next entries come in 1999: an episode of the CBS
series ``L.A. Doctors,'' about a teenager who abuses masculinizing
hormones, and
``\href{https://www.nytimes3xbfgragh.onion/2019/10/09/movies/boys-dont-cry-anniversary.html}{Boys
Don't Cry},'' about the life and murder of Brandon Teena, played by
Hilary Swank. ``Five years ago, the kind of roles I'm doing would have
\href{https://www.nytimes3xbfgragh.onion/2015/09/04/movies/who-gets-to-play-the-transgender-part-in-hollywood.html}{gone
to cisgender actors},'' says
\href{https://www.instagram.com/theogermaine/?hl=en}{Theo Germaine}, 27,
of their recent parts as young trans men on Netflix's
``\href{https://www.nytimes3xbfgragh.onion/2019/09/26/business/media/netflix-ryan-murphy-the-politician.html}{The
Politician}'' and Showtime's
``\href{https://www.nytimes3xbfgragh.onion/2020/01/23/arts/television/david-lynch-netflix-work-in-progress-showtime.html}{Work
in Progress}'' (Germaine identifies as nonbinary and uses both male and
gender-neutral pronouns). Germaine is correct, but the reality is
starker: Five years ago, these roles mostly didn't exist. When a
transmasculine character did pop up, he was often a victim, his story
limited to and by trans trauma; Smith describes seeing ``Boys Don't
Cry'' while in high school as both affirming and terrifying.

But in the last year, we've witnessed more trans male and nonbinary
actors onscreen than ever before. Even more important is what the actors
and their roles represent. They are reflecting back the reality of trans
male and nonbinary lives while mainstreaming long-marginalized
characters and narratives. They are introducing multidimensional
characters whose gender intersects with other facets of identity ---
race, class, sexual orientation, disability. Through their performances
and social media, the actors are updating and expanding the very idea of
the leading man.

Why is this vital? Let me start with the most basic reason: survival.
The actors are creating characters that audiences have never seen before
at a time when right-wing politicians are trying to strip trans people
of not only their rights (the
\href{https://www.nytimes3xbfgragh.onion/2018/07/05/us/military-transgender-recruits.html}{military's
recent restrictions} surrounding transgender troops and recruits, for
example) but their humanity (think of all the so-called
\href{https://www.nytimes3xbfgragh.onion/2019/07/23/us/north-carolina-transgender-bathrooms.html}{bathroom
bills}). A paradox of America 2020: There's been a swift advancement of
trans visibility and equality, even as anti-trans violence has become
what both the
\href{https://www.hrc.org/resources/a-national-epidemic-fatal-anti-trans-violence-in-the-united-states-in-2019}{Human
Rights Campaign} and the
\href{https://www.nytimes3xbfgragh.onion/2019/09/27/us/transgender-women-deaths.html}{American
Medical Association} call an epidemic, and an unprecedented acceptance
of trans folks, even as the Supreme Court considers whether someone's
gender identity is grounds for termination from employment. More than
half of trans male adolescents have attempted suicide, according to a
2018 study published in the journal Pediatrics. ``There's a reason for
that,'' says \href{https://www.scotttschofield.com/}{Scott Turner
Schofield}, who stars in Amazon's new
``\href{https://www.amazon.com/Studio-City/dp/B0831S9K6R}{Studio
City}.'' ``We're raised to believe there's something wrong with us.
We're raised to believe we're the only one.'' So when Smith's character
came out on ``Queen Sugar,'' Twitter lit up with the hashtag
\href{https://www.instagram.com/explore/tags/firsttimeisawme/}{\#FirstTimeISawMe}.
Progress --- social, cultural, political --- always begins with the
self.

Image

``It's necessary for people to see themselves onscreen,'' says Dasani,
center. ``It's necessary for people to see multiple versions of
masculinity.''Credit...Andy Freeberg

\hypertarget{2-superheroes}{%
\subsubsection{2. Superheroes}\label{2-superheroes}}

Two and a half years ago, \href{https://www.leosheng.com/}{Leo Sheng}
was in Ann Arbor, Mich., about to start a master's in social work when a
casting agent messaged him on Instagram. ``Acting was not on my radar,''
he says. Sheng flew to New York for an audition. As he boarded the plane
back to Michigan, his phone rang: He got the part. In ``Adam,'' which
premiered last August, Sheng plays Ethan, a young trans man so
emotionally grounded that he becomes the ballast for the cis characters
flailing all around him, thus flipping a trans narrative trope. Soon
after, he was cast in ``The L Word: Generation Q'' as Micah, an adjunct
professor of social work.

Sheng, now 23, recognizes the potential for social and political change
in acting: Through characters like Ethan and Micah, he's helping
Hollywood revise its depictions of trans men, catching up to trans lives
as they're actually lived. Story lines are moving past transition into
love, friendship, work, family --- the everything-ness of a man's life.
Sheng and the other actors are portraying men not defined by crisis or
fear, or hormones and chest-binding, but in the midst of full and
(mostly) happy lives --- ``a type of happiness a lot of people want to
know is possible,'' Sheng says.

Sheng uses social media to further complicate the narrative, engaging in
an ongoing deconstruction of who and what defines the male self. He
posts about going to the gym and his evolving relationship to muscles.
He wants people to see trans male bodies as they are, whether ripped or
soft, hairy or smooth, boyish or dad-ish, scarred or not. Sheng recently
posted about his period, a frankness that drew praise but also online
attacks about his identity.

Something we can't forget: Even as the actors appear in more and more
celebrated projects, some people continue to deny their existence. The
English actor, writer and director Jake Graf, 42, says in the past trans
men were invisible, both onscreen and in broader society, in part
because many could choose whether to disclose their identity: ``Largely
due to our physicality, we've been afforded the luxury of living that
unseen, under-the-radar, stealth life.'' Their reasons were complex and
understandable (personal safety, social and financial stability, for
example), but one consequence has been that there is now far less
awareness of trans men than of trans women. Society has a long and
unfortunate history of gazing at and fetishizing trans women, but that
has been less the case with trans men. That's a generalization, of
course, but only in service of sharing a point many of the actors made
to me: They now want to be seen; they now want people to know they
exist. ``Trans women have historically been more visible,'' Graf says.
``Trans men have been out there doing things much more quietly, which is
great for them, but not great for visibility.''

Graf used to audition without disclosing his identity, and casting
directors saw him as another guy in the gaggle. Only after coming out
could he stand out, booking roles in 2018's
``\href{https://www.nytimes3xbfgragh.onion/2018/09/19/movies/colette-keira-knightley-costumes.html}{Colette}''
and 2015's
``\href{https://www.nytimes3xbfgragh.onion/2016/02/18/movies/oscars-2016-the-danish-girl-production-design.html}{The
Danish Girl}'' (based on a novel I wrote). With his square jaw and
British charm, Graf embodies the classic leading man while also
subverting the very notion. He started making short films, highlighting
the fine-grain details of ordinary trans lives: a young man visiting his
gynecologist; an older man recalling life before the queer and trans
rights movements --- a multiplicity of stories Hollywood is only now
incorporating.

``There's no one version of a trans guy in Hollywood anymore,'' says
Elliot Fletcher, 23. From 2016 to 2018, Fletcher took on three
consecutive trans roles that, viewed together, proved groundbreaking. On
MTV's ``Faking It,'' he played a high schooler navigating his gender
identity and sexuality. On Freeform's
``\href{https://www.nytimes3xbfgragh.onion/watching/recommendations/the-fosters}{The
Fosters},'' he was the sweetly rebellious boyfriend. On Showtime's
``\href{https://www.nytimes3xbfgragh.onion/watching/recommendations/shameless-us}{Shameless},''
Fletcher plays an L.G.B.T.Q. activist who is simultaneously insulting,
raunchy and endearing. The role he'd really love, though, is the next
Spider-Man. With the right glasses, he could pass for Peter Parker. When
I asked the actors about their dream roles, most said they want to play
a superhero. A superhero implies someone elite, a status long denied to
trans and gender-nonconforming people.

Image

Laura Dreyfuss, Theo Germaine and Ben Platt in ``The Politician''
(2019).Credit...Netflix

\hypertarget{3-i-am-my-own-masculinity}{%
\subsubsection{3. I Am My Own
Masculinity}\label{3-i-am-my-own-masculinity}}

In ``The Politician,'' Germaine's character, James, toggles between
running the student-body campaign of his best friend, Payton, and
sleeping with Payton's girlfriend. ``I was like, `Oh my God, people are
going to hate me because I'm not playing a nice person,''' Germaine
says. In fact, James represents an evolution of the trans male character
beyond hero, saint or martyr to someone allowed to be ruthless and
deceitful. The character breakdown describes James as trans, but his
gender identity is never explicitly defined --- it's so unremarkable,
it's never remarked upon. It's what makes the character so
transformational; James refuses to take on the burdens of definition or
explanation. If you look at him only through the lens of gender
identity, he won't engage. He transfers questions of identity from him
to you. As a nonbinary actor, Germaine says they still feel some
pressure to justify and self-explain but hope audiences are ready to
move past fixed ideas of how a trans or nonbinary person should look.

Viewers seem to be. Germaine's James takes us to a place beyond
``passing'' --- a word the actors don't like but also acknowledge as a
real factor in how trans men are depicted. Hollywood still mostly
employs actors who can be perceived as cis. On the one hand, this
represents progress: The actors want casting directors to see them for
any relevant part, whether cis or trans. But it's also a privilege that
excludes many. ``There are so many different ways to express gender,''
Sheng says. ``I would love to see more nonbinary, genderqueer and trans
folks who can't `pass' be given opportunities.''

Recently Chella Man, 21, joined the cast of DC Universe's series
``Titans,'' playing Jericho, who is mute, biracial and bisexual. Soon
after, Man, who identifies as genderqueer, modeled for Calvin Klein
flexing his biceps in black boxer briefs. It's a radical act, he says,
``to showcase my flat-chested, penis- less body.'' He says he received a
lot of love and a lot of transphobia, neither of which surprised him. He
\href{https://www.instagram.com/p/B2Z7egEHLaZ/}{posted} on Instagram
from the shoot with the caption: ``No visible underwear bulge. Jewish
and Asian history and representation in my DNA and on my skin. Top
surgery scars out and proud. Visible cochlear implants paired with my
DEAF AF tattoo.'' Man followed this with an older image of himself ---
in track pants and shirtless, pre-top surgery. A bicep tattoo gives the
photo its caption and meaning: ``I Am My Own Masculinity.'' Man is
saying through image and ink that he can define his own masculinity,
rather than let it define him.

This, then, is what comes next: shifting from a past where gender was
handed to us by society's cues and prompts to a future of expressing who
we are in terms we control. ``Masculinity stems from gender, which is
socially constructed,'' says Man. ``Anyone has the potential to unlearn
social constructs and/or redefine what they may mean to them.''
Collectively, the actors are engaged in this conversation about gender
and identity, leading us to a day when those conversations are no longer
necessary.

Image

Leo Sheng as Micah Lee and Freddy Miyares as Jose Garcia in ``The L
Word: Generation Q'' (2020).Credit...Erica Parise/Showtime

\hypertarget{4-the-gates-of-paramount}{%
\subsubsection{4. The Gates of
Paramount}\label{4-the-gates-of-paramount}}

At the beginning of 2019,
\href{https://www.nytimes3xbfgragh.onion/2014/08/31/magazine/can-jill-soloway-do-justice-to-the-trans-movement.html}{Jill
Soloway}, the creator of Amazon's
``\href{https://www.nytimes3xbfgragh.onion/watching/recommendations/watching-tv-transparent?auto=true}{Transparent},''
invited Schofield, who declined to give his age, for a hike in Griffith
Park. In the chaparral above Los Angeles, they discussed organizing a
new group under the umbrella of 5050 by 2020, a strategic initiative
working toward gender parity across all Hollywood professions that
Soloway helps lead. Soloway, 54, who identifies as nonbinary, says that
many trans men and nonbinary people have a unique perspective on the
issues of equality, opportunity and the post-\#MeToo discussion of
masculinity and its privileges. Smith agrees: As he transitioned, he had
to ask himself what kind of man he wanted to be, ``examining that
earlier on and more intensely than cis men.'' He believes the importance
society places on masculinity might be more problematic than masculinity
in and of itself. Dasani, who also declined to give his age, echoed that
with a story from his time in film school, before transitioning: In a
study group with three film bros, he found himself ignored. After
transitioning, he noticed what he said wasn't devalued in a way he felt
it had been before. ``I remember clocking this --- be the guy who makes
room for other voices at the table.''

In April 2019, about 30 actors, writers, directors and editors met in a
boardroom on the Paramount lot. They gathered around an imposing
executive table, the kind that has long excluded them. The cohort's
goals are both practical (networking, professional development) and
inspirational (support, friendship). For some, it's the only time
they've been in a space with so many like themselves. As far as anyone
knows, it's a first for Hollywood. ``There's so much tenderness in the
room,'' says Dasani of the now monthly meetings. The symbolism of the
Paramount lot isn't lost: For a long time, those gates have been closed
to many communities. When I ask Dasani about this moment of increasing
representation, he corrects me. ``I hope it's more than a moment. I hope
it's a cultural shift.'' A shift to ensure the gates never close again.

Grooming: Christina Guerra and Hailey Adickes at Celestine Agency.

Advertisement

\protect\hyperlink{after-bottom}{Continue reading the main story}

\hypertarget{site-index}{%
\subsection{Site Index}\label{site-index}}

\hypertarget{site-information-navigation}{%
\subsection{Site Information
Navigation}\label{site-information-navigation}}

\begin{itemize}
\tightlist
\item
  \href{https://help.nytimes3xbfgragh.onion/hc/en-us/articles/115014792127-Copyright-notice}{©~2020~The
  New York Times Company}
\end{itemize}

\begin{itemize}
\tightlist
\item
  \href{https://www.nytco.com/}{NYTCo}
\item
  \href{https://help.nytimes3xbfgragh.onion/hc/en-us/articles/115015385887-Contact-Us}{Contact
  Us}
\item
  \href{https://www.nytco.com/careers/}{Work with us}
\item
  \href{https://nytmediakit.com/}{Advertise}
\item
  \href{http://www.tbrandstudio.com/}{T Brand Studio}
\item
  \href{https://www.nytimes3xbfgragh.onion/privacy/cookie-policy\#how-do-i-manage-trackers}{Your
  Ad Choices}
\item
  \href{https://www.nytimes3xbfgragh.onion/privacy}{Privacy}
\item
  \href{https://help.nytimes3xbfgragh.onion/hc/en-us/articles/115014893428-Terms-of-service}{Terms
  of Service}
\item
  \href{https://help.nytimes3xbfgragh.onion/hc/en-us/articles/115014893968-Terms-of-sale}{Terms
  of Sale}
\item
  \href{https://spiderbites.nytimes3xbfgragh.onion}{Site Map}
\item
  \href{https://help.nytimes3xbfgragh.onion/hc/en-us}{Help}
\item
  \href{https://www.nytimes3xbfgragh.onion/subscription?campaignId=37WXW}{Subscriptions}
\end{itemize}
