Sections

SEARCH

\protect\hyperlink{site-content}{Skip to
content}\protect\hyperlink{site-index}{Skip to site index}

\href{https://www.nytimes3xbfgragh.onion/section/health}{Health}

\href{https://myaccount.nytimes3xbfgragh.onion/auth/login?response_type=cookie\&client_id=vi}{}

\href{https://www.nytimes3xbfgragh.onion/section/todayspaper}{Today's
Paper}

\href{/section/health}{Health}\textbar{}Wuhan Coronavirus Looks
Increasingly Like a Pandemic, Experts Say

\url{https://nyti.ms/2GOL83q}

\begin{itemize}
\item
\item
\item
\item
\item
\item
\end{itemize}

\href{https://www.nytimes3xbfgragh.onion/news-event/coronavirus?action=click\&pgtype=Article\&state=default\&region=TOP_BANNER\&context=storylines_menu}{The
Coronavirus Outbreak}

\begin{itemize}
\tightlist
\item
  live\href{https://www.nytimes3xbfgragh.onion/2020/08/01/world/coronavirus-covid-19.html?action=click\&pgtype=Article\&state=default\&region=TOP_BANNER\&context=storylines_menu}{Latest
  Updates}
\item
  \href{https://www.nytimes3xbfgragh.onion/interactive/2020/us/coronavirus-us-cases.html?action=click\&pgtype=Article\&state=default\&region=TOP_BANNER\&context=storylines_menu}{Maps
  and Cases}
\item
  \href{https://www.nytimes3xbfgragh.onion/interactive/2020/science/coronavirus-vaccine-tracker.html?action=click\&pgtype=Article\&state=default\&region=TOP_BANNER\&context=storylines_menu}{Vaccine
  Tracker}
\item
  \href{https://www.nytimes3xbfgragh.onion/interactive/2020/07/29/us/schools-reopening-coronavirus.html?action=click\&pgtype=Article\&state=default\&region=TOP_BANNER\&context=storylines_menu}{What
  School May Look Like}
\item
  \href{https://www.nytimes3xbfgragh.onion/live/2020/07/31/business/stock-market-today-coronavirus?action=click\&pgtype=Article\&state=default\&region=TOP_BANNER\&context=storylines_menu}{Economy}
\end{itemize}

Advertisement

\protect\hyperlink{after-top}{Continue reading the main story}

Supported by

\protect\hyperlink{after-sponsor}{Continue reading the main story}

Global health

\hypertarget{wuhan-coronavirus-looks-increasingly-like-a-pandemic-experts-say}{%
\section{Wuhan Coronavirus Looks Increasingly Like a Pandemic, Experts
Say}\label{wuhan-coronavirus-looks-increasingly-like-a-pandemic-experts-say}}

Rapidly rising caseloads alarm researchers, who fear the virus may make
its way across the globe. But scientists cannot yet predict how many
deaths may result.

\includegraphics{https://static01.graylady3jvrrxbe.onion/images/2020/02/02/world/02virus-pandemic-sub/02virus-pandemic-sub-articleLarge.jpg?quality=75\&auto=webp\&disable=upscale}

\href{https://www.nytimes3xbfgragh.onion/by/donald-g-mcneil-jr}{\includegraphics{https://static01.graylady3jvrrxbe.onion/images/2018/06/13/multimedia/author-donald-g-mcneil-jr/author-donald-g-mcneil-jr-thumbLarge-v4.png}}

By
\href{https://www.nytimes3xbfgragh.onion/by/donald-g-mcneil-jr}{Donald
G. McNeil Jr.}

\begin{itemize}
\item
  Published Feb. 2, 2020Updated Feb. 20, 2020
\item
  \begin{itemize}
  \item
  \item
  \item
  \item
  \item
  \item
  \end{itemize}
\end{itemize}

\href{https://cn.nytimes3xbfgragh.onion/china/20200203/coronavirus-pandemic-china/}{阅读简体中文版}\href{https://cn.nytimes3xbfgragh.onion/china/20200203/coronavirus-pandemic-china/zh-hant/}{閱讀繁體中文版}

The
\href{https://www.nytimes3xbfgragh.onion/2020/02/20/world/asia/japan-coronavirus-clusters.html}{Wuhan
coronavirus} spreading from China is now likely to become a pandemic
that circles the globe, according to many of the world's leading
infectious disease experts.

The prospect is daunting. A pandemic --- an ongoing epidemic on two or
more continents --- may well have global consequences, despite the
extraordinary travel restrictions and quarantines now imposed by
\href{https://www.nytimes3xbfgragh.onion/2020/02/20/world/asia/japan-coronavirus-clusters.html}{China}
and other countries, including the United States.

Scientists do not yet know how lethal the new coronavirus is, however,
so there is uncertainty about how much damage a pandemic might cause.
But there is growing consensus that the pathogen is readily transmitted
between humans.

The Wuhan coronavirus is spreading more like influenza, which is highly
transmissible, than like its slow-moving viral cousins, SARS and MERS,
scientists have found.

``It's very, very transmissible, and it almost certainly is going to be
a pandemic,'' said Dr. Anthony S. Fauci, director of the National
Institute of Allergy and Infectious Disease.

``But will it be catastrophic? I don't know.''

In the last three weeks, the number of lab-confirmed cases has soared
from about 50 in China to more than
\href{https://www.nytimes3xbfgragh.onion/2020/02/02/world/asia/china-coronavirus.html}{17,000
in at least 23 countries; there have been more than 360 deaths}.

But various epidemiological models estimate that the real number of
cases is 100,000 or even more. While that expansion is not as rapid as
that of flu or measles, it is an enormous leap beyond what virologists
saw when SARS and MERS emerged.

When SARS was vanquished in July 2003 after spreading for nine months,
only 8,098 cases had been confirmed. MERS has been circulating since
2012, but there have been only about 2,500 known cases.

The biggest uncertainty now, experts said, is how many people around the
world will die. SARS killed about 10 percent of those who got it, and
MERS now kills about one of three.

The \href{https://wwwnc.cdc.gov/eid/article/12/1/05-0979_article}{1918
``Spanish flu''} killed only about 2.5 percent of its victims --- but
because it infected so many people and medical care was much cruder
then, an estimated 50 million died, perhaps even more.

By contrast, the highly transmissible H1N1 ``swine flu'' pandemic of
2009
killed\href{https://www.thelancet.com/journals/laninf/article/PIIS1473-3099(12)70121-4/fulltext}{about
285,000}, fewer than seasonal flu normally does, and had a relatively
low fatality rate, estimated at .02 percent.

The mortality rate for known cases of the Wuhan coronavirus has been
running about 2 percent, although that is likely to drop as more tests
are done and more mild cases are found.

\includegraphics{https://static01.graylady3jvrrxbe.onion/images/2020/02/01/science/31virus-pandemic/31virus-pandemic-articleLarge.jpg?quality=75\&auto=webp\&disable=upscale}

It is ``increasingly unlikely that the virus can be contained,'' said
Dr. Thomas R. Frieden, a former director of the Centers for Disease
Control and Prevention who now runs Resolve to Save Lives, a nonprofit
devoted to fighting epidemics.

\hypertarget{latest-updates-global-coronavirus-outbreak}{%
\section{\texorpdfstring{\href{https://www.nytimes3xbfgragh.onion/2020/08/01/world/coronavirus-covid-19.html?action=click\&pgtype=Article\&state=default\&region=MAIN_CONTENT_1\&context=storylines_live_updates}{Latest
Updates: Global Coronavirus
Outbreak}}{Latest Updates: Global Coronavirus Outbreak}}\label{latest-updates-global-coronavirus-outbreak}}

Updated 2020-08-01T19:54:00.494Z

\begin{itemize}
\tightlist
\item
  \href{https://www.nytimes3xbfgragh.onion/2020/08/01/world/coronavirus-covid-19.html?action=click\&pgtype=Article\&state=default\&region=MAIN_CONTENT_1\&context=storylines_live_updates\#link-3ac56579}{Top
  officials work to break impasse over jobless benefit.}
\item
  \href{https://www.nytimes3xbfgragh.onion/2020/08/01/world/coronavirus-covid-19.html?action=click\&pgtype=Article\&state=default\&region=MAIN_CONTENT_1\&context=storylines_live_updates\#link-8796723}{The
  virus picks up dangerous speed in the Midwest, and in areas that had
  seen success.}
\item
  \href{https://www.nytimes3xbfgragh.onion/2020/08/01/world/coronavirus-covid-19.html?action=click\&pgtype=Article\&state=default\&region=MAIN_CONTENT_1\&context=storylines_live_updates\#link-25930521}{Thousands
  in Berlin protest Germany's coronavirus measures.}
\end{itemize}

\href{https://www.nytimes3xbfgragh.onion/2020/08/01/world/coronavirus-covid-19.html?action=click\&pgtype=Article\&state=default\&region=MAIN_CONTENT_1\&context=storylines_live_updates}{See
more updates}

More live coverage:
\href{https://www.nytimes3xbfgragh.onion/live/2020/07/31/business/stock-market-today-coronavirus?action=click\&pgtype=Article\&state=default\&region=MAIN_CONTENT_1\&context=storylines_live_updates}{Markets}

``It is therefore likely that it will spread, as flu and other organisms
do, but we still don't know how far, wide or deadly it will be.''

In the early days of the 2009 flu pandemic, ``they were talking about
Armageddon in Mexico,'' Dr. Fauci said. (That virus first emerged in
pig-farming areas in Mexico's Veracruz State.) ``But it turned out to
not be that severe.''

An accurate estimate of the virus's lethality will not be possible until
certain kinds of studies can be done: blood tests to see how many people
have antibodies, household studies to learn how often it infects family
members, and genetic sequencing to determine whether some strains are
more dangerous than others.

Closing borders to highly infectious pathogens never succeeds
completely, experts said, because all frontiers are somewhat porous.
Nonetheless, closings and rigorous screening may slow the spread, which
will buy time for the development of drug treatments and vaccines.

Other important unknowns include who is most at risk, whether coughing
or contaminated surfaces are more likely to transmit the virus, how fast
the virus can mutate and whether it will fade out when the weather
warms.

\href{https://www.nytimes3xbfgragh.onion/interactive/2020/world/coronavirus-maps.html}{}

\includegraphics{https://static01.graylady3jvrrxbe.onion/images/2020/03/03/world/coronavirus-map-promo/coronavirus-map-promo-articleLarge-v672.png}

\hypertarget{coronavirus-map-tracking-the-global-outbreak}{%
\subsection{Coronavirus Map: Tracking the Global
Outbreak}\label{coronavirus-map-tracking-the-global-outbreak}}

The virus has infected more than 17,624,000 people and has been detected
in nearly every country.

The effects of a pandemic would probably be harsher in some countries
than in others. While the United States and other wealthy countries may
be able to detect and quarantine the first carriers, countries with
fragile health care systems will not. The virus has already reached
\href{https://www.phnompenhpost.com/national/first-case-coronavirus-reported-kingdom}{Cambodia},
\href{https://qz.com/india/1793841/indias-first-confirmed-case-of-coronavirus-reported-in-kerala/}{India},
\href{https://www.thestar.com.my/news/nation/2020/01/30/coronavirus-eighth-positive-case-in-m039sia-confirmed-thursday-jan-30}{Malaysia},
\href{https://www.reuters.com/article/us-china-health-nepal/nepal-confirms-first-case-of-new-coronavirus-idUSKBN1ZN1S2}{Nepal},
\href{https://www.reuters.com/article/us-china-health-philippines/philippines-confirms-first-case-of-new-coronavirus-health-minister-idUSKBN1ZT0S0}{the
Philippines} and rural
\href{https://www.themoscowtimes.com/2020/01/31/russia-reports-first-coronavirus-cases-a69123}{Russia}.

``This looks far more like H1N1's spread than SARS, and I am
increasingly alarmed,'' said Dr. Peter Piot, director of the London
School of Hygiene and Tropical Medicine. ``Even 1 percent mortality
would mean 10,000 deaths in each million people.''

Other experts were more cautious.

Dr. Michael Ryan, head of emergency responses for the World Health
Organization, said in
\href{https://www.statnews.com/2020/02/01/top-who-official-says-not-too-late-to-stop-coronavirus-outbreak/}{an
interview with STAT News} on Saturday that there was ``evidence to
suggest this virus can still be contained'' and that the world needed to
``keep trying.''

Dr. W. Ian Lipkin, a virus-hunter at the Columbia University Mailman
School of Public Health who is in China advising its Center for Disease
Control and Prevention, said that although the virus is clearly being
transmitted through casual contact, labs are still behind in processing
samples.

Image

In Hyderabad, India, doctors left an isolation ward for people kept
under observation after returning from China.Credit...Mahesh
Kumar/Associated Press

But life in China has radically changed in the last two weeks. Streets
are deserted, public events are canceled, and citizens are wearing masks
and washing their hands, Dr. Lipkin said. All of that may have slowed
down what lab testing indicated was exponential growth in the infection.

It's unclear exactly how accurate tests done in overwhelmed Chinese
laboratories are. On the one hand, Chinese state media have reported
test kit shortages and processing bottlenecks, which could produce an
undercount.

But Dr. Lipkin said he knew of one lab running 5,000 samples a day,
which might produce some false-positive results, inflating the count.
``You can't possibly do quality control at that rate,'' he said.

Anecdotal reports from China, and
\href{https://www.scientificamerican.com/article/study-reports-first-case-of-coronavirus-spread-by-asymptomatic-person/}{one
published study from Germany}, indicate that some people infected with
the Wuhan coronavirus can pass it on before they show symptoms. That may
make border-screening much harder, scientists said.

Epidemiological modeling released Friday by the
\href{https://www.nytimes3xbfgragh.onion/2020/02/10/world/europe/coronavirus-europe.html}{European
Center for Disease Prevention} and Control estimated that 75 percent of
infected people reaching
\href{https://www.nytimes3xbfgragh.onion/2020/02/10/world/europe/coronavirus-europe.html}{Europe}
from China would still be in the incubation periods upon arrival, and
therefore not detected by airport screening, which looks for fevers,
coughs and breathing difficulties.

But if thermal cameras miss victims who are beyond incubation and
actively infecting others, the real number of missed carriers may be
higher than 75 percent.

\href{https://www.nytimes3xbfgragh.onion/news-event/coronavirus?action=click\&pgtype=Article\&state=default\&region=MAIN_CONTENT_3\&context=storylines_faq}{}

\hypertarget{the-coronavirus-outbreak-}{%
\subsubsection{The Coronavirus Outbreak
›}\label{the-coronavirus-outbreak-}}

\hypertarget{frequently-asked-questions}{%
\paragraph{Frequently Asked
Questions}\label{frequently-asked-questions}}

Updated July 27, 2020

\begin{itemize}
\item ~
  \hypertarget{should-i-refinance-my-mortgage}{%
  \paragraph{Should I refinance my
  mortgage?}\label{should-i-refinance-my-mortgage}}

  \begin{itemize}
  \tightlist
  \item
    \href{https://www.nytimes3xbfgragh.onion/article/coronavirus-money-unemployment.html?action=click\&pgtype=Article\&state=default\&region=MAIN_CONTENT_3\&context=storylines_faq}{It
    could be a good idea,} because mortgage rates have
    \href{https://www.nytimes3xbfgragh.onion/2020/07/16/business/mortgage-rates-below-3-percent.html?action=click\&pgtype=Article\&state=default\&region=MAIN_CONTENT_3\&context=storylines_faq}{never
    been lower.} Refinancing requests have pushed mortgage applications
    to some of the highest levels since 2008, so be prepared to get in
    line. But defaults are also up, so if you're thinking about buying a
    home, be aware that some lenders have tightened their standards.
  \end{itemize}
\item ~
  \hypertarget{what-is-school-going-to-look-like-in-september}{%
  \paragraph{What is school going to look like in
  September?}\label{what-is-school-going-to-look-like-in-september}}

  \begin{itemize}
  \tightlist
  \item
    It is unlikely that many schools will return to a normal schedule
    this fall, requiring the grind of
    \href{https://www.nytimes3xbfgragh.onion/2020/06/05/us/coronavirus-education-lost-learning.html?action=click\&pgtype=Article\&state=default\&region=MAIN_CONTENT_3\&context=storylines_faq}{online
    learning},
    \href{https://www.nytimes3xbfgragh.onion/2020/05/29/us/coronavirus-child-care-centers.html?action=click\&pgtype=Article\&state=default\&region=MAIN_CONTENT_3\&context=storylines_faq}{makeshift
    child care} and
    \href{https://www.nytimes3xbfgragh.onion/2020/06/03/business/economy/coronavirus-working-women.html?action=click\&pgtype=Article\&state=default\&region=MAIN_CONTENT_3\&context=storylines_faq}{stunted
    workdays} to continue. California's two largest public school
    districts --- Los Angeles and San Diego --- said on July 13, that
    \href{https://www.nytimes3xbfgragh.onion/2020/07/13/us/lausd-san-diego-school-reopening.html?action=click\&pgtype=Article\&state=default\&region=MAIN_CONTENT_3\&context=storylines_faq}{instruction
    will be remote-only in the fall}, citing concerns that surging
    coronavirus infections in their areas pose too dire a risk for
    students and teachers. Together, the two districts enroll some
    825,000 students. They are the largest in the country so far to
    abandon plans for even a partial physical return to classrooms when
    they reopen in August. For other districts, the solution won't be an
    all-or-nothing approach.
    \href{https://bioethics.jhu.edu/research-and-outreach/projects/eschool-initiative/school-policy-tracker/}{Many
    systems}, including the nation's largest, New York City, are
    devising
    \href{https://www.nytimes3xbfgragh.onion/2020/06/26/us/coronavirus-schools-reopen-fall.html?action=click\&pgtype=Article\&state=default\&region=MAIN_CONTENT_3\&context=storylines_faq}{hybrid
    plans} that involve spending some days in classrooms and other days
    online. There's no national policy on this yet, so check with your
    municipal school system regularly to see what is happening in your
    community.
  \end{itemize}
\item ~
  \hypertarget{is-the-coronavirus-airborne}{%
  \paragraph{Is the coronavirus
  airborne?}\label{is-the-coronavirus-airborne}}

  \begin{itemize}
  \tightlist
  \item
    The coronavirus
    \href{https://www.nytimes3xbfgragh.onion/2020/07/04/health/239-experts-with-one-big-claim-the-coronavirus-is-airborne.html?action=click\&pgtype=Article\&state=default\&region=MAIN_CONTENT_3\&context=storylines_faq}{can
    stay aloft for hours in tiny droplets in stagnant air}, infecting
    people as they inhale, mounting scientific evidence suggests. This
    risk is highest in crowded indoor spaces with poor ventilation, and
    may help explain super-spreading events reported in meatpacking
    plants, churches and restaurants.
    \href{https://www.nytimes3xbfgragh.onion/2020/07/06/health/coronavirus-airborne-aerosols.html?action=click\&pgtype=Article\&state=default\&region=MAIN_CONTENT_3\&context=storylines_faq}{It's
    unclear how often the virus is spread} via these tiny droplets, or
    aerosols, compared with larger droplets that are expelled when a
    sick person coughs or sneezes, or transmitted through contact with
    contaminated surfaces, said Linsey Marr, an aerosol expert at
    Virginia Tech. Aerosols are released even when a person without
    symptoms exhales, talks or sings, according to Dr. Marr and more
    than 200 other experts, who
    \href{https://academic.oup.com/cid/article/doi/10.1093/cid/ciaa939/5867798}{have
    outlined the evidence in an open letter to the World Health
    Organization}.
  \end{itemize}
\item ~
  \hypertarget{what-are-the-symptoms-of-coronavirus}{%
  \paragraph{What are the symptoms of
  coronavirus?}\label{what-are-the-symptoms-of-coronavirus}}

  \begin{itemize}
  \tightlist
  \item
    Common symptoms
    \href{https://www.nytimes3xbfgragh.onion/article/symptoms-coronavirus.html?action=click\&pgtype=Article\&state=default\&region=MAIN_CONTENT_3\&context=storylines_faq}{include
    fever, a dry cough, fatigue and difficulty breathing or shortness of
    breath.} Some of these symptoms overlap with those of the flu,
    making detection difficult, but runny noses and stuffy sinuses are
    less common.
    \href{https://www.nytimes3xbfgragh.onion/2020/04/27/health/coronavirus-symptoms-cdc.html?action=click\&pgtype=Article\&state=default\&region=MAIN_CONTENT_3\&context=storylines_faq}{The
    C.D.C. has also} added chills, muscle pain, sore throat, headache
    and a new loss of the sense of taste or smell as symptoms to look
    out for. Most people fall ill five to seven days after exposure, but
    symptoms may appear in as few as two days or as many as 14 days.
  \end{itemize}
\item ~
  \hypertarget{does-asymptomatic-transmission-of-covid-19-happen}{%
  \paragraph{Does asymptomatic transmission of Covid-19
  happen?}\label{does-asymptomatic-transmission-of-covid-19-happen}}

  \begin{itemize}
  \tightlist
  \item
    So far, the evidence seems to show it does. A widely cited
    \href{https://www.nature.com/articles/s41591-020-0869-5}{paper}
    published in April suggests that people are most infectious about
    two days before the onset of coronavirus symptoms and estimated that
    44 percent of new infections were a result of transmission from
    people who were not yet showing symptoms. Recently, a top expert at
    the World Health Organization stated that transmission of the
    coronavirus by people who did not have symptoms was ``very rare,''
    \href{https://www.nytimes3xbfgragh.onion/2020/06/09/world/coronavirus-updates.html?action=click\&pgtype=Article\&state=default\&region=MAIN_CONTENT_3\&context=storylines_faq\#link-1f302e21}{but
    she later walked back that statement.}
  \end{itemize}
\end{itemize}

Still, asymptomatic carriers ``are not normally major drivers of
epidemics,'' Dr. Fauci said. Most people get ill from someone they know
to be sick --- a family member, a co-worker or a patient, for example.

The virus's most vulnerable target is Africa, many experts said. More
than 1 million expatriate Chinese work there, mostly on mining, drilling
or engineering projects. Also, many Africans work and study in China and
other countries where the virus has been found.

If anyone on the continent has the virus now, ``I'm not sure the
diagnostic systems are in place to detect it,'' said Dr. Daniel Bausch,
head of scientific programs for the American Society of Tropical
Medicine and Hygiene, who is consulting with the W.H.O. on the outbreak.

South Africa and Senegal could probably diagnose it, he said. Nigeria
and some other countries have asked the W.H.O. for the genetic materials
and training they need to perform diagnostic tests, but that will take
time.

At least four African countries have suspect cases quarantined,
according to
\href{https://www.scmp.com/news/china/article/3048310/china-coronavirus-african-nations-quarantine-symptomatic-passengers}{an
article published Friday in The South China Morning Post}. They have
sent samples to France, Germany, India and South Africa for testing.

\textbf{\emph{{[}}\href{http://on.fb.me/1paTQ1h}{\emph{Like the Science
Times page on Facebook.}}} ****** \emph{\textbar{} Sign up for the}
\textbf{\href{http://nyti.ms/1MbHaRU}{\emph{Science Times
newsletter.}}\emph{{]}}}

At the moment, it seems unlikely that the virus will spread widely in
countries with vigorous, alert public health systems, said Dr. William
Schaffner, a preventive medicine specialist at Vanderbilt University
Medical Center.

``Every doctor in the U.S. has this top of mind,'' he said. ``Any
patient with fever or respiratory problems will get two questions. `Have
you been to China? Have you had contact with anyone who has?' If the
answer is yes, they'll be put in isolation right away.''

Assuming the virus spreads globally, tourism to and trade with countries
besides China may be affected --- and the urgency to find ways to halt
the virus and prevent deaths will grow.

Image

Men in protective suits greeted a plane carrying 32 Mongolian citizens
evacuated from Wuhan, China, as it arrived in
Ulaanbaatar.Credit...Byambasuren Byamba-Ochir/Agence France-Presse ---
Getty Images

It is possible that the Wuhan coronavirus will fade out as weather
warms. Many viruses, like flu, measles and norovirus, thrive in cold,
dry air. The SARS outbreak began in winter, and MERS transmission also
peaks then, though that may be related to transmission in newborn
camels.

Four mild coronaviruses cause about a quarter of the nation's common
colds, which also peak in winter.

But even if an outbreak fades in June, there could be a second wave in
the fall, as has occurred in every major flu pandemic, including those
that began in 1918 and 2009.

By that time, some remedies might be on hand, although they will need
rigorous testing and perhaps political pressure to make them available
and affordable.

In China, several
\href{https://www.sciencemag.org/news/2020/01/can-anti-hiv-combination-or-other-existing-drugs-outwit-new-coronavirus}{antiviral
drugs are being prescribed}. A common combination is pills containing
lopinavir and ritonavir with infusions of interferon, a signaling
protein that wakes up the immune system.

In the United States, the combination is sold as Kaletra by AbbVie for
H.I.V. therapy, and it is relatively expensive. In India, a dozen
generic makers produce the drugs at rock-bottom prices for use against
H.I.V. in Africa, and their products are W.H.O.-approved.

Another option may be
\href{https://www.gilead.com/news-and-press/company-statements/gilead-sciences-statement-on-the-company-ongoing-response-to-the-2019-new-coronavirus}{an
experimental drug, remdesivir}, on which the patent is held by Gilead.
The drug has not yet been approved for use against any disease.
Nonetheless, there is some evidence that it works against coronaviruses,
and Gilead has donated doses to China.

Several American companies are
\href{https://www.nytimes3xbfgragh.onion/2020/01/28/health/coronavirus-vaccine.html}{working
on a vaccine}, using various combinations of their own funds, taxpayer
money and foundation grants.

Although modern gene-chemistry techniques have made it possible to build
vaccine candidates within just days, medical ethics require that they
then be carefully tested on animals and small numbers of healthy humans
for safety and effectiveness.

That aspect of the process cannot be sped up, because dangerous side
effects may take time to appear and because human immune systems need
time to produce the antibodies that show whether a vaccine is working.

Whether or not what is being tried in China will be acceptable elsewhere
will depend on how rigorously Chinese doctors run their clinical trials.

``In God we trust,'' Dr. Schaffner said. ``All others must provide
data.''

Advertisement

\protect\hyperlink{after-bottom}{Continue reading the main story}

\hypertarget{site-index}{%
\subsection{Site Index}\label{site-index}}

\hypertarget{site-information-navigation}{%
\subsection{Site Information
Navigation}\label{site-information-navigation}}

\begin{itemize}
\tightlist
\item
  \href{https://help.nytimes3xbfgragh.onion/hc/en-us/articles/115014792127-Copyright-notice}{©~2020~The
  New York Times Company}
\end{itemize}

\begin{itemize}
\tightlist
\item
  \href{https://www.nytco.com/}{NYTCo}
\item
  \href{https://help.nytimes3xbfgragh.onion/hc/en-us/articles/115015385887-Contact-Us}{Contact
  Us}
\item
  \href{https://www.nytco.com/careers/}{Work with us}
\item
  \href{https://nytmediakit.com/}{Advertise}
\item
  \href{http://www.tbrandstudio.com/}{T Brand Studio}
\item
  \href{https://www.nytimes3xbfgragh.onion/privacy/cookie-policy\#how-do-i-manage-trackers}{Your
  Ad Choices}
\item
  \href{https://www.nytimes3xbfgragh.onion/privacy}{Privacy}
\item
  \href{https://help.nytimes3xbfgragh.onion/hc/en-us/articles/115014893428-Terms-of-service}{Terms
  of Service}
\item
  \href{https://help.nytimes3xbfgragh.onion/hc/en-us/articles/115014893968-Terms-of-sale}{Terms
  of Sale}
\item
  \href{https://spiderbites.nytimes3xbfgragh.onion}{Site Map}
\item
  \href{https://help.nytimes3xbfgragh.onion/hc/en-us}{Help}
\item
  \href{https://www.nytimes3xbfgragh.onion/subscription?campaignId=37WXW}{Subscriptions}
\end{itemize}
