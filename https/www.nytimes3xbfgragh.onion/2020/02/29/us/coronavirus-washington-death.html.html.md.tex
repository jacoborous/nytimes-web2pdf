Sections

SEARCH

\protect\hyperlink{site-content}{Skip to
content}\protect\hyperlink{site-index}{Skip to site index}

\href{https://www.nytimes3xbfgragh.onion/section/us}{U.S.}

\href{https://myaccount.nytimes3xbfgragh.onion/auth/login?response_type=cookie\&client_id=vi}{}

\href{https://www.nytimes3xbfgragh.onion/section/todayspaper}{Today's
Paper}

\href{/section/us}{U.S.}\textbar{}Washington State Declares Emergency
Amid Coronavirus Death and Illnesses at Nursing Home

\url{https://nyti.ms/2TlM1q0}

\begin{itemize}
\item
\item
\item
\item
\item
\end{itemize}

\hypertarget{the-coronavirus-outbreak}{%
\subsubsection{\texorpdfstring{\href{https://www.nytimes3xbfgragh.onion/news-event/coronavirus?name=styln-coronavirus-national\&region=TOP_BANNER\&variant=undefined\&block=storyline_menu_recirc\&action=click\&pgtype=Article\&impression_id=12f52010-e398-11ea-950f-95281846e31d}{The
Coronavirus
Outbreak}}{The Coronavirus Outbreak}}\label{the-coronavirus-outbreak}}

\begin{itemize}
\tightlist
\item
  live\href{https://www.nytimes3xbfgragh.onion/2020/08/21/world/covid-19-coronavirus.html?name=styln-coronavirus-national\&region=TOP_BANNER\&variant=undefined\&block=storyline_menu_recirc\&action=click\&pgtype=Article\&impression_id=12f52011-e398-11ea-950f-95281846e31d}{Latest
  Updates}
\item
  \href{https://www.nytimes3xbfgragh.onion/interactive/2020/us/coronavirus-us-cases.html?name=styln-coronavirus-national\&region=TOP_BANNER\&variant=undefined\&block=storyline_menu_recirc\&action=click\&pgtype=Article\&impression_id=12f52012-e398-11ea-950f-95281846e31d}{Maps
  and Cases}
\item
  \href{https://www.nytimes3xbfgragh.onion/interactive/2020/science/coronavirus-vaccine-tracker.html?name=styln-coronavirus-national\&region=TOP_BANNER\&variant=undefined\&block=storyline_menu_recirc\&action=click\&pgtype=Article\&impression_id=12f52013-e398-11ea-950f-95281846e31d}{Vaccine
  Tracker}
\item
  \href{https://www.nytimes3xbfgragh.onion/2020/08/19/us/colleges-closing-covid.html?name=styln-coronavirus-national\&region=TOP_BANNER\&variant=undefined\&block=storyline_menu_recirc\&action=click\&pgtype=Article\&impression_id=12f54720-e398-11ea-950f-95281846e31d}{Colleges
  Closing}
\item
  \href{https://www.nytimes3xbfgragh.onion/live/2020/08/20/business/stock-market-today-coronavirus?name=styln-coronavirus-national\&region=TOP_BANNER\&variant=undefined\&block=storyline_menu_recirc\&action=click\&pgtype=Article\&impression_id=12f54721-e398-11ea-950f-95281846e31d}{Economy}
\end{itemize}

Advertisement

\protect\hyperlink{after-top}{Continue reading the main story}

Supported by

\protect\hyperlink{after-sponsor}{Continue reading the main story}

\hypertarget{washington-state-declares-emergency-amid-coronavirus-death-and-illnesses-at-nursing-home}{%
\section{Washington State Declares Emergency Amid Coronavirus Death and
Illnesses at Nursing
Home}\label{washington-state-declares-emergency-amid-coronavirus-death-and-illnesses-at-nursing-home}}

A person in the Seattle area died. Two others tested positive for the
virus in a long term care center where dozens of people have reported
feeling ill.

\includegraphics{https://static01.graylady3jvrrxbe.onion/images/2020/02/29/us/29virus-northwest/merlin_169802310_af06ddc8-3884-4a37-aac6-1170130d0332-articleLarge.jpg?quality=75\&auto=webp\&disable=upscale}

By \href{https://www.nytimes3xbfgragh.onion/by/mike-baker}{Mike Baker},
\href{https://www.nytimes3xbfgragh.onion/by/nicholas-bogel-burroughs}{Nicholas
Bogel-Burroughs} and
\href{https://www.nytimes3xbfgragh.onion/by/karen-weise}{Karen Weise}

\begin{itemize}
\item
  Feb. 29, 2020
\item
  \begin{itemize}
  \item
  \item
  \item
  \item
  \item
  \end{itemize}
\end{itemize}

SEATTLE --- Concerns about the coronavirus intensified across the
Pacific Northwest on Saturday, after a person from the Seattle area died
and as two new cases emerged inside a nursing care center in Kirkland,
Wash., where dozens of other people were reported to be feeling sick.

Gov. Jay Inslee of Washington declared a state of emergency, and said
officials were considering canceling sporting events, closing schools
and taking any other steps needed to slow the spread of the virus. ``If
--- and this is a big if --- there is a social distancing strategy that
becomes necessary, the emergency declaration would give us some legal
authority,'' Mr. Inslee said.

At two schools that have had ties to cases, in Oregon and in Washington,
officials announced that they were shuttering buildings for several
days, and ordering deep cleanings.

And in Kirkland, where a health care worker in her 40s at the Life Care
Center, a long-term nursing home, and a resident of the center, in her
70s, were reported on Saturday to have tested positive for the virus,
health officials expressed alarm at the possibility of more cases. Among
288 residents and workers at Life Care, more than 50 people --- 25
health workers and 27 residents --- have shown symptoms of respiratory
illness or have been hospitalized for pneumonia, local health officials
said.

``We are very concerned about an outbreak in a setting where there are
many older people,'' said Dr. Jeff Duchin, the health officer for Public
Health in Seattle and King County. A team of federal health workers was
dispatched to Kirkland to assist local workers, and many more
coronavirus tests were expected to be conducted in the coming days.

Older people are much more likely to face serious illness if infected
with the coronavirus. They are also much more likely to die. An
\href{https://jamanetwork.com/journals/jama/fullarticle/2762130}{analysis
of Chinese patients} found that nearly 15 percent of infected people
over age 80 died; of those in their 70s, 8 percent died.

\hypertarget{latest-updates-the-coronavirus-outbreak}{%
\section{\texorpdfstring{\href{https://www.nytimes3xbfgragh.onion/2020/08/21/world/covid-19-coronavirus.html?action=click\&pgtype=Article\&state=default\&region=MAIN_CONTENT_1\&context=storylines_live_updates}{Latest
Updates: The Coronavirus
Outbreak}}{Latest Updates: The Coronavirus Outbreak}}\label{latest-updates-the-coronavirus-outbreak}}

Updated 2020-08-21T10:13:38.790Z

\begin{itemize}
\tightlist
\item
  \href{https://www.nytimes3xbfgragh.onion/2020/08/21/world/covid-19-coronavirus.html?action=click\&pgtype=Article\&state=default\&region=MAIN_CONTENT_1\&context=storylines_live_updates\#link-4690b6aa}{Shutdowns,
  warnings and scoldings follow gatherings on college campuses.}
\item
  \href{https://www.nytimes3xbfgragh.onion/2020/08/21/world/covid-19-coronavirus.html?action=click\&pgtype=Article\&state=default\&region=MAIN_CONTENT_1\&context=storylines_live_updates\#link-324af071}{As
  he accepts the Democratic nomination, Biden knocks Trump's pandemic
  response.}
\item
  \href{https://www.nytimes3xbfgragh.onion/2020/08/21/world/covid-19-coronavirus.html?action=click\&pgtype=Article\&state=default\&region=MAIN_CONTENT_1\&context=storylines_live_updates\#link-35890b73}{Hundreds
  of doctors in Kenya go on strike over their pay and protective gear.}
\end{itemize}

\href{https://www.nytimes3xbfgragh.onion/2020/08/21/world/covid-19-coronavirus.html?action=click\&pgtype=Article\&state=default\&region=MAIN_CONTENT_1\&context=storylines_live_updates}{See
more updates}

More live coverage:
\href{https://www.nytimes3xbfgragh.onion/live/2020/08/20/business/stock-market-today-coronavirus?action=click\&pgtype=Article\&state=default\&region=MAIN_CONTENT_1\&context=storylines_live_updates}{Markets}

On Saturday afternoon, workers in plastic protective gear and masks
could be seen rolling a patient, also in a mask, to an ambulance outside
the nursing home, a low slung building on a side street tucked among
small condo complexes, and surrounded by pine trees. Officials said that
testing for the virus was being conducted at an area hospital, and that
it was not open for visits from patients' family members or vendors as a
precaution.

Nancy Butner, the northwest divisional vice president for Life Care
Centers of America, said that many of the residents were showing only
respiratory symptoms that were not necessarily tied to the coronavirus.
In an interview, she said that residents were mostly staying in their
rooms, and that despite public health officials' warnings of a possible
``outbreak'' at the nursing home, the mood inside was relaxed.

``We are encouraging people to remain in their rooms,'' she said. ``We
have the equipment and supplies to take care of them, and people are
doing what they need to do.''

Chad Bergevin, who lives opposite the center, said he had learned of the
situation in a text message from a neighbor. ``It was like, `Wow, this
is literally less than a football field away from my house,''' he said,
adding that he was surprised to see people still seeming to come and go
near the center. ``I'm sorry, if it were me, I'd have the place on
lockdown,'' he said.

The indications of a possible spread, and the involvement of a nursing
home marked a new, urgent phase in the response to the virus in the
United States, where 70 cases have been reported, and until Saturday,
none had been fatal. Most of the cases could be explained by overseas
travel or contact with someone who had been ill. This week, though, new
cases,
\href{https://www.nytimes3xbfgragh.onion/2020/02/28/us/coronavirus-solano-county.html}{in
California, Oregon and Washington}, were the first in the United States
in which the cause was mysterious and unknown --- a sign, experts
warned, that the virus, which has killed more than 2,800 people
worldwide and has sickened more than 86,000 others, might now be
spreading in the United States.

In the Northwest, especially, health officials were putting in place new
precautions given the new cases. They were already discussing the
possibility that they might recommend cancellations of public events.
They began warning that life in the coming weeks may undergo dramatic
change.

By Saturday, 10 people have been treated in Washington State, including
the first case of coronavirus that was diagnosed in the United States, a
man in his 30s who had traveled in China and has since recovered;
several patients were treated at a Spokane area hospital after returning
from a cruise ship in Japan; and the first known coronavirus fatality in
the United States, a man in his 50s whose death was announced Saturday.

In announcing the death at a news conference, President Trump said the
victim was a ``wonderful woman'' in her 50s, but local officials later
said the patient had been a man in his 50s. The Centers for Disease
Control and Prevention later said that the patient was, in fact, a man,
and that the agency had incorrectly described the patient as a woman.

Few details were known about the man who died, except that he had
underlying health conditions and had been a patient at a hospital in
Kirkland. He was not known to have traveled abroad, or to have had
contact with anyone who had tested positive for the virus, adding to
\href{https://www.nytimes3xbfgragh.onion/2020/02/28/us/coronavirus-solano-county.html}{growing
signs that the coronavirus may be spreading} in the United States. He
also had no known connection to the nursing home, officials said.

The new cases added to the fears of some residents. Noelle Salazar, an
author in Bothell, Wash., was recovering from a surgery, and realized
that she had been in the same hospital as the man who died.

\href{https://www.nytimes3xbfgragh.onion/news-event/coronavirus?action=click\&pgtype=Article\&state=default\&region=MAIN_CONTENT_3\&context=storylines_faq}{}

\hypertarget{the-coronavirus-outbreak-}{%
\subsubsection{The Coronavirus Outbreak
›}\label{the-coronavirus-outbreak-}}

\hypertarget{frequently-asked-questions}{%
\paragraph{Frequently Asked
Questions}\label{frequently-asked-questions}}

Updated August 17, 2020

\begin{itemize}
\item ~
  \hypertarget{why-does-standing-six-feet-away-from-others-help}{%
  \paragraph{Why does standing six feet away from others
  help?}\label{why-does-standing-six-feet-away-from-others-help}}

  \begin{itemize}
  \tightlist
  \item
    The coronavirus spreads primarily through droplets from your mouth
    and nose, especially when you cough or sneeze. The C.D.C., one of
    the organizations using that measure,
    \href{https://www.nytimes3xbfgragh.onion/2020/04/14/health/coronavirus-six-feet.html?action=click\&pgtype=Article\&state=default\&region=MAIN_CONTENT_3\&context=storylines_faq}{bases
    its recommendation of six feet} on the idea that most large droplets
    that people expel when they cough or sneeze will fall to the ground
    within six feet. But six feet has never been a magic number that
    guarantees complete protection. Sneezes, for instance, can launch
    droplets a lot farther than six feet,
    \href{https://jamanetwork.com/journals/jama/fullarticle/2763852}{according
    to a recent study}. It's a rule of thumb: You should be safest
    standing six feet apart outside, especially when it's windy. But
    keep a mask on at all times, even when you think you're far enough
    apart.
  \end{itemize}
\item ~
  \hypertarget{i-have-antibodies-am-i-now-immune}{%
  \paragraph{I have antibodies. Am I now
  immune?}\label{i-have-antibodies-am-i-now-immune}}

  \begin{itemize}
  \tightlist
  \item
    As of right
    now,\href{https://www.nytimes3xbfgragh.onion/2020/07/22/health/covid-antibodies-herd-immunity.html?action=click\&pgtype=Article\&state=default\&region=MAIN_CONTENT_3\&context=storylines_faq}{that
    seems likely, for at least several months.} There have been
    frightening accounts of people suffering what seems to be a second
    bout of Covid-19. But experts say these patients may have a
    drawn-out course of infection, with the virus taking a slow toll
    weeks to months after initial exposure. People infected with the
    coronavirus typically
    \href{https://www.nature.com/articles/s41586-020-2456-9}{produce}
    immune molecules called antibodies, which are
    \href{https://www.nytimes3xbfgragh.onion/2020/05/07/health/coronavirus-antibody-prevalence.html?action=click\&pgtype=Article\&state=default\&region=MAIN_CONTENT_3\&context=storylines_faq}{protective
    proteins made in response to an
    infection}\href{https://www.nytimes3xbfgragh.onion/2020/05/07/health/coronavirus-antibody-prevalence.html?action=click\&pgtype=Article\&state=default\&region=MAIN_CONTENT_3\&context=storylines_faq}{.
    These antibodies may} last in the body
    \href{https://www.nature.com/articles/s41591-020-0965-6}{only two to
    three months}, which may seem worrisome, but that's perfectly normal
    after an acute infection subsides, said Dr. Michael Mina, an
    immunologist at Harvard University. It may be possible to get the
    coronavirus again, but it's highly unlikely that it would be
    possible in a short window of time from initial infection or make
    people sicker the second time.
  \end{itemize}
\item ~
  \hypertarget{im-a-small-business-owner-can-i-get-relief}{%
  \paragraph{I'm a small-business owner. Can I get
  relief?}\label{im-a-small-business-owner-can-i-get-relief}}

  \begin{itemize}
  \tightlist
  \item
    The
    \href{https://www.nytimes3xbfgragh.onion/article/small-business-loans-stimulus-grants-freelancers-coronavirus.html?action=click\&pgtype=Article\&state=default\&region=MAIN_CONTENT_3\&context=storylines_faq}{stimulus
    bills enacted in March} offer help for the millions of American
    small businesses. Those eligible for aid are businesses and
    nonprofit organizations with fewer than 500 workers, including sole
    proprietorships, independent contractors and freelancers. Some
    larger companies in some industries are also eligible. The help
    being offered, which is being managed by the Small Business
    Administration, includes the Paycheck Protection Program and the
    Economic Injury Disaster Loan program. But lots of folks have
    \href{https://www.nytimes3xbfgragh.onion/interactive/2020/05/07/business/small-business-loans-coronavirus.html?action=click\&pgtype=Article\&state=default\&region=MAIN_CONTENT_3\&context=storylines_faq}{not
    yet seen payouts.} Even those who have received help are confused:
    The rules are draconian, and some are stuck sitting on
    \href{https://www.nytimes3xbfgragh.onion/2020/05/02/business/economy/loans-coronavirus-small-business.html?action=click\&pgtype=Article\&state=default\&region=MAIN_CONTENT_3\&context=storylines_faq}{money
    they don't know how to use.} Many small-business owners are getting
    less than they expected or
    \href{https://www.nytimes3xbfgragh.onion/2020/06/10/business/Small-business-loans-ppp.html?action=click\&pgtype=Article\&state=default\&region=MAIN_CONTENT_3\&context=storylines_faq}{not
    hearing anything at all.}
  \end{itemize}
\item ~
  \hypertarget{what-are-my-rights-if-i-am-worried-about-going-back-to-work}{%
  \paragraph{What are my rights if I am worried about going back to
  work?}\label{what-are-my-rights-if-i-am-worried-about-going-back-to-work}}

  \begin{itemize}
  \tightlist
  \item
    Employers have to provide
    \href{https://www.osha.gov/SLTC/covid-19/standards.html}{a safe
    workplace} with policies that protect everyone equally.
    \href{https://www.nytimes3xbfgragh.onion/article/coronavirus-money-unemployment.html?action=click\&pgtype=Article\&state=default\&region=MAIN_CONTENT_3\&context=storylines_faq}{And
    if one of your co-workers tests positive for the coronavirus, the
    C.D.C.} has said that
    \href{https://www.cdc.gov/coronavirus/2019-ncov/community/guidance-business-response.html}{employers
    should tell their employees} -\/- without giving you the sick
    employee's name -\/- that they may have been exposed to the virus.
  \end{itemize}
\item ~
  \hypertarget{what-is-school-going-to-look-like-in-september}{%
  \paragraph{What is school going to look like in
  September?}\label{what-is-school-going-to-look-like-in-september}}

  \begin{itemize}
  \tightlist
  \item
    It is unlikely that many schools will return to a normal schedule
    this fall, requiring the grind of
    \href{https://www.nytimes3xbfgragh.onion/2020/06/05/us/coronavirus-education-lost-learning.html?action=click\&pgtype=Article\&state=default\&region=MAIN_CONTENT_3\&context=storylines_faq}{online
    learning},
    \href{https://www.nytimes3xbfgragh.onion/2020/05/29/us/coronavirus-child-care-centers.html?action=click\&pgtype=Article\&state=default\&region=MAIN_CONTENT_3\&context=storylines_faq}{makeshift
    child care} and
    \href{https://www.nytimes3xbfgragh.onion/2020/06/03/business/economy/coronavirus-working-women.html?action=click\&pgtype=Article\&state=default\&region=MAIN_CONTENT_3\&context=storylines_faq}{stunted
    workdays} to continue. California's two largest public school
    districts --- Los Angeles and San Diego --- said on July 13, that
    \href{https://www.nytimes3xbfgragh.onion/2020/07/13/us/lausd-san-diego-school-reopening.html?action=click\&pgtype=Article\&state=default\&region=MAIN_CONTENT_3\&context=storylines_faq}{instruction
    will be remote-only in the fall}, citing concerns that surging
    coronavirus infections in their areas pose too dire a risk for
    students and teachers. Together, the two districts enroll some
    825,000 students. They are the largest in the country so far to
    abandon plans for even a partial physical return to classrooms when
    they reopen in August. For other districts, the solution won't be an
    all-or-nothing approach.
    \href{https://bioethics.jhu.edu/research-and-outreach/projects/eschool-initiative/school-policy-tracker/}{Many
    systems}, including the nation's largest, New York City, are
    devising
    \href{https://www.nytimes3xbfgragh.onion/2020/06/26/us/coronavirus-schools-reopen-fall.html?action=click\&pgtype=Article\&state=default\&region=MAIN_CONTENT_3\&context=storylines_faq}{hybrid
    plans} that involve spending some days in classrooms and other days
    online. There's no national policy on this yet, so check with your
    municipal school system regularly to see what is happening in your
    community.
  \end{itemize}
\end{itemize}

``We weren't in the same section, but it's not comforting,'' she said
while recuperating at home. ``I'm a little on edge right now for sure.''

Like many Americans, she has begun to take extra precautions in recent
days: She wiped down her shopping cart at a grocery store for the first
time, bought extra vitamins and nonperishable food and canceled a
Pilates class to avoid getting too close to others.

Around the region, it was clear that residents were bracing. At a
big-box store north of Seattle on early Saturday, checkout lines were
unusually long, snaking down aisles with carts loaded with all sorts of
supplies.

In Oregon, a state that until Friday had not reported any cases of the
coronavirus, officials say an employee of Forest Hills Elementary School
in Lake Oswego, a suburb of Portland, appeared to have contracted the
virus more than a week ago. The school would be closed for several days,
and was being cleaned, but parents said they were uncertain and scared.

Gov. Kate Brown of Oregon said that she expected more cases, and that
her state might take more aggressive action if the outbreak got more
severe. But, in the meantime, she said people did not need to take
drastic action.

``I'm wanting to convey to Oregonians, and frankly folks on the entire
West Coast: stay calm, continue on your daily lives and follow public
health precautions,'' Ms. Brown said.

Dr. Dean Sidelinger, Oregon's state health officer, said a broader
closure of schools was an option the state could pursue at some point.
``If we do notice spread in our community or multiple cases, that is
certainly something we would consider on a case-by-case basis,'' Dr.
Sidelinger said.

Back at the school in Lake Oswego, parents were weighing how to go
forward.

``I really don't know how to process it at this point,'' said Danielle
Gaustad, a mother of three children, ages 3, 5 and 18. Her 5-year-old,
who attends Forest Hills Elementary, had pneumonia several weeks ago,
and her 3-year-old has severe asthma. ``When people don't understand an
illness, and clearly no one understands coronavirus at this point,
everybody gets scared,'' Ms. Gaustad said.

Though the school has said it intends to reopen in a few days, Ms.
Gaustad said she would not allow her children to go back to school this
week. ``I don't know when I will, honestly,'' she said. ``It's scary.''

Mike Baker reported from Seattle, Nicholas Bogel-Burroughs from New York
and Karen Weise from Kirkland, Wash. Knvul Sheikh contributed reporting
from New York, Claire Cain Miller from Lake Oswego, Ore., and Mitch
Smith from Chicago.

Advertisement

\protect\hyperlink{after-bottom}{Continue reading the main story}

\hypertarget{site-index}{%
\subsection{Site Index}\label{site-index}}

\hypertarget{site-information-navigation}{%
\subsection{Site Information
Navigation}\label{site-information-navigation}}

\begin{itemize}
\tightlist
\item
  \href{https://help.nytimes3xbfgragh.onion/hc/en-us/articles/115014792127-Copyright-notice}{©~2020~The
  New York Times Company}
\end{itemize}

\begin{itemize}
\tightlist
\item
  \href{https://www.nytco.com/}{NYTCo}
\item
  \href{https://help.nytimes3xbfgragh.onion/hc/en-us/articles/115015385887-Contact-Us}{Contact
  Us}
\item
  \href{https://www.nytco.com/careers/}{Work with us}
\item
  \href{https://nytmediakit.com/}{Advertise}
\item
  \href{http://www.tbrandstudio.com/}{T Brand Studio}
\item
  \href{https://www.nytimes3xbfgragh.onion/privacy/cookie-policy\#how-do-i-manage-trackers}{Your
  Ad Choices}
\item
  \href{https://www.nytimes3xbfgragh.onion/privacy}{Privacy}
\item
  \href{https://help.nytimes3xbfgragh.onion/hc/en-us/articles/115014893428-Terms-of-service}{Terms
  of Service}
\item
  \href{https://help.nytimes3xbfgragh.onion/hc/en-us/articles/115014893968-Terms-of-sale}{Terms
  of Sale}
\item
  \href{https://spiderbites.nytimes3xbfgragh.onion}{Site Map}
\item
  \href{https://help.nytimes3xbfgragh.onion/hc/en-us}{Help}
\item
  \href{https://www.nytimes3xbfgragh.onion/subscription?campaignId=37WXW}{Subscriptions}
\end{itemize}
