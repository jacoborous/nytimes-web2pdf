Sections

SEARCH

\protect\hyperlink{site-content}{Skip to
content}\protect\hyperlink{site-index}{Skip to site index}

\href{https://www.nytimes3xbfgragh.onion/section/books/review}{Book
Review}

\href{https://myaccount.nytimes3xbfgragh.onion/auth/login?response_type=cookie\&client_id=vi}{}

\href{https://www.nytimes3xbfgragh.onion/section/todayspaper}{Today's
Paper}

\href{/section/books/review}{Book Review}\textbar{}Pop-Up Brothels,
Severed Tongues and Creepy Nursery Rhymes

\url{https://nyti.ms/329XTiI}

\begin{itemize}
\item
\item
\item
\item
\item
\end{itemize}

Advertisement

\protect\hyperlink{after-top}{Continue reading the main story}

Supported by

\protect\hyperlink{after-sponsor}{Continue reading the main story}

\href{/column/crime}{Crime}

\hypertarget{pop-up-brothels-severed-tongues-and-creepy-nursery-rhymes}{%
\section{Pop-Up Brothels, Severed Tongues and Creepy Nursery
Rhymes}\label{pop-up-brothels-severed-tongues-and-creepy-nursery-rhymes}}

\includegraphics{https://static01.graylady3jvrrxbe.onion/images/2020/02/23/books/review/23Crime/23Crime-articleLarge.jpg?quality=75\&auto=webp\&disable=upscale}

By Marilyn Stasio

\begin{itemize}
\item
  Feb. 20, 2020
\item
  \begin{itemize}
  \item
  \item
  \item
  \item
  \item
  \end{itemize}
\end{itemize}

\emph{Editor's note: We're thrilled that our longtime columnist, Marilyn
Stasio --- who's been recovering from a traffic accident --- is finally
back in our pages with her crisp, sharp, tartly funny takes on the
latest crime novels. We've missed her, and we know you have, too.}

Yuck! That's the English word that unavoidably springs to mind when
handling a novel by Lars Kepler, the pen name of the Swedish
husband-and-wife writing team of Alexandra Coelho Ahndoril and Alexander
Ahndoril. \textbf{THE RABBIT HUNTER (Knopf, 512 pp., \$27.95)} is a
graphic example of their stomach-churning style. Let's put it this way:
It's not the killings that are disturbing --- it's what happens to the
body parts.

The translation by Neil Smith adheres to the authors' sanguinary style
with descriptive accounts of subsequent murders by a killer or killers
who seem to have it in for politicians. Given the high rank of the
targeted victims, the case falls to Saga Bauer, a Security Police
officer with a specialty in counterterrorism and a penchant for showing
up at crime scenes dressed in a black leather bodysuit.

Kepler injects the sweet voice of a child chanting a nursery rhyme about
bunnies before each execution-style death. (``Ten little rabbits, all
dressed in white / Tried to get to heaven on the end of a kite.
\ldots{}'') But that's more a grace note than a plot point in a story
that hops from political terrorism to psycho-killer suspense in a
heartbeat. No fluffy tales here.

Whatever will they come up with next? In \textbf{MANY RIVERS TO CROSS
(Morrow, 377 pp., \$28.99),} Peter Robinson's new mystery featuring his
simpatico police detective, Alan Banks, it's ``pop-up brothels.'' These
floating escort agencies materialize out of nowhere to service patrons
who trawl the dark web to find them --- only to quietly fold their tents
and disappear once the police get wind of them.

``They can be quite sophisticated,'' observes Banks, ever the master of
British reserve and understatement. But these enterprises are part of a
pervasive flesh-peddling racket that victimizes young women who are
lured to England with promises of respectable jobs, only to be tricked
out as prostitutes. That old story is given a new twist when Zelda, one
of these involuntary recruits, turns out to be a ``super-recognizer,''
an individual gifted (or cursed) with extraordinary abilities to place a
face. As always, Robinson approaches his characters with immense
compassion. But it's Zelda's uncanny skills, not her vulnerable
humanity, that are of intense interest to the police --- and even more
so to the criminals who hold her life in their hands.

We all need other people, but some of us sadly find ourselves in need of
suspiciously helpful strangers --- call them Samaritans --- like the
ones we meet in C. J. Tudor's eerie thriller, \textbf{THE OTHER PEOPLE
(Ballantine, 324 pp., \$27)}.

Gabe finds himself accepting the dubious assistance of such a shadowy
syndicate following the disappearance of his 5-year-old daughter, Izzy.
Driving home from work one evening, Gabe is stunned to see a child who
looks exactly like Izzy in the back seat of a car that passes him on the
highway. It must be an illusion, he tells himself --- surely his
daughter is at home with his wife --- until the little girl looks
directly at him and mouths the word: ``Daddy!'' Unnervingly, upon his
arrival home, he finds the police on his doorstep.

Perhaps worse than the finality of death, ``missing is limbo,'' he
reflects, stung by the pain of studying old photographs of absent loved
ones, ``their hairstyles becoming more dated, their smiles more frozen
with each missed birthday and Christmas.''

Fast forward three years: The car Gabe saw on the highway is pulled from
a lake, a rotted body in the trunk, and Gabe is thrown back into the old
nightmare. Better make that a fresh nightmare, because at the hands of
Tudor, the real pain is yet to come.

``I recognize these ears.''

Now, there's a laugh line for you. But Brynn Callahan, the gutsy heroine
of Susan Furlong's latest mystery, \textbf{SHATTERED JUSTICE
(Kensington, 293 pp., \$26),} doesn't crack a smile when she says it. A
sheriff's deputy in a remote Tennessee community known as Bone Gap,
Brynn had seen those severed ears --- ``bloodstained, blue-tinged flesh,
strung up to dry like anemic chili peppers'' --- just the night before,
at a local bar. In a healthier state and sporting a silver horseshoe
stud, they had belonged to a male stripper who had danced at a hen party
and staggered off at the end of the night with one of the guests. This
morning, they turned up at a children's playground, displayed under the
crude scribble: ``Hear no evil.''

What's next? A severed tongue, accompanied by a warning to ``Speak no
evil''? Well, yes, because that's the kind of humor that gets grim
laughs in this gritty series, a real find, if you ask me. The thickly
forested setting is gorgeous, once you look past the armed militia
encampments pitched in the woods. And the locals are just quirky enough
to make you forget they can also be dangerous. But the sturdy wildflower
in this treacherous terrain is Brynn, who lives with a dog named Wilco,
``once the best damn HRD (human remains detection) dog in the entire
Middle Eastern conflict.'' The question is, are these two veterans tough
enough to survive on the home front?

Advertisement

\protect\hyperlink{after-bottom}{Continue reading the main story}

\hypertarget{site-index}{%
\subsection{Site Index}\label{site-index}}

\hypertarget{site-information-navigation}{%
\subsection{Site Information
Navigation}\label{site-information-navigation}}

\begin{itemize}
\tightlist
\item
  \href{https://help.nytimes3xbfgragh.onion/hc/en-us/articles/115014792127-Copyright-notice}{©~2020~The
  New York Times Company}
\end{itemize}

\begin{itemize}
\tightlist
\item
  \href{https://www.nytco.com/}{NYTCo}
\item
  \href{https://help.nytimes3xbfgragh.onion/hc/en-us/articles/115015385887-Contact-Us}{Contact
  Us}
\item
  \href{https://www.nytco.com/careers/}{Work with us}
\item
  \href{https://nytmediakit.com/}{Advertise}
\item
  \href{http://www.tbrandstudio.com/}{T Brand Studio}
\item
  \href{https://www.nytimes3xbfgragh.onion/privacy/cookie-policy\#how-do-i-manage-trackers}{Your
  Ad Choices}
\item
  \href{https://www.nytimes3xbfgragh.onion/privacy}{Privacy}
\item
  \href{https://help.nytimes3xbfgragh.onion/hc/en-us/articles/115014893428-Terms-of-service}{Terms
  of Service}
\item
  \href{https://help.nytimes3xbfgragh.onion/hc/en-us/articles/115014893968-Terms-of-sale}{Terms
  of Sale}
\item
  \href{https://spiderbites.nytimes3xbfgragh.onion}{Site Map}
\item
  \href{https://help.nytimes3xbfgragh.onion/hc/en-us}{Help}
\item
  \href{https://www.nytimes3xbfgragh.onion/subscription?campaignId=37WXW}{Subscriptions}
\end{itemize}
