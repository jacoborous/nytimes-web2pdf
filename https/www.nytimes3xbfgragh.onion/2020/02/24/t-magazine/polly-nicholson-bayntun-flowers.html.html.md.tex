The Tulip Revivalist

\url{https://nyti.ms/37PGEV7}

\begin{itemize}
\item
\item
\item
\item
\item
\item
\end{itemize}

\includegraphics{https://static01.graylady3jvrrxbe.onion/images/2020/02/24/t-magazine/24tmag-bayntun-slide-84LE/24tmag-bayntun-slide-84LE-articleLarge.jpg?quality=75\&auto=webp\&disable=upscale}

Sections

\protect\hyperlink{site-content}{Skip to
content}\protect\hyperlink{site-index}{Skip to site index}

\hypertarget{the-tulip-revivalist}{%
\section{The Tulip Revivalist}\label{the-tulip-revivalist}}

On her estate in the English countryside, one horticultural historian is
cultivating a small empire of almost extinct varieties that once bloomed
centuries ago.

The Victorian conservatory attached to the main house on Polly
Nicholson's estate remains frost-free year-round, allowing her to grow
plants like succulents from Cornwall and the Isles of
Scilly.Credit...Simon Upton

Supported by

\protect\hyperlink{after-sponsor}{Continue reading the main story}

By \href{https://www.nytimes3xbfgragh.onion/by/nancy-hass}{Nancy Hass}

\begin{itemize}
\item
  Feb. 24, 2020
\item
  \begin{itemize}
  \item
  \item
  \item
  \item
  \item
  \item
  \end{itemize}
\end{itemize}

THE FLORAL DESIGNER and organic grower
\href{https://www.bayntunflowers.co.uk/pages/polly-nicholson}{Polly
Nicholson} understands why the tulip rarely gets its due. Despite its
simple shape, it is the most freighted of blooms. That may be why the
genus Tulipa, of all the flowers she grows, consumes her.

Unlike the rose, which is associated with all things sweet --- romantic
love, a pleasant perfume --- the tulip has \emph{baggage}. In
17th-century
\href{https://www.nytimes3xbfgragh.onion/2007/05/13/travel/09tulipswebonly2.html}{Holland},
it was responsible for the frenzy called tulipomania, which drove bulb
prices to absurd levels and is now shorthand for ruinous economic
bubbles. It is also associated with the corner deli or grocery store,
where these days tulips stand as unglamorous commoners, identical
bunches of primary reds and yellows shipped from stadium-size fields in
the Netherlands, bound with rubber bands and jammed into green plastic
buckets.

What people forget, says Nicholson, as she walks along the paths of
Blacklands, her 120-acre estate in Wiltshire, England, a 90-minute drive
west of London, is that tulips were an aesthetic fixation long before
tulipomania --- and remained so long after. Starting in the early 16th
century, when garden styles changed to favor mostly green landscapes,
tulips, with their vast array of shapes, hues, sizes and markings ---
candy-cane swirls, flames of aubergine and scarlet, ivory double blooms
edged in fuchsia --- were the object of unparalleled desire throughout
much of Europe. Whether solid-color varieties planted in borders with
hyacinths and white narcissus by Louis XIV in the gardens surrounding
the Grand Trianon at Versailles or the one-off, elaborately streaked
sorts husbanded by members of the more than 100 tulip societies that
flourished in England from the 17th to mid-19th century, tulips were
long acknowledged as the most poetic and mysterious of blooms.
``Consider it,'' Nicholson says, as we stand in the shadow of a clipped
copper beech hedge, trumpet-shaped Ivory Floradale and Doll's Minuet, a
magenta bloom streaked with green --- nodding on whippetlike stems in
the distance: ``We can always recognize a tulip, even though they can be
so different, like a lion from a tiger. No other flower is quite so
deeply coded in our awareness.''

\includegraphics{https://static01.graylady3jvrrxbe.onion/images/2020/02/24/t-magazine/24tmag-bayntun-slide-YXLU/24tmag-bayntun-slide-YXLU-articleLarge.jpg?quality=75\&auto=webp\&disable=upscale}

Image

Nicholson does much of her flower arranging in the coach house. She also
uses the space to dry flowers such as Allium christophii and schubertii
for the wreaths she makes and sells.Credit...Simon Upton

Image

A junk-shop vase overflows with forget-me-nots in what was formerly the
estate's coach house.Credit...Simon Upton

Nicholson, 50, is on a mission to return the tulip to its rightful place
in history and the public imagination. Her garden spreads over 4.5 acres
and includes a parterre created in a defunct walled tennis court beside
the 18th-century, 10-bedroom limestone manor house she shares with her
husband, Ed, who used to work in finance, and their four children.
Although she grows nearly 200 other flower species, supplying designers
with sweet peas, delphiniums and China asters, it is tulips, she
insists, that embody our inchoate longing for novelty and surprise.
Perhaps no other flower is as obsessively categorized by botanists:
There are 100 types of wild tulips and 15 categories grouped by bloom
time and shape. And, with over 5,000 distinct cultivars, no other bloom
has a wider range of colors, from the slender white-and-crimson
viridiflora Flaming Spring Green and the raspberry-striped Hemisphere to
the peony-like, double-fringed claret-red Belfort and Blackjack, a
single flower the color of a half-healed bruise. Such complexity,
variation and potential for downright strangeness may explain why it was
they, not the rose or the hyacinth or the peony, that once brutally
hobbled an entire financial system.

Nicholson is among a handful of private growers whose passion is the
resurrection of nearly extinct heirloom tulips, including the
mahogany-and-butterscotch-streaked Absalon, first registered in 1780,
and the velvety lavender-pink La Joyeuse, from 1863. Outside her acres,
it would take a time machine set to the 19th century to find fields as
thick with the endangered blooms she prefers (she shows them annually
with Britain's only remaining club dedicated to the flower, the
\href{https://www.tulipsociety.co.uk/}{Wakefield and North of England
Tulip Society}). Her specialty is the rare heritage varieties called
florists' tulips --- until the 1800s, the term ``florist'' was used not
for flower sellers but to describe a particular intense sort of amateur
grower --- but she also cultivates broken tulips, those whose petals
bear flame-like striations: just the sort that set off the Dutch craze.
The fever stemmed from the fact that such markings, memorialized in
dozens of paintings by Dutch still-life artists and countless
illustrated books of the time, emerged randomly from solid-color bulbs;
no one could discern what caused the hues to ``break.'' Although
commercial hybridizers have created imitations of this aesthetically
pleasing strain, real broken tulips cannot be propagated by seed, only
by collecting the tiny offshoots that form on the main bulb; their
delicacy and rarity make them exceedingly precious.

In 1927, the British mycologist Dorothy Cayley discovered that the
spontaneous coloration of broken tulips was caused by an aphid-borne
virus, the reason the flowers are now largely extinct. Today, Nicholson
is among the few breeders who are growing them, in an isolated patch
away from the other tulips to prevent contamination. ``There is a
risk,'' she says, ``but they are so beautiful that it's hard to see them
as flawed.'' After bloom time in April and May of each year, she digs
the bulbs up to tease away the tiny bulblets that form on each side,
readying them for fall planting. Every spring, the peacock-bright
feathered striations surprise her: magenta twisted with antique white,
tangerine and black exploding into hot pink, violet tinged with
butterscotch. They wave in the long grass like velvet on fire.

Image

Crab apple trees preside over fragrant cloud-pruned
osmanthus.Credit...Simon Upton

Image

An avenue of copper beech trees leads the way to the 120-acre property's
parkland.Credit...Simon Upton

Image

Historic tulips, including Insulinde, Absalon and George Grappe, on
Nicholson's dining-room table beneath a Murano glass chandelier. On the
walls hang engravings of tulips by Giovanni Battista Piranesi and
Basilius Besler.Credit...Simon Upton

NICHOLSON'S DECISION TO turn her private garden into a commercial
operation evolved slowly. She and her husband bought the estate, which
includes a coach house, stables and a dovecote with 1,500 limestone
perches, in 2005, forsaking a townhouse in London's Ravenscourt Park.
The property is on the outskirts of the market town of Calne, once home
to the C\&T Harris bacon factory and, for roughly a decade following its
closing, severe unemployment; the two-lane highway that runs through the
municipality is lined with battered buff-colored two-story attached
houses. ``Our friends thought we were daft to be buying here, in an area
that's not as smart as the Cotswolds,'' she says, ``but what is
wonderful is that this is a town with people living here, working in the
area, not just bankers buzzing back and forth from the city. It's a much
more real existence.''

Their property, down an easily missed fork lined with loosely pruned box
hedges and beech trees, is majestic though understated. Nicholson comes
from a long line of antiquarian book dealers (her business is called
\href{https://www.instagram.com/bayntunflowers/?hl=en}{Bayntun Flowers},
for George Bayntun, her great-great-grandfather, the founder of the
famed, 126-year-old \href{http://www.georgebayntun.com/}{eponymous
bookstore} in Bath). After university, where she majored in English
literature and minored in medieval studies, she spent years in the books
and manuscripts department of Sotheby's, dreaming of flowers and of
moving to the country.

But soon after she moved to Blacklands, with its rolling meadows,
endless lawns and groves of meticulously trimmed cedar of Lebanon, oak
and beech trees, friends began asking her to pillage her cutting garden
to make bouquets or create arrangements for private parties. Nicholson
had always thought of herself more as a flower grower than a creator of
landscapes, and in response, she began to scale up operations to include
a herd of 70 black Hebridean sheep, to keep the grass trimmed, and
several part-time staffers, including craftspeople who make plant
supports from bent willow branches and gardeners who hand-clip her
half-mile of trees and countless topiaries. She invites groups for
lectures in her vast greenhouse and holds workshops in the old coach
house; the thick stone walls keep the buckets of flowers cool even in
August and provide a shady place to dry spidery alliums for the wreaths
she makes and sells.

Despite the constant sense of industry, Bayntun may seem to be more
productive than immensely profitable. Nicholson revels in a garden that
does more than just provide private enjoyment; it brings fresh energy to
a working town, jolts history alive and leaves the land more fertile
than it was before. In championing long-unseen tulips, laboriously
coaxing them back into existence and then slipping them into bouquets
that find their way to smart London events, Nicholson lends a frisson to
her industrious country life: With every fierce flame or ragged fringe,
the past flickers, brilliantly, briefly, into view.

Advertisement

\protect\hyperlink{after-bottom}{Continue reading the main story}

\hypertarget{site-index}{%
\subsection{Site Index}\label{site-index}}

\hypertarget{site-information-navigation}{%
\subsection{Site Information
Navigation}\label{site-information-navigation}}

\begin{itemize}
\tightlist
\item
  \href{https://help.nytimes3xbfgragh.onion/hc/en-us/articles/115014792127-Copyright-notice}{©~2020~The
  New York Times Company}
\end{itemize}

\begin{itemize}
\tightlist
\item
  \href{https://www.nytco.com/}{NYTCo}
\item
  \href{https://help.nytimes3xbfgragh.onion/hc/en-us/articles/115015385887-Contact-Us}{Contact
  Us}
\item
  \href{https://www.nytco.com/careers/}{Work with us}
\item
  \href{https://nytmediakit.com/}{Advertise}
\item
  \href{http://www.tbrandstudio.com/}{T Brand Studio}
\item
  \href{https://www.nytimes3xbfgragh.onion/privacy/cookie-policy\#how-do-i-manage-trackers}{Your
  Ad Choices}
\item
  \href{https://www.nytimes3xbfgragh.onion/privacy}{Privacy}
\item
  \href{https://help.nytimes3xbfgragh.onion/hc/en-us/articles/115014893428-Terms-of-service}{Terms
  of Service}
\item
  \href{https://help.nytimes3xbfgragh.onion/hc/en-us/articles/115014893968-Terms-of-sale}{Terms
  of Sale}
\item
  \href{https://spiderbites.nytimes3xbfgragh.onion}{Site Map}
\item
  \href{https://help.nytimes3xbfgragh.onion/hc/en-us}{Help}
\item
  \href{https://www.nytimes3xbfgragh.onion/subscription?campaignId=37WXW}{Subscriptions}
\end{itemize}
