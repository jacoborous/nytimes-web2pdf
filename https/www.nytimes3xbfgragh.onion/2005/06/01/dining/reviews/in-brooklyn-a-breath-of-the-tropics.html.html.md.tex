Sections

SEARCH

\protect\hyperlink{site-content}{Skip to
content}\protect\hyperlink{site-index}{Skip to site index}

\href{https://www.nytimes3xbfgragh.onion/section/style}{Style}

\href{https://myaccount.nytimes3xbfgragh.onion/auth/login?response_type=cookie\&client_id=vi}{}

\href{https://www.nytimes3xbfgragh.onion/section/todayspaper}{Today's
Paper}

\href{/section/style}{Style}\textbar{}In Brooklyn, a Breath of the
Tropics

\begin{itemize}
\item
\item
\item
\item
\item
\end{itemize}

Advertisement

\protect\hyperlink{after-top}{Continue reading the main story}

Supported by

\protect\hyperlink{after-sponsor}{Continue reading the main story}

\$25 AND UNDER

\hypertarget{in-brooklyn-a-breath-of-the-tropics}{%
\section{In Brooklyn, a Breath of the
Tropics}\label{in-brooklyn-a-breath-of-the-tropics}}

By Dana Bowen

\begin{itemize}
\item
  June 1, 2005
\item
  \begin{itemize}
  \item
  \item
  \item
  \item
  \item
  \end{itemize}
\end{itemize}

YOU need patience to really appreciate the Islands, a Jamaican takeout
spot across from the Brooklyn Museum of Art that is so small it shares
an awning with the Key Food grocery next door.

When one customer approached the wood countertop, pivoting his cellphone
from his mouth to order oxtail, Shawn Letchford, one of the owners, told
him to wait his turn.

"I'm working on these nice people's food," she said, acknowledging the
standing-room-only crowd and turning back to her kiosk of a kitchen,
where a tropical landscape hung at eye level.

And work she did, for 10 more minutes. She plucked thyme leaves from
their stalks and calmly stirred them into a simmering yellow curry. Two
chopped bell peppers, eight jumbo shrimp, one unanswered phone call and
a few sultanas later, she was ready to take his order.

The Islands is not for customers unwilling to compromise their cravings.
After one woman ordered stew peas, Marilyn Reid, the other owner and
cook, told her to come back later. In four visits, pepper pot soup was
never available and the closest I came to the fried chicken (\$7.50 for
a small order, \$9 for large) was a peppery drumstick leftover from an
early dinner rush. It was delicious.

The restaurant, which the two women, childhood friends from Jamaica,
opened four year ago, probably isn't suited to Republicans, either. Ms.
Reid is an ardent Air America Radio listener, and the station's hosts
wax liberal and loud over the sound system.

The quirks make the Islands feel like a home kitchen. Ms. Letchford, a
longtime professional chef, and Ms. Reid, who worked in the corporate
world, sometimes sip wine as they mince and chop. They quiz customers
about their spice tolerance and tickle their children.

But quirks don't make it worth the visit; the food does. The women
prepare meals with such unwavering care that the one night my friends
and I received anything subpar -\/- a partially cold roti and Calypso
shrimp sauce with an irksome texture -\/- Ms. Reid apologized without
any of us having to complain. "Next time," she said, "come in earlier."

If you don't feel like waiting for food downstairs or outside under a
canopy of fairy lights, there are steep stairs leading to a dining loft,
lighted with candles and decorated with seashells, parrots and floral
prints. It feels like a secret hideaway, with just four tables and
ceilings so low that most diners slouch to their seats. Once there,
you'll want to sip brown-bag wine or beer, or homemade concoctions like
gingery limeade sweetened with brown sugar, and sorrel so heavily
steeped and spiced it seemed spiked.

Dining in is a chance to order dishes that don't travel well, like the
delicately fried snapper doused with a sweet, allspice-studded marinade.
Or the made-to-order curries, which pair well with the fragrant roti
(\$3). Each is the size of a large pizza, and arrives folded in
quarters, steaming hot. The dhal puri version, which seals crushed
chickpeas between paper-thin sheets of bread, is a thrilling way to wrap
curried vegetables and tofu (\$7 small, \$10 large). The plain or
spinach roti are better for soaking up the thicker curry clinging to
tender cubes of goat (\$8, \$10).

But the codfish roti, made by mashing re-hydrated salt cod into the
dough, is the most extraordinary, enlivening a soupy shrimp curry (\$7,
\$10) with just a hint of brine.

The house jerk sauce is fantastic, too -\/- a riddle of thyme, rosemary,
scotch bonnets and who knows what else (they aren't telling). The
chicken was moist and tongue-numbingly good (\$7.50, \$9). But my
favorite was the leg of lamb (\$12, \$15), stuffed and slathered with
chunky jerk seasoning and roasted in a re-fabbed Boston Market oven. It
was lean and intense every time, with herby pan drippings that made
anything it touched -\/- especially a custardy macaroni and cheese (\$4)
-\/- even better.

Each meal arrives with cooked cabbage, piles of sweetly seasoned coconut
rice with pigeon peas and thyme, and plantains fried in a skillet
crusted over with decades' worth of fat.

You won't have room for bread pudding (\$4), but go ahead and order it
anyway. It's rummy and rich, and they'll probably be out of it the next
time you visit.

The Islands 803 Washington Avenue (Eastern Parkway), Brooklyn;
(718)398-3575.

BEST DISHES -\/- Jerk leg of lamb; curry goat; curry shrimp; bread
pudding. PRICE RANGE -\/- Appetizers, \$4 to \$8; entrees, \$7.50 to
\$25; sides, \$3 to \$4. CREDIT CARDS -\/- Cash only. HOURS -\/- Noon to
10:30 p.m., daily. WHEELCHAIR ACCESS -\/- Takeout area is accessible.
Restrooms and tables are upstairs.

Advertisement

\protect\hyperlink{after-bottom}{Continue reading the main story}

\hypertarget{site-index}{%
\subsection{Site Index}\label{site-index}}

\hypertarget{site-information-navigation}{%
\subsection{Site Information
Navigation}\label{site-information-navigation}}

\begin{itemize}
\tightlist
\item
  \href{https://help.nytimes3xbfgragh.onion/hc/en-us/articles/115014792127-Copyright-notice}{©~2020~The
  New York Times Company}
\end{itemize}

\begin{itemize}
\tightlist
\item
  \href{https://www.nytco.com/}{NYTCo}
\item
  \href{https://help.nytimes3xbfgragh.onion/hc/en-us/articles/115015385887-Contact-Us}{Contact
  Us}
\item
  \href{https://www.nytco.com/careers/}{Work with us}
\item
  \href{https://nytmediakit.com/}{Advertise}
\item
  \href{http://www.tbrandstudio.com/}{T Brand Studio}
\item
  \href{https://www.nytimes3xbfgragh.onion/privacy/cookie-policy\#how-do-i-manage-trackers}{Your
  Ad Choices}
\item
  \href{https://www.nytimes3xbfgragh.onion/privacy}{Privacy}
\item
  \href{https://help.nytimes3xbfgragh.onion/hc/en-us/articles/115014893428-Terms-of-service}{Terms
  of Service}
\item
  \href{https://help.nytimes3xbfgragh.onion/hc/en-us/articles/115014893968-Terms-of-sale}{Terms
  of Sale}
\item
  \href{https://spiderbites.nytimes3xbfgragh.onion}{Site Map}
\item
  \href{https://help.nytimes3xbfgragh.onion/hc/en-us}{Help}
\item
  \href{https://www.nytimes3xbfgragh.onion/subscription?campaignId=37WXW}{Subscriptions}
\end{itemize}
