Sections

SEARCH

\protect\hyperlink{site-content}{Skip to
content}\protect\hyperlink{site-index}{Skip to site index}

\href{https://www.nytimes3xbfgragh.onion/section/technology}{Technology}

\href{https://myaccount.nytimes3xbfgragh.onion/auth/login?response_type=cookie\&client_id=vi}{}

\href{https://www.nytimes3xbfgragh.onion/section/todayspaper}{Today's
Paper}

\href{/section/technology}{Technology}\textbar{}Those Dimples May Be
Digits

\begin{itemize}
\item
\item
\item
\item
\item
\end{itemize}

Advertisement

\protect\hyperlink{after-top}{Continue reading the main story}

Supported by

\protect\hyperlink{after-sponsor}{Continue reading the main story}

\hypertarget{those-dimples-may-be-digits}{%
\section{Those Dimples May Be
Digits}\label{those-dimples-may-be-digits}}

By \href{https://www.nytimes3xbfgragh.onion/by/andy-newman}{Andy Newman}

\begin{itemize}
\item
  May 3, 2001
\item
  \begin{itemize}
  \item
  \item
  \item
  \item
  \item
  \end{itemize}
\end{itemize}

See the article in its original context from\\
May 3, 2001, Section G, Page
1\href{https://store.nytimes3xbfgragh.onion/collections/new-york-times-page-reprints?utm_source=nytimes\&utm_medium=article-page\&utm_campaign=reprints}{Buy
Reprints}

\href{http://timesmachine.nytimes3xbfgragh.onion/timesmachine/2001/05/03/131512.html}{View
on timesmachine}

TimesMachine is an exclusive benefit for home delivery and digital
subscribers.

DETECTIVE SEAN BOERO and his partner were walking the casino floor at
the Atlantic City Hilton at 3 o'clock one recent morning when, behind a
pile of red chips at a craps table, they spotted a familiar face.

It was a petty thief and card cheat named Sammy with a long criminal
record. It was a safe bet that he posed some kind of threat to public or
corporate safety at the casino, but the detectives had nothing on him.
They called headquarters to see if there were any warrants for his
arrest. No dice.

That might have been the end of it. But Detective Boero, a veteran state
trooper assigned to New Jersey's Division of Gaming Enforcement, had a
hunch.

He went back to his office, called up Sammy's digital mug shot on a
computer and hit Search. Up popped a bunch of mug shots of men who
looked, to varying degrees, like Sammy (whose full name the detective
would not disclose for publication). Several were of Sammy himself under
different names, Detective Boero said. He checked for warrants on those
names.

Two of Sammy's alter egos were wanted for theft.

Detective Boero raced back to the casino. It was 6 a.m. Sammy had
switched tables, but he still had his chips. The detective approached
and gave Sammy the news.

''He just smiled and said, 'You got me,' '' Detective Boero said.

Score one for face-recognition software.

Face recognition -\/- which works by converting video images, photos or
even police composites of faces into strings of numbers, then comparing
them with other strings of numbers that stand for known faces -\/- is
very much in its infancy.

It works reasonably well at confirming the identity of someone who wants
to log on to a computer or pass through a security checkpoint. But it
has yet to yield consistent results in law-enforcement and surveillance
settings, where, depending on lighting, the camera angle and quality,
and other variables, its powers of discrimination can be more akin to
Mr. Magoo's than Superman's.

The technology burst into the public consciousness, and caused dismay
among many privacy advocates, after the Super Bowl in January, when it
was revealed that the faces of everyone entering the stadium had been
compared against a mug shot file of criminals and terrorists. And such
systems are peering into more and more faces every day.

About 100 casinos across the country now use cameras hooked up to facial
recognition systems to search their floors for known cheats,
purse-snatchers and coin-cup swipers, according to the International
Biometric Group, a consulting firm in New York. (Face recognition is one
of several types of biometric systems, which use physical measurements
to identify people.)

Illinois uses it to catch people using borrowed or stolen identities
applying for drivers' licenses. A couple of dozen police departments in
the United States and Canada use it as the state police in Atlantic City
do, to help figure out who their suspects really are. In West Virginia,
a computer is churning through millions of Web pages a day looking for
photographs of missing children. And election officials in Mexico and
Uganda are using it to prevent double voting.

Face recognition is still a tiny industry. Only about \$10 million worth
of face-recognition software was sold last year, according to the
International Biometric Group.

But Jim Wayman, the director of the Biometric Test Center at San Jose
State University, which has tested face recognition for the federal
government, said its use would continue to grow as long as people did
not expect too much from it too soon.

''Even if you use it to say, 'Look, I don't know who this guy is,' but
instead of looking at a thousand pictures you can narrow it down to a
hundred, that's a good thing,'' Mr. Wayman said. In essence, he said, it
creates a smaller haystack.

Facial recognition was developed at universities over the past decade at
the request (and with the financial support) of the Defense Department,
which thought that it could be used to tighten security at border
crossings and fight drug smuggling.

The system that Detective Boero uses runs software designed by Viisage
Technology, a company in Littleton, Mass. It works by describing faces
in terms of how much they resemble each of 128 archetypal faces. The
archetypes, ghostly-looking things known as eigenfaces, have been
created by combining hundreds of real faces in a databank and then using
mathematical formulas to analyze similarities and differences.

Viisage's chief rival, the Visionics Corporation, based in Jersey City,
uses a system called local feature analysis. It involves plotting out
the relative positions of a dozen or more points at places where the
curvature of the face changes -\/- the features that make a face unique
-\/- and trying to find other faces with similar constellations of
critical points.

Both systems, after analyzing a face, display the stored faces that it
resembles in descending order of similarity.

Under ideal conditions, most systems do a decent job of recognizing
faces. But a battery of tests completed by the Defense Department last
year found that differences in lighting, camera angle and distance, and
camera quality all substantially affected face recognition's
performance. The ability to make a match also declines as time goes by
and people's faces change with age.

One test, using two photos of a cooperative subject taken about a year
apart under somewhat different lighting, found that the Visionics
system, generally the best performer of the five systems in the tests,
picked the correct person out of a file of 227 faces only 55 percent of
the time.

Visionics had the correct person ranked in the top 10 percent of matches
about 90 percent of the time. But many mug shot files include tens of
thousands of images, so a face could appear thousands of images from the
top of the list and still be in the top 10 percent.

When the systems tried to match a head-on photo with one of the same
person turned 45 degrees, both Viisage and Visionics picked the correct
person out of a file of 1,200 images less than a quarter of the time.

Software makers note that surveillance cameras taking 30 frames a second
offer many chances to match an image. But a test using people walking
past a video camera (without looking straight at it) found that even the
best systems ranked the correct person first in a field of 165 images
only about 35 percent of the time.

It is little wonder, then, that many customers who were expecting
miracles from their face-recognition systems, which can cost anywhere
from \$15,000 to many times that, have been disappointed.

Within hours of installing a face-recognition system in 1997, the Los
Angeles County sheriff's department used it in a carjacking case in
which the only leads were a sketch of the suspect, a possible first name
and a guess about what neighborhood he lived in. The system's choice for
the second-best match turned out to be the culprit.

But the system has not solved any more cases, said Sergeant Bill Conley.
''It has not proved to be the end-all that we had hoped initially,'' he
said.

Capt. Dan McCoy of the Santa Ana, Calif., police department, which has
had a face-recognition system for two years, said it was useless at
making identifications from grainy store-surveillance videos.

Face-recognition technology works best when it gets a lot of help -\/-
in places like casinos, where hundreds of cameras monitor the
well-lighted floors and guards can pan, tilt and zoom to get a good
shot.

In March, at the Trump Marina casino in Atlantic City, a customer at a
transaction window was acting ''a little squirrelly'' as the cashier
processed his credit card, recalled Richard Santoro, Trump's vice
president for corporate security.

The cashier notified security, and a camera operator zoomed in. A row of
mug shots popped up on a computer screen. One was of the same man, taken
a few weeks before at another Trump casino when he was arrested under a
different name on fraud charges. Meanwhile, a check of his new
credentials showed that they had been reported stolen.

Security guards called the police, who charged the man with credit card
fraud and theft by deception, Mr. Santoro said.

At the Super Bowl in Tampa, Fla., in January, a face-recognition system
showed that 19 petty criminals were among the 70,000 fans in attendance,
police officials said. No arrests were made, but the police were
impressed enough that they plan to hook up a face-recognition system for
the outdoor security cameras in Ybor City, the city's popular nightlife
district.

The biggest face-recognition installation in public space is in Newham,
a rough neighborhood in London's East End. There, 300 security cameras
are trained on the borough's five main shopping areas, and at any time,
images from 10 of the cameras are being fed into a face-recognition
system. When the system spots any of 100 or so local hoodlums, an alarm
goes off in a monitoring room; if warranted, an officer will be
dispatched to trail or question the person. Newham officials credit the
system with a 30 percent drop in crime.

But even in Newham, face recognition has not led to any arrests. Rather,
said Bob Lack, the official in charge of the system, criminals have
heard about the system and taken their business elsewhere.

While Mr. Lack said that Newham residents overwhelmingly supported the
face-recognition system, facial recognition has raised concerns about
privacy in Britain and even more so in the United States. To the extent
that the technology does work, civil libertarians worry that it will
enhance the government's ability to spy on its residents.

The police are not allowed to stop someone on the street for no reason
and demand identification, but face recognition could in effect allow
them to do just that, said Barry Steinhardt, the associate director of
the American Civil Liberties Union. ''It's a wonderful tool for police
who want to film a demonstration and compare faces to drivers'
licenses,'' he said.

For every Orwellian scenario that facial recognition conjures, however,
there is a potential benefit. George Davis, the program director of
Anser, a nonprofit group that uses the technology to search the Web
(unsuccessfully, so far) for images of missing children, said that a
federal agency was interested in his proposal to search seized child
pornography videos or photos for missing children or for adults from
sex-offender registries.

It would not be a big stretch, Mr. Davis said, to pass legislation
creating a database of photos of all the parents in the country who are
suspected of abducting their children, then comparing those photos to
drivers' license records. ''You do that, you find half the missing kids
in the country,'' he said.

Mr. Wayman of San Jose State said there was one thing he could predict
with confidence about face recognition. ''It will certainly be more
accurate in the future,'' he said. ''There's only one way it can go.''

Advertisement

\protect\hyperlink{after-bottom}{Continue reading the main story}

\hypertarget{site-index}{%
\subsection{Site Index}\label{site-index}}

\hypertarget{site-information-navigation}{%
\subsection{Site Information
Navigation}\label{site-information-navigation}}

\begin{itemize}
\tightlist
\item
  \href{https://help.nytimes3xbfgragh.onion/hc/en-us/articles/115014792127-Copyright-notice}{©~2020~The
  New York Times Company}
\end{itemize}

\begin{itemize}
\tightlist
\item
  \href{https://www.nytco.com/}{NYTCo}
\item
  \href{https://help.nytimes3xbfgragh.onion/hc/en-us/articles/115015385887-Contact-Us}{Contact
  Us}
\item
  \href{https://www.nytco.com/careers/}{Work with us}
\item
  \href{https://nytmediakit.com/}{Advertise}
\item
  \href{http://www.tbrandstudio.com/}{T Brand Studio}
\item
  \href{https://www.nytimes3xbfgragh.onion/privacy/cookie-policy\#how-do-i-manage-trackers}{Your
  Ad Choices}
\item
  \href{https://www.nytimes3xbfgragh.onion/privacy}{Privacy}
\item
  \href{https://help.nytimes3xbfgragh.onion/hc/en-us/articles/115014893428-Terms-of-service}{Terms
  of Service}
\item
  \href{https://help.nytimes3xbfgragh.onion/hc/en-us/articles/115014893968-Terms-of-sale}{Terms
  of Sale}
\item
  \href{https://spiderbites.nytimes3xbfgragh.onion}{Site Map}
\item
  \href{https://help.nytimes3xbfgragh.onion/hc/en-us}{Help}
\item
  \href{https://www.nytimes3xbfgragh.onion/subscription?campaignId=37WXW}{Subscriptions}
\end{itemize}
