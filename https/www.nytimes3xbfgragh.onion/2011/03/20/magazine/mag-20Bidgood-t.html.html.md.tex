Sections

SEARCH

\protect\hyperlink{site-content}{Skip to
content}\protect\hyperlink{site-index}{Skip to site index}

\href{https://myaccount.nytimes3xbfgragh.onion/auth/login?response_type=cookie\&client_id=vi}{}

\href{https://www.nytimes3xbfgragh.onion/section/todayspaper}{Today's
Paper}

A Gay Cult Classic Re-Emerges

\begin{itemize}
\item
\item
\item
\item
\item
\end{itemize}

Advertisement

\protect\hyperlink{after-top}{Continue reading the main story}

Supported by

\protect\hyperlink{after-sponsor}{Continue reading the main story}

Read More

\hypertarget{a-gay-cult-classic-re-emerges}{%
\section{A Gay Cult Classic
Re-Emerges}\label{a-gay-cult-classic-re-emerges}}

\includegraphics{https://static01.graylady3jvrrxbe.onion/images/2011/03/20/magazine/mag-20Bidgood-t_CA0/mag-20Bidgood-t_CA0-articleLarge.jpg?quality=75\&auto=webp\&disable=upscale}

By William Van Meter

\begin{itemize}
\item
  March 18, 2011
\item
  \begin{itemize}
  \item
  \item
  \item
  \item
  \item
  \end{itemize}
\end{itemize}

``It's like I survived a train wreck and everybody else died and my name
is associated with that disaster,'' says James Bidgood of his first and
last film, ``Pink Narcissus,'' a 1971 erotic fantasia that contrasts a
hustler's satin reveries with sordid reality. On March 21, it will be
shown at the IFC Center in New York as part of ``Queer/Art/Film.''
Bidgood, 77, is debating whether or not to go. ``I can't look at it.''

``Pink Narcissus'' may be a benchmark of underground gay cinema, but for
decades its director was forgotten. Rumors circulated that the film was
by Andy Warhol or Kenneth Anger. Bidgood didn't get any credit and
didn't want any. It was the making and unmaking of his artistic career.

\href{https://www.nytimes3xbfgragh.onion/slideshow/2011/03/20/magazine/bidgood.html}{}

\hypertarget{queer-classic}{%
\subsection{Queer Classic}\label{queer-classic}}

6 Photos

View Slide Show ›

\includegraphics{https://static01.graylady3jvrrxbe.onion/images/2011/03/20/magazine/bidgood-slide-S457/bidgood-slide-S457-jumbo.jpg?quality=75\&auto=webp\&disable=upscale}

Peter Yang for The New York Times

Bidgood moved to New York in 1951, working as a female impersonator and
taking photographs for men's physique magazines like Muscleboy and
Adonis. ``There was no art,'' Bidgood laments. ``They were badly lit and
uninteresting. Playboy had girls in furs, feathers and lights. They had
faces like beautiful angels. I didn't understand why boy pictures
weren't like that.'' Bidgood's first series showed a man swimming
through a shimmering undersea cave, which he built in his living room.

``Pink Narcissus'' grew out of these fanciful soft-core tableaux. Most
of the film was shot in his apartment near Times Square, between 1963
and 1970. Neighborhood prostitutes made up the cast. Bidgood designed
the sets and costumes and did the makeup. In the editing stage, a
now-defunct production company that had invested in the project took
over. ``One of the props in my apartment was an ax,'' he recalls. ``I
was going to edit their bodies.'' Bidgood replaced his name in the
credits with Anonymous. The film was released but ignored. Bidgood took
a decade-long break from his art.

``No one lives in squalor out of caprice,'' Bidgood said when I met with
him in the cramped fourth-floor walk-up where he now lives and works.
Shelves holding paint and glitter line the walls. The toaster oven
contains miniature golden cherubs for a photo project commissioned by
the shoe designer Christian Louboutin.

Bidgood works continuously, but his output is small; much of his recent
work was financed by a grant from Creative Capital. ``Had it not been
about what it was about,'' he said of ``Pink Narcissus,'' ``this person
might have gotten a call from Hollywood.''

Advertisement

\protect\hyperlink{after-bottom}{Continue reading the main story}

\hypertarget{site-index}{%
\subsection{Site Index}\label{site-index}}

\hypertarget{site-information-navigation}{%
\subsection{Site Information
Navigation}\label{site-information-navigation}}

\begin{itemize}
\tightlist
\item
  \href{https://help.nytimes3xbfgragh.onion/hc/en-us/articles/115014792127-Copyright-notice}{©~2020~The
  New York Times Company}
\end{itemize}

\begin{itemize}
\tightlist
\item
  \href{https://www.nytco.com/}{NYTCo}
\item
  \href{https://help.nytimes3xbfgragh.onion/hc/en-us/articles/115015385887-Contact-Us}{Contact
  Us}
\item
  \href{https://www.nytco.com/careers/}{Work with us}
\item
  \href{https://nytmediakit.com/}{Advertise}
\item
  \href{http://www.tbrandstudio.com/}{T Brand Studio}
\item
  \href{https://www.nytimes3xbfgragh.onion/privacy/cookie-policy\#how-do-i-manage-trackers}{Your
  Ad Choices}
\item
  \href{https://www.nytimes3xbfgragh.onion/privacy}{Privacy}
\item
  \href{https://help.nytimes3xbfgragh.onion/hc/en-us/articles/115014893428-Terms-of-service}{Terms
  of Service}
\item
  \href{https://help.nytimes3xbfgragh.onion/hc/en-us/articles/115014893968-Terms-of-sale}{Terms
  of Sale}
\item
  \href{https://spiderbites.nytimes3xbfgragh.onion}{Site Map}
\item
  \href{https://help.nytimes3xbfgragh.onion/hc/en-us}{Help}
\item
  \href{https://www.nytimes3xbfgragh.onion/subscription?campaignId=37WXW}{Subscriptions}
\end{itemize}
