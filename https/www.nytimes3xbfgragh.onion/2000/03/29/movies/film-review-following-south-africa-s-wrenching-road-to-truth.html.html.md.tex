Sections

SEARCH

\protect\hyperlink{site-content}{Skip to
content}\protect\hyperlink{site-index}{Skip to site index}

\href{https://www.nytimes3xbfgragh.onion/section/movies}{Movies}

\href{https://myaccount.nytimes3xbfgragh.onion/auth/login?response_type=cookie\&client_id=vi}{}

\href{https://www.nytimes3xbfgragh.onion/section/todayspaper}{Today's
Paper}

\href{/section/movies}{Movies}\textbar{}FILM REVIEW; Following South
Africa's Wrenching Road to Truth

\begin{itemize}
\item
\item
\item
\item
\item
\end{itemize}

Advertisement

\protect\hyperlink{after-top}{Continue reading the main story}

Supported by

\protect\hyperlink{after-sponsor}{Continue reading the main story}

FILM REVIEW

\hypertarget{film-review-following-south-africas-wrenching-road-to-truth}{%
\section{FILM REVIEW; Following South Africa's Wrenching Road to
Truth}\label{film-review-following-south-africas-wrenching-road-to-truth}}

\begin{itemize}
\tightlist
\item
  Long Night's Journey Into Day\\
  Directed by Deborah Hoffmann, Frances Reid Documentary Not Rated 1h
  34m
\end{itemize}

By Elvis Mitchell

\begin{itemize}
\item
  March 29, 2000
\item
  \begin{itemize}
  \item
  \item
  \item
  \item
  \item
  \end{itemize}
\end{itemize}

See the article in its original context from\\
March 29, 2000, Section E, Page
6\href{https://store.nytimes3xbfgragh.onion/collections/new-york-times-page-reprints?utm_source=nytimes\&utm_medium=article-page\&utm_campaign=reprints}{Buy
Reprints}

\href{http://timesmachine.nytimes3xbfgragh.onion/timesmachine/2000/03/29/502812.html}{View
on timesmachine}

TimesMachine is an exclusive benefit for home delivery and digital
subscribers.

''Long Night's Journey Into Day'' is a beautiful and often disturbing
reflection on the nature of truth and forgiveness. This documentary,
directed by Frances Reid and Deborah Hoffmann, captures the mandate of
one of the most unusual social phenomena of the last century, the Truth
and Reconciliation Committee of South Africa.

The committee's function, after the fall of apartheid, was to sift
through the crimes of an era of systematic racism and decide who would
be allowed to seek amnesty.

''This process is not about pillorying,'' says Archbishop Desmond Tutu,
the committee's chairman. ''It's actually about getting to the truth, so
we can heal.'' To be granted amnesty, the applicants have to tell the
truth. According to on-screen text at the film's opening, more than
7,000 sought amnesty, and the documentary chooses four stories to tell.

The filmmakers' structure is simple and direct, like the hearings
themselves. They start with a case that many may be familiar with, that
of the American student and rights advocate Amy Biehl, who was killed by
black South Africans. Biehl was attracted to South Africa because of her
love of the work of Nelson Mandela, and her death was particularly
tragic; she was the victim of resentment that couldn't have had less to
do with her.

Here we get to see how the hearings were held, in front of a crowd and
broadcast on television so the entire country witnessed the accused's
contrition. You may feel a chill pass through you when Biehl's parents
find it in their hearts to approve the application for amnesty by the
convicted killers of their child. The mother of one of the accused is as
heartbroken as Biehl's parents, and the process doesn't buy her a
moment's peace.

The directors recognize that the drama in this story is so potent it
doesn't require any icing and that there's a bigger story between the
lines, a subtext about psychological violence visited on a country,
generation after generation.

The subtle complications of ''Long Night's Journey Into Day'' take hold
as another applicant, Robert McBride, convicted of a bombing that took
the lives of three whites, talks about his suffering, describing the
crude gradations of dehumanization that were accepted in South Africa:
''colored'' and ''nonwhite.'' Driven to strike back, he says on the
stand that he has ''served a sentence longer than any apartheid
minister.''

This attempt to bring closure to the jagged wound of apartheid has so
far to go that each of the stories sends a jolt of hurt to your nervous
system.

Each of the episodes is more wrenching than the one before it. The most
heart-rending is the last, which reports on the shooting of seven
''terrorists'' by the police, including video scenes of the corpses
being callously jerked by rope across bloody asphalt. (The victims'
mothers become hysterical during the hearing when the tape is run.)

The biggest shock comes at the revelation that the secret police
essentially recruited angry young black men, trained them to become
revolutionaries and then killed them. Of the 25 police officers involved
in this grim event, only two seek amnesty, and one is a black who
infiltrated the group and instigated its members' actions.

''Night'' offers its own definition of justice. Archbishop Tutu calls it
''restorative justice,'' noting the need to confront these events in the
light of day. It's not totally curative, because justice can never
really be delivered in these circumstances; the ache of racism and its
violent aftermath still remain. This is an issue that the film addresses
with as much integrity as the committee shows in its own efforts.

LONG NIGHT'S JOURNEY INTO DAY

Directed by Frances Reid and Deborah Hoffmann; written (in English,
Xhosa and Afrikaans, with English subtitles) by Antijie Krog; directors
of photography, Ms. Reid and Ezra Jwili; edited by Ms. Hoffman; music by
Lebo M; produced by Ms. Reid; released by Iris Films/Cinemax Reel Life.
At the Film Forum, 209 West Houston Street, South Village. Running time:
94 minutes. This film is not rated.

WITH: Peter and Linda Biehl, Mongezi Manqina, Evelyn Manqina, Easy
Nofemela, Eric Taylor, Nomonde Calata, Nyameka Goniwe, George Bizos,
Robert McBride, Sharon Welgemoed, Thapelo Mbelo, Jann Turner, Archbishop
Desmond Tutu and Pumla Gobodo-Madikizela.

Advertisement

\protect\hyperlink{after-bottom}{Continue reading the main story}

\hypertarget{site-index}{%
\subsection{Site Index}\label{site-index}}

\hypertarget{site-information-navigation}{%
\subsection{Site Information
Navigation}\label{site-information-navigation}}

\begin{itemize}
\tightlist
\item
  \href{https://help.nytimes3xbfgragh.onion/hc/en-us/articles/115014792127-Copyright-notice}{©~2020~The
  New York Times Company}
\end{itemize}

\begin{itemize}
\tightlist
\item
  \href{https://www.nytco.com/}{NYTCo}
\item
  \href{https://help.nytimes3xbfgragh.onion/hc/en-us/articles/115015385887-Contact-Us}{Contact
  Us}
\item
  \href{https://www.nytco.com/careers/}{Work with us}
\item
  \href{https://nytmediakit.com/}{Advertise}
\item
  \href{http://www.tbrandstudio.com/}{T Brand Studio}
\item
  \href{https://www.nytimes3xbfgragh.onion/privacy/cookie-policy\#how-do-i-manage-trackers}{Your
  Ad Choices}
\item
  \href{https://www.nytimes3xbfgragh.onion/privacy}{Privacy}
\item
  \href{https://help.nytimes3xbfgragh.onion/hc/en-us/articles/115014893428-Terms-of-service}{Terms
  of Service}
\item
  \href{https://help.nytimes3xbfgragh.onion/hc/en-us/articles/115014893968-Terms-of-sale}{Terms
  of Sale}
\item
  \href{https://spiderbites.nytimes3xbfgragh.onion}{Site Map}
\item
  \href{https://help.nytimes3xbfgragh.onion/hc/en-us}{Help}
\item
  \href{https://www.nytimes3xbfgragh.onion/subscription?campaignId=37WXW}{Subscriptions}
\end{itemize}
