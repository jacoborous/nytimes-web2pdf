Sections

SEARCH

\protect\hyperlink{site-content}{Skip to
content}\protect\hyperlink{site-index}{Skip to site index}

\href{https://www.nytimes3xbfgragh.onion/section/nyregion}{New York}

\href{https://myaccount.nytimes3xbfgragh.onion/auth/login?response_type=cookie\&client_id=vi}{}

\href{https://www.nytimes3xbfgragh.onion/section/todayspaper}{Today's
Paper}

\href{/section/nyregion}{New York}\textbar{}After 9/11, Parcels of
Money, and Dismay

\begin{itemize}
\item
\item
\item
\item
\item
\end{itemize}

Advertisement

\protect\hyperlink{after-top}{Continue reading the main story}

Supported by

\protect\hyperlink{after-sponsor}{Continue reading the main story}

\hypertarget{after-911-parcels-of-money-and-dismay}{%
\section{After 9/11, Parcels of Money, and
Dismay}\label{after-911-parcels-of-money-and-dismay}}

This article was reported and written by
\href{https://www.nytimes3xbfgragh.onion/by/edward-wyatt}{Edward Wyatt},
\href{https://www.nytimes3xbfgragh.onion/by/david-w-chen}{David W. Chen}
and \href{https://www.nytimes3xbfgragh.onion/by/charles-v-bagli}{Charles
V. Bagli}

\begin{itemize}
\item
  Dec. 30, 2002
\item
  \begin{itemize}
  \item
  \item
  \item
  \item
  \item
  \end{itemize}
\end{itemize}

See the article in its original context from\\
December 30, 2002, Section A, Page
1\href{https://store.nytimes3xbfgragh.onion/collections/new-york-times-page-reprints?utm_source=nytimes\&utm_medium=article-page\&utm_campaign=reprints}{Buy
Reprints}

\href{http://timesmachine.nytimes3xbfgragh.onion/timesmachine/2002/12/30/620904.html}{View
on timesmachine}

TimesMachine is an exclusive benefit for home delivery and digital
subscribers.

In the months after nearly 2,800 people died in the terrorist attack
that destroyed the World Trade Center, the Bush administration pledged
\$21.4 billion to address the emergency. The extraordinary allocation of
federal disaster relief, while less than local elected officials had
petitioned for, was meant to reassure, restore and restart an
emotionally and economically ravaged New York City and surrounding
region.

More than 15 months later, some \$4.5 billion to \$5 billion has made
its way from Washington to New York. Significant sums of money were made
available almost immediately, and they made possible early progress in
providing for victims and establishing a sense that the city could
rebound.

Roughly \$600 million was sent to cover the costs of the monumental task
of cleaning up the site. An additional \$401 million has been
distributed to some 10,000 downtown businesses to keep them afloat. And
\$493 million in tax-free bonds have been approved to help finance major
construction projects, including three residential complexes around
ground zero.

But many victims, elected officials, business executives and others are
both confused and angry about why, more than a year after the most
serious terrorist attack on American soil, less than a quarter of the
federal government's promise of financial assistance has been realized,
why hundreds of millions of dollars that are in the hands of New York
officials have gone unclaimed, and why firm decisions have yet to be
made on how additional billions of dollars will actually be spent.

The explanations, based on an examination of financial filings and
interviews with government officials, real estate experts, watchdog
organizations and downtown residents and business people, are varied.

Some \$1 billion of the \$21.4 billion in aid, it turns out, will buy
insurance for the companies that conducted the cleanup, which they would
use to defend themselves against any lawsuits. Aides to some New York
elected officials who have tracked the money also believe that some of
New York's promised millions have gone to victims and institutions in
Pennsylvania and Washington, the other two sites of devastation wrought
by terrorists on Sept. 11.

Moreover, federal officials never intended to deliver another huge chunk
of the money -\/- some \$5 billion to rebuild the transportation network
downtown -\/- for years, determined to wait until there was a fully
formulated plan. No one doubts that the government will make good on its
pledge. But the jobs that such a project would bring and the tangible
benefits to the city remain far off.

Further, many businesses and residents were shocked to learn that
hundreds of millions of dollars in grants available to them were subject
to federal taxes, a fact that both reduced the real value of that aid
and discouraged some from taking advantage of it. Many people have also
failed to realize that about a quarter of the total aid package -\/-
some \$5.5 billion -\/- was made up of tax breaks, not real dollars. So
the ultimate value of that portion is the subject of great debate, with
some economists and real estate experts in the city worried that it is
largely an illusion.

Distrust among the government agencies involved in controlling the aid,
and self-confessed bureaucratic missteps by the Federal Emergency
Management Agency, which is in charge of channeling nearly \$9 billion
to New York, has also led to delays in payments and disillusionment
among potential recipients.

And finally, some of the money has not made its way to New York because
local and state officials have not figured out exactly how to take
advantage of all the available types of aid. Other bids for financing
certain projects -\/- \$90 million to help assess the long-term health
of thousands of trade center rescue workers, \$980 million to reimburse
the city and state for the cost of deploying its uniformed forces and
future fire department training, have been stalled.

''You can look at the trees, and see many things that should have been
done better or quicker,'' said Senator Charles E. Schumer, a New York
Democrat who has been involved in negotiating the aid packages. ''But if
you look at the forest, it's a large and unprecedented sum of money
that's basically being used in the way that it was needed and intended.
The spending of the money has been far from perfect, but over all, it's
been pretty good.''

Determining precisely how much money has made it to New York and
actually been distributed is difficult. An examination by The New York
Times of the dozens of sources and uses of the promised \$21.4 billion
in disaster aid found significant disparities in accounting for the
money, both among the federal officials handling it and the
organizations, people and institutions in New York and the region
receiving it.

The White House's Office of Management and Budget has one set of figures
for how much has been given, and congressional staff members have a
different set, and there is some belief that double counting has
occurred. Beyond that, the number of government agencies and
organizations involved in the effort is daunting, and the money was
appropriated in different pieces of legislation. Indeed, a tabulation by
The Times of the total amount pledged by various federal agencies adds
up to only \$21.3 billion.

Responsibility for definitively tracking where it has all gone remains
unclear. For instance, last September, an official with the federal
General Accounting Office told New York officials that the O.M.B.
appeared to be so uncertain of how much had been spent that it asked the
General Accounting Office -\/- an unusual request since it is the
O.M.B.'s job to monitor federal expenditures.

But establishing some specific answers and general trends was possible,
as was identifying unanswered questions about enormous amounts of money.

It is clear, for instance, that FEMA, after much confusion and repeated
complaints, has spent \$84 million in its program to provide emergency
rent and mortgage assistance. But FEMA says it still has more than \$360
million that, absent specific requests by New York officials, has no
determined fate.

The Agency

Much-Disputed Role In Delivering Aid

Of all the federal agencies involved in New York's recovery effort, none
has played a greater or more disputed role than the Federal Emergency
Management Agency. The agency, which many officials at local and federal
levels now believe was a poor choice to play such a large role in a
unique economic disaster, was charged with handling \$8.8 billion in aid
meant to address multiple needs.

Currently, some \$2.3 billion of that total has been paid out, or is
close to being claimed by recipients, including roughly \$600 million
for those involved in cleaning up ground zero, \$62 million for the city
medical examiner's office and \$16.7 million for people who lost their
jobs after Sept. 11.

Officials with the agency, after months of requests from elected
officials, recently released what they said was the definitive
accounting of the agency's money. As part of that listing, FEMA
officials said that in addition to the \$2.3 billion paid, \$6.1 billion
has been earmarked for certain programs or projected as the likely costs
for a range of efforts. One of those projections, for instance, involves
a possible \$800 million payment to the Port Authority; another involves
the \$91 million allocation for an environmental cleanup program.

And finally, the agency said the fate of some \$366 million was as yet
undetermined.

But while the agency has drawn praise for its prompt and complete
payment for the cleanup effort, virtually every other aspect of its
performance and its accounting has been the object of criticism and
debate -\/- the rate and speed with which it compensated people who
suffered economically, its failures to publicize some of its aid
programs, and the delayed public disclosures about where its money had
gone.

''Is it moving along as expeditiously as we hoped? No,'' said
Representative John E. Sweeney, Republican from New York, who added that
he thought New York officials had often gotten ''the runaround'' from
FEMA. ''But I think the pressure needs to be constant from Congress.''
He added, ''I'm not ready to say that I am absolutely frustrated, but I
think the next six months are going to be rather critical.''

For all the acrimony and suspicion, though, FEMA officials have sought
to dispel any notion that New York will not get the full amount of money
it was promised.

''FEMA will spend the \$8.8 billion on this disaster, no question about
it,'' said Brad Gair, the agency's federal recovery officer in New York.
''It will all be spent.''

For FEMA, which typically tackles hurricanes, earthquakes and ice storms
in rural or suburban settings, the terror attack in Lower Manhattan was
a disaster without precedent in terms of scale, cost and landscape. And
some of the most intense criticism of the agency has grown out of the
issue of whether its models for distributing aid work in a situation
where much of the damage was not just physical, but economic. The
victims, too, were spread over different boroughs and states.

The agency's handling of its program to help people avoid eviction or
foreclosure has been a representative problem. The agency has admitted
bungling the program, sending out the wrong applications and having
people without knowledge of New York's geography review aid requests.
When it was revealed that the agency was turning down applicants at a
rate never before seen in a disaster, it was forced to adjust its
requirements.

To date, some \$84 million has been paid out in the program -\/- less
than half the \$175 million that has been dedicated. And the \$91
million program intended to help people clean their downtown apartments
has so far paid out only \$19 million, in part because eligible
residents were unaware of it.

''We feel FEMA has done a good job getting the money out the door,'' Mr.
Gair said.

The modest amounts paid out in some of FEMA's programs, though, raise
the question of whether it is doing enough to distribute the cash. But
they have also provoked great curiosity over whether FEMA will make any
leftover money available for alternative uses.

The \$366 million that by its own account is unallocated could grow by
hundreds of millions if its payouts continue to run below projections.

The priorities are myriad for the remaining money. Senator Hillary
Rodham Clinton, for instance, wants \$90 million to monitor the
long-term health of rescue workers who assisted in the trade center
recovery effort. Representative Carolyn B. Maloney, Democrat of New
York, wants FEMA to live up to its promise to provide \$33 million in
funds for mental health services in New York's schools.

Mayor Michael R. Bloomberg and Gov. George E. Pataki want FEMA to
reimburse the city and the state \$980 million for expenses like the
cost of deploying uniformed personnel and for future fire department
training costs that are not already covered under FEMA regulations.

So far, those efforts have stalled. Further, numerous officials have
begun to express alarm about whether FEMA's \$8.8 billion will be enough
in the end.

Mr. Gair has conceded that New York may have more needs than FEMA can
cover.

''New York shouldn't have to choose whether it needs to take care of
fire, police or rebuilding its infrastructure,'' Mrs. Maloney said.
''Historically, the response to disasters was based on need, and this is
the first disaster where it's been based on how much money is
available.''

The debate over how much money remains available has focused more on the
\$21.4 billion, and whether it would prove adequate. After all, that
figure was not based on any assessment, but was rather a number that
grew out of early conversations between Mrs. Clinton, Mr. Schumer and
President Bush days after the attacks.

Securing any additional funds will not be easy, given the concerns over
the ballooning federal deficit.

''Part of what we have to do is to show that we're not being greedy,
we're not overreaching,'' said Mrs. Clinton. ''But if you look at every
other disaster, we are being conservative in saying what we need. So I
am confident that we will be able to make the case.''

The Bonds

Unused Tax Breaks In Uncertain Times

One of the biggest question marks in the federal aid effort is the \$5.5
billion Liberty Zone tax package intended to stimulate employment and
the construction of new office towers, residential buildings and retail
shops in Lower Manhattan.

The \$5.5 billion sum, which sounded a lot like cash to anxious New
Yorkers back in March, is the value over time of the tax breaks accorded
to employers and construction projects that contractors and
entrepreneurs might undertake. It was not, and will never be, an
infusion of cash.

Now, some state and city officials believe that the package was an
ill-conceived device for addressing the damage to the city's economy.
They say that the usefulness of the aid is dependent upon the
willingness of people to invest in office buildings and other business
opportunities at a moment when the future of downtown, and especially
the market for office space, is seriously clouded.

To date, there are only four projects in line for a total of \$493
million in tax-exempt Liberty Bonds, and three of them are residential
complexes. State and city officials, as well as many real estate and
business experts, say the tax enticements could not possibly be used by
the end of 2004, the deadline established by Congress.

''The jury is out on how much of this will be used and how effective it
will be,'' Mr. Schumer said.

Even estimating the total potential value of the package has proved
difficult. The bond plan was trumpeted as a \$5.5 billion funding
package when it was announced, but the O.M.B. now puts its value at \$5
billion. And a report done for the city by the accounting firm
PriceWaterhouseCoopers said that the entire package of tax incentives
was worth just \$3.8 billion.

''Our number's a more realistic estimate,'' one New York government
official said. ''But we don't want to appear ungrateful.''

In the weeks after the attacks, Liberty Bonds emerged as a
reconstruction tool in meetings between the White House, elected
officials, the New York City Partnership and the Real Estate Board of
New York. Those involved assumed that the bonds could be used to replace
much of an estimated 25 million square feet of office space that had
been damaged or destroyed.

The economic stimulus package provides tax credits of \$2,400 per
employee in 2002 and 2003 for every business south of Canal Street with
fewer than 200 employees. One provision allows the city, the state and
other agencies to refinance certain debts, while another grants
accelerated tax benefits for investment in new office space, equipment
and technology.

Finally, the legislation allows the city and the state to issue up to
\$8 billion in tax-free bonds for developers of office buildings,
housing and retail projects below Canal Street. Of that, up to \$2
billion in bonds can be used for commercial, but not residential,
projects outside Lower Manhattan.

But clearly, the greatest concerns about what the aid will mean in real
terms involves the city's current economic climate.

With the economy weak and layoffs sweeping Wall Street, there is little
demand for new office towers in Lower Manhattan. Indeed, there is more
vacant space in Manhattan now -\/- 46.4 million square feet -\/- than
there is office space in all of San Francisco. That has led some
officials to question the value of the Liberty Zone benefits as tools
for rapidly reviving the downtown economy.

''No one can argue that hard cash for our immediate needs is much better
than tax relief that might come later in an amount that might never be
quantified,'' said Mrs. Maloney. ''Lower Manhattan is economically still
reeling from Sept. 11. We need help now.''

The city and state have so far approved the use of the tax-exempt bonds
for three residential towers in Lower Manhattan and an office building
at Atlantic Center in Brooklyn for the Bank of New York. But in no case
has the bond financing been completed, and experts say there is little
current appetite for applying for more.

State and city officials say it is clear that they will not be able to
use most of the bonds by the 2004 deadline. Recognizing that few
commercial projects will qualify, city officials have been working with
Mrs. Clinton to see if some of the bonds could be used for residential
projects outside Lower Manhattan.

City and state officials say that they will almost certainly ask
Congress to extend the deadline beyond 2004. ''If demand perks up, we'll
push for an extension of the deadline past 2004,'' said a senior state
official. ''Otherwise, we'll let it expire and ask them to give us some
more cash.''

The Grants

Many Are Offered, But Few Are Taken

In devising the aid package for New York, legislators in Washington and
rebuilding officials in the city found a way to channel nearly \$4
billion through the federal Department of Housing and Urban Development
and the Small Business Administration. The housing agency has the power
to make block grants of money to address needs of particular urgency if
existing conditions pose a serious and immediate threat to a community's
health and welfare.

Thus empowered, officials decided that \$3.66 billion in grants of
various kinds would flow through the two agencies. But only about \$690
million, or less than one-fifth of the total, has been paid to downtown
businesses and residents.

The most successful part of that financing effort has distributed \$401
million to thousands of businesses large and small, giving some an
opportunity they otherwise would not have had: the ability to survive in
a precarious economic environment. And some \$41 million has been used
to make grants of up to \$14,500 to any family willing to remain in
Lower Manhattan for two years. That has at least played some role in
helping return occupancy rates in residential buildings around ground
zero to what they were before the attack.

But there are a variety of reasons why so much of the allocated money
has been unused. Many small businesses felt the application process was
too burdensome and confusing, and effectively gave up applying. Other
businesses and residents discovered that some of the grants were subject
to federal taxes and opted not to pursue the more-diminished amounts of
money.

Further, some businesses found that because they were just outside the
designated disaster zone, they were ineligible for money that many
elected officials in New York believe was clearly intended to help them.

In Greenwich Village, for example, tourism dropped drastically. In
Chinatown, restrictions on truck movements halted deliveries to many
businesses, particularly garment factories. Businesses in much of those
neighborhoods were entitled to grants totaling only a few days'
receipts.

A second small-business program, the Small Firm Attraction and Retention
Grant program, is scheduled to distribute up to \$291 million, but only
\$18.6 million has been distributed so far. Officials from the Empire
State Development Corporation say they are trying to promote the program
more to address a lack of applicants.

And even the \$41 million used to entice people to stay in Lower
Manhattan amounts to a fraction of the money that was allocated for that
effort. In all, officials had set aside \$280 million for the retention
program. But only 8,500 grants have been approved among 28,000
applications that were submitted as of early December. So far, only some
2,000 residents of the 50,000 eligible have actually received the
grants.

Charles A. Gargano, the chairman of the Empire State Development
Corporation, said so little of the money for attracting companies has
reached its intended targets because ''we are in an economic situation
where lots of companies are not making moves and are tightening their
belts.'' However, he added, ''the programs have been successful in
helping a substantial number of businesses.''

Rebuilding officials have had an easier time reaching out to big
companies. A category of grants intended to benefit companies with more
than 200 employees pays cash for a guarantee that a company will move to
or remain in Lower Manhattan. That program, with a budget of \$320
million, has so far paid 63 grants totaling \$215 million.

Yet that program, too, has come under fire from fiscal watchdogs who
question the need to pay grants to institutions -\/- like Pace
University and the American Stock Exchange -\/- that were at small risk
of leaving Lower Manhattan. The program has also had limited success in
attracting new companies to downtown.

The distribution of HUD money has been complicated for months by the
decision by Congress to use two different agencies to deliver it.
Congress gave responsibility for \$700 million in aid to Empire State
Development, the economic development arm of the Pataki administration.
It then opted to direct \$2.7 billion through the Lower Manhattan
Development Corporation, the special entity set up by Mr. Pataki and
former Mayor Rudolph W. Giuliani to oversee the rebuilding effort.

When Empire State Development requested some \$800 million in additional
money to continue its grant programs and retention efforts, it had to go
through the Lower Manhattan organization, which balked and turned over
only \$350 million. It then angered its counterpart by asking for a full
accounting of the money Empire State had already handed out.

As it stands, \$1.3 billion of the \$2.7 billion given to the Lower
Manhattan agency still remains unspent and unallocated. Various people
have their eyes on the money, not least Mr. Bloomberg, who laid out an
ambitious plan to build housing, revitalize business districts and
improve transportation in Lower Manhattan. The mayor said he intended to
draw on many sources for that money, among them the development
corporation.

The Wish List

A Grand Vision And a Hard Reality

The greatest chunk of hard money Washington promised to New York was
meant to cover the costs of rebuilding, and perhaps vastly improving,
the transportation hub downtown. While most everyone understood the
money would come only after perhaps years of planning and preparation,
it was viewed as one of the most vital components of aid if New York was
to prosper.

But in the end, the package of transportation aid pledged was barely
half what local officials desired.

For decades, downtown property owners and economic development officials
had argued that the area suffered from a shortcoming that afflicted none
of the country's other large central business districts: the lack of a
rail connection to the region's major airports. Some also wanted better
connections to the suburbs to the east and north enjoyed by Midtown. And
there was talk of a grand point of arrival to house the newly connected
transportation systems.

Thus, state and city officials went to Washington with a transportation
wish list including those and nine other projects, with a total price
tag of \$7.5 billion. But federal officials balked at some of the
efforts, like a new ferry terminal in Hoboken, N.J., that clearly had
little relationship to Sept. 11. Other projects appeared to overlap: the
Metropolitan Transportation Authority and the Port Authority of New York
and New Jersey wanted to build a point of arrival, but their proposed
stations were two blocks apart.

A compromise totaling \$4.5 billion was reached, but left out some
projects on the wish list. Officials in Washington left it to New York
to decide which ones. The money was a mix of nearly \$3 billion from
FEMA and roughly \$2 billion more issued as part of a federal Department
of Transportation grant.

''You figure out what is best and we are going to help fund that for
you,'' Michael Brown, the deputy director of FEMA, told New York
officials last summer.

Washington is still waiting. In October, Mr. Pataki sent a list of six
projects, but no cost projections, to FEMA and the Federal Transit
Administration, which will oversee the transit financing.

''We need further clarification defining the scope of the projects and
the dollar amounts attached to each,'' said Kristi M. Clemens, a
spokeswoman for the transit administration.

Rebuilding officials say they are nearing an agreement, and promise that
a full transportation plan will be offered early next year.

''The question now is,'' said Mr. Schumer, ''are we going to think in a
large-scale way that will change the face of downtown, or will we allow
the money to be cannibalized into a lot of worthy but smaller-scale
projects that ultimately won't change downtown?''

That the money will eventually reach New York may be inevitable, but its
delayed arrival underscores how the federal government's emergency
commitment to New York is in fact an extended work in progress.

First Aid

More on Pages B4-B5:

The Agency

The Bonds

The Grants

The Wish List

Advertisement

\protect\hyperlink{after-bottom}{Continue reading the main story}

\hypertarget{site-index}{%
\subsection{Site Index}\label{site-index}}

\hypertarget{site-information-navigation}{%
\subsection{Site Information
Navigation}\label{site-information-navigation}}

\begin{itemize}
\tightlist
\item
  \href{https://help.nytimes3xbfgragh.onion/hc/en-us/articles/115014792127-Copyright-notice}{©~2020~The
  New York Times Company}
\end{itemize}

\begin{itemize}
\tightlist
\item
  \href{https://www.nytco.com/}{NYTCo}
\item
  \href{https://help.nytimes3xbfgragh.onion/hc/en-us/articles/115015385887-Contact-Us}{Contact
  Us}
\item
  \href{https://www.nytco.com/careers/}{Work with us}
\item
  \href{https://nytmediakit.com/}{Advertise}
\item
  \href{http://www.tbrandstudio.com/}{T Brand Studio}
\item
  \href{https://www.nytimes3xbfgragh.onion/privacy/cookie-policy\#how-do-i-manage-trackers}{Your
  Ad Choices}
\item
  \href{https://www.nytimes3xbfgragh.onion/privacy}{Privacy}
\item
  \href{https://help.nytimes3xbfgragh.onion/hc/en-us/articles/115014893428-Terms-of-service}{Terms
  of Service}
\item
  \href{https://help.nytimes3xbfgragh.onion/hc/en-us/articles/115014893968-Terms-of-sale}{Terms
  of Sale}
\item
  \href{https://spiderbites.nytimes3xbfgragh.onion}{Site Map}
\item
  \href{https://help.nytimes3xbfgragh.onion/hc/en-us}{Help}
\item
  \href{https://www.nytimes3xbfgragh.onion/subscription?campaignId=37WXW}{Subscriptions}
\end{itemize}
