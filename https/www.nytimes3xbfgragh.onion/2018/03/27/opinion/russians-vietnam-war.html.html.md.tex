Sections

SEARCH

\protect\hyperlink{site-content}{Skip to
content}\protect\hyperlink{site-index}{Skip to site index}

\href{https://myaccount.nytimes3xbfgragh.onion/auth/login?response_type=cookie\&client_id=vi}{}

\href{https://www.nytimes3xbfgragh.onion/section/todayspaper}{Today's
Paper}

\href{/section/opinion}{Opinion}\textbar{}Why Were the Russians in
Vietnam?

\url{https://nyti.ms/2GivpaI}

\begin{itemize}
\item
\item
\item
\item
\item
\end{itemize}

Advertisement

\protect\hyperlink{after-top}{Continue reading the main story}

Supported by

\protect\hyperlink{after-sponsor}{Continue reading the main story}

\href{/section/opinion}{Opinion}

\href{/column/vietnam-67}{Vietnam '67}

\hypertarget{why-were-the-russians-in-vietnam}{%
\section{Why Were the Russians in
Vietnam?}\label{why-were-the-russians-in-vietnam}}

By Sergey Radchenko

\begin{itemize}
\item
  March 27, 2018
\item
  \begin{itemize}
  \item
  \item
  \item
  \item
  \item
  \end{itemize}
\end{itemize}

\includegraphics{https://static01.graylady3jvrrxbe.onion/images/2018/03/27/opinion/27Vietnam-Radchenko/27Vietnam-Radchenko-articleLarge.jpg?quality=75\&auto=webp\&disable=upscale}

We know today why it took the United States so long to disengage from
Vietnam: Leaving meant showing weakness in the face of the global
Communist menace, prompting a backlash at home and the loss of America's
credibility among allies.

But if America's involvement is well understood, the same cannot be said
of its superpower nemesis, the Soviet Union. What did the Russians stand
to gain from backing a remote jungle war, sending advisers, matériel and
money to help the North Vietnamese, even when doing so not only put
Soviet-American relations on ice but also risked causing a global
conflagration?

Was it Vietnam's geopolitical importance? Or, perhaps, Moscow's
preoccupation with spreading revolutionary ideology? We often attribute
to the other side more foresight and purpose than we allow our own.
There were in fact remarkable parallels between American and Soviet
involvement in Vietnam. Like the United States, what Moscow was most
concerned about was its credibility as an ally and a superpower and the
domestic and international legitimacy that such credibility afforded.

Nikita Khrushchev, who pioneered the Soviet pivot to the third world in
the 1950s, had limited interest in and little patience with the North
Vietnamese, whom he eyed with suspicion, especially after Hanoi began to
tilt visibly to China's side in the unfolding Sino-Soviet split.

North Vietnam's siding with China was a tactical move in the absence of
better options. Khrushchev himself precipitated this shift by refusing
to provide aid. But he blamed his loss of North Vietnam on the imaginary
machinations of ``Chinese half-breeds'' in the Vietnamese party
leadership. For Khrushchev, the problem of Vietnam was only an aspect of
his broader struggle with China, and a rather peripheral aspect at that.

All of that changed when Khrushchev was ousted by his colleagues in a
palace coup in October 1964. His successors, Leonid Brezhnev and Alexei
Kosygin, wanted to prove that they were truly committed to an ally in
need by providing military aid. The underlying rationale was that the
new Soviet leadership faced a deficit of political legitimacy. Aiding
Vietnam in a war against ``imperialism'' helped them in being recognized
--- by their people, their clients and allies and the broader world ---
as the legitimate heirs to the leadership of the socialist camp. For the
same reason, Moscow attempted to improve relations with China.

Mao Zedong, however, was not inclined to reciprocate. This became clear
during Kosygin's February 1965 trip to Beijing. Kosygin, the Soviet
premier, spoke of the need for ``united action'' to help Hanoi's war
effort. Mao responded to his pleas with hostile sarcasm, telling Kosygin
that the Sino-Soviet struggle would last for 10,000 years. ``The U.S.
and the U.S.S.R. are now deciding the world's destiny,'' Mao said
acidly. ``Well, go ahead and decide.'' He appeared unconcerned by the
new round of escalation in Vietnam: ``So what? What is horrible about
the fact that some number of people died?'' --- and countered Kosygin's
worries about the deepening conflict with optimistic calls for a
``revolutionary war.''

Even as Moscow's relations with China continued to deteriorate, Hanoi
moved away from a pro-Chinese stand to a semblance of neutrality. That
was because the North Vietnamese needed the Soviet weapons, especially
advanced antiaircraft missiles, to protect themselves against American
bombing. But the Chinese Cultural Revolution also helped. The Vietnamese
leaders resented Beijing's effort to stir up radicalism among the
sizable community of Chinese living in North Vietnam. ``As paradoxical
as it sounds,'' remarked one Politburo member, Nguyen Van Vinh, in 1967,
at the height of American involvement, the Vietnamese ``do not fear the
Americans but fear the Chinese comrades.''

The tensions between Beijing and Hanoi became much more pronounced in
1971, after Henry Kissinger's secret trip to China and the announcement
of Nixon's impending visit. The North Vietnamese were not consulted and
felt betrayed. But there was an even bigger problem: The Chinese and the
Vietnamese had very different ideas about their relative importance. The
Chinese leaders regarded the North Vietnamese as underlings. They helped
them. They instructed them. But they expected deference in return. The
Vietnamese refused to defer. After years of fighting against the United
States, they felt entitled to claim revolutionary leadership, at least
in Southeast Asia.

This was the message that Gen. Vo Nguyen Giap brought to Moscow in
December 1971 just as the Vietnamese geared up for the Spring Offensive
to deal the final blow to South Vietnam. Giap promised that the joint
Soviet-Vietnamese victory in Vietnam would herald Hanoi's rise to ranks
of the leader of, and the socialist bridgehead to the third world. ``We
would like to carry on this mission together with the Soviet Union,
because no one can do it without the Soviet Union,'' he said. The Soviet
leaders embraced the message, especially after Giap promised to allow
the Soviet Union naval rights in then-still American-controlled Cam Ranh
Bay.

There were dangers in supporting Hanoi's militant attitude. The restart
of major fighting in March 1972 threatened to derail the move toward the
Soviet-American détente. After the Americans responded to Hanoi's Spring
Offensive by massive bombing raids, several in the Soviet leadership,
including Kosygin, proposed to cancel the upcoming summit in Moscow.
``Are you kidding?,'' Brezhnev asked. ``Why not!,'' Kosygin countered.
``This would be the right kind of a bomb.'' ``It'll be a bomb all
right,'' Brezhnev commented, ``but who will it affect more?''

Brezhnev regarded détente as a personal achievement and was not willing
to sacrifice it for the sake of Vietnam. At the same time, though, he
was also unwilling to pressure Vietnam for the sake of a better
relationship with the United States, the idea known to Kissinger and
Nixon as ``linkage.'' What the American duo did not quite understand was
that Vietnam was an important element of Brezhnev's bid for global
leadership. Soviet support for Hanoi was what made the Soviet Union a
true superpower and America's equal.

Nixon later recalled being taken aback during the May 1972 Moscow Summit
when Brezhnev, ``who had just been laughing and slapping me on the back,
started shouting angrily,'' accusing the United States of committing
terrible crimes in Vietnam. Brezhnev did it because he had to defend his
credibility in front of his own colleagues and also the North
Vietnamese. ``I don't remember that I or my comrades ever had to speak
to anyone so sharply and so harshly as we spoke to Nixon about
Vietnam,'' Brezhnev later told General Secretary Le Duan and Prime
Minister Pham Van Dong.

Sino-Vietnamese relations at that time had reached new lows. By the
summer of 1973, as the United States was completing its disengagement,
Le Duan worried about China, telling Brezhnev that he thought Mao
planned ``to invade all of Indochina and Southeast Asia if the
circumstances were right.'' Brezhnev promised to help defend Vietnam ---
this time against its northern neighbor.

The costs of postwar reconstruction were enormous. Le Duan and Pham Van
Dong were upfront with Brezhnev about Hanoi's expectations: There would
have to be a major Soviet aid effort to help ``industrialize'' Vietnam
in order to show Southeast Asia the practical benefits of socialist
orientation. ``We have nothing,'' Le Duan told Brezhnev, suggesting that
everything would have to come from the Soviet bloc for the next 10 to 15
years.

Brezhnev agreed to cancel all of Hanoi's debts. Credits kept coming,
though, and by 1990 Vietnam had received more than \$11 billion dollars
in aid, most of which was never repaid. Subsidizing Vietnam became a
serious burden on the Soviet economy in 1980s, contributing to Moscow's
insolvency.

The Vietnam War ended with the Soviet-Vietnamese victory but, for Moscow
at least, it was a Pyrrhic victory. Maintaining clients was good for
one's credibility as a superpower and for the leaders' political
legitimacy, but it wasn't good for the state budget. Russia's policy in
recent years, in particular its operations in Syria, is reminiscent of
the Cold War-era quest for legitimacy in Vietnam. The long-term
consequences of this renewed quest will be equally dire.

Advertisement

\protect\hyperlink{after-bottom}{Continue reading the main story}

\hypertarget{site-index}{%
\subsection{Site Index}\label{site-index}}

\hypertarget{site-information-navigation}{%
\subsection{Site Information
Navigation}\label{site-information-navigation}}

\begin{itemize}
\tightlist
\item
  \href{https://help.nytimes3xbfgragh.onion/hc/en-us/articles/115014792127-Copyright-notice}{©~2020~The
  New York Times Company}
\end{itemize}

\begin{itemize}
\tightlist
\item
  \href{https://www.nytco.com/}{NYTCo}
\item
  \href{https://help.nytimes3xbfgragh.onion/hc/en-us/articles/115015385887-Contact-Us}{Contact
  Us}
\item
  \href{https://www.nytco.com/careers/}{Work with us}
\item
  \href{https://nytmediakit.com/}{Advertise}
\item
  \href{http://www.tbrandstudio.com/}{T Brand Studio}
\item
  \href{https://www.nytimes3xbfgragh.onion/privacy/cookie-policy\#how-do-i-manage-trackers}{Your
  Ad Choices}
\item
  \href{https://www.nytimes3xbfgragh.onion/privacy}{Privacy}
\item
  \href{https://help.nytimes3xbfgragh.onion/hc/en-us/articles/115014893428-Terms-of-service}{Terms
  of Service}
\item
  \href{https://help.nytimes3xbfgragh.onion/hc/en-us/articles/115014893968-Terms-of-sale}{Terms
  of Sale}
\item
  \href{https://spiderbites.nytimes3xbfgragh.onion}{Site Map}
\item
  \href{https://help.nytimes3xbfgragh.onion/hc/en-us}{Help}
\item
  \href{https://www.nytimes3xbfgragh.onion/subscription?campaignId=37WXW}{Subscriptions}
\end{itemize}
