Sections

SEARCH

\protect\hyperlink{site-content}{Skip to
content}\protect\hyperlink{site-index}{Skip to site index}

\href{https://www.nytimes3xbfgragh.onion/section/us}{U.S.}

\href{https://myaccount.nytimes3xbfgragh.onion/auth/login?response_type=cookie\&client_id=vi}{}

\href{https://www.nytimes3xbfgragh.onion/section/todayspaper}{Today's
Paper}

\href{/section/us}{U.S.}\textbar{}Ex-Playboy Model Karen McDougal Sues
to Speak on Alleged Trump Affair

\url{https://nyti.ms/2ucW7jR}

\begin{itemize}
\item
\item
\item
\item
\item
\item
\end{itemize}

Advertisement

\protect\hyperlink{after-top}{Continue reading the main story}

Supported by

\protect\hyperlink{after-sponsor}{Continue reading the main story}

\hypertarget{ex-playboy-model-karen-mcdougal-sues-to-speak-on-alleged-trump-affair}{%
\section{Ex-Playboy Model Karen McDougal Sues to Speak on Alleged Trump
Affair}\label{ex-playboy-model-karen-mcdougal-sues-to-speak-on-alleged-trump-affair}}

\includegraphics{https://static01.graylady3jvrrxbe.onion/images/2018/03/21/autossell/21lawsuit1/20lawsuit1-articleLarge.jpg?quality=75\&auto=webp\&disable=upscale}

By \href{https://www.nytimes3xbfgragh.onion/by/jim-rutenberg}{Jim
Rutenberg} and
\href{https://www.nytimes3xbfgragh.onion/by/rebecca-r-ruiz}{Rebecca R.
Ruiz}

\begin{itemize}
\item
  March 20, 2018
\item
  \begin{itemize}
  \item
  \item
  \item
  \item
  \item
  \item
  \end{itemize}
\end{itemize}

A former Playboy model who claimed she had an affair with Donald J.
Trump sued on Tuesday to be released from a 2016 legal agreement
restricting her ability to speak, becoming the second woman this month
to challenge Trump allies' efforts during the presidential campaign to
bury stories about extramarital relationships.

\emph{(Read the}
\href{https://static01.graylady3jvrrxbe.onion/files/2018/us/mcdougal-complaint-exhibits.pdf?action=click\&module=Intentional\&pgtype=Article}{\emph{complaint}}\emph{.)}

The model,
\href{https://www.nytimes3xbfgragh.onion/2019/12/05/us/fox-news-mcdougal.html}{Karen
McDougal}, is suing The National Enquirer's parent company, which paid
her \$150,000 and whose chief executive is a friend of President
Trump's. The other woman, the adult entertainment star
\href{https://www.nytimes3xbfgragh.onion/2018/01/12/us/trump-stephanie-clifford-stormy-daniels.html}{Stephanie
Clifford}, better known as Stormy Daniels, was paid \$130,000 to stay
quiet by the president's personal lawyer, Michael D. Cohen. She filed a
lawsuit earlier this month.

Both women, who argue that their contracts are invalid, are trying to
get around clauses requiring them to resolve disputes in secretive
arbitration proceedings rather than in open court. Mr. Trump has denied
the affairs, which both women have described as consensual.

In other potential legal trouble for the president, a Manhattan judge on
Tuesday denied a move by Mr. Trump's lawyers to block a defamation suit
from Summer Zervos, a former ``Apprentice'' contestant. She accused Mr.
Trump of sexually harassing her after she appeared on his reality show.
He has called her and over a dozen other women who accused him of
harassment ``liars.''

The decision can allow lawyers to subpoena documents and call on other
women to testify in depositions. The judge, Jennifer G. Schecter, cited
the Paula Jones harassment case that led to Bill Clinton's impeachment
and said, ``It is settled that the president of the United States has no
immunity and is `subject to the laws' for purely private acts.''

Ms. McDougal, in a lawsuit filed in Los Angeles Superior Court, claims
that Mr. Cohen was secretly involved in her talks with the tabloid
company, American Media Inc., and that A.M.I. and her lawyer at the time
misled her about the deal. She also asserts that after she spoke last
month with
\href{https://www.newyorker.com/news/news-desk/donald-trump-a-playboy-model-and-a-system-for-concealing-infidelity-national-enquirer-karen-mcdougal}{The
New Yorker}, which obtained notes she kept on Mr. Trump, A.M.I. warned
that ``any further disclosures would breach Karen's contract'' and
``cause considerable monetary damages.''

In an email to The New York Times, her new lawyer, Peter K. Stris,
accused A.M.I. of ``a multifaceted effort to silence Karen McDougal.''

``The lawsuit filed today aims to restore her right to her own voice,''
he said, adding, ``We intend to invalidate the so-called contract that
American Media Inc. imposed on Karen so she can move forward with the
private life she deserves.''

Shortly after Ms. McDougal filed her suit, CNN announced that she would
sit for an interview with Anderson Cooper on Thursday. Ms. Clifford was
to appear on ``60 Minutes'' three days later to discuss her relationship
with Mr. Trump and the efforts Mr. Cohen undertook on his client's
behalf to pay for her silence.

Mr. Trump joined a legal effort last week seeking some
\href{https://www.nytimes3xbfgragh.onion/2018/03/16/us/trumps-stormy-daniels-lawsuit.html}{\$20
million} in penalties tied to Ms. Clifford's agreement.

\includegraphics{https://static01.graylady3jvrrxbe.onion/images/2018/03/08/us/20lawsuit2/merlin_135212859_67b6f89c-1e4e-48f2-a1c9-dda1af2eb065-articleLarge.jpg?quality=75\&auto=webp\&disable=upscale}

The court dispute has drawn public attention to an issue that was
previously sidelined. And both women's suits could provide more fodder
for federal complaints from the watchdog group
\href{http://www.commoncause.org/press/press-releases/trump-attorney-appears-to-have-made-illegal-contributions-to-trump-camoaign-when-he-made-alleged-hush-money-payment-to-stormy-daniels.html}{Common
Cause} that the payoffs were, effectively, illegal campaign
contributions.

Ms. Clifford and Ms. McDougal tell strikingly similar stories about
their experiences with Mr. Trump, which included alleged trysts at the
same Lake Tahoe golf tournament in 2006, dates at the same Beverly Hills
hotel and promises of apartments as gifts. Their stories first surfaced
in the
\href{https://www.wsj.com/articles/national-enquirer-shielded-donald-trump-from-playboy-models-affair-allegation-1478309380}{The
Wall Street Journal} four days before the election, but got little
traction in the swirl of news that followed Mr. Trump's victory. The
women even shared the same Los Angeles lawyer, Keith Davidson, who has
long worked for clients who sell their stories to the tabloids.

Ms. McDougal negotiated with the country's leading tabloid news
provider, A.M.I., which is known to buy and bury stories that might
damage allies of its chief executive, David J. Pecker --- a practice
known as ``catch and kill.''

Ms. McDougal's legal complaint alleges that she did not know about the
practice, or about Mr. Pecker's friendship with Mr. Trump, when she
began talking to company representatives, shortly after Mr. Trump locked
up the Republican nomination.

A.M.I. has previously acknowledged that Mr. Trump had been friends with
Mr. Pecker, but said that he had never tried to influence coverage at
the company's publications.

In a statement on Tuesday, A.M.I. said that its contract with Ms.
McDougal was valid and that it looked forward ``to reaching an amicable
resolution.'' It added that while she had given the company ``editorial
discretion to publish her life story,'' she had been ``free to respond
to press inquiries about her relationship with President Trump since
2016.''

A.M.I. amended her contract after the election, allowing her to answer
``legitimate'' questions from the press. But the lawsuit contends that
the parameters were unclear to her, and her lawyers argue that A.M.I.
can continue to control her responses.

Ms. McDougal has said that she was ambivalent about selling her story on
the tabloid news market, but felt that her hand was forced after a hint
of the alleged affair appeared in May 2016 on social media. Convinced
something more would come out, she was determined to tell her story on
her terms, her suit says.

A mutual friend connected her to Mr. Davidson, who, she said, told her
the story could be worth millions. He arranged an interview with Dylan
Howard, A.M.I.'s chief content officer, in Los Angeles. Mr. Davidson
told her before the interview that A.M.I. would put \$500,000 in an
escrow account for her, and that ``a seven-figure publishing contract
awaited her,'' the complaint reads.

Mr. Howard spent several hours pressing Ms. McDougal on the details of
her story. But several days later, the media company declined to buy it,
the complaint reads, and ``Mr. Davidson revealed that, in fact, there
was no money in escrow.''

A spokesman for Mr. Davidson said on Tuesday that the lawyer ``fulfilled
his obligations and zealously advocated for Ms. McDougal to accomplish
her stated goals at that time,'' but that commenting further would
``violate attorney-client privilege.''

A.M.I. told The Times last month that it decided not to print Ms.
McDougal's story because it could not verify important details, though
it acknowledged discussing her allegations with Mr. Cohen, the
president's lawyer, saying it did so as part of its reporting process.

The tabloid company showed renewed interest in the story in summer 2016,
when Ms. McDougal began talks with ABC News. This time, A.M.I. offered a
different deal.

Mr. Davidson informed her that A.M.I. would buy her story but not
publish it because of Mr. Pecker's relationship with Mr. Trump, the suit
says. The payment would be \$150,000, with Mr. Davidson and others
involved on her behalf taking 45 percent. More alluring to Ms. McDougal,
who is now a fitness specialist, was that the media company would
feature her on its covers and in regular health and fitness columns, the
complaint says.

As A.M.I. and Mr. Davidson pushed her to sign the deal on Aug. 5, Ms.
McDougal expressed misgivings. But, her suit says, Mr. Davidson and Mr.
Howard argued in an urgent Skype call that the deal to promote her would
``kick start and revitalize'' her career, given that she was ``old
now.'' She was 45.

In all, they said, the contract would obligate A.M.I. to run more than
100 columns or articles and at least two covers featuring her. When she
asked Mr. Davidson what she should do if her story leaked, he responded
in an email, ``IF YOU DENY YOU ARE SAFE,'' and urged her to sign as soon
as possible, according to the court documents.

The Times
\href{https://www.nytimes3xbfgragh.onion/2018/02/18/us/politics/michael-cohen-trump.html}{reported
last month} that Mr. Davidson sent Mr. Cohen an email on Aug. 5, 2016,
asking him to call. Mr. Davidson then told Mr. Cohen over the phone that
the deal had been completed, according to a person familiar with the
conversation.

The timeline provided in the lawsuit shows that Mr. Davidson's email
came as he and A.M.I. were still hashing out the terms of the deal,
which Ms. McDougal did not sign until the following day, Aug. 6. Mr.
Cohen told The Times last month that he did not recall the
communications.

After signing the contract, Ms. McDougal grew frustrated when she did
not hear about columns or cover shoots for several weeks. She later
figured out why. Though the agreement explicitly mentioned ``a monthly
column'' on aging and fitness for OK! and Star, and ``four posts each
month'' on Radar Online, it only gave A.M.I. ``the right'' to print
them. It was not an obligation.

``She was tricked into signing it while being misled as to its contents
(including by her own lawyer, on whose advice she was entitled to
rely),'' the lawsuit reads.

So far, A.M.I. has run one cover and roughly two dozen columns or posts
featuring her. The company said that it had been trying to schedule a
photo shoot for another cover but implied that Ms. McDougal felt she had
not been paid enough.

Her legal team denied that she was seeking more money.

Mr. Stris contends that his client was misled and that the contract was
executed under fraudulent circumstances, giving her the right to sue in
court rather than proceed in arbitration.

Advertisement

\protect\hyperlink{after-bottom}{Continue reading the main story}

\hypertarget{site-index}{%
\subsection{Site Index}\label{site-index}}

\hypertarget{site-information-navigation}{%
\subsection{Site Information
Navigation}\label{site-information-navigation}}

\begin{itemize}
\tightlist
\item
  \href{https://help.nytimes3xbfgragh.onion/hc/en-us/articles/115014792127-Copyright-notice}{©~2020~The
  New York Times Company}
\end{itemize}

\begin{itemize}
\tightlist
\item
  \href{https://www.nytco.com/}{NYTCo}
\item
  \href{https://help.nytimes3xbfgragh.onion/hc/en-us/articles/115015385887-Contact-Us}{Contact
  Us}
\item
  \href{https://www.nytco.com/careers/}{Work with us}
\item
  \href{https://nytmediakit.com/}{Advertise}
\item
  \href{http://www.tbrandstudio.com/}{T Brand Studio}
\item
  \href{https://www.nytimes3xbfgragh.onion/privacy/cookie-policy\#how-do-i-manage-trackers}{Your
  Ad Choices}
\item
  \href{https://www.nytimes3xbfgragh.onion/privacy}{Privacy}
\item
  \href{https://help.nytimes3xbfgragh.onion/hc/en-us/articles/115014893428-Terms-of-service}{Terms
  of Service}
\item
  \href{https://help.nytimes3xbfgragh.onion/hc/en-us/articles/115014893968-Terms-of-sale}{Terms
  of Sale}
\item
  \href{https://spiderbites.nytimes3xbfgragh.onion}{Site Map}
\item
  \href{https://help.nytimes3xbfgragh.onion/hc/en-us}{Help}
\item
  \href{https://www.nytimes3xbfgragh.onion/subscription?campaignId=37WXW}{Subscriptions}
\end{itemize}
