Sections

SEARCH

\protect\hyperlink{site-content}{Skip to
content}\protect\hyperlink{site-index}{Skip to site index}

\href{https://myaccount.nytimes3xbfgragh.onion/auth/login?response_type=cookie\&client_id=vi}{}

\href{https://www.nytimes3xbfgragh.onion/section/todayspaper}{Today's
Paper}

\href{/section/opinion}{Opinion}\textbar{}The End of South Vietnam

\url{https://nyti.ms/2uxWLbz}

\begin{itemize}
\item
\item
\item
\item
\item
\item
\end{itemize}

Advertisement

\protect\hyperlink{after-top}{Continue reading the main story}

Supported by

\protect\hyperlink{after-sponsor}{Continue reading the main story}

\href{/section/opinion}{Opinion}

\href{/column/vietnam-67}{Vietnam '67}

\hypertarget{the-end-of-south-vietnam}{%
\section{The End of South Vietnam}\label{the-end-of-south-vietnam}}

By Dien Huynh

\begin{itemize}
\item
  March 30, 2018
\item
  \begin{itemize}
  \item
  \item
  \item
  \item
  \item
  \item
  \end{itemize}
\end{itemize}

\includegraphics{https://static01.graylady3jvrrxbe.onion/images/2018/03/30/opinion/30Vietnam-Huynhphoto/30Vietnam-Huynhphoto-articleLarge.jpg?quality=75\&auto=webp\&disable=upscale}

\textbf{Go Cong, South Vietnam --- 1970}

A few months into 10th grade, I began to find it hard to pay attention
to my homework. I was distracted by the radio and the news about the
war. My town was just south of Saigon, near the coast. I was so scared,
confused and disillusioned; I started thinking that when I finished my
high school diploma I would be 18 and would have no choice but to enlist
in the Army of the Republic of Vietnam, or face the prospect of being
drafted.

If I did it right, I would end up a first lieutenant --- not a great
prospect in a war where young, lower-ranking officers were dying at an
unbelievably high rate. I had seen many of my high school friends ahead
of me not make it through the first six months of combat after
graduating from the Thu Duc Military Academy.

By the middle of the school year, I decided to write a letter to the
principal asking to have my report cards and high school documents, so
that I could submit them to the Air Force academy. I wanted to become a
helicopter mechanic. After I passed the physical test and the written
entry exam, the Air Force issued me an acceptance paper good until the
end of the year. That winter was the last time I celebrated the New Year
with my family.

\textbf{Phu Cat --- 1972 to 1975}

In the summer of 1972, I received a diploma and was assigned to work on
CH-47 Chinooks, big twin-rotored cargo helicopters. I was sent to Phu
Cat Air Base in central Vietnam, which had been full of American troops
until their recent return to the United States. For the United States,
the war was ending; for us, it continued.

My unit's job was to support ground troops fighting North Vietnamese
Army units in the Central Highlands, near the border with Laos. On the
first day out to work on the runway, we had to carry M-16s and wear
helmets, because the N.V.A. was close enough to drop mortar shells on us
from nearby hills. The attacks happened every morning, while we were out
there getting the helicopters ready for their missions --- working under
fire like that became our daily routine. Even worse, sometimes I would
have to fly on the helicopters during the missions, often taking fire as
we flew close over the mountains.

After six months I got my first two-week vacation. I went home to visit
my family and had second thoughts about going back.

\textbf{Con Son Island and the Midway --- 1975}

On April 28, 1975, I went back to the air base after overextending my
four-day leave. Military police officers immediately handcuffed me and
locked me in a cell. I remembered that the last time I said goodbye to
my girlfriend, her oldest sister told me that whatever happened they
would try to go to the main gate of the base and wait for me there.

On April 29, around 2 a.m., as I was sleeping, military police officers
opened the door and threw four helicopter pilots into my cell. I asked
them what was going on. They said they got caught trying to steal
airplanes. I was puzzled. Then they told me that the war was fast coming
to an end, and we were losing: The N.V.A. had already taken over most of
the cities, and they were marching to Saigon, the capital.

Early the next morning, the police came in, opened the jail cells, gave
us all our IDs and left in a hurry. I thought about what my girlfriend's
sister had said, so I ran toward to the main gate. As I got closer to
the gate, I could see Air Force security and military police with tanks
and guns trying to prevent people outside from coming to the base. It
was like a war zone; they were shooting at each other. I dropped to the
ground to avoid being shot, then crawled and crouched to get to the
gate. The gunfire continued. People were running in all directions,
panicked and scared. I couldn't find my girlfriend or her family.

Around 11 a.m., a radio announcement came over the loudspeakers. Saigon
had fallen. Then things really got out of control. Soldiers ripped off
their uniforms, grabbing whatever civilian clothes they could find.
Intelligence personnel burned their documents. I saw overloaded
helicopters hover and crash, with people coming out bloodied, badly
injured. Smoke came from the runway. I didn't know what to do.

I ran toward the hangars and bumped into the pilot who was in the jail
cell with me the previous night. We ran down to the end of the runway
and jumped into an empty UH-1 ``Huey'' helicopter. As soon as we started
the engine, about 20 guys jumped on. The weight was too much; we could
barely hover at 10 feet. Fortunately there was another Huey parked
nearby, and half of us jumped in and took off. Airborne, we flew out to
the sea, like birds escaping a net. I would not set foot in Vietnam
again for almost 20 years.

We landed on an island called Con Son, where the South Vietnamese
government incarcerated political prisoners. There were some
high-ranking Air Force officials there; they had arrived the night
before from Saigon with their families. We refueled and made a plan. We
sent two of the several Hueys on the island to look for an American
aircraft carrier, which was stationed farther out in the South China
Sea.

While we waited, we met the newly freed prisoners; in exchange for some
money and jewelry, they fed us lunch with food they had grown on the
island. They had no idea what had just happened in Saigon.

A few hours later, the Hueys returned --- they had found the carrier.
There were Chinooks on the island as well, and we all boarded them and
flew east. I sat in the tail, with a headset, and listened in on the
radio. Vietcong operatives had found our transmissions, and a female
voice kept calling us, pleading with us to come back to Saigon, saying
that our families were waiting for us, that our country would take us
back with open arms and big rewards.

I could hear the pilots arguing in the cockpit. They had in fact left
their families behind. Suddenly I felt the helicopter turn around.

I screamed into the headset: ``No, no, we can never trust them! All
those words are just lies! We all will be caught and sent to jail! Maybe
executed!'' Finally they turned back around, and we tried to catch up
with our group.

A few minutes later, we arrived at the aircraft carrier Midway. As we
approached, I saw one Chinook and a couple of Hueys floating upside down
in the water near the ship. I thought some of us had missed the landing
and crashed.

The flight deck below us was crowded with people, planes and
helicopters. We had to circle for a while, but eventually the deck crew
allowed us to land. As soon as we were out, crewmen took away our
helmets, headphones and handguns, and directed the passengers off the
flight deck. Then a group of us helped them push the Chinook off the
side into the water to make room for more.

I watched it bob in the water. For a moment I felt lost. Then I turned
and followed the line of refugees filing into the ship, as the sun faded
on the horizon.

Advertisement

\protect\hyperlink{after-bottom}{Continue reading the main story}

\hypertarget{site-index}{%
\subsection{Site Index}\label{site-index}}

\hypertarget{site-information-navigation}{%
\subsection{Site Information
Navigation}\label{site-information-navigation}}

\begin{itemize}
\tightlist
\item
  \href{https://help.nytimes3xbfgragh.onion/hc/en-us/articles/115014792127-Copyright-notice}{©~2020~The
  New York Times Company}
\end{itemize}

\begin{itemize}
\tightlist
\item
  \href{https://www.nytco.com/}{NYTCo}
\item
  \href{https://help.nytimes3xbfgragh.onion/hc/en-us/articles/115015385887-Contact-Us}{Contact
  Us}
\item
  \href{https://www.nytco.com/careers/}{Work with us}
\item
  \href{https://nytmediakit.com/}{Advertise}
\item
  \href{http://www.tbrandstudio.com/}{T Brand Studio}
\item
  \href{https://www.nytimes3xbfgragh.onion/privacy/cookie-policy\#how-do-i-manage-trackers}{Your
  Ad Choices}
\item
  \href{https://www.nytimes3xbfgragh.onion/privacy}{Privacy}
\item
  \href{https://help.nytimes3xbfgragh.onion/hc/en-us/articles/115014893428-Terms-of-service}{Terms
  of Service}
\item
  \href{https://help.nytimes3xbfgragh.onion/hc/en-us/articles/115014893968-Terms-of-sale}{Terms
  of Sale}
\item
  \href{https://spiderbites.nytimes3xbfgragh.onion}{Site Map}
\item
  \href{https://help.nytimes3xbfgragh.onion/hc/en-us}{Help}
\item
  \href{https://www.nytimes3xbfgragh.onion/subscription?campaignId=37WXW}{Subscriptions}
\end{itemize}
