Sections

SEARCH

\protect\hyperlink{site-content}{Skip to
content}\protect\hyperlink{site-index}{Skip to site index}

\href{https://www.nytimes3xbfgragh.onion/section/food}{Food}

\href{https://myaccount.nytimes3xbfgragh.onion/auth/login?response_type=cookie\&client_id=vi}{}

\href{https://www.nytimes3xbfgragh.onion/section/todayspaper}{Today's
Paper}

\href{/section/food}{Food}\textbar{}For Casual Indian Restaurants, It's
Party Time

\url{https://nyti.ms/2tJhs43}

\begin{itemize}
\item
\item
\item
\item
\item
\item
\end{itemize}

Advertisement

\protect\hyperlink{after-top}{Continue reading the main story}

Supported by

\protect\hyperlink{after-sponsor}{Continue reading the main story}

Critic's Notebook

\hypertarget{for-casual-indian-restaurants-its-party-time}{%
\section{For Casual Indian Restaurants, It's Party
Time}\label{for-casual-indian-restaurants-its-party-time}}

\includegraphics{https://static01.graylady3jvrrxbe.onion/images/2018/03/14/dining/14INDIAN1sub2/merlin_134614173_ccc82fb5-b219-4c98-ac3e-808fe3f896bb-articleLarge.jpg?quality=75\&auto=webp\&disable=upscale}

By \href{http://www.nytimes3xbfgragh.onion/by/pete-wells}{Pete Wells}

\begin{itemize}
\item
  March 13, 2018
\item
  \begin{itemize}
  \item
  \item
  \item
  \item
  \item
  \item
  \end{itemize}
\end{itemize}

Some new restaurants you just know are going to be copied like Chanel
handbags. I was fairly sure, for instance, that New York's dining future
was full of backless stools, long waits, chicken-liver toasts and
precisely engineered pub burgers after one meal at
\href{https://www.nytimes3xbfgragh.onion/2006/01/25/dining/reviews/25rest.html}{the
Spotted Pig}.

I'll admit, though, that when
\href{https://www.nytimes3xbfgragh.onion/2015/09/09/dining/hungry-city-babu-ji-east-village.html}{Babu
Ji} showed up on Avenue B in the summer of 2015, I had no clue it would
turn out to be a template for Indian restaurants in the city. The
blanketing sweetness that covered up the spices in several dishes caused
me to write the place off. I should have looked more closely; I should
have asked myself why tables were so hard to come by. But I didn't. Babu
Ji closed less than two years later, after
\href{https://ny.eater.com/2017/3/2/14789116/babu-ji-closing-labor-lawsuits-threats}{two
wage-theft lawsuits} were filed against the owners, and that, I thought,
was the end of that.

\includegraphics{https://static01.graylady3jvrrxbe.onion/images/2018/03/14/dining/14Indian2sub/merlin_134781333_b4b5a7b2-6f0f-4e17-8826-4fff7dec6f0e-articleLarge.jpg?quality=75\&auto=webp\&disable=upscale}

So much for my second career as a clairvoyant: The lawsuits were
settled, and Babu Ji was resurrected at
\href{http://www.babujinyc.com/}{another address}, near Union Square.
Meanwhile, over the past year I've been eating, often very happily, in a
number of new Indian places in Manhattan that are casual, not too
expensive, and reminiscent of Babu Ji in one way or another. I think of
them --- \href{https://www.rahinyc.com/}{Rahi},
\href{http://badshahny.com/}{Badshah}, \href{http://oldmonknyc.com/}{Old
Monk} and \href{https://www.aroqanyc.com/}{aRoqa} --- as the Baby Ji
restaurants.

Some borrow menu items without, thankfully, emulating the original's
sweet tooth, which persists at its second location. What ties them
together, and what I think will be Babu Ji's legacy, has less to do with
cooking than with finding a style of casual, inexpensive dining that's
in tune with current sensibilities. They're learning to create
atmospheres and present cooking in ways that resonate with a modern
audience, the way
\href{https://www.nytimes3xbfgragh.onion/2013/04/03/dining/reviews/restaurant-review-hanjan-in-manhattan.html}{Hanjan}
and
\href{https://www.nytimes3xbfgragh.onion/2017/05/09/dining/atoboy-review-korean-restaurant-nyc.html}{Atoboy}
have done with Korean food or
\href{https://www.nytimes3xbfgragh.onion/2017/07/25/dining/atla-review-mexican-restaurant-noho.html}{Atla}
does with Mexican.

Image

Upstairs at Babu Ji, Bollywood musical numbers are projected on the
brick wall.Credit...Casey Kelbaugh for The New York Times

This is not an issue for the city's fancier Indian restaurants, such as
\href{https://www.nytimes3xbfgragh.onion/2016/05/25/dining/indian-accent-restaurant-review.html}{Indian
Accent},
\href{https://www.nytimes3xbfgragh.onion/2011/03/30/dining/reviews/30rest.html}{Junoon}
or
\href{https://www.nytimes3xbfgragh.onion/2010/08/04/dining/reviews/04rest.html}{Tamarind
Tribeca}. Manhattan has a long tradition of formal Indian restaurants,
often overseen by cooks who learned to present their country's cuisine
in a fine-dining idiom by hard-core training in India's extensive system
of hotel kitchens. At the other end of the scale are the no-frills
places where value makes style irrelevant. The middle ground, though, is
ripe for the Baby Ji rebellion.

In retrospect, it should have been obvious that the source of Babu Ji's
popularity is the way it mimics a big, informal dinner at a friend's
house. Its owners, Jessi Singh and his wife, Jennifer, project
gloriously hallucinatory Bollywood numbers on the wall. You're
encouraged to ``help yourself'' to beer from a refrigerator case in the
dining room. (Fear not, this will be recorded on your check.) On top of
the refrigerator is a stuffed peacock. Gallery-quality framed
photographic portraits of Indian men set the geographic theme without
falling into cliché. The food, overseen by Mr. Singh, is dressed for a
party: The kitchen is fond of tossing flowers, sprouts and other
garnishes on dishes that would otherwise be a study in browns.

None of the second-generation places go quite as hard as Babu Ji. But
there is Hindi hip-hop at Badshah, in Hell's Kitchen, along with a
spray-painted mural of tigers and skyscrapers by Carl Joseph Gabriel,
and drinks served in canning jars. At Rahi, in Greenwich Village,
surreal and cartoonlike figures by the street-art duo
\href{http://www.yokandsheryo.com/}{Yok and Sheryo} crawl along the
walls, and there's more high-energy work in the back from a New Delhi
gallery of emerging Indian artists. Flowers and sprouts are rampant.

Image

The restaurant aRoqa, in Chelsea, has the shadowy lighting and sleek
design of a cocktail lounge.Credit...Casey Kelbaugh for The New York
Times

Over in Chelsea, \href{https://www.aroqanyc.com/}{aRoqa} is the most
cosmopolitan of the bunch, with moody cocktail-bar lighting and a
swooping ceiling of bent wooden slats. The chef, Gaurav Anand, lightens
the mood by serving rice-and-corn cakes in the luggage compartment of a
tiny carrier tricycle. Dry ice makes an appearance, as do squeezable
syringes, for injecting various chutneys into molded drums of paneer.
Needless to say, there will be flowers.

Old Monk, which took over Babu Ji's original space in the East Village,
is decorated with a different set of photographic portraits of men. This
time they are monks from around Asia; one is taking a picture and
another is holding a smartphone to his ear. The beer fridge is gone, but
there is a long beer list, drawn from the more mainstream wing of the
craft-brewing movement, like Fat Tire, Flying Dog, etc. (The wine list
takes more chances.)

One of Babu Ji's more clever innovations is offering a \$62 fixed-price
package of dishes as a tasting menu. It's not a true tasting menu in the
style of, say,
\href{https://www.nytimes3xbfgragh.onion/2015/06/17/dining/restaurant-review-blanca-in-bushwick-brooklyn.html}{Blanca},
but the term has cachet with modern diners, who end up trusting the
kitchen to choose what turns out to be a well-rounded, traditional
family-style meal.

Image

At aRoqa, they try to inject fun into the proceedings by serving corn
cakes, called paddu, in the back of a tiny tricycle.Credit...Casey
Kelbaugh for The New York Times

Old Monk has kept this idea, in a \$55, four-course dinner called You're
in Good Hands. I didn't try it, because my head was turned by the rest
of Navjot Arora's menu: fine pork-stuffed Tibetan momos with a ferocious
garlic-chile sambal; tandoori lamb chops marinated in rum and ginger; a
slow-cooked dal of mixed lentils that is inspired by Sikh temple cooking
and is very delicious.

Badshah's chef, Charles Mani, used to cook at Babu Ji, and even claims
to have come up with its General Tso's Cauliflower, a spin on the
classic Chinese-Indian fried cauliflower in chile sauce. In his new job,
he calls it Badshah Cauliflower. I've eaten just one quick dinner at
Badshah so far, and while I was content with the Kashmiri-style goat
curry, I was less thrilled by the refrigerator-cold sauces spooned over
hot potato croquettes.

Image

Some restaurants evoke India through modern photography; others, like
Rahi, use work by contemporary artists.Credit...Casey Kelbaugh for The
New York Times

The most exciting food in this group, I think, belongs to Rahi. Chintan
Pandya, the chef, trained under chefs from the
\href{https://www.oberoihotels.com/}{Oberoi hotel group}, and comes to
Rahi from Junoon, where he was executive chef. The cooking isn't as
consistent from night to night as it should be, and Mr. Pandya can
sometimes follow his creative impulses right over the cliff; my initial
skepticism about tandoori lamb chops smeared with wasabi did not melt
away when I tasted it.

Image

The Tulsi Chicken appetizer at Rahi is coated with a fresh basil sauce.
Note the flowers.Credit...Casey Kelbaugh for The New York Times

More often, the flavors are vivid and unexpected. With a chaat of fried
artichoke hearts and edamame in a fruity and sour sauce of tamarind and
pomegranate molasses, Mr. Pandya showed that he could infuse non-Indian
ingredients with flavors that are very true to Indian cooking. There is
a captivating appetizer of dark-meat chicken in a basil-chile sauce
called Tulsi Chicken, and an inexplicably good snack of Melba toasts
under chopped shishito peppers mixed with melted Amul cheese, a
processed and highly shelf-stable product that's everywhere in India.
And I'm slightly in awe of his tandoori skate, a pristine hunk of fish
cooked so it just slides off its cartilage and coated with a yogurt
sauce so rife with cinnamon and cloves that it tastes like A.1. Sauce
that some gifted cook had improved almost beyond recognition.

Over the weekend, I went to a new place that in some respects fits right
in. \href{https://www.thebombaybreadbar.com/}{The Bombay Bread Bar} is a
quick conversion of
\href{https://www.nytimes3xbfgragh.onion/2018/02/15/dining/floyd-cardoz-paowalla-closing-bombay-bread-bar.html}{Floyd
Cardoz}'s SoHo restaurant Paowalla. I don't have the nerve to call it a
Baby Ji, though. Mr. Cardoz practically invented fun, casual,
inexpensive Indian dining years ago at the old Bread Bar, below Tabla,
and he brings some of his old tricks to his new place.

Image

At Rahi, skate wing on the bone is roasted in the tandoor with a yogurt
sauce spiked with cinnamon and cloves.Credit...Casey Kelbaugh for The
New York Times

But I can't help noticing that the menu is easier to scan; that the
cooking, as good as ever, has moved toward small, colorful plates; that
the prices stand firmly in the middle ground; and that the drab,
businesslike design of Paowalla has been engulfed by paper marigolds,
fruit-patterned oilcloths and a mural painted in comics style by the
Pakistani-raised Canadian artist
\href{https://www.thestar.com/entertainment/books/2017/08/01/they-gave-her-bad-advice-toronto-artist-maria-qamar-turned-it-into-a-book-trust-no-aunty.html}{Maria
Qamar}. I'm not quite sure what it depicts, but it looks like a pair of
Bollywood actors.

\textbf{aRoqa} 206 Ninth Avenue (West 23rd Street), Chelsea;
646-678-5471; \href{https://www.aroqanyc.com/}{aroqanyc.com}.

\textbf{Babu Ji} 22 East 13th Street (University Place), Greenwich
Village; 212-951-1082; \href{http://www.babujinyc.com/}{babujinyc.com}.

\textbf{Badshah} 788 Ninth Avenue (West 52nd Street), Hell's Kitchen;
646-649-2407; \href{http://badshahny.com/}{badshahny.com}.

\textbf{The Bombay Bread Bar} 195 Spring Street (Sullivan Street), SoHo;
212-235-1098;
\href{https://www.thebombaybreadbar.com/}{thebombaybreadbar.com}.

\textbf{Old Monk} 175 Avenue B (East 11th Street), East Village;
646-559-2922; \href{http://oldmonknyc.com/}{oldmonknyc.com}.

\textbf{Rahi} 60 Greenwich Avenue (Seventh Avenue), Greenwich Village;
212-373-8900; \href{https://www.rahinyc.com/}{rahinyc.com}.

\href{https://www.facebookcorewwwi.onion/nytfood/}{Follow NYT Food on
Facebook}, \href{https://instagram.com/nytfood}{Instagram},
\href{https://twitter.com/nytfood}{Twitter} and
\href{https://www.pinterest.com/nytfood/}{Pinterest}.
\href{https://www.nytimes3xbfgragh.onion/newsletters/cooking}{Get
regular updates from NYT Cooking, with recipe suggestions, cooking tips
and shopping advice}.

Advertisement

\protect\hyperlink{after-bottom}{Continue reading the main story}

\hypertarget{site-index}{%
\subsection{Site Index}\label{site-index}}

\hypertarget{site-information-navigation}{%
\subsection{Site Information
Navigation}\label{site-information-navigation}}

\begin{itemize}
\tightlist
\item
  \href{https://help.nytimes3xbfgragh.onion/hc/en-us/articles/115014792127-Copyright-notice}{©~2020~The
  New York Times Company}
\end{itemize}

\begin{itemize}
\tightlist
\item
  \href{https://www.nytco.com/}{NYTCo}
\item
  \href{https://help.nytimes3xbfgragh.onion/hc/en-us/articles/115015385887-Contact-Us}{Contact
  Us}
\item
  \href{https://www.nytco.com/careers/}{Work with us}
\item
  \href{https://nytmediakit.com/}{Advertise}
\item
  \href{http://www.tbrandstudio.com/}{T Brand Studio}
\item
  \href{https://www.nytimes3xbfgragh.onion/privacy/cookie-policy\#how-do-i-manage-trackers}{Your
  Ad Choices}
\item
  \href{https://www.nytimes3xbfgragh.onion/privacy}{Privacy}
\item
  \href{https://help.nytimes3xbfgragh.onion/hc/en-us/articles/115014893428-Terms-of-service}{Terms
  of Service}
\item
  \href{https://help.nytimes3xbfgragh.onion/hc/en-us/articles/115014893968-Terms-of-sale}{Terms
  of Sale}
\item
  \href{https://spiderbites.nytimes3xbfgragh.onion}{Site Map}
\item
  \href{https://help.nytimes3xbfgragh.onion/hc/en-us}{Help}
\item
  \href{https://www.nytimes3xbfgragh.onion/subscription?campaignId=37WXW}{Subscriptions}
\end{itemize}
