Sections

SEARCH

\protect\hyperlink{site-content}{Skip to
content}\protect\hyperlink{site-index}{Skip to site index}

\href{https://www.nytimes3xbfgragh.onion/section/opinion/sunday}{Sunday
Review}

\href{https://myaccount.nytimes3xbfgragh.onion/auth/login?response_type=cookie\&client_id=vi}{}

\href{https://www.nytimes3xbfgragh.onion/section/todayspaper}{Today's
Paper}

\href{/section/opinion/sunday}{Sunday Review}\textbar{}It's Hard to Be
Hungry on Spring Break

\url{https://nyti.ms/2FQQfy2}

\begin{itemize}
\item
\item
\item
\item
\item
\end{itemize}

Advertisement

\protect\hyperlink{after-top}{Continue reading the main story}

Supported by

\protect\hyperlink{after-sponsor}{Continue reading the main story}

\href{/section/opinion}{Opinion}

\href{/column/on-campus}{On Campus}

\hypertarget{its-hard-to-be-hungry-on-spring-break}{%
\section{It's Hard to Be Hungry on Spring
Break}\label{its-hard-to-be-hungry-on-spring-break}}

By Anthony Abraham Jack

\begin{itemize}
\item
  March 17, 2018
\item
  \begin{itemize}
  \item
  \item
  \item
  \item
  \item
  \end{itemize}
\end{itemize}

\includegraphics{https://static01.graylady3jvrrxbe.onion/images/2018/03/18/opinion/sunday/18jack/18jack-articleLarge.jpg?quality=75\&auto=webp\&disable=upscale}

The phrase ``spring break'' conjures up images of college students
lounging on beaches by day and hitting the clubs at night. Many students
do, in fact, travel from campus to far-flung places. On the assumption
that most students leave, schools generally shut down. But this
assumption is outdated, especially as colleges enroll a greater number
of academically talented students from poor families.

I met Valeria, an engaging sociology major from the Midwest, while
conducting a study on social class at elite colleges that included
white, black, Latino and Asian students. (Valeria is not her real name;
the terms of my research protocol require that I use pseudonyms for all
students.) In our conversations, she described one aspect of how it felt
to be a poor student on a rich campus: ``There's always famine during
spring break.''

This problem is more complicated and widespread than people realize.
Data I collected in 2016 on colleges that have adopted no-loan financial
aid policies, which is one way of measuring a school's commitment to
lower-income students, reveal that roughly one in four kept their
cafeterias open during spring break the same way they do when classes
are in session. At Harvard, where I teach, it was not until
\href{https://www.nytimes3xbfgragh.onion/2015/04/12/education/edlife/first-generation-students-unite.html}{2015}
that the administration opened the dining halls during the break (a
project I was involved in).

Some colleges, like Smith and Carleton, charge students additional fees
to stay on campus during this period. Now, a daily rate of \$10 or \$15
might not seem like much to some. But to many lower-income students, it
is substantial.

Spring break is a luxury that many students can't afford. In a sense,
though, it is one that many colleges make them buy anyway.

I faced this reality as a student at Amherst in the mid-2000s. Valentine
Hall, our only cafeteria, closed. I could not
\href{http://www.nytimes3xbfgragh.onion/2007/05/27/education/27grad.html}{afford}
to go home to Miami, the choice destination for many of my peers.
Instead, I foraged through campus job postings to pick up extra shifts
to pay off meals I put on my credit card.

Today's students face a similar fate. With no access to a kitchen in
which to cook or store food, Michael, a slim, reserved first-generation
college student, told me, ``I just go to Family Dollar to buy things
that I can microwave.'' Many students I spoke to rationed food they took
from the cafeteria, oftentimes with the help of sympathetic cafeteria
workers.

Spencer, the daughter of refugees, recounted how she ``stole food the
day before they closed everything --- I took a bunch of bread and things
that are not perishable.'' Michelle, reflective and thoughtful, imported
a strategy she used when she and her family were homeless in New York:
She found a soup kitchen near campus. With comedic seriousness, Arianna,
who has a Southern California vibe, told me, ``Spring break is the real
`Hunger Games.'\emph{''}

Colleges are not unaware of what cafeteria closings can do to students.
Some make special concessions for athletes, international students and
those with certain campus jobs. Lower-income students are often not one
of these protected classes. This institutional oversight brings about
\href{https://www.washingtonpost.com/local/education/for-the-poor-in-the-ivy-league-a-full-ride-isnt-always-what-they-imagined/2016/05/16/5f89972a-114d-11e6-81b4-581a5c4c42df_story.html?utm_term=.e90c56a36e1b}{real
pain}. Worse, the strategies students adopt in response amplify their
sense of isolation and difference. Miranda, her words heavy with
emotion, yelled: ``We don't have a kitchen. It's really frustrating.
What the hell are we supposed to do?''

At the 2016 \href{http://www.1vyg.org}{1vyG} conference, where
first-generation college students from the Ivy League came together,
Molly, a woman with a pixie haircut, stood up in a room of 200 people to
discuss the realities of being a poor college student. After pausing for
just a moment, she revealed how she made ends meet during spring break:
She increased her online dating activity to secure meals. Banking on men
paying for the first date, she felt that her best option was to use
Tinder as if it were OpenTable.

At elite colleges like Harvard, food insecurity --- not knowing where
your next meal is coming from --- is more episodic, happening mostly
around breaks. That is not the case at many other colleges
\href{https://studentsagainsthunger.org/hunger-on-campus/}{around the
country}. At the University of Hawaii at Manoa, public health
researchers found in 2009 that
\href{https://www.cambridge.org/core/journals/public-health-nutrition/article/div-classtitlefood-insecurity-prevalence-among-college-students-at-the-university-of-hawaii-at-mnoadiv/21D2F99685FB0C06061003AB6B9DEE62}{21
percent} students there experienced this reality firsthand. A more
recent George Washington University survey revealed that one in five
first-generation college students reported being ``food insecure'' three
or more times a week. In a study done by California State University,
college officials estimated that 21 percent of their students struggled
with food insecurity. The reality is that students at
\href{https://www.nytimes3xbfgragh.onion/2015/12/04/opinion/hungry-homeless-and-in-college.html}{state
and community colleges} bear this burden
\href{https://www.nytimes3xbfgragh.onion/2018/01/14/opinion/hunger-college-food-insecurity.html}{most
acutely}.

Food insecurity undercuts academic performance. But its effects go
beyond lowering grades. Hunger in the midst of plenty weakens students'
sense of belonging and undercuts their social, emotional and physical
well-being. Knowing one's peers are away relaxing while you scrounge for
food makes poor students not only keenly aware of their own economic
disadvantage but also of what their colleges make them endure because of
it.

Some colleges are reversing their decision to close cafeterias. Amherst
now opens Valentine Hall during break. Connecticut College, which in
2015 charged fees to eat and stay on campus, no longer does for breaks
in the academic year. Through valiant efforts by student activists,
other colleges have opened
\href{http://www.npr.org/sections/thesalt/2016/10/14/497948224/more-colleges-open-food-pantries-to-address-campus-hunger}{food
banks}. Virginia Commonwealth University (in 2014) and George Washington
University (in 2016) opened pantries to provide students with healthy
food options. Columbia opened a food bank last year. We need more
changes like these on individual campuses. But for systemic change, more
robust interventions are needed.

Battling food insecurity in college calls for national policy changes.
Increases to federal Pell Grants would provide students with resources
for expenses associated with being in college, those covered by tuition
and the many incidentals that are not. Expanding college students'
eligibility for
\href{https://www.benefits.gov/benefits/benefit-details/361}{SNAP} is
equally important. These changes would allow students to focus energy
and time on academics instead of strategizing about ways to secure food.
Given the present presidential administration, however, I am not
optimistic that reducing such inequalities will become a priority
anytime soon.

To borrow from the actress Viola Davis, diversity
``\href{http://www.etonline.com/media/video/exclusive_viola_davis_says_hollywood_diversity_issue_is_not_a_hashtag-181257/}{is
not a hashtag}'' to be celebrated when recruiting poor students and put
on the back burner once they are on campus. It is one thing to extend
coveted invitations to them. It's another to really prepare for their
arrival.

Advertisement

\protect\hyperlink{after-bottom}{Continue reading the main story}

\hypertarget{site-index}{%
\subsection{Site Index}\label{site-index}}

\hypertarget{site-information-navigation}{%
\subsection{Site Information
Navigation}\label{site-information-navigation}}

\begin{itemize}
\tightlist
\item
  \href{https://help.nytimes3xbfgragh.onion/hc/en-us/articles/115014792127-Copyright-notice}{©~2020~The
  New York Times Company}
\end{itemize}

\begin{itemize}
\tightlist
\item
  \href{https://www.nytco.com/}{NYTCo}
\item
  \href{https://help.nytimes3xbfgragh.onion/hc/en-us/articles/115015385887-Contact-Us}{Contact
  Us}
\item
  \href{https://www.nytco.com/careers/}{Work with us}
\item
  \href{https://nytmediakit.com/}{Advertise}
\item
  \href{http://www.tbrandstudio.com/}{T Brand Studio}
\item
  \href{https://www.nytimes3xbfgragh.onion/privacy/cookie-policy\#how-do-i-manage-trackers}{Your
  Ad Choices}
\item
  \href{https://www.nytimes3xbfgragh.onion/privacy}{Privacy}
\item
  \href{https://help.nytimes3xbfgragh.onion/hc/en-us/articles/115014893428-Terms-of-service}{Terms
  of Service}
\item
  \href{https://help.nytimes3xbfgragh.onion/hc/en-us/articles/115014893968-Terms-of-sale}{Terms
  of Sale}
\item
  \href{https://spiderbites.nytimes3xbfgragh.onion}{Site Map}
\item
  \href{https://help.nytimes3xbfgragh.onion/hc/en-us}{Help}
\item
  \href{https://www.nytimes3xbfgragh.onion/subscription?campaignId=37WXW}{Subscriptions}
\end{itemize}
