Sections

SEARCH

\protect\hyperlink{site-content}{Skip to
content}\protect\hyperlink{site-index}{Skip to site index}

\href{https://myaccount.nytimes3xbfgragh.onion/auth/login?response_type=cookie\&client_id=vi}{}

\href{https://www.nytimes3xbfgragh.onion/section/todayspaper}{Today's
Paper}

\href{/section/opinion}{Opinion}\textbar{}Was the Vietnam War Necessary?

\url{https://nyti.ms/2Gj2RC5}

\begin{itemize}
\item
\item
\item
\item
\item
\item
\end{itemize}

Advertisement

\protect\hyperlink{after-top}{Continue reading the main story}

Supported by

\protect\hyperlink{after-sponsor}{Continue reading the main story}

\href{/section/opinion}{Opinion}

\href{/column/vietnam-67}{Vietnam '67}

\hypertarget{was-the-vietnam-war-necessary}{%
\section{Was the Vietnam War
Necessary?}\label{was-the-vietnam-war-necessary}}

By Mark Atwood Lawrence

\begin{itemize}
\item
  March 29, 2018
\item
  \begin{itemize}
  \item
  \item
  \item
  \item
  \item
  \item
  \end{itemize}
\end{itemize}

\includegraphics{https://static01.graylady3jvrrxbe.onion/images/2018/03/29/opinion/29Vietnam-Lawrence/29Vietnam-Lawrence-articleLarge.jpg?quality=75\&auto=webp\&disable=upscale}

Apocalyptic rhetoric rolled across the airwaves on Sept. 29, 1967, as it
usually does when presidents go before the American public to explain
why the nation is fighting in a war. As he spoke to his national
television audience, President Lyndon Johnson sought to bolster flagging
public support for the Vietnam War by highlighting the calamities that
might befall the United States if the Communists prevailed.

Johnson conceded that no one could see the future with certainty, but he
left little doubt that he believed the rest of Southeast Asia would
quickly fall to Communism once Vietnam did. Even worse, he said success
in one region would embolden America's enemies to unleash new aggression
elsewhere, creating immense perils for ``our children and for our
grandchildren.''

``I am convinced,'' Johnson declared, ``that by seeing this struggle
through now, we are greatly reducing the chances of a much larger war
--- perhaps a nuclear war.''

Johnson's words conveyed confidence, even passion, about the necessity
of the war he had chosen to fight. But was he correct in his dire
assessment of the stakes in Vietnam?

Developments over the next few years suggested that he wasn't. The
Communist takeover of South Vietnam in 1975 unquestionably contributed
to Communist victories in neighboring Cambodia and Laos, precisely as
Johnson had predicted. But his nightmare vision of regional and global
catastrophe proved badly overstated. America's alliances remained
intact, and American-Soviet relations moved toward détente, not war.
Most troubling, Johnson had reason in 1967 to believe that defeat might
have precisely such unremarkable consequences.

Two weeks before his speech, Johnson had received an unusual C.I.A.
study examining the likely implications of a Communist victory in
Vietnam. The 33-page report, a distillation of the opinions of more than
30 C.I.A. officers, concluded that a failure in Vietnam would not open
the way to devastating setbacks, much less lead to a major war. On the
contrary, the study asserted, ``such risks are probably more limited and
controllable than most previous argument had indicated.''

Richard Helms, the director of central intelligence, knew this
conclusion would not sit well with a president who had ordered American
forces into combat two and a half years earlier and steadily increased
the American commitment ever since. In a covering note, Helms assured
Johnson that the C.I.A. was not arguing that the United States should
end the war any time soon. ``We are not defeatist out here,'' he wrote
from his office in Langley, Va. Yet the report plainly suggested that
the scale of America's involvement in Vietnam was out of line with that
country's actual importance to the United States.

The study, titled
``\href{https://www.cia.gov/library/readingroom/docs/DOC_0001166443.pdf}{Implications
of an Unfavorable Outcome in Vietnam},'' acknowledged that a Communist
victory would amount to ``a rather dramatic demonstration that there are
certain limits on U.S. power, a discovery that would be unexpected for
many, disconcerting for some, and encouraging to others.'' But none of
this, Johnson was told, would amount to a disastrous blow to American
security. For one thing, the report said, it would hardly come as a
surprise that a highly motivated and well-supplied guerrilla movement
could defeat a militarily superior power. ``This is not a novel
discovery,'' the C.I.A. pointed out.

More important, the report added, defeat in Vietnam would do nothing to
undermine the ``essential strength'' of the United States, which would
clearly remain the ``weightiest single factor'' in global affairs as
long as it gave fresh indications of its determination to play its
accustomed role. To reinforce the point, the C.I.A. looked to the past:
``Historically, great powers have repeatedly absorbed setbacks without
permanent diminution of the role which they subsequently played'' in
international affairs.

The report also questioned the contention that defeat in Vietnam would
cause governments around the world to doubt the willingness of the
United States to live up to its international commitments. Since the
Eisenhower years, leaders had justified their decisions to escalate
American involvement in Vietnam in part on the supposition that its
alliances could unravel --- and Communist challenges to American
interests could multiply --- if other nations had reason to question
Washington's resolve when it confronted challenges.

In fact, the C.I.A. study asserted that damage to national credibility
would be limited and temporary. The Soviets would continue to act with
``their usual caution'' and would pull back in the face of new
indications of American strength, according to the report. It even
speculated that willingness to accept defeat in Vietnam might improve
the reputation of the United States with its NATO allies, who might view
Washington's willingness to cut its losses in a region of secondary
importance as a sign of maturity rather than unreliability.

A secretary's notation on the document, which was declassified in 1993,
indicates that Johnson read the study, but there is no record of his
reaction. None of his aides could recall speaking with him about it.
It's not difficult, though, to imagine why he would have ignored it.

Most obviously, he might have dismissed the document as irrelevant,
since he remained convinced that the United States was on course to
achieve its basic objectives in Vietnam. That, after all, would be the
central theme of a major effort undertaken by the administration later
in the year to re-energize the nation's commitment to the war.

Political calculations may also have played a role. Just as in earlier
years, Johnson feared that backing down in Vietnam would jeopardize his
re-election in 1968 and the prospects of the Democratic Party more
broadly. The party had already suffered setbacks in the 1966
congressional elections, and bigger disasters loomed if Republicans
could paint their rivals as irresolute cold warriors. Closely related,
perhaps, was Johnson's concern for his own reputation and legacy. A man
of towering ambitions, he shuddered at the thought of becoming the first
president to lose a war.

Or it may be that Johnson simply could not see past pervasive
assumptions about the monumental stakes of the war. To be sure, he had
occasionally questioned the importance of Vietnam during private
conversations with aides and was well aware that many prominent
Americans in and out of government had long doubted the need to fight
there.

But he was also steeped in a Cold War mind-set that assumed a defeat
anywhere in the world could unleash domino effects, irrevocably damage
American credibility, and invite larger wars. To defy this conventional
wisdom, particularly at a time when thousands of Americans had already
died to uphold it, would have required exceptional intellectual and
political courage.

It also, though, might have been the right thing to do. Instead of
seriously reckoning with the issues raised in the report, Johnson
continued to insist that core American interests demanded success in
Vietnam, even after events forced him in 1968 to open negotiations. His
successor, Richard Nixon, was even more adamant about the dangers to
American power internationally if the Communists won, and 20,000 more
Americans died in the failed effort to prevent that outcome after
Johnson left office.

Only after 1975 would Americans see what the C.I.A. had pointed to eight
years earlier: Defeat in Vietnam, whatever the political and social
traumas it produced for the United States, brought no geopolitical
disaster.

Advertisement

\protect\hyperlink{after-bottom}{Continue reading the main story}

\hypertarget{site-index}{%
\subsection{Site Index}\label{site-index}}

\hypertarget{site-information-navigation}{%
\subsection{Site Information
Navigation}\label{site-information-navigation}}

\begin{itemize}
\tightlist
\item
  \href{https://help.nytimes3xbfgragh.onion/hc/en-us/articles/115014792127-Copyright-notice}{©~2020~The
  New York Times Company}
\end{itemize}

\begin{itemize}
\tightlist
\item
  \href{https://www.nytco.com/}{NYTCo}
\item
  \href{https://help.nytimes3xbfgragh.onion/hc/en-us/articles/115015385887-Contact-Us}{Contact
  Us}
\item
  \href{https://www.nytco.com/careers/}{Work with us}
\item
  \href{https://nytmediakit.com/}{Advertise}
\item
  \href{http://www.tbrandstudio.com/}{T Brand Studio}
\item
  \href{https://www.nytimes3xbfgragh.onion/privacy/cookie-policy\#how-do-i-manage-trackers}{Your
  Ad Choices}
\item
  \href{https://www.nytimes3xbfgragh.onion/privacy}{Privacy}
\item
  \href{https://help.nytimes3xbfgragh.onion/hc/en-us/articles/115014893428-Terms-of-service}{Terms
  of Service}
\item
  \href{https://help.nytimes3xbfgragh.onion/hc/en-us/articles/115014893968-Terms-of-sale}{Terms
  of Sale}
\item
  \href{https://spiderbites.nytimes3xbfgragh.onion}{Site Map}
\item
  \href{https://help.nytimes3xbfgragh.onion/hc/en-us}{Help}
\item
  \href{https://www.nytimes3xbfgragh.onion/subscription?campaignId=37WXW}{Subscriptions}
\end{itemize}
