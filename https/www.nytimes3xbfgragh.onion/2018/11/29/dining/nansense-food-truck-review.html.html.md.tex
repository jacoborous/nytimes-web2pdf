Sections

SEARCH

\protect\hyperlink{site-content}{Skip to
content}\protect\hyperlink{site-index}{Skip to site index}

\href{https://www.nytimes3xbfgragh.onion/section/food}{Food}

\href{https://myaccount.nytimes3xbfgragh.onion/auth/login?response_type=cookie\&client_id=vi}{}

\href{https://www.nytimes3xbfgragh.onion/section/todayspaper}{Today's
Paper}

\href{/section/food}{Food}\textbar{}At Nansense, Afghan Comfort Comes
From an Unlikely Place: a Former Mail Truck

\url{https://nyti.ms/2DRLZzZ}

\begin{itemize}
\item
\item
\item
\item
\item
\item
\end{itemize}

Advertisement

\protect\hyperlink{after-top}{Continue reading the main story}

Supported by

\protect\hyperlink{after-sponsor}{Continue reading the main story}

\href{/column/hungry-city}{Hungry City}

\hypertarget{at-nansense-afghan-comfort-comes-from-an-unlikely-place-a-former-mail-truck}{%
\section{At Nansense, Afghan Comfort Comes From an Unlikely Place: a
Former Mail
Truck}\label{at-nansense-afghan-comfort-comes-from-an-unlikely-place-a-former-mail-truck}}

\href{https://www.nytimes3xbfgragh.onion/slideshow/2018/11/29/dining/nansense-food-truck-brooklyn.html}{}

\hypertarget{mantu-dumplings-on-new-yorks-streets}{%
\subsection{Mantu Dumplings, on New York's
Streets}\label{mantu-dumplings-on-new-yorks-streets}}

10 Photos

View Slide Show ›

\includegraphics{https://static01.graylady3jvrrxbe.onion/images/2018/11/28/dining/28HUNGRY-slide-K99F/28HUNGRY-slide-K99F-articleLarge.jpg?quality=75\&auto=webp\&disable=upscale}

Caitlin Ochs for The New York Times

\begin{itemize}
\tightlist
\item
  Nansense\\
  **NYT Critic's Pick Afghan \$ 625 Atlantic Avenue 833-626-7367
\end{itemize}

By Ligaya Mishan

\begin{itemize}
\item
  Nov. 29, 2018
\item
  \begin{itemize}
  \item
  \item
  \item
  \item
  \item
  \item
  \end{itemize}
\end{itemize}

Two years ago, Mohibullah Rahmati, known as Mo, bought a mail truck at a
government auction, hoping to equip it with a kitchen and bring his
mother's Afghan recipes to the streets of New York.

She and his father had fled their hometown, Tashkurghan (today Kholm),
in northern Afghanistan after the Soviet invasion of 1979. With two
young children (Mo's older brothers) in tow, they made their way to
India, then Arizona, before settling in Woodside, Queens, where Mo was
born.

At the time, Afghan refugees were just beginning to find their footing
in the city, taking over the sidewalk pushcarts that were once a
near-monopoly of the Greek immigrants who had arrived before them. They
sold coffee and bagels, but never Afghan food.

\includegraphics{https://static01.graylady3jvrrxbe.onion/images/2018/11/28/dining/28hungry2/merlin_146916354_54d58a75-b13e-4d8f-8524-992fb577188e-articleLarge.jpg?quality=75\&auto=webp\&disable=upscale}

Now, on the corner of 20th Street and Sixth Avenue in Chelsea, the
Nansense truck offers mantu, dumplings with sheer skins, corners pinched
at the top. Inside is ground beef at a plush burger-worthy ratio: 80
percent lean, 20 percent fat. Mr. Rahmati browns the meat unadorned,
then works in salt, pepper, crushed coriander seeds and cilantro. This
gets folded into thin won-ton wrappers along with curls of raw onion,
which grow slack and faintly sweet as the dumplings steam.

The mantu are fortifying, the heat penned up inside the skins, which
hiss when pierced. But what makes them great are their ornaments:
garlicky yogurt, oddly punchy and soothing at once, and a rubble of
split peas in a rich korma of tomato paste and onions caramelized to a
near burn. There's cumin in there, gently warming, along with the
slightest shock of chile: a baring of the teeth.

The rest of the menu is devoted to more versions of korma, made with
kofta --- rough rounds of ground beef and onions bound by chickpea flour
--- chicken thighs cooked in their own fat, or potatoes and peas, melded
but still distinct. Each comes in a box with kabuli, carrots and raisins
given a gloss in a pan with cardamom and sugar, and a bracing fistful of
raw red onions and tomatoes, piqued by lime, with a flutter of dried
mint and cilantro.

One last dish is mashawa, a soup thick as stew and built for winter.
Mung beans, black-eyed peas and chickpeas are softened and fused with
rice, carrots, turnips and potatoes. It's suffused with dill, which adds
an unexpected sunniness. With each spoonful, I felt like I could walk a
hundred miles --- or sleep for hours.

Mr. Rahmati, 32, still lives in Woodside, in the building where he grew
up, although his parents have moved to Flushing, where there's a larger
Afghan-born population. He struggled for months to get a mobile vendor
permit, all the while supporting himself as an Uber driver.

Image

Chicken korma comes with kabuli, carrots and raisins given a gloss in a
pan with cardamom and sugar, and a bracing fistful of raw red onions and
tomatoes, piqued by lime, with a flutter of dried mint and
cilantro.Credit...Caitlin Ochs for The New York Times

Last January, he finally rolled out the mail truck-turned-food truck,
now painted chocolate brown, with the name Nansense in big yellow block
letters --- a pun on naan, the Afghan word for bread. (The name is
hidden in the logo, too, a grinning face whose eyes spell ``naan'' in
Dari, with ``sense'' in English scrawled down the tongue.)

That first winter was rough. Mr. Rahmati woke at 4 every morning and
headed to a commissary kitchen in Long Island City, Queens, to start
preparing the day's meals. Few customers were willing to wait on the
sidewalk in the cold. So when the temperatures started dropping last
month, he opened a weekend stand at
\href{https://www.smorgasburg.com/}{Smorgasburg}'s indoor market in
Downtown Brooklyn.

But every Wednesday, the truck comes back out to Chelsea. Mr. Rahmati
works the stoves alone, which for the time being limits his ambitions.
He briefly offered a kebab burger on naan khasa, an Afghan flatbread,
but couldn't manage it without another cook to watch the grill.

The bread, furrowed and beautifully chewy, is still available, however.
It, too, is a Rahmati production --- it comes from New Kabul Bakery in
College Point, Queens, which Mo's older brothers, Wais and Abdul
Rahmati, took over this year, with Abdul as head baker. The bread is
tricky to make: You have to use all 10 fingers, pressing the dough away,
then lifting back and pressing it away again, all down its length, so
that pockets of air form, bringing lightness.

``I still can't get it right,'' Mo said with a laugh. ``But my brother
has mastered it.''

Advertisement

\protect\hyperlink{after-bottom}{Continue reading the main story}

\hypertarget{site-index}{%
\subsection{Site Index}\label{site-index}}

\hypertarget{site-information-navigation}{%
\subsection{Site Information
Navigation}\label{site-information-navigation}}

\begin{itemize}
\tightlist
\item
  \href{https://help.nytimes3xbfgragh.onion/hc/en-us/articles/115014792127-Copyright-notice}{©~2020~The
  New York Times Company}
\end{itemize}

\begin{itemize}
\tightlist
\item
  \href{https://www.nytco.com/}{NYTCo}
\item
  \href{https://help.nytimes3xbfgragh.onion/hc/en-us/articles/115015385887-Contact-Us}{Contact
  Us}
\item
  \href{https://www.nytco.com/careers/}{Work with us}
\item
  \href{https://nytmediakit.com/}{Advertise}
\item
  \href{http://www.tbrandstudio.com/}{T Brand Studio}
\item
  \href{https://www.nytimes3xbfgragh.onion/privacy/cookie-policy\#how-do-i-manage-trackers}{Your
  Ad Choices}
\item
  \href{https://www.nytimes3xbfgragh.onion/privacy}{Privacy}
\item
  \href{https://help.nytimes3xbfgragh.onion/hc/en-us/articles/115014893428-Terms-of-service}{Terms
  of Service}
\item
  \href{https://help.nytimes3xbfgragh.onion/hc/en-us/articles/115014893968-Terms-of-sale}{Terms
  of Sale}
\item
  \href{https://spiderbites.nytimes3xbfgragh.onion}{Site Map}
\item
  \href{https://help.nytimes3xbfgragh.onion/hc/en-us}{Help}
\item
  \href{https://www.nytimes3xbfgragh.onion/subscription?campaignId=37WXW}{Subscriptions}
\end{itemize}
