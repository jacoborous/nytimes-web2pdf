Sections

SEARCH

\protect\hyperlink{site-content}{Skip to
content}\protect\hyperlink{site-index}{Skip to site index}

\href{/section/business}{Business}\textbar{}Pesticide Studies Won
E.P.A.'s Trust, Until Trump's Team Scorned `Secret Science'

\url{https://nyti.ms/2o67QKY}

\begin{itemize}
\item
\item
\item
\item
\item
\item
\end{itemize}

\includegraphics{https://static01.graylady3jvrrxbe.onion/images/2018/08/26/business/26human-1/26human-1-articleLarge.jpg?quality=75\&auto=webp\&disable=upscale}

\hypertarget{pesticide-studies-won-epas-trust-until-trumps-team-scorned-secret-science}{%
\section{Pesticide Studies Won E.P.A.'s Trust, Until Trump's Team
Scorned `Secret
Science'}\label{pesticide-studies-won-epas-trust-until-trumps-team-scorned-secret-science}}

Backed by agrochemical companies, the current administration and
Congress are moving to curb the role of human health studies in
regulation.

A strawberry field in California's Salinas Valley, where a yearslong
study has linked pesticides to ailments in children.Credit...Carlos
Chavarría for The New York Times

Supported by

\protect\hyperlink{after-sponsor}{Continue reading the main story}

By \href{https://www.nytimes3xbfgragh.onion/by/danny-hakim}{Danny Hakim}
and \href{https://www.nytimes3xbfgragh.onion/by/eric-lipton}{Eric
Lipton}

\begin{itemize}
\item
  Aug. 24, 2018
\item
  \begin{itemize}
  \item
  \item
  \item
  \item
  \item
  \item
  \end{itemize}
\end{itemize}

SALINAS, Calif. --- José Camacho once worked the fields here in the
Salinas Valley, known as ``the Salad Bowl of the World'' for its
abundance of lettuce and vegetables. His wife still does.

But back in 2000, Mr. Camacho, who is 63, got an unusual phone call. He
was asked if he wanted to work for a new project studying the effects of
pesticides on the children of farm workers.

``This seemed really crazy,'' he recalled saying at the time, since he
barely spoke English. ``A research study?''

The project, run by scientists from the University of California,
Berkeley, and
\href{https://cfpub.epa.gov/ncer_abstracts/index.cfm/fuseaction/display.abstractDetail/abstract/9234/report/2015}{funded
in part} by the Environmental Protection Agency, is still going all
these years later. Known as Chamacos, Spanish for ``children,'' it has
linked pesticides sprayed on fruit and vegetable crops with
\href{https://www.ncbi.nlm.nih.gov/pubmed/26634937}{respiratory
complications},
\href{https://onlinelibrary.wiley.com/doi/epdf/10.1111/j.1742-7843.2007.00171.x}{developmental
disorders} and \href{https://www.ncbi.nlm.nih.gov/pubmed/28557711}{lower
I.Q.s} among children of farm workers. State and federal regulators have
cited its findings to help justify proposed restrictions on everything
from
\href{https://www.regulations.gov/document?D=EPA-HQ-OPP-2015-0653-0454}{insecticides}
to flame-retardant chemicals.

But the Trump administration wants to restrict how human studies like
Chamacos are used in rule-making. A government proposal this year,
called Strengthening Transparency in Regulatory Science, could stop them
from being used to justify regulating pesticides,
\href{http://www.sciencemag.org/news/2018/05/epa-s-secret-science-rule-could-undermine-agency-s-war-lead}{lead}
and pollutants like
\href{https://insideclimatenews.org/news/22032018/epa-air-pollution-soot-rules-scott-pruitt-secret-science-policy-health-regulations}{soot},
and undermine
\href{https://www.hsph.harvard.edu/news/features/six-cities-air-pollution-study-turns-20/}{foundational
research} behind national air-quality rules. The E.P.A., which has
funded these kinds of studies, is now labeling many of them ``secret
science.''

Studying disease trends in specific groups of people --- a branch of
medicine
\href{https://www.cdc.gov/ophss/csels/dsepd/ss1978/lesson1/section2.html}{known
as epidemiology} --- started to gain currency at the E.P.A. in recent
years. These studies can be difficult because they require adjusting for
all the various substances people are exposed to beyond pesticides. But
researchers had amassed years of data from a wave of compelling chemical
studies begun in the 1990s, giving regulators a new body of research to
incorporate into their decision-making.

Under the Obama administration, the E.P.A., which had long favored tests
on rats and other laboratory animals in its pesticide regulation, began
considering epidemiological studies more seriously. The agency leaned on
this type of research in proposing to ban an insecticide called
chlorpyrifos in late 2016, and has been
\href{https://www.nytimes3xbfgragh.onion/2018/08/09/us/politics/chlorpyrifos-pesticide-ban-epa-court.html?rref=collection\%2Fbyline\%2Feric-lipton\&action=click\&contentCollection=undefined\&region=stream\&module=stream_unit\&version=latest\&contentPlacement=1\&pgtype=collection}{repeatedly
prodded to take action} on the chemical by federal courts.

But weeks after Donald J. Trump was elected president, CropLife America,
the main agrochemical trade group,
\href{http://www.croplifeamerica.org/news/2017/10/26/cla-petitions-epa-to-stop-using-studies-that-are-not-backed-by-sound-science-or-quality-data}{petitioned
the E.P.A.} to ``halt regulatory decisions that are highly influenced
and/or determined by the results of epidemiological studies'' unless
universities were forced to share more of their data.

\includegraphics{https://static01.graylady3jvrrxbe.onion/images/2018/08/26/business/26human-2/merlin_139889724_76511819-8ea6-4eba-96f9-76db548a347f-articleLarge.jpg?quality=75\&auto=webp\&disable=upscale}

Industry leaders aggressively challenged such studies in high-level
meetings and emails with E.P.A. leaders, according to thousands of pages
of documents obtained through Freedom of Information Act requests. One
trade group invited a top E.P.A. official to meet with its Washington
lobbyist last year, complaining that ``carefully controlled'' animal
studies were giving way to ``conclusions reflected in epidemiological
papers.''

Gary W. Van Sickle, executive director of the California Specialty Crops
Council,
\href{https://www.documentcloud.org/documents/4639868-2017-9-21-California-Speciality-Crops-Council.html}{wrote
to the agency} last September that ``there have been serious flaws with
E.P.A.'s conclusion to use these data.''

The council, representing growers of crops as diverse as carrots,
garlic, pears and peppers, cited ``inappropriate use of the
epidemiology.''

The E.P.A., whose new leadership is seeded with industry veterans, has
responded. In
\href{https://www.documentcloud.org/documents/4639871-2018-07-10-Atazine-Draft-Human-Health-Assessment.html}{a
mid-July assessment of atrazine}, a widely used weed killer
\href{https://www.nytimes3xbfgragh.onion/2015/02/24/business/international/a-pesticide-banned-or-not-underscores-trans-atlantic-trade-sensitivities.html}{long
banned in Europe}, the agency reviewed and dismissed 12 recent
epidemiological studies linking the herbicide to such ailments as
\href{https://www.ncbi.nlm.nih.gov/pubmed/26025512}{childhood leukemia}
and \href{https://www.ncbi.nlm.nih.gov/pubmed/25939349}{Parkinson's
disease}. It echoed the
\href{http://www.atrazine.com/human-health.aspx}{conclusions of research
funded by Syngenta}, atrazine's manufacturer, finding the chemical
\href{https://academic.oup.com/toxsci/article/123/2/441/1683036}{unlikely
to cause cancer}.

Before scandals
\href{https://www.nytimes3xbfgragh.onion/2018/07/05/climate/scott-pruitt-epa-trump.html}{forced
Scott Pruitt out last month} as head of the E.P.A., he proposed the
transparency regulation. It would ban many epidemiological studies, and
other outside research, unless more data behind the studies was made
public. In doing so, he revived a strategy advanced for years by
\href{http://www.sciencemag.org/news/2015/03/house-approves-epa-secret-science-bills-despite-white-house-veto-threat}{congressional
Republicans} and corporate interests like tobacco companies.

``\href{https://www.epa.gov/newsreleases/epa-administrator-pruitt-proposes-rule-strengthen-science-used-epa-regulations}{The
era of secret science at E.P.A. is coming to an end},'' Mr. Pruitt
proclaimed at the time. The agency's new acting administrator, Andrew R.
Wheeler,
\href{https://www.eenews.net/eenewspm/stories/1060089083/search?keyword=wheeler}{says
he's moving forward} with the proposal, as the agency re-evaluates a
class of widely used insecticides, called organophosphates, that have
been the subject of numerous epidemiological studies like Chamacos.

Nancy B. Beck, a chemical industry veteran who is the E.P.A.'s deputy
assistant administrator, said there was no attempt to thwart
epidemiology, adding that the agency was committed to ``the best
available science in the most transparent manner.''

But academics and state health officials say universities are being
\href{https://www.documentcloud.org/documents/4639885-2018-8-7-Harvard-Letter-to-EPA-Document-Pm-03.html}{pressured
to release data} that would ultimately divulge the identities of study
participants, a strategy once used by tobacco companies seeking to
undermine research on the dangers of smoking. While participant data is
shared with regulators in drug trials, academics fear that the E.P.A.'s
proposal would additionally require divulging confidential personal
information, potentially violating privacy regulations for federally
funded research.

Image

Ana Lilia Sanchez, a farmworker and the mother of a participant in the
Salinas Valley study, said her family took precautions to avoid
pesticide contamination.Credit...Carlos Chavarría for The New York Times

``It is a naked attempt to use a false claim that something nefarious is
going on with these studies in an effort to allow industry to challenge
conclusions that are not in their favor,'' said James Kelly, a manager
of environmental surveillance at the Minnesota Department of Health.

\hypertarget{a-wave-of-studies-an-uneasy-industry}{%
\subsection{A Wave of Studies, an Uneasy
Industry}\label{a-wave-of-studies-an-uneasy-industry}}

An advertisement in a Nebraska student newspaper was looking for people
who wanted to ``earn extra money.'' Thirty-six college student
volunteers and others from the community who responded were
\href{https://www.documentcloud.org/documents/4641196-Human-Studies-of-Pesticides.html}{paid
\$460 to drink gelatin capsules} filled with the pesticide chlorpyrifos,
at up to 300 times levels the E.P.A. considered safe, without a full
discussion of the risks.

Sponsored by Dow Chemical,
\href{https://www.documentcloud.org/documents/4639889-1999-Human-Tests-Chlorpyfios-Kisicki.html}{this
study}, conducted in 1998, was one of the last of its kind. That year,
the E.P.A. banned the use of studies exposing people to pesticides, and
it
\href{https://www.regulations.gov/document?D=EPA-HQ-OPP-2010-0785-0037}{continues
to severely restrict} them.

Epidemiology, which has been used to examine everything from the effects
of climate change to childhood obesity, offered a way to continue
studying disease trends,
\href{https://archive.epa.gov/pesticides/regulating/laws/fqpa/web/html/fqpahigh.html}{amid
new legal requirements} to examine how pesticides particularly affect
infants and children. And it could do so by tracking people during their
normal lives instead of treating them as if they were lab rats. Chamacos
and other studies began almost immediately, although it took decades to
collect sufficient data and study how participants changed over time.

One study by Columbia University researchers
\href{https://www.ncbi.nlm.nih.gov/pubmed/17116700}{linked an
insecticide to developmental delays} in toddlers. Another, by scientists
at the University of California, Los Angeles,
\href{https://www.niehs.nih.gov/research/supported/success/ritz/index.cfm}{connected
pesticides to Parkinson's disease}. Academics at the University of
Rochester found that pesticides
\href{https://onlinelibrary.wiley.com/doi/abs/10.1111/j.1365-2605.2005.00620.x}{lower
sperm counts} in men, while researchers from the Harvard School of
Public Health
\href{https://jamanetwork.com/journals/jamainternalmedicine/article-abstract/2659557?redirect=true}{found
lower fertility} in women.

By 2015, there was a growing body of research, often funded in part by
the E.P.A. The agency decided that year to consult epidemiology more
seriously in its evaluation of glyphosate, the world's most popular weed
killer and the active ingredient in Monsanto's Roundup.

``This is a watershed event in our Program, and one which I feel
particularly proud to be a part (go epi!!),'' Carol Christensen, then an
E.P.A. epidemiologist, wrote in a 2015 email to a colleague --- using
``epi'' as shorthand for epidemiology. ``In the 35 year history of our
program, this will be the FIRST time epi studies are actively considered
in the decision making.''

Yet even then, there was friction over what to make of studies aiming to
determine whether glyphosate causes cancer.

One E.P.A. division, the Office of Research and Development, closely
examined epidemiological research and came to believe either that
glyphosate was likely to cause cancer or that there was at least some
evidence suggesting a problem. But another division, the Office of
Pesticide Programs, was dismissive of epidemiological studies and
determined that glyphosate was not a carcinogen, a view that prevailed
at the E.P.A., according to interviews, emails and
\href{https://www.documentcloud.org/documents/4641115-Cogliano-Memo.html}{an
internal memo} obtained by The New York Times. Those involved in the
agency's debates on epidemiology spoke on the condition of anonymity
because the discussions weren't public.

Monsanto said in a statement that ``we cannot speak to the internal
E.P.A. discussions'' but emphasized the agency's ultimate finding that
glyphosate was not likely to cause cancer.

The cancer question received renewed attention this month when a
California jury
\href{https://www.nbcnews.com/news/us-news/jury-orders-monsanto-pay-290m-roundup-trial-n899811}{awarded
\$289 million} to a groundskeeper who alleged that the chemical had
sickened him. In his closing argument, the plaintiff's attorney, R.
Brent Wisner, called epidemiology one of
``\href{https://www.baumhedlundlaw.com/pdf/monsanto-documents/johnson-trial/Johnson-Day-Eighteen-A-8-7-18.pdf}{the
three pillars of cancer science}'' that the case relied on.

At the E.P.A., the debate swung in favor of epidemiology. While such
studies are often complex and can be of varying quality, the agency was
reluctant in the past to give them as much weight as lab experiments on
animals. But by the Obama administration's final months, the agency
moved for the first time to ban a pesticide largely because of
epidemiological research.

The pesticide, chlorpyrifos, was the same one ingested years earlier by
unwitting Nebraskans. It is applied to crops like apples, oranges and
strawberries to combat insects like spider mites and sap-sucking bugs.

In California alone, chlorpyrifos was sprayed on
\href{https://www.cdpr.ca.gov/docs/pur/pur16rep/chmrpt16.pdf}{640,000
acres in 2016}, according to state data. And research from Salinas, and
the Chamacos study, became a central element in the E.P.A.'s
recommendation.

``There is a breadth of information available on the potential adverse
neurodevelopmental effects in infants and children as a result of
prenatal exposure to chlorpyrifos,''
\href{https://www.documentcloud.org/documents/4641119-2016-11-03-EPA-Chlorpyrifos-Revised-Human-Health.html}{the
agency concluded in 2016}, also citing epidemiological research from
Columbia University and the Icahn School of Medicine at Mount Sinai.

The pesticide industry's reaction was loud and intense.

Monsanto, in emails with the E.P.A., was dismissive of critical
epidemiological research related to Roundup, writing that ``such studies
are well known to be
\href{https://www.documentcloud.org/documents/4641181-Page1703.html}{prone
to a number of biases}.''

Image

A Trump administration proposal would prevent the E.P.A. from using many
epidemiological studies, like the one in Salinas, unless more data
behind them was made public.Credit...Carlos Chavarría for The New York
Times

Dow Chemical said in reports submitted to the E.P.A. that
``\href{https://www.documentcloud.org/documents/4641129-2014-9-Dow-on-EPIDE-Pages-From-PANNA-MERGED.html}{the
evidence from these studies is insufficient}'' and called chlorpyrifos a
``\href{https://www.documentcloud.org/documents/4641128-2017-1-16-Dow-Petition-Mass-Comment-Campaign.html}{proven
first-line of defense}'' against new pest outbreaks.

A month after taking over the E.P.A., Mr. Pruitt acted. He disregarded
agency scientists and
\href{https://www.nytimes3xbfgragh.onion/2017/03/29/us/politics/epa-insecticide-chlorpyrifos.html}{rejected
the proposed chlorpyrifos ban},
\href{https://www.nytimes3xbfgragh.onion/2017/08/18/us/politics/epa-agriculture-industry.html}{later
calling} for
``\href{https://www.documentcloud.org/documents/3935290-EPA-HQ-2017-005731-Redacted.html\#document/p32/a369099}{a
new day}, a new future, for a common-sense approach to environmental
protection.''

\hypertarget{view-from-the-field}{%
\subsection{View From the Field}\label{view-from-the-field}}

Ana Lilia Sanchez, 50, has worked in the fields in Salinas more than
half her life, and one of her daughters has been a Chamacos study
participant.

Ms. Sanchez has learned to watch for drifting droplets or the whir of a
helicopter spraying overhead.

``Sometimes when we feel it, or we hear it, we start talking about it,''
she said recently, sitting with her 5-month-old granddaughter at her
home on a Salinas cul-de-sac. ``Why wouldn't they tell us, you know, to
get out of here, to not come today?'' she asked. ``Women, they cover
themselves, but men are working in short sleeves, so they are more
exposed.''

Insecticides like chlorpyrifos are organophosphates, from
\href{https://www.epa.gov/sites/production/files/documents/rmpp_6thed_ch5_organophosphates.pdf}{the
same chemical family} as nerve agents
\href{https://www.nytimes3xbfgragh.onion/2018/07/26/world/asia/japan-cult-execute-tokyo-subway-attack.html}{like
sarin} and
\href{https://www.nytimes3xbfgragh.onion/2018/03/13/world/europe/uk-russia-spy-poisoning.html}{Novichok},
the Russian-developed compound linked to recent attacks in Britain.
While the safety of insecticides is extensively tested, long-term health
impacts, or even
\href{https://www.nytimes3xbfgragh.onion/2017/09/21/business/monsanto-dicamba-weed-killer.html}{how
far pesticides drift}, are the subject of continuing disagreement.

Ms. Sanchez showers after work, before touching her granddaughter.

``I also put my clothes aside,'' she said. ``We separate the clothes we
use when we're working, both my husband and I, and wash them separately
so they're not contaminated.''

While \href{https://www.nature.com/articles/jes20177}{some human studies
examine} potential harm from pesticide residue found on fruits and
vegetables, the
\href{https://cerch.berkeley.edu/research-programs/chamacos-study}{Chamacos}
project is more personal, following hundreds of children in the heart of
where American food is grown. California has
\href{https://www.cdfa.ca.gov/Statistics/PDFs/2016-17AgReport.pdf}{the
nation's largest agricultural industry} and uses
\href{https://www.cdpr.ca.gov/docs/pur/pur16rep/16sum.htm\#pestuse}{more
than 200 million pounds of pesticides annually}.

Image

Brenda Eskenazi, the director of the Salinas Valley project, said that
``well-controlled epidemiologic studies'' were essential for
understanding ``how things affect human health.''Credit...Carlos
Chavarría for The New York Times

For locals, pesticides are part of life. ``It's a big difference from
when I was working,'' Mr. Camacho said, while standing in a strawberry
field framed on three sides by distant hills. Men and women were bent
over nearby, pulling weeds. ``My supervisor would say: `That's not
dangerous. Just keep working.' There was no information.''

Chamacos is built on an unsettling premise: What happens to children of
pregnant mothers certain to have pesticides in their bloodstreams? The
E.P.A. and other government agencies have spent millions of dollars
funding Chamacos.

Half the Chamacos children have been tracked since before birth.
Researchers have collected 350,000 samples of blood, urine, breast milk
and even household dust and spent nearly two decades studying maturing
children. They perform neurodevelopmental and physical assessments and
study factors like diet and school performance. After nearly two
decades, the study's data appears in more than
\href{https://www.ncbi.nlm.nih.gov/myncbi/browse/collection/51228867/}{160
academic papers}.

During a visit to the Chamacos office in Salinas, Brenda Eskenazi, the
director of the project and a professor of epidemiology at Berkeley, was
testing out brain monitoring equipment, wearing what looked like a black
swim cap strewn with knobs and wiring. She has long been fascinated with
cognitive development, going back to when she saw a Woodstock reveler
--- one having a bad acid trip --- dive into pavement.

``Why did he do that?'' Ms. Eskenazi remembers wondering at the time.
``What was he thinking? What's going on in that brain?''

``Any science is imperfect,'' she said, but stressed that
``well-controlled epidemiologic studies'' were essential for
understanding ``how things affect human health.'' She added, ``Otherwise
you're just making huge assumptions that a rodent is the same as a
human.''

\hypertarget{a-bitter-debate}{%
\subsection{A Bitter Debate}\label{a-bitter-debate}}

The day after Mr. Pruitt made his March 2017 decision to reject a ban on
chlorpyrifos, he hosted top executives from one of the nation's largest
farming and pesticide trade organizations for a closed-door
conversation.

Near the top of the meeting agenda was ``Epidemiology Study Policy'' in
the aftermath of the ``chlorpyrifos matter,'' according to internal
records.

Image

McKinnon Elementary School in Salinas. The pesticide industry contends
that epidemiological studies are prone to biases and not as reliable as
testing on lab animals.Credit...Carlos Chavarría for The New York Times

``There are no guideposts, if you will, for what is a legitimate, useful
epidemiology study and what is not,'' Jay Vroom, CropLife America's
president, said in an interview, explaining what he had told agency
officials at this and other meetings.

In a subsequent letter to the E.P.A., a CropLife America lobbyist said
the agency was relying on a
``\href{https://www.documentcloud.org/documents/4641204-2017-7-24-Crop-Life-CLA-Comments-Chlorpyrifos.html}{shortsighted
approach},'' and the group submitted formal proposals to curb the
embrace of epidemiology the E.P.A. undertook under the Obama
administration.

Mr. Pruitt responded with his proposal, made this past spring, to ban
epidemiological and other studies that did not make study details
public, including at least some information on study participants.

Academics have resisted previous requests to review their data,
\href{https://www.epa.gov/ingredients-used-pesticide-products/chlorpyrifos-epas-seven-year-quest-columbias-raw-data}{notably
at Columbia University}. In a 2016 letter to the agency, a university
official wrote that it could not provide ``extensive individual level
data to E.P.A. in a way that ensures the confidentiality'' of ``our
research subjects.''

\href{https://publichealth.gwu.edu/departments/environmental-and-occupational-health/david-michaels-phd}{David
Michaels}, an epidemiologist at George Washington University's School of
Public Health and head of the Occupational Safety and Health
Administration during the Obama administration, said Mr. Pruitt's plan
was not about transparency but about discrediting studies that made
pesticides look bad.

``The underlying justification for this `transparency' proposal is a
caricature of how science really works,'' Mr. Michaels said at a recent
hearing. ``The cynical approach proposed by E.P.A. can be best described
as `weaponized transparency.'''

It is no coincidence, he said, that the term ``secret science'' was also
used in the 1970s when the tobacco industry was trying to forestall
critical research about smoking.

Researchers have had wins. This month, a
\href{https://www.nytimes3xbfgragh.onion/2018/08/09/us/politics/chlorpyrifos-pesticide-ban-epa-court.html?rref=collection\%2Fbyline\%2Feric-lipton\&action=click\&contentCollection=undefined\&region=stream\&module=stream_unit\&version=latest\&contentPlacement=1\&pgtype=collection}{federal
appeals court} ordered the E.P.A. to ban chlorpyrifos, citing findings
from human studies. The Trump administration is mulling whether to
appeal.

But epidemiologists are unsettled. In mid-July, after nearly two decades
of work on Chamacos, the E.P.A. emailed Ms. Eskenazi requesting ``the
original data'' from her research, citing ``uncertainty around
neurodevelopmental effects associated'' with pesticides she has studied.
The agency made a similar request to Columbia.

Ms. Eskenazi, worried about her study participants' privacy, alerted
university lawyers. She is now concerned that the E.P.A. may try to
undermine her study's repeated findings that some pesticides may be
harming children.

``I knew this was going to come sooner or later,'' she said. ``And here
it is.''

Advertisement

\protect\hyperlink{after-bottom}{Continue reading the main story}

\hypertarget{site-index}{%
\subsection{Site Index}\label{site-index}}

\hypertarget{site-information-navigation}{%
\subsection{Site Information
Navigation}\label{site-information-navigation}}

\begin{itemize}
\tightlist
\item
  \href{https://help.nytimes3xbfgragh.onion/hc/en-us/articles/115014792127-Copyright-notice}{©~2020~The
  New York Times Company}
\end{itemize}

\begin{itemize}
\tightlist
\item
  \href{https://www.nytco.com/}{NYTCo}
\item
  \href{https://help.nytimes3xbfgragh.onion/hc/en-us/articles/115015385887-Contact-Us}{Contact
  Us}
\item
  \href{https://www.nytco.com/careers/}{Work with us}
\item
  \href{https://nytmediakit.com/}{Advertise}
\item
  \href{http://www.tbrandstudio.com/}{T Brand Studio}
\item
  \href{https://www.nytimes3xbfgragh.onion/privacy/cookie-policy\#how-do-i-manage-trackers}{Your
  Ad Choices}
\item
  \href{https://www.nytimes3xbfgragh.onion/privacy}{Privacy}
\item
  \href{https://help.nytimes3xbfgragh.onion/hc/en-us/articles/115014893428-Terms-of-service}{Terms
  of Service}
\item
  \href{https://help.nytimes3xbfgragh.onion/hc/en-us/articles/115014893968-Terms-of-sale}{Terms
  of Sale}
\item
  \href{https://spiderbites.nytimes3xbfgragh.onion}{Site Map}
\item
  \href{https://help.nytimes3xbfgragh.onion/hc/en-us}{Help}
\item
  \href{https://www.nytimes3xbfgragh.onion/subscription?campaignId=37WXW}{Subscriptions}
\end{itemize}
