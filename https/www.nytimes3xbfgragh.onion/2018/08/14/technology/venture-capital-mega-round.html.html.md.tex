\href{/section/technology}{Technology}\textbar{}\$100 Million Was Once
Big Money for a Start-Up. Now, It's Common.

\url{https://nyti.ms/2Moaidt}

\begin{itemize}
\item
\item
\item
\item
\item
\end{itemize}

\includegraphics{https://static01.graylady3jvrrxbe.onion/images/2018/08/14/business/14Megaround.illo/merlin_142167144_85875c61-1730-460b-89c0-431b035d6ab2-articleLarge.jpg?quality=75\&auto=webp\&disable=upscale}

Sections

\protect\hyperlink{site-content}{Skip to
content}\protect\hyperlink{site-index}{Skip to site index}

\hypertarget{100-million-was-once-big-money-for-a-start-up-now-its-common}{%
\section{\$100 Million Was Once Big Money for a Start-Up. Now, It's
Common.}\label{100-million-was-once-big-money-for-a-start-up-now-its-common}}

Known as a mega-round in Silicon Valley, large-scale fund raising is
producing a frenzy around tech companies with enough reach and momentum
to absorb a large check.

Credit...Jon Reinfurt

Supported by

\protect\hyperlink{after-sponsor}{Continue reading the main story}

By \href{https://www.nytimes3xbfgragh.onion/by/erin-griffith}{Erin
Griffith}

\begin{itemize}
\item
  Aug. 14, 2018
\item
  \begin{itemize}
  \item
  \item
  \item
  \item
  \item
  \end{itemize}
\end{itemize}

\href{https://cn.nytimes3xbfgragh.onion/technology/20180820/venture-capital-mega-round/}{阅读简体中文版}\href{https://cn.nytimes3xbfgragh.onion/technology/20180820/venture-capital-mega-round/zh-hant/}{閱讀繁體中文版}

In late April, when Mike Massaro set out to get \$40 million to \$75
million in funding for his payments start-up, Flywire, he contacted a
small group of investors he already knew. But word quickly got around,
and other investors flooded his inbox with \$200 million of investment
offers, half of which he turned down.

Gusto, a payroll and benefits software company, raised \$140 million in
July, but could have done five times that, according to Joshua Reeves,
its chief executive and founder.

Convene, a real estate services start-up, recently obtained \$152
million and turned away more than \$100 million of additional
investment. Soon after, another wave of hopeful investors called, asking
if the company would be looking for more financing, according to Ryan
Simonetti, Convene's chief executive.

Start-ups raising \$100 million or more from investors --- known as a
mega-round in Silicon Valley --- used to be a rarity. But now, they are
practically routine, producing a frenzy around tech companies with
enough scale and momentum to absorb a large check.

The jump in oversize investments is led by relatively new investors,
including the Japanese conglomerate SoftBank, Chinese companies and
sovereign wealth funds. They see a chance to capitalize on tech's
incursion into just about every industry, and want to put their money
down before the young companies go public.

By entering the tech market, they have all but eliminated talk in
Silicon Valley about an investment bubble --- a leading concern a couple
of years ago --- because the money now seems almost limitless.

For the start-ups, the pots of money are changing the normal way of
building a tech company. They must move even faster, expand their
ambitions and collect more investment money than ever --- even if they
might not be ready. They risk becoming too reliant on funding and never
finding a path to profit.

\includegraphics{https://static01.graylady3jvrrxbe.onion/images/2018/08/15/business/15MEGAROUNDS-1/merlin_142325541_0f8bb19b-9320-4f8b-b72f-e56de17cf869-articleLarge.jpg?quality=75\&auto=webp\&disable=upscale}

``If your competitor is going to raise \$150 million and you want to be
conservative and only raise \$20 million, you're going to get run
over,'' said Bill Gurley, a managing partner at Benchmark Capital.

Investors participated in a record 273 mega-rounds last year, according
to the data provider Crunchbase. This year is on pace to easily eclipse
that, with 268 completed in the first seven months of the year. In July,
start-ups reached more than 50 financing deals worth a combined \$15
billion, a new monthly high.

In the last 10 days, Letgo, an online classifieds ads company, raised
\$500 million. Actifio, a data storage company, took in \$100 million.
MyDreamPlus, a co-working space start-up, secured \$120 million. And
Klook, a travel activity booking site, got \$200 million.

These mega-rounds have become so common that CB Insights, which tracks
start-up investments, has even debated lifting its definition of a
mega-round to \$200 million or more, according to Anand Sanwal, the
firm's chief executive.

Many of the new investors, including SoftBank's \$93 billion Vision
Fund, manage funds so large they dwarf the entire traditional venture
capital market in the United States. These giant funds are looking for
start-ups that can take large sums of money with one shot. Writing lots
of small checks is too time-consuming, and the returns from small bets
will not make a difference for a such a big fund. So investors are
competing to back any start-up that shows promise and the ability to put
\$100 million or more to use.

``As soon as they feel like they have a winner, they will really put a
lot of resources behind it,'' said Mr. Sanwal of CB Insights.

SoftBank's deal-making has affected every part of the venture capital
market. The arrival of its Vision Fund, which has a minimum investment
size of \$100 million, has prompted a number of traditional venture
capital firms, including Sequoia Capital, to build larger pools of money
to compete. Funds from seven different firms are raising capital,
according to the data provider Pitchbook.

But the Vision Fund is not even the most active mega-round investor. In
the first seven months of 2018, Tencent Holdings participated in 31
rounds of funding of \$100 million or more, compared with 18 for
SoftBank, according to CB Insights. GIC and Temasek Holdings, investment
funds associated with the government of Singapore, as well as Alibaba
and Sequoia Capital China, have also been among the most active
mega-round investors this year.

Image

Investors flooded the inbox of Mike Massaro, the chief executive of
Flywire, with \$200 million of investment offers, half of which he
turned down.Credit...Cody O'Loughlin for The New York Times

As a result, early investors must make sure their portfolio companies
are friendly with the large funds, laying groundwork for a potential
investment in the future.

``It feels like there is a bit of a beauty pageant that early-stage
investors put on for the mega-funds,'' said Patricia Nakache, a general
partner at Trinity Ventures.

The hot funding market is pushing high-growth start-ups to change their
plans. Flywire was not going to pursue more investment money until next
year. It still had \$15 million in the bank from a previous funding
round. But the company saw ``investment heat'' in the payments industry
and Mr. Massaro thought more money would help Flywire grow even faster.

Funding rounds can take as long as six months to put together, but
Flywire's round wrapped up in just over two. As a result of the capital,
the company will complete some of its hiring plans in half the time it
previously planned and expand into a new geographic market two years
early.

Moving aggressively is not a choice. Well-funded start-ups, called
``super-haves'' by some investors, can afford to pay their employees
more and lower their prices, losing money in the short term to win more
customers.

Convene, the real estate start-up, competes tangentially with WeWork, an
office rental company that has raised more than \$8 billion in funding
from SoftBank and other investors. Convene was not planning to chase
more capital until later this year, but the company's investors
encouraged it to move sooner. Business deals move faster today than they
did five years ago, said Mr. Simonetti, the company's chief executive,
and having extra capital on hand helps with speed.

``There's definitely a little bit of this, `Let's overcapitalize the
company a little bit so we can move quicker and faster,' '' he said.

The large investors also demand big ideas from the start-ups. Softbank's
Vision Fund team pushed Tina Sharkey, chief executive of e-commerce
start-up Brandless, to share the most grandiose, ambitious version of
her business road map.

``They were like, `Come on, show us your real plan,' '' she said.
Brandless had barely been operating for a year before Ms. Sharkey
outlined an expansive vision to use machine learning, data, curation,
and community-building to create efficiencies. ``We didn't have the
gumption to say that to anyone else,'' she said. In July, SoftBank
invested \$224 million into her company.

``In today's hyper-connected world, companies need to hire, scale and
enter new markets faster than ever before or risk being surpassed by
others,'' said Jeff Housenbold, a managing partner at SoftBank
Investment Advisers.

Few venture investors foresee a slowdown in the pace of mega-rounds.
Those who once cautioned of a tech bubble and subsequent crash have
given up on their warnings. In 2015, Mr. Gurley of Benchmark predicted
``dead unicorns,'' referring to start-ups valued at \$1 billion or more.
But since 2015, the number of start-ups worth \$1 billion or more has
ballooned to 258 from 80, according to CB Insights. Excess funding is
tied to inflated valuations, which may create problems when overvalued
companies eventually try to go public.

Mr. Gurley said he was done trying to sound the alarm. ``You have to
adjust to the reality and play the game on the field,'' he said.

Annie Lamont, a managing partner at the venture fund Oak HC/FT, expected
a drop-off in start-up valuations and funding three years ago, but it
never happened. Now, she expects more of the same, partly because most
the companies can easily get more money and few are worried about a
downturn.

``The fear of a correction is not occurring,'' she said. If any
start-ups do ``vaporize,'' she said, ``I think people are going to
ignore them and roll right on to the next one.''

Mitchell Green, a managing partner at Lead Edge Capital, does not
imagine a slowdown unless interest rates increase significantly, a
change that could prompt investors to move money into tax-free bonds.
``It ain't going to stop,'' he said. ``There's too much money.''

Mr. Reeves, of software start-up Gusto, acknowledged that founders who
obtain outsize sums of capital can get caught up in a ``growth at any
cost'' mentality. That is why he chose not to maximize his funding round
despite the intense interest. ``It's up to the founder to realize that's
a distraction,'' he said. ``Success is not having more money or a bigger
team, but having more customers or revenue.''

Mr. Reeves is staying cautious, even when few others seem worried about
a bubble. ``The most likely time to have some type of correction, or
change, or realization of the cycle turning, is when you're not talking
about it,'' he said.

Advertisement

\protect\hyperlink{after-bottom}{Continue reading the main story}

\hypertarget{site-index}{%
\subsection{Site Index}\label{site-index}}

\hypertarget{site-information-navigation}{%
\subsection{Site Information
Navigation}\label{site-information-navigation}}

\begin{itemize}
\tightlist
\item
  \href{https://help.nytimes3xbfgragh.onion/hc/en-us/articles/115014792127-Copyright-notice}{©~2020~The
  New York Times Company}
\end{itemize}

\begin{itemize}
\tightlist
\item
  \href{https://www.nytco.com/}{NYTCo}
\item
  \href{https://help.nytimes3xbfgragh.onion/hc/en-us/articles/115015385887-Contact-Us}{Contact
  Us}
\item
  \href{https://www.nytco.com/careers/}{Work with us}
\item
  \href{https://nytmediakit.com/}{Advertise}
\item
  \href{http://www.tbrandstudio.com/}{T Brand Studio}
\item
  \href{https://www.nytimes3xbfgragh.onion/privacy/cookie-policy\#how-do-i-manage-trackers}{Your
  Ad Choices}
\item
  \href{https://www.nytimes3xbfgragh.onion/privacy}{Privacy}
\item
  \href{https://help.nytimes3xbfgragh.onion/hc/en-us/articles/115014893428-Terms-of-service}{Terms
  of Service}
\item
  \href{https://help.nytimes3xbfgragh.onion/hc/en-us/articles/115014893968-Terms-of-sale}{Terms
  of Sale}
\item
  \href{https://spiderbites.nytimes3xbfgragh.onion}{Site Map}
\item
  \href{https://help.nytimes3xbfgragh.onion/hc/en-us}{Help}
\item
  \href{https://www.nytimes3xbfgragh.onion/subscription?campaignId=37WXW}{Subscriptions}
\end{itemize}
