Sections

SEARCH

\protect\hyperlink{site-content}{Skip to
content}\protect\hyperlink{site-index}{Skip to site index}

\href{https://www.nytimes3xbfgragh.onion/section/politics}{Politics}

\href{https://myaccount.nytimes3xbfgragh.onion/auth/login?response_type=cookie\&client_id=vi}{}

\href{https://www.nytimes3xbfgragh.onion/section/todayspaper}{Today's
Paper}

\href{/section/politics}{Politics}\textbar{}A Trump Endorsement Can
Decide a Race. Here's How to Get One.

\url{https://nyti.ms/2P8wXbq}

\begin{itemize}
\item
\item
\item
\item
\item
\item
\end{itemize}

Advertisement

\protect\hyperlink{after-top}{Continue reading the main story}

Supported by

\protect\hyperlink{after-sponsor}{Continue reading the main story}

\hypertarget{a-trump-endorsement-can-decide-a-race-heres-how-to-get-one}{%
\section{A Trump Endorsement Can Decide a Race. Here's How to Get
One.}\label{a-trump-endorsement-can-decide-a-race-heres-how-to-get-one}}

\includegraphics{https://static01.graylady3jvrrxbe.onion/images/2018/08/28/us/politics/28trump-endorsements1-print/28trump-endorsements1-articleLarge.jpg?quality=75\&auto=webp\&disable=upscale}

By \href{https://www.nytimes3xbfgragh.onion/by/jonathan-martin}{Jonathan
Martin} and
\href{https://www.nytimes3xbfgragh.onion/by/maggie-haberman}{Maggie
Haberman}

\begin{itemize}
\item
  Aug. 27, 2018
\item
  \begin{itemize}
  \item
  \item
  \item
  \item
  \item
  \item
  \end{itemize}
\end{itemize}

WASHINGTON --- Before she agreed to run for the Senate in Arizona this
year, Representative Martha McSally reached an agreement with White
House officials: President Trump would remain on the sidelines and not
endorse one of her more conservative competitors.

But a few weeks ago, Ms. McSally and other establishment Republicans
were worried enough about her prospects that they returned to the White
House with a new appeal, according to multiple party officials familiar
with the conversations: Could Mr. Trump drop his neutrality and endorse
her candidacy after all?

Ms. McSally's shifting requests illustrate Mr. Trump's ability to play
kingmaker and effectively decide competitive primaries. But, more
consequentially, they demonstrate the willingness of mainstream
Republicans like Ms. McSally, who will not say whether she even voted
for Mr. Trump in 2016, to link themselves to the president if they want
to win.

The Republican nominating season will largely conclude on Tuesday with
the Arizona Senate race and the Florida governor's contest, leaving a
paradox looming over Washington: Even as
\href{https://www.nytimes3xbfgragh.onion/2018/08/22/us/politics/mueller-investigation-cohen-manafort.html}{legal
questions swirl around the presidency}, Mr. Trump's grip on G.O.P.
primary voters is as strong as it has been since he seized the party's
nomination a little over two years ago.

Mr. Trump's outsized influence offers him a measure of political
insurance. And some of his leading allies are already warning Republican
officeholders who may be faced with an impeachment vote in a
Democratic-controlled Congress to be fully mindful of the president's
popularity with their shared base.

``He'll attack you, your money will dry up and you will lose your
primary,'' Corey Lewandowski, Mr. Trump's former campaign manager, said
about the consequences Republican lawmakers would face if they turned on
the president. ``Go ask Mark Sanford what fighting with the president
will get you.''

Mr. Sanford, a South Carolina congressman,
\href{https://www.nytimes3xbfgragh.onion/2018/06/12/us/politics/mark-sanford-trump.html}{lost
a primary} to the Trump-endorsed Katie Arrington in June. Mr.
Lewandowski also invoked Senators Jeff Flake of Arizona and Bob Corker
of Tennessee, two Trump critics who retired rather than seek re-election
this year. And he brought up Senator Ben Sasse of Nebraska, another
frequent detractor of the president, to pointedly note that Mr. Trump's
high command is well aware Mr. Sasse's term is up in 2020.

Mr. Trump's endorsements have included many candidates who did not face
serious primary challenges. But they have also proved decisive in some
races, or at least ensured victory for candidates in tight races, such
as Brian Kemp, the Republican nominee for governor in Georgia.

The endorsements may arrive in the form of a tweet weeks or months
before any votes are cast or the day before or even on Election Day. The
most full-throated endorsements are accompanied by rallies with Mr.
Trump
\href{https://www.nytimes3xbfgragh.onion/2018/07/24/us/politics/kemp-cagle-georgia.html}{or
the vice president}, exciting the party faithful and prompting them to
turn out for the vote.

To the
\href{https://www.nytimes3xbfgragh.onion/2018/07/20/us/politics/trump-endorsement-kemp.html}{dismay
of some party leaders}, though, a handful of the candidates Mr. Trump
has pushed through the primary may prove weaker general election
candidates than their rivals would have been, forcing the party to spend
more money this fall.

\includegraphics{https://static01.graylady3jvrrxbe.onion/images/2018/08/28/us/politics/28trump-endorsements2-print/merlin_142833030_ce3b795c-36a8-4de8-9c98-7fa3a06d1978-articleLarge.jpg?quality=75\&auto=webp\&disable=upscale}

Mr. Trump has relished his role as the decider in chief, growing more
emboldened about intervening in intraparty contests as nearly all of the
candidates he has backed in primaries this year have won. (One recent
misfire: the Wyoming governor's race last week,
\href{https://www.nytimes3xbfgragh.onion/2018/08/21/us/politics/wyoming-alaska-primaries.html}{where
his preferred candidate came in second}.)

But the president has also complained to advisers that leaders like
Mitch McConnell, the Senate majority leader, are laying on the flattery
about how he's the only figure who can settle races, but not more
aggressively defending his embattled administration in public, according
to a senior White House official.

Mr. Trump has, as with so much else, redefined the political role of the
presidency to fit his own unconventional style rather than bowing to the
customs of the office.

``Traditionally in the two-party system, politicians and their team try
to make their base bigger,'' said Haley Barbour, the former Mississippi
governor and Republican National Committee chairman. ``Trump keeps
trying to make his base harder.''

Inserting himself in race after race with little warning, Mr. Trump has
demolished the tradition of presidents remaining neutral in primaries,
\href{https://www.nytimes3xbfgragh.onion/2018/07/20/us/politics/trump-endorsement-kemp.html}{upended
the well-laid plans} of state and national power brokers and rewarded
allies, new and old, in their nomination fights. In doing so, he is
creating more loyalists who owe their success to him.

The president has found success wading into contested governor's
primaries in
\href{https://www.nytimes3xbfgragh.onion/2018/06/26/us/politics/primaries-henry-mcmaster-trump.html}{South
Carolina},
\href{https://www.nytimes3xbfgragh.onion/2018/07/18/us/politics/trump-brian-kemp-georgia.html}{Georgia}
and
\href{https://www.nytimes3xbfgragh.onion/2018/08/14/us/politics/kansas-kobach-colyer.html}{Kansas},
and he may have had the most impact with his early support for
Representative Ron DeSantis, who is leading in the polls in the Florida
governor's primary.

``If a president wants to reshape the party in his own image, there's no
better way to do that than to impact the outcome of primaries,'' said
Kris Kobach, the Kansas secretary of state and nominee for governor
\href{https://www.nytimes3xbfgragh.onion/2018/08/06/us/politics/kris-kobach-trump-kansas.html}{who
got a late endorsement}in his primary from Mr. Trump.

The president's endorsements have largely been done to repay supporters,
but Mr. Trump has also sought to solidify new alliances by lending a
hand to some former critics. For example, he helped ensure that Senator
Dean Heller of Nevada would not have a primary challenger, propelled
Representative Martha Roby of Alabama in her runoff, and he even got
behind Mitt Romney's Senate bid in Utah. Each of them, to varying
degrees, had criticized the president in 2016.

The White House political director, Bill Stepien, said that Mr. Trump's
endorsement is powerful, ``but there are many tools in his political
tool kit,'' including not engaging openly in a race but working behind
the scenes as ``party leader.'' He said that Nevada was an important
example of where the president and his team worked behind the scenes to
\href{https://www.nytimes3xbfgragh.onion/2018/03/16/us/politics/bannon-republican-senate-primary-challengers.html}{clear
the Senate primary field}by nudging a Heller challenger to run for the
House instead.

Mr. Lewandowski was even blunter, recalling a Las Vegas dinner he
attended in 2017 with Mr. Heller, in which he bluntly told the senator
that if he did not fall in line with Mr. Trump, the president's allies
would aggressively target him. ``I said, `With all due respect, I will
not miss,''' Mr. Lewandowski recalled.

This is not to say that Mr. Trump has totally forgiven and forgotten
those who denounced his campaign --- or that the president's advisers do
not strategically highlight past critiques when they believe it is best
for him to stay out of races.

This year few Republican candidates, for example, were as aggressive in
lobbying for Mr. Trump's endorsement as Representative Diane Black of
Tennessee, who came in third in her state's primary for governor this
month. She approached the president at a White House event, had some of
his most high-profile congressional allies weigh in on her behalf, and
even deployed some West Wing officials who are friendly to her.

Image

Mr. Trump campaigned for Representative Ron DeSantis, whom he endorsed
for the Republican nomination for governor of Florida, last
month.Credit...Doug Mills/The New York Times

But most of Mr. Trump's aides wanted him to stay out of the race, and
they were able to keep him sidelined in part by reminding him of what
Ms. Black said after the video of Mr. Trump boasting about groping women
was released in 2016 (``I would've yanked my son by the ear if he had
talked that way when he was a teenager much less an adult,'' she said at
the time).

In the early months of his administration, Mr. Trump's approach to
politics was, like most everything else, haphazard.

But since then, Mr. Stepien and Johnny DeStefano, the head of
presidential personnel, have imposed a measure of order to the process
(or at least as much as somebody as impulsive as Mr. Trump will
tolerate), with the chief of staff, John F. Kelly, guaranteeing them
face time with the president that they struggled to get in 2017.

Thick binders have been assembled on all high-level Republican
candidates and sitting members of Congress. The binders include their
votes, their behavior on social media and what they said about Mr. Trump
as he sought the presidency.

In the most significant races, a survey has been sent to the candidates
asking where they stand on issues and requesting that they grade the
president. The candidates are judged on several criteria, including how
quickly they respond.

Last Thursday, for example, Mr. Trump offered his support for Senator
Cindy Hyde-Smith of Mississippi, giving her a leg up against former
state Senator Chris McDaniel, a hard-right Republican, in
\href{https://www.nytimes3xbfgragh.onion/2018/06/01/us/politics/chris-mcdaniel-mississippi-trump.html}{a
November special election there}.

But the president blessed Ms. Hyde-Smith only after she met a series of
requests from the White House.

``There were clear expectations and benchmarks and things like that that
the senator was supposed to hit,'' said Brad White, Ms. Hyde-Smith's
chief of staff, explaining: ``They looked at things from fund-raising to
the type of senator she was to votes. It was very methodical.''

But even as his staff has introduced some discipline and organization to
the process, Mr. Trump has on occasion still gone his own way, because
of the many outside influences on his thinking.

It was largely overshadowed by the other, more serious blows he absorbed
last week, but the president suffered his first loss in a primary this
year when he endorsed
\href{https://www.nytimes3xbfgragh.onion/2018/08/21/us/politics/wyoming-foster-friess.html}{the
Republican donor Foster Friess}in the Wyoming governor's race only to
see Mr. Friess be handily defeated.

Mr. Trump, who tweeted his support only on the morning of the primary,
was cajoled into the endorsement by his eldest son, Donald Trump Jr.,
who prodded him a few times on it, according to two Republican officials
familiar with the conversations. (The younger Mr. Trump
\href{https://trib.com/opinion/columns/trump-jr-friess-right-choice-for-wyoming/article_94e2311c-cad9-53d5-b72d-651c95190986.html}{wrote
an op-ed endorsing Mr. Friess} this month.)

As for Ms. McSally's race in Arizona --- where she faces Joe Arpaio, the
former Maricopa County sheriff, and Kelli Ward, a one-time state senator
--- Mr. Trump's advisers determined that she would probably win the
primary even without the president's support. Her pleas, they decided,
were aimed only at saving a few million dollars she would need to spend
to ensure victory without the president's blessing.

Advertisement

\protect\hyperlink{after-bottom}{Continue reading the main story}

\hypertarget{site-index}{%
\subsection{Site Index}\label{site-index}}

\hypertarget{site-information-navigation}{%
\subsection{Site Information
Navigation}\label{site-information-navigation}}

\begin{itemize}
\tightlist
\item
  \href{https://help.nytimes3xbfgragh.onion/hc/en-us/articles/115014792127-Copyright-notice}{©~2020~The
  New York Times Company}
\end{itemize}

\begin{itemize}
\tightlist
\item
  \href{https://www.nytco.com/}{NYTCo}
\item
  \href{https://help.nytimes3xbfgragh.onion/hc/en-us/articles/115015385887-Contact-Us}{Contact
  Us}
\item
  \href{https://www.nytco.com/careers/}{Work with us}
\item
  \href{https://nytmediakit.com/}{Advertise}
\item
  \href{http://www.tbrandstudio.com/}{T Brand Studio}
\item
  \href{https://www.nytimes3xbfgragh.onion/privacy/cookie-policy\#how-do-i-manage-trackers}{Your
  Ad Choices}
\item
  \href{https://www.nytimes3xbfgragh.onion/privacy}{Privacy}
\item
  \href{https://help.nytimes3xbfgragh.onion/hc/en-us/articles/115014893428-Terms-of-service}{Terms
  of Service}
\item
  \href{https://help.nytimes3xbfgragh.onion/hc/en-us/articles/115014893968-Terms-of-sale}{Terms
  of Sale}
\item
  \href{https://spiderbites.nytimes3xbfgragh.onion}{Site Map}
\item
  \href{https://help.nytimes3xbfgragh.onion/hc/en-us}{Help}
\item
  \href{https://www.nytimes3xbfgragh.onion/subscription?campaignId=37WXW}{Subscriptions}
\end{itemize}
