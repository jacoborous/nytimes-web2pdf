Sections

SEARCH

\protect\hyperlink{site-content}{Skip to
content}\protect\hyperlink{site-index}{Skip to site index}

\href{https://www.nytimes3xbfgragh.onion/section/us}{U.S.}

\href{https://myaccount.nytimes3xbfgragh.onion/auth/login?response_type=cookie\&client_id=vi}{}

\href{https://www.nytimes3xbfgragh.onion/section/todayspaper}{Today's
Paper}

\href{/section/us}{U.S.}\textbar{}Parkland Parent Lori Alhadeff Wins
Election to School Board in Florida

\url{https://nyti.ms/2oi9JnV}

\begin{itemize}
\item
\item
\item
\item
\item
\end{itemize}

Advertisement

\protect\hyperlink{after-top}{Continue reading the main story}

Supported by

\protect\hyperlink{after-sponsor}{Continue reading the main story}

\hypertarget{parkland-parent-lori-alhadeff-wins-election-to-school-board-in-florida}{%
\section{Parkland Parent Lori Alhadeff Wins Election to School Board in
Florida}\label{parkland-parent-lori-alhadeff-wins-election-to-school-board-in-florida}}

\includegraphics{https://static01.graylady3jvrrxbe.onion/images/2018/08/29/us/29parkland/merlin_138189456_4b66b913-5846-409e-85c4-35d8a07ff883-articleLarge.jpg?quality=75\&auto=webp\&disable=upscale}

By \href{https://www.nytimes3xbfgragh.onion/by/patricia-mazzei}{Patricia
Mazzei}

\begin{itemize}
\item
  Aug. 28, 2018
\item
  \begin{itemize}
  \item
  \item
  \item
  \item
  \item
  \end{itemize}
\end{itemize}

PARKLAND, Fla. --- When Alyssa Alhadeff and Alaina Petty, both 14, died
this year in a mass shooting at Marjory Stoneman Douglas High School in
Parkland, Fla., their grief-stricken parents publicly mourned their
daughters and questioned politicians over what could be done to prevent
another tragedy.

Then they entered the political arena themselves,
\href{https://www.nytimes3xbfgragh.onion/2018/05/15/us/parkland-parents-school-board.html}{launching
campaigns} for the elected body with the most power over their
daughters' education: the Broward County School Board.

On Tuesday, one of them, Lori Alhadeff, won her election to the board,
promising to focus on security and hold accountable administrators she
sees as responsible for failing to quickly
\href{https://www.nytimes3xbfgragh.onion/2018/08/11/us/back-to-school-security-guns.html}{adopt
safety measures} after the shooting, which left 17 students and
educators dead on Feb. 14. The other parent, Ryan Petty, was trailing
late into the night behind the incumbent, Donna Korn, who would need at
least 50 percent of the vote to avoid a runoff.

``I'm so excited to honor my daughter, Alyssa, and the 16 other victims
in the Parkland shooting, and to be able to be the voice of change,''
Ms. Alhadeff, 43, said after her victory.

The Parkland shooting drew much more attention than most other school
killings in part because student survivors and victims' families in the
affluent and mostly liberal community almost immediately
\href{https://www.nytimes3xbfgragh.onion/2018/08/15/us/politics/parkland-students-voting.html}{engaged
in gun politics}, speaking loudly and forcefully to state and federal
lawmakers. Several families started their own school safety and
awareness efforts, and some parents even appeared in television ads for
Democratic candidates for Florida governor, advocating stricter gun
control.

At the local level, however, the attention turned to the school
district, where Parkland families emerged as critics of Superintendent
Robert W. Runcie's administration, saying he should have been more
transparent about the education system's dealings with the former
student accused in the killings, and taken more decisive steps to secure
school campuses in the wake of the shooting.

{[}\emph{\href{https://www.nytimes3xbfgragh.onion/2018/08/28/us/politics/florida-arizona-election-results.html}{Get
more results from Tuesday's primary in Florida.}}{]}

Mr. Runcie's employment appeared at risk if his critics won school board
seats. But most of the incumbents supportive of the superintendent
fended off their challengers on Tuesday. Ms. Alhadeff has been less
critical than Mr. Petty of the superintendent, saying she will wait to
make up her mind until she takes her seat and reviews the inner workings
of his administration.

Mr. Runcie has defended his actions, acknowledging some mistakes but
noting that no playbook exists for how to prevent or respond to a
devastating school rampage.

``There's no easy fix, and it's going to take us a while to go through
this very difficult process,'' he told reporters in Parkland this month.
``But at this moment in our journey, what we need in this school system,
in our community, is stability in leadership.''

Ms. Alhadeff won a Parkland-based seat where the incumbent, Abby
Freedman, chose not to seek re-election. One of her opponents, Tennille
Doe-Decoste, is the mother of the best friend of Joaquin Oliver, one of
the students killed at Stoneman Douglas. Another candidate, Richard
Mendelson, a former Stoneman Douglas teacher, is a friend of the family
of Aaron Feis, a coach who died in the shooting. Mr. Mendelson lost to
an incumbent, Laurie Rich Levinson.

Mr. Petty tried to oust Ms. Korn, a seven-year incumbent, in a
countywide seat. He faced criticism for old posts on Twitter that Ms.
Korn said showed him to be bigoted. He apologized for his past comments.

A group of Parkland victims' families, Stand With Parkland, had called
for wholesale change on the school board. Some parents took particular
issue with a speech Ms. Korn delivered to district employees this month,
in which she said the previous school year had been ``amazing.''

``It was difficult to hear those words,'' said April Schentrup, whose
16-year-old daughter, Carmen, was killed in the shooting. Ms. Schentrup
was a public school principal at the time of the shooting, but was
reassigned afterward to oversee district safety and security. ``We know
it wasn't the best year in Broward schools,'' she said.

Ms. Korn apologized, but the controversy underscored the ugly tone some
school board races took in the weeks that followed. The police were even
called after candidates and their volunteers verbally tussled at early
voting polls.

Advertisement

\protect\hyperlink{after-bottom}{Continue reading the main story}

\hypertarget{site-index}{%
\subsection{Site Index}\label{site-index}}

\hypertarget{site-information-navigation}{%
\subsection{Site Information
Navigation}\label{site-information-navigation}}

\begin{itemize}
\tightlist
\item
  \href{https://help.nytimes3xbfgragh.onion/hc/en-us/articles/115014792127-Copyright-notice}{©~2020~The
  New York Times Company}
\end{itemize}

\begin{itemize}
\tightlist
\item
  \href{https://www.nytco.com/}{NYTCo}
\item
  \href{https://help.nytimes3xbfgragh.onion/hc/en-us/articles/115015385887-Contact-Us}{Contact
  Us}
\item
  \href{https://www.nytco.com/careers/}{Work with us}
\item
  \href{https://nytmediakit.com/}{Advertise}
\item
  \href{http://www.tbrandstudio.com/}{T Brand Studio}
\item
  \href{https://www.nytimes3xbfgragh.onion/privacy/cookie-policy\#how-do-i-manage-trackers}{Your
  Ad Choices}
\item
  \href{https://www.nytimes3xbfgragh.onion/privacy}{Privacy}
\item
  \href{https://help.nytimes3xbfgragh.onion/hc/en-us/articles/115014893428-Terms-of-service}{Terms
  of Service}
\item
  \href{https://help.nytimes3xbfgragh.onion/hc/en-us/articles/115014893968-Terms-of-sale}{Terms
  of Sale}
\item
  \href{https://spiderbites.nytimes3xbfgragh.onion}{Site Map}
\item
  \href{https://help.nytimes3xbfgragh.onion/hc/en-us}{Help}
\item
  \href{https://www.nytimes3xbfgragh.onion/subscription?campaignId=37WXW}{Subscriptions}
\end{itemize}
