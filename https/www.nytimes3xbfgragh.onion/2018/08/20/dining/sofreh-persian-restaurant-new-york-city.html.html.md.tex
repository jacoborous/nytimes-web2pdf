Sections

SEARCH

\protect\hyperlink{site-content}{Skip to
content}\protect\hyperlink{site-index}{Skip to site index}

\href{https://www.nytimes3xbfgragh.onion/section/food}{Food}

\href{https://myaccount.nytimes3xbfgragh.onion/auth/login?response_type=cookie\&client_id=vi}{}

\href{https://www.nytimes3xbfgragh.onion/section/todayspaper}{Today's
Paper}

\href{/section/food}{Food}\textbar{}At 59, a Gutsy Chef Makes Her
Restaurant Debut

\url{https://nyti.ms/2ORNfFM}

\begin{itemize}
\item
\item
\item
\item
\item
\item
\end{itemize}

Advertisement

\protect\hyperlink{after-top}{Continue reading the main story}

Supported by

\protect\hyperlink{after-sponsor}{Continue reading the main story}

\hypertarget{at-59-a-gutsy-chef-makes-her-restaurant-debut}{%
\section{At 59, a Gutsy Chef Makes Her Restaurant
Debut}\label{at-59-a-gutsy-chef-makes-her-restaurant-debut}}

You won't necessarily find kebabs at Sofreh, Nasim Alikhani's new
restaurant in Brooklyn, but a bold, modern take on Persian cuisine.

\includegraphics{https://static01.graylady3jvrrxbe.onion/images/2018/08/22/dining/22sofreh1/merlin_142072224_4e94f6ba-00c2-4f83-9090-431217fc5416-articleLarge.jpg?quality=75\&auto=webp\&disable=upscale}

\href{https://www.nytimes3xbfgragh.onion/by/melissa-clark}{\includegraphics{https://static01.graylady3jvrrxbe.onion/images/2018/06/21/multimedia/author-melissa-clark/author-melissa-clark-thumbLarge.png}}

By \href{https://www.nytimes3xbfgragh.onion/by/melissa-clark}{Melissa
Clark}

\begin{itemize}
\item
  Aug. 20, 2018
\item
  \begin{itemize}
  \item
  \item
  \item
  \item
  \item
  \item
  \end{itemize}
\end{itemize}

In Persian cooking, herbs are measured in epic terms --- by the quart,
not the tablespoon.

Parsley, dill, cilantro, fenugreek and mint: Used in profusion, alone or
in combination, they play the role of the vegetable rather than a
garnish, adding their woodsy fragrance at every opportunity. Such was
fate of the feathery greens that on a recent afternoon sat in front of
Nasim Alikhani, the owner of \href{http://www.sofrehnyc.com/}{Sofreh}
restaurant in Prospect Heights, Brooklyn.

Ms. Alikhani was cooking spicy fish in tamarind sauce from the south of
Iran, which she learned from her best friend, Minoo Falsafi.

``I'm from Isfahan, which is a desert city, traditionally there's no
fish,'' she said as she used both hands to scoop the herbs into a hot
skillet.

``When Minoo first made this dish for me, I thought, `Wow, this is so
delicious,''' she said. ``It really opened my eyes. So what I'm doing
now is her recipe.'' The herbs sputtered and hissed.

``Except I'm changing it.''

Image

Ms. Alikhani simmers herbs, onions, turmeric and tamarind for a Persian
fish dish.Credit...Ellen Silverman for The New York Times

Image

The finished dish: cod with a spicy tamarind herb sauce.Credit...Ellen
Silverman for The New York Times

Ms. Alikhani was a law student at the time of the Iranian revolution.
She moved to New York by herself at age 23, without speaking much
English. Eventually, she became a nanny for an Iranian family.

``That job saved me,'' she recounted, as she caramelized onions in a pan
next to the herbs. ``I was in a bad way, very lonely and isolated.''

To assuage her homesickness, she started cooking regularly for the first
time in her life. She found Persian ingredients at markets in Queens,
and with practice she built her skills.

``It was always very simple things, rice and lentils,'' she said. ``But
for the children, it was something new. They loved it.''

After nine years in New York, she was able to return to Iran to visit
her family. Tasting Persian food again after such a long absence made
her realize its wealth, complexity and diversity. She saw how the
flavors build on one another, how experienced cooks coaxed different
nuances out of herbs, spices, and all manner of meats and vegetables
depending on the technique they used.

Whenever she went back to visit, she traveled across the country,
cooking, eating, writing down recipes and immersing herself in the
history of the ancient cuisine.

She learned, for example, that the fish in tamarind sauce probably
originated with the African population living around Abadan, near the
water. Then the dish traveled to Isfahan and beyond as people relocated
after the Iran-Iraq war, taking their ingredients and traditions with
them.

\includegraphics{https://static01.graylady3jvrrxbe.onion/images/2018/08/22/dining/22sofreh2/merlin_142072491_0c80303a-2922-4a23-b05f-e3bfdf3b0ece-articleLarge.jpg?quality=75\&auto=webp\&disable=upscale}

By then, she had married, and had two children. Back in New York with
her family, she tried out all her new recipes on her friends and
neighbors, even catering huge Persian feasts for hundreds of people. The
idea to open a restaurant was always in the back of her mind, but she
put it off year after year, waiting for her children to grow up.

She started the process in earnest when they were in middle school, but
it wasn't until they were out of college that she finally was able to
open Sofreh in June. Just after she opened, Ms. Alikhani turned 59.

She interrupted her story as the herbs and onions were starting to
shrivel and blacken.

``You see, they are really burning,'' she said, pointing to the sticky
brown coating on the bottom of both pans. ``I do this intentionally.
This is how to build the flavors.'' She added that this is not how the
sauce is made in Iran, where it's cooked much more slowly at a lower
heat. But she prefers the complexity of the dark caramelization.

This kind of updated, gutsy touch is evident across her menu, where
she's not afraid to rebel against tradition.

``I'm going to upset some people because I'm going to do pork and I'm
going to do scallops, which you can't get in Iran,'' she said.

Image

From left, Soroosh Golbabae, one of Sofreh's chefs, and Ms. Alikhani
working dinner service. Credit...Ellen Silverman for The New York Times

During the day before the restaurant opens, she cooks alongside her two
chefs, Ali Saboor and Soroosh Golbabae, simmering intricate sauces;
preparing fresh yogurt; and preserving fruits and vegetables like
garlic, lemons, cherries and grapes.

Her chefs take over entirely once customers start arriving, allowing Ms.
Alikhani to float from table to table in her small restaurant, as if she
were hosting a giant dinner party in her home.

``Originally, I wanted to serve everyday food like my family would make,
\href{https://www.nytimes3xbfgragh.onion/2016/04/08/t-magazine/food/persian-frittata-michael-pollan-samin-nosrat.html}{kuku},
salads, rice,'' she told me as she put some seared cod into the herb
sauce to gently simmer, allowing the flavors to meld. (Kuku, an herb
frittata-like dish, is a classic of Persian cuisine.)

``Then I realized, I cannot start with this,'' she said. ``I felt I
needed to show my guests more special occasion food, how Persians throw
parties. Because when we have a party, we throw a feast.''

Recipe:
\textbf{\href{https://cooking.nytimes3xbfgragh.onion/recipes/1019481-persian-cod-with-herbs-and-tamarind}{Persian
Cod With Herbs and Tamarind}}

\emph{Follow} \emph{\href{https://twitter.com/nytfood}{NYT Food on
Twitter}} \emph{and}
\emph{\href{https://www.instagram.com/nytcooking/}{NYT Cooking on
Instagram},}
\emph{\href{https://www.facebookcorewwwi.onion/nytcooking/}{Facebook}}
\emph{and}
\emph{\href{https://www.pinterest.com/nytcooking/}{Pinterest}.}
\emph{\href{https://www.nytimes3xbfgragh.onion/newsletters/cooking}{Get
regular updates from NYT Cooking, with recipe suggestions, cooking tips
and shopping advice}.}

Advertisement

\protect\hyperlink{after-bottom}{Continue reading the main story}

\hypertarget{site-index}{%
\subsection{Site Index}\label{site-index}}

\hypertarget{site-information-navigation}{%
\subsection{Site Information
Navigation}\label{site-information-navigation}}

\begin{itemize}
\tightlist
\item
  \href{https://help.nytimes3xbfgragh.onion/hc/en-us/articles/115014792127-Copyright-notice}{©~2020~The
  New York Times Company}
\end{itemize}

\begin{itemize}
\tightlist
\item
  \href{https://www.nytco.com/}{NYTCo}
\item
  \href{https://help.nytimes3xbfgragh.onion/hc/en-us/articles/115015385887-Contact-Us}{Contact
  Us}
\item
  \href{https://www.nytco.com/careers/}{Work with us}
\item
  \href{https://nytmediakit.com/}{Advertise}
\item
  \href{http://www.tbrandstudio.com/}{T Brand Studio}
\item
  \href{https://www.nytimes3xbfgragh.onion/privacy/cookie-policy\#how-do-i-manage-trackers}{Your
  Ad Choices}
\item
  \href{https://www.nytimes3xbfgragh.onion/privacy}{Privacy}
\item
  \href{https://help.nytimes3xbfgragh.onion/hc/en-us/articles/115014893428-Terms-of-service}{Terms
  of Service}
\item
  \href{https://help.nytimes3xbfgragh.onion/hc/en-us/articles/115014893968-Terms-of-sale}{Terms
  of Sale}
\item
  \href{https://spiderbites.nytimes3xbfgragh.onion}{Site Map}
\item
  \href{https://help.nytimes3xbfgragh.onion/hc/en-us}{Help}
\item
  \href{https://www.nytimes3xbfgragh.onion/subscription?campaignId=37WXW}{Subscriptions}
\end{itemize}
