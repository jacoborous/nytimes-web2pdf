Sections

SEARCH

\protect\hyperlink{site-content}{Skip to
content}\protect\hyperlink{site-index}{Skip to site index}

\href{https://www.nytimes3xbfgragh.onion/section/food}{Food}

\href{https://myaccount.nytimes3xbfgragh.onion/auth/login?response_type=cookie\&client_id=vi}{}

\href{https://www.nytimes3xbfgragh.onion/section/todayspaper}{Today's
Paper}

\href{/section/food}{Food}\textbar{}At Kopitiam, Malaysian Food Powers
Through Some Growing Pains

\url{https://nyti.ms/2N0kYeH}

\begin{itemize}
\item
\item
\item
\item
\item
\item
\end{itemize}

Advertisement

\protect\hyperlink{after-top}{Continue reading the main story}

Supported by

\protect\hyperlink{after-sponsor}{Continue reading the main story}

\href{/column/restaurant-review}{Restaurant Review}

\hypertarget{at-kopitiam-malaysian-food-powers-through-some-growing-pains}{%
\section{At Kopitiam, Malaysian Food Powers Through Some Growing
Pains}\label{at-kopitiam-malaysian-food-powers-through-some-growing-pains}}

\href{https://www.nytimes3xbfgragh.onion/slideshow/2018/09/25/dining/kopitiam-nyc.html}{}

\hypertarget{malaysias-coffee-and-tea-culture-take-root-in-chinatown}{%
\subsection{Malaysia's Coffee and Tea Culture Take Root in
Chinatown}\label{malaysias-coffee-and-tea-culture-take-root-in-chinatown}}

10 Photos

View Slide Show ›

\includegraphics{https://static01.graylady3jvrrxbe.onion/images/2018/09/19/dining/26REST-slide-G5GU/26REST-slide-G5GU-articleLarge.jpg?quality=75\&auto=webp\&disable=upscale}

Jeenah Moon for The New York Times

\begin{itemize}
\tightlist
\item
  Kopitiam\\
  **NYT Critic's Pick ★ Malaysian \$ 151 East Broadway 646-609-3785
\end{itemize}

By \href{https://www.nytimes3xbfgragh.onion/by/pete-wells}{Pete Wells}

\begin{itemize}
\item
  Sept. 25, 2018
\item
  \begin{itemize}
  \item
  \item
  \item
  \item
  \item
  \item
  \end{itemize}
\end{itemize}

Sometimes when I am waiting, and waiting, for a barista to steam milk
for my cortado, I like to pass the time by imagining what life would be
like if Howard Schultz had boarded the wrong plane in 1981 and flown to
Penang instead of Milan.

Starbucks, the company he joined after that trip and would
\href{https://www.nytimes3xbfgragh.onion/2018/06/04/business/dealbook/starbucks-howard-schultz.html}{eventually
lead}, might teach its employees the fine arts of Malaysian kopi, brewed
by pouring hot water over grounds in the bottom of a sock, rather than
show them how to operate an espresso machine. Instead of white-chocolate
mocha lattes, we would drink bek-kopi, white coffee, brewed from beans
that are lightly roasted in palm oil margarine. Venti would just be
Italian for a number, while kopi-C kosong, or strong black coffee with
unsweetened condensed milk, would have entered the American lexicon,
along with kopi tarik, the same drink made with sweetened condensed
milk. There would be no competitions where baristas drew swans and
elephants in milk froth, and in their place we would have
\href{https://www.youtube.com/watch?v=ij20GFJKJbU}{exhibitions of coffee
and tea pulling}, the art of cooling the beverage while coaxing the milk
in it to form a foamy head by pouring it from one pot to another in
long, swooping, steamy cataracts.

In other words, the places where we get our caffeine fix might look a
lot like \href{https://www.kopitiamnyc.com/}{Kopitiam}, on East Broadway
in Chinatown.

Kyo Pang founded Kopitiam three years ago as a homage to the coffee and
tea parlors of Malaysia, where she grew up. A typical Malaysian kopitiam
is a ramshackle hangout where people drop in to read the morning paper,
grab breakfast, catch up on neighborhood gossip and do all the other
things that classically accompany a cup of tea or coffee. Something of
that spirit pervades Kopitiam, too.

\includegraphics{https://static01.graylady3jvrrxbe.onion/images/2018/09/26/dining/26REST1/26REST1-articleLarge.jpg?quality=75\&auto=webp\&disable=upscale}

\emph{{[}Read about some of the}
\href{https://www.nytimes3xbfgragh.onion/2018/11/15/nyregion/best-new-nyc-restaurants.html?action=click\&module=Intentional\&pgtype=Article}{\emph{best
new restaurants in New York City}} \emph{(for now).{]}}

The original Kopitiam,
\href{https://www.nytimes3xbfgragh.onion/2015/12/09/dining/hungry-city-kopitiam-lower-east-side.html}{reviewed
by Ligaya Mishan} in a 2015 Hungry City column, was situated around the
corner on Canal Street and had just four seats at a single counter. In
June, Ms. Pang and her new business partner, Moonlynn Tsai, relocated to
a larger, if still fairly intimate, space a few steps above the
sidewalk. You order at the counter and receive a number on a metal
stand, a kind of promissory note for the food and beverages. If you show
up on a weekday in the morning or afternoon, when seats are not hard to
come by and the room has a languid, unhurried air, you could get just a
cup of bek-kopi and linger over it with a newspaper. You will probably
want more than that, though.

From 9 a.m., when the doors open, until 10 p.m., when they close, you
can get nasi lemak, Malaysia's national dish and a particular favorite
at breakfast. I don't know another kitchen in the city where the
fragrances of coconut and pandan leaf infuse the rice as elegantly, or
where the tiny dried fish, which Kopitiam fries with peanuts, form a
caramelized crust that erases the distinction between sweet and savory.

Or your breakfast could be a bowl of two eggs boiled just long enough to
turn the whites opaque while leaving the yolks free to billow into
mushroom-soy broth underneath. Or it might be fish ball soup, grape-size
globes of ground fish in a cloudy white broth; if the grains of white
pepper on the surface aren't intense enough for you, you can stir in a
little fish sauce spiced with bird's eye chiles. Rice vermicelli can be
added, too, but the soup is easier to appreciate in pure liquid form.

For mornings when nothing but a cannonball dive into sugar will do,
Kopitiam is ready with a sculptural pile of thick-cut French toast
battered with Milo malt-chocolate powder, a Malaysian passion, with
streams of sweetened condensed milk playing the part that in this
country is normally taken by maple syrup. Eat the whole stack and you
will know the answer to the musical question posed by the Cramps:
\href{https://www.youtube.com/watch?v=c-S9pPFkfLY}{How far can too far
go?}

At 10:30 a.m. Ms. Pang, the chef, adds a gang of new dishes to the
breakfast crew, and all of them stick around until closing time. There
are fresh little oysters embedded in a thin browned omelet, rice noodles
in a chilled sesame-oil broth with a growl of chile heat and spicy skate
steamed in banana leaves, the classic acoustic version of the dish that
Max Ng plays with amplifiers and wah-wah pedals at
\href{https://www.nytimes3xbfgragh.onion/2017/10/31/dining/momofuku-ssam-bar-review.html}{Momofuku
Ssam Bar}. Any of these could make a small meal.

But one of the most inviting aspects of Kopitiam is Ms. Pang's fidelity
to the Malaysian love of snacks. Pandan leaves wrapped around a kind of
chicken sausage; ground five-spice pork rolled inside bean-curd skins
and fried; fried strips of mackerel sausage with a curry-leaf sauce ---
any of them makes a fine appetizer, but each could just as easily be the
entire goal of a between-meals stop at Kopitiam. For that matter, so
could the small sweets like pulut: triangular, pandan-wrapped bundles of
sticky rice tinted with blue morning glory flowers and spread with
toasted coconut sugar or house-made coconut jam.

Image

Kyo Pang, who founded Kopitiam three years ago as a homage to the coffee
and tea parlors of Malaysia, where she grew up.Credit...Jeenah Moon for
The New York Times

Ms. Pang is devoted to cakes as well. You get the sense that if it were
physically possible, she'd bake every cake she knows every day, but she
settles for one or two in an unpredictable rotation. You may find the
caramel-brown honeycomb cake, with its intriguing structure of hollow
vertical columns at the bottom, or you may find her coconut cake, or
there may be some fresh surprise.

Snacks and sweets tend to get edged out of the spotlight in New York
restaurants, where there's always pressure to order something big. This
is, after all, a city that turned the casual tapas bar into the peculiar
genre known as the tapas restaurant. There is no such pressure inside
Kopitiam, at least on weekdays.

Weekend brunch is another story, as it often is. The menu is the same,
but the crowds thicken. A sign-in sheet is kept outside the door, next
to the pink pay phone (25 cents; it works), and hopeful customers sit on
ceramic garden stools at the top of the stoop listening for their names.
When you're finally called, you may feel silly getting nothing more than
tannic milky tea and a bundle of blue sticky rice.

And when demand peaks, chaos can creep in, suggesting that Kopitiam is
still figuring out the mechanics of its larger space. Servers may pop by
anxiously to ask, ``Did you get all your food?'' and the answer may be
no. Drink orders may stack up. Once the white coffee was off the menu;
another time, no coffee at all was being poured; and for a short time
one day, both coffee and tea were unavailable, an awkward state of
affairs for a coffee and tea parlor.

Even with a backlog of tickets, though, the food trickling out of the
kitchen is almost unfailingly terrific. Maybe the stir-fried duck
tongues I ate last time could have slid from their cartilaginous
framework more easily. Then again, maybe I wasn't in the right mood for
duck tongues.

\emph{Follow} \emph{\href{https://twitter.com/nytfood}{NYT Food on
Twitter}} \emph{and}
\emph{\href{https://www.instagram.com/nytcooking/}{NYT Cooking on
Instagram},}
\emph{\href{https://www.facebookcorewwwi.onion/nytcooking/}{Facebook}}
\emph{and}
\emph{\href{https://www.pinterest.com/nytcooking/}{Pinterest}.}
\emph{\href{https://www.nytimes3xbfgragh.onion/newsletters/cooking}{Get
regular updates from NYT Cooking, with recipe suggestions, cooking tips
and shopping advice}.}

Advertisement

\protect\hyperlink{after-bottom}{Continue reading the main story}

\hypertarget{site-index}{%
\subsection{Site Index}\label{site-index}}

\hypertarget{site-information-navigation}{%
\subsection{Site Information
Navigation}\label{site-information-navigation}}

\begin{itemize}
\tightlist
\item
  \href{https://help.nytimes3xbfgragh.onion/hc/en-us/articles/115014792127-Copyright-notice}{©~2020~The
  New York Times Company}
\end{itemize}

\begin{itemize}
\tightlist
\item
  \href{https://www.nytco.com/}{NYTCo}
\item
  \href{https://help.nytimes3xbfgragh.onion/hc/en-us/articles/115015385887-Contact-Us}{Contact
  Us}
\item
  \href{https://www.nytco.com/careers/}{Work with us}
\item
  \href{https://nytmediakit.com/}{Advertise}
\item
  \href{http://www.tbrandstudio.com/}{T Brand Studio}
\item
  \href{https://www.nytimes3xbfgragh.onion/privacy/cookie-policy\#how-do-i-manage-trackers}{Your
  Ad Choices}
\item
  \href{https://www.nytimes3xbfgragh.onion/privacy}{Privacy}
\item
  \href{https://help.nytimes3xbfgragh.onion/hc/en-us/articles/115014893428-Terms-of-service}{Terms
  of Service}
\item
  \href{https://help.nytimes3xbfgragh.onion/hc/en-us/articles/115014893968-Terms-of-sale}{Terms
  of Sale}
\item
  \href{https://spiderbites.nytimes3xbfgragh.onion}{Site Map}
\item
  \href{https://help.nytimes3xbfgragh.onion/hc/en-us}{Help}
\item
  \href{https://www.nytimes3xbfgragh.onion/subscription?campaignId=37WXW}{Subscriptions}
\end{itemize}
