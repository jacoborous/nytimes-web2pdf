Sections

SEARCH

\protect\hyperlink{site-content}{Skip to
content}\protect\hyperlink{site-index}{Skip to site index}

\href{https://www.nytimes3xbfgragh.onion/section/books/review}{Book
Review}

\href{https://myaccount.nytimes3xbfgragh.onion/auth/login?response_type=cookie\&client_id=vi}{}

\href{https://www.nytimes3xbfgragh.onion/section/todayspaper}{Today's
Paper}

\href{/section/books/review}{Book Review}\textbar{}February's Book Club
Pick: The Dickensian Conditions of Life in a For-Profit Lockup

\url{https://nyti.ms/2QgtrfQ}

\begin{itemize}
\item
\item
\item
\item
\item
\item
\end{itemize}

Advertisement

\protect\hyperlink{after-top}{Continue reading the main story}

Supported by

\protect\hyperlink{after-sponsor}{Continue reading the main story}

Nonfiction

\hypertarget{februarys-book-club-pick-the-dickensian-conditions-of-life-in-a-for-profit-lockup}{%
\section{February's Book Club Pick: The Dickensian Conditions of Life in
a For-Profit
Lockup}\label{februarys-book-club-pick-the-dickensian-conditions-of-life-in-a-for-profit-lockup}}

\includegraphics{https://static01.graylady3jvrrxbe.onion/images/2018/10/07/books/review/07Blakeslee-COVER/merlin_144320568_122dfc9a-667b-469c-81ae-9d01dbf97097-articleLarge.jpg?quality=75\&auto=webp\&disable=upscale}

Buy Book ▾

\begin{itemize}
\tightlist
\item
  \href{https://www.amazon.com/gp/search?index=books\&tag=NYTBSREV-20\&field-keywords=American+Prison\%3A+A+Reporter\%E2\%80\%99s+Undercover+Journey+Into+the+Business+of+Punishment+Shane+Bauer}{Amazon}
\item
  \href{https://du-gae-books-dot-nyt-du-prd.appspot.com/buy?title=American+Prison\%3A+A+Reporter\%E2\%80\%99s+Undercover+Journey+Into+the+Business+of+Punishment\&author=Shane+Bauer}{Apple
  Books}
\item
  \href{https://www.anrdoezrs.net/click-7990613-11819508?url=https\%3A\%2F\%2Fwww.barnesandnoble.com\%2Fw\%2F\%3Fean\%3D9780735223585}{Barnes
  and Noble}
\item
  \href{https://www.anrdoezrs.net/click-7990613-35140?url=https\%3A\%2F\%2Fwww.booksamillion.com\%2Fp\%2FAmerican\%2BPrison\%253A\%2BA\%2BReporter\%25E2\%2580\%2599s\%2BUndercover\%2BJourney\%2BInto\%2Bthe\%2BBusiness\%2Bof\%2BPunishment\%2FShane\%2BBauer\%2F9780735223585}{Books-A-Million}
\item
  \href{https://bookshop.org/a/3546/9780735223585}{Bookshop}
\item
  \href{https://www.indiebound.org/book/9780735223585?aff=NYT}{Indiebound}
\end{itemize}

When you purchase an independently reviewed book through our site, we
earn an affiliate commission.

By Nate Blakeslee

\begin{itemize}
\item
  Oct. 1, 2018
\item
  \begin{itemize}
  \item
  \item
  \item
  \item
  \item
  \item
  \end{itemize}
\end{itemize}

\emph{(This book was selected as one of The New York Times Book Review's
10 Best Books of 2018. For the rest of the list, click}
\href{https://www.nytimes3xbfgragh.onion/2018/11/29/books/review/best-books.html}{\emph{here}}\emph{.)}

\textbf{AMERICAN PRISON}\\
\textbf{A Reporter's Undercover Journey Into the Business of
Punishment}\\
By Shane Bauer\\
Illustrated. 351 pp. Penguin Press. \$28.

Two weeks after Shane Bauer, a senior reporter for Mother Jones
magazine, began his undercover stint as a prison guard at Winn
Correctional Center in rural Louisiana, an inmate jumped the razor-wire
fence and sprinted into the surrounding woods. There were no officers in
the guard towers to witness his escape; the private prison company hired
by the state of Louisiana to run the facility, Corrections Corporation
of America (CCA), had decided to save money by leaving those posts
empty. An alarm sounded in the control room, but a guard ostensibly
monitoring the battery of surveillance cameras saw nothing. Rather than
review the footage, she simply switched off the alarm and returned her
attention to whatever was occupying it beforehand. Hours passed before
anybody noticed that the inmate was gone.

The situation at Winn went downhill from there, as Bauer revealed in a
\href{https://www.motherjones.com/politics/2016/06/cca-private-prisons-corrections-corporation-inmates-investigation-bauer/}{35,000-word
exposé} that ran in Mother Jones in the summer of 2016, an article that
immediately became one of the most celebrated achievements in that
venerable publication's recent renaissance. ``American Prison'' reprises
that page-turning narrative, and adds not only the fascinating back
story of CCA, the nation's first private prison company, but also an
eye-opening examination of the history of corrections as a profit-making
enterprise, of which the advent of the private prisons that now house 8
percent of American inmates is only the latest chapter.

Image

Bauer's reporting has inevitably sparked comparisons to Ted Conover's
book ``Newjack,'' for which the renowned journalist spent a year in the
late 1990s working at New York's infamous Sing Sing prison to better
understand what it's like to pursue a career as a professional jailer.
But this is not what Bauer is writing about, because a career is not
what a company like CCA offers its employees. At Sing Sing, which is
staffed by state employees, new hires spend two months at a training
academy, enjoy good pay and benefits guaranteed by a strong union, are
supervised by seasoned officers and can look forward to a decent
retirement. At Winn, Bauer gets four weeks of training --- when his
instructors show up, that is --- and the pay starts at \$9 per hour. He
soon discovers that all guards earn that rate, no matter how long they
have worked at Winn. The only way to earn more is to make rank, but most
don't stay long enough to get promoted. Turnover is so high and staffing
so short that Bauer himself is asked to begin training cadets less than
seven weeks into his tenure.

The company's main concern seems to be maintaining parity with the local
Walmart, where the pay is comparable, and the conditions presumably less
anxiety-inducing. ``People say \ldots{} we'll hire anybody,'' the
prison's head of training tells Bauer and his fellow cadets. ``Which is
not really true, but if you come here and you breathing and you got a
valid driver's license and you willing to work, then we're willing to
hire you.'' (Yes, that is an exact quote --- Bauer carried a recording
device concealed in a pen.) If you like that line, you'll love this
book; the sheer number of forehead-slapping quotes from Bauer's
superiors and fellow guards alone are worth the price of admission.

Every management decision at Winn, Bauer discovers, is dictated by one
imperative: maintaining profitability by squeezing expenses. This begins
with the low pay, which leads to staffing shortages dire enough to
threaten the safety of both guards and inmates. But the crisis at Winn
goes much deeper. During his four-month tenure, Bauer documents a dozen
stabbings; scores of ``use of force'' incidents (far more than at
comparable state-run units); cell doors that can be opened by inmates;
atrocious medical care; and a seemingly preventable inmate suicide. He
records guards shamelessly admitting that they trained bloodhounds by
using actual inmates, beat inmates outside the view of cameras and
routinely failed to perform the most basic elements of their jobs.

``Ain't no order here,'' a convict says. ``Inmates run this bitch,
son.'' It is less of a boast than a complaint, because the situation is
dangerous for everybody involved. If Conover set out to discover what
it's like to be in charge of a prison, Bauer asks a different question:
What is it like to work --- or serve time --- in a prison where nobody
is in charge?

His survey of profit-driven incarceration begins in the mid-19th century
and strikes a familiar theme, that mass incarceration in the South was
simply slavery by another name. But Bauer adds new details, especially
about the history of convict leasing, in which entire prisons --- filled
mostly with African-American inmates --- were rented out to individuals
or companies to provide a captive work force. Convicts did more than
plant cotton. The textile mill inside Texas' first penitentiary became
the largest factory in the state, and inmates were used throughout the
South to dig mines and build railroads, generally working under horrible
conditions. Death rates were staggeringly high; convicts, unlike slaves,
cost nothing to replace. As much as anything, this is the story of the
South trying to compete with Northern industry without disturbing the
region's existing power structure, which is to say, without labor
unions. Inmates were the original scabs.

Image

Shane BauerCredit...Ted Ely

Prisons operated by companies like CCA (recently rebranded as
CoreCivic), which was founded in 1983 and is now a \$3.04 billion
publicly traded concern, don't typically grow crops or manufacture
anything of value. Inmates themselves are the commodities, and money is
made by persuading legislators that a private operator can confine and
care for them more cheaply --- in Winn's case, \$34 per inmate per day
--- than the state. Of course, penny-pinching and staff shortages are
found at state-run lockups, too (especially in places like Louisiana),
but there is another consideration when a profit-seeking middleman gets
involved: What happens when setting a prisoner free is detrimental to a
company's bottom line? Bauer discovers that a Winn inmate was held for a
full year after he was eligible for release, ostensibly because he had
no address in Louisiana that would take him in --- a technicality that
presumably earned the company an additional \$12,410 from his continued
incarceration.

It's not just the convicts who are being exploited. Most of the guards
at Winn, like Bauer himself, are afraid of their charges and resentful
of the chaos that makes their jobs more dangerous. Bauer is a generous
narrator with a nice ear for detail, and his colleagues come across as
sympathetic characters, with a few notable exceptions. In a wonderful
twist, he interviews a number of them after his deception is eventually
exposed. How much loyalty does \$9 per hour buy? About as much as you'd
imagine; most are all too happy to help pull the curtain back on CCA.

Bauer's takeaway is that private prisons like Winn can't be fixed, that
the profit motive inevitably drives companies to take risks and cut
corners. He's not the only one to draw that conclusion; after Mother
Jones ran his article, Bauer was invited to Washington to talk to
federal officials studying the efficacy of private prisons. Not long
after, the Obama administration announced that the Department of Justice
would no longer contract with CCA or its ilk. It turned out to be a
short-lived mandate, reversed almost immediately by the incoming
attorney general, Jeff Sessions. Today the industry is thriving thanks
to a bull market in immigrant detention.

As for Winn, conditions at the unit continued to deteriorate until
personnel from a state-run prison stormed in and temporarily regained
control. Two weeks after Bauer left, CCA voluntarily withdrew from its
contract, essentially admitting that the facility was a lost cause.
Louisiana officials didn't see it that way, however. Another company,
LaSalle Corrections, promptly took over management of Winn, though the
state is no longer paying CCA's bargain rate of \$34 per day.

LaSalle agreed to do the job for \$24.

Advertisement

\protect\hyperlink{after-bottom}{Continue reading the main story}

\hypertarget{site-index}{%
\subsection{Site Index}\label{site-index}}

\hypertarget{site-information-navigation}{%
\subsection{Site Information
Navigation}\label{site-information-navigation}}

\begin{itemize}
\tightlist
\item
  \href{https://help.nytimes3xbfgragh.onion/hc/en-us/articles/115014792127-Copyright-notice}{©~2020~The
  New York Times Company}
\end{itemize}

\begin{itemize}
\tightlist
\item
  \href{https://www.nytco.com/}{NYTCo}
\item
  \href{https://help.nytimes3xbfgragh.onion/hc/en-us/articles/115015385887-Contact-Us}{Contact
  Us}
\item
  \href{https://www.nytco.com/careers/}{Work with us}
\item
  \href{https://nytmediakit.com/}{Advertise}
\item
  \href{http://www.tbrandstudio.com/}{T Brand Studio}
\item
  \href{https://www.nytimes3xbfgragh.onion/privacy/cookie-policy\#how-do-i-manage-trackers}{Your
  Ad Choices}
\item
  \href{https://www.nytimes3xbfgragh.onion/privacy}{Privacy}
\item
  \href{https://help.nytimes3xbfgragh.onion/hc/en-us/articles/115014893428-Terms-of-service}{Terms
  of Service}
\item
  \href{https://help.nytimes3xbfgragh.onion/hc/en-us/articles/115014893968-Terms-of-sale}{Terms
  of Sale}
\item
  \href{https://spiderbites.nytimes3xbfgragh.onion}{Site Map}
\item
  \href{https://help.nytimes3xbfgragh.onion/hc/en-us}{Help}
\item
  \href{https://www.nytimes3xbfgragh.onion/subscription?campaignId=37WXW}{Subscriptions}
\end{itemize}
