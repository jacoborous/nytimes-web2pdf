Sections

SEARCH

\protect\hyperlink{site-content}{Skip to
content}\protect\hyperlink{site-index}{Skip to site index}

\href{https://www.nytimes3xbfgragh.onion/section/food}{Food}

\href{https://myaccount.nytimes3xbfgragh.onion/auth/login?response_type=cookie\&client_id=vi}{}

\href{https://www.nytimes3xbfgragh.onion/section/todayspaper}{Today's
Paper}

\href{/section/food}{Food}\textbar{}A Brooklyn Favorite for Jamaican
Food Beats the Odds

\url{https://nyti.ms/2MZAbxL}

\begin{itemize}
\item
\item
\item
\item
\item
\item
\end{itemize}

Advertisement

\protect\hyperlink{after-top}{Continue reading the main story}

Supported by

\protect\hyperlink{after-sponsor}{Continue reading the main story}

\href{/column/restaurant-review}{Restaurant Review}

\hypertarget{a-brooklyn-favorite-for-jamaican-food-beats-the-odds}{%
\section{A Brooklyn Favorite for Jamaican Food Beats the
Odds}\label{a-brooklyn-favorite-for-jamaican-food-beats-the-odds}}

\href{https://www.nytimes3xbfgragh.onion/slideshow/2018/06/26/dining/the-islands-restaurant-brooklyn.html}{}

\hypertarget{a-second-life-for-a-jamaican-kitchen}{%
\subsection{A Second Life for a Jamaican
Kitchen}\label{a-second-life-for-a-jamaican-kitchen}}

8 Photos

View Slide Show ›

\includegraphics{https://static01.graylady3jvrrxbe.onion/images/2018/06/27/dining/27rest-1/merlin_139824474_c6248451-0dab-4262-a222-6b0dc941ac28-articleLarge.jpg?quality=75\&auto=webp\&disable=upscale}

Cole Wilson for The New York Times

\begin{itemize}
\tightlist
\item
  The Islands\\
  **NYT Critic's Pick ★ Caribbean \$ 671 Washington Avenue 718-398-3575
\end{itemize}

By \href{https://www.nytimes3xbfgragh.onion/by/pete-wells}{Pete Wells}

\begin{itemize}
\item
  June 26, 2018
\item
  \begin{itemize}
  \item
  \item
  \item
  \item
  \item
  \item
  \end{itemize}
\end{itemize}

The Islands
\href{https://www.nytimes3xbfgragh.onion/2005/06/01/dining/reviews/in-brooklyn-a-breath-of-the-tropics.html}{used
to be a two-story restaurant} on the Crown Heights side of Washington
Avenue in Brooklyn, near the top of the hill, within sight of the
Brooklyn Museum. Upstairs was a cramped, angular, shadowy dining room
with a few tables under a ceiling so low it made tall people stoop.
Downstairs was a cramped, angular, shadowy kitchen where oxtails, goat
curry and other Jamaican dishes were cooked and sold.

To the right of the door, behind a counter, was the restaurant's single
stove. It had three burners that were almost always going. No matter
when I went there it seemed that about a quarter of the things on the
menu had not finished cooking, but complaining would have been cloddish,
both because the limitations on capacity were obvious and because
whatever was finished was bound to be both nuanced and powerful.

The whole operation was one of those precarious marvels of adaptive
survival, like a pine growing from a crack in the side of a cliff. So
the outlook was grim when news came two summers ago that the structures
housing the Islands and another business were being torn down so a new
apartment building could be constructed. This neighborhood and the ones
to the south between Flatbush and Ralph Avenues owe much of their
character to black- and immigrant-owned businesses, including a large
swath of Caribbean restaurants such as the Islands. With developers
sniffing at this part of Brooklyn, many of them are at risk.

Marilyn Reid, who owns the Islands with Shuon Letchford, had other
ideas. She had had other ideas since 2001, when she opened the
restaurant. ``We were always looking because we had outgrown it by the
second day,'' she said in a phone interview. ``We knew.''

Before closing the original space in March 2017, she and Ms. Letchford
had found a much bigger one downhill on Washington Avenue. Preparing it
took a few months more than she had hoped, but the second incarnation of
the Islands has been up and running since January.

\includegraphics{https://static01.graylady3jvrrxbe.onion/images/2018/06/27/dining/27rest-9/merlin_139824579_63bb32bb-3c18-404b-9f1a-0e4a331df94f-articleLarge.jpg?quality=75\&auto=webp\&disable=upscale}

\emph{{[}Read about some of the}
\href{https://www.nytimes3xbfgragh.onion/2018/11/15/nyregion/best-new-nyc-restaurants.html?action=click\&module=Intentional\&pgtype=Article}{\emph{best
new restaurants in New York City}} \emph{(for now).{]}}

The new kitchen has two stoves and 10 burners. Nevertheless, what's
available when I've gone has always been several dishes shy of the
complete menu. The stewed okra with cod is especially hard to pin down.
At the same time, the comfort of the bigger dining room and the table
service, an impossibility in the first space, have encouraged me to
branch out beyond my usual favorites.

The Islands is best known for long-simmered stews but the kitchen also
knows a thing or two about frying. The flattened fritters of salt cod
seasoned with fresh hot chiles, known as stamp and go, were crunchy and
golden. Some Caribbean cooks deep-fry fish escovitch to oblivion, on the
theory that spicy vinegar heals all wounds. The escovitch I had at the
Islands was a whole red snapper, obviously fresh and fried just until
firm. The vinegar dressing, meanwhile, was seasoned with not just hot
chiles but pickled bell peppers, carrots and cucumbers, making this
escovitch more interesting than usual.

From the old days, I knew the curried goat would be very tender and
richly seasoned, although like most of the cooking at the Islands, it
was never scalp-piercingly spicy. Now I've found a new standby in the
curried vegetables --- a very different thing, of course, but the golden
raisins, turmeric and allspice may do even more for cabbage and carrots
than they do for goat meat.

Barbecue chicken was ordered in a rush verging on panic because whatever
I'd first wanted wasn't ready yet. The sauce is nothing like the
ketchup-based sludge I'd imagined. Made with raisins and allspice, it
tastes more like a glaze for a holiday ham, and it's lovely.

I prefer it to the jerk chicken, which will sound like heresy to some
customers. Like barbecue, jerk can describe either a sauce or a process,
and the process needs to involve wood smoke. The Islands bakes its jerk
chicken in the oven, and while I like its sauce, when I want jerk
chicken I will cross Eastern Parkway and keep the windows down until I
smell burning allspice logs.

But I have no quarrel with the jerk shrimp or leg of lamb at the Islands
--- there's no place to get an allspice-smoked version of either, as far
as I know. The lamb is especially alluring, stuffed with jerk spices and
slowly roasted until its outer crust falls, upon slicing, into a dark
rubble that in New Orleans would be called debris.

Rice and peas, fried plantains and stewed cabbage, slippery with coconut
oil, are all mandatory in my book. Others insist on baked macaroni and
cheese, served in a brick the size of a pound of butter. When I taste
the bread crumbs on top, and wonder if they've been seasoned with dried
onions, I think those people may have a point.

Image

The new restaurant has a larger dining room and table
service.Credit...Cole Wilson for The New York Times

All of this food is brought to the table on oval platters that could
hold a small Thanksgiving turkey.

Since opening in January, the new dining room has been embellished bit
by bit. The walls are painted in watermelon and aquamarine. Small
hurricane lamps decorate the tables. A couple of weeks ago, plaid cloth
napkins made their debut in the middle of dinner service. Menus, which
used to be kept inside a clear plastic sheath, are now presented inside
folders made of textured handmade paper, held in place with stickers of
bumblebees and butterflies.

The bar, across the dining room from the kitchen, is still unoccupied
while Ms. Reid waits for a liquor license to come through. A rumored
roof deck has not materialized, either. In the meantime there is an
excellent limeade sweetened with demerara sugar and a dark purple,
fruity sorrel punch, which can give the impression it has been lightly
spiked with rum.

Credit for the food at the Islands is shared. There are two chefs,
Delroy Henry and Ronald Porter. Most of the recipes, though, are Ms.
Letchford's, and she does a lot of the cooking. Like Ms. Reid, all of
them were raised in Jamaica before moving to New York, but only Ms.
Letchford did any professional cooking there. Ms. Reid is often in the
kitchen too. ``I do the light stuff,'' she said. ``I tend to stay with
salads, seafood.''

Is the food as good as before? I used to be impressed that anything at
all could be cooked at the old Islands. That so much of it was actively
delicious seemed like a miracle. It's less astonishing now that the
kitchen has 10 burners and room to turn around.

But I suspect the only thing that's changed is me. And when I'm in the
mood for something miraculous, I'll just remind myself that the Islands
is still in business. You can't stop gentrification, but sometimes you
can fool it while it's looking the other way.

\href{https://www.facebookcorewwwi.onion/nytfood/}{\emph{Follow NYT Food
on Facebook}}\emph{,}
\href{https://instagram.com/nytfood}{\emph{Instagram}}\emph{,}
\href{https://twitter.com/nytfood}{\emph{Twitter}} \emph{and}
\href{https://www.pinterest.com/nytfood/}{\emph{Pinterest}}\emph{.}
\href{https://www.nytimes3xbfgragh.onion/newsletters/cooking}{\emph{Get
regular updates from NYT Cooking, with recipe suggestions, cooking tips
and shopping advice}}\emph{.}

Advertisement

\protect\hyperlink{after-bottom}{Continue reading the main story}

\hypertarget{site-index}{%
\subsection{Site Index}\label{site-index}}

\hypertarget{site-information-navigation}{%
\subsection{Site Information
Navigation}\label{site-information-navigation}}

\begin{itemize}
\tightlist
\item
  \href{https://help.nytimes3xbfgragh.onion/hc/en-us/articles/115014792127-Copyright-notice}{©~2020~The
  New York Times Company}
\end{itemize}

\begin{itemize}
\tightlist
\item
  \href{https://www.nytco.com/}{NYTCo}
\item
  \href{https://help.nytimes3xbfgragh.onion/hc/en-us/articles/115015385887-Contact-Us}{Contact
  Us}
\item
  \href{https://www.nytco.com/careers/}{Work with us}
\item
  \href{https://nytmediakit.com/}{Advertise}
\item
  \href{http://www.tbrandstudio.com/}{T Brand Studio}
\item
  \href{https://www.nytimes3xbfgragh.onion/privacy/cookie-policy\#how-do-i-manage-trackers}{Your
  Ad Choices}
\item
  \href{https://www.nytimes3xbfgragh.onion/privacy}{Privacy}
\item
  \href{https://help.nytimes3xbfgragh.onion/hc/en-us/articles/115014893428-Terms-of-service}{Terms
  of Service}
\item
  \href{https://help.nytimes3xbfgragh.onion/hc/en-us/articles/115014893968-Terms-of-sale}{Terms
  of Sale}
\item
  \href{https://spiderbites.nytimes3xbfgragh.onion}{Site Map}
\item
  \href{https://help.nytimes3xbfgragh.onion/hc/en-us}{Help}
\item
  \href{https://www.nytimes3xbfgragh.onion/subscription?campaignId=37WXW}{Subscriptions}
\end{itemize}
