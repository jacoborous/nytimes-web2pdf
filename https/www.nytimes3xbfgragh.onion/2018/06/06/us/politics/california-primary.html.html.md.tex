Sections

SEARCH

\protect\hyperlink{site-content}{Skip to
content}\protect\hyperlink{site-index}{Skip to site index}

\href{https://www.nytimes3xbfgragh.onion/section/politics}{Politics}

\href{https://myaccount.nytimes3xbfgragh.onion/auth/login?response_type=cookie\&client_id=vi}{}

\href{https://www.nytimes3xbfgragh.onion/section/todayspaper}{Today's
Paper}

\href{/section/politics}{Politics}\textbar{}Gavin Newsom and John Cox to
Compete in California Election for Governor

\url{https://nyti.ms/2HogVpy}

\begin{itemize}
\item
\item
\item
\item
\item
\item
\end{itemize}

Advertisement

\protect\hyperlink{after-top}{Continue reading the main story}

Supported by

\protect\hyperlink{after-sponsor}{Continue reading the main story}

\hypertarget{gavin-newsom-and-john-cox-to-compete-in-california-election-for-governor}{%
\section{Gavin Newsom and John Cox to Compete in California Election for
Governor}\label{gavin-newsom-and-john-cox-to-compete-in-california-election-for-governor}}

\includegraphics{https://static01.graylady3jvrrxbe.onion/images/2018/06/06/us/06CALIFORNIA-dypt/06NEWSOM-COMBO-videoSixteenByNine3000.jpg}

By \href{http://www.nytimes3xbfgragh.onion/by/adam-nagourney}{Adam
Nagourney} and
\href{http://www.nytimes3xbfgragh.onion/by/alexander-burns}{Alexander
Burns}

\begin{itemize}
\item
  June 6, 2018
\item
  \begin{itemize}
  \item
  \item
  \item
  \item
  \item
  \item
  \end{itemize}
\end{itemize}

LOS ANGELES --- Gavin Newsom, the Democratic lieutenant governor and
former mayor of San Francisco, took a major step Tuesday in his bid to
become California's next governor,
\href{https://vote.sos.ca.gov/returns/governor}{capturing one of two
spots}on the November ballot as the state moved closer to the end of the
era of Gov. Jerry Brown.

John Cox, a Republican businessman backed by President Trump, captured
the other spot, setting up what is --- at best --- a very long-shot bid
for Mr. Cox in a decidedly Democratic state where Mr. Trump lost by
nearly four million votes.

Mr. Cox's showing represented a major tactical victory for national
Republicans as they seek to protect seven Republican-held congressional
seats in California that Democrats are targeting as they try to
recapture the House. Republican leaders, including Kevin McCarthy, the
House majority leader who comes from central California, had feared that
having no Republicans running for a high-profile statewide office would
diminish turnout among party voters in the fall.

Importantly, Democrats seemed poised to avoid the disaster they feared
in House races: Being shut out of the November balloting under the
state's so-called ``top-two'' primary system, in which only the top two
finishers advance to the general election. But many of the districts had
crowded primaries and in some of them votes were still being counted
early Wednesday morning.

The\href{https://www.nytimes3xbfgragh.onion/interactive/2018/06/01/us/elections/california-house-primary.html}{most-watched
races here} were seven congressional districts that Hillary Clinton
carried in 2016 and that are now held by Republicans. Democrats are
aiming to capture those seats in November, a linchpin of their strategy
to take back control of the House.

{[}\emph{\href{https://www.nytimes3xbfgragh.onion/2018/06/05/us/politics/primary-elections-new-jersey.html}{Here
are results from New Jersey and the other states that voted
Tuesday.}}{]}

The November race between Mr. Newsom and Mr. Cox promises to be, in
part, a fight over Mr. Trump, and one in which the liberal Democrats who
embraced Mr. Newsom have a clear advantage. The election is taking place
at a critical time as California is enmeshed in a protracted fight with
the Trump administration on range of battlefields, including
environmental protections, immigration and offshore oil drilling. And on
Tuesday night, both candidates invoked Mr. Trump in dueling remarks to
supporters.

\includegraphics{https://static01.graylady3jvrrxbe.onion/images/2018/06/06/us/06california6/06california6-articleLarge.jpg?quality=75\&auto=webp\&disable=upscale}

``It looks like voters will have a real choice this November --- between
a governor who is going to stand up to Donald Trump and a foot soldier
in his war on California,'' Mr. Newsom told hundreds of supporters at a
San Francisco nightclub, as he pledged to push for guaranteed health
care for all and ``a Marshall Plan for affordable housing.''

Mr. Cox, speaking to friends and donors in San Diego, continually
painted Mr. Newsom as ``part of the status quo'' and knocked the
Democrat's attacks on Mr. Trump.

``It wasn't Donald Trump that made California the highest-taxed state in
the country, it was Gavin Newsom and the Democrats,'' Mr. Cox said.

Running far behind in the governor's primary race was Antonio R.
Villaraigosa, a Democrat and former Los Angeles mayor.

In the race for United States Senate, Dianne Feinstein easily won a spot
on the November ballot in what by every indication looks like an easy
race this fall --- no matter who ends up running against her.

Among the seven highly competitive House races in California, Democrats
battled for months to avoid getting shut out from the November ballot
under the state's ``top two'' election system: The two leading
vote-getters, regardless of party, will go on to face each other in
November.

Image

Antonio Villaraigosa spoke to supporters at his primary night party in
Los Angeles on Tuesday.Credit...Melissa Lyttle for The New York Times

{[}\emph{\href{https://www.nytimes3xbfgragh.onion/interactive/2018/06/05/us/elections/results-california-primary-elections.html?action=click\&module=Intentional\&pgtype=Article}{Get
live results from all the races in
California}}\href{https://www.nytimes3xbfgragh.onion/interactive/2018/06/05/us/elections/results-california-primary-elections.html?action=click\&module=Intentional\&pgtype=Article}{.}{]}

California may be the single most important battleground for Democrats
in their drive to claim a majority in Congress. Mr. Trump is intensely
unpopular in the state, and broad backlash against his administration
could help Democrats seize perhaps a third of the 23 seats they need to
regain power.

Yet California's unusual open-primary system has become a difficult
obstacle for Democrats, as a horde of candidates on the left have
divided up Democratic votes and threatened to let Republicans monopolize
the general election.

The national Democratic Party has spent millions in California in recent
weeks to attack Republican candidates in television ads, aiming to drive
down their support and create more space for Democratic candidates to
rise. And party leaders in Washington backed Gil Cisneros, a Navy
veteran who won the California lottery, and Harley Rouda, a wealthy real
estate executive, for a pair of Republican-held districts in Orange
County that Mrs. Clinton carried in the presidential election.

Both Mr. Cisneros and Mr. Rouda appeared to stand a good chance of
making it into the general election, but both races were still too close
overnight for the top two finishers to be determined.

Voting took place across the state under a cloud of confusion as voters
tried to navigate their way through the top two system. And in a
potentially unnerving sign for some Democrats, the Los Angeles County
clerk revealed Tuesday night that a printing error had improperly left
about 119,000 names off voting rosters in the area --- a development
that Mr. Villaraigosa called ``infuriating'' as he urged affected voters
to cast provisional ballots.

Earlier in the day, at Laguna Beach City Hall, Aggie Dougherty had to
thumb through the sample ballot packet she carried with her to remember
which Democrat she had chosen after more than a dozen candidates
inundated the 48th Congressional District with campaign material in
their bid to unseat Dana Rohrabacher, a particularly embattled
Republican.

Image

Gavin Newsom took photos with voters in Oakland on Tuesday.Credit...Jim
Wilson/The New York Times

Ms. Dougherty, 67, a bookkeeper, settled on Harley Rouda, the candidate
endorsed by the Democratic Congressional Campaign Committee. Still, as
she went into the voting booth, she realized she had to check to
remember whom she had picked off the long list of Democrats.

``Oh, right,'' she said. ``Harley.''

Turnout appeared light during much of the day. A June primary
historically has not drawn voters to the polls in particularly high
numbers --- even one that has drawn this kind of national attention.
About 2.5 million votes had been received by mail as of Tuesday.
(California voters are permitted to vote by mail through the end of
Election Day.) Which is not to say the candidates were not trying to
pique voters' interest.

``I literally could not go through my work day without getting flooded
with calls,'' said Tim Cain, 52, a video game developer in Orange
County. ``I basically said, my phone is no longer available.''

With voting in progress, Mr. Trump prodded California conservatives to
support Mr. Cox again on Tuesday morning,
\href{https://twitter.com/realDonaldTrump/status/1003985427970314246}{promising
on Twitter} that the long-shot Republican would ``make a BIG
difference'' as governor. And the president encouraged Republicans to
turn out in the congressional elections,
\href{https://twitter.com/realDonaldTrump/status/1003987298072002565}{offering
a version} of the argument his party is expected to deliver across the
country this fall: ``Keep our country out of the hands of High Tax, High
Crime Nancy Pelosi.''

Mr. Cox repeatedly aligned himself with Mr. Trump in his remarks to
supporters Tuesday night. Even his closing message carried a slight echo
of Mr. Trump's ``Make America Great Again'' campaign slogan: ``Let's
together make California the Golden State once again.''

Robert DeRose, a close friend of Mr. Cox, said he believed the
Republican could win in November if he made it to the general election.
Mr. DeRose contended that dissatisfaction with taxes would lead many
voters to support Mr. Cox.

Image

Supporters of Mr. Cox, a Republican businessman backed by President
Trump, held signs on stage at a primary night party in San Diego on
Tuesday.Credit...Andrew Cullen for The New York Times

``This state chases businesspeople away, people like me,'' Mr. DeRose
said.

In the governor's race, Mr. Newsom, 50, had long been viewed as a
leading candidate to replace Mr. Brown, a Democrat retiring at the end
of the year. He has spent much of the past 15 years preparing for this
moment, and that was evident in the strength of his fund-raising and a
broad base of support on Tuesday.

But it was a somber night for Mr. Villaraigosa, who campaigned
energetically and had a huge burst of financial support from fellow
supporters of charter schools, including Michael Bloomberg, the former
mayor of New York, and Eli Broad, the Los Angeles philanthropist.

In an emotional concession speech that seemed aimed at unifying his
party, Mr. Villaraigosa congratulated Mr. Newsom and Mr. Cox and
encouraged his supporters to ``get behind the winner.''

``Gavin, thank you for caring enough about this state to put your hat in
the ring, to run for governor in this state,'' Mr. Villaraigosa, flanked
by his family, told a crowd of supporters.

The contest for governor marks the end of a long chapter in California
history. Mr. Brown, 80, is stepping down because of term limits. He has
served two terms now --- and two terms in the 1970s --- and leaves
office popular and generally respected. But Mr. Brown has struck a
decidedly moderate note during his years in Sacramento --- he was well
known for pushing back at what he saw as excesses by the Legislature
when it came to spending or lawmaking --- at a time when energy in the
Democratic Party was moving to the left.

For the general election, the map of important congressional races in
California extends well beyond the Southern California seats where
Democrats feared a ``top two'' fiasco. The party is also choosing
candidates to oppose vulnerable Republicans in the Central Valley and
elsewhere in the suburbs around Los Angeles, where Mr. Trump's policies
on immigration, taxes and health care have put sitting lawmakers in deep
peril.

Anneliese Gelberg, 21, wanted to vote
\href{https://www.nytimes3xbfgragh.onion/2018/05/30/us/politics/jess-phoenix-congress-climate-change.html}{for
Jess Phoenix}, one of three female Democratic candidates running for the
House seat in California's 25th Congressional District north of Los
Angeles. For one, Ms. Gelberg said, she was more inclined to vote for
women. She also liked Ms. Phoenix's policies.

But rather than casting her vote for Ms. Phoenix on Tuesday, she said
she voted for one of Ms. Phoenix's competitors --- Bryan Caforio.

``I knew that she didn't have a lot of backing or support,'' Ms. Gelberg
said, over a lunch of grilled cheese and fries. In the end, she added,
she wanted a Democrat to beat the Republican incumbent, Steve Knight,
and she thought Mr. Caforio had a better chance.

Advertisement

\protect\hyperlink{after-bottom}{Continue reading the main story}

\hypertarget{site-index}{%
\subsection{Site Index}\label{site-index}}

\hypertarget{site-information-navigation}{%
\subsection{Site Information
Navigation}\label{site-information-navigation}}

\begin{itemize}
\tightlist
\item
  \href{https://help.nytimes3xbfgragh.onion/hc/en-us/articles/115014792127-Copyright-notice}{©~2020~The
  New York Times Company}
\end{itemize}

\begin{itemize}
\tightlist
\item
  \href{https://www.nytco.com/}{NYTCo}
\item
  \href{https://help.nytimes3xbfgragh.onion/hc/en-us/articles/115015385887-Contact-Us}{Contact
  Us}
\item
  \href{https://www.nytco.com/careers/}{Work with us}
\item
  \href{https://nytmediakit.com/}{Advertise}
\item
  \href{http://www.tbrandstudio.com/}{T Brand Studio}
\item
  \href{https://www.nytimes3xbfgragh.onion/privacy/cookie-policy\#how-do-i-manage-trackers}{Your
  Ad Choices}
\item
  \href{https://www.nytimes3xbfgragh.onion/privacy}{Privacy}
\item
  \href{https://help.nytimes3xbfgragh.onion/hc/en-us/articles/115014893428-Terms-of-service}{Terms
  of Service}
\item
  \href{https://help.nytimes3xbfgragh.onion/hc/en-us/articles/115014893968-Terms-of-sale}{Terms
  of Sale}
\item
  \href{https://spiderbites.nytimes3xbfgragh.onion}{Site Map}
\item
  \href{https://help.nytimes3xbfgragh.onion/hc/en-us}{Help}
\item
  \href{https://www.nytimes3xbfgragh.onion/subscription?campaignId=37WXW}{Subscriptions}
\end{itemize}
