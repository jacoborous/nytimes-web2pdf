\href{/section/politics}{Politics}\textbar{}Corey Stewart, Virginia
Senate Nominee, Evokes Trump on Racial Issues

\url{https://nyti.ms/2HXLzq1}

\begin{itemize}
\item
\item
\item
\item
\item
\end{itemize}

\includegraphics{https://static01.graylady3jvrrxbe.onion/images/2018/06/18/us/politics/18coreystewart_p1/merlin_139525080_f78d2f87-d433-4af5-a4fd-ad48ad04563d-articleLarge.jpg?quality=75\&auto=webp\&disable=upscale}

Sections

\protect\hyperlink{site-content}{Skip to
content}\protect\hyperlink{site-index}{Skip to site index}

\hypertarget{corey-stewart-virginia-senate-nominee-evokes-trump-on-racial-issues}{%
\section{Corey Stewart, Virginia Senate Nominee, Evokes Trump on Racial
Issues}\label{corey-stewart-virginia-senate-nominee-evokes-trump-on-racial-issues}}

Mr. Stewart, a Minnesota transplant who has defended the Confederacy and
praised white nationalists, has seized opportunities to gain media
attention, especially with harsh remarks about immigrants.

Corey Stewart, the Republican candidate for Senate in Virginia, has
grown emboldened by President Trump's success and is eager to test his
style of campaigning.Credit...T.J. Kirkpatrick for The New York Times

Supported by

\protect\hyperlink{after-sponsor}{Continue reading the main story}

By \href{https://www.nytimes3xbfgragh.onion/by/trip-gabriel}{Trip
Gabriel} and
\href{https://www.nytimes3xbfgragh.onion/by/jonathan-martin}{Jonathan
Martin}

\begin{itemize}
\item
  June 17, 2018
\item
  \begin{itemize}
  \item
  \item
  \item
  \item
  \item
  \end{itemize}
\end{itemize}

HAYMARKET, Va. --- Speaking to a racially mixed audience in August 2016,
Corey Stewart, the far-right provocateur who won the Republican Senate
nomination in Virginia last week, hailed the renaming of a middle school
to honor a local black philanthropist instead of a former governor with
segregationist roots.

``As we've become safer, as we have become more educated, as we have
become more prosperous, Prince William County has also become more
diverse,'' Mr. Stewart said enthusiastically about this booming Northern
Virginia jurisdiction he leads.

But six months later, he was singing a notably different and harsher
tune.

Mr. Stewart stood beside the controversial statue of Robert E. Lee in
Charlottesville, and raised his voice to denounce protesters calling for
the monument's removal. ``They have no respect for our heritage,'' he
said, calling the Confederate general ``a great American.''

Mr. Stewart himself was a Minnesota transplant, who grew up in Duluth
and went to law school in St. Paul. But his defense of the Confederacy,
which was a central theme in his 2017 run for governor, and his
invocation of diversity, which drew applause on his home turf,
illustrate a blunt approach to racial issues that mimics President
Trump's. Both men have praised white nationalists in the past while
talking about race to suit their purposes --- Mr. Trump often takes
credit for low black unemployment --- and have especially used attacks
on immigrants to get attention and stand out among more conventional
politicians.

Interviews with more than two dozen Virginians present a portrait of Mr.
Stewart as a conservative politician-on-the-make turned Trumpian
agitator thirsty for cable television bookings. His evolution is a
revealing case study about the incentives for Republicans in the Trump
era, but also the risks they run: Longtime colleagues described Mr.
Stewart as a brazenly cynical politician who could drag down the party
as he tests whether a campaign of attacks on immigrants, schoolyard
taunts and made-for-media bombast can prove effective in a state race.

{[}\emph{\href{https://www.nytimes3xbfgragh.onion/2018/06/13/us/politics/corey-stewart-virginia.html}{Read
more about Corey Stewart, Republicans and the fringe right.}}{]}

``I have given up trying to figure out what part of Corey Stewart's
statements represent his core beliefs and what represents the things he
talks about to get publicity,'' said Martin E. Nohe, a fellow Republican
on the Prince William County Board of Supervisors, where Mr. Stewart
serves as chairman and Mr. Nohe is vice chairman.

Frank J. Principi, a Democrat on the county board, said Mr. Stewart had
told him that he saw himself in the White House one day and wanted to
use statewide election in Virginia to get there. ``He is a political
opportunist,'' Mr. Principi said. ``His rhetoric is over the top in
order to attract that media attention.''

\includegraphics{https://static01.graylady3jvrrxbe.onion/images/2018/06/17/us/politics/17coreystewart2/merlin_139642056_a8fe5a49-f5e8-427c-ac11-ac9595f5ede7-articleLarge.jpg?quality=75\&auto=webp\&disable=upscale}

Mr. Stewart, in an interview Friday, rejected these colleagues'
descriptions of him. ``It's my job to represent the concerns and address
the concerns my constituents have, and those concerns change over
time,'' he said.

But he did admit that much of his message is designed to outrage --- to
win attention in the news media and to overcome a lack of money in the
bank.

``I am deliberately edgy, I'm not going to deny that,'' he said. ``To
win a successful campaign against an opponent that has a lot more name
recognition and a lot more money, I have to be edgy.'' (Asked if he
indeed saw himself in the White House someday, he said: ``I'll be happy
with Senate. And we have a great president.'')

Mr. Stewart looks unlikely to defeat Senator Tim Kaine, who has more
than \$10 million on hand and is popular in Virginia. But
\href{https://www.nytimes3xbfgragh.onion/2018/06/13/us/corey-stewart-virginia-republicans.html}{he
could impair the Republican ticket and brand} --- he has vowed to ``kick
Tim Kaine's teeth in'' --- in his determination to pursue a
bomb-throwing strategy in a state that has rapidly shifted from reliably
red to safely blue because of the very demographic shifts he outlined in
2016 at what had been Mills E. Godwin Middle School.

\emph{{[}For more coverage of race,}
\emph{\href{http://www.nytimes3xbfgragh.onion/newsgraphics/2016/race-related/}{sign
up here}} \emph{to have our Race/Related newsletter delivered weekly to
your inbox.{]}}

Mr. Stewart, 49, has long eyed higher office. ``I'm an ambitious fellow,
I'll be frank about that,'' he said in 2011, when he was considering an
earlier Senate race. And as he pursued those ambitions, he has become a
very different kind of politician than when he first ran for local
office: an immigrant's husband who has bashed undocumented immigrants
and embraced white nationalists; a Midwesterner who attended Confederate
flag-bedecked balls and defended the controversial rebel statues in his
adopted state; and an aspiring politician who has turned on his party's
leaders, deeming them ``flaccid'' and unable to please their wives.

While Mr. Stewart's nomination for Senate was his third try for
statewide office in five years --- he ran for lieutenant governor in
2013 --- he has found more success in local politics. He has repeatedly
won election as a county supervisor on issues like promoting development
and building ball fields. But his post in Prince William is also where
he discovered the potency of racial politics.

Once a sparsely populated enclave and political backwater, Prince
William County --- 30 miles southwest of Washington --- has grown at a
torrid pace in recent decades as families in search of good schools and
a front yard pushed into the exurbs.

It is now Virginia's second-largest jurisdiction, with 463,000
residents, including many who are foreign-born. In fact, Prince William
is a majority-minority county, just shy of a quarter African-American, a
quarter Hispanic and nearly 10 percent Asian.

Mr. Stewart was first elected chairman of the county board in 2006, and
he quickly pounced on the illegal immigration issue, as the population
boom brought an influx of Hispanics. He led an effort on the board to
empower the police to check the immigration status of anyone they had
cause to believe was in the country illegally. The crackdown drew
immense local and even national media coverage.

``Corey found his lightning in a bottle,'' Mr. Nohe said. ``It got him a
level of attention he could not have imagined from outside our county.''

\includegraphics{https://static01.graylady3jvrrxbe.onion/images/2018/06/14/us/14virginia/14virginia-videoSixteenByNine3000-v3.jpg}

Mr. Stewart's zeal for the media spotlight --- beginning with
immigration and continuing with his hyperbolic comments about the
Confederacy --- prompts eye-rolling among Prince William Republicans,
who chuckle over what they called ``election year Corey.''

But not everyone is amused. Last year, Mr. Stewart used an alt-right
slur, ``cuckservative,'' to suggest that an opponent in the Republican
governor's primary had failed to put the white race first.

``When he gives his flaming speeches, to him it's all a big joke,'' said
Willie Deutsch, a Republican on the Prince William County School Board,
recalling the reference. ``To me, that makes it worse.''

After losing the nomination for lieutenant governor in 2013, Mr. Stewart
seized an opportunity to become the chairman of Mr. Trump's presidential
campaign in Virginia in 2015.

The role enabled him to be politically active across the state, and when
Mr. Trump captured the party's nomination, Mr. Stewart was well
positioned as one of the few elected officials to side with the
anti-establishment nominee early in the race.

Image

Mr. Stewart, then chairman of President Trump's Virginia campaign,
gathered with supporters of Mr. Trump in October 2016.Credit...Carolyn
Kaster/Associated Press

But Mr. Stewart clashed with other Trump officials, and his tenure came
to a humiliating end less than a month before the election when he
showed up at the Republican National Committee's headquarters to protest
what he said was the party's insufficient support for nominee.

Republican and Trump campaign officials warned him against it, multiple
Republicans recalled, but he appeared anyway --- wielding a hand-held
microphone that resembled a CB radio and rallied about a dozen activists
before nearly as many reporters and photographers.

``That was all built around building his persona,'' said Chris LaCivita,
a longtime Republican strategist who was advising the committee at the
time.

It was also the last straw for many in Mr. Trump's orbit who had grown
tired of Mr. Stewart's quest for headlines: He was immediately fired
from the campaign, and to this day a number of Republicans jostle for
credit over who made the decision to throw him overboard.

``He is an egomaniacal, narcissistic buffoon,'' said Michael Rubino, a
top Trump adviser in Virginia who is close to Corey Lewandowski, the
president's former campaign manager. ``The MAGA agenda is but a vehicle
for him to continue his years-old self-aggrandizement campaign.''

But Mr. Stewart was untroubled by his termination and Mr. Trump's loss
in Virginia, the only Southern state Hillary Clinton won, and
immediately turned to a run for governor. And much as he did with
immigration after being elected to the county board, he quickly seized
on another racially charged issue.

Mr. Stewart's embrace of the Confederacy reached an apogee in his 2017
campaign. He showed up to an Old South ball in Danville, Va., and,
surrounded by men in re-enactment regalia and women in hoop skirts,
declared the Confederate flag a symbol of ``our heritage,'' not of hate.
And he appeared with the white nationalist Jason Kessler, who went on to
organize the torch-led protests in Charlottesville that turned deadly.

His stance troubles some people back in Prince William. Outside a Food
Lion in Haymarket, Charles Dubissette, 42, recalled with distaste a
video of the ball with Mr. Stewart expressing pride in the Confederate
flag, which played on the news this week after
\href{https://www.nytimes3xbfgragh.onion/elections/results/virginia-senate-primary-election}{his
primary victory}.

``He's trying to pander to his perceived constituents, but the dynamics
of all that are changing,'' said Mr. Dubissette, who sells real estate.
``Prince William County is not what it used to be. He ought to take that
into consideration if he plans on succeeding.''

After losing the governor's race, Mr. Stewart almost immediately began
running for the Senate. In December, he showed up in Montgomery, Ala.,
at the election night party for Roy S. Moore, the Republican Senate
candidate there who was accused of making sexual advances to underage
girls. Mr. Stewart stood with reporters in the back of a thronged room
and held an impromptu news conference for those who recognized him.

Image

Supporters of Mr. Stewart at his primary night party in Woodbridge, Va.,
on Tuesday.Credit...Tom Brenner/The New York Times

In this campaign, Mr. Stewart has only about \$160,000 on hand, and
Senate Republicans indicated last week that they were keeping him at
arm's length. Mr. Trump's advisers are also wary of him, in part because
of his stunt at the R.N.C. but also because they want the president to
be seen with candidates likely to win.

But what alarms some Republicans in Virginia --- where the party has not
won statewide in nearly a decade --- is what Mr. Stewart may mean for
some of their House candidates. Already, the state party is planning to
set up federal fund-raising accounts that will let donors give to the
endangered House lawmakers and not to Mr. Stewart, according to a
Republican familiar with the planning.

Yet his inability to raise money may only encourage him to amplify his
message to capture in media attention what he cannot purchase in
advertising, some Republicans say.

``He knows how this works,'' said J. Tucker Martin, a Republican
strategist here. ``He'll say something provocative, press will have a
field day with it, rinse and repeat.''

If it all sounds familiar, Mr. Stewart says he intends to keep up the
Trumpian embrace.

``I ran on fighting to support the president, and that's exactly what
I'm going to do in the general,'' he said. ``I will be tying myself
very, very closely to the president and his success.''

Advertisement

\protect\hyperlink{after-bottom}{Continue reading the main story}

\hypertarget{site-index}{%
\subsection{Site Index}\label{site-index}}

\hypertarget{site-information-navigation}{%
\subsection{Site Information
Navigation}\label{site-information-navigation}}

\begin{itemize}
\tightlist
\item
  \href{https://help.nytimes3xbfgragh.onion/hc/en-us/articles/115014792127-Copyright-notice}{©~2020~The
  New York Times Company}
\end{itemize}

\begin{itemize}
\tightlist
\item
  \href{https://www.nytco.com/}{NYTCo}
\item
  \href{https://help.nytimes3xbfgragh.onion/hc/en-us/articles/115015385887-Contact-Us}{Contact
  Us}
\item
  \href{https://www.nytco.com/careers/}{Work with us}
\item
  \href{https://nytmediakit.com/}{Advertise}
\item
  \href{http://www.tbrandstudio.com/}{T Brand Studio}
\item
  \href{https://www.nytimes3xbfgragh.onion/privacy/cookie-policy\#how-do-i-manage-trackers}{Your
  Ad Choices}
\item
  \href{https://www.nytimes3xbfgragh.onion/privacy}{Privacy}
\item
  \href{https://help.nytimes3xbfgragh.onion/hc/en-us/articles/115014893428-Terms-of-service}{Terms
  of Service}
\item
  \href{https://help.nytimes3xbfgragh.onion/hc/en-us/articles/115014893968-Terms-of-sale}{Terms
  of Sale}
\item
  \href{https://spiderbites.nytimes3xbfgragh.onion}{Site Map}
\item
  \href{https://help.nytimes3xbfgragh.onion/hc/en-us}{Help}
\item
  \href{https://www.nytimes3xbfgragh.onion/subscription?campaignId=37WXW}{Subscriptions}
\end{itemize}
