Sections

SEARCH

\protect\hyperlink{site-content}{Skip to
content}\protect\hyperlink{site-index}{Skip to site index}

\href{https://www.nytimes3xbfgragh.onion/section/climate}{Climate}

\href{https://myaccount.nytimes3xbfgragh.onion/auth/login?response_type=cookie\&client_id=vi}{}

\href{https://www.nytimes3xbfgragh.onion/section/todayspaper}{Today's
Paper}

\href{/section/climate}{Climate}\textbar{}Uranium Miners Pushed Hard for
a Comeback. They Got Their Wish.

\url{https://nyti.ms/2FBmK3y}

\begin{itemize}
\item
\item
\item
\item
\item
\item
\end{itemize}

\hypertarget{climate-and-environment}{%
\subsubsection{\texorpdfstring{\href{https://www.nytimes3xbfgragh.onion/section/climate?name=styln-climate\&region=TOP_BANNER\&variant=undefined\&block=storyline_menu_recirc\&action=click\&pgtype=Article\&impression_id=3d0de680-e388-11ea-b57e-737919ae4111}{Climate
and
Environment}}{Climate and Environment}}\label{climate-and-environment}}

\begin{itemize}
\tightlist
\item
  \href{https://www.nytimes3xbfgragh.onion/2020/08/17/climate/alaska-oil-drilling-anwr.html?name=styln-climate\&region=TOP_BANNER\&variant=undefined\&block=storyline_menu_recirc\&action=click\&pgtype=Article\&impression_id=3d0e0d90-e388-11ea-b57e-737919ae4111}{Arctic
  Refuge}
\item
  \href{https://www.nytimes3xbfgragh.onion/interactive/2020/climate/trump-environment-rollbacks.html?name=styln-climate\&region=TOP_BANNER\&variant=undefined\&block=storyline_menu_recirc\&action=click\&pgtype=Article\&impression_id=3d0e0d91-e388-11ea-b57e-737919ae4111}{Trump's
  Changes}
\item
  \href{https://www.nytimes3xbfgragh.onion/interactive/2020/04/19/climate/climate-crash-course-1.html?name=styln-climate\&region=TOP_BANNER\&variant=undefined\&block=storyline_menu_recirc\&action=click\&pgtype=Article\&impression_id=3d0e0d92-e388-11ea-b57e-737919ae4111}{Climate
  101}
\item
  \href{https://www.nytimes3xbfgragh.onion/interactive/2018/08/30/climate/how-much-hotter-is-your-hometown.html?name=styln-climate\&region=TOP_BANNER\&variant=undefined\&block=storyline_menu_recirc\&action=click\&pgtype=Article\&impression_id=3d0e0d93-e388-11ea-b57e-737919ae4111}{Is
  Your Hometown Hotter?}
\end{itemize}

Advertisement

\protect\hyperlink{after-top}{Continue reading the main story}

Supported by

\protect\hyperlink{after-sponsor}{Continue reading the main story}

\hypertarget{uranium-miners-pushed-hard-for-a-comeback-they-got-their-wish}{%
\section{Uranium Miners Pushed Hard for a Comeback. They Got Their
Wish.}\label{uranium-miners-pushed-hard-for-a-comeback-they-got-their-wish}}

\includegraphics{https://static01.graylady3jvrrxbe.onion/images/2018/01/14/science/14CLI-URANIUM1/00CLI-URANIUM1-articleLarge.jpg?quality=75\&auto=webp\&disable=upscale}

By \href{http://www.nytimes3xbfgragh.onion/by/hiroko-tabuchi}{Hiroko
Tabuchi}

\begin{itemize}
\item
  Jan. 13, 2018
\item
  \begin{itemize}
  \item
  \item
  \item
  \item
  \item
  \item
  \end{itemize}
\end{itemize}

\emph{Want the latest climate news in your inbox? You can}
\href{https://www.nytimes3xbfgragh.onion/newsletters/climate-change}{\emph{sign
up here}} \emph{to receive} \emph{\textbf{Climate Fwd:}}\emph{, our new
email newsletter.}

MONUMENT VALLEY, Utah --- Garry Holiday grew up among the abandoned
mines that dot the Navajo Nation's red landscape, remnants of a time
when uranium helped cement America's status as a nuclear superpower and
fueled its nuclear energy program.

It left a toxic legacy. All but a few of the 500 abandoned mines still
await cleanup. Mining tainted the local groundwater. Mr. Holiday's
father succumbed to respiratory disease after years of hacking the ore
from the earth.

But now, emboldened by the Trump administration's embrace of corporate
interests, the uranium mining industry is renewing a push into the areas
adjacent to Mr. Holiday's Navajo Nation home: the Grand Canyon watershed
to the west, where a new uranium mine is preparing to open, and the
Bears Ears National Monument to the north.

\href{https://www.nytimes3xbfgragh.onion/2017/12/04/us/trump-bears-ears.html}{The
Trump administration is set to shrink Bears Ears by 85 percent} next
month, potentially opening more than a million acres to mining, drilling
and other industrial activity. But even as Interior Secretary Ryan Zinke
declared last month that ``there is no mine within Bears Ears,'' there
were more than 300 uranium mining claims inside the monument, according
to data from Utah's Bureau of Land Management office that was reviewed
by The New York Times.

The vast majority of those claims fall neatly outside the new boundaries
of Bears Ears set by the administration. And an examination of local
B.L.M. records, including those not yet entered into the agency's
\href{https://reports.blm.gov/content/lr2000/about/}{land and mineral
use authorizations database}, shows that about a third of the claims are
linked to Energy Fuels, a Canadian uranium producer. Energy Fuels also
owns the Grand Canyon mine, where groundwater has already flooded the
main shaft.

Energy Fuels, together with other mining groups, lobbied extensively for
a reduction of Bears Ears, preparing maps that marked the areas it
wanted removed from the monument and distributing them during a visit to
the monument by Mr. Zinke in May.

\includegraphics{https://static01.graylady3jvrrxbe.onion/images/2018/01/14/science/14CLI-URANIUM2/merlin_131484305_dd191576-d8e0-43df-9831-b9419a63dc96-articleLarge.jpg?quality=75\&auto=webp\&disable=upscale}

Energy Fuels' lobbying campaign, elements of which were
\href{https://www.washingtonpost.com/national/health-science/uranium-firm-urged-trump-officials-to-shrink-bears-ears-national-monument/2017/12/08/2eea39b6-dc31-11e7-b1a8-62589434a581_story.html}{first
reported by The Washington Post}, is part of a wider effort by the
long-ailing uranium industry to make a comeback.

The Uranium Producers of America, an industry group, is pushing the
Environmental Protection Agency to withdraw
\href{https://www.epa.gov/radiation/40-cfr-part-192-proposed-rulemaking-and-background-documents}{regulations
proposed by the Obama administration} to strengthen groundwater
protections at uranium mines. Mining groups have also waged a six-year
legal battle against a moratorium on new uranium mining on more than a
million acres of land adjacent to the Grand Canyon.

For the Navajo, the drive for new mines is a painful flashback.

``Back then, we didn't know it was dangerous --- nobody told us,'' Mr.
Holiday said, as he pointed to the gashes of discolored rocks that mark
where the old uranium mines cut into the region's mesas. ``Now they
know. They know.''

Supporters of the mining say that a revival of domestic uranium
production, which has
\href{https://www.eia.gov/todayinenergy/detail.php?id=26472}{declined by
90 percent since 1980} amid slumping prices and foreign competition,
will make the United States a larger player in the global uranium
market.

It would expand the country's energy independence, they say, and give a
lift to nuclear power, still a pillar of carbon-free power generation.
Canada, Kazakhstan, Australia, Russia and a few other countries now
\href{https://www.eia.gov/energyexplained/index.cfm?page=nuclear_where}{supply
most of America's nuclear fuel}.

The dwindling domestic market was thrust into the spotlight by the
contentious
\href{https://www.nytimes3xbfgragh.onion/2017/11/14/us/politics/uranium-one-hillary-clinton.html}{2010
decision under the Obama administration} that allowed Russia's nuclear
agency to buy Uranium One, a company that has amassed production
facilities in the United States.
\href{https://www.nytimes3xbfgragh.onion/2017/11/13/us/politics/justice-department-uranium-one-special-counsel.html}{The
Justice Department is examining allegations} that donations to the
Clinton Foundation were tied to that decision.

\href{https://www.nytimes3xbfgragh.onion/section/climate?action=click\&pgtype=Article\&state=default\&region=MAIN_CONTENT_1\&context=storylines_keepup}{}

\hypertarget{climate-and-environment-}{%
\subsubsection{Climate and Environment
›}\label{climate-and-environment-}}

\hypertarget{keep-up-on-the-latest-climate-news}{%
\paragraph{Keep Up on the Latest Climate
News}\label{keep-up-on-the-latest-climate-news}}

Updated Aug. 18, 2020

Here's what you need to know this week:

\begin{itemize}
\item
  \begin{itemize}
  \tightlist
  \item
    Five automakers
    \href{https://www.nytimes3xbfgragh.onion/2020/08/17/climate/california-automakers-pollution.html?action=click\&pgtype=Article\&state=default\&region=MAIN_CONTENT_1\&context=storylines_keepup}{sealed
    a binding agreement} with California to follow the state's stricter
    tailpipe emissions rules.
  \item
    The Trump
    administration\href{https://www.nytimes3xbfgragh.onion/2020/08/13/climate/trump-methane.html?action=click\&pgtype=Article\&state=default\&region=MAIN_CONTENT_1\&context=storylines_keepup}{eliminated
    a major methane rule}, even as leaks are worsening, in a decision
    that researchers warned ignored science.
  \item
    Climate change leaders said
    \href{https://www.nytimes3xbfgragh.onion/2020/08/12/climate/kamala-harris-environmental-justice.html?action=click\&pgtype=Article\&state=default\&region=MAIN_CONTENT_1\&context=storylines_keepup}{the
    vice-presidential choice of Kamala Harris} signaled that Democrats
    will have a focus on environmental justice.
  \end{itemize}
\end{itemize}

``If we consider nuclear a clean energy, if people are serious about
that, domestic uranium has to be in the equation,'' said Jon J. Indall,
a lawyer for Uranium Producers of America. ``But the proposed
regulations would have had a devastating impact on our industry.''

``Countries like Kazakhstan, they're not under the same environmental
standards. We want a level playing field.''

\hypertarget{scaling-back-a-monument}{%
\subsection{Scaling back a monument}\label{scaling-back-a-monument}}

Image

Interior Secretary Ryan Zinke visiting the Bears Ears National Monument
in Utah last year. Mr. Zinke has said mining was not a consideration in
the decision to shrink the monument.Credit...Scott G Winterton/The
Deseret News, via Associated Press

The trip was one of the earliest made by Mr. Zinke to the vast lands he
oversees as secretary of the interior: a visit to Bears Ears, where he
struck a commanding figure, touring the rugged terrain on horseback.

A notable presence on Mr. Zinke's trip was Energy Fuels, the Canadian
uranium producer. Company executives openly lobbied for shrinking Bears
Ears' borders, handing out the map that marked the pockets the company
wanted removed: areas adjacent to its White Mesa Mill, just to the east
of the monument, and its Daneros Mine, which it is developing just to
the west.

``They wanted to talk to anyone who'd listen,'' said Commissioner Phil
Lyman of San Juan County, Utah, a Republican who participated in the
tour and is sympathetic to Energy Fuels' position. ``They were there
representing their business interest.''

Mr. Zinke has insisted that mining played no role in the decision to
shrink Bears Ears, and a department spokeswoman said he had met with
interested parties on all sides.

But President Trump has prioritized scrapping environmental regulations
to help revitalize domestic energy production. His
\href{https://www.federalregister.gov/documents/2017/05/01/2017-08908/review-of-designations-under-the-antiquities-act}{executive
order instructing Mr. Zinke to review Bears Ears} said that improper
monument designations could ``create barriers to achieving energy
independence.''

In theory, even after President Barack Obama established Bears Ears in
2016, mining companies could have developed any of the claims within it,
given proper local approvals. But companies say that expanding the
sites, or even building roads to access them, would have required
special permits, driving up costs.

Energy Fuels said it had sold its Bears Ears claims to a smaller
company, Encore Energy, in 2016. But Encore issued shares to Energy
Fuels in return, making Energy Fuels
\href{http://encoreenergycorp.com/share-structure/}{Encore's largest
shareholder}, with
\href{http://encoreenergycorp.com/corporate/board-of-directors/}{a seat
on its board}.

Curtis Moore, an Energy Fuels spokesman, said the company had played
only a small part in the decision to shrink Bears Ears. The company
proposed scaling back the monument by just 2.5 percent, he said, and was
prepared to support a ban within the rest of the original boundaries.

Yet two weeks after Mr. Zinke's visit,
\href{https://www.regulations.gov/document?D=DOI-2017-0002-99711}{Energy
Fuels wrote to the Interior Department} arguing there were many other
known uranium deposits within Bears Ears ``that could provide valuable
energy and mineral resources in the future'' and urging the department
to shrink the monument away from any ``existing or future operations.''

\href{https://www.congress.gov/bill/115th-congress/house-bill/4532/text}{A
bill introduced last month by Representative John Curtis}, Republican of
Utah, would codify Mr. Trump's cuts to the monument while banning
further drilling or mining within the original boundaries. But
environmental groups say the bill has little chance of passing at all,
let alone before the monument is scaled back next month.

``Come February, anyone can place a mining claim on the land,'' said
Greg Zimmerman, deputy director at the Center for Western Priorities, a
conservation group.

\hypertarget{new-mine-new-challenges}{%
\subsection{New mine, new challenges}\label{new-mine-new-challenges}}

Image

Workers at Energy Fuels' new Canyon Mine, a few miles south of the Grand
Canyon, had to pump contaminated groundwater into open ponds, where they
used industrial sprayers to speed evaporation.Credit...Caitlin O'Hara
for The New York Times

Image

Left, the headframe atop the shaft at Canyon Mine. Right, a woodpecker
near the shuttered Orphan Mine, a partially reclaimed Superfund site
near the Grand Canyon's South Rim.Credit...Caitlin O'Hara for The New
York Times

At the end of a dirt road just six miles from the Grand Canyon's South
Rim, the uranium industry's renewed ambitions, and challenges, are on
display.

Three decades after exploratory drilling uncovered uranium deposits,
production at Energy Fuels' Canyon Mine is finally starting up, the
wheel above a 1,500-foot shaft slowly turning during a recent visit. The
company calls Canyon Mine a ``high-grade'' project, with the potential
to compete with mines overseas.

It is already running into trouble.

As workers drilled into the formations that make up the region's
distinct rock layers last year, they hit shallow groundwater. The water
flooded the mine's shaft, forcing workers to pump the runoff --- by then
contaminated with uranium --- into open ponds, where they used
industrial sprayers to speed evaporation. Those sprayers were present
during a recent visit, and water could be seen from outside the compound
continuing to pour into a large evaporation pond.

Energy Fuels officials said hitting shallow groundwater was to be
expected, and rejected concerns that contamination could escape.

Still, Fred Tillman, an environmental engineer with the United States
Geological Survey, said during a recent visit to the mine that the
groundwater flows in the region were too complex to rule out the risk of
contamination.

``There are these big unknowns about the potential impacts on cultural
resources, on biological resources, on water resources,'' Dr. Tillman
said.

\hypertarget{a-senator-steps-in}{%
\subsection{A senator steps in}\label{a-senator-steps-in}}

Image

Traces from uranium mining are still visible on the side of a mesa in
Monument Valley, where Mr. Holiday lives.Credit...Caitlin O'Hara for The
New York Times

Even as troubles persist on the ground, the industry pushback has
continued.

In court, mining groups led by the National Mining Association have
challenged a 20-year moratorium on mining in the Grand Canyon watershed,
established in 2012 by the Obama administration. (The Canyon Mine
predates the moratorium.)

A federal court of appeals
\href{http://www.biologicaldiversity.org/programs/public_lands/mining/Grand_Canyon_Uranium_Mining/pdfs/9thCirOpinionAffirmingGrandCanyonWithdrawal-12-12-2017.pdf}{upheld
the moratorium last month}. But the United States Forest Service
\href{https://www.fs.fed.us/sites/default/files/eo-13783-usda-final-report-10.11.17.pdf}{has
recommended rolling back the protections}, meaning the Trump
administration could soon reverse them on its own.

The Arizona Chamber of Commerce, which represents mining interests, also
backed an effort to defeat a separate proposal that would have
permanently banned mining on 1.7 million acres surrounding the Grand
Canyon. An Energy Fuels executive
\href{https://gosar.house.gov/sites/gosar.house.gov/files/harold\%20r\%20roberts\%20testimony\%20april\%2011\%202016\%20(2).pdf}{testified
in Congress against the ban}.

And with the help of Republican senators like John Barrasso of Wyoming,
the industry has pressed the E.P.A. to withdraw
\href{https://www.epa.gov/radiation/40-cfr-part-192-proposed-rulemaking-and-background-documents}{an
Obama-era proposal} that would strengthen groundwater protections at
uranium mines.

Image

Senator John BarrassoCredit...Pete Marovich for The New York Times

Senator Barrasso has received
\href{https://www.opensecrets.org/members-of-congress/industries?cid=N00006236\&cycle=CAREER\&type=I}{more
than \$350,000 in campaign contributions from mining groups} over his
career. His office did not respond to requests for comment.

The proposal would regulate a mining method called in-situ recovery,
which involves injecting a solution into aquifers containing uranium and
bringing that solution to the surface for processing --- a method
\href{https://www.nrdc.org/sites/default/files/uranium-mining-report.pdf}{criticized
by environmentalists} as posing wider contamination risks.

The rule is unnecessary and will cost jobs, Mr. Indall, the lawyer for
Uranium Producers of America, said in
\href{http://theupa.org/_resources/news/UPA-Comments-EPA-Regulatory-Review-May-15-2017.pdf}{a
letter to Scott Pruitt}, the E.P.A. administrator. In July, U.P.A.
representatives
\href{http://www.theupa.org/_resources/news/UPA-July-25-2017-Press-Release.pdf}{met
with Mr. Pruitt and Energy Secretary Rick Perry} to press their case.

The E.P.A. said in a statement that it was ``reviewing options for next
steps.''

Last month,
\href{https://www.epw.senate.gov/public/_cache/files/7/b/7b0204a4-681d-4f61-ace0-c76a791467ba/70422F424D1DF2650D3BFD02D7571E04.12.14.17-barrasso-pruitt-letter-isr.pdf}{Mr.
Barrasso again called on the agency to withdraw the rule}, calling it
``unreasonably burdensome.''

\hypertarget{a-town-still-struggles}{%
\subsection{A town still struggles}\label{a-town-still-struggles}}

Image

Tommy Rock discovered that the people of Sanders, Ariz., had been
exposed to potentially dangerous levels of uranium in their drinking
water for years. The Sanders school district had to shut off
fountains.Credit...Caitlin O'Hara for The New York Times

The Navajo town of Sanders, Ariz., a dusty outpost with a single
stoplight, is a reminder of uranium's lasting environmental legacy.

In Sanders, hundreds of people were exposed to potentially dangerous
levels of uranium in their drinking water for years, until testing by a
doctoral researcher at Northern Arizona University named Tommy Rock
exposed the contamination.

``I was shocked,'' Mr. Rock said. ``I wasn't expecting that reading at
all.''

Mr. Rock and other scientists say they suspect a link to the 1979 breach
of a wastewater pond at
\href{https://cumulis.epa.gov/supercpad/cursites/csitinfo.cfm?id=0600819}{a
uranium mill in Church Rock, N.M., now a Superfund site}. That accident
is considered the single largest release of radioactive material in
American history, surpassing the crisis at Three Mile Island.

It wasn't until 2003, however, that testing by state regulators picked
up uranium levels in Sanders's tap water. Still, the community was not
told. Erin Jordan, a spokeswoman for the Arizona Department of
Environmental Quality, said the department had urged the now-defunct
local water company for years to address the contamination, but it had
been up to that company to notify its customers.

Only in 2015, after Mr. Rock raised the alarm, did local regulators
issue a public notice.

The town's school district, whose wells were also contaminated with
uranium, received little state or federal assistance. It shut off its
water fountains and handed out bottled water to its 800 elementary and
middle-school students.

The schools finally installed filters last May. Parents remain on edge.

``I still don't trust the water,'' said Shanon Sangster, who still sends
her 10-year-old daughter, Shania, to school with bottled water. ``It's
like we are all scarred by it, by the uranium.''

Image

A family filled water containers in the Navajo Nation. The nearest
untainted well is 25 miles from their home.Credit...Caitlin O'Hara for
The New York Times

Advertisement

\protect\hyperlink{after-bottom}{Continue reading the main story}

\hypertarget{site-index}{%
\subsection{Site Index}\label{site-index}}

\hypertarget{site-information-navigation}{%
\subsection{Site Information
Navigation}\label{site-information-navigation}}

\begin{itemize}
\tightlist
\item
  \href{https://help.nytimes3xbfgragh.onion/hc/en-us/articles/115014792127-Copyright-notice}{©~2020~The
  New York Times Company}
\end{itemize}

\begin{itemize}
\tightlist
\item
  \href{https://www.nytco.com/}{NYTCo}
\item
  \href{https://help.nytimes3xbfgragh.onion/hc/en-us/articles/115015385887-Contact-Us}{Contact
  Us}
\item
  \href{https://www.nytco.com/careers/}{Work with us}
\item
  \href{https://nytmediakit.com/}{Advertise}
\item
  \href{http://www.tbrandstudio.com/}{T Brand Studio}
\item
  \href{https://www.nytimes3xbfgragh.onion/privacy/cookie-policy\#how-do-i-manage-trackers}{Your
  Ad Choices}
\item
  \href{https://www.nytimes3xbfgragh.onion/privacy}{Privacy}
\item
  \href{https://help.nytimes3xbfgragh.onion/hc/en-us/articles/115014893428-Terms-of-service}{Terms
  of Service}
\item
  \href{https://help.nytimes3xbfgragh.onion/hc/en-us/articles/115014893968-Terms-of-sale}{Terms
  of Sale}
\item
  \href{https://spiderbites.nytimes3xbfgragh.onion}{Site Map}
\item
  \href{https://help.nytimes3xbfgragh.onion/hc/en-us}{Help}
\item
  \href{https://www.nytimes3xbfgragh.onion/subscription?campaignId=37WXW}{Subscriptions}
\end{itemize}
