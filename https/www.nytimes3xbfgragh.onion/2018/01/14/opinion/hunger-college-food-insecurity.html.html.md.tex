Sections

SEARCH

\protect\hyperlink{site-content}{Skip to
content}\protect\hyperlink{site-index}{Skip to site index}

\href{https://myaccount.nytimes3xbfgragh.onion/auth/login?response_type=cookie\&client_id=vi}{}

\href{https://www.nytimes3xbfgragh.onion/section/todayspaper}{Today's
Paper}

\href{/section/opinion}{Opinion}\textbar{}It's Hard to Study if You're
Hungry

\url{https://nyti.ms/2EHNHB3}

\begin{itemize}
\item
\item
\item
\item
\item
\end{itemize}

Advertisement

\protect\hyperlink{after-top}{Continue reading the main story}

Supported by

\protect\hyperlink{after-sponsor}{Continue reading the main story}

\href{/section/opinion}{Opinion}

\href{/column/on-campus}{On Campus}

\hypertarget{its-hard-to-study-if-youre-hungry}{%
\section{It's Hard to Study if You're
Hungry}\label{its-hard-to-study-if-youre-hungry}}

By Sara Goldrick-Rab

\begin{itemize}
\item
  Jan. 14, 2018
\item
  \begin{itemize}
  \item
  \item
  \item
  \item
  \item
  \end{itemize}
\end{itemize}

\includegraphics{https://static01.graylady3jvrrxbe.onion/images/2018/01/15/opinion/15oncampus-goldrick/15oncampus-goldrick-articleLarge.jpg?quality=75\&auto=webp\&disable=upscale}

Last fall, students at two of the nation's premier historically black
colleges, Spelman and Morehouse,
\href{https://www.bustle.com/p/spelman-morehouse-students-are-going-on-hunger-strike-to-protest-food-insecurity-on-college-campuses-3242622}{went
on a hunger strike}. They weren't protesting policymakers in Washington.
They were pressuring their schools to allow students to donate unused
meal plan vouchers to those on campus who needed them.

These students recognized a real problem, one that plagues all sorts of
colleges and universities, especially the community colleges and state
schools that most Americans attend.

An estimated half of all college
students\href{http://journals.sagepub.com/doi/abs/10.3102/0013189X17741303?ai=1gvoi\&mi=3ricys\&af=R}{struggle
with food insecurity}, even at elite flagship universities like the
\href{https://alumni.berkeley.edu/california-magazine/just-in/2016-05-11/hunger-uc-berkeley-sizeable-share-students-are-financially}{University
of California}, Berkeley, and selective private schools like
\href{http://www.swipehunger.org/northwestern}{Northwestern University}.
Former foster youth, L.G.B.T. students and students of color are at
\href{http://www.ucop.edu/global-food-initiative/_files/food-housing-security.pdf}{substantially
increased risk}. Food insecurity is
\href{http://www.tandfonline.com/doi/full/10.1080/00091383.2016.1121081?scroll=top\&needAccess=true\&}{strongly
linked}to lower graduation rates.

The new economics of college led us into this mess. The cost of higher
education is at
\href{http://time.com/money/4543839/college-costs-record-2016/}{an
all-time high}, which is in sharp contrast to the declining income and
wealth of most American families. And while a college degree is no
guarantee of employment, it still greatly
\href{https://cew.georgetown.edu/cew-reports/americas-divided-recovery/}{increases
the odds of a middle-class life}. It makes sense that students work hard
to go to college to achieve stability, and it is tragic that many fail
to complete degrees because they cannot escape poverty long enough to
focus on their studies.

As a researcher who studies how college students live, I hear frequently
from people who say that struggling a bit to get through college is fine
--- in fact, it's better than fine because it teaches you to work hard
for what you want. After all, they had side jobs in college; they ate
Ramen noodles. That's just how it goes.

But what is happening today is very different. For decades, many
students survived on little to afford college. But over time, the
situation worsened to the point where now, hunger and homelessness
routinely undermine students' very ability to learn. Even though a far
greater percentage of
\href{https://nces.ed.gov/fastfacts/display.asp?id=31}{college students
qualify for financial aid} than in the past, colleges and states have
\href{https://www.cbpp.org/research/state-budget-and-tax/a-lost-decade-in-higher-education-funding}{fewer
dollars per student} to allocate to them.

Students can't trust in a government safety net, either. It used to be
the case that relatively few low-income women with children attended
college, but those who did could receive welfare while in school. Today,
\href{https://iwpr.org/publications/4-8-million-college-students-are-raising-children/}{one
in four college students have a child,} and yet most of these parents
can't get aid (or affordable child care) because of federal work
requirements that require them to work 20 to 30 hours a week to get cash
assistance.

Food stamps have onerous requirements, too. Students without children
who qualify for food stamps
\href{https://www.fns.usda.gov/snap/facts-about-snap}{often cannot
receive them} without working 20 hours a week on top of going to school.
While that might sound easy, it isn't --- students are competing in a
difficult job market for part-time, low-wage jobs. They are at a
disadvantage because they lack flexibility and, often, experience. And
through all of this, the
\href{https://www.washingtonpost.com/news/wonk/wp/2017/12/29/the-u-s-has-one-of-the-stingiest-minimum-wage-policies-of-any-wealthy-nation/}{value
of the real minimum wage} continues to decline. No wonder so many
children are growing up in poverty.

In New York, where a forthcoming study by researchers at the City
University of New York reports that 30 percent of community college
students and 22 percent of four-year college students are food insecure,
Gov. Andrew M. Cuomo recently proposed that every public college open a
free campus food pantry. This is a modest but welcome step, since even
mere acknowledgment of the problem among top policymakers is rare.

The
\href{https://partnersforourchildren.org/blog/poc-co-sponsors-federal-briefing-about-food-insecurity-college-campuses}{first
federal briefing} on college food insecurity took place just last month,
and Gov. Jerry Brown of California is the only state leader to put
substantial funding toward the problem, with a \$7.5 million investment.

But while quick fixes are useful for the students who need food now,
they are not long-term, preventive solutions. Charitable donations of
cans of food and cartons of milk must be supplemented with changes to
how food is distributed and priced on campus, and access to the SNAP
food stamp program should be broadened for students.

Colleges themselves have a responsibility to do better, and they can
take a cue from their students. The hunger strikers at Spelman and
Morehouse were part of a growing coalition of youth who have joined
\href{http://www.swipehunger.org/}{Swipe Out Hunger}, a nonprofit that
advocates donating unused meal credits. Spelman and Morehouse both
agreed to allocate
\href{https://thegrio.com/2017/11/11/morehouse-spelman-students-hunger-strike-ends-free-meals-campus/}{14,000
free meals per year} to students in need. Bunker Hill Community College
in Boston is one of several schools that now distribute meal vouchers,
and Houston Community College is
\href{http://wihopelab.com/publications/Addressing-Basic-Needs-Security-in-Higher-Education.pdf}{providing
grocery scholarships}.

Fundamentally, financial aid must be reformed to address the
\href{https://tcf.org/content/report/the-real-price-of-college/}{real
price} of college, which cannot be calculated without factoring in food
and shelter. Living expenses are educational expenses.

After all, it's impossible to learn when you're starving.

Advertisement

\protect\hyperlink{after-bottom}{Continue reading the main story}

\hypertarget{site-index}{%
\subsection{Site Index}\label{site-index}}

\hypertarget{site-information-navigation}{%
\subsection{Site Information
Navigation}\label{site-information-navigation}}

\begin{itemize}
\tightlist
\item
  \href{https://help.nytimes3xbfgragh.onion/hc/en-us/articles/115014792127-Copyright-notice}{©~2020~The
  New York Times Company}
\end{itemize}

\begin{itemize}
\tightlist
\item
  \href{https://www.nytco.com/}{NYTCo}
\item
  \href{https://help.nytimes3xbfgragh.onion/hc/en-us/articles/115015385887-Contact-Us}{Contact
  Us}
\item
  \href{https://www.nytco.com/careers/}{Work with us}
\item
  \href{https://nytmediakit.com/}{Advertise}
\item
  \href{http://www.tbrandstudio.com/}{T Brand Studio}
\item
  \href{https://www.nytimes3xbfgragh.onion/privacy/cookie-policy\#how-do-i-manage-trackers}{Your
  Ad Choices}
\item
  \href{https://www.nytimes3xbfgragh.onion/privacy}{Privacy}
\item
  \href{https://help.nytimes3xbfgragh.onion/hc/en-us/articles/115014893428-Terms-of-service}{Terms
  of Service}
\item
  \href{https://help.nytimes3xbfgragh.onion/hc/en-us/articles/115014893968-Terms-of-sale}{Terms
  of Sale}
\item
  \href{https://spiderbites.nytimes3xbfgragh.onion}{Site Map}
\item
  \href{https://help.nytimes3xbfgragh.onion/hc/en-us}{Help}
\item
  \href{https://www.nytimes3xbfgragh.onion/subscription?campaignId=37WXW}{Subscriptions}
\end{itemize}
