Sections

SEARCH

\protect\hyperlink{site-content}{Skip to
content}\protect\hyperlink{site-index}{Skip to site index}

\href{https://www.nytimes3xbfgragh.onion/section/world/asia}{Asia
Pacific}

\href{https://myaccount.nytimes3xbfgragh.onion/auth/login?response_type=cookie\&client_id=vi}{}

\href{https://www.nytimes3xbfgragh.onion/section/todayspaper}{Today's
Paper}

\href{/section/world/asia}{Asia Pacific}\textbar{}China's Leader Says
Party Must Control `All Tasks,' and Asian Markets Slump

\url{https://nyti.ms/2QYoWKZ}

\begin{itemize}
\item
\item
\item
\item
\item
\item
\end{itemize}

Advertisement

\protect\hyperlink{after-top}{Continue reading the main story}

Supported by

\protect\hyperlink{after-sponsor}{Continue reading the main story}

\hypertarget{chinas-leader-says-party-must-control-all-tasks-and-asian-markets-slump}{%
\section{China's Leader Says Party Must Control `All Tasks,' and Asian
Markets
Slump}\label{chinas-leader-says-party-must-control-all-tasks-and-asian-markets-slump}}

\includegraphics{https://static01.graylady3jvrrxbe.onion/images/2018/12/19/world/19china-xi-print/merlin_148270290_38177170-aee6-4f49-a922-1da68e9260bc-articleLarge.jpg?quality=75\&auto=webp\&disable=upscale}

By \href{https://www.nytimes3xbfgragh.onion/by/chris-buckley}{Chris
Buckley} and
\href{https://www.nytimes3xbfgragh.onion/by/steven-lee-myers}{Steven Lee
Myers}

\begin{itemize}
\item
  Dec. 18, 2018
\item
  \begin{itemize}
  \item
  \item
  \item
  \item
  \item
  \item
  \end{itemize}
\end{itemize}

\href{https://cn.nytimes3xbfgragh.onion/china/20181219/xi-jinping-speech-china/}{阅读简体中文版}\href{https://cn.nytimes3xbfgragh.onion/china/20181219/xi-jinping-speech-china/zh-hant/}{閱讀繁體中文版}

BEIJING --- Facing deepening tensions abroad and anxieties at home,
China's leader, Xi Jinping, delivered an unabashed defense of his
policies on Tuesday, using a key anniversary to argue that his recipe of
guided growth under strong Communist Party control must not waver.

Mr. Xi made his case to some 3,000 officials and guests gathered in the
imposing Great Hall of the People in Beijing to commemorate 40 years
since China embarked on far-reaching economic changes after decades of
upheaval and malaise under Mao Zedong.

The resonant date had inspired expectations among some analysts and
investors that Mr. Xi would give clearer priorities to counter
\href{https://www.nytimes3xbfgragh.onion/2018/12/14/business/china-economy-xi-jinping.html?action=click\&module=RelatedCoverage\&pgtype=Article\&region=Footer}{economic
headwinds} and trade tensions that have flared with the United States.
But he offered none, referring only obliquely to the economic and
diplomatic challenges confronting China.

Instead, he used the meeting, broadcast live on Chinese television, to
stress that only the party's dominance would allow China to continue
\href{https://www.nytimes3xbfgragh.onion/interactive/2018/11/25/world/asia/china-economy-strategy.html?action=click\&module=RelatedCoverage\&pgtype=Article\&region=Footer}{its
stunning transformation} into the decades ahead. The first lesson from
40 years of reform, he said, was the need to maintain party leadership
``over all tasks.''

``It was precisely because we've adhered to the centralized and united
leadership of the party that we were able to achieve this great historic
transition,'' Mr. Xi said.

Mr. Xi's speech, lasting nearly one and half hours, came at a pivotal,
potentially fraught moment in the country, when all the contradictions
in its governance appeared in stark relief. Mr. Xi's political power is
as great as that of any leader in decades, yet his party's tightening of
controls over the economy and ever more aspects of society suggest a
deep-seated insecurity at the highest levels.

Mr. Xi's government has been forced to make some compromises with the
United States as President Trump's trade demands have escalated. But
Beijing has also intensified corporate espionage and reacted with
unbridled fury when American prosecutors sought to extradite
\href{https://www.nytimes3xbfgragh.onion/2018/12/14/business/huawei-meng-hsbc-canada.html?module=inline}{an
executive of Huawei}, the Chinese telecommunications giant, who was
recently arrested in Canada. China quickly arrested two Canadians,
apparently in retaliation.

Mr. Xi said that a country of China's size and influence was right to
hold ``lofty aspirations.''

``China will never develop itself by sacrificing the interests of other
countries,'' Mr. Xi said, but he added that China also would not
``abandon its own legitimate rights and interests.''

Throughout his speech, Mr. Xi performed similar rhetorical swerves,
promising both greater openness and assertiveness, both strong state
companies and prospering private businesses.

The government's
\href{https://www.nytimes3xbfgragh.onion/2018/12/16/world/asia/xinjiang-china-forced-labor-camps-uighurs.html}{intensifying
repression of Muslims in Xinjiang},
\href{https://www.nytimes3xbfgragh.onion/2018/12/13/world/asia/china-religion-crackdown.html}{crackdown
on Christians} and secretive detention of
\href{https://www.nytimes3xbfgragh.onion/2018/10/05/world/europe/meng-hongwei-missing-interpol.html}{the
Chinese chief of Interpol} have clouded its global standing at a time
when it aspires to play a larger international role.

Mr. Xi's speech risked leaving Chinese officials no clearer about his
policy agenda at a time when relations with the United States in
particular have deteriorated badly.

\includegraphics{https://static01.graylady3jvrrxbe.onion/images/2018/12/19/world/19china-xi-2/merlin_148271085_8a7c9a40-9bd8-44bc-adb5-3ba36353e113-articleLarge.jpg?quality=75\&auto=webp\&disable=upscale}

``When everything is a priority, nothing is a priority,'' Yuen Yuen Ang,
a professor of political science at the University of Michigan, Ann
Arbor, who studies China, said by email after watching the speech.
``Today, many policy goals in China are in tension with one another.
Which ones take precedence? This is what officials will need to know to
carry out their work on a practical level.''

Even as Mr. Xi spoke,
\href{https://www.nytimes3xbfgragh.onion/2018/12/18/business/stock-markets.html}{stock
markets dropped in Asia}. Though such speeches are not China's usual
vehicle for announcing specific policy measures, some investors had been
hoping for signals that Beijing would take further steps to liberalize
the economy or ease tensions with Washington.

Mr. Xi and Mr. Trump agreed early this month to
\href{https://www.nytimes3xbfgragh.onion/2018/12/01/world/trump-xi-g20-merkel.html}{call
a truce in disputes over trade and investment}, and to allow 90 days to
reach an agreement.

But Mr. Xi's speech on Tuesday was likely to dampen hopes of a
breakthrough, said Ryan L. Hass, a former director for China at the
National Security Council who is now a fellow at the Brookings
Institution.

Mr. Hass said his Chinese contacts had ``described the speech as the
place where Xi would send a signal to Trump on his own terms about the
market openings and other reforms on the horizon.''

``If those messages were embedded in the speech,'' he added, ``they
appear to have been well concealed.''

Mr. Xi warned that the future contained ``all kinds of risks and
challenges,'' but he said repeatedly that the party had expertly guided
the country thus far and must continue to do so. He emphasized twice
that the party had been ``completely correct'' in its embrace of
economic overhauls, a remark that brushed over the many internal
debates, as well as ups and downs, that accompanied those changes.

Mr. Xi called for revitalizing Marxist-Leninist doctrine, a reflection
of the party's fears that it could lose its grip over a younger,
increasingly wired and well-traveled generation. ``Let contemporary
Chinese Marxism shine even more brilliant rays of truth,'' he said.

According to Julian B. Gewirtz, a scholar at the Weatherhead Center for
International Affairs at Harvard, who watched the speech while visiting
Beijing, ``This was a speech about the party more than anything else.''

It remains to be seen whether Mr. Xi's remarks will reassure Chinese
private companies, as he has tried in recent weeks to do. Business
leaders and economists have complained about meddlesome officials, heavy
and capricious tax burdens, restrictions on investment and banks that
prefer to channel loans to big state companies that enjoy the patronage
of party leaders. They have welcomed Mr. Xi's promises, but also warned
that the economy remains troubled by risks.

``Of all the anniversaries related to the reforms --- 20 years, 25
years, 35 years --- this 40th anniversary is perhaps the least
optimistic I have seen,'' said Ding Xueliang, a professor emeritus at
the Hong Kong University of Science and Technology who has long studied
China's reforms.

``People in very senior positions also have no clear idea of the
direction,'' he added. ``Not a single person in the past half year who I
talked with in China, not a single person, said he or she is clear about
the next stage.''

Image

A billboard in the southern Chinese city of Shenzhen this week featured
Deng Xiaoping, the late Chinese leader who presided over economic
reforms in the 1980s.Credit...Nicolas Asfouri/Agence France-Presse ---
Getty Images

Adding to the anxiety were signs that the government was tightening the
release of local economic data amid a sharp slowdown. Last month, the
southern province of Guangdong stopped releasing the results of a
monthly purchasing managers' index --- a survey that takes the
temperature of China's important factory sector --- citing a notice from
the National Bureau of Statistics. The bureau said on Tuesday that the
province had violated regulations on statistics gathering.

In his speech, Mr. Xi repeatedly touted the huge advances China has made
since ``reform and opening'' began in 1978, rattling off detailed
statistics on personal incomes, education and life expectancy. Gone are
the days when food and clothing were rationed, he said.

``Hunger, food shortages and poverty, which plagued the Chinese people
for thousands of years, have been generally left behind,'' he said.

Mr. Xi paid tribute to Deng Xiaoping, the former leader who presided
over the reforms in the 1980s. In past anniversaries of the reform era,
Deng stood out in tributes and displays, but Mr. Xi has
\href{https://www.nytimes3xbfgragh.onion/2018/11/05/world/asia/china-xi-jinping-deng-xiaoping.html}{shifted
the spotlight} to his own achievements since he became party leader in
2012.

Mr. Xi has repeatedly promised to ensure that China offers businesses
and foreign investors an open, fair market, but many have become
skeptical that he will follow through. Instead, many say, Mr. Xi's drive
to extend party control, stifle public debate and defend the state
sector have stymied economic liberalization.

``Pledges to reform are sincere, but simultaneous pledges to prevent all
instability too often nullify progress,'' said Daniel H. Rosen, a
founding partner of Rhodium Group, an economic analysis firm that helps
keep a
\href{https://rhg.com/research/the-china-dashboard-tracking-chinas-economic-reforms/}{running
scorecard} on China's promised changes. ``Reform necessarily means some
instability, and trying to have it both ways will not work.''

The occasion of the speech on Tuesday was the anniversary of a party
meeting in 1978, when Deng and other veteran leaders who had fallen
during Mao's Cultural Revolution began to reassert their power and lay
out ideas for restoring the economy after decades of strife.

The meeting now features in the party's heavily mythologized history of
that time as a watershed, although it was only years afterward that
``reform and opening up'' became an official party formula.

Before Mr. Xi's speech, Chinese economists who favor market reforms had
openly voiced frustration with what they said was the slow, muddled pace
of change. They appear likely to be disappointed, and even worried.

``We're sincerely hoping that this big meeting will be able to sound a
clarion call for deepening reform,'' Xiang Songzuo, a senior economist
at Renmin University in Beijing, said at a forum in Shanghai over the
weekend.

He cited an estimate from researchers at an unidentified official
institute who concluded that China's real rate of economic growth this
year could be just 1.67 percent, or even lower. That projection is at
the very low end of economists' estimates, but Chinese growth is widely
believed to be lower than official estimates, which forecast an
expansion of 6.5 percent this year.

If there was no strongly reformist call from leaders, Professor Xiang
said, ``My final conclusion will be that China's economy is headed for a
plight that will last for a considerable time and be very, very
difficult.'' He did not respond to emails or messages after Mr. Xi's
speech.

Advertisement

\protect\hyperlink{after-bottom}{Continue reading the main story}

\hypertarget{site-index}{%
\subsection{Site Index}\label{site-index}}

\hypertarget{site-information-navigation}{%
\subsection{Site Information
Navigation}\label{site-information-navigation}}

\begin{itemize}
\tightlist
\item
  \href{https://help.nytimes3xbfgragh.onion/hc/en-us/articles/115014792127-Copyright-notice}{©~2020~The
  New York Times Company}
\end{itemize}

\begin{itemize}
\tightlist
\item
  \href{https://www.nytco.com/}{NYTCo}
\item
  \href{https://help.nytimes3xbfgragh.onion/hc/en-us/articles/115015385887-Contact-Us}{Contact
  Us}
\item
  \href{https://www.nytco.com/careers/}{Work with us}
\item
  \href{https://nytmediakit.com/}{Advertise}
\item
  \href{http://www.tbrandstudio.com/}{T Brand Studio}
\item
  \href{https://www.nytimes3xbfgragh.onion/privacy/cookie-policy\#how-do-i-manage-trackers}{Your
  Ad Choices}
\item
  \href{https://www.nytimes3xbfgragh.onion/privacy}{Privacy}
\item
  \href{https://help.nytimes3xbfgragh.onion/hc/en-us/articles/115014893428-Terms-of-service}{Terms
  of Service}
\item
  \href{https://help.nytimes3xbfgragh.onion/hc/en-us/articles/115014893968-Terms-of-sale}{Terms
  of Sale}
\item
  \href{https://spiderbites.nytimes3xbfgragh.onion}{Site Map}
\item
  \href{https://help.nytimes3xbfgragh.onion/hc/en-us}{Help}
\item
  \href{https://www.nytimes3xbfgragh.onion/subscription?campaignId=37WXW}{Subscriptions}
\end{itemize}
