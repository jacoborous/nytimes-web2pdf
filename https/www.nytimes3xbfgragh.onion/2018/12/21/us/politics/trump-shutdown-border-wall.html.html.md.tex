Sections

SEARCH

\protect\hyperlink{site-content}{Skip to
content}\protect\hyperlink{site-index}{Skip to site index}

\href{https://www.nytimes3xbfgragh.onion/section/politics}{Politics}

\href{https://myaccount.nytimes3xbfgragh.onion/auth/login?response_type=cookie\&client_id=vi}{}

\href{https://www.nytimes3xbfgragh.onion/section/todayspaper}{Today's
Paper}

\href{/section/politics}{Politics}\textbar{}Government Shuts Down as
Talks Fail to Break Impasse

\url{https://nyti.ms/2GzEfWo}

\begin{itemize}
\item
\item
\item
\item
\item
\item
\end{itemize}

Advertisement

\protect\hyperlink{after-top}{Continue reading the main story}

Supported by

\protect\hyperlink{after-sponsor}{Continue reading the main story}

\hypertarget{government-shuts-down-as-talks-fail-to-break-impasse}{%
\section{Government Shuts Down as Talks Fail to Break
Impasse}\label{government-shuts-down-as-talks-fail-to-break-impasse}}

\includegraphics{https://static01.graylady3jvrrxbe.onion/images/2018/12/24/us/22dc-cong-1/22dc-cong-1-videoSixteenByNine3000-v2.jpg}

By
\href{https://www.nytimes3xbfgragh.onion/by/julie-hirschfeld-davis}{Julie
Hirschfeld Davis} and
\href{https://www.nytimes3xbfgragh.onion/by/emily-cochrane}{Emily
Cochrane}

\begin{itemize}
\item
  Dec. 21, 2018
\item
  \begin{itemize}
  \item
  \item
  \item
  \item
  \item
  \item
  \end{itemize}
\end{itemize}

WASHINGTON --- The federal government shut down early Saturday after
congressional and White House officials failed to find a compromise on a
spending bill that hinged on President Trump's demands for \$5.7 billion
for a border wall.

It is the third shutdown in two years of unified Republican rule in
Washington, and it will stop work at nine federal departments and
several other agencies. Hundreds of thousands of government employees
are affected.

Any hope of a compromise ended about 8:30 p.m. Friday, when both the
House and the Senate had adjourned with no solution in sight. Talks are
expected to begin again on Saturday.

A burst of late-afternoon activity could not break the deadlock, even as
Vice President Mike Pence met with Senator Chuck Schumer of New York,
the Democratic leader, and senior House Republicans, searching for a
solution to a logjam that Mr. Trump has shown little interest in
breaking.

While the president has been unwilling to consider dropping his demand
to fund his signature campaign promise, Mr. Pence and other White House
officials were discussing a number of potential compromises that would
force him to do just that, omitting spending on a wall and instead
adding money for other security measures at the border, according to
several officials with knowledge of the talks.

Late Friday, as his budget director ordered the carrying out of shutdown
plans, President Trump told the country in a
\href{https://twitter.com/realDonaldTrump/status/1076308614372048897}{video
on Twitter} that ``we're going to have a shutdown.''

``There's nothing we can do about that because we need the Democrats to
give us their votes,'' he said in the video.

As in previous government shutdowns, it will not affect core government
functions like the Postal Service, the military, the Department of
Veterans Affairs and entitlement programs, including Social Security,
Medicaid, Medicare and food stamps.

But about 380,000 workers would be sent home and would not be paid.
Another 420,000 considered too essential to be furloughed would be
forced, like the Border Patrol officers, to work without pay.

The Departments of Agriculture, Commerce, Homeland Security, Housing and
Urban Development, Interior --- which includes national parks ---
Justice, State, Transportation and Treasury would all be affected. NASA
would also be hit.

\includegraphics{https://static01.graylady3jvrrxbe.onion/images/2018/12/21/us/politics/-22dc-cong-2/merlin_148393485_93caf949-0d2e-48f3-b804-cd44ecf41cfc-articleLarge.jpg?quality=75\&auto=webp\&disable=upscale}

There had been a glimmer of progress late in the day when the Senate
voted, 48 to 47, with Mr. Pence breaking a tie, to begin debating
stopgap spending legislation passed Thursday night by the House that
would keep the government running through Feb. 8 and provide \$5.7
billion to begin construction of the wall on the southwestern border.

But the vote was more a repudiation of Mr. Trump's proposal than an
endorsement of it. Senators in both parties conceded that the measure
could not pass the chamber, where major legislation requires bipartisan
support, and said they were advancing it only to allow negotiations
between the White House and congressional leaders in both parties to
proceed on a compromise that all sides could accept.

Senator Mitch McConnell, Republican of Kentucky and the majority leader,
said the Senate had approved the measure ``in order to preserve maximum
flexibility for productive conversations to continue between the White
House and our Democratic colleagues.''

Mr. Schumer said the vote only underscored what Democrats had been
telling Mr. Trump since last week, when the president declared during a
combative Oval Office meeting that he would be proud to shut down the
government and shoulder the blame if he could not win support to fund
his border wall.

``His wall does not have 60 votes here in the Senate, let alone 50 votes
--- that much is now clear,'' Mr. Schumer said. ``We are willing to
continue discussions'' on proposals to keep the government funded, he
added.

In his Friday night video, Mr. Trump appeared to be moderating his
position slightly, calling for ``great border security with a wall, or a
slat fence, or whatever you want to call it --- but we need a great
barrier.''

``Let's be bipartisan and let's get it done,'' Mr. Trump said. ``The
shutdown hopefully will not last long.''

The shutdown is an appropriate end to a period of unified Republican
rule of the White House and both chambers of Congress that has been
marked by dysfunction and infighting, a mercurial president whose
shifting positions and whims have scuttled legislative deals.

There was no clear sense of where talks might lead on Saturday. The
House and Senate remained on standby, planning to reconvene but devoid,
for the moment, of any measure that would reopen the government and
bridge the divide.

Friday's Senate vote unfolded as Mr. Pence, along with Mick Mulvaney,
Mr. Trump's budget director and incoming chief of staff, and Jared
Kushner, his son-in-law, huddled in the Capitol to jump-start
negotiations. The prospect of a deal has been hampered by the
president's refusal to budge on the wall, or to indicate what
alternatives he would be willing to accept to keep the government open.

\includegraphics{https://static01.graylady3jvrrxbe.onion/images/2018/12/22/us/politics/22dc-trump-sub/22dc-trump-sub-videoSixteenByNine3000.jpg}

Among the options discussed behind closed doors were proposals that
would allocate \$1.6 billion to a total of \$2.5 billion to border
security, none of which could be spent on a wall. But it was not clear
that conservatives in the House, who insisted on Thursday on adding the
\$5.7 billion for the physical barrier the president has demanded to the
stopgap spending measure, would back that solution.

``What I want is real money for the wall,'' said Representative Jim
Jordan of Ohio, a founder of the conservative House Freedom Caucus, who
declined to say how much funding he would consider sufficient, but said
\$1.6 billion was not enough.

In a meeting in his office just off the Senate floor, Mr. Schumer flatly
told Mr. Pence, Mr. Mulvaney and Mr. Kushner, who had requested to meet
with him, that any measure that included money for the wall could not
pass the Senate, and urged them to consider agreeing to one that omitted
it but included funding for other forms of border security, according to
a spokesman.

Complicating the chances of such a deal was the president's own refusal
to detail his bottom line in negotiations. During a meeting with
Republican senators on Friday morning, Mr. Trump would not provide
specifics about what kind of plan he could support, including how much
money he would accept for fortifying the border, despite their repeated
efforts to ascertain his conditions for a deal, according to a Senate
official briefed on the session who insisted on anonymity to describe
it. The president talked at length about the wall and repeatedly pressed
the senators about eliminating the filibuster so they could fund it with
51 votes.

Senator John Cornyn of Texas, the chamber's No. 2 Republican, appeared
practically giddy at the prospect that Mr. Trump's aides were engaging
in serious talks with lawmakers, saying, ``The fact that that's
happening represents real progress,'' and adding that he was ``so happy
about it.''

Without a strategy for averting a shutdown, Mr. Trump spent the day
maneuvering to ensure that Democrats would shoulder the blame,
notwithstanding his public courting of the dysfunctional denouement.

He began the day warning on Twitter that a partial government shutdown
``will last for a very long time.''

``If enough Dems don't vote, it will be a Democrat Shutdown!'' Mr. Trump
wrote. ``House Republicans were great yesterday!''

He was referring to a nearly party-line vote in the House on Thursday
night to add the wall funding to the stopgap spending bill despite
almost certain defeat of the measure once it reached the Senate. House
Republicans also added roughly \$8 billion in disaster aid for farmers,
a critical sweetener that helped advance a bill that they feared until
the last moment might not have enough votes to pass.

Yet the only certainty to emerge was an intense round of political
blame-shifting. House passage of the wall funding did change the
dynamics of the fight, putting Senate Democrats in the position of being
the spoilers of a measure to keep the government running.

Image

Senator Bob Corker of Tennessee delayed his vote, threatening to vote
against the measure.Credit...Erin Schaff for The New York Times

Democrats, who believe their leverage will only grow when they assume
the majority in the House in January, did not appear to be cowed by the
tactic.

``Abandon your shutdown strategy,'' Mr. Schumer said on the Senate
floor, addressing his remarks to the president. ``You're not getting the
wall today, next week or on Jan. 3 when Democrats take control of the
House.''

The president also urged Mr. McConnell to pursue what is known as the
``nuclear option'' and abolish a rule that allows any senator to block
final votes on legislation, often used by the minority party to thwart
major bills. The tactic was used by Senate Democrats to lower the
threshold to 51 votes and end a Republican blockade of President Barack
Obama's judicial nominees. Senate Republicans then used the same move to
end filibusters of Supreme Court nominees.

``Mitch, use the Nuclear Option and get it done!''
\href{https://twitter.com/realDonaldTrump/status/1076100620581511170}{Mr.
Trump tweeted}. ``Our Country is counting on you!''

Mr. McConnell has long said that there was no support for dismantling
the 60-vote requirement on legislation, and he and a number of senior
Republican senators released statements on Friday morning before the
meeting with Mr. Trump making it clear it would not happen.

``The leader has said for years that the votes are not there in the
conference to use the nuclear option,'' David Popp, his spokesman, said
in a statement. ``Just this morning, several senators put out statements
confirming their opposition, and confirming that there is not a majority
in the conference to go down that road.''

Even as Mr. Trump mocked Democratic opposition and objections to his
vision of a wall at the border with Mexico (``It's like the wheel, there
is nothing better,'' Mr. Trump wrote), he seemed to acknowledge that the
wall funding proposal was not getting through the Senate.

``No matter what happens today in the Senate, Republican House Members
should be very proud of themselves,'' Mr. Trump wrote. ``They flew back
to Washington from all parts of the World in order to vote for Border
Security and the Wall.''

``We will get it done, one way or the other!'' the president wrote in
another tweet.

With funding set to expire, the nine federal departments and several
other agencies were beginning to prepare themselves. Some agencies will
have enough money in the pipeline to carry them into the new year, but
thousands of government workers are expected to be furloughed or
required to work through the holidays without pay.

``It's actually part of what you do when you sign up for any public
service position,'' Representative Mark Meadows, Republican of North
Carolina and chairman of the Freedom Caucus, told reporters on Thursday.
``It's not lost on me in terms of the potential hardship.''

Several House lawmakers blamed Representative Nancy Pelosi, Democrat of
California, and Mr. Schumer for the impending shutdown, arguing that
they were unwilling to compromise on border security. But with Democrats
set to reclaim the House majority in two weeks, there is little
motivation for Ms. Pelosi to acquiesce to the president's demands.

In the aftermath of Mr. Trump's insistence that he would own a
government shutdown, House Democratic aides had already begun crafting
legislation that would reopen the government come Jan. 3 and the
swearing-in of new members.

Advertisement

\protect\hyperlink{after-bottom}{Continue reading the main story}

\hypertarget{site-index}{%
\subsection{Site Index}\label{site-index}}

\hypertarget{site-information-navigation}{%
\subsection{Site Information
Navigation}\label{site-information-navigation}}

\begin{itemize}
\tightlist
\item
  \href{https://help.nytimes3xbfgragh.onion/hc/en-us/articles/115014792127-Copyright-notice}{©~2020~The
  New York Times Company}
\end{itemize}

\begin{itemize}
\tightlist
\item
  \href{https://www.nytco.com/}{NYTCo}
\item
  \href{https://help.nytimes3xbfgragh.onion/hc/en-us/articles/115015385887-Contact-Us}{Contact
  Us}
\item
  \href{https://www.nytco.com/careers/}{Work with us}
\item
  \href{https://nytmediakit.com/}{Advertise}
\item
  \href{http://www.tbrandstudio.com/}{T Brand Studio}
\item
  \href{https://www.nytimes3xbfgragh.onion/privacy/cookie-policy\#how-do-i-manage-trackers}{Your
  Ad Choices}
\item
  \href{https://www.nytimes3xbfgragh.onion/privacy}{Privacy}
\item
  \href{https://help.nytimes3xbfgragh.onion/hc/en-us/articles/115014893428-Terms-of-service}{Terms
  of Service}
\item
  \href{https://help.nytimes3xbfgragh.onion/hc/en-us/articles/115014893968-Terms-of-sale}{Terms
  of Sale}
\item
  \href{https://spiderbites.nytimes3xbfgragh.onion}{Site Map}
\item
  \href{https://help.nytimes3xbfgragh.onion/hc/en-us}{Help}
\item
  \href{https://www.nytimes3xbfgragh.onion/subscription?campaignId=37WXW}{Subscriptions}
\end{itemize}
