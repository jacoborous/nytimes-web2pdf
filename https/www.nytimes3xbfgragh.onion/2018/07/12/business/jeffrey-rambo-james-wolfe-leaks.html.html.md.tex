Sections

SEARCH

\protect\hyperlink{site-content}{Skip to
content}\protect\hyperlink{site-index}{Skip to site index}

\href{https://www.nytimes3xbfgragh.onion/section/business}{Business}

\href{https://myaccount.nytimes3xbfgragh.onion/auth/login?response_type=cookie\&client_id=vi}{}

\href{https://www.nytimes3xbfgragh.onion/section/todayspaper}{Today's
Paper}

\href{/section/business}{Business}\textbar{}Border Agent Who Questioned
Reporter Is Investigated for Computer Misuse

\url{https://nyti.ms/2JgYKTL}

\begin{itemize}
\item
\item
\item
\item
\item
\end{itemize}

Advertisement

\protect\hyperlink{after-top}{Continue reading the main story}

Supported by

\protect\hyperlink{after-sponsor}{Continue reading the main story}

\hypertarget{border-agent-who-questioned-reporter-is-investigated-for-computer-misuse}{%
\section{Border Agent Who Questioned Reporter Is Investigated for
Computer
Misuse}\label{border-agent-who-questioned-reporter-is-investigated-for-computer-misuse}}

\includegraphics{https://static01.graylady3jvrrxbe.onion/images/2018/07/13/business/13rambo1/merlin_141013587_33b2bf61-c310-4bd0-9600-dc5c1d989a2e-articleLarge.jpg?quality=75\&auto=webp\&disable=upscale}

By \href{https://www.nytimes3xbfgragh.onion/by/scott-shane}{Scott Shane}
and \href{https://www.nytimes3xbfgragh.onion/by/ron-nixon}{Ron Nixon}

\begin{itemize}
\item
  July 12, 2018
\item
  \begin{itemize}
  \item
  \item
  \item
  \item
  \item
  \end{itemize}
\end{itemize}

A Border Patrol agent who obtained the confidential travel records of a
Washington journalist and used them to press her about her sources last
year is under investigation for misuse of government computer systems,
according to an official briefed on the inquiry.

The agent, Jeffrey A. Rambo, who usually worked in the San Diego area,
was temporarily assigned at the time to the National Targeting Center, a
facility in Sterling, Va., operated by Customs and Border Protection
that stores data on the travel of millions of Americans and foreigners.
Such information is supposed to be used only under strict rules by
immigration and law enforcement officials.

Now the Department of Homeland Security's inspector general and
investigators from the border agency are examining whether Mr. Rambo
used the travel data improperly or illegally and whether anyone else was
involved. Press advocates have expressed alarm that a government
official would use sensitive private information in what they say
amounted to a blackmail attempt against a journalist.

On June 1, 2017, Mr. Rambo, 33, contacted Ali Watkins, a reporter for
Politico at the time who now works at The New York Times, saying that he
needed to meet her in Washington immediately. He told Ms. Watkins that
he worked for the government but declined to give his name or agency.

In a lengthy conversation at a bar near Dupont Circle, Mr. Rambo claimed
to be helping the F.B.I. with investigations into leaks of sensitive
material to journalists. He eventually revealed that he knew the details
of a trip to Spain that Ms. Watkins had taken with James A. Wolfe,
security director of the Senate Intelligence Committee, who was then her
boyfriend.

\href{https://www.nytimes3xbfgragh.onion/2018/06/24/business/media/james-wolfe-ali-watkins-leaks-reporter.html}{According
to accounts Ms. Watkins provided to friends} and editors, Mr. Rambo
hinted that he might disclose their relationship to The Washington Post
and pressed her to become his informant and report to him on other
journalists and their sources.

Ms. Watkins rejected the request and returned the next day to the bar
where they had talked and learned his identity from a credit card
receipt. The episode was widely reported last month after
\href{https://www.nytimes3xbfgragh.onion/2018/06/07/us/politics/times-reporter-phone-records-seized.html}{Mr.
Wolfe was arrested} and accused of lying to F.B.I. leak investigators
about his contacts with Ms. Watkins and other reporters. He has pleaded
not guilty.

News of Mr. Wolfe's arrest also revealed that Ms. Watkins's email and
phone records
\href{https://www.nytimes3xbfgragh.onion/2018/06/24/business/media/james-wolfe-ali-watkins-leaks-reporter.html}{had
been secretly seized by the Trump administration} as part of the leak
inquiry. The Times subsequently conducted an internal review of Ms.
Watkins's actions, and she
\href{https://www.nytimes3xbfgragh.onion/2018/07/03/business/media/ali-watkins-times-reporter-memo.html}{was
recently reassigned} to its New York newsroom.

Mr. Rambo's actions raised several questions: What was a California
border patrol agent doing in the Washington area? Was he really helping
the F.B.I. with leak investigations, as he claimed? Was his anonymous
approach to Ms. Watkins, which violated law enforcement standards, part
of an authorized operation or the work of a rogue agent?

Law enforcement officials have said they can find no evidence that Mr.
Rambo was officially assigned to work on leak investigations. Officials
at Customs and Border Protection and its parent agency, Homeland
Security, have declined to answer questions about Mr. Rambo's role,
citing the internal investigation.

Mr. Rambo has not replied to repeated requests for comment. But several
government officials, speaking on condition of anonymity, have supplied
some basic facts.

The officials confirmed that Mr. Rambo had been assigned to the National
Targeting Center, which is why he was in the Washington area and might
have had access to Ms. Watkins' private travel information.

It remains unclear whether Mr. Rambo handled or heard about an official
F.B.I. request to the center for Mr. Wolfe's travel records, and, if so,
whether that led to the discovery that Ms. Watkins was his traveling
companion. According to Ms. Watkins's accounts, Mr. Rambo spoke with
enthusiasm to her about Mr. Trump's crackdown on leaks, telling her that
``we're finally going to be able to drain the swamp,'' raising the
possibility that he had searched the database for her records on his own
initiative.

Either way, for Mr. Rambo, the venture into combating leaks appears to
be the latest expression of an entrepreneurial personality. Public
records and internet archives show that starting in his teenage years,
he has embarked on ambitious enterprises, although they have produced
only modest results.

When he was 16, records show, he helped start what he called an ``online
consulting company,'' called Rambo Harrington \& Hopkins, to advise
small businesses how to use the web. In 2003, he announced in a
\href{http://web.archive.org/web/20030209075010/http://www.brandergy.com/}{news
release} that his business had ``evolved into Brandergy, Inc., a new
firm with a new sense of direction and purpose.''

``The new firm, where Rambo finds himself as Managing Director and CEO,
will focus all of its attention to the Branding world,'' said the
release, written when Mr. Rambo was 17.

But Brandergy went nowhere, according to Kwan Skinner, an experienced
programmer whom Mr. Rambo announced as ``Operations Director and Chief
Technology Officer.'' Mr. Skinner said in an interview that he had
supplied software to Mr. Rambo but had never received the payments he
was promised, and that he had eventually dropped out of the effort.

``I don't think he was doing anything deliberately dishonest or
manipulative,'' Mr. Skinner said of Brandergy. ``It just never took
off.''

Mr. Rambo joined the border patrol in 2007, serving in and around San
Diego. While there, he created a website, jefframbo.com, where he
periodically posted blog items recounting his thoughts on law
enforcement, current events and developments in San Diego, sometimes
posting photographs.

\href{http://web.archive.org/web/20131005004011/http://jefframbo.com/2011/07/01/justice-anyone/}{One
2011 post} hinted at a difficult upbringing.

``We were deprived of cousins to call over for playtime because they
were in graves for traveling the wrong path in life,'' he wrote. ``We
were deprived of fathers that were serving prison sentences. We were
deprived of going outside after certain hours due to fears of
drive-by's. We were deprived of a mother due to an addiction. We were
deprived of our happiness in result of too many beatings. We were
deprived of our brothers who liked the wrong colors and paid a price for
it.''

In 2014, again showing a flair for promotional entrepreneurship, Mr.
Rambo announced that he and a partner would soon open a microbrewery,
Social Jack's Brewing Company, in San Diego's Little Italy neighborhood.
He had lined up a location and an experienced manager, and the plans
were
\href{https://sandiego.eater.com/2014/2/19/6276997/new-little-italy-brewery-wants-to-get-social-with-you}{written
up in San Diego Eater}, a website, and described in
\href{https://ksr-video.imgix.net/projects/889648/video-351836-h264_high.mp4}{a
captivating video}.

Once again, the business appears to have failed quickly, leaving
multiple claims against Mr. Rambo for start-up loans he had not repaid,
court records show.

Mr. Rambo's blog went silent for a long stretch, but in 2016 he used it
to promote a new idea: to turn business failure into a kind of success.

``Have you ever set out to achieve greatness only to find yourself alone
in a corner, with nothing to show but the resulting fear of
embarrassment, regret, and worst of all \ldots{} shame?'' he wrote. ``If
there were ever fifty words to surmise what has become of my life, you
just read them. But the most important words are those to come, which
detail my rebound and will inspire you toward greatness of your own.''

Nothing appears to have followed. But early this year, months after his
encounter with Ms. Watkins, he added
\href{https://web.archive.org/web/20170912110924/http://jefframbo.com/}{a
nicely designed, blue-and-red logo} to the blog.

``Dear Failure with Jeff Rambo,'' it said. The website is no longer
online and can be viewed only in archived versions.

Advertisement

\protect\hyperlink{after-bottom}{Continue reading the main story}

\hypertarget{site-index}{%
\subsection{Site Index}\label{site-index}}

\hypertarget{site-information-navigation}{%
\subsection{Site Information
Navigation}\label{site-information-navigation}}

\begin{itemize}
\tightlist
\item
  \href{https://help.nytimes3xbfgragh.onion/hc/en-us/articles/115014792127-Copyright-notice}{©~2020~The
  New York Times Company}
\end{itemize}

\begin{itemize}
\tightlist
\item
  \href{https://www.nytco.com/}{NYTCo}
\item
  \href{https://help.nytimes3xbfgragh.onion/hc/en-us/articles/115015385887-Contact-Us}{Contact
  Us}
\item
  \href{https://www.nytco.com/careers/}{Work with us}
\item
  \href{https://nytmediakit.com/}{Advertise}
\item
  \href{http://www.tbrandstudio.com/}{T Brand Studio}
\item
  \href{https://www.nytimes3xbfgragh.onion/privacy/cookie-policy\#how-do-i-manage-trackers}{Your
  Ad Choices}
\item
  \href{https://www.nytimes3xbfgragh.onion/privacy}{Privacy}
\item
  \href{https://help.nytimes3xbfgragh.onion/hc/en-us/articles/115014893428-Terms-of-service}{Terms
  of Service}
\item
  \href{https://help.nytimes3xbfgragh.onion/hc/en-us/articles/115014893968-Terms-of-sale}{Terms
  of Sale}
\item
  \href{https://spiderbites.nytimes3xbfgragh.onion}{Site Map}
\item
  \href{https://help.nytimes3xbfgragh.onion/hc/en-us}{Help}
\item
  \href{https://www.nytimes3xbfgragh.onion/subscription?campaignId=37WXW}{Subscriptions}
\end{itemize}
