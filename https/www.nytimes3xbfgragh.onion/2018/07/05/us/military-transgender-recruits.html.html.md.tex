Sections

SEARCH

\protect\hyperlink{site-content}{Skip to
content}\protect\hyperlink{site-index}{Skip to site index}

\href{https://www.nytimes3xbfgragh.onion/section/us}{U.S.}

\href{https://myaccount.nytimes3xbfgragh.onion/auth/login?response_type=cookie\&client_id=vi}{}

\href{https://www.nytimes3xbfgragh.onion/section/todayspaper}{Today's
Paper}

\href{/section/us}{U.S.}\textbar{}Ban Was Lifted, but Transgender
Recruits Still Can't Join Up

\url{https://nyti.ms/2lW4ARt}

\begin{itemize}
\item
\item
\item
\item
\item
\item
\end{itemize}

Advertisement

\protect\hyperlink{after-top}{Continue reading the main story}

Supported by

\protect\hyperlink{after-sponsor}{Continue reading the main story}

\hypertarget{ban-was-lifted-but-transgender-recruits-still-cant-join-up}{%
\section{Ban Was Lifted, but Transgender Recruits Still Can't Join
Up}\label{ban-was-lifted-but-transgender-recruits-still-cant-join-up}}

\includegraphics{https://static01.graylady3jvrrxbe.onion/images/2018/07/06/us/06transgender-print-2/merlin_140740203_a17e9dd9-b1c7-460e-b681-8f62193b134d-articleLarge.jpg?quality=75\&auto=webp\&disable=upscale}

By \href{http://www.nytimes3xbfgragh.onion/by/dave-philipps}{Dave
Philipps}

\begin{itemize}
\item
  July 5, 2018
\item
  \begin{itemize}
  \item
  \item
  \item
  \item
  \item
  \item
  \end{itemize}
\end{itemize}

Nicholas Bade showed up at an Air Force recruiting office on an icy
morning in January, determined to be one of the first transgender
recruits to enlist in the military.

He was in top shape, and had earned two martial arts black belts. He had
already aced the military aptitude test, and organized the stack of
medical records required to show he was stable and healthy enough to
serve. So he expected to be called for basic training in a month, maybe
two at the most.

Six months later, he's still waiting. And so are nearly all other
transgender recruits who have tried to join up since a federal court
ordered the Trump administration not to ban them from the military.

The Obama administration announced a plan in 2016 for the armed services
to begin accepting transgender recruits at the start of this year. But
before the plan could take effect, President Trump abruptly reversed
course,
\href{https://twitter.com/i/moments/890200249884651520?lang=en}{announcing
on Twitter} in July 2017 that the military would ``no longer accept or
allow transgender individuals to serve in any capacity,'' because the
military ``cannot be burdened with the tremendous medical costs and
disruption that transgender in the military would entail.'' Military
leaders were given little notice of the change, which has left a wake of
controversy and confusion.

Civil rights groups immediately sued, claiming that a blanket ban was
unconstitutional, and the courts blocked the new rules. Three federal
judges hearing separate cases issued injunctions against the ban last
fall that cleared the way --- in theory at least --- for transgender
recruits to start enlisting on Jan. 1.

Since then, scores have applied --- but it appears almost none are being
accepted.

The Defense Department refused requests for statistics on transgender
enlistments. But Sparta, an organization for transgender recruits,
troops and veterans, says that out of its 140 members who are trying to
enlist, only two have made it into the service since Jan. 1.

Others have been stymied by the Military Entrance Processing Command,
which has rejected some of the applicants and kept others in limbo for
months by requesting ever more detailed medical documentation. Other
advocates said the Sparta members' experiences probably reflected the
overall picture for transgender enlistment.

The applicants are being stalled or turned away at a time when some
branches of the military face
\href{https://www.military.com/daily-news/2018/04/21/goal-80000-recruits-year-wont-be-met-army-secretary-says.html}{a
shortage of recruits}, and when recruiters have been ordered to work
Saturdays to try to make up the shortfall.

``I'm now on round five of rejections,'' said Mr. Bade, 38, a waiter and
martial arts instructor who lives in Chicago. ``Each time, they say they
need even more medical information. My last one was a minor document
from years ago.''

Mr. Bade began taking hormones in 2014, and had breast-removal surgery a
year later. He has had so few issues since then, he said, that he often
forgets he is transgender. His ambition is to become a dog handler in
the Air Force's security forces, but he is beginning to wonder if it
will ever happen.

Other applicants now in limbo say their transgender status rarely
hinders them in civilian life. One is a rugby coach. One is a substitute
teacher. One repairs tractors and heaves bales of hay for the cattle
that he and his grandmother keep on a small hillside farm in Appalachia.
Another moves 200-pound tanks of carbon dioxide for a job creating
special effects for Broadway shows.

Most say that military recruiters have supported their enlistment, but
their applications have gotten hung up in the medical review.

``We're hesitant to speak up, because we don't want to be treated as
special, but this has become a huge headache,'' said one 26-year-old who
is trying to join the Coast Guard Reserve. He said he has spent months
gathering medical notes, lab results, hormone records and doctors'
credentials going back four years to support his application. He asked
not to be identified for fear that any public attention would hurt his
chances of acceptance.

Transgender groups like Sparta initially hailed the court injunctions
last fall as victories. But their optimism has melted as months have
passed with so few recruits actually being allowed to enlist. Most
advocacy groups are trying to be patient, chalking the delays up to the
inevitable inertia of a giant bureaucracy forced to change. But some are
beginning to question whether the delays are evidence of a concerted
effort to keep transgender recruits out, despite the court rulings.

\includegraphics{https://static01.graylady3jvrrxbe.onion/images/2018/07/06/us/06transgender-print-1/merlin_140496852_f42f69d4-79f7-4238-b5de-8303ffa8e5e6-articleLarge.jpg?quality=75\&auto=webp\&disable=upscale}

``We've heard people are meeting with mystifying obstacles,'' said
Shannon Minter, a lawyer with the National Center for Lesbian Rights,
which
\href{https://www.nytimes3xbfgragh.onion/2017/10/30/us/military-transgender-ban.html}{sued}
the Trump administration over the ban. ``We want to give the military
the benefit of the doubt, but at this point so few applicants have been
accepted, there is reason to be concerned that there is some passive
resistance to the injunctions, and people are getting slow-walked.''

Mr. Minter also worries that the military may seize on unrelated medical
issues as a pretext for rejecting transgender recruits.

One applicant in Ohio spent five months submitting more and more medical
records, and then was rejected in late May because of knee surgery he
had as an infant. The applicant, who asked not to be named because he
still hopes to join the military, said he was dumbfounded at the
rejection, because he has had no issues stemming from the surgery for 25
years.

The Defense Department declined to make any officials available for
interview, citing pending litigation. It refused to say how long
recruits have been kept waiting or how many have been rejected on
medical grounds. But it said in a written statement that it ``continues
to comply with the court order,'' and that ``the time it takes to review
each individual record will vary based upon the individual.''

Thousands of transgender troops, who officially came out or transitioned
in the military when the Obama administration decided in 2016
\href{https://www.nytimes3xbfgragh.onion/2016/07/01/us/transgender-military.html}{to
lift a ban}, are serving now. A
\href{https://www.rand.org/pubs/research_reports/RR1530.html}{RAND
Corporation study} in 2016 estimated their number at between 2,000 and
11,000. Many are in demanding jobs and have deployed overseas.

Leaders of the Army, Marines, Air Force, Navy and Coast Guard told
Congress this spring that they have seen no issues with the transgender
troops. ``As long as they can meet the standard of what their particular
occupation was, I think we'll move forward,'' Gen. Robert Neller, the
commandant of the Marine Corps, said in his testimony.

But the Trump administration continues to oppose any transgender
military service. Before it was blocked by the court injunctions, the
administration sought not only to keep transgender troops from joining,
but to discharge those already in the ranks. Defense Secretary Jim
Mattis issued a
\href{https://media.defense.gov/2018/Mar/23/2001894037/-1/-1/0/MILITARY-SERVICE-BY-TRANSGENDER-INDIVIDUALS.PDF}{memo}
in February saying their presence threatened to ``undermine readiness,
disrupt unit cohesion, and impose an unreasonable burden on the
military.''

Last month, the Justice Department filed a
\href{https://www.documentcloud.org/documents/4568643-TransportRoom.html}{motion}
to overturn one of the injunctions, arguing that the panel of Defense
Department experts who created the Trump administration policy had the
necessary authority to ban particular categories of recruits, and that
the court had ``provided scant explanation for disregarding that
reasoned and reasonable military assessment.''

Opponents of transgender service have argued that transgender recruits
\href{https://hartzler.house.gov/media-center/press-releases/hartzler-statement-ndaa-amendment-reverse-obama-transgender-policy}{could
shoulder the Pentagon with huge medical costs}, and could be sidelined
from duty for long periods by surgical procedures.

Those eager to enlist counter that transgender people serve without
problems now in police and fire departments and in federal law
enforcement. For many, they say, the only continuing medical care they
need are inexpensive hormone doses that they can administer themselves
at home.

Regulations for transgender recruits require them to show that they have
been mentally and physically stable for 18 months before enlisting; a
similar standard is applied to recruits who have had other medical
procedures. Applicants must also have a civilian doctor certify that
their transition is complete and does not limit their ability to serve.

``I think the requirements are reasonable,'' said Paula Neira, who heads
the Center for Transgender Health at Johns Hopkins Medicine. Ms. Neira
is a former Navy officer who transitioned after she left the military in
1991; she helped write the Obama-era guidelines that were kept in place
by the courts.

The long delays, she said, are less likely to be caused by an
intentional and illegal effort to exclude transgender recruits than by
simple bureaucratic caution over a new policy.

``There is no one doing these assessments that is an expert in
transgender health, so they have to figure things out as they go
along,'' she said. ``If you are that far outside your expertise, you are
going to be very conservative.''

If the medical evaluations continue to drag on, she said, there could
well be cause for alarm. But she urged patience.

``I know how hard it is to wait --- I waited for 25 years,'' she said.
``If it had been different, I'd still be in the Navy. But it took so
long to change the regulations that the clock ran out on me.''

Advertisement

\protect\hyperlink{after-bottom}{Continue reading the main story}

\hypertarget{site-index}{%
\subsection{Site Index}\label{site-index}}

\hypertarget{site-information-navigation}{%
\subsection{Site Information
Navigation}\label{site-information-navigation}}

\begin{itemize}
\tightlist
\item
  \href{https://help.nytimes3xbfgragh.onion/hc/en-us/articles/115014792127-Copyright-notice}{©~2020~The
  New York Times Company}
\end{itemize}

\begin{itemize}
\tightlist
\item
  \href{https://www.nytco.com/}{NYTCo}
\item
  \href{https://help.nytimes3xbfgragh.onion/hc/en-us/articles/115015385887-Contact-Us}{Contact
  Us}
\item
  \href{https://www.nytco.com/careers/}{Work with us}
\item
  \href{https://nytmediakit.com/}{Advertise}
\item
  \href{http://www.tbrandstudio.com/}{T Brand Studio}
\item
  \href{https://www.nytimes3xbfgragh.onion/privacy/cookie-policy\#how-do-i-manage-trackers}{Your
  Ad Choices}
\item
  \href{https://www.nytimes3xbfgragh.onion/privacy}{Privacy}
\item
  \href{https://help.nytimes3xbfgragh.onion/hc/en-us/articles/115014893428-Terms-of-service}{Terms
  of Service}
\item
  \href{https://help.nytimes3xbfgragh.onion/hc/en-us/articles/115014893968-Terms-of-sale}{Terms
  of Sale}
\item
  \href{https://spiderbites.nytimes3xbfgragh.onion}{Site Map}
\item
  \href{https://help.nytimes3xbfgragh.onion/hc/en-us}{Help}
\item
  \href{https://www.nytimes3xbfgragh.onion/subscription?campaignId=37WXW}{Subscriptions}
\end{itemize}
