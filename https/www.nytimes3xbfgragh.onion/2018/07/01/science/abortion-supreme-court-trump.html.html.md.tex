Sections

SEARCH

\protect\hyperlink{site-content}{Skip to
content}\protect\hyperlink{site-index}{Skip to site index}

\href{https://www.nytimes3xbfgragh.onion/section/health}{Health}

\href{https://myaccount.nytimes3xbfgragh.onion/auth/login?response_type=cookie\&client_id=vi}{}

\href{https://www.nytimes3xbfgragh.onion/section/todayspaper}{Today's
Paper}

\href{/section/health}{Health}\textbar{}Bulwark Against an Abortion Ban?
Medical Advances

\url{https://nyti.ms/2lM9Uqg}

\begin{itemize}
\item
\item
\item
\item
\item
\item
\end{itemize}

Advertisement

\protect\hyperlink{after-top}{Continue reading the main story}

Supported by

\protect\hyperlink{after-sponsor}{Continue reading the main story}

\hypertarget{bulwark-against-an-abortion-ban-medical-advances}{%
\section{Bulwark Against an Abortion Ban? Medical
Advances}\label{bulwark-against-an-abortion-ban-medical-advances}}

\includegraphics{https://static01.graylady3jvrrxbe.onion/images/2018/07/02/science/02-jp-abortion-print/00ABORTION1-articleLarge.jpg?quality=75\&auto=webp\&disable=upscale}

By \href{http://www.nytimes3xbfgragh.onion/by/pam-belluck}{Pam Belluck}
and \href{http://www.nytimes3xbfgragh.onion/by/jan-hoffman}{Jan Hoffman}

\begin{itemize}
\item
  July 1, 2018
\item
  \begin{itemize}
  \item
  \item
  \item
  \item
  \item
  \item
  \end{itemize}
\end{itemize}

As partisans on both sides of the abortion divide contemplate a Supreme
Court with two Trump appointees, one thing is certain: America even
without legal abortion would be very different from America before
abortion was legal.

The moment Justice Anthony M. Kennedy announced his retirement,
\href{https://www.nytimes3xbfgragh.onion/2018/06/27/us/politics/kennedy-abortion-roe-v-wade.html}{speculation
swirled that Roe v. Wade,} the landmark 1973 ruling that legalized
abortion, would be overturned. Most legal experts say that day is years
away, if it arrives at all. A more likely scenario, they predict, is
that a rightward-shifting court would uphold efforts to restrict
abortion, which would encourage some states to further limit access.

Even then, a full-fledged return to an era of back-alley, coat-hanger
abortions seems improbable. In the decades since Roe was decided, a
burst of scientific innovation has produced more effective, simpler and
safer ways to prevent pregnancies and to stop them after conception ---
advances that have contributed to an abortion rate that has already
plunged by half since the 1980s.

``We're in a new world now,'' said
\href{https://www.northeastern.edu/law/faculty/directory/ahmed.html}{Aziza
Ahmed}, a law professor at Northeastern University who writes about
reproductive rights law. ``The majority of American women are on some
form of contraception. We take it for granted that we can control when
and how we want to reproduce. We see pregnancy as within the realm that
we can control.''

Women have powerful tools at hand: improved intrauterine devices and
hormonal implants that can prevent pregnancy for years at a time;
inexpensive home pregnancy tests able to detect pregnancy very early;
and morning-after pills, some even available over the counter, which can
prevent pregnancy if taken up to five days after unprotected sex.

Medication abortions enable women up to 10 weeks pregnant to take two
pills, the first supervised by a doctor and the second at home, to
terminate a pregnancy without surgery. In 2013, nearly a quarter of
abortions were accomplished with medication, up from 10 percent in 2004.
Even in countries that have banned virtually all abortions, including
some in Latin America, women have managed to get these drugs from
websites and abortion rights organizations that ship them.

And the Affordable Care Act, which has so far defied repeated repeal
attempts, has made birth control available to poor and working-class
women, and also to those with private coverage through employers, with
its requirement that most insurers cover the full cost of contraception.
Apps and telemedicine services are making birth control pills and other
methods available without even a visit to a doctor.

Still, legal changes that make abortion less available would have
profound effects on millions of women, disproportionately affecting
African-Americans, Latinas and women struggling economically. And access
to contraception can be problematic for low-income single women in the
19 states, including Texas and Florida, that have still not expanded
Medicaid coverage for poor single adults.

Despite the new drugs and technologies, nearly half of all pregnancies
in the United States are unintended, a higher rate than in many other
developed countries.

A report this year by a committee of the National Academies of Sciences,
Engineering and Medicine found that three-quarters of women who have
abortions are poor or low-income, and 61 percent are women of color.
Such women bear the brunt of state laws that restrict abortion,
including those requiring multiple appointments or waiting periods or
that limit which providers can perform abortions.

Such hurdles and delays could eventually threaten the consistently high
level of safety in abortion procedures, experts said. ``We found that
more and more regulations on abortion and abortion procedures reduced
the quality of care,'' said the committee's co-chairwoman, Dr. Helene
Gayle, president and chief executive of the
\href{http://cct.org/}{Chicago Community Trust}.

``The people most impacted are the immigrant women already under siege,
low-income women, women of color, transgender and queer women,'' said
Jessica González-Rojas***,*** executive director of the
\href{http://www.latinainstitute.org/}{National Latina Institute for
Reproductive Health}, which works with
\href{https://www.nuestrotexas.org/}{women in the Rio Grande Valley in
Texas}. ``Having a Supreme Court friendly to these restrictive laws
makes it a de facto ban on that kind of health care, abortion and
contraception. Legal access without real access is not access at all.''

In some states, though, the impact of anti-abortion laws can be hard to
measure. A recent report on 2014 data by the Guttmacher Institute, a
research group that supports abortion rights, found that while the
national abortion rate had reached its lowest since the Roe v. Wade
decision, the rate rose modestly in six states --- five of which had
introduced restrictive abortion laws.

The report also found that in states where the number of abortion
clinics had increased, women were not necessarily having more abortions.
New Jersey went from having 24 clinics in 2011 to 41 in 2014, but
abortions declined from about 47,000 to about 44,000 during that time.

Overall, abortion rates have declined almost steadily since 1981, when
the rate was 29.3 per 1,000 women. In 2014, there were an estimated
926,200 abortions --- a rate of 14.6 per 1,000 women, ages 15 to 44.

\hypertarget{back-when-abortion-was-banned}{%
\subsection{Back When Abortion Was
Banned}\label{back-when-abortion-was-banned}}

When abortion was outlawed, initially by state laws in the 19th century,
women still managed to obtain them, sometimes with doctors or midwives,
sometimes with unlicensed abortionists.

``Making abortions illegal didn't stop them ever,'' said
\href{https://as.nyu.edu/history/people.linda-gordon.html}{Linda
Gordon,} a professor of history at New York University.

The
so-called\href{https://www.pbs.org/wgbh/americanexperience/features/pill-anthony-comstocks-chastity-laws/}{Comstock
obscenity laws}, passed from 1873 through the early 1900s, made it
illegal to give, sell, mail or transport any item used for contraception
or abortion.

After that, ``Margaret Sanger built a movement by compromising,'' Dr.
Gordon said. ``They would campaign for legalization of contraception but
not abortion.''

Even after the birth control pill went on the market in 1960,
contraceptives were only provided to women who were married. ``When I
was in college, there was a wedding ring that was shared among young
women when they wanted to see a doctor to get contraception,'' Dr.
Gordon said.

Abortions were often arranged through networks of
\href{https://www.npr.org/2017/05/19/529175737/50-years-ago-a-network-of-clergy-helped-women-seeking-abortion}{clergy}
or \href{https://www.cwluherstory.org/jane-stories-articles/}{women} who
helped people find, travel to and pay for providers.

\href{http://www.ihhcpar.rutgers.edu/about_us/members.asp?v=2\&i=726}{Johanna
Schoen}, a professor at Rutgers University in New Brunswick, N.J., who
specializes in the history of women's reproductive health, said the fate
and rate of abortion will be intertwined with the availability of
contraception, and whether anti-abortion political forces also take aim
at birth control.

Professor Schoen said many European countries have low abortion rates
because birth control and sex education are widely available. ``But in
the United States, the same people who are trying to restrict abortions
have tried to restrict contraception, too.''

\hypertarget{death-by-a-thousand-cuts}{%
\subsection{Death by a Thousand Cuts?}\label{death-by-a-thousand-cuts}}

\href{https://www.law.columbia.edu/faculty/carol-sanger}{Carol Sanger},
who teaches reproductive rights at Columbia Law School, predicts that
the Supreme Court won't overrule Roe v. Wade anytime soon. Developing a
case that would be a direct assault takes years, she said.

``You don't say, `Kennedy's out, Roe's overturned,''' Ms. Sanger said.

The doctrine of precedent, known as stare decisis, ``to stand by things
decided,'' is sturdy. Circumstances must be extraordinary, a law
unworkable, for a court to overrule itself, Ms. Sanger said. ``It stands
for the idea that the substance of our law doesn't blow back and forth
just because we get a new administration.''

Another reason Roe v. Wade may not be struck down? ``You can do a heck
of a lot of damage without overturning it,'' she said.

\includegraphics{https://static01.graylady3jvrrxbe.onion/images/2018/06/29/science/00ABORTION2/merlin_133782650_b2d10ef2-a113-4497-aab0-38db6355d8c9-articleLarge.jpg?quality=75\&auto=webp\&disable=upscale}

In the 45 years since the ruling, anti-abortion activists have largely
focused on lobbying state legislatures for laws that delay or limit
access to abortion, including mandating parental notification by
teenagers, longer waiting periods, and strict requirements for clinics.

Among such initiatives, said Susan Swayze Liebel of
the\href{https://www.sba-list.org/}{Susan B. Anthony List}, an
anti-abortion organization, ``fetal pain'' laws have become a ``top
priority.'' Some 20 states have enacted these laws, which assert that a
fetus can feel pain at 20 weeks after conception ---
\href{https://www.nytimes3xbfgragh.onion/2013/09/17/health/complex-science-at-issue-in-politics-of-fetal-pain.html}{a
claim refuted by most medical experts}.

Roe barred most legal restrictions on abortions until fetuses were
considered able to survive outside the womb, believed then to be 24
weeks after a woman's last menstrual period (about 22 weeks after
conception). These laws seek to shorten abortion deadlines by two weeks,
and while more than 90 percent of abortions occur much earlier, in the
first trimester, fetal pain laws serve as potent political rallying
cries.

Numerous lawsuits about abortion restrictions are currently in state and
\href{https://www.texastribune.org/2018/06/14/lawsuit-challenges-texas-anti-abortion-laws-based-2013-supreme-court-r/}{federal
courts}, primed to wound Roe with a thousand cuts.

Like the fetal pain laws, the intention of one such category is to roll
back viability dates, which goes to the heart of Roe. Iowa just
\href{https://www.npr.org/sections/thetwo-way/2018/05/15/611327234/groups-file-lawsuit-to-block-iowas-new-heartbeat-abortion-law}{enacted
a law} banning most abortions after six weeks, when a fetal heartbeat
can be detected. Mississippi recently passed a ban on abortions after 15
weeks. Abortion providers swiftly sued after the laws were passed.
Legislators who sponsored the laws said they relished such court
clashes, hoping to reach the Supreme Court.

Another category is
\href{https://www.kxan.com/news/us-politics/texas/providers-activists-file-lawsuit-challenging-decades-of-texas-abortion-law/1240137484}{TRAP
laws}: targeted regulation of abortion providers.
\href{https://www.nytimes3xbfgragh.onion/2018/05/29/us/politics/supreme-court-wont-hear-challenge-to-restrictive-arkansas-abortion-law.html}{An
Arkansas law}, for example, requires providers of medication abortions
to have a contract with an obstetrician/gynecologist with hospital
admitting privileges.
\href{https://www.nytimes3xbfgragh.onion/2018/05/24/opinion/supreme-court-abortion.html}{The
Supreme Court} declined to consider an appeal by the plaintiffs, Planned
Parenthood of Arkansas and Eastern Oklahoma, of an appellate court
ruling that upheld the law. The clinics say finding such doctors who
will work with them has been impossible. The case is back in federal
court, where a judge has blocked the law until July 2. If the law takes
effect, Arkansas will likely lose two of its three clinics.

Another cluster of laws aims to limit abortions based on possible
reasons for having them, including sex selection and fetal diagnoses of
conditions such as Down syndrome. Indiana's version, signed by
then-Governor Mike Pence in 2016, was
\href{http://www.abajournal.com/news/article/7th_circuit_blocks_disability_abortion_ban_partial_dissent_labels_abortion}{recently
struck down}by the United States Court of Appeals for the Seventh
Circuit, 2-1.

As abortion court battles unfold, both sides say they will redouble
their political efforts. Noting that the Senate this year did not pass a
20-week abortion ban, Mrs. Liebel of Susan B. Anthony's List said, ``The
focus for our political activity is to go door-to-door in seven states
and flip some key Senate seats to be pro-life.''

And Nancy Northup, president and chief executive of the
\href{https://www.reproductiverights.org/}{Center for Reproductive
Rights}, which focuses on laws concerning reproductive freedom, noted
that some eight states have enshrined the right to abortion, should Roe
fall. ``So we are also looking at advocacy alternatives, such as
friendlier state laws and federal legislation to protect women's
health,'' she said.

Advertisement

\protect\hyperlink{after-bottom}{Continue reading the main story}

\hypertarget{site-index}{%
\subsection{Site Index}\label{site-index}}

\hypertarget{site-information-navigation}{%
\subsection{Site Information
Navigation}\label{site-information-navigation}}

\begin{itemize}
\tightlist
\item
  \href{https://help.nytimes3xbfgragh.onion/hc/en-us/articles/115014792127-Copyright-notice}{©~2020~The
  New York Times Company}
\end{itemize}

\begin{itemize}
\tightlist
\item
  \href{https://www.nytco.com/}{NYTCo}
\item
  \href{https://help.nytimes3xbfgragh.onion/hc/en-us/articles/115015385887-Contact-Us}{Contact
  Us}
\item
  \href{https://www.nytco.com/careers/}{Work with us}
\item
  \href{https://nytmediakit.com/}{Advertise}
\item
  \href{http://www.tbrandstudio.com/}{T Brand Studio}
\item
  \href{https://www.nytimes3xbfgragh.onion/privacy/cookie-policy\#how-do-i-manage-trackers}{Your
  Ad Choices}
\item
  \href{https://www.nytimes3xbfgragh.onion/privacy}{Privacy}
\item
  \href{https://help.nytimes3xbfgragh.onion/hc/en-us/articles/115014893428-Terms-of-service}{Terms
  of Service}
\item
  \href{https://help.nytimes3xbfgragh.onion/hc/en-us/articles/115014893968-Terms-of-sale}{Terms
  of Sale}
\item
  \href{https://spiderbites.nytimes3xbfgragh.onion}{Site Map}
\item
  \href{https://help.nytimes3xbfgragh.onion/hc/en-us}{Help}
\item
  \href{https://www.nytimes3xbfgragh.onion/subscription?campaignId=37WXW}{Subscriptions}
\end{itemize}
