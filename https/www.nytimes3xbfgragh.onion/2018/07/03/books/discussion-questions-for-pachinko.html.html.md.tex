Sections

SEARCH

\protect\hyperlink{site-content}{Skip to
content}\protect\hyperlink{site-index}{Skip to site index}

\href{https://www.nytimes3xbfgragh.onion/section/books}{Books}

\href{https://myaccount.nytimes3xbfgragh.onion/auth/login?response_type=cookie\&client_id=vi}{}

\href{https://www.nytimes3xbfgragh.onion/section/todayspaper}{Today's
Paper}

\href{/section/books}{Books}\textbar{}Discussion Questions for
`Pachinko'

\url{https://nyti.ms/2IOTJl2}

\begin{itemize}
\item
\item
\item
\item
\item
\end{itemize}

Advertisement

\protect\hyperlink{after-top}{Continue reading the main story}

Supported by

\protect\hyperlink{after-sponsor}{Continue reading the main story}

Now Read This

\hypertarget{discussion-questions-for-pachinko}{%
\section{Discussion Questions for
`Pachinko'}\label{discussion-questions-for-pachinko}}

\includegraphics{https://static01.graylady3jvrrxbe.onion/images/2017/11/03/books/MinJinLee1/MinJinLee1-articleLarge.jpg?quality=75\&auto=webp\&disable=upscale}

July 3, 2018

\begin{itemize}
\item
\item
\item
\item
\item
\end{itemize}

\emph{Our July pick for the PBS NewsHour-New York Times book club, ``Now
Read This'' is}
\href{https://www.nytimes3xbfgragh.onion/2017/02/02/books/review/pachinko-min-jin-lee.html?rref=collection\%2Fspotlightcollection\%2Fnow-read-this\&action=click\&contentCollection=books\&region=rank\&module=package\&version=highlights\&contentPlacement=2\&pgtype=collection}{\emph{Min
Jin Lee's historical novel ``Pachinko.''}} \emph{Become a member of the
book club by joining our}
\href{https://www.facebookcorewwwi.onion/groups/NowReadThisBookClub}{\emph{Facebook
group}}\emph{, or by signing up to}
\href{https://pbs.us1.list-manage.com/subscribe?u=8aa1c620fd96b27384151c36e\&id=2fe6581b35}{\emph{our
newsletter}}\emph{. Learn more about the book club}
\href{https://www.pbs.org/newshour/arts/what-is-now-read-this}{\emph{here}}\emph{.}

Below are questions to help guide your discussions as you read the book
over the next month. You can also submit your own questions for Min Jin
Lee on our
\href{https://www.facebookcorewwwi.onion/groups/NowReadThisBookClub}{Facebook
page}, which she will answer on the NewsHour broadcast at the end of the
month.

1. The book's first line reads: ``History has failed us, but no
matter.'' Why do you think Min Jin Lee chose to begin the book this way?

2. The inciting incident in the book comes when Sunja, the daughter of a
boardinghouse owner, is seduced by Hansu, the mysterious and wealthy
stranger. How does that moment reverberate through the generations?

3. What role does shame play in the novel?

4. How does being in exile and being perceived as foreign affect how
Sunja's family members see themselves?

5. Sunja is told early on that ``a woman's life is endless work and
suffering \ldots{} For a woman, the man you marry will determine the
quality of your life completely.'' How do the women in this book have or
not have agency? And how do they struggle to reclaim it?

6. How did the book make you think differently about migration, if at
all?

7. Did you know much about the Japanese occupation of Korea from 1910
through the end of World War II before reading this book? Or about
Korean culture in Japan?

8. ``There was more to being something than just blood,'' Min Jin Lee
writes. How do the characters grapple with this idea throughout the
book?

9. The epigraph for the third section of ``Pachinko,'' from Benedict
Anderson, describes a nation as ``an imagined political community.'' Do
you agree?

10. Which character throughout the four generations do you identify with
most, and why?

11. How did the book make you think differently about what makes a
family?

12. At one point in the novel, Min Jin Lee writes: ``You want to see a
very bad man? Make an ordinary man successful beyond his imagination.
Let's see how good he is when he can do whatever he wants.'' How does
that apply to characters in the book and the larger historical events
happening around them?

13. Did you identify at all with Noa's efforts to ``pass'' as an
identity different than his own --- as Japanese instead of Korean ---
and if not, did it feel relevant to today?

14. ``We cannot help but be interested in the stories of people that
history pushes aside so thoughtlessly,'' Min Jin Lee writes. Do you
think ``Pachinko'' is an effort to reclaim those stories?

15. After finishing the book, why do you think Min Jin Lee chose the
title ``Pachinko,'' from the game common in Japan? How does she compare
the game of Pachinko to the game of life?

Advertisement

\protect\hyperlink{after-bottom}{Continue reading the main story}

\hypertarget{site-index}{%
\subsection{Site Index}\label{site-index}}

\hypertarget{site-information-navigation}{%
\subsection{Site Information
Navigation}\label{site-information-navigation}}

\begin{itemize}
\tightlist
\item
  \href{https://help.nytimes3xbfgragh.onion/hc/en-us/articles/115014792127-Copyright-notice}{©~2020~The
  New York Times Company}
\end{itemize}

\begin{itemize}
\tightlist
\item
  \href{https://www.nytco.com/}{NYTCo}
\item
  \href{https://help.nytimes3xbfgragh.onion/hc/en-us/articles/115015385887-Contact-Us}{Contact
  Us}
\item
  \href{https://www.nytco.com/careers/}{Work with us}
\item
  \href{https://nytmediakit.com/}{Advertise}
\item
  \href{http://www.tbrandstudio.com/}{T Brand Studio}
\item
  \href{https://www.nytimes3xbfgragh.onion/privacy/cookie-policy\#how-do-i-manage-trackers}{Your
  Ad Choices}
\item
  \href{https://www.nytimes3xbfgragh.onion/privacy}{Privacy}
\item
  \href{https://help.nytimes3xbfgragh.onion/hc/en-us/articles/115014893428-Terms-of-service}{Terms
  of Service}
\item
  \href{https://help.nytimes3xbfgragh.onion/hc/en-us/articles/115014893968-Terms-of-sale}{Terms
  of Sale}
\item
  \href{https://spiderbites.nytimes3xbfgragh.onion}{Site Map}
\item
  \href{https://help.nytimes3xbfgragh.onion/hc/en-us}{Help}
\item
  \href{https://www.nytimes3xbfgragh.onion/subscription?campaignId=37WXW}{Subscriptions}
\end{itemize}
