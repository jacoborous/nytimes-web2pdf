Sections

SEARCH

\protect\hyperlink{site-content}{Skip to
content}\protect\hyperlink{site-index}{Skip to site index}

\href{https://www.nytimes3xbfgragh.onion/section/technology/personaltech}{Personal
Tech}

\href{https://myaccount.nytimes3xbfgragh.onion/auth/login?response_type=cookie\&client_id=vi}{}

\href{https://www.nytimes3xbfgragh.onion/section/todayspaper}{Today's
Paper}

\href{/section/technology/personaltech}{Personal Tech}\textbar{}I
Downloaded the Information That Facebook Has on Me. Yikes.

\begin{itemize}
\item
\item
\item
\item
\item
\item
\end{itemize}

Advertisement

\protect\hyperlink{after-top}{Continue reading the main story}

Supported by

\protect\hyperlink{after-sponsor}{Continue reading the main story}

\href{/column/tech-fix}{Tech Fix}

\hypertarget{i-downloaded-the-information-that-facebook-has-on-me-yikes}{%
\section{I Downloaded the Information That Facebook Has on Me.
Yikes.}\label{i-downloaded-the-information-that-facebook-has-on-me-yikes}}

\includegraphics{https://static01.graylady3jvrrxbe.onion/images/2018/04/11/autossell/user/user-videoSixteenByNineJumbo1600.jpg}

By \href{https://www.nytimes3xbfgragh.onion/by/brian-x-chen}{Brian X.
Chen}

\begin{itemize}
\item
  April 11, 2018
\item
  \begin{itemize}
  \item
  \item
  \item
  \item
  \item
  \item
  \end{itemize}
\end{itemize}

\href{https://www.nytimes3xbfgragh.onion/es/2018/04/12/facebook-index-informacion/}{Leer
en español}

When I downloaded a copy of my
\href{https://www.nytimes3xbfgragh.onion/2019/03/13/technology/facebook-data-subpoenas.html}{Facebook
data} last week, I didn't expect to see much. My profile is sparse, I
rarely post anything on the site, and I seldom click on ads. (I'm what
some call
\href{https://www.urbandictionary.com/define.php?term=Facebook\%20lurker}{a
Facebook ``lurker.''})

But when I opened my file, it was like opening Pandora's box.

With a few clicks, I learned that about 500 advertisers --- many that I
had never heard of, like Bad Dad, a motorcycle parts store, and Space
Jesus, an electronica band --- had my contact information, which could
include my email address, phone number and full name. Facebook also had
my entire phone book, including the number to ring my apartment buzzer.
The social network had even kept a permanent record of the roughly 100
people I had deleted from my friends list over the last 14 years,
including my exes.

There was so much that Facebook knew about me --- more than I wanted to
know. But after looking at the totality of what the Silicon Valley
company had obtained about yours truly, I decided to try to better
understand how and why my data was collected and stored. I also sought
to find out how much of my data could be removed.

How Facebook collects and treats personal information was central this
week when Mark Zuckerberg, the company's chief executive,
\href{https://www.nytimes3xbfgragh.onion/2018/04/10/us/politics/mark-zuckerberg-testimony.html}{answered
questions in Congress} about data privacy and his responsibilities to
users. During his testimony, Mr. Zuckerberg repeatedly said Facebook has
a \href{https://www.facebookcorewwwi.onion/help/131112897028467}{tool
for downloading your data} that ``allows people to see and take out all
the information they've put into Facebook.'' (Those who want to download
their own Facebook data can use
\href{https://www.facebookcorewwwi.onion/help/131112897028467}{this
link}.)

But that's an overstatement. Most basic information, like my birthday,
could not be deleted. More important, the pieces of data that I found
objectionable, like the record of people I had unfriended, could not be
removed from Facebook, either.

``They don't delete anything, and that's a general policy,'' said
Gabriel Weinberg, the founder of
\href{https://www.nytimes3xbfgragh.onion/2014/04/03/technology/personaltech/sweeping-away-a-search-history.html}{DuckDuckGo},
which offers internet privacy tools. He added that data was kept around
to eventually help brands serve targeted ads.

Beth Gautier, a Facebook spokeswoman, put it this way: ``When you delete
something, we remove it so it's not visible or accessible on Facebook.''
She added: ``You can also delete your account whenever you want. It may
take up to 90 days to delete all backups of data on our servers.''

Digging through your Facebook files is an exercise I highly recommend if
you care about how your personal information is stored and used. Here's
what I learned.

\href{https://www.nytimes3xbfgragh.onion/interactive/2018/04/11/technology/facebook-sells-ads-life-details.html}{}

\includegraphics{https://static01.graylady3jvrrxbe.onion/images/2018/04/11/us/facebook-sells-ads-life-details-promo-1523471252800/facebook-sells-ads-life-details-promo-1523471252800-articleLarge.png}

\hypertarget{how-facebook-lets-brands-and-politicians-target-you}{%
\subsection{How Facebook Lets Brands and Politicians Target
You}\label{how-facebook-lets-brands-and-politicians-target-you}}

A history of the steps the company took to become an advertising giant.

\hypertarget{facebook-retains-more-data-than-we-think}{%
\subsection{Facebook Retains More Data Than We
Think}\label{facebook-retains-more-data-than-we-think}}

When you download a copy of your Facebook data, you will see a folder
containing multiple subfolders and files. The most important one is the
``index'' file, which is essentially a raw data set of your Facebook
account, where you can click through your profile, friends list,
timeline and messages, among other features.

One surprising part of my index file was a section called Contact Info.
This contained the 764 names and phone numbers of everyone in my
iPhone's address book. Upon closer inspection, it turned out that
Facebook had stored my entire phone book because I had uploaded it when
setting up
\href{https://www.nytimes3xbfgragh.onion/2018/03/27/technology/personaltech/facebook-messenger-uninstall.html}{Facebook's
messaging app, Messenger}.

This was unsettling. I had hoped Messenger would use my contacts list to
find others who were also using the app so that I could connect with
them easily --- and hold on to the relevant contact information only for
the people who were on Messenger. Yet Facebook kept the entire list,
including the phone numbers for my car mechanic, my apartment door
buzzer and a pizzeria.

This felt unnecessary, though Facebook holds on to your phone book
partly to keep it synchronized with your contacts list on Messenger and
to help find people who newly sign up for the messaging service. I opted
to turn off synchronizing and
\href{https://www.facebookcorewwwi.onion/invite_history.php}{deleted all
my phone book entries}.

My Facebook data also revealed how little the social network forgets.
For instance, in addition to recording the exact date I signed up for
Facebook in 2004, there was a record of when I deactivated Facebook in
October 2010, only to reactivate it four days later --- something I
barely remember doing.

Facebook also kept a history of each time I opened Facebook over the
last two years, including which device and web browser I used. On some
days, it even logged my locations, like when I was at a hospital two
years ago or when I visited Tokyo last year.

Facebook keeps a log of this data as a security measure to flag
suspicious logins from unknown devices or locations, similar to how
banks send a fraud alert when your credit card number is used in a
suspicious location. This practice seemed reasonable, so I didn't try to
purge this information.

But what bothered me was the data that I had explicitly deleted but that
lingered in plain sight. On my friends list, Facebook had a record of
``Removed Friends,'' a dossier of the 112 people I had removed along
with the date I clicked the ``Unfriend'' button. Why should Facebook
remember the people I've cut off from my life?

Facebook's explanation was dissatisfying. The company said it might use
my list of deleted friends so that those people did not appear in my
feed with the feature ``On This Day,'' which resurfaces memories from
years past to help people reminisce. I'd rather have the option to
delete the list of deleted friends for good.

\includegraphics{https://static01.graylady3jvrrxbe.onion/images/2018/04/12/business/12TECHFIX-1web/12TECHFIX-1web-articleLarge.jpg?quality=75\&auto=webp\&disable=upscale}

\hypertarget{the-ad-industry-has-eyes-everywhere}{%
\subsection{The Ad Industry Has Eyes
Everywhere}\label{the-ad-industry-has-eyes-everywhere}}

What Facebook retained about me isn't remotely as creepy as the sheer
number of advertisers that have my information in their databases. I
found this out when I clicked on the Ads section in my Facebook file,
which loaded a history of the dozen ads I had clicked on while browsing
the social network.

Lower down, there was a section titled ``Advertisers with your contact
info,'' followed by a list of roughly 500 brands, the overwhelming
majority of which I had never interacted with. Some brands sounded
obscure and sketchy --- one was called ``Microphone Check,'' which
turned out to be a radio show. Other brands were more familiar, like
Victoria's Secret Pink, Good Eggs or AARP.

Facebook said unfamiliar advertisers might appear on the list because
they might have obtained my contact information from elsewhere, compiled
it into a list of people they wanted to target and uploaded that list
into Facebook. Brands can upload their customer lists into a tool called
Custom Audiences, which helps them find those same people's Facebook
profiles to serve them ads.

Brands can obtain your information in many different ways. Those
include:

■ Buying information from a
\href{https://www.nytimes3xbfgragh.onion/2012/06/17/technology/acxiom-the-quiet-giant-of-consumer-database-marketing.html}{data
provider like Acxiom}, which has amassed one of the world's largest
commercial databases on consumers. Brands can buy different types of
customer data sets from a provider, like contact information for people
who belong to a certain demographic, and take that information to
Facebook to serve targeted ads, said Michael Priem, chief executive of
Modern Impact, an advertising firm in Minneapolis.

Last month, Facebook announced that it
\href{https://www.nytimes3xbfgragh.onion/2018/03/28/technology/facebook-privacy-security-settings.html?rref=collection\%2Fbyline\%2Fnatasha-singer\&action=click\&contentCollection=undefined\&region=stream\&module=stream_unit\&version=latest\&contentPlacement=5\&pgtype=collection}{was
limiting its practice} of allowing advertisers to target ads using
information from third-party data brokers like Acxiom.

■ Using
\href{https://www.nytimes3xbfgragh.onion/2016/02/18/technology/personaltech/free-tools-to-keep-those-creepy-online-ads-from-watching-you.html}{tracking
technologies} like web cookies and invisible pixels that load in your
web browser to collect information about your browsing activities. There
are many different trackers on the web, and Facebook offers 10 different
trackers to help brands harvest your information, according to Ghostery,
which offers privacy tools that block ads and trackers. The advertisers
can take some pieces of data that they have collected with trackers and
upload them into the Custom Audiences tool to serve ads to you on
Facebook.

■ Getting your information in simpler ways, too. Someone you shared
information with could share it with another entity. Your credit card
loyalty program, for example, could share your information with a hotel
chain, and that hotel chain could serve you ads on Facebook.

The upshot? Even a Facebook lurker, like myself, who has barely clicked
on any digital ads can have personal information exposed to an enormous
number of advertisers. This was not entirely surprising, but seeing the
list of unfamiliar brands with my contact information in my Facebook
file was a dose of reality.

I tried to contact some of these advertisers, like Very Important
Puppets, a toymaker, to ask them about what they did with my data. They
did not respond.

\hypertarget{what-about-google}{%
\subsection{What About Google?}\label{what-about-google}}

Let's be clear: Facebook is just the tip of the iceberg when it comes to
what information tech companies have collected on me.

Knowing this, I also downloaded copies of my Google data with a tool
called \href{https://takeout.google.com/settings/takeout}{Google
Takeout}. The data sets were exponentially larger than my Facebook data.
For my personal email account alone, Google's archive of my data
measured eight gigabytes, enough to hold about 2,000 hours of music. By
comparison, my Facebook data was about 650 megabytes, the equivalent of
about 160 hours of music.

Here was the biggest surprise in what Google collected on me: In a
folder labeled Ads, Google kept a history of many news articles I had
read, like a Newsweek story about
\href{http://www.newsweek.com/apple-employees-call-911-walk-glass-walls-829998}{Apple
employees walking into glass walls} and a
\href{https://www.nytimes3xbfgragh.onion/2018/02/14/technology/personaltech/valentines-tech-couples-together.html}{New
York Times story} about the editor of our Modern Love column. I didn't
click on ads for either of these stories, but the search giant logged
them because the sites had loaded ads served by Google.

In another folder, labeled Android, Google had a record of apps I had
opened on an Android phone since 2015, along with the date and time.
This felt like an extraordinary level of detail.

Google did not immediately respond to a request for comment.

On a brighter note, I downloaded an archive of my LinkedIn data. The
data set was less than half a megabyte and contained exactly what I had
expected: spreadsheets of my LinkedIn contacts and information I had
added to my profile.

Yet that offered little solace. Be warned: Once you see the vast amount
of data that has been collected about you, you won't be able to unsee
it.

Advertisement

\protect\hyperlink{after-bottom}{Continue reading the main story}

\hypertarget{site-index}{%
\subsection{Site Index}\label{site-index}}

\hypertarget{site-information-navigation}{%
\subsection{Site Information
Navigation}\label{site-information-navigation}}

\begin{itemize}
\tightlist
\item
  \href{https://help.nytimes3xbfgragh.onion/hc/en-us/articles/115014792127-Copyright-notice}{©~2020~The
  New York Times Company}
\end{itemize}

\begin{itemize}
\tightlist
\item
  \href{https://www.nytco.com/}{NYTCo}
\item
  \href{https://help.nytimes3xbfgragh.onion/hc/en-us/articles/115015385887-Contact-Us}{Contact
  Us}
\item
  \href{https://www.nytco.com/careers/}{Work with us}
\item
  \href{https://nytmediakit.com/}{Advertise}
\item
  \href{http://www.tbrandstudio.com/}{T Brand Studio}
\item
  \href{https://www.nytimes3xbfgragh.onion/privacy/cookie-policy\#how-do-i-manage-trackers}{Your
  Ad Choices}
\item
  \href{https://www.nytimes3xbfgragh.onion/privacy}{Privacy}
\item
  \href{https://help.nytimes3xbfgragh.onion/hc/en-us/articles/115014893428-Terms-of-service}{Terms
  of Service}
\item
  \href{https://help.nytimes3xbfgragh.onion/hc/en-us/articles/115014893968-Terms-of-sale}{Terms
  of Sale}
\item
  \href{https://spiderbites.nytimes3xbfgragh.onion}{Site Map}
\item
  \href{https://help.nytimes3xbfgragh.onion/hc/en-us}{Help}
\item
  \href{https://www.nytimes3xbfgragh.onion/subscription?campaignId=37WXW}{Subscriptions}
\end{itemize}
