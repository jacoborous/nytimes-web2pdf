\href{/section/food}{Food}\textbar{}The New Noma: Frequently Asked
Questions

\url{https://nyti.ms/2Kcz10n}

\begin{itemize}
\item
\item
\item
\item
\item
\item
\end{itemize}

\includegraphics{https://static01.graylady3jvrrxbe.onion/images/2018/04/25/dining/25noma/25noma-articleLarge.jpg?quality=75\&auto=webp\&disable=upscale}

Sections

\protect\hyperlink{site-content}{Skip to
content}\protect\hyperlink{site-index}{Skip to site index}

Critic's Notebook

\hypertarget{the-new-noma-frequently-asked-questions}{%
\section{The New Noma: Frequently Asked
Questions}\label{the-new-noma-frequently-asked-questions}}

What's it like to eat at the second incarnation of the rule-defying
Copenhagen restaurant? Our critic explains it all for you.

At the new version of Noma, in Copenhagen, sea-snail broth is meant to
be sipped from a shell with pickled flowers around the
lip.Credit...Ditte Isager for The New York Times

Supported by

\protect\hyperlink{after-sponsor}{Continue reading the main story}

By \href{http://www.nytimes3xbfgragh.onion/by/pete-wells}{Pete Wells}

\begin{itemize}
\item
  April 24, 2018
\item
  \begin{itemize}
  \item
  \item
  \item
  \item
  \item
  \item
  \end{itemize}
\end{itemize}

\textbf{What is the ``new Noma''?}

It's what the tasting-menu set calls the compound in Copenhagen where
the chef René Redzepi
\href{https://www.nytimes3xbfgragh.onion/2015/09/16/dining/noma-rene-redzepi-urban-farm.html}{recently
transplanted} the restaurant that invented New Nordic cuisine. The
original Noma operated in a 1765 warehouse for dried fish and whale oil
on a city pier from 2003 until February 2017. The
\href{http://noma.dk/}{new Noma} served its first customers on Feb. 16,
2018.

\includegraphics{https://static01.graylady3jvrrxbe.onion/images/2018/04/25/dining/25Noma13/25Noma13-articleLarge.jpg?quality=75\&auto=webp\&disable=upscale}

\textbf{Will I be able to get a reservation?}

You might. Less than a day after its online ticketing system opened, in
November, journalists, cooks, locals and destination-restaurant pilgrims
from around the world had booked every seat through the end of April.
This critic, not very quick on the draw, failed to get a table, but a
former colleague who has written about Noma did, and offered a seat at
his table. Sporadically, the restaurant seems to find additional space
and puts new tickets up for sale. More usefully, the second batch of
tickets, running through the end of September, has not quite sold out
yet. A table for eight, in particular, is up for grabs on many dates.

Image

The restaurant is one of several buildings, including a greenhouse, on
the bank of a pond in the city's Christiania section.Credit...Signe
Birck for The New York Times

\textbf{Where is it?}

The mailing address is Refshalevej 96, 1432 Copenhagen K, Denmark. More
generally, the new Noma is in the part of the city called Christiania,
where fortified walls were built on landfill in the 1600s to defend the
city. The area was neglected until it was taken over by residents in
1971, first as a playground and then as the base of Freetown, an
experimental anarchist community that proclaimed itself self-governing
and self-sufficient. ``It is so far the biggest opportunity to build a
society from scratch,'' one of the founders wrote. ``For those who feel
the beating of the pioneer heart there can be no doubt as to the purpose
of Christiania.''

\textbf{Does the pioneer heart still beat in Christiania?}

Yes, despite continuing conflict over hashish vendors on Pusher Street.
The Noma complex, parts of which were still behind plywood early this
month, is an island within the island of Christiania. It is cordoned off
by a chain-link fence through which you can see the closest neighbor's
home, a makeshift yurt in the woods. At the edge of the property is a
pond where swans, mallards and coots paddle around. It is as if Mr.
Redzepi had located Copenhagen's back door and walked through it,
carrying the restaurant with him.

Image

The architect Bjarke Ingels conceived the kitchen and other areas as
huts connected by glass ceilings.Credit...Ditte Isager for The New York
Times

\textbf{Does Noma look like a restaurant?}

Not as the word is commonly understood, no. With its rooftop garden, its
cluster of outbuildings and a main structure where work is carried on in
a cluster of ``huts'' connected by glass-roofed corridors, it more
closely resembles the campus of a tech firm or a small progressive
college.

\textbf{Who designed it?}

The Danish architecture firm \href{http://www.big.dk/}{BIG}, led by
Bjarke Ingels. (BIG is also responsible for the trash-burning power
plant that you can see from the dining room, and that will have a
2,000-foot-long ski run on its roof when it is finished.) According to
Mr. Ingels, the layout was inspired by the clustered structures on a
traditional Danish farmstead. Mr. Redzepi, who as the son of a Muslim
father of Albanian descent and a Christian mother from Denmark is
something of an outsider in Danish culture, has told people that the
central building where the kitchen and dining room sit is derived from
Viking longhouses.

Image

The menu's oceanic theme is foreshadowed by sea creatures preserved in
jars.Credit...Ditte Isager for The New York Times

\textbf{If I get a reservation, will eating at Noma make me feel like a
Viking?}

Probably not, unless you precede your meal with a score on Pusher
Street. Viking longhouses were windowless, underfurnished, smoky and
probably smelly, given that farm animals slept in them. Noma's dining
room, by contrast, is meticulously carpentered together, from the peaked
ceiling to the bare floor, out of sanded oak and Douglas fir. It has
spindle-legged, custom-built Danish-modern tables and chairs beside
glass walls with a view of the pond and woods. The kitchen, which you
can see from the dining room, is well ventilated. From time to time you
might hear chords of whatever music the cooks are listening to. The
animals living inside the complex --- king crabs and mollusks in shades
of pink and aquamarine generally seen only on the residents of
\href{https://www.youtube.com/watch?v=RjUVOBory70}{Bikini Bottom} ---
are kept in tanks with no noticeable odor. The 42 or so diners are
typically well groomed and carefully, if not formally, dressed. Although
they are not given to marauding and pillaging, they are not particularly
solemn, either. Noma is not a place of worship.

\textbf{What will happen when I arrive?}

You may look for a sign. You will not see one, but if you see
greenhouses and a long concrete bunker built into a 17th-century earthen
rampart, you are in the right place. Somebody will greet you at the
gate, perhaps Ali Sonko, who immigrated to Denmark from Gambia, started
at Noma as a dishwasher and is now one of Mr. Redzepi's partners. If you
are a repeat customer, he may hug you before leading you to the
restaurant. When you enter, most of the kitchen and dining room staff,
including Mr. Redzepi, will be standing inside the door. They will act
as if they have been particularly looking forward to your arrival and
had all the time in the world to greet you. It is a little like meeting
the von Trapp children. Once this ritual is over, they return to their
posts and you are brought to your table. A glass of sparkling wine will
probably materialize quickly.

Image

Noma seats about 42 diners in comfortable, understated
style.Credit...Ditte Isager for The New York Times

\textbf{How are meals at the new Noma different?}

The menus are more tightly focused. The old Noma restricted itself to
ingredients that grew in the Nordic countries. The new one narrows the
scope even more, with three major themes a year that stick to what Mr.
Redzepi thinks is best at the time. The menu in late spring and summer
will revolve around plants, although it will not necessarily be
vegetarian. ``We will have things we think belong, like ants and
snails,'' Mr. Redzepi said during one of his frequent trips to the
dining room. ``They're there, in the garden.'' Foragers and hunters will
supply the late-fall and early-winter kitchen with wild mushrooms, nuts,
game birds, deer, moose, bear. Every course in the current menu, in
effect until late spring, contains something from the ocean.

\textbf{How many courses are in the tasting menu?}

There are around 20 dishes, a few of which come at the same time.
Matchstick strips of the firmer bits of a mahogany clam, decorated with
seaweed fronds and little salt-preserved unripe gooseberries and black
currant buds, arrive in its shell in a bath of mussel juice and oil
pressed from black-currant wood at the same moment as a bay scallop and
its roe, pulled from Norwegian seas by a diver named Roderick Sloan, who
must be immune to hypothermia.

\textbf{Is that good?}

Unless Mr. Sloan is a friend of yours, the scallop is likely to be the
sweetest you have ever tasted. The clam is briny and tart and chewy, and
affects you like a splash of Norwegian water in the face.

\textbf{Will I see a menu before I eat?}

In a manner of speaking, yes. You won't see a printed menu until the
meal is over, but hanging to the right of the entrance is a framed
beachcomber's collage of shells, seaweeds, starfish, sea horses and
other saltwater creatures. Nearly every course is represented somewhere.

Image

In lieu of a paper menu by the door, Noma displays a beachcomber's
collage representing nearly every course, and some decorative creatures
that are not served.

Credit...Ditte Isager for The New York Times

\textbf{Is Noma serving sea horse and starfish?}

Both are decorative items in the collage, like the polished rocks,
although the test kitchen gave ground starfish the old college try. ``We
did not enjoy it,'' Mr. Redzepi said, emphatically. Instead cooks paint
a starfish on the plate with a pearlescent sauce of egg yolks and
pumpkin-seed oil and cover it with wild Danish trout roe. Studded with
tiny flecks of dried fermented plum, it is wonderful to eat, although
you could see trompe l'oeil plating as a small betrayal of the
all-natural ethos that animates most of Noma's cooking.

\textbf{What is the black-and-white shell, the size of a soapdish, in
the bottom-left corner of the collage?}

A horse mussel. Horse mussels are almost never eaten, not even by
horses. After throwing away the nondelicious parts, which Mr. Redzepi
estimates at roughly two-thirds to three-quarters of the animal, Noma's
cooks stew the rest with chanterelles they preserved in oil last year.
It tastes meaty, a bit like lamb, or at least more like lamb than
anything else on the menu, and its flavor is considerably perked up by
some tart foraged mirabelle plums, salted and dried before the winter.

Image

Many of the shellfish are procured by a diver from waters near the
Arctic Circle.Credit...Ditte Isager for The New York Times

\textbf{Should I start eating horse mussels?}

If you can get them the way Noma cooks them, sure.

\textbf{Are regular blue mussels not on the menu?}

They are, and they must be among the best mussels on earth. Four or five
of their stout bellies have been joined together after being separated
from the stringier bits, which are ground up in a smoked butter sauce
that is insanely good.

Image

A ring of pink lumpfish roe is visible beneath a
jellyfish.Credit...Ditte Isager for The New York Times

\textbf{Does the all-seafood menu get monotonous?}

Not for a minute. At around two hours, the meal skates along briskly and
pulls a greater variety of flavors out of the Nordic waters than another
restaurant would get by importing seafood from around the world.

\textbf{How do all these courses get to my table?}

Mr. Redzepi will bring one or two, stopping to chat about, say, snails
and starfish. Others will be brought by cooks. You will probably be
served by Lars Korby, who helps herd the wine collection; James
Spreadbury, a kindly Australian who manages the restaurant; Mette
Soberg, who as head of Noma's research and development department works
out the first drafts of many dishes. There is no discernible hierarchy
in the service staff, although Mr. Redzepi is obviously the boss.

Image

The kitchen layout with free-standing work stations was first tried out
at a pop-up in Mexico.Credit...Ditte Isager for The New York Times

\textbf{What happened to the famous fermentation laboratory?}

The laboratory and its director, David Zilber, used to work out of a
shipping container outside the old restaurant. Now both have moved
indoors, and have been given a walk-in refrigerator stacked with various
garums and ``peaso,'' a relative of miso made with yellow peas.
Something fermented turns up in every course on the current menu,
including the esoteric juices --- saffron and Arctic thyme is a typical
example --- that can be had in place of a wine pairing. The restaurant's
work with fermentation is so extensive that it fills a book,
\href{https://www.workman.com/products/the-noma-guide-to-fermentation-foundations-of-flavor}{``The
Noma Guide to Fermentation,''} by Mr. Zilber and Mr. Redzepi, to be
published this year.

\textbf{I neither know nor care how fermentation works. Will I still
appreciate the food?}

If you can enjoy wine or cheese without understanding their metabolic
underpinnings, you will be fine. But when you are eating something at
Noma that tastes like much more than the sum of its parts, when you
realize that few of the restaurant's many imitators load as much depth
and complexity into their cooking, when you start to lose your bearings
and can't quite figure out what is happening, it can be helpful to
recall that just out of sight is an entire room full of special sauce.

Image

Every course contains something from the ocean; one of the desserts is a
plankton cake.Credit...Ditte Isager for The New York Times

\textbf{Does fermentation make everything at Noma taste great?}

You may reach an answer in the negative if you drink the green liquid of
plankton and raw pumpkin-seed milk thinned with yogurt whey. And then
you may find yourself reflecting that to explore the boundaries of
deliciousness it is sometimes necessary to go beyond them.

\textbf{Will I avoid plankton if I get the wine pairing?}

Not necessarily. One current dessert is a plankton mousse under a
toasted-milk crumble. It does not taste weird at all. Neither do the two
desserts made with kelp, but none of them is as likable over the short
term (and probably the long term, too) as the icy cloudberry soup with
snowdrifts of frozen yogurt and tiny candied pine cones, as chewy as
jelly beans.

\textbf{All these small portions look as if they'd been put on the plate
by a team of synchronized hummingbirds. Will I get enough to eat?}

Just before the finale of desserts, when you may be second-guessing Mr.
Redzepi's decision not to serve any bread with this menu, something
close to perfect happens. It is a dish called ``head of the cod.'' It is
not an entire head, but the meatiest chunks on sharp blades of bone that
have been as carefully trimmed as any Frenched rack of lamb. The fish
has been brushed with seaweed and mushroom glazes reminiscent of soy and
miso and then grilled, something like the way yellowtail collar is
cooked in an izakaya. There are four cuts and three garnishes, so you
have the option of, say, dipping the cheek in horseradish oil and
dredging the tongue in a tart pesto made from ground Danish wood ants.
The fish is soft, extravagantly rich, and by the time you have found the
last shred of flesh you are ready for something sweet.

Image

The final course before dessert is sections of cod head with grilled
ramson leaves.Credit...Ditte Isager for The New York Times

\textbf{What is Noma all about?}

The cod-head dish sums it up. Noma's strategy in all things is to get
rid of any received notions of luxury in restaurants and replace them
with something seen as more humble (pottery spun on a wheel instead of
Bernardaud porcelain), eccentric (natural wines rather than blue-chip
Bordeaux), overlooked (horse mussels instead of no horse mussels), or
undervalued (cod heads for lobster). Mr. Redzepi and his colleagues have
rebuilt the template of high-end, destination dining piece by piece with
stuff that has been thought about, considered and chosen for a reason.
Sometimes the cook or dining-room worker bringing the food to your table
tells you the reason, but even when you're not told, you can still sense
that everything has a purpose. That's what Noma is about as a business
serving food. As an aesthetic project, it is also about questioning
received hierarchies of value. The stray plant in your backyard or
window box is a weed only if you pull it out. Let it grow and it could
be a wildflower, or a tasty addition to tonight's salad.

\textbf{How much does it cost?}

About \$375 for lunch or dinner without drinks, paid in advance when you
reserve on the restaurant's website. Wine pairings cost about \$166 and
the slate of juices runs about \$133. Each night four students, randomly
chosen from a waiting list, are seated and charged about \$165 a person,
including wine or juice pairings.

\textbf{Isn't \$375 a lot of money?}

Yes. Noma is redefining luxury, not abolishing it.

\href{https://www.facebookcorewwwi.onion/nytfood/}{\emph{Follow NYT Food
on Facebook}}\emph{,}
\href{https://instagram.com/nytfood}{\emph{Instagram}}\emph{,}
\href{https://twitter.com/nytfood}{\emph{Twitter}} \emph{and}
\href{https://www.pinterest.com/nytfood/}{\emph{Pinterest}}\emph{.}
\href{https://www.nytimes3xbfgragh.onion/newsletters/cooking}{\emph{Get
regular updates from NYT Cooking, with recipe suggestions, cooking tips
and shopping advice}}\emph{.}

Advertisement

\protect\hyperlink{after-bottom}{Continue reading the main story}

\hypertarget{site-index}{%
\subsection{Site Index}\label{site-index}}

\hypertarget{site-information-navigation}{%
\subsection{Site Information
Navigation}\label{site-information-navigation}}

\begin{itemize}
\tightlist
\item
  \href{https://help.nytimes3xbfgragh.onion/hc/en-us/articles/115014792127-Copyright-notice}{©~2020~The
  New York Times Company}
\end{itemize}

\begin{itemize}
\tightlist
\item
  \href{https://www.nytco.com/}{NYTCo}
\item
  \href{https://help.nytimes3xbfgragh.onion/hc/en-us/articles/115015385887-Contact-Us}{Contact
  Us}
\item
  \href{https://www.nytco.com/careers/}{Work with us}
\item
  \href{https://nytmediakit.com/}{Advertise}
\item
  \href{http://www.tbrandstudio.com/}{T Brand Studio}
\item
  \href{https://www.nytimes3xbfgragh.onion/privacy/cookie-policy\#how-do-i-manage-trackers}{Your
  Ad Choices}
\item
  \href{https://www.nytimes3xbfgragh.onion/privacy}{Privacy}
\item
  \href{https://help.nytimes3xbfgragh.onion/hc/en-us/articles/115014893428-Terms-of-service}{Terms
  of Service}
\item
  \href{https://help.nytimes3xbfgragh.onion/hc/en-us/articles/115014893968-Terms-of-sale}{Terms
  of Sale}
\item
  \href{https://spiderbites.nytimes3xbfgragh.onion}{Site Map}
\item
  \href{https://help.nytimes3xbfgragh.onion/hc/en-us}{Help}
\item
  \href{https://www.nytimes3xbfgragh.onion/subscription?campaignId=37WXW}{Subscriptions}
\end{itemize}
