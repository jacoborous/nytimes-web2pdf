Sections

SEARCH

\protect\hyperlink{site-content}{Skip to
content}\protect\hyperlink{site-index}{Skip to site index}

\href{https://www.nytimes3xbfgragh.onion/section/technology}{Technology}

\href{https://myaccount.nytimes3xbfgragh.onion/auth/login?response_type=cookie\&client_id=vi}{}

\href{https://www.nytimes3xbfgragh.onion/section/todayspaper}{Today's
Paper}

\href{/section/technology}{Technology}\textbar{}Inside Uber's
Aggressive, Unrestrained Workplace Culture

\url{https://nyti.ms/2lx4TRC}

\begin{itemize}
\item
\item
\item
\item
\item
\item
\end{itemize}

Advertisement

\protect\hyperlink{after-top}{Continue reading the main story}

Supported by

\protect\hyperlink{after-sponsor}{Continue reading the main story}

\hypertarget{inside-ubers-aggressive-unrestrained-workplace-culture}{%
\section{Inside Uber's Aggressive, Unrestrained Workplace
Culture}\label{inside-ubers-aggressive-unrestrained-workplace-culture}}

\includegraphics{https://static01.graylady3jvrrxbe.onion/images/2017/02/22/business/23UBER1/23UBER1-articleInline.jpg?quality=75\&auto=webp\&disable=upscale}

By \href{http://www.nytimes3xbfgragh.onion/by/mike-isaac}{Mike Isaac}

\begin{itemize}
\item
  Feb. 22, 2017
\item
  \begin{itemize}
  \item
  \item
  \item
  \item
  \item
  \item
  \end{itemize}
\end{itemize}

\href{https://www.nytimes3xbfgragh.onion/es/2017/02/27/uber-y-su-cultura-de-trabajo-agresiva/}{Leer
en español}

SAN FRANCISCO --- When new employees join Uber, they are asked to
subscribe to 14 core company values, including making bold bets, being
``obsessed'' with the customer, and ``always be hustlin'.'' The
ride-hailing service particularly emphasizes ``meritocracy,'' the idea
that the best and brightest will rise to the top based on their efforts,
even if it means stepping on toes to get there.

Those values have helped propel Uber to one of Silicon Valley's biggest
success stories. The company is valued at close to \$70 billion
\href{https://www.nytimes3xbfgragh.onion/2016/06/02/technology/uber-investment-saudi-arabia.html}{by
private investors} and now operates in more than 70 countries.

Yet the focus on pushing for the best result has also fueled what
current and former Uber employees describe as a Hobbesian environment at
the company, in which workers are sometimes pitted against one another
and where a blind eye is turned to infractions from top performers.

Interviews with more than 30 current and former Uber employees, as well
as reviews of internal emails, chat logs and tape-recorded meetings,
paint a picture of an often unrestrained workplace culture. Among the
most egregious accusations from employees, who either witnessed or were
subject to incidents and who asked to remain anonymous because of
confidentiality agreements and fear of retaliation: One Uber manager
groped female co-workers' breasts at a company retreat in Las Vegas. A
director shouted a homophobic slur at a subordinate during a heated
confrontation in a meeting. Another manager threatened to beat an
underperforming employee's head in with a baseball bat.

Until this week, this culture was only whispered about in Silicon
Valley. Then on Sunday, Susan Fowler, an engineer who left Uber in
December, published
\href{https://www.susanjfowler.com/blog/2017/2/19/reflecting-on-one-very-strange-year-at-uber}{a
blog post} about her time at the company. She detailed a history of
discrimination and sexual harassment by her managers, which she said was
shrugged off by Uber's human resources department. Ms. Fowler said the
culture was stoked --- and even fostered --- by those at the top of the
company.

``It seemed like every manager was fighting their peers and attempting
to undermine their direct supervisor so that they could have their
direct supervisor's job,'' Ms. Fowler wrote. ``No attempts were made by
these managers to hide what they were doing: They boasted about it in
meetings, told their direct reports about it, and the like.''

\includegraphics{https://static01.graylady3jvrrxbe.onion/images/2017/02/23/business/20uber-1/20uber-1-articleInline.jpg?quality=75\&auto=webp\&disable=upscale}

Her revelations have spurred hand-wringing over how unfriendly Silicon
Valley workplaces can be to women and provoked an internal crisis at
Uber. The company's chief executive, Travis Kalanick, has opened
\href{https://www.nytimes3xbfgragh.onion/2017/02/19/business/uber-sexual-harassment-investigation.html}{an
internal investigation} into the accusations and has brought in the
board member Arianna Huffington and the former attorney general Eric H.
Holder Jr. to look into harassment issues and the human resources
department.

To contain the fallout, Mr. Kalanick also began more disclosure. On
Monday, he said that 15.1 percent of Uber's engineering, product
management and scientist roles were filled by women, and that those
numbers had not changed substantively over the past year.

Mr. Kalanick also held a 90-minute
\href{https://newsroom.uber.com/ariannaupdate/}{all-hands meeting on
Tuesday}, during which he and other executives were besieged with dozens
of questions and pleas from employees who were aghast at --- or strongly
identified with --- Ms. Fowler's story and demanded change.

In what was described by five attendees as an emotional moment, and
according to a video of the meeting reviewed by The New York Times, Mr.
Kalanick apologized to employees for leading the company and the culture
to this point. ``What I can promise you is that I will get better every
day,'' he said. ``I can tell you that I am authentically and fully
dedicated to getting to the bottom of this.''

Some Uber employees said Mr. Kalanick's speedy efforts were positive.
``I am pleased with how quickly Travis has responded to this,'' Aimee
Lucido, an Uber software engineer, wrote
\href{https://medium.com/@hadrad1000/reflecting-on-susan-fowlers-reflections-e2dccb374b47\#.gwyl18wzu}{in
a blog post}. ``We are better situated to handle this sort of problem
than we have ever been in the past.''

As chief executive, Mr. Kalanick has long set the tone for Uber. Under
him, Uber has taken a pugnacious approach to business, flouting local
laws and criticizing competitors in a race to expand as quickly as
possible. Mr. Kalanick, 40, has made pointed displays of ego: In a
\href{http://www.gq.com/story/uber-cab-confessions?currentPage=1}{GQ
article in 2014}, he referred to Uber as ``Boob-er'' because of how the
company helped him attract women.

\href{https://www.nytimes3xbfgragh.onion/interactive/2017/02/22/technology/document-Internal-Memo-From-Travis-Kalanick.html}{}

\includegraphics{https://static01.graylady3jvrrxbe.onion/images/2017/02/22/technology/image-Internal-Memo-From-Travis-Kalanick/image-Internal-Memo-From-Travis-Kalanick-videoLarge.gif}

\hypertarget{internal-memo-from-ubers-chief-travis-kalanick}{%
\subsection{Internal Memo From Uber's Chief, Travis
Kalanick}\label{internal-memo-from-ubers-chief-travis-kalanick}}

Below is the full text of an internal memo sent to Uber employees this
week by the company's chief, Travis Kalanick.

That tone has been echoed in Uber's workplace. At least two former Uber
workers said they had notified Thuan Pham, the company's chief technical
officer, of workplace harassment at the hands of managers and colleagues
in 2016. One also emailed Mr. Kalanick.

Uber also faces at least three lawsuits in at least two countries from
former employees alleging sexual harassment or verbal abuse at the hands
of managers, according to legal documents reviewed by The Times. Other
current and former employees said they were considering legal action
against the company.

Liane Hornsey, Uber's chief human resources officer, said in a
statement, ``We are totally committed to healing wounds of the past and
building a better workplace culture for everyone.''

Uber's aggressive culture began with its 2009 founding, when Mr.
Kalanick and another founder, Garrett Camp, created a start-up that
would let customers hail a cab with little more than a few taps of their
smartphone --- bypassing many of the headaches people had with the taxi
industry. Mr. Kalanick also started putting into place what eventually
became Uber's 14 core values, inspired by
\href{https://www.amazon.jobs/principles}{the leadership principles} at
one of the biggest public tech companies,
\href{https://www.nytimes3xbfgragh.onion/2015/08/16/technology/inside-amazon-wrestling-big-ideas-in-a-bruising-workplace.html}{Amazon}.

To grow quickly, Uber kept its structure decentralized, emphasizing
autonomy among regional offices. General managers are encouraged to ``be
themselves,'' another of Uber's core values, and are empowered to make
decisions without intense supervision from the company's San Francisco
headquarters. The top priority: Achieve growth and revenue targets.

While Uber is now the dominant ride-hailing company in the United
States, and is rapidly growing in South America, India and other
countries, its explosive growth has come at a cost internally. As Uber
hired more employees, its internal politics became more convoluted.
Getting ahead, employees said, often involved undermining departmental
leaders or colleagues.

Image

Arianna Huffington, an Uber board member, was brought in to look into
harassment issues and the human resources department.Credit...Andrew
Burton/Getty Images

Workers like Ms. Fowler who went to human resources with their problems
said they were often left stranded. She and a half-dozen others said
human resources often made excuses for top performers because of their
ability to improve the health of the business. Occasionally, problematic
managers who were the subject of numerous complaints were shuffled
around different regions; firings were less common.

One group appeared immune to internal scrutiny, the current and former
employees said. Members of the group, called the A-Team and composed of
executives who were personally close to Mr. Kalanick, were shielded from
much accountability over their actions.

One member of the A-Team was Emil Michael, senior vice president for
business, who was caught up in a public scandal over comments he made in
2014 about
\href{https://bits.blogs.nytimes3xbfgragh.onion/2014/11/18/emil-michael-of-uber-proposes-digging-into-journalists-private-lives/?_r=0}{digging
into the private lives of journalists} who opposed the company. Mr.
Kalanick defended Mr. Michael, saying he believed Mr. Michael could
learn from his mistakes.

Uber's aggressive workplace culture spilled out at a global all-hands
meeting in late 2015 in Las Vegas, where the company hired Beyoncé to
perform at the rooftop bar of the Palms Hotel. Between bouts of drinking
and gambling, Uber employees used cocaine in the bathrooms at private
parties, said three attendees, and a manager groped several female
employees. (The manager was terminated within 12 hours.) One employee
hijacked a private shuttle bus, filled it with friends and took it for a
joy ride, the attendees said.

At the Las Vegas outing, Mr. Kalanick also held a companywide lecture
reviewing Uber's 14 core values, the attendees said. During the lecture,
Mr. Kalanick pulled onstage employees who he believed exemplified each
of the values. One of those was Mr. Michael.

Since Ms. Fowler's blog post, several Uber employees have said they are
considering leaving the company. Some are waiting until their equity
compensation from Uber, which is restricted stock units, is vested.
Others said they had started sending résumés to competitors.

Still other employees said they were hopeful that Uber could change. Mr.
Kalanick has promised to deliver a diversity report to better detail the
number of women and minorities who work at Uber, and the company is
holding listening sessions with employees.

At the Tuesday all-hands meeting, Ms. Huffington, the Uber board member,
also vowed that the company would make another change. According to
attendees and video of the meeting, Ms. Huffington said there would no
longer be hiring of ``brilliant jerks.''

Advertisement

\protect\hyperlink{after-bottom}{Continue reading the main story}

\hypertarget{site-index}{%
\subsection{Site Index}\label{site-index}}

\hypertarget{site-information-navigation}{%
\subsection{Site Information
Navigation}\label{site-information-navigation}}

\begin{itemize}
\tightlist
\item
  \href{https://help.nytimes3xbfgragh.onion/hc/en-us/articles/115014792127-Copyright-notice}{©~2020~The
  New York Times Company}
\end{itemize}

\begin{itemize}
\tightlist
\item
  \href{https://www.nytco.com/}{NYTCo}
\item
  \href{https://help.nytimes3xbfgragh.onion/hc/en-us/articles/115015385887-Contact-Us}{Contact
  Us}
\item
  \href{https://www.nytco.com/careers/}{Work with us}
\item
  \href{https://nytmediakit.com/}{Advertise}
\item
  \href{http://www.tbrandstudio.com/}{T Brand Studio}
\item
  \href{https://www.nytimes3xbfgragh.onion/privacy/cookie-policy\#how-do-i-manage-trackers}{Your
  Ad Choices}
\item
  \href{https://www.nytimes3xbfgragh.onion/privacy}{Privacy}
\item
  \href{https://help.nytimes3xbfgragh.onion/hc/en-us/articles/115014893428-Terms-of-service}{Terms
  of Service}
\item
  \href{https://help.nytimes3xbfgragh.onion/hc/en-us/articles/115014893968-Terms-of-sale}{Terms
  of Sale}
\item
  \href{https://spiderbites.nytimes3xbfgragh.onion}{Site Map}
\item
  \href{https://help.nytimes3xbfgragh.onion/hc/en-us}{Help}
\item
  \href{https://www.nytimes3xbfgragh.onion/subscription?campaignId=37WXW}{Subscriptions}
\end{itemize}
