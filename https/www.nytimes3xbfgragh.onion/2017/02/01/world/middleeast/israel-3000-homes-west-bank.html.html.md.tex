Sections

SEARCH

\protect\hyperlink{site-content}{Skip to
content}\protect\hyperlink{site-index}{Skip to site index}

\href{https://www.nytimes3xbfgragh.onion/section/world/middleeast}{Middle
East}

\href{https://myaccount.nytimes3xbfgragh.onion/auth/login?response_type=cookie\&client_id=vi}{}

\href{https://www.nytimes3xbfgragh.onion/section/todayspaper}{Today's
Paper}

\href{/section/world/middleeast}{Middle East}\textbar{}Israel Defiantly
Cranks West Bank Settlement Plans Into High Gear

\url{https://nyti.ms/2jUgjx1}

\begin{itemize}
\item
\item
\item
\item
\item
\end{itemize}

Advertisement

\protect\hyperlink{after-top}{Continue reading the main story}

Supported by

\protect\hyperlink{after-sponsor}{Continue reading the main story}

\hypertarget{israel-defiantly-cranks-west-bank-settlement-plans-into-high-gear}{%
\section{Israel Defiantly Cranks West Bank Settlement Plans Into High
Gear}\label{israel-defiantly-cranks-west-bank-settlement-plans-into-high-gear}}

\includegraphics{https://static01.graylady3jvrrxbe.onion/images/2017/02/02/world/02Jerusalem/02Jerusalem-articleLarge.jpg?quality=75\&auto=webp\&disable=upscale}

By \href{http://www.nytimes3xbfgragh.onion/by/ian-fisher}{Ian Fisher}
and \href{https://www.nytimes3xbfgragh.onion/by/isabel-kershner}{Isabel
Kershner}

\begin{itemize}
\item
  Feb. 1, 2017
\item
  \begin{itemize}
  \item
  \item
  \item
  \item
  \item
  \end{itemize}
\end{itemize}

AMONA OUTPOST, West Bank --- In a major acceleration of new settlement
construction plans, Israel has approved thousands more housing units in
the occupied West Bank and, for the first time in years, has called for
the establishment of an entire new settlement there.

Together, the moves intensified Israel's defiance of international
pressure, and opened a
\href{https://www.nytimes3xbfgragh.onion/2017/01/24/world/middleeast/israel-settlement-expansion-west-bank.html}{forceful
new phase of Israeli expansion} into land the Palestinians claim for a
future state.

Even amid images of struggle and frustration on Wednesday as hundreds of
police officers moved to clear out the illegal Amona outpost in the West
Bank, the long-term aims of right-wing politicians who have called for
the aggressive expansion of settlements have seldom taken as large a
leap forward, after years of rebuke from the Obama administration.

``We are in a new era, where life in Judea and Samaria goes back to its
normal and proper course,'' the defense minister, Avigdor Lieberman,
said in a statement, using the biblical name for the West Bank.

Near midnight on Tuesday, the Israeli government approved 3,000 more
settler housing units in the occupied West Bank. That roughly doubled
the amount of proposed
\href{https://www.nytimes3xbfgragh.onion/2017/01/24/world/middleeast/israel-settlement-expansion-west-bank.html}{new
housing units announced in recent days}. Then, on Wednesday, Prime
Minister Benjamin Netanyahu, who has come under heavy pressure from
rival politicians on the right to take bolder steps to expand
settlements, announced that he would promote the establishment of an
entirely new West Bank settlement.

Palestinians reacted with weary opposition, in the long absence of any
real hope for the renewal of talks working toward a two-state solution,
with a full Palestinian state alongside Israel.

``This is a government of settlers that has abandoned the two-state
solution and fully embraced the settler agenda,'' said Husam Zomlot, the
strategic affairs adviser to Mahmoud Abbas, the president of the
Palestinian Authority.

It was a revealing and dramatic day on the chilly hilltops of the West
Bank, now occupied for 50 years after Israel's capture of it from Jordan
in the Arab-Israeli War of 1967.

The new construction announcements seemed timed to soothe hard feelings
among the Israeli right as hundreds of soldiers and police officers
converged early Wednesday on the unauthorized settlement outpost of
Amona to evacuate it, days ahead of a court-ordered deadline for its
demolition and after more than a decade of legal wrangling. It was
built, the courts here say, on privately owned Palestinian land and has
become a minefield for Israeli politicians.

But as young activists barricaded themselves inside some trailer homes
and tried to resist the evacuation, settler leaders appeared largely
upbeat: Despite the evacuation, they said the day's events represented
only a minor setback in what they see as a larger battle, in which many
Israelis doubt there is any deal the Palestinians will ever accept.

\includegraphics{https://static01.graylady3jvrrxbe.onion/images/2017/02/02/world/02Jerusalem3/02Jerusalem3-articleLarge.jpg?quality=75\&auto=webp\&disable=upscale}

Shilo Adler, the head of the Yesha Council, which represents settlers in
the area, said the transition to the Trump administration in the United
States had provided an unprecedented opportunity for wider expansion ---
an opportunity he said should be pressed especially hard before Mr.
Netanyahu is to meet Mr. Trump in Washington on Feb. 15.

``Now we have a historical time to build in all of Judea and Samaria,''
he said. ``Take this very bad story, and think what we can do now, like
after the rain.''

Mr. Netanyahu's office said he had promised the settlers about six weeks
ago that he would establish a new settlement. On Wednesday, as another
sweetener to compensate for the removal of Amona, he appointed a team to
begin work on locating a site for it.

During previous American administrations, Israel made a commitment not
to build new settlements. For years, Israel made a point of describing
housing developments and outposts dotting the West Bank as new
``neighborhoods'' of existing settlements.

World leaders have denounced the settlements in the West Bank, home to
an estimated 400,000 Israeli settlers, arguing that they are choking off
the hopes for two states. In December, the United Nations Security
Council
\href{https://www.nytimes3xbfgragh.onion/2016/12/23/world/middleeast/israel-settlements-un-vote.html}{rejected
settlement building} as a ``flagrant violation'' under international law
--- a position that the United States tacitly supported in the waning
days of the Obama administration.

Mr. Trump seems not to share former President Barack Obama's opposition:
He has said nothing about the new construction, and his administration
has shown signs of tightening ties between the two countries.

The latest plans for the new units in about a dozen settlements came a
week after Israel approved 2,500 homes in the West Bank and 566 in East
Jerusalem. At the same time, the Israeli Parliament is scheduled to vote
next week on legislation that would retroactively legalize scores of
other settlement homes and outposts built on private Palestinian land
and prevent any future evacuations and demolitions.

At the hilltop outpost of Amona, about 3,000 soldiers and police
officers took part in the operation to evacuate about 40 families who
lived in the outpost and hundreds of supporters, who lit fires and
littered the roads with large rocks to try to prevent the authorities
from advancing.

The government had been working to conduct the evacuation without
bloodshed, and hundreds of Israeli police officers, wearing caps and
blue fleece jackets but carrying no weapons, moved into position in the
early morning.

Around 2 p.m., the police began taking away settlers who would not leave
voluntarily, ripping up the makeshift barricades and smashing the
windows of trailers used by activists.

As the police tried to gain entry to one house, people inside responded
by throwing some kind of liquid, and one man screamed, ``You are
supposed to protect us, not break into our homes!''

\includegraphics{https://static01.graylady3jvrrxbe.onion/images/2017/02/02/world/02Jerusalem4/02Jerusalem4-videoSixteenByNine3000.jpg}

Ayelet Videl, 35, who moved to the windy outpost from Jerusalem nine
years ago, said she had packed a few bags, but not the entire house. She
was waiting for a final order to leave, and left later in the day.

``I didn't believe this terrible thing would happen,'' said Ms. Videl,
who had sent her four children, all born in Amona, to their
grandparents' house in central Israel. ``This is our land, this is our
forefathers' land. For 50 years, they've related to it in a confused
way. They should have declared sovereignty over it.''

Ms. Videl's husband, Hillel, had to be carried out by security forces.

By evening, with about half the outpost emptied, the police had reported
at least 20 injuries from objects being thrown at them, and they said
that about a dozen people described as rioters had been arrested.

The new settlement announcements could help ease the pressure on Mr.
Netanyahu, who is
\href{https://www.nytimes3xbfgragh.onion/2017/01/03/world/middleeast/israel-netanyahu-graft-investigation.html}{under
investigation on several fronts} and is trying to push back against
politicians further to the right. The education minister, Naftali
Bennett, is pressing for legislation --- not yet fully embraced by Mr.
Netanyahu --- to take the drastic step of the first annexation of a West
Bank settlement,
\href{https://www.nytimes3xbfgragh.onion/2017/01/30/world/middleeast/the-sleepy-israeli-settlement-thats-fast-becoming-a-flash-point.html}{Ma'ale
Adumim}, just east of Jerusalem.

Speaking in the Parliament on Wednesday as the outpost evacuation began,
Mr. Bennett said of Amona, ``We lost the battle, but we are winning the
campaign for the land of Israel.''

Mr. Netanyahu is also now pushing for the contentious legislation that
would retroactively legalize the illegal outposts, although he
originally opposed it. Israel's attorney general has said that the bill
is unconstitutional and contravenes international law, and that he would
refuse to defend any challenges in court.

``Instead of making peace with the Palestinians, Prime Minister
Netanyahu and his cabinet spend time making peace with the settlers,
which at the end of the day, is their preferred partner for the future
of the Jewish state,'' said Mitchell Barak, a pollster and political
consultant.

Mr. Zomlot, the adviser to Mr. Abbas, said Mr. Netanyahu was using this
time of political transition in the United States to test how the new
administration's stance might differ from that of Mr. Obama.

There are already signs that Mr. Trump intends to be more sympathetic to
Israel's claims: He appointed as ambassador to Israel
\href{https://www.nytimes3xbfgragh.onion/2016/12/16/world/middleeast/david-friedman-us-ambassador-israel.html}{David
M. Friedman}, who opposes a two-state solution and has supported
settlements.

Mr. Trump has also promised to move the American Embassy to Jerusalem
--- a move that Palestinians and Arab leaders have denounced as de facto
recognition of Israel's annexation of East Jerusalem after capturing it
from Jordan in the 1967 war. Mr. Trump has since said that the move
requires further study.

Nonetheless, Mr. Zomlot said his ``working assumption'' was that the
Trump administration would ultimately fall more in line with past
American administrations, which have seen two states as the only
solution.

``We are looking forward to working with this administration to find a
formula for peace --- the ultimate deal, as Trump called it,'' he said.

Advertisement

\protect\hyperlink{after-bottom}{Continue reading the main story}

\hypertarget{site-index}{%
\subsection{Site Index}\label{site-index}}

\hypertarget{site-information-navigation}{%
\subsection{Site Information
Navigation}\label{site-information-navigation}}

\begin{itemize}
\tightlist
\item
  \href{https://help.nytimes3xbfgragh.onion/hc/en-us/articles/115014792127-Copyright-notice}{©~2020~The
  New York Times Company}
\end{itemize}

\begin{itemize}
\tightlist
\item
  \href{https://www.nytco.com/}{NYTCo}
\item
  \href{https://help.nytimes3xbfgragh.onion/hc/en-us/articles/115015385887-Contact-Us}{Contact
  Us}
\item
  \href{https://www.nytco.com/careers/}{Work with us}
\item
  \href{https://nytmediakit.com/}{Advertise}
\item
  \href{http://www.tbrandstudio.com/}{T Brand Studio}
\item
  \href{https://www.nytimes3xbfgragh.onion/privacy/cookie-policy\#how-do-i-manage-trackers}{Your
  Ad Choices}
\item
  \href{https://www.nytimes3xbfgragh.onion/privacy}{Privacy}
\item
  \href{https://help.nytimes3xbfgragh.onion/hc/en-us/articles/115014893428-Terms-of-service}{Terms
  of Service}
\item
  \href{https://help.nytimes3xbfgragh.onion/hc/en-us/articles/115014893968-Terms-of-sale}{Terms
  of Sale}
\item
  \href{https://spiderbites.nytimes3xbfgragh.onion}{Site Map}
\item
  \href{https://help.nytimes3xbfgragh.onion/hc/en-us}{Help}
\item
  \href{https://www.nytimes3xbfgragh.onion/subscription?campaignId=37WXW}{Subscriptions}
\end{itemize}
