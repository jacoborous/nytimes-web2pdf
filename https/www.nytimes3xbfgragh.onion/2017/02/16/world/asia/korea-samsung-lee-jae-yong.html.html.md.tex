Sections

SEARCH

\protect\hyperlink{site-content}{Skip to
content}\protect\hyperlink{site-index}{Skip to site index}

\href{https://www.nytimes3xbfgragh.onion/section/world/asia}{Asia
Pacific}

\href{https://myaccount.nytimes3xbfgragh.onion/auth/login?response_type=cookie\&client_id=vi}{}

\href{https://www.nytimes3xbfgragh.onion/section/todayspaper}{Today's
Paper}

\href{/section/world/asia}{Asia Pacific}\textbar{}Lee Jae-yong, Samsung
Heir, Is Arrested on Bribery Charges

\url{https://nyti.ms/2lncm84}

\begin{itemize}
\item
\item
\item
\item
\item
\end{itemize}

Advertisement

\protect\hyperlink{after-top}{Continue reading the main story}

Supported by

\protect\hyperlink{after-sponsor}{Continue reading the main story}

\hypertarget{lee-jae-yong-samsung-heir-is-arrested-on-bribery-charges}{%
\section{Lee Jae-yong, Samsung Heir, Is Arrested on Bribery
Charges}\label{lee-jae-yong-samsung-heir-is-arrested-on-bribery-charges}}

\includegraphics{https://static01.graylady3jvrrxbe.onion/images/2017/02/17/world/17Samsung/17Samsung-articleLarge.jpg?quality=75\&auto=webp\&disable=upscale}

By \href{http://www.nytimes3xbfgragh.onion/by/choe-sang-hun}{Choe
Sang-Hun}

\begin{itemize}
\item
  Feb. 16, 2017
\item
  \begin{itemize}
  \item
  \item
  \item
  \item
  \item
  \end{itemize}
\end{itemize}

SEOUL, South Korea --- The de facto leader of Samsung, Lee Jae-yong, was
arrested Friday on bribery charges, a dramatic turn in South Korea's
decades-old struggle to end collusive ties between the government and
powerful family-controlled conglomerates.

Mr. Lee, Samsung's vice chairman, was taken to a jail outside Seoul, the
capital, soon after a judge at the Seoul Central District Court issued
an arrest warrant early Friday.

He is accused of paying \$36 million in bribes to President Park
Geun-hye's secretive confidante, Choi Soon-sil, in return for political
favors. Those are alleged to include government support for a merger of
two Samsung affiliates in 2015 that helped Mr. Lee, 48, inherit
corporate control from his incapacitated father, Lee Kun-hee, the
chairman.

Samsung, whose market capitalization accounts for one-fourth of the
value of all listed companies in South Korea, is a potent national
symbol of power, wealth and technological innovation. Mr. Lee is the
first head of the conglomerate ever to be arrested on corruption
charges. Other charges against him include embezzlement, illegal
transfer of property abroad and committing perjury during a
parliamentary hearing.

Analysts say his case is a test of whether the country's relatively
youthful democracy and judicial system
\href{https://www.nytimes3xbfgragh.onion/2017/01/02/world/asia/south-korea-park-geun-hye-samsung.html?_r=0\%20//}{are
ready to crack down on} the white-collar crimes of family-owned
conglomerates, or chaebol, among which Samsung is the biggest and most
profitable.

Mr. Lee, who has not been convicted of any crime, is also the most
prominent businessman to be ensnared in the special prosecutor's
broadening investigation into a corruption scandal that led
\href{https://www.nytimes3xbfgragh.onion/2016/12/09/world/asia/south-korea-president-park-geun-hye-impeached.html}{Parliament
to vote to impeach Ms. Park} in December.

Ms. Park's presidential powers remain suspended, and the
\href{https://www.nytimes3xbfgragh.onion/2017/01/03/world/asia/south-korea-president-impeachment-trial.html}{Constitutional
Court} is expected to rule in the coming weeks on whether she should be
reinstated or formally removed from office.

South Koreans have grown increasingly fed up with corruption scandals.
When huge crowds took to the streets in recent months to demand Ms.
Park's impeachment, they also called for the arrest of Mr. Lee and other
chaebol leaders.

All presidents, including Ms. Park, have entered office vowing to end
favoritism and collusion with the chaebol, but they all later
backtracked, arguing that the corporate titans were too important to the
national economy to be arrested or given long sentences.

While Mr. Lee's arrest was welcome news to many in South Korea, some
fear that even if he is convicted he will be pardoned.

The arrest is a hard-won victory for the special prosecutor, Park
Young-soo, who has been struggling to establish a bribery case against
Mr. Lee and Ms. Park.

Mr. Lee, who also goes by the name Jay Y. Lee in the West, had survived
the prosecutor's first attempt to arrest him last month, when
\href{https://www.nytimes3xbfgragh.onion/2017/01/18/world/asia/samsung-korea-president-impeachment.html}{a
court in Seoul ruled} that there was not enough evidence of bribery. But
investigators have since collected what they called more incriminating
evidence and again asked the court for an arrest warrant.

``Given the newly presented criminal charges and the additional evidence
collected, the legal grounds and need for arresting him are
recognized,'' the judge, Han Jeong-seok, said Friday when he issued the
arrest warrant.

The prosecutor must indict Mr. Lee within 20 days.

In a statement, Samsung said, ``We will do our best to ensure that the
truth is revealed in the court proceedings.''

The prosecutor is preparing to bring bribery charges against Ms. Park as
well, although she cannot be indicted while she is in office. If she is
removed and indicted, she will be the first South Korean leader on trial
for corruption since two military dictators, Chun Doo-hwan and Roh
Tae-woo, were convicted of bribery in the mid-1990s.

Samsung was the most generous among a handful of conglomerates that each
contributed millions of dollars to two foundations controlled by Ms.
Choi, the president's confidante, or signed lucrative contracts with
companies run by Ms. Choi or her associates.

In November, state prosecutors indicted Ms. Choi
\href{https://www.nytimes3xbfgragh.onion/2016/11/20/world/asia/park-geun-hye-south-korea-extortion-accomplice-prosecutors.html}{on
extortion charges}, saying she leveraged her connections with Ms. Park
to coerce Samsung and others.

They identified Ms. Park as an accomplice, but they brought no charges
against the businesses, which they saw as victims of extortion. But the
special prosecutor, who has since taken over the investigation from
state prosecutors, has called Samsung's contributions bribes that were
exchanged for political favors from Ms. Park.

Mr. Lee and Samsung argued that the ``donations'' Samsung paid out to
Ms. Choi were coerced, suggesting that the company was extorted. Ms.
Park also denies any wrongdoing, saying that the donations from Samsung
and other business were voluntary.

Mr. Lee's arrest comes at a bad time for Samsung. His father, Lee
Kun-hee, the chairman, has remained
\href{https://www.nytimes3xbfgragh.onion/2014/05/12/business/international/samsungs-chairman-has-surgery-after-heart-attack.html}{incapacitated
since a heart attack} in 2014. Under the younger Mr. Lee's youthful but
largely untested leadership, Samsung has tried to shed its stodgy image,
vowing repeatedly to improve its corporate governance. That promise has
sounded increasingly hollow in recent weeks, as new details of Samsung's
ties with Ms. Park and Ms. Choi have emerged.

Mr. Lee's arrest also followed Samsung's global recall of its Galaxy
Note 7 smartphones, the most ambitious product launched under his
leadership, which
\href{https://www.nytimes3xbfgragh.onion/2016/09/03/business/samsung-galaxy-note-battery.html}{were
prone to catch fire}.

While his father remained bedridden, Mr. Lee and loyal executives have
tried to speed up a transfer of management control over the
conglomerate. Now those efforts have effectively backfired, with the
prosecutor asserting that the 2015 merger, a crucial piece in the
succession plan, was facilitated through bribery.

Samsung's dozens of subsidiaries are run by professional managers. But
in the cultlike corporate culture of the chaebol, only the so-called
owner chairman can decide on multibillion-dollar investments, the kind
of huge bets that have allowed Samsung to move quickly into a new market
or stay ahead of its competitors in smartphones, memory chips and
flat-panel screens.

Pro-business groups warned that Mr. Lee's arrest would create a
``management vacuum'' at Samsung, leaving it leaderless and shy of
investment. Many Koreans also fear that Samsung's troubles would hurt
the national economy. The conglomerate's main company, Samsung
Electronics, alone accounts for 20 percent of the country's exports.

Over the decades, the South Korean government and the judiciary have
often cited the impact on the economy when they decided not to arrest
chaebol chairmen accused of white-collar crimes or gave them light or
suspended sentences.

Mr. Lee's father was twice convicted of bribery and tax evasion but has
never spent a day in jail. Each time, he was pardoned by the president
and returned to management. At least six of the nation's top 10 chaebol,
which generate revenue equivalent to more than 80 percent of gross
domestic product, are led by men once convicted of white-collar crimes.

Advertisement

\protect\hyperlink{after-bottom}{Continue reading the main story}

\hypertarget{site-index}{%
\subsection{Site Index}\label{site-index}}

\hypertarget{site-information-navigation}{%
\subsection{Site Information
Navigation}\label{site-information-navigation}}

\begin{itemize}
\tightlist
\item
  \href{https://help.nytimes3xbfgragh.onion/hc/en-us/articles/115014792127-Copyright-notice}{©~2020~The
  New York Times Company}
\end{itemize}

\begin{itemize}
\tightlist
\item
  \href{https://www.nytco.com/}{NYTCo}
\item
  \href{https://help.nytimes3xbfgragh.onion/hc/en-us/articles/115015385887-Contact-Us}{Contact
  Us}
\item
  \href{https://www.nytco.com/careers/}{Work with us}
\item
  \href{https://nytmediakit.com/}{Advertise}
\item
  \href{http://www.tbrandstudio.com/}{T Brand Studio}
\item
  \href{https://www.nytimes3xbfgragh.onion/privacy/cookie-policy\#how-do-i-manage-trackers}{Your
  Ad Choices}
\item
  \href{https://www.nytimes3xbfgragh.onion/privacy}{Privacy}
\item
  \href{https://help.nytimes3xbfgragh.onion/hc/en-us/articles/115014893428-Terms-of-service}{Terms
  of Service}
\item
  \href{https://help.nytimes3xbfgragh.onion/hc/en-us/articles/115014893968-Terms-of-sale}{Terms
  of Sale}
\item
  \href{https://spiderbites.nytimes3xbfgragh.onion}{Site Map}
\item
  \href{https://help.nytimes3xbfgragh.onion/hc/en-us}{Help}
\item
  \href{https://www.nytimes3xbfgragh.onion/subscription?campaignId=37WXW}{Subscriptions}
\end{itemize}
