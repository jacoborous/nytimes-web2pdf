Sections

SEARCH

\protect\hyperlink{site-content}{Skip to
content}\protect\hyperlink{site-index}{Skip to site index}

\href{https://www.nytimes3xbfgragh.onion/section/politics}{Politics}

\href{https://myaccount.nytimes3xbfgragh.onion/auth/login?response_type=cookie\&client_id=vi}{}

\href{https://www.nytimes3xbfgragh.onion/section/todayspaper}{Today's
Paper}

\href{/section/politics}{Politics}\textbar{}Settlements for Company Sins
Can No Longer Aid Other Projects, Sessions Says

\url{https://nyti.ms/2t4Emh7}

\begin{itemize}
\item
\item
\item
\item
\item
\end{itemize}

Advertisement

\protect\hyperlink{after-top}{Continue reading the main story}

Supported by

\protect\hyperlink{after-sponsor}{Continue reading the main story}

\hypertarget{settlements-for-company-sins-can-no-longer-aid-other-projects-sessions-says}{%
\section{Settlements for Company Sins Can No Longer Aid Other Projects,
Sessions
Says}\label{settlements-for-company-sins-can-no-longer-aid-other-projects-sessions-says}}

\includegraphics{https://static01.graylady3jvrrxbe.onion/images/2017/06/10/science/10SETTLE3/10SETTLE3-articleLarge.jpg?quality=75\&auto=webp\&disable=upscale}

By
\href{https://www.nytimes3xbfgragh.onion/by/tatiana-schlossberg}{Tatiana
Schlossberg} and
\href{http://www.nytimes3xbfgragh.onion/by/hiroko-tabuchi}{Hiroko
Tabuchi}

\begin{itemize}
\item
  June 9, 2017
\item
  \begin{itemize}
  \item
  \item
  \item
  \item
  \item
  \end{itemize}
\end{itemize}

When companies settle claims of wrongdoing, they are often compelled to
pay for environmental or community development projects as well as pay
fines and direct compensation to victims. Sometimes the third-party
payments are only marginally related to the damages caused by the
company's actions.

To settle claims from the Gulf oil spill,
\href{https://www.nytimes3xbfgragh.onion/interactive/2015/07/02/us/bp-oil-spill-settlement-background.html?mcubz=1}{BP
was required to spend billions} on coastal restoration projects that
were not directly related to spill damage.
\href{https://www.nytimes3xbfgragh.onion/2017/03/30/business/volkswagen-diesel-penalties.html?mcubz=1}{Volkswagen
is financing electric vehicle charging stations} under its settlement of
the diesel emissions cheating scandal.
\href{https://www.nytimes3xbfgragh.onion/2015/09/11/us/duke-energy-to-pay-fine-over-power-plant-violations.html?mcubz=1}{Duke
Energy paid for soil restoration} on federal land as part of its
compensation for air pollution violations at some of its power plants in
North Carolina.

That longstanding practice is now under attack on two fronts,
potentially jeopardizing a source of financing for initiatives across
the country that supporters say have paid great environmental and social
dividends. Critics say the practice effectively creates ``slush funds''
for favored organizations or causes.

Attorney General Jeff Sessions,
\href{https://www.eenews.net/assets/2017/06/07/document_gw_01.pdf}{in a
memo issued this week}, directed the
\href{https://www.nytimes3xbfgragh.onion/topic/organization/us-justice-department}{Justice
Department} to no longer include funding for projects managed by outside
groups in settlements with corporate wrongdoers. The settlement money
will instead go exclusively to the federal Treasury or to victims of the
company's actions, Mr. Sessions said.

``It has come to my attention that certain previous settlement
agreements involving the department included payments to various
nongovernmental, third-party organizations as a condition of settlement
with the United States,'' Mr. Sessions said. ``These third-party
organizations were neither victims nor parties to the lawsuits. The
department will no longer engage in this practice.''

The policy applies only to future cases.

A
\href{https://www.congress.gov/bill/115th-congress/house-bill/732/text/ih}{bill
with similar intent}, sponsored by Robert W. Goodlatte, Republican of
Virginia, passed a House committee in February. It would prevent the
government from using settlement money from civil cases for purposes
other than direct victim compensation or remediation, like cleanups of
environmental disasters. A
\href{https://www.congress.gov/bill/114th-congress/house-bill/5063/text}{version
of the bill} passed the House last year, but died in the Senate.

This year, groups including the Competitive Enterprise Institute and
Americans for Prosperity, both closely linked to the libertarian
billionaire brothers Charles G. and David H. Koch, wrote to President
Trump criticizing a recent settlement between the Obama administration's
Justice Department and Volkswagen.

\href{https://www.nytimes3xbfgragh.onion/2016/06/28/business/volkswagen-settlement-diesel-scandal.html}{The
\$14.7 billion settlement}, over Volkswagen's use of ``defeat devices''
to cheat emissions rules, included almost \$2 billion that Volkswagen
was required to invest in electric vehicle charging stations and other
clean vehicle technology. The settlement also directed Volkswagen to pay
\$2.7 billion to programs that would reduce pollution from diesel cars
and trucks. Volkswagen had been accused of manufacturing cars that
spewed harmful nitrogen oxides at up to 40 times the levels allowed
under the
\href{http://topics.nytimes3xbfgragh.onion/top/reference/timestopics/subjects/c/clean_air_act/index.html?inline=nyt-classifier}{Clean
Air Act}.

Some of the money was in effect going to pay for clean air initiatives
championed by President Barack Obama, the conservative groups said,
initiatives that Congress twice refused to fund.

``Having been twice spurned by lawmakers, the Obama administration
leveraged the Volkswagen settlement,'' the groups charged. All
settlement money, they argued, should ``instead be deposited into the
U.S. Treasury.''

The groups said that Congress has not authorized or provided money for
electric vehicle infrastructure. They said the plan represents ``an
end-run around Congress's lawmaking power.''

Such settlement funds can be appropriate in cases of environmental
wrongdoing, some environmental lawyers said, because the victims are
often scattered over large areas or are difficult to identify and
compensate directly. At times, the government has ordered a polluter to
pay for unrelated environmental restoration projects as restitution; the
attorney general's order would prohibit such deals.

Environmental groups say that settlement funds have gone to critical
projects nationwide. The 2012 settlement with BP, in which it agreed to
pay to resolve criminal charges, included more than \$2 billion for
projects aimed at environmental restoration in the Gulf of Mexico, and
research into spill prevention and other topics. A 2015 settlement with
Duke Energy allocated \$4.4 million to stream and river ecosystem
restoration in North Carolina.

``You're killing something that's worked really well --- which is
getting violators who've broken the law, in some cases in a criminal
way, to agree to fund projects to make the air or water cleaner,'' said
Eric Schaeffer, executive director of the Environmental Integrity
Project and the former director of civil enforcement at the
\href{https://www.nytimes3xbfgragh.onion/topic/organization/environmental-protection-agency}{Environmental
Protection Agency}. ``What's wrong with that?''

Mr. Sessions' new policy directs Justice Department lawyers not to
include similar payments in settlements with corporations. ``The goals
of any settlement are, first and foremost, to compensate victims,
redress harm, or punish and deter unlawful conduct,'' he said in his
memo, dated June 5.

The House bill, known as the Stop Settlement Slush Funds Act of 2017,
would turn that policy into law, making it difficult for future
administrations to reverse.

Mr. Goodlatte, who co-sponsored the bill,
\href{https://judiciary.house.gov/press-release/statement-house-judiciary-committee-chairman-bob-goodlatte-markup-h-r-732-stop-settlement-slush-funds-act-2017/}{cited
the government's settlement} with Volkswagen as an example of the abuse
of settlement funds. He also referred to a deal that required Credit
Suisse to originate affordable housing loans that generate a minimum of
\$240 million in credit as part of
\href{https://www.nytimes3xbfgragh.onion/2016/12/23/business/dealbook/credit-suisse-mortgage-investigation.html}{a
settlement related to its handling of mortgages} before the global
financial crisis. Mr. Goodlatte called that ``effectively a donation in
the guise of a loan.''

Credit Suisse declined to comment.

Frank Holleman, a senior lawyer for the Southern Environmental Law
Center, said that if settlement money for environmental violations goes
to the Treasury Department, it may be spent on something else, and
prevent restoration of or protection of an affected community or
ecosystem.

``You can't just dump money in the river and it gets clean,'' he said.
``You have to contribute to a nonprofit that does the work to make it
that way. It`s not just being thrown away or given to these entities ---
it's payments for a particular service.''

The money has gone to a variety of environmental projects.

In 2015,
\href{https://www.nytimes3xbfgragh.onion/topic/company/duke-energy-corporation}{Duke
Energy}, a North Carolina utility, reached a settlement with the
Department of Justice over claims that it had violated the Clean Air Act
at five of its power plants in the state. The settlement, which
contained provisions to prevent further air pollution, including
shutting down coal-fired electricity generating units, also required the
company to pay a civil penalty and \$4.4 million toward environmental
mitigation projects.

The mitigation projects included ecosystem restoration --- soil
restoration in national forests, and restoration of a trout population,
or revegetation of Red Spruce trees --- and electric vehicle charging
infrastructure and energy efficiency programs in poor areas.

Neil Nissan, a spokesman for Duke Energy, said the utility does not take
a position on the policy or the bill.

One of the projects the
\href{https://www.nytimes3xbfgragh.onion/interactive/2015/07/02/us/bp-oil-spill-settlement-background.html?mcubz=1}{BP
settlement money} has funded is the restoration of more than 13 miles of
beach along a geographical feature known as the Caminada Headlands,
which will provide storm protection for Port Fourchon, La., and other
nearby areas. BP declined to comment.

David M. Uhlmann, a professor at the University of Michigan Law School
and the former chief of the environmental crimes section at the Justice
Department, said, ``The attorney general's directive prohibits those
laudable third-party payments and, in doing so, needlessly undermines
public health and environmental protection efforts.''

He added, ''Congress can and should rely on the Justice Department and
regulatory agencies to ensure that companies who commit regulatory
violations make appropriate payments to remedy the harm they have
caused.''

Advertisement

\protect\hyperlink{after-bottom}{Continue reading the main story}

\hypertarget{site-index}{%
\subsection{Site Index}\label{site-index}}

\hypertarget{site-information-navigation}{%
\subsection{Site Information
Navigation}\label{site-information-navigation}}

\begin{itemize}
\tightlist
\item
  \href{https://help.nytimes3xbfgragh.onion/hc/en-us/articles/115014792127-Copyright-notice}{©~2020~The
  New York Times Company}
\end{itemize}

\begin{itemize}
\tightlist
\item
  \href{https://www.nytco.com/}{NYTCo}
\item
  \href{https://help.nytimes3xbfgragh.onion/hc/en-us/articles/115015385887-Contact-Us}{Contact
  Us}
\item
  \href{https://www.nytco.com/careers/}{Work with us}
\item
  \href{https://nytmediakit.com/}{Advertise}
\item
  \href{http://www.tbrandstudio.com/}{T Brand Studio}
\item
  \href{https://www.nytimes3xbfgragh.onion/privacy/cookie-policy\#how-do-i-manage-trackers}{Your
  Ad Choices}
\item
  \href{https://www.nytimes3xbfgragh.onion/privacy}{Privacy}
\item
  \href{https://help.nytimes3xbfgragh.onion/hc/en-us/articles/115014893428-Terms-of-service}{Terms
  of Service}
\item
  \href{https://help.nytimes3xbfgragh.onion/hc/en-us/articles/115014893968-Terms-of-sale}{Terms
  of Sale}
\item
  \href{https://spiderbites.nytimes3xbfgragh.onion}{Site Map}
\item
  \href{https://help.nytimes3xbfgragh.onion/hc/en-us}{Help}
\item
  \href{https://www.nytimes3xbfgragh.onion/subscription?campaignId=37WXW}{Subscriptions}
\end{itemize}
