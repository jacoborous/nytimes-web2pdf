Sections

SEARCH

\protect\hyperlink{site-content}{Skip to
content}\protect\hyperlink{site-index}{Skip to site index}

\href{https://www.nytimes3xbfgragh.onion/section/technology}{Technology}

\href{https://myaccount.nytimes3xbfgragh.onion/auth/login?response_type=cookie\&client_id=vi}{}

\href{https://www.nytimes3xbfgragh.onion/section/todayspaper}{Today's
Paper}

\href{/section/technology}{Technology}\textbar{}Women in Tech Speak
Frankly on Culture of Harassment

\url{https://nyti.ms/2usF1td}

\begin{itemize}
\item
\item
\item
\item
\item
\item
\end{itemize}

Advertisement

\protect\hyperlink{after-top}{Continue reading the main story}

Supported by

\protect\hyperlink{after-sponsor}{Continue reading the main story}

\hypertarget{women-in-tech-speak-frankly-on-culture-of-harassment}{%
\section{Women in Tech Speak Frankly on Culture of
Harassment}\label{women-in-tech-speak-frankly-on-culture-of-harassment}}

\includegraphics{https://static01.graylady3jvrrxbe.onion/images/2017/07/01/us/01VALLEYWOMEN1/01VALLEYWOMEN1-articleLarge-v2.jpg?quality=75\&auto=webp\&disable=upscale}

By \href{http://www.nytimes3xbfgragh.onion/by/katie-benner}{Katie
Benner}

\begin{itemize}
\item
  June 30, 2017
\item
  \begin{itemize}
  \item
  \item
  \item
  \item
  \item
  \item
  \end{itemize}
\end{itemize}

\href{https://www.nytimes3xbfgragh.onion/es/2017/07/04/hay-un-desequilibrio-de-poder-enorme-mujeres-hablan-del-acoso-sexual-en-silicon-valley}{Leer
en español}

Their stories came out slowly, even hesitantly, at first. Then in a
rush.

One female entrepreneur recounted how she had been propositioned by a
Silicon Valley venture capitalist while seeking a job with him, which
she did not land after rebuffing him. Another showed the increasingly
suggestive messages she had received from a start-up investor. And one
chief executive described how she had faced numerous sexist comments
from an investor while raising money for her online community website.

What happened afterward was often just as disturbing, the women told The
New York Times. Many times, the investors' firms and colleagues ignored
or played down what had happened when the situations were brought to
their attention. Saying anything, the women were warned, might lead to
ostracism.

Now some of these female entrepreneurs have decided to take that risk.
More than two dozen women in the technology start-up industry spoke to
The Times in recent days about being sexually harassed. Ten of them
named the investors involved, often providing corroborating messages and
emails, and pointed to high-profile venture capitalists such as Chris
Sacca of Lowercase Capital and Dave McClure of 500 Startups.

The disclosures came after the tech news site The Information
\href{https://www.theinformation.com/silicon-valley-women-tell-of-vcs-unwanted-advances}{reported}
that female entrepreneurs had been preyed upon by a venture capitalist,
Justin Caldbeck of Binary Capital. The new accounts underscore how
sexual harassment in the tech start-up ecosystem goes beyond one firm
and is pervasive and ingrained. Now their speaking out suggests a
cultural shift in Silicon Valley, where such predatory behavior had
often been murmured about but rarely exposed.

The tech industry has long suffered a gender imbalance, with companies
such as \href{https://www.google.com/diversity/}{Google} and Facebook
acknowledging how few women were in their ranks. Some female engineers
have started to speak out on the issue, including a former Uber engineer
who detailed
\href{https://www.nytimes3xbfgragh.onion/2017/02/19/business/uber-sexual-harassment-investigation.html}{a
pattern of sexual harassment} at the company, setting off internal
investigations that spurred the resignation in June of Uber's chief
executive, Travis Kalanick.

Most recently, the revelations about Mr. Caldbeck of Binary Capital have
triggered an outcry. The investor has been accused of sexually harassing
entrepreneurs while he worked at three different venture firms in the
past seven years, often in meetings in which the women were presenting
their companies to him.

Several of Silicon Valley's top venture capitalists and technologists,
including Reid Hoffman, a founder of LinkedIn, condemned Mr. Caldbeck's
behavior last week and called for investors to
\href{https://www.linkedin.com/pulse/human-rights-women-entrepreneurs-reid-hoffman}{sign
a ``decency pledge.''} Binary has since collapsed, with Mr. Caldbeck
leaving the firm and investors pulling money out of its funds.

The chain of events has emboldened more women to talk publicly about the
treatment they said they had endured from tech investors.

``Female entrepreneurs are a critical part of the fabric of Silicon
Valley,'' said Katrina Lake, founder and chief executive of the
\href{https://www.nytimes3xbfgragh.onion/2017/05/10/business/dealbook/as-department-stores-close-stitch-fix-expands-online.html}{online
clothing start-up Stitch Fix}, who was one of the women targeted by Mr.
Caldbeck. ``It's important to expose the type of behavior that's been
reported in the last few weeks, so the community can recognize and
address these problems.''

The women's experiences help explain why the venture capital and
start-up ecosystem --- which underpins the tech industry and has spawned
companies such as Google, Facebook and Amazon --- has been so lopsided
in terms of gender.

Most venture capitalists and entrepreneurs are men, with female
entrepreneurs receiving \$1.5 billion in funding last year versus \$58.2
billion for men, according to the data firm PitchBook. Many of the
investors hold outsize power, since entrepreneurs need their money to
turn ideas and innovations into a business. And because the venture
industry operates with few disclosure requirements, people have kept
silent about investors who cross the lines with entrepreneurs.

\includegraphics{https://static01.graylady3jvrrxbe.onion/images/2017/07/01/us/01VALLEYWOMEN2/02VALLEYWOMEN2-articleInline.jpg?quality=75\&auto=webp\&disable=upscale}

Some venture capitalists' abuse of power has come to light in recent
years. In 2015, Ellen Pao took her former employer, the prestigious
venture firm Kleiner Perkins Caufield \& Byers, to trial for allegations
of gender discrimination, leveling accusations of professional
retaliation after spurned sexual advances. Ms. Pao
\href{https://www.nytimes3xbfgragh.onion/2015/03/28/technology/ellen-pao-kleiner-perkins-case-decision.html}{lost
the case}, but it
\href{http://www.nytimes3xbfgragh.onion/2012/06/03/technology/lawsuit-against-kleiner-perkins-is-shaking-silicon-valley.html}{sparked
a debate} about whether women in tech should publicly call out unequal
treatment.

``Having had several women come out earlier, including Ellen Pao and me,
most likely paved the way and primed the industry that these things
indeed happen,'' said Gesche Haas, an entrepreneur who said she was
\href{https://medium.com/dreamers-doers/the-journey-of-going-public-c50e69a6d648}{propositioned
for sex} by an investor, Pavel Curda, in 2014. Mr. Curda has
\href{http://www.businessinsider.com/angel-investor-pavel-curda-apologizes-for-sexual-messages-2014-8}{since
apologized}.

Some of the entrepreneurs who spoke with The Times said they were often
touched without permission by investors or advisers.

At a mostly male tech gathering in Las Vegas in 2009, Susan Wu, an
entrepreneur and investor, said that Mr. Sacca, an investor and former
Google executive, touched her face without her consent in a way that
made her uncomfortable. Ms. Wu said she was also propositioned by Mr.
Caldbeck while fund-raising in 2010 and worked hard to avoid him later
when they crossed paths.

``There is such a massive imbalance of power that women in the industry
often end up in distressing situations,'' Ms. Wu said.

After being contacted by The Times, Mr. Sacca wrote in
\href{https://medium.com/@sacca/i-have-more-work-to-do-c775c5d56ca1}{a
blog post} on Thursday: ``I now understand I personally contributed to
the problem. I am sorry.'' In a statement to The Times, he added that he
was ``grateful to Susan and the other brave women sharing their stories.
I'm confident the result of their courage will be long-overdue, lasting
change.''

After the publication of this article, Mr. Sacca contacted The Times
again to amend his original statement, adding: ``I dispute Susan's
account from 2009.''

Many of the women also said they believed they had limited ability to
push back against inappropriate behavior, often because they needed
funding, a job or other help.

In 2014, Sarah Kunst, 31, an entrepreneur, said she discussed a
potential job at 500 Startups, a start-up incubator in San Francisco.
During the recruiting process, Mr. McClure, a founder of 500 Startups
and an investor, sent her a Facebook message that read in part, ``I was
getting confused figuring out whether to hire you or hit on you.''

Ms. Kunst, who now runs a fitness start-up, said she declined Mr.
McClure's advance. When she later discussed the message with one of Mr.
McClure's colleagues, she said 500 Startups ended its conversations with
her.

500 Startups said Mr. McClure, who did not respond to a request for
comment, was no longer in charge of day-to-day operations after an
internal investigation.

``After being made aware of instances of Dave having inappropriate
behavior with women in the tech community, we have been making changes
internally,'' 500 Startups said. ``He recognizes he has made mistakes
and has been going through counseling to work on addressing changes in
his previous unacceptable behavior.''

Image

Lindsay Meyer in her home in San Francisco. She said a venture
capitalist groped and kissed her. ``I felt like I had to tolerate it
because this is the cost of being a nonwhite female founder,'' she
said.Credit...Jim Wilson/The New York Times

Rachel Renock, the chief executive of
\href{https://www.wethos.co/\#get-involved-section}{Wethos}, described a
similar situation in which she faced sexist comments while seeking
financing for her online community site. While she and her female
partners were fund-raising in March, one investor told them that they
should marry for money, that he liked it when women fought back because
he would always win, and that they needed more attractive photos of
themselves in their presentation.

They put up with the comments, Ms. Renock said, because they ``couldn't
imagine a world in which that \$500,000 wasn't on the table anymore.''
Ms. Renock declined to name the investor. Wethos raised the \$500,000
from someone else and is still fund-raising.

Wendy Dent, 43, whose company Cinemmerse makes an app for smart watches,
said she was sent increasingly flirtatious messages by a start-up
adviser, Marc Canter, as she was trying to start her company in 2014.
Mr. Canter, who had founded a software company in the 1980s that became
known as Macromedia, initially agreed to help her find a co-founder. But
over time, his messages became sexual in nature.

In one message, reviewed by The Times, he wrote that she was a
``sorceress casting a spell.'' In another, he commented on how she
looked in a blue dress and added, ``Know what I'm thinking? Why am I
sending you this --- in private?''

Mr. Canter, in an interview, said that Ms. Dent ``came on strong to me,
asking for help'' and that she had used her sexuality publicly. He said
he disliked her ideas so he behaved the way he did to make her go away.

Some entrepreneurs were asked to not speak about the behavior they
experienced.

At a start-up competition in 2014 in San Francisco, Lisa Curtis, an
entrepreneur, pitched her food start-up, Kuli Kuli, and was told her
idea had won the most plaudits from the audience, opening the door to
possible investment. As she stepped off the stage, an investor named
Jose De Dios, said, ``Of course you won. You're a total babe.''

Ms. Curtis later posted on Facebook about the exchange and got a call
from a different investor. He said ``that if I didn't take down the
post, no one in Silicon Valley would give me money again,'' she said.
Ms. Curtis deleted the post.

In a statement, Mr. De Dios said he ``unequivocally did not make a
defamatory remark.''

Often, change happens only when there is a public revelation, some of
the women said. In the case of Mr. Caldbeck and Binary, the investor and
the firm
\href{https://twitter.com/caldbeckj/status/878343782231531520}{have
apologized}, as has Mr. Caldbeck's previous employer, the venture
capital firm Lightspeed Venture Partners, which had received complaints
about him.

``We regret we did not take stronger action,''
\href{https://twitter.com/lightspeedvp/status/879731401242705920}{Lightspeed
said on Twitter} on Tuesday. ``It is clear now that we should have done
more.''

Lindsay Meyer, an entrepreneur in San Francisco, said Mr. Caldbeck put
\$25,000 of his own money into her fitness start-up in 2015. That gave
Mr. Caldbeck reason to constantly text her; in those messages, reviewed
by The Times, he asked if she was attracted to him and why she would
rather be with her boyfriend than him. At times, he groped and kissed
her, she said.

``I felt like I had to tolerate it because this is the cost of being a
nonwhite female founder,'' said Ms. Meyer, who is Asian-American.

But even after she reached out to a mentor, who alerted one of Binary's
investors, Legacy Venture, to Mr. Caldbeck's actions, little changed.
Legacy went on to invest in Binary's new fund. Binary and Mr. Caldbeck
declined to comment.

``We failed to follow up on information about Mr. Caldbeck's personal
behavior,'' Legacy said in a statement. ``We regret this oversight and
are determined to do better.''

Advertisement

\protect\hyperlink{after-bottom}{Continue reading the main story}

\hypertarget{site-index}{%
\subsection{Site Index}\label{site-index}}

\hypertarget{site-information-navigation}{%
\subsection{Site Information
Navigation}\label{site-information-navigation}}

\begin{itemize}
\tightlist
\item
  \href{https://help.nytimes3xbfgragh.onion/hc/en-us/articles/115014792127-Copyright-notice}{©~2020~The
  New York Times Company}
\end{itemize}

\begin{itemize}
\tightlist
\item
  \href{https://www.nytco.com/}{NYTCo}
\item
  \href{https://help.nytimes3xbfgragh.onion/hc/en-us/articles/115015385887-Contact-Us}{Contact
  Us}
\item
  \href{https://www.nytco.com/careers/}{Work with us}
\item
  \href{https://nytmediakit.com/}{Advertise}
\item
  \href{http://www.tbrandstudio.com/}{T Brand Studio}
\item
  \href{https://www.nytimes3xbfgragh.onion/privacy/cookie-policy\#how-do-i-manage-trackers}{Your
  Ad Choices}
\item
  \href{https://www.nytimes3xbfgragh.onion/privacy}{Privacy}
\item
  \href{https://help.nytimes3xbfgragh.onion/hc/en-us/articles/115014893428-Terms-of-service}{Terms
  of Service}
\item
  \href{https://help.nytimes3xbfgragh.onion/hc/en-us/articles/115014893968-Terms-of-sale}{Terms
  of Sale}
\item
  \href{https://spiderbites.nytimes3xbfgragh.onion}{Site Map}
\item
  \href{https://help.nytimes3xbfgragh.onion/hc/en-us}{Help}
\item
  \href{https://www.nytimes3xbfgragh.onion/subscription?campaignId=37WXW}{Subscriptions}
\end{itemize}
