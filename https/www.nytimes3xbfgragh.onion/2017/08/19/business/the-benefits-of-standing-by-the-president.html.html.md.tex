Sections

SEARCH

\protect\hyperlink{site-content}{Skip to
content}\protect\hyperlink{site-index}{Skip to site index}

\href{https://www.nytimes3xbfgragh.onion/section/business}{Business}

\href{https://myaccount.nytimes3xbfgragh.onion/auth/login?response_type=cookie\&client_id=vi}{}

\href{https://www.nytimes3xbfgragh.onion/section/todayspaper}{Today's
Paper}

\href{/section/business}{Business}\textbar{}The Benefits of Standing by
the President

\url{https://nyti.ms/2vNsG6D}

\begin{itemize}
\item
\item
\item
\item
\item
\item
\end{itemize}

Advertisement

\protect\hyperlink{after-top}{Continue reading the main story}

Supported by

\protect\hyperlink{after-sponsor}{Continue reading the main story}

\hypertarget{the-benefits-of-standing-by-the-president}{%
\section{The Benefits of Standing by the
President}\label{the-benefits-of-standing-by-the-president}}

\includegraphics{https://static01.graylady3jvrrxbe.onion/images/2017/08/20/business/20BLACKSTONE1/20BLACKSTONE1-articleInline-v2.jpg?quality=75\&auto=webp\&disable=upscale}

By
\href{http://www.nytimes3xbfgragh.onion/by/jessica-silver-greenberg}{Jessica
Silver-Greenberg},
\href{http://www.nytimes3xbfgragh.onion/by/ben-protess}{Ben Protess} and
\href{http://www.nytimes3xbfgragh.onion/by/michael-corkery}{Michael
Corkery}

\begin{itemize}
\item
  Aug. 19, 2017
\item
  \begin{itemize}
  \item
  \item
  \item
  \item
  \item
  \item
  \end{itemize}
\end{itemize}

He heaped praise on Jared Kushner at a private gathering of bankers and
corporate executives in December, congratulating President Trump's
son-in-law on the surprise election triumph.

He stood up again in May before a group of corporate leaders on the 39th
floor of Citigroup's offices to remind them of all the good the Trump
administration could do for the economy and the country.

And at a meeting on Monday with his employees, as Mr. Trump's support in
corporate America began to crumble over remarks about white
nationalists, he condemned the violence in Charlottesville, Va., but not
the president's response to it. By week's end, a rebellion among
corporate leaders led to the disbanding of business advisory councils to
the president.

Stephen A. Schwarzman, the chief executive of the private equity giant
Blackstone and the leader of one of the councils, has not been alone on
Wall Street in his embrace of the Trump presidency, particularly after
the corporate world endured eight years of Obama-era regulation. But in
each of these private meetings, recounted by people who attended them,
Mr. Schwarzman emerged as one of the president's most respected and
reliable allies in high finance.

People close to Mr. Schwarzman say he does not view himself as a member
of the president's inner circle, but rather as an independent
businessman who gives the White House advice on trade and the economy.

\includegraphics{https://static01.graylady3jvrrxbe.onion/images/2017/08/20/business/20BLACKSTONE2/20BLACKSTONE2-articleInline.jpg?quality=75\&auto=webp\&disable=upscale}

But Mr. Schwarzman's stature in both the world of finance and in Mr.
Trump's Washington helped Blackstone nail down one of the biggest deals
on Wall Street this year --- its selection by Saudi Arabia to
\href{https://www.nytimes3xbfgragh.onion/2017/05/20/business/dealbook/saudi-arabia-to-invest-20-billion-in-infrastructure-mostly-in-us.html}{manage
a new \$20 billion fund}, according to a person with knowledge of the
selection process.

In May, while the president was visiting Saudi Arabia, Blackstone
announced the agreement to manage the fund, the largest in the world to
invest in infrastructure projects. The announcement was made at the
royal palace in Riyadh as Mr. Trump and Mr. Kushner looked on.

Blackstone has noted that it has a long-running relationship with Saudi
Arabia, which had invested in Blackstone before, and said Mr.
Schwarzman's support for the president had nothing to do with the
infrastructure deal, which it believes would have happened regardless of
who was president. The agreement was the ``culmination of a year's
discussions'' with the Saudis that began during the Obama
administration, the company said.

Still, interviews with four people briefed on the deal revealed that the
Saudis had been discussing a possible partnership with a number of other
firms as well, and formally decided on Blackstone --- a fund-raising
juggernaut that manages funds larger than the economies of some nations
--- only after Mr. Schwarzman had started advising the president.

In addition to Mr. Schwarzman's prominence, the Saudi sovereign wealth
fund was drawn to the firm's record of generating huge investment
returns and building new business lines, from real estate to hedge
funds, according to the people with knowledge of the deal.

There is no suggestion that Blackstone did anything wrong. Instead, the
company's experience illustrates the incentives that corporate leaders
have to develop strong ties with Mr. Trump --- the country's businessman
in chief --- and the reputational risks associated with those
relationships when Mr. Trump veers off course, as he did this past week.

``Public service is a core value for people of my generation,'' Mr.
Schwarzman said in a statement. ``It's a great privilege to be asked to
help the country --- even if it occasionally comes with some degree of
criticism.''

Mr. Trump's visit to Saudi Arabia pushed the Blackstone deal forward so
that it could be announced while the president was there, two people
briefed on the agreement said. And the new fund, which plans to invest
in projects like aging bridges and roads primarily in the United States,
is expected to benefit from any federal infrastructure plan that may
materialize under the Trump administration.

A Blackstone spokeswoman said that while a federal plan ``would be
helpful, our business is not at all dependent upon it since state and
local governments --- which build the vast majority of projects --- are
already pursuing billions of dollars in public-private partnerships.''

Some of the other investment firms that were in discussions about a
Saudi partnership, including Brookfield and the Carlyle Group, had more
experience in managing infrastructure funds and are still in talks with
the Saudis according to the people who were not authorized to speak
about a private deal.

But the Saudi sovereign wealth fund saw that as a benefit for
Blackstone: Without its own infrastructure fund, the company would have
the flexibility to build one from scratch to the liking of the Saudis.
Blackstone, which has invested in infrastructure projects but does not
have a stand-alone fund, is expected to raise at least another \$20
billion for the Saudi fund.

Other deals involving chief executives with ties to Mr. Trump were
announced during his visit to Saudi Arabia.

Image

Andrew Liveris, the chairman and chief executive of Dow Chemical,
reached an agreement to invest \$100 million in a Saudi manufacturing
facility.Credit...Brendan Smialowski/Getty Images

Andrew Liveris, the chairman and chief executive of Dow Chemical,
reached an agreement to invest \$100 million in a Saudi manufacturing
facility. Mr. Liveris, who led the president's manufacturing council
\href{https://www.nytimes3xbfgragh.onion/2017/08/16/business/trumps-council-ceos.html}{until
it was disbanded this past week}, has done business in Saudi Arabia for
years. And before the president's visit, Mr. Liveris offered to
introduce Mr. Kushner to the Saudi energy minister, according to two
people with knowledge of the matter.

In all, there were more than 40 signed agreements between Saudi Arabia
and largely American corporations, including General Electric and the
defense contractor Lockheed Martin. The deal signings, which Mr. Trump
said were valued at nearly \$400 billion, came on the same day that
about 50 chief executives from the United States and Saudi Arabia had
gathered at the Four Seasons Hotel in Riyadh to discuss business
opportunities.

It is routine for the Commerce Department to advocate American companies
--- and for trade missions to lead to the signing of partnerships ---
but the scale of the Saudi event was unusual compared with previous
United States-Saudi gatherings, according to people who attended.

The Saudi sovereign wealth fund did not respond to requests for comment.

While Mr. Schwarzman's support for the president caused a public
relations headache for Blackstone this past week, friends say he is not
the type of corporate leader to express regrets about taking on a
prominent role in Washington.

``Steve Schwarzman is not a person who second guesses himself, and I
don't believe anything catches him off guard,'' said Kathryn S. Wylde,
president and chief executive of the Partnership for New York City, a
business group focused on economic policy.

Ms. Wylde said she had spoken this month with Mr. Schwarzman, who is
co-chairman of the group's board, and found him to be in good spirits.
``He was speaking from a position of strength,'' she said.

\hypertarget{an-unlikely-ally}{%
\subsection{An Unlikely Ally}\label{an-unlikely-ally}}

Mr. Schwarzman and Mr. Trump are hardly old friends.

Mr. Schwarzman, a Republican billionaire, did not support Mr. Trump
during the election and gave no contributions to his campaign. Two of
Mr. Schwarzman's top deputies at Blackstone are big Democratic donors.

But while Mr. Schwarzman's alliance with Mr. Trump is new, his ties to
Mr. Kushner, the president's son-in-law, run deeper. Mr. Kushner and his
wife, Ivanka Trump, attended Mr. Schwarzman's 70th birthday party in
February at his home in Palm Beach, Fla., near Mr. Trump's Mar-a-Lago
estate.

In 2013, well before Mr. Trump was even a candidate, Blackstone financed
the purchase of a few warehouses and industrial buildings by Mr.
Kushner's family company, according to a person briefed on the
transaction.

Blackstone also made a loan, which has since been paid off, to Kushner
Companies on a Rector Street property in Manhattan. And last summer, an
entity controlled by Blackstone lent \$376 million to Mr. Kushner's
company to purchase a large property in Brooklyn that the Jehovah's
Witnesses had operated for many years, real estate records show.

Image

Jared Kushner, left, Mr. Trump's son-in-law, and Jon Gray, a senior
Blackstone executive.Credit...From left: Getty Images; Karsten Moran for
The New York Times

Mr. Kushner is also friendly with Jon Gray, a senior Blackstone
executive who runs its real estate business. (Blackstone is the largest
commercial real estate investor in the world.)

Mr. Kushner and Mr. Gray, a Democrat, have been photographed together at
Manhattan social events, and before the election, Mr. Kushner urged the
staff at the Commercial Observer newspaper, which Mr. Kushner used to
run, to place Mr. Gray higher on its list of ``Power 100'' real estate
executives, according to a former employee with knowledge of the list.
In 2016, Mr. Gray was No. 1 on that list.

Blackstone said it had no knowledge of efforts to influence the ranking,
but Mr. Gray often lands at the top of lists of leading players in real
estate.

Separately, Mr. Kushner and Ms. Trump invested up to \$500,000 in a fund
that Blackstone manages. The couple is now in the process of divesting,
according to the couple's financial disclosure. Blackstone declined to
comment on specific investors, but a spokeswoman noted that the firm had
thousands of investors across its many funds.

Shortly after the election, Mr. Trump asked Mr. Schwarzman to lead the
Strategic and Policy Forum, a group of executives from big banks and
other companies that would advise the president on economic issues. The
group met only twice before being disbanded this past week in the wake
of the president's comments about Charlottesville.

In those two meetings, the group discussed issues important to business
like infrastructure and regulations. Within hours of the first meeting
in February, Mr. Trump signed an executive order seeking to roll back
Obama-era financial rules.

Weeks after the group met in April, Mr. Schwarzman addressed the board
of the Partnership for New York City. He made a case that Mr. Trump was
good for business and in turn the country, according to two people who
attended. Unlike many people in Washington, Mr. Schwarzman said, Mr.
Trump could accomplish tax policy reform and an infrastructure overhaul.

Mr. Schwarzman speaks with Mr. Trump as much as once a week, typically
about the economy though also about social policy, including a
conversation in which Mr. Schwarzman
\href{https://www.nytimes3xbfgragh.onion/2017/04/22/us/politics/donald-trump-white-house.html?mcubz=1}{advised
the president} to continue shielding young undocumented immigrants from
deportation, according to a person briefed on their calls. The two men
sometimes go weeks without talking, said the person, who added that they
do not discuss Blackstone's business.

\hypertarget{a-royal-agreement}{%
\subsection{A Royal Agreement}\label{a-royal-agreement}}

Blackstone manages about \$370 billion. But the private equity firm is
always on the hunt for more --- and Saudi Arabia was ripe for the
picking.

The kingdom has been looking to diversify its economy beyond fossil
fuels and to make investments in other areas like tech and
infrastructure.

Last year, the Saudi sovereign wealth fund began soliciting bids from
multiple investment firms to manage an infrastructure fund, according to
two people briefed on the matter. Blackstone and other American asset
managers were among those to have discussions with the kingdom, the
people said.

In May 2016, Mr. Schwarzman flew to Riyadh to speak with Mohammed bin
Salman, then the deputy crown prince of the Saudi royal family. The
prince, who has since ascended to become first in line to the throne,
was overseeing the sovereign wealth fund, known as the public investment
fund. Mr. Schwarzman and the prince met again about a month later in New
York, in part to discuss infrastructure, a person briefed on the meeting
said.

It was not until months later that the selection process formally gained
momentum --- and by then, a lot had changed. The newly elected Trump
administration signaled that its policies, particularly the president's
hard line against Iran, would be more amenable to the Saudis than
President Barack Obama's agenda for the region.

In March, the Saudis hired an American adviser to help vet the
investment firms vying for the infrastructure deal. By April, Blackstone
stood out as the winner, a person briefed on the deal said. (Some of the
other firms are still discussing other infrastructure projects with the
kingdom.)

The timeline for announcing the infrastructure fund manager was suddenly
accelerated when the White House said in early May that Mr. Trump would
make Saudi Arabia the destination of his first foreign trip as
president.

It was a big moment for the kingdom. After years of complex relations
with Mr. Obama, Mr. Trump offered the Saudis a more straightforward
alliance.

``They felt they could do business with the Trump administration without
any real focus on thorny issues such as human rights or questions of
governance that have complicated bilateral ties with other presidencies
in the past,'' said Kristian Coates Ulrichsen, a fellow for the Middle
East at Rice University's Baker Institute for Public Policy.

The Saudis scrambled to put together a business meeting on the same
weekend as Mr. Trump's visit --- when the world's news media would be
glued to the new president's first foreign trip.

Dow Chemical's chief executive, Mr. Liveris, worked with the Saudis to
organize a meeting that would showcase the country's many opportunities
for global businesses.

By then, Blackstone was exchanging formal documents with the Saudi
sovereign wealth fund. But if it wanted the infrastructure deal to
proceed, Blackstone had to agree to a nonbinding version of the deal in
time for the president's visit.

Image

Kirill Dmitriev, left, the chief executive of a Russian sovereign wealth
fund, and Michael Corbat, the chief executive of Citigroup.Credit...From
left: Denis Sinyakov and Ueslei Marcelino/Reuters

Dozens of chief executives from across the United States faced pressure
over the meeting. Some of them, speaking on the condition of anonymity,
said they had felt they had no choice but to go if they wanted to do
business in Saudi Arabia. One executive said that he had planned to send
a subordinate, but that an event organizer had told him that he should
attend.

The guest list included an oil executive, defense contractors and a
college president. Michael Corbat, the chief executive of Citigroup, was
also at the meeting. In April, his bank had received a capital markets
license in Saudi Arabia that would allow it do more business there,
after being frozen out for many years.

Also invited was Kirill Dmitriev, the chief executive of a Russian
sovereign wealth fund who has spoken publicly of his support for Mr.
Trump, according to a private list of attendees. Mr. Dmitriev's fund was
created as part of VEB, a bank wholly owned by the Russian state that
\href{https://www.nytimes3xbfgragh.onion/2017/06/04/business/vnesheconombank-veb-bank-russia-trump-kushner.html}{has
figured in the federal investigation into Russian meddling} in the
election.

Mr. Schwarzman and other American executives had joined Mr. Dmitriev's
international advisory board seven years earlier, but most resigned
after Moscow's military intervention in Crimea.

The celebration in Saudi Arabia stretched throughout the weekend,
highlighted by a lunch at the royal palace and a dinner at the home of
Yasir Al Rumayyan, the managing director of Saudi Arabia's Public
Investment Fund, the fund that is working with Blackstone, according to
people who attended the dinner.

The day after the deals were announced, Mr. Trump gave a speech,
thanking the Saudis for their hospitality and their willingness to
cooperate on terrorism and business matters.

The president also called the deals that Americans companies had struck
``blessed news.''

Image

Mr. Trump spoke in Riyadh in May.Credit...Stephen Crowley/The New York
Times

Advertisement

\protect\hyperlink{after-bottom}{Continue reading the main story}

\hypertarget{site-index}{%
\subsection{Site Index}\label{site-index}}

\hypertarget{site-information-navigation}{%
\subsection{Site Information
Navigation}\label{site-information-navigation}}

\begin{itemize}
\tightlist
\item
  \href{https://help.nytimes3xbfgragh.onion/hc/en-us/articles/115014792127-Copyright-notice}{©~2020~The
  New York Times Company}
\end{itemize}

\begin{itemize}
\tightlist
\item
  \href{https://www.nytco.com/}{NYTCo}
\item
  \href{https://help.nytimes3xbfgragh.onion/hc/en-us/articles/115015385887-Contact-Us}{Contact
  Us}
\item
  \href{https://www.nytco.com/careers/}{Work with us}
\item
  \href{https://nytmediakit.com/}{Advertise}
\item
  \href{http://www.tbrandstudio.com/}{T Brand Studio}
\item
  \href{https://www.nytimes3xbfgragh.onion/privacy/cookie-policy\#how-do-i-manage-trackers}{Your
  Ad Choices}
\item
  \href{https://www.nytimes3xbfgragh.onion/privacy}{Privacy}
\item
  \href{https://help.nytimes3xbfgragh.onion/hc/en-us/articles/115014893428-Terms-of-service}{Terms
  of Service}
\item
  \href{https://help.nytimes3xbfgragh.onion/hc/en-us/articles/115014893968-Terms-of-sale}{Terms
  of Sale}
\item
  \href{https://spiderbites.nytimes3xbfgragh.onion}{Site Map}
\item
  \href{https://help.nytimes3xbfgragh.onion/hc/en-us}{Help}
\item
  \href{https://www.nytimes3xbfgragh.onion/subscription?campaignId=37WXW}{Subscriptions}
\end{itemize}
