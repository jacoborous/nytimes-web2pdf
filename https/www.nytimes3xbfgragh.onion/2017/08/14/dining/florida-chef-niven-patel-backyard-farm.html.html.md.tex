Sections

SEARCH

\protect\hyperlink{site-content}{Skip to
content}\protect\hyperlink{site-index}{Skip to site index}

\href{https://www.nytimes3xbfgragh.onion/section/food}{Food}

\href{https://myaccount.nytimes3xbfgragh.onion/auth/login?response_type=cookie\&client_id=vi}{}

\href{https://www.nytimes3xbfgragh.onion/section/todayspaper}{Today's
Paper}

\href{/section/food}{Food}\textbar{}For Niven Patel, Farm-to-Table
Cooking Means Taro and Mangoes

\begin{itemize}
\item
\item
\item
\item
\item
\item
\end{itemize}

Advertisement

\protect\hyperlink{after-top}{Continue reading the main story}

Supported by

\protect\hyperlink{after-sponsor}{Continue reading the main story}

The Chef

\hypertarget{for-niven-patel-farm-to-table-cooking-means-taro-and-mangoes}{%
\section{For Niven Patel, Farm-to-Table Cooking Means Taro and
Mangoes}\label{for-niven-patel-farm-to-table-cooking-means-taro-and-mangoes}}

\href{https://www.nytimes3xbfgragh.onion/slideshow/2017/08/14/dining/the-homesteader.html}{}

\hypertarget{the-homesteader}{%
\subsection{The Homesteader}\label{the-homesteader}}

11 Photos

View Slide Show ›

\includegraphics{https://static01.graylady3jvrrxbe.onion/images/2017/08/16/dining/16TROPICAL-slide-Y4KA/16TROPICAL-slide-Y4KA-articleLarge.jpg?quality=75\&auto=webp\&disable=upscale}

John Taggart for The New York Times

By Rachel Wharton

\begin{itemize}
\item
  Aug. 14, 2017
\item
  \begin{itemize}
  \item
  \item
  \item
  \item
  \item
  \item
  \end{itemize}
\end{itemize}

HOMESTEAD, Fla. --- August may not be the ideal time to visit Niven
Patel's backyard farm.

It's the middle of monsoon season here in South Florida, which means
that Mr. Patel's two-acre property --- which produces about 15 percent
of the produce for his restaurant, \href{http://gheemiami.com/}{Ghee
Indian Kitchen}, a half-hour north in greater Miami --- is besieged by
mosquitoes, rain and humidity so intense that you can't grow tomatoes
past April.

But summer does have its payoffs for those who choose to live at the
fertile edge of the Everglades. At a recent dinner party at Ranchopatel
--- Mr. Patel's gray stone house on Avocado Drive --- the chef cut
waist-high taro leaves, dug his own white turmeric roots, plucked
teardrop Indian eggplants to stuff with crushed peanuts, and filled a
wheelbarrow with 200 pounds of mangoes destined for lassis and chutney.

His four towering mango trees --- along with three avocado, a sapodilla
and 12 lychee trees by the driveway --- were actually what drew Mr.
Patel to this property four years ago, when he was still the chef de
cuisine at \href{https://michaelsgenuine.com/}{Michael's Genuine Food \&
Drink} in Miami.

``As soon as I drove in, I saw the lychee trees,'' Mr. Patel said, ``and
it just felt right.''

Mr. Patel, who turns 34 on Wednesday, was wise to trust his instincts.
Located in the only part of the contiguous United States with a tropical
monsoon climate --- like those in India and Southeast Asia ---
Ranchopatel gave him not just sweet fruit, but a new direction.

Homestead is part of one of Florida's most diverse agricultural regions.
Mr. Patel's neighbors include ordinary owners of one-story homes with
impossibly lush lawns, as well as farmers who send conventional
supermarket vegetables north in December; large-scale exotic fruit
producers with groves of Thai white guavas or Pakistani mangoes; and
small-scale growers with roots across the Asian continent, the Caribbean
or South America. Many of those growers cultivate just a few acres,
their yards mixed carpets of foreign tree fruits, vegetables and herbs
that creep right up to the road.

These crops --- hard-to-find tastes of home like Cambodian rice paddy
herb, Vietnamese coriander or moringa leaves --- quietly make their way
up the East Coast and across the Midwest, said Valerie Imbruce, an
economic botanist who studied the area for her 2015 book,
``\href{http://www.cornellpress.cornell.edu/book/?GCOI=80140100191750}{From
Farm to Canal Street}.'' There, you'll mainly find them in niche grocery
stores, she said, ``tucked away in the refrigerated section next to the
bottled drinks.''

Go to an Indian market in New York City, for example, and you are likely
to spot green papaya or luffa gourd from Jalaram Produce, which Mahendra
Raolji has run in Homestead for 22 years. ``He is the biggest Indian
farmer in probably the U.S.,'' Mr. Patel said, ``and he's 10 minutes
away.''

That's handy for Mr. Patel, who blends local sourcing with his family's
home cooking. Born in Georgia and raised in Florida, Mr. Patel traces
his roots to the Indian state of Gujarat, as does his wife, Shivani.

At his four-month-old restaurant --- a roomy modern space where jars of
ground Ranchopatel chiles serve as both pantry and folk art --- Mr.
Patel may marinate the tropical game fish called wahoo in fresh
turmeric, serving a still-raw slice alongside bhel puri, a mix of puffed
rice with green mango and herby chutney. Charred local okra pods are
seasoned with black mustard seed and soaked with preserved heirloom
tomatoes, which he buys from farmer friends in spring. (The Florida
vegetable season is reversed, Mr. Patel explained. Crops that thrive in
extraordinarily hot and humid climates, like okra, beans and eggplant,
are the only vegetables that grow during the summer; Floridians import
fresh tomatoes and everything else from farmers up north.)

Many Miami chefs work with local farmers or seek out special or native
ingredients, but with Ghee, Mr. Patel is a game changer for the region.
He is the first chef to draw such a sharp line from his cooking to the
unique potential of his agricultural landscape, linking his heritage to
his literal backyard.

``What he's cooking and what he's doing, and in this climate and in this
zone, are just like a match made in heaven,'' said Michael Schwartz, the
executive chef and owner of Michael's Genuine and a farm-to-table
pioneer in Miami.

``For me, it's inspiring,'' Mr. Schwartz said. ``He's certainly got
everyone thinking about new possibilities.''

Mr. Patel was still working for Mr. Schwartz when he decided to open
Ghee. ``I never in my wildest dreams thought I would open up an Indian
restaurant,'' said Mr. Patel, who has spent a majority of his 13-year
career cooking modern American food, much of it farm-to-table.

He first fell for hands-on growing six years ago while working as a chef
at a restaurant on Grand Cayman that kept an expansive kitchen garden.
After buying his house in Homestead in 2013, Mr. Patel had the land to
create his own: He recruited his kitchen staff at Michael's to help
build out 14 32-foot-long raised beds just past the swimming pool.

Friends came to see, and he served them what his family always ate at
home: basmati rice with roasted garlic, ghee and chiles; a simple okra
curry; the hand-rolled Gujarati flatbread called rotli; a warm yogurt
soup flavored with homegrown curry leaf and thickened with grated
tomato, cucumber and tender, fresh pigeon peas.

His guests were blown away by the food, and Mr. Patel by their reaction:
``I thought, `We have to do this; we have to make a place that's like
eating at home.'''

Home, in fact, is where Ghee's staff came to train, learning the
intricacies of the Indian kitchen from Mr. Patel, his wife and her
parents, who also live at Ranchopatel. (His father-in-law is a co-farmer
and the handyman, his mother-in-law is the restaurant's second culinary
guru, and Ms. Patel works the dining room after spending the day at her
full-time job in human resources at a resort.)

``We would cook up a feast,'' said Mr. Patel, who eventually started
summoning people over to eat it --- chefs, brewers, farmers or food
producers, some of whom he knew only via Instagram. These free-form
dinners became legend, spread by social media posts showing sprouted
mung beans or spices being ground by a hand-powered stone mill.

Mr. Patel still lures crowds to Ranchopatel when he can. That backyard
dinner with the wheelbarrow full of mangoes, for example, was a feast
for about 30 on the family's one day off that week. The guests included
a South Beach chef, a lawyer who runs one of Miami's best-read food
blogs, the owner of an artisan tea company and most of Mr. Patel's
staff, who pitched in to grill whole queen snappers slicked with
ginger-chile spice paste and to fill flatbreads with the purple Indian
yam called ratalu.

These connections are important to Mr. Patel, whose goal --- beyond
hiring a farmer to convert the rest of his yard so he can grow more
produce for the restaurant by fall --- is to foster the deep links
between agriculture, food professionals and diners that exist in many
other cities.

His sleep-deprived family might like it if the Ranchopatel gatherings
got a little smaller, though. ``I invite too many people,'' he said,
``and they get mad at me.''

Recipes:
\href{https://cooking.nytimes3xbfgragh.onion/recipes/1018887-eggplant-ravaiya}{\textbf{Eggplant
Ravaiya}} \textbar{}
\href{https://cooking.nytimes3xbfgragh.onion/recipes/1018886-roasted-mango-or-banana-lassi}{\textbf{Roasted
Mango or Banana Lassi}}

\href{https://www.facebookcorewwwi.onion/nytfood/}{\emph{Follow NYT Food
on Facebook}}\emph{,}
\href{https://instagram.com/nytfood}{\emph{Instagram}}\emph{,}
\href{https://twitter.com/nytfood}{\emph{Twitter}} \emph{and}
\href{https://www.pinterest.com/nytfood/}{\emph{Pinterest}}\emph{.}
\href{https://www.nytimes3xbfgragh.onion/newsletters/cooking}{\emph{Get
regular updates from NYT Cooking, with recipe suggestions, cooking tips
and shopping advice}}\emph{.}

Advertisement

\protect\hyperlink{after-bottom}{Continue reading the main story}

\hypertarget{site-index}{%
\subsection{Site Index}\label{site-index}}

\hypertarget{site-information-navigation}{%
\subsection{Site Information
Navigation}\label{site-information-navigation}}

\begin{itemize}
\tightlist
\item
  \href{https://help.nytimes3xbfgragh.onion/hc/en-us/articles/115014792127-Copyright-notice}{©~2020~The
  New York Times Company}
\end{itemize}

\begin{itemize}
\tightlist
\item
  \href{https://www.nytco.com/}{NYTCo}
\item
  \href{https://help.nytimes3xbfgragh.onion/hc/en-us/articles/115015385887-Contact-Us}{Contact
  Us}
\item
  \href{https://www.nytco.com/careers/}{Work with us}
\item
  \href{https://nytmediakit.com/}{Advertise}
\item
  \href{http://www.tbrandstudio.com/}{T Brand Studio}
\item
  \href{https://www.nytimes3xbfgragh.onion/privacy/cookie-policy\#how-do-i-manage-trackers}{Your
  Ad Choices}
\item
  \href{https://www.nytimes3xbfgragh.onion/privacy}{Privacy}
\item
  \href{https://help.nytimes3xbfgragh.onion/hc/en-us/articles/115014893428-Terms-of-service}{Terms
  of Service}
\item
  \href{https://help.nytimes3xbfgragh.onion/hc/en-us/articles/115014893968-Terms-of-sale}{Terms
  of Sale}
\item
  \href{https://spiderbites.nytimes3xbfgragh.onion}{Site Map}
\item
  \href{https://help.nytimes3xbfgragh.onion/hc/en-us}{Help}
\item
  \href{https://www.nytimes3xbfgragh.onion/subscription?campaignId=37WXW}{Subscriptions}
\end{itemize}
