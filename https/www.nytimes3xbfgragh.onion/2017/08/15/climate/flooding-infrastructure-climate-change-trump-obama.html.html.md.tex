Sections

SEARCH

\protect\hyperlink{site-content}{Skip to
content}\protect\hyperlink{site-index}{Skip to site index}

\href{https://www.nytimes3xbfgragh.onion/section/climate}{Climate}

\href{https://myaccount.nytimes3xbfgragh.onion/auth/login?response_type=cookie\&client_id=vi}{}

\href{https://www.nytimes3xbfgragh.onion/section/todayspaper}{Today's
Paper}

\href{/section/climate}{Climate}\textbar{}Trump Signs Order Rolling Back
Environmental Rules on Infrastructure

\url{https://nyti.ms/2uZIGPe}

\begin{itemize}
\item
\item
\item
\item
\item
\end{itemize}

\hypertarget{climate-and-environment}{%
\subsubsection{\texorpdfstring{\href{https://www.nytimes3xbfgragh.onion/section/climate?name=styln-climate\&region=TOP_BANNER\&variant=undefined\&block=storyline_menu_recirc\&action=click\&pgtype=Article\&impression_id=d385cc40-e388-11ea-b524-5975769cbb40}{Climate
and
Environment}}{Climate and Environment}}\label{climate-and-environment}}

\begin{itemize}
\tightlist
\item
  \href{https://www.nytimes3xbfgragh.onion/2020/08/17/climate/alaska-oil-drilling-anwr.html?name=styln-climate\&region=TOP_BANNER\&variant=undefined\&block=storyline_menu_recirc\&action=click\&pgtype=Article\&impression_id=d385f350-e388-11ea-b524-5975769cbb40}{Arctic
  Refuge}
\item
  \href{https://www.nytimes3xbfgragh.onion/interactive/2020/climate/trump-environment-rollbacks.html?name=styln-climate\&region=TOP_BANNER\&variant=undefined\&block=storyline_menu_recirc\&action=click\&pgtype=Article\&impression_id=d38de290-e388-11ea-b524-5975769cbb40}{Trump's
  Changes}
\item
  \href{https://www.nytimes3xbfgragh.onion/interactive/2020/04/19/climate/climate-crash-course-1.html?name=styln-climate\&region=TOP_BANNER\&variant=undefined\&block=storyline_menu_recirc\&action=click\&pgtype=Article\&impression_id=d38de291-e388-11ea-b524-5975769cbb40}{Climate
  101}
\item
  \href{https://www.nytimes3xbfgragh.onion/interactive/2018/08/30/climate/how-much-hotter-is-your-hometown.html?name=styln-climate\&region=TOP_BANNER\&variant=undefined\&block=storyline_menu_recirc\&action=click\&pgtype=Article\&impression_id=d38e09a0-e388-11ea-b524-5975769cbb40}{Is
  Your Hometown Hotter?}
\end{itemize}

Advertisement

\protect\hyperlink{after-top}{Continue reading the main story}

Supported by

\protect\hyperlink{after-sponsor}{Continue reading the main story}

\hypertarget{trump-signs-order-rolling-back-environmental-rules-on-infrastructure}{%
\section{Trump Signs Order Rolling Back Environmental Rules on
Infrastructure}\label{trump-signs-order-rolling-back-environmental-rules-on-infrastructure}}

\includegraphics{https://static01.graylady3jvrrxbe.onion/images/2017/08/16/science/16FLOOD2/16FLOOD2-articleLarge.jpg?quality=75\&auto=webp\&disable=upscale}

By \href{https://www.nytimes3xbfgragh.onion/by/lisa-friedman}{Lisa
Friedman}

\begin{itemize}
\item
  Aug. 15, 2017
\item
  \begin{itemize}
  \item
  \item
  \item
  \item
  \item
  \end{itemize}
\end{itemize}

WASHINGTON --- President Trump announced on Tuesday that he had signed a
sweeping executive order to eliminate and streamline some permitting
regulations and to speed construction of roads, bridges and pipelines,
declaring that the moves would fix a ``badly broken'' infrastructure
system in America and bring manufacturing jobs back to the country.

In
\href{https://www.nytimes3xbfgragh.onion/2017/08/15/us/politics/trump-press-conference-charlottesville.html}{an
explosive news conference} overshadowed by questions about his response
to a \href{100000005365230/draft/editing}{white nationalist rally} over
the weekend in Charlottesville, Va., Mr. Trump tried several times to
steer the conversation back to infrastructure.

The president has vowed to pass a
\href{https://www.nytimes3xbfgragh.onion/2017/06/03/us/politics/trump-plans-to-shift-infrastructure-funding-to-cities-states-and-business.html}{\$1
trillion package} to revitalize the nation's infrastructure, but so far
\href{https://www.nytimes3xbfgragh.onion/2017/07/23/us/trump-infrastructure-program.html}{no
such legislation exists}. Mr. Trump said he was not worried about his
chances of winning support for his plan, even in the wake of a failed
effort to overhaul health care.

``We will end up getting health care, but we'll get the infrastructure,
and actually infrastructure is something that I think we'll have
bipartisan support on,'' Mr. Trump said. ``I actually think Democrats
will go along with the infrastructure.''

Flanked by Transportation Secretary Elaine L. Chao and Treasury
Secretary Steven Mnuchin at Trump Tower in Manhattan, Mr. Trump unveiled
a head-to-toe-length flow chart purporting to show the permitting
regulations required to build a highway in a state he would not name
that he claimed took 17 years.

``This is what we will bring it down to --- this is less than two
years,'' Mr. Trump said, dropping the paper to the ground and revealing
a new flow chart about a quarter of the size.

\href{https://www.nytimes3xbfgragh.onion/section/climate?action=click\&pgtype=Article\&state=default\&region=MAIN_CONTENT_1\&context=storylines_keepup}{}

\hypertarget{climate-and-environment-}{%
\subsubsection{Climate and Environment
›}\label{climate-and-environment-}}

\hypertarget{keep-up-on-the-latest-climate-news}{%
\paragraph{Keep Up on the Latest Climate
News}\label{keep-up-on-the-latest-climate-news}}

Updated Aug. 18, 2020

Here's what you need to know this week:

\begin{itemize}
\item
  \begin{itemize}
  \tightlist
  \item
    Five automakers
    \href{https://www.nytimes3xbfgragh.onion/2020/08/17/climate/california-automakers-pollution.html?action=click\&pgtype=Article\&state=default\&region=MAIN_CONTENT_1\&context=storylines_keepup}{sealed
    a binding agreement} with California to follow the state's stricter
    tailpipe emissions rules.
  \item
    The Trump
    administration\href{https://www.nytimes3xbfgragh.onion/2020/08/13/climate/trump-methane.html?action=click\&pgtype=Article\&state=default\&region=MAIN_CONTENT_1\&context=storylines_keepup}{eliminated
    a major methane rule}, even as leaks are worsening, in a decision
    that researchers warned ignored science.
  \item
    Climate change leaders said
    \href{https://www.nytimes3xbfgragh.onion/2020/08/12/climate/kamala-harris-environmental-justice.html?action=click\&pgtype=Article\&state=default\&region=MAIN_CONTENT_1\&context=storylines_keepup}{the
    vice-presidential choice of Kamala Harris} signaled that Democrats
    will have a focus on environmental justice.
  \end{itemize}
\end{itemize}

``We used to have the greatest infrastructure anywhere in the world, and
today we're like a third-world country,'' he said. ``No longer will we
allow the infrastructure of our magnificent country to crumble and
decay.''

A key element of the new executive order rolls back standards set by
former President Barack Obama that required the federal government to
account for climate change and sea-level rise when building
infrastructure.

It also puts in place what the White House called a ``one federal
decision policy'' under which one lead federal agency works with others
to complete environmental reviews and other permitting decisions for a
given project. All decisions on federal permits will have to be made
within 90 days, and agencies will have a two-year goal to process
environmental reviews for major projects.

``It's going to be a very streamlined process, and by the way, if it
doesn't meet environmental safeguards, we're not going to approve it,''
Mr. Trump said.

\includegraphics{https://static01.graylady3jvrrxbe.onion/images/2017/08/16/science/16FLOODsub/16FLOOD-articleLarge.jpg?quality=75\&auto=webp\&disable=upscale}

Some Republicans cheered the executive order.
\href{https://robbishop.house.gov/}{Representative Rob Bishop of Utah}
called it a strong foundation upon which Congress can move forward an
infrastructure bill. ``It's encouraging to have a president who
understands that regulatory reform is a precondition for any successful
infrastructure policy,'' he said in a statement.

Rolling back the
\href{https://obamawhitehouse.archives.gov/the-press-office/2015/01/30/executive-order-establishing-federal-flood-risk-management-standard-and-}{Federal
Flood Risk Management Standard}, established by Mr. Obama in an
executive order in 2015, is the latest effort by the Trump
administration to unravel the former president's climate change agenda.
Building trade groups and Republican lawmakers had criticized the rule
as costly and overly burdensome.

But environmental activists, flood plain managers and some conservatives
had urged the Trump administration to preserve it, arguing that it
protected critical infrastructure and taxpayer dollars.

``The Trump administration's decision to overturn this is a disaster for
taxpayers and the environment,'' said
\href{http://www.rstreet.org/people/eli-lehrer/}{Eli Lehrer}, president
of the R Street Institute, a free-market think tank in Washington. He
described the Obama order as a common-sense measure to prevent taxpayer
money from being sunk into projects threatened by flooding.

A White House official said that Mr. Trump's order reinstated the
previous flood management standard, issued by President Jimmy Carter in
1977, but that it did not prohibit state and local agencies from using
more stringent standards if they chose.

\href{https://www.nytimes3xbfgragh.onion/interactive/2017/05/18/climate/antarctica-ice-melt-climate-change-flood.html}{}

\includegraphics{https://static01.graylady3jvrrxbe.onion/images/2017/05/18/climate/antarctica-image-promo-chapter-2/antarctica-image-promo-chapter-2-square640.png}

\hypertarget{looming-floods-threatened-cities}{%
\subsection{Looming Floods, Threatened
Cities}\label{looming-floods-threatened-cities}}

Antarctica's potential collapse could damage coastal cities across the
globe.

The Obama-era rule gave federal agencies three options to flood-proof
new infrastructure projects. They could use the best available climate
change science; they could require that standard projects like roads and
railways be built two feet above the national 100-year flood elevation
standard and critical buildings like hospitals be built three feet
higher; or they could require infrastructure to be built to at least the
500-year flood plain. The order did not regulate private development.

\href{https://abraham.house.gov/}{Representative Ralph Abraham of
Louisiana}, a Republican who sponsored legislation that would have
blocked Mr. Obama's flood standard, said he was thrilled by Mr. Trump's
decision. He acknowledged that Louisiana was inundated with catastrophic
flooding last year, but called it an isolated event. The bigger threat,
he said, is from costly regulations.

He estimated the rule would have increased the cost of a new home by 25
percent to 30 percent in Louisiana because most of the state would be
put in a federal flood plain.

``We had more than our share of tragedy down here with the water, but we
already have problems meeting requirements,'' Mr. Abraham said. ``The
new plan would make it so costly for my Louisiana residents.''

The Obama administration had estimated that the more stringent standards
would increase construction costs by 0.25 percent to 1.25 percent, but
save taxpayers money in the long run.

\href{https://curbelo.house.gov/}{Representative Carlos Curbelo}, a
Florida Republican who has called for addressing the threat posed by
climate change, criticized Mr. Trump's decision.

``This executive order is not fiscally conservative,'' he said in a
statement. ``It's irresponsible and it will lead to taxpayer dollars
being wasted on projects that may not be built to endure the flooding we
are already seeing and know is only going to get worse.''

Advertisement

\protect\hyperlink{after-bottom}{Continue reading the main story}

\hypertarget{site-index}{%
\subsection{Site Index}\label{site-index}}

\hypertarget{site-information-navigation}{%
\subsection{Site Information
Navigation}\label{site-information-navigation}}

\begin{itemize}
\tightlist
\item
  \href{https://help.nytimes3xbfgragh.onion/hc/en-us/articles/115014792127-Copyright-notice}{©~2020~The
  New York Times Company}
\end{itemize}

\begin{itemize}
\tightlist
\item
  \href{https://www.nytco.com/}{NYTCo}
\item
  \href{https://help.nytimes3xbfgragh.onion/hc/en-us/articles/115015385887-Contact-Us}{Contact
  Us}
\item
  \href{https://www.nytco.com/careers/}{Work with us}
\item
  \href{https://nytmediakit.com/}{Advertise}
\item
  \href{http://www.tbrandstudio.com/}{T Brand Studio}
\item
  \href{https://www.nytimes3xbfgragh.onion/privacy/cookie-policy\#how-do-i-manage-trackers}{Your
  Ad Choices}
\item
  \href{https://www.nytimes3xbfgragh.onion/privacy}{Privacy}
\item
  \href{https://help.nytimes3xbfgragh.onion/hc/en-us/articles/115014893428-Terms-of-service}{Terms
  of Service}
\item
  \href{https://help.nytimes3xbfgragh.onion/hc/en-us/articles/115014893968-Terms-of-sale}{Terms
  of Sale}
\item
  \href{https://spiderbites.nytimes3xbfgragh.onion}{Site Map}
\item
  \href{https://help.nytimes3xbfgragh.onion/hc/en-us}{Help}
\item
  \href{https://www.nytimes3xbfgragh.onion/subscription?campaignId=37WXW}{Subscriptions}
\end{itemize}
