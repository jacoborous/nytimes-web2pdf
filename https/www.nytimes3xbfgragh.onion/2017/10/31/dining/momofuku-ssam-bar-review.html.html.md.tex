Sections

SEARCH

\protect\hyperlink{site-content}{Skip to
content}\protect\hyperlink{site-index}{Skip to site index}

\href{https://www.nytimes3xbfgragh.onion/section/food}{Food}

\href{https://myaccount.nytimes3xbfgragh.onion/auth/login?response_type=cookie\&client_id=vi}{}

\href{https://www.nytimes3xbfgragh.onion/section/todayspaper}{Today's
Paper}

\href{/section/food}{Food}\textbar{}Momofuku Ssam Bar Keeps Evolving
Under a Singaporean Chef

\url{https://nyti.ms/2z1s8Kp}

\begin{itemize}
\item
\item
\item
\item
\item
\item
\end{itemize}

Advertisement

\protect\hyperlink{after-top}{Continue reading the main story}

Supported by

\protect\hyperlink{after-sponsor}{Continue reading the main story}

\href{/column/restaurant-review}{Restaurant Review}

\hypertarget{momofuku-ssam-bar-keeps-evolving-under-a-singaporean-chef}{%
\section{Momofuku Ssam Bar Keeps Evolving Under a Singaporean
Chef}\label{momofuku-ssam-bar-keeps-evolving-under-a-singaporean-chef}}

\href{https://www.nytimes3xbfgragh.onion/slideshow/2017/10/31/dining/momofuku-ssam-bar-nyc.html}{}

\hypertarget{an-early-david-chang-hit-changes-direction}{%
\subsection{An Early David Chang Hit Changes
Direction}\label{an-early-david-chang-hit-changes-direction}}

6 Photos

View Slide Show ›

\includegraphics{https://static01.graylady3jvrrxbe.onion/images/2017/11/01/dining/01RESTMOMOFUKU-slide-TFQ8/01RESTMOMOFUKU-slide-TFQ8-articleLarge.jpg?quality=75\&auto=webp\&disable=upscale}

Benjamin Norman for The New York Times

\begin{itemize}
\tightlist
\item
  Momofuku Ssam Bar\\
  ★★★ American;Korean;Southeast Asian \$\$\$ 207 Second Avenue
  212-254-3500
\end{itemize}

\href{https://www.opentable.com/single.aspx?ref=4201\&rid=275446}{Reserve
a Table}

When you make a reservation at an independently reviewed restaurant
through our site, we earn an affiliate commission.

By \href{http://www.nytimes3xbfgragh.onion/by/pete-wells}{Pete Wells}

\begin{itemize}
\item
  Oct. 31, 2017
\item
  \begin{itemize}
  \item
  \item
  \item
  \item
  \item
  \item
  \end{itemize}
\end{itemize}

When I see something wrapped in a banana leaf, I am almost always
interested in what's inside. Unwrapping one such package at
\href{https://ssambar.momofuku.com/}{Momofuku Ssam Bar} recently I was
extra interested, because the restaurant has a new chef named Max Ng who
is a citizen of Singapore, and one of the dishes he has brought with him
from that country is a whole skate wing cooked in a banana leaf.

The wing was rubbed top and bottom with sambal belacan, a Malaysian
sauce built on a foundation of fermented-shrimp paste. It was a little
spicy and strongly, disorientingly funky in a way that gave me the
feeling of trying to stand up in a small boat in high surf. On the side,
Mr. Ng served a bowl of congee to act as a kind of buffer, although it
was by no means bland.

Under the orange belacan I could taste the green-tea flavor of the
banana leaf, drawn out by steam. Mr. Ng had intensified that, too, by
charring the leaf so aggressively that it was still smoldering when it
came to the table.

Smoke in restaurants is usually found in captivity these days, caged
within a glass or a cloche or a bag and carefully liberated, to oohs and
aahs, before it drifts toward the ceiling and is gone. This is a
favorite modernist party trick that anyone can do with the right smoker
and pump. It is kind of cool the first time you see it.

The smoke rising from the banana-leaf skate was the old, Promethean
kind, and if I'm reading its message correctly it signaled another shift
in direction for Ssam Bar. Founded by David Chang in the East Village in
2006, the restaurant evolved so quickly in its first two years that it
took three New York Times reviews just to keep up.

In the very beginning, Ssam Bar sold three kinds of Korean burritos.
Dana Bowen,
\href{http://www.nytimes3xbfgragh.onion/2006/10/25/dining/reviews/25unde.html}{writing
for The Times's cheap-eats column,} called them ``enjoyable enough,''
then spent the rest of her space on the after-hours experiments that Mr.
Chang and three other cooks were serving late at night, like a whole
roasted pork butt, to be pulled apart with tongs that quickly become as
joyously greasy as everything and everybody else at the table.

By early 2007,
\href{http://www.nytimes3xbfgragh.onion/2007/02/21/dining/reviews/21rest.html}{when
Frank Bruni weighed in}, the after-hours experiments had taken over the
menu, which offered previously unimagined combinations of kimchi,
country ham, hamachi, rice sticks, fish sauce and organ meats. Ssam Bar
had become ``a nearly full-fledged restaurant in near-perfect sync with
the times,'' Mr. Bruni wrote. Apparently it wasn't done fledging yet; he
returned in 2008 to
\href{http://www.nytimes3xbfgragh.onion/2008/12/03/dining/reviews/03rest.html}{promote
it from two stars to three}.

Review overkill? No. Ssam Bar wasn't just changing its menu. It rewrote
the rules by which critically acclaimed restaurants were supposed to
operate, stripping away comforts (chairs with backs, sound systems with
a ``low'' setting) and amenities (reservations, unshared tables), and
gambling that everybody would be too stunned by the food to complain.

It worked. Ssam Bar mounted a guerrilla attack on the dining
establishment, and it won before anybody quite knew what was going on,
Mr. Chang included.

\includegraphics{https://static01.graylady3jvrrxbe.onion/images/2017/11/01/dining/01RESTMomofuku3/01RESTMomofuku3-articleLarge.jpg?quality=75\&auto=webp\&disable=upscale}

There followed a long period of refinement and maturation, which is more
than some guerrilla outfits can say. The wine list now takes more than
20 seconds to read, and is especially worth exploring for fans of gamay
and sparkling wine. (Or both --- there's a sparkling gamay.) Under
Matthew Rudofker, the executive chef from 2010 until May, Ssam Bar
expanded beyond pork shoulder to other sizable cuts of meat: rib-eyes,
briskets, whole ducks.

Over the years, like many of Mr. Chang's restaurants, Ssam Bar was
increasingly infiltrated by modern techniques, fermentation,
cutting-edge plating styles and umami-building tricks. The kitchen
worked at a very high level, but at times it felt slightly deracinated,
as if it drew much of its inspiration from other islands in
\href{https://momofuku.com/}{the Momofuku archipelago}.

Carefully but confidently, Mr. Ng is moving away from that style and
toward one of his own. He has gone back to the traditions of Asia,
particularly street food and Singaporean cuisine. Like the char on that
smoldering banana leaf, his techniques tend to be premodern, even though
he has worked for nobody but Mr. Chang since moving to the United States
in 2011, starting with an externship at Ssam Bar during culinary school
and rising to chef de cuisine at
\href{https://www.nytimes3xbfgragh.onion/2015/10/14/dining/restaurant-review-momofuku-ko-east-village.html?_r=0}{Momofuku
Ko}.

Mr. Ng has been trying his hand at the fish-shaped cakes from Japan
known as taiyaki. He has a terrific idea for what to do with them: stuff
them with foie gras. In his hands they're almost French, because he
fills the molds with croissant dough rather than pancake batter and
glazes the crust with honey and white port. Over the top he strews some
candied puffed rice that would make a fine breakfast; the whole dish
would make a fine breakfast, come to think of it, although you'd stand
up afterward knowing your day had already peaked.

Sizzling eggs arrive not quite set in a cast-iron pan hot enough to
finish cooking them at the table. The heat also softens a slice of pork
terrine, which melts to coat the eggs. When I first tried this
open-faced omelet, the eggs had been poured around a handful of
chanterelles. Now they've been replaced by smoked bluefish.

To make one of the most original corn dishes I've seen in a long time,
Mr. Ng cleaves the cobs lengthwise into quarters and fries them until
they curl. They are dusted with spice. You pick one up with your fingers
and swipe it through a black aioli of squid ink or a white streak of
whipped ricotta, which will melt like butter. Then you eat the kernels
from the cob as if you were gnawing the meat off a baby back rib.

These new dishes live alongside some legacies of Mr. Rudofker's regime.
There are cured sardines on long fingers of toast spread with hozon, the
fermented chickpea paste that in this case plays the role of butter very
well.

Big cuts of meat are still offered. They have to be ordered ahead, but a
variation of the rotisserie duck can sometimes be had on the spur of the
moment. The breast is rubbed heartily with five-spice powder and fanned
out over rice, the idea being to wrap it in lettuce with fried shallots
and gochujang, without getting too distracted when the duck's fried
bones show up about 15 minutes later.

As if he had realized that a restaurant that acts like a teenage punk is
not as endearing once it reaches middle age, Mr. Chang redid Ssam Bar
last year. The seats and bar stools have backs. The communal tables have
been replaced by smaller ones where you sit only with people you know.
Somehow the noise has been tamed.

You won't have tinnitus by the time you get to Mr. Ng's Singaporean
coconut pie, which has a smooth, sweet, wobbly, pandan-scented filling
like chess pie, a dome of whipped cream made from coconut milk and a
dark brown drizzle of coconut-sugar caramel.

\emph{\href{https://www.facebookcorewwwi.onion/nytfood/}{Follow NYT Food
on Facebook},} \emph{\href{https://instagram.com/nytfood}{Instagram},}
\emph{\href{https://twitter.com/nytfood}{Twitter}} \emph{and}
\emph{\href{https://www.pinterest.com/nytfood/}{Pinterest}.}
\emph{\href{https://www.nytimes3xbfgragh.onion/newsletters/cooking}{Get
regular updates from NYT Cooking, with recipe suggestions, cooking tips
and shopping advice}.}

Advertisement

\protect\hyperlink{after-bottom}{Continue reading the main story}

\hypertarget{site-index}{%
\subsection{Site Index}\label{site-index}}

\hypertarget{site-information-navigation}{%
\subsection{Site Information
Navigation}\label{site-information-navigation}}

\begin{itemize}
\tightlist
\item
  \href{https://help.nytimes3xbfgragh.onion/hc/en-us/articles/115014792127-Copyright-notice}{©~2020~The
  New York Times Company}
\end{itemize}

\begin{itemize}
\tightlist
\item
  \href{https://www.nytco.com/}{NYTCo}
\item
  \href{https://help.nytimes3xbfgragh.onion/hc/en-us/articles/115015385887-Contact-Us}{Contact
  Us}
\item
  \href{https://www.nytco.com/careers/}{Work with us}
\item
  \href{https://nytmediakit.com/}{Advertise}
\item
  \href{http://www.tbrandstudio.com/}{T Brand Studio}
\item
  \href{https://www.nytimes3xbfgragh.onion/privacy/cookie-policy\#how-do-i-manage-trackers}{Your
  Ad Choices}
\item
  \href{https://www.nytimes3xbfgragh.onion/privacy}{Privacy}
\item
  \href{https://help.nytimes3xbfgragh.onion/hc/en-us/articles/115014893428-Terms-of-service}{Terms
  of Service}
\item
  \href{https://help.nytimes3xbfgragh.onion/hc/en-us/articles/115014893968-Terms-of-sale}{Terms
  of Sale}
\item
  \href{https://spiderbites.nytimes3xbfgragh.onion}{Site Map}
\item
  \href{https://help.nytimes3xbfgragh.onion/hc/en-us}{Help}
\item
  \href{https://www.nytimes3xbfgragh.onion/subscription?campaignId=37WXW}{Subscriptions}
\end{itemize}
