Sections

SEARCH

\protect\hyperlink{site-content}{Skip to
content}\protect\hyperlink{site-index}{Skip to site index}

\href{https://www.nytimes3xbfgragh.onion/section/food}{Food}

\href{https://myaccount.nytimes3xbfgragh.onion/auth/login?response_type=cookie\&client_id=vi}{}

\href{https://www.nytimes3xbfgragh.onion/section/todayspaper}{Today's
Paper}

\href{/section/food}{Food}\textbar{}The Extras Are the Stars on the
Korean Menu at Atoboy

\url{https://nyti.ms/2ps1Hap}

\begin{itemize}
\item
\item
\item
\item
\item
\item
\end{itemize}

Advertisement

\protect\hyperlink{after-top}{Continue reading the main story}

Supported by

\protect\hyperlink{after-sponsor}{Continue reading the main story}

\href{/column/restaurant-review}{Restaurant Review}

\hypertarget{the-extras-are-the-stars-on-the-korean-menu-at-atoboy}{%
\section{The Extras Are the Stars on the Korean Menu at
Atoboy}\label{the-extras-are-the-stars-on-the-korean-menu-at-atoboy}}

\href{https://www.nytimes3xbfgragh.onion/slideshow/2017/05/09/dining/atoboy-nyc-review.html}{}

\hypertarget{atoboy}{%
\subsection{Atoboy}\label{atoboy}}

10 Photos

View Slide Show ›

\includegraphics{https://static01.graylady3jvrrxbe.onion/images/2017/05/10/dining/10REST-ATOBOY-slide-MTNI/10REST-ATOBOY-slide-MTNI-articleLarge.jpg?quality=75\&auto=webp\&disable=upscale}

Linda Xiao for The New York Times

\begin{itemize}
\tightlist
\item
  Atoboy\\
  ★★ Korean \$\$ 43 East 28th Street 646-476-7217
\end{itemize}

\href{https://resy.com/cities/ny/atoboy?utm_source=nyt\&utm_medium=restoprofile\&utm_campaign=affiliates\&aff_id=c1fe784}{Reserve
a Table}

When you make a reservation at an independently reviewed restaurant
through our site, we earn an affiliate commission.

By \href{http://www.nytimes3xbfgragh.onion/by/pete-wells}{Pete Wells}

\begin{itemize}
\item
  May 9, 2017
\item
  \begin{itemize}
  \item
  \item
  \item
  \item
  \item
  \item
  \end{itemize}
\end{itemize}

When all of the little plates of pickles and vegetables and whatnot land
on the table at the start of a Korean meal, it's like winning a game
show. They are called banchan, and they shower down out of nowhere, like
balloons. Fermented daikon and wavy cabbage leaves, pink with chile
paste. Pinwheels of soy-stained lotus root. Silvery, crunchy stir-fried
anchovies no bigger than paper matches.

If I'm in a restaurant, I pay close attention to these dishes. The
better they are, the higher my hopes for the rest of the meal. They
serve as a quick reminder from the outset of how varied Korean cuisine
can be, how it can pack both nuance and power. And, of course, I like
them because at most Korean restaurants they're free.

It was my strong feelings about banchan that made me suspicious when I
first heard that a new restaurant called
\href{http://atoboynyc.com/}{Atoboy} had made them the foundation of its
menu. What could this be but another ploy to charge us for something
that used to come with the cost of the meal? We've lost the battle over
bread. I was ready to draw a line in the sand over banchan.

After three dinners at Atoboy (ah-TOE-boy) since last fall and one quick
snack at the bar, I'm happy to admit that I was wrong. This is no scam.
Junghyun Park, the chef, uses banchan as a starting point, then goes on
to build smart, surprising dishes around it.

Look at what he does with those pinwheels of lotus root, laying them
over a bed of tofu in soft, tiny curds like ricotta. It's a great dish,
half crunchy and half creamy, both halves humming along under a dressing
of chile oil and sesame oil.

Fans of the barbecue at
\href{https://www.nytimes3xbfgragh.onion/2014/12/17/dining/critics-notebook-pete-wells-explores-korean-restaurants-in-queens.html?_r=0}{Kang
Ho Dang Baekjeong} will remember the pan of corn in melted cheese that
sits at the edge of the domed grill and is basted by the hot drippings
that slide down from the sizzling meat. Mr. Park has a wonderful take on
this, using bacon for its fat and taleggio for its tang, mixing in the
fermented soybean paste called doenjang to make everything more intense.
When a bowl of it showed up, it got side-eyes at my table. Corn out of
season? Nobody who tasted it stayed skeptical.

Mr. Park, 32, was raised in Seoul. Bypassing culinary school, he went on
a grand tour and learned to cook at restaurants in England and
Australia. At the end, he returned to his hometown to work in the
kitchen at Jungsik, a westernized Korean restaurant with global
aspirations. When
\href{http://www.nytimes3xbfgragh.onion/2012/02/29/dining/reviews/jungsik-intribeca-reinterprets-korean-cuisine.html}{Jungsik}
set up an outpost in Lower Manhattan, Mr. Park was dispatched to serve
as its chef de cuisine.

Finally ready to go out on his own, he opened Atoboy on East 28th Street
last summer with his wife, Ellia, a veteran of the dining rooms at
\href{https://www.nytimes3xbfgragh.onion/2015/06/24/dining/restaurant-reviews-noreetuh-the-eddy-in-the-east-village.html}{Noreetuh}
and
\href{http://www.nytimes3xbfgragh.onion/2010/01/20/dining/reviews/20rest.html}{Maialino}.
She manages the restaurant.

\includegraphics{https://static01.graylady3jvrrxbe.onion/images/2017/05/10/dining/10REST1-1494029307033/10REST1-1494029307033-articleInline.jpg?quality=75\&auto=webp\&disable=upscale}

Like Jungsik, Atoboy favors the subtler end of the Korean flavor
spectrum, but it avoids fussing and tweezing. Mr. Park's novel and
lovely sea bass tartare, under a shimmering pale-green layer of chopped
kiwi, mint and fermented spring garlic, could easily slide onto the menu
at Jungsik if he decked out the plate with some foofaraw. (I'm glad he
doesn't.)

The low-key atmosphere that the Parks have set down, though, has little
in common with the carpeted environs of Jungsik, which affects a
formality so stiff around the spine that the servers might be wearing
back braces. Atoboy is also a deliberate step away from the controlled
chaos of the Korean restaurants a few blocks north on 32nd Street.
Between its patched walls of raw concrete, simple wooden tables are
arranged in two symmetrical rows with a narrow path to the kitchen down
the middle.

It looks like a downtown wine bar, and in fact wine is one of the chief
attractions. The list is brief but manages to take you to places you
wouldn't expect to visit in a Korean restaurant. There's a concentration
of bottles from Nikolaihof, Gut Oggau and other fine Austrian producers;
a small flock of Loire winemakers such as Olga Raffault; Champagnes from
Charles Ellner and other independent labels; and some respected West
Coast names like Pedroncelli and Dirty \& Rowdy. Everything I've tried
has cohabited amicably with Mr. Park's food.

Although all 18 dishes on the menu can be ordered on their own, servers
suggest a \$36 meal made of one from each of three categories ---
roughly, cold salads, warm appetizers and plates where seafood or meat
do most of the work. They're served more or less in that order but
without pauses between the courses, and the plates pile up in the middle
of the table.

This makes it too easy for nuanced dishes, like fried and braised
sunchokes with oyster mushrooms and oranges dressed in a creamy truffle
emulsion, to get lost. The servers insist that everything is meant for
sharing. Heard that one before? But some dishes, like the delicate egg
custard in a smoky dashi with morels and soybeans, were worse for wear
and tear after they were doled out to individual plates.

It's almost impossible to get worked up about any of this, though,
because the \$36 menu is such a fair deal, and because so many of the
dishes work so well and do things you didn't know Korean food could do.

Mr. Park's take on yuk hwe is memorable, the beef tartare cut into long
skinny threads that are tossed with soy sauce, dotted with nettle cream
and topped with what seem to be julienne potato chips. Braised mackerel,
usually cooked in soy sauce, is simmered in a complex green-chile broth.
Little drums of octopus leg, a dot of parsley oil in the center of each
one, are ringed around a kind of hash of kimchi and chorizo --- two
things that were meant to be together, though I never would have
guessed.

After these shifting, swirling flavors, the desserts have a clarifying
effect. The ginger panna cotta with pink grapefruit is just as
refreshing as it sounds. There's also one that uses
\href{https://mykoreankitchen.com/sujeonggwa-korean-cinnamon-punch/}{sujeonggwa},
the sweet cinnamon punch ladled out at the end of Korean meals. It's
frozen into a granita and then spooned over burrata, lychee yogurt and
candied walnuts for a cheese course you won't find anywhere but Atoboy.

\href{https://www.facebookcorewwwi.onion/nytfood/}{\emph{Follow NYT Food
on Facebook}}\emph{,}
\href{https://instagram.com/nytfood}{\emph{Instagram}}\emph{,}
\href{https://twitter.com/nytfood}{\emph{Twitter}} \emph{and}
\href{https://www.pinterest.com/nytfood/}{\emph{Pinterest}}\emph{.}
\href{https://www.nytimes3xbfgragh.onion/newsletters/cooking}{\emph{Get
regular updates from NYT Cooking, with recipe suggestions, cooking tips
and shopping advice}}\emph{.}

Advertisement

\protect\hyperlink{after-bottom}{Continue reading the main story}

\hypertarget{site-index}{%
\subsection{Site Index}\label{site-index}}

\hypertarget{site-information-navigation}{%
\subsection{Site Information
Navigation}\label{site-information-navigation}}

\begin{itemize}
\tightlist
\item
  \href{https://help.nytimes3xbfgragh.onion/hc/en-us/articles/115014792127-Copyright-notice}{©~2020~The
  New York Times Company}
\end{itemize}

\begin{itemize}
\tightlist
\item
  \href{https://www.nytco.com/}{NYTCo}
\item
  \href{https://help.nytimes3xbfgragh.onion/hc/en-us/articles/115015385887-Contact-Us}{Contact
  Us}
\item
  \href{https://www.nytco.com/careers/}{Work with us}
\item
  \href{https://nytmediakit.com/}{Advertise}
\item
  \href{http://www.tbrandstudio.com/}{T Brand Studio}
\item
  \href{https://www.nytimes3xbfgragh.onion/privacy/cookie-policy\#how-do-i-manage-trackers}{Your
  Ad Choices}
\item
  \href{https://www.nytimes3xbfgragh.onion/privacy}{Privacy}
\item
  \href{https://help.nytimes3xbfgragh.onion/hc/en-us/articles/115014893428-Terms-of-service}{Terms
  of Service}
\item
  \href{https://help.nytimes3xbfgragh.onion/hc/en-us/articles/115014893968-Terms-of-sale}{Terms
  of Sale}
\item
  \href{https://spiderbites.nytimes3xbfgragh.onion}{Site Map}
\item
  \href{https://help.nytimes3xbfgragh.onion/hc/en-us}{Help}
\item
  \href{https://www.nytimes3xbfgragh.onion/subscription?campaignId=37WXW}{Subscriptions}
\end{itemize}
