Sections

SEARCH

\protect\hyperlink{site-content}{Skip to
content}\protect\hyperlink{site-index}{Skip to site index}

\href{https://www.nytimes3xbfgragh.onion/section/world/middleeast}{Middle
East}

\href{https://myaccount.nytimes3xbfgragh.onion/auth/login?response_type=cookie\&client_id=vi}{}

\href{https://www.nytimes3xbfgragh.onion/section/todayspaper}{Today's
Paper}

\href{/section/world/middleeast}{Middle East}\textbar{}\$110 Billion
Weapons Sale to Saudis Has Jared Kushner's Personal Touch

\url{https://nyti.ms/2qyZYnw}

\begin{itemize}
\item
\item
\item
\item
\item
\item
\end{itemize}

Advertisement

\protect\hyperlink{after-top}{Continue reading the main story}

Supported by

\protect\hyperlink{after-sponsor}{Continue reading the main story}

\hypertarget{110-billion-weapons-sale-to-saudis-has-jared-kushners-personal-touch}{%
\section{\$110 Billion Weapons Sale to Saudis Has Jared Kushner's
Personal
Touch}\label{110-billion-weapons-sale-to-saudis-has-jared-kushners-personal-touch}}

\includegraphics{https://static01.graylady3jvrrxbe.onion/images/2017/05/19/world/19arms/19arms-articleInline.jpg?quality=75\&auto=webp\&disable=upscale}

By \href{http://www.nytimes3xbfgragh.onion/by/mark-landler}{Mark
Landler}, \href{http://www.nytimes3xbfgragh.onion/by/eric-schmitt}{Eric
Schmitt} and
\href{http://www.nytimes3xbfgragh.onion/by/matt-apuzzo}{Matt Apuzzo}

\begin{itemize}
\item
  May 18, 2017
\item
  \begin{itemize}
  \item
  \item
  \item
  \item
  \item
  \item
  \end{itemize}
\end{itemize}

WASHINGTON --- On the afternoon of May 1, President Trump's son-in-law,
Jared Kushner, welcomed a high-level delegation of Saudis to a gilded
reception room next door to the White House and delivered a brisk pep
talk: ``Let's get this done today.''

Mr. Kushner was referring to a \$100 billion-plus arms deal that the
administration hoped to seal with Saudi Arabia in time to announce it
during
\href{https://www.nytimes3xbfgragh.onion/2017/05/04/us/politics/trump-to-visit-saudi-arabia-and-israel-in-first-foreign-trip.html}{Mr.
Trump's visit to the kingdom} this weekend. The two sides discussed a
shopping list that included planes, ships and precision-guided bombs.
Then an American official raised the idea of the Saudis' buying a
sophisticated radar system designed to shoot down ballistic missiles.

Sensing that the cost might be a problem, several administration
officials said, Mr. Kushner picked up the phone and called Marillyn A.
Hewson --- the chief executive of Lockheed Martin, which makes the radar
system --- and asked her whether she could cut the price. As his guests
watched slack-jawed, Ms. Hewson told him she would look into it,
officials said.

Mr. Kushner's personal intervention in the arms sale is further evidence
of the Trump White House's readiness to dispense with custom in favor of
informal, hands-on deal making. It also offers a window into how the
administration hopes to change America's position in the Middle East,
emphasizing hard power and haggling over traditional diplomacy.

The Trump administration is expected to frame the deal, worth about
\$110 billion over 10 years, as a symbol of America's renewed commitment
to security in the Persian Gulf. But former officials pointed out that
President Barack Obama, whose arms sales to Saudi Arabia totaled \$115
billion, had already approved several of the weapons in the package.

``Both sides have an incentive to herald this as a new era in Gulf
cooperation,'' said Derek H. Chollet, who served as assistant secretary
of defense for international security affairs under Mr. Obama. ``I see
this as largely continuity.''

\includegraphics{https://static01.graylady3jvrrxbe.onion/images/2017/05/19/world/middleeast/19saudi-arms-vid-pic/19saudi-arms-vid-pic-videoSixteenByNine3000.jpg}

What has changed, Mr. Chollet said, is that the House of Saud is now
dealing directly with a member of the Trump family. ``It's quite normal
for them to sit down with the son-in-law of a president and do a deal,''
he said. ``It's more normal for them than any previous administration.''

The White House and Lockheed declined to comment on the call between Mr.
Kushner and Ms. Hewson, or on the broader arms sale.

While Mr. Kushner's middle-of-the-meeting call to a military contractor
was unorthodox, current and former officials said, it did not appear to
raise legal issues. Lockheed is the prime contractor for the antimissile
system, known as Terminal High Altitude Area Defense, or Thaad. Instead,
the episode was reminiscent of
\href{https://www.nytimes3xbfgragh.onion/2017/02/03/business/lockheed-lowers-price-on-f-35-fighters-after-prodding-by-trump.html}{Lockheed's
decision in February} to cut the price of F-35 fighter jets it was
selling to the Pentagon after Mr. Trump complained to Ms. Hewson that
the planes were too expensive.

Mr. Kushner, White House officials said, began building ties to members
of the Saudi royal family during the transition. He was at the table
when his father-in-law
\href{https://www.nytimes3xbfgragh.onion/2017/03/14/world/middleeast/mohammed-bin-salman-saudi-arabia-trump.html}{hosted
the deputy crown prince}, Mohammed bin Salman, at a lunch in the State
Dining Room in March. And he offered a strategic overview of the
Saudi-American relationship at the meeting this month, according to an
agenda obtained by The New York Times.

But officials emphasized that Mr. Kushner's work on the deal was part of
a governmentwide effort that includes the State Department, the Defense
Department and the National Security Council.

They also said the arms sale would be only one element of Mr. Trump's
busy two-day stop in Saudi Arabia, which will also include a meeting
with King Salman at the royal court, a conference with Persian Gulf
allies, a broader summit meeting with the leaders of Muslim countries,
and a visit to a new center dedicated to combating terrorism and
extremism.

The showcase event will be a speech in which the officials said Mr.
Trump would seek to unify the Muslim world against the scourge of
extremism. Stephen Miller, Mr. Trump's senior policy adviser, is writing
the speech, which officials said would serve as an answer to the
landmark address to the Islamic world that
\href{http://www.nytimes3xbfgragh.onion/2009/06/05/world/middleeast/05prexy.html}{Mr.
Obama gave in Cairo} in 2009.

White House officials have consulted Mr. Obama's speech and predicted a
starkly different tone from Mr. Trump. His goal, they said, will be to
unify America's allies around a common set of objectives, including a
harder line against Iran and a pledge to share the security burden in
the region. The speech will not include any apology for America's role.

After a strained relationship with Mr. Obama, Saudi officials have
expressed delight at Mr. Trump's tough rhetoric on Iran. This White
House is viewed as more sympathetic to the military campaign that Saudi
Arabia and the United Arab Emirates are carrying out against the
Houthis, Iranian-backed rebels who are waging an insurgency in
neighboring Yemen.

The Obama administration put a hold on precision-guided munitions it had
agreed to sell the Saudis out of fear that they would be used to bomb
civilians in Yemen. The Trump administration has freed up those weapons,
which are part of the \$110 billion package.

The package also includes ``maritime assets,'' meaning ships, so the
Saudis can assume more of the burden of policing the Persian Gulf and
Red Sea against Iranian aggression. It does not include high-end items
like the advanced F-35 fighter, whose sale to Saudi Arabia would alarm
Israel.

Mr. Trump is not expected to raise human rights concerns with the
Saudis, in keeping with his approach to strongmen in Turkey, Egypt,
China and the Philippines. The president, his aides said, does not
believe the United States gets results by lecturing other countries.

Given that, and the big-ticket arms sale, most analysts and former
officials predicted that Mr. Trump's visit to Saudi Arabia would be a
success. It could end up being the highlight of his nine-day,
four-country tour, particularly since he will be going later to a NATO
summit meeting in Brussels, where the other attendees will watch for
evidence that he still wants to mothball the alliance.

Even in Israel, where Mr. Trump is likely to be welcomed with open arms,
tensions have surfaced over his
\href{https://www.nytimes3xbfgragh.onion/2017/05/16/world/middleeast/israel-trump-classified-intelligence-russia.html}{sharing
classified Israeli intelligence} during a meeting with Russia's foreign
minister and ambassador to the United States, and
\href{https://www.nytimes3xbfgragh.onion/2017/05/15/world/middleeast/emirati-prince-trump.html}{a
smaller flap} over the political status of the Western Wall.

Still, the Saudi visit is not without risk. Mr. Obama made Riyadh, the
Saudi capital, his first stop in the Middle East in June 2009, hoping to
enlist the Saudis in a new Israeli-Palestinian peace effort. King
Salman's predecessor, King Abdullah, rebuffed the young president.

For now, the White House
\href{https://www.nytimes3xbfgragh.onion/2017/05/17/us/politics/trump-iran-nuclear-deal.html}{is
not abandoning the Iran nuclear agreement}, which is reviled in Saudi
Arabia. Though experts say the Saudis understand the administration's
reluctance to act precipitously, some critics worry that it will make
Mr. Trump more eager to accommodate the Saudis in other areas, like
their campaign in Yemen.

``We'd been saying for two years that this is not a conflict you're
going to win militarily,'' said Jeffrey Prescott, a senior director for
Iran, Iraq, Syria and Gulf nations on Mr. Obama's National Security
Council. ``We had been trying to use the leverage we had to get the
Saudis and Emiratis to the table to negotiate.''

``One of the things to look at,'' Mr. Prescott added, ``is whether we're
getting into someone else's conflict.''

Advertisement

\protect\hyperlink{after-bottom}{Continue reading the main story}

\hypertarget{site-index}{%
\subsection{Site Index}\label{site-index}}

\hypertarget{site-information-navigation}{%
\subsection{Site Information
Navigation}\label{site-information-navigation}}

\begin{itemize}
\tightlist
\item
  \href{https://help.nytimes3xbfgragh.onion/hc/en-us/articles/115014792127-Copyright-notice}{©~2020~The
  New York Times Company}
\end{itemize}

\begin{itemize}
\tightlist
\item
  \href{https://www.nytco.com/}{NYTCo}
\item
  \href{https://help.nytimes3xbfgragh.onion/hc/en-us/articles/115015385887-Contact-Us}{Contact
  Us}
\item
  \href{https://www.nytco.com/careers/}{Work with us}
\item
  \href{https://nytmediakit.com/}{Advertise}
\item
  \href{http://www.tbrandstudio.com/}{T Brand Studio}
\item
  \href{https://www.nytimes3xbfgragh.onion/privacy/cookie-policy\#how-do-i-manage-trackers}{Your
  Ad Choices}
\item
  \href{https://www.nytimes3xbfgragh.onion/privacy}{Privacy}
\item
  \href{https://help.nytimes3xbfgragh.onion/hc/en-us/articles/115014893428-Terms-of-service}{Terms
  of Service}
\item
  \href{https://help.nytimes3xbfgragh.onion/hc/en-us/articles/115014893968-Terms-of-sale}{Terms
  of Sale}
\item
  \href{https://spiderbites.nytimes3xbfgragh.onion}{Site Map}
\item
  \href{https://help.nytimes3xbfgragh.onion/hc/en-us}{Help}
\item
  \href{https://www.nytimes3xbfgragh.onion/subscription?campaignId=37WXW}{Subscriptions}
\end{itemize}
