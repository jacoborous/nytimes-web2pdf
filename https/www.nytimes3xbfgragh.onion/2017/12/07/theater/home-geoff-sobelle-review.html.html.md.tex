Sections

SEARCH

\protect\hyperlink{site-content}{Skip to
content}\protect\hyperlink{site-index}{Skip to site index}

\href{https://www.nytimes3xbfgragh.onion/section/theater}{Theater}

\href{https://myaccount.nytimes3xbfgragh.onion/auth/login?response_type=cookie\&client_id=vi}{}

\href{https://www.nytimes3xbfgragh.onion/section/todayspaper}{Today's
Paper}

\href{/section/theater}{Theater}\textbar{}Review: Dreaming of `Home,'
With a Magical Guide in His Underwear

\url{https://nyti.ms/2B16RUi}

\begin{itemize}
\item
\item
\item
\item
\item
\item
\end{itemize}

Advertisement

\protect\hyperlink{after-top}{Continue reading the main story}

Supported by

\protect\hyperlink{after-sponsor}{Continue reading the main story}

\hypertarget{review-dreaming-of-home-with-a-magical-guide-in-his-underwear}{%
\section{Review: Dreaming of `Home,' With a Magical Guide in His
Underwear}\label{review-dreaming-of-home-with-a-magical-guide-in-his-underwear}}

\includegraphics{https://static01.graylady3jvrrxbe.onion/images/2017/12/08/arts/08HOME1/08HOME1-articleLarge.jpg?quality=75\&auto=webp\&disable=upscale}

\begin{itemize}
\tightlist
\item
  Next Wave Festival: HOME\\
  Off Broadway, Dance Play, Experimental/Perf. Art, Play 1 hr. and 35
  min. Closing Date: Dec. 10, 2017 Brooklyn Academy of Music - Harvey
  Theater, 651 Fulton St. 718-636-4100
\end{itemize}

By \href{http://www.nytimes3xbfgragh.onion/by/ben-brantley}{Ben
Brantley}

\begin{itemize}
\item
  Dec. 7, 2017
\item
  \begin{itemize}
  \item
  \item
  \item
  \item
  \item
  \item
  \end{itemize}
\end{itemize}

Geoff Sobelle knows where you dream. Even more unsettlingly, he knows
that territory you inhabit on the edge of dreams --- when you're
suddenly half-awake in the middle of the night and aren't sure where you
are.

You're home, right? Or so you try to reassure yourself, as you squint to
make out familiar objects in the dark. But which of the many homes and
way stations you've slept in is this one?

Before your real location has come into focus, you've taken mental
inventory of a whole succession of beds, occupied by different versions
of you, at different ages, perhaps in combination with different
partners.

Yeah, I know, Proust wrote all about this stuff in the opening pages of
\href{http://www.authorama.com/remembrance-of-things-past-1.html}{``Remembrance
of Things Past.''} (``For a long time I used to go to bed early,'' it
begins.) But don't think I've ever seen this particular form of
nocturnal disorientation summoned as evocatively --- on a stage, in real
(unreal) time --- as it is in ``Home,'' Mr. Sobelle's essential new
performance piece at the
\href{https://www.bam.org/physical-theater/2017/home}{Brooklyn Academy
of Music.}

If you have any intention of seeing ``Home,'' which runs only through
Sunday at the Harvey Theater, you might want to stop reading now. Part
of the effectiveness of this latest offering from Mr. Sobelle, whose
earlier credits include the inspired installation work
\href{https://www.nytimes3xbfgragh.onion/2014/11/07/theater/geoff-sobelles-the-object-lesson-at-bam.html?_r=0}{``The
Object Lesson,''} relies on conjuring tricks.

\includegraphics{https://static01.graylady3jvrrxbe.onion/images/2017/12/08/arts/08HOME3-sub/08HOME4-articleLarge.jpg?quality=75\&auto=webp\&disable=upscale}

This production, directed with a spontaneous air of seamlessness by Lee
Sunday Evans, seems to keep pulling apparitions out of air, just as your
mind does when it's feeling tired and unguarded. That semi-waking
sensation I mentioned above is given full, fluid life early in the show,
and it involves little more than a simple single bed, the middle-aged
Mr. Sobelle and interchangeable alter-egos who include a towheaded boy,
a young woman and an older woman.

The entire sequence lasts maybe five minutes, and yet it feels as if it
covers not just your lifetime but those of at least several other people
as well. And, oh, you know that other unnerving staple of nighttime
fantasies, the one in which you're in a public place in your underwear?

Mr. Sobelle has that one covered (or uncovered), too, as he stands
center stage in his white boxers and T-shirt, modestly draping himself
in sheer plastic tarpaulins, looking both slightly alarmed and supremely
regal. Watching him in such moments, you are sure to feel an embarrassed
empathy for Mr. Sobelle, awash in your own instinctive fears of being on
undignified and unprotected display.

Not to worry, though. Mr. Sobelle will soon have an entire house ---
custom built, room by room, before your astonished eyes --- to shelter
him. But how much of a sanctuary is a house, any house, finally?

Looked at from a longer view, which is how Mr. Sobelle's vision works,
it's just a temporary refuge through which many travelers are probably
destined to pass. As to any illusions you might have about the
permanence of where you lay your hat, well, just remember that anything
that can be assembled can be leveled even more quickly.

I wasn't speaking in metaphors about that house being built onstage. At
the center of Steven Dufala's uncanny set for ``Home'' is a two-story
suburban-style dwelling (with complete kitchen and bathroom). Even
though you watch it being put together, it still seems to materialize of
the shadows, just like the place you once lived with Mom and Dad, as it
shows up in your dreams.

Don't make the mistake of thinking that this house is private property.
This building belongs if not to the ages, at least to several successive
generations of tenants.

Image

Sophie Bortolussi, left, and Jennifer Kidwell, center, in the crowded
kitchen of ``Home.''Credit...Sara Krulwich/The New York Times

Image

Further developments in the construction of ``Home.''Credit...Sara
Krulwich/The New York Times

These folks go about their daily business of brushing their teeth,
taking out the garbage, unclogging the toilet, changing clothes and
putting away the groceries, just as you or I might on an average, boring
day. But they do it in multiples, so that as many as seven people are
inhabiting the house at the same time, performing much the same tasks,
but unaware of one another's existence.

And that's before things get really crowded.

``Home'' admits to a cast of only seven, including Mr. Sobelle. That is
a deceptive number. You are a cast member, too, whether you wind up on
the stage or not. (And be warned: that is a possibility, but nothing
that involves you in your underwear.)

A highly skilled creative team, which includes Christopher Kuhl
(lighting) and Brandon Wolcott (sound), extends the borders of this
work's title property through subliminal sensory effects. After all, as
Mr. Sobelle points out in a written introduction in the program, there's
a reason that theaters are referred to as houses; they are places where
we settle in for a spell, as occupants and owners of seats we
presumptuously think of as ``ours.''

If you are not the first person ever to live where you are living now,
``Home'' is guaranteed to elicit a familiar sense of being haunted.
Surely on some level, conscious or not, you've thought about the
existences that preceded yours in this spot, and felt both their weight
and their ephemerality.

Those who like anchors of annotation with their artistic experiences
will be pleased to learn that this production features a alto-harp and
guitar-strumming troubadour in the form of
\href{http://www.elvisperkinssound.net/}{Elvis Perkins,} who shows up to
sing gnomically of the follies of identifying too closely with our
places of residence.

Mr. Perkins has a certain droll charm. But for me, his presence was
superfluous. Mr. Sobelle and company have landscaped their ghost house
so precisely as an of-the-moment phenomenon that no explanation is
required.

And as I looked at the (spoiler) ruins of what was once a sturdy edifice
as the show concluded, I cast a prophetic thought toward them, one I
knew would be fulfilled: ``I'll see you in my dreams.''

Advertisement

\protect\hyperlink{after-bottom}{Continue reading the main story}

\hypertarget{site-index}{%
\subsection{Site Index}\label{site-index}}

\hypertarget{site-information-navigation}{%
\subsection{Site Information
Navigation}\label{site-information-navigation}}

\begin{itemize}
\tightlist
\item
  \href{https://help.nytimes3xbfgragh.onion/hc/en-us/articles/115014792127-Copyright-notice}{©~2020~The
  New York Times Company}
\end{itemize}

\begin{itemize}
\tightlist
\item
  \href{https://www.nytco.com/}{NYTCo}
\item
  \href{https://help.nytimes3xbfgragh.onion/hc/en-us/articles/115015385887-Contact-Us}{Contact
  Us}
\item
  \href{https://www.nytco.com/careers/}{Work with us}
\item
  \href{https://nytmediakit.com/}{Advertise}
\item
  \href{http://www.tbrandstudio.com/}{T Brand Studio}
\item
  \href{https://www.nytimes3xbfgragh.onion/privacy/cookie-policy\#how-do-i-manage-trackers}{Your
  Ad Choices}
\item
  \href{https://www.nytimes3xbfgragh.onion/privacy}{Privacy}
\item
  \href{https://help.nytimes3xbfgragh.onion/hc/en-us/articles/115014893428-Terms-of-service}{Terms
  of Service}
\item
  \href{https://help.nytimes3xbfgragh.onion/hc/en-us/articles/115014893968-Terms-of-sale}{Terms
  of Sale}
\item
  \href{https://spiderbites.nytimes3xbfgragh.onion}{Site Map}
\item
  \href{https://help.nytimes3xbfgragh.onion/hc/en-us}{Help}
\item
  \href{https://www.nytimes3xbfgragh.onion/subscription?campaignId=37WXW}{Subscriptions}
\end{itemize}
