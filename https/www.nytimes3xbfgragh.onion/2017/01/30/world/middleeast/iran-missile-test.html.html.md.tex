Sections

SEARCH

\protect\hyperlink{site-content}{Skip to
content}\protect\hyperlink{site-index}{Skip to site index}

\href{https://www.nytimes3xbfgragh.onion/section/world/middleeast}{Middle
East}

\href{https://myaccount.nytimes3xbfgragh.onion/auth/login?response_type=cookie\&client_id=vi}{}

\href{https://www.nytimes3xbfgragh.onion/section/todayspaper}{Today's
Paper}

\href{/section/world/middleeast}{Middle East}\textbar{}Iran Launches a
Missile, Testing Trump's Vows of Strict Enforcement

\url{https://nyti.ms/2jP6sst}

\begin{itemize}
\item
\item
\item
\item
\item
\item
\end{itemize}

Advertisement

\protect\hyperlink{after-top}{Continue reading the main story}

Supported by

\protect\hyperlink{after-sponsor}{Continue reading the main story}

\hypertarget{iran-launches-a-missile-testing-trumps-vows-of-strict-enforcement}{%
\section{Iran Launches a Missile, Testing Trump's Vows of Strict
Enforcement}\label{iran-launches-a-missile-testing-trumps-vows-of-strict-enforcement}}

By \href{http://www.nytimes3xbfgragh.onion/by/david-e-sanger}{David E.
Sanger}

\begin{itemize}
\item
  Jan. 30, 2017
\item
  \begin{itemize}
  \item
  \item
  \item
  \item
  \item
  \item
  \end{itemize}
\end{itemize}

WASHINGTON --- Iran conducted its first missile test since President
Trump took office, American and Israeli officials said Monday, posing an
early test of whether the Trump administration will make good on its
promises to strictly enforce all aspects of the Iranian nuclear deal and
a side agreement on missile testing.

The Iranian missile traveled about 600 miles, but its re-entry vehicle
reportedly exploded before the flight was complete. It is unclear
whether that was accidental or a deliberate detonation.

Israel's United Nations ambassador, Danny Danon, accused Iran of
violating a Security Council resolution passed in 2015, shortly after
the nuclear accord was reached in Vienna.

While the agreement itself dealt only with Iran's nuclear program,
then-Secretary of State John Kerry negotiated for days with the Iranians
about a revised Security Council resolution on missiles. That last
concession by the United States, China, Russia, Germany and France
sealed the entire diplomatic package.

The previous United Nations resolution had specifically prohibited such
tests, with wording the Iranians were eager to jettison. Under the new
resolution, Iran was ``called upon not to undertake any activity related
to ballistic missiles designed to be capable of delivering nuclear
weapons, including launches using such ballistic missile technology.''

Iran has said that because it does not have a nuclear weapons program,
its missiles are not designed to be able to deliver nuclear warheads.

At the White House on Monday, Sean Spicer, the press secretary,
acknowledged that the missile launch had taken place but said nothing
about how the new administration would react.

``We're aware that Iran fired that missile,'' he said. ``We're looking
into the exact nature of it.''

During the campaign, Mr. Trump often condemned the nuclear deal and at
times conflated Iran's nuclear program and its missile program.

Inside the administration, there seems to be little appetite now to tear
up the 2015 agreement, despite Mr. Trump's criticism of it. The new
defense secretary, Jim Mattis, said during his confirmation hearing that
while he wished the accord would prohibit Iranian nuclear activity for
longer than it does, the world was safer with the agreement.

The missile launch opens up new possibilities for the administration: It
could join Israel in seeking new sanctions that are missile-related,
steering clear of the core nuclear deal. But it is unclear whether there
is any desire for that in the European Union, China or Russia, all
signatories to the accord.

Advertisement

\protect\hyperlink{after-bottom}{Continue reading the main story}

\hypertarget{site-index}{%
\subsection{Site Index}\label{site-index}}

\hypertarget{site-information-navigation}{%
\subsection{Site Information
Navigation}\label{site-information-navigation}}

\begin{itemize}
\tightlist
\item
  \href{https://help.nytimes3xbfgragh.onion/hc/en-us/articles/115014792127-Copyright-notice}{©~2020~The
  New York Times Company}
\end{itemize}

\begin{itemize}
\tightlist
\item
  \href{https://www.nytco.com/}{NYTCo}
\item
  \href{https://help.nytimes3xbfgragh.onion/hc/en-us/articles/115015385887-Contact-Us}{Contact
  Us}
\item
  \href{https://www.nytco.com/careers/}{Work with us}
\item
  \href{https://nytmediakit.com/}{Advertise}
\item
  \href{http://www.tbrandstudio.com/}{T Brand Studio}
\item
  \href{https://www.nytimes3xbfgragh.onion/privacy/cookie-policy\#how-do-i-manage-trackers}{Your
  Ad Choices}
\item
  \href{https://www.nytimes3xbfgragh.onion/privacy}{Privacy}
\item
  \href{https://help.nytimes3xbfgragh.onion/hc/en-us/articles/115014893428-Terms-of-service}{Terms
  of Service}
\item
  \href{https://help.nytimes3xbfgragh.onion/hc/en-us/articles/115014893968-Terms-of-sale}{Terms
  of Sale}
\item
  \href{https://spiderbites.nytimes3xbfgragh.onion}{Site Map}
\item
  \href{https://help.nytimes3xbfgragh.onion/hc/en-us}{Help}
\item
  \href{https://www.nytimes3xbfgragh.onion/subscription?campaignId=37WXW}{Subscriptions}
\end{itemize}
