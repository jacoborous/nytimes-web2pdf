Sections

SEARCH

\protect\hyperlink{site-content}{Skip to
content}\protect\hyperlink{site-index}{Skip to site index}

\href{https://www.nytimes3xbfgragh.onion/section/world/middleeast}{Middle
East}

\href{https://myaccount.nytimes3xbfgragh.onion/auth/login?response_type=cookie\&client_id=vi}{}

\href{https://www.nytimes3xbfgragh.onion/section/todayspaper}{Today's
Paper}

\href{/section/world/middleeast}{Middle East}\textbar{}Trump Renews Vow
for Jerusalem Embassy, a Gift of Uncertain Value

\url{https://nyti.ms/2k6rF0E}

\begin{itemize}
\item
\item
\item
\item
\item
\end{itemize}

Advertisement

\protect\hyperlink{after-top}{Continue reading the main story}

Supported by

\protect\hyperlink{after-sponsor}{Continue reading the main story}

\hypertarget{trump-renews-vow-for-jerusalem-embassy-a-gift-of-uncertain-value}{%
\section{Trump Renews Vow for Jerusalem Embassy, a Gift of Uncertain
Value}\label{trump-renews-vow-for-jerusalem-embassy-a-gift-of-uncertain-value}}

\includegraphics{https://static01.graylady3jvrrxbe.onion/images/2017/01/20/us/20JERUSALEM/20JERUSALEM-articleInline.jpg?quality=75\&auto=webp\&disable=upscale}

By \href{http://www.nytimes3xbfgragh.onion/by/ian-fisher}{Ian Fisher}
and \href{https://www.nytimes3xbfgragh.onion/by/isabel-kershner}{Isabel
Kershner}

\begin{itemize}
\item
  Jan. 19, 2017
\item
  \begin{itemize}
  \item
  \item
  \item
  \item
  \item
  \end{itemize}
\end{itemize}

JERUSALEM --- It started, as it has in American presidential races for
decades, as a campaign line, one that weary Israelis and Palestinians
hear but rarely take seriously: Donald J. Trump promised to move his
nation's embassy to Jerusalem from Tel Aviv.

But by Thursday, the eve of Mr. Trump's inauguration, those decades of
promises seemed very real --- with reverberations far beyond stone and
cement.

Mr. Trump himself made perhaps his strongest statement on the issue on
Thursday, telling a conservative Israeli news outlet, ``You know I'm not
a person who breaks promises.''

Palestinians protested around the West Bank on Thursday, and many
Israeli Jews wondered if this was a gift that could be politely pushed
away. Moving the embassy is not even close to the top of the list of
concerns for even right-leaning Israelis who oppose the establishment of
a Palestinian state.

Many worry it would only set off new fighting with the Palestinians as
well as the rest of the Arab world, a big price tag for a symbolic
change that would hardly move the ball on the broader conflict.

``I don't know what's in it for Trump,'' said Akiva Eldar, a longtime
Israeli columnist and co-author of
\href{http://www.haaretz.com/jewish/books/the-jerusalem-hijack-1.96576}{a
book on the issue of moving the embassy}. Mr. Eldar's thesis was that
this was largely a concern for American politicians, not Israelis or
Palestinians --- and even within the United States, it was not generally
advocated by those with experience on the ground.

``If you talk to serious people, if you ask the secret service, they say
don't do it,'' Mr. Eldar noted. ``They don't think it's worth it.
Everything is so fragile right now.''

Amid the West Bank protests on Thursday, Saeb Erekat, the chief
Palestinian negotiator, said that moving the embassy would be a ``red
line'' that, if crossed, could kill a two-state solution, reduce
American influence in the peace process, and possibly set off a new
round of violence.

``Why would a president-elect decide to begin his presidency by playing
with the blood of Palestinians and Israelis?'' Mr. Erekat asked in an
interview. ``Why? For whose sake?''

He said that if Mr. Trump were to follow through, the Palestine
Liberation Organization would repeal its recognition of Israel,
considered a baseline condition for peace talks. ``I hope he does not do
it,'' Mr. Erekat added.

Jerusalem is the seat of Israel's government, but it is not home to any
foreign embassies. American policy, like that of many other nations, has
long been that the future of the holy city can be determined only as
part of a broader peace agreement and that putting the embassy there
would prejudge the outcome.

The Israeli leadership considers Jerusalem, including territory captured
from Jordan in the 1967 war, its united and eternal capital. But
Palestinians see Jerusalem as the capital of their future state.

If Mr. Trump does in some way move the embassy to Jerusalem --- and
there are various situations, including having his ambassador simply set
up shop in the existing United States Consulate --- it would be seen as
a victory for Israel's right wing, which has been gaining political
dominance over many years (and power in Washington: The leaders of two
Israeli settlements, condemned by past administrations, are attending
Mr. Trump's inauguration).

It would also be considered a rebuke to the Obama administration's
\href{https://www.nytimes3xbfgragh.onion/2016/12/23/world/middleeast/israel-settlements-un-vote.html}{refusal
last month to veto a United Nations resolution} condemning settlement
building in the occupied West Bank as it warned of the slow strangling
of a two-state peace.

``It is hard for me to imagine any Israeli on the right or even on the
left who thinks Jerusalem is not the capital of Israel and would not
welcome in their heart recognition of that by the U.S.,'' said Moshe
Arens, a former defense and foreign minister from the Likud, the
governing party led by Prime Minister Benjamin Netanyahu.

But even some on the right have questioned the move's value and timing.
In December, Avigdor Lieberman, the hard-line defense minister, appeared
to say it was not a priority.

``We saw in all the American election campaigns that they say they will
move the embassy to Jerusalem,'' Mr. Lieberman said at a forum in
Washington. ``We have enough challenges around us, and I think it would
be a mistake to turn the matter of the embassy into a central one.''

Privately, other conservatives have questioned whether the symbolic
value of an embassy move is worth the risks in terms of a potential
Palestinian uprising and deterioration in Israel's relationship with
other Arab neighbors like Egypt and Jordan. But as Mr. Trump's promise
seems more inevitable, some have publicly cheered it.

``We welcome it,'' Danny Danon, the Israeli ambassador to the United
Nations --- who opposes a Palestinian state --- said in an interview on
Thursday. ``Jerusalem was the capital of the Jewish people 3,000 years
ago.''

There is as much discussion of whether a new Trump administration would
move the embassy as exactly how, and some believe that could either
soften the blow or make it harder for Palestinians and Arab states to
accept the change quietly.

The United States Embassy, and its hundreds of employees, are in Tel
Aviv, as are those of nearly all other countries. The United States
Consulate General's home and office, which handles relations with
Palestinians, is in West Jerusalem, and a United States consular office
is close to the line that divides East and West Jerusalem and provides
visas and similar services.

Experts agree that neither of these locations could house a modern,
secure American embassy immediately. So a range of options are being
discussed.

One could be as simple as changing the sign on the consulate in West
Jerusalem, which processes the visas for Israelis, as the State
Department looked for and built a more permanent embassy --- something
that would take years. In another, Mr. Trump's nominee as ambassador,
David Friedman, who is aligned with the Israeli far right, supports the
settlements and has a residence in Jerusalem, would work out of the West
Jerusalem consulate --- without an immediate declaration of it as the
embassy.

Neither of those options is acceptable to the Palestinians, Mr. Erekat
said. The symbolism remains too strong, favoring Israeli claims against
those of Palestinians. The question of capitals, he said, is not for
other nations to decide, but something to be negotiated between the two
principals.

``This will destroy us as Palestinian moderates,'' he said. ``This will
bring extremism to the region.''

Advertisement

\protect\hyperlink{after-bottom}{Continue reading the main story}

\hypertarget{site-index}{%
\subsection{Site Index}\label{site-index}}

\hypertarget{site-information-navigation}{%
\subsection{Site Information
Navigation}\label{site-information-navigation}}

\begin{itemize}
\tightlist
\item
  \href{https://help.nytimes3xbfgragh.onion/hc/en-us/articles/115014792127-Copyright-notice}{©~2020~The
  New York Times Company}
\end{itemize}

\begin{itemize}
\tightlist
\item
  \href{https://www.nytco.com/}{NYTCo}
\item
  \href{https://help.nytimes3xbfgragh.onion/hc/en-us/articles/115015385887-Contact-Us}{Contact
  Us}
\item
  \href{https://www.nytco.com/careers/}{Work with us}
\item
  \href{https://nytmediakit.com/}{Advertise}
\item
  \href{http://www.tbrandstudio.com/}{T Brand Studio}
\item
  \href{https://www.nytimes3xbfgragh.onion/privacy/cookie-policy\#how-do-i-manage-trackers}{Your
  Ad Choices}
\item
  \href{https://www.nytimes3xbfgragh.onion/privacy}{Privacy}
\item
  \href{https://help.nytimes3xbfgragh.onion/hc/en-us/articles/115014893428-Terms-of-service}{Terms
  of Service}
\item
  \href{https://help.nytimes3xbfgragh.onion/hc/en-us/articles/115014893968-Terms-of-sale}{Terms
  of Sale}
\item
  \href{https://spiderbites.nytimes3xbfgragh.onion}{Site Map}
\item
  \href{https://help.nytimes3xbfgragh.onion/hc/en-us}{Help}
\item
  \href{https://www.nytimes3xbfgragh.onion/subscription?campaignId=37WXW}{Subscriptions}
\end{itemize}
