Sections

SEARCH

\protect\hyperlink{site-content}{Skip to
content}\protect\hyperlink{site-index}{Skip to site index}

\href{https://www.nytimes3xbfgragh.onion/section/world/middleeast}{Middle
East}

\href{https://myaccount.nytimes3xbfgragh.onion/auth/login?response_type=cookie\&client_id=vi}{}

\href{https://www.nytimes3xbfgragh.onion/section/todayspaper}{Today's
Paper}

\href{/section/world/middleeast}{Middle East}\textbar{}As Protests
Flare, Iran Bids Farewell to Rafsanjani

\url{https://nyti.ms/2ieukHK}

\begin{itemize}
\item
\item
\item
\item
\item
\end{itemize}

Advertisement

\protect\hyperlink{after-top}{Continue reading the main story}

Supported by

\protect\hyperlink{after-sponsor}{Continue reading the main story}

\hypertarget{as-protests-flare-iran-bids-farewell-to-rafsanjani}{%
\section{As Protests Flare, Iran Bids Farewell to
Rafsanjani}\label{as-protests-flare-iran-bids-farewell-to-rafsanjani}}

\includegraphics{https://static01.graylady3jvrrxbe.onion/images/2017/01/11/world/11iran4/11iran4-articleInline-v2.jpg?quality=75\&auto=webp\&disable=upscale}

By \href{http://www.nytimes3xbfgragh.onion/by/thomas-erdbrink}{Thomas
Erdbrink}

\begin{itemize}
\item
  Jan. 10, 2017
\item
  \begin{itemize}
  \item
  \item
  \item
  \item
  \item
  \end{itemize}
\end{itemize}

TEHRAN --- Iranians bade farewell to
\href{http://www.nytimes3xbfgragh.onion/2017/01/08/world/middleeast/ayatollah-rafsanjani-dead.html}{Ayatollah
Akbar Hashemi Rafsanjani} on Tuesday, with the sprawling state funeral
veering slightly off script when groups of mourners started shouting
opposition slogans.

The authorities were forced to raise the volume on the loudspeakers
playing lamentation songs after some in the crowds took up cries of
``Oh, Hussein, Mir Hussein,'' a reference to a former presidential
candidate, Mir Hussein Moussavi, who has been under house arrest since
2011.

Some of the chants were aimed at Russia, Iran's ally in the Syrian
conflict. Video clips on social media showed mourners shouting ``Death
to Russia'' and ``the Russian Embassy is the den of espionage,'' as they
passed the embassy's complex in the heart of Tehran. People also called
for the release of
\href{http://www.nytimes3xbfgragh.onion/2017/01/09/world/middleeast/the-hunger-strike-the-protest-tactic-of-gandhi-is-vexing-irans-penal-overseers.html?ref=world}{hunger
strikers} in Iranian prisons.

State television, broadcasting the funeral live, airbrushed the
protests, which were nevertheless allowed to proceed without police
intervention.

\includegraphics{https://static01.graylady3jvrrxbe.onion/images/2017/01/11/world/11Iran2/11Iran2-articleLarge.jpg?quality=75\&auto=webp\&disable=upscale}

Mr. Rafsanjani, 82, who
\href{http://www.nytimes3xbfgragh.onion/2017/01/08/world/middleeast/iran-ali-akbar-hashemi-rafsanjani-dies.html}{died
on Sunday}, was laid to rest after an elaborate ceremony that lasted
several days. Right after his demise, his body was placed in a coffin
that was put on public display in the modest house of the late founder
of the Islamic republic, Ayatollah Ruhollah Khomeini.

For two days, mourners had filed through the northern Tehran site,
untouched since Mr. Khomeini died in 1989. A religious chanter brought
the crowds to tears as he recalled how Mr. Rafsanjani helped to oust
Shah Mohammed Reza Pahlavi in the 1979 revolution. ``Our sheikh was so
wise, he made the shah leave, leave,'' the chanter sang.

Men gathering on the ground floor bowed their heads in respect, while on
the first floor --- the women's section --- mourners in black chadors
peeked down. Qassem Soleimani, the general of the Quds Force of the
Revolutionary Guards who runs Iran's operations in Iraq and Syria, paid
his respects, some people said, showing clips of him on their cellphones
as proof.

Because of Mr. Rafsanjani's close relationship with Ayatollah Khomeini,
he was accorded the honor of being buried in the late leader's
\href{https://www.google.co.uk/search?q=khomeini+mausoleum+sciolino\&biw=1718\&bih=912\&tbm=isch\&tbo=u\&source=univ\&sa=X\&ved=0ahUKEwiMqsuqw7fRAhWrLMAKHfWlBHgQsAQIGQ}{mausoleum},
in a golden cage. Before the interment, all Iranians were invited to
gather around the campus of the University of Tehran, in the central
part of the city, where Iran's supreme leader, Ayatollah Ali Khamenei,
led a prayer.

People showed up early, some wearing scarves around their faces to
protect them from the morning cold. Families passed by, pushing
strollers carrying babies wearing woolen hats. Students took videos with
their cellphones. Shiite clerics in traditional winter robes made of
camel's hair held prayer beads.

\includegraphics{https://static01.graylady3jvrrxbe.onion/images/2017/01/11/world/11iran-video/11iran-video-videoSixteenByNineJumbo1600.jpg}

There were so many people --- 2.5 million by official estimates --- that
many of the dignitaries and family members invited to the campus were
marooned in their cars amid the crowds. Some hid behind curtains; others
waved at the collection of camera phones.

One of Mr. Rafsanjani's daughters, Faezeh Hashemi, was photographed
sticking her head out of the window of a bus and flashing a victory
sign. She and her brother Mehdi have been harassed by hard-liners for
their growing support of reformists and moderates seeking change in
Iran. The daughter, an activist for women's rights and personal
freedoms, was jailed in 2011 for making ``anti-regime propaganda,''
while her brother was given leave to attend the funeral from prison,
where he was sent on embezzlement charges.

In recent years their father, long a staunch conservative, became an
unexpected hero to Iran's middle class. Mr. Rafsanjani sympathized with
some demands made by protesters during the so-called
\href{http://www.nytimes3xbfgragh.onion/2009/06/14/world/middleeast/14iran.html}{Green
Revolution}, the antigovernment demonstrations following the disputed
re-election victory of President Mahmoud Ahmadinejad in 2009. They saw
him as a lone voice representing their beliefs in Iran's establishment.

Such deviations from the official line were put aside by the authorities
on Tuesday. In death it seemed that Mr. Rafsanjani was to be remembered
for his revolutionary credentials, not for his criticisms. Potential
troublemakers were not invited. The former reformist president, Mohammad
Khatami, who was supported by Mr. Rafsanjani, was told not to attend,
local websites said.

The same apparatus that normally churns out posters showing Uncle Sam
with blood dripping from his teeth to burn during state-backed
anti-American demonstrations, now printed pictures of Mr. Rafsanjani,
extolling him as ``a man of history, who is immortal.''

Image

Iran's supreme leader, Ayatollah Ali Khamenei, center, led a prayer on
Tuesday over the casket of Mr. Rafsanjani, with President Hassan
Rouhani, center left, at the Tehran University campus in
Tehran.Credit...Office of the Iranian Supreme Leader, via Associated
Press

In the teeming streets, scenes clashed incongruously. At one point,
Ayatollah Khamenei could be heard through loudspeakers saying prayers
for Mr. Rafsanjani while protesters chanted opposition slogans. Some
wore green wristbands, the color of the opposition, and flashed victory
signs.

Supporters of the establishment tried to drown out the slogans by
shouting ``Allahu akbar,'' meaning ``God is great,'' but for the most
part they were overmatched. On state television, sound engineers at one
point forgot to lower the volume when people shouted, ``Hail to
Khatami.''

``Hashemi's death is a great worry to us,'' said Leili Farhang, a
26-year-old university graduate, who emphasized that she was unemployed
``like many of my generation.'' She and her friends had showed up in
front of the Tehran University campus ``to pay respect to a man who
respected us.''

It was hard, she and her friends agreed, to come up with the name of
anybody within Iran's establishment to replace Mr. Rafsanjani. Not one
has his weight and stature, they concluded: ``He will be missed.''

It took hours for the body to arrive at the South Tehran mausoleum,
because of ``the millions that have come out to honor the ayatollah,''
Khabarfori, an Iranian online news channel, said on the Telegram
messaging app.

Inside the mausoleum, state television showed, a marching band played
the national anthem, after which Mr. Rafsanjani's coffin was placed next
to Mr. Khomeini's, as planned.

Advertisement

\protect\hyperlink{after-bottom}{Continue reading the main story}

\hypertarget{site-index}{%
\subsection{Site Index}\label{site-index}}

\hypertarget{site-information-navigation}{%
\subsection{Site Information
Navigation}\label{site-information-navigation}}

\begin{itemize}
\tightlist
\item
  \href{https://help.nytimes3xbfgragh.onion/hc/en-us/articles/115014792127-Copyright-notice}{©~2020~The
  New York Times Company}
\end{itemize}

\begin{itemize}
\tightlist
\item
  \href{https://www.nytco.com/}{NYTCo}
\item
  \href{https://help.nytimes3xbfgragh.onion/hc/en-us/articles/115015385887-Contact-Us}{Contact
  Us}
\item
  \href{https://www.nytco.com/careers/}{Work with us}
\item
  \href{https://nytmediakit.com/}{Advertise}
\item
  \href{http://www.tbrandstudio.com/}{T Brand Studio}
\item
  \href{https://www.nytimes3xbfgragh.onion/privacy/cookie-policy\#how-do-i-manage-trackers}{Your
  Ad Choices}
\item
  \href{https://www.nytimes3xbfgragh.onion/privacy}{Privacy}
\item
  \href{https://help.nytimes3xbfgragh.onion/hc/en-us/articles/115014893428-Terms-of-service}{Terms
  of Service}
\item
  \href{https://help.nytimes3xbfgragh.onion/hc/en-us/articles/115014893968-Terms-of-sale}{Terms
  of Sale}
\item
  \href{https://spiderbites.nytimes3xbfgragh.onion}{Site Map}
\item
  \href{https://help.nytimes3xbfgragh.onion/hc/en-us}{Help}
\item
  \href{https://www.nytimes3xbfgragh.onion/subscription?campaignId=37WXW}{Subscriptions}
\end{itemize}
