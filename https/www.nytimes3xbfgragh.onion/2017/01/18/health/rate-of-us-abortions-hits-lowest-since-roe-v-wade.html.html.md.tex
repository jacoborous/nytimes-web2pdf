Sections

SEARCH

\protect\hyperlink{site-content}{Skip to
content}\protect\hyperlink{site-index}{Skip to site index}

\href{https://www.nytimes3xbfgragh.onion/section/health}{Health}

\href{https://myaccount.nytimes3xbfgragh.onion/auth/login?response_type=cookie\&client_id=vi}{}

\href{https://www.nytimes3xbfgragh.onion/section/todayspaper}{Today's
Paper}

\href{/section/health}{Health}\textbar{}Rate of U.S. Abortions Hits
Lowest Since Roe v. Wade

\url{https://nyti.ms/2k0lWcC}

\begin{itemize}
\item
\item
\item
\item
\item
\end{itemize}

Advertisement

\protect\hyperlink{after-top}{Continue reading the main story}

Supported by

\protect\hyperlink{after-sponsor}{Continue reading the main story}

\hypertarget{rate-of-us-abortions-hits-lowest-since-roe-v-wade}{%
\section{Rate of U.S. Abortions Hits Lowest Since Roe v.
Wade}\label{rate-of-us-abortions-hits-lowest-since-roe-v-wade}}

\includegraphics{https://static01.graylady3jvrrxbe.onion/images/2017/01/19/science/24SCI-NUMBER/24SCI-NUMBER-articleLarge.jpg?quality=75\&auto=webp\&disable=upscale}

By \href{http://www.nytimes3xbfgragh.onion/by/jan-hoffman}{Jan Hoffman}

\begin{itemize}
\item
  Jan. 18, 2017
\item
  \begin{itemize}
  \item
  \item
  \item
  \item
  \item
  \end{itemize}
\end{itemize}

The rate of abortions performed in the United States has fallen lower
than during any year since 1973, when the Supreme Court legalized the
procedure, according to a new
\href{http://onlinelibrary.wiley.com/doi/10.1363/psrh.12015/full}{report}by
the Guttmacher Institute, a research group that supports abortion
rights.

The latest numbers, for 2014, continue a trend of declining abortion
rates for most years since 1981.

In 2014, there were an estimated 926,200 abortions --- a rate of 14.6
per 1,000 women of childbearing age (15 to 44) --- compared with 1.06
million abortions in 2011, the year of the last Guttmacher report, or
16.9 per 1,000. In 1973, the year of the Roe v. Wade decision, the rate
was
\href{https://www.guttmacher.org/sites/default/files/article_files/4000608.pdf}{16.3}.
In 1981, the rate was 29.3.

Researchers suggested that increased use of long-term birth control,
such as intrauterine devices and contraceptive implants, contributed to
the most recent decline. In particular, the proportion of clients at
federally funded family planning clinics who sought such methods
increased to 11 percent in 2014 from 7 percent in 2011. Because women
who rely on these clinics are disproportionately young and poor and
account for a majority of unintended pregnancies, researchers said, even
a moderate increase in reliance on these methods could have an effect on
the abortion rate.

The impact of restrictive anti-abortion laws and the shuttering of
clinics on abortion rates was unclear. For example, the abortion rate
rose modestly in six states: Kansas, Arkansas, Michigan, Mississippi,
North Carolina and Vermont. Yet between 2001 and 2014, with the
exception of Vermont, all of these states introduced more restrictive
abortion laws.

``The underlying purpose of these restrictions was to reduce access to
abortion, but clearly that doesn't always happen,'' said Rachel K.
Jones, a principal research scientist at Guttmacher and the lead author
of the report.

Nationally, there was a 6 percent decline in the number of clinics
performing abortions. In the Midwest, the percentage of closing clinics
was 22 percent; in the South, 13 percent. By contrast, the percentage of
clinics in the Northeast increased 14 percent.

``But an increased number of clinics doesn't mean you're going to have
more abortions,'' said Ms. Jones.

In New Jersey, the number of clinics rose to 41 in 2014 from 24 in 2011.
Yet the number of abortions in the state during that period declined, to
about 44,000 in 2014 from about 47,000 in 2011.

For the first time, researchers asked about the incidence of
self-induced abortion: 12 percent of clinics reported having treated at
least one patient who had tried to end her own pregnancy.

Advertisement

\protect\hyperlink{after-bottom}{Continue reading the main story}

\hypertarget{site-index}{%
\subsection{Site Index}\label{site-index}}

\hypertarget{site-information-navigation}{%
\subsection{Site Information
Navigation}\label{site-information-navigation}}

\begin{itemize}
\tightlist
\item
  \href{https://help.nytimes3xbfgragh.onion/hc/en-us/articles/115014792127-Copyright-notice}{©~2020~The
  New York Times Company}
\end{itemize}

\begin{itemize}
\tightlist
\item
  \href{https://www.nytco.com/}{NYTCo}
\item
  \href{https://help.nytimes3xbfgragh.onion/hc/en-us/articles/115015385887-Contact-Us}{Contact
  Us}
\item
  \href{https://www.nytco.com/careers/}{Work with us}
\item
  \href{https://nytmediakit.com/}{Advertise}
\item
  \href{http://www.tbrandstudio.com/}{T Brand Studio}
\item
  \href{https://www.nytimes3xbfgragh.onion/privacy/cookie-policy\#how-do-i-manage-trackers}{Your
  Ad Choices}
\item
  \href{https://www.nytimes3xbfgragh.onion/privacy}{Privacy}
\item
  \href{https://help.nytimes3xbfgragh.onion/hc/en-us/articles/115014893428-Terms-of-service}{Terms
  of Service}
\item
  \href{https://help.nytimes3xbfgragh.onion/hc/en-us/articles/115014893968-Terms-of-sale}{Terms
  of Sale}
\item
  \href{https://spiderbites.nytimes3xbfgragh.onion}{Site Map}
\item
  \href{https://help.nytimes3xbfgragh.onion/hc/en-us}{Help}
\item
  \href{https://www.nytimes3xbfgragh.onion/subscription?campaignId=37WXW}{Subscriptions}
\end{itemize}
