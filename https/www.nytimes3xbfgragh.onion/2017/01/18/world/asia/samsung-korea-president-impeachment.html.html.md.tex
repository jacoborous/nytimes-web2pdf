Sections

SEARCH

\protect\hyperlink{site-content}{Skip to
content}\protect\hyperlink{site-index}{Skip to site index}

\href{https://www.nytimes3xbfgragh.onion/section/world/asia}{Asia
Pacific}

\href{https://myaccount.nytimes3xbfgragh.onion/auth/login?response_type=cookie\&client_id=vi}{}

\href{https://www.nytimes3xbfgragh.onion/section/todayspaper}{Today's
Paper}

\href{/section/world/asia}{Asia Pacific}\textbar{}In a Blow to
Prosecutor, South Korean Court Blocks Arrest of Samsung Group Leader

\url{https://nyti.ms/2k1E0Do}

\begin{itemize}
\item
\item
\item
\item
\item
\end{itemize}

Advertisement

\protect\hyperlink{after-top}{Continue reading the main story}

Supported by

\protect\hyperlink{after-sponsor}{Continue reading the main story}

\hypertarget{in-a-blow-to-prosecutor-south-korean-court-blocks-arrest-of-samsung-group-leader}{%
\section{In a Blow to Prosecutor, South Korean Court Blocks Arrest of
Samsung Group
Leader}\label{in-a-blow-to-prosecutor-south-korean-court-blocks-arrest-of-samsung-group-leader}}

\includegraphics{https://static01.graylady3jvrrxbe.onion/images/2017/01/19/world/19korea/19korea-articleInline.jpg?quality=75\&auto=webp\&disable=upscale}

By \href{http://www.nytimes3xbfgragh.onion/by/choe-sang-hun}{Choe
Sang-Hun}

\begin{itemize}
\item
  Jan. 18, 2017
\item
  \begin{itemize}
  \item
  \item
  \item
  \item
  \item
  \end{itemize}
\end{itemize}

SEOUL, South Korea --- A South Korean court on Thursday blocked a
prosecutor's attempt to arrest Jay Y. Lee, the leader of Samsung, saying
there was not enough evidence that Mr. Lee had bribed President Park
Geun-hye, in a scandal that led to her impeachment.

A justice on the Central District Court in Seoul, Cho Eui-yeon, rejected
the prosecutor's request to issue an arrest warrant, saying said it was
``difficult to recognize the need'' to incarcerate Mr. Lee.

Mr. Lee, a third-generation scion and vice chairman of Samsung, one of
the world's biggest conglomerates, was immediately released from a
detention center outside Seoul, where he had been waiting for the court
to decide whether he should be formally arrested.

South Koreans have paid keen attention to the fate of Mr. Lee. Some
analysts said his case was a test of whether the country's relatively
youthful
\href{https://www.nytimes3xbfgragh.onion/2017/01/02/world/asia/south-korea-park-geun-hye-samsung.html}{democracy
and judicial system} are ready to crack down on the white-collar crimes
of family-owned conglomerates. No Samsung leader has ever been jailed,
though the company has been investigated many times for corruption.

The court's decision is likely to anger many South Koreans who have held
weekend rallies calling for Ms. Park's ouster and the arrest of business
tycoons on corruption charges.

The special prosecutor called the court decision ``very regrettable.''
But he has yet to announce whether he will offer more evidence in a
renewed effort to have Mr. Lee arrested. He can also indict Mr. Lee on
bribery or lesser charges without arresting him.

**``**We will take necessary steps and persist in our investigation
without wavering,'' said Lee Kyu-chul, a spokesman for the special
prosecutor, without elaborating.

Samsung welcomed the court's decision. For now, the ruling allows Mr.
Lee to continue to lead Samsung. It dealt a blow to the special
prosecutor who had tried to build a bribery case against Mr. Lee and Ms.
Park.

Mr. Lee's father has twice been convicted of bribery and tax evasion but
has never spent a day in prison. Each time, he received a presidential
pardon and returned to management.

Mr. Lee, 48, was accused of paying \$36 million to Ms. Park's secretive
confidante, Choi Soon-sil. The special prosecutor and Mr. Lee's lawyers
have been arguing over how to characterize the money.

In November, state prosecutors indicted Ms. Choi on extortion charges,
saying she leveraged her connections with Ms. Park to coerce Samsung and
scores of other big businesses to contribute tens of millions of dollars
to two foundations Ms. Choi controlled or to companies run by her or her
associates.

They identified Ms. Park as an accomplice, but they brought no charges
against the businesses, which they saw as victims of extortion. But the
special prosecutor, Park Young-soo, who took over the investigation from
state prosecutors last month, has called Samsung's contributions bribes
that were exchanged for political favors from Ms. Park.

That includes government support for a merger of two Samsung affiliates
in 2015, which helped Mr. Lee inherit corporate control from his
incapacitated father, the chairman, Lee Kun-hee, according to the
prosecutor.

Pro-business groups accused the prosecutor of overreaching in an attempt
to find a high-profile scapegoat to soothe a public infuriated over Ms.
Park's corruption scandal and fed up with decades of collusive ties
between the government and the
\href{https://www.nytimes3xbfgragh.onion/2017/01/16/business/lee-jae-yong-samsung.html}{chaebol}.

Mr. Lee was the most prominent businessman to be ensnared in the special
prosecutor's broadening investigation into the corruption scandal that
led to Ms. Park's
\href{https://www.nytimes3xbfgragh.onion/2016/12/09/world/asia/south-korea-president-park-geun-hye-impeached.html}{impeachment
by Parliament} last month. Ms. Park's presidential powers remained
suspended, while the Constitutional Court
\href{https://www.nytimes3xbfgragh.onion/2017/01/03/world/asia/south-korea-president-impeachment-trial.html}{is
expected to rule} in coming weeks whether she should be reinstated or
formally removed from office.

``We have been too lenient toward chaebol corruption,'' said Moon
Jae-in, an opposition politician who leads in polls on contenders to
replace Ms. Park if she is removed.

Speaking to a group of foreign reporters hours before the court's
decision, Mr. Moon said Samsung was typical of a chaebol whose top boss
wielded ``imperial powers'' over his sprawling business group but was
''seldom held accountable'' for corruption or managerial failures.

Ms. Park denies any wrongdoing. Mr. Lee and Samsung have also denied
bribery; they argued that the ``donations'' Samsung paid out to Ms. Choi
were coerced, not meant as a quid pro quo for political favors from Ms.
Park.

Advertisement

\protect\hyperlink{after-bottom}{Continue reading the main story}

\hypertarget{site-index}{%
\subsection{Site Index}\label{site-index}}

\hypertarget{site-information-navigation}{%
\subsection{Site Information
Navigation}\label{site-information-navigation}}

\begin{itemize}
\tightlist
\item
  \href{https://help.nytimes3xbfgragh.onion/hc/en-us/articles/115014792127-Copyright-notice}{©~2020~The
  New York Times Company}
\end{itemize}

\begin{itemize}
\tightlist
\item
  \href{https://www.nytco.com/}{NYTCo}
\item
  \href{https://help.nytimes3xbfgragh.onion/hc/en-us/articles/115015385887-Contact-Us}{Contact
  Us}
\item
  \href{https://www.nytco.com/careers/}{Work with us}
\item
  \href{https://nytmediakit.com/}{Advertise}
\item
  \href{http://www.tbrandstudio.com/}{T Brand Studio}
\item
  \href{https://www.nytimes3xbfgragh.onion/privacy/cookie-policy\#how-do-i-manage-trackers}{Your
  Ad Choices}
\item
  \href{https://www.nytimes3xbfgragh.onion/privacy}{Privacy}
\item
  \href{https://help.nytimes3xbfgragh.onion/hc/en-us/articles/115014893428-Terms-of-service}{Terms
  of Service}
\item
  \href{https://help.nytimes3xbfgragh.onion/hc/en-us/articles/115014893968-Terms-of-sale}{Terms
  of Sale}
\item
  \href{https://spiderbites.nytimes3xbfgragh.onion}{Site Map}
\item
  \href{https://help.nytimes3xbfgragh.onion/hc/en-us}{Help}
\item
  \href{https://www.nytimes3xbfgragh.onion/subscription?campaignId=37WXW}{Subscriptions}
\end{itemize}
