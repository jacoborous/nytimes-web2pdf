Sections

SEARCH

\protect\hyperlink{site-content}{Skip to
content}\protect\hyperlink{site-index}{Skip to site index}

\href{https://www.nytimes3xbfgragh.onion/section/us}{U.S.}

\href{https://myaccount.nytimes3xbfgragh.onion/auth/login?response_type=cookie\&client_id=vi}{}

\href{https://www.nytimes3xbfgragh.onion/section/todayspaper}{Today's
Paper}

\href{/section/us}{U.S.}\textbar{}Women's March on Washington Opens
Contentious Dialogues About Race

\url{https://nyti.ms/2ibJqxv}

\begin{itemize}
\item
\item
\item
\item
\item
\item
\end{itemize}

Advertisement

\protect\hyperlink{after-top}{Continue reading the main story}

Supported by

\protect\hyperlink{after-sponsor}{Continue reading the main story}

\hypertarget{womens-march-on-washington-opens-contentious-dialogues-about-race}{%
\section{Women's March on Washington Opens Contentious Dialogues About
Race}\label{womens-march-on-washington-opens-contentious-dialogues-about-race}}

\includegraphics{https://static01.graylady3jvrrxbe.onion/images/2017/01/09/us/09womensmarch/09womensmarch-articleInline.jpg?quality=75\&auto=webp\&disable=upscale}

By \href{https://www.nytimes3xbfgragh.onion/by/farah-stockman}{Farah
Stockman}

\begin{itemize}
\item
  Jan. 9, 2017
\item
  \begin{itemize}
  \item
  \item
  \item
  \item
  \item
  \item
  \end{itemize}
\end{itemize}

Many thousands of women are expected to converge on the nation's capital
for the
\href{https://www.nytimes3xbfgragh.onion/2016/11/19/us/womens-march-on-washington.html}{Women's
March on Washington} the day after Donald J. Trump's inauguration.
Jennifer Willis no longer plans to be one of them.

Ms. Willis, a 50-year-old wedding minister from South Carolina, had
looked forward to taking her daughters to the march. Then she read a
post on the Facebook page for the march that made her feel unwelcome
because she is white.

The post, written by a black activist from Brooklyn who is a march
volunteer, advised ``white allies'' to listen more and talk less. It
also chided those who, it said, were only now waking up to racism
because of the election.

``You don't just get to join because now you're scared, too,'' read the
post. ``I was born scared.''

Stung by the tone, Ms. Willis canceled her trip.

``This is a women's march,'' she said. ``We're supposed to be allies in
equal pay, marriage, adoption. Why is it now about, `White women don't
understand black women'?''

If all goes as planned, the Jan. 21 march will be a momentous display of
unity in protest of a president whose treatment of women came to
dominate the campaign's final weeks. But long before the first buses
roll to Washington and
\href{https://womensmarch.squarespace.com/sisters}{sister
demonstrations}take place in other cities, contentious conversations
about race have erupted nearly every day among marchers, exhilarating
some and alienating others.

In Tennessee, emotions ran high when organizers changed the name of the
local march from ``Women's March on Washington-Nashville'' to ``Power
Together Tennessee, in solidarity with Women's March on Washington.''
While many applauded the name change, which was meant to signal the
start of a new social justice movement in Nashville, some complained
that the event had turned from a march for all women into a march for
black women.

In Louisiana, the first state coordinator gave up her volunteer role in
part because there were no minority women in leadership positions at
that time.

``I got a lot of flak locally when I stepped down, from white women who
said that I'm alienating a lot of white women,'' said Candice Huber, a
bookstore owner in New Orleans, who is white. ``They said, `Why do you
have to be so divisive?'''

In some ways, the discord is by design. Even as they are working to
ensure a smooth and unified march next week, the national organizers
said they made a deliberate decision to highlight the plight of minority
and undocumented immigrant women and provoke uncomfortable discussions
about race.

``This was an opportunity to take the conversation to the deep places,''
said Linda Sarsour, a Muslim who heads the Arab American Association of
New York and is one of \href{https://www.womensmarch.com/team/}{four
co-chairwomen} of the national march. ``Sometimes you are going to upset
people.''

The post that offended Ms. Willis was part of that effort. So was the
quotation posted on the march's Facebook page from Bell Hooks, the black
feminist, about forging a stronger sisterhood by ``confronting the ways
women --- through sex, class and race --- dominated and exploited other
women.''

In response, a New Jersey woman wrote: ``I'm starting to feel not very
welcome in this endeavor.''

A debate then ensued about whether white women were just now
experiencing what minority women experience daily, or were having a hard
time yielding control. A young white woman from Baltimore wrote with
bitterness that white women who might have been victims of rape and
abuse were being ``asked to check their privilege,'' a catchphrase that
refers to people acknowledging their advantages, but which even some
liberal women find unduly confrontational.

No one involved with the march fears that the rancor will dampen
turnout; even many of those who expressed dismay at the tone of the
discussion said they still intended to join what is sure to be the
largest demonstration yet against the Trump presidency.

``I will march,'' one wrote on the march's Facebook page, ``Hoping that
someday soon a sense of unity will occur before it's too late.''

But these debates over race also reflect deeper questions about the
future of progressivism in the age of Trump. Should the march highlight
what divides women, or what unites them? Is there room for women who
have never heard of ``white privilege''?

And at a time when a presidential candidate ran against political
correctness and won --- with half of white female voters supporting him
--- is this the time to tone down talk about race or to double down?

\href{https://www.nytimes3xbfgragh.onion/interactive/2017/01/10/us/politics/womens-march-guide.html}{}

\includegraphics{https://static01.graylady3jvrrxbe.onion/images/2017/01/11/opinion/11womensmarch-1/16bazelonWeb-articleLarge.jpg}

\hypertarget{womens-march-on-washington-what-you-need-to-know}{%
\subsection{Women's March on Washington: What You Need to
Know}\label{womens-march-on-washington-what-you-need-to-know}}

Many thousands of people are expected to march in protest of the new
president on Jan. 21, his first full day in office.

``If your short-term goal is to get as many people as possible at the
march, maybe you don't want to alienate people,'' said Anne Valk, the
author of ``Radical Sisters,'' a book about racial and class differences
in the women's movement. ``But if your longer-term goal is to use the
march as a catalyst for progressive social and political change, then
that has to include thinking about race and class privilege.''

The discord also reflects the variety of women's rights and liberal
causes being represented at the march, as well as a generational divide.

Many older white women spent their lives fighting for rights like
workplace protections that younger women now take for granted. Many
young activists have spent years protesting police tactics and criminal
justice policies --- issues they feel too many white liberals have
ignored.

``Yes, equal pay is an issue,'' Ms. Sarsour said. ``But look at the
ratio of what white women get paid versus black women and Latina
women.''

For too long, the march organizers said, the women's rights movement
focused on issues that were important to well-off white women, such as
the ability to work outside the home and attain the same high-powered
positions that men do. But minority women, they said, have had different
priorities. Black women who have worked their whole lives as maids might
care more about the minimum wage or police brutality than about seeing a
woman in the White House. Undocumented immigrant women might care about
abortion rights, they said, but not nearly as much as they worry about
being deported.

This brand of feminism --- frequently referred to as
\href{https://www.washingtonpost.com/news/in-theory/wp/2015/09/24/why-intersectionality-cant-wait/?utm_term=.57d0384a6d96}{``intersectionality''}
--- asks white women to acknowledge that they have had it easier. It
speaks candidly about the history of racism, even within the feminist
movement itself. The organizers of the 1913 suffrage march on Washington
asked black women to march at the back of the parade.

The issue of race has followed the march from its inception. The day
after the election, Bob Bland, a fashion designer in New York, floated
the idea of a march in Washington on Facebook. Within hours, 3,000
people said they would join. Then a friend called to tell Ms. Bland that
\href{https://www.washingtonpost.com/national/it-started-with-a-grandmother-in-hawaii-now-the-womens-march-on-washington-is-poised-to-be-the-biggest-inauguration-demonstration/2017/01/03/8af61686-c6e2-11e6-bf4b-2c064d32a4bf_story.html?utm_term=.3aa2a18a7aab}{a
woman in Hawaii} with a similar page had collected pledges from 12,000
people.

``I thought, `Wow, let's merge,''' Ms. Bland recalled.

As the effort grew, a number of comments on Facebook implored Ms. Bland,
who is white, to include minority women on the leadership team. Ms.
Bland felt strongly that it was the right thing to do. Within three days
of the election, Carmen Perez, a Hispanic activist working on juvenile
justice, and Tamika D. Mallory, a gun control activist who is black,
joined Ms. Bland.

\href{https://www.nytimes3xbfgragh.onion/topic/person/gloria-steinem}{Gloria
Steinem}, honorary co-chairwoman of the march along with
\href{https://www.nytimes3xbfgragh.onion/2016/11/07/opinion/campaign-stops/harry-belafonte-what-do-we-have-to-lose-everything.html}{Harry
Belafonte}, lauded their approach. ``Sexism is always made worse by
racism --- and vice versa,'' she said in an email.

Ms. Steinem, who plans to participate in a town hall meeting during the
march with Alicia Garza, a co-founder of Black Lives Matter, said even
contentious conversations about race were a ``good thing.''

``It's about knowing each other,'' she wrote. ``Which is what movements
and marches are for.''

But the tone of the discussion, particularly online, can become so raw
that some would-be marchers feel they are no longer welcome.

Ms. Willis, the South Carolina wedding minister, had been looking
forward to the salve of rallying with people who share her values, a
rarity in her home state, where she said she had been insulted and
shouted at for marrying gay couples.

But then she read a post by ShiShi Rose, a 27-year-old blogger from
Brooklyn.

``Now is the time for you to be listening more, talking less,'' Ms. Rose
wrote. ``You should be reading our books and understanding the roots of
racism and white supremacy. Listening to our speeches. You should be
drowning yourselves in our poetry.''

It rubbed Ms. Willis the wrong way.

``How do you know that I'm not reading black poetry?'' she asked in an
interview. Ms. Willis says that she understands being born white gives
her advantages, and that she is always open to learning more about the
struggles of others.

But, she said, ``The last thing that is going to make me endeared to
you, to know you and love you more, is if you are sitting there wagging
your finger at me.''

Ms. Rose said in an interview that the intention of the post was not to
weed people out but rather to make them understand that they had a lot
of learning to do.

``I needed them to understand that they don't just get to join the march
and not check their privilege constantly,'' she said.

That phrase --- check your privilege --- exasperates Ms. Willis. She
asked a reporter: ``Can you please tell me what that means?''

Advertisement

\protect\hyperlink{after-bottom}{Continue reading the main story}

\hypertarget{site-index}{%
\subsection{Site Index}\label{site-index}}

\hypertarget{site-information-navigation}{%
\subsection{Site Information
Navigation}\label{site-information-navigation}}

\begin{itemize}
\tightlist
\item
  \href{https://help.nytimes3xbfgragh.onion/hc/en-us/articles/115014792127-Copyright-notice}{©~2020~The
  New York Times Company}
\end{itemize}

\begin{itemize}
\tightlist
\item
  \href{https://www.nytco.com/}{NYTCo}
\item
  \href{https://help.nytimes3xbfgragh.onion/hc/en-us/articles/115015385887-Contact-Us}{Contact
  Us}
\item
  \href{https://www.nytco.com/careers/}{Work with us}
\item
  \href{https://nytmediakit.com/}{Advertise}
\item
  \href{http://www.tbrandstudio.com/}{T Brand Studio}
\item
  \href{https://www.nytimes3xbfgragh.onion/privacy/cookie-policy\#how-do-i-manage-trackers}{Your
  Ad Choices}
\item
  \href{https://www.nytimes3xbfgragh.onion/privacy}{Privacy}
\item
  \href{https://help.nytimes3xbfgragh.onion/hc/en-us/articles/115014893428-Terms-of-service}{Terms
  of Service}
\item
  \href{https://help.nytimes3xbfgragh.onion/hc/en-us/articles/115014893968-Terms-of-sale}{Terms
  of Sale}
\item
  \href{https://spiderbites.nytimes3xbfgragh.onion}{Site Map}
\item
  \href{https://help.nytimes3xbfgragh.onion/hc/en-us}{Help}
\item
  \href{https://www.nytimes3xbfgragh.onion/subscription?campaignId=37WXW}{Subscriptions}
\end{itemize}
