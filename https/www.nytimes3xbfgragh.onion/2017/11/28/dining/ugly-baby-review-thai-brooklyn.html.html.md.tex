Sections

SEARCH

\protect\hyperlink{site-content}{Skip to
content}\protect\hyperlink{site-index}{Skip to site index}

\href{https://www.nytimes3xbfgragh.onion/section/food}{Food}

\href{https://myaccount.nytimes3xbfgragh.onion/auth/login?response_type=cookie\&client_id=vi}{}

\href{https://www.nytimes3xbfgragh.onion/section/todayspaper}{Today's
Paper}

\href{/section/food}{Food}\textbar{}At Ugly Baby, the Name Isn't All
That's Unusual

\url{https://nyti.ms/2id5v1l}

\begin{itemize}
\item
\item
\item
\item
\item
\item
\end{itemize}

Advertisement

\protect\hyperlink{after-top}{Continue reading the main story}

Supported by

\protect\hyperlink{after-sponsor}{Continue reading the main story}

\href{/column/restaurant-review}{Restaurant Review}

\hypertarget{at-ugly-baby-the-name-isnt-all-thats-unusual}{%
\section{At Ugly Baby, the Name Isn't All That's
Unusual}\label{at-ugly-baby-the-name-isnt-all-thats-unusual}}

\href{https://www.nytimes3xbfgragh.onion/slideshow/2017/11/28/dining/ugly-baby-nyc.html}{}

\hypertarget{ugly-baby}{%
\subsection{Ugly Baby}\label{ugly-baby}}

7 Photos

View Slide Show ›

\includegraphics{https://static01.graylady3jvrrxbe.onion/images/2017/11/29/dining/29REST-slide-HI53/29REST-slide-HI53-articleLarge.jpg?quality=75\&auto=webp\&disable=upscale}

Danny Ghitis for The New York Times

\begin{itemize}
\tightlist
\item
  Ugly Baby\\
  ★★ Thai \$\$ 407 Smith Street 347-689-3075
\end{itemize}

By \href{http://www.nytimes3xbfgragh.onion/by/pete-wells}{Pete Wells}

\begin{itemize}
\item
  Nov. 28, 2017
\item
  \begin{itemize}
  \item
  \item
  \item
  \item
  \item
  \item
  \end{itemize}
\end{itemize}

If you've ever hired a contractor, someone probably told you that you
can get work that's good, fast or cheap, but not all three. This may be
an immutable law for kitchen cabinets, but it is not always true of Thai
restaurants, as I learned at a new place in Brooklyn with the memorable
name \href{https://www.uglybabynyc.com/}{Ugly Baby}.

We were still asking ourselves whether we'd ordered too much when the
first plate landed: four little fried coconut cakes, called tue ka ko,
with sweet black beans and taro sticks embedded in their tops under a
handful of chopped peanuts. Imagine coming across a snack that combines
the best qualities of the doughnut, the mini-muffin and
\href{https://www.orwashers.com/}{Orwashers'} toasted-coconut macaroon,
and finding it not at some hipster's stand at
\href{https://www.smorgasburg.com/}{Smorgasburg} but on a dim sum
trolley, and you have a pretty good idea of Ugly Baby's tue ka ko.

It was the perfect dessert for a meal we hadn't eaten yet, I decided,
just as the next dish touched down. This was kang hoh, rice vermicelli
tangled up with tender pork in a tamarind-soured dry curry from northern
Thailand. The bits of fried pork skin on top, curled like Fritos, gave
me something to munch on while I wondered how much of my tongue the
curry would shut down.

Not too much, it turned out; the steady drive of the heat just made me
want more. But by the time that became clear, the table was full. It had
taken about five minutes. Nothing cost more than \$25. All of it was
very good and a few things were so exceptional that they are now pinned
to my map of great Thai dishes in the city, up on the mental wall with
such standbys as the duck curry at
\href{http://www.nytimes3xbfgragh.onion/2010/09/08/dining/reviews/08under.html}{Ayada}
and the papaya salad with crispy ground catfish at
\href{http://www.nytimes3xbfgragh.onion/2004/11/03/dining/a-thai-pilgrimage-leads-to-queens.html?_r=0}{Sripraphai}.

\includegraphics{https://static01.graylady3jvrrxbe.onion/images/2017/11/29/dining/29REST-slide-X3ZS/29REST-slide-X3ZS-articleLarge.jpg?quality=75\&auto=webp\&disable=upscale}

Kang kua supparod is one of them. The menu terms it a mushroom pineapple
curry, but you will get a sharper idea if you picture mushroom slices
and squares of pressed tofu paddling around in a creamy yellow sauce of
crushed pineapple and coconut milk. You can see the lime leaves that
flavor it, but you only taste the lemongrass, galangal, chiles and other
complications.

Another is a kao tod nom klook, a warm rice salad with roots in Laos.
The rice is cooked with a dry curry and a nontrivial quantity of chiles,
then seared to imprint a chewy, golden crust on it. Stirred with short
bits of long beans, cilantro, mint, peanuts and, crucially, some tangy
pink crumbles of fermented pork sausage, it performs best when wrapped
in lettuce with a basil leaf, a dried chile and anything else you happen
to discover on the plate.

Ugly Baby is in Carroll Gardens, across Smith Street from the hole in
the earth where the F and G tracks begin to climb toward their high-rise
crawl over the Gowanus Canal. Open since August, the restaurant traces
its lineage back to two restaurants that had brief lives not far away in
Red Hook. One, Chiang Mai, a pop-up inside another establishment, was
supposed to be temporary. The other, Kao Soy, wasn't, but the two owners
found that their partnership was not built for the long haul.

What the three restaurants share is Sirichai Sreparplarn. He shared chef
duties at the first two and takes the wheel himself at Ugly Baby, which
is named after the Thai superstition that evil spirits won't bother
harming unattractive infants. The walls are painted in a James
Rosenquist palette with quick, energetic brush strokes. The cooks and
servers wear cartoon colored T-shirts, sneakers and aprons, except for
the younger cook whose apron is printed with a monochromatic photograph
of hand-lettered cassettes. There's buzzy guitar pop playing at volumes
that can make the restaurant sound like a bar, even though the
refrigerator at the entrance to the open kitchen isn't stocked with wine
and beer yet.

When
\href{https://www.nytimes3xbfgragh.onion/2015/02/11/dining/restaurant-review-kao-soy-in-red-hook-brooklyn.html}{I
reviewed Kao Soy in 2015}, Mr. Sreparplarn told me that he and his
co-chef were passionate about serving ``the real food from the north.''
He has let himself have the run of the country at Ugly Baby, although he
still gives the north its due.

In his larb, made with duck in the style of the Isan city of Udon Thani,
smashed dried chiles do not so much coat the meat as embed themselves in
it. As you chew, you summon up new waves of heat that the fresh leaves
of Vietnamese coriander and bites of raw cabbage and cucumbers can do
only so much to mitigate.

Image

In Ugly Baby's kang hoh, rice vermicelli are tangled up with tender pork
in a tamarind-soured dry curry.Credit...Danny Ghitis for The New York
Times

Its only rival for sheer atomic power comes from southern Thailand. Kua
kling is a beef curry that is stir-fried until it is completely dry and
there is almost no hope of separating the meat from the chiles. Mr.
Sreparplarn's version is cooked with needles of lime leaf and clusters
of green peppercorns on their stalks. I tasted them right away, although
it took five minutes before I was able to tell anybody about it.

Southern Thai cooking isn't all scorched earth, though. Ugly Baby has a
soothing fried sea bream rubbed with fresh turmeric and garlic, a
favorite way to cook fish in that part of the country. Another calming
seafood dish, this one from central Thailand, looks like a soup, but its
tea-colored broth --- soured with tamarind and brightened by sticks of
ginger --- is simply meant to flavor the fillets of red snapper in it,
and keep them hot.

Some of these dishes are rare in New York. You can't say that about the
grilled chicken skewers marinated in coconut-peanut sauce, but Ugly
Baby's might be the best available version of the dish, made with meaty
thighs instead of stringy white meat.

Like Mr. Sreparplarn's first two restaurants, Ugly Baby serves a rich
kao soy, with beef in the coconut broth instead of chicken this time. I
have to admit that I miss the crisp haystack of fried noodles and green
papaya fritters that the first restaurant piled on top, and I wish Mr.
Sreparplarn added in more pickled mustard greens and less chile oil (or
served condiments on the side and let you decide how sour and spicy to
make the soup, as Kao Soy did).

The one dessert, when it is available, is rice pudding with durian. You
may love this. Or, like me, you may try a spoonful to see if you still
think durian tastes like boiled pearl onions. (Yes.) Or you may want to
keep your distance. If you're in that group, and you haven't ordered the
coconut mini-muffins, now's your chance.

\emph{\href{https://www.facebookcorewwwi.onion/nytfood/}{Follow NYT Food
on Facebook},} \emph{\href{https://instagram.com/nytfood}{Instagram},}
\emph{\href{https://twitter.com/nytfood}{Twitter}} \emph{and}
\emph{\href{https://www.pinterest.com/nytfood/}{Pinterest}.}
\emph{\href{https://www.nytimes3xbfgragh.onion/newsletters/cooking}{Get
regular updates from NYT Cooking, with recipe suggestions, cooking tips
and shopping advice}.}

Advertisement

\protect\hyperlink{after-bottom}{Continue reading the main story}

\hypertarget{site-index}{%
\subsection{Site Index}\label{site-index}}

\hypertarget{site-information-navigation}{%
\subsection{Site Information
Navigation}\label{site-information-navigation}}

\begin{itemize}
\tightlist
\item
  \href{https://help.nytimes3xbfgragh.onion/hc/en-us/articles/115014792127-Copyright-notice}{©~2020~The
  New York Times Company}
\end{itemize}

\begin{itemize}
\tightlist
\item
  \href{https://www.nytco.com/}{NYTCo}
\item
  \href{https://help.nytimes3xbfgragh.onion/hc/en-us/articles/115015385887-Contact-Us}{Contact
  Us}
\item
  \href{https://www.nytco.com/careers/}{Work with us}
\item
  \href{https://nytmediakit.com/}{Advertise}
\item
  \href{http://www.tbrandstudio.com/}{T Brand Studio}
\item
  \href{https://www.nytimes3xbfgragh.onion/privacy/cookie-policy\#how-do-i-manage-trackers}{Your
  Ad Choices}
\item
  \href{https://www.nytimes3xbfgragh.onion/privacy}{Privacy}
\item
  \href{https://help.nytimes3xbfgragh.onion/hc/en-us/articles/115014893428-Terms-of-service}{Terms
  of Service}
\item
  \href{https://help.nytimes3xbfgragh.onion/hc/en-us/articles/115014893968-Terms-of-sale}{Terms
  of Sale}
\item
  \href{https://spiderbites.nytimes3xbfgragh.onion}{Site Map}
\item
  \href{https://help.nytimes3xbfgragh.onion/hc/en-us}{Help}
\item
  \href{https://www.nytimes3xbfgragh.onion/subscription?campaignId=37WXW}{Subscriptions}
\end{itemize}
