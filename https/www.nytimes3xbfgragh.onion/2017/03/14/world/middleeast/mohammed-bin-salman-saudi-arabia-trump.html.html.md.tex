Sections

SEARCH

\protect\hyperlink{site-content}{Skip to
content}\protect\hyperlink{site-index}{Skip to site index}

\href{https://www.nytimes3xbfgragh.onion/section/world/middleeast}{Middle
East}

\href{https://myaccount.nytimes3xbfgragh.onion/auth/login?response_type=cookie\&client_id=vi}{}

\href{https://www.nytimes3xbfgragh.onion/section/todayspaper}{Today's
Paper}

\href{/section/world/middleeast}{Middle East}\textbar{}Trump Meets Saudi
Prince as U.S. and Kingdom Seek Warmer Relations

\url{https://nyti.ms/2mH1D6A}

\begin{itemize}
\item
\item
\item
\item
\item
\end{itemize}

Advertisement

\protect\hyperlink{after-top}{Continue reading the main story}

Supported by

\protect\hyperlink{after-sponsor}{Continue reading the main story}

\hypertarget{trump-meets-saudi-prince-as-us-and-kingdom-seek-warmer-relations}{%
\section{Trump Meets Saudi Prince as U.S. and Kingdom Seek Warmer
Relations}\label{trump-meets-saudi-prince-as-us-and-kingdom-seek-warmer-relations}}

\includegraphics{https://static01.graylady3jvrrxbe.onion/images/2017/03/15/us/15PREXY-01/15PREXY-01-articleInline.jpg?quality=75\&auto=webp\&disable=upscale}

By
\href{https://www.nytimes3xbfgragh.onion/by/julie-hirschfeld-davis}{Julie
Hirschfeld Davis}

\begin{itemize}
\item
  March 14, 2017
\item
  \begin{itemize}
  \item
  \item
  \item
  \item
  \item
  \end{itemize}
\end{itemize}

WASHINGTON --- President Trump hosted Deputy Crown Prince Mohammed bin
Salman of Saudi Arabia for lunch at the White House on Tuesday, moving
to forge a warmer relationship with the kingdom after a period of
tension between the United States and a longstanding ally.

The lunch was an early effort by Mr. Trump to engage with Prince
Mohammed, the defense minister of Saudi Arabia. Hopes are high in Riyadh
for improved relations with the United States after strained diplomacy
between the Obama administration and the Saudis, particularly over the
nuclear deal with Iran.

The visit --- initially expected to be a short meet-and-greet but turned
at the last moment into a formal lunch --- was a chance for the two men
to also discuss Yemen, where a civil war has pitted Iranian-aligned
Houthi rebels against a Saudi-led coalition of mostly Sunni Arab
countries with American support, and where the United States is
\href{https://www.nytimes3xbfgragh.onion/2017/03/03/world/middleeast/yemen-us-airstrikes-al-qaeda.html}{stepping
up} a campaign against Al Qaeda. Mr. Trump faces a decision on whether
to resume arms sales to the Saudis.

Mr. Trump, a new American president eager to break with his predecessor,
and Prince Mohammed, a young, ambitious leader jockeying for
\href{https://www.nytimes3xbfgragh.onion/2016/10/16/world/rise-of-saudi-prince-shatters-decades-of-royal-tradition.html}{influence
in his kingdom}, each see the other as a crucial ally on a variety of
pressing issues. Neither spoke to reporters as they shook hands in the
Oval Office, or later, when they took seats in the State Dining Room for
a lunch with senior aides.

The president was expected to urge Saudi Arabia to support
\href{https://www.nytimes3xbfgragh.onion/2016/12/15/us/politics/syria-safe-zones-donald-trump.html}{safe
zones in Syria}, which the administration has argued would be an
alternative to accepting thousands of refugees from a country that has
been ripped apart by
\href{https://www.nytimes3xbfgragh.onion/2016/09/19/world/middleeast/syria-civil-war-bashar-al-assad-refugees-islamic-state.html}{six
years of civil war}.

Mr. Trump and members of his inner circle regard Saudi Arabia as a vital
component of the White House
\href{https://www.nytimes3xbfgragh.onion/2017/02/09/world/middleeast/trump-arabs-palestinians-israel.html}{strategy}
to get Middle East allies to help break the deadlock in the conflict
between Israelis and Palestinians. That approach is said to be favored
by Jared Kushner, Mr. Trump's son-in-law and senior adviser, who has
been tasked with forging a peace between the two sides.

The president and his top aides ``see Saudi Arabia as a crucial part of
the Middle East and an important country to have a positive relationship
with, even if there are irritants,'' said Simon Henderson, the director
of the Gulf and Energy Policy Program at the Washington Institute for
Near East Policy*.* ``This is at odds with the Obama administration, so
they want to make that clear distinction.''

Saudi Arabia and its Persian Gulf neighbors have been optimistic about
Mr. Trump's presidency, largely because of their deep frustration at
what they called Mr. Obama's refusal to forcefully engage in Middle
Eastern issues like the war in Syria. They are encouraged by Mr. Trump's
business background, his lack of interest in human rights and, most
importantly, his vow to take a
\href{https://www.nytimes3xbfgragh.onion/2017/02/07/world/middleeast/trump-iran-ayatollah-ali-khamenei.html}{hard
line against Iran}.

``They were happy to see Obama go,'' Bruce Riedel, a senior fellow at
the Brookings Institution, said of the Saudis. Mr. Riedel said the
kingdom had lost confidence in Mr. Obama after the Arab Spring swept
across several countries in the Middle East and North Africa in 2011 and
because his administration often pressured Middle Eastern leaders, such
as President Abdel Fattah el-Sisi of Egypt, on human rights concerns.

``Trump has made it clear he is not worried about supporting human
rights or freedom; he's made clear that Sisi is going to be his best
friend in Egypt; that all those difficult questions about gender
equality and the like are going to be off the table for the next four
years, and that Iran is very much on the table,'' Mr. Riedel said. ``As
the Saudis look at Trump, they see they don't need to worry about any of
that.''

Still, like many other leaders around the world, the Saudis view Mr.
Trump with some degree of wariness, uncertain about the basic competency
of his administration and eager to size up a president who has no
experience in handling geopolitical affairs. Mr. Trump also sent some
mixed signals to the Islamic world in the opening days of his
presidency, including in signing
\href{https://www.nytimes3xbfgragh.onion/2017/01/27/us/politics/trump-syrian-refugees.html}{a
travel ban} targeting predominantly Muslim countries, which excluded
Saudi Arabia but was widely regarded as the fulfillment of a campaign
promise to enact a ``Muslim ban.''

Mr. Riedel said the Trump administration --- and particularly Jim
Mattis, the secretary of defense --- ``recognizes that we need to
clarify that signal with the Saudis, and the best way to do it is with
the king's favorite son.''

Prince Mohammed, 31, is second in line to the throne. He oversees Saudi
Aramco, the state-owned oil company, and serves as defense minister,
putting him in charge of the Saudi-led intervention in Yemen. Saudi
officials see the Houthis in Yemen as a national threat and would like
greater American assistance in the fight against them. Saudi Arabia is a
major buyer of American weapons.

Prince Mohammed is also the guiding force behind a plan, known as Vision
2030, to transform the kingdom and reduce its dependence on oil. While
Prince Mohammed is in Washington, his father, King Salman, is touring
Asia in a trip aimed at attracting foreign investment to the kingdom.

Mr. Trump had been scheduled to spend much of his day on Tuesday with
Chancellor Angela Merkel of Germany. A snowstorm that blanketed much of
the northeastern United States prompted Ms. Merkel to delay her visit
until Friday, leaving Mr. Trump's lunch hour available for Prince
Mohammed.

Joining the president for the meetings were Vice President Mike Pence;
H. R. McMaster, the national security adviser; Stephen K. Bannon, the
president's chief strategist; Reince Priebus, the White House chief of
staff; and Mr. Kushner. Mr. Mattis is expected to meet with Prince
Mohammed later this week.

Advertisement

\protect\hyperlink{after-bottom}{Continue reading the main story}

\hypertarget{site-index}{%
\subsection{Site Index}\label{site-index}}

\hypertarget{site-information-navigation}{%
\subsection{Site Information
Navigation}\label{site-information-navigation}}

\begin{itemize}
\tightlist
\item
  \href{https://help.nytimes3xbfgragh.onion/hc/en-us/articles/115014792127-Copyright-notice}{©~2020~The
  New York Times Company}
\end{itemize}

\begin{itemize}
\tightlist
\item
  \href{https://www.nytco.com/}{NYTCo}
\item
  \href{https://help.nytimes3xbfgragh.onion/hc/en-us/articles/115015385887-Contact-Us}{Contact
  Us}
\item
  \href{https://www.nytco.com/careers/}{Work with us}
\item
  \href{https://nytmediakit.com/}{Advertise}
\item
  \href{http://www.tbrandstudio.com/}{T Brand Studio}
\item
  \href{https://www.nytimes3xbfgragh.onion/privacy/cookie-policy\#how-do-i-manage-trackers}{Your
  Ad Choices}
\item
  \href{https://www.nytimes3xbfgragh.onion/privacy}{Privacy}
\item
  \href{https://help.nytimes3xbfgragh.onion/hc/en-us/articles/115014893428-Terms-of-service}{Terms
  of Service}
\item
  \href{https://help.nytimes3xbfgragh.onion/hc/en-us/articles/115014893968-Terms-of-sale}{Terms
  of Sale}
\item
  \href{https://spiderbites.nytimes3xbfgragh.onion}{Site Map}
\item
  \href{https://help.nytimes3xbfgragh.onion/hc/en-us}{Help}
\item
  \href{https://www.nytimes3xbfgragh.onion/subscription?campaignId=37WXW}{Subscriptions}
\end{itemize}
