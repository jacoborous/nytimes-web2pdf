Sections

SEARCH

\protect\hyperlink{site-content}{Skip to
content}\protect\hyperlink{site-index}{Skip to site index}

\href{https://www.nytimes3xbfgragh.onion/section/world/asia}{Asia
Pacific}

\href{https://myaccount.nytimes3xbfgragh.onion/auth/login?response_type=cookie\&client_id=vi}{}

\href{https://www.nytimes3xbfgragh.onion/section/todayspaper}{Today's
Paper}

\href{/section/world/asia}{Asia Pacific}\textbar{}North Korea, Citing
Kim Jong-nam Dispute, Blocks Malaysians From Exiting

\url{https://nyti.ms/2n8MgDb}

\begin{itemize}
\item
\item
\item
\item
\item
\end{itemize}

Advertisement

\protect\hyperlink{after-top}{Continue reading the main story}

Supported by

\protect\hyperlink{after-sponsor}{Continue reading the main story}

\hypertarget{north-korea-citing-kim-jong-nam-dispute-blocks-malaysians-from-exiting}{%
\section{North Korea, Citing Kim Jong-nam Dispute, Blocks Malaysians
From
Exiting}\label{north-korea-citing-kim-jong-nam-dispute-blocks-malaysians-from-exiting}}

\includegraphics{https://static01.graylady3jvrrxbe.onion/images/2017/03/08/world/08malaysia-1/08malaysia-1-articleInline.jpg?quality=75\&auto=webp\&disable=upscale}

By
\href{https://www.nytimes3xbfgragh.onion/by/richard-c-paddock}{Richard
C. Paddock}

\begin{itemize}
\item
  March 7, 2017
\item
  \begin{itemize}
  \item
  \item
  \item
  \item
  \item
  \end{itemize}
\end{itemize}

BANGKOK --- North Korea said Tuesday that it was barring all Malaysians
from leaving the country until there was a ``fair settlement'' of a
dispute over the assassination in Kuala Lumpur of
\href{https://www.nytimes3xbfgragh.onion/2017/02/15/world/asia/kim-jong-nam-assassination-north-korea.html}{Kim
Jong-nam}, the half brother of North Korea's leader.

Malaysia responded in kind, with Prime Minister Najib Razak instructing
the police to prevent all North Koreans from leaving Malaysia until he
was assured of the safety of Malaysians in North Korea.

The developments were a drastic escalation in the diplomatic dispute
over
\href{https://www.nytimes3xbfgragh.onion/2017/02/22/world/asia/kim-jong-nam-assassination-korea-malaysia.html}{Mr.
Kim's killing}. The Malaysian police have said that several North
Koreans are suspects.

``This abhorrent act, effectively holding our citizens hostage, is in
total disregard of all international law and diplomatic norms,'' Mr.
Najib said of North Korea's action.

Mr. Najib convened an emergency meeting of the National Security Council
in the evening.

In its statement on Tuesday, North Korea said it would ``temporarily ban
the exit of Malaysian citizens'' until the safety of North Korean
diplomats and citizens in Malaysia is ``fully guaranteed through the
fair settlement of the case that occurred in Malaysia,'' the state-run
Korean Central News Agency reported.

It was unclear what resolution to the Kim case North Korea was seeking.
But it has rejected the findings of the Malaysian police that
\href{https://www.nytimes3xbfgragh.onion/2017/03/02/world/asia/kim-jong-nam-malaysia.html}{Mr.
Kim was poisoned} by VX nerve agent at the Malaysian capital's
international airport on Feb. 13, and it has demanded that his body be
handed over to the North Korean Embassy.

The Malaysian police want to question several North Koreans in the case,
including a diplomat.

Malaysian officials said there were 11 Malaysians in the North who could
be affected by the North Korean ban, including embassy staff members,
their family and two workers for the United Nations.

After the security council meeting, Mr. Najib
\href{https://twitter.com/NajibRazak/status/839133732489441280}{posted
on his Twitter account}: ``I know that the family and friends of our
fellow Malaysians detained in North Korea are anxiously anticipating
news of their loved ones.'' He added in a
\href{https://twitter.com/NajibRazak/status/839135526946914304}{second
posting}: ``You can rest assured that we are doing our very best to
secure their safe return.''

About 1,000 North Koreans are believed to live and work in Malaysia;
until Monday, they had been allowed to enter the country without a visa.

``As a peace-loving nation, Malaysia is committed to maintaining
friendly relations with all countries,'' Mr. Najib said on Tuesday.
``However, protecting our citizens is my first priority, and we will not
hesitate to take all measures necessary when they are threatened.''

Mr. Kim, the elder half brother of the North Korean leader, Kim Jong-un,
was killed when two women rubbed poison on his face at Kuala Lumpur
International Airport, the Malaysian police said. The women, one from
Vietnam and one from Indonesia, have been arrested and
\href{https://www.nytimes3xbfgragh.onion/2017/02/28/world/asia/north-korea-kim-jong-nam-death.html}{charged
with murder}.

The Malaysian police, who conducted an autopsy of Mr. Kim's body over
North Korea's objections, concluded that he had been poisoned by
\href{https://www.nytimes3xbfgragh.onion/2017/02/24/world/asia/vx-nerve-agent-kim-jong-nam.html}{VX
nerve agent}, a banned chemical weapon known to be in North Korea's
arsenal. North Korea has suggested that he
\href{https://www.nytimes3xbfgragh.onion/2017/03/01/world/asia/kim-jong-nam-assassination-north-korea-visa-malaysia.html}{died
of heart failure} and accused Malaysia of working with other countries
to defame North Korea.

``Once it denied responsibility for the assassination, North Korea had
no option but to push back in a tit-for-tat escalation,'' Kim Yong-hyun,
a professor of North Korean studies at Dongguk University in Seoul,
South Korea, said on Tuesday. ``Offense is the best defense for the
North.''

Preventing the Malaysians from leaving North Korea would also give the
government continuing leverage over Malaysia. If the Malaysians had been
free to leave, Malaysia could have broken off diplomatic relations
without any significant political cost.

That would have led to the closing of the North Korean Embassy, with at
least one suspect who has taken refuge there no longer safe from arrest.

The suspect, Kim Uk-il, an employee of the state-owned North Korean
airline, Air Koryo, could be arrested if the embassy were closed. A
second suspect who the police say may be hiding at the embassy, Ri Ji-u,
also known as James, would also be subject to arrest.

A third suspect, Hyon Kwang-song, a second secretary at the embassy, has
diplomatic immunity and could not be arrested.

``If we break diplomatic ties, then all the embassy staff have to leave
Malaysia, but the staff with diplomatic immunity at the time of the
offense is still safe and must be allowed to leave,'' said Sivananthan
Nithyanantham, a Malaysian lawyer who has served as counsel at the
International Criminal Court in The Hague. ``The airline worker then
loses his sanctuary and will be liable to arrest.''

The police are seeking seven North Korean men in connection with Mr.
Kim's killing. The other four are believed to have returned to North
Korea.

Khalid Abu Bakar, Malaysia's top police official, confirmed at a news
conference on Tuesday that at least two suspects had
\href{https://www.nytimes3xbfgragh.onion/2017/03/01/world/asia/malaysia-kim-jong-nam-embassy-immunity.html}{taken
refuge at the North Korean Embassy} and that North Korea had refused a
request to hand them over.

``The North Korean authorities are not cooperating with us in this
investigation,'' he said.

He said the police would wait as long as necessary to arrest Mr. Kim,
the airline employee, and Mr. Ri, if he is there.

``If it takes five years, we will wait outside,'' he said. ``Definitely
somebody will come out.''

North Korea has denied responsibility for the killing and has not
acknowledged that the victim was Kim Jong-nam.

Lim Kit Siang, a leader of Malaysia's opposition Democratic Action
Party, called on Parliament to adopt an emergency motion condemning what
he called
\href{https://blog.limkitsiang.com/2017/03/07/parliament-should-adopt-an-all-party-emergency-motion-tomorrow-to-condemn-north-koreas-hostage-terrorism-and-to-call-on-north-korea-regime-to-immediately-revoke-the-ban-on-m/\#more-36730}{North
Korea's ``hostage terrorism''} and urging the North Koreans to let the
Malaysians leave.

North Korea's statement on Tuesday described the exit ban as temporary.
But the North Korean government has been accused of playing hostage
politics before, partly to complicate negotiations over its nuclear arms
and missile development. In 2014,
\href{https://www.nytimes3xbfgragh.onion/2014/05/30/world/asia/north-korea-agrees-to-investigate-fate-of-japanese-abducted-decades-ago.html}{North
Korea said it would reopen an investigation} into Japanese citizens it
was accused of abducting during the Cold War, but it halted that inquiry
last year
\href{https://www.nytimes3xbfgragh.onion/2016/02/14/world/asia/north-korea-japan-abductions.html}{in
retaliation for sanctions} imposed by Japan over a rocket launch.

Duyeon Kim, a Seoul-based nonresident fellow at Georgetown University's
Institute for the Study of Diplomacy, said on Tuesday that North Korea
was ``playing dirty and not diplomatically, apparently hoping this might
force Malaysia to reverse its findings'' about Mr. Kim's killing.

Malaysia, however, showed every intention of pressing ahead with its
contention that VX nerve agent had been used in the Kim assassination.

On Tuesday, the Malaysians presented their formal report about their
findings to the executive council of the Organization for the
Prohibition of Chemical Weapons, the group based in The Hague that
monitors compliance with the Chemical Weapons Convention, which Malaysia
has signed.

In a
\href{https://www.opcw.org/fileadmin/OPCW/EC/84/en/Malaysia_ec84_statement.pdf}{statement
to the executive council}, Malaysia noted that it did not ``produce,
stockpile, import, export or use'' VX or any other such chemical weapon.

``Malaysia strongly condemns the use of such a chemical by anyone,
anywhere and under any circumstances,'' the statement said. ``Its use at
a public place could have endangered the general public.''

North Korea, which has not signed the Chemical Weapons Convention, is
believed to have a large stockpile of VX despite its denials.

Tuesday's developments follow the tit-for-tat expulsion of ambassadors
between the two countries. Kang Chol, North Korea's ambassador to
Malaysia,
\href{https://www.nytimes3xbfgragh.onion/2017/03/06/world/asia/vx-north-korea-kim-jong-nam.html}{was
expelled} on Monday over what Malaysia considered to be insulting
comments. North Korea responded by formally expelling Malaysia's
ambassador, Mohamad Nizan Mohamad, though he had already been recalled
to Malaysia for consultations.

Advertisement

\protect\hyperlink{after-bottom}{Continue reading the main story}

\hypertarget{site-index}{%
\subsection{Site Index}\label{site-index}}

\hypertarget{site-information-navigation}{%
\subsection{Site Information
Navigation}\label{site-information-navigation}}

\begin{itemize}
\tightlist
\item
  \href{https://help.nytimes3xbfgragh.onion/hc/en-us/articles/115014792127-Copyright-notice}{©~2020~The
  New York Times Company}
\end{itemize}

\begin{itemize}
\tightlist
\item
  \href{https://www.nytco.com/}{NYTCo}
\item
  \href{https://help.nytimes3xbfgragh.onion/hc/en-us/articles/115015385887-Contact-Us}{Contact
  Us}
\item
  \href{https://www.nytco.com/careers/}{Work with us}
\item
  \href{https://nytmediakit.com/}{Advertise}
\item
  \href{http://www.tbrandstudio.com/}{T Brand Studio}
\item
  \href{https://www.nytimes3xbfgragh.onion/privacy/cookie-policy\#how-do-i-manage-trackers}{Your
  Ad Choices}
\item
  \href{https://www.nytimes3xbfgragh.onion/privacy}{Privacy}
\item
  \href{https://help.nytimes3xbfgragh.onion/hc/en-us/articles/115014893428-Terms-of-service}{Terms
  of Service}
\item
  \href{https://help.nytimes3xbfgragh.onion/hc/en-us/articles/115014893968-Terms-of-sale}{Terms
  of Sale}
\item
  \href{https://spiderbites.nytimes3xbfgragh.onion}{Site Map}
\item
  \href{https://help.nytimes3xbfgragh.onion/hc/en-us}{Help}
\item
  \href{https://www.nytimes3xbfgragh.onion/subscription?campaignId=37WXW}{Subscriptions}
\end{itemize}
