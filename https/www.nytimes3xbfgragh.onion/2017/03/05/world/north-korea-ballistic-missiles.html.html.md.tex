Sections

SEARCH

\protect\hyperlink{site-content}{Skip to
content}\protect\hyperlink{site-index}{Skip to site index}

\href{https://www.nytimes3xbfgragh.onion/section/world/asia}{Asia
Pacific}

\href{https://myaccount.nytimes3xbfgragh.onion/auth/login?response_type=cookie\&client_id=vi}{}

\href{https://www.nytimes3xbfgragh.onion/section/todayspaper}{Today's
Paper}

\href{/section/world/asia}{Asia Pacific}\textbar{}North Korea's Launch
of Ballistic Missiles Raises New Worries

\url{https://nyti.ms/2n3PFDb}

\begin{itemize}
\item
\item
\item
\item
\item
\end{itemize}

Advertisement

\protect\hyperlink{after-top}{Continue reading the main story}

Supported by

\protect\hyperlink{after-sponsor}{Continue reading the main story}

\hypertarget{north-koreas-launch-of-ballistic-missiles-raises-new-worries}{%
\section{North Korea's Launch of Ballistic Missiles Raises New
Worries}\label{north-koreas-launch-of-ballistic-missiles-raises-new-worries}}

\includegraphics{https://static01.graylady3jvrrxbe.onion/images/2017/03/06/world/06MISSILES/06MISSILES-articleInline.jpg?quality=75\&auto=webp\&disable=upscale}

By \href{http://www.nytimes3xbfgragh.onion/by/choe-sang-hun}{Choe
Sang-Hun}

\begin{itemize}
\item
  March 5, 2017
\item
  \begin{itemize}
  \item
  \item
  \item
  \item
  \item
  \end{itemize}
\end{itemize}

SEOUL, South Korea --- North Korea launched four ballistic missiles from
its long-range rocket launch site on Monday morning, the South Korean
military said. The launch prompted South Korean security officials to
call for the early deployment of an advanced American missile defense
system that has provoked China.

The missiles took off from Tongchang-ri, in northwest North Korea, and
flew an average of 620 miles before falling into the sea between North
Korea and Japan, said Noh Jae-chon, a South Korean military spokesman.
The type of missile fired was not immediately clear, but Mr. Noh said it
was unlikely that they were intercontinental ballistic missiles, which
the North had recently threatened to test launch.

During a meeting of the National Security Council, Hwang Kyo-ahn, the
acting president of South Korea, called for the early deployment of the
American missile defense system known as Thaad, or Terminal
High-Altitude Area Defense.

The United States and South Korea have agreed to complete the Thaad
deployment within the year. They say it is meant to protect South Korea
and American military sites there from North Korean missiles. But China
says Thaad would undermine its own nuclear deterrent and has hinted at
economic retaliation against South Korea.

Mr. Hwang also called on his government to look aggressively for ``ways
to effectively strengthen the United States' extended deterrence'' for
South Korea, referring to Washington's ability to deter attacks on its
allies with the help of its nuclear forces. Mr. Hwang did not elaborate,
but his comment came days after The New York Times
\href{https://www.nytimes3xbfgragh.onion/2017/03/04/world/asia/north-korea-missile-program-sabotage.html?rref=collection\%2Fsectioncollection\%2Fasia\&action=click\&contentCollection=asia\&region=stream\&module=stream_unit\&version=latest\&contentPlacement=1\&pgtype=sectionfront\&_r=0}{reported
that President Trump's} national security deputies recently discussed
various options against North Korea, including the possibility of
reintroducing nuclear weapons to South Korea as a bold warning.

``If North Korea gets a hold of nuclear weapons, its consequences are
too horrible to think about,'' Mr. Hwang said.

In his New Year's Day speech, the North Korean leader, Kim Jong-un, said
his country was in the ``final stage'' of preparing for its first ICBM
test. In February, the
\href{https://www.nytimes3xbfgragh.onion/2017/02/11/world/asia/north-korea-missile-test-trump.html}{North
launched a ballistic missile} that the United States Strategic Command
determined was not a threat to the United States, but North Korea has
said it is ready to test launch an ICBM.

The North's missile launching came as the United States and South Korea
were conducting their annual joint military exercise. North Korea calls
such drills a rehearsal for invasion and has often responded by
conducting missile tests.

On Thursday, the North Korean military called the joint exercise a drill
for ``nuclear war'' and vowed to take unspecified strong measures. The
next day, the North's main state-run newspaper, Rodong Sinmun, hinted at
more missile tests, saying, ``New strategic weapons of our own style
will soar into the sky.''

North Korea has boasted of an ability to strike the continental United
States with a nuclear-tipped missile. It has never tested a missile
capable of flying across the Pacific, although it has displayed what
outside analysts said were ICBMs during military parades in recent
years. Strong doubt also remains over the North's claim that it can
manufacture a nuclear warhead small enough to be fitted onto such a
missile.

But its test on Feb. 12 demonstrated its advancing ballistic missile
technology. The test involved Pukguksong-2, a new intermediate-range
ballistic missile that the North said can carry a nuclear warhead.

The multiple missile launchings illustrated the frustration of the
United Nations Security Council over its inability to halt or contain
North Korea's nuclear ambitions with punitive economic sanctions.

An
\href{https://assets.documentcloud.org/documents/3482392/NORTH-KOREA-REPORT.pdf}{investigative
report} released a week ago by a panel of experts concluded that the
country's leaders had developed an international smuggling network to
foil the sanctions and outmaneuver enforcement measures. The report
described a matrix of North Korean companies with bogus identities used
to accrue cash, technologies and materials for the government's weapons
development.

In remarks to reporters on Monday morning, Yoshihide Suga, the chief
cabinet secretary to Japan's prime minister, Shinzo Abe, said the
missiles appeared to have fallen into the sea in an exclusive economic
zone around Japan. Mr. Suga called the missile launch a ``serious threat
to our security'' as well as ``extremely problematic behavior from the
viewpoint of security of aircraft and ships.'' He said the government
had protested to North Korea.

``We just cannot accept such repeated provocations,'' he said.

Advertisement

\protect\hyperlink{after-bottom}{Continue reading the main story}

\hypertarget{site-index}{%
\subsection{Site Index}\label{site-index}}

\hypertarget{site-information-navigation}{%
\subsection{Site Information
Navigation}\label{site-information-navigation}}

\begin{itemize}
\tightlist
\item
  \href{https://help.nytimes3xbfgragh.onion/hc/en-us/articles/115014792127-Copyright-notice}{©~2020~The
  New York Times Company}
\end{itemize}

\begin{itemize}
\tightlist
\item
  \href{https://www.nytco.com/}{NYTCo}
\item
  \href{https://help.nytimes3xbfgragh.onion/hc/en-us/articles/115015385887-Contact-Us}{Contact
  Us}
\item
  \href{https://www.nytco.com/careers/}{Work with us}
\item
  \href{https://nytmediakit.com/}{Advertise}
\item
  \href{http://www.tbrandstudio.com/}{T Brand Studio}
\item
  \href{https://www.nytimes3xbfgragh.onion/privacy/cookie-policy\#how-do-i-manage-trackers}{Your
  Ad Choices}
\item
  \href{https://www.nytimes3xbfgragh.onion/privacy}{Privacy}
\item
  \href{https://help.nytimes3xbfgragh.onion/hc/en-us/articles/115014893428-Terms-of-service}{Terms
  of Service}
\item
  \href{https://help.nytimes3xbfgragh.onion/hc/en-us/articles/115014893968-Terms-of-sale}{Terms
  of Sale}
\item
  \href{https://spiderbites.nytimes3xbfgragh.onion}{Site Map}
\item
  \href{https://help.nytimes3xbfgragh.onion/hc/en-us}{Help}
\item
  \href{https://www.nytimes3xbfgragh.onion/subscription?campaignId=37WXW}{Subscriptions}
\end{itemize}
