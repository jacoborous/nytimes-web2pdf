Sections

SEARCH

\protect\hyperlink{site-content}{Skip to
content}\protect\hyperlink{site-index}{Skip to site index}

\href{https://myaccount.nytimes3xbfgragh.onion/auth/login?response_type=cookie\&client_id=vi}{}

\href{https://www.nytimes3xbfgragh.onion/section/todayspaper}{Today's
Paper}

\href{/section/upshot}{The Upshot}\textbar{}State I.R.A. Plans Are
Ready, if Congress Doesn't Interfere

\url{https://nyti.ms/2lDygAd}

\begin{itemize}
\item
\item
\item
\item
\item
\item
\end{itemize}

Advertisement

\protect\hyperlink{after-top}{Continue reading the main story}

Supported by

\protect\hyperlink{after-sponsor}{Continue reading the main story}

Upshot

\href{/column/economic-view}{Economic View}

\hypertarget{state-ira-plans-are-ready-if-congress-doesnt-interfere}{%
\section{State I.R.A. Plans Are Ready, if Congress Doesn't
Interfere}\label{state-ira-plans-are-ready-if-congress-doesnt-interfere}}

\includegraphics{https://static01.graylady3jvrrxbe.onion/images/2017/03/05/business/05VIEW/05VIEW-articleLarge.jpg?quality=75\&auto=webp\&disable=upscale}

By Richard H. Thaler

\begin{itemize}
\item
  March 3, 2017
\item
  \begin{itemize}
  \item
  \item
  \item
  \item
  \item
  \item
  \end{itemize}
\end{itemize}

Thirty-nine million Americans work for an employer without a
payroll-deduction retirement savings plan, and many of them are saving
little or nothing. In the absence of a federal plan for this problem,
states including California, Connecticut, Illinois, Maryland and Oregon
have taken it upon themselves to create their own solutions.

This flurry of state-level innovation might be cause for celebration,
except for one major impediment: Congress may kill the nascent plans.
Why? Republicans, who typically call for less regulation, say the state
programs won't be sufficiently regulated. You can't make this stuff up.

The backdrop behind this battle is the long struggle that many Americans
--- even those with workplace retirement plans --- are waging to save
money. Some people fail to join their company's plan, or if they join,
they save too little.

The good news is that by using behavioral economics, we know how to fix
those problems: While giving workers the ability to opt out from these
arrangements, automatically enroll them in a plan and increase
contributions over time, and offer a sensible low-cost default
investment fund. All of that makes saving easy and increases workers'
savings rates substantially.

I've written about these issues extensively, including in the book
``\href{http://www.nytimes3xbfgragh.onion/2008/08/24/books/review/Friedman-t.html}{Nudge}''
with Cass Sunstein, a Harvard law professor. Most large companies now
incorporate these features, so there is hope things will improve for
many people.

But those whose employers offer no plan are at greater risk. This is the
situation for 60 percent of workers in the bottom quarter of the income
distribution. Many of them work for smaller firms that don't offer their
employees a 401(k) or any other plan.

This is another problem we know how to mitigate using behavioral
principles utilizing individual retirement accounts instead of 401(k)s.
A 2006 proposal by the
\href{https://www.brookings.edu/wp-content/uploads/2016/06/07_automatic_ira_iwry.pdf}{Brookings
Institution} and the Heritage Foundation suggested a basic format for
such plans, called the ``Automatic I.R.A.'' The idea is to create a
program to make things as easy and inexpensive as possible for both
employers and workers.

Basically, firms of a certain size that do not offer a retirement plan
would have to enroll their employees in I.R.A.s, allowing them to opt
out, of course. The government would engage private-sector portfolio
managers to manage the money, and by pooling contributions, fees can be
kept low. Britain has rolled out a similar
\href{https://www.nestpensions.org.uk/schemeweb/NestWeb/includes/public/docs/nest-insight-2015,pdf.pdf}{plan}over
the past few years that is working well --- fewer than 10 percent of
workers are opting out. New Zealand has had a similar
\href{http://www.kiwisaver.govt.nz/new/about/summary/}{plan}since 2006.

The Obama administration included a proposal to create such a program
throughout the United States in each of its annual budgets, but it
stalled in Congress. As a result, the states have begun to create their
own automatic I.R.A. programs. There is a catch, though. Because the new
plans are based on I.R.A.s, which haven't traditionally been part of
workplace plans, the states asked for guidance from the Labor
Department. They sought to confirm that employers offering I.R.A.s under
the states' programs would not be subject to the requirements of the
Employee Retirement Income Security Act (Erisa) and that the state
programs themselves would not be seen as attempting to circumvent Erisa.
Erisa imposes safeguards, including the requirement that employers meet
their fiduciary responsibility to run retirement plans in the best
interest of their workers.

The department spent more than a year studying the issue, proposing
rules, studying public comments and issuing final
\href{https://www.federalregister.gov/documents/2016/08/30/2016-20639/savings-arrangements-established-by-states-for-non-governmental-employees}{regulations}
in 2016 that accommodated the state plans. In a tiny but crucial change,
the new rules said the state plans merely had to be ``voluntary'' as
opposed to ``completely voluntary,'' the requirement for traditional
I.R.A.s used in a workplace setting. This shift made it possible to
offer the essential automatic enrollment provision, which nudges people
to save, and only requires an active choice for those who do not wish to
save. Alternatively, where workers have to go out of their way to sign
up to save, usage is low, especially for I.R.A.s. The states planning to
start their own plans were happy, and other states have been expressing
interest in following suit.

So far, so good. Power was devolving to individual states; regulatory
burdens were easing; and American workers were being given a chance to
easily save for retirement and become part of what President George W.
Bush called the ``ownership society.''

Yet some Republicans in Congress are now trying to override the new
rule.

Their tactic uses the obscure
\href{https://www.nytimes3xbfgragh.onion/2017/01/30/us/politics/congressional-review-act-obama-regulations.html?_r=0}{Congressional
Review Act}, which allows Congress to undo recent regulations with a
simple majority vote in the House of Representatives and the Senate. By
rule, they must do so within 60 ``Congressional days'' of the issuance
of a regulation, defined as days that Congress is in session. The House
has already passed a
\href{https://www.congress.gov/bill/115th-congress/house-joint-resolution/66/text}{resolution}
of disapproval revoking the new rules, but the Senate has not yet taken
up the issue.

This is odd, considering that giving more power to the states is a
traditional Republican mantra, often mentioned as part of plans to
replace the Affordable Care Act with local alternatives. What's more,
helping people save used to be an issue with bipartisan support. The
federal Pension Protection Act of 2006, which had wide support from both
parties (and
\href{http://www.nytimes3xbfgragh.onion/2007/02/14/opinion/15talkingpoints.html}{from
me}, too), included rules that encouraged employers to adopt automatic
enrollment and automatic savings increases, with opt-out provisions.

Yet in a news
\href{http://edworkforce.house.gov/news/documentsingle.aspx?DocumentID=401321}{release}
and op-ed piece in
\href{http://www.latimes.com/opinion/opinion-la/la-ed-state-ira-blowback-20170213-story.html}{The
Los Angeles Times}, the sponsors of the move to stop the state plans in
the House, Francis Rooney, a Republican from Florida, and Tim Walberg, a
Republican from Michigan, described the Labor Department's rule as a
``last-minute regulatory loophole.'' That's an interesting choice of
words: The rule took about two years to complete.

The news release says ``hardworking Americans could be forced into
government-run plans,'' failing to recognize that automatic enrollment
includes an easy way to opt out at any time. It works very well for
millions of people in 401(k) plans.

Another claim (without any supporting evidence) is that these state
plans will discourage firms from offering their own retirement plans. Of
course, that is possible, but most successful businesses eventually
offer a 401(k) plan because it can allow for employer matching funds and
has much higher contribution limits than I.R.A.s, both attractive
features for people in top management.

The congressmen also say that the states cannot be trusted to administer
the new programs. Certainly, many state-sponsored defined-benefit
pension plans have been inadequately funded. However, it is misleading
to conflate underfunded defined-benefit plans with I.R.A.-based plans
that would be fully funded by employee contributions held by
private-sector custodians. States have long administered college saving
529 plans, which have some similar features, without any crises of which
I am aware.

One could plausibly argue that these savings plans should be national in
scope, and not relegated to the states. If that is the reasoning, the
Senate would be better advised to enact legislation enabling a national
automatic I.R.A. program. Furthermore, if states are not capable of
administering a plan as simple as this, what can be the wisdom of
handing off to them a much more complicated problem such as health care,
as many Republicans in Congress have proposed?

The savings plan issue will soon come before the Senate. If it votes to
override the rule using the strategy made possible by the Congressional
Review Act, the state plans will be blocked and the Labor Department
will be forbidden from considering new versions. Legislators can help
Americans save for retirement by simply doing nothing.

Advertisement

\protect\hyperlink{after-bottom}{Continue reading the main story}

\hypertarget{site-index}{%
\subsection{Site Index}\label{site-index}}

\hypertarget{site-information-navigation}{%
\subsection{Site Information
Navigation}\label{site-information-navigation}}

\begin{itemize}
\tightlist
\item
  \href{https://help.nytimes3xbfgragh.onion/hc/en-us/articles/115014792127-Copyright-notice}{©~2020~The
  New York Times Company}
\end{itemize}

\begin{itemize}
\tightlist
\item
  \href{https://www.nytco.com/}{NYTCo}
\item
  \href{https://help.nytimes3xbfgragh.onion/hc/en-us/articles/115015385887-Contact-Us}{Contact
  Us}
\item
  \href{https://www.nytco.com/careers/}{Work with us}
\item
  \href{https://nytmediakit.com/}{Advertise}
\item
  \href{http://www.tbrandstudio.com/}{T Brand Studio}
\item
  \href{https://www.nytimes3xbfgragh.onion/privacy/cookie-policy\#how-do-i-manage-trackers}{Your
  Ad Choices}
\item
  \href{https://www.nytimes3xbfgragh.onion/privacy}{Privacy}
\item
  \href{https://help.nytimes3xbfgragh.onion/hc/en-us/articles/115014893428-Terms-of-service}{Terms
  of Service}
\item
  \href{https://help.nytimes3xbfgragh.onion/hc/en-us/articles/115014893968-Terms-of-sale}{Terms
  of Sale}
\item
  \href{https://spiderbites.nytimes3xbfgragh.onion}{Site Map}
\item
  \href{https://help.nytimes3xbfgragh.onion/hc/en-us}{Help}
\item
  \href{https://www.nytimes3xbfgragh.onion/subscription?campaignId=37WXW}{Subscriptions}
\end{itemize}
