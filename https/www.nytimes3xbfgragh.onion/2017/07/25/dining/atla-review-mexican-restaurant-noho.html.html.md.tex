Sections

SEARCH

\protect\hyperlink{site-content}{Skip to
content}\protect\hyperlink{site-index}{Skip to site index}

\href{https://www.nytimes3xbfgragh.onion/section/food}{Food}

\href{https://myaccount.nytimes3xbfgragh.onion/auth/login?response_type=cookie\&client_id=vi}{}

\href{https://www.nytimes3xbfgragh.onion/section/todayspaper}{Today's
Paper}

\href{/section/food}{Food}\textbar{}At Atla, Mexican for Every Moment of
the Day

\url{https://nyti.ms/2tWWFVn}

\begin{itemize}
\item
\item
\item
\item
\item
\item
\end{itemize}

Advertisement

\protect\hyperlink{after-top}{Continue reading the main story}

Supported by

\protect\hyperlink{after-sponsor}{Continue reading the main story}

\href{/column/restaurant-review}{Restaurant Review}

\hypertarget{at-atla-mexican-for-every-moment-of-the-day}{%
\section{At Atla, Mexican for Every Moment of the
Day}\label{at-atla-mexican-for-every-moment-of-the-day}}

\href{https://www.nytimes3xbfgragh.onion/slideshow/2017/07/25/dining/atla-nyc-review.html}{}

\hypertarget{atla}{%
\subsection{Atla}\label{atla}}

12 Photos

View Slide Show ›

\includegraphics{https://static01.graylady3jvrrxbe.onion/images/2017/07/26/dining/26REST-ATLA-slide-B35O/26REST-ATLA-slide-B35O-articleLarge.jpg?quality=75\&auto=webp\&disable=upscale}

Daniel Krieger for The New York Times

\begin{itemize}
\tightlist
\item
  Atla\\
  ★★ Mexican \$\$ 372 Lafayette Street
\end{itemize}

\href{https://resy.com/cities/ny/atla?utm_source=nyt\&utm_medium=restoprofile\&utm_campaign=affiliates\&aff_id=c1fe784}{Reserve
a Table}

When you make a reservation at an independently reviewed restaurant
through our site, we earn an affiliate commission.

By \href{http://www.nytimes3xbfgragh.onion/by/pete-wells}{Pete Wells}

\begin{itemize}
\item
  July 25, 2017
\item
  \begin{itemize}
  \item
  \item
  \item
  \item
  \item
  \item
  \end{itemize}
\end{itemize}

Since Atla ambled up to the corner of Lafayette and Great Jones Streets
this spring, all the people who run to every new opening in town have
been telling me they want to eat chilaquiles for breakfast there; they
want the chicken enchiladas for lunch; they want to hide from the
afternoon sun with a tall glass of iced tepache, a tingling, off-dry
agua fresca Atla makes by letting sugar ferment with pineapple rinds.

Sometimes, under their breath, they query a few of the prices at this
casual Mexican cafe (``Fourteen dollars for radishes?''), but I haven't
talked to anyone who doesn't think Atla is an extremely likable place.
Despite its fairly modest goals, or maybe because of them, Atla is one
of the least divisive restaurants Manhattan has seen in some time. It's
as if people, worn out by the sheer effort of being scandalized by the
news, had decided that 2017 would be more bearable if we could all just
find one thing to agree about. Atla is that thing.

Opened by Enrique Olvera, the chef best known for Pujol in Mexico City,
together with his proxy in New York, Daniela Soto-Innes, and their chef
de cuisine, Hugo Vera, Atla knows just what it's doing without seeming
to try. This combination is as attractive in a restaurant as it is in a
person.

One thing it is not trying to do is make a big statement. Mr. Olvera and
Ms. Soto-Innes made their statement already, at Cosme, their more
elaborate restaurant on East 21st Street, which is now nearly as well
known as Pujol. Having proved at Cosme that they have original things to
say about Mexican cuisine, they are set free at Atla to serve Mexican
food that New Yorkers may want to eat every day. Which is not to say
that this is the everyday cooking of Mexico.

While Cosme specializes in the kind of layered, multicomponent dishes
with which chefs make their mark, Atla's menu is simpler. The
chilaquiles are just chilaquiles, tortilla chips in red or green salsa
under white onions and a tangy crema. The chips are thick enough to keep
their crunch, and they get more of it from a sprinkling of toasted
flaxseeds.

\includegraphics{https://static01.graylady3jvrrxbe.onion/images/2017/07/26/dining/26REST1/26REST1-articleInline.jpg?quality=75\&auto=webp\&disable=upscale}

Flaxseeds are not a standard garnish for chilaquiles, but they are
symptomatic of the contemporary strain of nutrient-conscious eating Atla
displays. It has less in common with the sturdy cooking of traditional
Mexican restaurants than with the modified health-food aesthetic of
places like Dimes that cater to the yoga-mat crowd.

Besides a regular cafe con leche Atla serves two versions with no dairy;
one is made with cashew milk (a little odd) and another with coconut
milk (wonderful). In any form, the cafe con leche is made with strong
espresso, sweet enough to get along without sugar.

Chia bowls usually bore me, despite their exemplary levels of omega-3
fatty acids, but the one at Atla held my attention. It's stirred with
sweet Mexican cinnamon, like a semisolid horchata, and topped with
candied ginger, pumpkin seeds and nuts.

A number of dishes are flat-out salads, like the fine if not
soul-stirring quinoa with cucumbers and pico de gallo. Others are
salad-adjacent. Those \$14 radishes, cool, crisp and skinny, surround a
smooth avocado dip. The chunky guacamole is pounded with nearly enough
tarragon, mint and basil to qualify as an herb salad. When it arrives it
is nearly invisible under a single chile-dusted corn chip the color of a
wet bluestone sidewalk, and the size of a flip-flop.

A smaller, rounder blue masa chip, spread with farmer cheese and capers,
is the foundation of the arctic char tostada. As an attempt to introduce
Russ \& Daughters to Mexico it is all right, but it doesn't go beyond
that. The ceviche verde, on the other hand, is more dynamic than you
expect, thanks to the fresh ginger in its tart green sauce, among other
tricks.

Depending on your metabolism, all this will leave you feeling either
refreshed and ready to meet the day or slightly hungry. To make a lunch
of the chilaquiles I needed to fortify the plate with poached chicken,
bringing a \$14 dish to \$19.

There are more substantial things to eat. Gray sole fried in a crunchy
panko crust that tastes of garlic and butter is a complete meal. It
comes with a small cucumber salad, like a Viennese schnitzel, although
in Vienna they probably wouldn't bathe the cucumbers in herb juice and
add green chiles and cilantro. Nor would they sell you a \$3 plate of
warm tortillas so you can make your own tacos. Maybe they should,
though.

Image

The pambazo is the traditional sandwich of chorizo and soft potatoes on
a soft roll dipped in guajillo salsa.Credit...Daniel Krieger for The New
York Times

The pambazo, meanwhile, is the traditional sandwich of chorizo and
potatoes on a soft roll dipped in guajillo salsa. It's more or less the
pambazo you would get out of a Oaxacan lunch truck if you were lucky
enough to find one parked on Lafayette Street.

Still, the overall lightness of the food has some bearing on the one
real debate you could have about Atla. The question is not whether to
go, but when. (People plan their meals at Cosme a week ahead or more,
and while Atla takes reservations, going there tends to be a
spur-of-the-moment decision.)

The dining room's corner space is wrapped in plate glass. By day this
invites the sunshine in and provides a wide-screen view of the endless
fashion parade of NoHo. By night, when the eggs and other breakfast
dishes roll off the menu and the room fills up with people more
interested in mezcal and the ``overproof margarita'' than coffee and
agua fresca, the windows become sounding boards.

At times like this you may notice all the other hard surfaces, like the
stone floor; you may wonder why only some of the seats have backs; you
may be frustrated in your effort to keep all your plates and glasses on
the small round table.

The very qualities that make Atla ideal for a relaxed breakfast or
lunch, in other words, make it a dinner destination that is best
approached with a moderate appetite, a resilient lower back and a
suspicion that whatever your friends are saying probably isn't important
anyway. Better still, treat it as a drinking spot where food is an added
attraction.

For dessert there is a curious tamal topped with queso fresco. It's more
salty than sweet, a minor-key reprise of the great corn-husk meringue at
Cosme. The most appealing dessert I had was a roasted sweet potato under
a caramelized pool of sweetened condensed milk. It is off the menu now,
but I am telling you about it anyway because I suspect it will be back.
I know I will.

\href{https://www.facebookcorewwwi.onion/nytfood/}{\emph{Follow NYT Food
on Facebook}}\emph{,}
\href{https://instagram.com/nytfood}{\emph{Instagram}}\emph{,}
\href{https://twitter.com/nytfood}{\emph{Twitter}} \emph{and}
\href{https://www.pinterest.com/nytfood/}{\emph{Pinterest}}\emph{.}
\href{https://www.nytimes3xbfgragh.onion/newsletters/cooking}{\emph{Get
regular updates from NYT Cooking, with recipe suggestions, cooking tips
and shopping advice}}\emph{.}

Advertisement

\protect\hyperlink{after-bottom}{Continue reading the main story}

\hypertarget{site-index}{%
\subsection{Site Index}\label{site-index}}

\hypertarget{site-information-navigation}{%
\subsection{Site Information
Navigation}\label{site-information-navigation}}

\begin{itemize}
\tightlist
\item
  \href{https://help.nytimes3xbfgragh.onion/hc/en-us/articles/115014792127-Copyright-notice}{©~2020~The
  New York Times Company}
\end{itemize}

\begin{itemize}
\tightlist
\item
  \href{https://www.nytco.com/}{NYTCo}
\item
  \href{https://help.nytimes3xbfgragh.onion/hc/en-us/articles/115015385887-Contact-Us}{Contact
  Us}
\item
  \href{https://www.nytco.com/careers/}{Work with us}
\item
  \href{https://nytmediakit.com/}{Advertise}
\item
  \href{http://www.tbrandstudio.com/}{T Brand Studio}
\item
  \href{https://www.nytimes3xbfgragh.onion/privacy/cookie-policy\#how-do-i-manage-trackers}{Your
  Ad Choices}
\item
  \href{https://www.nytimes3xbfgragh.onion/privacy}{Privacy}
\item
  \href{https://help.nytimes3xbfgragh.onion/hc/en-us/articles/115014893428-Terms-of-service}{Terms
  of Service}
\item
  \href{https://help.nytimes3xbfgragh.onion/hc/en-us/articles/115014893968-Terms-of-sale}{Terms
  of Sale}
\item
  \href{https://spiderbites.nytimes3xbfgragh.onion}{Site Map}
\item
  \href{https://help.nytimes3xbfgragh.onion/hc/en-us}{Help}
\item
  \href{https://www.nytimes3xbfgragh.onion/subscription?campaignId=37WXW}{Subscriptions}
\end{itemize}
