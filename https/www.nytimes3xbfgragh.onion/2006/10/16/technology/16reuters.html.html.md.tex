Sections

SEARCH

\protect\hyperlink{site-content}{Skip to
content}\protect\hyperlink{site-index}{Skip to site index}

\href{https://www.nytimes3xbfgragh.onion/section/technology}{Technology}

\href{https://myaccount.nytimes3xbfgragh.onion/auth/login?response_type=cookie\&client_id=vi}{}

\href{https://www.nytimes3xbfgragh.onion/section/todayspaper}{Today's
Paper}

\href{/section/technology}{Technology}\textbar{}The Reporter Is Real,
but the World He Covers Isn't

\begin{itemize}
\item
\item
\item
\item
\item
\end{itemize}

Advertisement

\protect\hyperlink{after-top}{Continue reading the main story}

Supported by

\protect\hyperlink{after-sponsor}{Continue reading the main story}

\hypertarget{the-reporter-is-real-but-the-world-he-covers-isnt}{%
\section{The Reporter Is Real, but the World He Covers
Isn't}\label{the-reporter-is-real-but-the-world-he-covers-isnt}}

\includegraphics{https://static01.graylady3jvrrxbe.onion/images/2006/10/16/business/16reuters.1_600x374.jpg?quality=75\&auto=webp\&disable=upscale}

By Andrew Adam Newman

\begin{itemize}
\item
  Oct. 16, 2006
\item
  \begin{itemize}
  \item
  \item
  \item
  \item
  \item
  \end{itemize}
\end{itemize}

In preparing to open a Reuters bureau on a bustling island, Adam Pasick
has been introducing himself to residents and interviewing
entrepreneurs. After finishing such interviews, Mr. Pasick often
levitates for a moment, then flies over buildings.

Mr. Pasick, a Reuters technology reporter who was formerly earthbound
with the news agency, is heading up Reuters' first virtual news bureau
inside the online role-playing game Second Life. While many independent
journalists and bloggers have published inside such virtual worlds,
Reuters is the first established news agency to dispatch a full-time
reporter to do so.

Created by Linden Labs of San Francisco, Second Life is a realistic
world where avatars --- animated representations --- for its more than
850,000 players interact. Avatars buy islands on which to build homes or
businesses, and sell one another everything from homes designed by Adobe
Photoshop to virtual sneakers. The currency is Linden dollars; \$1 buys
280 of them. In September, user-to-user transactions totaled \$7.1
million.

``The fact that it's in a virtual world doesn't change things as much as
you'd think,'' said Mr. Pasick, 30, a Michigan native based in London.
``It's not any different than when Reuters opens up a bureau in a part
of the world that has a fast-growing economy that we weren't in before.
The laws of supply and demand hold true, it has a currency exchange,
people open businesses and get paid for goods and services.''

Mr. Pasick's avatar, Adam Reuters, was modeled after the reporter, and
sports a press pass so others know he buys his pixels by the barrel. He
will set up shop in a virtual building made to look like a hybrid of
Reuters' London and Times Square buildings.

While players who drop in (flying is one of only a few superhuman
aspects of Second Life) can access Reuters news from the real world, the
articles Mr. Pasick files will be strictly about --- and addressed to
--- Second World players. One of his first examines Second Life's
biggest lender, who charges 40 percent annual interest. His dispatches
will be posted at secondlife.reuters.com.

``I've been playing in Second Life since it was a relatively small
community,'' said Thomas H. Glocer, Reuters' chief executive. Mr. Glocer
allowed that some might question the wisdom of parachuting the legendary
155-year-old news agency into such a geekfest.

``This is a very serious, old brand that stands for things and has
principles, but that doesn't take itself so seriously that it wouldn't
play in a gaming space,'' Mr. Glocer said. ``This appeals to a younger
demographic. Even for people who don't go in and play in Second Life, it
shows Reuters has a certain with-it-ness.''

Edward Castronova, author of ``Synthetic Worlds: The Business and
Culture of Online Games,'' said it made sense for Reuters to hang a
shingle in Second Life. ``It's an easy way for a blue-chip, traditional
organization to get virtual world credibility,'' he said.

Reuters would get more exposure in the most popular online fantasy game,
World of Warcraft, which has more than seven million subscribers, but
that game's players are decidedly less civilized. ``It would be more
fun, but Reuters would be more likely to end up in a dragon's belly,''
Mr. Castronova said. ANDREW ADAM NEWMAN

Advertisement

\protect\hyperlink{after-bottom}{Continue reading the main story}

\hypertarget{site-index}{%
\subsection{Site Index}\label{site-index}}

\hypertarget{site-information-navigation}{%
\subsection{Site Information
Navigation}\label{site-information-navigation}}

\begin{itemize}
\tightlist
\item
  \href{https://help.nytimes3xbfgragh.onion/hc/en-us/articles/115014792127-Copyright-notice}{©~2020~The
  New York Times Company}
\end{itemize}

\begin{itemize}
\tightlist
\item
  \href{https://www.nytco.com/}{NYTCo}
\item
  \href{https://help.nytimes3xbfgragh.onion/hc/en-us/articles/115015385887-Contact-Us}{Contact
  Us}
\item
  \href{https://www.nytco.com/careers/}{Work with us}
\item
  \href{https://nytmediakit.com/}{Advertise}
\item
  \href{http://www.tbrandstudio.com/}{T Brand Studio}
\item
  \href{https://www.nytimes3xbfgragh.onion/privacy/cookie-policy\#how-do-i-manage-trackers}{Your
  Ad Choices}
\item
  \href{https://www.nytimes3xbfgragh.onion/privacy}{Privacy}
\item
  \href{https://help.nytimes3xbfgragh.onion/hc/en-us/articles/115014893428-Terms-of-service}{Terms
  of Service}
\item
  \href{https://help.nytimes3xbfgragh.onion/hc/en-us/articles/115014893968-Terms-of-sale}{Terms
  of Sale}
\item
  \href{https://spiderbites.nytimes3xbfgragh.onion}{Site Map}
\item
  \href{https://help.nytimes3xbfgragh.onion/hc/en-us}{Help}
\item
  \href{https://www.nytimes3xbfgragh.onion/subscription?campaignId=37WXW}{Subscriptions}
\end{itemize}
