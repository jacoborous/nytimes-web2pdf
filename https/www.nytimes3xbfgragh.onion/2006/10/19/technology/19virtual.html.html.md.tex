Sections

SEARCH

\protect\hyperlink{site-content}{Skip to
content}\protect\hyperlink{site-index}{Skip to site index}

\href{https://www.nytimes3xbfgragh.onion/section/technology}{Technology}

\href{https://myaccount.nytimes3xbfgragh.onion/auth/login?response_type=cookie\&client_id=vi}{}

\href{https://www.nytimes3xbfgragh.onion/section/todayspaper}{Today's
Paper}

\href{/section/technology}{Technology}\textbar{}A Virtual World but Real
Money

\begin{itemize}
\item
\item
\item
\item
\item
\end{itemize}

Advertisement

\protect\hyperlink{after-top}{Continue reading the main story}

Supported by

\protect\hyperlink{after-sponsor}{Continue reading the main story}

\hypertarget{a-virtual-world-but-real-money}{%
\section{A Virtual World but Real
Money}\label{a-virtual-world-but-real-money}}

\includegraphics{https://static01.graylady3jvrrxbe.onion/images/2006/10/18/business/19virtual.600.jpg?quality=75\&auto=webp\&disable=upscale}

By \href{https://www.nytimes3xbfgragh.onion/by/richard-siklos}{Richard
Siklos}

\begin{itemize}
\item
  Oct. 19, 2006
\item
  \begin{itemize}
  \item
  \item
  \item
  \item
  \item
  \end{itemize}
\end{itemize}

\textbf{Correction Appended}

It has a population of a million. The ``people'' there make friends,
build homes and run businesses. They also play sports, watch movies and
do a lot of other familiar things. They even have their own currency,
convertible into American dollars.

But residents also fly around, walk underwater and make themselves look
beautiful, or like furry animals, dragons, or practically anything ---
or anyone --- they wish.

This parallel universe, an online service called Second Life that allows
computer users to create a new and improved digital version of
themselves, began in 1999 as a kind of online video game.

But now, the budding fake world is not only attracting a lot more
people, it is taking on a real world twist: big business interests are
intruding on digital utopia. The Second Life online service is fast
becoming a three-dimensional test bed for corporate marketers, including
Sony BMG Music Entertainment, Sun Microsystems, Nissan, Adidas/Reebok,
Toyota and Starwood Hotels.

The sudden rush of real companies into so-called virtual worlds mirrors
the evolution of the Internet itself, which moved beyond an educational
and research network in the 1990's to become a commercial proposition
--- but not without complaints from some quarters that the medium's
purity would be lost.

Already, the Internet is the fastest-growing advertising medium, as
traditional forms of marketing like television commercials and print
advertising slow. For businesses, these early forays into virtual worlds
could be the next frontier in the blurring of advertising and
entertainment.

Unlike other popular online video games like World of Warcraft that are
competitive fantasy games, these sites meld elements of the most popular
forms of new media: chat rooms, video games, online stores,
user-generated content sites like YouTube.com and social networking
sites like MySpace.com.

Philip Rosedale, the chief executive of Linden Labs, the San Francisco
company that operates Second Life, said that until a few months ago only
one or two real world companies had dipped their toes in the synthetic
water. Now, more than 30 companies are working on projects there, and
dozens more are considering them. ``It's taken off in a way that is kind
of surreal,'' Mr. Rosedale said, with no trace of irony.

Beginning a promotional venture in a virtual world is still a relatively
inexpensive proposition compared with the millions spent on other media.
In Second Life, a company like Nissan or its advertising agency could
buy an ``island'' for a one-time fee of \$1,250 and a monthly rate of
\$195 a month. For its new campaign built around its Sentra car, the
company then needed to hire some computer programmers to create a
gigantic driving course and design digital cars that people ``in world''
could actually drive, as well as some billboards and other promotional
spots throughout the virtual world that would encourage people to visit
Nissan Island.

Image

Screen grabs from the Second Life universe, which is fast becoming a
three-dimensional test bed for corporate marketers, including Sony BMG
Music Entertainment, Sun Microsystems, Nissan, Adidas/Reebok, Toyota and
Starwood Hotels.

Virtual world proponents --- including a roster of Linden Labs investors
that includes Jeffrey P. Bezos, the founder of Amazon.com; Mitchell D.
Kapor, the software pioneer; and Pierre Omidyar, the eBay founder ---
say that the entire Internet is moving toward being a three-dimensional
experience that will become more realistic as computing technology
advances.

Entering Second Life, people's digital alter-egos --- known as avatars
--- are able to move around and do everything they do in the physical
world, but without such bothers as the laws of physics. ``When you are
at Amazon.com you are actually there with 10,000 concurrent other
people, but you cannot see them or talk to them,'' Mr. Rosedale said.
``At Second Life, everything you experience is inherently experienced
with others.''

Second Life is the largest and best known of several virtual worlds
created to attract a crowd. The cable TV network MTV, for example, just
began Virtual Laguna Beach, where fans of its show, ``Laguna Beach: The
Real O.C.,'' can fashion themselves after the show's characters and hang
out in their faux settings.

Unlike Second Life, which emphasizes a hands-off approach and has little
say over who sets up shop inside its simulated world, MTV's approach is
to bring in advertisers as partners.

In Second Life, retailers like Reebok, Nike, Amazon and American Apparel
have all set up shops to sell digital as well as real world versions of
their products. Last week, Sun Microsystems unveiled a new pavilion
promoting its products, and I.B.M. alumni held a virtual world reunion.

This week, the performer Ben Folds is to promote a new album with two
virtual appearances. At one, he will play the opening party for Aloft,
an elaborate digital prototype for a new chain of hotels planned by
Starwood Hotels and Resorts. The same day, Mr. Folds will also
``appear'' at a new facility his music label's parent company, Sony BMG,
is opening at a complex called Media Island.

Meanwhile, Nissan is introducing its Nissan promotion, featuring a
gigantic vending machine dispensing cars people can ``drive'' around.

And some of this is likely to be covered for the outside world by such
business news outlets as CNet and Reuters, which now have reporters
embedded full-time in the virtual realm.

All this attention has some Second Lifers concerned that their digital
paradise will never be the same, like a Wal-Mart coming to town or a
Starbucks opening in the neighborhood. ``The phase it is in now is just
using it as a hype and marketing thing,'' said Catherine A. Fitzpatrick,
50, a member of Second Life who in the real world is a Russian
translator in Manhattan.

In her second life, Ms. Fitzpatrick's digital alter-ego is a figure
well-known to other participants called Prokofy Neva, who runs a
business renting ``real estate'' to other players. ``The next phase,''
she said, ``will be they try to compete with other domestic products ---
the people who made sneakers in the world are now in danger of being
crushed by Adidas.''

Image

Philip Rosedale, the chief executive of Linden Labs, the San Francisco
company that operates Second Life. More than 30 companies are working on
projects on Second Life, and dozens more are considering
them.Credit...Heidi Schumann for The New York Times

Mr. Rosedale says such concerns are overstated, because there are no
advantages from economies of scale for big corporations in Second Life,
and people can avoid places like Nissan Island as easily as they can
avoid going to Nissan's Web site. There is no limit to what can be built
in Second Life, just as there is no limit to how many Web sites populate
the Internet.

Linden Labs makes most of its money leasing ``land'' to tenants, Mr.
Rosedale said, at an average of roughly \$20 per month per ``acre'' or
\$195 a month for a private ``island.'' The land mass of Second Life is
growing about 8 percent a month, a spokeswoman said, and now totals
``60,000 acres,'' the equivalent of about 95 square miles in the
physical world. Linden Labs, a private company, does not disclose its
revenue.

Despite the surge of outside business activity in Second Life, Linden
Labs said corporate interests still owned less than 5 percent of the
virtual world's real estate.

As many as 10,000 people are in the virtual world at a time, and they
are engaged in a gamut of ventures: everything from holding charity
fund-raisers to selling virtual helicopters to operating sex clubs.
Linden also makes money on exchanging United States dollars for what it
calls Linden dollars for around 400 Linden dollars for \$1 (people can
load up on them with a credit card). A typical article of clothing ---
say a shirt --- would cost around 200 Linden dollars, or 50 cents. As
evidence of the growth of its ``economy,'' Second Life's Web site tracks
how much money changes hands each day. It recently reached as much as
\$500,000 a day and is growing as much as 15 percent a month.

On Tuesday, a Congressional committee said it was investigating whether
virtual assets and incomes should be taxed.

But many inhabitants simply hang out for free. For advertisers worried
about the effectiveness of the 30-second TV spot and the clutter of real
world billboards and Internet pop-up ads, Second Life is appealing
because it is a place where people literally immerse themselves in their
products.

Steve F. Kerho, director of interactive marketing and media for Nissan
USA, said the Second Life campaign was part of a growing interest in
online video games. ``We're just trying to follow our consumer, that's
where they're spending their time,'' Mr. Kerho said. ``But there has to
be something in it for them --- it's got to be fun; it's got to be
playful.''

Projects like the Aloft hotel, an offshoot of Starwood's W Hotels brand,
are designed to promote the venture but also to give its designers
feedback from prospective guests before the first real hotel opens in
2008.

The new Sony BMG building has rooms devoted to popular musicians like
Justin Timberlake and DMX, allowing fans to mingle, listen to tunes or
watch videos. Sony BMG is also toying with renting residences in the
complex, as well as selling music downloads that people can listen to
throughout the simulated world.

Sibley Verbeck, chief executive of the Electric Sheep Company, a
consultancy that designed the Aloft and Sony BMG projects, said the
flurry of corporate interest stemmed from the 10 to 20 percent growth in
the number of people who had gone into virtual worlds each month for the
last three years. Though exact numbers are difficult to come by, the
figure should top a few million by next year, he said.

Image

Catherine A. Fitzpatrick, 50, a member of Second Life who in the real
world is a Russian translator in Manhattan. Her digital alter-ego is
Prokofy Neva, who runs a business renting ``real estate'' to other
players.Credit...Ozier Muhammad/The New York Times

The spread of these worlds, however, is limited by access to high-speed
Internet connections and, in Second Life's case, software that is
challenging to master and only runs on certain models of computers.

``If it doesn't crash and burn then it will become real,'' he said. ``So
now's the time to start experimenting and learning ahead of your
competition.''

As part of that process, businesses are learning that different rules
apply when they venture into an arena where audiences are in control.
``Users are the content --- that's the thing that everybody has a hard
time getting over,'' said Michael Wilson, the chief executive of Makena
Technologies, which operates the virtual world There.com and helped
build Virtual Laguna Beach.

For example, Sun Microsystems kicked off the opening of its Second Life
venue with a press conference online hosted by executives and Mr.
Rosedale of Linden Labs. But by the time the event was in full swing,
several members of the audience had either walked or flown onto the
stage, where they were running roughshod over the proceedings.

Even Mr. Rosedale got in on the act: he conjured a pair of sunglasses
that he superimposed on a video image of a Sun representative talking on
a screen behind the stage. (In virtual world lingo, such high jinks are
known as ``griefing.'')

Some corporate events have been met with protests by placard-waving
avatars. And there is even a group called the Second Life Liberation
Army that has staged faux ``attacks'' on Reebok and American Apparel
stores. (The S.L.L.A. says it is fighting for voting rights for avatars
--- as well as stock in Linden Labs.)

Companies in this new environment have to get used to the idea that they
may never know exactly who they are dealing with. Most of those in
Second Life have chosen their names from a whimsical menu of supplied
surnames, resulting in monikers like Snoopybrown Zamboni and Bitmason
Pimpernel; males posing as female avatars and vice versa are not
uncommon.

Another issue companies have to contend with is that their brands may
already be in these virtual worlds, but illegally. Henry Jenkins, a
professor at the Massachusetts Institute of Technology Media Lab, said
one Second Life habitué created a virtual reproduction of the Ikea
catalog to help people decorate their digital pads.

Mr. Verbeck of Electric Sheep said copyright infringement was rampant.
His company runs an online boutique where Second Life residents sell
each other pixelized creations of everything from body parts to home
furnishings to roller skates --- many of them unauthorized knockoffs.

So far, the boutique has not had many requests to stop selling fake
products. But ``we did have a request from the Salvador Dali Museum ---
which was great,'' Mr. Verbeck said. ``Second Life is so surreal that it
was perfect.''

Advertisement

\protect\hyperlink{after-bottom}{Continue reading the main story}

\hypertarget{site-index}{%
\subsection{Site Index}\label{site-index}}

\hypertarget{site-information-navigation}{%
\subsection{Site Information
Navigation}\label{site-information-navigation}}

\begin{itemize}
\tightlist
\item
  \href{https://help.nytimes3xbfgragh.onion/hc/en-us/articles/115014792127-Copyright-notice}{©~2020~The
  New York Times Company}
\end{itemize}

\begin{itemize}
\tightlist
\item
  \href{https://www.nytco.com/}{NYTCo}
\item
  \href{https://help.nytimes3xbfgragh.onion/hc/en-us/articles/115015385887-Contact-Us}{Contact
  Us}
\item
  \href{https://www.nytco.com/careers/}{Work with us}
\item
  \href{https://nytmediakit.com/}{Advertise}
\item
  \href{http://www.tbrandstudio.com/}{T Brand Studio}
\item
  \href{https://www.nytimes3xbfgragh.onion/privacy/cookie-policy\#how-do-i-manage-trackers}{Your
  Ad Choices}
\item
  \href{https://www.nytimes3xbfgragh.onion/privacy}{Privacy}
\item
  \href{https://help.nytimes3xbfgragh.onion/hc/en-us/articles/115014893428-Terms-of-service}{Terms
  of Service}
\item
  \href{https://help.nytimes3xbfgragh.onion/hc/en-us/articles/115014893968-Terms-of-sale}{Terms
  of Sale}
\item
  \href{https://spiderbites.nytimes3xbfgragh.onion}{Site Map}
\item
  \href{https://help.nytimes3xbfgragh.onion/hc/en-us}{Help}
\item
  \href{https://www.nytimes3xbfgragh.onion/subscription?campaignId=37WXW}{Subscriptions}
\end{itemize}
