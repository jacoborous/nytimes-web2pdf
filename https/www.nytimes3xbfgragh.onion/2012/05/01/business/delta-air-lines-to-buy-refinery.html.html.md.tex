Sections

SEARCH

\protect\hyperlink{site-content}{Skip to
content}\protect\hyperlink{site-index}{Skip to site index}

\href{https://www.nytimes3xbfgragh.onion/section/business}{Business}

\href{https://myaccount.nytimes3xbfgragh.onion/auth/login?response_type=cookie\&client_id=vi}{}

\href{https://www.nytimes3xbfgragh.onion/section/todayspaper}{Today's
Paper}

\href{/section/business}{Business}\textbar{}Delta Buys Refinery to Get
Control of Fuel Costs

\begin{itemize}
\item
\item
\item
\item
\item
\end{itemize}

Advertisement

\protect\hyperlink{after-top}{Continue reading the main story}

Supported by

\protect\hyperlink{after-sponsor}{Continue reading the main story}

\hypertarget{delta-buys-refinery-to-get-control-of-fuel-costs}{%
\section{Delta Buys Refinery to Get Control of Fuel
Costs}\label{delta-buys-refinery-to-get-control-of-fuel-costs}}

By \href{https://www.nytimes3xbfgragh.onion/by/jad-mouawad}{Jad Mouawad}

\begin{itemize}
\item
  April 30, 2012
\item
  \begin{itemize}
  \item
  \item
  \item
  \item
  \item
  \end{itemize}
\end{itemize}

Delta Air Lines said on Monday that it had
\href{http://news.delta.com/index.php?s=43\&item=1601}{agreed to buy a
refinery} near Philadelphia from ConocoPhillips to offset the risk of
higher jet fuel prices.

Delta said that it would spend \$150 million to acquire the Trainer
refinery, which has been shuttered for six months, after receiving \$30
million from the state of Pennsylvania as part of a deal to support job
creation.

The airline said it would spend \$100 million more to refurbish the
plant to increase its output of jet fuel.

Richard H. Anderson, Delta's chief executive, said the investment was a
modest one, equivalent to the list price of a new wide-body plane like a
Boeing 777. The company estimated that it would reduce its annual fuel
expense by \$300 million, once the refinery was refurbished and
operating again.

To achieve similar fuel savings, Delta would have to buy 60
new-generation narrow-body planes like the Boeing 737, a capital
investment that would total \$2.5 billion, according to a regulatory
filing.

\includegraphics{https://static01.graylady3jvrrxbe.onion/images/2012/05/01/business/DELTA/DELTA-jumbo.jpg?quality=75\&auto=webp\&disable=upscale}

Delta said it had also struck a three-year agreement with BP to supply
crude oil to the refinery.

As part of the deal, the details of which were not released, Delta said
it would exchange gasoline, diesel and other petroleum products produced
at the refinery for jet fuel from other sources like BP and Phillips 66.

Combined with the jet fuel produced at Trainer, Delta said these deals
would provide 80 percent of its fuel needs in the United States. The
purchase ``is an innovative approach to managing our largest expense,''
Mr. Anderson said in a statement.

The jet fuel bill at Delta, as at other airlines, soared in recent years
as prices for crude oil increased. On average, fuel accounts for about a
third of an airline's operating costs, a share that has been rising for
much of the last decade.

The International Air Transport Association estimates that the global
airline industry's fuel bill will grow by \$40 billion this year, from
about \$177 billion in 2011.

These rising fuel costs have forced painful restructurings for airlines
in recent years, helping to push many of them into bankruptcy and
spurring consolidations across the industry. The airlines have set up
elaborate hedging strategies to try to counter the rising fuel costs.
But the hedges backfired when crude oil prices rose or fell in
unexpected ways. Buying a refinery will not erase Delta's fuel bill. The
airline will still need to buy crude oil at world market prices. But in
justifying the purchase, Delta said the cost of manufacturing jet fuel
had risen more rapidly than crude oil costs. It also said that demand
for jet fuel and diesel, both by-products of the refining process, had
been rising steadily in recent years, while demand for gasoline was
falling, adding to the pressure on jet fuel prices.

Last year, Delta spent \$12 billion on fuel, about \$3 billion more than
the previous year, as oil prices rebounded from their postrecession
lows. Fuel make up 40 percent of Delta's costs.

Image

Richard Anderson, Delta's chief, called it a modest
investment.Credit...Andrew Harrer/Bloomberg News

The company said it expected the acquisition of the refinery to be
completed in the first half of 2012. The airline said changes to the
plant's infrastructure were to be completed by the end of the third
quarter, when jet fuel production would begin. Delta estimated that its
fuel savings this year would be more than \$100 million.

Analysts have been
\href{http://www.nytimes3xbfgragh.onion/2012/04/05/business/deltas-puzzling-interest-in-buying-an-oil-refinery.html}{puzzled
by Delta's interest} in the refinery, a notoriously difficult business
that is chronically unprofitable. Despite the skepticism from oil
specialists, investors seem to support Delta's unusual approach. The
company's shares have gained more than 10 percent since the beginning of
April as word of its interest in the refinery started to spread.

The Trainer refinery is on the Delaware River, midway between
Philadelphia and Wilmington, Del. It has a crude oil capacity of 185,000
barrels a day, and processes mainly light, low-sulfur crude oil.

Trainer has historically been geared to the gasoline market in the
Northeast. But as consumption dropped and light crude oil costs rose
faster than other types of crude oil, Conoco's plant has struggled.

Two refineries in the Philadelphia region --- the Trainer refinery and
Sunoco's Marcus Hook --- and one in the Virgin Islands, accounting for
half of the East Coast's refining capacity, have shut down since
September.

Delta said Trainer's jet fuel output would be 32 percent of the total
daily capacity, up from its current 14 percent. Its gasoline production
will drop to 43 percent, from 52 percent today, Delta said.

To assuage concerns that an airline has never owned or run a refinery,
Delta said the plant would be led by executives with refining
experience, including Jeffrey Warmann, who ran Murphy Oil's refinery in
Meraux, La.

Advertisement

\protect\hyperlink{after-bottom}{Continue reading the main story}

\hypertarget{site-index}{%
\subsection{Site Index}\label{site-index}}

\hypertarget{site-information-navigation}{%
\subsection{Site Information
Navigation}\label{site-information-navigation}}

\begin{itemize}
\tightlist
\item
  \href{https://help.nytimes3xbfgragh.onion/hc/en-us/articles/115014792127-Copyright-notice}{©~2020~The
  New York Times Company}
\end{itemize}

\begin{itemize}
\tightlist
\item
  \href{https://www.nytco.com/}{NYTCo}
\item
  \href{https://help.nytimes3xbfgragh.onion/hc/en-us/articles/115015385887-Contact-Us}{Contact
  Us}
\item
  \href{https://www.nytco.com/careers/}{Work with us}
\item
  \href{https://nytmediakit.com/}{Advertise}
\item
  \href{http://www.tbrandstudio.com/}{T Brand Studio}
\item
  \href{https://www.nytimes3xbfgragh.onion/privacy/cookie-policy\#how-do-i-manage-trackers}{Your
  Ad Choices}
\item
  \href{https://www.nytimes3xbfgragh.onion/privacy}{Privacy}
\item
  \href{https://help.nytimes3xbfgragh.onion/hc/en-us/articles/115014893428-Terms-of-service}{Terms
  of Service}
\item
  \href{https://help.nytimes3xbfgragh.onion/hc/en-us/articles/115014893968-Terms-of-sale}{Terms
  of Sale}
\item
  \href{https://spiderbites.nytimes3xbfgragh.onion}{Site Map}
\item
  \href{https://help.nytimes3xbfgragh.onion/hc/en-us}{Help}
\item
  \href{https://www.nytimes3xbfgragh.onion/subscription?campaignId=37WXW}{Subscriptions}
\end{itemize}
