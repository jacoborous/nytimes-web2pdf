Sections

SEARCH

\protect\hyperlink{site-content}{Skip to
content}\protect\hyperlink{site-index}{Skip to site index}

\href{https://www.nytimes3xbfgragh.onion/pages/dining/index.html}{Dining
\& Wine}

\href{https://myaccount.nytimes3xbfgragh.onion/auth/login?response_type=cookie\&client_id=vi}{}

\href{https://www.nytimes3xbfgragh.onion/section/todayspaper}{Today's
Paper}

\href{/pages/dining/index.html}{Dining \& Wine}\textbar{}A Map of Your
Taste Buds Shaped Like Italy

\begin{itemize}
\item
\item
\item
\item
\item
\end{itemize}

Advertisement

\protect\hyperlink{after-top}{Continue reading the main story}

Supported by

\protect\hyperlink{after-sponsor}{Continue reading the main story}

Restaurant Review

\hypertarget{a-map-of-your-taste-buds-shaped-like-italy}{%
\section{A Map of Your Taste Buds Shaped Like
Italy}\label{a-map-of-your-taste-buds-shaped-like-italy}}

\includegraphics{https://static01.graylady3jvrrxbe.onion/images/2012/02/15/dining/15REST_SPAN/15JPREST1-articleLarge.jpg?quality=75\&auto=webp\&disable=upscale}

By \href{https://www.nytimes3xbfgragh.onion/by/pete-wells}{Pete Wells}

\begin{itemize}
\item
  Feb. 14, 2012
\item
  \begin{itemize}
  \item
  \item
  \item
  \item
  \item
  \end{itemize}
\end{itemize}

IS it crazy to fall for a restaurant because of a handful of chickpeas?
Tumbling around with pickled currants under crunchy stalks of grilled
octopus leg, these chickpeas were smaller and sweeter than usual, less
starchy and grainy. Tasting one was like encountering a goldfinch if the
only birds you'd ever seen were pigeons.

Still, they were just chickpeas. Is it more logical to fall for a
restaurant because of sliced bread in a basket? It was remarkable stuff,
with the gradually unfolding nuances of taste that are achieved only
through a slow and patient fermentation of dough with wild yeast.

But other restaurants serve great bread. So let's blame it on the salumi
board, with satiny pink and white folds of lonza and capocollo and lardo
that melt on the tongue into a lasting impression of salt, pig fat and
time. The meats, cured and aged in the basement of Il Buco Alimentari e
Vineria, are among the finest salumi in the country.

In truth, there were a dozen little tastes that made me fall for this
restaurant, which opened on Great Jones Street four months ago and has
become New York's most complete realization so far of a powerful myth:
the simple and convivial spot that tastes just like Italy.

That myth is an old one. You go to a small town in Umbria and find your
way to some modest trattoria, or maybe a market with a few tables. Then
lunch arrives, and the top of your head comes off. Meanwhile everybody
around you has their elbows on the table and is acting like food this
good is no big thing. And you say, why not? Why couldn't we have a place
just like this back home?

The answer could fill a book, but the abridged version is that the
modest trattoria can't exist without the town. Fantastic elemental
cooking requires fantastic elements, and those have to come from the
baker, the grocer, the butcher and a dozen other local trades people and
merchants.

Still, the myth echoes in many places, from the pursed-lipped pieties of
Alice Waters's cathedral to the open-throated revels of Mario Batali's
pagan temple.

Very few believers have tried to live out the myth by recreating the
entire Italian village. But in a sense this is what Donna Lennard and
her partners have done at Il Buco Alimentari e Vineria, and that is what
makes it such an inviting place to have a pastry in the morning, a
sandwich and a bowl of soup at noon, and a small feast at night.

First, she built the grocery, the Alimentari part of the name, which
carries those earthy chickpeas, along with salt crystals and olive oil
with a pepper rasp that catches in the back of your throat. The shop
seems at first to be the whole show.

Keep walking. Down a few steps and past a colossal sculptural chandelier
that would not look lost hanging in the Guggenheim rotunda is a bright,
busy room where people are tearing at pieces of bread, eating off one
another's plates and talking with their elbows on the tables.

Every cliché of the rustic Italian restaurant is on display here: the
terra-cotta floor tiles, the handblown wine bottles, the rickety chairs,
the overworked servers who are always rushing to somebody else's table.
Get their attention, though, and they will reliably guide you toward the
good stuff on the menu and on the wine list.

\href{https://www.nytimes3xbfgragh.onion/slideshow/2012/02/15/dining/20120215_REST.html}{}

\hypertarget{il-buco-alimentari-e-vineria}{%
\subsection{Il Buco Alimentari e
Vineria}\label{il-buco-alimentari-e-vineria}}

12 Photos

View Slide Show ›

\includegraphics{https://static01.graylady3jvrrxbe.onion/images/2012/02/15/dining/20120215_REST-slide-NIFS/20120215_REST-slide-NIFS-jumbo.jpg?quality=75\&auto=webp\&disable=upscale}

Dave Sanders for The New York Times

With six to eight selections from a number of producers, the wine list
lets you drink your way through some bottles you might overlook on a
more comprehensive list, like the spry Tyrolean whites of Garlider, or
the spicy Calabrian reds of Odoardi.

The sooner you order, the faster you'll get a basket of Kamel Saci's
breads. He makes a dozen kinds, letting the dough gain flavor and
structure during a rise that lasts at least 18 hours, sometimes twice
that. His bread is not something you munch absent-mindedly while waiting
for the first course. It is the first course.

The Alimentari, with its \$9 chocolate bars, might be too precious for
everyday cooks. But it lets Justin Smillie, the chef of the space in the
back (the Vineria), do amazing things with dishes that on paper sound
commonplace.

A few drops of specially imported Sicilian anchovy sauce, called
colatura di alici, propel a roasted short rib for two to a height of
deliciousness that short ribs rarely scale. With a forceful crust of
peppercorns and coriander seeds, and crunchy bits of meat that are this
restaurant's answer to Kansas City's burnt ends, this may be one of the
best new dishes in town. Actually, since the same meat reappears as a
sandwich filling at lunch, it may be two of the best new dishes.

Bucatini alla gricia is a classic bare-bones production of pecorino,
pepper and cured jowl. It is what Romans put together when there is
nothing else in the house. When it's cooked fearlessly, like it is here,
you don't need anything else in the house. Anyone still under the
impression that dried pasta is inherently inferior to fresh will be
quickly and permanently set straight.

And if that doesn't work, they should proceed to the spaghetti with
bottarga to study the way the mullet roe sauce wraps itself around every
strand, sheathing it in creamy maritime intensity. On the night I tried
it, I only wished that the spaghetti had had a little more snap;
sometimes the pasta station here seems to lose track of time, and a bowl
of noodles that should be tensely coiled is allowed to unwind just a
bit.

Not that Mr. Smillie spends his days opening packages of spaghetti. He
is, after all, the chef who was fired from a doomed East Village
gastropub for spending too much at the Greenmarket. (More compatibly, he
also cooked under Jonathan Waxman and Dan Silverman.) Somehow in
February he hunted down the ripest persimmon I've tasted in a year,
sending out wedges of it with a grilled quail that had not a hint of the
liver overtones quail sometimes gets. And he found tiny beets to plant
like tulip bulbs in a snowy drift of ricotta curds, with shiny tongues
of white grapefruit on the side.

Still, what makes Il Buco Alimentari e Vineria stand out are the
products made in this village-inside-a-restaurant, and above all the
cured meats. When Americans first conceived the myth of the great,
simple Italian restaurant back home, the one thing that was most out of
reach would have been salumi of this quality.

Ms. Lennard first entered the cured-meats business about a decade ago at
her first restaurant, Il Buco, a block away on Bond Street. At the time,
only a few people in the United States were trying to replicate
Italian-style salumi. As they set out to study a very intricate craft,
they discovered that even fewer Americans were raising the old breeds of
hogs whose thick sheath of sweet, creamy fat is essential to great
salumi.

If you somehow assembled the right skills and the right pork, you could
still be stopped by the law. Ask Ms. Lennard about that. In 2006, the
city's health department told her that the temperatures in Il Buco's
basement curing facility were all wrong.

``They made my guys chop all this gorgeous prosciutto, some of it two
years old, into little pieces, and pour bleach over it, and put it into
plastic bags, and poor Bernardo in the morning had to throw them into
the garbage truck,'' she said by phone.

Bernardo Flores, the butcher in Ms. Lennard's village, now has high-tech
curing equipment in the basement of the new restaurant, imported from
Italy at some expense. To devise a safety plan that keeps the inspectors
at bay, she hired Christopher Lee, a longtime salumi maker at Chez
Panisse.

The work they are doing now, with pork from some of the best pig farmers
on the East Coast, is the kind of exceptional reward Manhattan bestows
on people who are stubborn, tireless and have the right timing. And on
the rest of us, too, if we are lucky.

Advertisement

\protect\hyperlink{after-bottom}{Continue reading the main story}

\hypertarget{site-index}{%
\subsection{Site Index}\label{site-index}}

\hypertarget{site-information-navigation}{%
\subsection{Site Information
Navigation}\label{site-information-navigation}}

\begin{itemize}
\tightlist
\item
  \href{https://help.nytimes3xbfgragh.onion/hc/en-us/articles/115014792127-Copyright-notice}{©~2020~The
  New York Times Company}
\end{itemize}

\begin{itemize}
\tightlist
\item
  \href{https://www.nytco.com/}{NYTCo}
\item
  \href{https://help.nytimes3xbfgragh.onion/hc/en-us/articles/115015385887-Contact-Us}{Contact
  Us}
\item
  \href{https://www.nytco.com/careers/}{Work with us}
\item
  \href{https://nytmediakit.com/}{Advertise}
\item
  \href{http://www.tbrandstudio.com/}{T Brand Studio}
\item
  \href{https://www.nytimes3xbfgragh.onion/privacy/cookie-policy\#how-do-i-manage-trackers}{Your
  Ad Choices}
\item
  \href{https://www.nytimes3xbfgragh.onion/privacy}{Privacy}
\item
  \href{https://help.nytimes3xbfgragh.onion/hc/en-us/articles/115014893428-Terms-of-service}{Terms
  of Service}
\item
  \href{https://help.nytimes3xbfgragh.onion/hc/en-us/articles/115014893968-Terms-of-sale}{Terms
  of Sale}
\item
  \href{https://spiderbites.nytimes3xbfgragh.onion}{Site Map}
\item
  \href{https://help.nytimes3xbfgragh.onion/hc/en-us}{Help}
\item
  \href{https://www.nytimes3xbfgragh.onion/subscription?campaignId=37WXW}{Subscriptions}
\end{itemize}
