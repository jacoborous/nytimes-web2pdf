Sections

SEARCH

\protect\hyperlink{site-content}{Skip to
content}\protect\hyperlink{site-index}{Skip to site index}

\href{https://www.nytimes3xbfgragh.onion/section/business}{Business}

\href{https://myaccount.nytimes3xbfgragh.onion/auth/login?response_type=cookie\&client_id=vi}{}

\href{https://www.nytimes3xbfgragh.onion/section/todayspaper}{Today's
Paper}

\href{/section/business}{Business}\textbar{}Refinery Gets a Look From
Delta, Perplexing Analysts

\begin{itemize}
\item
\item
\item
\item
\item
\end{itemize}

Advertisement

\protect\hyperlink{after-top}{Continue reading the main story}

Supported by

\protect\hyperlink{after-sponsor}{Continue reading the main story}

\hypertarget{refinery-gets-a-look-from-delta-perplexing-analysts}{%
\section{Refinery Gets a Look From Delta, Perplexing
Analysts}\label{refinery-gets-a-look-from-delta-perplexing-analysts}}

By \href{https://www.nytimes3xbfgragh.onion/by/jad-mouawad}{Jad Mouawad}

\begin{itemize}
\item
  April 4, 2012
\item
  \begin{itemize}
  \item
  \item
  \item
  \item
  \item
  \end{itemize}
\end{itemize}

Running an airline is a tough business. But running an oil refinery can
be even more punishing.

So refining experts were puzzled this week when Delta Air Lines emerged
as a possible buyer of a refinery near Philadelphia that ConocoPhillips
is trying to sell.

``It's a little like a rabbi buying a church,'' said Tom Kloza, the
publisher and chief oil analyst at the Oil Price Information Service,
which first reported Delta's possible interest on Monday. ``It's so
counterintuitive.''

Delta declined to comment. A spokesman for Conoco also declined to
comment on a potential bid for its Trainer, Pa., refinery, but said the
company was ``continuing efforts to find a buyer.'' The airline's bid
was confirmed by a person familiar with the situation who declined to be
identified because the deal was not final.

Rising fuel costs have forced a painful restructuring for airlines in
recent years, helping to push many of them into bankruptcy and spurring
consolidations across the sector. Jet fuel now accounts for about a
third of an airline's operating costs, a share that has been steadily
rising along with the price of crude oil.

The International Air Transport Association has estimated that the
global industry's fuel bill would grow by \$40 billion this year, up
from about \$177 billion in 2011.

In response, the airlines have set up elaborate hedging strategies to
try to control their fuel bills. But these hedges can sometimes backfire
if crude oil prices rise or fall in unexpected ways. Delta said that its
fuel bill last year was \$3 billion higher than in 2010 as oil prices
rebounded from their postrecession lows.

Still, the company's financial recovery in recent years has been solid,
thanks to its purchase of Northwest Airlines and efforts to fly fewer,
but fuller, planes. At the end of 2011, the company had \$3.6 billion in
cash and short-term investments and an undrawn credit line of \$1.8
billion. Last year, it posted a profit of \$854 million, up 44 percent
from the previous year.

\includegraphics{https://static01.graylady3jvrrxbe.onion/images/2012/04/05/business/Delta/Delta-jumbo.jpg?quality=75\&auto=webp\&disable=upscale}

On average, it paid \$2.97 per gallon for jet fuel in December, up from
\$2.47 a year earlier.

But if the airline business can be unpredictable, the refining industry
has been equally troubled. Except for a brief period from 2004 to 2008,
refining has been an unprofitable business, which explains why no new
refinery has been built in the United States in decades.

The refiners' hardships on the East Coast are especially pronounced.
Since September, two refineries in the Philadelphia region --- Conoco's
Trainer refinery and Sunoco's Marcus Hook --- and one in the Virgin
Islands, all supplying the East Coast market, have closed. These three
refineries account for half of the East Coast's refining capacity. The
Trainer refinery, which can handle 185,000 barrels per day, was closed
by Conoco in September, and employed 400 people.

So far, these closings have not much affected gasoline prices, since
supplies were readily available elsewhere. That could change if Sunoco
goes ahead with its plans to close another refinery near Philadelphia by
July if it can't find a buyer for that operation as well, according to a
recent report by the United States Energy Information Administration.

These refineries have suffered because the type of crude oil they
process --- light, sweet crude from the Atlantic --- happens to be the
most expensive. In recent months, the price for its benchmark, Brent
crude, has soared well above other grades of oil that are refined in
other parts of the country. Even as crude oil prices soared, and
gasoline demand dropped, refiners have faced a glut of gasoline-refining
capacity around the world.

``Refining is a tough business, probably tougher than the airlines,''
said Aaron Brady, a refining analyst at IHS CERA, an energy consulting
firm. ``These refineries on the East Coast are shutting down because
they are not competitive. They are just not making money. They need
investments to be reconfigured and be competitive. Why would an airline
do that? That seems strange to me.''

Analysts could not think of a precedent in which a domestic airline ever
owned a refinery.

Also, refiners cannot produce just one type of petroleum product. There
is very little wiggle room to the chemistry involved in processing a
barrel of oil into products like jet fuel, gasoline, diesel or heating
oil. So an airline, in theory, would end up with a raft of other
products even as it tried to maximize production of jet fuel.

Some oil analysts speculated that Delta would bring other partners ---
including perhaps a merchant bank that trades in crude oil --- into the
deal. One way that the East Coast refineries might be profitable would
be to bring in crude oil produced from the Bakken shale field of North
Dakota, which is currently selling at a discount. Without a pipeline,
this would mean using a combination of barges, trains and trucks, an
expensive proposition that justifies the discount for these resources.

``Bakken oil is discounted because it is landlocked,'' Mr. Brady said.
``It might look like a great opportunity now, but once it is not
landlocked anymore, it might not be discounted anymore.''

Advertisement

\protect\hyperlink{after-bottom}{Continue reading the main story}

\hypertarget{site-index}{%
\subsection{Site Index}\label{site-index}}

\hypertarget{site-information-navigation}{%
\subsection{Site Information
Navigation}\label{site-information-navigation}}

\begin{itemize}
\tightlist
\item
  \href{https://help.nytimes3xbfgragh.onion/hc/en-us/articles/115014792127-Copyright-notice}{©~2020~The
  New York Times Company}
\end{itemize}

\begin{itemize}
\tightlist
\item
  \href{https://www.nytco.com/}{NYTCo}
\item
  \href{https://help.nytimes3xbfgragh.onion/hc/en-us/articles/115015385887-Contact-Us}{Contact
  Us}
\item
  \href{https://www.nytco.com/careers/}{Work with us}
\item
  \href{https://nytmediakit.com/}{Advertise}
\item
  \href{http://www.tbrandstudio.com/}{T Brand Studio}
\item
  \href{https://www.nytimes3xbfgragh.onion/privacy/cookie-policy\#how-do-i-manage-trackers}{Your
  Ad Choices}
\item
  \href{https://www.nytimes3xbfgragh.onion/privacy}{Privacy}
\item
  \href{https://help.nytimes3xbfgragh.onion/hc/en-us/articles/115014893428-Terms-of-service}{Terms
  of Service}
\item
  \href{https://help.nytimes3xbfgragh.onion/hc/en-us/articles/115014893968-Terms-of-sale}{Terms
  of Sale}
\item
  \href{https://spiderbites.nytimes3xbfgragh.onion}{Site Map}
\item
  \href{https://help.nytimes3xbfgragh.onion/hc/en-us}{Help}
\item
  \href{https://www.nytimes3xbfgragh.onion/subscription?campaignId=37WXW}{Subscriptions}
\end{itemize}
