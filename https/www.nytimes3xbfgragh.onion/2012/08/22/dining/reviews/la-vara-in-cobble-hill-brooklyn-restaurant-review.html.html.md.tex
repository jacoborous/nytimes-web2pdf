Sections

SEARCH

\protect\hyperlink{site-content}{Skip to
content}\protect\hyperlink{site-index}{Skip to site index}

\href{https://www.nytimes3xbfgragh.onion/pages/dining/index.html}{Dining
\& Wine}

\href{https://myaccount.nytimes3xbfgragh.onion/auth/login?response_type=cookie\&client_id=vi}{}

\href{https://www.nytimes3xbfgragh.onion/section/todayspaper}{Today's
Paper}

\href{/pages/dining/index.html}{Dining \& Wine}\textbar{}A Reunion of
Flavors From Spain

\url{https://nyti.ms/SiDCOu}

\begin{itemize}
\item
\item
\item
\item
\item
\end{itemize}

Advertisement

\protect\hyperlink{after-top}{Continue reading the main story}

Supported by

\protect\hyperlink{after-sponsor}{Continue reading the main story}

Restaurant Review

\hypertarget{a-reunion-of-flavors-from-spain}{%
\section{A Reunion of Flavors From
Spain}\label{a-reunion-of-flavors-from-spain}}

\includegraphics{https://static01.graylady3jvrrxbe.onion/images/2012/08/22/dining/22REST1_SPAN/22REST1_SPAN-articleLarge.jpg?quality=75\&auto=webp\&disable=upscale}

By \href{https://www.nytimes3xbfgragh.onion/by/pete-wells}{Pete Wells}

\begin{itemize}
\item
  Aug. 21, 2012
\item
  \begin{itemize}
  \item
  \item
  \item
  \item
  \item
  \end{itemize}
\end{itemize}

APPEARANCES count in food, but great cooking is sometimes hard to spot
at first glance. When they arrive at the table, there is nothing
prepossessing about the fried artichokes served at La Vara, a
four-month-old Spanish restaurant in Cobble Hill, Brooklyn. I took one
look and thought, O.K., fried artichokes. I know what these are all
about.

So what was it that made me want to betray my friends by creating a
diversion (Hey, out there on Clinton Street!
\href{http://www.nytimes3xbfgragh.onion/2012/06/24/fashion/literary-brooklyn-gets-its-leading-man.html}{Is
that Martin Amis?}) so I could polish them off all by myself? It clearly
has something to do with the way all the luxuriously soft qualities of
confited artichokes are wrapped inside those crunchy fried leaves and
hearts, and just as much to do with that sauce, a creamy aioli with
force-multiplying jabs of anchovy.

Nor did I swoon at the arrival of La Vara's fideuà, thin noodles cooked
paella fashion in a seafood broth. Brown, broken and floppy, the noodles
look like a halfhearted attempt to disguise last night's pasta. All
leftovers should fare so well. Surrounded by tiny squid, clams and
shrimp, cooked until done and not an instant longer, the fideuà tastes a
little like toasted durum wheat and a lot like the dripping-wet haul
from a Valencian fishing net.

La Vara is the newest vessel in the small and nimble Spanish armada
commanded by Alex Raij and her husband and co-chef, Eder Montero.
Txikito and El Quinto Pino in Manhattan, which make up the rest of their
fleet, have done much to broaden the range of Spanish flavors in the
city. With La Vara, Ms. Raij and Mr. Montero are exploring new
territory, the vast legacy of the Jews and Muslims who shared the
Iberian Peninsula with Christians for centuries.

This three-way marriage, known as la convivencia, had its tensions, and
the breakup was ugly, but it did wonderful things for the country's
kitchens. Its imprint on Spain's cuisine isn't always evident. La Vara
urges, and rewards, a closer inspection.

During their long rule of Spain, as Claudia Roden writes in ``The Food
of Spain,'' the Muslims introduced ingredients that remain staples, like
rice, artichokes, eggplants, bitter oranges, cumin and saffron. They
brought skewers, noodles (fideuà comes from an Arabic word for pasta,
al-fidawsh), the shatteringly crisp savory pastries known as bricks and
chickpea-spinach stews.

All are on offer at La Vara. I'm not convinced that sweet-and-savory
chopped greens with pine nuts and currants are the ideal stuffing for a
brick. But grilled chicken hearts made plenty of sense on a skewer, and
while they could have been more tender, their pepper, caraway and
coriander seed seasoning was inviting, as was the herb salad alongside,
vibrant with lime. Another gift from the Moors was deep-fried fish, a
term that doesn't suggest just how delicious the long ribbons of fried,
marinated, pimentón-dusted skate called raya en adobo turn out to be.

\href{https://www.nytimes3xbfgragh.onion/slideshow/2012/08/22/dining/20120822-REST.html}{}

\hypertarget{la-vara}{%
\subsection{La Vara}\label{la-vara}}

8 Photos

View Slide Show ›

\includegraphics{https://static01.graylady3jvrrxbe.onion/images/2012/08/22/dining/20120822-REST-slide-0HOZ/20120822-REST-slide-0HOZ-jumbo.jpg?quality=75\&auto=webp\&disable=upscale}

Karsten Moran for The New York Times

But is La Vara good for the Jews? The history of Jewish cooking in Spain
is fraught, to put it mildly, especially after 1492, when Ferdinand and
Isabella compelled the Jews to convert or leave the kingdom. Converts
were often watched for suspiciously pork-free kitchens and smoke-free
chimneys on the Sabbath. Forced into hiding, Spanish-Jewish cuisine
virtually disappeared.

Traces remain, though, and Ms. Raij and Mr. Montero have
\href{http://www.tabletmag.com/jewish-life-and-religion/96987/cooking-up-spains-jewish-past}{excavated
some of them}. Those fried artichokes, for instance, may remind you of
the carciofi alla giudea placed in the Roman repertory by Sephardic
Jews.

One of La Vara's two best desserts is an almond cake called Torta de
Santiago. Thought to be Jewish in origin, it survived in, of all places,
Catholic convents. Understandably, La Vara skips the cake's traditional
ornament, a cross outlined in sugar. (The other standout is the Egipcio,
an improved Pop-Tart with date-and-walnut paste inside an exquisitely
tender semolina shortbread.)

The restaurant doesn't force-feed you any of this history, not in menu
footnotes or in carefully memorized tableside lectures. What most people
will notice is simply how inviting the place is behind its glass
storefront, brick walls glowing under white light, and how it manages to
feel intimate without feeling cramped. At any given table, the plates
are scattered all over, because La Vara serves most things as small
tapas-size dishes.

~Sometimes this works and sometimes it doesn't. If the bar isn't crowded
and time isn't tight, you could certainly stop for a glass of fino
sherry and crunch on a dish of excellent fried chickpeas, a marvelously
crisp and creamy croqueta or a single oil-cured sardine under sliced
radish pickles.

But the pleasures of eating tapas-style can get lost at a table for
four, where you may find that just as you realize how much you're
enjoying a dish, the person next to you has managed to stab the last
forkful. (See the Amis stratagem, above.) When will New York restaurants
stop peddling the myth of ``small plates meant for sharing''? Small
plates are meant for hoarding. This might be why the most satisfying
meal I had here was at a table for two. (If you have eaten out in the
last decade, you can guess that while the prices at La Vara may seem
low, the check won't.)

In truth, I was content to let others finish a few items at La Vara,
including the strange stewed bacon sandwiches, the thin and meek ajo
blanco with strands of squid, and the braised beef tongue in a watery
tomato-caper sauce. But such dishes were far outnumbered by the ones I
competed avidly for. If the griddled red shrimp came 12 to an order
instead of 2, I could have eaten them all, and sucked the bittersweet
juice from the head of each one. And I need to return for a second run
at the pasta called gurullos, if only to figure out how any pasta can be
as fluffy as an Italian grandmother's prizewinning gnocchi.

La Vara, by the way, was the name of a Jewish newspaper published in New
York until it ceased in 1948. It was written in Ladino, a
Castilian-Hebrew hybrid that was to Spain what Yiddish was to Eastern
Europe. The language has all but vanished from the city, but it remains
on the headstones of Sephardic graveyards in Queens and the Bronx, out
in plain sight for those who know where to look.

Advertisement

\protect\hyperlink{after-bottom}{Continue reading the main story}

\hypertarget{site-index}{%
\subsection{Site Index}\label{site-index}}

\hypertarget{site-information-navigation}{%
\subsection{Site Information
Navigation}\label{site-information-navigation}}

\begin{itemize}
\tightlist
\item
  \href{https://help.nytimes3xbfgragh.onion/hc/en-us/articles/115014792127-Copyright-notice}{©~2020~The
  New York Times Company}
\end{itemize}

\begin{itemize}
\tightlist
\item
  \href{https://www.nytco.com/}{NYTCo}
\item
  \href{https://help.nytimes3xbfgragh.onion/hc/en-us/articles/115015385887-Contact-Us}{Contact
  Us}
\item
  \href{https://www.nytco.com/careers/}{Work with us}
\item
  \href{https://nytmediakit.com/}{Advertise}
\item
  \href{http://www.tbrandstudio.com/}{T Brand Studio}
\item
  \href{https://www.nytimes3xbfgragh.onion/privacy/cookie-policy\#how-do-i-manage-trackers}{Your
  Ad Choices}
\item
  \href{https://www.nytimes3xbfgragh.onion/privacy}{Privacy}
\item
  \href{https://help.nytimes3xbfgragh.onion/hc/en-us/articles/115014893428-Terms-of-service}{Terms
  of Service}
\item
  \href{https://help.nytimes3xbfgragh.onion/hc/en-us/articles/115014893968-Terms-of-sale}{Terms
  of Sale}
\item
  \href{https://spiderbites.nytimes3xbfgragh.onion}{Site Map}
\item
  \href{https://help.nytimes3xbfgragh.onion/hc/en-us}{Help}
\item
  \href{https://www.nytimes3xbfgragh.onion/subscription?campaignId=37WXW}{Subscriptions}
\end{itemize}
