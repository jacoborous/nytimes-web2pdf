Sections

SEARCH

\protect\hyperlink{site-content}{Skip to
content}\protect\hyperlink{site-index}{Skip to site index}

\href{https://www.nytimes3xbfgragh.onion/pages/dining/index.html}{Dining
\& Wine}

\href{https://myaccount.nytimes3xbfgragh.onion/auth/login?response_type=cookie\&client_id=vi}{}

\href{https://www.nytimes3xbfgragh.onion/section/todayspaper}{Today's
Paper}

\href{/pages/dining/index.html}{Dining \& Wine}\textbar{}Like a Kid in a
Vegetable Store

\url{https://nyti.ms/10Q0k4X}

\begin{itemize}
\item
\item
\item
\item
\item
\end{itemize}

Advertisement

\protect\hyperlink{after-top}{Continue reading the main story}

Supported by

\protect\hyperlink{after-sponsor}{Continue reading the main story}

\hypertarget{like-a-kid-in-a-vegetable-store}{%
\section{Like a Kid in a Vegetable
Store}\label{like-a-kid-in-a-vegetable-store}}

\includegraphics{https://static01.graylady3jvrrxbe.onion/images/2012/11/28/dining/28DIRTCANDY_SPAN/28DIRTCANDY_SPAN-articleLarge.jpg?quality=75\&auto=webp\&disable=upscale}

By \href{https://www.nytimes3xbfgragh.onion/by/pete-wells}{Pete Wells}

\begin{itemize}
\item
  Nov. 27, 2012
\item
  \begin{itemize}
  \item
  \item
  \item
  \item
  \item
  \end{itemize}
\end{itemize}

DID you hear the one about the rutabaga? No, probably not. Vegetables
don't tend to inspire great jokes. On the contrary, they conjure their
very own brand of humorlessness, the tight-lipped officiousness of grim
nutritionists that is captured in the phrase ``Eat your vegetables.''

Since opening Dirt Candy in the East Village almost four years ago, the
chef
\href{http://dinersjournal.blogs.nytimes3xbfgragh.onion/2012/08/16/q-a-amanda-cohen-of-dirt-candy/}{Amanda
Cohen} has been waging war on the ``eat your vegetables'' mind-set,
using humor as one of her weapons. Here she is on her restaurant's Web
site on the subject of cabbage: ``It's the wino of the vegetable world:
smelly, unloved and looking at it makes you feel sad. A cabbage salad
sounds even worse. Cabbage that's ... raw? It sounds like a plated
suicide note.''

Humor is so integral to Ms. Cohen's work that she may be the only chef
in America who could publish her first
\href{http://www.youtube.com/watch?feature=player_embedded\&v=1u7GfLGNG3s}{cookbook}
in comic-book form and make the decision seem not just sensible but
inevitable. Drawn by Ryan Dunlavey with facial expressions that range
from mild exasperation to howling rage, Ms. Cohen's cartoon character
delivers bitingly funny monologues about appearing on ``Iron Chef'' and
rolling out pasta dough. Along the way, she and her staff beat the
daylights out of a wizard and a fairy who are idiotic enough to suggest
that pickles are made by magic.

But Ms. Cohen's chief weapon in the battle against vegetable scolds is
the food she serves at Dirt Candy, a vegetarian restaurant that is
thrillingly free of high-minded ideals. Ms. Cohen isn't necessarily out
to save a small planet or prevent heart disease. Her goals are possibly
more subversive, and she achieves them with vegetable cookery that is
original, clever, visually arresting and, above all, a lot of fun to
eat.

Take the dish called
\href{http://www.dirtcandynyc.com/?p=2950}{Cauliflower!} The punctuation
is the restaurant's, but I'll go along with the emotion. Ms. Cohen gives
cauliflower florets a long bath in maple smoke, dips them in cornflakes,
fries them to a golden crisp and serves them on waffles. She's making a
visual pun on the chicken-leg shape of the florets and stems. But these
smoky and crunchy nuggets also invite you to see vegetables as an
indulgence, a pleasure that is not quite guilty but not entirely
innocent, either.

Eating at Dirt Candy can be like going to a child's birthday party in a
country where all the children love vegetables.
\href{http://www.dirtcandynyc.com/?p=4268}{Eggplant tiramisù}, with
layers of rosemary lady fingers sandwiching a sweet mascarpone whipped
with grilled eggplant, sounds like a dare: daring you to try it, daring
you to like it. While you consider your options, a server brings the
final element of the dish, a white frizz of cotton candy with a
startlingly pure and piney taste of rosemary. It lifts up the eggplant,
the dessert and the party.

\href{https://www.nytimes3xbfgragh.onion/slideshow/2012/11/28/dining/20121128-DIRTCANDY.html}{}

\hypertarget{dirt-candy}{%
\subsection{Dirt Candy}\label{dirt-candy}}

8 Photos

View Slide Show ›

\includegraphics{https://static01.graylady3jvrrxbe.onion/images/2012/11/28/dining/20121128-DIRTCANDY-slide-F7RA/20121128-DIRTCANDY-slide-F7RA-jumbo.jpg?quality=75\&auto=webp\&disable=upscale}

Michael Nagle for The New York Times

One way of cooking vegetables is to go deep, to concentrate flavors
until you have the world's most carroty carrot, and leave it at that.
Ms. Cohen builds layers of flavor from the outside in. During the fall,
she constructs a \href{http://www.dirtcandynyc.com/?p=4914}{savory tart}
using tomatoes in every way imaginable, to intensify the effect. There
are dried tomatoes in a pink, biscuit-like crust; there are peeled and
marinated cherry tomatoes on top of that crust, with a creamy layer of
smoked feta; there is a ribbon of sweet and chewy tomato leather around
the tart; finally, off to the side, there is a salad of fresh and
dehydrated cherry tomatoes. The whole was more than the sum of its
parts, but all the parts were good to begin with.

The same isn't true yet of a new dish (Onions!) built around small
Chinese-derived cakes of smoked scallions and chives. One night they
were overwhelmingly smoky; on another night the smoke was more reticent
but the cakes themselves tasted flat. Both nights, a salad of grilled
scallions, red onions and herbs could have used a livelier, more
decisive dressing. But it's very difficult to criticize a dish that
includes the world's tiniest onion rings, each just big enough to fit on
the tip of a chopstick.

These elaborate plates, each of which holds about a half-dozen
components and can be made in vegan form upon request, emerge from a
visibly cramped kitchen. This may seem like a marvel, but it pales next
to the logistical feat Ms. Cohen and her staff pulled off early this
month. Like hundreds of restaurants downtown, Dirt Candy lost power for
four days after Hurricane Sandy. Every perishable, including stocks,
jams and pickles, had to be tossed onto the sidewalk in trash bags.
Almost all of it was replenished in just two days, which meant
scheduling times and temperatures for the single oven down to the
minute. It must have been like re-enacting the invasion of Normandy in a
kiddie pool.

It also involved the same kind of spatial planning that allows Dirt
Candy to fit 18 seats into a spare, modernist dining room where even one
more might be too much. Wineglasses are stored on a sea-green glass
ledge above the tables, and customers waiting for a table are stored
outside on Ninth Street.

Vegetarian cooks in the United States have not always found much
nourishment in American cuisine. Some looked to health food and
ideology, while others have looked abroad.

Ms. Cohen dabbles in overseas influences.
\href{http://www.dirtcandynyc.com/?p=4607}{Cabbage}!, a lonesome hobo no
more, is reimagined as the star of a marvelous Asian noodle salad, with
kohlrabi threads standing in for noodles, fried walnuts carrying a spicy
blast of Sichuan peppercorn, and purple fried won tons getting their
color and flavor from juiced cabbage. A new item on the menu, Beans!,
surrounds haricots verts and crisp tofu with a deeply intriguing sauce
in which saffron, ginger, cilantro, coriander seed, cumin and Urfa biber
pepper offer echoes of Morocco.

But there is no mistaking a Dirt Candy dish for one that you would find
in another country. Ms. Cohen is not adapting the vegetarian cuisine of
some other culture. She is inventing her own.

Advertisement

\protect\hyperlink{after-bottom}{Continue reading the main story}

\hypertarget{site-index}{%
\subsection{Site Index}\label{site-index}}

\hypertarget{site-information-navigation}{%
\subsection{Site Information
Navigation}\label{site-information-navigation}}

\begin{itemize}
\tightlist
\item
  \href{https://help.nytimes3xbfgragh.onion/hc/en-us/articles/115014792127-Copyright-notice}{©~2020~The
  New York Times Company}
\end{itemize}

\begin{itemize}
\tightlist
\item
  \href{https://www.nytco.com/}{NYTCo}
\item
  \href{https://help.nytimes3xbfgragh.onion/hc/en-us/articles/115015385887-Contact-Us}{Contact
  Us}
\item
  \href{https://www.nytco.com/careers/}{Work with us}
\item
  \href{https://nytmediakit.com/}{Advertise}
\item
  \href{http://www.tbrandstudio.com/}{T Brand Studio}
\item
  \href{https://www.nytimes3xbfgragh.onion/privacy/cookie-policy\#how-do-i-manage-trackers}{Your
  Ad Choices}
\item
  \href{https://www.nytimes3xbfgragh.onion/privacy}{Privacy}
\item
  \href{https://help.nytimes3xbfgragh.onion/hc/en-us/articles/115014893428-Terms-of-service}{Terms
  of Service}
\item
  \href{https://help.nytimes3xbfgragh.onion/hc/en-us/articles/115014893968-Terms-of-sale}{Terms
  of Sale}
\item
  \href{https://spiderbites.nytimes3xbfgragh.onion}{Site Map}
\item
  \href{https://help.nytimes3xbfgragh.onion/hc/en-us}{Help}
\item
  \href{https://www.nytimes3xbfgragh.onion/subscription?campaignId=37WXW}{Subscriptions}
\end{itemize}
