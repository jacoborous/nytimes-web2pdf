Sections

SEARCH

\protect\hyperlink{site-content}{Skip to
content}\protect\hyperlink{site-index}{Skip to site index}

\href{https://myaccount.nytimes3xbfgragh.onion/auth/login?response_type=cookie\&client_id=vi}{}

\href{https://www.nytimes3xbfgragh.onion/section/todayspaper}{Today's
Paper}

\hypertarget{the-coronavirus-outbreak}{%
\subsubsection{\texorpdfstring{\href{https://www.nytimes3xbfgragh.onion/news-event/coronavirus?name=styln-coronavirus-markets\&region=TOP_BANNER\&variant=undefined\&block=storyline_menu_recirc\&action=click\&pgtype=LegacyCollection\&impression_id=f4eb6140-e38a-11ea-8da8-0956b6b90375}{The
Coronavirus
Outbreak}}{The Coronavirus Outbreak}}\label{the-coronavirus-outbreak}}

\begin{itemize}
\tightlist
\item
  live\href{https://www.nytimes3xbfgragh.onion/2020/08/20/world/coronavirus-covid.html?name=styln-coronavirus-markets\&region=TOP_BANNER\&variant=undefined\&block=storyline_menu_recirc\&action=click\&pgtype=LegacyCollection\&impression_id=f4eb6141-e38a-11ea-8da8-0956b6b90375}{Latest
  Updates}
\item
  \href{https://www.nytimes3xbfgragh.onion/interactive/2020/us/coronavirus-us-cases.html?name=styln-coronavirus-markets\&region=TOP_BANNER\&variant=undefined\&block=storyline_menu_recirc\&action=click\&pgtype=LegacyCollection\&impression_id=f4eb6142-e38a-11ea-8da8-0956b6b90375}{Maps
  and Cases}
\item
  \href{https://www.nytimes3xbfgragh.onion/interactive/2020/science/coronavirus-vaccine-tracker.html?name=styln-coronavirus-markets\&region=TOP_BANNER\&variant=undefined\&block=storyline_menu_recirc\&action=click\&pgtype=LegacyCollection\&impression_id=f4eb6143-e38a-11ea-8da8-0956b6b90375}{Vaccine
  Tracker}
\item
  \href{https://www.nytimes3xbfgragh.onion/2020/08/19/us/colleges-closing-covid.html?name=styln-coronavirus-markets\&region=TOP_BANNER\&variant=undefined\&block=storyline_menu_recirc\&action=click\&pgtype=LegacyCollection\&impression_id=f4eb6144-e38a-11ea-8da8-0956b6b90375}{Colleges
  Closing}
\item
  \href{https://www.nytimes3xbfgragh.onion/live/2020/08/20/business/stock-market-today-coronavirus?name=styln-coronavirus-markets\&region=TOP_BANNER\&variant=undefined\&block=storyline_menu_recirc\&action=click\&pgtype=LegacyCollection\&impression_id=f4eb6145-e38a-11ea-8da8-0956b6b90375}{Economy}
\end{itemize}

\hypertarget{robinhood-raises-200-million-from-investors-live-updates}{%
\section{Robinhood Raises \$200 Million From Investors: Live
Updates}\label{robinhood-raises-200-million-from-investors-live-updates}}

Last Updated

Aug. 17, 2020, 5:24 p.m. ET

Aug. 17, 2020, 5:24 p.m. ET

\hypertarget{heres-what-you-need-to-know}{%
\subsubsection{Here's what you need to
know:}\label{heres-what-you-need-to-know}}

\begin{itemize}
\item
  \protect\hyperlink{new-unemployment-benefits-authorized-by-president-trump-wont-come-until-late-august}{}

  New unemployment benefits authorized by President Trump won't come
  until late August.
\item
  \protect\hyperlink{most-unemployed-americans-doubt-they-will-return-to-their-jobs}{}

  Most unemployed Americans doubt they will return to their jobs.
\item
  \protect\hyperlink{wall-street-starts-the-week-with-a-gain-but-cant-break-through-to-a-record}{}

  Wall Street starts the week with a gain, but can't break through to a
  record.
\item
  \protect\hyperlink{europes-big-oil-companies-move-toward-an-electric-future}{}

  Europe's big oil companies move toward an electric future.
\end{itemize}

\hypertarget{robinhood-a-stock-trading-app-is-valued-at-112-billion-in-its-latest-funding-round}{%
\subsection{\texorpdfstring{\protect\hyperlink{robinhood-a-stock-trading-app-is-valued-at-11-2-billion-in-its-latest-funding-round}{Robinhood,
a stock trading app, is valued at \$11.2 billion in its latest funding
round.}}{Robinhood, a stock trading app, is valued at \$11.2 billion in its latest funding round.}}\label{robinhood-a-stock-trading-app-is-valued-at-112-billion-in-its-latest-funding-round}}

Copied to clipboard.

\includegraphics{https://static01.graylady3jvrrxbe.onion/images/2020/08/17/business/17markets-brf-robinhood/merlin_173900673_796f80b3-2187-42f6-aeca-a8703effedcc-articleLarge.jpg?quality=75\&auto=webp\&disable=upscale}

\textbf{Robinhood}, a stock trading app whose popularity has surged in
the pandemic, has raised another \$200 million in funding,
\href{https://blog.robinhood.com/news/2020/8/17/series-g}{the company
said on Monday}, bringing its funding total to \$800 million in recent
months, and more than \$1 billion since it was founded seven years ago.

The new round of funding, led by the hedge fund \textbf{D1 Capital
Partners}, values the start-up at \$11.2 billion.

Use of Robinhood's app has exploded in recent months as people have been
bored during the pandemic and stock market volatility has made day
trading into an exciting hobby. In May, the company announced it had 13
million accounts.

But the company's growth has also attracted critics. In March, the
company's service crashed several times in one week, leaving customers
unable to transact as the stock market plummeted.

And traders who have lost money point to features in the app that
\href{https://www.nytimes3xbfgragh.onion/2020/07/08/technology/robinhood-risky-trading.html}{make
stock trading feel like a game}. Ashton Kutcher, one of Robinhood's
investors, has even compared it to gambling,
\href{https://www.nytimes3xbfgragh.onion/2020/07/08/technology/robinhood-risky-trading.html}{The
New York Times previously reported}. (Mr. Kutcher said he was not
insinuating Robinhood was a gambling platform.)

In recent months, Robinhood has emphasized its education materials for
responsible investing. The company said it planned to use the new money
to expand its customer service team by hiring hundreds of people. (It
does not have a phone number for customers to call.)

Robinhood has been a disruptive force in the investment world. It does
not charge its users any fees for trading; rather, it makes money by
selling the transaction to larger Wall Street firms. Last year, Charles
Schwab, TD Ameritrade and E*Trade dropped their trading fees to compete.

--- \href{https://www.nytimes3xbfgragh.onion/by/erin-griffith}{Erin
Griffith}

\hypertarget{new-unemployment-benefits-authorized-by-president-trump-wont-come-until-late-august}{%
\subsection{\texorpdfstring{\protect\hyperlink{new-unemployment-benefits-authorized-by-president-trump-wont-come-until-late-august}{New
unemployment benefits authorized by President Trump won't come until
late
August.}}{New unemployment benefits authorized by President Trump won't come until late August.}}\label{new-unemployment-benefits-authorized-by-president-trump-wont-come-until-late-august}}

Copied to clipboard.

\includegraphics{https://static01.graylady3jvrrxbe.onion/images/2020/08/17/business/17markets-brf-FEMA/17markets-brf-FEMA-articleLarge.jpg?quality=75\&auto=webp\&disable=upscale}

Most American workers expecting \$300 or \$400 in extra weekly
unemployment benefits from the federal government are unlikely to see it
until the end of August, the Federal Emergency Management Agency said on
Monday.

President Trump signed an executive order earlier this month, saying he
was
\href{https://www.nytimes3xbfgragh.onion/2020/08/08/us/politics/trump-stimulus-bill-coronavirus.html}{bypassing
Congress to deliver emergency pandemic aid} by directing F.E.M.A to use
federal disaster assistance funds in order to increase the benefits of
workers left unemployed by the pandemic recession.

On Monday, F.E.M.A. officials said seven states had thus far been
approved to receive the money: Arizona, Iowa, Louisiana, New Mexico,
Colorado, Missouri and Utah. They also laid out the timetable for those
funds in a new
``\href{https://www.fema.gov/sites/default/files/2020-08/fema_supplement-lost-wages-payments-under-other-needs-assistance_faq_0.pdf}{frequently
asked questions''} guidance about the program.

The guidance, citing the Labor Department, estimates that states will
need an average of three weeks from the date of Mr. Trump's order, Aug.
8, to adjust their unemployment systems in order to be eligible for the
grants, which will supply \$300 per week of federal funds per worker,
with an option for states to kick in an additional \$100 per week.
States that win approval, it says, will see money flow after one
business day.

Initial grants will provide only three weeks worth of benefits per
state, the guidance said, and additional funds will be provided on a
week-to-week basis ``in order to ensure that funding remains available
for the states who apply for the grant assistance.''

Some state leaders have expressed reservations about applying for the
money. South Dakota's governor has
\href{https://www.cnn.com/2020/08/16/politics/south-dakota-trump-unemployment-benefits/index.html}{opted
not to join the program}.

--- \href{https://www.nytimes3xbfgragh.onion/by/jim-tankersley}{Jim
Tankersley}

\hypertarget{advertisement}{%
\subsubsection{Advertisement}\label{advertisement}}

\protect\hyperlink{after-dfp-ad-mid1}{Continue reading the main story}

\hypertarget{most-unemployed-americans-doubt-they-will-return-to-their-jobs}{%
\subsection{\texorpdfstring{\protect\hyperlink{most-unemployed-americans-doubt-they-will-return-to-their-jobs}{Most
unemployed Americans doubt they will return to their
jobs.}}{Most unemployed Americans doubt they will return to their jobs.}}\label{most-unemployed-americans-doubt-they-will-return-to-their-jobs}}

Copied to clipboard.

As the pandemic-induced economic crisis drags on, jobless Americans are
becoming more pessimistic about their prospects for getting back to
work.

Nearly six in 10 Americans who are out of work because of the pandemic
say they do not expect to return to their old jobs, according to a
survey this month for The New York Times by the online research platform
SurveyMonkey. That's up from half who said the same a month ago.

Of those who are still out of work, 13 percent anticipate returning to
their old jobs in the next month, down from 22 percent a month earlier.

The growing pessimism comes as hiring has slowed and other measures of
economic activity have lost momentum. The Times survey adds to the
evidence of a stall: The share of those surveyed who reported that they
had returned to work fell slightly in August, perhaps reflecting the new
wave of business closures in response to the virus. And overall consumer
confidence dipped. Only 24 percent of Americans now say they are better
off than a year ago, the lowest share in the survey's three and a half
years.

Economists say that if a large share of Americans are unable to return
to their old jobs, the recovery will be slower. The longer the crisis
lasts, the more likely that becomes: More than half of job seekers in
the Times survey report having been out of work for five months or
longer, consistent with other data showing rising levels of long-term
unemployment.

--- \href{https://www.nytimes3xbfgragh.onion/by/ben-casselman}{Ben
Casselman}

\hypertarget{wall-street-starts-the-week-with-a-gain-but-cant-break-through-to-a-record}{%
\subsection{\texorpdfstring{\protect\hyperlink{wall-street-starts-the-week-with-a-gain-but-cant-break-through-to-a-record}{Wall
Street starts the week with a gain, but can't break through to a
record.}}{Wall Street starts the week with a gain, but can't break through to a record.}}\label{wall-street-starts-the-week-with-a-gain-but-cant-break-through-to-a-record}}

Copied to clipboard.

Wall Street continued to tread water just below a record, with
technology stocks again leading the gains but the market's cautious tone
continuing to keep the rally limited.

The S\&P 500 rose about a quarter of a percent. European stocks were
also modestly higher.

For days, the S\&P 500 has toyed with ---~but failed to close above
---~its February record of 3,386.15. The index did cross above that
threshold for a short time on Monday, but again failed to hold onto the
high.

As they often have in recent days, technology stocks fared better than
the broader market, with the Nasdaq composite rising 1 percent.

Earlier Monday, Japanese authorities reported that the economy fell 7.8
percent in the second quarter, an annualized drop of 27.8 percent. It
was the third straight quarter of contraction for Japan, the world's
third-largest economy after the United States and China. Even before the
pandemic,
\href{https://www.nytimes3xbfgragh.onion/2020/06/20/business/japan-unemployment.html}{Japan's
economy} was weakened by a tax increase, slowing demand from China and a
series of natural disasters last fall.

But there are signs the worst may be over. By late in the second
quarter, analysts said, the full effects of Japan's economic stimulus
package, including cash handouts and zero-interest loans, began to be
felt, keeping joblessness and bankruptcies low.

The prospect of an economic recovery --- in Japan, China or the United
States --- has helped lift share prices around the world, after they
suffered a staggering decline earlier this year. On Wall Street, the
S\&P 500 is up about 50 percent since the depths of the market slump in
March, despite the millions unemployed and thousands of businesses still
shuttered.

That recovery has also been fueled by trillions of dollars pumped into
the financial markets by the
\href{https://www.nytimes3xbfgragh.onion/live/2020/08/12/business/stock-market-today-coronavirus/want-the-economy-to-rebound-wear-a-mask-federal-reserve-official-says}{Federal
Reserve} and spending by the government to cushion the worst of the
downturn. And though the virus continues to exact a toll on the American
economy, and cases are surging in many states, investors have largely
looked the other way in recent weeks.

--- \href{https://www.nytimes3xbfgragh.onion/by/kevin-granville}{Kevin
Granville} and Mohammed Hadi

\hypertarget{europes-big-oil-companies-move-toward-an-electric-future}{%
\subsection{\texorpdfstring{\protect\hyperlink{europes-big-oil-companies-move-toward-an-electric-future}{Europe's
big oil companies move toward an electric
future.}}{Europe's big oil companies move toward an electric future.}}\label{europes-big-oil-companies-move-toward-an-electric-future}}

Copied to clipboard.

\includegraphics{https://static01.graylady3jvrrxbe.onion/images/2020/08/03/business/00oilclimate1/merlin_156670797_8d377543-4ca9-4ab0-85b0-e05fc89fc0b4-articleLarge.jpg?quality=75\&auto=webp\&disable=upscale}

This may turn out to be the year that oil giants, especially in Europe,
started looking more like
\href{https://www.nytimes3xbfgragh.onion/2020/08/04/business/energy-environment/bp-renewable-investment.html}{electric
companies}.

Late last month, \textbf{Royal Dutch Shell} won a deal to build a vast
wind farm off the coast of the Netherlands. Earlier in the year,
France's \textbf{Total}, which owns a battery maker, agreed to make
several large investments in solar power in Spain and a wind farm off
Scotland. Total also bought an electric and natural gas utility in Spain
and is joining Shell and \textbf{BP} in expanding its electric vehicle
charging business.

At the same time, the companies are ditching plans to drill more wells
as they chop back capital budgets. Shell recently said it would delay
new fields in the Gulf of Mexico and in the North Sea, while BP has
promised not to hunt for oil in any new countries.

Prodded by
\href{https://www.nytimes3xbfgragh.onion/2019/10/07/business/energy-environment/oil-companies-climate-change-profits.html}{governments}
and
\href{https://www.nytimes3xbfgragh.onion/2020/01/31/business/church-of-england-climate-change.html}{investors}
to address climate change concerns about their products, Europe's oil
companies are accelerating their production of cleaner energy ---
usually electricity, sometimes
\href{https://www.nytimes3xbfgragh.onion/2019/10/07/business/energy-environment/oil-companies-climate-change-profits.html}{hydrogen}
--- and promoting natural gas, which they argue can be a cleaner
transition fuel from coal and oil to renewables.

For some executives, the sudden
\href{https://www.nytimes3xbfgragh.onion/2020/07/30/business/shell-and-total-report-big-drop-in-profits-but-made-clean-energy-investments.html?searchResultPosition=8}{plunge
in demand for oil} caused by the pandemic --- and the accompanying
collapse in earnings --- is another warning that unless they change the
composition of their businesses, they risk being dinosaurs headed for
extinction.

``What the world wants from energy is changing,'' said
\href{https://www.nytimes3xbfgragh.onion/2019/10/04/business/bp-ceo-bob-dudley-bernard-looney.html}{Bernard
Looney}, a 29-year BP veteran who became chief executive in February,
``and so we need to change, quite frankly, what we offer the world.''

--- \href{https://www.nytimes3xbfgragh.onion/by/stanley-reed}{Stanley
Reed}

\hypertarget{advertisement-1}{%
\subsubsection{Advertisement}\label{advertisement-1}}

\protect\hyperlink{after-dfp-ad-mid2}{Continue reading the main story}

\hypertarget{heres-what-you-need-to-know-for-the-week-ahead}{%
\subsection{\texorpdfstring{\protect\hyperlink{heres-what-you-need-to-know-for-the-week-ahead}{Here's
what you need to know for the week
ahead.}}{Here's what you need to know for the week ahead.}}\label{heres-what-you-need-to-know-for-the-week-ahead}}

Copied to clipboard.

🗣 The \textbf{Democratic National Convention} will take place mostly
virtually, spread out over four nights, starting tonight. Speakers
\href{https://www.washingtonpost.com/politics/unconventional-democratic-convention-will-juggle-hundreds-of-live-feeds-to-re-create-the-feel-of-a-party-celebration/2020/08/15/02f887de-de60-11ea-809e-b8be57ba616e_story.html}{include}
former President Barack Obama, Hillary Clinton and Senator Bernie
Sanders. Senator Kamala Harris of California, the Democratic
vice-presidential candidate, will speak on Wednesday, and Joe Biden will
wrap it up on Thursday. The Times has a
\href{https://www.nytimes3xbfgragh.onion/2020/08/17/us/politics/democratic-national-convention-speakers-schedule.html}{guide
for how to watch}, and will offer live analysis throughout.

🛍 Retail earnings are in the spotlight this week, with \textbf{Home
Depot}, \textbf{Kohl's} and \textbf{Walmart} reporting on Tuesday;
\textbf{Lowe's} and \textbf{Target} on Wednesday; and \textbf{TJX} on
Thursday.

💰 Other noteworthy reports include \textbf{Norway's sovereign wealth
fund} on Tuesday; the shipping giant \textbf{A. P. Moller-Maersk} and
the chip maker \textbf{Nvidia} on Wednesday; and the heavy machinery
manufacturer \textbf{Deere \& Company} on Friday.

\textbf{🏦} Investors will have a chance to scrutinize the latest minutes
of recent meetings at the \textbf{U.S. Federal Reserve}, released on
Wednesday, and the \textbf{European Central Bank}, due on Thursday, for
clues as to what monetary policymakers are thinking about whether more
stimulus is needed.

--- \href{https://www.nytimes3xbfgragh.onion/by/jason-karaian}{Jason
Karaian}

\hypertarget{the-worlds-biggest-bike-maker-steers-around-trade-obstacles}{%
\subsection{\texorpdfstring{\protect\hyperlink{the-worlds-biggest-bike-maker-steers-around-trade-obstacles}{The
world's biggest bike maker steers around trade
obstacles.}}{The world's biggest bike maker steers around trade obstacles.}}\label{the-worlds-biggest-bike-maker-steers-around-trade-obstacles}}

Copied to clipboard.

\includegraphics{https://static01.graylady3jvrrxbe.onion/images/2020/08/06/business/06taiwan-bikes-2/merlin_174895581_ae9ec662-dc68-402a-b02c-b5df8ff01588-articleLarge.jpg?quality=75\&auto=webp\&disable=upscale}

The pandemic has caused a surge in bicycle sales around the world,
resulting in an international
\href{https://www.nytimes3xbfgragh.onion/2020/05/18/nyregion/bike-shortage-coronavirus.html}{bike
shortage}. And the world's largest bike maker, \textbf{Giant}, expects
its supplies to remain tight for some time to come.

After President Trump started his trade war with China in 2018, Giant
moved some of its manufacturing for the American market from China to
the company's home base in Taiwan to avoid the added tariffs. The
following year, the European Union imposed
\href{https://www.bicycleretailer.com/international/2019/01/18/eu-imposes-stiff-anti-dumping-duties-chinese-e-bikes\#.XyNs2xMzZmg}{antidumping
duties} on electric bikes from China, so Giant began making those in
Taiwan, too.

But when the pandemic caused demand for bikes to jump, Giant needed to
reverse course. With its Taiwan facility already under strain, the
company had little choice but to crank up production in China, even it
meant bearing the extra cost of tariffs.

``There's nowhere else in the world that can go like China from zero to
100 in an instant, like a sports car. \emph{Shyeew}!'' Giant's
chairwoman, Bonnie Tu, said in an interview.

The Trump administration this year
\href{https://ustr.gov/sites/default/files/enforcement/301Investigations/\%24200_Billion_Exclusions_Granted_April_2020.pdf}{temporarily
lifted tariffs} on a variety of Chinese-made goods that are deemed
strategically unimportant.
\href{https://www.federalregister.gov/documents/2020/02/05/2020-02225/notice-of-product-exclusions-chinas-acts-policies-and-practices-related-to-technology-transfer}{Bicycles
made the list}, which made it easier for Giant to go back to producing
some of its bikes for the U.S. market in China.

But \href{https://peopleforbikes.org/exclusions/}{the tariff pause} for
certain types of bikes expired this month, meaning Giant may need to
adjust its supply arrangements yet again.

Today all of Giant's factories are running nearly at full steam. Despite
the rush of first-time bike buyers, Ms. Tu does not plan to ``blindly''
invest in new manufacturing capacity.

``Every boom ends someday,'' she said. ``It's just a question of whether
it ends quickly or slowly.''

--- \href{https://www.nytimes3xbfgragh.onion/by/raymond-zhong}{Raymond
Zhong}

\hypertarget{kids-with-few-places-to-go-make-roblox-a-hit}{%
\subsection{\texorpdfstring{\protect\hyperlink{kids-with-few-places-to-go-make-roblox-a-hit}{Kids
with few places to go make Roblox a
hit.}}{Kids with few places to go make Roblox a hit.}}\label{kids-with-few-places-to-go-make-roblox-a-hit}}

Copied to clipboard.

\includegraphics{https://static01.graylady3jvrrxbe.onion/images/2020/08/17/business/00roblox1-print/merlin_175488597_aaf83b6d-54d7-4633-929f-5814cb11ebec-articleLarge.jpg?quality=75\&auto=webp\&disable=upscale}

The coronavirus has created some pandemic winners, as people shop in
droves
on\href{https://www.nytimes3xbfgragh.onion/live/2020/07/30/business/stock-market-today-coronavirus\#amazons-earnings-double-as-sales-surge}{}\textbf{\href{https://www.nytimes3xbfgragh.onion/live/2020/07/30/business/stock-market-today-coronavirus\#amazons-earnings-double-as-sales-surge}{Amazon}},
\href{https://www.nytimes3xbfgragh.onion/2020/05/06/technology/peloton-boom-workout-virus.html}{buy}\textbf{\href{https://www.nytimes3xbfgragh.onion/2020/05/06/technology/peloton-boom-workout-virus.html}{Peloton}}\href{https://www.nytimes3xbfgragh.onion/2020/05/06/technology/peloton-boom-workout-virus.html}{bikes}
to exercise at home and head to
\href{https://www.nytimes3xbfgragh.onion/2020/03/24/style/drive-in-theaters-coronavirus.html}{drive-in
movies}. For children, there are pandemic victors, too --- and chief
among them is \textbf{Roblox}, a 14-year-old online gaming site and app
with Lego-like characters and millions of virtual worlds to explore.

Since February, the number of active players on Roblox has jumped about
35 percent, reaching 164 million in July,
\href{https://blog.rtrack.live/index.php/2020/08/02/roblox-continues-upward-ascent-with-164-million-monthly-active-users/}{according
to RTrack}, a site that tracks Roblox data. About three-quarters of
American children ages 9 to 12 are now on the platform, according to
Roblox. And players spent three billion hours on the site and app in
July, twice as much as they did in February, the company said.

With so much time at home starting in March, Garvey Mortley began
logging more hours in the online universe, building virtual houses,
adopting digital pets and racing other players in obstacle courses. She
said she now plays Roblox on her laptop for up to five hours a day while
chatting with friends on her phone, up from an hour or two before
\href{https://www.nytimes3xbfgragh.onion/news-event/coronavirus?name=styln-coronavirus-national\&region=TOP_BANNER\&variant=1_Show\&block=storyline_menu_recirc\&action=click\&pgtype=Interactive\&impression_id=cfb8d020-dd71-11ea-bf10-75191c97e453}{the
pandemic}. ``It's like my main passion,'' said Garvey, 12. ``It's pretty
diverse, and you can meet people around the world.''

Roblox is free to play, but gamers pay real money --- often \$5 or \$10
at a time --- to become premium members and to buy an in-game currency
called Robux, which lets them buy clothing, weapons and even hot air
balloons for their characters.

``At a time like this, where people are housebound, being able to escape
into the digital world and have these kinds of fun, imaginative
experiences with a friend, is very, very relevant,'' said Craig Donato,
Roblox's chief business officer.

--- \href{https://www.nytimes3xbfgragh.onion/by/kellen-browning}{Kellen
Browning}

\hypertarget{advertisement-2}{%
\subsubsection{Advertisement}\label{advertisement-2}}

\protect\hyperlink{after-dfp-ad-mid3}{Continue reading the main story}

\hypertarget{nursing-homes-with-safety-problems-are-deploying-lobbyists-with-trump-ties}{%
\subsection{\texorpdfstring{\protect\hyperlink{nursing-homes-with-safety-problems-are-deploying-lobbyists-with-trump-ties}{Nursing
homes with safety problems are deploying lobbyists with Trump
ties.}}{Nursing homes with safety problems are deploying lobbyists with Trump ties.}}\label{nursing-homes-with-safety-problems-are-deploying-lobbyists-with-trump-ties}}

Copied to clipboard.

\includegraphics{https://static01.graylady3jvrrxbe.onion/images/2020/08/13/business/00Virus-NursingHomeLobby-Kirkland-01/merlin_175635402_7f931de3-acb5-4688-9dd8-3dcb8fce07b6-articleLarge.jpg?quality=75\&auto=webp\&disable=upscale}

Nursing homes have been the center of America's coronavirus pandemic,
with more than 62,000 residents and staff dying from Covid-19 at nursing
homes and other long-term care facilities, about 40 percent of the
country's Covid-19 fatalities.

Now, the lightly regulated industry is campaigning in Washington for
federal help that could increase its profits.

Some of the country's largest nursing-home companies --- including those
with long histories of safety violations and misusing public funds ---
have assembled a fleet of lobbyists, many with close ties to the Trump
administration.

\begin{itemize}
\item
  Eliezer Scheiner, a nursing-home owner and major donor to President
  Trump, recently retained
  \href{http://ballardpartners.com/the-team/brian-d-ballard/}{Brian
  Ballard}, a friend of the president who used to lobby on behalf of Mr.
  Trump's business.
\item
  \textbf{Genesis Healthcare}, the largest nursing-home chain in the
  United States, hired two former top White House aides, including
  \href{https://www.cozen.com/people/bios/schultz-james}{Jim Schultz}, a
  former special assistant to Mr. Trump.
\item
  \textbf{LifeCare Centers of America}, whose Kirkland, Wash., facility
  had the country's first coronavirus outbreak in March, brought on four
  former Republican Senate aides.
\item
  The industry's main trade group enlisted Haley Barbour, a former
  chairman of the Republic National Committee.
\end{itemize}

It is hardly unusual for embattled industries to seek help from
Washington. But the fact that individual nursing-home companies are
hiring lobbyists, not just relying on trade associations, reflects the
ambitious nature of the industry's mobilization.

Nursing homes --- many of which were in deep
\href{https://www.nytimes3xbfgragh.onion/2020/05/07/business/coronavirus-nursing-homes.html}{financial
trouble} even before the pandemic --- are also on the hunt for
government cash infusions through the federal economic rescue that
became law in March, as well as any future stimulus bills.

Among the industry's biggest goals, though, is for the federal
government to block residents and their families from suing nursing
homes for wrongful deaths and other malpractice claims --- even those
that have nothing to do with Covid-19.

---
\href{https://www.nytimes3xbfgragh.onion/by/jessica-silver-greenberg}{Jessica
Silver-Greenberg} and
\href{https://www.nytimes3xbfgragh.onion/by/jesse-drucker}{Jesse
Drucker}

\hypertarget{site-index}{%
\subsection{Site Index}\label{site-index}}

\hypertarget{site-information-navigation}{%
\subsection{Site Information
Navigation}\label{site-information-navigation}}

\begin{itemize}
\tightlist
\item
  \href{https://help.nytimes3xbfgragh.onion/hc/en-us/articles/115014792127-Copyright-notice}{©~2020~The
  New York Times Company}
\end{itemize}

\begin{itemize}
\tightlist
\item
  \href{https://www.nytco.com/}{NYTCo}
\item
  \href{https://help.nytimes3xbfgragh.onion/hc/en-us/articles/115015385887-Contact-Us}{Contact
  Us}
\item
  \href{https://www.nytco.com/careers/}{Work with us}
\item
  \href{https://nytmediakit.com/}{Advertise}
\item
  \href{http://www.tbrandstudio.com/}{T Brand Studio}
\item
  \href{https://www.nytimes3xbfgragh.onion/privacy/cookie-policy\#how-do-i-manage-trackers}{Your
  Ad Choices}
\item
  \href{https://www.nytimes3xbfgragh.onion/privacy}{Privacy}
\item
  \href{https://help.nytimes3xbfgragh.onion/hc/en-us/articles/115014893428-Terms-of-service}{Terms
  of Service}
\item
  \href{https://help.nytimes3xbfgragh.onion/hc/en-us/articles/115014893968-Terms-of-sale}{Terms
  of Sale}
\item
  \href{https://spiderbites.nytimes3xbfgragh.onion}{Site Map}
\item
  \href{https://help.nytimes3xbfgragh.onion/hc/en-us}{Help}
\item
  \href{https://www.nytimes3xbfgragh.onion/subscription?campaignId=37WXW}{Subscriptions}
\end{itemize}
