Sections

SEARCH

\protect\hyperlink{site-content}{Skip to
content}\protect\hyperlink{site-index}{Skip to site index}

\href{https://myaccount.nytimes3xbfgragh.onion/auth/login?response_type=cookie\&client_id=vi}{}

\href{https://www.nytimes3xbfgragh.onion/section/todayspaper}{Today's
Paper}

\hypertarget{the-coronavirus-outbreak}{%
\subsubsection{\texorpdfstring{\href{https://www.nytimes3xbfgragh.onion/news-event/coronavirus?name=styln-coronavirus-markets\&region=TOP_BANNER\&variant=undefined\&block=storyline_menu_recirc\&action=click\&pgtype=LegacyCollection\&impression_id=f8de16e0-e398-11ea-86e2-89c09cd5c287}{The
Coronavirus
Outbreak}}{The Coronavirus Outbreak}}\label{the-coronavirus-outbreak}}

\begin{itemize}
\tightlist
\item
  live\href{https://www.nytimes3xbfgragh.onion/2020/08/21/world/covid-19-coronavirus.html?name=styln-coronavirus-markets\&region=TOP_BANNER\&variant=undefined\&block=storyline_menu_recirc\&action=click\&pgtype=LegacyCollection\&impression_id=f8de3df0-e398-11ea-86e2-89c09cd5c287}{Latest
  Updates}
\item
  \href{https://www.nytimes3xbfgragh.onion/interactive/2020/us/coronavirus-us-cases.html?name=styln-coronavirus-markets\&region=TOP_BANNER\&variant=undefined\&block=storyline_menu_recirc\&action=click\&pgtype=LegacyCollection\&impression_id=f8de3df1-e398-11ea-86e2-89c09cd5c287}{Maps
  and Cases}
\item
  \href{https://www.nytimes3xbfgragh.onion/interactive/2020/science/coronavirus-vaccine-tracker.html?name=styln-coronavirus-markets\&region=TOP_BANNER\&variant=undefined\&block=storyline_menu_recirc\&action=click\&pgtype=LegacyCollection\&impression_id=f8de3df2-e398-11ea-86e2-89c09cd5c287}{Vaccine
  Tracker}
\item
  \href{https://www.nytimes3xbfgragh.onion/2020/08/19/us/colleges-closing-covid.html?name=styln-coronavirus-markets\&region=TOP_BANNER\&variant=undefined\&block=storyline_menu_recirc\&action=click\&pgtype=LegacyCollection\&impression_id=f8de3df3-e398-11ea-86e2-89c09cd5c287}{Colleges
  Closing}
\item
  \href{https://www.nytimes3xbfgragh.onion/live/2020/08/20/business/stock-market-today-coronavirus?name=styln-coronavirus-markets\&region=TOP_BANNER\&variant=undefined\&block=storyline_menu_recirc\&action=click\&pgtype=LegacyCollection\&impression_id=f8de3df4-e398-11ea-86e2-89c09cd5c287}{Economy}
\end{itemize}

\hypertarget{jobless-claims-unexpectedly-jump-to-11-million}{%
\section{Jobless Claims Unexpectedly Jump to 1.1
Million}\label{jobless-claims-unexpectedly-jump-to-11-million}}

This briefing is no longer being updated. Follow live updates
\href{https://www.nytimes3xbfgragh.onion/2020/08/20/world/coronavirus-covid.html}{here}.

Last Updated

Aug. 21, 2020, 6:13 a.m. ET

Aug. 21, 2020, 6:13 a.m. ET

\hypertarget{heres-what-you-need-to-know}{%
\subsubsection{Here's what you need to
know:}\label{heres-what-you-need-to-know}}

\begin{itemize}
\item
  \protect\hyperlink{american-airlines-to-stop-flights-to-15-cities-after-government-aid-ends}{}

  American Airlines to stop flights to 15 cities after government aid
  ends.
\item
  \protect\hyperlink{1-1-million-filed-new-claims-for-state-unemployment-benefits-last-week}{}

  1.1 million filed new claims for state unemployment benefits last
  week.
\item
  \protect\hyperlink{the-latest-delta-to-block-middle-seats-until-next-year-airbnb-bans-parties}{}

  The latest: Delta to block middle seats until next year, Airbnb bans
  parties.
\item
  \protect\hyperlink{the-economic-outlook-remains-tenuous-with-congress-holding-a-key}{}

  The economic outlook remains tenuous, with Congress holding a key.
\item
  \protect\hyperlink{11-states-have-signed-on-to-the-white-house-plan-for-unemployment-pay}{}

  11 states have signed on to the White House plan for unemployment pay.
\end{itemize}

\hypertarget{american-airlines-to-stop-flights-to-15-cities-after-government-aid-ends}{%
\subsection{\texorpdfstring{\protect\hyperlink{american-airlines-to-stop-flights-to-15-cities-after-government-aid-ends}{American
Airlines to stop flights to 15 cities after government aid
ends.}}{American Airlines to stop flights to 15 cities after government aid ends.}}\label{american-airlines-to-stop-flights-to-15-cities-after-government-aid-ends}}

Copied to clipboard.

\includegraphics{https://static01.graylady3jvrrxbe.onion/images/2020/08/20/business/20-markets-brf-AmericanAirlines/merlin_174608667_6b4a0035-c8d2-4336-8e52-408180b5b0d7-articleLarge.jpg?quality=75\&auto=webp\&disable=upscale}

\textbf{American Airlines} said it would drop flights to 15 small and
medium-sized cities in October because of low demand and the end of a
requirement from a March stimulus law that airlines maintain minimum
service to destinations they flew to before the pandemic.

Starting Oct. 7, the airline will stop flying to cities like New Haven,
Conn.; Dubuque, Iowa; Joplin, Mo.; and Kalamazoo and Battle Creek, Mich.
The suspension will continue through at least Nov. 3 and the airline
said it could revise its plans if Congress extends
\href{https://www.nytimes3xbfgragh.onion/2020/04/14/business/coronavirus-airlines-bailout-treasury-department.html}{the
\$25 billion it gave the industry to forestall layoffs}.

That funding, which expires on Sept. 30, has preserved tens of thousands
of jobs and provided airlines flexibility by taking care of payroll
expenses. A union-led effort to secure funds for another six months had
gained
\href{https://www.nytimes3xbfgragh.onion/live/2020/08/05/business/stock-market-today-coronavirus/republican-senators-back-an-extension-of-support-for-airlines}{bipartisan
support} in recent weeks before stimulus negotiations between
congressional leaders and the Trump administration stalled. American
received \$5.8 billion under the payroll-support program, more than any
other airline.

The March law, the CARES Act, also gave Transportation Secretary Elaine
Chao the authority to require airlines that received the aid to maintain
some portion of the flight schedules they operated before the pandemic
decimated air travel.

Demand started to bounce back in May and June before stalling in July.
While the travel has picked up again this month, only about 30 percent
as many people are currently flying compared with this time last year.

Other cities that will lose service by American include Huntington,
W.Va.; Roswell, N.M.; Sioux City, Iowa; and Greenville, N.C. American
said it planned to post its full October schedule next Saturday.

--- \href{https://www.nytimes3xbfgragh.onion/by/niraj-chokshi}{Niraj
Chokshi}

\hypertarget{the-producer-of-unhinged-makes-a-big-bet-on-audiences-returning-to-theaters}{%
\subsection{\texorpdfstring{\protect\hyperlink{the-producer-of-unhinged-makes-a-big-bet-on-audiences-returning-to-theaters.html}{The
producer of `Unhinged' makes a big bet on audiences returning to
theaters.}}{The producer of `Unhinged' makes a big bet on audiences returning to theaters.}}\label{the-producer-of-unhinged-makes-a-big-bet-on-audiences-returning-to-theaters}}

Copied to clipboard.

\includegraphics{https://static01.graylady3jvrrxbe.onion/images/2020/08/21/business/20Virus-Unhinged1-print/merlin_175534284_d95546d6-4e4a-4742-acde-02bd0c569b7d-articleLarge.jpg?quality=75\&auto=webp\&disable=upscale}

The pandemic, which closed theaters around the world,
\href{https://www.nytimes3xbfgragh.onion/2020/06/12/business/media/tenet-release-delayed.html}{wreaked
havoc} on the premiere dates of movies like ``Tenet'' and ``Wonder Woman
1984.''

But unlike dozens of other movies,
\href{https://www.nytimes3xbfgragh.onion/2020/08/20/movies/unhinged-review.html}{``Unhinged,''}
a gritty action film starring
\href{https://www.nytimes3xbfgragh.onion/2019/06/27/arts/television/loudest-voice-roger-ailes-russell-crowe.html}{Russell
Crowe}, will indeed have a theatrical release in the United States this
summer, signaling a return to business for Hollywood.

The original plan was for a July 1 premiere. The date was pushed back
twice more before the film was scheduled to appear on Friday in more
than 1,800 theaters in the United States and Canada, a release that came
about largely because of the stubbornness of Mark Gill, an independent
film producer who was hellbent on getting it to the big screen.

``Unhinged'' will be the largest new offering in North American theaters
this weekend, when 26 percent of theaters in the United States and
Canada are scheduled to be open, according to the National Association
of Theatre Owners, a trade association.

While other Hollywood executives decided they had no choice but to
reschedule, Mr. Gill made a priority of beating the competition to
theaters. ``There is no question that the `first-mover' advantage has
been a big deal,'' he said.

He believes in the film --- a throwback to the thrillers of decades ago
that sent Charles Bronson into scenery-trampling mode --- but said he
could not predict how many moviegoers would show up this weekend, when
theaters across the country have heightened safety restrictions.

``It's all bananas,'' Mr. Gill said. ``That's the crazy thing. I have no
idea.''

--- \href{https://www.nytimes3xbfgragh.onion/by/nicole-sperling}{Nicole
Sperling}

\hypertarget{advertisement}{%
\subsubsection{Advertisement}\label{advertisement}}

\protect\hyperlink{after-dfp-ad-mid1}{Continue reading the main story}

\hypertarget{11-million-filed-new-claims-for-state-unemployment-benefits-last-week}{%
\subsection{\texorpdfstring{\protect\hyperlink{1-1-million-filed-new-claims-for-state-unemployment-benefits-last-week}{1.1
million filed new claims for state unemployment benefits last
week.}}{1.1 million filed new claims for state unemployment benefits last week.}}\label{11-million-filed-new-claims-for-state-unemployment-benefits-last-week}}

Copied to clipboard.

Initial weekly unemployment claims,

both regular and those under the Pandemic Unemployment Assistance
program

6 million

1.1 million regular claims last week after falling below 1 million the
week before

5

4

3

2

1

0

Feb.

March

April

May

June

July

Aug.

Initial weekly unemployment claims, both regular and those under the
Pandemic Unemployment Assistance program

6 million

5

1.1 million regular claims last week after falling below 1 million the
week before

4

3

2

1

0

Feb.

March

April

May

June

July

Aug.

Pandemic Unemployment Assistance extends eligibility to some workers who
would not otherwise be able to apply for unemployment benefits, such as
part-time and self-employed workers. Regular claims are seasonally
adjusted but P.U.A. claims are not.

Source: Labor Department

By Ella Koeze

The number of Americans filing for unemployment insurance unexpectedly
rose last week, a sign of the job market's fragility five months after
the coronavirus pandemic began to devastate the economy.

The economy remains challenging for many American workers, with the
unemployment rate at 10.2 percent and sectors like leisure and
hospitality experiencing huge losses in employment.

Last week, 1.1 million workers filed new claims for state unemployment
benefits, the \href{https://www.dol.gov/ui/data.pdf}{Labor Department
reported Thursday}, compared with 971,000 the previous week.

``It definitely suggests that momentum in the recovery is slowing,''
said Scott Anderson, chief economist at Bank of the West.

``We're still at a historically high level of claims,'' he added, noting
that the weekly peak during the Great Recession stood at 665,000. ``The
labor market is in the I.C.U.''

There were 543,000 new claims last week for Pandemic Unemployment
Assistance, a separate program aimed at self-employed people, gig
workers and others not covered by traditional unemployment benefits.
That number, unlike the figures for state claims, is not seasonally
adjusted.

Despite the weak number for jobless claims, Gus Faucher, chief economist
at the PNC Financial Services Group, pointed to pockets of strength in
the economy.

``We see continued improvement with housing starts increasing, consumer
spending increasing, and industrial production increasing,'' he said.
``But the pace of improvement is slowing.''

Mr. Faucher noted that the data for initial jobless claims in Thursday's
report was affected by a large seasonal adjustment linked to the
beginning of the school year. Without that, new claims would have
totaled 891,510.

---
\href{https://www.nytimes3xbfgragh.onion/by/nelson-d-schwartz}{Nelson D.
Schwartz}

\hypertarget{the-latest-delta-to-block-middle-seats-until-next-year-airbnb-bans-parties}{%
\subsection{\texorpdfstring{\protect\hyperlink{the-latest-delta-to-block-middle-seats-until-next-year-airbnb-bans-parties}{The
latest: Delta to block middle seats until next year, Airbnb bans
parties.}}{The latest: Delta to block middle seats until next year, Airbnb bans parties.}}\label{the-latest-delta-to-block-middle-seats-until-next-year-airbnb-bans-parties}}

Copied to clipboard.

\begin{itemize}
\item
  \textbf{Delta Air Lines} said it would continue to block middle seats
  through Jan. 6, the first major airline to commit to such limits over
  the busy holiday season. At the same time, Delta plans to raise
  seating limits in its main cabins to 75 percent, from 60 percent,
  starting in October.
\item
  \textbf{Airbnb} on Thursday
  \href{https://news.airbnb.com/airbnb-announces-global-party-ban/}{announced}
  a ban on parties and events at its listings worldwide, and an
  occupancy cap of 16. The company said in a statement that the ban was
  in the interest of public health in the midst of the pandemic. ``Some
  have chosen to take bar and club behavior to homes, sometimes rented
  through our platform. We think such conduct is incredibly
  irresponsible --- we do not want that type of business, and anyone
  engaged in or allowing that behavior does not belong on our
  platform,'' the company said.
\item
  \textbf{Qantas Airways}, Australia's national carrier, swung to a loss
  of 1.96 billion Australian dollars, or \$1.4 billion, for the 2020
  financial year, as the pandemic devastated the airline industry
  worldwide. The loss was driven primarily by a write-down of assets,
  including its A380 fleet, and restructuring costs intended to help it
  survive during the pandemic, the airline said in a statement on
  Thursday. ``The impact of Covid on all airlines is clear,'' the
  company's chief executive, Alan Joyce, said in the statement. ``It's
  devastating and it will be a question of survival for many.''
\item
  \textbf{Estée Lauder Companies} said Thursday that it would cut about
  1,500 to 2,000 jobs globally and close 10 to 15 percent of its stores.
  It reported that in the fourth quarter net sales fell to \$2.43
  billion from \$3.59 billion a year ago. The company operates cosmetic
  and skin care brands like MAC Cosmetics, La Mer and Jo Malone London.
\end{itemize}

\hypertarget{the-economic-outlook-remains-tenuous-with-congress-holding-a-key}{%
\subsection{\texorpdfstring{\protect\hyperlink{the-economic-outlook-remains-tenuous-with-congress-holding-a-key}{The
economic outlook remains tenuous, with Congress holding a
key.}}{The economic outlook remains tenuous, with Congress holding a key.}}\label{the-economic-outlook-remains-tenuous-with-congress-holding-a-key}}

Copied to clipboard.

\includegraphics{https://static01.graylady3jvrrxbe.onion/images/2020/08/20/business/20markets-brf-outlook/merlin_174846558_24df5698-fbff-419e-a070-3f44e473f26a-articleLarge.jpg?quality=75\&auto=webp\&disable=upscale}

Even as nearly one million Americans file new state claims for
unemployment benefits each week, the
\href{https://www.nytimes3xbfgragh.onion/2020/08/18/business/stock-market-record.html}{stock
market is hitting record highs}. Hotels and airports are nearly empty
and many restaurants remain closed, but
\href{https://www.reuters.com/article/us-usa-economy-housingstarts/us-housing-starts-surge-in-july-in-rare-pandemic-bright-spot-idUSKCN25E21O}{home
building is booming} and
\href{https://www.nytimes3xbfgragh.onion/2020/08/14/business/retail-sales-coronavirus.html}{retail
sales are back} to levels that preceded the pandemic.

The crosscurrents in the economy are striking, but economists warn that
conditions could easily deteriorate if Washington doesn't offer more
support.

Republicans and Democrats have been unable to agree on a new coronavirus
relief package to augment the CARES Act, passed in March. A \$600 weekly
federal supplement to state unemployment insurance expired at the end of
last month, and a
\href{https://www.nytimes3xbfgragh.onion/2020/08/13/business/economy/unemployment-benefits-coronavirus.html}{\$300-a-week
replacement} engineered by President Trump is having trouble getting off
the ground.

``Federal support is crucial to underpinning the virtuous cycle we've
had,'' said Michael Gapen, chief U.S. economist at Barclays. ``The
longer negotiations stall, the more likely there will be a hiccup in
spending.''

``As long as we get a package in September, the outlook is OK,'' Mr.
Gapen said. But Republicans and Democrats remain far apart, even as the
economy hangs in the balance.

The power of the government's efforts to help are apparent in the data,
Mr. Gapen said. Personal income in June totaled \$19.9 trillion at an
annual rate, up from \$19.1 trillion in February, before the pandemic.
``With unemployment at 10 percent, that's only happening with federal
transfer payments,'' Mr. Gapen said.

---
\href{https://www.nytimes3xbfgragh.onion/by/nelson-d-schwartz}{Nelson D.
Schwartz}

\hypertarget{advertisement-1}{%
\subsubsection{Advertisement}\label{advertisement-1}}

\protect\hyperlink{after-dfp-ad-mid2}{Continue reading the main story}

\hypertarget{finding-a-job-after-a-long-search-but-settling-for-less-pay}{%
\subsection{\texorpdfstring{\protect\hyperlink{finding-a-job-after-a-long-search-but-settling-for-less-pay}{Finding
a job after a long search, but settling for less
pay.}}{Finding a job after a long search, but settling for less pay.}}\label{finding-a-job-after-a-long-search-but-settling-for-less-pay}}

Copied to clipboard.

\includegraphics{https://static01.graylady3jvrrxbe.onion/images/2020/08/20/business/20markets-brf-vance/merlin_175914816_b7e78dd7-c736-404c-8f21-594972044b00-articleLarge.jpg?quality=75\&auto=webp\&disable=upscale}

After spending up to six hours a day submitting more than 600
applications since being furloughed this spring and then laid off in
late July, Sonia Vance, 42, finally landed a new job.

In a few weeks, she starts as an eyewear consultant in California, Md.,
earning \$16 an hour. It pays far less than the dream job she had before
--- a \$48,000-a-year human resources role at a staffing company --- but
it comes with health insurance.

The cushion is comforting, because Ms. Vance must now go to work each
day in an office, despite health issues that she fears could complicate
a recovery if she catches the coronavirus.

Reflecting the experience of millions who scrambled for work after their
careers evaporated in the pandemic, Ms. Vance said the past few months
had been ``heartbreaking and very emotional.''

This week, she moved from Maryville, Tenn., and will stay temporarily
with a friend. She is finishing up bankruptcy paperwork and expects to
lose her mobile home.

``You do feel relief that you have a job, but there's also a sense of
shame and embarrassment,'' she said. ``You're out there doing everything
you can to be a good member of society and to take care of your own, but
it just takes a few months to wipe out all of your hard work.''

--- \href{https://www.nytimes3xbfgragh.onion/by/tiffany-hsu}{Tiffany
Hsu}

\hypertarget{without-school-plays-and-assemblies-a-technicians-livelihood-withers}{%
\subsection{\texorpdfstring{\protect\hyperlink{without-school-plays-and-assemblies-a-technicians-livelihood-withers}{Without
school plays and assemblies, a technician's livelihood
withers.}}{Without school plays and assemblies, a technician's livelihood withers.}}\label{without-school-plays-and-assemblies-a-technicians-livelihood-withers}}

Copied to clipboard.

This should be a time of keen anticipation for David Leske. A lighting
and sound technician in Ridgway, Pa., he works in local schools to make
plays, assemblies and other shows come to life.

But a few weeks before the school year is to begin, the coronavirus
pandemic is still preventing large indoor gatherings. In some cases,
schools are sticking to online instruction.

``Our local district has no intention of doing school plays,'' Mr. Leske
said. ``The high school auditorium is now a storage area.''

Mr. Leske, 52, said that work began to dry up in March and that the
Pandemic Unemployment Assistance program --- which covers independent
contractors, self-employed workers and others who don't qualify for
regular state benefits --- had been crucial to keeping him afloat,
especially with the \$600 weekly federal supplement that expired at the
end of last month.

He expects to be out of work through September 2021 as schools hold off
on plays and assemblies.

``That extra \$600 is what's been keeping us alive,'' Mr. Leske said.
Without it, he and his wife have been forced to tap their savings.
``It's scary,'' he said.

---
\href{https://www.nytimes3xbfgragh.onion/by/nelson-d-schwartz}{Nelson D.
Schwartz}

\hypertarget{11-states-have-signed-on-to-the-white-house-plan-for-unemployment-pay}{%
\subsection{\texorpdfstring{\protect\hyperlink{11-states-have-signed-on-to-the-white-house-plan-for-unemployment-pay}{11
states have signed on to the White House plan for unemployment
pay.}}{11 states have signed on to the White House plan for unemployment pay.}}\label{11-states-have-signed-on-to-the-white-house-plan-for-unemployment-pay}}

Copied to clipboard.

\includegraphics{https://static01.graylady3jvrrxbe.onion/images/2020/08/20/business/20markets-brf-fema1/merlin_175492773_a796ab95-9fc0-47a0-8f71-61982a863156-articleLarge.jpg?quality=75\&auto=webp\&disable=upscale}

Eleven states have been approved for a
\href{https://www.nytimes3xbfgragh.onion/2020/08/13/business/economy/unemployment-benefits-coronavirus.html}{Federal
Emergency Management Agency program}that will give unemployed workers an
additional \$300 in weekly benefits, but many others are holding off on
applying as they try to understand what the program entails.

Acting on an
\href{https://www.whitehouse.gov/presidential-actions/memorandum-authorizing-needs-assistance-program-major-disaster-declarations-related-coronavirus-disease-2019/}{Aug.
8 memorandum from President Trump}, the federal agency
\href{https://www.fema.gov/press-release/20200819/fema-announces-lost-wages-grant-maryland}{said
it had approved} Arizona, Colorado, Idaho, Iowa, Louisiana, Maryland,
Missouri, Montana, New Mexico, Oklahoma and Utah for access to three
weeks of funds.

Mr. Trump's executive action caps spending on the program at \$44
billion. Officials from FEMA and the Labor Department said on a
conference call with reporters on Thursday that FEMA had approved \$2.4
billion in grants so far and that an additional eight states had applied
for funds.

The officials said the \$44 billion should be enough to cover an
estimated four to five weeks of benefits. They said that most states
were expected to take part and that none had been rejected.

Arizona was the first state to make the so-called lost wages payments,
sending \$96 million to 320,000 people on Monday and Tuesday. But the
timeline for payments ``will be all over the map,'' potentially taking
several weeks, depending on how long states need to reprogram their
systems, said John Pallasch, the assistant secretary for employment and
training at the Labor Department.

Among the changes that must be factored in: The new program offers a
smaller benefit than the weekly \$600 federal supplement that expired
\href{https://www.nytimes3xbfgragh.onion/2020/08/08/business/economy/lost-unemployment-benefits.html?action=click\&module=RelatedLinks\&pgtype=Article}{at
the end of July}. And only people who already qualify to receive at
least \$100 in unemployment benefits each week are eligible for the
federal funds, which are retroactive to Aug. 1.

The program gives states the option to contribute \$100 in additional
funds, increasing the weekly benefit to \$400. But some states are
nervous about chipping in, citing
\href{https://www.nytimes3xbfgragh.onion/2020/08/10/us/politics/virus-stimulus-congress-trump.html?action=click\&module=RelatedLinks\&pgtype=Article}{severe
budget shortfalls}.

States have until Sept. 10 to apply for the funds.

--- \href{https://www.nytimes3xbfgragh.onion/by/tiffany-hsu}{Tiffany
Hsu}

\hypertarget{advertisement-2}{%
\subsubsection{Advertisement}\label{advertisement-2}}

\protect\hyperlink{after-dfp-ad-mid3}{Continue reading the main story}

\hypertarget{us-stocks-drift-as-jobless-claims-increase}{%
\subsection{\texorpdfstring{\protect\hyperlink{us-stocks-drift-as-jobless-claims-increase}{U.S.
stocks drift as jobless claims
increase.}}{U.S. stocks drift as jobless claims increase.}}\label{us-stocks-drift-as-jobless-claims-increase}}

Copied to clipboard.

\begin{itemize}
\item
  The S\&P 500 \textbf{climbed on Thursday}, lifted by more gains in
  technology stocks.
\item
  The Labor Department's report on the health of the job market served
  as \textbf{a reminder of the fragility of the economic recovery}: The
  tally of new claims for state unemployment benefits unexpectedly rose
  to over 1.1 million.
\item
  Investors were initially cautious, after
  \href{https://www.nytimes3xbfgragh.onion/2020/08/19/business/economy/fed-meeting-minutes-coronavirus.html}{minutes
  from the Federal Reserve'}s last meeting revealed \textbf{concerns
  that the U.S. economy needed more financial support} from Congress.
\item
  The Fed minutes \textbf{rattled Wall Street}, which has been soaring
  lately. But
  \textbf{\href{https://www.nytimes3xbfgragh.onion/2020/08/19/technology/apple-2-trillion.html}{Apple}}
  has reached \textbf{a valuation of \$2 trillion}, another milestone
  for the maker of iPhones, Mac computers and Apple Watches, punctuating
  how
  \href{https://www.nytimes3xbfgragh.onion/2020/03/23/technology/coronavirus-facebook-amazon-youtube.html}{the
  pandemic has been a bonanza for the tech giants}.
\item
  On Thursday, the major European indexes were \textbf{all down more
  than 1 percent}. Asian indexes broadly lost ground.
\end{itemize}

--- \href{https://www.nytimes3xbfgragh.onion/by/kevin-granville}{Kevin
Granville}

\hypertarget{with-government-help-companies-save-jobs-by-sharing-them}{%
\subsection{\texorpdfstring{\protect\hyperlink{with-government-help-companies-save-jobs-by-sharing-them}{With
government help, companies save jobs by sharing
them.}}{With government help, companies save jobs by sharing them.}}\label{with-government-help-companies-save-jobs-by-sharing-them}}

Copied to clipboard.

\includegraphics{https://static01.graylady3jvrrxbe.onion/images/2020/08/17/business/00workshare1/merlin_175429827_0658653d-4b1c-4eff-98e2-cb14348470e8-articleLarge.jpg?quality=75\&auto=webp\&disable=upscale}

Work sharing programs are extraordinarily popular among economists,
Republican and Democratic policymakers, employers and workers --- at
least those who have heard of them. The problem is that few have, even
though economists say work sharing is one of the best ways to strengthen
the labor market during a downturn.

Of the roughly 30 million people receiving unemployment benefits,
\href{https://www.dol.gov/ui/data.pdf}{only 451,000} --- just 1.5
percent --- are getting them through a shared work program.

Congress sweetened the program's appeal during the pandemic, promising
as part of the CARES Act that the federal government would
\href{https://wdr.doleta.gov/directives/attach/UIPL/UIPL_21-20.pdf}{pick
up the cost} from the states through the end of the year, without an
overall cap, but nearly half of all states still don't have such a
program.

``I'm sick of this being the `best kept secret,''' Suzan LeVine,
commissioner of Washington's Employment Security Department, said of the
program, officially titled short-time compensation. ``It is the diamond
in the rough of the unemployment benefits system.''

Washington State, which started its program in 1983, has vastly expanded
participation since the pandemic. From March to August last year, 688
businesses took part; now 3,560 are doing so. One in nine Washington
workers receiving state jobless benefits is getting them through work
sharing.

Ted Brown Music is one business taking part. Although program rules can
vary by state, companies must apply individually, and file a separate
plan for each unit or category of workers. Ted Brown Music was approved
within two weeks.

Now 150 of its employees are taking part. They are paid an hourly wage
for the time they work, and receive state unemployment benefits for the
hours they don't.

Jim Stevens, who joined the company in 1970 and knew its founder, was
laid off for six weeks after the pandemic hit. ``That was just
terrible,'' said Mr. Stevens, a salesman in the flagship Tacoma store
that his wife, Ellie, manages. ``I've never been unemployed for any
major period in my life.'' He was later brought back to work 28 hours a
week under the work sharing program.

--- \href{https://www.nytimes3xbfgragh.onion/by/patricia-cohen}{Patricia
Cohen}

\hypertarget{site-index}{%
\subsection{Site Index}\label{site-index}}

\hypertarget{site-information-navigation}{%
\subsection{Site Information
Navigation}\label{site-information-navigation}}

\begin{itemize}
\tightlist
\item
  \href{https://help.nytimes3xbfgragh.onion/hc/en-us/articles/115014792127-Copyright-notice}{©~2020~The
  New York Times Company}
\end{itemize}

\begin{itemize}
\tightlist
\item
  \href{https://www.nytco.com/}{NYTCo}
\item
  \href{https://help.nytimes3xbfgragh.onion/hc/en-us/articles/115015385887-Contact-Us}{Contact
  Us}
\item
  \href{https://www.nytco.com/careers/}{Work with us}
\item
  \href{https://nytmediakit.com/}{Advertise}
\item
  \href{http://www.tbrandstudio.com/}{T Brand Studio}
\item
  \href{https://www.nytimes3xbfgragh.onion/privacy/cookie-policy\#how-do-i-manage-trackers}{Your
  Ad Choices}
\item
  \href{https://www.nytimes3xbfgragh.onion/privacy}{Privacy}
\item
  \href{https://help.nytimes3xbfgragh.onion/hc/en-us/articles/115014893428-Terms-of-service}{Terms
  of Service}
\item
  \href{https://help.nytimes3xbfgragh.onion/hc/en-us/articles/115014893968-Terms-of-sale}{Terms
  of Sale}
\item
  \href{https://spiderbites.nytimes3xbfgragh.onion}{Site Map}
\item
  \href{https://help.nytimes3xbfgragh.onion/hc/en-us}{Help}
\item
  \href{https://www.nytimes3xbfgragh.onion/subscription?campaignId=37WXW}{Subscriptions}
\end{itemize}
