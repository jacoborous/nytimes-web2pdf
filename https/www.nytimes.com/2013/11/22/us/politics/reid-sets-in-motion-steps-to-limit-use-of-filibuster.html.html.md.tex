Sections

SEARCH

\protect\hyperlink{site-content}{Skip to
content}\protect\hyperlink{site-index}{Skip to site index}

\href{https://www.nytimes.com/section/politics}{Politics}

\href{https://myaccount.nytimes.com/auth/login?response_type=cookie\&client_id=vi}{}

\href{https://www.nytimes.com/section/todayspaper}{Today's Paper}

\href{/section/politics}{Politics}\textbar{}In Landmark Vote, Senate
Limits Use of the Filibuster

\url{https://nyti.ms/17Qt6DG}

\begin{itemize}
\item
\item
\item
\item
\item
\item
\end{itemize}

Advertisement

\protect\hyperlink{after-top}{Continue reading the main story}

Supported by

\protect\hyperlink{after-sponsor}{Continue reading the main story}

\hypertarget{in-landmark-vote-senate-limits-use-of-the-filibuster}{%
\section{In Landmark Vote, Senate Limits Use of the
Filibuster}\label{in-landmark-vote-senate-limits-use-of-the-filibuster}}

\includegraphics{https://static01.nyt.com/images/2013/11/22/us/politics/22judges/22judges-articleLarge.jpg?quality=75\&auto=webp\&disable=upscale}

By \href{https://www.nytimes.com/by/jeremy-w-peters}{Jeremy W. Peters}

\begin{itemize}
\item
  Nov. 21, 2013
\item
  \begin{itemize}
  \item
  \item
  \item
  \item
  \item
  \item
  \end{itemize}
\end{itemize}

WASHINGTON --- The Senate approved the most fundamental alteration of
its rules in more than a generation on Thursday, ending the minority
party's ability to filibuster most presidential nominees in response to
the partisan gridlock that has plagued Congress for much of the Obama
administration.

Furious Republicans accused Democrats of a power grab, warning them that
they would deeply regret their action if they lost control of the Senate
next year and the White House in years to come. Invoking the Founding
Fathers and the meaning of the Constitution, Republicans said Democrats
were trampling the minority rights the framers intended to protect. But
when the vote was called, Senator Harry Reid, the majority leader who
was initially reluctant to force the issue, prevailed 52 to 48.

Under the change, the Senate will be able to cut off debate on executive
and judicial branch nominees with a simple majority rather than rounding
up a supermajority of 60 votes. The new precedent established by the
Senate on Thursday does not apply to Supreme Court nominations or
legislation itself.

It represented the culmination of years of frustration over what
Democrats denounced as a Republican campaign to stall the machinery of
Congress, stymie President Obama's agenda and block his choices for
cabinet posts and federal judgeships by insisting that virtually
everything the Senate approves be done by a supermajority.

After repeatedly threatening to change the rules, Mr. Reid decided to
follow through when Republicans refused this week to back down from
their effort to keep Mr. Obama from filling any of three vacancies on
the most powerful appeals court in the country.

\includegraphics{https://static01.nyt.com/images/2013/11/22/us/politics/22judges2/22judges2-articleLarge.jpg?quality=75\&auto=webp\&disable=upscale}

This was the final straw for some Democratic holdouts against limiting
the filibuster, providing Mr. Reid with the votes he needed to impose a
new standard certain to reverberate through the Senate for years.

``There has been unbelievable, unprecedented obstruction,'' Mr. Reid
said as he set in motion the steps for the vote on Thursday. ``The
Senate is a living thing, and to survive it must change as it has over
the history of this great country. To the average American, adapting the
rules to make the Senate work again is just common sense.''

Republicans accused Democrats of irreparably damaging the character of
an institution that in many ways still operates as it did in the 19th
century, and of disregarding the constitutional prerogative of the
Senate as a body of ``advice and consent'' on presidential nominations.

``You think this is in the best interest of the United States Senate and
the American people?'' asked the Republican leader, Senator Mitch
McConnell, sounding incredulous.

``I say to my friends on the other side of the aisle, you'll regret
this. And you may regret it a lot sooner than you think,'' he added.

Mr. Obama applauded the Senate's move. ``Today's pattern of obstruction,
it just isn't normal,'' he told reporters at the White House. ``It's not
what our founders envisioned. A deliberate and determined effort to
obstruct everything, no matter what the merits, just to refight the
results of an election is not normal, and for the sake of future
generations we can't let it become normal.''

Image

Filibusters have spiked under the Republican Senate minority in recent
years, with an increasing number being used against high-level
presidential nominees.

Only three Democrats voted against the measure.

The changes will apply to all 1,183 executive branch nominations that
require Senate confirmation --- not just cabinet positions but hundreds
of high- and midlevel federal agency jobs and government board seats.

This fight was a climax to the bitter debate between the parties over
electoral mandates and the consequences of presidential elections.
Republicans, through their frequent use of the various roadblocks that
congressional procedure affords them, have routinely thwarted Democrats.
Democrats, in turn, have accused Republicans of effectively trying to
nullify the results of a presidential election they lost, whether by
trying to dismantle his health care law or keep Mr. Obama from filling
his cabinet.

Republicans saw their battle as fighting an overzealous president who,
left to his own devices, would stack a powerful and underworked court
with judges sympathetic to his vision of big-government liberalism,
pushing its conservative tilt sharply left. The court is of immense
political importance to both parties because it often decides questions
involving White House and federal agency policy.

Republicans proposed eliminating three of its 11 full-time seats. When
Democrats balked, the Republicans refused to confirm any more judges,
saying they were exercising their constitutional check against the
executive.

Senator Pat Roberts, Republican of Kansas, said Democrats had undercut
the minority party's rights forever. ``We have weakened this body
permanently, undermined it for the sake of an incompetent
administration,'' he said. ``What a tragedy.''

With the filibuster rules now rewritten --- the most significant change
since the Senate lowered its threshold to break a filibuster from
two-thirds of the body to three-fifths, or 60 votes, in 1975 --- the
Senate can proceed with approving a backlog of presidential nominations.

\includegraphics{https://static01.nyt.com/images/2013/11/21/us/video-obama-connected-112113/video-obama-connected-112113-videoSmall.jpg}

There are now 59 nominees to executive branch positions and 17 nominees
to the federal judiciary awaiting confirmation votes. The Senate acted
immediately on Thursday when it voted with just 55 senators affirming to
move forward on the nomination of Patricia A. Millett, a Washington
lawyer nominated to the Washington appeals court. Two other nominees to
that court, Cornelia T. L. Pillard and Robert L. Wilkins, are expected
to be confirmed when the Senate returns from its Thanksgiving recess
next month.

The filibuster or threats to use it have frustrated presidents and
majority parties since the early days of the republic. Over the years,
and almost always after the minority had made excessive use of it, the
Senate has adjusted the rules. Until 1917, the year Woodrow Wilson
derided his antiwar antagonists as ``a little group of willful men'' who
had rendered the government helpless through blocking everything in
front of it, there was no rule to end debate. From 1917 to 1975, the bar
for cutting off debate was set at two-thirds of the Senate.

Some would go even further than Thursday's action. Senator Jeff Merkley,
Democrat of Oregon, said that he would like to see the next fight on the
filibuster to be to require senators to actually stand on the floor and
talk if they wanted to stall legislation.

The gravity of the situation was reflected in an unusual scene on the
Senate floor: Nearly all 100 senators were in their seats, rapt, as
their two leaders debated.

As the two men went back and forth, Mr. McConnell appeared to realize
there was no way to persuade Mr. Reid to change his mind. As many
Democrats wore large grins, Republicans looked dour as they lost on a
futile, last-ditch parliamentary attempt by Mr. McConnell to overrule
the majority vote.

When Mr. McConnell left the chamber, he said, ``I think it's a time to
be sad about what's been done to the United States Senate.''

Advertisement

\protect\hyperlink{after-bottom}{Continue reading the main story}

\hypertarget{site-index}{%
\subsection{Site Index}\label{site-index}}

\hypertarget{site-information-navigation}{%
\subsection{Site Information
Navigation}\label{site-information-navigation}}

\begin{itemize}
\tightlist
\item
  \href{https://help.nytimes.com/hc/en-us/articles/115014792127-Copyright-notice}{©~2020~The
  New York Times Company}
\end{itemize}

\begin{itemize}
\tightlist
\item
  \href{https://www.nytco.com/}{NYTCo}
\item
  \href{https://help.nytimes.com/hc/en-us/articles/115015385887-Contact-Us}{Contact
  Us}
\item
  \href{https://www.nytco.com/careers/}{Work with us}
\item
  \href{https://nytmediakit.com/}{Advertise}
\item
  \href{http://www.tbrandstudio.com/}{T Brand Studio}
\item
  \href{https://www.nytimes.com/privacy/cookie-policy\#how-do-i-manage-trackers}{Your
  Ad Choices}
\item
  \href{https://www.nytimes.com/privacy}{Privacy}
\item
  \href{https://help.nytimes.com/hc/en-us/articles/115014893428-Terms-of-service}{Terms
  of Service}
\item
  \href{https://help.nytimes.com/hc/en-us/articles/115014893968-Terms-of-sale}{Terms
  of Sale}
\item
  \href{https://spiderbites.nytimes.com}{Site Map}
\item
  \href{https://help.nytimes.com/hc/en-us}{Help}
\item
  \href{https://www.nytimes.com/subscription?campaignId=37WXW}{Subscriptions}
\end{itemize}
