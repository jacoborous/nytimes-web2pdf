Sections

SEARCH

\protect\hyperlink{site-content}{Skip to
content}\protect\hyperlink{site-index}{Skip to site index}

\href{https://www.nytimes.com/section/arts}{Arts}

\href{https://myaccount.nytimes.com/auth/login?response_type=cookie\&client_id=vi}{}

\href{https://www.nytimes.com/section/todayspaper}{Today's Paper}

\href{/section/arts}{Arts}\textbar{}Albert Broccoli, Film Producer, Dies
at 87

\begin{itemize}
\item
\item
\item
\item
\item
\end{itemize}

Advertisement

\protect\hyperlink{after-top}{Continue reading the main story}

Supported by

\protect\hyperlink{after-sponsor}{Continue reading the main story}

\hypertarget{albert-broccoli-film-producer-dies-at-87}{%
\section{Albert Broccoli, Film Producer, Dies at
87}\label{albert-broccoli-film-producer-dies-at-87}}

By \href{https://www.nytimes.com/by/dinitia-smith}{Dinitia Smith}

\begin{itemize}
\item
  June 29, 1996
\item
  \begin{itemize}
  \item
  \item
  \item
  \item
  \item
  \end{itemize}
\end{itemize}

See the article in its original context from\\
June 29, 1996, Section 1, Page
26\href{https://store.nytimes.com/collections/new-york-times-page-reprints?utm_source=nytimes\&utm_medium=article-page\&utm_campaign=reprints}{Buy
Reprints}

\href{http://timesmachine.nytimes.com/timesmachine/1996/06/29/049654.html}{View
on timesmachine}

TimesMachine is an exclusive benefit for home delivery and digital
subscribers.

About the Archive

This is a digitized version of an article from The Times's print
archive, before the start of online publication in 1996. To preserve
these articles as they originally appeared, The Times does not alter,
edit or update them.

Occasionally the digitization process introduces transcription errors or
other problems; we are continuing to work to improve these archived
versions.

Albert (Cubby) Broccoli, who was a producer of the James Bond films, one
of the most successful movie series of all time, died on Thursday at his
home in Beverly Hills, Calif. He was 87.

In the late 1950's, Mr. Broccoli (pronounced like the vegetable) and his
partner, Harry Saltzman, bought the screen rights to the novels of Ian
Fleming, and proceeded to make Mr. Fleming's character, James Bond,
Agent 007, a household name. The 17 Bond films Mr. Broccoli was
associated with were reported to have earned \$1 billion world wide.

James Bond, played by a succession of actors -\/- Sean Connery, George
Lazenby, Roger Moore, Timothy Dalton and Pierce Brosnan -\/- was the
quintessential cold war hero, a dashing connoisseur of dry martinis (he
liked them shaken, not stirred) and beautiful women, who fought a
succession of monolithic enemies with all the gadgetry available to the
modern industrial age.

He was the father of the modern action hero, the progenitor of
characters later played by Arnold Schwarzenegger and Sylvester Stallone.

Mr. Broccoli could not have been more different from his cinematic
creation. Albert Romolo Broccoli was born in in 1909, the son of
immigrants from Calabria. He was nicknamed Cubby because he was a chubby
child. The family was in the vegetable business, and Mr. Broccoli said
one of his uncles brought the first broccoli seeds into the United
States in the 1870's.

For a while, Mr. Broccoli, too, worked in the vegetable business. Then
in 1933, he became manager of a family coffin business, but he found
that the work depressed him. While visiting a cousin, who was a
Hollywood agent, he met Cary Grant, who became his friend.

Mr. Broccoli realized that he wanted to get into the movie business, and
obtained a job in the mail room at 20th Century Fox. He later worked on
the Howard Hughes's film "The Outlaw." He eventually became an agent and
then, with Irving Allen, began producing films in England.

In the 1950's, when he and Mr. Saltzman tried to get financing for their
first James Bond movie, they were turned down everywhere, according to
Lee Pfeiffer, author of "The Incredible World of 007," because the
character was thought to be too sexually aggressive and too British for
American audiences. Arthur Krim, then head of United Artists, agreed to
give them \$1 million to make the first Bond movie, "Dr. No," in 1962.

Mr. Broccoli and Mr. Saltzman auditioned several actors for the lead.
But when Mr. Broccoli's wife saw a film clip of an unknown actor named
Sean Connery, she is said to have cried: "Take that one! He's gorgeous!"

"Dr. No" made Mr. Connery a star, and he went on to appear in other Bond
films, including "From Russia With Love," "Goldfinger" and
"Thunderball."

In the films, Mr. Broccoli, together with Richard Maibaum, who was a
writer of many Bond movies, transformed an essentially British character
into an international figure.

In 1976, Mr. Broccolli and Mr. Saltzman, who died in 1993, broke up
their partnership, and Mr. Broccoli retained the rights to produce the
series. He went on to make "The Spy Who Loved Me, "For Your Eyes Only,"
"Octopussy" and "License to Kill." The most recent Bond film, last
year's "Goldeneye," with Pierce Brosnan, was produced by his daughter
Barbara Broccoli and his stepson, Michael G. Wilson.

Besides the Bond films, Mr. Broccoli's production credits included "Call
Me Bwana," starring Bob Hope, and "Chitty Chitty Bang Bang," based on a
children's story by Ian Fleming.

In addition to his daughter and stepson, he is survived by his wife,
Dana; another daughter, Tina; a son, Tony, and five grandchildren, all
of Los Angeles.

Advertisement

\protect\hyperlink{after-bottom}{Continue reading the main story}

\hypertarget{site-index}{%
\subsection{Site Index}\label{site-index}}

\hypertarget{site-information-navigation}{%
\subsection{Site Information
Navigation}\label{site-information-navigation}}

\begin{itemize}
\tightlist
\item
  \href{https://help.nytimes.com/hc/en-us/articles/115014792127-Copyright-notice}{©~2020~The
  New York Times Company}
\end{itemize}

\begin{itemize}
\tightlist
\item
  \href{https://www.nytco.com/}{NYTCo}
\item
  \href{https://help.nytimes.com/hc/en-us/articles/115015385887-Contact-Us}{Contact
  Us}
\item
  \href{https://www.nytco.com/careers/}{Work with us}
\item
  \href{https://nytmediakit.com/}{Advertise}
\item
  \href{http://www.tbrandstudio.com/}{T Brand Studio}
\item
  \href{https://www.nytimes.com/privacy/cookie-policy\#how-do-i-manage-trackers}{Your
  Ad Choices}
\item
  \href{https://www.nytimes.com/privacy}{Privacy}
\item
  \href{https://help.nytimes.com/hc/en-us/articles/115014893428-Terms-of-service}{Terms
  of Service}
\item
  \href{https://help.nytimes.com/hc/en-us/articles/115014893968-Terms-of-sale}{Terms
  of Sale}
\item
  \href{https://spiderbites.nytimes.com}{Site Map}
\item
  \href{https://help.nytimes.com/hc/en-us}{Help}
\item
  \href{https://www.nytimes.com/subscription?campaignId=37WXW}{Subscriptions}
\end{itemize}
