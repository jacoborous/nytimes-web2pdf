Sections

SEARCH

\protect\hyperlink{site-content}{Skip to
content}\protect\hyperlink{site-index}{Skip to site index}

\href{https://www.nytimes.com/section/us}{U.S.}

\href{https://myaccount.nytimes.com/auth/login?response_type=cookie\&client_id=vi}{}

\href{https://www.nytimes.com/section/todayspaper}{Today's Paper}

\href{/section/us}{U.S.}\textbar{}Travel From New York City Seeded Wave
of U.S. Outbreaks

\url{https://nyti.ms/3cdCeum}

\begin{itemize}
\item
\item
\item
\item
\item
\item
\end{itemize}

\href{https://www.nytimes.com/news-event/coronavirus?action=click\&pgtype=Article\&state=default\&region=TOP_BANNER\&context=storylines_menu}{The
Coronavirus Outbreak}

\begin{itemize}
\tightlist
\item
  live\href{https://www.nytimes.com/2020/08/01/world/coronavirus-covid-19.html?action=click\&pgtype=Article\&state=default\&region=TOP_BANNER\&context=storylines_menu}{Latest
  Updates}
\item
  \href{https://www.nytimes.com/interactive/2020/us/coronavirus-us-cases.html?action=click\&pgtype=Article\&state=default\&region=TOP_BANNER\&context=storylines_menu}{Maps
  and Cases}
\item
  \href{https://www.nytimes.com/interactive/2020/science/coronavirus-vaccine-tracker.html?action=click\&pgtype=Article\&state=default\&region=TOP_BANNER\&context=storylines_menu}{Vaccine
  Tracker}
\item
  \href{https://www.nytimes.com/interactive/2020/07/29/us/schools-reopening-coronavirus.html?action=click\&pgtype=Article\&state=default\&region=TOP_BANNER\&context=storylines_menu}{What
  School May Look Like}
\item
  \href{https://www.nytimes.com/live/2020/07/31/business/stock-market-today-coronavirus?action=click\&pgtype=Article\&state=default\&region=TOP_BANNER\&context=storylines_menu}{Economy}
\end{itemize}

Advertisement

\protect\hyperlink{after-top}{Continue reading the main story}

Supported by

\protect\hyperlink{after-sponsor}{Continue reading the main story}

\hypertarget{travel-from-new-york-city-seeded-wave-of-us-outbreaks}{%
\section{Travel From New York City Seeded Wave of U.S.
Outbreaks}\label{travel-from-new-york-city-seeded-wave-of-us-outbreaks}}

The coronavirus outbreak in New York City became the primary source of
infections around the United States, researchers have found.

By \href{https://www.nytimes.com/by/benedict-carey}{Benedict Carey} and
\href{https://www.nytimes.com/by/james-glanz}{James Glanz}

\begin{itemize}
\item
  Published May 7, 2020Updated July 16, 2020
\item
  \begin{itemize}
  \item
  \item
  \item
  \item
  \item
  \item
  \end{itemize}
\end{itemize}

New York City's
\href{https://www.nytimes.com/2020/07/16/travel/virus-vacation.html}{coronavirus}
outbreak grew so large by early March that the city became the primary
source of new infections in the United States, new research reveals, as
thousands of infected people traveled from the city and seeded outbreaks
around the country.

The research indicates that a wave of infections swept from
\href{https://www.nytimes.com/2020/07/14/nyregion/coronavirus-ny-travel-cuomo.html}{New
York City} through much of the country before the city began setting
social distancing limits to stop the growth. That helped to fuel
outbreaks in Louisiana,
\href{https://www.nytimes.com/2020/07/14/nyregion/coronavirus-ny-travel-cuomo.html}{Texas},
Arizona and as far away as the West Coast.

The findings are drawn from geneticists' tracking signature mutations of
the virus,
\href{https://www.nytimes.com/2020/07/16/travel/virus-vacation.html}{travel}
histories of infected people and models of the outbreak by infectious
disease experts.

``We now have enough data to feel pretty confident that New York was the
primary gateway for the rest of the country,'' said Nathan Grubaugh, an
epidemiologist at the Yale School of Public Health.

Early analysis of genetic samples indicates that more infections across
the country came from a line of the virus associated with the outbreak
in New York City, shown in red, than from a line associated with the
outbreak in Washington State, shown in yellow.

Percent of genetic samples

related to each area

Washington State

New York City

100\%

50\%

0\%

50\%

100\%

States on the West Coast

Washington

53\%

42\%

California

32\%

50\%

Viruses that spread from Washington early on carry a distinct genetic
signature.

~

But even on the West Coast, samples related to New York are common.

Oregon

30\%

50\%

Alaska

80\%

0\%

Other Western states

Wyoming

31\%

69\%

Texas

4\%

70\%

Arizona

6\%

84\%

Utah

9\%

89\%

Idaho

0\%

98\%

Midwestern states

Illinois

27\%

45\%

Minnesota

15\%

72\%

Wisconsin

4\%

78\%

Ohio

0\%

88\%

Iowa

0\%

100\%

Southern states

Georgia

30\%

0\%

Virginia

11\%

78\%

Every sample from Louisiana, a hot spot for the virus, was related to
New York.

Louisiana

0\%

100\%

Northeastern states

Connecticut

12\%

81\%

New Jersey

7\%

93\%

Maryland

0\%

92\%

The genetic line associated with New York can be traced back to Europe.

New York

1\%

94\%

Massachusetts

0\%

94\%

Early analysis of genetic samples indicates that more infections across
the country came from a line of the virus associated with the outbreak
in New York City, shown in red, than from a line associated with the
outbreak in Washington State, shown in yellow.

Percent of genetic samples related to each area

Washington State

New York City

100\%

50\%

0\%

50\%

100\%

States on the West Coast

Washington

53\%

42\%

California

32\%

50\%

Viruses that spread from Washington early on carry a distinct genetic
signature.

~

But even on the West Coast, samples related to New York are common.

Oregon

30\%

50\%

Alaska

80\%

0\%

Other Western states

Wyoming

31\%

69\%

Texas

4\%

70\%

Arizona

6\%

84\%

Utah

9\%

89\%

Idaho

0\%

98\%

Midwestern states

Illinois

27\%

45\%

Minnesota

15\%

72\%

Wisconsin

4\%

78\%

Ohio

0\%

88\%

Iowa

0\%

100\%

Southern states

Georgia

0\%

30\%

Virginia

11\%

78\%

Every sample from Louisiana, a hot spot for the virus, was related to
New York.

Louisiana

0\%

100\%

Northeastern states

Connecticut

12\%

81\%

New Jersey

7\%

93\%

Maryland

0\%

92\%

The genetic line associated with New York can be traced back to Europe.

New York

1\%

94\%

Massachusetts

0\%

94\%

Early analysis of genetic samples indicates that more infections across
the country came from a line of the virus associated with the outbreak
in New York City, shown in red, than from a line associated with the
outbreak in Washington State, shown in yellow.

Percent of genetic samples

related to each area

Washington

New York City

100\%

50\%

0\%

50\%

100\%

States on the West Coast

Washington

53\%

42\%

California

32\%

50\%

Viruses that spread from Washington early on carry a distinct genetic
signature.

~

But even on the West Coast, samples related to New York are common.

Oregon

30\%

50\%

80\%

Alaska

0\%

Other Western states

Wyoming

31\%

69\%

Texas

4\%

70\%

84\%

Arizona

6\%

89\%

Utah

9\%

98\%

Idaho

0\%

Midwestern states

Illinois

27\%

45\%

Minnesota

15\%

72\%

78\%

Wisconsin

4\%

88\%

Ohio

0\%

100\%

Iowa

0\%

Southern states

Georgia

30\%

0\%

78\%

Virginia

11\%

Every sample from Louisiana, a hot spot for the virus, was related to
New York.

100\%

Louisiana

0\%

Northeastern states

81\%

Connecticut

12\%

93\%

New Jersey

7\%

92\%

Maryland

0\%

The genetic line associated with New York can be traced back to Europe.

94\%

New York

1\%

94\%

Massachusetts

0\%

Note: Scientists have thus far sequenced only a small fraction of total
infections, so the distribution of genetic lines could change as more
samples are analyzed. States with fewer than 10 samples were left off
the chart. Georgia, Illinois, Maryland, Massachusetts, New Jersey,
Oregon and Wyoming all had fewer than 20 samples. Percentages may not
add up to 100 because additional genetic lines are omitted.

Source: Nextstrain

By Derek Watkins

The central role of New York's outbreak shows that decisions made by
state and federal officials --- including waiting to impose distancing
measures and to limit international flights --- helped shape the
trajectory of the outbreak and allowed it to grow in the rest of the
country.

The city joins other densely populated urban hot spots around the world,
starting with Wuhan, China, and then Milan, that have become vectors for
the
\href{https://www.nytimes.com/2020/07/04/health/239-experts-with-one-big-claim-the-coronavirus-is-airborne.html}{virus's
spread}.

Travel from other American cities also sparked infections across the
country, including from an early outbreak centered in the Seattle area
that seeded infections in
\href{https://www.nytimes.com/2020/04/22/us/coronavirus-sequencing.html}{more
than a dozen states}, researchers say. Even if New York had managed to
slow the virus, it probably would have continued to spread from
elsewhere, they say.

\emph{{[}}\href{https://www.nytimes.com/2020/07/10/travel/state-travel-restrictions.html}{\emph{Thinking
of traveling within the US? Here's where you can go}}\emph{.{]}}

But the Seattle outbreak proved to be a squall before the larger storm
gathering in New York, where, at the end of February, thousands of
infected people packed trains and restaurants, thronged tourist
attractions and passed through its three major airports.

During crucial weeks in March, New York's political leaders
\href{https://www.nytimes.com/2020/04/08/nyregion/new-york-coronavirus-response-delays.html}{waited
to take aggressive action}, even after identifying hundreds of cases,
giving the virus a head start. And by mid-March, when President Trump
restricted travel from Europe, the restrictions were essentially
pointless, the data suggest, as the disease was already spreading widely
within the country.

\hypertarget{latest-updates-global-coronavirus-outbreak}{%
\section{\texorpdfstring{\href{https://www.nytimes.com/2020/08/01/world/coronavirus-covid-19.html?action=click\&pgtype=Article\&state=default\&region=MAIN_CONTENT_1\&context=storylines_live_updates}{Latest
Updates: Global Coronavirus
Outbreak}}{Latest Updates: Global Coronavirus Outbreak}}\label{latest-updates-global-coronavirus-outbreak}}

Updated 2020-08-02T07:42:09.613Z

\begin{itemize}
\tightlist
\item
  \href{https://www.nytimes.com/2020/08/01/world/coronavirus-covid-19.html?action=click\&pgtype=Article\&state=default\&region=MAIN_CONTENT_1\&context=storylines_live_updates\#link-34047410}{The
  U.S. reels as July cases more than double the total of any other
  month.}
\item
  \href{https://www.nytimes.com/2020/08/01/world/coronavirus-covid-19.html?action=click\&pgtype=Article\&state=default\&region=MAIN_CONTENT_1\&context=storylines_live_updates\#link-780ec966}{Top
  U.S. officials work to break an impasse over the federal jobless
  benefit.}
\item
  \href{https://www.nytimes.com/2020/08/01/world/coronavirus-covid-19.html?action=click\&pgtype=Article\&state=default\&region=MAIN_CONTENT_1\&context=storylines_live_updates\#link-2bc8948}{Its
  outbreak untamed, Melbourne goes into even greater lockdown.}
\end{itemize}

\href{https://www.nytimes.com/2020/08/01/world/coronavirus-covid-19.html?action=click\&pgtype=Article\&state=default\&region=MAIN_CONTENT_1\&context=storylines_live_updates}{See
more updates}

More live coverage:
\href{https://www.nytimes.com/live/2020/07/31/business/stock-market-today-coronavirus?action=click\&pgtype=Article\&state=default\&region=MAIN_CONTENT_1\&context=storylines_live_updates}{Markets}

Acting earlier would most likely have blunted the virus's march across
the country, researchers say.

``It means that we missed the boat early on, and the vast majority in
this country is coming from domestic spread,'' said Kristian Andersen, a
professor in the department of immunology and microbiology at Scripps
Research. ``I keep hearing that it's somebody else's fault. That's not
true. It's not somebody else's fault, it's our own fault.''

A lack of testing obscured the
\href{https://www.nytimes.com/2020/04/23/us/coronavirus-early-outbreaks-cities.html}{true
extent} of the outbreak for months, and officials acted on incomplete
and sometimes conflicting information. The enormous growth of New York's
outbreak partly reflects its volume of international visitors,
especially from Europe, where
\href{https://www.nytimes.com/2020/04/08/science/new-york-coronavirus-cases-europe-genomes.html}{most
of its infections came from}.

Dani Lever, communications director for Gov. Andrew M. Cuomo, criticized
federal authorities, describing an ``enormous failure by the federal
government to leave New York and the East Coast exposed to flights from
Europe, while at the same time instilling a false sense of security by
telling the State of New York that we had no Covid cases throughout the
entire month of February.''

A White House spokesman, Judd Deere, said that Mr. Trump had acted
quickly. The president blocked most visitors from Europe starting on
March 13, more than a month after he restricted travel from China.

``Just as he acted early on to cut off travel from the source of the
virus, President Trump was advised by his health and infectious disease
experts that he should cut off travel from Europe --- an action he took
decisively without delay to save lives while Democrats and the media
criticized him and the global health community still did not fully
comprehend the level of transmission or spread,'' Mr. Deere said.

The travel that helped spread the virus includes New York residents who
left the city and non-New Yorkers who visited or passed through the
city. Now that infections are dispersed around the country, travel from
New York is no longer a main factor shaping the progression of the
epidemic, researchers said.

As states around the nation begin to relax their restrictions, the
findings demonstrate that it is difficult, if not impossible, to prevent
those actions from affecting the rest of the nation.

\includegraphics{https://static01.nyt.com/images/2020/03/01/nyregion/07virus-nyc-hub-pix2/merlin_169794105_27491074-3827-4adb-b815-b8076493df46-articleLarge.jpg?quality=75\&auto=webp\&disable=upscale}

Geneticists have analyzed and shared more than 2,000 samples of the
virus from infected people. As the virus infects new people and
replicates, it picks up mutations along the way. These mutations
typically do not change the behavior of the virus, but they can provide
a signature of a virus's origin.

Most samples taken in Texas, Ohio, Louisiana, Idaho, Wisconsin and many
other states carry distinct mutations that can be traced back to viruses
introduced into New York.

Over all, Dr. Grubaugh estimated, infections spreading from New York
account for 60 to 65 percent of the sequenced viruses across the
country.

Other scientists said that they would like to see more samples before
calculating precise figures. But they agreed that New York's prominence
in seeding the national spread appears to have begun in early March, two
weeks before stay-at-home orders were put in place.

``New York acted as the Grand Central Station for this virus, with the
opportunity to move from there in so many directions, to so many
places,'' said David Engelthaler, head of the infectious disease branch
of the Translational Genomics Research Institute in Arizona.

The most commonly detected viruses tied to New York have a distinct
genetic signature linking them to outbreaks in Europe. Those spreading
from Washington State have a signature linking them directly to China.

Benjamin M. Branham, a spokesman for the Port **** Authority of New York
and New Jersey, which controls the airports, said that before the flight
restrictions from Europe, the federal government's Customs and Border
Protection ``only screened passengers from China,'' not from Europe.

At this stage, scientists say, genetic fingerprints alone are not
sufficient for pinpointing the source of the viruses. But travel
patterns and case histories of early known cases support the idea, they
said.

\href{https://www.nytimes.com/news-event/coronavirus?action=click\&pgtype=Article\&state=default\&region=MAIN_CONTENT_3\&context=storylines_faq}{}

\hypertarget{the-coronavirus-outbreak-}{%
\subsubsection{The Coronavirus Outbreak
›}\label{the-coronavirus-outbreak-}}

\hypertarget{frequently-asked-questions}{%
\paragraph{Frequently Asked
Questions}\label{frequently-asked-questions}}

Updated July 27, 2020

\begin{itemize}
\item ~
  \hypertarget{should-i-refinance-my-mortgage}{%
  \paragraph{Should I refinance my
  mortgage?}\label{should-i-refinance-my-mortgage}}

  \begin{itemize}
  \tightlist
  \item
    \href{https://www.nytimes.com/article/coronavirus-money-unemployment.html?action=click\&pgtype=Article\&state=default\&region=MAIN_CONTENT_3\&context=storylines_faq}{It
    could be a good idea,} because mortgage rates have
    \href{https://www.nytimes.com/2020/07/16/business/mortgage-rates-below-3-percent.html?action=click\&pgtype=Article\&state=default\&region=MAIN_CONTENT_3\&context=storylines_faq}{never
    been lower.} Refinancing requests have pushed mortgage applications
    to some of the highest levels since 2008, so be prepared to get in
    line. But defaults are also up, so if you're thinking about buying a
    home, be aware that some lenders have tightened their standards.
  \end{itemize}
\item ~
  \hypertarget{what-is-school-going-to-look-like-in-september}{%
  \paragraph{What is school going to look like in
  September?}\label{what-is-school-going-to-look-like-in-september}}

  \begin{itemize}
  \tightlist
  \item
    It is unlikely that many schools will return to a normal schedule
    this fall, requiring the grind of
    \href{https://www.nytimes.com/2020/06/05/us/coronavirus-education-lost-learning.html?action=click\&pgtype=Article\&state=default\&region=MAIN_CONTENT_3\&context=storylines_faq}{online
    learning},
    \href{https://www.nytimes.com/2020/05/29/us/coronavirus-child-care-centers.html?action=click\&pgtype=Article\&state=default\&region=MAIN_CONTENT_3\&context=storylines_faq}{makeshift
    child care} and
    \href{https://www.nytimes.com/2020/06/03/business/economy/coronavirus-working-women.html?action=click\&pgtype=Article\&state=default\&region=MAIN_CONTENT_3\&context=storylines_faq}{stunted
    workdays} to continue. California's two largest public school
    districts --- Los Angeles and San Diego --- said on July 13, that
    \href{https://www.nytimes.com/2020/07/13/us/lausd-san-diego-school-reopening.html?action=click\&pgtype=Article\&state=default\&region=MAIN_CONTENT_3\&context=storylines_faq}{instruction
    will be remote-only in the fall}, citing concerns that surging
    coronavirus infections in their areas pose too dire a risk for
    students and teachers. Together, the two districts enroll some
    825,000 students. They are the largest in the country so far to
    abandon plans for even a partial physical return to classrooms when
    they reopen in August. For other districts, the solution won't be an
    all-or-nothing approach.
    \href{https://bioethics.jhu.edu/research-and-outreach/projects/eschool-initiative/school-policy-tracker/}{Many
    systems}, including the nation's largest, New York City, are
    devising
    \href{https://www.nytimes.com/2020/06/26/us/coronavirus-schools-reopen-fall.html?action=click\&pgtype=Article\&state=default\&region=MAIN_CONTENT_3\&context=storylines_faq}{hybrid
    plans} that involve spending some days in classrooms and other days
    online. There's no national policy on this yet, so check with your
    municipal school system regularly to see what is happening in your
    community.
  \end{itemize}
\item ~
  \hypertarget{is-the-coronavirus-airborne}{%
  \paragraph{Is the coronavirus
  airborne?}\label{is-the-coronavirus-airborne}}

  \begin{itemize}
  \tightlist
  \item
    The coronavirus
    \href{https://www.nytimes.com/2020/07/04/health/239-experts-with-one-big-claim-the-coronavirus-is-airborne.html?action=click\&pgtype=Article\&state=default\&region=MAIN_CONTENT_3\&context=storylines_faq}{can
    stay aloft for hours in tiny droplets in stagnant air}, infecting
    people as they inhale, mounting scientific evidence suggests. This
    risk is highest in crowded indoor spaces with poor ventilation, and
    may help explain super-spreading events reported in meatpacking
    plants, churches and restaurants.
    \href{https://www.nytimes.com/2020/07/06/health/coronavirus-airborne-aerosols.html?action=click\&pgtype=Article\&state=default\&region=MAIN_CONTENT_3\&context=storylines_faq}{It's
    unclear how often the virus is spread} via these tiny droplets, or
    aerosols, compared with larger droplets that are expelled when a
    sick person coughs or sneezes, or transmitted through contact with
    contaminated surfaces, said Linsey Marr, an aerosol expert at
    Virginia Tech. Aerosols are released even when a person without
    symptoms exhales, talks or sings, according to Dr. Marr and more
    than 200 other experts, who
    \href{https://academic.oup.com/cid/article/doi/10.1093/cid/ciaa939/5867798}{have
    outlined the evidence in an open letter to the World Health
    Organization}.
  \end{itemize}
\item ~
  \hypertarget{what-are-the-symptoms-of-coronavirus}{%
  \paragraph{What are the symptoms of
  coronavirus?}\label{what-are-the-symptoms-of-coronavirus}}

  \begin{itemize}
  \tightlist
  \item
    Common symptoms
    \href{https://www.nytimes.com/article/symptoms-coronavirus.html?action=click\&pgtype=Article\&state=default\&region=MAIN_CONTENT_3\&context=storylines_faq}{include
    fever, a dry cough, fatigue and difficulty breathing or shortness of
    breath.} Some of these symptoms overlap with those of the flu,
    making detection difficult, but runny noses and stuffy sinuses are
    less common.
    \href{https://www.nytimes.com/2020/04/27/health/coronavirus-symptoms-cdc.html?action=click\&pgtype=Article\&state=default\&region=MAIN_CONTENT_3\&context=storylines_faq}{The
    C.D.C. has also} added chills, muscle pain, sore throat, headache
    and a new loss of the sense of taste or smell as symptoms to look
    out for. Most people fall ill five to seven days after exposure, but
    symptoms may appear in as few as two days or as many as 14 days.
  \end{itemize}
\item ~
  \hypertarget{does-asymptomatic-transmission-of-covid-19-happen}{%
  \paragraph{Does asymptomatic transmission of Covid-19
  happen?}\label{does-asymptomatic-transmission-of-covid-19-happen}}

  \begin{itemize}
  \tightlist
  \item
    So far, the evidence seems to show it does. A widely cited
    \href{https://www.nature.com/articles/s41591-020-0869-5}{paper}
    published in April suggests that people are most infectious about
    two days before the onset of coronavirus symptoms and estimated that
    44 percent of new infections were a result of transmission from
    people who were not yet showing symptoms. Recently, a top expert at
    the World Health Organization stated that transmission of the
    coronavirus by people who did not have symptoms was ``very rare,''
    \href{https://www.nytimes.com/2020/06/09/world/coronavirus-updates.html?action=click\&pgtype=Article\&state=default\&region=MAIN_CONTENT_3\&context=storylines_faq\#link-1f302e21}{but
    she later walked back that statement.}
  \end{itemize}
\end{itemize}

``It is a combination, still, of what genomic epidemiology and
shoe-leather epidemiology is going to tell us,'' Dr. Engelthaler said.

Scientists modeling the progression of the disease nationally said the
prominence of New York as a national hub was broadly consistent with
their findings, although the picture was still emerging.

``I would say this is not surprising in a sense,'' said Alessandro
Vespignani, director of the Network Science Institute at Northeastern
University in Boston. ``The picture emerging is consistent with
numerical models.''

Earlier research by Dr. Vespignani showed just how rapidly, and
invisibly, the outbreak exploded in New York. By March 1, when the first
coronavirus case was confirmed in New York, the city probably had
\href{https://www.nytimes.com/2020/04/23/us/coronavirus-early-outbreaks-cities.html}{over
10,000 undetected infections}, his research group showed.

New York and Washington State are not the only sources of the outbreak.
Other large domestic hubs contributed to the spread, scientists believe,
and a more diverse genetic mix is still seen in some places around the
country, particularly in the Midwest and parts of the South.

Even as domestic travel began to drive the outbreak, some infections
were still seeded around the country by international travelers,
geneticists said. It is possible, experts said, that some of the virus
samples attributed to New York may have instead been seeded in other
cities by direct flights from Europe, or from travelers laying over in
New York before traveling elsewhere.

For that reason, some scientists said they would like to see more
samples before linking the majority of infections in the United States
to New York.

``I think that's probably the story line that's going to emerge, but I'd
like to see more data,'' said Harm van Bakel, a geneticist at Mount
Sinai in New York.

A New York Times analysis of travel data supports the idea that the
chains of infection originated in New York, experts said. The number of
cases across the country was closely related to how many travelers each
place received from New York in early March, based on anonymized
cellphone tracking data from Cuebiq, a data intelligence company.

``It looks like most of the domestic spread is basically people
traveling out from New York,'' said Dr. Kari Stefansson, founder and
chief executive of deCODE Genetics, a leading genome analysis firm based
in Reykjavik, Iceland.

Last week, Dr. Andersen of Scripps Research and other scientists
analyzing the outbreak in New Orleans reported that all of the samples
taken from New Orleans were from the line linked back to New York. The
virus swept through the area in March and has killed more than 1,000
people.

``You can figure out, with travel patterns, that the most likely thing
to have happened is those came into New Orleans directly from New
York,'' Dr. Grubaugh said.

Josh Holder, Michael Crowley and Derek Watkins contributed reporting.

Advertisement

\protect\hyperlink{after-bottom}{Continue reading the main story}

\hypertarget{site-index}{%
\subsection{Site Index}\label{site-index}}

\hypertarget{site-information-navigation}{%
\subsection{Site Information
Navigation}\label{site-information-navigation}}

\begin{itemize}
\tightlist
\item
  \href{https://help.nytimes.com/hc/en-us/articles/115014792127-Copyright-notice}{©~2020~The
  New York Times Company}
\end{itemize}

\begin{itemize}
\tightlist
\item
  \href{https://www.nytco.com/}{NYTCo}
\item
  \href{https://help.nytimes.com/hc/en-us/articles/115015385887-Contact-Us}{Contact
  Us}
\item
  \href{https://www.nytco.com/careers/}{Work with us}
\item
  \href{https://nytmediakit.com/}{Advertise}
\item
  \href{http://www.tbrandstudio.com/}{T Brand Studio}
\item
  \href{https://www.nytimes.com/privacy/cookie-policy\#how-do-i-manage-trackers}{Your
  Ad Choices}
\item
  \href{https://www.nytimes.com/privacy}{Privacy}
\item
  \href{https://help.nytimes.com/hc/en-us/articles/115014893428-Terms-of-service}{Terms
  of Service}
\item
  \href{https://help.nytimes.com/hc/en-us/articles/115014893968-Terms-of-sale}{Terms
  of Sale}
\item
  \href{https://spiderbites.nytimes.com}{Site Map}
\item
  \href{https://help.nytimes.com/hc/en-us}{Help}
\item
  \href{https://www.nytimes.com/subscription?campaignId=37WXW}{Subscriptions}
\end{itemize}
