Sections

SEARCH

\protect\hyperlink{site-content}{Skip to
content}\protect\hyperlink{site-index}{Skip to site index}

\href{https://www.nytimes.com/section/arts}{Arts}

\href{https://myaccount.nytimes.com/auth/login?response_type=cookie\&client_id=vi}{}

\href{https://www.nytimes.com/section/todayspaper}{Today's Paper}

\href{/section/arts}{Arts}\textbar{}John Driscoll, Scholar, Art Dealer
and Collector, Dies at 70

\url{https://nyti.ms/3cY892d}

\begin{itemize}
\item
\item
\item
\item
\item
\end{itemize}

\href{https://www.nytimes.com/news-event/coronavirus?action=click\&pgtype=Article\&state=default\&region=TOP_BANNER\&context=storylines_menu}{The
Coronavirus Outbreak}

\begin{itemize}
\tightlist
\item
  live\href{https://www.nytimes.com/2020/08/03/world/coronavirus-covid-19.html?action=click\&pgtype=Article\&state=default\&region=TOP_BANNER\&context=storylines_menu}{Latest
  Updates}
\item
  \href{https://www.nytimes.com/interactive/2020/us/coronavirus-us-cases.html?action=click\&pgtype=Article\&state=default\&region=TOP_BANNER\&context=storylines_menu}{Maps
  and Cases}
\item
  \href{https://www.nytimes.com/interactive/2020/science/coronavirus-vaccine-tracker.html?action=click\&pgtype=Article\&state=default\&region=TOP_BANNER\&context=storylines_menu}{Vaccine
  Tracker}
\item
  \href{https://www.nytimes.com/2020/08/02/us/covid-college-reopening.html?action=click\&pgtype=Article\&state=default\&region=TOP_BANNER\&context=storylines_menu}{College
  Reopening}
\item
  \href{https://www.nytimes.com/live/2020/08/03/business/stock-market-today-coronavirus?action=click\&pgtype=Article\&state=default\&region=TOP_BANNER\&context=storylines_menu}{Economy}
\end{itemize}

Advertisement

\protect\hyperlink{after-top}{Continue reading the main story}

Supported by

\protect\hyperlink{after-sponsor}{Continue reading the main story}

Those We've Lost

\hypertarget{john-driscoll-scholar-art-dealer-and-collector-dies-at-70}{%
\section{John Driscoll, Scholar, Art Dealer and Collector, Dies at
70}\label{john-driscoll-scholar-art-dealer-and-collector-dies-at-70}}

He purchased and expanded Babcock Galleries, New York's oldest art
seller, while building a renowned pottery collection. He died of the
novel coronavirus.

\includegraphics{https://static01.nyt.com/images/2020/05/27/obituaries/21Driscoll1/merlin_172537107_d6e0d48f-932d-4188-babe-f13b3664862d-articleLarge.jpg?quality=75\&auto=webp\&disable=upscale}

\href{https://www.nytimes.com/by/richard-sandomir}{\includegraphics{https://static01.nyt.com/images/2018/12/10/multimedia/author-richard-sandomir/author-richard-sandomir-thumbLarge.png}}

By \href{https://www.nytimes.com/by/richard-sandomir}{Richard Sandomir}

\begin{itemize}
\item
  May 22, 2020
\item
  \begin{itemize}
  \item
  \item
  \item
  \item
  \item
  \end{itemize}
\end{itemize}

\emph{This obituary is part of a series about people who have died in
the coronavirus pandemic. Read about others}
\href{https://www.nytimes.com/interactive/2020/obituaries/people-died-coronavirus-obituaries.html}{\emph{here}}\emph{.}

\href{http://www.driscollbabcock.com/exhibitions/installation/john-driscoll-1949-2020\#works}{John
Driscoll's} first visit to a museum, the Minneapolis Institute of Art,
at age 10, had been a tedious affair until he came upon
\href{https://collections.artsmia.org/art/529/lucretia-rembrandt-harmensz-van-rijn}{``Lucretia,''}
Rembrandt's painting of the suicide of a Roman nobleman's wife.

Her gown is bloody. A dagger is still in her right hand.

``And I can tell you right now that painting went through me like a
freight train,'' Dr. Driscoll recalled in 2019. ``I don't know if I was
shaking on the outside, but I was shaking on the inside. I was
mesmerized.''

He soon began craving more of that type of sensory experience, and he
would go on to find it often in a career in art that included owning the
venerable \href{http://www.driscollbabcock.com/}{Driscoll Babcock
Galleries} in Manhattan and assembling a major collection of English
studio pottery.

He died on April 10 in Peekskill, N.Y., at 70. His death, which was not
widely reported at the time, was caused by the new coronavirus,
according to his wife, Marylyn Dintenfass, a painter.

When Dr. Driscoll bought Babcock Galleries --- on East 80th Street near
Madison Avenue --- in 1987, he had already held several curatorial jobs
and been a partner in a Boston gallery. At Babcock --- New York's oldest
gallery, having opened in 1852 --- he widened its focus, combining
historical and more recent work under one roof. Where the gallery had
specialized in early 20th-century art and the American Modernism of
artists like Marsden
\href{https://www.nga.gov/collection/artist-info.1375.html}{Hartley}, he
added 19th-century paintings, in particular landscapes by the Hudson
River School, as well as the work of post-World War II artists.

``He was one of the first dealers who had come as a scholar from the
museum world and one of the first to combine earlier historical art with
more contemporary art,'' Barbara Haskell, a curator at the
\href{https://whitney.org}{Whitney Museum of American Art}, said in an
interview.

Dr. Driscoll's biography ``John Frederick Kensett: An American Master''
(1985) --- one of many books he wrote --- helped establish him as a
leading authority on that
\href{https://www.metmuseum.org/toah/hd/kens/hd_kens.htm}{Hudson River
artist known for his luminist}paintings.

John Paul Driscoll was born on Oct. 28, 1949, in Madison, Minn., and
grew up nearby in Clarkfield, a farming village. His mother, Vida
(Loafman) Driscoll, owned dry cleaning shops, a cafe and a uniform
store. His father, Paul, was a high school teacher and coach.

With his father's encouragement, John hunted for agates and Native
American artifacts and collected coins and stamps. He grew so enamored
with the beauty of early American coins that he started his own
coin-dealing business when he was about 12.

``His father once read in the paper that there was a coin show in
town,'' Ms. Dintenfass said in a phone interview. ``And after telling
John, who was 14 or 15, he saw that John's name was on it --- that he'd
put the whole thing together without telling his parents.''

After earning a bachelor's degree in history and art history at the
University of Minnesota, Morris, Dr. Driscoll studied for a master's in
art history at Penn State University, where he also earned a Ph.D. in
American art.

His interest in collecting ceramics, in particular British pottery,
began when he was a graduate assistant at the Museum of Art at Penn
State \href{https://palmermuseum.psu.edu/}{(now the Palmer Museum of
Art)}.

Captivated by the pots, he found he could afford them even on his modest
salary. Paying from \$4 to \$80 for individual pieces, he bought works
by significant contemporary potters like
\href{http://www.artnet.com/artists/lucie-rie/}{Lucie Rie},
\href{https://www.oxfordceramics.com/artists/89-john-ward/overview/}{John
Ward},
\href{https://www.20thcenturyforum.com/t11375-annette-fuchs}{Annette
Fuchs}and Hans Coper. Soon he traveled to England to visit potters,
including Ms. Rie and Mr. Ward, spending \$3,000 on 50 to 60 pots.

Dr. Driscoll continued to amass a world-class collection of pottery
privately as he rose to registrar at the Penn State museum; curator of
the private art holdings of the
\href{https://www.mfa.org/give/gifts-art/lane-collection}{William H.
Lane Foundation}in Leominster, Mass.; guest curator at the
\href{https://www.worcesterart.org/}{Worcester Art Museum} in
Massachusetts; and partner at Driscoll \& Walsh Fine Arts in Boston in
the mid-1980s. After acquiring Babcock in New York, he renamed it
\href{http://www.driscollbabcock.com/}{Driscoll Babcock Galleries} in
2012.

Dr. Driscoll sold works both to private collectors and to museums,
including the Metropolitan Museum of Art, the National Gallery of Art,
the Detroit Institute of Arts and the Minneapolis Institute of Arts,
where he first saw ``Lucretia.''

In addition to his wife, he is survived by his daughters, Emily and
Gillian Driscoll; his stepdaughters, Sharon Katz and Elana Amsterdam;
his stepsons, Robert and Marc Katz; his brothers, Charles and Robert;
and five grandchildren. His marriage to Jeanne Baker ended in divorce.

In 2015, Dr. Driscoll arranged the purchase of one of two surviving
versions of \href{https://www.mmam.org/collections}{Emanuel Leutze's
``Washington Crossing the Delaware},'' from 1851. A collector who had
lent it to the White House for 35 years sold it to the founders of the
Minnesota Marine Art Museum in Winona.

``It's probably the most famous American painting west of the Hudson
River,'' Dr. Driscoll
\href{https://www.startribune.com/washington-crossing-the-delaware-lands-in-winona-museum/297329091/}{said
when it was unveiled}. ``At auction, this picture would have pulled out
not only art collectors but ultrapatriots who are very wealthy.''

After his death, Mary Burrichter, one of the museum's founders, lauded
his help as an adviser.

``We would get comments from guests like, `This museum belongs on the
East Coast or West Coast; why is it in a small town of 27,000 and not in
a major city?'''
\href{https://www.startribune.com/minnesota-born-art-scholar-john-driscoll-dies-of-covid-19-complications/569964762/}{she
told the Star Tribune of Minneapolis.} ``John just loved that
feedback.''

TK TEXT

\href{https://www.nytimes.com/interactive/2020/obituaries/people-died-coronavirus-obituaries.html?action=click\&pgtype=Article\&state=default\&region=BELOW_MAIN_CONTENT\&context=covid_obits_promo}{}

\hypertarget{those-weve-lost}{%
\section{Those We've Lost}\label{those-weve-lost}}

The coronavirus pandemic has taken an incalculable death toll. This
series is designed to put names and faces to the numbers.

Read more

\includegraphics{https://static01.nyt.com/images/2020/07/30/obituaries/30Pedro/30Pedro-square640.jpg}

\hypertarget{bernaldina-josuxe9-pedro}{%
\section{Bernaldina José Pedro}\label{bernaldina-josuxe9-pedro}}

d. Boa Vista, Brazil

Leader among the Indigenous Macuxi

\includegraphics{https://static01.nyt.com/images/2020/07/31/obituaries/31Swing/merlin_175167783_8913bc90-0d64-43f3-a655-1bb1bf1601c9-square640.jpg}

\hypertarget{john-eric-swing}{%
\section{John Eric Swing}\label{john-eric-swing}}

d. Fountain Valley, Calif.

Champion of Filipino-Americans

\includegraphics{https://static01.nyt.com/images/2020/07/27/obituaries/27Victor/merlin_175001436_38b11f8e-227a-4e2c-9821-7618af9b2524-square640.jpg}

\hypertarget{victor-victor}{%
\section{Victor Victor}\label{victor-victor}}

d. Santo Domingo, Dominican Republic

Beloved musician of the Dominican Republic

\includegraphics{https://static01.nyt.com/images/2020/07/31/obituaries/31Negron/merlin_175160169_516322ae-fd23-4969-b6b2-193ced371105-square640.jpg}

\hypertarget{dr-eddie-negruxf3n}{%
\section{Dr. Eddie Negrón}\label{dr-eddie-negruxf3n}}

d. Fort Walton Beach, Fla.

Internist on Florida's Emerald Coast

\includegraphics{https://static01.nyt.com/images/2020/07/30/obituaries/30Dobson/merlin_175115928_f6b9271c-8f05-4fe1-a38a-5ca4a58f8935-square640.jpg}

\hypertarget{dobby-dobson}{%
\section{Dobby Dobson}\label{dobby-dobson}}

d. Coral Springs, Fla.

Jamaican singer and songwriter

\includegraphics{https://static01.nyt.com/images/2020/08/01/obituaries/28Gonzalez/merlin_175002771_beb57888-3951-409a-ae13-03a94b2e962e-square640.jpg}

\hypertarget{waldemar-gonzalez}{%
\section{Waldemar Gonzalez}\label{waldemar-gonzalez}}

d. White Plains, N.Y.

Teacher and social worker

Advertisement

\protect\hyperlink{after-bottom}{Continue reading the main story}

\hypertarget{site-index}{%
\subsection{Site Index}\label{site-index}}

\hypertarget{site-information-navigation}{%
\subsection{Site Information
Navigation}\label{site-information-navigation}}

\begin{itemize}
\tightlist
\item
  \href{https://help.nytimes.com/hc/en-us/articles/115014792127-Copyright-notice}{©~2020~The
  New York Times Company}
\end{itemize}

\begin{itemize}
\tightlist
\item
  \href{https://www.nytco.com/}{NYTCo}
\item
  \href{https://help.nytimes.com/hc/en-us/articles/115015385887-Contact-Us}{Contact
  Us}
\item
  \href{https://www.nytco.com/careers/}{Work with us}
\item
  \href{https://nytmediakit.com/}{Advertise}
\item
  \href{http://www.tbrandstudio.com/}{T Brand Studio}
\item
  \href{https://www.nytimes.com/privacy/cookie-policy\#how-do-i-manage-trackers}{Your
  Ad Choices}
\item
  \href{https://www.nytimes.com/privacy}{Privacy}
\item
  \href{https://help.nytimes.com/hc/en-us/articles/115014893428-Terms-of-service}{Terms
  of Service}
\item
  \href{https://help.nytimes.com/hc/en-us/articles/115014893968-Terms-of-sale}{Terms
  of Sale}
\item
  \href{https://spiderbites.nytimes.com}{Site Map}
\item
  \href{https://help.nytimes.com/hc/en-us}{Help}
\item
  \href{https://www.nytimes.com/subscription?campaignId=37WXW}{Subscriptions}
\end{itemize}
