Sections

SEARCH

\protect\hyperlink{site-content}{Skip to
content}\protect\hyperlink{site-index}{Skip to site index}

\href{https://www.nytimes.com/section/politics}{Politics}

\href{https://myaccount.nytimes.com/auth/login?response_type=cookie\&client_id=vi}{}

\href{https://www.nytimes.com/section/todayspaper}{Today's Paper}

\href{/section/politics}{Politics}\textbar{}Trump Foresees Virus Death
Toll as High as 100,000 in the United States

\url{https://nyti.ms/2YytkDF}

\begin{itemize}
\item
\item
\item
\item
\item
\end{itemize}

\href{https://www.nytimes.com/news-event/coronavirus?action=click\&pgtype=Article\&state=default\&region=TOP_BANNER\&context=storylines_menu}{The
Coronavirus Outbreak}

\begin{itemize}
\tightlist
\item
  live\href{https://www.nytimes.com/2020/08/04/world/coronavirus-cases.html?action=click\&pgtype=Article\&state=default\&region=TOP_BANNER\&context=storylines_menu}{Latest
  Updates}
\item
  \href{https://www.nytimes.com/interactive/2020/us/coronavirus-us-cases.html?action=click\&pgtype=Article\&state=default\&region=TOP_BANNER\&context=storylines_menu}{Maps
  and Cases}
\item
  \href{https://www.nytimes.com/interactive/2020/science/coronavirus-vaccine-tracker.html?action=click\&pgtype=Article\&state=default\&region=TOP_BANNER\&context=storylines_menu}{Vaccine
  Tracker}
\item
  \href{https://www.nytimes.com/2020/08/02/us/covid-college-reopening.html?action=click\&pgtype=Article\&state=default\&region=TOP_BANNER\&context=storylines_menu}{College
  Reopening}
\item
  \href{https://www.nytimes.com/live/2020/08/04/business/stock-market-today-coronavirus?action=click\&pgtype=Article\&state=default\&region=TOP_BANNER\&context=storylines_menu}{Economy}
\end{itemize}

Advertisement

\protect\hyperlink{after-top}{Continue reading the main story}

Supported by

\protect\hyperlink{after-sponsor}{Continue reading the main story}

\hypertarget{trump-foresees-virus-death-toll-as-high-as-100000-in-the-united-states}{%
\section{Trump Foresees Virus Death Toll as High as 100,000 in the
United
States}\label{trump-foresees-virus-death-toll-as-high-as-100000-in-the-united-states}}

But even as President Trump acknowledged that the coronavirus has been
deadlier than he had previously predicted, he pressed to reopen the
country.

\includegraphics{https://static01.nyt.com/images/2020/05/04/us/politics/03dc-virus-trump-print/03dc-virus-trump-print-articleLarge.jpg?quality=75\&auto=webp\&disable=upscale}

\href{https://www.nytimes.com/by/peter-baker}{\includegraphics{https://static01.nyt.com/images/2018/06/13/multimedia/peter-baker/peter-baker-thumbLarge-v2.png}}

By \href{https://www.nytimes.com/by/peter-baker}{Peter Baker}

\begin{itemize}
\item
  Published May 3, 2020Updated May 4, 2020
\item
  \begin{itemize}
  \item
  \item
  \item
  \item
  \item
  \end{itemize}
\end{itemize}

WASHINGTON ---
\href{https://www.nytimes.com/2020/05/04/us/politics/trump-coronavirus-death-toll.html}{President
Trump} predicted on Sunday night that
\href{https://www.nytimes.com/interactive/2020/us/coronavirus-us-cases.html}{the
death toll from the coronavirus pandemic} ravaging the country may reach
as high as 100,000 in the United States, twice as many as he had
forecast just two weeks ago, even as he pressed states to reopen the
shuttered economy.

Mr. Trump, who last month forecast that fatalities from the outbreak
could be kept
``\href{https://www.whitehouse.gov/briefings-statements/remarks-president-trump-vice-president-pence-members-coronavirus-task-force-press-briefing-24/}{substantially
below the 100,000}'' mark and even as low as 50,000, acknowledged that
the
\href{https://www.nytimes.com/2020/05/04/us/politics/trump-coronavirus-death-toll.html}{virus}
has proved more devastating than expected. But nonetheless, he said that
parks, beaches and some businesses should begin reopening now and that
schools should resume classes in person by this fall.

``We're going to lose anywhere from 75, 80 to 100,000 people,'' the
president said in a virtual ``town hall'' meeting at the Lincoln
Memorial
\href{https://www.foxnews.com/politics/trump-virtual-town-hall-america-together-returning-to-work-coronavirus-lincoln-memorial}{hosted
by Fox News}. ``That's a horrible thing. We shouldn't lose one person
over this.'' But he credited himself with preventing the toll from being
worse. ``If we didn't do it, the minimum we would have lost was a
million two, a million four, a million five, that's the minimum. We
would have lost probably higher, it's possible higher than 2.2''
million.

The death toll passed 67,000 on Sunday, more than the total American
deaths in the Vietnam War and already higher than the president's
earlier prediction. More than 1,000 additional deaths have been
announced every day since April 2 and while the rate appears to have
peaked, it has not begun to fall in a significant, sustained way. The
model embraced by the White House a month ago had assumed the death rate
would begin to fall substantially by mid-April.

Despite that, Mr. Trump indicated again that he favored lifting
stay-at-home orders and other restrictions that have cratered the
economy and put
\href{https://www.nytimes.com/2020/04/30/business/economy/coronavirus-unemployment-claims.html}{more
than 30 million people out of work}, arguing that the government had
armed itself enough against the virus to be prepared to curb any
additional outbreak even after people begin emerging from their homes to
re-enter workplaces and other public spaces.

``At some point we have to open our country,'' the president said. ``And
people are going to be safe. We've learned a lot. We've learned about
the tremendous contagion. But we have no choice. We can't stay closed as
a country. We're not going to have a country left.''

Mr. Trump asserted again that the virus would eventually fade. ``This
virus will pass,'' he said. ``It will go. Will it come back? It might.
It could. Some people say yes. But it will pass.'' While he has
previously expressed doubt about a second wave in the fall anticipated
by public health experts, he conceded that it could happen. ``We may
have to put out a fire,'' he said.

\hypertarget{latest-updates-global-coronavirus-outbreak}{%
\section{\texorpdfstring{\href{https://www.nytimes.com/2020/08/04/world/coronavirus-cases.html?action=click\&pgtype=Article\&state=default\&region=MAIN_CONTENT_1\&context=storylines_live_updates}{Latest
Updates: Global Coronavirus
Outbreak}}{Latest Updates: Global Coronavirus Outbreak}}\label{latest-updates-global-coronavirus-outbreak}}

Updated 2020-08-04T19:15:37.345Z

\begin{itemize}
\tightlist
\item
  \href{https://www.nytimes.com/2020/08/04/world/coronavirus-cases.html?action=click\&pgtype=Article\&state=default\&region=MAIN_CONTENT_1\&context=storylines_live_updates\#link-4825b93}{Public
  and private schools in Maryland and elsewhere are divided over
  in-person instruction.}
\item
  \href{https://www.nytimes.com/2020/08/04/world/coronavirus-cases.html?action=click\&pgtype=Article\&state=default\&region=MAIN_CONTENT_1\&context=storylines_live_updates\#link-4d1eafa8}{N.Y.C.'s
  health commissioner resigns after clashing with the mayor over the
  virus.}
\item
  \href{https://www.nytimes.com/2020/08/04/world/coronavirus-cases.html?action=click\&pgtype=Article\&state=default\&region=MAIN_CONTENT_1\&context=storylines_live_updates\#link-6b644638}{`Long
  days, long nights': Washington prepares for a prolonged fight over
  virus relief.}
\end{itemize}

\href{https://www.nytimes.com/2020/08/04/world/coronavirus-cases.html?action=click\&pgtype=Article\&state=default\&region=MAIN_CONTENT_1\&context=storylines_live_updates}{See
more updates}

More live coverage:
\href{https://www.nytimes.com/live/2020/08/04/business/stock-market-today-coronavirus?action=click\&pgtype=Article\&state=default\&region=MAIN_CONTENT_1\&context=storylines_live_updates}{Markets}

The president's appearance on Fox, in which he sat at a distance from
the hosts at the foot of the Abraham Lincoln statue and took questions
sent by video from around the country, came in the middle of a furious
debate in the United States about how and when the states should begin
restoring a semblance of everyday life. The program was titled ``America
Together: Returning to Work.''

As of Friday,
\href{https://www.nytimes.com/interactive/2020/us/states-reopen-map-coronavirus.html}{more
than a dozen states had begun to reopen} their economies and public life
while many others had set plans to do so under certain conditions and
with certain precautions, in some cases over the warnings of public
health specialists who feared that moving too quickly would reignite a
wave of infections.

Mr. Trump predicted that
\href{https://www.nytimes.com/2020/05/02/us/politics/vaccines-coronavirus-research.html}{a
vaccine would be developed by the end of 2020}, which would be sooner
than some public health experts anticipate and
\href{https://www.nytimes.com/2020/04/18/health/coronavirus-america-future.html}{much
faster than any other vaccine} for such a major virus. ``We are very
confident that we're going to have a vaccine at the end of the year, by
the end of the year,'' he said. Even if it is developed that soon,
though, he did not say whether it could be approved and produced in
sufficient quantities for widespread use by then.

The president confirmed that he was warned about the virus,
\href{https://www.nytimes.com/2020/05/03/world/europe/backlash-china-coronavirus.html}{which
originated in China}, in an intelligence briefing in January, but
asserted that it was characterized as if ``it was not a big deal.'' He
said intelligence agencies would release information about his briefings
as early as Monday.

``On Jan. 23, I was told that there could be a virus coming in but it
was of no real import,'' Mr. Trump said. ``In other words, it wasn't,
`Oh, we've got to do something, we've got to do something.' It was a
brief conversation and it was only on Jan. 23. Shortly thereafter, I
closed the country to China. We had 23 people in the room and I was the
only one in the room who wanted to close it down.''

Mr. Trump was referring to his decision on Jan. 30 to
\href{https://www.nytimes.com/2020/01/31/business/china-travel-coronavirus.html}{block
entry by most foreign nationals coming from China}, a move that in fact
was supported by a number of his advisers and came only after major
American airlines had already canceled flights. Some public health
advisers have said that the travel limits helped slow the spread to the
United States but that the Trump administration did not use the extra
time to adequately prepare by ramping up testing and producing medical
equipment.

Mr. Trump said his travel limit, which did not apply to Americans or
legal residents, was not driven by the Jan. 23 warning. ``I didn't do it
because of what they said,'' he said. ``They said it very matter of
factly. It was not a big deal.''

In forecasting the toll of the virus, the White House had relied on
models by the Institute for Health Metrics and Evaluation at the
University of Washington, which last month had predicted 60,415 deaths
by the first week of August. Last week, the institute
\href{https://covid19.healthdata.org/united-states-of-america}{increased
its estimate to 72,433} by early August. But now the toll looks likely
to pass that number within a week.

``It looks like we're headed to a number substantially below the
100,000,'' Mr. Trump had said on April 10. ``That would be the low mark.
And I hope that bears out.'' He said a lower number would amount to a
victory for him. ``Hard to believe that if you had 60,000 --- you could
never be happy, but that's a lot fewer than we were originally told and
thinking.'' As late as April 20, he said ``we're going toward 50 or
60,000 people.''

Vice President Mike Pence and Treasury Secretary Steven Mnuchin appeared
with Mr. Trump on Sunday's broadcast. Mr. Pence acknowledged making a
mistake by
\href{https://www.nytimes.com/2020/04/28/us/politics/coronavirus-pence-mask.html}{not
wearing a mask during a visit to the Mayo Clinic} last week despite the
medical center's policy.

Because masks are meant to protect other people and he has been tested
regularly, Mr. Pence said, he was in keeping with federal guidelines.
``I didn't think it was necessary,'' he said. ``But I should have wore
the mask at the Mayo Clinic.''

\href{https://www.nytimes.com/news-event/coronavirus?action=click\&pgtype=Article\&state=default\&region=MAIN_CONTENT_3\&context=storylines_faq}{}

\hypertarget{the-coronavirus-outbreak-}{%
\subsubsection{The Coronavirus Outbreak
›}\label{the-coronavirus-outbreak-}}

\hypertarget{frequently-asked-questions}{%
\paragraph{Frequently Asked
Questions}\label{frequently-asked-questions}}

Updated August 4, 2020

\begin{itemize}
\item ~
  \hypertarget{i-have-antibodies-am-i-now-immune}{%
  \paragraph{I have antibodies. Am I now
  immune?}\label{i-have-antibodies-am-i-now-immune}}

  \begin{itemize}
  \tightlist
  \item
    As of right
    now,\href{https://www.nytimes.com/2020/07/22/health/covid-antibodies-herd-immunity.html?action=click\&pgtype=Article\&state=default\&region=MAIN_CONTENT_3\&context=storylines_faq}{that
    seems likely, for at least several months.} There have been
    frightening accounts of people suffering what seems to be a second
    bout of Covid-19. But experts say these patients may have a
    drawn-out course of infection, with the virus taking a slow toll
    weeks to months after initial exposure. People infected with the
    coronavirus typically
    \href{https://www.nature.com/articles/s41586-020-2456-9}{produce}
    immune molecules called antibodies, which are
    \href{https://www.nytimes.com/2020/05/07/health/coronavirus-antibody-prevalence.html?action=click\&pgtype=Article\&state=default\&region=MAIN_CONTENT_3\&context=storylines_faq}{protective
    proteins made in response to an
    infection}\href{https://www.nytimes.com/2020/05/07/health/coronavirus-antibody-prevalence.html?action=click\&pgtype=Article\&state=default\&region=MAIN_CONTENT_3\&context=storylines_faq}{.
    These antibodies may} last in the body
    \href{https://www.nature.com/articles/s41591-020-0965-6}{only two to
    three months}, which may seem worrisome, but that's perfectly normal
    after an acute infection subsides, said Dr. Michael Mina, an
    immunologist at Harvard University. It may be possible to get the
    coronavirus again, but it's highly unlikely that it would be
    possible in a short window of time from initial infection or make
    people sicker the second time.
  \end{itemize}
\item ~
  \hypertarget{im-a-small-business-owner-can-i-get-relief}{%
  \paragraph{I'm a small-business owner. Can I get
  relief?}\label{im-a-small-business-owner-can-i-get-relief}}

  \begin{itemize}
  \tightlist
  \item
    The
    \href{https://www.nytimes.com/article/small-business-loans-stimulus-grants-freelancers-coronavirus.html?action=click\&pgtype=Article\&state=default\&region=MAIN_CONTENT_3\&context=storylines_faq}{stimulus
    bills enacted in March} offer help for the millions of American
    small businesses. Those eligible for aid are businesses and
    nonprofit organizations with fewer than 500 workers, including sole
    proprietorships, independent contractors and freelancers. Some
    larger companies in some industries are also eligible. The help
    being offered, which is being managed by the Small Business
    Administration, includes the Paycheck Protection Program and the
    Economic Injury Disaster Loan program. But lots of folks have
    \href{https://www.nytimes.com/interactive/2020/05/07/business/small-business-loans-coronavirus.html?action=click\&pgtype=Article\&state=default\&region=MAIN_CONTENT_3\&context=storylines_faq}{not
    yet seen payouts.} Even those who have received help are confused:
    The rules are draconian, and some are stuck sitting on
    \href{https://www.nytimes.com/2020/05/02/business/economy/loans-coronavirus-small-business.html?action=click\&pgtype=Article\&state=default\&region=MAIN_CONTENT_3\&context=storylines_faq}{money
    they don't know how to use.} Many small-business owners are getting
    less than they expected or
    \href{https://www.nytimes.com/2020/06/10/business/Small-business-loans-ppp.html?action=click\&pgtype=Article\&state=default\&region=MAIN_CONTENT_3\&context=storylines_faq}{not
    hearing anything at all.}
  \end{itemize}
\item ~
  \hypertarget{what-are-my-rights-if-i-am-worried-about-going-back-to-work}{%
  \paragraph{What are my rights if I am worried about going back to
  work?}\label{what-are-my-rights-if-i-am-worried-about-going-back-to-work}}

  \begin{itemize}
  \tightlist
  \item
    Employers have to provide
    \href{https://www.osha.gov/SLTC/covid-19/standards.html}{a safe
    workplace} with policies that protect everyone equally.
    \href{https://www.nytimes.com/article/coronavirus-money-unemployment.html?action=click\&pgtype=Article\&state=default\&region=MAIN_CONTENT_3\&context=storylines_faq}{And
    if one of your co-workers tests positive for the coronavirus, the
    C.D.C.} has said that
    \href{https://www.cdc.gov/coronavirus/2019-ncov/community/guidance-business-response.html}{employers
    should tell their employees} -\/- without giving you the sick
    employee's name -\/- that they may have been exposed to the virus.
  \end{itemize}
\item ~
  \hypertarget{should-i-refinance-my-mortgage}{%
  \paragraph{Should I refinance my
  mortgage?}\label{should-i-refinance-my-mortgage}}

  \begin{itemize}
  \tightlist
  \item
    \href{https://www.nytimes.com/article/coronavirus-money-unemployment.html?action=click\&pgtype=Article\&state=default\&region=MAIN_CONTENT_3\&context=storylines_faq}{It
    could be a good idea,} because mortgage rates have
    \href{https://www.nytimes.com/2020/07/16/business/mortgage-rates-below-3-percent.html?action=click\&pgtype=Article\&state=default\&region=MAIN_CONTENT_3\&context=storylines_faq}{never
    been lower.} Refinancing requests have pushed mortgage applications
    to some of the highest levels since 2008, so be prepared to get in
    line. But defaults are also up, so if you're thinking about buying a
    home, be aware that some lenders have tightened their standards.
  \end{itemize}
\item ~
  \hypertarget{what-is-school-going-to-look-like-in-september}{%
  \paragraph{What is school going to look like in
  September?}\label{what-is-school-going-to-look-like-in-september}}

  \begin{itemize}
  \tightlist
  \item
    It is unlikely that many schools will return to a normal schedule
    this fall, requiring the grind of
    \href{https://www.nytimes.com/2020/06/05/us/coronavirus-education-lost-learning.html?action=click\&pgtype=Article\&state=default\&region=MAIN_CONTENT_3\&context=storylines_faq}{online
    learning},
    \href{https://www.nytimes.com/2020/05/29/us/coronavirus-child-care-centers.html?action=click\&pgtype=Article\&state=default\&region=MAIN_CONTENT_3\&context=storylines_faq}{makeshift
    child care} and
    \href{https://www.nytimes.com/2020/06/03/business/economy/coronavirus-working-women.html?action=click\&pgtype=Article\&state=default\&region=MAIN_CONTENT_3\&context=storylines_faq}{stunted
    workdays} to continue. California's two largest public school
    districts --- Los Angeles and San Diego --- said on July 13, that
    \href{https://www.nytimes.com/2020/07/13/us/lausd-san-diego-school-reopening.html?action=click\&pgtype=Article\&state=default\&region=MAIN_CONTENT_3\&context=storylines_faq}{instruction
    will be remote-only in the fall}, citing concerns that surging
    coronavirus infections in their areas pose too dire a risk for
    students and teachers. Together, the two districts enroll some
    825,000 students. They are the largest in the country so far to
    abandon plans for even a partial physical return to classrooms when
    they reopen in August. For other districts, the solution won't be an
    all-or-nothing approach.
    \href{https://bioethics.jhu.edu/research-and-outreach/projects/eschool-initiative/school-policy-tracker/}{Many
    systems}, including the nation's largest, New York City, are
    devising
    \href{https://www.nytimes.com/2020/06/26/us/coronavirus-schools-reopen-fall.html?action=click\&pgtype=Article\&state=default\&region=MAIN_CONTENT_3\&context=storylines_faq}{hybrid
    plans} that involve spending some days in classrooms and other days
    online. There's no national policy on this yet, so check with your
    municipal school system regularly to see what is happening in your
    community.
  \end{itemize}
\end{itemize}

The Fox town hall came on a day when Mr. Trump lashed out at former
President
\href{https://www.nytimes.com/topic/person/george-w-bush}{George W.
Bush}, who called for national unity in a three-minute video message
posted on Saturday.

\includegraphics{https://static01.nyt.com/images/2020/05/03/us/politics/03dc-virus-trump3/03dc-virus-trump3-articleLarge.jpg?quality=75\&auto=webp\&disable=upscale}

``Let us remember how small our differences are in the face of this
shared threat,''
\href{https://twitter.com/TheBushCenter/status/1256607729151619073}{Mr.
Bush said in the video}, which was set against music and photographs of
medical workers helping victims of the virus and of ordinary Americans
wearing masks. ``In the final analysis, we are not partisan combatants.
We are human beings, equally vulnerable and equally wonderful in the
sight of God. We rise or fall together, and we are determined to rise.''

While Mr. Bush never mentioned Mr. Trump's name, the sitting president
clearly took the message as an implicit rebuke. In a Twitter message,
Mr. Trump paraphrased a Fox News personality saying, ``Oh bye the way, I
appreciate the message from former President Bush, but where was he
during Impeachment calling for putting partisanship aside.''

Mr. Trump then added in his own voice: ``He was nowhere to be found in
speaking up against the greatest Hoax in American history!''

Hours later, Mr. Trump went after another predecessor, reposting a tweet
from a pro-Trump website accusing former President Barack Obama of
plotting against him. ``Evidence has surfaced that indicates Barack
Obama was the one running the Russian hoax,'' said the original message
retweeted by the president.

Mr. Bush's video message was part of a series of videos aired online as
part of a 24-hour live-streamed project, ``\href{https://unite.us/}{The
Call to Unite},'' that also featured Oprah Winfrey, Tim Shriver, Julia
Roberts, Martin Luther King III, Sean Combs, Quincy Jones, Naomi Judd,
Andrew Yang and others.

Mr. Bush's office said he had no response to Mr. Trump's message. ``The
video was a part of an event called `A Call to Unite,''' said Freddy
Ford, the former president's chief of staff. ``I hope those covering it
will resist the temptation to use it as a call to divide.'' Mr. Obama's
office had no comment.

Advertisement

\protect\hyperlink{after-bottom}{Continue reading the main story}

\hypertarget{site-index}{%
\subsection{Site Index}\label{site-index}}

\hypertarget{site-information-navigation}{%
\subsection{Site Information
Navigation}\label{site-information-navigation}}

\begin{itemize}
\tightlist
\item
  \href{https://help.nytimes.com/hc/en-us/articles/115014792127-Copyright-notice}{©~2020~The
  New York Times Company}
\end{itemize}

\begin{itemize}
\tightlist
\item
  \href{https://www.nytco.com/}{NYTCo}
\item
  \href{https://help.nytimes.com/hc/en-us/articles/115015385887-Contact-Us}{Contact
  Us}
\item
  \href{https://www.nytco.com/careers/}{Work with us}
\item
  \href{https://nytmediakit.com/}{Advertise}
\item
  \href{http://www.tbrandstudio.com/}{T Brand Studio}
\item
  \href{https://www.nytimes.com/privacy/cookie-policy\#how-do-i-manage-trackers}{Your
  Ad Choices}
\item
  \href{https://www.nytimes.com/privacy}{Privacy}
\item
  \href{https://help.nytimes.com/hc/en-us/articles/115014893428-Terms-of-service}{Terms
  of Service}
\item
  \href{https://help.nytimes.com/hc/en-us/articles/115014893968-Terms-of-sale}{Terms
  of Sale}
\item
  \href{https://spiderbites.nytimes.com}{Site Map}
\item
  \href{https://help.nytimes.com/hc/en-us}{Help}
\item
  \href{https://www.nytimes.com/subscription?campaignId=37WXW}{Subscriptions}
\end{itemize}
