Sections

SEARCH

\protect\hyperlink{site-content}{Skip to
content}\protect\hyperlink{site-index}{Skip to site index}

\href{https://www.nytimes.com/section/obituaries}{Obituaries}

\href{https://myaccount.nytimes.com/auth/login?response_type=cookie\&client_id=vi}{}

\href{https://www.nytimes.com/section/todayspaper}{Today's Paper}

\href{/section/obituaries}{Obituaries}\textbar{}Tendol Gyalzur, Refugee
Who Founded Orphanages in Tibet, Dies

\url{https://nyti.ms/2Z9c5cf}

\begin{itemize}
\item
\item
\item
\item
\item
\end{itemize}

\href{https://www.nytimes.com/news-event/coronavirus?action=click\&pgtype=Article\&state=default\&region=TOP_BANNER\&context=storylines_menu}{The
Coronavirus Outbreak}

\begin{itemize}
\tightlist
\item
  live\href{https://www.nytimes.com/2020/08/03/world/coronavirus-covid-19.html?action=click\&pgtype=Article\&state=default\&region=TOP_BANNER\&context=storylines_menu}{Latest
  Updates}
\item
  \href{https://www.nytimes.com/interactive/2020/us/coronavirus-us-cases.html?action=click\&pgtype=Article\&state=default\&region=TOP_BANNER\&context=storylines_menu}{Maps
  and Cases}
\item
  \href{https://www.nytimes.com/interactive/2020/science/coronavirus-vaccine-tracker.html?action=click\&pgtype=Article\&state=default\&region=TOP_BANNER\&context=storylines_menu}{Vaccine
  Tracker}
\item
  \href{https://www.nytimes.com/2020/08/02/us/covid-college-reopening.html?action=click\&pgtype=Article\&state=default\&region=TOP_BANNER\&context=storylines_menu}{College
  Reopening}
\item
  \href{https://www.nytimes.com/live/2020/08/03/business/stock-market-today-coronavirus?action=click\&pgtype=Article\&state=default\&region=TOP_BANNER\&context=storylines_menu}{Economy}
\end{itemize}

Advertisement

\protect\hyperlink{after-top}{Continue reading the main story}

Supported by

\protect\hyperlink{after-sponsor}{Continue reading the main story}

Those We've Lost

\hypertarget{tendol-gyalzur-refugee-who-founded-orphanages-in-tibet-dies}{%
\section{Tendol Gyalzur, Refugee Who Founded Orphanages in Tibet,
Dies}\label{tendol-gyalzur-refugee-who-founded-orphanages-in-tibet-dies}}

After losing her parents during the failed 1959 Tibetan uprising, she
fled to India, became a nurse in Germany and later worked with the
Chinese government to aid needy children in her homeland.

\includegraphics{https://static01.nyt.com/images/2020/05/19/obituaries/00Gyalzur/00Gyalzur-articleLarge-v2.jpg?quality=75\&auto=webp\&disable=upscale}

By Stephen Kurczy

\begin{itemize}
\item
  May 15, 2020
\item
  \begin{itemize}
  \item
  \item
  \item
  \item
  \item
  \end{itemize}
\end{itemize}

\emph{This obituary is part of a series about people who have died in
the coronavirus pandemic. Read about others}
\href{https://www.nytimes.com/series/people-who-have-died-of-the-coronavirus}{\emph{here}}\emph{.}

Tendol Gyalzur lost her parents and brother during the 1959 Tibetan
uprising and as a child crossed the Himalayas on foot and on horseback
to safety. But she returned to Tibet after more than three decades to
start the region's first private orphanages, which have taken in more
than 300 children.

Mrs. Gyalzur died on May 3 in Chur, Switzerland. She was believed to be
69. The cause was Covid-19, her son, Songtsen Gyalzur, said.

With assistance from the Tibet Development Fund, a Chinese-controlled
nonprofit, and using family savings, Mrs. Gyalzur opened Tibet's first
private orphanage in 1993 in Lhasa, the capital, accepting children from
a variety of ethnic groups.

``It was a big concern of hers to show that children are children and
people, people, no matter what ethnicity or religion,'' said Tanja
Polli, author of ``One Life for the Children of Tibet: The Unbelievable
Story of Tendol Gyalzur'' (2019).

She started a second orphanage in 1997 in her husband's hometown,
Shangri-La (also known as Zhongdian), in China's southwestern Yunnan
Province. In 2002, she began supporting a school for the children of
nomadic herders in western Sichuan Province.

After 25 years, amid a
\href{https://www.nytimes.com/2016/04/29/world/asia/china-foreign-ngo-law.html}{clampdown}
on the work of foreign organizations, Mrs. Gyalzur handed control of her
centers to the Chinese government, although she continued to visit them.

In a tense political environment, Mrs. Gyalzur was known for her ability
to cooperate with Chinese officials for more than a quarter-century to
get her job done. That cooperation, with a government accused of
\href{https://www.nytimes.com/2012/11/03/world/asia/un-rights-official-faults-china-on-tibetan-suppression.html}{human
rights abuses}, drew criticism from within the Tibetan diaspora. Tibet,
which is an autonomous region of China, is a hotbed of antigovernment
protests calling for greater independence and the return of the Dalai
Lama, the exiled spiritual leader.

``There's always the people who say that anyone working with the Chinese
regime is a sort of `blood traitor,''' said Tsering Shakya, a professor
at the University of British Columbia and author of ``The Dragon in the
Land of Snows: A History of Modern Tibet Since 1947'' (1999). He saw
Mrs. Gyalzur as pragmatic. ``You can have your own political stance, but
being able to effect change at the ground level is necessary.''

Tenzin Dolkar, who later began using Tendol, a variation of her first
name, was born in the central Tibetan town of Shigatse (also known as
Xigaze) **** on Dec. 2, 1951, according to her Swiss passport. But her
true birth date was unknown. When she arrived at a refugee camp in
northern India after fleeing Tibet with a migrant caravan, her age was
determined based on the number of her baby teeth. Even the names of her
parents and brother, and the details of their deaths, were lost amid a
chaotic departure from Tibet.

She was transferred to an orphanage in Dharamsala, India, run by the
sister of the Dalai Lama and chosen as one of a dozen children sent in
1963 to a children's home outside Munich as part of an initiative of the
Tibetan government-in-exile, according to Ms. Polli. Mrs. Gyalzur and
her peers were exhorted by the Dalai Lama to be ``the seeds of the
future Tibet.''

In Germany, Mrs. Gyalzur earned a nursing degree and met her future
husband, a fellow Tibetan refugee named Losang Gyalzur, whom she married
in 1972. She is survived by him; another son, Ghaden; and four
grandchildren.

The couple moved to Switzerland. Mrs. Gyalzur returned to Tibet for the
first time 18 years later and was moved to open her orphanage by the
sight of street children rummaging through the trash for food in Lhasa.

``It was the first time in my life,'' she said
\href{https://www.csmonitor.com/World/Making-a-difference/2009/1019/p07s01-lign.html}{in
an interview} for The Christian Science Monitor ** in 2009, ``that I
realized that the only thing I wanted to do was to fight for the rights
of these abandoned children.''

\href{https://www.nytimes.com/interactive/2020/obituaries/people-died-coronavirus-obituaries.html?action=click\&pgtype=Article\&state=default\&region=BELOW_MAIN_CONTENT\&context=covid_obits_promo}{}

\hypertarget{those-weve-lost}{%
\section{Those We've Lost}\label{those-weve-lost}}

The coronavirus pandemic has taken an incalculable death toll. This
series is designed to put names and faces to the numbers.

Read more

\includegraphics{https://static01.nyt.com/images/2020/07/30/obituaries/30Pedro/30Pedro-square640.jpg}

\hypertarget{bernaldina-josuxe9-pedro}{%
\section{Bernaldina José Pedro}\label{bernaldina-josuxe9-pedro}}

d. Boa Vista, Brazil

Leader among the Indigenous Macuxi

\includegraphics{https://static01.nyt.com/images/2020/07/31/obituaries/31Swing/merlin_175167783_8913bc90-0d64-43f3-a655-1bb1bf1601c9-square640.jpg}

\hypertarget{john-eric-swing}{%
\section{John Eric Swing}\label{john-eric-swing}}

d. Fountain Valley, Calif.

Champion of Filipino-Americans

\includegraphics{https://static01.nyt.com/images/2020/07/27/obituaries/27Victor/merlin_175001436_38b11f8e-227a-4e2c-9821-7618af9b2524-square640.jpg}

\hypertarget{victor-victor}{%
\section{Victor Victor}\label{victor-victor}}

d. Santo Domingo, Dominican Republic

Beloved musician of the Dominican Republic

\includegraphics{https://static01.nyt.com/images/2020/07/31/obituaries/31Negron/merlin_175160169_516322ae-fd23-4969-b6b2-193ced371105-square640.jpg}

\hypertarget{dr-eddie-negruxf3n}{%
\section{Dr. Eddie Negrón}\label{dr-eddie-negruxf3n}}

d. Fort Walton Beach, Fla.

Internist on Florida's Emerald Coast

\includegraphics{https://static01.nyt.com/images/2020/07/30/obituaries/30Dobson/merlin_175115928_f6b9271c-8f05-4fe1-a38a-5ca4a58f8935-square640.jpg}

\hypertarget{dobby-dobson}{%
\section{Dobby Dobson}\label{dobby-dobson}}

d. Coral Springs, Fla.

Jamaican singer and songwriter

\includegraphics{https://static01.nyt.com/images/2020/08/01/obituaries/28Gonzalez/merlin_175002771_beb57888-3951-409a-ae13-03a94b2e962e-square640.jpg}

\hypertarget{waldemar-gonzalez}{%
\section{Waldemar Gonzalez}\label{waldemar-gonzalez}}

d. White Plains, N.Y.

Teacher and social worker

Advertisement

\protect\hyperlink{after-bottom}{Continue reading the main story}

\hypertarget{site-index}{%
\subsection{Site Index}\label{site-index}}

\hypertarget{site-information-navigation}{%
\subsection{Site Information
Navigation}\label{site-information-navigation}}

\begin{itemize}
\tightlist
\item
  \href{https://help.nytimes.com/hc/en-us/articles/115014792127-Copyright-notice}{©~2020~The
  New York Times Company}
\end{itemize}

\begin{itemize}
\tightlist
\item
  \href{https://www.nytco.com/}{NYTCo}
\item
  \href{https://help.nytimes.com/hc/en-us/articles/115015385887-Contact-Us}{Contact
  Us}
\item
  \href{https://www.nytco.com/careers/}{Work with us}
\item
  \href{https://nytmediakit.com/}{Advertise}
\item
  \href{http://www.tbrandstudio.com/}{T Brand Studio}
\item
  \href{https://www.nytimes.com/privacy/cookie-policy\#how-do-i-manage-trackers}{Your
  Ad Choices}
\item
  \href{https://www.nytimes.com/privacy}{Privacy}
\item
  \href{https://help.nytimes.com/hc/en-us/articles/115014893428-Terms-of-service}{Terms
  of Service}
\item
  \href{https://help.nytimes.com/hc/en-us/articles/115014893968-Terms-of-sale}{Terms
  of Sale}
\item
  \href{https://spiderbites.nytimes.com}{Site Map}
\item
  \href{https://help.nytimes.com/hc/en-us}{Help}
\item
  \href{https://www.nytimes.com/subscription?campaignId=37WXW}{Subscriptions}
\end{itemize}
