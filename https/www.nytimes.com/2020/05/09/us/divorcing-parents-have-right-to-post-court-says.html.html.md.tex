Sections

SEARCH

\protect\hyperlink{site-content}{Skip to
content}\protect\hyperlink{site-index}{Skip to site index}

\href{https://www.nytimes.com/section/us}{U.S.}

\href{https://myaccount.nytimes.com/auth/login?response_type=cookie\&client_id=vi}{}

\href{https://www.nytimes.com/section/todayspaper}{Today's Paper}

\href{/section/us}{U.S.}\textbar{}Divorcing Parents Have a Right to Post
Their Stories Online, Court Says

\url{https://nyti.ms/2xMYtZ1}

\begin{itemize}
\item
\item
\item
\item
\item
\end{itemize}

Advertisement

\protect\hyperlink{after-top}{Continue reading the main story}

Supported by

\protect\hyperlink{after-sponsor}{Continue reading the main story}

\hypertarget{divorcing-parents-have-a-right-to-post-their-stories-online-court-says}{%
\section{Divorcing Parents Have a Right to Post Their Stories Online,
Court
Says}\label{divorcing-parents-have-a-right-to-post-their-stories-online-court-says}}

A ruling in Massachusetts finds that involuntary nondisparagement
orders, commonly used to keep spouses from discussing their cases on
social media, are unconstitutional.

\includegraphics{https://static01.nyt.com/images/2020/05/09/us/00divorce/00divorce-articleLarge.jpg?quality=75\&auto=webp\&disable=upscale}

\href{https://www.nytimes.com/by/ellen-barry}{\includegraphics{https://static01.nyt.com/images/2018/10/08/multimedia/author-ellen-barry/author-ellen-barry-thumbLarge.png}}

By \href{https://www.nytimes.com/by/ellen-barry}{Ellen Barry}

\begin{itemize}
\item
  May 9, 2020
\item
  \begin{itemize}
  \item
  \item
  \item
  \item
  \item
  \end{itemize}
\end{itemize}

The acrimonious split of Masha and Ronnie Shak ended up where many
divorces do these days --- on Facebook.

As the proceedings unfolded, Mr. Shak offered a running commentary on
social media, shared with the couple's rabbi, assistant rabbi and
members of their synagogue, court documents show.

He created a GoFundMe page entitled ``Help me KEEP MY SON.'' He called
his ex-wife an ``evil liar.'' He illustrated the posts with a video of
their one-year-old son, and told their friends to unfriend her.

That was until a probate court judge banned Mr. Shak from posting on
social media about his divorce, a common practice known as a
nondisparagement order.

A
\href{https://www.courtlistener.com/pdf/2020/05/07/shak_v._shak.pdf}{ruling
this week by the Massachusetts Supreme Judicial Cour}t, stemming from
the Shaks' divorce, found such bans to be unconstitutional, a decision
that could have broad implications in the state.

``As important as it is to protect a child from the emotional and
psychological harm that might follow from one parent's use of vulgar or
disparaging words about the other, merely reciting that interest is not
enough to satisfy the heavy burden'' of restricting speech, Justice
Kimberly S. Budd wrote
in\href{https://www.courtlistener.com/pdf/2020/05/07/shak_v._shak.pdf}{a
13-page ruling}.

Jennifer M. Lamanna, a lawyer who represented Mr. Shak in the appeal,
called the ruling a ``game-changer'' because family and probate judges
in the state frequently give such orders, and treat violations as
contempt of court, carrying severe penalties.

``There are thousands of these out there, which is why this is, for
Massachusetts purposes, a landmark ruling,'' she said. ``People ask for
them routinely and they are just handed out.''

She said the orders, used for decades to control disparaging speech,
have been expanded in recent years to focus on social media.

Under such orders, she said, ``my client could write a nasty letter to
everyone he knows, but he's not allowed to put it up on social media.
You can whisper in your synagogue, make nasty remarks about your
ex-wife, but you can't put it up on Facebook.''

Ms. Shak's attorney, Richard M. Novitch, said the ruling had an
immediate, negative effect, prompting Mr. Shak to resume his postings on
social media. ``Within the last 24 hours of the Shak case being issued
by the S.J.C., he's right back at it, blowing up on social media,'' he
said. ``There's nothing that stops him.''

While Mr. Novitch called the decision ``constitutionally sound,'' he
said that ``common sense would suggest that children should be insulated
from the combat between parents.''

``It will give license to a lot of bad actors to say what they want,
regardless of where and when and the circumstances,'' he said.

The case underscored the role social media can play in modern divorce,
as dueling parties try to win support from their circle of
acquaintances.

Shortly after filing for divorce and seeking to remove Mr. Shak from
their shared home, Ms. Shak filed a motion to prohibit him from posting
disparaging remarks about her on social media. Two family court judges
complied, with the second, George F. Phelan, issuing an order preventing
both Mr. and Ms. Shak from posting ``any disparagement of the other
party'' on social media until their son reached the age of 14.

Judge Phelan's ruling prevented both spouses from using four specific
expletives, as well as ``other pejoratives involving any gender,''
noting that ``the Court acknowledges the impossibility of listing herein
all of the opprobrious vitriol and their permutations within the human
lexicon.''

It also banned the
\href{https://www.nytimes.com/2020/05/10/us/coronavirus-giving-birth-new-mothers.html}{parents}
from posting photographs of their son in poses the judge considered
inappropriate.

``The court finds that the father's posing, taking and posting of the
photo of the parties' child (then less than one year old) with a
cigarette in his mouth was in poor taste, even if intended as a joke,
and causes the Court to question the father's maturity,'' the judge
wrote.

But Judge Phelan also put the order on hold, to be reviewed on
constitutional grounds by the Supreme Judicial Court. And this week, the
court found it unconstitutional.

An order preventing someone from carrying out a certain kind of speech,
known as ``prior restraint,'' is legal in the United States when the
threat of damage caused by that speech is compelling. But though the
state does have an interest in protecting children from ``being exposed
to disparagement between their parents,'' it is not grave enough to
justify restricting freedom of speech, the ruling said.

The ruling noted that one spouse, if offended by the other's speech, has
the option of suing for defamation or seeking a harassment prevention
order. It also noted that the judges' ruling does not apply to voluntary
nondisparagement agreements.

``What are people with common sense going to do? They're going to go out
in the hallway and reach an accord in which each agrees not to disparage
the other,'' said Mr. Novitch, Ms. Shak's attorney. ``It will be based
on the agreement of the parties, not on judicial fiat.''

Ruth A. Bourquin, a senior attorney from the American Civil Liberties
Union,
t\href{https://www.ma-appellatecourts.org/pdf/SJC-12748/SJC-12748_05_Amicus_ACLU_Brief.pdf}{he
co-author of an amicus brief supporting Mr. Shak}, said she was relieved
by the Massachusetts Supreme Judicial Court ruling. ``We're so grateful
that the S.J.C. reiterated the first amendment principles, and
recognized that they applied here,'' she said, comparing social media to
``the new town square.''

``That's what it is,'' she said. ``Just because it's bigger doesn't mean
we can say that the rights of free speech don't apply. Having a
government actor say you can say this, and not say that, is a somewhat
scary alternative.''

Advertisement

\protect\hyperlink{after-bottom}{Continue reading the main story}

\hypertarget{site-index}{%
\subsection{Site Index}\label{site-index}}

\hypertarget{site-information-navigation}{%
\subsection{Site Information
Navigation}\label{site-information-navigation}}

\begin{itemize}
\tightlist
\item
  \href{https://help.nytimes.com/hc/en-us/articles/115014792127-Copyright-notice}{©~2020~The
  New York Times Company}
\end{itemize}

\begin{itemize}
\tightlist
\item
  \href{https://www.nytco.com/}{NYTCo}
\item
  \href{https://help.nytimes.com/hc/en-us/articles/115015385887-Contact-Us}{Contact
  Us}
\item
  \href{https://www.nytco.com/careers/}{Work with us}
\item
  \href{https://nytmediakit.com/}{Advertise}
\item
  \href{http://www.tbrandstudio.com/}{T Brand Studio}
\item
  \href{https://www.nytimes.com/privacy/cookie-policy\#how-do-i-manage-trackers}{Your
  Ad Choices}
\item
  \href{https://www.nytimes.com/privacy}{Privacy}
\item
  \href{https://help.nytimes.com/hc/en-us/articles/115014893428-Terms-of-service}{Terms
  of Service}
\item
  \href{https://help.nytimes.com/hc/en-us/articles/115014893968-Terms-of-sale}{Terms
  of Sale}
\item
  \href{https://spiderbites.nytimes.com}{Site Map}
\item
  \href{https://help.nytimes.com/hc/en-us}{Help}
\item
  \href{https://www.nytimes.com/subscription?campaignId=37WXW}{Subscriptions}
\end{itemize}
