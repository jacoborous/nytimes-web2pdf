Sections

SEARCH

\protect\hyperlink{site-content}{Skip to
content}\protect\hyperlink{site-index}{Skip to site index}

\href{https://www.nytimes.com/section/science}{Science}

\href{https://myaccount.nytimes.com/auth/login?response_type=cookie\&client_id=vi}{}

\href{https://www.nytimes.com/section/todayspaper}{Today's Paper}

\href{/section/science}{Science}\textbar{}Dr. John Murray, Pioneering
Lung Expert, Dies at 92

\url{https://nyti.ms/3aBsVn6}

\begin{itemize}
\item
\item
\item
\item
\item
\end{itemize}

\href{https://www.nytimes.com/news-event/coronavirus?action=click\&pgtype=Article\&state=default\&region=TOP_BANNER\&context=storylines_menu}{The
Coronavirus Outbreak}

\begin{itemize}
\tightlist
\item
  live\href{https://www.nytimes.com/2020/08/03/world/coronavirus-covid-19.html?action=click\&pgtype=Article\&state=default\&region=TOP_BANNER\&context=storylines_menu}{Latest
  Updates}
\item
  \href{https://www.nytimes.com/interactive/2020/us/coronavirus-us-cases.html?action=click\&pgtype=Article\&state=default\&region=TOP_BANNER\&context=storylines_menu}{Maps
  and Cases}
\item
  \href{https://www.nytimes.com/interactive/2020/science/coronavirus-vaccine-tracker.html?action=click\&pgtype=Article\&state=default\&region=TOP_BANNER\&context=storylines_menu}{Vaccine
  Tracker}
\item
  \href{https://www.nytimes.com/2020/08/02/us/covid-college-reopening.html?action=click\&pgtype=Article\&state=default\&region=TOP_BANNER\&context=storylines_menu}{College
  Reopening}
\item
  \href{https://www.nytimes.com/live/2020/08/03/business/stock-market-today-coronavirus?action=click\&pgtype=Article\&state=default\&region=TOP_BANNER\&context=storylines_menu}{Economy}
\end{itemize}

Advertisement

\protect\hyperlink{after-top}{Continue reading the main story}

Supported by

\protect\hyperlink{after-sponsor}{Continue reading the main story}

Those we've lost

\hypertarget{dr-john-murray-pioneering-lung-expert-dies-at-92}{%
\section{Dr. John Murray, Pioneering Lung Expert, Dies at
92}\label{dr-john-murray-pioneering-lung-expert-dies-at-92}}

Internationally recognized in pulmonology, he helped define acute
respiratory distress syndrome, a condition that led to his death,
attributed to the coronavirus.

\includegraphics{https://static01.nyt.com/images/2020/04/03/obituaries/01MURRAY/01MURRAY-articleLarge.jpg?quality=75\&auto=webp\&disable=upscale}

\href{https://www.nytimes.com/by/john-schwartz}{\includegraphics{https://static01.nyt.com/images/2018/02/16/multimedia/author-john-schwartz/author-john-schwartz-thumbLarge.jpg}}

By \href{https://www.nytimes.com/by/john-schwartz}{John Schwartz}

\begin{itemize}
\item
  Published April 2, 2020Updated April 16, 2020
\item
  \begin{itemize}
  \item
  \item
  \item
  \item
  \item
  \end{itemize}
\end{itemize}

\emph{This obituary is part of a series about people who have died in
the coronavirus pandemic. Read about others}
\href{https://www.nytimes.com/series/people-who-have-died-of-the-coronavirus}{\emph{here}}\emph{.}

Dr. John Murray, an internationally recognized expert in pulmonology,
helped the medical world understand a deadly condition known as acute
respiratory distress syndrome.

On March 24, the condition he helped define led to his death. He was 92.

The University of California, San Francisco School of Medicine, where
Dr. Murray was a professor emeritus, said the cause of death, ``sadly
and ironically,'' was respiratory failure resulting from acute
respiratory distress syndrome caused by the novel coronavirus. He lived
in Paris for much of the year with his wife, the novelist Diane Johnson.

Dr. Murray served as chief of pulmonary and critical care from 1966 to
1989 at the institution now known as Zuckerberg San Francisco General
Hospital and Trauma Center. After retiring in 1994, he continued to work
as an attending physician in the intensive care unit and to teach.

The medical school's
\href{https://medschool.ucsf.edu/remembering-john-f-murray-md}{statement}
credited him with leaving ``indelible marks on the clinical practice of
pulmonary medicine, the process of selecting and training fellows in
pulmonary disease, and on lung disease research.''

When Dr. Murray began his career, pulmonology was largely focused on
tuberculosis. His research and promotion of specialty training expanded
the field to encompass a much wider range of diseases throughout the
body and their effects on the lungs, said Dr. Philip Hopewell, a
professor of medicine at the University of California, San Francisco.
That made Dr. Murray a ``bridging figure between the old breed of chest
physicians and the new, modern breed,'' he said.

Much of Dr. Murray's best-known research focused on pulmonary disease
and AIDS, which he encountered at San Francisco General in 1981, and on
defining acute respiratory distress syndrome.

``He played an outsized role in forming this new branch of medicine,
which is now carrying the brunt of the outbreak,'' said Dr. Courtney
Broaddus, a colleague and past chief of San Francisco General's
pulmonary division, referring to the coronavirus pandemic. Dr. Broaddus
edited the current edition of one of Dr. Murray's best-known works
within the profession, ``Murray \& Nadel's Textbook of Respiratory
Medicine.''

When Dr. Broaddus was about to begin her position as an attending
physician in the intensive care unit, she said, she went to Dr. Murray
for advice. ``He got up from his desk and took me to meet someone,'' she
said in an email. ``I expected it would be a respiratory therapist,
someone to show me details of the mechanical ventilators. To my
surprise, he wanted me to meet the social worker.''

Social workers tracked down the identities of patients who came in
without identification, found their families and helped coordinate care.
``This story captures for me how John led the team,'' she said.
``Everyone was valued.''

Dr. Robert M. Wachter, head of the department of medicine at the
University of California, San Francisco, said that as a resident he had
noticed Dr. Murray's distinguished appearance --- he always wore bow
ties --- in an environment in which ``other faculty wore Hawaiian print
shirts.''

Dr. Murray's son, Douglas, explained: ``He didn't want his ties to droop
onto the patients when he was bending over them.''

John Frederic Murray was born on June 8, 1927, in Mineola, N.Y., on Long
Island, to Frederic S. and Dorothy (Hanna) Murray. His ancestors on his
father's side had owned an estate on the East Side of Manhattan in the
section now known as Murray Hill.

His father, a cartoonist, moved the family to Los Angeles, where he
became best known for a Hollywood-focused syndicated comic strip,
``\href{https://www.comicskingdom.com/trending/blog/2014/05/22/ask-the-archivist-seein-stars}{Seein'
Stars},'' which he produced under the name Feg Murray. Dorothy Murray
was a homemaker.

Dr. Murray graduated from Stanford University in 1949 with a bachelor of
arts degree and from Stanford's medical school in 1953

He married Sarah Sherman in 1949. They divorced in 1967. He married Ms.
Johnson in 1969.

Along with Ms. Johnson, his survivors include two children from his
first marriage, Douglas and Elizabeth Murray; his stepsons Kevin and
Simon Johnson; his stepdaughters Darcy Tell and Amanda Johnson; and 14
grandchildren. Another son from the first marriage, James, died in 2018.

After residencies that took him to San Francisco General and Kings
County Hospital in Brooklyn, a research fellowship in London and
teaching positions at the University of California, Los Angeles, Dr.
Murray made his way back to San Francisco in 1966.

A hiker and outdoorsman for many years, he played tennis into his 80s
and participated with great enthusiasm in a monthly Shakespeare reading
group in Paris.

Dr. Murray contracted the coronavirus while he was in a weakened state
after undergoing radiation treatments for prostate cancer. In the
hospital, he had repeatedly asked about his blood oxygen levels before
slipping into a coma.

Along with his textbooks, he wrote books for a general audience,
including ``Intensive Care: A Doctor's Journal,'' originally published
in 2000, and ``How Aging Works: What Science Can Do About It'' (2015).

When nurses at Zuckerberg San Francisco General Hospital learned last
week that Dr. Murray was seriously ill, many clipped on bow ties in his
honor.

\emph{Marlise Simons contributed reporting.}

\href{https://www.nytimes.com/interactive/2020/obituaries/people-died-coronavirus-obituaries.html?action=click\&pgtype=Article\&state=default\&region=BELOW_MAIN_CONTENT\&context=covid_obits_promo}{}

\hypertarget{those-weve-lost}{%
\section{Those We've Lost}\label{those-weve-lost}}

The coronavirus pandemic has taken an incalculable death toll. This
series is designed to put names and faces to the numbers.

Read more

\includegraphics{https://static01.nyt.com/images/2020/07/30/obituaries/30Pedro/30Pedro-square640.jpg}

\hypertarget{bernaldina-josuxe9-pedro}{%
\section{Bernaldina José Pedro}\label{bernaldina-josuxe9-pedro}}

d. Boa Vista, Brazil

Leader among the Indigenous Macuxi

\includegraphics{https://static01.nyt.com/images/2020/07/31/obituaries/31Swing/merlin_175167783_8913bc90-0d64-43f3-a655-1bb1bf1601c9-square640.jpg}

\hypertarget{john-eric-swing}{%
\section{John Eric Swing}\label{john-eric-swing}}

d. Fountain Valley, Calif.

Champion of Filipino-Americans

\includegraphics{https://static01.nyt.com/images/2020/07/27/obituaries/27Victor/merlin_175001436_38b11f8e-227a-4e2c-9821-7618af9b2524-square640.jpg}

\hypertarget{victor-victor}{%
\section{Victor Victor}\label{victor-victor}}

d. Santo Domingo, Dominican Republic

Beloved musician of the Dominican Republic

\includegraphics{https://static01.nyt.com/images/2020/07/31/obituaries/31Negron/merlin_175160169_516322ae-fd23-4969-b6b2-193ced371105-square640.jpg}

\hypertarget{dr-eddie-negruxf3n}{%
\section{Dr. Eddie Negrón}\label{dr-eddie-negruxf3n}}

d. Fort Walton Beach, Fla.

Internist on Florida's Emerald Coast

\includegraphics{https://static01.nyt.com/images/2020/07/30/obituaries/30Dobson/merlin_175115928_f6b9271c-8f05-4fe1-a38a-5ca4a58f8935-square640.jpg}

\hypertarget{dobby-dobson}{%
\section{Dobby Dobson}\label{dobby-dobson}}

d. Coral Springs, Fla.

Jamaican singer and songwriter

\includegraphics{https://static01.nyt.com/images/2020/08/01/obituaries/28Gonzalez/merlin_175002771_beb57888-3951-409a-ae13-03a94b2e962e-square640.jpg}

\hypertarget{waldemar-gonzalez}{%
\section{Waldemar Gonzalez}\label{waldemar-gonzalez}}

d. White Plains, N.Y.

Teacher and social worker

Advertisement

\protect\hyperlink{after-bottom}{Continue reading the main story}

\hypertarget{site-index}{%
\subsection{Site Index}\label{site-index}}

\hypertarget{site-information-navigation}{%
\subsection{Site Information
Navigation}\label{site-information-navigation}}

\begin{itemize}
\tightlist
\item
  \href{https://help.nytimes.com/hc/en-us/articles/115014792127-Copyright-notice}{©~2020~The
  New York Times Company}
\end{itemize}

\begin{itemize}
\tightlist
\item
  \href{https://www.nytco.com/}{NYTCo}
\item
  \href{https://help.nytimes.com/hc/en-us/articles/115015385887-Contact-Us}{Contact
  Us}
\item
  \href{https://www.nytco.com/careers/}{Work with us}
\item
  \href{https://nytmediakit.com/}{Advertise}
\item
  \href{http://www.tbrandstudio.com/}{T Brand Studio}
\item
  \href{https://www.nytimes.com/privacy/cookie-policy\#how-do-i-manage-trackers}{Your
  Ad Choices}
\item
  \href{https://www.nytimes.com/privacy}{Privacy}
\item
  \href{https://help.nytimes.com/hc/en-us/articles/115014893428-Terms-of-service}{Terms
  of Service}
\item
  \href{https://help.nytimes.com/hc/en-us/articles/115014893968-Terms-of-sale}{Terms
  of Sale}
\item
  \href{https://spiderbites.nytimes.com}{Site Map}
\item
  \href{https://help.nytimes.com/hc/en-us}{Help}
\item
  \href{https://www.nytimes.com/subscription?campaignId=37WXW}{Subscriptions}
\end{itemize}
