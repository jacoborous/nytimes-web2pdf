Sections

SEARCH

\protect\hyperlink{site-content}{Skip to
content}\protect\hyperlink{site-index}{Skip to site index}

\href{https://www.nytimes.com/section/us}{U.S.}

\href{https://myaccount.nytimes.com/auth/login?response_type=cookie\&client_id=vi}{}

\href{https://www.nytimes.com/section/todayspaper}{Today's Paper}

\href{/section/us}{U.S.}\textbar{}Donald Kennedy, Who Led Stanford in
1980s, Dies at 88

\url{https://nyti.ms/3eMvEws}

\begin{itemize}
\item
\item
\item
\item
\item
\end{itemize}

\href{https://www.nytimes.com/news-event/coronavirus?action=click\&pgtype=Article\&state=default\&region=TOP_BANNER\&context=storylines_menu}{The
Coronavirus Outbreak}

\begin{itemize}
\tightlist
\item
  live\href{https://www.nytimes.com/2020/08/03/world/coronavirus-covid-19.html?action=click\&pgtype=Article\&state=default\&region=TOP_BANNER\&context=storylines_menu}{Latest
  Updates}
\item
  \href{https://www.nytimes.com/interactive/2020/us/coronavirus-us-cases.html?action=click\&pgtype=Article\&state=default\&region=TOP_BANNER\&context=storylines_menu}{Maps
  and Cases}
\item
  \href{https://www.nytimes.com/interactive/2020/science/coronavirus-vaccine-tracker.html?action=click\&pgtype=Article\&state=default\&region=TOP_BANNER\&context=storylines_menu}{Vaccine
  Tracker}
\item
  \href{https://www.nytimes.com/2020/08/02/us/covid-college-reopening.html?action=click\&pgtype=Article\&state=default\&region=TOP_BANNER\&context=storylines_menu}{College
  Reopening}
\item
  \href{https://www.nytimes.com/live/2020/08/03/business/stock-market-today-coronavirus?action=click\&pgtype=Article\&state=default\&region=TOP_BANNER\&context=storylines_menu}{Economy}
\end{itemize}

Advertisement

\protect\hyperlink{after-top}{Continue reading the main story}

Supported by

\protect\hyperlink{after-sponsor}{Continue reading the main story}

those we've lost

\hypertarget{donald-kennedy-who-led-stanford-in-1980s-dies-at-88}{%
\section{Donald Kennedy, Who Led Stanford in 1980s, Dies at
88}\label{donald-kennedy-who-led-stanford-in-1980s-dies-at-88}}

As president, he expanded the university and battled a government
inquiry into its research expenses. He also headed the F.D.A. under
President Carter.

\includegraphics{https://static01.nyt.com/images/2020/04/24/obituaries/24Kennedy1/merlin_171817122_902d640a-0bcf-4e10-8f8d-363b6fdda2cb-articleLarge.jpg?quality=75\&auto=webp\&disable=upscale}

\href{https://www.nytimes.com/by/sam-roberts}{\includegraphics{https://static01.nyt.com/images/2018/02/20/multimedia/author-sam-roberts/author-sam-roberts-thumbLarge.jpg}}

By \href{https://www.nytimes.com/by/sam-roberts}{Sam Roberts}

\begin{itemize}
\item
  Published April 23, 2020Updated April 27, 2020
\item
  \begin{itemize}
  \item
  \item
  \item
  \item
  \item
  \end{itemize}
\end{itemize}

Donald Kennedy, a neurobiologist who headed the Food and Drug
Administration before becoming president of Stanford University, where
he oversaw major expansions of its campus and curriculum and weathered a
crisis over research spending, died on April 21 in Redwood City, Calif.
He was 88.

His death, at a residential care facility, was caused by complications
of the new coronavirus, his wife, Robin Kennedy, said. He had suffered a
severe stroke in 2015.

\href{https://www.stanford.edu/}{Stanford} had been Dr. Kennedy's life
since 1960, when, not yet 30, he joined its faculty as an assistant
professor of biology. And except for a stint in the late 1970s as head
of the F.D.A. under President Jimmy Carter, he remained wedded to the
university, becoming provost and then president in 1980, beginning an
11-year tenure.

It was a productive one. During his presidency the university opened the
\href{http://shc.stanford.edu/}{Stanford Humanities Center} and campuses
in Oxford, England; Kyoto, Japan; and Washington; diversified the
Western culture curriculum; and raised \$1.2 billion in a five-year
centennial campaign, although by the end of the decade the university
was facing deficits.

His tenure also coincided with fiery debates over antiwar protests and
academic freedom by both professors and students, divestiture of the
university's holdings in companys doing business in South Africa, and
\$160 million in damage inflicted by the Loma Prieta Earthquake in 1989.

A would-be writer who had become a neurobiologist in college
adventitiously, Dr. Kennedy found his leadership under the microscope in
the early 1990s, when the university was accused --- and later cleared
--- of improperly billing the Navy for research expenses.

The accusations were aired by federal auditors and Representative
\href{https://www.nytimes.com/2019/02/07/us/politics/john-dingell-dead-longest-congressman.html}{John
D. Dingell Jr.,} a tenacious Michigan Democrat, who said that Stanford
may have billed the government for as much as \$200 million in improper
expenses on research contracts for over a decade.

By 1994, Stanford had agreed that a total of about \$3 million had been
inadvertently billed to the government, but the
\href{https://www.nytimes.com/1994/10/19/us/navy-settles-a-fraud-case-on-stanford-research-costs.html}{federal
auditors concluded} that there was no evidence of misrepresentation by
the university.

Still, the damage was done to Stanford's reputation, and Dr. Kennedy
resigned in 1991, attributing the government accusations to political
and personal vendettas and acknowledging that they had contributed to
his decision to step down.

``It is very difficult, I have concluded, for a person identified with a
problem to be the spokesman for its solution,'' he said in announcing
\href{https://www.nytimes.com/1991/07/30/us/stanford-chief-quits-amid-furor-on-use-of-federal-money.html}{his
resignation}. He went on to edit the journal Science.

But he had his ardent supporters on the Bay Area campus, where he was
known to bike to work and engage with students. Among them was his
protégé Cory Booker, the future senator from New Jersey, whom Dr.
Kennedy had encouraged to apply for a Rhodes Scholarship.

\includegraphics{https://static01.nyt.com/images/2020/04/24/obituaries/24Kennedy2/merlin_171819561_33f5ac34-bf08-4596-9f86-6628e0c91477-articleLarge.jpg?quality=75\&auto=webp\&disable=upscale}

``To watch him lead through the indirect cost crisis, through
professional and personal attacks, under tremendous stress and strain,
with clouds amassed over his head and challenges raining on him,"
Senator Booker wrote in the foreword to Dr. Kennedy's memoir, ``A Place
in the Sun'' (2017), ``was a study in leadership, character, and
discipline, always better shown in times of crisis than when all is
going well.''

Donald Kennedy was born on Aug. 18, 1931, in Manhattan to William and
Barbara (Bean) Kennedy. His father was a writer, an editor and an
assistant dean of the Harvard Business School. His mother was a teacher
and journalist.

As his father repeatedly switched jobs, Donald was raised in about a
half-dozen locales, including Greenwich, Conn., by the time he was 15.

After graduating from the Dublin School in New Hampshire, he enrolled in
Harvard University intending to major in English and be a writer; at one
point he received an A on a 5,000-word final paper in creative writing.
But, as he recalled in his memoir, his professor, perhaps pointing him
toward a more profitable profession, asked him over sherry one night: ``
Tell me, Don. What else interests you?'''

``Surprised by the question, I gathered my wits and responded, `Well,
biology and natural history, I guess.'''

```Biology,' he said. `That sounds like a wonderful choice.'''

He earned is bachelor's degree in 1952, followed by a master's and a
doctorate, all three from Harvard. And all in biology.

Dr. Kennedy was recruited to the F.D.A. in 1977 by Joseph A. Califano
Jr., the secretary of health, education and welfare. During his tenure
there the agency's proposed ban on saccharin, the artificial sweetener,
was defeated, but overall his record won plaudits from industry
representatives and consumer advocates alike.

Image

Dr. Kennedy was a familiar, and often informal, presence on the Stanford
campus.~Credit...Stanford News Service

He returned to Stanford briefly as provost before he was named
president.

Dr. Kennedy was a familiar presence on campus, not only biking to the
quadrangle but also inviting students to join him on his morning runs up
to the Dish, the radio antenna in the foothills of the campus.

``Kennedy is not someone whom students hear once when they arrive and
then once when they graduate,'' The Stanford Daily, the student
newspaper, editorialized in 1991.

A former student, Ingrid Schwontes Jackoway, was quoted as saying in an
alumni publication: ``I will never forget Donald Kennedy getting up on
the lab table at the front of the lecture hall and assuming a quadruped
position to demonstrate to us the concepts of dorsal, ventral, cephalo
and caudal. His first concern was always with teaching effectively, not
preserving his dignity.''

Dr. Kennedy's marriage to Jeanne Dewey ended in divorce. In addition to
his wife, Robin Hamill, who was associate counsel at Stanford when they
married in 1987, he is survived by two daughters from his first
marriage, Page Kennedy Rochon and Julia Kennedy Tussing; two
stepchildren, Cameron Kennedy and Jamie Hamill; and nine grandchildren.
A brother, Dorsey, died before him.

Dr. Kennedy was the editor in chief of Science, the weekly journal of
the American Association for the Advancement of Science, from 2000 to
2008. But even there he was not immune to controversy. Researchers had
fabricated their findings in several articles, and a reported sighting
of an extinct ivory-billed woodpecker appeared to have been mistaken.

Among his other books were ``The Cold and the Dark: The World After
Nuclear War'' (1984) with Carl Sagan and Paul R. Ehrlich, and
\href{https://www.nytimes.com/1998/01/04/books/beyond-the-culture-wars.html}{``Academic
Duty''} (1997). At his death he was Bing professor for environmental
science emeritus at Stanford.

Shortly after he became president, Dr. Kennedy told the student radio
station, KZSU, that he intended to keep his perspective despite the
pressures of the job.

``The president is ultimately the person to whom the problems come,'' he
said. ``What you need then is to walk around, or visit a dormitory, or
to give a class, or to meet a student who wants to come in and talk
about a career choice. I find those occasions very uplifting because
they're not automatically negative. They're not the kind of problems
that are programmed for the president's desk because they haven't been
solved by anybody else. Instead, they're the kinds of things that go on
around here day by day, and make this a terrific place.''

\href{https://www.nytimes.com/interactive/2020/obituaries/people-died-coronavirus-obituaries.html?action=click\&pgtype=Article\&state=default\&region=BELOW_MAIN_CONTENT\&context=covid_obits_promo}{}

\hypertarget{those-weve-lost}{%
\section{Those We've Lost}\label{those-weve-lost}}

The coronavirus pandemic has taken an incalculable death toll. This
series is designed to put names and faces to the numbers.

Read more

\includegraphics{https://static01.nyt.com/images/2020/07/30/obituaries/30Pedro/30Pedro-square640.jpg}

\hypertarget{bernaldina-josuxe9-pedro}{%
\section{Bernaldina José Pedro}\label{bernaldina-josuxe9-pedro}}

d. Boa Vista, Brazil

Leader among the Indigenous Macuxi

\includegraphics{https://static01.nyt.com/images/2020/07/31/obituaries/31Swing/merlin_175167783_8913bc90-0d64-43f3-a655-1bb1bf1601c9-square640.jpg}

\hypertarget{john-eric-swing}{%
\section{John Eric Swing}\label{john-eric-swing}}

d. Fountain Valley, Calif.

Champion of Filipino-Americans

\includegraphics{https://static01.nyt.com/images/2020/07/27/obituaries/27Victor/merlin_175001436_38b11f8e-227a-4e2c-9821-7618af9b2524-square640.jpg}

\hypertarget{victor-victor}{%
\section{Victor Victor}\label{victor-victor}}

d. Santo Domingo, Dominican Republic

Beloved musician of the Dominican Republic

\includegraphics{https://static01.nyt.com/images/2020/07/31/obituaries/31Negron/merlin_175160169_516322ae-fd23-4969-b6b2-193ced371105-square640.jpg}

\hypertarget{dr-eddie-negruxf3n}{%
\section{Dr. Eddie Negrón}\label{dr-eddie-negruxf3n}}

d. Fort Walton Beach, Fla.

Internist on Florida's Emerald Coast

\includegraphics{https://static01.nyt.com/images/2020/07/30/obituaries/30Dobson/merlin_175115928_f6b9271c-8f05-4fe1-a38a-5ca4a58f8935-square640.jpg}

\hypertarget{dobby-dobson}{%
\section{Dobby Dobson}\label{dobby-dobson}}

d. Coral Springs, Fla.

Jamaican singer and songwriter

\includegraphics{https://static01.nyt.com/images/2020/08/01/obituaries/28Gonzalez/merlin_175002771_beb57888-3951-409a-ae13-03a94b2e962e-square640.jpg}

\hypertarget{waldemar-gonzalez}{%
\section{Waldemar Gonzalez}\label{waldemar-gonzalez}}

d. White Plains, N.Y.

Teacher and social worker

Advertisement

\protect\hyperlink{after-bottom}{Continue reading the main story}

\hypertarget{site-index}{%
\subsection{Site Index}\label{site-index}}

\hypertarget{site-information-navigation}{%
\subsection{Site Information
Navigation}\label{site-information-navigation}}

\begin{itemize}
\tightlist
\item
  \href{https://help.nytimes.com/hc/en-us/articles/115014792127-Copyright-notice}{©~2020~The
  New York Times Company}
\end{itemize}

\begin{itemize}
\tightlist
\item
  \href{https://www.nytco.com/}{NYTCo}
\item
  \href{https://help.nytimes.com/hc/en-us/articles/115015385887-Contact-Us}{Contact
  Us}
\item
  \href{https://www.nytco.com/careers/}{Work with us}
\item
  \href{https://nytmediakit.com/}{Advertise}
\item
  \href{http://www.tbrandstudio.com/}{T Brand Studio}
\item
  \href{https://www.nytimes.com/privacy/cookie-policy\#how-do-i-manage-trackers}{Your
  Ad Choices}
\item
  \href{https://www.nytimes.com/privacy}{Privacy}
\item
  \href{https://help.nytimes.com/hc/en-us/articles/115014893428-Terms-of-service}{Terms
  of Service}
\item
  \href{https://help.nytimes.com/hc/en-us/articles/115014893968-Terms-of-sale}{Terms
  of Sale}
\item
  \href{https://spiderbites.nytimes.com}{Site Map}
\item
  \href{https://help.nytimes.com/hc/en-us}{Help}
\item
  \href{https://www.nytimes.com/subscription?campaignId=37WXW}{Subscriptions}
\end{itemize}
