Sections

SEARCH

\protect\hyperlink{site-content}{Skip to
content}\protect\hyperlink{site-index}{Skip to site index}

\href{https://www.nytimes.com/section/arts/design}{Art \& Design}

\href{https://myaccount.nytimes.com/auth/login?response_type=cookie\&client_id=vi}{}

\href{https://www.nytimes.com/section/todayspaper}{Today's Paper}

\href{/section/arts/design}{Art \& Design}\textbar{}Michael McKinnell,
84, Dies; Architect of a Monumental City Hall

\url{https://nyti.ms/2V1y7Kn}

\begin{itemize}
\item
\item
\item
\item
\item
\item
\end{itemize}

\href{https://www.nytimes.com/news-event/coronavirus?action=click\&pgtype=Article\&state=default\&region=TOP_BANNER\&context=storylines_menu}{The
Coronavirus Outbreak}

\begin{itemize}
\tightlist
\item
  live\href{https://www.nytimes.com/2020/08/03/world/coronavirus-covid-19.html?action=click\&pgtype=Article\&state=default\&region=TOP_BANNER\&context=storylines_menu}{Latest
  Updates}
\item
  \href{https://www.nytimes.com/interactive/2020/us/coronavirus-us-cases.html?action=click\&pgtype=Article\&state=default\&region=TOP_BANNER\&context=storylines_menu}{Maps
  and Cases}
\item
  \href{https://www.nytimes.com/interactive/2020/science/coronavirus-vaccine-tracker.html?action=click\&pgtype=Article\&state=default\&region=TOP_BANNER\&context=storylines_menu}{Vaccine
  Tracker}
\item
  \href{https://www.nytimes.com/2020/08/02/us/covid-college-reopening.html?action=click\&pgtype=Article\&state=default\&region=TOP_BANNER\&context=storylines_menu}{College
  Reopening}
\item
  \href{https://www.nytimes.com/live/2020/08/03/business/stock-market-today-coronavirus?action=click\&pgtype=Article\&state=default\&region=TOP_BANNER\&context=storylines_menu}{Economy}
\end{itemize}

Advertisement

\protect\hyperlink{after-top}{Continue reading the main story}

Supported by

\protect\hyperlink{after-sponsor}{Continue reading the main story}

those we've lost

\hypertarget{michael-mckinnell-84-dies-architect-of-a-monumental-city-hall}{%
\section{Michael McKinnell, 84, Dies; Architect of a Monumental City
Hall}\label{michael-mckinnell-84-dies-architect-of-a-monumental-city-hall}}

His and Gerhard Kallmann's sculptural and public-minded design for
Boston helped spur the cityscape's revival in the late 1960s. He died of
the coronavirus.

\includegraphics{https://static01.nyt.com/images/2020/04/06/obituaries/04mckinnell-virus-lost-image2/04mckinnell-virus-lost-image2-articleLarge.jpg?quality=75\&auto=webp\&disable=upscale}

By Joseph Giovannini

\begin{itemize}
\item
  Published April 4, 2020Updated April 16, 2020
\item
  \begin{itemize}
  \item
  \item
  \item
  \item
  \item
  \item
  \end{itemize}
\end{itemize}

\emph{This obituary is part of a series about people who have died in
the coronavirus pandemic. Read about others}
\href{https://www.nytimes.com/series/people-who-have-died-of-the-coronavirus}{\emph{here}}\emph{.}

Michael McKinnell, whose heroically sculptural and democratically open
design for Boston City Hall catalyzed the city's urban revival in the
late 1960s and embodied the era's idealism and civic activism, died on
March 27 in Beverly, Mass. He was 84.

\includegraphics{https://static01.nyt.com/images/2020/04/06/obituaries/01McKINNELL1/merlin_171111573_ccc42cad-a96c-4a19-812e-5bf5fcb08efb-articleLarge.jpg?quality=75\&auto=webp\&disable=upscale}

His wife and architectural partner, Stephanie Mallis, said the cause was
the coronavirus.

In 1962, the British-born Mr. McKinnell was a 26-year-old graduate
student in architecture at Columbia University working as a teaching
assistant to the German-born architect
\href{https://www.nytimes.com/2012/06/25/arts/design/gerhard-kallmann-architect-dies-at-97.html}{Gerhard
Kallmann} when, almost on a lark, the two entered a competition to
design a new Boston City Hall. Vying with 255 other submissions, they
won.

``They were as amazed as anyone that they prevailed,'' Ms. Mallis said.
Neither had ever built anything.

Except for fellow architects, few Americans had seen anything quite like
the winning proposal --- certainly not the city's mayor, John F.
Collins, who reportedly exclaimed on first seeing the design, ``What the
{[}expletive{]} is that?''

It wasn't the proper government structure of Boston's staid red-brick
tradition; rather, it was a proudly monumental building that would
command the vast plaza of the new Government Center complex with
thunderous authority.

Image

Mr. McKinnell said of the massive concrete Boston City Hall that if the
architects had their way, ``We would have used concrete to make the
light switches.''Credit...David L Ryan/The Boston Globe, via Getty
Images

What Mayor Collins saw was the exposed musculature of mighty concrete
piers supporting a massive cornice line of offices. The building was to
stand at the edge of a fanning plaza whose bricks entered the ground
floor of the structure like a carpet, running up three flights to the
City Council chambers.

Spaces and forms interpenetrated. Sculptural concrete projections that
housed the chambers and the mayor's office protruded from a modular
facade of offices. A brick amphitheater of stairs on the ground floor
accommodated gatherings of citizens, even spontaneous sit-ins; the vast
plaza in front, inspired by the Campo of Siena, Italy, anticipated the
thousands of protesters of those riotous times.

It was a benevolent structure that took the side of the people,
guaranteeing citizens free access through porous perimeters in that
cradle of American democracy.

Twenty years Mr. McKinnell's senior, Mr. Kallmann was best known for
publishing an essay about two divergent trends in architecture: one in
which buildings were composed rigorously within a controlling grid; the
other, propelled by French and British Brutalists, in which designs in
concrete embraced sometimes violent breakout forms. The
Kallmann-McKinnell design synthesized the two approaches, the rational
and the complex.

Both men had been educated in British architecture schools that taught
that architecture's mission was to build well in the service of a moral
and social purpose. And both had been imbued with the belief that
concrete was, as one theorist said, ``the stone of our time.''

Built in concrete and brick, their robust City Hall design implicitly
critiqued the thin commercial glass-and-steel buildings then chilling
cities across America. It was a statement of protest against what Mr.
McKinnell called the ``degenerate frippery and surface concerns'' of
``cosmetic'' architecture.

Its deliberate monumentality expressed a generosity toward the public
rather than a corporate idea of minimal expense, and its exposed
concrete surfaces gave the building a sense of authenticity --- an ``all
thoroughness,'' Mr. McKinnell said in an interview, admitting that had
they had their way, ``we would have used concrete to make the light
switches.''

Image

The City Hall building commands the vast plaza of Boston's Government
Center complex.Credit...John Tlumacki/The Boston Globe via Getty Images

Like the Sydney Opera House, the structure landed on postcards, the two
men's design becoming a symbol of the new Boston. And though it proved
controversial with some members of the public, who found the rugged
materials and bold forms confrontational, it was, for most architects
and critics, the masterpiece that would hover over the rest of Mr.
McKinnell's nearly 60-year career.

The two architects quickly won a half-dozen commissions to design a
series of buildings in and around Boston, all built in concrete. Both
received appointments to the faculty of Harvard's Graduate School of
Design.

Their careers were inextricably linked. They even sat opposite each
other at a partners desk, passing drawings back and forth as they
conceptualized and developed projects.

Mr. Kallmann was more the theorist: He talked about architecture. Mr.
McKinnell was interested in materials, in how parts fit: He talked about
the building. He always carried a six-inch ruler in his pocket,
explaining that architecture was about measure and scaling the building
to the human body.

Their emphasis on constructional rigor would earn Mr. Kallmann and Mr.
McKinnell the sobriquet ``Column and Mechanical'' in one of their
Harvard classes.

But by 1974, as concrete grew unpopular and a national recession took
hold, Mr. McKinnell and Mr. Kallmann found themselves out of work,
waiting for the phone to ring in a garret office on Tremont Street
overlooking Boston Common. They thought of returning to England.

At the same time, the softer architectural blandishments of
postmodernism, promising the comforts of architectural history and
tradition, were challenging modernists.

Image

The American Academy of Arts and Sciences in Cambridge, Mass., designed
by the firm Kallmann McKinnell \& Wood, evoked the
turn-of-the-20th-century architecture of Frank Lloyd Wright and
others.Credit...Wendy Maeda/The Boston Globe via Getty Images

In 1978, the second act for Kallmann McKinnell \& Wood, as their firm
was then called, arrived in the form of a commission for the American
Academy of Arts and Sciences, to be built on a hill in residential
Cambridge, Mass. Its directors told the architects, ``Not one square
inch of concrete, inside or out.''

The commission was for a house for academics, a grove of academia, and
having learned the lesson of survival --- that they should design not
for an audience of architects but for clients and the public --- Mr.
Kallmann and Mr. McKinnell embarked in a new direction that would last
the rest of their careers. Mr. Kallmann died in 2012.

They built the Academy with a sloping roof, like the houses around it,
and they consulted history books to evoke the turn-of-the-20th-century
architecture of Frank Lloyd Wright, the brothers Charles Sumner Greene
and Henry Mather Greene, and others who used natural materials like wood
and stone. For the Academy, City Hall's arresting vertical lines were
tranquilized into calming horizontals. And Mr. McKinnell relaxed the
interiors, realizing that a space for artists, philosophers and
historians needed to have ``talking places'' --- nooks and crannies for
informal conversation.

The critic Ada Louise Huxtable of The New York Times
\href{https://www.nytimes.com/1981/09/20/arts/architecture-view-classical-clarity-in-an-academic-design-cambridge-mass.html}{wrote
approvingly} of the project. ``The architects have felt free to find
their answers in terms of existing building types and historical
precedents,'' she wrote, ``rather than through the invention of new
forms. The design goes beyond modernism.''

But it was a period of style wars in architecture, and modernists
believed Mr. McKinnell and Mr. Kallmann had stepped onto a slippery
architectural slope. Indeed, years of postmodernist buildings followed
in their portfolio, designs that indulged historicism and traditions
less probingly than in the Academy building.

``Yes, it was possible to have the rigor and systematic nature of
modernism together with some symbolic and representational elements,''
Andrea Leers, a friend and teaching colleague of Mr. McKinnell's, said
in an interview. ``But to most of us, it was a kind of capitulation.''

If the firm lost the attention of modernists, it gained the support of a
larger audience, and Kallmann McKinnell \& Wood went on to add scores of
employees and to attract commissions to design embassies, academic
buildings and museums around the world.

Noel Michael McKinnell was born on Dec. 25, 1935, in Salford,
Manchester, England, to Ronald and Marguerite McKinnell. His father was
an accountant who had fought in both world wars; his mother was a
homemaker.

Mr. McKinnell's marriage to Jane D'Espo in 1961 ended in divorce.
Besides Ms. Mallis, whom he married in 2003, he is survived by two
daughters from his first marriage, Caitlin McKinnell and Phoebe
McKinnell; a sister, Sheila Sharman; and four grandchildren.

After Mr. McKinnell left his firm several years ago and moved to
Rockport, Mass., he and Ms. Mallis continued to design buildings,
completing institutional projects in Israel. They also both painted.

He remained the dedicated architect and builder even on his deathbed.
After he refused life support in the hospital in Beverly, he spoke by
telephone with his wife, herself isolated at home because of the
coronavirus. He described to her the design for a grave site, to be
created in their backyard garden in Rockport overlooking the sea. It
would be a simple square of white roses, his favorite flower, planted in
a lower terrace on an axis with another flower bed on the upper one.

As he described it, she drew the design. The gravestone itself, carved
with both their names, would be granite, not concrete.

\href{https://www.nytimes.com/interactive/2020/obituaries/people-died-coronavirus-obituaries.html?action=click\&pgtype=Article\&state=default\&region=BELOW_MAIN_CONTENT\&context=covid_obits_promo}{}

\hypertarget{those-weve-lost}{%
\section{Those We've Lost}\label{those-weve-lost}}

The coronavirus pandemic has taken an incalculable death toll. This
series is designed to put names and faces to the numbers.

Read more

\includegraphics{https://static01.nyt.com/images/2020/07/30/obituaries/30Pedro/30Pedro-square640.jpg}

\hypertarget{bernaldina-josuxe9-pedro}{%
\section{Bernaldina José Pedro}\label{bernaldina-josuxe9-pedro}}

d. Boa Vista, Brazil

Leader among the Indigenous Macuxi

\includegraphics{https://static01.nyt.com/images/2020/07/31/obituaries/31Swing/merlin_175167783_8913bc90-0d64-43f3-a655-1bb1bf1601c9-square640.jpg}

\hypertarget{john-eric-swing}{%
\section{John Eric Swing}\label{john-eric-swing}}

d. Fountain Valley, Calif.

Champion of Filipino-Americans

\includegraphics{https://static01.nyt.com/images/2020/07/27/obituaries/27Victor/merlin_175001436_38b11f8e-227a-4e2c-9821-7618af9b2524-square640.jpg}

\hypertarget{victor-victor}{%
\section{Victor Victor}\label{victor-victor}}

d. Santo Domingo, Dominican Republic

Beloved musician of the Dominican Republic

\includegraphics{https://static01.nyt.com/images/2020/07/31/obituaries/31Negron/merlin_175160169_516322ae-fd23-4969-b6b2-193ced371105-square640.jpg}

\hypertarget{dr-eddie-negruxf3n}{%
\section{Dr. Eddie Negrón}\label{dr-eddie-negruxf3n}}

d. Fort Walton Beach, Fla.

Internist on Florida's Emerald Coast

\includegraphics{https://static01.nyt.com/images/2020/07/30/obituaries/30Dobson/merlin_175115928_f6b9271c-8f05-4fe1-a38a-5ca4a58f8935-square640.jpg}

\hypertarget{dobby-dobson}{%
\section{Dobby Dobson}\label{dobby-dobson}}

d. Coral Springs, Fla.

Jamaican singer and songwriter

\includegraphics{https://static01.nyt.com/images/2020/08/01/obituaries/28Gonzalez/merlin_175002771_beb57888-3951-409a-ae13-03a94b2e962e-square640.jpg}

\hypertarget{waldemar-gonzalez}{%
\section{Waldemar Gonzalez}\label{waldemar-gonzalez}}

d. White Plains, N.Y.

Teacher and social worker

Advertisement

\protect\hyperlink{after-bottom}{Continue reading the main story}

\hypertarget{site-index}{%
\subsection{Site Index}\label{site-index}}

\hypertarget{site-information-navigation}{%
\subsection{Site Information
Navigation}\label{site-information-navigation}}

\begin{itemize}
\tightlist
\item
  \href{https://help.nytimes.com/hc/en-us/articles/115014792127-Copyright-notice}{©~2020~The
  New York Times Company}
\end{itemize}

\begin{itemize}
\tightlist
\item
  \href{https://www.nytco.com/}{NYTCo}
\item
  \href{https://help.nytimes.com/hc/en-us/articles/115015385887-Contact-Us}{Contact
  Us}
\item
  \href{https://www.nytco.com/careers/}{Work with us}
\item
  \href{https://nytmediakit.com/}{Advertise}
\item
  \href{http://www.tbrandstudio.com/}{T Brand Studio}
\item
  \href{https://www.nytimes.com/privacy/cookie-policy\#how-do-i-manage-trackers}{Your
  Ad Choices}
\item
  \href{https://www.nytimes.com/privacy}{Privacy}
\item
  \href{https://help.nytimes.com/hc/en-us/articles/115014893428-Terms-of-service}{Terms
  of Service}
\item
  \href{https://help.nytimes.com/hc/en-us/articles/115014893968-Terms-of-sale}{Terms
  of Sale}
\item
  \href{https://spiderbites.nytimes.com}{Site Map}
\item
  \href{https://help.nytimes.com/hc/en-us}{Help}
\item
  \href{https://www.nytimes.com/subscription?campaignId=37WXW}{Subscriptions}
\end{itemize}
