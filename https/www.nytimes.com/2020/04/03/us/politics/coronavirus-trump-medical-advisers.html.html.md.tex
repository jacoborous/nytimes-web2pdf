Sections

SEARCH

\protect\hyperlink{site-content}{Skip to
content}\protect\hyperlink{site-index}{Skip to site index}

\href{https://www.nytimes.com/section/politics}{Politics}

\href{https://myaccount.nytimes.com/auth/login?response_type=cookie\&client_id=vi}{}

\href{https://www.nytimes.com/section/todayspaper}{Today's Paper}

\href{/section/politics}{Politics}\textbar{}Tensions Persist Between
Trump and Medical Advisers Over the Coronavirus

\url{https://nyti.ms/34aSdpC}

\begin{itemize}
\item
\item
\item
\item
\item
\end{itemize}

\href{https://www.nytimes.com/news-event/coronavirus?action=click\&pgtype=Article\&state=default\&region=TOP_BANNER\&context=storylines_menu}{The
Coronavirus Outbreak}

\begin{itemize}
\tightlist
\item
  live\href{https://www.nytimes.com/2020/08/01/world/coronavirus-covid-19.html?action=click\&pgtype=Article\&state=default\&region=TOP_BANNER\&context=storylines_menu}{Latest
  Updates}
\item
  \href{https://www.nytimes.com/interactive/2020/us/coronavirus-us-cases.html?action=click\&pgtype=Article\&state=default\&region=TOP_BANNER\&context=storylines_menu}{Maps
  and Cases}
\item
  \href{https://www.nytimes.com/interactive/2020/science/coronavirus-vaccine-tracker.html?action=click\&pgtype=Article\&state=default\&region=TOP_BANNER\&context=storylines_menu}{Vaccine
  Tracker}
\item
  \href{https://www.nytimes.com/interactive/2020/07/29/us/schools-reopening-coronavirus.html?action=click\&pgtype=Article\&state=default\&region=TOP_BANNER\&context=storylines_menu}{What
  School May Look Like}
\item
  \href{https://www.nytimes.com/live/2020/07/31/business/stock-market-today-coronavirus?action=click\&pgtype=Article\&state=default\&region=TOP_BANNER\&context=storylines_menu}{Economy}
\end{itemize}

Advertisement

\protect\hyperlink{after-top}{Continue reading the main story}

Supported by

\protect\hyperlink{after-sponsor}{Continue reading the main story}

\hypertarget{tensions-persist-between-trump-and-medical-advisers-over-the-coronavirus}{%
\section{Tensions Persist Between Trump and Medical Advisers Over the
Coronavirus}\label{tensions-persist-between-trump-and-medical-advisers-over-the-coronavirus}}

The president's public refusal to wear a mask was the latest way he has
cast doubt on their recommendations.

\includegraphics{https://static01.nyt.com/images/2020/04/03/us/politics/03dc-virus-trump1/merlin_171264774_d842ebd0-d2c8-460c-b6cb-86316fe28fee-articleLarge.jpg?quality=75\&auto=webp\&disable=upscale}

By \href{https://www.nytimes.com/by/peter-baker}{Peter Baker},
\href{https://www.nytimes.com/by/maggie-haberman}{Maggie Haberman} and
\href{https://www.nytimes.com/by/james-glanz}{James Glanz}

\begin{itemize}
\item
  April 3, 2020
\item
  \begin{itemize}
  \item
  \item
  \item
  \item
  \item
  \end{itemize}
\end{itemize}

WASHINGTON --- Rarely has the schism between President Trump and his own
public health advisers over the coronavirus pandemic been put on display
quite so starkly. Even as he announced a new federal recommendation on
Friday that Americans wear masks when out in public, he immediately
disavowed it: ``I am choosing not to do it.''

The striking dichotomy underscored how often Mr. Trump has been at odds
with the medical experts seeking to guide his handling of the outbreak
as well as some of the governors fighting it on the front lines, despite
his move to extend social distancing guidelines through April 30 and his
acknowledgment that the death toll could be staggering.

While the health specialists and some governors press for a more
aggressive, uniform national approach to the virus, the president has
resisted expanding limits on daily life and sought to shift blame to the
states for being unprepared to deal with the coronavirus. While they
sound the alarm and call for more federal action, Mr. Trump has
deflected responsibility and left it to others to take a more aggressive
stance.

Some of the president's health advisers in recent days have argued that
restrictions on social interaction and economic activity that have shut
down much of the nation need to be expanded to all 50 states and that
more Americans need to adopt them. Mr. Trump, by contrast, has
characterized the crisis as generally limited to hot spots like New
York, California and Michigan and has expressed no support for a
nationwide lockdown. ``I would leave it to the governors,'' he said on
Friday.

As hospitals cope with shortages of medical equipment, the
administration on Friday also rewrote the federal government's stated
mission for its stockpile of supplies to make clear that it sees itself
as playing a secondary role to the states. Where the federal government
once said the stockpile ``ensures that the right medicines and supplies
get to those who need the most,'' the revised version said the federal
stockpile's role was merely to ``supplement state and local supplies.''

The tension over the scale of the federal response comes as the
president defends his administration's reaction to the pandemic that has
now infected more than 270,000 people in the United States and has
\href{https://www.nytimes.com/interactive/2020/us/coronavirus-us-cases.html}{killed
more than 7,000}. New polls showed that public support for Mr. Trump's
handling of the crisis has begun to slip, a worrisome development for a
president seeking re-election in the fall.

Mr. Trump's decision to take a back seat to the states by leaving it to
them to decide whether to shut down public life and insisting they take
the lead in addressing shortages amounts to a remarkable deference by a
president who typically makes himself the center of the action. It also
contrasts with his own self-description as a wartime president leading a
great battle against an invisible enemy.

It underscores both pragmatic and political imperatives for Mr. Trump,
reflecting a traditional federalist approach that eschews imposing a
one-size-fits-all national standard on states. But it also shows the
president's desire to blame the governors rather than accept any
responsibility for shortages of ventilators, masks and other critical
supplies.

The most fundamental point of conflict centers over how broadly the
virtual lockdown of many states in the Midwest and on the East and West
Coasts should be expanded. Dr. Anthony S. Fauci, the director of the
National Institute of Allergy and Infectious Diseases, said stay-at-home
orders should be extended to the entire nation.

``I don't understand why that's not happening,''
\href{https://www.cnn.com/2020/04/02/politics/fauci-stay-home-coronavirus-states-cnntv/index.html}{Dr.
Fauci said Thursday night on CNN}. ``The tension between federally
mandated versus states' rights to do what they want is something I don't
want to get into. But if you look at what is going on in this country, I
don't understand why we're not doing that. We really should be.''

\hypertarget{latest-updates-global-coronavirus-outbreak}{%
\section{\texorpdfstring{\href{https://www.nytimes.com/2020/08/01/world/coronavirus-covid-19.html?action=click\&pgtype=Article\&state=default\&region=MAIN_CONTENT_1\&context=storylines_live_updates}{Latest
Updates: Global Coronavirus
Outbreak}}{Latest Updates: Global Coronavirus Outbreak}}\label{latest-updates-global-coronavirus-outbreak}}

Updated 2020-08-02T07:42:09.613Z

\begin{itemize}
\tightlist
\item
  \href{https://www.nytimes.com/2020/08/01/world/coronavirus-covid-19.html?action=click\&pgtype=Article\&state=default\&region=MAIN_CONTENT_1\&context=storylines_live_updates\#link-34047410}{The
  U.S. reels as July cases more than double the total of any other
  month.}
\item
  \href{https://www.nytimes.com/2020/08/01/world/coronavirus-covid-19.html?action=click\&pgtype=Article\&state=default\&region=MAIN_CONTENT_1\&context=storylines_live_updates\#link-780ec966}{Top
  U.S. officials work to break an impasse over the federal jobless
  benefit.}
\item
  \href{https://www.nytimes.com/2020/08/01/world/coronavirus-covid-19.html?action=click\&pgtype=Article\&state=default\&region=MAIN_CONTENT_1\&context=storylines_live_updates\#link-2bc8948}{Its
  outbreak untamed, Melbourne goes into even greater lockdown.}
\end{itemize}

\href{https://www.nytimes.com/2020/08/01/world/coronavirus-covid-19.html?action=click\&pgtype=Article\&state=default\&region=MAIN_CONTENT_1\&context=storylines_live_updates}{See
more updates}

More live coverage:
\href{https://www.nytimes.com/live/2020/07/31/business/stock-market-today-coronavirus?action=click\&pgtype=Article\&state=default\&region=MAIN_CONTENT_1\&context=storylines_live_updates}{Markets}

His comments came after a telling interchange between Mr. Trump and Dr.
Deborah L. Birx, the White House pandemic response coordinator, at the
briefing on Thursday. Dr. Birx expressed concern that too many Americans
were not following the guidelines.

\includegraphics{https://static01.nyt.com/images/2020/04/03/us/politics/03dc-virus-trump2/merlin_171235779_e4d29f0d-47f9-40e0-a189-38bfd1aba030-articleLarge.jpg?quality=75\&auto=webp\&disable=upscale}

``I can tell by the curve and, as it is today, that not every American
is following it,'' she said. ``And so this is really a call to action.
We see Spain, we see Italy, we see France, we see Germany, when we see
others beginning to bend their curves. We can bend ours, but it means
everybody has to take that same responsibility as Americans.''

When she returned to the issue a few minutes later, Mr. Trump tried to
recalibrate her remarks.

``But, Deborah, aren't you referring to just a few states?'' he said.
``Because many of those states are dead flat.''

``Yes, there are states that are dead flat,'' she agreed. ``But you
know, what changes the curve is a new Detroit, a new Chicago, a new New
Orleans, a new Colorado. Those change the curves because it all of a
sudden spikes with the number of new cases.'' In other words, without
taking action, ``dead flat'' states can suddenly become hot spots.

The interplay was a rare instance of Mr. Trump doing in real time on
camera what officials have repeatedly denied that he does behind the
scenes --- attempting to water down the effect of what the medical
experts were saying.

\href{https://abcnews.go.com/Politics/candid-talk-dr-fauci-future-restrictions-white-house/story?id=69901189}{In
a video that leaked online last week,} Dr. Fauci was seen telling
colleagues at the National Institutes of Health that he regularly made
suggestions for the president's prepared remarks before the daily
briefings but that Mr. Trump ``almost always'' ignores them.

Where Dr. Fauci and the president have differed most strongly is on the
therapeutic potential of chloroquines to treat people who have the
coronavirus. Mr. Trump has said the drugs, which are approved by the
Food and Drug Administration for off-label uses aside from their
intended treatment of ailments like lupus and rheumatoid arthritis,
could be a ``game-changer.''

But Dr. Fauci has repeatedly sounded a note of skepticism, much to the
president's frustration. ``I think we've got to be careful that we don't
make that majestic leap to assume that this is a knockout drug,'' Dr.
Fauci said Friday in an interview on ``Fox \& Friends.''

Mr. Trump has also tried in recent days to blame states for shortages of
medical equipment. ``They should have had more ventilators,'' he said on
Friday.

Jared Kushner, the president's son-in-law and senior adviser, said at
Thursday's briefing that the federal stockpile was not for states to
rely on. ``The notion of the federal stockpile was it's supposed to be
our stockpile,'' Mr. Kushner said. ``It's not supposed to be states'
stockpiles that they then use.''

A day later, on Friday, the description on
\href{https://www.phe.gov/about/sns/Pages/default.aspx}{the Department
of Health and Human Services' website for its Strategic National
Stockpile} was altered evidently to reflect that viewpoint.

Previously, the website said: ``Strategic National Stockpile is the
nation's largest supply of lifesaving pharmaceuticals and medical
supplies for use in a public health emergency severe enough to cause
local supplies to run out.''

Image

Dr. Anthony S. Fauci, the director of the National Institute of Allergy
and Infectious Diseases, said stay-at-home orders should be extended to
the entire nation.Credit...Doug Mills/The New York Times

``When state, local, tribal and territorial responders request federal
assistance to support their response efforts,'' it continued, ``the
stockpile ensures that the right medicines and supplies get to those who
need the most during an emergency.'' It went on to say the stockpile
``contains enough supplies to respond to multiple large-scale
emergencies simultaneously.''

\href{https://www.nytimes.com/news-event/coronavirus?action=click\&pgtype=Article\&state=default\&region=MAIN_CONTENT_3\&context=storylines_faq}{}

\hypertarget{the-coronavirus-outbreak-}{%
\subsubsection{The Coronavirus Outbreak
›}\label{the-coronavirus-outbreak-}}

\hypertarget{frequently-asked-questions}{%
\paragraph{Frequently Asked
Questions}\label{frequently-asked-questions}}

Updated July 27, 2020

\begin{itemize}
\item ~
  \hypertarget{should-i-refinance-my-mortgage}{%
  \paragraph{Should I refinance my
  mortgage?}\label{should-i-refinance-my-mortgage}}

  \begin{itemize}
  \tightlist
  \item
    \href{https://www.nytimes.com/article/coronavirus-money-unemployment.html?action=click\&pgtype=Article\&state=default\&region=MAIN_CONTENT_3\&context=storylines_faq}{It
    could be a good idea,} because mortgage rates have
    \href{https://www.nytimes.com/2020/07/16/business/mortgage-rates-below-3-percent.html?action=click\&pgtype=Article\&state=default\&region=MAIN_CONTENT_3\&context=storylines_faq}{never
    been lower.} Refinancing requests have pushed mortgage applications
    to some of the highest levels since 2008, so be prepared to get in
    line. But defaults are also up, so if you're thinking about buying a
    home, be aware that some lenders have tightened their standards.
  \end{itemize}
\item ~
  \hypertarget{what-is-school-going-to-look-like-in-september}{%
  \paragraph{What is school going to look like in
  September?}\label{what-is-school-going-to-look-like-in-september}}

  \begin{itemize}
  \tightlist
  \item
    It is unlikely that many schools will return to a normal schedule
    this fall, requiring the grind of
    \href{https://www.nytimes.com/2020/06/05/us/coronavirus-education-lost-learning.html?action=click\&pgtype=Article\&state=default\&region=MAIN_CONTENT_3\&context=storylines_faq}{online
    learning},
    \href{https://www.nytimes.com/2020/05/29/us/coronavirus-child-care-centers.html?action=click\&pgtype=Article\&state=default\&region=MAIN_CONTENT_3\&context=storylines_faq}{makeshift
    child care} and
    \href{https://www.nytimes.com/2020/06/03/business/economy/coronavirus-working-women.html?action=click\&pgtype=Article\&state=default\&region=MAIN_CONTENT_3\&context=storylines_faq}{stunted
    workdays} to continue. California's two largest public school
    districts --- Los Angeles and San Diego --- said on July 13, that
    \href{https://www.nytimes.com/2020/07/13/us/lausd-san-diego-school-reopening.html?action=click\&pgtype=Article\&state=default\&region=MAIN_CONTENT_3\&context=storylines_faq}{instruction
    will be remote-only in the fall}, citing concerns that surging
    coronavirus infections in their areas pose too dire a risk for
    students and teachers. Together, the two districts enroll some
    825,000 students. They are the largest in the country so far to
    abandon plans for even a partial physical return to classrooms when
    they reopen in August. For other districts, the solution won't be an
    all-or-nothing approach.
    \href{https://bioethics.jhu.edu/research-and-outreach/projects/eschool-initiative/school-policy-tracker/}{Many
    systems}, including the nation's largest, New York City, are
    devising
    \href{https://www.nytimes.com/2020/06/26/us/coronavirus-schools-reopen-fall.html?action=click\&pgtype=Article\&state=default\&region=MAIN_CONTENT_3\&context=storylines_faq}{hybrid
    plans} that involve spending some days in classrooms and other days
    online. There's no national policy on this yet, so check with your
    municipal school system regularly to see what is happening in your
    community.
  \end{itemize}
\item ~
  \hypertarget{is-the-coronavirus-airborne}{%
  \paragraph{Is the coronavirus
  airborne?}\label{is-the-coronavirus-airborne}}

  \begin{itemize}
  \tightlist
  \item
    The coronavirus
    \href{https://www.nytimes.com/2020/07/04/health/239-experts-with-one-big-claim-the-coronavirus-is-airborne.html?action=click\&pgtype=Article\&state=default\&region=MAIN_CONTENT_3\&context=storylines_faq}{can
    stay aloft for hours in tiny droplets in stagnant air}, infecting
    people as they inhale, mounting scientific evidence suggests. This
    risk is highest in crowded indoor spaces with poor ventilation, and
    may help explain super-spreading events reported in meatpacking
    plants, churches and restaurants.
    \href{https://www.nytimes.com/2020/07/06/health/coronavirus-airborne-aerosols.html?action=click\&pgtype=Article\&state=default\&region=MAIN_CONTENT_3\&context=storylines_faq}{It's
    unclear how often the virus is spread} via these tiny droplets, or
    aerosols, compared with larger droplets that are expelled when a
    sick person coughs or sneezes, or transmitted through contact with
    contaminated surfaces, said Linsey Marr, an aerosol expert at
    Virginia Tech. Aerosols are released even when a person without
    symptoms exhales, talks or sings, according to Dr. Marr and more
    than 200 other experts, who
    \href{https://academic.oup.com/cid/article/doi/10.1093/cid/ciaa939/5867798}{have
    outlined the evidence in an open letter to the World Health
    Organization}.
  \end{itemize}
\item ~
  \hypertarget{what-are-the-symptoms-of-coronavirus}{%
  \paragraph{What are the symptoms of
  coronavirus?}\label{what-are-the-symptoms-of-coronavirus}}

  \begin{itemize}
  \tightlist
  \item
    Common symptoms
    \href{https://www.nytimes.com/article/symptoms-coronavirus.html?action=click\&pgtype=Article\&state=default\&region=MAIN_CONTENT_3\&context=storylines_faq}{include
    fever, a dry cough, fatigue and difficulty breathing or shortness of
    breath.} Some of these symptoms overlap with those of the flu,
    making detection difficult, but runny noses and stuffy sinuses are
    less common.
    \href{https://www.nytimes.com/2020/04/27/health/coronavirus-symptoms-cdc.html?action=click\&pgtype=Article\&state=default\&region=MAIN_CONTENT_3\&context=storylines_faq}{The
    C.D.C. has also} added chills, muscle pain, sore throat, headache
    and a new loss of the sense of taste or smell as symptoms to look
    out for. Most people fall ill five to seven days after exposure, but
    symptoms may appear in as few as two days or as many as 14 days.
  \end{itemize}
\item ~
  \hypertarget{does-asymptomatic-transmission-of-covid-19-happen}{%
  \paragraph{Does asymptomatic transmission of Covid-19
  happen?}\label{does-asymptomatic-transmission-of-covid-19-happen}}

  \begin{itemize}
  \tightlist
  \item
    So far, the evidence seems to show it does. A widely cited
    \href{https://www.nature.com/articles/s41591-020-0869-5}{paper}
    published in April suggests that people are most infectious about
    two days before the onset of coronavirus symptoms and estimated that
    44 percent of new infections were a result of transmission from
    people who were not yet showing symptoms. Recently, a top expert at
    the World Health Organization stated that transmission of the
    coronavirus by people who did not have symptoms was ``very rare,''
    \href{https://www.nytimes.com/2020/06/09/world/coronavirus-updates.html?action=click\&pgtype=Article\&state=default\&region=MAIN_CONTENT_3\&context=storylines_faq\#link-1f302e21}{but
    she later walked back that statement.}
  \end{itemize}
\end{itemize}

But after the revisions, first noticed by the journalist Laura Bassett,
the website on Friday said that the role of the stockpile was to
``supplement state and local supplies during public health
emergencies.''

``Many states have products stockpiled as well,'' it said.

``The supplies, medicines and devices for lifesaving care contained in
the stockpile,'' the site added, ``can be used as a short-term stopgap
buffer when the immediate supply of adequate amounts of these materials
may not be immediately available.''

The explosive growth of the virus in many cities over the past two weeks
has made clear that the United States has not been following the
trajectory of places like Taiwan, Japan and Hong Kong that have kept
outbreaks relatively contained so far. And the country has not begun to
see the number of new cases level off yet, as Italy has.

Several scientists said it was too early to make ironclad statements
about whether social distancing was having a powerful effect. In a few
cities that acted early, including New York, San Francisco and Seattle,
new reported cases have begun to slow, providing some optimism that
control measures work.

``The growth rate in New York City is slowing. We do have evidence that
measures we put in place two or three weeks ago may be having an
effect,'' said Jeffrey Shaman, a professor of environmental health
sciences at Columbia University. Data from Seattle and San Francisco, he
said, shows ``they've slowed it in spots.''

``But whether they're going to hold onto it is an open question,'' he
added.

The number of cases and deaths in New York City has continued to rise
quickly in recent days. More than 30,000 new cases in the metro area
were reported since Monday for a total of more than 100,000 cases
overall.

The United States has seen new hot spots in New Orleans, Indianapolis,
Chicago, Detroit and other cities that did not significantly reduce how
much people traveled until mid- to late-March, leaving open a critical
window for exponential growth.

Florida, which took longer than most of the country to issue a
stay-at-home order and reduce the distances that people traveled,
reported increasing cases this week in the Miami, Tampa and Jacksonville
areas. Experts say the
\href{https://www.nytimes.com/interactive/2020/04/02/us/coronavirus-social-distancing.html}{delays
in keeping people at home} in Florida and much of the Southeast could
make those areas more vulnerable to outbreaks in coming weeks.

The low testing rate among the population can also muddle any assessment
of the effect of distancing measures so far, said Lauren Ancel Meyers, a
professor of biology and statistics at the University of Texas at
Austin.

``In many of these other places, where social distancing measures were
enacted very recently, it would be very difficult to see it in the data
yet,'' Dr. Meyers said. ``Even if it's effective.''

Peter Baker reported from Washington, and Maggie Haberman and James
Glanz from New York. Josh Katz contributed reporting from New York.

Advertisement

\protect\hyperlink{after-bottom}{Continue reading the main story}

\hypertarget{site-index}{%
\subsection{Site Index}\label{site-index}}

\hypertarget{site-information-navigation}{%
\subsection{Site Information
Navigation}\label{site-information-navigation}}

\begin{itemize}
\tightlist
\item
  \href{https://help.nytimes.com/hc/en-us/articles/115014792127-Copyright-notice}{©~2020~The
  New York Times Company}
\end{itemize}

\begin{itemize}
\tightlist
\item
  \href{https://www.nytco.com/}{NYTCo}
\item
  \href{https://help.nytimes.com/hc/en-us/articles/115015385887-Contact-Us}{Contact
  Us}
\item
  \href{https://www.nytco.com/careers/}{Work with us}
\item
  \href{https://nytmediakit.com/}{Advertise}
\item
  \href{http://www.tbrandstudio.com/}{T Brand Studio}
\item
  \href{https://www.nytimes.com/privacy/cookie-policy\#how-do-i-manage-trackers}{Your
  Ad Choices}
\item
  \href{https://www.nytimes.com/privacy}{Privacy}
\item
  \href{https://help.nytimes.com/hc/en-us/articles/115014893428-Terms-of-service}{Terms
  of Service}
\item
  \href{https://help.nytimes.com/hc/en-us/articles/115014893968-Terms-of-sale}{Terms
  of Sale}
\item
  \href{https://spiderbites.nytimes.com}{Site Map}
\item
  \href{https://help.nytimes.com/hc/en-us}{Help}
\item
  \href{https://www.nytimes.com/subscription?campaignId=37WXW}{Subscriptions}
\end{itemize}
