Sections

SEARCH

\protect\hyperlink{site-content}{Skip to
content}\protect\hyperlink{site-index}{Skip to site index}

\href{https://www.nytimes.com/section/nyregion}{New York}

\href{https://myaccount.nytimes.com/auth/login?response_type=cookie\&client_id=vi}{}

\href{https://www.nytimes.com/section/todayspaper}{Today's Paper}

\href{/section/nyregion}{New York}\textbar{}They Filed for Unemployment
Last Month. They Haven't Seen a Dime.

\url{https://nyti.ms/2Vk5zNQ}

\begin{itemize}
\item
\item
\item
\item
\item
\end{itemize}

\href{https://www.nytimes.com/news-event/coronavirus?action=click\&pgtype=Article\&state=default\&region=TOP_BANNER\&context=storylines_menu}{The
Coronavirus Outbreak}

\begin{itemize}
\tightlist
\item
  live\href{https://www.nytimes.com/2020/08/04/world/coronavirus-cases.html?action=click\&pgtype=Article\&state=default\&region=TOP_BANNER\&context=storylines_menu}{Latest
  Updates}
\item
  \href{https://www.nytimes.com/interactive/2020/us/coronavirus-us-cases.html?action=click\&pgtype=Article\&state=default\&region=TOP_BANNER\&context=storylines_menu}{Maps
  and Cases}
\item
  \href{https://www.nytimes.com/interactive/2020/science/coronavirus-vaccine-tracker.html?action=click\&pgtype=Article\&state=default\&region=TOP_BANNER\&context=storylines_menu}{Vaccine
  Tracker}
\item
  \href{https://www.nytimes.com/2020/08/02/us/covid-college-reopening.html?action=click\&pgtype=Article\&state=default\&region=TOP_BANNER\&context=storylines_menu}{College
  Reopening}
\item
  \href{https://www.nytimes.com/live/2020/08/04/business/stock-market-today-coronavirus?action=click\&pgtype=Article\&state=default\&region=TOP_BANNER\&context=storylines_menu}{Economy}
\end{itemize}

Advertisement

\protect\hyperlink{after-top}{Continue reading the main story}

Supported by

\protect\hyperlink{after-sponsor}{Continue reading the main story}

\hypertarget{they-filed-for-unemployment-last-month-they-havent-seen-a-dime}{%
\section{They Filed for Unemployment Last Month. They Haven't Seen a
Dime.}\label{they-filed-for-unemployment-last-month-they-havent-seen-a-dime}}

Crashing websites, disconnected calls and problems with state-issued
debit cards are common for the 1.2 million New Yorkers seeking relief.

\includegraphics{https://static01.nyt.com/images/2020/04/17/nyregion/17NYVIRUS-UNEMPLOYED1/merlin_171623994_88007ca7-4427-45ee-b170-6e1cfe3d715b-articleLarge.jpg?quality=75\&auto=webp\&disable=upscale}

\href{https://www.nytimes.com/by/matthew-haag}{\includegraphics{https://static01.nyt.com/images/2018/06/14/multimedia/author-matthew-haag/author-matthew-haag-thumbLarge.jpg}}

By \href{https://www.nytimes.com/by/matthew-haag}{Matthew Haag}

\begin{itemize}
\item
  April 17, 2020
\item
  \begin{itemize}
  \item
  \item
  \item
  \item
  \item
  \end{itemize}
\end{itemize}

Gov. Andrew M. Cuomo has repeatedly promised to fix New York's archaic
unemployment-insurance system, which has been overwhelmed by an
\href{https://www.nytimes.com/2020/05/14/business/economy/coronavirus-unemployment-claims.html}{unprecedented
wave of claims}.

The state has teamed up with Google to overhaul the online application,
added thousands of workers at call centers while expanded call-volume
capacity, and vowed to address outstanding unemployment claims within 72
hours.

Carly Keohane has yet to benefit from any of the improvements.

Ms. Keohane, who lost her waitressing job in Rochester, N.Y., has been
waiting a month to receive \$2,124 in unemployment payments as a direct
deposit into her bank account.

But the state instead told her that the money had been deposited on a
state-issued debit card, which she never received. She cannot reach
anyone on the phone to find out where it is.

``I call the Department of Labor every single day, and I know the
options by heart now,'' said Ms. Keohane, 31, whose checking account was
down to \$10.35. ``It would be OK if I just knew where the money was.''

As the
\href{https://www.nytimes.com/2020/04/15/nyregion/coronavirus-face-masks-andrew-cuomo.html}{coronavirus
pandemic} and near-nationwide stay-at-home orders exact an astonishing
toll on the U.S. economy,
\href{https://www.nytimes.com/2020/04/04/nyregion/coronavirus-ny-unemployment-benefits.html}{state
unemployment systems have cratered} under a never-before-seen
\href{https://www.nytimes.com/2020/04/02/business/economy/coronavirus-unemployment-claims.html}{deluge
of jobless claims}. In the past four weeks,
\href{https://www.nytimes.com/2020/04/16/business/economy/unemployment-numbers-coronavirus.html}{about
22 million workers filed jobless claims}, including about 1.2 million
New Yorkers.

Unemployment systems, some of which rely on an antiquated computer
programming language that has largely gone the way of dinosaurs, were
not built for such a rush of claimants.

They also were not built for a new class of workers --- independent
contractors and the self-employed --- who are eligible for assistance
during the outbreak.

The results have been disastrous and maddening. Many people have had
their online applications crash before they could hit submit, requiring
them to start over from scratch. They have endured hourslong wait times
over several days only to be randomly disconnected, or connected with
representatives who say they cannot address their problems.

In other states,
\href{https://www.kansascity.com/news/local/article242013491.html}{including
Kansas} and Missouri, applicants say that they are still waiting for
unemployment payments to arrive, and that they have experienced long
wait times on the phone, as well as busy signals, disconnections and
error-prone online applications.

\hypertarget{latest-updates-global-coronavirus-outbreak}{%
\section{\texorpdfstring{\href{https://www.nytimes.com/2020/08/04/world/coronavirus-cases.html?action=click\&pgtype=Article\&state=default\&region=MAIN_CONTENT_1\&context=storylines_live_updates}{Latest
Updates: Global Coronavirus
Outbreak}}{Latest Updates: Global Coronavirus Outbreak}}\label{latest-updates-global-coronavirus-outbreak}}

Updated 2020-08-04T20:50:09.557Z

\begin{itemize}
\tightlist
\item
  \href{https://www.nytimes.com/2020/08/04/world/coronavirus-cases.html?action=click\&pgtype=Article\&state=default\&region=MAIN_CONTENT_1\&context=storylines_live_updates\#link-1228a480}{Novavax
  sees encouraging results from two studies of its experimental
  vaccine.}
\item
  \href{https://www.nytimes.com/2020/08/04/world/coronavirus-cases.html?action=click\&pgtype=Article\&state=default\&region=MAIN_CONTENT_1\&context=storylines_live_updates\#link-4825b93}{Public
  and private schools in Maryland and elsewhere are divided over
  in-person instruction.}
\item
  \href{https://www.nytimes.com/2020/08/04/world/coronavirus-cases.html?action=click\&pgtype=Article\&state=default\&region=MAIN_CONTENT_1\&context=storylines_live_updates\#link-50f7386d}{The
  United Nations calls on policymakers to `plan thoroughly for school
  reopenings.'}
\end{itemize}

\href{https://www.nytimes.com/2020/08/04/world/coronavirus-cases.html?action=click\&pgtype=Article\&state=default\&region=MAIN_CONTENT_1\&context=storylines_live_updates}{See
more updates}

More live coverage:
\href{https://www.nytimes.com/live/2020/08/04/business/stock-market-today-coronavirus?action=click\&pgtype=Article\&state=default\&region=MAIN_CONTENT_1\&context=storylines_live_updates}{Markets}

Without unemployment assistance, they have relied on friends, family and
savings, if they have any, to survive.

For New York applicants lucky enough to get through and submit a claim,
some have been jolted awake at 2 a.m. by calls from the state Labor
Department seeking to confirm their identities.

Speaking in Albany on Thursday, the secretary to the governor, Melissa
DeRosa, said New York had been staggering under the weight of more than
a million claims for unemployment insurance, about four times the number
of people who lost jobs in the 2008 financial crisis.

``We are going to continue doing everything we can to bring the system
up to deal with this scale,'' she said.

A Labor Department spokesman said on Friday that after the agency made
changes to its unemployment system, including updating its application,
its call center had made more than 470,000 follow-up calls to New
Yorkers who had not submitted completed claims.

\includegraphics{https://static01.nyt.com/images/2020/04/17/nyregion/17NYVIRUS-UNEMPLOYED2/merlin_171623991_6544e554-b07b-493a-9c41-f4951ae738bf-articleLarge.jpg?quality=75\&auto=webp\&disable=upscale}

Ms. Keohane was saving for a down payment on house. Instead, she has
withdrawn all of her money to pay for groceries and diapers and wipes
for her 2-year-old son.

She has debated whether to get groceries from a food pantry but she
cannot bring herself to do it.

``It's not right for me to have to go there,'' she said. ``There are
people who are more needy than me.''

Amy Berryman, a playwright who was let go from a wine bar in Manhattan
last month, has not received the debit card the state said it sent her
weeks ago. Every week when she has to certify her unemployment claim,
she asks that her payment be deposited into her bank account. It never
has been.

``I'm trying to spend \$50 a week or less,'' said Ms. Berryman, 31, as
she stood in line at a grocery store to buy fresh produce, which she has
been using to make lots of soup.

The \$2.2 trillion federal stimulus that Congress passed last month set
aside especially generous benefits for those who recently lost jobs:
\$600 a week on top of what states offer for unemployment. (The maximum
weekly unemployment in New York is \$504.)

But the stimulus has exacerbated the problem for states, which are now
responsible for administering an enormous expansion of unemployment
benefits for previously ineligible workers. For the first time,
independent contractors and self-employed workers qualify for relief.

But in New York and other states, those workers are facing an extra set
of head-scratching bureaucratic obstacles.

Self-employed New Yorkers, for instance, must first apply for
traditional unemployment benefits even though they are not eligible.
Once the state denies their claim, they can then pursue the new
pandemic-related benefits available to them.

Jennifer Walsh, a self-employed hair stylist in upstate New York who
stopped working on March 14, submitted her application more than two
weeks ago. She is still waiting to be denied.

``Why is this even a step?'' said Ms. Walsh, who added that many of her
friends in the hair business were in the same situation. ``I understand
this is a new process for everyone, but in the meantime we are broke and
we have no answers.''

While she waits, Ms. Walsh has been using credit cards and her savings
to buy food and pay bills. ``That will only go so far,'' she said.

\href{https://www.nytimes.com/news-event/coronavirus?action=click\&pgtype=Article\&state=default\&region=MAIN_CONTENT_3\&context=storylines_faq}{}

\hypertarget{the-coronavirus-outbreak-}{%
\subsubsection{The Coronavirus Outbreak
›}\label{the-coronavirus-outbreak-}}

\hypertarget{frequently-asked-questions}{%
\paragraph{Frequently Asked
Questions}\label{frequently-asked-questions}}

Updated August 4, 2020

\begin{itemize}
\item ~
  \hypertarget{i-have-antibodies-am-i-now-immune}{%
  \paragraph{I have antibodies. Am I now
  immune?}\label{i-have-antibodies-am-i-now-immune}}

  \begin{itemize}
  \tightlist
  \item
    As of right
    now,\href{https://www.nytimes.com/2020/07/22/health/covid-antibodies-herd-immunity.html?action=click\&pgtype=Article\&state=default\&region=MAIN_CONTENT_3\&context=storylines_faq}{that
    seems likely, for at least several months.} There have been
    frightening accounts of people suffering what seems to be a second
    bout of Covid-19. But experts say these patients may have a
    drawn-out course of infection, with the virus taking a slow toll
    weeks to months after initial exposure. People infected with the
    coronavirus typically
    \href{https://www.nature.com/articles/s41586-020-2456-9}{produce}
    immune molecules called antibodies, which are
    \href{https://www.nytimes.com/2020/05/07/health/coronavirus-antibody-prevalence.html?action=click\&pgtype=Article\&state=default\&region=MAIN_CONTENT_3\&context=storylines_faq}{protective
    proteins made in response to an
    infection}\href{https://www.nytimes.com/2020/05/07/health/coronavirus-antibody-prevalence.html?action=click\&pgtype=Article\&state=default\&region=MAIN_CONTENT_3\&context=storylines_faq}{.
    These antibodies may} last in the body
    \href{https://www.nature.com/articles/s41591-020-0965-6}{only two to
    three months}, which may seem worrisome, but that's perfectly normal
    after an acute infection subsides, said Dr. Michael Mina, an
    immunologist at Harvard University. It may be possible to get the
    coronavirus again, but it's highly unlikely that it would be
    possible in a short window of time from initial infection or make
    people sicker the second time.
  \end{itemize}
\item ~
  \hypertarget{im-a-small-business-owner-can-i-get-relief}{%
  \paragraph{I'm a small-business owner. Can I get
  relief?}\label{im-a-small-business-owner-can-i-get-relief}}

  \begin{itemize}
  \tightlist
  \item
    The
    \href{https://www.nytimes.com/article/small-business-loans-stimulus-grants-freelancers-coronavirus.html?action=click\&pgtype=Article\&state=default\&region=MAIN_CONTENT_3\&context=storylines_faq}{stimulus
    bills enacted in March} offer help for the millions of American
    small businesses. Those eligible for aid are businesses and
    nonprofit organizations with fewer than 500 workers, including sole
    proprietorships, independent contractors and freelancers. Some
    larger companies in some industries are also eligible. The help
    being offered, which is being managed by the Small Business
    Administration, includes the Paycheck Protection Program and the
    Economic Injury Disaster Loan program. But lots of folks have
    \href{https://www.nytimes.com/interactive/2020/05/07/business/small-business-loans-coronavirus.html?action=click\&pgtype=Article\&state=default\&region=MAIN_CONTENT_3\&context=storylines_faq}{not
    yet seen payouts.} Even those who have received help are confused:
    The rules are draconian, and some are stuck sitting on
    \href{https://www.nytimes.com/2020/05/02/business/economy/loans-coronavirus-small-business.html?action=click\&pgtype=Article\&state=default\&region=MAIN_CONTENT_3\&context=storylines_faq}{money
    they don't know how to use.} Many small-business owners are getting
    less than they expected or
    \href{https://www.nytimes.com/2020/06/10/business/Small-business-loans-ppp.html?action=click\&pgtype=Article\&state=default\&region=MAIN_CONTENT_3\&context=storylines_faq}{not
    hearing anything at all.}
  \end{itemize}
\item ~
  \hypertarget{what-are-my-rights-if-i-am-worried-about-going-back-to-work}{%
  \paragraph{What are my rights if I am worried about going back to
  work?}\label{what-are-my-rights-if-i-am-worried-about-going-back-to-work}}

  \begin{itemize}
  \tightlist
  \item
    Employers have to provide
    \href{https://www.osha.gov/SLTC/covid-19/standards.html}{a safe
    workplace} with policies that protect everyone equally.
    \href{https://www.nytimes.com/article/coronavirus-money-unemployment.html?action=click\&pgtype=Article\&state=default\&region=MAIN_CONTENT_3\&context=storylines_faq}{And
    if one of your co-workers tests positive for the coronavirus, the
    C.D.C.} has said that
    \href{https://www.cdc.gov/coronavirus/2019-ncov/community/guidance-business-response.html}{employers
    should tell their employees} -\/- without giving you the sick
    employee's name -\/- that they may have been exposed to the virus.
  \end{itemize}
\item ~
  \hypertarget{should-i-refinance-my-mortgage}{%
  \paragraph{Should I refinance my
  mortgage?}\label{should-i-refinance-my-mortgage}}

  \begin{itemize}
  \tightlist
  \item
    \href{https://www.nytimes.com/article/coronavirus-money-unemployment.html?action=click\&pgtype=Article\&state=default\&region=MAIN_CONTENT_3\&context=storylines_faq}{It
    could be a good idea,} because mortgage rates have
    \href{https://www.nytimes.com/2020/07/16/business/mortgage-rates-below-3-percent.html?action=click\&pgtype=Article\&state=default\&region=MAIN_CONTENT_3\&context=storylines_faq}{never
    been lower.} Refinancing requests have pushed mortgage applications
    to some of the highest levels since 2008, so be prepared to get in
    line. But defaults are also up, so if you're thinking about buying a
    home, be aware that some lenders have tightened their standards.
  \end{itemize}
\item ~
  \hypertarget{what-is-school-going-to-look-like-in-september}{%
  \paragraph{What is school going to look like in
  September?}\label{what-is-school-going-to-look-like-in-september}}

  \begin{itemize}
  \tightlist
  \item
    It is unlikely that many schools will return to a normal schedule
    this fall, requiring the grind of
    \href{https://www.nytimes.com/2020/06/05/us/coronavirus-education-lost-learning.html?action=click\&pgtype=Article\&state=default\&region=MAIN_CONTENT_3\&context=storylines_faq}{online
    learning},
    \href{https://www.nytimes.com/2020/05/29/us/coronavirus-child-care-centers.html?action=click\&pgtype=Article\&state=default\&region=MAIN_CONTENT_3\&context=storylines_faq}{makeshift
    child care} and
    \href{https://www.nytimes.com/2020/06/03/business/economy/coronavirus-working-women.html?action=click\&pgtype=Article\&state=default\&region=MAIN_CONTENT_3\&context=storylines_faq}{stunted
    workdays} to continue. California's two largest public school
    districts --- Los Angeles and San Diego --- said on July 13, that
    \href{https://www.nytimes.com/2020/07/13/us/lausd-san-diego-school-reopening.html?action=click\&pgtype=Article\&state=default\&region=MAIN_CONTENT_3\&context=storylines_faq}{instruction
    will be remote-only in the fall}, citing concerns that surging
    coronavirus infections in their areas pose too dire a risk for
    students and teachers. Together, the two districts enroll some
    825,000 students. They are the largest in the country so far to
    abandon plans for even a partial physical return to classrooms when
    they reopen in August. For other districts, the solution won't be an
    all-or-nothing approach.
    \href{https://bioethics.jhu.edu/research-and-outreach/projects/eschool-initiative/school-policy-tracker/}{Many
    systems}, including the nation's largest, New York City, are
    devising
    \href{https://www.nytimes.com/2020/06/26/us/coronavirus-schools-reopen-fall.html?action=click\&pgtype=Article\&state=default\&region=MAIN_CONTENT_3\&context=storylines_faq}{hybrid
    plans} that involve spending some days in classrooms and other days
    online. There's no national policy on this yet, so check with your
    municipal school system regularly to see what is happening in your
    community.
  \end{itemize}
\end{itemize}

Ms. DeRosa said on Thursday that roughly 275,000 New Yorkers still had
outstanding unemployment claims, most of them self-employed, which
requires additional paperwork and confirmation.

State officials said on Friday that the federal government was requiring
New York State to confirm that those workers were not eligible for
traditional unemployment before processing their claims for pandemic
assistance. The state is working to create a single unemployment
application for such workers.

But challenges with New York's unemployment system are just the start of
problems for many people who are out of work. More than a half-dozen New
Yorkers who recently lost their jobs told The New York Times that they
asked that unemployment payments be deposited in their checking
accounts, but received debit cards instead.

James Colón, who was let go from the Strand bookstore in Manhattan last
month, received one of the cards, issued by Key Bank, a regional bank
based in Cleveland. Its online banking system worked the first day, but
now shows an error message when he tries to log on.

Without access to Key Bank's site, he cannot transfer the money into his
checking account to pay May rent. No one at Key Bank has been able to
resolve the problem, he said.

A Key Bank representative did not immediately respond to questions about
its unemployment benefits card. Other states, including Washington and
Indiana, also disperse unemployment assistance onto the bank's cards.

The New York Labor Department had temporarily suspended direct deposit
payments because of back-end problems, state officials said. During that
period, the state issued the debit cards to ensure claimants received
payments.

``This situation is entirely unacceptable --- we expect all our
contractors to responsibly and reliably deliver benefits for New
Yorkers,'' a Labor Department spokesman said. ``We will ensure every
single New Yorker who is entitled to unemployment benefits gets them.''

Bobbie de Matos, who lost her job as a server at a table-tennis themed
bar in Manhattan, received a Key Bank card, which she did not request.
It also does not work.

After calling the bank over many days, including holding for four hours
on one call, Ms. de Matos said she finally reached a representative who
told her that the card had not been assigned to her or anyone.

She needed to ask the state's Labor Department to fix the issue, the
person told her. But the state said it was an error with the bank. A new
card is supposed to arrive in the mail soon.

She is hoping everything will be cleared up by next Friday, when she is
scheduled to move from Manhattan to Brooklyn and will need to pay the
movers.

``It's a complete mess,'' said Ms. de Matos, 23.

Long before the stay-at-home orders, Melvin Taylor II was let go from a
production position in New York City. He received a Key Bank card in the
mail late last year for his unemployment benefits.

Right as mass layoffs and furloughs began about a month ago, Key Bank
alerted him that it had detected potential fraud on his card and
automatically canceled it.

Mr. Taylor said he had not been able to reach a bank representative to
order a replacement card.

``You'd be on the phone three hours, 59 minutes and 27 seconds, and then
the phone would cut off,'' Mr. Taylor said.

He has resorted to searching through coats and pants for loose change
--- he found about \$20 --- and has experimented with cheap and filling
rice and pasta recipes.

``There are a lot of different spices that you can put in rice,'' he
said.

Jesse McKinley contributed reporting.

Advertisement

\protect\hyperlink{after-bottom}{Continue reading the main story}

\hypertarget{site-index}{%
\subsection{Site Index}\label{site-index}}

\hypertarget{site-information-navigation}{%
\subsection{Site Information
Navigation}\label{site-information-navigation}}

\begin{itemize}
\tightlist
\item
  \href{https://help.nytimes.com/hc/en-us/articles/115014792127-Copyright-notice}{©~2020~The
  New York Times Company}
\end{itemize}

\begin{itemize}
\tightlist
\item
  \href{https://www.nytco.com/}{NYTCo}
\item
  \href{https://help.nytimes.com/hc/en-us/articles/115015385887-Contact-Us}{Contact
  Us}
\item
  \href{https://www.nytco.com/careers/}{Work with us}
\item
  \href{https://nytmediakit.com/}{Advertise}
\item
  \href{http://www.tbrandstudio.com/}{T Brand Studio}
\item
  \href{https://www.nytimes.com/privacy/cookie-policy\#how-do-i-manage-trackers}{Your
  Ad Choices}
\item
  \href{https://www.nytimes.com/privacy}{Privacy}
\item
  \href{https://help.nytimes.com/hc/en-us/articles/115014893428-Terms-of-service}{Terms
  of Service}
\item
  \href{https://help.nytimes.com/hc/en-us/articles/115014893968-Terms-of-sale}{Terms
  of Sale}
\item
  \href{https://spiderbites.nytimes.com}{Site Map}
\item
  \href{https://help.nytimes.com/hc/en-us}{Help}
\item
  \href{https://www.nytimes.com/subscription?campaignId=37WXW}{Subscriptions}
\end{itemize}
