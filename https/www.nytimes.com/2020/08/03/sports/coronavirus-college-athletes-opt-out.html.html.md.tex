Sections

SEARCH

\protect\hyperlink{site-content}{Skip to
content}\protect\hyperlink{site-index}{Skip to site index}

\href{https://www.nytimes.com/section/sports}{Sports}

\href{https://myaccount.nytimes.com/auth/login?response_type=cookie\&client_id=vi}{}

\href{https://www.nytimes.com/section/todayspaper}{Today's Paper}

\href{/section/sports}{Sports}\textbar{}A College Athlete Calls His
Coach to Opt Out. And Ends Up on the Outs.

\url{https://nyti.ms/33rBbp1}

\begin{itemize}
\item
\item
\item
\item
\item
\end{itemize}

Advertisement

\protect\hyperlink{after-top}{Continue reading the main story}

Supported by

\protect\hyperlink{after-sponsor}{Continue reading the main story}

on college football

\hypertarget{a-college-athlete-calls-his-coach-to-opt-out-and-ends-up-on-the-outs}{%
\section{A College Athlete Calls His Coach to Opt Out. And Ends Up on
the
Outs.}\label{a-college-athlete-calls-his-coach-to-opt-out-and-ends-up-on-the-outs}}

Kassidy Woods, a redshirt sophomore receiver at Washington State, was
concerned about the pandemic. The coach was sympathetic until he learned
he was joining a players' rights initiative.

\includegraphics{https://static01.nyt.com/images/2020/08/03/sports/03collegefootball-3/03collegefootball-3-articleLarge.jpg?quality=75\&auto=webp\&disable=upscale}

\href{https://www.nytimes.com/by/billy-witz}{\includegraphics{https://static01.nyt.com/images/2018/02/16/multimedia/author-billy-witz/author-billy-witz-thumbLarge.jpg}}

By \href{https://www.nytimes.com/by/billy-witz}{Billy Witz}

\begin{itemize}
\item
  Aug. 3, 2020
\item
  \begin{itemize}
  \item
  \item
  \item
  \item
  \item
  \end{itemize}
\end{itemize}

College athletes have begun
\href{https://www.nytimes.com/2020/06/12/sports/ncaafootball/george-floyd-protests-college-sports.html}{challenging}
a longstanding pillar, that the college sports industrial complex must
hum along --- as if straight from the pages of ``Das Kapital'' --- on
the fuel of exploited labor. Their labor.

Yet, to better understand how the modern-day dynamic works --- and why
players are more stridently
\href{https://www.nytimes.com/2020/08/02/sports/ncaafootball/coronavirus-college-football-pac-12.html}{calling
for a voice}in matters like social justice, how their images are used,
straight-up pay and playing during the pandemic --- all that's necessary
is to listen to a five-minute, nine-second recording of a phone call
between Nick Rolovich, the new football coach at Washington State, and
Kassidy Woods, a redshirt sophomore receiver.

It lays clear --- not with an iron fist, but a velvet hammer --- just
who is in charge.

It begins amiably.

``What's up, coach?''

``Kass, how are you doing? What's up?''

Woods, who was competing for a starting position, had called to tell
Rolovich that he was opting out of the season. Woods explained that he
had been diagnosed with the sickle cell trait when he enrolled at
Washington State and with so much uncertainty about the coronavirus's
lingering effects, he did not feel comfortable playing.

``I've got nothing wrong with that,'' Rolovich replied.

Then he asked Woods a question: was he joining the Pac-12 Conference
unity group?

Rolovich was referring to the Pac-12 football players who announced
Sunday they were threatening to sit out the season unless their demands,
including more concrete health and safety protocols and measures that
would amount to a redistribution of much of the wealth that players
generate for their schools, were met.

``Yes, sir,'' Woods said.

Well, the coach said, that would be a problem.

Woods's scholarship would be honored for this year, as is required for
anyone who opts out for health reasons, but if he was part of this
organized effort, it was going to be handled differently, the coach
said. Woods could not work out with the team because it would send a
mixed message and his locker should be emptied by Monday.

Rolovich then urged Woods to tell others they would face the same
consequences. (Dallas Hobbs, a redshirt junior defensive end, soon found
out he needed to empty his locker, too, he said.)

And then the conversation concluded as if it they had discussed dessert
options in the dining hall.

``All right. Appreciate you, coach,'' Woods said.

``How's your family?'' Rolovich asked.

``They're doing good. I already talked to them about it,'' Woods
answered.

``Cool,'' said Rolovich, who closed the call by saying he would see
Woods on a team Zoom call on Sunday night.

\includegraphics{https://static01.nyt.com/images/2020/08/04/sports/04college-football-print-1/03college-football-1-articleLarge.jpg?quality=75\&auto=webp\&disable=upscale}

When I spoke with Woods on Monday --- he had sent me a recording of the
phone call on Sunday night --- he said he was devastated, but resolute.
He had hoped to become a starter this season and work toward a career in
the N.F.L., and had no complaints about his place on the team.

Indeed, Woods was emerging as a leader. He (along with Hobbs)
represented the football team on the Student-Athlete Advisory Committee,
served as the social chair of the recently formed Black Student-Athlete
Association, and represented Washington State at the Black
Student-Athlete Summit in January at the University of Texas. And Woods
also served as the team's unofficial barber, commandeering a chair in
the Cougars' athletic complex and putting to use the skills his mother,
a hairdresser, taught him.

\hypertarget{the-games-resume}{%
\subsubsection{The Games Resume}\label{the-games-resume}}

\hypertarget{sports-and-the-virus}{%
\paragraph{Sports and the Virus}\label{sports-and-the-virus}}

Updated Aug. 3, 2020

Here's what's happening as the world of sports slowly comes back to
life:

\begin{itemize}
\item
  \begin{itemize}
  \tightlist
  \item
    On all but the two biggest courts, automated line calls
    \href{https://www.nytimes.com/2020/08/03/sports/tennis/us-open-hawkeye-line-judges.html?action=click\&pgtype=Article\&state=default\&region=MAIN_CONTENT_2\&context=storylines_keepup}{will
    replace human judges} at the U.S. Open to reduce the number of
    people on site during the pandemic.
  \item
    Mets star Yoenis Cespedes is healthy, but
    \href{https://www.nytimes.com/2020/08/02/sports/baseball/Yoenis-cespedes-opt-out-rule.html?action=click\&pgtype=Article\&state=default\&region=MAIN_CONTENT_2\&context=storylines_keepup}{has
    decided to opt out} of the 2020 baseball season for Covid-related
    reasons.
  \item
    Britain tried to bring fans back to indoor sports.
    \href{https://www.nytimes.com/2020/08/02/sports/snooker-world-championship.html?action=click\&pgtype=Article\&state=default\&region=MAIN_CONTENT_2\&context=storylines_keepup}{It
    lasted a day}.
  \end{itemize}
\end{itemize}

He'd been introduced to Washington State President Kirk Schultz through
a Black Student-Athlete Association video conference call and had built
up a relationship with the athletic director, Pat Chun.

But by Monday, Woods said he felt abandoned.

He'd called Chun hoping he could still be part of the team, but Woods
said the athletic director backed the coach. (Rolovich and Chun declined
an interview request.) What also upset him, he said, is that several
teammates were cowed into not opting out because he said they felt
threatened.

``A lot of them have reached out --- `Man, I'm sorry,''' Woods said.
``If you're here for me, just opt out. If we all did, what is he going
to do --- cut everybody from the team? You say you love me, say I'm your
brother, but me and Dallas are pretty much ostracized from the team.''

He added: ``It's all about the movement. Me and Dallas have been nothing
but a service to Washington State. Our coaches don't have anything bad
to say about me. I don't have anything bad to say about them except for
dismissing me for being part of this movement.''

Woods said his disquiet goes back to late June, when a teammate he was
living with texted several days before Woods headed back to campus to
say he had tested positive for the coronavirus. Woods said nobody from
the school notified him --- or of any other cases.

He also expressed discomfort with signing a liability waiver when he
reported for voluntary workouts on July 1. And when Washington State
announced on July 23 that virtually all learning would be remote, Woods
said he and his teammates wondered why they were on campus preparing for
football.

Image

On the call, Coach Nick Rolovich appears understanding until he hears
Woods is joining a Pac-12 Conference players' rights
initiative.Credit...Pete Caster/Lewiston Tribune, via Associated Press

I asked, if his relationship with Rolovich was good, why did he feel the
need to record the call?

Even though he and Hobbs had spoken with Rolovich about the unity
group's broad plans without any pushback, Woods said his growing
distrust over the waiver, the way he found out about the roommate's test
and the practices while students were attending remotely left him unsure
how the conversation was going to unfold. He wanted to have a record for
his parents to hear.

``I knew I was standing up for something,'' Woods said. ``You don't
really know how it's going to go.''

Woods's feelings of abandonment, though, are not complete. He said he
has received support from players around the country. And his parents
and his six siblings have firmly encouraged him. In fact, his mother,
Jerline, made public her son's circumstance as a rebuttal to a reporter
who tweeted that no players had been cut.

``You're putting all this on your back --- a target --- maybe teams
don't touch you,'' said his father John Woods Jr., a basketball captain
at Missouri in the late 1990s, who encouraged his son to make the
recording public.

But he said that times are different.

``He's just standing up for his First-Amendment rights that need to be
addressed,'' his father said. ``He didn't do anything wrong and he
stands by that. Twenty-five years ago, we wanted to do that, but now
they've got this platform where it's OK.''

He continued: ``We can't just dribble, be quiet, run, you've got your
scholarship you should be happy. You can't get away with that and
intimidate players into not saying those things and make them feel like,
`oh, it's me.' Those days are over.''

Advertisement

\protect\hyperlink{after-bottom}{Continue reading the main story}

\hypertarget{site-index}{%
\subsection{Site Index}\label{site-index}}

\hypertarget{site-information-navigation}{%
\subsection{Site Information
Navigation}\label{site-information-navigation}}

\begin{itemize}
\tightlist
\item
  \href{https://help.nytimes.com/hc/en-us/articles/115014792127-Copyright-notice}{©~2020~The
  New York Times Company}
\end{itemize}

\begin{itemize}
\tightlist
\item
  \href{https://www.nytco.com/}{NYTCo}
\item
  \href{https://help.nytimes.com/hc/en-us/articles/115015385887-Contact-Us}{Contact
  Us}
\item
  \href{https://www.nytco.com/careers/}{Work with us}
\item
  \href{https://nytmediakit.com/}{Advertise}
\item
  \href{http://www.tbrandstudio.com/}{T Brand Studio}
\item
  \href{https://www.nytimes.com/privacy/cookie-policy\#how-do-i-manage-trackers}{Your
  Ad Choices}
\item
  \href{https://www.nytimes.com/privacy}{Privacy}
\item
  \href{https://help.nytimes.com/hc/en-us/articles/115014893428-Terms-of-service}{Terms
  of Service}
\item
  \href{https://help.nytimes.com/hc/en-us/articles/115014893968-Terms-of-sale}{Terms
  of Sale}
\item
  \href{https://spiderbites.nytimes.com}{Site Map}
\item
  \href{https://help.nytimes.com/hc/en-us}{Help}
\item
  \href{https://www.nytimes.com/subscription?campaignId=37WXW}{Subscriptions}
\end{itemize}
