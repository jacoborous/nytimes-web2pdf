Sections

SEARCH

\protect\hyperlink{site-content}{Skip to
content}\protect\hyperlink{site-index}{Skip to site index}

\href{https://myaccount.nytimes.com/auth/login?response_type=cookie\&client_id=vi}{}

\href{https://www.nytimes.com/section/todayspaper}{Today's Paper}

\href{/section/opinion}{Opinion}\textbar{}I'd Need Evidence Before I Got
a Covid-19 Vaccine. It Doesn't Exist Yet.

\url{https://nyti.ms/33p3e8z}

\begin{itemize}
\item
\item
\item
\item
\item
\item
\end{itemize}

Advertisement

\protect\hyperlink{after-top}{Continue reading the main story}

\href{/section/opinion}{Opinion}

Supported by

\protect\hyperlink{after-sponsor}{Continue reading the main story}

\hypertarget{id-need-evidence-before-i-got-a-covid-19-vaccine-it-doesnt-exist-yet}{%
\section{I'd Need Evidence Before I Got a Covid-19 Vaccine. It Doesn't
Exist
Yet.}\label{id-need-evidence-before-i-got-a-covid-19-vaccine-it-doesnt-exist-yet}}

Scientists need to show us the data. And that's exactly what they're
working on.

By Natalie Dean

Dr. \href{https://www.nataliexdean.com/}{Dean} is an assistant professor
of biostatistics at the University of Florida.

\begin{itemize}
\item
  Aug. 3, 2020
\item
  \begin{itemize}
  \item
  \item
  \item
  \item
  \item
  \item
  \end{itemize}
\end{itemize}

\includegraphics{https://static01.nyt.com/images/2020/08/05/opinion/03deanWeb/merlin_174998976_25186e70-3ea0-4b30-bbaa-73aba64fd000-articleLarge.jpg?quality=75\&auto=webp\&disable=upscale}

Coronavirus vaccines are
\href{https://www.nytimes.com/interactive/2020/science/coronavirus-vaccine-tracker.html}{rapidly
advancing} through the development pipeline. The University of Oxford's
vaccine is in large trials in
\href{https://www.ox.ac.uk/news/2020-05-22-oxford-covid-19-vaccine-begin-phase-iiiii-human-trials}{Britain},
\href{https://www.ox.ac.uk/news/2020-06-28-trial-oxford-covid-19-vaccine-starts-brazil}{Brazil}
and
\href{https://www.ovg.ox.ac.uk/news/trial-of-oxford-covid-19-vaccine-in-south-africa-begins}{South
Africa}. In the United States, researchers just began enrolling around
30,000 volunteers to test
\href{https://www.nih.gov/news-events/news-releases/phase-3-clinical-trial-investigational-vaccine-covid-19-begins}{Moderna's
vaccine}, and more trials are starting every day.
\href{https://www.hhs.gov/about/news/2020/06/16/fact-sheet-explaining-operation-warp-speed.html}{Operation
Warp Speed} has set an ambitious goal of delivering 300 million doses of
a safe, effective vaccine by January.

But the concept of developing a vaccine at ``warp speed''
\href{https://www.nytimes.com/2020/07/18/health/coronavirus-anti-vaccine.html}{makes
many people uncomfortable}. In a
\href{https://apnorc.org/projects/expectations-for-a-covid-19-vaccine/}{May
survey}, 49 percent of the Americans polled said they plan to get a
coronavirus vaccine when one is available, 20 percent do not, and 31
percent indicated that they were not sure. The World Health Organization
considers ``vaccine hesitancy'' a major threat to global health, and
poor uptake would jeopardize the impact of a coronavirus vaccine.

This hesitancy isn't surprising. Why should we expect Americans to agree
to a vaccine before one is even available? ``I think it's reasonable to
be skeptical about a vaccine that doesn't exist yet,'' Dr. Paul Offit,
the director of the Vaccine Education Center at Children's Hospital of
Philadelphia,
\href{https://www.today.com/health/when-will-vaccine-be-ready-covid-19-speed-causes-safety-t187727}{told
Today}.

I'm a vaccine researcher, and even I would place myself in the ``not
sure'' bucket. What we have right now is a collection of
\href{https://news.harvard.edu/gazette/story/2020/05/vaccines-found-that-may-protect-against-covid-19-in-animal-models/}{animal
data}, immune response data and safety data based on early trials and
from similar vaccines for other diseases. The evidence that would
convince me to get a Covid-19 vaccine, or to recommend that my loved
ones get vaccinated, does not yet exist.

That data can be generated by the large trials that are just beginning,
known as Phase III or efficacy trials.
\href{https://www.forbes.com/sites/stevensalzberg/2020/08/02/start-vaccinating-now/\#647e8baecf6e}{Some
have argued} that we already have enough safety and immune response data
to start vaccinating people now. But this would be a big mistake.

This is how Phase III trials work: Thousands of healthy adult volunteers
are randomized to receive either a new Covid-19 vaccine or a control ---
a placebo or an already licensed vaccine for another disease. Then they
go about their normal lives. They do not know what they have received
(known as ``blinding'') so the two groups behave similarly in terms of
risk taking.

Participants are monitored for side effects and contacted regularly to
ask about symptoms and to be tested for infection. The goal is to
compare the rates of disease or infection across the two groups to
measure how well the vaccine prevents Covid-19 ``in the field.''

It is possible that some Covid-19 vaccines may not
\href{https://www.biorxiv.org/content/10.1101/2020.05.13.093195v1}{prevent
infection entirely}, but they could still prepare a person's immune
system so that, if infected, they would experience milder symptoms, or
even none at all. That's similar to the flu vaccine: It's not perfect,
but we
\href{https://www.cdc.gov/flu/vaccines-work/vaccineeffect.htm}{advise
people to get it} because it reduces intensive care admissions and
deaths.

How many people need to be protected by a vaccine before it's
recommended for widespread use? Ideally, rates of disease will be 70
percent lower in vaccinated people than in unvaccinated people. The
World Health Organization says a vaccine should be at minimum
\href{https://www.who.int/blueprint/priority-diseases/key-action/WHO_Target_Product_Profiles_for_COVID-19_web.pdf}{50
percent} effective, averaged across age groups. (We know from influenza
that vaccines don't always work as well on
\href{https://www.cdc.gov/flu/spotlights/2019-2020/vaccine-stronger-immune.htm}{older
adults} whose immune systems have declined.)

This benchmark is crucial because a weak vaccine might be worse than no
vaccine at all. We do not want people who are only slightly protected to
behave as if they are invulnerable, which could exacerbate transmission.
It is also costly to roll out a vaccine, diverting attention away from
other efforts that we know work, like mask-wearing, and from testing
better vaccines.

The last thing Phase III trials do is examine safety. Earlier trials do
this, too, but larger trials allow us to detect rarer side effects. One
of those rare effects researchers are paying attention to is a
paradoxical phenomenon known as
\href{https://www.pnas.org/content/117/15/8218}{immune enhancement}, in
which a vaccinated person's immune system overreacts to infection.
Researchers can test for this by comparing the rates of disease severe
enough to require hospitalization across the two groups. A clear signal
that hospitalization is higher among vaccinated participants would mark
the end of a vaccine.

The speed of the trials depends on how quickly we can detect a
difference between the two groups. If two vaccinated people became sick
versus 10 who got a placebo, it could be because of chance. But if it
were 20 compared to 100, we would feel much more confident that the
vaccine was working.

Key to getting a quick result is placing the trial in outbreak hot spots
where people are most likely to be infected. We can even target the
highest-risk people within those areas, using mobile teams to travel to
neighborhoods, bringing the trial directly to the people. Some trials
\href{https://abc11.com/covdi-19-vaccine-clinical-trials-covid-19-volunteers-needed-wake-research/6337324/}{explicitly
prioritize} essential workers like health care workers or grocery
employees. Others are simply focused on enrolling large numbers of
participants as fast as possible.

Combining those efforts, it could take as little as three to six months
to generate enough convincing safety and efficacy data for companies to
apply for expedited review by the Food and Drug Administration.

There are ways for vaccines to be approved without definitive efficacy
data, based on
\href{https://www.nature.com/articles/npjvaccines201613}{animal} or
immune response data instead, but the bar is extremely high, and for
good reason. A precondition is that efficacy trials are not possible,
typically because the disease is so rare or sporadic that it would
require hundreds of thousands of participants to be followed for many
years to tell if the vaccine is effective (rabies, for example). That is
not the situation here.

While there is promising data from smaller trials that measured the
antibody response in people who got a vaccine, it's not enough to
approve a vaccine. We don't know the
\href{https://www.nature.com/articles/d41586-020-02174-y}{level of
antibodies needed} to prevent infection from this virus. There is a
history of vaccines with promising immune response data that
\href{https://www.fda.gov/media/102332/download}{did not pan out in the
field}.

With this in mind, the F.D.A. has
\href{https://www.fda.gov/regulatory-information/search-fda-guidance-documents/development-and-licensure-vaccines-prevent-covid-19}{committed}
to the need for traditional efficacy trial data to approve Covid-19
vaccines. And it follows the W.H.O.'s recommendation, stating that
vaccines must be at least 50 percent effective to be approved.

I worry nonetheless that public pressure may mount to approve a product
that doesn't meet our standards. Other countries may decide to approve
vaccines based on weaker evidence. Russia, for example,
\href{https://www.wsj.com/articles/russia-seeks-to-register-first-coronavirus-vaccine-in-august-11596047326}{claims}
to be on track to approve a vaccine in just a few weeks.

We must resist the desire to rush out a product. Creating vaccines is
hard, and we should be prepared for the reality that some promising ones
will not meet the F.D.A.'s criteria. Researchers and the government
should also commit to
\href{https://www.thenation.com/article/society/vaccine-coronavirus-october-surprise/}{transparency}
so that people can see the results for themselves to understand the
regulatory decisions.

Waiting for a better vaccine to come along may feel like torture, but it
is the right move. With so many potential shots on goal, scientists are
optimistic that a safe and effective vaccine is out there. We can't
afford to jeopardize the public's health and hard-earned trust by
approving anything short of that.

\href{https://www.nataliexdean.com/}{Natalie Dean} is an assistant
professor of biostatistics at the University of Florida.

\emph{The Times is committed to publishing}
\href{https://www.nytimes.com/2019/01/31/opinion/letters/letters-to-editor-new-york-times-women.html}{\emph{a
diversity of letters}} \emph{to the editor. We'd like to hear what you
think about this or any of our articles. Here are some}
\href{https://help.nytimes.com/hc/en-us/articles/115014925288-How-to-submit-a-letter-to-the-editor}{\emph{tips}}\emph{.
And here's our email:}
\href{mailto:letters@nytimes.com}{\emph{letters@nytimes.com}}\emph{.}

\emph{Follow The New York Times Opinion section on}
\href{https://www.facebook.com/nytopinion}{\emph{Facebook}}\emph{,}
\href{http://twitter.com/NYTOpinion}{\emph{Twitter (@NYTopinion)}}
\emph{and}
\href{https://www.instagram.com/nytopinion/}{\emph{Instagram}}\emph{.}

Advertisement

\protect\hyperlink{after-bottom}{Continue reading the main story}

\hypertarget{site-index}{%
\subsection{Site Index}\label{site-index}}

\hypertarget{site-information-navigation}{%
\subsection{Site Information
Navigation}\label{site-information-navigation}}

\begin{itemize}
\tightlist
\item
  \href{https://help.nytimes.com/hc/en-us/articles/115014792127-Copyright-notice}{©~2020~The
  New York Times Company}
\end{itemize}

\begin{itemize}
\tightlist
\item
  \href{https://www.nytco.com/}{NYTCo}
\item
  \href{https://help.nytimes.com/hc/en-us/articles/115015385887-Contact-Us}{Contact
  Us}
\item
  \href{https://www.nytco.com/careers/}{Work with us}
\item
  \href{https://nytmediakit.com/}{Advertise}
\item
  \href{http://www.tbrandstudio.com/}{T Brand Studio}
\item
  \href{https://www.nytimes.com/privacy/cookie-policy\#how-do-i-manage-trackers}{Your
  Ad Choices}
\item
  \href{https://www.nytimes.com/privacy}{Privacy}
\item
  \href{https://help.nytimes.com/hc/en-us/articles/115014893428-Terms-of-service}{Terms
  of Service}
\item
  \href{https://help.nytimes.com/hc/en-us/articles/115014893968-Terms-of-sale}{Terms
  of Sale}
\item
  \href{https://spiderbites.nytimes.com}{Site Map}
\item
  \href{https://help.nytimes.com/hc/en-us}{Help}
\item
  \href{https://www.nytimes.com/subscription?campaignId=37WXW}{Subscriptions}
\end{itemize}
