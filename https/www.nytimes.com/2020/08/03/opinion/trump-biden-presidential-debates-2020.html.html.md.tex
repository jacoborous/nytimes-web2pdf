Sections

SEARCH

\protect\hyperlink{site-content}{Skip to
content}\protect\hyperlink{site-index}{Skip to site index}

\href{https://myaccount.nytimes.com/auth/login?response_type=cookie\&client_id=vi}{}

\href{https://www.nytimes.com/section/todayspaper}{Today's Paper}

\href{/section/opinion}{Opinion}\textbar{}Let's Scrap the Presidential
Debates

\url{https://nyti.ms/2EKUVtz}

\begin{itemize}
\item
\item
\item
\item
\item
\item
\end{itemize}

Advertisement

\protect\hyperlink{after-top}{Continue reading the main story}

\href{/section/opinion}{Opinion}

Supported by

\protect\hyperlink{after-sponsor}{Continue reading the main story}

\hypertarget{lets-scrap-the-presidential-debates}{%
\section{Let's Scrap the Presidential
Debates}\label{lets-scrap-the-presidential-debates}}

They've become unrevealing quip contests.

By Elizabeth Drew

Ms. Drew is a journalist based in Washington.

\begin{itemize}
\item
  Aug. 3, 2020
\item
  \begin{itemize}
  \item
  \item
  \item
  \item
  \item
  \item
  \end{itemize}
\end{itemize}

\includegraphics{https://static01.nyt.com/images/2020/08/03/opinion/03drew1/03drew1-articleLarge.jpg?quality=75\&auto=webp\&disable=upscale}

Nervous managers of the
\href{https://ballotpedia.org/Presidential_debates,_2020}{scheduled 2020
presidential debates} are shuffling the logistics and locations to deal
with the threat of the coronavirus. But here's a better idea: Scrap them
altogether. And not for health reasons.

The debates have never made sense as a
\href{https://www.nytimes.com/2019/12/19/opinion/presidential-debate-alternatives.html}{test
for presidential leadership}. In fact, one could argue that they reward
precisely the opposite of what we want in a president. When we were
serious about the presidency, we wanted intelligence, thoughtfulness,
knowledge, empathy and, to be sure, likability. It should also go
without saying, dignity.

Yet the debates play an outsize role in campaigns and weigh more heavily
on the verdict than their true value deserves.

Perhaps the most substantive televised debate of all was
\href{https://learning.blogs.nytimes.com/2011/09/26/septe-26-1960-first-televised-presidential-debate/}{the
first one}, between John F. Kennedy and Richard Nixon, which Nixon was
considered to have won on substance on the radio, while the cooler and
more appealing Kennedy won on television. Since these weren't true
debates, the concept of ``winning'' one of these odd encounters was
always amorphous. (To be sure, many questions by panels of journalists
were designed less to stimulate debate than to challenge one of the
candidates.)

\includegraphics{https://static01.nyt.com/images/2020/08/04/opinion/03drew3/merlin_75655960_325e36f8-8421-4c4e-b014-1b44f7d93169-articleLarge.jpg?quality=75\&auto=webp\&disable=upscale}

Over time, the debates came to resemble professional wrestling matches,
and more substantive debates were widely panned in the press. Points
went to snappy comebacks and one-liners. Witty remarks drew laughs from
the audience and got repeated for days and remembered for years.

Some of them have been less than hilarious, but they did the job of
dominating reaction to a debate. Whatever substance existed was largely
ignored. In 1980, when Ronald Reagan debated the incumbent Jimmy Carter,
Carter made a serious point about Reagan's position on Medicare, and
Reagan's riposte,
\href{https://www.nytimes.com/politics/first-draft/2014/10/28/on-this-day-there-you-go-again/}{``There
you go again,''} a non-answer if ever there was one, brought down the
house and that was that.

In the first 1984 debate, Reagan, seeking re-election and at 73, the
oldest person to be nominated for the presidency, seemed tired and
tended to wander off mentally at times. His lackluster performance
caused panic among his staff. Democratic supporters of former Vice
President Walter Mondale saw an opening.

But another debate soon followed. Thoroughly prepared, Reagan
\href{http://content.time.com/time/specials/packages/article/0,28804,1844704_1844706_1844612,00.html}{got
off the crack}, ``I will not make age an issue of this campaign. I am
not going to exploit, for political purposes, my opponent's youth and
inexperience.''

Image

President Ronald Reagan during a debate with Walter F. Mondale in
October 1984.Credit...David Longstreath/Associated Press

The audience roared and Mr. Mondale feigned a laugh, knowing he was
cooked. Not even Reagan's ending of that debate, reminiscing about
driving along the Pacific Coast and musing about time capsules, was
enough to undermine his political prospects. Reagan's ``joke'' aimed at
nullifying the age issue dominated the post-debate chatter.

But what is the point or relevance of the carefully prepared one-liner?
It's as spontaneous as a can of sardines. It's usually delivered from a
memory chip in the mind, having been fashioned and rehearsed with aides.
When is a president called upon to put down an interlocutor, be it a
member of Congress or a foreign leader?

This, by the way, isn't written out of any concern that Donald Trump
will prevail over Joe Biden in the debates; Mr. Biden has done just fine
in a long string of such contests. The point is that ``winning'' a
debate, however assessed, should be irrelevant, as are the debates
themselves.

The better way to pay attention to and choose among the presidential
candidates is to follow the long campaign that so many complain about.
The reason for such moaning has always been a mystery, because unless
the campaign is taking place in your living room, you can simply switch
it off.

The key words are ``pay attention to,'' because over the stretch of
2015-2016 it wasn't impossible to see the implications of a Trump
presidency. Not just the vulgarity but the ignorance and insensitivity
and extreme narcissism were apparent more than a year before Election
Day.

Moreover, we didn't need the debates to tell us that Trump had chosen to
be the P.T. Barnum of American politics. For him, it was (and still is)
all about the show, about distracting the public from reality. It was
obvious that Mr. Trump had no real affinity for the working-class people
whose votes he was chasing. Nothing in his life suggested that his heart
was with struggling workers and farmers. It wasn't impossible to know
that he wasn't the skilled businessman he professed to be. His
bankruptcies and shady business practices and discrimination against
Black tenants were no secret.

The debates took us nowhere nearer the realities about arguably the most
disastrous president in our history. They became simply another tool in
his arsenal.

The party conventions, also vestigial organs of a political system that
no longer exists, are close to being done away with, if not for the
reasons they should be. There's no reason not to throw the presidential
debates on the trash heap of useless (at best) rituals that are no help
in our making such a fateful decision.

Elizabeth Drew, a political journalist who for many years covered
Washington for The New Yorker, is the author of ``Washington Journal:
Reporting Watergate and Richard Nixon's Downfall.''

\emph{The Times is committed to publishing}
\href{https://www.nytimes.com/2019/01/31/opinion/letters/letters-to-editor-new-york-times-women.html}{\emph{a
diversity of letters}} \emph{to the editor. We'd like to hear what you
think about this or any of our articles. Here are some}
\href{https://help.nytimes.com/hc/en-us/articles/115014925288-How-to-submit-a-letter-to-the-editor}{\emph{tips}}\emph{.
And here's our email:}
\href{mailto:letters@nytimes.com}{\emph{letters@nytimes.com}}\emph{.}

\emph{Follow The New York Times Opinion section on}
\href{https://www.facebook.com/nytopinion}{\emph{Facebook}}\emph{,}
\href{http://twitter.com/NYTOpinion}{\emph{Twitter (@NYTopinion)}}
\emph{and}
\href{https://www.instagram.com/nytopinion/}{\emph{Instagram}}\emph{.}

Advertisement

\protect\hyperlink{after-bottom}{Continue reading the main story}

\hypertarget{site-index}{%
\subsection{Site Index}\label{site-index}}

\hypertarget{site-information-navigation}{%
\subsection{Site Information
Navigation}\label{site-information-navigation}}

\begin{itemize}
\tightlist
\item
  \href{https://help.nytimes.com/hc/en-us/articles/115014792127-Copyright-notice}{©~2020~The
  New York Times Company}
\end{itemize}

\begin{itemize}
\tightlist
\item
  \href{https://www.nytco.com/}{NYTCo}
\item
  \href{https://help.nytimes.com/hc/en-us/articles/115015385887-Contact-Us}{Contact
  Us}
\item
  \href{https://www.nytco.com/careers/}{Work with us}
\item
  \href{https://nytmediakit.com/}{Advertise}
\item
  \href{http://www.tbrandstudio.com/}{T Brand Studio}
\item
  \href{https://www.nytimes.com/privacy/cookie-policy\#how-do-i-manage-trackers}{Your
  Ad Choices}
\item
  \href{https://www.nytimes.com/privacy}{Privacy}
\item
  \href{https://help.nytimes.com/hc/en-us/articles/115014893428-Terms-of-service}{Terms
  of Service}
\item
  \href{https://help.nytimes.com/hc/en-us/articles/115014893968-Terms-of-sale}{Terms
  of Sale}
\item
  \href{https://spiderbites.nytimes.com}{Site Map}
\item
  \href{https://help.nytimes.com/hc/en-us}{Help}
\item
  \href{https://www.nytimes.com/subscription?campaignId=37WXW}{Subscriptions}
\end{itemize}
