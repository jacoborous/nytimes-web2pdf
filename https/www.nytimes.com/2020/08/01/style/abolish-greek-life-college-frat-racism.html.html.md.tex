Sections

SEARCH

\protect\hyperlink{site-content}{Skip to
content}\protect\hyperlink{site-index}{Skip to site index}

\href{https://www.nytimes.com/section/style}{Style}

\href{https://myaccount.nytimes.com/auth/login?response_type=cookie\&client_id=vi}{}

\href{https://www.nytimes.com/section/todayspaper}{Today's Paper}

\href{/section/style}{Style}\textbar{}The War on Frats

\url{https://nyti.ms/39PAc3k}

\begin{itemize}
\item
\item
\item
\item
\item
\item
\end{itemize}

\href{https://www.nytimes.com/news-event/george-floyd-protests-minneapolis-new-york-los-angeles?action=click\&pgtype=Article\&state=default\&region=TOP_BANNER\&context=storylines_menu}{Race
and America}

\begin{itemize}
\tightlist
\item
  \href{https://www.nytimes.com/interactive/2020/07/03/us/george-floyd-protests-crowd-size.html?action=click\&pgtype=Article\&state=default\&region=TOP_BANNER\&context=storylines_menu}{Black
  Lives Matter Movement}
\item
  \href{https://www.nytimes.com/interactive/2020/06/28/us/i-cant-breathe-police-arrest.html?action=click\&pgtype=Article\&state=default\&region=TOP_BANNER\&context=storylines_menu}{History
  of `I Can't Breathe'}
\item
  \href{https://www.nytimes.com/interactive/2020/06/10/upshot/black-lives-matter-attitudes.html?action=click\&pgtype=Article\&state=default\&region=TOP_BANNER\&context=storylines_menu}{How
  Public Opinion Shifted}
\item
  \href{https://www.nytimes.com/interactive/2020/07/16/us/black-lives-matter-protests-louisville-breonna-taylor.html?action=click\&pgtype=Article\&state=default\&region=TOP_BANNER\&context=storylines_menu}{45
  Days in Louisville}
\end{itemize}

Advertisement

\protect\hyperlink{after-top}{Continue reading the main story}

Supported by

\protect\hyperlink{after-sponsor}{Continue reading the main story}

\hypertarget{the-war-on-frats}{%
\section{The War on Frats}\label{the-war-on-frats}}

Groups of fraternity brothers and sorority sisters are working to kick
their organizations off campus.

By \href{https://www.nytimes.com/by/ezra-marcus}{Ezra Marcus}

\includegraphics{https://static01.nyt.com/images/2020/07/30/fashion/30abolishfrats-students-1/30abolishfrats-students--articleLarge.jpg?quality=75\&auto=webp\&disable=upscale}

Aug. 1, 2020

In the past month, hundreds of students have dropped out of their
fraternities and sororities at Vanderbilt University. They have
gathered, digitally, using group-run Instagram activist pages. They have
written searing
\href{https://vanderbilthustler.com/33211/featured/schulman-drop/}{op-eds}
\href{https://vanderbilthustler.com/33241/featured/guest-editorial-a-message-from-the-former-brothers-of-delta-tau-delta/}{condemning}
their own organizations for the student newspaper, The Vanderbilt
Hustler.

And they have
\href{https://www.change.org/p/vanderbilt-university-abolish-ifc-and-panhellenic-organizations-at-vanderbilt-university?utm_content=cl_sharecopy_23388057_en-US\%3A6\&recruiter=316950555\&recruited_by_id=962f0bd0-1382-11e5-9061-9518c3aa4d04\&utm_source=share_petitio}{petitioned}
the administration to ban Greek organizations from campus.

The mass action, which has taken place while students have been away
from the Nashville campus for the summer and isolated because of the
pandemic, has been accelerated by a handful of racist incidents that
\href{https://news.vanderbilt.edu/2020/07/07/vanderbilt-university-statement-on-greek-life/}{have
been surfaced} in videos and on social media.

But students said their real reasons have deeper roots: that Greek life
is exclusionary, racist and misogynist, as well as resistant to reform
because of the hierarchical nature of the national Greek organizations,
which control local chapters.

Similar ``Abolish Greek Life'' movements have sprung up at other
universities around the country, including at the University of
Richmond, Duke, Emory, American University, Northwestern and the
University of North Carolina.

Emma Heck, 21, a senior at Emory who recently dropped out of the Pi Beta
Phi sorority, said, ``The national organizations are always going to
prohibit any real change.'' Max Ratelle, 21, a rising senior at Tufts,
said he dropped out of his fraternity because reform felt futile.
``We're just going to see history repeat itself over and over again,''
he said.

On Wednesday, the
\href{https://www.instagram.com/p/CDPpAbbFJSc/}{governing panel} of
sororities at Tufts announced in a statement that rush (when students
become acquainted with the different fraternities or sororities on
campus) would not take place in the fall as they ``decide what the best
course of action is for Greek Life at Tufts'' and continue to examine
``the structurally and situationally problematic nature of Greek Life.''

The movement at Vanderbilt has been the biggest so far, with many
students leaving several prominent fraternity and sorority chapters
there, including Delta Tau Delta and Kappa Kappa Gamma.

Both national organizations said that membership numbers remained
healthy; Delta Tau Delta said that ``approximately a third'' of the
Vanderbilt chapter had disaffiliated, and Kappa Kappa Gamma said ``a
majority of our women at Vanderbilt University remain members.'' In both
cases, formal disaffiliation requires that each student submit
paperwork; at Kappa Kappa Gamma, there is a waiting period of several
weeks.

Both organizations stated their commitment to supporting remaining
members in efforts to address and reform issues within the Greek system
and outside it.

Taylor Thompson, 21, a rising senior at Vanderbilt University, was one
of the first to leave the Kappa Kappa Gamma sorority **** in late May,
after the death of George Floyd in police custody. As protests flared
around the country, Ms. Thompson, who is Black, said there were no
efforts from her sorority sisters to discuss anti-racist action.

``Nothing was being talked about in our group chat except for, like, a
trip to Vegas,'' Ms. Thompson said. She sent the chat a message
expressing ``disappointment that whenever something like this happens,
I'm the first person to bring it up or another person of color is,'' she
said, and urged her sisters, most of whom are white, to share resources
and make donations related to the protests.

At first, reception was positive. Lots of her sisters ``liked'' her
comment, and the conversation flowed for an hour or so. But it soon fell
off track.

She and four other women of color decided to quit. ``I didn't want to
continue to have to spend all my time educating all the girls around
me,'' Ms. Thompson said. ``We've had countless, you know, diversity
inclusion sessions and workshops, and everybody is, quote unquote,
trying. But the fruits of that labor don't really show up when it means
the most.''

Image

From left, Taylor Thompson, Marty Grady, Kelly Morgan, Grace Jennings,
Lyndsey Delouya, Katherine Deegan, Emma Pinto.Credit...Photo
Illustration by The New York Times

\hypertarget{the-jock-line}{%
\subsection{The Jock Line}\label{the-jock-line}}

Vanderbilt is an unlikely place for an anti-Greek life movement. John
Hechinger, the author of
``\href{https://www.publicaffairsbooks.com/titles/john-hechinger/true-gentlemen/9781610396837/}{True
Gentlemen: The Broken Pledge of America's Fraternities},'' said
Vanderbilt had been ``a real stronghold'' of Greek power in the country.
The university has hosted fraternities since 1873, the year it was
founded.

Today, according to
\href{https://www.vanderbilt.edu/greek_life/about-us/}{Vanderbilt}, more
than 35 percent of the nearly 7,000 undergraduates there belong to a
Greek life organization, which are housed in 25 on-campus buildings. But
there is a historical precedent for students walking out of their
fraternities and sororities. During the civil rights movement in the
1960s, students rejected Greek life as a bastion of reactionary politics
and racism, and dropped their affiliation en masse. Some local chapters
disbanded.

In 1968, a group of student activists occupied a Columbia University
administration building during a protest. According to the historian
Paul Cronin, these students faced off in a violent clash with a
counterrevolutionary group calling itself the Majority Coalition, which
consisted mainly of conservative athletes and fraternity brothers. (``A
row of clean-shaven white men, mostly wearing jackets and ties, punched
away as students and outsiders tried to bash through what they called
the Jock Line,'' Mr. Cronin
\href{https://www.politico.com/news/magazine/2020/06/07/barr-protesters-columbia-1968-304556}{wrote}
in Politico.)

One of those in the Majority Coalition is the current attorney general,
William P. Barr, who belonged to the Sigma Nu fraternity.

Attorney General Barr is far from the only powerful government figure
with Greek ties. Eighteen United States presidents, both Democratic and
Republican, have belonged to fraternities, along with scores of other
politicians and titans of industry. Vanderbilt fraternity alumni include
William Bain, the co-founder of the consultancy giant Bain \& Company,
the Republican Senator Lamar Alexander and the Democratic governor of
Kentucky, Andy Beshear.

The promise of networking connections and camaraderie is a large part of
the draw. At many schools, fraternities and sororities run the social
scene and throw the biggest parties. Since 1984, when the drinking age
rose to 21 nationwide, fraternities became the ``unofficial bartenders''
of many campuses, Mr. Hechinger said.

But with the pandemic preventing many students from going back to campus
in the fall, Greek organizations have less to offer in a social sense.
Fraternity and sorority dues, about 50 percent of which often go to the
national organizations, are harder to justify.

Ms. Thompson helps run the Instagram account
\href{https://www.instagram.com/abolishvandyifcandpanhellenic/}{@abolishvandyifcandpanhellenic},
which urges students to drop their Greek affiliation and publishes
anonymous and signed submissions from students about their negative
experiences with Greek life. In early July someone sent her a video in
which a white frat brother from Delta Kappa Epsilon yelled a racist slur
at several white Kappa Alpha Theta sisters, one of whom was wearing what
appeared to be a mock durag. Ms. Thompson published it on her own
Instagram on July 3.

Within hours, dozens of members of Kappa Alpha Theta began dropping
their affiliation; some began calling for a vote to remove the
organization's charter so that it could no longer operate on campus. The
sorority soon received an email from the advisory board chair for its
chapter, Mary Lee Bartlett, who graduated from Vanderbilt in 1985 and
works as a liaison between the current students and the national
organization.

``PLEASE! Zip your lips on these topics!'' Ms. Bartlett wrote in the
email, the phrase highlighted in gold for emphasis. She urged the
current sorority sisters not to speak to friends or family about either
the video or ``the interest members have expressed in either
surrendering the Charter and/or individually resigning.''

Someone leaked Ms. Bartlett's email to Emma Pinto, a Vanderbilt senior
who left Zeta Tau Alpha, and she posted that on Instagram as well. For
many angry students, it was a clear-cut example of the way the national
organizations put their reputations ahead of accountability. (Neither
Kappa Alpha Theta or Ms. Bartlett responded to requests for comment from
The New York Times.)

The email, Ms. Thompson said, made her and others ``critically examine''
why the organization would want to ``put a gag order on the girls in
that sorority.''

``What values,'' she said, ``do these organizations hold?''

Image

From left, Riya Doshi, Dannah Seecoomar, Jackie Miller, Katherine
Strauss, Gracie Pitman, Shane Ausmus, Edie Duncan.Credit...Photo
Illustration by The New York Times

\hypertarget{matters-of-concern}{%
\subsection{`Matters of Concern'}\label{matters-of-concern}}

In the past, the national organizations have been a moderating force on
Greek life, stepping in to limit hazing or try to prevent racist party
themes, Mr. Hechinger said. Now, though, many students think the
nationals are a barrier to reform.

A rising senior at Vanderbilt, who was allowed to speak anonymously for
privacy reasons, was, until late June, in a high-ranking leadership
position in the fraternity Delta Tau Delta. He said that a push for
outright abolishment could have been avoided if the administration and
national organizations had been more flexible about student concerns.

He said that initially he felt it was his responsibility as a leader in
his organization and ``someone with bargaining chips with the
administration'' to push for reforms to Greek life on campus, rather
than walking away completely.

He and several other fraternity members wrote up a policy memo and
arranged a meeting with Vanderbilt administrators, calling for reforms
that included a ban on Greek social dues and for redistribution of
campus resources. ``I was asking them to sign off on a housing
application that would allow Greek houses to be applied for by any
campus organization,'' he said, in order to ``redistribute some of the
social capital on campus.''

The meeting, which happened on June 29 with Kristin Torrey, the director
of Greek life at Vanderbilt, left several of the brothers feeling
dismissed. ``She just showed, like, total animosity and unwillingness to
change,'' the senior recalled.

Ms. Torrey did not respond to a request for comment from The Times. A
spokeswoman for Vanderbilt said the school was ``available to work with
students as they navigate reforms, while respecting students' autonomy
to create, sustain and lead various organizations as a part of the
college experience.''

A week later, a group of students from Delta Tau Delta had a call with a
representative from the national chapter, in which they expressed
concerns about fraternity conduct and felt similarly dismissed. After
that, many of the brothers, including all of its senior leadership,
decided to quit the fraternity. Twenty-seven of
them\href{https://vanderbilthustler.com/33241/featured/guest-editorial-a-message-from-the-former-brothers-of-delta-tau-delta/}{signed
a letter} in The Vanderbilt Hustler, calling for the end of Greek life
on campus.

``Our genuine efforts towards meaningful reform have been met with
systemic apathy and animosity,'' they wrote. ``Because of our failed
attempt at reform, those of us who have disaffiliated are adamant in our
call for the abolition of historically white Greek Life. To all those
harmed by Delta Tau Delta, we extend a sincere apology --- there is no
reversing the damage we have caused.''

Jack Kreman, the chief executive officer of Delta Tau Delta
International Fraternity, wrote in a statement that the national
organization ``believes calls to abolish fraternities fall short of
truly dealing with campus-wide cultural challenges,'' and reiterated its
commitment ``to working with the remaining members to address matters of
concern.''

Hundreds of Vanderbilt students began talking in lengthy group chats and
collaborating in Google Docs; according to Ms. Pinto, there is no
designated leader among them. ``This was collective organizing and
collective action,'' she said.

On July 7, three of the highest ranking fraternity brothers at
Vanderbilt --- Callen DiGiovanni, who was the student president of the
Interfraternity Council; Joshua Allen, who was the student attorney
general; and Alex Snape, who was student vice president of housing ---
wrote
\href{https://medium.com/@callendigi/we-resigned-from-thevanderbilt-university-interfraternity-council-a3a3378fe4b3}{a
Medium post resigning from their positions}.

``To the students and alumni who have been harmed by our organization,
we sincerely apologize,'' they wrote. ``We know that words don't erase
the past, but hope that our action today will help this University move
beyond this toxic culture.''

Image

From left, Sabina Smith, Claire Conway, Cedoni Francis, Charlotte
Hoigard, Angi Axelrode, Danielle Chari, Aaron NiedermanCredit...Photo
Illustration by The New York Times

\hypertarget{value-and-respect}{%
\subsection{Value and Respect}\label{value-and-respect}}

The Vanderbilt administration has promised to conduct an internal review
of Greek life next semester.

The university said in a statement: ``We respect the right of students
to join or disaffiliate with any registered student organization.'' The
statement reiterated that all members of student organizations are
expected to ``adhere to high standards of conduct aligned with our
commitment to a safe, welcoming and inclusive campus for everyone" and
that ``when student conduct violations occur, we take action,
investigate and hold perpetrators accountable --- both individuals and
organizations.''

But just because pressure is building on students to walk away from the
Greek system does not mean that the majority of those in fraternities or
sororities want to leave them.

In college student newspapers, some have criticized the Abolish Greek
Life movement for painting fraternities and sororities with too broad a
brush. ``AGL is wrong to characterize every Greek chapter as
irredeemable,'' wrote Jared Bauman, a Vanderbilt student, in a
Vanderbilt Hustler
\href{https://vanderbilthustler.com/33363/featured/guest-editorial-how-abolish-greek-life-gets-it-wrong/}{op-ed},
referring to the Abolish Greek Life movement. ``My fraternity might not
be perfect, but it's a far cry from the image of abject depravity that
AGL projects.''

Similar debates are taking place at other schools. ``The current
movement to abolish the Greek life system at Duke fails to consider
people like
me,''\href{https://www.dukechronicle.com/article/2020/07/being-gay-in-greek-life-ato}{wrote}
a senior named AJ Whitney in a letter to The Duke Chronicle. Mr. Whitney
belongs to the Alpha Tau Omega fraternity, and wrote that he is openly
gay. ``I have never felt more accepted, valued and respected by any
other community, both at Duke and throughout my life, than by my Taus,''
he wrote.

Most students who are members of historically Black fraternities and
sororities have no plans to drop; at Vanderbilt and other schools, the
Abolish Greek Life movement is targeted at historically white Greek life
organizations, which fall into either the Interfraternity Council or the
National Panhellenic Conference. (Most historically Black Greek
organizations belong to the National Pan-Hellenic Council.)

For much of their history, fraternities and sororities were segregated;
the charters of many organizations explicitly prohibited nonwhite
non-Christians from joining.

Though white fraternities and sororities were officially desegregated by
the end of the 1960s, many local
chapters\href{https://academic.oup.com/socpro/article-abstract/57/4/653/1667281?redirectedFrom=fulltext}{continued}
to informally prioritize new white members.

In an interview, Shelby Hart, a Vanderbilt junior who belongs to the
Black sorority Alpha Kappa Alpha, said that Black Greek life can provide
a separation from the prejudice of white Greek life. ``I have heard of
many occurrences where even people of color within these organizations
faced racism through other members using racial slurs,'' she said. ``I
know people who've been called the N-word on campus by other students.''

``The university is doing a very strategic higher-ed method that a lot
of universities apply when there is some type of prejudice scandal that
occurs on campus,' she said. ``They say that this `does not align with
our values.'''

This, she believes, is an attempt to dodge accountability and avoid
substantive changes. ``The rhetoric the university has used makes it
seem as if these events are isolated, and that these events do not
reflect Vanderbilt's culture,'' she said. ``However, they do.''

Advertisement

\protect\hyperlink{after-bottom}{Continue reading the main story}

\hypertarget{site-index}{%
\subsection{Site Index}\label{site-index}}

\hypertarget{site-information-navigation}{%
\subsection{Site Information
Navigation}\label{site-information-navigation}}

\begin{itemize}
\tightlist
\item
  \href{https://help.nytimes.com/hc/en-us/articles/115014792127-Copyright-notice}{©~2020~The
  New York Times Company}
\end{itemize}

\begin{itemize}
\tightlist
\item
  \href{https://www.nytco.com/}{NYTCo}
\item
  \href{https://help.nytimes.com/hc/en-us/articles/115015385887-Contact-Us}{Contact
  Us}
\item
  \href{https://www.nytco.com/careers/}{Work with us}
\item
  \href{https://nytmediakit.com/}{Advertise}
\item
  \href{http://www.tbrandstudio.com/}{T Brand Studio}
\item
  \href{https://www.nytimes.com/privacy/cookie-policy\#how-do-i-manage-trackers}{Your
  Ad Choices}
\item
  \href{https://www.nytimes.com/privacy}{Privacy}
\item
  \href{https://help.nytimes.com/hc/en-us/articles/115014893428-Terms-of-service}{Terms
  of Service}
\item
  \href{https://help.nytimes.com/hc/en-us/articles/115014893968-Terms-of-sale}{Terms
  of Sale}
\item
  \href{https://spiderbites.nytimes.com}{Site Map}
\item
  \href{https://help.nytimes.com/hc/en-us}{Help}
\item
  \href{https://www.nytimes.com/subscription?campaignId=37WXW}{Subscriptions}
\end{itemize}
