Sections

SEARCH

\protect\hyperlink{site-content}{Skip to
content}\protect\hyperlink{site-index}{Skip to site index}

\href{https://www.nytimes.com/section/science}{Science}

\href{https://myaccount.nytimes.com/auth/login?response_type=cookie\&client_id=vi}{}

\href{https://www.nytimes.com/section/todayspaper}{Today's Paper}

\href{/section/science}{Science}\textbar{}Can Humans Give Coronavirus to
Bats, and Other Wildlife?

\url{https://nyti.ms/3fr6ZNn}

\begin{itemize}
\item
\item
\item
\item
\item
\end{itemize}

\href{https://www.nytimes.com/news-event/coronavirus?action=click\&pgtype=Article\&state=default\&region=TOP_BANNER\&context=storylines_menu}{The
Coronavirus Outbreak}

\begin{itemize}
\tightlist
\item
  live\href{https://www.nytimes.com/2020/08/03/world/coronavirus-covid-19.html?action=click\&pgtype=Article\&state=default\&region=TOP_BANNER\&context=storylines_menu}{Latest
  Updates}
\item
  \href{https://www.nytimes.com/interactive/2020/us/coronavirus-us-cases.html?action=click\&pgtype=Article\&state=default\&region=TOP_BANNER\&context=storylines_menu}{Maps
  and Cases}
\item
  \href{https://www.nytimes.com/interactive/2020/science/coronavirus-vaccine-tracker.html?action=click\&pgtype=Article\&state=default\&region=TOP_BANNER\&context=storylines_menu}{Vaccine
  Tracker}
\item
  \href{https://www.nytimes.com/2020/08/02/us/covid-college-reopening.html?action=click\&pgtype=Article\&state=default\&region=TOP_BANNER\&context=storylines_menu}{College
  Reopening}
\item
  \href{https://www.nytimes.com/live/2020/08/03/business/stock-market-today-coronavirus?action=click\&pgtype=Article\&state=default\&region=TOP_BANNER\&context=storylines_menu}{Economy}
\end{itemize}

Advertisement

\protect\hyperlink{after-top}{Continue reading the main story}

Supported by

\protect\hyperlink{after-sponsor}{Continue reading the main story}

\hypertarget{can-humans-give-coronavirus-to-bats-and-other-wildlife}{%
\section{Can Humans Give Coronavirus to Bats, and Other
Wildlife?}\label{can-humans-give-coronavirus-to-bats-and-other-wildlife}}

Federal agencies suggest caution in U.S. bat research to avoid
transmitting the novel coronavirus to wildlife.

\includegraphics{https://static01.nyt.com/images/2020/08/04/science/29VIRUS-BATS1/29VIRUS-BATS1-articleLarge.jpg?quality=75\&auto=webp\&disable=upscale}

\href{https://www.nytimes.com/by/james-gorman}{\includegraphics{https://static01.nyt.com/images/2018/02/16/multimedia/author-james-gorman/author-james-gorman-thumbLarge.jpg}}

By \href{https://www.nytimes.com/by/james-gorman}{James Gorman}

\begin{itemize}
\item
  Aug. 1, 2020
\item
  \begin{itemize}
  \item
  \item
  \item
  \item
  \item
  \end{itemize}
\end{itemize}

Many people worry about bats as a source of viruses, including the one
that has caused a worldwide pandemic. But another question is surfacing:
Could humans pass the novel coronavirus to wildlife, specifically North
American bats?

It may seem like the last pandemic worry right now, far down the line
after concerns about getting sick and staying employed. But as the
spread of the novel coronavirus has made clear, the more careful we are
about viruses passing among species, the better off we are.

The scientific consensus is that the virus originated in bats in China
or neighboring countries. A recent paper tracing the genetic lineage of
the novel
virus\href{https://www.nature.com/articles/s41564-020-0771-4}{found
evidence that it probably evolved in bats into its current form.}The
researchers also concluded that either this coronavirus or others that
could make the jump to humans are likely present in bat populations now
--- we just haven't found them yet.

So why worry about infecting new bats with the current virus? The
federal government considers it a legitimate concern both for bat
populations, which have been devastated by a fungal disease called
white-nose syndrome, and for humans, given potential problems down the
road.

The U.S. Geological Survey and the Fish and Wildlife Service, two
agencies involved in research on bats, took the issue seriously enough
to convene a panel of 12 experts to analyze the likelihood of
human-to-bat transmission of the virus, SARS-CoV-2, in North America.

\hypertarget{latest-updates-global-coronavirus-outbreak}{%
\section{\texorpdfstring{\href{https://www.nytimes.com/2020/08/03/world/coronavirus-covid-19.html?action=click\&pgtype=Article\&state=default\&region=MAIN_CONTENT_1\&context=storylines_live_updates}{Latest
Updates: Global Coronavirus
Outbreak}}{Latest Updates: Global Coronavirus Outbreak}}\label{latest-updates-global-coronavirus-outbreak}}

Updated 2020-08-04T05:55:16.339Z

\begin{itemize}
\tightlist
\item
  \href{https://www.nytimes.com/2020/08/03/world/coronavirus-covid-19.html?action=click\&pgtype=Article\&state=default\&region=MAIN_CONTENT_1\&context=storylines_live_updates\#link-4547638f}{Fauci
  defends Birx after she is criticized by Trump.}
\item
  \href{https://www.nytimes.com/2020/08/03/world/coronavirus-covid-19.html?action=click\&pgtype=Article\&state=default\&region=MAIN_CONTENT_1\&context=storylines_live_updates\#link-15e7f995}{Trump
  derides Democrats as lawmakers and administration officials try to
  break stimulus impasse.}
\item
  \href{https://www.nytimes.com/2020/08/03/world/coronavirus-covid-19.html?action=click\&pgtype=Article\&state=default\&region=MAIN_CONTENT_1\&context=storylines_live_updates\#link-e5a2cda}{The
  deadline for 2020 census counting has been moved up by a month.}
\end{itemize}

\href{https://www.nytimes.com/2020/08/03/world/coronavirus-covid-19.html?action=click\&pgtype=Article\&state=default\&region=MAIN_CONTENT_1\&context=storylines_live_updates}{See
more updates}

More live coverage:
\href{https://www.nytimes.com/live/2020/08/03/business/stock-market-today-coronavirus?action=click\&pgtype=Article\&state=default\&region=MAIN_CONTENT_1\&context=storylines_live_updates}{Markets}

Another team of scientists, mostly from the two agencies, assessed the
expert opinions and issued a report
\href{https://doi.org/10.3133/ofr20201060}{in June. They concluded} that
there is some risk, although how much is hard to pin down. Taking
precautions, like wearing masks, gloves and protective clothing, could
significantly cut it down.

Kevin Olival, a vice president for research at EcoHealth Alliance, an
independent group and an author of the report, said that as the virus
began to spread around the globe, ``there was a real concern that not
only North American but wildlife populations all over the world could be
exposed.''

While the group studied interactions between North American bats and
scientific researchers, Dr. Olival said wildlife-control workers and
people who rehabilitate injured bats, for example, may come into contact
with bats even more than researchers do.

Evaluating risk meant trying to cope with unknowns piled on unknowns:
the risk of an infected research scientist or wildlife worker
encountering bats; the risk of the bats becoming infected in that
situation; the risk of an infected bat passing the virus onto other bats
so that the virus becomes established in the population.

The authors of the paper concluded there was a risk of humans infecting
bats with the novel coronavirus. How much risk? You might say little, or
small, or unknown, but this report is from two federal agencies, so it
describes the risk as ``non-negligible.''

Although the issue of how bat researchers should conduct their work may
seem narrow, the potential consequences are broad. The report notes that
if SARS-CoV-2 became established in North American bats, it would allow
the virus to keep propagating in animals even if it didn't cause
disease. And the virus could potentially spill back over to humans after
this pandemic is contained.

\includegraphics{https://static01.nyt.com/images/2020/08/04/science/30VIRUS-BATS2/merlin_174432909_1e54f4e4-198c-4402-a758-4854e18ea426-articleLarge.jpg?quality=75\&auto=webp\&disable=upscale}

Another concern involves how readily the coronavirus might spread from
bats to other kinds of wildlife or domestic animals, including pets.
Scientists have already shown that domestic cats and big cats can become
infected, and domestic cats can infect each other. Ferrets are easily
infected, as are minks. On the suspicion that they may be passing the
disease to people,
\href{https://www.nytimes.com/reuters/2020/07/16/world/europe/16reuters-health-coronavirus-spain-minks.html}{Spain
and the Netherlands have slaughtered thousands of minks at fur farms.}

A small number of infected pets has gotten a good deal of publicity. But
public health authorities like the Centers for Disease Control and
Prevention have said that, although information is limited,
\href{https://www.cdc.gov/coronavirus/2019-ncov/daily-life-coping/pets.html}{the
risk of pets spreading the virus to people} is low. They do recommend
that any person who has Covid-19 take the same precautions with their
pets that they would with human family members.
\href{https://www.nationalgeographic.com/animals/2020/07/first-dog-to-test-positive-for-covid-in-us-dies/}{National
Geographic reported Thursday that the first U. S. dog known to have
tested positive for SARS-CoV-2, had died}. The dog, Buddy, apparently
had lymphoma.

\href{https://www.nytimes.com/news-event/coronavirus?action=click\&pgtype=Article\&state=default\&region=MAIN_CONTENT_3\&context=storylines_faq}{}

\hypertarget{the-coronavirus-outbreak-}{%
\subsubsection{The Coronavirus Outbreak
›}\label{the-coronavirus-outbreak-}}

\hypertarget{frequently-asked-questions}{%
\paragraph{Frequently Asked
Questions}\label{frequently-asked-questions}}

Updated August 3, 2020

\begin{itemize}
\item ~
  \hypertarget{im-a-small-business-owner-can-i-get-relief}{%
  \paragraph{I'm a small-business owner. Can I get
  relief?}\label{im-a-small-business-owner-can-i-get-relief}}

  \begin{itemize}
  \tightlist
  \item
    The
    \href{https://www.nytimes.com/article/small-business-loans-stimulus-grants-freelancers-coronavirus.html?action=click\&pgtype=Article\&state=default\&region=MAIN_CONTENT_3\&context=storylines_faq}{stimulus
    bills enacted in March} offer help for the millions of American
    small businesses. Those eligible for aid are businesses and
    nonprofit organizations with fewer than 500 workers, including sole
    proprietorships, independent contractors and freelancers. Some
    larger companies in some industries are also eligible. The help
    being offered, which is being managed by the Small Business
    Administration, includes the Paycheck Protection Program and the
    Economic Injury Disaster Loan program. But lots of folks have
    \href{https://www.nytimes.com/interactive/2020/05/07/business/small-business-loans-coronavirus.html?action=click\&pgtype=Article\&state=default\&region=MAIN_CONTENT_3\&context=storylines_faq}{not
    yet seen payouts.} Even those who have received help are confused:
    The rules are draconian, and some are stuck sitting on
    \href{https://www.nytimes.com/2020/05/02/business/economy/loans-coronavirus-small-business.html?action=click\&pgtype=Article\&state=default\&region=MAIN_CONTENT_3\&context=storylines_faq}{money
    they don't know how to use.} Many small-business owners are getting
    less than they expected or
    \href{https://www.nytimes.com/2020/06/10/business/Small-business-loans-ppp.html?action=click\&pgtype=Article\&state=default\&region=MAIN_CONTENT_3\&context=storylines_faq}{not
    hearing anything at all.}
  \end{itemize}
\item ~
  \hypertarget{what-are-my-rights-if-i-am-worried-about-going-back-to-work}{%
  \paragraph{What are my rights if I am worried about going back to
  work?}\label{what-are-my-rights-if-i-am-worried-about-going-back-to-work}}

  \begin{itemize}
  \tightlist
  \item
    Employers have to provide
    \href{https://www.osha.gov/SLTC/covid-19/standards.html}{a safe
    workplace} with policies that protect everyone equally.
    \href{https://www.nytimes.com/article/coronavirus-money-unemployment.html?action=click\&pgtype=Article\&state=default\&region=MAIN_CONTENT_3\&context=storylines_faq}{And
    if one of your co-workers tests positive for the coronavirus, the
    C.D.C.} has said that
    \href{https://www.cdc.gov/coronavirus/2019-ncov/community/guidance-business-response.html}{employers
    should tell their employees} -\/- without giving you the sick
    employee's name -\/- that they may have been exposed to the virus.
  \end{itemize}
\item ~
  \hypertarget{should-i-refinance-my-mortgage}{%
  \paragraph{Should I refinance my
  mortgage?}\label{should-i-refinance-my-mortgage}}

  \begin{itemize}
  \tightlist
  \item
    \href{https://www.nytimes.com/article/coronavirus-money-unemployment.html?action=click\&pgtype=Article\&state=default\&region=MAIN_CONTENT_3\&context=storylines_faq}{It
    could be a good idea,} because mortgage rates have
    \href{https://www.nytimes.com/2020/07/16/business/mortgage-rates-below-3-percent.html?action=click\&pgtype=Article\&state=default\&region=MAIN_CONTENT_3\&context=storylines_faq}{never
    been lower.} Refinancing requests have pushed mortgage applications
    to some of the highest levels since 2008, so be prepared to get in
    line. But defaults are also up, so if you're thinking about buying a
    home, be aware that some lenders have tightened their standards.
  \end{itemize}
\item ~
  \hypertarget{what-is-school-going-to-look-like-in-september}{%
  \paragraph{What is school going to look like in
  September?}\label{what-is-school-going-to-look-like-in-september}}

  \begin{itemize}
  \tightlist
  \item
    It is unlikely that many schools will return to a normal schedule
    this fall, requiring the grind of
    \href{https://www.nytimes.com/2020/06/05/us/coronavirus-education-lost-learning.html?action=click\&pgtype=Article\&state=default\&region=MAIN_CONTENT_3\&context=storylines_faq}{online
    learning},
    \href{https://www.nytimes.com/2020/05/29/us/coronavirus-child-care-centers.html?action=click\&pgtype=Article\&state=default\&region=MAIN_CONTENT_3\&context=storylines_faq}{makeshift
    child care} and
    \href{https://www.nytimes.com/2020/06/03/business/economy/coronavirus-working-women.html?action=click\&pgtype=Article\&state=default\&region=MAIN_CONTENT_3\&context=storylines_faq}{stunted
    workdays} to continue. California's two largest public school
    districts --- Los Angeles and San Diego --- said on July 13, that
    \href{https://www.nytimes.com/2020/07/13/us/lausd-san-diego-school-reopening.html?action=click\&pgtype=Article\&state=default\&region=MAIN_CONTENT_3\&context=storylines_faq}{instruction
    will be remote-only in the fall}, citing concerns that surging
    coronavirus infections in their areas pose too dire a risk for
    students and teachers. Together, the two districts enroll some
    825,000 students. They are the largest in the country so far to
    abandon plans for even a partial physical return to classrooms when
    they reopen in August. For other districts, the solution won't be an
    all-or-nothing approach.
    \href{https://bioethics.jhu.edu/research-and-outreach/projects/eschool-initiative/school-policy-tracker/}{Many
    systems}, including the nation's largest, New York City, are
    devising
    \href{https://www.nytimes.com/2020/06/26/us/coronavirus-schools-reopen-fall.html?action=click\&pgtype=Article\&state=default\&region=MAIN_CONTENT_3\&context=storylines_faq}{hybrid
    plans} that involve spending some days in classrooms and other days
    online. There's no national policy on this yet, so check with your
    municipal school system regularly to see what is happening in your
    community.
  \end{itemize}
\item ~
  \hypertarget{is-the-coronavirus-airborne}{%
  \paragraph{Is the coronavirus
  airborne?}\label{is-the-coronavirus-airborne}}

  \begin{itemize}
  \tightlist
  \item
    The coronavirus
    \href{https://www.nytimes.com/2020/07/04/health/239-experts-with-one-big-claim-the-coronavirus-is-airborne.html?action=click\&pgtype=Article\&state=default\&region=MAIN_CONTENT_3\&context=storylines_faq}{can
    stay aloft for hours in tiny droplets in stagnant air}, infecting
    people as they inhale, mounting scientific evidence suggests. This
    risk is highest in crowded indoor spaces with poor ventilation, and
    may help explain super-spreading events reported in meatpacking
    plants, churches and restaurants.
    \href{https://www.nytimes.com/2020/07/06/health/coronavirus-airborne-aerosols.html?action=click\&pgtype=Article\&state=default\&region=MAIN_CONTENT_3\&context=storylines_faq}{It's
    unclear how often the virus is spread} via these tiny droplets, or
    aerosols, compared with larger droplets that are expelled when a
    sick person coughs or sneezes, or transmitted through contact with
    contaminated surfaces, said Linsey Marr, an aerosol expert at
    Virginia Tech. Aerosols are released even when a person without
    symptoms exhales, talks or sings, according to Dr. Marr and more
    than 200 other experts, who
    \href{https://academic.oup.com/cid/article/doi/10.1093/cid/ciaa939/5867798}{have
    outlined the evidence in an open letter to the World Health
    Organization}.
  \end{itemize}
\end{itemize}

As to the susceptibility of North American bats, Dr. Olival was not
aware of any published work on whether they can be infected with the
virus. Researchers in Hong Kong have reported that
\href{https://www.nature.com/articles/s41591-020-0912-6}{in a lab the
coronavirus infected the intestinal cells of Chinese rufous horseshoe
bats.} A report this month
\href{https://www.thelancet.com/journals/lanmic/article/PIIS2666-5247(20)30089-6/fulltext}{in
The Lancet found that fruit bats could become infected with the virus}.

Beyond bats, Dr. Olival said that scientists should be concerned about
how they conduct research on wildlife in general and consider what
precautions to take to avoid potentially infecting one species or
another. One step, he said, would be evaluating research goals to weigh
what level of contact would be necessary.

In some cases, he said, observation and data recording could be done
without handling animals. If not, gloves and other precautions make
sense, although some ``old-school'' researchers have balked at the
suggestions, he said.

He said his group continues to recommend, ``the highest level of
personal protective equipment when you work with wildlife, because it's
not just a risk that you will pick up something from the wildlife, but
that you don't give something back to them.''

He acknowledged that research precautions with wildlife will have a very
small effect, given the greater number of people who hunt wildlife or
come into contact in other ways. Education efforts are underway to try
to change some of those practices; in addition that, he said,
researchers ``should set some kind of standard.''

Advertisement

\protect\hyperlink{after-bottom}{Continue reading the main story}

\hypertarget{site-index}{%
\subsection{Site Index}\label{site-index}}

\hypertarget{site-information-navigation}{%
\subsection{Site Information
Navigation}\label{site-information-navigation}}

\begin{itemize}
\tightlist
\item
  \href{https://help.nytimes.com/hc/en-us/articles/115014792127-Copyright-notice}{©~2020~The
  New York Times Company}
\end{itemize}

\begin{itemize}
\tightlist
\item
  \href{https://www.nytco.com/}{NYTCo}
\item
  \href{https://help.nytimes.com/hc/en-us/articles/115015385887-Contact-Us}{Contact
  Us}
\item
  \href{https://www.nytco.com/careers/}{Work with us}
\item
  \href{https://nytmediakit.com/}{Advertise}
\item
  \href{http://www.tbrandstudio.com/}{T Brand Studio}
\item
  \href{https://www.nytimes.com/privacy/cookie-policy\#how-do-i-manage-trackers}{Your
  Ad Choices}
\item
  \href{https://www.nytimes.com/privacy}{Privacy}
\item
  \href{https://help.nytimes.com/hc/en-us/articles/115014893428-Terms-of-service}{Terms
  of Service}
\item
  \href{https://help.nytimes.com/hc/en-us/articles/115014893968-Terms-of-sale}{Terms
  of Sale}
\item
  \href{https://spiderbites.nytimes.com}{Site Map}
\item
  \href{https://help.nytimes.com/hc/en-us}{Help}
\item
  \href{https://www.nytimes.com/subscription?campaignId=37WXW}{Subscriptions}
\end{itemize}
