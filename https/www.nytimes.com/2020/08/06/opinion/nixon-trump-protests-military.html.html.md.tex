Sections

SEARCH

\protect\hyperlink{site-content}{Skip to
content}\protect\hyperlink{site-index}{Skip to site index}

\href{https://myaccount.nytimes.com/auth/login?response_type=cookie\&client_id=vi}{}

\href{https://www.nytimes.com/section/todayspaper}{Today's Paper}

\href{/section/opinion}{Opinion}\textbar{}Who Was Behind the Largest
Mass Arrest in U.S. History?

\href{https://nyti.ms/31nkb0B}{https://nyti.ms/31nkb0B}

\begin{itemize}
\item
\item
\item
\item
\item
\end{itemize}

Advertisement

\protect\hyperlink{after-top}{Continue reading the main story}

\href{/section/opinion}{Opinion}

Supported by

\protect\hyperlink{after-sponsor}{Continue reading the main story}

\hypertarget{who-was-behind-the-largest-mass-arrest-in-us-history}{%
\section{Who Was Behind the Largest Mass Arrest in U.S.
History?}\label{who-was-behind-the-largest-mass-arrest-in-us-history}}

Washington's police chief took the blame. But Nixon was behind the
decision.

By Lawrence Roberts

Mr. Roberts is the author of ``Mayday 1971: A White House at War, a
Revolt in the Streets, and the Untold History of America's Biggest Mass
Arrest.''

\begin{itemize}
\item
  Aug. 6, 2020
\item
  \begin{itemize}
  \item
  \item
  \item
  \item
  \item
  \end{itemize}
\end{itemize}

\includegraphics{https://static01.nyt.com/images/2020/08/07/opinion/07Roberts/06Roberts-articleLarge.jpg?quality=75\&auto=webp\&disable=upscale}

In the spring of 1971, Richard Nixon found himself in a situation not
unlike President Trump's. His approval rating was falling --- in Mr.
Nixon's case, to
\href{https://www.nytimes.com/1971/04/01/archives/gallup-poll-finds-support-for-nixon-at-50-lowest-yet.html}{a
first-term low} --- just as an energetic social movement was hitting the
streets. Like Mr. Trump, Mr. Nixon was tempted to use military force to
counter those dissenters. And like the current president, Mr. Nixon and
his aides found a way around the Pentagon's resistance.

The occasion was the most audacious plan yet by the six-year-old
movement against the Vietnam War. A group called the Mayday Tribe
organized a traffic blockade of Washington under the slogan ``If the
government won't stop the war, we'll stop the government.''

As the Mayday action unfolded on May 3, twin-engine Chinook helicopters
roared down by the Washington Monument, disgorging troops from the 82nd
Airborne Division, who trotted off to the Capitol and other hot spots.
In all, the administration summoned 10,000 soldiers and Marines, turning
``the center of the nation's capital into an armed camp with thousands
of troops lining the bridges and principal streets, helicopters whirring
overhead and helmeted police charging crowds of civilians with
nightsticks and tear gas,''
\href{https://www.nytimes.com/1971/05/04/archives/empty-victory.html}{according
to a New York Times report}. More than 12,000 people were swept up over
three days, the largest mass arrest in U.S. history.

John Dean, the Nixon aide who flipped on his boss in the Watergate
scandal, wrote
\href{https://www.nytimes.com/2020/07/31/opinion/trump-nixon-authoritarianism.html}{recently}
in The Times: ``Never once did I hear anyone in the Nixon White House or
Justice Department suggest using United States military forces, or any
federal officers outside the military, to quell civil unrest or
disorder. Nor have I found any evidence of such activity after the fact,
when digging through the historical record.''

Mr. Dean and I were there on Mayday (he was inside the White House; I
was on the streets). He has suggested that the troops were called by
city officials, not Mr. Nixon, and in any case weren't used offensively
to quell the blockade. I also dug through the historical record, for a
new book on those events, and came to quite a different conclusion. What
I found in White House tapes, in minutes of planning meetings and in the
papers of Mr. Nixon's aides, including those of his chief of staff, H.R.
Haldeman, and his chief domestic adviser, John Ehrlichman, left no doubt
that a half century ago, a president under siege resorted to military
force and mass arrests for political gain.

The Mayday protest was the finale of an extraordinary season of dissent.
After Mr. Nixon expanded the Vietnam War into Laos, hundreds of
thousands of protesters arrived in Washington for a variety of events.
Among them were Vietnam veterans, ``flower children,'' self-styled
revolutionaries and pacifists. Veterans hurled medals onto the Capitol's
steps. Quakers held pray-ins. A mass march, almost surely the biggest
the city had seen, stretched along the National Mall. Then, on the first
weekend in May, more than 40,000 people gathered by the Potomac River
for the Mayday action.

The antiwar movement had already helped turn public opinion against Mr.
Nixon's conduct of the war. He was determined to deny activists a
victory that could cause further political damage. He blasted them in
private with rants like ``Little bastards are draft dodgers,
country-haters or don't-cares.''(If Mr. Nixon had access to Twitter, his
tweets would have been eerily similar to Mr. Trump's.) He instructed
aides to ensure the blockade would fail and, as one put it, didn't care
if it took 100,000 troops, and if they came up short, ``someone will be
in big trouble.''

Mr. Nixon's men convened a war council with representatives of the
police, the military and the National Guard. Presiding was the deputy
attorney general, Richard Kleindienst. Washington didn't yet have home
rule, so the police chief, Jerry V. Wilson, answered to the White House.
Mr. Kleindienst and Mr. Ehrlichman batted away objections from Chief
Wilson and Army Lt. Gen. Hugh Exton, who questioned Mr. Kleindienst's
demand for 10,000 regular troops, given that thousands of police and
guardsmen were already available. They suggested such force might do
more to inflame the situation than calm it. Separately, Pentagon
officials told Mr. Kleindienst that his plan ``to combat dissent,'' as
they characterized it, might not comport with the rules. They reminded
him of the 1878 Posse Comitatus Act, which generally bans active duty
troops from law enforcement.

Mr. Kleindienst overrode their concerns with
\href{https://www.justice.gov/file/20836/download}{an opinion} from the
Justice Department's legal counsel, William Rehnquist, who had been his
protégé in their home state, Arizona. Mr. Rehnquist said the act didn't
apply; the president had ``inherent constitutional authority'' to use
troops ``to protect the functioning of the government.'' (Mr. Rehnquist
would be named to the Supreme Court by Mr. Nixon later that year and
elevated to chief justice under President Ronald Reagan.)

Mr. Kleindienst faced another obstacle. David Packard, the deputy
secretary of defense, pointed out the procedures a president should
follow, under the Insurrection Act, in calling forth the military: a
\href{https://www.everycrsreport.com/files/20060814_RS22266_c6617a8c1fc8c51828f9ab4d4a42de8366358c24.pdf}{formal
order} that demonstrators disperse and, if they don't, an executive
order to send in troops. Mr. Nixon's predecessor, Lyndon Johnson,
\href{https://history.army.mil/html/books/030/30-20/CMH_Pub_30-20.pdf}{had
done this} during the riots in Washington in 1968 after the
assassination of Martin Luther King Jr. The White House, however, wanted
to keep its involvement under wraps.
\href{https://www.nixonlibrary.gov/sites/default/files/virtuallibrary/documents/haldeman-diaries/37-hrhd-audiotape-ac07b-19710430-pa.pdf}{According
to Mr. Haldeman's diary}, Mr. Nixon let Mr. Packard know he wanted
troops sent without any public presidential action. The White House
spread the fake news that city officials had requested the military
help.

In contrast, Mr. Trump has been open about
\href{https://www.washingtonpost.com/opinions/2020/06/05/how-trump-came-brink-deploying-active-duty-troops-washington/}{his
desire to send troops} to ``dominate'' streets in cities with Black
Lives Matter protests. After Defense Secretary Mark Esper and the
chairman of the Joint Chiefs of Staff, Gen. Mark Milley,
\href{https://www.nytimes.com/2020/06/04/us/politics/trump-military-protests.html}{stood
in the way} of using active-duty military, the president dispatched
forces from agencies including Customs and Border Protection. In June,
those agents cleared peaceful demonstrators from Lafayette Square
outside the White House for the president's now-famous photo op in front
of a church. In Portland, Ore., they used tear gas and other riot tools
to disperse largely peaceful protesters outside the federal courthouse.

During the 1971 Mayday action, as 12,000 people tried to snarl rush-hour
traffic with nonviolent civil disobedience, a majority of the regular
troops fended off protesters at bridges and federal buildings, or
guarded large groups of detained protesters. Most soldiers didn't
confront demonstrators directly, but their presence and hardware
bolstered the authoritarian tactics and escalated tensions. A police
dragnet swept up 7,000 people that Monday, including many young people
just walking on the streets wearing hippie-style clothing, and took in
more than 5,000 other demonstrators over the next two days. My research
confirmed that Mr. Nixon gave the order to make the mass arrests. He
made it clear later to a group of conservative members of Congress:
``The point is, I had the responsibility,'' he told them. ``I approved
this plan.''

As criticism mounted that the dragnet was unconstitutional (courts
ultimately agreed, awarding detainees millions in damages), Mr. Nixon's
involvement was suspected. The White House denied it. Aides instructed
the police chief, Mr. Wilson, to take the heat. ``I wish to emphasize
the fact that I made all tactical decisions relating to the recent
disorders,'' he said in a public statement. ``I took these steps because
I felt they were necessary to protect the safety of law-abiding citizens
and to maintain order in the city.'' The tapes show Mr. Nixon's men were
delighted.

``Wilson went to the mat today,'' Mr. Ehrlichman confirmed to Mr. Nixon.
``Good for him!'' the president said. Mr. Ehrlichman added, ``We
programmed him to do this this morning, and he did better than you could
possibly have programmed.'' He went on: ``He has never let us down
yet.''

No military leader expressed second thoughts in the weeks after Mayday.

But in June, after General Milley accompanied Mr. Trump to Lafayette
Square wearing combat fatigues as protesters were dispersed by federal
agents and police, he said he regretted taking part.

``We must hold dear the principle of an apolitical military that is so
deeply rooted in the very essence of our republic,'' General Milley
\href{https://www.youtube.com/watch?v=7AKmmApwi0M}{told graduates} of
National Defense University. ``And this is not easy. It takes time and
work and effort, but it may be the most important thing each and every
one of us does every single day. And my second piece of advice is very
simple: Embrace the Constitution.''

\href{http://www.lawrenceproberts.com/}{Lawrence Roberts}, a former
editor at ProPublica and The Washington Post, is the author of
``\href{https://www.hmhco.com/shop/books/mayday-1971/9781328766724}{Mayday
1971:} A White House at War, a Revolt in the Streets, and the Untold
History of America's Biggest Mass Arrest.''

\begin{center}\rule{0.5\linewidth}{\linethickness}\end{center}

\emph{The Times is committed to publishing}
\href{https://www.nytimes.com/2019/01/31/opinion/letters/letters-to-editor-new-york-times-women.html}{\emph{a
diversity of letters}} \emph{to the editor. We'd like to hear what you
think about this or any of our articles. Here are some}
\href{https://help.nytimes.com/hc/en-us/articles/115014925288-How-to-submit-a-letter-to-the-editor}{\emph{tips}}\emph{.
And here's our email:}
\href{mailto:letters@nytimes.com}{\emph{letters@nytimes.com}}\emph{.}

\emph{Follow The New York Times Opinion section on}
\href{https://www.facebook.com/nytopinion}{\emph{Facebook}}\emph{,}
\href{http://twitter.com/NYTOpinion}{\emph{Twitter (@NYTopinion)}}
\emph{and}
\href{https://www.instagram.com/nytopinion/}{\emph{Instagram}}\emph{.}

Advertisement

\protect\hyperlink{after-bottom}{Continue reading the main story}

\hypertarget{site-index}{%
\subsection{Site Index}\label{site-index}}

\hypertarget{site-information-navigation}{%
\subsection{Site Information
Navigation}\label{site-information-navigation}}

\begin{itemize}
\tightlist
\item
  \href{https://help.nytimes.com/hc/en-us/articles/115014792127-Copyright-notice}{©~2020~The
  New York Times Company}
\end{itemize}

\begin{itemize}
\tightlist
\item
  \href{https://www.nytco.com/}{NYTCo}
\item
  \href{https://help.nytimes.com/hc/en-us/articles/115015385887-Contact-Us}{Contact
  Us}
\item
  \href{https://www.nytco.com/careers/}{Work with us}
\item
  \href{https://nytmediakit.com/}{Advertise}
\item
  \href{http://www.tbrandstudio.com/}{T Brand Studio}
\item
  \href{https://www.nytimes.com/privacy/cookie-policy\#how-do-i-manage-trackers}{Your
  Ad Choices}
\item
  \href{https://www.nytimes.com/privacy}{Privacy}
\item
  \href{https://help.nytimes.com/hc/en-us/articles/115014893428-Terms-of-service}{Terms
  of Service}
\item
  \href{https://help.nytimes.com/hc/en-us/articles/115014893968-Terms-of-sale}{Terms
  of Sale}
\item
  \href{https://spiderbites.nytimes.com}{Site Map}
\item
  \href{https://help.nytimes.com/hc/en-us}{Help}
\item
  \href{https://www.nytimes.com/subscription?campaignId=37WXW}{Subscriptions}
\end{itemize}
