Sections

SEARCH

\protect\hyperlink{site-content}{Skip to
content}\protect\hyperlink{site-index}{Skip to site index}

\href{https://www.nytimes.com/section/opinion/sunday}{Sunday Review}

\href{https://myaccount.nytimes.com/auth/login?response_type=cookie\&client_id=vi}{}

\href{https://www.nytimes.com/section/todayspaper}{Today's Paper}

\href{/section/opinion/sunday}{Sunday Review}\textbar{}A Song That
Changed Music Forever

\href{https://nyti.ms/33EG5PD}{https://nyti.ms/33EG5PD}

\begin{itemize}
\item
\item
\item
\item
\item
\end{itemize}

Advertisement

\protect\hyperlink{after-top}{Continue reading the main story}

\href{/section/opinion}{Opinion}

Supported by

\protect\hyperlink{after-sponsor}{Continue reading the main story}

\hypertarget{a-song-that-changed-music-forever}{%
\section{A Song That Changed Music
Forever}\label{a-song-that-changed-music-forever}}

100 years ago, Mamie Smith recorded a seminal blues hit that gave voice
to outrage at violence against Black Americans.

By David Hajdu

Mr. Hajdu is a cultural historian and music critic.

\begin{itemize}
\item
  Aug. 8, 2020, 11:00 a.m. ET
\item
  \begin{itemize}
  \item
  \item
  \item
  \item
  \item
  \end{itemize}
\end{itemize}

\includegraphics{https://static01.nyt.com/images/2020/08/09/opinion/08hajdu/merlin_55302586_2b17a26d-671b-4ba0-b149-c51bad2394a2-articleLarge.jpg?quality=75\&auto=webp\&disable=upscale}

On Aug. 10, 1920, two African-American musicians, Mamie Smith and Perry
Bradford, went into a New York studio and changed the course of music
history. Ms. Smith, then a modestly successful singer from Cincinnati
who had made only one other record, a sultry ballad that fizzled in the
marketplace, recorded a new song by Mr. Bradford called
``\href{https://www.youtube.com/watch?v=qaz4Ziw_CfQ}{Crazy Blues}.'' A
boisterous cry of outrage by a woman driven mad by mistreatment, the
song spoke with urgency and fire to Black listeners across the country
who had been ravaged by the abuses of race-hate groups, the police and
military forces in the preceding year --- the notorious
``\href{https://time.com/5636454/what-is-red-summer/}{Red Summer}'' of
1919.

``Crazy Blues'' became a hit record of unmatched proportions and
profound impact. Within a month of its release, it sold some 75,000
copies and would be reported to sell more than two million over time. It
established the blues as a popular art and prepared the way for a
century of Black expression in the fiery core of American music.

As a record, something made for private listening in the home, ``Crazy
Blues'' was able to say things rarely heard in public performances.
Seemingly a song about a woman whose man has left her, it reveals
itself, on close listening, to be a song about a woman moved to kill her
abusive partner. As a work of blues, it used the language of domestic
strife to tell a story of violence and subjugation that Black Americans
also knew outside the home, in a world of white oppression. The blues
worked on multiple levels simultaneously and partly in code, with ``my
man'' or ``the man'' translatable as ``the white man'' or ``white
people.''

Ms. Smith, a skilled contralto with a keen sense of drama, brought
clarity and panache to words that would strike today's listeners as
conventional only because they have been replicated and emulated in
countless variations over the past century: ``I can't sleep at night/ I
can't eat a bite/ 'Cause the man I love/ he don't treat me right.''

Out of her mind with despair, the singer turns to violence against her
oppressor for relief in the chorus that gives the song its title: ``Now
the doctor's gonna do all that he can/ But what you're gonna need is an
undertaker man/ I ain't had nothin' but bad news/ Now I've got the crazy
blues.''

That a woman was singing made the song an acutely potent message of
protest against the forces of authority, be they male or white, domestic
or sociopolitical.

With ``Crazy Blues,'' Mamie Smith opened the door to a surge of
powerfully voiced female singers who defied the conventions of singerly
gentility to make the blues a popular phenomenon in the 1920s. Indeed,
the blues became a full-blown craze, with listeners of every color able
to buy and listen at home to music marketed as ``race records.'' The
form was initially associated almost exclusively with women such as Ms.
Smith, Ma Rainey, Ethel Waters and Bessie Smith. They and many more
women made hundreds of records that sold millions of copies over more
than a decade --- well before the great bluesman Robert Johnson stepped
into a recording studio for the first time, in November 1936.

There had been some blues recordings before ``Crazy Blues,'' nearly all
instrumentals or records, often made by white musicians, of songs of
various kinds with the word ``Blues'' in the title. A feeling of
veracity as Black expression was part of the secret of ``Crazy Blues.''
But so was the song's disturbing but powerful ending, in which Ms. Smith
sings allegorically of the darkening circumstances: ``There's a change
in the ocean/ change in the deep blue sea.'' In the concluding verse,
she speaks of changing the way she responds. She has decided to ``go and
get some hop,'' she announces, and ``get myself a gun and shoot myself a
cop.''

It was an idea at once abhorrent and cathartic. Recorded in the wake of
horrific violence against African-Americans, ``Crazy Blues'' was not
only an outlet for exasperation in the face of ``nothin' but bad news.''
It was also a rallying cry in Black musical language and a call for
redress through reciprocal violence --- one that broke daringly out of
domestic allegory into a literal sphere where the police and the
military claimed the only prerogative to shoot at will.

One hundred years later, the blues endures as the essence of American
music, from rock 'n' roll and three-chord country songs to hip-hop and
contemporary R\&B. If in a 2020 hit like Chris Brown and Young Thug's
``Go Crazy,'' the title means to party, not to feel blue, we should
remember that Mamie Smith's ``Crazy Blues'' was also a dance tune:
People were not only moved by it; they moved to it.

From its earliest days, the blues has always done many and sometimes
contradictory things at the same time, as both an outlet for rage and a
release from it. Hatred and violence have hardly disappeared from the
American landscape, but neither has the blues.

David Hajdu (\href{https://twitter.com/davidhajdu_}{@davidhajdu}) is the
music critic for The Nation, a professor at the Columbia University
Graduate School of Journalism and the author of the forthcoming
``Adrianne Geffel: A Fiction.''

\emph{The Times is committed to publishing}
\href{https://www.nytimes.com/2019/01/31/opinion/letters/letters-to-editor-new-york-times-women.html}{\emph{a
diversity of letters}} \emph{to the editor. We'd like to hear what you
think about this or any of our articles. Here are some}
\href{https://help.nytimes.com/hc/en-us/articles/115014925288-How-to-submit-a-letter-to-the-editor}{\emph{tips}}\emph{.
And here's our email:}
\href{mailto:letters@nytimes.com}{\emph{letters@nytimes.com}}\emph{.}

\emph{Follow The New York Times Opinion section on}
\href{https://www.facebook.com/nytopinion}{\emph{Facebook}}\emph{,}
\href{http://twitter.com/NYTOpinion}{\emph{Twitter (@NYTopinion)}}
\emph{and}
\href{https://www.instagram.com/nytopinion/}{\emph{Instagram}}\emph{.}

Advertisement

\protect\hyperlink{after-bottom}{Continue reading the main story}

\hypertarget{site-index}{%
\subsection{Site Index}\label{site-index}}

\hypertarget{site-information-navigation}{%
\subsection{Site Information
Navigation}\label{site-information-navigation}}

\begin{itemize}
\tightlist
\item
  \href{https://help.nytimes.com/hc/en-us/articles/115014792127-Copyright-notice}{©~2020~The
  New York Times Company}
\end{itemize}

\begin{itemize}
\tightlist
\item
  \href{https://www.nytco.com/}{NYTCo}
\item
  \href{https://help.nytimes.com/hc/en-us/articles/115015385887-Contact-Us}{Contact
  Us}
\item
  \href{https://www.nytco.com/careers/}{Work with us}
\item
  \href{https://nytmediakit.com/}{Advertise}
\item
  \href{http://www.tbrandstudio.com/}{T Brand Studio}
\item
  \href{https://www.nytimes.com/privacy/cookie-policy\#how-do-i-manage-trackers}{Your
  Ad Choices}
\item
  \href{https://www.nytimes.com/privacy}{Privacy}
\item
  \href{https://help.nytimes.com/hc/en-us/articles/115014893428-Terms-of-service}{Terms
  of Service}
\item
  \href{https://help.nytimes.com/hc/en-us/articles/115014893968-Terms-of-sale}{Terms
  of Sale}
\item
  \href{https://spiderbites.nytimes.com}{Site Map}
\item
  \href{https://help.nytimes.com/hc/en-us}{Help}
\item
  \href{https://www.nytimes.com/subscription?campaignId=37WXW}{Subscriptions}
\end{itemize}
