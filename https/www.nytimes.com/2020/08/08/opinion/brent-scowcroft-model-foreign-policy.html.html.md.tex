Sections

SEARCH

\protect\hyperlink{site-content}{Skip to
content}\protect\hyperlink{site-index}{Skip to site index}

\href{https://myaccount.nytimes.com/auth/login?response_type=cookie\&client_id=vi}{}

\href{https://www.nytimes.com/section/todayspaper}{Today's Paper}

\href{/section/opinion}{Opinion}\textbar{}Brent Scowcroft Didn't Always
Follow `the Scowcroft Model'

\href{https://nyti.ms/30F1juP}{https://nyti.ms/30F1juP}

\begin{itemize}
\item
\item
\item
\item
\item
\end{itemize}

Advertisement

\protect\hyperlink{after-top}{Continue reading the main story}

\href{/section/opinion}{Opinion}

Supported by

\protect\hyperlink{after-sponsor}{Continue reading the main story}

\hypertarget{brent-scowcroft-didnt-always-follow-the-scowcroft-model}{%
\section{Brent Scowcroft Didn't Always Follow `the Scowcroft
Model'}\label{brent-scowcroft-didnt-always-follow-the-scowcroft-model}}

As national security adviser, he voiced strong opinions and acted on
them, especially when it came to Beijing and Moscow.

By James Mann

Mr. Mann is the author of ``The Great Rift: Dick Cheney, Colin Powell
and the Broken Friendship that Defined an Era.''

\begin{itemize}
\item
  Aug. 8, 2020, 4:02 p.m. ET
\item
  \begin{itemize}
  \item
  \item
  \item
  \item
  \item
  \end{itemize}
\end{itemize}

\includegraphics{https://static01.nyt.com/images/2020/08/08/opinion/08mann-pix/merlin_104153902_5d976a03-a2cf-4e13-aa87-cc7b83a2815f-articleLarge.jpg?quality=75\&auto=webp\&disable=upscale}

It became known in foreign policy circles as ``the Scowcroft model.''
Brent Scowcroft, the former national security adviser for Presidents
Gerald Ford and George H.W. Bush, who
\href{https://www.nytimes.com/2020/08/07/us/politics/brent-scowcroft-dead.html}{died
Friday}, was frequently praised for establishing a paradigm for doing
that job that many of his successors attempted (or claimed) to follow.
The irony is that the real Brent Scowcroft, a man of strong views,
didn't always fit the paradigm himself.

Under the Scowcroft model, the national security adviser shouldn't
become a strong advocate for his or her own ideas on foreign policy.
Rather, the national security adviser's main task should be to collect
the policy recommendations of others in the administration and make sure
that the various, often conflicting positions of the State Department,
the Pentagon, the C.I.A. and other foreign-policy agencies are passed on
to the president in a fair and balanced way. In this model, the national
security adviser should stay home handling the meetings and the paper
flow and let the secretary of state travel the world and speak for the
United States.

The ``Scowcroft model'' wasn't drawn up out of thin air. It was a
reaction to the modus operandi of Henry Kissinger, who was for a time
Mr. Scowcroft's boss. As President Richard Nixon's national security
adviser, Mr. Kissinger became the dominant force in that
administration's approach to the world. Mr. Kissinger reduced Mr.
Nixon's secretary of state, William Rogers, to an almost marginal
figure, who was not even allowed to be in the room for Mr. Nixon and Mr.
Kissinger's meeting with Mao Zedong in 1972. At the beginning of Mr.
Nixon's second term, Mr. Kissinger took on the job of secretary of
state, while keeping his portfolio as national security adviser.

Mr. Scowcroft's role as a new, more modest sort of national security
adviser began of necessity under President Ford, who sought to
circumscribe Mr. Kissinger by taking away his national security position
and giving it to Mr. Scowcroft. But there wasn't really a ``Scowcroft
model'' yet. Jimmy Carter's national security adviser, Zbigniew
Brzezinski, often sought to be a powerful, activist national security
adviser like Mr. Kissinger. And Robert McFarlane, one of Ronald Reagan's
string of national security advisers, was clearly trying to channel Mr.
Kissinger when, after leaving office but on behalf of the
administration, he made a secret visit to Iran in 1986, which he wrongly
hoped would be akin to Mr. Kissinger's groundbreaking secret trip to
China in 1971.

It was under President George H.W. Bush that the ``Scowcroft model''
took hold. It was in those years that Mr. Scowcroft (and others)
articulated the notion of a national security adviser who steps back,
coordinates and lets the cabinet secretaries take center stage.
Condoleezza Rice, who had worked for Mr. Scowcroft and considered him a
mentor, specifically cited him as a model even before she took the job
for President George W. Bush. Ms. Rice was far from alone. Over the
years, I've listened to various national security advisers of both
parties, including one of President Barack Obama's national security
advisers, Tom Donilon, say that they were trying to do their jobs in
line with the Scowcroft model.

The record shows that in real life, Mr. Scowcroft himself was both far
more opinionated and more of an activist than the model bearing his name
would suggest. He was not merely a neutral referee**.** He was a man of
determined beliefs, who sometimes voiced strong disapproval of those
whose ideas were different.

The best example was China. Mr. Scowcroft believed deeply in
perpetuating the secretive, anti-Soviet relationship with Beijing that
had been forged under Mr. Nixon and Mr. Kissinger, and he saw most
things connected to China through that lens. Early on, he angrily
rebuked Winston Lord, the U.S. ambassador to Beijing, for inviting a
Chinese dissident to a large dinner with the president --- and Mr. Lord,
once a rising star, never got another job in the administration.

In June 1989, after China's bloody crackdown on the Tiananmen Square
protests, the Bush administration announced a suspension of all
high-level exchanges with Chinese officials. That was the public policy.
In private, Mr. Scowcroft made a secret visit to Beijing that same month
for talks with the Chinese leader Deng Xiaoping. Mr. Scowcroft made a
second trip six months later, and, to his later regret, was photographed
clinking glasses at a banquet with top Chinese leaders.

On these China trips, Mr. Scowcroft was carrying out the wishes of his
boss, President Bush. Other officials in the administration, notably
Secretary of State James Baker, had reservations about the China policy,
but Mr. Scowcroft didn't try to draw them out, and they knew better than
to question too much on China.

On policy toward the Soviet Union, similarly, Mr. Scowcroft was far less
detached and more opinionated than the ``Scowcroft model'' might
suggest. In Ronald Reagan's final two years as president, Mr. Reagan and
Secretary of State George Shultz came to believe that Soviet President
Mikhail S. Gorbachev was different from past Soviet leaders, and they
pursued a series of agreements with him. Mr. Scowcroft, along with Mr.
Nixon and Mr. Kissinger, opposed Mr. Mr. Reagan's new, more dovish
Soviet policy. After Mr. Bush took office, the hawkish Mr. Scowcroft led
the way in putting everything on hold for the better part of a year.

To be sure, in some other instances, Mr. Scowcroft did act in ways that
matched the idealized model. Before the Persian Gulf War, Colin Powell,
as chairman of the Joint Chiefs, began expressing private qualms about
the idea of military action against Iraq, arguing instead for a policy
of containment. Mr. Scowcroft, who thought containment wouldn't work,
allowed General Powell to come into the White House to put forth his
ideas directly to Bush --- and then the administration moved ahead
toward war anyway.

Twelve years later, in private life, Mr. Scowcroft strongly opposed
President George W. Bush's war to remove Saddam Hussein from power. He
had first aired his opinions on television and was then persuaded to put
them in writing in
\href{https://www.wsj.com/articles/SB1029371773228069195}{an op-ed} for
The Wall Street Journal. At the time, in 2002, I was in the middle of a
series of interviews with Mr. Scowcroft for a book on the two Bush
administrations. He expressed surprise at the furor his antiwar op-ed
had created and even more at the fact that he had suddenly become
something of a hero to the political left. ``Twenty-five years ago, I
was a leading hawk,'' Mr. Scowcroft told me. ``I feel the same way about
things, and now I'm a leading dove.''

He had firm judgments, and as national security adviser, he acted on
them. In other words, not even Brent Scowcroft could conform to the
Scowcroft model.

\href{http://james-mann.com/}{James Mann}
(\href{https://twitter.com/byjamesmann?lang=en}{@byjamesmann}), a fellow
at Johns Hopkins School of Advanced International Studies, is the author
of ``The Great Rift: Dick Cheney, Colin Powell and the Broken Friendship
that Defined an Era'' and ``Rise of the Vulcans: The History of Bush's
War Cabinet.''

\emph{The Times is committed to publishing}
\href{https://www.nytimes.com/2019/01/31/opinion/letters/letters-to-editor-new-york-times-women.html}{\emph{a
diversity of letters}} \emph{to the editor. We'd like to hear what you
think about this or any of our articles. Here are some}
\href{https://help.nytimes.com/hc/en-us/articles/115014925288-How-to-submit-a-letter-to-the-editor}{\emph{tips}}\emph{.
And here's our email:}
\href{mailto:letters@nytimes.com}{\emph{letters@nytimes.com}}\emph{.}

\emph{Follow The New York Times Opinion section on}
\href{https://www.facebook.com/nytopinion}{\emph{Facebook}}\emph{,}
\href{http://twitter.com/NYTOpinion}{\emph{Twitter (@NYTopinion)}}
\emph{and}
\href{https://www.instagram.com/nytopinion/}{\emph{Instagram}}\emph{.}

Advertisement

\protect\hyperlink{after-bottom}{Continue reading the main story}

\hypertarget{site-index}{%
\subsection{Site Index}\label{site-index}}

\hypertarget{site-information-navigation}{%
\subsection{Site Information
Navigation}\label{site-information-navigation}}

\begin{itemize}
\tightlist
\item
  \href{https://help.nytimes.com/hc/en-us/articles/115014792127-Copyright-notice}{©~2020~The
  New York Times Company}
\end{itemize}

\begin{itemize}
\tightlist
\item
  \href{https://www.nytco.com/}{NYTCo}
\item
  \href{https://help.nytimes.com/hc/en-us/articles/115015385887-Contact-Us}{Contact
  Us}
\item
  \href{https://www.nytco.com/careers/}{Work with us}
\item
  \href{https://nytmediakit.com/}{Advertise}
\item
  \href{http://www.tbrandstudio.com/}{T Brand Studio}
\item
  \href{https://www.nytimes.com/privacy/cookie-policy\#how-do-i-manage-trackers}{Your
  Ad Choices}
\item
  \href{https://www.nytimes.com/privacy}{Privacy}
\item
  \href{https://help.nytimes.com/hc/en-us/articles/115014893428-Terms-of-service}{Terms
  of Service}
\item
  \href{https://help.nytimes.com/hc/en-us/articles/115014893968-Terms-of-sale}{Terms
  of Sale}
\item
  \href{https://spiderbites.nytimes.com}{Site Map}
\item
  \href{https://help.nytimes.com/hc/en-us}{Help}
\item
  \href{https://www.nytimes.com/subscription?campaignId=37WXW}{Subscriptions}
\end{itemize}
