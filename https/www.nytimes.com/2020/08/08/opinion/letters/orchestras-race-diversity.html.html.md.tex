Sections

SEARCH

\protect\hyperlink{site-content}{Skip to
content}\protect\hyperlink{site-index}{Skip to site index}

\href{https://myaccount.nytimes.com/auth/login?response_type=cookie\&client_id=vi}{}

\href{https://www.nytimes.com/section/todayspaper}{Today's Paper}

\href{/section/opinion}{Opinion}\textbar{}How to Diversify Orchestras

\href{https://nyti.ms/2XHSgav}{https://nyti.ms/2XHSgav}

\begin{itemize}
\item
\item
\item
\item
\item
\end{itemize}

Advertisement

\protect\hyperlink{after-top}{Continue reading the main story}

\href{/section/opinion}{Opinion}

Supported by

\protect\hyperlink{after-sponsor}{Continue reading the main story}

letters

\hypertarget{how-to-diversify-orchestras}{%
\section{How to Diversify
Orchestras}\label{how-to-diversify-orchestras}}

Readers offer their ideas in response to the music critic Anthony
Tommasini's suggestion that blind auditions be ended.

Aug. 8, 2020, 12:00 p.m. ET

\begin{itemize}
\item
\item
\item
\item
\item
\end{itemize}

\includegraphics{https://static01.nyt.com/images/2020/07/19/arts/19race-tommasini-2/19race-tommasini-2-articleLarge.jpg?quality=75\&auto=webp\&disable=upscale}

\textbf{To the Editor:}

Re
``\href{https://www.nytimes.com/2020/07/16/arts/music/blind-auditions-orchestras-race.html}{Make
Orchestras More Diverse? End Blind Auditions},'' by Anthony Tommasini
(Critic's Notebook, July 19):

Mr. Tommasini notes that blind auditions have been transformative ---
the number of women in major orchestras has come to more accurately
reflect life. I am a violist in the Metropolitan Opera, which today is
about 50-50 women to men.

When I entered the orchestra in 1987, I was about the 25th woman in the
orchestra out of about 100 regular full-time players. After my audition,
when the committee met me, more than one man on the audition committee
said, ``But you're a girl!'' Bringing down the screen will allow
prejudice in, not remove it.

There is no easy answer, and considering that we all have implicit bias,
I would argue that we should focus on the pipeline. It takes about a
decade to prepare for an audition that you will win. That puts the
spotlight on our elementary schools. As a society we need to invest in
arts education in our primary public education institutions in every
neighborhood and in every state.

Désirée Elsevier\\
Dallas

\textbf{To the Editor:}

I applaud your recent articles regarding the profound lack of diversity
across the classical music profession. The urgency of addressing this
issue goes to a question of survival. Demographics are changing such
that growing audiences for the classical performing arts is tied to
diversifying those audiences, and that cannot happen without
diversifying the performers on our stages.

This work begins with the institutions that train artists --- the
conservatories and their preparatory schools --- requiring the
commitment of every institution like mine. We must change the face of
our faculty and students to reflect the diverse world in which our
graduates will operate.

This past year, the Peabody Institute's faculty was 14 percent Black and
Latinx, matched by a similar percentage of our student cohort. Greater
intentionality in faculty searches brings faculty of color, who in turn
help inspire students of color to apply. Potential students can be
recruited from the many magnet performing arts schools and elsewhere
throughout the country.

If every conservatory makes a long-term commitment to do this work, we
can change the face of our field. But we must hold our collective feet
to the fire.

Fred Bronstein\\
Baltimore\\
\emph{The writer is dean of the Peabody Institute of The Johns Hopkins
University.}

\textbf{To the Editor:}

I read with interest Anthony Tommasini's piece on ending blind orchestra
auditions to increase diversity. While I believe his heart is in the
right place, I strongly disagree with his method for getting there.

The way to increase diversity in the classical music world is to provide
affordable access to quality music instruction. For all. See Colombia's
extraordinary music training program.

Classical music instruction is expensive, and this is why whites
dominate the classical world. As students improve, they need better
teachers --- who cost more, often much more. They need to attend summer
music programs. These are expensive, the best ones are \emph{very}
expensive, and they may be far from the student's home, entailing
transportation expenses. The students need better, more expensive
instruments. The best college-level music programs are, you guessed it,
quite expensive.

A musically gifted student without those opportunities will simply not
be able to compete at upper levels. Ending blind auditions won't solve
this problem. Money and opportunity aimed at financially disadvantaged
students will.

If music lovers want top-quality musicians playing in a top-quality
orchestra, blind auditions are a must. Our challenge is to prepare a
larger, more diverse talent pool for the rigors of those auditions.

Lynn Lloyd\\
Mount Shasta, Calif.

\textbf{To the Editor:}

I imagine that Anthony Tommasini has received some flak over his
proposal to end blind orchestra auditions in order to promote racial
diversity in symphony orchestras.

I suggest a simple modification: Keep the blind audition but, when the
competition is close, have it identify the top two (or more) performers.
Their names would then be submitted to a special panel charged with
making the final decision and possessing background information on the
applicants. For the orchestra wishing to diversify, that decision would
be based on race.

This will guarantee, for both supporters and doubters, that the quality
of the orchestra is not being compromised. It will guarantee that Black
musicians so chosen are worthy of the position and not under any cloud.
And it will avoid risking deleterious blowups on the divisive subject of
affirmative action.

Nick Tingley\\
Greenwich, Conn.

\textbf{To the Editor:}

There is no question that Black and Latino musicians are
underrepresented in U.S. orchestras. Yet there are no quick fixes for a
complex set of social and economic factors that have blocked the entry
of people of color into the classical music profession.

Therefore, I don't believe that simple solutions, such as removing the
screen at orchestra auditions, will achieve diversity. A mandate to give
race higher priority in auditions may also have unintended consequences.
Someone who wins an audition based primarily on race or gender might not
get tenure, for example.

The preservation of this beautiful art form is very important. Classical
music will change and grow, new musicians and audiences will come to
love it, and excellence must remain our guiding principle. We do not
want to choose between artistic standards and diversity, but to embrace
both. In our rush toward diversity, let us always keep this in mind.

Linda Marianiello\\
Santa Fe, N.M.\\
\emph{The writer is executive director of the New Mexico Performing Arts
Society and former assistant principal flute for several orchestras.}

\textbf{To the Editor:}

There is a larger crisis underway than the use of blind auditions by a
few elite orchestras. If there are so many exceptional musicians
fighting for work, then we clearly need to create more jobs.

As any classical-music fan will tell you,
\href{https://www.nytimes.com/2003/05/14/arts/as-funds-disappear-so-do-orchestras.html}{U.S.
orchestras are disappearing}. Remember the Florida Philharmonic? San
Jose Symphony? Tulsa Philharmonic? Savannah Symphony Orchestra?

And Broadway's pit orchestras --- once brimming with top-notch musicians
--- are now mere repositories for synthesizers, amps and laptops.

Massive investment in classical music, which could put all worthy
musicians to work, would render the blind-audition controversy moot.

Aaron Johnson\\
Milton, Mass.\\
\emph{The writer is a violist.}

\textbf{To the Editor:}

Anthony Tommasini was absolutely right in noting that screens played an
important role in achieving greater gender equity in American
orchestras. The logic of removing them to accomplish fairness in hiring
racial-ethnic minorities not only overlooks their demonstrated value in
encouraging racial fairness in promotion and advancement, and the risk
of rolling back what has been accomplished in terms of gender equity,
but the many other factors that create racial inequities in hiring.

Among these, the single most important is lack of a diverse pool of
applicants. Orchestras must aggressively recruit racial-ethnic
minorities to apply for positions. Without diverse applicants, it is
impossible to succeed at hiring a diverse set of musicians.

There is a great deal of social science expertise on the topic of
achieving equity in hiring. I hope orchestras turn to experts and the
research literature for advice about how best to achieve the diversity
and fairness that surely are long overdue.

Abigail J. Stewart\\
Ann Arbor, Mich.\\
\emph{The writer is a professor of psychology and women's studies at the
University of Michigan.}

Advertisement

\protect\hyperlink{after-bottom}{Continue reading the main story}

\hypertarget{site-index}{%
\subsection{Site Index}\label{site-index}}

\hypertarget{site-information-navigation}{%
\subsection{Site Information
Navigation}\label{site-information-navigation}}

\begin{itemize}
\tightlist
\item
  \href{https://help.nytimes.com/hc/en-us/articles/115014792127-Copyright-notice}{©~2020~The
  New York Times Company}
\end{itemize}

\begin{itemize}
\tightlist
\item
  \href{https://www.nytco.com/}{NYTCo}
\item
  \href{https://help.nytimes.com/hc/en-us/articles/115015385887-Contact-Us}{Contact
  Us}
\item
  \href{https://www.nytco.com/careers/}{Work with us}
\item
  \href{https://nytmediakit.com/}{Advertise}
\item
  \href{http://www.tbrandstudio.com/}{T Brand Studio}
\item
  \href{https://www.nytimes.com/privacy/cookie-policy\#how-do-i-manage-trackers}{Your
  Ad Choices}
\item
  \href{https://www.nytimes.com/privacy}{Privacy}
\item
  \href{https://help.nytimes.com/hc/en-us/articles/115014893428-Terms-of-service}{Terms
  of Service}
\item
  \href{https://help.nytimes.com/hc/en-us/articles/115014893968-Terms-of-sale}{Terms
  of Sale}
\item
  \href{https://spiderbites.nytimes.com}{Site Map}
\item
  \href{https://help.nytimes.com/hc/en-us}{Help}
\item
  \href{https://www.nytimes.com/subscription?campaignId=37WXW}{Subscriptions}
\end{itemize}
