Sections

SEARCH

\protect\hyperlink{site-content}{Skip to
content}\protect\hyperlink{site-index}{Skip to site index}

\href{https://myaccount.nytimes.com/auth/login?response_type=cookie\&client_id=vi}{}

\href{https://www.nytimes.com/section/todayspaper}{Today's Paper}

\href{/section/opinion}{Opinion}\textbar{}America's Military Should
Confront Its Past, Not Bury It

\href{https://nyti.ms/33F9n0z}{https://nyti.ms/33F9n0z}

\begin{itemize}
\item
\item
\item
\item
\item
\end{itemize}

Advertisement

\protect\hyperlink{after-top}{Continue reading the main story}

\href{/section/opinion}{Opinion}

Supported by

\protect\hyperlink{after-sponsor}{Continue reading the main story}

\hypertarget{americas-military-should-confront-its-past-not-bury-it}{%
\section{America's Military Should Confront Its Past, Not Bury
It}\label{americas-military-should-confront-its-past-not-bury-it}}

The German military's infiltration by far-right extremists should be a
warning for how we confront our own troubled history.

\includegraphics{https://static01.nyt.com/images/2020/03/13/opinion/elliot-ackerman/elliot-ackerman-thumbLarge.png}

By Elliot Ackerman

Mr. Ackerman is a contributing opinion writer and a former U.S. Marine
and intelligence officer.

\begin{itemize}
\item
  Aug. 8, 2020
\item
  \begin{itemize}
  \item
  \item
  \item
  \item
  \item
  \end{itemize}
\end{itemize}

\includegraphics{https://static01.nyt.com/images/2020/08/05/opinion/05Ackerman1/05Ackerman1-articleLarge.jpg?quality=75\&auto=webp\&disable=upscale}

At the end of May, German police commandos arrived outside a rural home
owned by a sergeant major in that country's most elite special forces
unit, the KSK. Buried in the backyard they found a trove of weapons,
explosives and Nazi memorabilia. In response, Germany's defense minister
announced that she would disband one-quarter of the unit because of the
widespread infiltration of far-right extremists into its ranks. But as
several news reports have made clear, it is suspected that the
infiltration extends far beyond a single segment of the KSK.

Being a soldier in Germany has long been a fraught proposition given the
stain of its Nazi past, a history that, like the explosives in the
sergeant major's garden, the government has been attempting to bury for
decades.

Like the American military, the German
\href{https://www.dw.com/en/the-german-military-and-its-troubled-traditions/a-38863290}{Bundeswehr}
is an all-volunteer force, with conscription having ended in 2011; that,
combined with the public disapproval of Germany's participation in the
war in Afghanistan and an increasing number of other commitments abroad,
has
\href{https://foreignpolicy.com/2019/07/01/the-military-arent-heroes-or-villains-theyre-us/}{created
a widening civil-military divide}, much like the one that exists in the
United States.

Unlike the American military, though, the Bundeswehr is in many ways an
ahistoric organization, officially cut off from its complicated past.
The acceptable history of the German military is codified in the
Bundeswehr by its ``Traditionserlass'' (``tradition decree''). In that
document (first enacted in 1965; a new one was issued in 2018), the
current army purges its Wehrmacht past and traces its lineage instead to
dissident officers who attempted to assassinate Hitler in the failed
July 1944 Stauffenberg plot.

Given the enormity of the Nazis' crimes, Germany's disavowal as it
attempted to reestablish its military after World War II is
understandable. But the KSK revelations raise the possibility that in
scrubbing its military's history, the government failed to confront its
past, but rather buried it, and in doing so, left that history --- one
easily weaponized --- vulnerable to co-option by radicals, unchecked by
the sort of moral framework that the full, open engagement of a society
can provide.

America's military is now reckoning with chapters of its own past ---
from the genocide of Indigenous people that enabled the settling of the
continent, to our Civil War and beyond --- and I believe that Germany
now offers us a cautionary example of what can happen when a nation
buries its past too deeply. I worry that if our own military sets itself
too stridently against its complex history, it might unleash similarly
malicious forces.

I welcome many of the measures being taken to more fully render that
history --- proposals to redesignate bases named after Confederate
generals and, as the Marine Corps has done,
\href{https://www.nytimes.com/2020/06/06/us/marines-confederate-flag-ban.html}{banning}
the display of the Confederate flag on base. But I am convinced that a
much broader erasure of controversial figures and chapters of American
military history would be a mistake.

Take Robert E. Lee, the Confederate general and former superintendent of
West Point. Every young military officer learns about Lee. We learn
about the Battle of Chancellorsville when, in May of 1863, Lee made the
audacious decision to split his army and go on the offense against a
force twice its size and subsequently routed the Union Army in a victory
that became known as ``Lee's Miracle.'' We also learn how two months
later, at Gettysburg, Lee's same offensive spirit and blind faith in his
soldiers' abilities led him to order Pickett's Charge, his greatest
strategic blunder, one that cost him the battle and, some say, the war.
Our instructors taught this history as an example of how a commander's
attributes can be a strength in one case and a liability in another. We
studied Lee to understand the human element that, with all its attendant
complexity and contradictions, is omnipresent in war. You cannot
understand war by understanding only its moral exemplars.

Suppose we cast Lee's story outside the pantheon of American military
history, following the German model? What if we focused solely on the
military leaders who fought for causes of which we approved? That's a
dangerous proposition in a profession where your job is to understand
and anticipate your enemy's actions. Learning to think like a
Confederate, a Nazi or a card-carrying member of Al Qaeda requires a
psychological empathy and academic rigor to which ``safe spaces'' and
``trigger warnings'' are anathema.

Even if removed from military curriculums, Lee's story and many like it
will continue to be sought out and learned. But future tactical
disciples who find Lee outside of a structured education risk omitting
his failings. We want future soldiers to learn Gettysburg \emph{and}
Chancellorsville. To learn the brilliance and the failure.

That is what has happened in German military units like the KSK, where
Nazis like the brash
\href{https://en.wikipedia.org/wiki/Otto_Skorzeny}{Otto Skorzeny} ---
who led one of the most audacious commando missions of all time, the
rescue of Benito Mussolini, and an attempt to capture the Yugoslavian
dictator Marshal Tito --- remain unclaimed by the Bundeswehr, and are
venerated as spiritual fathers by the far right in unofficial, secretive
meetings rife with Nazi symbolism, rather than studied with clear
understandings of both tactical genius and ideological bigotry.

Much of what I learned about Germany's military I learned in the context
of our military. It should go without saying that this appreciation
wasn't ideological but tactical. In various military schools and
courses, my instructors assigned a range of military strategists: from
the Prussian general and military theorist Carl von Clausewitz, who
fought in the Napoleonic Wars and wrote the seminal text ``On War''
(``War is the continuation of politics by other means,'' he famously
said), to U.S. Adm. William McRaven, whose
\href{https://www.penguinrandomhouse.com/books/112624/spec-ops-by-william-h-mcraven/}{first
book} featured the Wehrmacht's 1940 commando raid on the Belgian fort
Eben Emael as a case study to demonstrate principles we'd later use on
raids in Iraq and Afghanistan. The tactical influence of the German Army
appears everywhere from the Marine Corps core doctrinal publication
``MCDP-1 Warfighting'' ** (spelled as one word, in the German way) to
the design of the standard-issue Kevlar helmets worn by soldiers in the
U.S. military.

Although Germany's airbrushed narrative has granted its military an
acceptable place in society, some historians believe it has helped
foster the current far-right extremism in its ranks. I spoke with Klaus
Schmider, a senior lecturer at the Royal Military Academy Sandhurst, who
believes that the German government has ``brought the current crisis on
themselves by refusing to give German soldiers a positive self-image as
soldiers.'' To be a soldier in Germany, he said, one must ``repeat a
mantra how being a soldier isn't really being a warrior.'' This goes
beyond the Nazi past: ``Even units which are able to trace their lineage
to the wars against Napoleon have recently been actively encouraged by
the ministry to empty any display cases with mementos from that period,
because of the eventual Prussian influence within the Wehrmacht,'' Dr.
Schmider said.

When I think of our nation's complicated past --- Confederate or
otherwise --- I prefer to associate those symbols with our society's
dead-enders. I would much rather see the ``Stars and Bars'' flown in a
backwater by one of those brittle souls being left behind by a
pluralistic, inclusive America, rather than unfurled in a basement one
night 10, 20 or 30 years in the future by a group of active-duty, if
disaffected, SEALs, Rangers or Marines who've appropriated it as their
own. The former would be troublesome, but the latter would be a threat
to our republic.

History teaches us that civil-military divides like those that exist in
the United States and Germany can become fertile soil for grievance. The
seeds of discontent exist in the pasts of both countries. But it is best
to leave those seeds scattered on the surface, where they can be picked
at and disregarded, instead of buried deep in the earth, where they can
eventually take root, breaking ground in twisted, unexpected ways.

Advertisement

\protect\hyperlink{after-bottom}{Continue reading the main story}

\hypertarget{site-index}{%
\subsection{Site Index}\label{site-index}}

\hypertarget{site-information-navigation}{%
\subsection{Site Information
Navigation}\label{site-information-navigation}}

\begin{itemize}
\tightlist
\item
  \href{https://help.nytimes.com/hc/en-us/articles/115014792127-Copyright-notice}{©~2020~The
  New York Times Company}
\end{itemize}

\begin{itemize}
\tightlist
\item
  \href{https://www.nytco.com/}{NYTCo}
\item
  \href{https://help.nytimes.com/hc/en-us/articles/115015385887-Contact-Us}{Contact
  Us}
\item
  \href{https://www.nytco.com/careers/}{Work with us}
\item
  \href{https://nytmediakit.com/}{Advertise}
\item
  \href{http://www.tbrandstudio.com/}{T Brand Studio}
\item
  \href{https://www.nytimes.com/privacy/cookie-policy\#how-do-i-manage-trackers}{Your
  Ad Choices}
\item
  \href{https://www.nytimes.com/privacy}{Privacy}
\item
  \href{https://help.nytimes.com/hc/en-us/articles/115014893428-Terms-of-service}{Terms
  of Service}
\item
  \href{https://help.nytimes.com/hc/en-us/articles/115014893968-Terms-of-sale}{Terms
  of Sale}
\item
  \href{https://spiderbites.nytimes.com}{Site Map}
\item
  \href{https://help.nytimes.com/hc/en-us}{Help}
\item
  \href{https://www.nytimes.com/subscription?campaignId=37WXW}{Subscriptions}
\end{itemize}
