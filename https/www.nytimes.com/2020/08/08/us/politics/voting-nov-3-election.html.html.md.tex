Sections

SEARCH

\protect\hyperlink{site-content}{Skip to
content}\protect\hyperlink{site-index}{Skip to site index}

\href{https://www.nytimes.com/section/politics}{Politics}

\href{https://myaccount.nytimes.com/auth/login?response_type=cookie\&client_id=vi}{}

\href{https://www.nytimes.com/section/todayspaper}{Today's Paper}

\href{/section/politics}{Politics}\textbar{}The Voting Will End Nov. 3.
The Legal Battle Probably Won't.

\href{https://nyti.ms/3gBbTIO}{https://nyti.ms/3gBbTIO}

\begin{itemize}
\item
\item
\item
\item
\item
\end{itemize}

\begin{itemize}
\item
  \href{https://www.nytimes.com/2020/08/07/us/elections/biden-vs-trump.html?action=click\&pgtype=Article\&state=default\&region=TOP_BANNER\&context=storylines_menu}{Election
  Updates}
\item
  \href{https://www.nytimes.com/interactive/2020/08/06/us/elections/results-tennessee-primary-elections.html?action=click\&pgtype=Article\&state=default\&region=TOP_BANNER\&context=storylines_menu}{Tennessee
  Results}
\item
  \href{https://www.nytimes.com/article/biden-vice-president-2020.html?action=click\&pgtype=Article\&state=default\&region=TOP_BANNER\&context=storylines_menu}{Biden's
  V.P. Search}
\item
  \href{https://www.nytimes.com/interactive/2019/us/politics/2020-presidential-candidates.html?action=click\&pgtype=Article\&state=default\&region=TOP_BANNER\&context=storylines_menu}{The
  Candidates}
\item
  \href{https://www.nytimes.com/newsletters/politics?action=click\&pgtype=Article\&state=default\&region=TOP_BANNER\&context=storylines_menu}{Politics
  Newsletter}
\end{itemize}

Advertisement

\protect\hyperlink{after-top}{Continue reading the main story}

Supported by

\protect\hyperlink{after-sponsor}{Continue reading the main story}

\hypertarget{the-voting-will-end-nov-3-the-legal-battle-probably-wont}{%
\section{The Voting Will End Nov. 3. The Legal Battle Probably
Won't.}\label{the-voting-will-end-nov-3-the-legal-battle-probably-wont}}

As the two parties clash over how to conduct an election in a pandemic,
President Trump's litigiousness and unfounded claims of fraud have
increased the likelihood of epic postelection court fights.

\includegraphics{https://static01.nyt.com/images/2020/08/09/us/politics/09DC-VOTING-print1/08DC-VOTING-articleLarge.jpg?quality=75\&auto=webp\&disable=upscale}

\href{https://www.nytimes.com/by/peter-baker}{\includegraphics{https://static01.nyt.com/images/2018/06/13/multimedia/peter-baker/peter-baker-thumbLarge-v2.png}}\href{https://www.nytimes.com/by/nick-corasaniti}{\includegraphics{https://static01.nyt.com/images/2018/06/13/multimedia/author-nick-corasaniti/author-nick-corasaniti-thumbLarge-v2.png}}\href{https://www.nytimes.com/by/michael-s-schmidt}{\includegraphics{https://static01.nyt.com/images/2018/06/12/multimedia/author-michael-s-schmidt/author-michael-s-schmidt-thumbLarge.png}}\href{https://www.nytimes.com/by/maggie-haberman}{\includegraphics{https://static01.nyt.com/images/2018/07/12/multimedia/author-maggie-haberman/author-maggie-haberman-thumbLarge.png}}

By \href{https://www.nytimes.com/by/peter-baker}{Peter Baker},
\href{https://www.nytimes.com/by/nick-corasaniti}{Nick Corasaniti},
\href{https://www.nytimes.com/by/michael-s-schmidt}{Michael S. Schmidt}
and \href{https://www.nytimes.com/by/maggie-haberman}{Maggie Haberman}

\begin{itemize}
\item
  Aug. 8, 2020, 1:18 p.m. ET
\item
  \begin{itemize}
  \item
  \item
  \item
  \item
  \item
  \end{itemize}
\end{itemize}

The stormy once-in-a-lifetime
\href{https://www.nytimes.com/2000/12/13/us/bush-prevails-single-vote-justices-end-recount-blocking-gore-after-5-week.html}{Florida
recount battle} that polarized the nation in 2000 and left the Supreme
Court to decide the presidency may soon look like a high school student
council election compared with what could be coming after this
November's election.

Imagine not just another Florida, but a dozen Floridas. Not just one set
of lawsuits but a vast array of them. And instead of two restrained
candidates staying out of sight and leaving the fight to surrogates, a
sitting president of the United States unleashing ALL CAPS Twitter
blasts from the Oval Office while seeking ways to use the power of his
office to intervene.

The possibility of an ugly November --- and perhaps even December and
January --- has emerged more starkly in recent days as President Trump
complains that the election will be rigged and Democrats accuse him of
trying to make that a self-fulfilling prophesy.

With about 85 days until Nov. 3, lawyers are already in court mounting
pre-emptive strikes and preparing for the larger, scorched-earth
engagements likely to come. Like the Trump campaign, Joseph R. Biden
Jr.'s campaign and its network of Democratic support groups are stocking
up on lawyers, and Democrats are gaming out worst-case scenarios,
including how to respond if Mr. Trump prematurely declares victory or
sends federal officers into the party's strongholds as an intimidation
tactic.

The emerging battle is the latest iteration of the long-running dispute
over voting rights, one shaped by the view that higher participation
will improve the Democratic Party's chances. Republicans, under cover of
dubious or unfounded claims about widespread fraud, are trying to
prevent steps that would make it easier for more people to vote and
Democrats are pressing more aggressively than ever to secure ballot
access and expand the electorate.

But that clash has been vastly complicated this year by the challenge of
holding a national election in the middle of a deadly pandemic, with a
greater reliance on mail-in voting that could prolong the counting in a
way that turns Election Day into Election Week or Election Month. And
the atmosphere has been inflamed by a president who is already using
words like
\href{https://twitter.com/realDonaldTrump/status/1290250416278532096}{``coup,''}
\href{https://twitter.com/realDonaldTrump/status/1266172570983940101}{``fraud''}
and
\href{https://twitter.com/realDonaldTrump/status/1285540318503407622?s=20}{``corrupt''}
to
\href{https://www.nytimes.com/2020/07/31/us/politics/trump-tweet-democracy.html}{delegitimize
the vote even before it happens}.

The battle is playing out on two tracks: defining the rules about how
the voting will take place, and preparing for fights over how the votes
should be counted and contesting the outcome.

``The big electoral crisis arises from the prospect of hundreds of
thousands of ballots not being counted in decisive states until a week
after the election or more,'' said Richard H. Pildes, a constitutional
scholar at New York University School of Law.

If the candidate who appears ahead on election night ends up losing
later on, he said, it will fuel suspicion, conspiracy theories and
polarization. ``I have no doubt the situation will be explosive,'' he
said.

Some flash points have already emerged:

\begin{itemize}
\item
  A long-troubled Postal Service,
  \href{https://www.nytimes.com/2020/07/31/us/politics/trump-usps-mail-delays.html}{now
  run by a Trump megadonor and seemingly overwhelmed} by the prospect of
  delivering tens of millions more votes cast by mail with an
  administration resistant to providing substantial new funding.
\item
  Concern among Democrats that Mr. Trump or Attorney General William P.
  Barr could use their bully pulpits to raise loud enough alarms about
  voter fraud to lead sympathetic state and local officials to slow or
  block adverse results.
\item
  Fights over
  \href{https://www.nytimes.com/2020/08/03/nyregion/nyc-congress-carolyn-maloney-ballots.html}{whether
  mailed ballots should be counted} if received by Election Day or
  simply postmarked by Election Day, not to mention what to do if the
  post office does not postmark them at all.
\item
  Fights over the use of drop boxes to return ballots and the number of
  polling places for in-person voting amid the risk of disease.
\item
  Fights over whether witnesses should still be required for absentee
  votes in a socially distant moment and what to do if signatures do not
  match those on file.
\end{itemize}

Already, by Mr. Pildes's count, party organizations, campaigns and
interest groups have filed 160 lawsuits across the country trying to
shape the rules of the election. About 40 have been filed in 17 states
by Mr. Trump's campaign and the Republican National Committee, some in
response to Democratic lawsuits, as part of an expansive \$20 million
litigation campaign against policies making it easier to vote on the
grounds that they could lead to fraud.

\href{https://twitter.com/realDonaldTrump/status/1290250416278532096}{``See
you in Court!''} Mr. Trump tweeted a few days ago to Nevada, which just
passed universal mail-in balloting legislation, under which the state
sends a mail-in ballot to every registered voter.

``They are just really efforts to throw tacks in front of the tires to
make it so states can't run their elections this time,'' said Michael
Waldman, the president of the Brennan Center for Justice and a former
aide to President Bill Clinton.

Democrats and their allies, led by Marc E. Elias, the general counsel of
the Democratic National Committee, are seeking to expand voting options,
particularly through mail-in voting. They have active litigation in
numerous battleground states, pursuing relief on deadlines, signature
and witness requirements, among others.

Republicans said their own court efforts were aimed at preventing
Democrats from changing the rules in the middle of the game.

``People are viewing it as an attack on vote-by-mail,'' said Justin
Riemer, the chief counsel for the Republican National Committee. But in
fact, he said, ``it's by and large protecting the safeguards that are in
place.''

Mr. Trump, who also made unfounded claims about fraud in the 2016
election even though he won, has signaled that he will not hesitate to
go back to court after Election Day if he does not like the result.
Unlike in 2000, when the Justice Department largely stayed on the
sidelines, Democrats worry that Mr. Barr will intervene with civil
suits, investigations or public statements, casting doubt on the result
if Mr. Trump appears to lose. And some Democrats say they are not sure
how Mr. Trump would respond, with the presidency on the line, to a court
ruling against him.

Some Democrats even express fear that Mr. Trump would send federal
agents into the streets as
\href{https://www.nytimes.com/2020/07/21/us/politics/trump-portland-federal-agents.html}{he
did in recent weeks in Portland, Ore.} Democrats have game-planned
situations in which Mr. Trump deploys immigration officers into Hispanic
neighborhoods to intimidate citizens shortly before the election and
suppress turnout.

``It is very, very much a concern,'' said Alex Padilla, the secretary of
state of California.

Mr. Trump's advisers dismiss such talk as overheated partisan messaging.
Justin Clark, the president's deputy campaign manager, said states like
California and Nevada trying to expand mail-in voting on the fly were
the ones setting the stage for a chaotic election.

``Rushing to implement universal vote-by-mail leads to delays in counts,
delays in results and uncertainty about who won an election,'' he said.

\href{https://www.nytimes.com/2020/08/04/nyregion/maloney-torres-ny-congressional-races.html}{It
took six weeks for the New York authorities} to determine the winners of
two House Democratic congressional primaries as they struggled with 10
times the normal number of absentee ballots, a case study in the
potential for a lengthy count in the fall even if not an example of
fraud as Mr. Trump has falsely claimed.

Mr. Clark is one of the party's top warriors on election fraud fights.
In a recording from 2019, he told fellow Republicans: ``Traditionally,
it's always been Republicans suppressing votes in places. Let's start
protecting our voters.''

Republicans, he said, should be more aggressive. ``Let's start playing
offense a little bit,'' he said then. ``That's what you're going to see
in 2020. It's going to be a much bigger program, a much more aggressive
program, a much better-funded program.''

He later said he was referring to false accusations made against
Republicans. A federal judge in 2018 lifted a consent decree in place
since 1982 that barred the Republican National Committee from certain
so-called ballot security efforts.

Asked about those comments, Mr. Clark said: ``Democrats have always
accused Republicans of voter suppression. The fact of the matter is all
Democrats have done this year is pushed crazy voting laws.''

The Trump team has also tried to halt another pillar of absentee voting
--- the drop box. In 2018 in Colorado, one of five states that already
votes nearly entirely by mail, 75 percent of ballots were returned
through a drop box or at a polling place. In Pennsylvania, the Trump
campaign sued against expanding the use of drop boxes, an action that
has concerned election officials across the country.

Jena Griswold, the secretary of state of Colorado, said the president's
attacks on the Postal Service and his refusal to devote enough resources
to fix its problems showed his disingenuous motives.

``You do all that and then you attack drop boxes, the alternative to
voting safely, it's a pattern of voter suppression,'' she said. ``It's a
pattern of voter suppression and I just think it's really
reprehensible.''

Others are looking to head off disqualifying ballots over procedural
issues like postmarks and the date of receipt. ``Voting shouldn't be a
game of gotcha,'' said Ann Jacobs, the chairwoman of the Wisconsin
Elections Commission.

The Help America Vote Act,
\href{https://www.nytimes.com/2002/10/17/us/2002-campaign-ballot-overhaul-congress-passes-bill-clean-up-election-system.html}{passed
on large bipartisan votes in 2002} in response to the Florida recount,
was meant to help states upgrade and standardize voting procedures. But
it gives the attorney general the power to file civil suits to enforce
its provisions and some critics said Mr. Barr could use that to step in.

Some Democrats said they were less worried about direct intervention by
Mr. Trump or Mr. Barr, but said they could use their positions to prod
sympathetic state and local officials to block votes while fostering a
narrative undercutting the credibility of a vote count going against the
president.

``The president has very little, if any, power with how elections are
conducted,'' said Mr. Elias, the Democratic lawyer. ``Trump's power is
that he has no shame and that shamelessness has infected his entire
political party.''

He added, ``You cannot imagine the party of George Bush or of John
McCain or Mitt Romney or even Reince Priebus saying out loud the things
Donald Trump screams out loud on Twitter, in the Oval Office and the
Rose Garden on daily and weekly basis.''

With the prospect of an extended and messy count lasting long past
Election Day, new attention is focusing on deadlines set by federal law.
Under the so-called safe harbor provision, states have until Dec. 8 to
resolve disputes over the results, meaning only five weeks --- the same
deadline that
\href{https://www.nytimes.com/2000/12/13/us/bush-prevails-single-vote-justices-end-recount-blocking-gore-after-5-week.html}{led
to the Florida recount being called off} in 2000 with George W. Bush in
the lead.

Senator Marco Rubio, Republican of Florida, warning about
\href{https://medium.com/@SenatorMarcoRubio/americans-should-expect-election-chaos-7fa8a9ac5aa1}{``a
nightmare scenario for our nation,''} introduced legislation on Thursday
\href{https://www.rubio.senate.gov/public/_cache/files/7c86cdcc-19c2-4abd-9164-bacc5e66a499/27CC6B97AB3A0618568E38775DD4B657.mcg20709.pdf}{extending
that deadline} to Jan. 1, giving states three-and-a-half more weeks to
count. The Electoral College would then meet Jan. 2 instead of Dec. 14,
still in time to provide their results to Congress to ratify the outcome
on Jan. 6 as scheduled.

In the end, it may depend on how close the count really is.

If ``it's clear one candidate or the other has a clear majority in the
Electoral College, then I don't think there's much Trump could do if
he's the loser except to complain,'' said Trevor Potter, the president
of the Campaign Legal Center and former chairman of the Federal Election
Commission. ``But if it's close, then I think there is the potential for
lots of mischief.''

\hypertarget{our-2020-election-guide}{%
\section{Our 2020 Election Guide}\label{our-2020-election-guide}}

Updated Aug. 7, 2020

\begin{itemize}
\item
  \begin{center}\rule{0.5\linewidth}{\linethickness}\end{center}

  \hypertarget{the-latest}{%
  \subsection{The Latest}\label{the-latest}}

  \begin{itemize}
  \tightlist
  \item
    \href{https://www.nytimes.com/2020/08/07/us/politics/russia-china-trump-biden-election-interference.html?action=click\&pgtype=Article\&state=default\&region=BELOW_MAIN_CONTENT\&context=storylines_guide}{Russia
    is using a range of techniques to denigrate Joe Biden}, American
    intelligence officials said, declaring that Moscow continues to try
    to interfere in the 2020 campaign to help President Trump.
  \end{itemize}
\item
  \begin{center}\rule{0.5\linewidth}{\linethickness}\end{center}

  \hypertarget{bidens-vp-search}{%
  \subsection{Biden's V.P. Search}\label{bidens-vp-search}}

  \begin{itemize}
  \tightlist
  \item
    \href{https://www.nytimes.com/article/biden-vice-president-2020.html?action=click\&pgtype=Article\&state=default\&region=BELOW_MAIN_CONTENT\&context=storylines_guide}{Here
    are 13 women} who have been under consideration to be Joe Biden's
    running mate, and why each might be chosen --- and might not be.
  \end{itemize}
\item
  \begin{center}\rule{0.5\linewidth}{\linethickness}\end{center}

  \hypertarget{keep-up-with-our-coverage}{%
  \subsection{Keep Up With Our
  Coverage}\label{keep-up-with-our-coverage}}

  \begin{itemize}
  \tightlist
  \item
    Get an
    \href{https://www.nytimes.com/newsletters/politics?action=click\&pgtype=Article\&state=default\&region=BELOW_MAIN_CONTENT\&context=storylines_guide}{email}
    recapping the day's news
  \end{itemize}

  \begin{itemize}
  \tightlist
  \item
    Download our mobile app on
    \href{https://apps.apple.com/us/app/nytimes/id284862083?ls=1\&mat_click_id=5c79ae7455014fd1bd66b5610c05b8f2-20191112-16948\&referrer=mat_click_id\%3D5c79ae7455014fd1bd66b5610c05b8f2-20191112-16948\%26link_click_id\%3D722930677036718082}{iOS}
    and
    \href{http://a.localytics.com/android?id=com.nytimes.android\&referrer=utm_source\%3Dother_nyt_mobile_web\%26utm_medium\%3DWeb\%2520page\%26utm_term\%3DGeneral\%2520Mobile\%2520Page\%26utm_campaign\%3DNYT\%2520Mobile\%2520General\%2520Page}{Android}
    and turn on Breaking News and Politics alerts
  \end{itemize}
\end{itemize}

Advertisement

\protect\hyperlink{after-bottom}{Continue reading the main story}

\hypertarget{site-index}{%
\subsection{Site Index}\label{site-index}}

\hypertarget{site-information-navigation}{%
\subsection{Site Information
Navigation}\label{site-information-navigation}}

\begin{itemize}
\tightlist
\item
  \href{https://help.nytimes.com/hc/en-us/articles/115014792127-Copyright-notice}{©~2020~The
  New York Times Company}
\end{itemize}

\begin{itemize}
\tightlist
\item
  \href{https://www.nytco.com/}{NYTCo}
\item
  \href{https://help.nytimes.com/hc/en-us/articles/115015385887-Contact-Us}{Contact
  Us}
\item
  \href{https://www.nytco.com/careers/}{Work with us}
\item
  \href{https://nytmediakit.com/}{Advertise}
\item
  \href{http://www.tbrandstudio.com/}{T Brand Studio}
\item
  \href{https://www.nytimes.com/privacy/cookie-policy\#how-do-i-manage-trackers}{Your
  Ad Choices}
\item
  \href{https://www.nytimes.com/privacy}{Privacy}
\item
  \href{https://help.nytimes.com/hc/en-us/articles/115014893428-Terms-of-service}{Terms
  of Service}
\item
  \href{https://help.nytimes.com/hc/en-us/articles/115014893968-Terms-of-sale}{Terms
  of Sale}
\item
  \href{https://spiderbites.nytimes.com}{Site Map}
\item
  \href{https://help.nytimes.com/hc/en-us}{Help}
\item
  \href{https://www.nytimes.com/subscription?campaignId=37WXW}{Subscriptions}
\end{itemize}
