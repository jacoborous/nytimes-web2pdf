Sections

SEARCH

\protect\hyperlink{site-content}{Skip to
content}\protect\hyperlink{site-index}{Skip to site index}

\href{/section/world/europe}{Europe}\textbar{}When Covid-19 Hit, Many
Elderly Were Left to Die

\href{https://nyti.ms/3fEcqZo}{https://nyti.ms/3fEcqZo}

\begin{itemize}
\item
\item
\item
\item
\item
\end{itemize}

\href{https://www.nytimes.com/news-event/coronavirus?action=click\&pgtype=Article\&state=default\&region=TOP_BANNER\&context=storylines_menu}{The
Coronavirus Outbreak}

\begin{itemize}
\tightlist
\item
  live\href{https://www.nytimes.com/2020/08/08/world/coronavirus-updates.html?action=click\&pgtype=Article\&state=default\&region=TOP_BANNER\&context=storylines_menu}{Latest
  Updates}
\item
  \href{https://www.nytimes.com/interactive/2020/us/coronavirus-us-cases.html?action=click\&pgtype=Article\&state=default\&region=TOP_BANNER\&context=storylines_menu}{Maps
  and Cases}
\item
  \href{https://www.nytimes.com/interactive/2020/science/coronavirus-vaccine-tracker.html?action=click\&pgtype=Article\&state=default\&region=TOP_BANNER\&context=storylines_menu}{Vaccine
  Tracker}
\item
  \href{https://www.nytimes.com/interactive/2020/world/coronavirus-tips-advice.html?action=click\&pgtype=Article\&state=default\&region=TOP_BANNER\&context=storylines_menu}{F.A.Q.}
\item
  \href{https://www.nytimes.com/live/2020/08/07/business/stock-market-today-coronavirus?action=click\&pgtype=Article\&state=default\&region=TOP_BANNER\&context=storylines_menu}{Markets
  \& Economy}
\end{itemize}

\includegraphics{https://static01.nyt.com/images/2020/08/09/world/09virus-nursinghomes-p5/00virus-nursinghomes-articleLarge.jpg?quality=75\&auto=webp\&disable=upscale}

Behind the Curve

\hypertarget{when-covid-19-hit-many-elderly-were-left-to-die}{%
\section{When Covid-19 Hit, Many Elderly Were Left to
Die}\label{when-covid-19-hit-many-elderly-were-left-to-die}}

At the Christalain nursing home in Brussels, at least 14 residents died
from the coronavirus.Credit...

Supported by

\protect\hyperlink{after-sponsor}{Continue reading the main story}

By \href{https://www.nytimes.com/by/matina-stevis-gridneff}{Matina
Stevis-Gridneff}, \href{https://www.nytimes.com/by/matt-apuzzo}{Matt
Apuzzo} and Monika Pronczuk

Photographs by Mauricio Lima

\begin{itemize}
\item
  Aug. 8, 2020, 5:00 a.m. ET
\item
  \begin{itemize}
  \item
  \item
  \item
  \item
  \item
  \end{itemize}
\end{itemize}

Warnings had piled up for years that nursing homes were vulnerable. The
pandemic sent them to the back of the line for equipment and care.

\begin{center}\rule{0.5\linewidth}{\linethickness}\end{center}

BRUSSELS --- Shirley Doyen was exhausted. The Christalain nursing home,
which she ran with her brother in an affluent neighborhood in Brussels,
was buckling from Covid-19. Eight residents had died in three weeks.
Some staff members had only gowns and goggles from Halloween doctor
costumes for protection.

Nor was help coming. Ms. Doyen had begged hospitals to collect her
infected residents. They refused. Sometimes she was told to administer
morphine and let death come. Once she was told to pray.

Then, in the early morning of April 10, it all got worse.

First, a resident died at 1:20 a.m. Three hours later, another died. At
5:30 a.m., still another. The night nurse had long since given up
calling ambulances.

Ms. Doyen arrived after dawn and discovered Addolorata Balducci, 89, in
distress from Covid-19. Ms. Balducci's son, Franco Pacchioli, demanded
that paramedics be called and begged them to take his mother to the
hospital. Instead, they gave her morphine.

``Your mother will die,'' the paramedics responded, Mr. Pacchioli
recalled. ``That's it.''

The paramedics left. Eight hours later, Ms. Balducci died.

Runaway coronavirus infections, medical gear shortages and government
inattention are woefully familiar stories in nursing homes around the
globe. But Belgium's response offers a gruesome twist: Paramedics and
hospitals sometimes flatly denied care to elderly people, even as
hospital beds sat unused.

Weeks earlier, the virus had overwhelmed hospitals in Italy. Determined
to prevent that from happening in Belgium, the authorities shunned and
all but ignored nursing homes. But while Italian doctors said they were
forced to ration care to the elderly because of shortages of space and
equipment, Belgium's hospital system never came under similar strain.

Even at the height of the outbreak in April, when Ms. Balducci was
turned away, intensive-care beds were no more than about 55 percent
full.

``They wouldn't accept old people,'' Ms. Doyen said. ``They had space,
and they didn't want them.''

Belgium now has, by some measures, the world's highest coronavirus death
rate, in part because of nursing homes. More than 5,700 nursing-home
residents have died,
\href{https://www.medrxiv.org/content/10.1101/2020.06.20.20136234v1.full.pdf}{according
to newly published data}. During the peak of the crisis, from March
through mid-May, residents accounted for two out of every three
coronavirus deaths.

\includegraphics{https://static01.nyt.com/images/2020/08/09/world/09virus-nursinghomes-02-p3/00virus-nursinghomes-02-articleLarge.jpg?quality=75\&auto=webp\&disable=upscale}

Image

At the end of an activity to work on residents' memory capacity at
Christalain.

Image

A memorial for Christalain's victims of Covid-19.

Of all the missteps by governments during the coronavirus pandemic, few
have had such an immediate and devastating impact as the failure to
protect nursing homes. Tens of thousands of older people died ---
casualties not only of the virus, but of more than a decade of ignored
warnings that nursing homes were vulnerable.

Public health officials around the world excluded nursing homes from
their pandemic preparedness plans and omitted residents from the
mathematical models used to guide their responses.

In recent months, the coronavirus outbreak in the United States has
dominated global attention, as the world's richest nation blundered its
way into the world's largest death toll. Some 40 percent of those
fatalities
\href{https://www.kff.org/health-costs/issue-brief/state-data-and-policy-actions-to-address-coronavirus/\#stateleveldata}{have
been linked} to long-term-care facilities. But even now, European
countries lead the world in per capita deaths, in part because of what
happened inside their nursing homes.

Spanish prosecutors
\href{https://www.nytimes.com/2020/03/25/world/europe/Spain-coronavirus-nursing-homes.html}{are
investigating cases} in which residents were abandoned to die. In
Sweden, overwhelmed emergency doctors have acknowledged
\href{https://www.dn.se/nyheter/sverige/overlakare-logn-att-patienter-inte-prioriterats-bort/}{turning
away elderly patients}.

In Britain, the
\href{https://www.independent.co.uk/news/health/coronavirus-care-homes-nhs-hospital-discharges-deaths-a9544671.html}{government
ordered thousands of older hospital patients} --- including some with
Covid-19 --- sent back to nursing homes to make room for an expected
crush of virus cases. (Similar policies were in effect in
\href{https://www.nytimes.com/2020/04/24/us/nursing-homes-coronavirus.html}{some
American states}.)

But by fixating on saving their hospitals, European leaders sometimes
left nursing-home residents and staff to fend for themselves.

``We thought about it, and we said, `Care homes are important,''' Matt
Keeling, a British emergency adviser, testified recently. ``We thought
they were being shielded, and we probably thought that was enough.''

It wasn't. Only
\href{https://www.ecdc.europa.eu/sites/default/files/documents/covid-19-long-term-care-facilities-surveillance-guidance.pdf}{about
a third of European nursing homes} had infectious-disease teams before
the Covid-19 pandemic. Most lacked in-house doctors and
\href{https://www.ecdc.europa.eu/sites/default/files/media/en/publications/Publications/healthcare-associated-infections-point-prevalence-survey-long-term-care-facilities-2013.pdf}{many
had no arrangements} with outside physicians to coordinate care.

Few countries embody this lethally ineffective pandemic response more
than Belgium, where government officials excluded nursing-home patients
from the testing policy until thousands were already dead. Nursing homes
were left waiting for proper masks and gowns. When masks did arrive from
the government, they came late and were sometimes defective.

\hypertarget{latest-updates-the-coronavirus-outbreak}{%
\section{\texorpdfstring{\href{https://www.nytimes.com/2020/08/07/world/covid-19-news.html?action=click\&pgtype=Article\&state=default\&region=MAIN_CONTENT_1\&context=storylines_live_updates}{Latest
Updates: The Coronavirus
Outbreak}}{Latest Updates: The Coronavirus Outbreak}}\label{latest-updates-the-coronavirus-outbreak}}

Updated 2020-08-08T12:04:28.992Z

\begin{itemize}
\tightlist
\item
  \href{https://www.nytimes.com/2020/08/07/world/covid-19-news.html?action=click\&pgtype=Article\&state=default\&region=MAIN_CONTENT_1\&context=storylines_live_updates\#link-1f86d03a}{As
  the U.S. relief talks falter again, Trump says he is prepared to act
  on his own.}
\item
  \href{https://www.nytimes.com/2020/08/07/world/covid-19-news.html?action=click\&pgtype=Article\&state=default\&region=MAIN_CONTENT_1\&context=storylines_live_updates\#link-3f64a70a}{Cuomo
  says N.Y. schools can reopen in-person but leaves it up to districts
  to determine if, when and how.}
\item
  \href{https://www.nytimes.com/2020/08/07/world/covid-19-news.html?action=click\&pgtype=Article\&state=default\&region=MAIN_CONTENT_1\&context=storylines_live_updates\#link-14e70066}{Thousands
  of cases went unreported in California when a computer server failed.}
\end{itemize}

\href{https://www.nytimes.com/2020/08/07/world/covid-19-news.html?action=click\&pgtype=Article\&state=default\&region=MAIN_CONTENT_1\&context=storylines_live_updates}{See
more updates}

More live coverage:
\href{https://www.nytimes.com/live/2020/08/07/business/stock-market-today-coronavirus?action=click\&pgtype=Article\&state=default\&region=MAIN_CONTENT_1\&context=storylines_live_updates}{Markets}

``Tape the masks to the bridge of your nose,'' regional health officials
advised in one email.

One nursing-home executive, bereft of options, ordered thousands of
ponchos after seeing animal-keepers wearing them in a countryside zoo.
Another home managed to get 5,000 masks from a staff member's father in
Vietnam. The precious cargo arrived through the embassy's diplomatic
pouch.

Belgian officials say denying care for the elderly was never their
policy. But in the absence of a national strategy, and with regional
officials bickering about who was in charge, officials now acknowledge
that some hospitals and emergency responders relied on vague advice and
guidelines to do just that.

The situation was so dire that the charity Médecins Sans Frontières
dispatched teams of experts more accustomed to working in war-hardened
countries. On March 25, when a team arrived at Val des Fleurs, a public
nursing home a few miles from European Union headquarters, they were
greeted by the stale smell of disinfectant and an eerie stillness,
pierced only by the song of a caged canary.

Image

A gym class at the Val des Fleurs nursing home in Brussels, where a team
from Médecins Sans Frontières intervened during the peak of the
pandemic.

Image

Cleaning up a corridor at Val des Fleurs. Some staff members showed
signs of trauma common in disaster zones, the medical charity found.

Image

The nursing-home compound seen from the corridor of residents' bedrooms.
When the M.S.F. team arrived, both the director and her deputy were sick
with Covid-19.

Seventeen people had died there in the past 10 days. There was no
protective equipment. Oxygen was running low. Half the staff was
infected. Others showed signs of trauma common in disaster zones, a
psychologist from the medical charity concluded.

The director and her deputy were sick with Covid-19, and the acting
chief collapsed in a chair, crying, as soon as the team met her.

``I never thought I would work with M.S.F. in my own country. That's
crazy. We are a rich country,'' said Marine Tondeur, a Belgian nurse who
has worked in South Sudan and Haiti.

Ms. Tondeur was horrified at her country's response.

``I feel a bit ashamed, actually, that we forgot those homes.''

\hypertarget{firefighters-in-pajamas}{%
\subsection{`Firefighters in Pajamas'}\label{firefighters-in-pajamas}}

In February, as the coronavirus was taking root in northern Italy,
Belgian officials expressed little alarm. Maggie De Block, Belgium's
federal health minister, spent the month playing down the risk. She saw
\href{https://plus.lesoir.be/282733/article/2020-02-25/maggie-de-block-sur-le-coronavirus-au-soir-il-faut-eviter-les-reflexes-de}{no
need to worry} about hospital capacity or testing capabilities.

``It isn't a very aggressive virus. You would have to sneeze in
someone's face to pass it on,''
\href{https://www.maggiedeblock.be/coronavirus-er-is-veel-paniekzaaierij-de-tijd-03-03-2020/}{she
said} on March 3, adding, ``If the temperature rises, it will probably
disappear.''

Even after the World Health Organization
\href{https://www.who.int/docs/default-source/coronaviruse/who-china-joint-mission-on-covid-19-final-report.pdf}{highlighted
the importance of creating plans} to protect nursing homes, a spokesman
for the health authority in Belgium's Dutch-speaking region said there
was no reason to worry.

``The risk of infection is very small for now,'' he said.

Yet the warning signs were there. Belgium has one of the world's largest
nursing-home populations per capita, and
\href{http://www.nsih.be/download/LTCF/Rapport/HALT-3_Nat\%20Rapport_FR_V3.pdf}{years
of research} has shown that respiratory illnesses like Covid-19 are
among the most common diseases in such facilities. Data from China
demonstrated that the elderly were most at risk from Covid-19.

Government reports as far back as 2006 had called for infectious-disease
training for nursing-home doctors and public help to stockpile
protective equipment. A separate report in 2009 recommended adding
nursing homes to the national pandemic plan. Both proposals went
nowhere.

So, at the beginning of March, nursing homes were effectively on their
own. Belgium's internal risk-assessment documents did not even mention
nursing homes among the top concerns.

Image

A resident waiting as a staff member cleaned her bedroom at the Val des
Roses nursing home in Brussels, where Médecins Sans Frontières also
intervened.

Image

A bedroom in the section for residents affected by mental confusion and
disorientation at Val des Roses.

Image

A recreation area for residents at Val des Roses.

``We have received no specific recommendations from the ministers,'' the
nursing-home association Femarbel wrote to its members.

Nursing homes around the world operate at the seams of local, regional
and national oversight, but Belgium magnifies that problem. Divided by
language and
\href{https://www.nytimes.com/2018/12/18/world/europe/right-wing-migration-belgium-collapse.html}{perpetually
difficult to govern}, Belgium has so many layers of bureaucracy that it
is sometimes referred to as an administrative lasagna.

The country has not one but nine health ministers, who answer to six
parliaments. The federal government takes a coordinating role in a
pandemic, but nursing homes are the purview of regional authorities.

So even when officials realized the threat posed by Covid-19, they could
not act decisively.

``We needed several weeks to figure out who was responsible,'' Pedro
Facon, a top federal health official, testified this month.

By the middle of March, with the coronavirus spreading rapidly, regional
governments offered nursing homes advice --- yet it was unhelpful on key
points. Government documents stressed the importance of masks, while
simultaneously declaring them all but unavailable.

``There are virtually no masks available on the market,'' one document
said. Caregivers were advised to reuse masks, withhold them from
administrative staff members, and scrounge for gear from nearby
hospitals.

And scrounge they did. At the Christalain home, Steve Doyen --- the
co-owner and Ms. Doyen's brother --- said he found a handful of gowns
and goggles through a friend who liked dressing up as a doctor for
Halloween.

Image

A manicure session at Christalain. Activities are slowly restarting.

Image

A walker resting near a message reminding residents and staff members to
disinfect their hands at Christalain.

Image

Support messages on a board at Christalain.

Worsening the problem, Belgium was unable to test even a fraction of
those infected. So the health authorities decided to test severely ill,
hospitalized patients. Everyone else was told to recover at home.

That meant leaving contagious people inside crowded, understaffed,
underequipped nursing homes.

``We got the impression quite early on that we would take the back
seat,'' said Lesley Moreels, the director of a public nursing home in
Brussels. ``We felt that we were going to be firefighters in pajamas.''

\hypertarget{test-return-infect}{%
\subsection{Test, Return, Infect}\label{test-return-infect}}

Belgium went into lockdown on March 18. Dozens of nursing-home residents
had already died. Three days later, Jacqueline Van Peteghem, a
91-year-old resident at the Christalain home, was sent to UZ Brussel, a
nearby hospital, where she was tested for Covid-19. Within days, her
test came back positive.

Shirley and Steve Doyen assumed Ms. Van Peteghem would remain
hospitalized for treatment and to prevent the disease from spreading to
scores of other residents. But her symptoms had stabilized, and Mr.
Doyen said that a hospital doctor declared her healthy enough to return
home.

So, on March 27, paramedics in hazmat suits delivered Ms. Van Peteghem,
on a stretcher, to the door of Christalain.

Mr. Doyen greeted them wearing a surgical mask.

``Is this mask all you have?'' the paramedics asked, Mr. Doyen recalled.

``Yes,'' he said.

``Good luck,'' they responded.

For the next hour, Christalain staff members watched as the paramedics
decontaminated themselves and their ambulance. Asked later about the
hospital's policies, the chief executive, Prof. Marc Noppen, said
infectious patients were not normally returned to nursing homes but that
it may have happened in some cases.

No one can be certain if Ms. Van Peteghem's return was the reason, but
Covid-19 infections in the home increased. Residents began dying. Ms.
Van Peteghem, who initially survived the virus, died last month.

Image

Jacqueline Van Peteghem's dinner table in June.

Image

A resident holding onto a staff member at Christalain.

Image

The view from a window at Christalain.

The Belgian authorities were aware of such problems, according to
internal documents. ``Some patients have returned from the hospital
infected,'' a government emergency committee wrote on March 25.
``Several hot spots have been caused this way.''

\href{https://www.nytimes.com/news-event/coronavirus?action=click\&pgtype=Article\&state=default\&region=MAIN_CONTENT_3\&context=storylines_faq}{}

\hypertarget{the-coronavirus-outbreak-}{%
\subsubsection{The Coronavirus Outbreak
›}\label{the-coronavirus-outbreak-}}

\hypertarget{frequently-asked-questions}{%
\paragraph{Frequently Asked
Questions}\label{frequently-asked-questions}}

Updated August 6, 2020

\begin{itemize}
\item ~
  \hypertarget{why-are-bars-linked-to-outbreaks}{%
  \paragraph{Why are bars linked to
  outbreaks?}\label{why-are-bars-linked-to-outbreaks}}

  \begin{itemize}
  \tightlist
  \item
    Think about a bar. Alcohol is flowing. It can be loud, but it's
    definitely intimate, and you often need to lean in close to hear
    your friend. And strangers have way, way fewer reservations about
    coming up to people in a bar. That's sort of the point of a bar.
    Feeling good and close to strangers. It's no surprise, then, that
    \href{https://www.nytimes.com/2020/07/02/us/coronavirus-bars.html?action=click\&pgtype=Article\&state=default\&region=MAIN_CONTENT_3\&context=storylines_faq}{bars
    have been linked to outbreaks in several states.} Louisiana health
    officials have tied
    \href{https://www.nytimes.com/2020/06/22/us/new-coronavirus-phase.html?action=click\&pgtype=Article\&state=default\&region=MAIN_CONTENT_3\&context=storylines_faq}{at
    least 100 coronavirus cases} to bars in the Tigerland nightlife
    district in Baton Rouge. Minnesota has traced 328 recent cases to
    bars across the state.
    \href{https://www.boisestatepublicradio.org/post/bars-large-venues-close-ada-county-after-surge-coronavirus-prompts-rollback\#stream/0}{In
    Idaho}, health officials shut down bars in Ada County after
    reporting clusters of infections among young adults who had visited
    several bars in downtown Boise. Governors in
    \href{https://www.nytimes.com/2020/07/01/us/california-coronavirus-reopening.html?action=click\&pgtype=Article\&state=default\&region=MAIN_CONTENT_3\&context=storylines_faq}{California},
    \href{https://www.nytimes.com/2020/06/14/us/coronavirus-united-states.html?action=click\&pgtype=Article\&state=default\&region=MAIN_CONTENT_3\&context=storylines_faq}{Texas
    and Arizona}, where coronavirus cases are soaring, have ordered
    hundreds of newly reopened bars to shut down. Less than two weeks
    after Colorado's bars reopened at limited capacity, Gov. Jared Polis
    \href{https://www.denverpost.com/2020/06/30/colorado-bars-closed-coronavirus/}{ordered
    them to close}.
  \end{itemize}
\item ~
  \hypertarget{i-have-antibodies-am-i-now-immune}{%
  \paragraph{I have antibodies. Am I now
  immune?}\label{i-have-antibodies-am-i-now-immune}}

  \begin{itemize}
  \tightlist
  \item
    As of right now,
    \href{https://www.nytimes.com/2020/07/22/health/covid-antibodies-herd-immunity.html?action=click\&pgtype=Article\&state=default\&region=MAIN_CONTENT_3\&context=storylines_faq}{that
    seems likely, for at least several months.} There have been
    frightening accounts of people suffering what seems to be a second
    bout of Covid-19. But experts say these patients may have a
    drawn-out course of infection, with the virus taking a slow toll
    weeks to months after initial exposure. People infected with the
    coronavirus typically
    \href{https://www.nature.com/articles/s41586-020-2456-9}{produce}
    immune molecules called antibodies, which are
    \href{https://www.nytimes.com/2020/05/07/health/coronavirus-antibody-prevalence.html?action=click\&pgtype=Article\&state=default\&region=MAIN_CONTENT_3\&context=storylines_faq}{protective
    proteins made in response to an
    infection}\href{https://www.nytimes.com/2020/05/07/health/coronavirus-antibody-prevalence.html?action=click\&pgtype=Article\&state=default\&region=MAIN_CONTENT_3\&context=storylines_faq}{.
    These antibodies may} last in the body
    \href{https://www.nature.com/articles/s41591-020-0965-6}{only two to
    three months}, which may seem worrisome, but that's perfectly normal
    after an acute infection subsides, said Dr. Michael Mina, an
    immunologist at Harvard University. It may be possible to get the
    coronavirus again, but it's highly unlikely that it would be
    possible in a short window of time from initial infection or make
    people sicker the second time.
  \end{itemize}
\item ~
  \hypertarget{im-a-small-business-owner-can-i-get-relief}{%
  \paragraph{I'm a small-business owner. Can I get
  relief?}\label{im-a-small-business-owner-can-i-get-relief}}

  \begin{itemize}
  \tightlist
  \item
    The
    \href{https://www.nytimes.com/article/small-business-loans-stimulus-grants-freelancers-coronavirus.html?action=click\&pgtype=Article\&state=default\&region=MAIN_CONTENT_3\&context=storylines_faq}{stimulus
    bills enacted in March} offer help for the millions of American
    small businesses. Those eligible for aid are businesses and
    nonprofit organizations with fewer than 500 workers, including sole
    proprietorships, independent contractors and freelancers. Some
    larger companies in some industries are also eligible. The help
    being offered, which is being managed by the Small Business
    Administration, includes the Paycheck Protection Program and the
    Economic Injury Disaster Loan program. But lots of folks have
    \href{https://www.nytimes.com/interactive/2020/05/07/business/small-business-loans-coronavirus.html?action=click\&pgtype=Article\&state=default\&region=MAIN_CONTENT_3\&context=storylines_faq}{not
    yet seen payouts.} Even those who have received help are confused:
    The rules are draconian, and some are stuck sitting on
    \href{https://www.nytimes.com/2020/05/02/business/economy/loans-coronavirus-small-business.html?action=click\&pgtype=Article\&state=default\&region=MAIN_CONTENT_3\&context=storylines_faq}{money
    they don't know how to use.} Many small-business owners are getting
    less than they expected or
    \href{https://www.nytimes.com/2020/06/10/business/Small-business-loans-ppp.html?action=click\&pgtype=Article\&state=default\&region=MAIN_CONTENT_3\&context=storylines_faq}{not
    hearing anything at all.}
  \end{itemize}
\item ~
  \hypertarget{what-are-my-rights-if-i-am-worried-about-going-back-to-work}{%
  \paragraph{What are my rights if I am worried about going back to
  work?}\label{what-are-my-rights-if-i-am-worried-about-going-back-to-work}}

  \begin{itemize}
  \tightlist
  \item
    Employers have to provide
    \href{https://www.osha.gov/SLTC/covid-19/standards.html}{a safe
    workplace} with policies that protect everyone equally.
    \href{https://www.nytimes.com/article/coronavirus-money-unemployment.html?action=click\&pgtype=Article\&state=default\&region=MAIN_CONTENT_3\&context=storylines_faq}{And
    if one of your co-workers tests positive for the coronavirus, the
    C.D.C.} has said that
    \href{https://www.cdc.gov/coronavirus/2019-ncov/community/guidance-business-response.html}{employers
    should tell their employees} -\/- without giving you the sick
    employee's name -\/- that they may have been exposed to the virus.
  \end{itemize}
\item ~
  \hypertarget{what-is-school-going-to-look-like-in-september}{%
  \paragraph{What is school going to look like in
  September?}\label{what-is-school-going-to-look-like-in-september}}

  \begin{itemize}
  \tightlist
  \item
    It is unlikely that many schools will return to a normal schedule
    this fall, requiring the grind of
    \href{https://www.nytimes.com/2020/06/05/us/coronavirus-education-lost-learning.html?action=click\&pgtype=Article\&state=default\&region=MAIN_CONTENT_3\&context=storylines_faq}{online
    learning},
    \href{https://www.nytimes.com/2020/05/29/us/coronavirus-child-care-centers.html?action=click\&pgtype=Article\&state=default\&region=MAIN_CONTENT_3\&context=storylines_faq}{makeshift
    child care} and
    \href{https://www.nytimes.com/2020/06/03/business/economy/coronavirus-working-women.html?action=click\&pgtype=Article\&state=default\&region=MAIN_CONTENT_3\&context=storylines_faq}{stunted
    workdays} to continue. California's two largest public school
    districts --- Los Angeles and San Diego --- said on July 13, that
    \href{https://www.nytimes.com/2020/07/13/us/lausd-san-diego-school-reopening.html?action=click\&pgtype=Article\&state=default\&region=MAIN_CONTENT_3\&context=storylines_faq}{instruction
    will be remote-only in the fall}, citing concerns that surging
    coronavirus infections in their areas pose too dire a risk for
    students and teachers. Together, the two districts enroll some
    825,000 students. They are the largest in the country so far to
    abandon plans for even a partial physical return to classrooms when
    they reopen in August. For other districts, the solution won't be an
    all-or-nothing approach.
    \href{https://bioethics.jhu.edu/research-and-outreach/projects/eschool-initiative/school-policy-tracker/}{Many
    systems}, including the nation's largest, New York City, are
    devising
    \href{https://www.nytimes.com/2020/06/26/us/coronavirus-schools-reopen-fall.html?action=click\&pgtype=Article\&state=default\&region=MAIN_CONTENT_3\&context=storylines_faq}{hybrid
    plans} that involve spending some days in classrooms and other days
    online. There's no national policy on this yet, so check with your
    municipal school system regularly to see what is happening in your
    community.
  \end{itemize}
\end{itemize}

The committee recommended testing nursing-home residents and
establishing locations to house Covid-19 patients who would otherwise be
returned to homes.

But national and regional authorities could not agree on those
recommendations, and the country remained a hodgepodge of policies.

For another two weeks, even as the government expanded its testing
capability, health advisers resisted adding nursing homes to the
national testing priority list. They worried that even the newfound
capacity would be unable to meet the demand under the broadened
criteria, according to documents and government officials.

``The federal government had tests. Hospitals had tests,'' said Dr.
Emmanuel André, a virologist who was tapped as a top government adviser
and who advocated for broader federal testing. ``But nursing homes?
There were no tests allowed.''

As a stopgap measure, Philippe De Backer, a minister who had been tapped
to expand testing, pushed out an initial batch of nursing-home tests in
early April. But he and others wanted residents formally added to the
testing priority list. Support for that change finally coalesced on
April 8. Mr. De Backer dialed into a conference call of the government's
risk-management group --- one of many committees that set policy in
Belgium.

``You can stop debating,'' he said. ``We're testing in care homes.''

When the first results were announced, one in five residents tested
positive. By then, more than 2,000 residents had already died.

As the testing debate unfolded in late March and early April, hospitals
quietly stopped taking infected patients from nursing homes.

The policy --- officially it was just advice --- took shape in a series
of memos from Belgian geriatric specialists.

``Unnecessary transfers are a risk for ambulance workers and emergency
rooms,'' read an early memo, signed by the Belgian Society for
Gerontology and Geriatrics and two major hospitals.

Extremely frail patients and the terminally ill should receive
palliative care and not be hospitalized, the memo said. The document
offered a complex flowchart for deciding when to hospitalize
nursing-home residents.

The gerontology society says that its advice --- drafted in case of an
overwhelmed hospital system --- was misunderstood. The society is not a
government agency, doctors there note, and it never intended to deny
hospital care for the elderly.

But that is what happened.

\hypertarget{do-not-admit}{%
\subsection{Do Not Admit}\label{do-not-admit}}

On the morning of April 9, Dr. André, the government adviser, was
preparing for the daily news briefing when one question, submitted in
advance by a journalist, caught him by surprise: Would nursing-home
residents soon be allowed to go to the hospital?

``Why is this question coming?'' Dr. André remembers thinking. ``Yes, of
course they can.''

But time and again, nursing-home residents with Covid-19 symptoms were
denied hospitalization, even when referred by doctors who had assessed
that they might recover.

``The decision not to accept residents in hospitals really shocked me,''
said Michel Hanset, a doctor in Brussels who tried in vain to admit
several nursing-home patients.

No data exists on how often this happened, but Médecins Sans Frontières
says about 30 percent of the homes it worked in during its deployment
reported this problem.

Image

A resident enjoying the sun on a balcony at Val des Fleurs.

Image

An aquarium in a recreation room at Val des Fleurs.

Image

A resident of Val des Fleurs, left, receiving a visit.

Government figures are also telling. During the first weeks of the
crisis, nearly two thirds of nursing-home residents' deaths occurred in
hospitals. But as the crisis worsened, and the geriatric memos began
circulating, that number plummeted.

At the peak of the outbreak, a mere 14 percent of gravely ill residents
made it to hospitals. The rest died in their nursing homes, according to
government data
\href{https://www.medrxiv.org/content/10.1101/2020.06.20.20136234v1}{compiled
by Belgian scientists} and released to The New York Times.

It is impossible to know how many deaths were preventable. But hospitals
always had space. Even at the peak of the pandemic, 1,100 of the
nation's 2,400 intensive care beds were free, according to Niel Hens, a
government adviser and University of Antwerp professor.

``Paramedics had been instructed by their referral hospital not to take
patients over a certain age, often 75 but sometimes as low as 65,''
Médecins Sans Frontières said in a July
\href{https://www.msf.org/sites/msf.org/files/2020-07/Left\%20behind\%20-\%20MSF\%20care\%20homes\%20in\%20Belgium\%20report.pdf}{report}.

Some senior regional and national officials acknowledge this problem.

``I heard from staff in care homes that emergency doctors were arriving,
taking residents and then they were sending them back to care homes,
saying they could not keep them in the hospital,'' Christie Morreale,
the top health official in Wallonia, Belgium's French-speaking region,
said in an interview.

Ms. De Block, the national health minister, declined to be interviewed
and did not respond to written questions. In interviews, senior hospital
doctors defended their policies. They said that nursing-home staff
sought hospital care for terminally ill patients who needed to be
comforted into death, not dragged to the hospital.

If nursing-home residents were denied admission, they say, it was
because a doctor determined that they were unlikely to survive.

``If you think medical treatment is of benefit for that patient, he or
she will be hospitalized,'' said Professor Noppen, the UZ Brussel
executive. ``It's as simple as that.''

Nursing-home administrators are adamant that was not the case.

``At a certain point, there was an implicit age limit,'' said Marijke
Verboven of Orpea group, which owns 60 homes around Belgium.

Mr. Moreels, whose nursing home, Val des Roses, also had an intervention
from a Médecins Sans Frontières team, agreed. ``The ambulance wouldn't
take them,'' he said. ``There was no detailed consultation. They just
said `Why did you even call us?'''

Image

A nursing home worker playing table tennis with a resident at Val des
Roses.

Image

A resident's bedroom at Val des Roses.

Image

A pet dog playing in a recreation room at Val des Roses.

The Brussels ambulance service denied any policy of refusing to take
nursing-home residents to the hospital. Yet even some doctors are
skeptical.

``We learned that people from care homes believed it was not even worth
calling an ambulance,'' said Dr. Charlotte Martin, the chief
epidemiologist at Saint-Pierre Hospital in central Brussels. ``They
should have been the first ones to get in the pipeline. And instead they
were just forgotten.''

At the Christalain home, activities resumed this summer and life inched
toward something resembling normal. But a shadow remains: 14 residents
have been confirmed to have died of Covid-19. Another, devastated and
confused from the quarantine, killed herself in April.

Mr. Pacchioli, whose mother died after being refused hospitalization, is
haunted by a question. ``Maybe it wasn't too late,'' he said. ``If she
had gone to the hospital, maybe she would have survived.''

The Médecins Sans Frontières teams concluded their nursing-home missions
in Belgium in mid-June. Some members returned to developing countries.
Others now work in another rich nation in crisis: the United States.

Today, Ms. De Block, Belgium's national health minister, speaks about
the nursing homes as if they were
\href{https://www.politico.eu/article/in-defense-of-belgium-coronavirus-covid19-pandemic-response/}{an
unfortunate footnote} in a story of a successful government response.
She notes with pride that Belgium never ran out of hospital beds.

``We took measures at the right moment,'' she said in an
\href{https://www.nieuwsblad.be/cnt/dmf20200528_04974067}{interview},
adding, ``We can be proud.''

Image

A resident waiting for his lunch at Val des Fleurs.

Reporting was contributed by David Kirkpatrick and Selam Gebrekidan in
London, Julia Echikson and Koba Ryckewaert in Brussels, and Christina
Anderson in Stockholm.

Advertisement

\protect\hyperlink{after-bottom}{Continue reading the main story}

\hypertarget{site-index}{%
\subsection{Site Index}\label{site-index}}

\hypertarget{site-information-navigation}{%
\subsection{Site Information
Navigation}\label{site-information-navigation}}

\begin{itemize}
\tightlist
\item
  \href{https://help.nytimes.com/hc/en-us/articles/115014792127-Copyright-notice}{©~2020~The
  New York Times Company}
\end{itemize}

\begin{itemize}
\tightlist
\item
  \href{https://www.nytco.com/}{NYTCo}
\item
  \href{https://help.nytimes.com/hc/en-us/articles/115015385887-Contact-Us}{Contact
  Us}
\item
  \href{https://www.nytco.com/careers/}{Work with us}
\item
  \href{https://nytmediakit.com/}{Advertise}
\item
  \href{http://www.tbrandstudio.com/}{T Brand Studio}
\item
  \href{https://www.nytimes.com/privacy/cookie-policy\#how-do-i-manage-trackers}{Your
  Ad Choices}
\item
  \href{https://www.nytimes.com/privacy}{Privacy}
\item
  \href{https://help.nytimes.com/hc/en-us/articles/115014893428-Terms-of-service}{Terms
  of Service}
\item
  \href{https://help.nytimes.com/hc/en-us/articles/115014893968-Terms-of-sale}{Terms
  of Sale}
\item
  \href{https://spiderbites.nytimes.com}{Site Map}
\item
  \href{https://help.nytimes.com/hc/en-us}{Help}
\item
  \href{https://www.nytimes.com/subscription?campaignId=37WXW}{Subscriptions}
\end{itemize}
