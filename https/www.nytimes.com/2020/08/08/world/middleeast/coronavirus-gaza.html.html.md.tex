Sections

SEARCH

\protect\hyperlink{site-content}{Skip to
content}\protect\hyperlink{site-index}{Skip to site index}

\href{https://www.nytimes.com/section/world/middleeast}{Middle East}

\href{https://myaccount.nytimes.com/auth/login?response_type=cookie\&client_id=vi}{}

\href{https://www.nytimes.com/section/todayspaper}{Today's Paper}

\href{/section/world/middleeast}{Middle East}\textbar{}Coronavirus
Spares Gaza, but Travel Restrictions Do Not

\href{https://nyti.ms/3abf33x}{https://nyti.ms/3abf33x}

\begin{itemize}
\item
\item
\item
\item
\item
\end{itemize}

\href{https://www.nytimes.com/news-event/coronavirus?action=click\&pgtype=Article\&state=default\&region=TOP_BANNER\&context=storylines_menu}{The
Coronavirus Outbreak}

\begin{itemize}
\tightlist
\item
  live\href{https://www.nytimes.com/2020/08/08/world/coronavirus-updates.html?action=click\&pgtype=Article\&state=default\&region=TOP_BANNER\&context=storylines_menu}{Latest
  Updates}
\item
  \href{https://www.nytimes.com/interactive/2020/us/coronavirus-us-cases.html?action=click\&pgtype=Article\&state=default\&region=TOP_BANNER\&context=storylines_menu}{Maps
  and Cases}
\item
  \href{https://www.nytimes.com/interactive/2020/science/coronavirus-vaccine-tracker.html?action=click\&pgtype=Article\&state=default\&region=TOP_BANNER\&context=storylines_menu}{Vaccine
  Tracker}
\item
  \href{https://www.nytimes.com/interactive/2020/world/coronavirus-tips-advice.html?action=click\&pgtype=Article\&state=default\&region=TOP_BANNER\&context=storylines_menu}{F.A.Q.}
\item
  \href{https://www.nytimes.com/live/2020/08/07/business/stock-market-today-coronavirus?action=click\&pgtype=Article\&state=default\&region=TOP_BANNER\&context=storylines_menu}{Markets
  \& Economy}
\end{itemize}

Advertisement

\protect\hyperlink{after-top}{Continue reading the main story}

Supported by

\protect\hyperlink{after-sponsor}{Continue reading the main story}

\hypertarget{coronavirus-spares-gaza-but-travel-restrictions-do-not}{%
\section{Coronavirus Spares Gaza, but Travel Restrictions Do
Not}\label{coronavirus-spares-gaza-but-travel-restrictions-do-not}}

The blockaded Gaza Strip has not recorded any cases of community
transmission of the coronavirus, but new restrictions on movement
continue to make life difficult.

\includegraphics{https://static01.nyt.com/images/2020/08/07/world/07gaza-12/07gaza-12-articleLarge.jpg?quality=75\&auto=webp\&disable=upscale}

By Adam Rasgon and Iyad Abuheweila

\begin{itemize}
\item
  Aug. 8, 2020Updated 9:12 a.m. ET
\item
  \begin{itemize}
  \item
  \item
  \item
  \item
  \item
  \end{itemize}
\end{itemize}

JERUSALEM --- There was barely enough space to move on the popular Omar
al-Mukhtar street in Gaza City on the eve of the Muslim celebration of
Eid al-Adha as throngs of Palestinians --- almost none with masks ---
crowded into colorful clothing shops and huddled around makeshift food
stands.

``If the virus were here, we wouldn't be so close to each other,'' said
Saber Siam, 28, a salesman at a clothing store selling imported items
from China and Turkey. ``You wouldn't find me holding a customer's arm
or kissing his cheek to encourage him to purchase our clothes.''

The blockaded Gaza Strip might be one of the only places in the world
where no cases of community transmission of the coronavirus have been
recorded --- an achievement attributed to the coastal enclave's
isolation as well as swift measures taken by its militant Hamas rulers.

The pandemic, however, hasn't left Gaza untouched.

Citing the need to combat the virus, the various governmental
authorities controlling the borders of Gaza have imposed new
restrictions on movement outside the territory. That has exacerbated an
already challenging situation for Palestinians who say they urgently
need to travel to Israel and the West Bank, as well as for those wishing
to escape the bleak economic reality by emigrating by way of Egypt.

In March, fearing the potentially disastrous consequences of an outbreak
in Gaza, Hamas authorities ordered all travelers returning to the
territory by way of Israel and Egypt to enter quarantine facilities for
three weeks. They could not leave quarantine until they had passed two
virus tests.

The system seems to have succeeded, sparing Gaza's health sector,
already devastated by years of war and conflict, from additional strain.
Medical officials detected all 78 known infections in the territory at
quarantine facilities.

Still, experts did not rule out the possibility of the pandemic
penetrating into the area's densely populated cities and towns.

``All it takes is one small mistake,'' said Gerald Rockenschaub, the
head of the World Health Organization's mission to the Palestinians.
``There's no guarantee the virus won't get inside.''

Mr. Rockenschaub also warned that Gaza lacked the resources to deal with
a widespread outbreak, noting that medical institutions carry only about
100 adult ventilators, most of which are already in use.

Hamas initially instituted other restrictions in Gaza. But it later
lifted many of them, enabling residents to follow significant parts of
their daily routines. They have been flocking to beaches, working out at
gyms, eating at restaurants, praying at mosques and shopping in markets,
among other activities.

\hypertarget{latest-updates-the-coronavirus-outbreak}{%
\section{\texorpdfstring{\href{https://www.nytimes.com/2020/08/07/world/covid-19-news.html?action=click\&pgtype=Article\&state=default\&region=MAIN_CONTENT_1\&context=storylines_live_updates}{Latest
Updates: The Coronavirus
Outbreak}}{Latest Updates: The Coronavirus Outbreak}}\label{latest-updates-the-coronavirus-outbreak}}

Updated 2020-08-08T12:04:28.992Z

\begin{itemize}
\tightlist
\item
  \href{https://www.nytimes.com/2020/08/07/world/covid-19-news.html?action=click\&pgtype=Article\&state=default\&region=MAIN_CONTENT_1\&context=storylines_live_updates\#link-1f86d03a}{As
  the U.S. relief talks falter again, Trump says he is prepared to act
  on his own.}
\item
  \href{https://www.nytimes.com/2020/08/07/world/covid-19-news.html?action=click\&pgtype=Article\&state=default\&region=MAIN_CONTENT_1\&context=storylines_live_updates\#link-3f64a70a}{Cuomo
  says N.Y. schools can reopen in-person but leaves it up to districts
  to determine if, when and how.}
\item
  \href{https://www.nytimes.com/2020/08/07/world/covid-19-news.html?action=click\&pgtype=Article\&state=default\&region=MAIN_CONTENT_1\&context=storylines_live_updates\#link-14e70066}{Thousands
  of cases went unreported in California when a computer server failed.}
\end{itemize}

\href{https://www.nytimes.com/2020/08/07/world/covid-19-news.html?action=click\&pgtype=Article\&state=default\&region=MAIN_CONTENT_1\&context=storylines_live_updates}{See
more updates}

More live coverage:
\href{https://www.nytimes.com/live/2020/08/07/business/stock-market-today-coronavirus?action=click\&pgtype=Article\&state=default\&region=MAIN_CONTENT_1\&context=storylines_live_updates}{Markets}

``We're glad we haven't had to confront the death we've heard about in
other countries,'' said Moath Abed, 29, an unemployed nurse residing in
Gaza City.

Israeli authorities have permitted Palestinians in need of emergency and
lifesaving medical treatment to use the Erez crossing --- the sole
pedestrian passageway between Israel and Gaza.

But they have tightened restrictions on movement for others in the
territory, creating problems for people like Munir Sabitan, 53, a
resident of Gaza City who works in kitchen installation.

Mr. Sabitan used to visit his wife and three children, who are Arab
citizens of Israel, with a merchant's permit. In March, though, Israel
froze those permits as the virus started spreading in its communities.

Now Mr. Sabitan is concerned that he will miss his daughter's wedding in
the Negev desert region if Israel doesn't soon grant him permission to
cross the border.

``The wedding was postponed twice, but it won't be again,'' he said,
noting that the new date was Aug. 17. ``I feel drained from this
experience. My daughter calls everyday and I tell her I'm still waiting
for permission.''

Gisha, an Israeli rights group that closely monitors Gaza, appealed to
the Israeli authorities on Mr. Sabitan's behalf, saying they were
applying ``a double standard'' to him because they had allowed immediate
family members of other Israelis to fly into the country to participate
in weddings, bar mitzvahs and funerals.

``Israel is effectively tightening its closure on Gaza under the guise
of the pandemic,'' the group said.

The authorities have continued to deny Mr. Sabitan entry, pointing to
the pandemic as well as the May decision by the Palestinian Authority,
which governs Palestinians in the West Bank, to halt coordination with
Israel to protest Israel's threats to annex parts of the West Bank.

Among the matters that the authority will not coordinate on are travel
permits for Palestinians in Gaza, making it harder to apply for them.

The Coordinator of Government Activities in the Territories, the Israeli
Defense Ministry arm responsible for issuing permits to Palestinians,
declined to comment on specific cases. But it said it has been working
``around the clock'' to ``provide the best and most appropriate
response'' for Gaza's needs.

``We note that the narrowing down of movement through the Erez crossing
for exceptional medical and humanitarian cases is solely meant to
prevent the spread of the coronavirus,'' it said.

There are tens of thousands of active cases in both Israel and Egypt,
while there are more than 6,000 among Palestinians in the West Bank.

Iyad al-Bozom, the spokesman for the Hamas-run Interior Ministry, said
that since the pandemic, authorities in Gaza have allowed Palestinians
with valid Israeli permits and ``urgent travel needs'' to leave the
enclave through Erez. He said that if Mr. Sabitan receives a permit, he
would be able to exit the territory.

For the same reasons, Neveen Zanon, 41, a resident of Rafah, has also
not been able to get permission to visit her father in Nablus, where he
is suffering from esophageal cancer.

\includegraphics{https://static01.nyt.com/images/2020/08/07/world/07gaza-09/07gaza-09-articleLarge.jpg?quality=75\&auto=webp\&disable=upscale}

``He barely has enough energy to speak on the phone,'' she said. ``I'm
worried he won't be with us by the time I get a permit to see him.''

\href{https://www.nytimes.com/news-event/coronavirus?action=click\&pgtype=Article\&state=default\&region=MAIN_CONTENT_3\&context=storylines_faq}{}

\hypertarget{the-coronavirus-outbreak-}{%
\subsubsection{The Coronavirus Outbreak
›}\label{the-coronavirus-outbreak-}}

\hypertarget{frequently-asked-questions}{%
\paragraph{Frequently Asked
Questions}\label{frequently-asked-questions}}

Updated August 6, 2020

\begin{itemize}
\item ~
  \hypertarget{why-are-bars-linked-to-outbreaks}{%
  \paragraph{Why are bars linked to
  outbreaks?}\label{why-are-bars-linked-to-outbreaks}}

  \begin{itemize}
  \tightlist
  \item
    Think about a bar. Alcohol is flowing. It can be loud, but it's
    definitely intimate, and you often need to lean in close to hear
    your friend. And strangers have way, way fewer reservations about
    coming up to people in a bar. That's sort of the point of a bar.
    Feeling good and close to strangers. It's no surprise, then, that
    \href{https://www.nytimes.com/2020/07/02/us/coronavirus-bars.html?action=click\&pgtype=Article\&state=default\&region=MAIN_CONTENT_3\&context=storylines_faq}{bars
    have been linked to outbreaks in several states.} Louisiana health
    officials have tied
    \href{https://www.nytimes.com/2020/06/22/us/new-coronavirus-phase.html?action=click\&pgtype=Article\&state=default\&region=MAIN_CONTENT_3\&context=storylines_faq}{at
    least 100 coronavirus cases} to bars in the Tigerland nightlife
    district in Baton Rouge. Minnesota has traced 328 recent cases to
    bars across the state.
    \href{https://www.boisestatepublicradio.org/post/bars-large-venues-close-ada-county-after-surge-coronavirus-prompts-rollback\#stream/0}{In
    Idaho}, health officials shut down bars in Ada County after
    reporting clusters of infections among young adults who had visited
    several bars in downtown Boise. Governors in
    \href{https://www.nytimes.com/2020/07/01/us/california-coronavirus-reopening.html?action=click\&pgtype=Article\&state=default\&region=MAIN_CONTENT_3\&context=storylines_faq}{California},
    \href{https://www.nytimes.com/2020/06/14/us/coronavirus-united-states.html?action=click\&pgtype=Article\&state=default\&region=MAIN_CONTENT_3\&context=storylines_faq}{Texas
    and Arizona}, where coronavirus cases are soaring, have ordered
    hundreds of newly reopened bars to shut down. Less than two weeks
    after Colorado's bars reopened at limited capacity, Gov. Jared Polis
    \href{https://www.denverpost.com/2020/06/30/colorado-bars-closed-coronavirus/}{ordered
    them to close}.
  \end{itemize}
\item ~
  \hypertarget{i-have-antibodies-am-i-now-immune}{%
  \paragraph{I have antibodies. Am I now
  immune?}\label{i-have-antibodies-am-i-now-immune}}

  \begin{itemize}
  \tightlist
  \item
    As of right now,
    \href{https://www.nytimes.com/2020/07/22/health/covid-antibodies-herd-immunity.html?action=click\&pgtype=Article\&state=default\&region=MAIN_CONTENT_3\&context=storylines_faq}{that
    seems likely, for at least several months.} There have been
    frightening accounts of people suffering what seems to be a second
    bout of Covid-19. But experts say these patients may have a
    drawn-out course of infection, with the virus taking a slow toll
    weeks to months after initial exposure. People infected with the
    coronavirus typically
    \href{https://www.nature.com/articles/s41586-020-2456-9}{produce}
    immune molecules called antibodies, which are
    \href{https://www.nytimes.com/2020/05/07/health/coronavirus-antibody-prevalence.html?action=click\&pgtype=Article\&state=default\&region=MAIN_CONTENT_3\&context=storylines_faq}{protective
    proteins made in response to an
    infection}\href{https://www.nytimes.com/2020/05/07/health/coronavirus-antibody-prevalence.html?action=click\&pgtype=Article\&state=default\&region=MAIN_CONTENT_3\&context=storylines_faq}{.
    These antibodies may} last in the body
    \href{https://www.nature.com/articles/s41591-020-0965-6}{only two to
    three months}, which may seem worrisome, but that's perfectly normal
    after an acute infection subsides, said Dr. Michael Mina, an
    immunologist at Harvard University. It may be possible to get the
    coronavirus again, but it's highly unlikely that it would be
    possible in a short window of time from initial infection or make
    people sicker the second time.
  \end{itemize}
\item ~
  \hypertarget{im-a-small-business-owner-can-i-get-relief}{%
  \paragraph{I'm a small-business owner. Can I get
  relief?}\label{im-a-small-business-owner-can-i-get-relief}}

  \begin{itemize}
  \tightlist
  \item
    The
    \href{https://www.nytimes.com/article/small-business-loans-stimulus-grants-freelancers-coronavirus.html?action=click\&pgtype=Article\&state=default\&region=MAIN_CONTENT_3\&context=storylines_faq}{stimulus
    bills enacted in March} offer help for the millions of American
    small businesses. Those eligible for aid are businesses and
    nonprofit organizations with fewer than 500 workers, including sole
    proprietorships, independent contractors and freelancers. Some
    larger companies in some industries are also eligible. The help
    being offered, which is being managed by the Small Business
    Administration, includes the Paycheck Protection Program and the
    Economic Injury Disaster Loan program. But lots of folks have
    \href{https://www.nytimes.com/interactive/2020/05/07/business/small-business-loans-coronavirus.html?action=click\&pgtype=Article\&state=default\&region=MAIN_CONTENT_3\&context=storylines_faq}{not
    yet seen payouts.} Even those who have received help are confused:
    The rules are draconian, and some are stuck sitting on
    \href{https://www.nytimes.com/2020/05/02/business/economy/loans-coronavirus-small-business.html?action=click\&pgtype=Article\&state=default\&region=MAIN_CONTENT_3\&context=storylines_faq}{money
    they don't know how to use.} Many small-business owners are getting
    less than they expected or
    \href{https://www.nytimes.com/2020/06/10/business/Small-business-loans-ppp.html?action=click\&pgtype=Article\&state=default\&region=MAIN_CONTENT_3\&context=storylines_faq}{not
    hearing anything at all.}
  \end{itemize}
\item ~
  \hypertarget{what-are-my-rights-if-i-am-worried-about-going-back-to-work}{%
  \paragraph{What are my rights if I am worried about going back to
  work?}\label{what-are-my-rights-if-i-am-worried-about-going-back-to-work}}

  \begin{itemize}
  \tightlist
  \item
    Employers have to provide
    \href{https://www.osha.gov/SLTC/covid-19/standards.html}{a safe
    workplace} with policies that protect everyone equally.
    \href{https://www.nytimes.com/article/coronavirus-money-unemployment.html?action=click\&pgtype=Article\&state=default\&region=MAIN_CONTENT_3\&context=storylines_faq}{And
    if one of your co-workers tests positive for the coronavirus, the
    C.D.C.} has said that
    \href{https://www.cdc.gov/coronavirus/2019-ncov/community/guidance-business-response.html}{employers
    should tell their employees} -\/- without giving you the sick
    employee's name -\/- that they may have been exposed to the virus.
  \end{itemize}
\item ~
  \hypertarget{what-is-school-going-to-look-like-in-september}{%
  \paragraph{What is school going to look like in
  September?}\label{what-is-school-going-to-look-like-in-september}}

  \begin{itemize}
  \tightlist
  \item
    It is unlikely that many schools will return to a normal schedule
    this fall, requiring the grind of
    \href{https://www.nytimes.com/2020/06/05/us/coronavirus-education-lost-learning.html?action=click\&pgtype=Article\&state=default\&region=MAIN_CONTENT_3\&context=storylines_faq}{online
    learning},
    \href{https://www.nytimes.com/2020/05/29/us/coronavirus-child-care-centers.html?action=click\&pgtype=Article\&state=default\&region=MAIN_CONTENT_3\&context=storylines_faq}{makeshift
    child care} and
    \href{https://www.nytimes.com/2020/06/03/business/economy/coronavirus-working-women.html?action=click\&pgtype=Article\&state=default\&region=MAIN_CONTENT_3\&context=storylines_faq}{stunted
    workdays} to continue. California's two largest public school
    districts --- Los Angeles and San Diego --- said on July 13, that
    \href{https://www.nytimes.com/2020/07/13/us/lausd-san-diego-school-reopening.html?action=click\&pgtype=Article\&state=default\&region=MAIN_CONTENT_3\&context=storylines_faq}{instruction
    will be remote-only in the fall}, citing concerns that surging
    coronavirus infections in their areas pose too dire a risk for
    students and teachers. Together, the two districts enroll some
    825,000 students. They are the largest in the country so far to
    abandon plans for even a partial physical return to classrooms when
    they reopen in August. For other districts, the solution won't be an
    all-or-nothing approach.
    \href{https://bioethics.jhu.edu/research-and-outreach/projects/eschool-initiative/school-policy-tracker/}{Many
    systems}, including the nation's largest, New York City, are
    devising
    \href{https://www.nytimes.com/2020/06/26/us/coronavirus-schools-reopen-fall.html?action=click\&pgtype=Article\&state=default\&region=MAIN_CONTENT_3\&context=storylines_faq}{hybrid
    plans} that involve spending some days in classrooms and other days
    online. There's no national policy on this yet, so check with your
    municipal school system regularly to see what is happening in your
    community.
  \end{itemize}
\end{itemize}

She said that when her mother was ill in 2011, she wasn't able to
receive a permit until after her funeral took place.

``I don't want to go through such a painful experience again,'' said Ms.
Zanon, who lives in a cramped two-room apartment with her husband and
six children.

In March Egypt and Hamas sealed the Rafah crossing between Egypt and
Gaza for people trying to leave Gaza. Mr. al-Bozom of the Interior
Ministry said Hamas was concerned that quarantine facilities could
become overburdened if too many Palestinians exited Gaza through Rafah
and then returned shortly thereafter.

But the new restriction has complicated the plans of many young
Palestinians hoping to flee the poor living conditions in Gaza, where
youth unemployment is more than 60 percent and poverty is rampant.

Nidal Kuhail, 26, had intended to quit his job at a restaurant in Gaza
City, where he earns \$13.22 per day, and move to Europe to either study
or work. Now his plans are on hold.

Image

Nidal Kuhail, right, at his job where he manages waiters at a restaurant
in Gaza City.Credit...Fatima Shbair for The New York Times

``There's no future or horizon here,'' said Mr. Kuhail, who has been
studying German at a cultural institute in Gaza. ``The jobs are so few
and so many people are despairing. I feel like I have no other choice
but to immigrate.''

In Gaza, 32 percent of Palestinians said they want to emigrate because
of the economic, political and social situation, according to a June
poll by the Palestinian Center for Policy and Survey Research.

Egypt opened Rafah in May 2018 after years of keeping it largely closed,
and thousands, and perhaps tens of thousands, of Palestinians in Gaza
have moved abroad.

Mr. Kuhail, who has only left Gaza once in his life, said he still was
hopeful he would find a way to Europe.

``I'll eventually make it,'' he said. ``I know when I do, I'll be in a
place where there are opportunities to build a decent life.''

Adam Rasgon reported from Jerusalem and Iyad Abuheweila from Gaza City.

Advertisement

\protect\hyperlink{after-bottom}{Continue reading the main story}

\hypertarget{site-index}{%
\subsection{Site Index}\label{site-index}}

\hypertarget{site-information-navigation}{%
\subsection{Site Information
Navigation}\label{site-information-navigation}}

\begin{itemize}
\tightlist
\item
  \href{https://help.nytimes.com/hc/en-us/articles/115014792127-Copyright-notice}{©~2020~The
  New York Times Company}
\end{itemize}

\begin{itemize}
\tightlist
\item
  \href{https://www.nytco.com/}{NYTCo}
\item
  \href{https://help.nytimes.com/hc/en-us/articles/115015385887-Contact-Us}{Contact
  Us}
\item
  \href{https://www.nytco.com/careers/}{Work with us}
\item
  \href{https://nytmediakit.com/}{Advertise}
\item
  \href{http://www.tbrandstudio.com/}{T Brand Studio}
\item
  \href{https://www.nytimes.com/privacy/cookie-policy\#how-do-i-manage-trackers}{Your
  Ad Choices}
\item
  \href{https://www.nytimes.com/privacy}{Privacy}
\item
  \href{https://help.nytimes.com/hc/en-us/articles/115014893428-Terms-of-service}{Terms
  of Service}
\item
  \href{https://help.nytimes.com/hc/en-us/articles/115014893968-Terms-of-sale}{Terms
  of Sale}
\item
  \href{https://spiderbites.nytimes.com}{Site Map}
\item
  \href{https://help.nytimes.com/hc/en-us}{Help}
\item
  \href{https://www.nytimes.com/subscription?campaignId=37WXW}{Subscriptions}
\end{itemize}
