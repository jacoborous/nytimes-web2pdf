Sections

SEARCH

\protect\hyperlink{site-content}{Skip to
content}\protect\hyperlink{site-index}{Skip to site index}

\href{https://www.nytimes.com/section/climate}{Climate}

\href{https://myaccount.nytimes.com/auth/login?response_type=cookie\&client_id=vi}{}

\href{https://www.nytimes.com/section/todayspaper}{Today's Paper}

\href{/section/climate}{Climate}\textbar{}When Disasters Overlap

\href{https://nyti.ms/33url5Z}{https://nyti.ms/33url5Z}

\begin{itemize}
\item
\item
\item
\item
\item
\end{itemize}

\href{https://www.nytimes.com/section/climate?action=click\&pgtype=Article\&state=default\&region=TOP_BANNER\&context=storylines_menu}{Climate
and Environment}

\begin{itemize}
\tightlist
\item
  \href{https://www.nytimes.com/interactive/2020/08/06/climate/climate-change-inequality-heat.html?action=click\&pgtype=Article\&state=default\&region=TOP_BANNER\&context=storylines_menu}{Extreme
  Heat}
\item
  \href{https://www.nytimes.com/interactive/2020/climate/trump-environment-rollbacks.html?action=click\&pgtype=Article\&state=default\&region=TOP_BANNER\&context=storylines_menu}{Trump's
  Changes}
\item
  \href{https://www.nytimes.com/interactive/2020/04/19/climate/climate-crash-course-1.html?action=click\&pgtype=Article\&state=default\&region=TOP_BANNER\&context=storylines_menu}{Climate
  101}
\item
  \href{https://www.nytimes.com/interactive/2018/08/30/climate/how-much-hotter-is-your-hometown.html?action=click\&pgtype=Article\&state=default\&region=TOP_BANNER\&context=storylines_menu}{Is
  Your Hometown Hotter?}
\item
  \href{https://www.nytimes.com/newsletters/climate-change?action=click\&pgtype=Article\&state=default\&region=TOP_BANNER\&context=storylines_menu}{Newsletter}
\end{itemize}

Advertisement

\protect\hyperlink{after-top}{Continue reading the main story}

Supported by

\protect\hyperlink{after-sponsor}{Continue reading the main story}

Climate Fwd:

\hypertarget{when-disasters-overlap}{%
\section{When Disasters Overlap}\label{when-disasters-overlap}}

Also this week, how to help a neighbor when it's hot

By \href{https://www.nytimes.com/by/christopher-flavelle}{Christopher
Flavelle} and Greta Moran

\begin{itemize}
\item
  Aug. 5, 2020
\item
  \begin{itemize}
  \item
  \item
  \item
  \item
  \item
  \end{itemize}
\end{itemize}

\emph{Welcome to the} \emph{\textbf{Climate Fwd:}} \emph{newsletter. The
New York Times climate team emails readers once a week with stories and
insights about climate change.}
\href{https://www.nytimes.com/newsletters/climate-change}{\emph{Sign up
here}} \emph{to get it in your inbox.}

\includegraphics{https://static01.nyt.com/images/2020/08/05/world/05cli-newsletter-1/merlin_175293939_6e0ec7a5-a7fc-4069-ab55-e417ae132bf5-articleLarge.jpg?quality=75\&auto=webp\&disable=upscale}

\href{https://www.nytimes.com/by/christopher-flavelle}{\includegraphics{https://static01.nyt.com/images/2019/06/28/climate/author-chris-flavelle/author-chris-flavelle-thumbLarge-v3.png}}

By \href{https://www.nytimes.com/by/christopher-flavelle}{Christopher
Flavelle}

If you're looking for something to read during this summer of
coronavirus and you tend toward the macabre, here's a suggestion: Sign
up for the Federal Emergency Management Agency's daily operations
briefing, released via email around 9:30 Eastern time most mornings, and
count the number of ongoing disasters.

Wednesday morning's edition included Hurricane Isaias, which had just
plowed up the East Coast, knocking out power for millions of households
across a dozen states; wildfires in California and Nevada; the risk of
``severe thunderstorms'' in the Central Plains; parts of Texas still
waiting for damage assessments from Hurricane Hanna last weekend; and,
of course, Covid-19, the disease caused by the novel coronavirus, which
has killed 155,204 Americans, according to the agency's latest count.

As Henry Fountain and I
\href{https://www.nytimes.com/2020/08/04/climate/hurricane-isaias-apple-fire-climate.html?action=click\&module=Top\%20Stories\&pgtype=Homepage}{wrote
this week}, this is what living with climate change will look like: Not
just an epic, Katrina- or Sandy-scale catastrophe every few years
(though probably that, too), but a relentless grind of overlapping
disasters, major and minor. The number of disasters that FEMA is
handling is about twice what it was three years ago, before Hurricane
Harvey struck Texas, and that doesn't include its pandemic response.
Disaster preparation and recovery have blurred into a single frenzied
motion, never ending but also never quite succeeding.

The consequences of that shift are only starting to become apparent.
Homeowners begin rebuilding after a flood, only to flood again; cities
watch their
\href{https://www.bloomberg.com/news/features/2018-05-02/the-u-s-climate-strategy-of-total-retreat-is-failing?sref=UBrhZ1ro}{tax
rolls shrink} as property values fall; emergency managers at every level
of government are exhausted. And then, there's the money: Federal
watchdogs have begun warning, with increasing urgency, that the nation's
disaster spending is
\href{https://www.gao.gov/assets/710/702215.pdf}{not sustainable}.

The additional pressure of the pandemic has focused new attention on why
disasters are so damaging in the United States: Underfunded emergency
and public health agencies, weak
\href{https://www.nytimes.com/2019/10/26/climate/building-codes-secret-deal.html}{home
construction standards} that make evacuation so frequently necessary,
and racial and income disparities that put
\href{https://www.nytimes.com/2020/05/17/climate/pollution-poverty-coronavirus.html}{some
communities} at
\href{https://www.nytimes.com/2020/07/24/climate/houston-flooding-race.html}{greater
risk}. But the growing toll of disasters might also generate the
pressure required to address those problems, experts say.

\href{https://www.nytimes.com/section/climate?action=click\&pgtype=Article\&state=default\&region=MAIN_CONTENT_1\&context=storylines_keepup}{}

\hypertarget{climate-and-environment-}{%
\subsubsection{Climate and Environment
›}\label{climate-and-environment-}}

\hypertarget{keep-up-on-the-latest-climate-news}{%
\paragraph{Keep Up on the Latest Climate
News}\label{keep-up-on-the-latest-climate-news}}

Updated Aug. 7, 2020

Here's what you need to know about the latest climate change news this
week:

\begin{itemize}
\item
  \begin{itemize}
  \tightlist
  \item
    The move toward
    \href{https://www.nytimes.com/2020/08/05/us/politics/pebble-mine-trump-alaska.html?action=click\&pgtype=Article\&state=default\&region=MAIN_CONTENT_1\&context=storylines_keepup}{opening
    the Pebble Mine} in Alaska has surfaced a rare dispute between
    Donald Trump Jr. and his father's administration.
  \item
    Scientists at NOAA
    \href{https://www.nytimes.com/2020/08/06/climate/hurricanes-noaa-prediction.html?action=click\&pgtype=Article\&state=default\&region=MAIN_CONTENT_1\&context=storylines_keepup}{updated
    their prediction} for the 2020 hurricane season, and now expect as
    many as 25 named storms.
  \item
    \href{https://www.nytimes.com/2020/08/04/climate/hurricane-isaias-apple-fire-climate.html?action=click\&pgtype=Article\&state=default\&region=MAIN_CONTENT_1\&context=storylines_keepup}{Twin
    emergencies on two coasts this week} --- Hurricane Isaias and the
    Apple Fire --- offer a preview of life in a warming world and the
    steady danger of overlapping disasters.
  \end{itemize}
\end{itemize}

``We have got to use this as a politically neutral, unifying effort to
instill resilience,'' said Brock Long, who ran FEMA until last year.
``If we don't make holistic changes in the emergency management and
public health industries, as a result of going through '17, '18 and now
Covid-19, then we are learning nothing.''

Officials should still have plenty of opportunities this year to work on
their disaster strategies. There are four months left in hurricane
season, and the worst storms usually don't hit until the fall.

\begin{center}\rule{0.5\linewidth}{\linethickness}\end{center}

Image

Brooklyn apartments with window AC units in in July.Credit...Holly
Pickett for The New York Times

\hypertarget{one-thing-you-can-do-help-a-neighbor-beat-the-heat}{%
\subsection{One thing you can do: Help a neighbor beat the
heat}\label{one-thing-you-can-do-help-a-neighbor-beat-the-heat}}

By Greta Moran

Cities around the world are facing
\href{https://www.nytimes.com/2019/07/18/climate/heatwave-climate-change.html}{more
frequent and more deadly heat waves}, and New York is no exception.
Officials took note this year, delivering thousands of air-conditioners
to low-income seniors, providing millions in aid for summer utility
bills, and modifying the city's cooling center program to account for
risks from the coronavirus.

But community organizers say the city's response could use some help in
one key area: communication. The most heat-vulnerable New Yorkers ---
seniors, people of color and people with chronic illnesses --- sometimes
don't know about the programs available, according to activists.

``I know a lot of people who don't realize they are there or have
trouble accessing them,'' Sonal Jessel, a policy and advocacy
coordinator at \href{https://www.weact.org/}{WE ACT for Environmental
Justice}, said of the various New York City resources.

Getting the word out doesn't just depend on the city. Community groups
and individuals can play an important role, too.

Ms. Jessel noted, for example, that the locations of cooling centers in
New York City can be found on an
\href{https://maps.nyc.gov/cooling-center/inactive.html?1596560400000}{online
map} or by calling 311. The cooling centers change location, though, and
you can't check the map if you you're not online. The phone line is more
accessible, but can be hard to navigate.

Ms. Jessel had suggested the city do more targeted outreach, like taking
advantage of communications tools that are already widely used within
target groups. ``We have a lot of immigrant populations that use
WhatsApp as a primary mode of communication,'' she said. ``How can we
find a way to get that information to them in the spaces that they are
already using?''

Mike Harrington, an assistant director at The New School's Tishman
Environment and Design Center, also said community outreach was crucial.
``People tend to trust community groups and people that they see almost
every day more than the city, or other political bodies,'' he said.

\href{https://www.nytimes.com/interactive/2018/08/30/climate/how-much-hotter-is-your-hometown.html}{}

\includegraphics{https://static01.nyt.com/images/2018/08/30/us/how-much-hotter-is-your-hometown-promo-1535677591454/how-much-hotter-is-your-hometown-promo-1535677591454-articleLarge-v2.jpg}

\hypertarget{how-much-hotter-is-your-hometown-than-when-you-were-born}{%
\subsection{How Much Hotter Is Your Hometown Than When You Were
Born?}\label{how-much-hotter-is-your-hometown-than-when-you-were-born}}

See how days at or above 90 degrees Fahrenheit have changed in your
lifetime and how much hotter it could get.

Mr. Harrington said groups distributing food aid in cities, which are
currently very active because of the coronavirus, could easily provide
information on extreme heat. This conversation could be as
straightforward as: ``Hey, here's your food. Also, next week there's
going to be a heat wave event. So, make sure if you don't have an AC
that you have somewhere you can go, or you have people you can reach out
to.''

New York City officials noted that the
\href{https://portal.311.nyc.gov/article/?kanumber=KA-03305}{GetCool Air
Conditioner Program} had installed more than 48,000 AC units for
low-income seniors, and that the city does support direct outreach
programs.

The city's Be a Buddy program, for instance, started in 2017, works with
community organizations to pair volunteers with residents in
heat-vulnerable neighborhoods and check in on them during heat waves.

``The buddy program is one of the core pieces of our Cool Neighborhood
Strategy, which is our overall heat-resiliency strategy for the city,''
said Jainey Bavishi, director of the Mayor's Office of Resiliency.
``Heat is often known as a silent killer because most people who die
from extreme heat actually die in their homes, so we want to make sure
we are checking in on those residents on hot days.''

However, the program is still in the pilot phase, yet to secure a
permanent spot in New York City's hotter future.

The good news is that you don't need to be in a formal program to be a
buddy. And you don't need to be in New York. As Mr. Harrington said, it
doesn't take long to check on a neighbor. Making sure they know where to
go to cool off and hydrate could make a big difference in the next heat
wave. You could even help them get a free air-conditioner.

\emph{We'd love your feedback on this newsletter. We read every message,
and reply to many! Please email thoughts and suggestions to}
\href{mailto:climateteam@nytimes.com?subject=Newsletter\%20Feedback}{\emph{climateteam@nytimes.com}}\emph{.}

\emph{If you like what we're doing, please spread the word and send this
to your friends. You can}
\href{https://www.nytimes.com/newsletters/climate-change}{\emph{sign up
here}} \emph{to get our newsletter delivered to your inbox each week.}

\emph{And be sure to check out}
\href{https://www.nytimes.com/newsletters}{\emph{our full assortment of
free newsletters}} \emph{from The Times.}

Advertisement

\protect\hyperlink{after-bottom}{Continue reading the main story}

\hypertarget{site-index}{%
\subsection{Site Index}\label{site-index}}

\hypertarget{site-information-navigation}{%
\subsection{Site Information
Navigation}\label{site-information-navigation}}

\begin{itemize}
\tightlist
\item
  \href{https://help.nytimes.com/hc/en-us/articles/115014792127-Copyright-notice}{©~2020~The
  New York Times Company}
\end{itemize}

\begin{itemize}
\tightlist
\item
  \href{https://www.nytco.com/}{NYTCo}
\item
  \href{https://help.nytimes.com/hc/en-us/articles/115015385887-Contact-Us}{Contact
  Us}
\item
  \href{https://www.nytco.com/careers/}{Work with us}
\item
  \href{https://nytmediakit.com/}{Advertise}
\item
  \href{http://www.tbrandstudio.com/}{T Brand Studio}
\item
  \href{https://www.nytimes.com/privacy/cookie-policy\#how-do-i-manage-trackers}{Your
  Ad Choices}
\item
  \href{https://www.nytimes.com/privacy}{Privacy}
\item
  \href{https://help.nytimes.com/hc/en-us/articles/115014893428-Terms-of-service}{Terms
  of Service}
\item
  \href{https://help.nytimes.com/hc/en-us/articles/115014893968-Terms-of-sale}{Terms
  of Sale}
\item
  \href{https://spiderbites.nytimes.com}{Site Map}
\item
  \href{https://help.nytimes.com/hc/en-us}{Help}
\item
  \href{https://www.nytimes.com/subscription?campaignId=37WXW}{Subscriptions}
\end{itemize}
