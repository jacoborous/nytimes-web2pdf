Sections

SEARCH

\protect\hyperlink{site-content}{Skip to
content}\protect\hyperlink{site-index}{Skip to site index}

\href{https://www.nytimes.com/section/sports/tennis}{Tennis}

\href{https://myaccount.nytimes.com/auth/login?response_type=cookie\&client_id=vi}{}

\href{https://www.nytimes.com/section/todayspaper}{Today's Paper}

\href{/section/sports/tennis}{Tennis}\textbar{}After It All, Serena
Williams Still Has No. 24 In Sight

\href{https://nyti.ms/3gF0fwx}{https://nyti.ms/3gF0fwx}

\begin{itemize}
\item
\item
\item
\item
\item
\end{itemize}

Advertisement

\protect\hyperlink{after-top}{Continue reading the main story}

Supported by

\protect\hyperlink{after-sponsor}{Continue reading the main story}

on tennis

\hypertarget{after-it-all-serena-williams-still-has-no-24-in-sight}{%
\section{After It All, Serena Williams Still Has No. 24 In
Sight}\label{after-it-all-serena-williams-still-has-no-24-in-sight}}

After the coronavirus paused the tours, Williams is gearing up for the
U.S. Open, her next opportunity to tie Margaret Court's Grand Slam
singles title record.

\includegraphics{https://static01.nyt.com/images/2020/08/09/sports/07tennis-serena-sub/merlin_160252602_261d1c0d-04de-4dbc-b545-949df424aa39-articleLarge.jpg?quality=75\&auto=webp\&disable=upscale}

\href{https://www.nytimes.com/by/christopher-clarey}{\includegraphics{https://static01.nyt.com/images/2018/09/10/multimedia/author-christopher-clarey/author-christopher-clarey-thumbLarge.png}}

By \href{https://www.nytimes.com/by/christopher-clarey}{Christopher
Clarey}

\begin{itemize}
\item
  Published Aug. 7, 2020Updated Aug. 8, 2020, 12:03 p.m. ET
\item
  \begin{itemize}
  \item
  \item
  \item
  \item
  \item
  \end{itemize}
\end{itemize}

After her latest long and unexpected break from the sport she once
dominated, Serena Williams will return to competition next week at a new
tournament, the Top Seed Open in Lexington, Ky.

What's different about this layoff is that Williams's comeback to
tour-level tennis will be part of everyone else's.

Professional players have been on hiatus because the coronavirus
pandemic shut down both the men's and women's circuits in early March.
The question is, how does that affect Williams's chances of winning the
United States Open, the Grand Slam tournament scheduled to begin on Aug.
31?

``I think she has the same chances that she has had since the birth of
her daughter,'' her coach, Patrick Mouratoglou, said in a telephone
interview from France this week before traveling to Kentucky. ``She
absolutely has the level. It still depends a great deal on her whether
she wins a Grand Slam. The Covid, for me, has changed absolutely nothing
in that department.''

Mouratoglou, who has coached Williams since 2012 and helped her win 10
of her 23 Grand Slam singles titles, spent much of the pandemic break at
his academy near Nice, France, starting an exhibition league, Ultimate
Tennis Showdown, designed to particularly appeal to younger,
non-hardcore tennis fans.

But he still believes in the old guard when it comes to women's tennis
and has remained adamant since Williams became a mother in 2017 that she
still has what it takes, even at this late stage in her career, to win
her 24th Grand Slam singles title and tie Margaret Court's record.

She has come agonizingly close. Since returning to the tour in 2018
after a difficult childbirth, she has reached four Grand Slam finals ---
two at Wimbledon and two at the U.S. Open --- losing all of them in
straight sets.

After winning her first tournament as a mother in January in Auckland,
New Zealand, she arrived with renewed momentum at this year's Australian
Open, only to play one of her shakiest recent matches in a third-round
loss to Wang Qiang: the highest-ranked Chinese player, whom she had
overwhelmed, 6-1, 6-0, at the 2019 U.S. Open.

There were doubts about Williams's fitness and ability to handle the
biggest moments before the pandemic, and those doubts remain as she
returns at age 38 with the U.S. Open again in her sights.

``She has been training very well,'' said Mouratoglou, who has not seen
Williams in person since early March but has received regular updates on
her practices. ``She is still motivated, and I think she will be ready.
The only thing that is missing are the matches, the opponent across the
net, and there is nothing that can replace the feeling of competition,
and that's exactly why we are going to Lexington instead of waiting. She
needs the matches. We'll see how many she gets.''

Williams, who has a history of blood clots and has had life-threatening
pulmonary embolisms that affected her lungs, could potentially face
greater risks than an average world-class athlete if she contracts
Covid-19, the disease caused by the virus.

``The best thing for her is definitely not to catch it,'' said
Mouratoglou, emphasizing that he is not a physician and not qualified to
comment on the potential risks.

Williams, who declined to be interviewed for this article through her
management team, last competed on Feb. 8, when she was upset by
Anastasija Sevastova of Latvia in a Fed Cup match.

Since then, she has mostly been at home in Palm Beach Gardens, Fla.,
with her husband Alexis Ohanian and their 2-year-old daughter Olympia.

She and Ohanian, a venture capitalist, have
\href{https://www.nytimes.com/2020/07/21/sports/soccer/angel-city-fc-nwsl.html}{invested
in a National Women's Soccer League expansion franchise} in Los Angeles
(Olympia is an investor, too).

Williams, who was active on social media during the shutdown,
\href{https://www.instagram.com/p/CDkFg76HjNX/}{often kept it
lighthearted}: singing along to the Frozen 2 soundtrack with Olympia and
posting a workout with her older sister Venus Williams where they
adopted Arnold Schwarzenegger accents and talked about ``pumping iron.''

But she also has ventured into deeper and more topical territory:
focusing on the Black Lives Matter movement and social justice.

On Instagram, \href{https://www.instagram.com/tv/CBHFg2fH-Uz/?hl=en}{she
interviewed Ohanian} at home about his decision in June to step down
from the board of directors of Reddit, the social network he co-founded,
and to call for an African-American to be chosen as his replacement. The
conversation focused on Ohanian's increased awareness of his ``white
privilege'' and his desire to ``lean into the pain'' of knowing that he
is ``racist because of a system I inherited.''

Williams \href{https://www.instagram.com/tv/CBuFfWIq4Q0/?hl=en}{later
interviewed Bryan Stevenson}, the founder and executive director of the
Equal Justice Initiative who has advocated for prisoners on death row
and for lowering the rate of incarceration in the United States.

She called him ``a super hero'' and talked about the resistance she and
her sister faced when they arrived on tour and eventually dominated in
the early 2000s.

\includegraphics{https://static01.nyt.com/images/2020/08/07/sports/07tennis-serena-web-3/merlin_10035020_ef7712e4-141e-45e5-ad30-b7584cca808d-articleLarge.jpg?quality=75\&auto=webp\&disable=upscale}

``When Venus and I were winning every week, Grand Slam finals every
time, it wasn't a celebration on tour,'' she said, reflecting on waiting
in the locker room when Venus was playing and listening to the crowd's
reaction.

If Serena heard loud cheers, she said her ``heart would sink'' because
she knew Venus had lost a point or the match.

``But if it was complete silence, then I would be like, `OK, she's
winning,''' Serena said.

The negative reaction at that early stage was certainly not all because
of race. The Williamses' initial Grand Slam duels and finals were often,
awkward constrained occasions because the sisters were so close off the
court (as they remain) and unable to compete with their customary fire.

But Serena emphasized the challenges that come with succeeding in a
predominantly white sport.

``I played not only against my opponent,'' she said. ``I played against
crowds. I played against fans, and I've played against people, and as
things have gone on, I've been able to have a tremendous amount of more
fans, and it's been a wonderful experience, but I worked really hard to
get this experience.''

Williams complained about at one stage being ``underpaid'': likely a
reference to having lower off-court earnings than Maria Sharapova
earlier in her career despite having a superior record. Williams also
expressed frustration at the way her own game is sometimes
characterized.

``Tennis is a mental game and Black people are athletic,'' she said,
referring to the
\href{https://www.nytimes.com/2020/06/30/sports/soccer/soccer-racism-broadcasting.html}{stereotype
long held by some} that Black athletes succeed because of strength and
athletic ability, while their white counterparts rely on their
intelligence. ``So whenever I would win, it's like, `I'm so athletic.'
No, actually I use my brain a lot more than I get credit for. I really
use my brain a lot out on the court. Yeah, I'm powerful, but the most
powerful players don't win 23 Grand Slams.''

Winning her 24th with a new generation of players rising would be
perhaps her finest achievement. The situation in which tennis finds
itself only makes the chase more intriguing.

``If she wins this U.S. Open as her 24th, it will be the toughest Grand
Slam title I think she's ever going to win or maybe anybody for that
matter,'' said Chris Evert, the ESPN analyst and an 18-time major
singles champion.

Evert said players have long complained about the Open coming near the
end of the season when they are tired and complained about the traffic
and hectic atmosphere in New York.

``That's a piece of cake compared to this,'' she said of the 2020 Open.

Image

Williams with her coach, Patrick Mouratoglou, at this year's Australian
Open.~Mouratoglou has not seen Williams in person since March but has
received regular updates on her training.Credit...Scott Barbour/EPA, via
Shutterstock

It will not have a full-strength field. Ashleigh Barty of Australia, the
No. 1 women's singles player, already has withdrawn. So have No. 5 Elina
Svitolina and No. 7 Kiki Bertens, and No. 2 Simona Halep is leaning that
way, too. But major threats remain. Will Williams's deep experience and
greater familiarity with comebacks give her an added edge against her
younger opponents? Or will she lack the runway to find top gear?

She could have returned for the doubleheader later this month in New
York: the Western \& Southern Open followed by the U.S. Open in a
so-called bubble with strict health and safety restrictions.

But she decided instead to give herself more matches, which came as
quite a surprise to Jon Sanders, tournament director of the new Top Seed
Open, a lower-tier WTA event.

``My initial response was, `This isn't real right?''' he said this week.
``But I was assured it was, and we are honored to have her want to make
this part of her U.S. Open preparation. We take it as a great
responsibility to make sure she stays safe when she is here.''

Williams, ranked No. 9 in singles, is the only top 10 player in the
tournament, which will be the first tour event in North America since
the pandemic and will be played without spectators. But the field has
ample star power with No. 11 Aryna Sabalenka of Belarus, No. 14 Johanna
Konta of Britain, American teens Amanda Anisimova and Coco Gauff and the
40-year-old Venus Williams.

During the break many players took part in exhibitions or World
TeamTennis, the mixed-gender league that played from July 12 to Aug. 2
at the Greenbrier resort in West Virginia to create a protected
environment.

Though Venus Williams and Sofia Kenin, the 21-year-old American who won
this year's Australian Open, played the W.T.T. season, Serena chose to
continue training at home.

``I think those who have managed to get some competition will have a
real advantage, because they will be operational very quickly, if not
right away,'' Mouratoglou said. ``Those who have not competed for six
months will be starting a bit over, but for someone who has so much
experience like Serena, I think that will be less of a problem.''

Players should also be particularly eager and likely healthier, at least
in the short term, after using the longest break of most of their
careers to heal nagging injuries. There is still a concern that the
demands of intense competition after a long layoff could lead quickly to
new injuries.

``After waiting so long, the motivation and enthusiasm will be far above
average, and that could compensate for some weaknesses,'' Mouratoglou
said.

Image

Williams congratulating Bianca Andreescu of Canada after the 2019 U.S.
Open.Credit...Usa Today Uspw/USA Today Sports, via Reuters

Often Williams's cheerleader in chief, Mouratoglou does acknowledge that
the opposition has played a role in Williams's Grand Slam drought. She
was beaten by tour veterans Angelique Kerber and Halep at the last two
Wimbledons and by young talents Naomi Osaka and Bianca Andreescu at the
last two U.S. Opens.

``They did play fabulous matches, and if they had not played as well,
Serena would have had the chance to come back,'' he said. ``The others
are of course progressing and are strong, and I am not trying to
undervalue them. But Serena is Serena, and the real Serena in full
possession of her powers and with a winning mind-set, the person who
will stop her is not yet born. Actually she is surely already born but
she's not ready yet.''

For Mouratoglou, the keys for Williams to break her streak in New York
are optimum fitness, quality matches in the lead-up and the right mental
approach, likely a new mental approach.

Blocking out No. 24 is not an option.

``When you have an elephant in the room, you can say you don't see it,
but it's not easy to believe it,'' he said.

Advertisement

\protect\hyperlink{after-bottom}{Continue reading the main story}

\hypertarget{site-index}{%
\subsection{Site Index}\label{site-index}}

\hypertarget{site-information-navigation}{%
\subsection{Site Information
Navigation}\label{site-information-navigation}}

\begin{itemize}
\tightlist
\item
  \href{https://help.nytimes.com/hc/en-us/articles/115014792127-Copyright-notice}{©~2020~The
  New York Times Company}
\end{itemize}

\begin{itemize}
\tightlist
\item
  \href{https://www.nytco.com/}{NYTCo}
\item
  \href{https://help.nytimes.com/hc/en-us/articles/115015385887-Contact-Us}{Contact
  Us}
\item
  \href{https://www.nytco.com/careers/}{Work with us}
\item
  \href{https://nytmediakit.com/}{Advertise}
\item
  \href{http://www.tbrandstudio.com/}{T Brand Studio}
\item
  \href{https://www.nytimes.com/privacy/cookie-policy\#how-do-i-manage-trackers}{Your
  Ad Choices}
\item
  \href{https://www.nytimes.com/privacy}{Privacy}
\item
  \href{https://help.nytimes.com/hc/en-us/articles/115014893428-Terms-of-service}{Terms
  of Service}
\item
  \href{https://help.nytimes.com/hc/en-us/articles/115014893968-Terms-of-sale}{Terms
  of Sale}
\item
  \href{https://spiderbites.nytimes.com}{Site Map}
\item
  \href{https://help.nytimes.com/hc/en-us}{Help}
\item
  \href{https://www.nytimes.com/subscription?campaignId=37WXW}{Subscriptions}
\end{itemize}
