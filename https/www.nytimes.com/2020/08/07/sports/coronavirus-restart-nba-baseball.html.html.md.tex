Sections

SEARCH

\protect\hyperlink{site-content}{Skip to
content}\protect\hyperlink{site-index}{Skip to site index}

\href{https://www.nytimes.com/section/sports}{Sports}

\href{https://myaccount.nytimes.com/auth/login?response_type=cookie\&client_id=vi}{}

\href{https://www.nytimes.com/section/todayspaper}{Today's Paper}

\href{/section/sports}{Sports}\textbar{}The Weird, Disturbing (and
Comforting) Return of Pro Sports

\href{https://nyti.ms/3kjRlqy}{https://nyti.ms/3kjRlqy}

\begin{itemize}
\item
\item
\item
\item
\item
\item
\end{itemize}

Advertisement

\protect\hyperlink{after-top}{Continue reading the main story}

Supported by

\protect\hyperlink{after-sponsor}{Continue reading the main story}

essay

\hypertarget{the-weird-disturbing-and-comforting-return-of-pro-sports}{%
\section{The Weird, Disturbing (and Comforting) Return of Pro
Sports}\label{the-weird-disturbing-and-comforting-return-of-pro-sports}}

The swirl of conflicting emotions about the return of sports, and
whether it will last, seems apt for these turbulent times of pandemic
and social unrest.

\includegraphics{https://static01.nyt.com/images/2020/08/07/sports/07virus-barry-essay-1/merlin_175347507_4775680e-0b1d-428b-ab2f-f74866de0bb2-articleLarge.jpg?quality=75\&auto=webp\&disable=upscale}

\href{https://www.nytimes.com/by/dan-barry}{\includegraphics{https://static01.nyt.com/images/2018/02/16/multimedia/author-dan-barry/author-dan-barry-thumbLarge.jpg}}

By \href{https://www.nytimes.com/by/dan-barry}{Dan Barry}

\begin{itemize}
\item
  Aug. 7, 2020
\item
  \begin{itemize}
  \item
  \item
  \item
  \item
  \item
  \item
  \end{itemize}
\end{itemize}

Finally, we thought: Sports.

After months of living in a state of paralytic dread, we could once
again find distraction in a jump shot, a slap shot, a moon shot. We
could set aside our fears and anxieties --- our grief --- for matters no
more urgent than a man on second with two out and the count
three-and-two.

Of course, the pandemic would keep our exuberant fandom in check. We
would still not be able to sit cheek-by-jowl in a ballpark, spraying
words, sauerkraut and who knows what else with every pitch. Or boo ---
in person! --- the Houston Astros for the cheating that forever taints
their 2017 championship. Or have the distinct honor of paying \$11 for a
beer at Citi Field.

But at least there would be sports from a safe distance. We could grab a
cold one, lean back and lose ourselves in a televised game of ultimate
and wonderful inconsequence.

If only we had the capacity of certain leaders for magical thinking.

Such tricks of the mind would allow us to take in the dystopian comedy
of it all --- the prerecorded crowd noises and the cardboard and video
stand-ins for fans --- without thinking too hard about the health
implications for the athletes; without being nagged by thoughts of: Is
this all right?

There is no question that sports have provided some diversion since they
stutter-stepped back into our lives. That offensive put-back by Anthony
Davis of the Los Angeles Lakers. That orange-and-white twirl of a
3-pointer by Sabrina Ionescu of the Liberty. That putt from another ZIP
code by Dustin Johnson. That pair of home runs by Aaron Judge of the
Yankees.

Nor is there any question that basketball, in particular, has insisted
that we not forget matters even larger than a contagion. W.N.B.A.
players have dedicated the season to Breonna Taylor, a Black medical
worker who was killed by police during the execution of a no-knock
warrant in Louisville. And N.B.A. players are wearing jerseys with words
of righteousness. Black Lives Matter. Equality. Justice. Vote.

\includegraphics{https://static01.nyt.com/images/2020/08/07/sports/07virus-barry-essay-2/merlin_174967407_29df324c-58cb-449a-921b-77d582a2736f-articleLarge.jpg?quality=75\&auto=webp\&disable=upscale}

Still, our sports cable package for the summer of 2020 has been on the W
\& D Channel: Weird and Disturbing.

Remember when a mourning dove flew across a baseball diamond in 2001 ---
just as Randy Johnson unleashed one of his trademark 100-mile-per-hour
fastballs? Ball met bird; bird ceased to exist.

Or when Rocky, the Denver Nuggets mascot, passed out while being lowered
from the rafters as part of an over-the-top stunt in 2013? He would
recover, but not before descending, seemingly lifeless, to the floor.

Well, multiply exploding bird and unconscious mascot by a thousand and
it's still nowhere close to matching the mind-boggling oddness of the
2020 sports world.

To begin with, hockey and basketball are not supposed to be ``in
season'' in August. Nor are their games supposed to be played in a
bio-dome (Please keep a safe social distance from the 1996 Pauly Shore
``comedy'' by the same name.).

It must be noted that these restricted venues for playoff contenders
have their charms. For example, they have allowed games to be played in
states of purity, free of T-shirt cannons, tossed seafood and the
Knicks.

\hypertarget{the-coronavirus-outbreak}{%
\subsubsection{The Coronavirus
Outbreak}\label{the-coronavirus-outbreak}}

\hypertarget{sports-and-the-virus}{%
\paragraph{Sports and the Virus}\label{sports-and-the-virus}}

Updated Aug. 7, 2020

Here's what's happening as the world of sports slowly comes back to
life:

\begin{itemize}
\item
  \begin{itemize}
  \tightlist
  \item
    Baseball
    \href{https://www.nytimes.com/2020/08/06/sports/baseball/mlb-safety-protocols.html?action=click\&pgtype=Article\&state=default\&region=MAIN_CONTENT_2\&context=storylines_keepup}{tightened
    its virus protocols} again: Players and staff members must wear
    masks in more places and cannot visit ``bars, lounges or malls''
    when they are home.
  \item
    With no live crowd noise as a buffer at baseball games,
    \href{https://www.nytimes.com/2020/08/06/sports/baseball/mlb-swearing.html?action=click\&pgtype=Article\&state=default\&region=MAIN_CONTENT_2\&context=storylines_keepup}{on-field
    sounds are easy to hear} on broadcasts --- and it's not all rated
    PG.
  \item
    The University of Connecticut
    \href{https://www.nytimes.com/2020/08/05/sports/ncaafootball/coronavirus-uconn-cancels-football.html?action=click\&pgtype=Article\&state=default\&region=MAIN_CONTENT_2\&context=storylines_keepup}{canceled
    its football season}, and Divisions II and III scrapped all of their
    fall championships.
  \end{itemize}
\end{itemize}

Ultimately, though, sports have often mirrored the unsettling and
surreal that continues to be lived beyond the playing field.

Image

Crowd noise has not been an issue in golf, with Dustin Johnson seemingly
having the entirety of T.P.C. River Highlands to himself back in
June.Credit...Rob Carr/Getty Images

Take the team benches in the Florida ``bubble'' where N.B.A. games are
being held and where players are routinely tested before being allowed
to compete. Each team has three rows of chairs separated by considerable
space. Players not in the game often wear face masks. They even have
their own Gatorade stations to ward against sharing drinks.

Then, with the wave of a coach's hand, they are sent into the game to
play ``man to man,'' breathing on and sweating against their opponents
as though it was 2019 --- old school.

Witnessing all this, virtually, are fans who have used a certain app
that allows their images to appear to be cheering from the stands. But
there is no uniformity in size, resulting in what looks like
Humpty-Dumpty-size spectators propped in chairs beside those of normal
proportions.

Once seen, it cannot be unseen.

The struggle to suspend disbelief is even harder when watching baseball.

Some of the empty ballparks have dispensed with pretense. At Nationals
Park in Washington, D.C., television viewers are treated to a blue
backdrop of empty seats bearing the Delta Air Lines logo, as if to
remind us of a time when people actually flew to destinations; when
people actually HAD destinations.

Citi Field, meanwhile, has filled seats with cardboard cutouts of Mets
fans who have donated to charity the sum of \$86. This is either a
reference to the team's last world championship --- 34 long years ago
--- or a signal that this plague year should just be eighty-sixed from
memory.

Watching a baseball broadcast, you can be lulled into thinking that
nothing has changed, an illusion fostered in part by the canned crowd
noise resounding in the background. But then come moments that jar.

Image

Mets fans had the option of purchasing cutouts bearing their picture to
be displayed during games at Citi Field. The \$86 cost does not include
the ability to retrieve them at the end of the season.Credit...Seth
Wenig/Associated Press

When Yankees relief pitcher Zack Britton entered the game against the
Philadelphia Phillies the other night, the broadcaster Michael Kay
explained his appearance this way:

``He's filling in for Aroldis Chapman, who was starting the season on
the Covid-19 list. He's back, he's had two negative tests in a row, but
they don't want to rush him. They want him to face some batters. So
Britton is the closer until Chapman gets back.''

The ``Covid-19 list?''

Delivered so matter-of-factly, Kay's report sounded as if Chapman had
begun the season on the injured list, gone through rehab, and was now
poised to rejoin the team. Only his ``injury'' was the coronavirus.

These are the moments that distract from the intended distractions.

In the last two weeks, 20 people associated with the Miami Marlins
(including 18 players) and 13 people with the St. Louis Cardinals
(including seven players) have tested positive for Covid-19. More than
simply disrupting the scheduling of a truncated season, these positive
tests have raised an acutely uncomfortable question:

Is all this worth it?

The other day, Mets outfielder Yoenis Cespedes failed to show up for a
game, an aberration not entirely out of character for either Cespedes or
the Mets. First came ridicule of Cespedes. Then came concern (Had he met
harm?). Then came word that he had opted out of the 2020 season because
of concerns about the coronavirus.

The skeptical among us might roll our eyes. We might note that Cespedes
was batting just .161. We might say that he had walked out on his
teammates. We might mutter that since he becomes a free agent at
season's end, this was nothing more than business.

Or we might raise our cold drinks in a toast to Cespedes for making a
decision that felt right for him --- then lean back and try in vain to
get comfortable as we watch others play through our collective pain.

Advertisement

\protect\hyperlink{after-bottom}{Continue reading the main story}

\hypertarget{site-index}{%
\subsection{Site Index}\label{site-index}}

\hypertarget{site-information-navigation}{%
\subsection{Site Information
Navigation}\label{site-information-navigation}}

\begin{itemize}
\tightlist
\item
  \href{https://help.nytimes.com/hc/en-us/articles/115014792127-Copyright-notice}{©~2020~The
  New York Times Company}
\end{itemize}

\begin{itemize}
\tightlist
\item
  \href{https://www.nytco.com/}{NYTCo}
\item
  \href{https://help.nytimes.com/hc/en-us/articles/115015385887-Contact-Us}{Contact
  Us}
\item
  \href{https://www.nytco.com/careers/}{Work with us}
\item
  \href{https://nytmediakit.com/}{Advertise}
\item
  \href{http://www.tbrandstudio.com/}{T Brand Studio}
\item
  \href{https://www.nytimes.com/privacy/cookie-policy\#how-do-i-manage-trackers}{Your
  Ad Choices}
\item
  \href{https://www.nytimes.com/privacy}{Privacy}
\item
  \href{https://help.nytimes.com/hc/en-us/articles/115014893428-Terms-of-service}{Terms
  of Service}
\item
  \href{https://help.nytimes.com/hc/en-us/articles/115014893968-Terms-of-sale}{Terms
  of Sale}
\item
  \href{https://spiderbites.nytimes.com}{Site Map}
\item
  \href{https://help.nytimes.com/hc/en-us}{Help}
\item
  \href{https://www.nytimes.com/subscription?campaignId=37WXW}{Subscriptions}
\end{itemize}
