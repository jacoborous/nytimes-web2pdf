Sections

SEARCH

\protect\hyperlink{site-content}{Skip to
content}\protect\hyperlink{site-index}{Skip to site index}

\href{https://www.nytimes.com/section/politics}{Politics}

\href{https://myaccount.nytimes.com/auth/login?response_type=cookie\&client_id=vi}{}

\href{https://www.nytimes.com/section/todayspaper}{Today's Paper}

\href{/section/politics}{Politics}\textbar{}How Joe Arpaio's Fate in
Arizona Could Be a Window Into Trump's

\url{https://nyti.ms/2EDJ7cl}

\begin{itemize}
\item
\item
\item
\item
\item
\end{itemize}

Advertisement

\protect\hyperlink{after-top}{Continue reading the main story}

Supported by

\protect\hyperlink{after-sponsor}{Continue reading the main story}

\hypertarget{how-joe-arpaios-fate-in-arizona-could-be-a-window-into-trumps}{%
\section{How Joe Arpaio's Fate in Arizona Could Be a Window Into
Trump's}\label{how-joe-arpaios-fate-in-arizona-could-be-a-window-into-trumps}}

The polarizing former sheriff of Maricopa County, a stylistic
doppelgänger to President Trump, is running for his old office in a
state where approval of both men has slid.

\includegraphics{https://static01.nyt.com/images/2020/08/03/us/politics/03arizona-arpaio1/03arizona-arpaio1-articleLarge.jpg?quality=75\&auto=webp\&disable=upscale}

By Hank Stephenson

\begin{itemize}
\item
  Aug. 2, 2020
\item
  \begin{itemize}
  \item
  \item
  \item
  \item
  \item
  \end{itemize}
\end{itemize}

PHOENIX --- After 24 years of doling out his punitive brand of justice
in Arizona's most populous county, Joe Arpaio, who billed himself as
``America's toughest sheriff,'' suffered a landslide defeat in 2016,
largely because of his hard-line immigration stances and his own
pugnacious defiance, which earned him a criminal conviction for contempt
of court.

Now he's trying to win back his old job.

Mr. Arpaio faces his first test in the Republican primary election on
Tuesday, when he must survive a three-way race that includes a challenge
from his former chief deputy,
\href{https://sheridan4sheriff2020.com/qualifications/}{Jerry Sheridan}.

Few in the state believe Mr. Arpaio, 88, can mount a successful comeback
and win in November, saying that he's too old, too out of touch or too
politically damaged to run a credible campaign in 2020.

There are signs that Mr. Arpaio, a former kingmaker in Republican
circles, may not even survive the primary. Nearly 80 percent of
Arizonans cast their ballots early by mail, and a recent poll of
Republicans who had already voted showed Mr. Arpaio and Mr. Sheridan
statistically tied.

Still, strategists and political operatives are monitoring Mr. Arpaio's
fate for signs of the broader implications for Arizona politics. The
former sheriff had closely aligned himself with
\href{https://www.nytimes.com/interactive/2020/us/elections/donald-trump.html}{President
Trump} on immigration, earning the president's praise. The two men are
stylistic doppelgängers who vilify undocumented immigrants and are
pushing a strident law-and-order message amid a nationwide movement to
stop police abuses against people of color.

When Mr. Trump issued the first pardon of his presidency, in August
2017,
\href{https://www.nytimes.com/2017/08/25/us/politics/joe-arpaio-trump-pardon-sheriff-arizona.html}{it
went to Mr. Arpaio}.

Mr. Arpaio's falling star among Republicans in the vast suburbs of
Maricopa County, which includes Phoenix and surrounding communities,
signals not only his fading appeal. It's also a sign of Mr. Trump's
downward arc in a state that has long leaned conservative but is now
considered a critical up-for-grabs presidential battleground.

\includegraphics{https://static01.nyt.com/images/2020/08/03/us/politics/03arizona-arpaio2/03arizona-arpaio2-articleLarge.jpg?quality=75\&auto=webp\&disable=upscale}

Recent polls show
\href{https://www.nytimes.com/interactive/2020/us/elections/joe-biden.html}{Joseph
R. Biden Jr.} holding a slight edge over Mr. Trump, who
\href{https://www.nytimes.com/elections/2016/results/arizona}{won the
state} by 3.5 percentage points in 2016. A New York Times/Siena College
poll in June
\href{https://www.nytimes.com/2020/06/25/upshot/poll-2020-biden-battlegrounds.html}{found
Mr. Biden ahead} of the president by seven percentage points.

``If you want to track the trajectory of Trumpism, you should study
Arizona circa 2006 to about 2016,'' said Kirk Adams, a Republican former
speaker of the Arizona House and former chief of staff to Gov. Doug
Ducey. ``Arizona was the precursor.''

Democrats know Mr. Arpaio's popularity has slipped among Republicans and
independents, and are eager to leverage Mr. Trump's fondness for him as
a way to bludgeon the president. Tom Perez, the chairman of the
Democratic National Committee, told reporters on a conference call in
late June that national Democrats would ``leave no ZIP code behind'' to
flip the newly minted battleground state.

``If the voters in Arizona want to know the difference between Donald
Trump and Joe Biden, look no further than Joe Arpaio,'' Mr. Perez said.
``The Obama-Biden administration prosecuted Joe Arpaio. Donald Trump
pardoned Joe Arpaio.''

The president is flagging in Arizona polls largely because he has
alienated suburban Republicans and independent voters, the same ones who
drove a spike into the heart of Mr. Arpaio's political career.

Mr. Trump's slump in the polls is firmly
\href{https://www.nytimes.com/2020/07/18/us/politics/trump-coronavirus-response-failure-leadership.html}{tethered
to the coronavirus}, while the most potent attack on Mr. Arpaio among
suburban voters is the millions in taxpayer dollars spent on legal
settlements largely related to the harsh
\href{https://www.azcentral.com/story/news/local/phoenix/2018/11/28/county-supervisors-approve-300-k-settlement-arpaio-era-jail-death/2138553002/}{conditions
in his jails} and his
\href{https://www.azcentral.com/story/news/local/phoenix/2018/02/01/county-paid-1-million-settle-arpaio-era-immigration-lawsuit/1086824001/}{immigration-related}
policing.

But the broad strokes of their problems are the same.

``People are tired of the drama,'' Mr. Adams said. ``They're just
flat-out fatigued from the daily reality TV show. Those people are prime
Biden voters. And as they go, so goes Maricopa County and so goes the
state.''

Mr. Trump's pull in Arizona, a traditionally conservative state where
anti-immigrant sentiment is prevalent, remains strong. The state's
quickly changing demographics give Democrats hope they can flip the
state, but that will depend on strong Latino turnout as well as
independents and moderate Republicans being sufficiently motivated to
vote the president out. Those swing voters could come back into the fold
for Mr. Trump if the coronavirus becomes a slightly lesser focus in the
race.

Barrett Marson, a Republican political consultant, said that while **
Mr. Arpaio was unlikely to win his old job back, it was less clear if
the same independent voters who soured on him have hit that point with
the president.

``The fates of those two are linked --- not tied together --- but
linked,'' he said. ``If you're put off by one, you're certainly put off
by the other.''

A decade ago, brashly charismatic populist personalities like Mr. Arpaio
and former Gov. Jan Brewer, also a Republican, dominated Arizona's
political landscape, as did the issue of immigration. But Arizona today
is a more image-focused state, better exemplified by Mr. Ducey, a
Republican who has tried to repair the state's reputation as an
intolerant desert backwoods, as well as its relationship with Mexico.

But for many people around the world, Arizona is still synonymous with
``Sheriff Joe,'' the outrageous lawman who once boasted that he was
``\href{https://www.newsweek.com/arpaio-trump-kkk-pardon-657134}{honored}''
to be compared to the Ku Klux Klan.

It's not the image local political and business leaders want.

\href{https://www.theplazaco.com/wp-content/uploads/2018/08/Sharon-Harper-Bio.pdf}{Sharon
Harper}, a developer and member of
\href{https://www.gplinc.org/leadership/}{Greater Phoenix Leadership}, a
group of mostly Republican political influencers, was the chair of Mr.
Ducey's re-election campaign in 2018 but is crossing the aisle in the
Maricopa County sheriff's race to support the Democratic incumbent, Paul
Penzone, a centrist former Phoenix police sergeant with a deliberately
understated style.

Image

Sharon Harper, a developer and member of Greater Phoenix Leadership, a
group of mostly Republican political influencers, is crossing the aisle
to support Sheriff Paul Penzone, the Democratic
incumbent.~Credit...Adriana Zehbrauskas for The New York Times

``I don't really think of it so much as supporting a Democrat as much as
I think about wanting to stand against bigotry, racism and the
divisiveness that I attribute to Arpaio when he was in that office,''
she said.

Mr. Ducey, who in 2014 earned Mr. Arpaio's then-coveted endorsement, has
not endorsed Mr. Arpaio or Mr. Penzone.

On the campaign trail, Mr. Arpaio travels the county in his mobile
headquarters,
\href{https://www.fhtimes.com/news/local_news/arpaio-campaign-hits-the-road/article_4a6b2e6a-c07d-11ea-8db5-ab2ad6fe3a1f.html}{a
gaudy 33-foot motor home} covered with images of him and the president.
He presses palms like it's 2019. He dons a face mask, but only
occasionally.

When the motor home pulled into a Walmart parking lot one recent morning
in Scottsdale, a small crowd gathered to gawk at the spectacle. Mr.
Arpaio still has loyalists.

``You're a legend in New York,'' Todd Hall, 42, a former New Yorker,
said as he loaded groceries into his car. ``Hell yeah, I'll vote for
you.''

Though Arizona's demographics are trending more Democratic, Mr. Arpaio
has not changed.

He promises that if he's elected again, he'll employ the same tactics,
some of them illegal, that he used to. If he wins, one of his first
priorities would be to reopen his Tent City jail, an outdoor detention
facility in the scorching desert that was subject to several lawsuits.

``I wanted to come back, not because I lost, but because there were so
many things that disappeared after I left,'' he said in an interview.
``The tents went down, everything went down. If there's any time in
history for a sheriff like me to get back, it's now, with all the
chaos.''

Image

Mr. Arpaio with supporters in Scottsdale, Ariz. He still has loyalists
despite alienating some moderate Republicans and independent
voters.~Credit...Adriana Zehbrauskas for The New York Times

Mr. Penzone, who is unchallenged in the Democratic primary, spent his
first four years in office trying to unravel Mr. Arpaio's legacy, ending
the practice of forcing male inmates to wear pink underwear and
\href{https://www.nytimes.com/2017/04/04/us/arpaio-tent-city-maricopa-sheriff-penzone.html}{shuttering
Tent City} --- which now houses a
\href{https://kjzz.org/content/550808/former-tent-city-jail-facility-house-drug-treatment-program}{drug
recovery program}.

Regaining the trust that Mr. Arpaio eroded with the Latino community,
and repairing the department's image, is a continuing project. Mr.
Penzone has
\href{https://ktar.com/story/2339892/sheriff-paul-penzone-said-hes-fired-employees-for-racial-profiling/}{fired,
demoted and disciplined} deputies accused of biased policing.

But the same Latino community groups and activists who drove Mr.
Arpaio's defeat say that while Mr. Penzone has put a friendlier face on
the department, he hasn't done enough to address systemic racism.

Mr. Penzone ended Mr. Arpaio's practice of
\href{https://www.azcentral.com/story/news/local/phoenix/2017/02/17/maricopa-county-sheriffs-office-courtesy-holds-federal-immigration-agents-ice/98072980/}{holding
inmates} suspected of being in the country illegally for extended
periods on behalf of federal immigration authorities. But he has left
intact the sheriff's office's agreement to work with U.S. Immigration
and Customs Enforcement in county jails.

\href{https://www.phoenix.gov/district8}{Carlos Garcia}, a Phoenix City
Council member and former executive director of Puente Human Rights
Movement, said this policy put undocumented people in danger of
deportation for even minor infractions. Mr. Garcia worked to elect Mr.
Penzone, but said he now had his doubts.

``I think in the yearning to get rid of Arpaio, we failed to really
understand who Paul Penzone was,'' Mr. Garcia said. ``I don't know if
`disappointed' is enough. He's done absolutely nothing except close Tent
City, and he did that for other reasons, not for the reasons it should
have been shut down. Unfortunately, Arpaio's culture remains.''

Mr. Penzone acknowledged that his office had not rooted out all of its
problems, but said he was working to address concerns about mistreatment
of inmates and biased policing.

``What you're seeing is an organizational change that's not going to
happen overnight,'' he said in an interview. ``You can't have 3,500
employees and root out all the bad ones overnight when they had 24 years
of previous leadership. But to say that what went on under that guy is
going on under this guy is just dishonest.''

Mr. Penzone's
\href{https://recorder.maricopa.gov/campaignfinance/campfinDocsresults.aspx?candidateid=1000885}{campaign
finance reports} read like an invitation list to a Chamber of Commerce
gala.

His top donors include the Arizona Diamondbacks' owner, Earl Kendrick,
and his wife, Randy, as well as the Arizona Cardinals' owner, Michael
Bidwill. Barry Goldwater Jr., a son of the famous Arizona conservative
Barry Goldwater, and Andrew McCain, a son of former Senator John McCain,
are also donors.

It's a fact that Mr. Arpaio is keenly aware of as he ticks off a list of
Republican donors to his Democratic opponent's campaign, noting that the
common thread is that they are allies of Mr. Ducey and Mr. McCain.

``They don't care about public safety --- they only care about their
reputations and how much beer they can sell and how much business
they're losing. And they don't like the reputation I'm giving to this
area?'' he said. ``They fear me. They know I'm going to come back and do
the job.''

Advertisement

\protect\hyperlink{after-bottom}{Continue reading the main story}

\hypertarget{site-index}{%
\subsection{Site Index}\label{site-index}}

\hypertarget{site-information-navigation}{%
\subsection{Site Information
Navigation}\label{site-information-navigation}}

\begin{itemize}
\tightlist
\item
  \href{https://help.nytimes.com/hc/en-us/articles/115014792127-Copyright-notice}{©~2020~The
  New York Times Company}
\end{itemize}

\begin{itemize}
\tightlist
\item
  \href{https://www.nytco.com/}{NYTCo}
\item
  \href{https://help.nytimes.com/hc/en-us/articles/115015385887-Contact-Us}{Contact
  Us}
\item
  \href{https://www.nytco.com/careers/}{Work with us}
\item
  \href{https://nytmediakit.com/}{Advertise}
\item
  \href{http://www.tbrandstudio.com/}{T Brand Studio}
\item
  \href{https://www.nytimes.com/privacy/cookie-policy\#how-do-i-manage-trackers}{Your
  Ad Choices}
\item
  \href{https://www.nytimes.com/privacy}{Privacy}
\item
  \href{https://help.nytimes.com/hc/en-us/articles/115014893428-Terms-of-service}{Terms
  of Service}
\item
  \href{https://help.nytimes.com/hc/en-us/articles/115014893968-Terms-of-sale}{Terms
  of Sale}
\item
  \href{https://spiderbites.nytimes.com}{Site Map}
\item
  \href{https://help.nytimes.com/hc/en-us}{Help}
\item
  \href{https://www.nytimes.com/subscription?campaignId=37WXW}{Subscriptions}
\end{itemize}
