Sections

SEARCH

\protect\hyperlink{site-content}{Skip to
content}\protect\hyperlink{site-index}{Skip to site index}

\href{https://www.nytimes.com/section/us}{U.S.}

\href{https://myaccount.nytimes.com/auth/login?response_type=cookie\&client_id=vi}{}

\href{https://www.nytimes.com/section/todayspaper}{Today's Paper}

\href{/section/us}{U.S.}\textbar{}Isaias Live Updates: Storm Grazes
Florida and Takes Aim at the Carolinas

\url{https://nyti.ms/2DeHuld}

\begin{itemize}
\item
\item
\item
\item
\item
\end{itemize}

Advertisement

\protect\hyperlink{after-top}{Continue reading the main story}

Supported by

\protect\hyperlink{after-sponsor}{Continue reading the main story}

LIVE UPDATES

Updated~

Aug. 2, 2020, 11:05 p.m. ET

Aug. 2, 2020, 11:05 p.m. ET

\hypertarget{isaias-live-updates-storm-grazes-florida-and-takes-aim-at-the-carolinas}{%
\section{Isaias Live Updates: Storm Grazes Florida and Takes Aim at the
Carolinas}\label{isaias-live-updates-storm-grazes-florida-and-takes-aim-at-the-carolinas}}

Storm watches have been extended as far north as New England. The storm
is on a track to reach land on Monday, with heavy rain and the potential
for damaging floods.

Right Now

Tropical Storm Isaias is 50 miles off the Florida coast near Cape
Canaveral.

\hypertarget{heres-what-you-need-to-know}{%
\subsubsection{Here's what you need to
know:}\label{heres-what-you-need-to-know}}

\begin{itemize}
\tightlist
\item
  \protect\hyperlink{link-6002fe80}{Tropical Storm Isaias scrapes the
  Florida coast.}
\item
  \protect\hyperlink{link-301f2153}{Astronauts splash down in the Gulf,
  avoiding the storm.}
\item
  \protect\hyperlink{link-17739c9a}{In South Carolina, ``it's a wait and
  see game.''}
\item
  \protect\hyperlink{link-e77dd06}{Is a face mask much use in a tropical
  storm? Not if it gets wet.}
\item
  \protect\hyperlink{link-6f11f4a6}{The storm is expected to soak the
  East Coast with potentially flooding rain.}
\item
  \protect\hyperlink{link-3c28a89f}{Florida is going through a summer of
  dread.}
\item
  \protect\hyperlink{link-2d06f1e8}{The battered Bahamas begins to take
  stock after a fresh pummeling.}
\end{itemize}

\includegraphics{https://static01.nyt.com/images/2020/08/02/us/02isaias-briefing-lead2/02isaias-briefing-lead2-videoSixteenByNine3000.jpg}

\subsection{}

Tropical Storm Isaias scrapes the Florida coast.

Floridians along the state's Atlantic coast hunkered down on Sunday as
Tropical Storm Isaias plowed northward just offshore, whipping the state
with high winds, rain and the threat of flash floods as it went.

At 11 p.m. Eastern time, the center of the storm was about 50 miles off
the Central Florida coast, near Cape Canaveral, and was moving
north-northwest at about nine miles an hour, according to the
\href{https://www.nhc.noaa.gov/text/refresh/MIATCPAT4+shtml/020856.shtml?}{National
Hurricane Center}. It had strengthened slightly from earlier in the day,
with sustained winds of 70 m.p.h., only 4 m.p.h. below hurricane
strength.

Isaias ---~(which is written Isaías in Spanish and pronounced
ees-ah-EE-ahs) ---~clobbered the Bahamas with hurricane conditions on
Saturday and early Sunday after hitting parts of Puerto Rico and the
Dominican Republic. It weakened to a tropical storm Saturday evening.

\href{https://www.nytimes.com/interactive/2020/07/31/us/hurricane-isaias-tracker-map.html}{}

\includegraphics{https://static01.nyt.com/images/2020/07/31/us/hurricane-isaias-tracker-map-promo-1596209917104/hurricane-isaias-tracker-map-promo-1596209917104-articleLarge-v6.jpg}

\hypertarget{hurricane-isaias-tracking-map}{%
\subsection{Hurricane Isaias Tracking
Map}\label{hurricane-isaias-tracking-map}}

Follow the storm's path as it approaches the Florida coast.

Flooding from the storm's heavy rains led to the death of at least one
person in Puerto Rico, the island's Department of Public Safety said on
Saturday in a statement. A woman who had been missing since Thursday
drowned near Rincón, in the northwest portion of the island.

The center of the storm skirted the coast of Florida on Sunday without
making landfall, and the southern part of the coast was left largely
unscathed, aside from scattered power outages. Only about 200 people in
Palm Beach County stayed at public shelters, out of a population of
\href{https://slack-redir.net/link?url=https\%3A\%2F\%2Fdiscover.pbcgov.org\%2Fpages\%2Fpbc_facts.aspx}{almost
1.5 million}, according to Bill Johnson, the county director of
emergency management.

``We are blessed that Hurricane Isaías spared us of significant
damage,'' Mr. Johnson said at a news conference Sunday. ``I am pleased
that this was more of an exercise than a real event --- something we
should all be grateful of.''

Forecasters said the storm would track northward and could fluctuate a
bit in strength before coming ashore in the Carolinas on Monday.
Hurricane watches were posted from South Santee River, S.C., north to
Surf City, N.C., and tropical storm watches are posted all the way to
Rhode Island. Forecasters said the storm had the potential to spawn
tornadoes in the Carolinas on Monday.

Complicating the emergency response to the storm, reported coronavirus
cases continue to rise sharply in Florida, Georgia and the Carolinas,
and health officials have warned that their health care systems could be
strained beyond capacity with the influx of new patients. The situation
would worsen if the storm knocks out power across wide areas or forces
evacuations of hospitals and nursing homes.

\hypertarget{-1}{%
\subsection{}\label{-1}}

Astronauts splash down in the Gulf, avoiding the storm.

\includegraphics{https://static01.nyt.com/images/2020/08/02/video/02vid-spacex-splash/02vid-spacex-splash-videoSixteenByNineJumbo1600.jpg}

Two long-haul travelers taking a red-eye flight from the International
Space Station arrived in the Gulf of Mexico off the Florida panhandle on
Sunday, working their way around Isaias' rough weather.

Robert L. Behnken and Douglas G. Hurley, the astronauts who blasted off
to the space station in May in the Crew Dragon capsule built and
operated by SpaceX, the rocket company started by Elon Musk, pushed off
from the orbiting outpost on Saturday night.

Suspended under four giant billowing orange-and-white parachutes, the
Crew Dragon settled into calm waters near Pensacola, Fla. at a gentle
pace of 15 miles per hour on Sunday afternoon. Two small SpaceX boats
arrived quickly to begin the operation to prepare the capsule to be
pulled out by the main recovery ship, where crews will tend to the
spacecraft's passengers.

It was the first water landing by NASA since 1975, when the agency's
crews were still flying in the Apollo modules used for the historic
American moon missions.

NASA and SpaceX selected seven potential sites in the Atlantic Ocean and
Gulf of Mexico where the capsule and its passengers could splash down.
But the track of Isaias ruled out the three in the Atlantic, so they
chose the site near Pensacola.

At the splashdown site, winds must be less than 10 miles an hour, and
there are additional constraints on waves and rain. In addition,
helicopters that take part in the recovery of the capsule must be able
to fly and land safely.

\hypertarget{-2}{%
\subsection{}\label{-2}}

In South Carolina, ``it's a wait and see game.''

\includegraphics{https://static01.nyt.com/images/2020/08/02/us/02isaias-briefing-SC/merlin_174752202_1982508d-a7bd-42ff-b9bd-3dc5246eeef9-articleLarge.jpg?quality=75\&auto=webp\&disable=upscale}

South Carolina's popular tourist destinations had their fingers crossed
on Sunday, hoping the storm would continue to be less fearsome than
forecast, as it has so far in Florida.

The \href{https://www.facebook.com/NWSCharlestonSC}{National Weather
Service is predicting} storm surges of 2 to 4 feet, some flooding and
perhaps a tornado in the Charleston area Sunday night and Monday as the
storm moves north offshore, toward an expected landfall farther up the
coast. City officials have opened four elevated garages where residents
can\href{https://charleston-sc.maps.arcgis.com/apps/MapSeries/index.html?appid=61d5ee562990480f922de9695872cd19}{park
free} in case of flooding.

Some business owners are worried that a one-two punch of the virus and a
powerful storm could push their establishments over the edge.
The\href{https://www.twomeetingstreetinn.com/}{Two Meeting Street Inn},
a waterfront bed and breakfast in Charleston, was shut down in March
over virus concerns and hoped to reopen Aug. 15, but may now have to
wait until September.

``It's been devastating for us,'' said Julie Spell Roberts, whose family
has owned the inn since 1946. ``Our biggest season is March, April and
May. Pretty much that's when the money is made to keep you afloat for
the rest of the year.''

Extreme weather is a threat the family has experience with. Ms. Spell
Roberts' mother remained at the inn during Hurricane Hugo, a
\href{https://www.weather.gov/chs/Hugo-30thAnniversary}{Category 4}
storm that ravaged the city in 1989.

To prepare, the family has taken the furniture off the porch and cleared
the property of anything that might smash a wall or window. ``What we
have learned over time is that you're foolish if you don't think that
Mother Nature is a formidable foe, because she is,'' Ms. Spell Roberts
said.

Farther north, Myrtle Beach is preparing for a ``lower to moderate
threat'' to arrive Monday night, said Steve Pfaff, a National Weather
Service meteorologist, with sustained winds around 50 to 60 miles an
hour and gusts up to 70. Four to six inches of rain could lead to flash
flooding in places with drainage problems, and some areas may get more.

For businesses away from the shoreline, ``it's a wait and see game,''
said Jay Slevin, general manager of Mellow Mushroom, a pizzeria a mile
and a half from the ocean in Myrtle Beach. ``We're not boarding anything
up,'' he said.

\hypertarget{-3}{%
\subsection{}\label{-3}}

Is a face mask much use in a tropical storm? Not if it gets wet.

Image

People walking to Juno Beach, Fla., on Sunday wore face masks despite
the rainy conditions.~Credit...Saul Martinez for The New York Times

In recent weeks as the coronavirus has been resurgent in many parts of
the country, experts and politicians alike have implored people to
protect themselves and others by always wearing a face mask in public.

Does that apply when you have to be out in the gusting wind and driving
rain? Our health columnist Tara Parker-Pope says probably not: Face
masks
\href{https://www.nursingtimes.net/clinical-archive/infection-control/the-effectiveness-of-surgical-face-masks-what-the-literature-shows-30-09-2003/}{aren't
as effective} when they are wet.

For one thing, it's much harder to breathe through a wet mask than a dry
one, Ms. Parker-Pope notes. And on top of that, a moist or wet mask
doesn't filter as well as a dry mask. The Centers for Disease Control
and Prevention, which recommends mask-wearing in general, says they
\href{https://www.cdc.gov/coronavirus/2019-ncov/prevent-getting-sick/cloth-face-cover-guidance.html}{should
not be worn when doing things that may get the mask wet.}

It doesn't take a tropical storm to drench a mask, of course. They can
become soaked with condensation from your breath or sweat from your
face, and some people think of wetting them deliberately to cool off in
hot weather. But the harm done is the same, wherever the moisture comes
from.

A paper surgical mask that gets soaked should probably be discarded, Ms.
Parker-Pope advises, but a cloth mask can be washed, dried and re-used.

If rain is coming down in buckets, social distancing is not likely to be
a problem, and any viral particles exhaled by an infected person
probably would be quickly diluted by gusting wind and rain. So there is
little need to wear a mask out in a rainstorm, Ms. Parker-Pope notes:
``In fact, you should take it off and keep it dry, so if you need to
duck into a store to wait out the storm, you have a dry mask to wear
indoors.''

\hypertarget{-4}{%
\subsection{}\label{-4}}

The storm is expected to soak the East Coast with potentially flooding
rain.

Image

Patrons of the Bogue Inlet Pier in Emerald Isle, N.C., walk past a red
flag signaling hazardous swimming conditions on Sunday. Heavy rains from
Tropical Storm Isaias are expected in the state early in the
week.Credit...Julia Wall/The News \& Observer, via Associated Press

While Isaias' gusting winds remain capable of significant damage, its
heavy rains may be the biggest punch the storm packs. Much of the East
Coast of the United States will get a soaking, forecasters say.

With 3 to 6 inches expected across the eastern Carolinas and Virginia
and isolated areas getting up to 8 inches, significant flash floods and
urban flooding is can be expected through the middle of the week, and
widespread minor to moderate river flooding is possible in the region.

Southeastern New York and New England can expect almost as much rain
when the storm reaches there Tuesday and Wednesday, while northeast
Florida and coastal Georgia are likely to get 1 to 3 inches before
Isaias moves away north, forecasters said.

Residents of inland North Carolina communities that have been swamped
before by river flooding are eyeing the storm's approach cautiously, and
\href{https://alerts.weather.gov/cap/wwacapget.php?x=NC125F5DA2F1C4.HurricaneLocalStatement.125F5DA42AA8NC.ILMHLSILM.3d2f501b896f05f00166a758bf00288a}{flash
flood warnings are already in effect} in some areas. Gov. Roy Cooper
declared a state of emergency on Friday.

``With the right protection and sheltering, we can keep people safe from
the storm while at the same time trying to avoid making the pandemic
worse,'' Mr. Cooper said on Twitter. ``A hurricane during a pandemic is
double trouble. But the state has been carefully preparing for this
scenario.''

A number of inland counties in North Carolina have been hit twice in
recent years by river flooding ---
\href{https://www.nytimes.com/2018/09/18/us/north-carolina-hurricanes-storms-history.html}{by
Hurricane Matthew in 2016 and Hurricane Florence in 2018}.

Justin Harrington, a manager of a Big Blue hardware store in inland
Kinston, N.C., said customers had started stocking up on generators,
flashlights and gas cans. So far, the store is planning to stay open,
``as long as it's not as bad as past storms.''

\hypertarget{-5}{%
\subsection{}\label{-5}}

Florida is going through a summer of dread.

Image

Utility crews worked on power lines in Palm Beach, Fla., on Sunday that
were damaged by Tropical Storm Isaias.Credit...Saul Martinez for The New
York Times

Summer in Florida, with its routine thunderstorms, sweaty nights and
unforgiving mosquitoes, is not for the faint of heart. But the summer of
2020 is proving especially despairing,
\href{https://www.nytimes.com/2020/08/02/us/florida-hurricane-isaias-coronavirus.html}{Patricia
Mazzei, our Miami bureau chief, writes}.

A public health crisis. An economic calamity, with more than a million
\href{https://www.nytimes.com/2020/04/23/us/florida-coronavirus-unemployment.html}{Floridians
out of work} and an unemployment payment system that was one of the
slowest in the country. And now an early debut of hurricane season, to
remind Florida that the inevitable convergence of the pandemic and the
weather is likely to play out again, and perhaps much more seriously
than this relatively mild storm, before this nightmare season ends.

It should be too early to worry much about storms in Florida, Ms. Mazzei
writes, but this annus horribilis would not have it any other way.

``It's just kind of been the way 2020's gone so far,'' said Howard
Tipton, the administrator for St. Lucie County, on Florida's Treasure
Coast. ``But we roll with it, right? We don't get to determine the cards
that we're dealt.''

Kevin Cho, 31, a Florida National Guard captain and a nurse practitioner
who treats Covid-19 patients in the intensive care units of several
Miami public hospitals, said the approach of the storm was ``really
stretching our limits.''

Many poor people contracting the disease ``are losing their jobs, and
now they're faced with a hurricane,'' he added. ``How could they prepare
for a hurricane when they have been exhausted of every resource they
have? This hurricane is only going to make things worse.''

\hypertarget{-6}{%
\subsection{}\label{-6}}

The battered Bahamas begins to take stock after a fresh pummeling.

Image

Residents covered a window with plywood in preparation for the arrival
of Hurricane Isaias in the Bahamas on Friday. The storm's heavy rains
may be the biggest punch the storm packs.Credit...Tim Aylen/Associated
Press

Isaias blew away from the Bahamas on Sunday after pummeling the
low-lying islands with heavy rain and high winds for most of the
weekend. The storm left parts of Grand Bahama drenched with more than a
foot of rain, and other islands in the archipelago were suffering from
minor flooding and downed trees and power lines.

No storm-related deaths have been reported in the country, where
memories are still raw from Hurricane Dorian, which ravaged Abaco and
Grand Bahama last year and left at least 74 people dead. Many residents
of the islands are still living in tents or in unrepaired houses that
were damaged during Dorian. According to a report in May by the
International Organization for Migration, the islands also lack an
adequate supply of hurricane shelters.

The coronavirus pandemic has made rebuilding more difficult, and
weakened the country's tourism-dependent economy, leaving the Bahamas
particularly vulnerable this hurricane season.

The country locked down in March after recording its first few
coronavirus infections, and it seemed to have the outbreak under
control, with just 104 cases by July 1, when it reopened for
international travel. Soon after that, though, cases began to surge,
with Grand Bahama emerging as a hot spot. The island was in the midst of
a new two-week lockdown when tropical storm warnings were posted on
Thursday.

The prime minister, Hubert Minnis, temporarily relaxed the restrictions
to allow residents to prepare, but fear of the virus remained a
significant obstacle. Some people were hesitant to venture out for
supplies, and many were afraid to seek refuge in shelters where there
was no clear pan for social distancing.

\hypertarget{-7}{%
\subsection{}\label{-7}}

Another storm may lurk behind Isaias, but on a less dangerous track.

Forecasters at the National Hurricane Center are tracking another
potential tropical cyclone that could develop in the western Atlantic in
the next five days.

But so far, they say, the odds of it becoming even a tropical
depression, with sustained winds up to 38 miles an hour, are only 60
percent. And its path would probably be well west and north of the major
islands of the Caribbean and the Bahamas, far from the continental
shore.

It would have to strengthen further still, to sustained winds of 39
m.p.h. or greater, to qualify as a named storm. Right now, it's just
``Disturbance 1'' ---~the only other system on the center's Atlantic map
besides Isaias.

Reporting was contributed by Nicholas Bogel-Burroughs, Kenneth Chang,
Melina Delkic, Rebecca Halleck, Patrick J. Lyons, Patricia Mazzei,
Giulia McDonnell Nieto del Rio, Christina Morales, Aimee Ortiz, Tara
Parker-Pope, Michael Roston, Rachel Knowles Scott, Lucy Tompkins and
Will Wright.

Advertisement

\protect\hyperlink{after-bottom}{Continue reading the main story}

\hypertarget{site-index}{%
\subsection{Site Index}\label{site-index}}

\hypertarget{site-information-navigation}{%
\subsection{Site Information
Navigation}\label{site-information-navigation}}

\begin{itemize}
\tightlist
\item
  \href{https://help.nytimes.com/hc/en-us/articles/115014792127-Copyright-notice}{©~2020~The
  New York Times Company}
\end{itemize}

\begin{itemize}
\tightlist
\item
  \href{https://www.nytco.com/}{NYTCo}
\item
  \href{https://help.nytimes.com/hc/en-us/articles/115015385887-Contact-Us}{Contact
  Us}
\item
  \href{https://www.nytco.com/careers/}{Work with us}
\item
  \href{https://nytmediakit.com/}{Advertise}
\item
  \href{http://www.tbrandstudio.com/}{T Brand Studio}
\item
  \href{https://www.nytimes.com/privacy/cookie-policy\#how-do-i-manage-trackers}{Your
  Ad Choices}
\item
  \href{https://www.nytimes.com/privacy}{Privacy}
\item
  \href{https://help.nytimes.com/hc/en-us/articles/115014893428-Terms-of-service}{Terms
  of Service}
\item
  \href{https://help.nytimes.com/hc/en-us/articles/115014893968-Terms-of-sale}{Terms
  of Sale}
\item
  \href{https://spiderbites.nytimes.com}{Site Map}
\item
  \href{https://help.nytimes.com/hc/en-us}{Help}
\item
  \href{https://www.nytimes.com/subscription?campaignId=37WXW}{Subscriptions}
\end{itemize}
