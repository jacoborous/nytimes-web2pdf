Sections

SEARCH

\protect\hyperlink{site-content}{Skip to
content}\protect\hyperlink{site-index}{Skip to site index}

\href{https://myaccount.nytimes.com/auth/login?response_type=cookie\&client_id=vi}{}

\href{https://www.nytimes.com/section/todayspaper}{Today's Paper}

How to Keep a Condor Wild

\url{https://nyti.ms/3k3kLZQ}

\begin{itemize}
\item
\item
\item
\item
\item
\end{itemize}

Advertisement

\protect\hyperlink{after-top}{Continue reading the main story}

Supported by

\protect\hyperlink{after-sponsor}{Continue reading the main story}

\href{/column/magazine-tip}{Tip}

\hypertarget{how-to-keep-a-condor-wild}{%
\section{How to Keep a Condor Wild}\label{how-to-keep-a-condor-wild}}

\includegraphics{https://static01.nyt.com/images/2020/08/09/magazine/09Mag-Tip-01/09Mag-Tip-01-articleLarge.jpg?quality=75\&auto=webp\&disable=upscale}

By Malia Wollan

\begin{itemize}
\item
  Aug. 4, 2020
\item
  \begin{itemize}
  \item
  \item
  \item
  \item
  \item
  \end{itemize}
\end{itemize}

``Don't treat condors as pets,'' says Tiana Williams-Claussen, director
of the Yurok Tribe Wildlife Department. Scavengers with a
nine-and-a-half-foot wingspan, condors are sacred in Yurok cosmology,
but it has been more than 100 years since they soared over the tribe's
ancestral territory, in northwestern California. Next spring,
Williams-Claussen and her team, along with state and federal government
partners, hope to release six juvenile birds born in captivity.
Creatures not raised in the wild often need time to learn and practice
how to exist without human help. ``Let them engage in natural behaviors
without being influenced by humans too close by,'' Williams-Claussen
says.

By the early 1980s, there were just 22 California condors left in the
wild. Desperate to save the birds from extinction, biologists captured
the last of them in 1987 to breed in zoos. The birds to be released will
join 337 others now flying free. They need to be monitored, periodically
captured and regularly fed by humans. Only trained condor biologists
should approach the birds or their nests. Even the experts should
proceed with caution. ``Keep out of sight as much as possible,'' says
Williams-Claussen, especially when putting out food; a condor that
associates food with humans might start following people around.

Trap the birds twice yearly for their health checks by using carrion to
lure them into an enclosure, then net them before sneaking up on them
and grabbing the back of their head. ``They've got a wicked beak,'' says
Williams-Claussen, who has been working to return condors to Yurok
territory for more than a decade. Proximity to such a strange big bird
will make your heart race. You might find yourself nervously doing baby
talk. You shouldn't. ``Don't coo at them,'' Williams-Claussen says.
``Don't pet them.''

In the nearly 40 years of breeding condors and reintroducing them to the
wild, birds have occasionally become habituated to humans in odd ways,
including lurking around campsites and lunging at hikers to steal their
shoelaces. Williams-Claussen feels confident that along her tribe's
remote section of coast, condors will have the space to find their niche
in the wilderness. ``When you actually see these birds soaring
overhead,'' she says, ``there is absolutely nothing like it.''

Advertisement

\protect\hyperlink{after-bottom}{Continue reading the main story}

\hypertarget{site-index}{%
\subsection{Site Index}\label{site-index}}

\hypertarget{site-information-navigation}{%
\subsection{Site Information
Navigation}\label{site-information-navigation}}

\begin{itemize}
\tightlist
\item
  \href{https://help.nytimes.com/hc/en-us/articles/115014792127-Copyright-notice}{©~2020~The
  New York Times Company}
\end{itemize}

\begin{itemize}
\tightlist
\item
  \href{https://www.nytco.com/}{NYTCo}
\item
  \href{https://help.nytimes.com/hc/en-us/articles/115015385887-Contact-Us}{Contact
  Us}
\item
  \href{https://www.nytco.com/careers/}{Work with us}
\item
  \href{https://nytmediakit.com/}{Advertise}
\item
  \href{http://www.tbrandstudio.com/}{T Brand Studio}
\item
  \href{https://www.nytimes.com/privacy/cookie-policy\#how-do-i-manage-trackers}{Your
  Ad Choices}
\item
  \href{https://www.nytimes.com/privacy}{Privacy}
\item
  \href{https://help.nytimes.com/hc/en-us/articles/115014893428-Terms-of-service}{Terms
  of Service}
\item
  \href{https://help.nytimes.com/hc/en-us/articles/115014893968-Terms-of-sale}{Terms
  of Sale}
\item
  \href{https://spiderbites.nytimes.com}{Site Map}
\item
  \href{https://help.nytimes.com/hc/en-us}{Help}
\item
  \href{https://www.nytimes.com/subscription?campaignId=37WXW}{Subscriptions}
\end{itemize}
