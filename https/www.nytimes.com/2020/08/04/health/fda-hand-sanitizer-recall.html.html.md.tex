Sections

SEARCH

\protect\hyperlink{site-content}{Skip to
content}\protect\hyperlink{site-index}{Skip to site index}

\href{https://www.nytimes.com/section/health}{Health}

\href{https://myaccount.nytimes.com/auth/login?response_type=cookie\&client_id=vi}{}

\href{https://www.nytimes.com/section/todayspaper}{Today's Paper}

\href{/section/health}{Health}\textbar{}F.D.A. Adds More Hand Sanitizers
to List of Products to Avoid

\url{https://nyti.ms/33sD84K}

\begin{itemize}
\item
\item
\item
\item
\item
\end{itemize}

\href{https://www.nytimes.com/news-event/coronavirus?action=click\&pgtype=Article\&state=default\&region=TOP_BANNER\&context=storylines_menu}{The
Coronavirus Outbreak}

\begin{itemize}
\tightlist
\item
  live\href{https://www.nytimes.com/2020/08/04/world/coronavirus-cases.html?action=click\&pgtype=Article\&state=default\&region=TOP_BANNER\&context=storylines_menu}{Latest
  Updates}
\item
  \href{https://www.nytimes.com/interactive/2020/us/coronavirus-us-cases.html?action=click\&pgtype=Article\&state=default\&region=TOP_BANNER\&context=storylines_menu}{Maps
  and Cases}
\item
  \href{https://www.nytimes.com/interactive/2020/science/coronavirus-vaccine-tracker.html?action=click\&pgtype=Article\&state=default\&region=TOP_BANNER\&context=storylines_menu}{Vaccine
  Tracker}
\item
  \href{https://www.nytimes.com/2020/08/02/us/covid-college-reopening.html?action=click\&pgtype=Article\&state=default\&region=TOP_BANNER\&context=storylines_menu}{College
  Reopening}
\item
  \href{https://www.nytimes.com/live/2020/08/04/business/stock-market-today-coronavirus?action=click\&pgtype=Article\&state=default\&region=TOP_BANNER\&context=storylines_menu}{Economy}
\end{itemize}

Advertisement

\protect\hyperlink{after-top}{Continue reading the main story}

Supported by

\protect\hyperlink{after-sponsor}{Continue reading the main story}

\hypertarget{fda-adds-more-hand-sanitizers-to-list-of-products-to-avoid}{%
\section{F.D.A. Adds More Hand Sanitizers to List of Products to
Avoid}\label{fda-adds-more-hand-sanitizers-to-list-of-products-to-avoid}}

The agency has moved to keep several of the products, which it found to
have inadequate concentrations of alcohol, from entering the United
States.

By \href{https://www.nytimes.com/by/derrick-bryson-taylor}{Derrick
Bryson Taylor}

\begin{itemize}
\item
  Aug. 4, 2020
\item
  \begin{itemize}
  \item
  \item
  \item
  \item
  \item
  \end{itemize}
\end{itemize}

\includegraphics{https://static01.nyt.com/images/2020/08/04/us/4xp-sanitizer/4xp-sanitizer-articleLarge-v2.jpg?quality=75\&auto=webp\&disable=upscale}

The Food and Drug Administration has expanded its list of hand
sanitizers that consumers should avoid to include products with
inadequate levels of alcohol in addition to those containing methanol.

The agency
\href{https://www.fda.gov/drugs/drug-safety-and-availability/fda-updates-hand-sanitizers-consumers-should-not-use\#products}{issued
an advisory} last week announcing that its tests had found four hand
sanitizers with ``concerningly low levels of ethyl alcohol or isopropyl
alcohol'' --- active ingredients in hand sanitizers.

The Centers for Disease Control and Prevention recommend that consumers
use alcohol-based hand sanitizers with at least 60 percent ethanol, if
soap and water are not available.

The four hand sanitizers the F.D.A. found to have inadequate
concentrations of ethanol are NeoNatural, Medicare Alcohol Antiseptic
Topical Solution, Datsen Hand Sanitizer and Alcohol Antiseptic 62
Percent Hand Sanitizer.

Three of the four products, all of which are manufactured in Mexico,
were added to an import alert to stop them from entering the United
States.

The F.D.A. also flagged several more products that had inadequate
amounts of benzalkonium chloride, a chemical with
\href{https://www.ncbi.nlm.nih.gov/pmc/articles/PMC6581159/}{antimicrobial
properties}.

By Tuesday, the F.D.A.'s list of hand sanitizers that consumers should
avoid had grown to 115.

In June, the agency warned consumers to
\href{https://www.nytimes.com/2020/06/22/health/fda-Eskbiochem-toxic-hand-sanitizer-virus.html}{avoid
nine hand sanitizer products} that were manufactured in Mexico because
they contained methanol, a substance that can be toxic if absorbed
through the skin or ingested. Substantial methanol exposure can lead to
nausea, vomiting, headaches, permanent blindness and seizures, among
other harmful effects.

During the coronavirus pandemic,
\href{https://www.nytimes.com/2020/03/11/smarter-living/wirecutter/coronavirus-hand-sanitizer.html}{sales
of hand sanitizers have soared} as consumers tried to observe health
officials' recommendations to frequently and thoroughly wash or sanitize
their hands to keep from contracting the virus.

According to Dr. Matthew G. Heinz, a hospital physician in Tucson,
Ariz., 60 percent alcohol is the minimum concentration for a hand
sanitizer to be effective. Lower concentrations mean diminished
disinfectant properties, he said.

``Depending on the exact concentration, it may almost have the same
effect as putting water on,'' Dr. Heinz said on Tuesday. ``If we're
talking something in the 15, 20, 25 percent range, you may not be able
to really kill anything.''

Asked how much sanitizer a person should use, Dr. Heinz recommend an
amount that would cover the hands entirely, not just the palms. But
using hand sanitizer repeatedly throughout the day, he said, is no
substitute for using warm water and soap for cleaning.

``After multiple uses, you can start diminishing the effectiveness of
the hand sanitizer,'' Dr. Heinz said. ``You really do need to actually
wash with soap and water for 20 seconds or more to kind of renew things.
You really can't just apply hand sanitizer 40 times throughout the day
and think that you're good.''

Advertisement

\protect\hyperlink{after-bottom}{Continue reading the main story}

\hypertarget{site-index}{%
\subsection{Site Index}\label{site-index}}

\hypertarget{site-information-navigation}{%
\subsection{Site Information
Navigation}\label{site-information-navigation}}

\begin{itemize}
\tightlist
\item
  \href{https://help.nytimes.com/hc/en-us/articles/115014792127-Copyright-notice}{©~2020~The
  New York Times Company}
\end{itemize}

\begin{itemize}
\tightlist
\item
  \href{https://www.nytco.com/}{NYTCo}
\item
  \href{https://help.nytimes.com/hc/en-us/articles/115015385887-Contact-Us}{Contact
  Us}
\item
  \href{https://www.nytco.com/careers/}{Work with us}
\item
  \href{https://nytmediakit.com/}{Advertise}
\item
  \href{http://www.tbrandstudio.com/}{T Brand Studio}
\item
  \href{https://www.nytimes.com/privacy/cookie-policy\#how-do-i-manage-trackers}{Your
  Ad Choices}
\item
  \href{https://www.nytimes.com/privacy}{Privacy}
\item
  \href{https://help.nytimes.com/hc/en-us/articles/115014893428-Terms-of-service}{Terms
  of Service}
\item
  \href{https://help.nytimes.com/hc/en-us/articles/115014893968-Terms-of-sale}{Terms
  of Sale}
\item
  \href{https://spiderbites.nytimes.com}{Site Map}
\item
  \href{https://help.nytimes.com/hc/en-us}{Help}
\item
  \href{https://www.nytimes.com/subscription?campaignId=37WXW}{Subscriptions}
\end{itemize}
