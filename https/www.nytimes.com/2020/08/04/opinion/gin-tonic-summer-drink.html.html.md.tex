Sections

SEARCH

\protect\hyperlink{site-content}{Skip to
content}\protect\hyperlink{site-index}{Skip to site index}

\href{https://myaccount.nytimes.com/auth/login?response_type=cookie\&client_id=vi}{}

\href{https://www.nytimes.com/section/todayspaper}{Today's Paper}

\href{/section/opinion}{Opinion}\textbar{}The Doggiest Days of Summer
Demand Gin and Tonics

\url{https://nyti.ms/30szAgV}

\begin{itemize}
\item
\item
\item
\item
\item
\item
\end{itemize}

Advertisement

\protect\hyperlink{after-top}{Continue reading the main story}

\href{/section/opinion}{Opinion}

Supported by

\protect\hyperlink{after-sponsor}{Continue reading the main story}

\hypertarget{the-doggiest-days-of-summer-demand-gin-and-tonics}{%
\section{The Doggiest Days of Summer Demand Gin and
Tonics}\label{the-doggiest-days-of-summer-demand-gin-and-tonics}}

They're refreshing and ridiculously easy to make, and come with just
enough bitterness to help you get through this miserable year.

\href{https://topics.nytimes.com/top/reference/timestopics/people/b/jennifer_finney_boylan/index.html}{\includegraphics{https://static01.nyt.com/images/2011/08/04/opinion/BOYLAN_NEW/BOYLAN_NEW-thumbLarge-v6.png}}

By
\href{https://topics.nytimes.com/top/reference/timestopics/people/b/jennifer_finney_boylan/index.html}{Jennifer
Finney Boylan}

Contributing Opinion Writer

\begin{itemize}
\item
  Aug. 4, 2020
\item
  \begin{itemize}
  \item
  \item
  \item
  \item
  \item
  \item
  \end{itemize}
\end{itemize}

\includegraphics{https://static01.nyt.com/images/2020/08/05/opinion/05boylan/05boylan-articleLarge.jpg?quality=75\&auto=webp\&disable=upscale}

They're called the
\href{https://www.almanac.com/content/what-are-dog-days-summer}{dog
days}, this stretch of July and August, because now is when Sirius, the
Dog Star, returns to the night sky. The dies caniculares ** are said to
bring about bad luck, lethargy and thunderstorms. Not to mention, you
know: cataclysmic national dysfunction.

Personally, I've always thought of this time as the dog days because
this is when my dog Chloe becomes an inert mass, a catatonic thing most
likely to be found passed out beneath the piano.

I try to make these days bearable for Chloe by putting ice cubes in her
water bowl.

As for me, the best antidote is the cocktail that epitomizes summer
itself: the gin and tonic.

When I spent my summers in Philadelphia, working as a bank teller, it
wasn't a soft pretzel or
\href{https://www.eater.com/2015/9/1/9211867/scrapple-goetta-livermush-what-is-it}{scrapple}or
a \href{https://mashable.com/2016/03/09/the-cult-of-wawa/}{Wawa} hoagie
that I dreamed about as I made my way out of the dense, humid city at
the end of the day. It was a G\&T made with Tanqueray, fizzed up with
Schweppes, garnished with a hearty green wedge of lime and served in a
tall glass filled with so much ice that it crackled as you poured in the
gin.

I can see it now, the beads of perspiration shimmering on the glass in
the 95 degree heat, the bubbles of the tonic roiling around the cubes
that had already begun to melt in the swampy twilight. From a neighbor's
yard, the sounds of boys playing baseball: the crack of a bat, the
leathery sock of a ball landing in a glove. From the stereo, Miles Davis
playing ``Summertime.''

Has there ever been a cocktail that so evokes the heat of summer as the
gin and tonic?

The
\href{https://www.kew.org/read-and-watch/just-the-tonic-history}{history}of
the G\&T is the opposite of refreshing. It was born out of colonialism,
from British outposts in various tropical locales, back when malaria was
widespread, and the antidote was quinine, dissolved in fizzy water. (The
choice to add gin came later.)

But ugly history aside, there's no denying that it is ridiculously easy
to make. Two ounces gin, four to six ounces tonic (to taste), a lime,
and you're done. It is easier to make one than it is to scramble an egg.

Or, if you wanted, you could make one that's very complex indeed. When
my wife and I went out for our 30th anniversary dinner several years
ago, the waiter put a whole menu in our hands that was just gin and
tonics. There were a dozen different gins from around the world; fancy
tonics that contained things like elderflower; and garnishes of lemons,
or grapefruit, or flower petals.

The
\href{https://www.barschool.net/us/blog/why-gin-so-popular-five-possible-theories}{gin
craze} of the last decade has brought us spirits like
\href{https://theginisin.com/gin-reviews/magellan-gin/}{Magellan}, a gin
the color of a robin's egg, the result of crushed iris blossoms. Or
\href{https://theginisin.com/gin-reviews/hendricks/}{Hendrick's}, from
Scotland, with its taste of cool cucumbers. Or
\href{https://www.ginfoundry.com/gin/deaths-door-gin/}{Death's Door},
from Wisconsin, containing faint notes of fennel. And then there's my
personal favorite,
\href{https://www.masterofmalt.com/gin/the-dingle-distillery/dingle-original-gin/}{Dingle},
made near my cousin's house near Ballyferriter, Ireland, with its
juniper, rowan, fuchsia, bog myrtle, heather, chervil, hawthorn,
angelica, and coriander.

Meaghan Dorman is bar director at
\href{http://www.raineslawroom.com/}{Raines Law Room} and
\href{https://www.dearirving.com/}{Dear Irving} in New York, as well as
a ``cocktails educator.'' She serves a G\&T that evokes what she calls
the ``Spanish style,'' containing AMASS gin from Los Angeles; touches of
cinnamon and vanilla from Licor 43; Pierre Ferrand Dry Curacao; and
Fever-Tree tonic. They're served in wine goblets, with a lot of ice and
garnished with a lime wedge and a dehydrated orange wheel.

It sounds spectacular --- but Ms. Dorman also sings the praises of the
drink made simply. The important thing, she told me, is to make sure the
tonic is fresh and cold. And always to add lime; if you don't have some
citrus to counterbalance the bitterness of the quinine and the piney
earthiness of the juniper berries, she says, don't bother.

A gin and tonic is not a meditative drink, like a
\href{https://www.nytimes.com/2019/06/12/opinion/negroni-2019.html}{Negroni}or
a single malt Scotch. It's a restorative drink, that refreshes after a
long day. It's a drink that, for many people, evokes days of youth ---
it is, Ms. Dorman observes, one of the first cocktails you make when you
decide you're going to make ``a grown-up drink.''

If many people's first experience with liquor is sweet or salty drinks
designed to cover up the taste of alcohol --- margaritas and daiquiris,
piña coladas and screwdrivers --- a gin and tonic is one of the first
drinks you might mix not to avoid the bitterness, but to meet it head
on.

It's one more reason the gin and tonic is the perfect drink for this
long, miserable summer, some of the doggiest days we've ever known. I
asked Ms. Dorman if she felt the gin and tonic, as a cocktail, was
appropriately bitter enough for this dark, miserable era we're all still
struggling to survive.

``Yeah,'' she said. ``I feel like, `Actually, I'll take two.'''

\emph{The Times is committed to publishing}
\href{https://www.nytimes.com/2019/01/31/opinion/letters/letters-to-editor-new-york-times-women.html}{\emph{a
diversity of letters}} \emph{to the editor. We'd like to hear what you
think about this or any of our articles. Here are some}
\href{https://help.nytimes.com/hc/en-us/articles/115014925288-How-to-submit-a-letter-to-the-editor}{\emph{tips}}\emph{.
And here's our email:}
\href{mailto:letters@nytimes.com}{\emph{letters@nytimes.com}}\emph{.}

\emph{Follow The New York Times Opinion section on}
\href{https://www.facebook.com/nytopinion}{\emph{Facebook}}\emph{,}
\href{http://twitter.com/NYTOpinion}{\emph{Twitter (@NYTopinion)}}
\emph{and}
\href{https://www.instagram.com/nytopinion/}{\emph{Instagram}}\emph{.}

Advertisement

\protect\hyperlink{after-bottom}{Continue reading the main story}

\hypertarget{site-index}{%
\subsection{Site Index}\label{site-index}}

\hypertarget{site-information-navigation}{%
\subsection{Site Information
Navigation}\label{site-information-navigation}}

\begin{itemize}
\tightlist
\item
  \href{https://help.nytimes.com/hc/en-us/articles/115014792127-Copyright-notice}{©~2020~The
  New York Times Company}
\end{itemize}

\begin{itemize}
\tightlist
\item
  \href{https://www.nytco.com/}{NYTCo}
\item
  \href{https://help.nytimes.com/hc/en-us/articles/115015385887-Contact-Us}{Contact
  Us}
\item
  \href{https://www.nytco.com/careers/}{Work with us}
\item
  \href{https://nytmediakit.com/}{Advertise}
\item
  \href{http://www.tbrandstudio.com/}{T Brand Studio}
\item
  \href{https://www.nytimes.com/privacy/cookie-policy\#how-do-i-manage-trackers}{Your
  Ad Choices}
\item
  \href{https://www.nytimes.com/privacy}{Privacy}
\item
  \href{https://help.nytimes.com/hc/en-us/articles/115014893428-Terms-of-service}{Terms
  of Service}
\item
  \href{https://help.nytimes.com/hc/en-us/articles/115014893968-Terms-of-sale}{Terms
  of Sale}
\item
  \href{https://spiderbites.nytimes.com}{Site Map}
\item
  \href{https://help.nytimes.com/hc/en-us}{Help}
\item
  \href{https://www.nytimes.com/subscription?campaignId=37WXW}{Subscriptions}
\end{itemize}
