Sections

SEARCH

\protect\hyperlink{site-content}{Skip to
content}\protect\hyperlink{site-index}{Skip to site index}

\href{https://www.nytimes.com/section/books/review}{Book Review}

\href{https://myaccount.nytimes.com/auth/login?response_type=cookie\&client_id=vi}{}

\href{https://www.nytimes.com/section/todayspaper}{Today's Paper}

\href{/section/books/review}{Book Review}\textbar{}What if the Meat We
Ate Was Human?

\url{https://nyti.ms/31vo6IX}

\begin{itemize}
\item
\item
\item
\item
\item
\end{itemize}

Advertisement

\protect\hyperlink{after-top}{Continue reading the main story}

Supported by

\protect\hyperlink{after-sponsor}{Continue reading the main story}

Fiction

\hypertarget{what-if-the-meat-we-ate-was-human}{%
\section{What if the Meat We Ate Was
Human?}\label{what-if-the-meat-we-ate-was-human}}

\includegraphics{https://static01.nyt.com/images/2020/07/30/books/review/Kraus1/Kraus1-articleLarge.jpg?quality=75\&auto=webp\&disable=upscale}

Buy Book ▾

\begin{itemize}
\tightlist
\item
  \href{https://www.amazon.com/gp/search?index=books\&tag=NYTBSREV-20\&field-keywords=Tender+Is+the+Flesh+Agustina+Bazterrica}{Amazon}
\item
  \href{https://du-gae-books-dot-nyt-du-prd.appspot.com/buy?title=Tender+Is+the+Flesh\&author=Agustina+Bazterrica}{Apple
  Books}
\item
  \href{https://www.anrdoezrs.net/click-7990613-11819508?url=https\%3A\%2F\%2Fwww.barnesandnoble.com\%2Fw\%2F\%3Fean\%3D9781982150921}{Barnes
  and Noble}
\item
  \href{https://www.anrdoezrs.net/click-7990613-35140?url=https\%3A\%2F\%2Fwww.booksamillion.com\%2Fp\%2FTender\%2BIs\%2Bthe\%2BFlesh\%2FAgustina\%2BBazterrica\%2F9781982150921}{Books-A-Million}
\item
  \href{https://bookshop.org/a/3546/9781982150921}{Bookshop}
\item
  \href{https://www.indiebound.org/book/9781982150921?aff=NYT}{Indiebound}
\end{itemize}

When you purchase an independently reviewed book through our site, we
earn an affiliate commission.

By Daniel Kraus

\begin{itemize}
\item
  Aug. 4, 2020, 5:00 a.m. ET
\item
  \begin{itemize}
  \item
  \item
  \item
  \item
  \item
  \end{itemize}
\end{itemize}

\textbf{TENDER IS THE FLESH}\\
By Agustina Bazterrica\\
Translated by Sarah Moses

``Carcass. Cut in half. Stunner. Slaughter line. Spray wash.'' From the
first words of the Argentine novelist Agustina Bazterrica's second
novel, ``Tender Is the Flesh,'' the reader is already the livestock in
the line, reeling, primordially aware that this book is a butcher's
block, and nothing that happens next is going to be pretty.

Marcos Tejo has been dealing ``heads'' of livestock for the Krieg
Processing Plant since a virus rendered all animals toxic for human
consumption. Things get hazy here --- Bazterrica's interest is less in
near-future world-building than in reflecting our grisly present --- but
the virus has led to ``the Transition,'' marked by the extermination of
as many animals as possible and, to satisfy man's innate craving, the
adoption of industrialized cannibalism.

The icky euphemism is ``special meat.'' No one says ``human meat,'' as
the bodies being eaten are not considered human. The prime fare, F.G.P.s
(First Generation Pure), are born and raised in captivity, artificially
inseminated, sold, butchered and plated however you'd like, from ``a
starter of fingers in a sherry reduction'' to a ``tongue \ldots{}
marinated in fine herbs, served over kimchi and lemon-dressed
potatoes.''

The setup sounds like the Charlton Heston teeth-gnasher ``Soylent
Green'' mated to Anthony Burgess's satirical novel ``The Wanting Seed,''
yet the prose feels like neither. Because of its banal and miserable
tone, given a muscular translation by Sarah Moses, ``Tender Is the
Flesh'' --- which won Argentina's Premio Clarin de Novela --- is, at
least in spates, more powerful than either forebear.

Within a cast of cadaver-cold characters, Marcos is the iciest. A career
of dehumanizing human livestock has dehumanized him; several chapters
pass before we're even sure of his name. The novel's first half puts the
reader in Marcos's shoes with a long series of procedural scenes. They
are as graceless as bludgeon strikes, but as effective too.

Marcos visits a breeding center where we learn of severing vocal cords
to make the livestock more docile --- ``meat doesn't talk,'' the owner
quips. In the dairy section, machines are ``suctioning the females'
udders.'' For ease of care, all pregnant females have their limbs
removed. Back at Krieg, livestock is sawed open, plucked of eyes and
tongues, flayed of skin, pillaged of entrails, flushed and quartered
until the result bears no resemblance to a human being --- entirely the
point, as we know from any grocery-store meat section.

It's surprising, though it shouldn't be, how easy it is to critique our
real-life factory-farm processes by mentally swapping a human for a pig
or cow. There really is no debate here; our process of mechanizing meat
production is morally appalling. If Bazterrica had stopped here, she'd
still have crafted one of the most potent indictments since ``Blood of
the Beasts,'' Georges Franju's palate-killing 1949 documentary about
Paris slaughterhouses.

Image

It's surprising, though it shouldn't be, how easy it is to critique our
real-life factory-farm processes by mentally swapping a human for a pig
or cow.

Of course, Bazterrica isn't writing a pamphlet. Her new world order
isn't so much woven into story as it is planted in front of us like a
gravestone. The conveyer-belt pacing therefore feels intentional: Our
murderous wrongs are repeated, and repeated, and to look away is to
refuse, deliberately, to bear witness.

As with Guy Montag in ``Fahrenheit 451,'' Marcos's deep-seated unease
with the modern world is awakened via a catalyst, in his case the gift
of an F.G.P. female. Great-tasting stock, worth a fortune. He can keep
her alive as ``domestic head,'' to slice off bits whenever he wants a
treat. But it's the last thing he wants. Marcos is haunted by the death
of his infant son, which emotionally disemboweledhis family. Funerals
are foregone these days, lest so-called Scavengers ransack the fresh
grave to feast on the corpse.

In his grief, Marcos increasingly identifies with the helpless.
Frequently he visits a dilapidated former zoo, where he encounters
frothing feral dogs he considers ``beautiful.'' He views the world of
predator and prey as natural, while the female in his barn is but a
device to damn him.

But then he starts to like her, and she likes him back. He cleans her
and names her Jasmine after her scent. He caresses the brand on her
forehead. Abruptly, we are faced with the prospect of a plot when,
halfway through, Marcos gets Jasmine pregnant --- a crime that could
land them both in the Municipal Slaughterhouse.

Readers want to root for a protagonist after such unmitigated bleakness.
Rediscover tenderness, Marcos, revolt against the system! But having
clobbered us dizzy, Bazterrica switches to rope-a-dope. Marcos's
off-scene impregnation of Jasmine feels like rape, particularly given
her childlike nature. Future-world sex is depicted as a drug to dull the
pain, cater-corner to cannibalism. Marcos's tryst with a meat-seller
happens atop a bloody butcher table, and big-game hunters (who now hunt
the most dangerous game: people) enjoy the tale of a brothel where you
can pay extra to eat your prostitute.

Though one can imagine cannibal culture being the apotheosis of
capitalism, Bazterrica's families, shackled by orthodoxy, can't exist
without the who-eats-whom hierarchies positioning some people as the
property of others. A worse barbarism forever waits in the wings.

But between Marcos and Jasmine, the barbarism is gussied up like love.
``He gets home tired,'' Bazterrica writes. ``Before opening Jasmine's
room, he takes a shower, otherwise she won't let him do so in peace.
She'll try to get under the water with him, kiss him, hug him. He
understands she's alone all day, that when he gets home she wants to
follow him around the house.''

This pet-language is an unmistakable allusion to Marcos's recollection
of having put down the family's viral dogs. The novel thus packages its
dilemma in tidy butcher paper: Is our supposed love of animals, and
occasionally theirs of us, a calculated transaction? If so, does it
differ from the transactions between humans? Not by much, Bazterrica
seems to say. Eventually the factory of our amity breaks down, and all
that's left is hunger.

Advertisement

\protect\hyperlink{after-bottom}{Continue reading the main story}

\hypertarget{site-index}{%
\subsection{Site Index}\label{site-index}}

\hypertarget{site-information-navigation}{%
\subsection{Site Information
Navigation}\label{site-information-navigation}}

\begin{itemize}
\tightlist
\item
  \href{https://help.nytimes.com/hc/en-us/articles/115014792127-Copyright-notice}{©~2020~The
  New York Times Company}
\end{itemize}

\begin{itemize}
\tightlist
\item
  \href{https://www.nytco.com/}{NYTCo}
\item
  \href{https://help.nytimes.com/hc/en-us/articles/115015385887-Contact-Us}{Contact
  Us}
\item
  \href{https://www.nytco.com/careers/}{Work with us}
\item
  \href{https://nytmediakit.com/}{Advertise}
\item
  \href{http://www.tbrandstudio.com/}{T Brand Studio}
\item
  \href{https://www.nytimes.com/privacy/cookie-policy\#how-do-i-manage-trackers}{Your
  Ad Choices}
\item
  \href{https://www.nytimes.com/privacy}{Privacy}
\item
  \href{https://help.nytimes.com/hc/en-us/articles/115014893428-Terms-of-service}{Terms
  of Service}
\item
  \href{https://help.nytimes.com/hc/en-us/articles/115014893968-Terms-of-sale}{Terms
  of Sale}
\item
  \href{https://spiderbites.nytimes.com}{Site Map}
\item
  \href{https://help.nytimes.com/hc/en-us}{Help}
\item
  \href{https://www.nytimes.com/subscription?campaignId=37WXW}{Subscriptions}
\end{itemize}
