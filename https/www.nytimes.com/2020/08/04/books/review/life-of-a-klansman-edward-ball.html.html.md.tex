Sections

SEARCH

\protect\hyperlink{site-content}{Skip to
content}\protect\hyperlink{site-index}{Skip to site index}

\href{https://www.nytimes.com/section/books/review}{Book Review}

\href{https://myaccount.nytimes.com/auth/login?response_type=cookie\&client_id=vi}{}

\href{https://www.nytimes.com/section/todayspaper}{Today's Paper}

\href{/section/books/review}{Book Review}\textbar{}`Life of a Klansman'
Tells Ugly Truths About America, Past and Present

\url{https://nyti.ms/33l7h5T}

\begin{itemize}
\item
\item
\item
\item
\item
\item
\end{itemize}

Advertisement

\protect\hyperlink{after-top}{Continue reading the main story}

Supported by

\protect\hyperlink{after-sponsor}{Continue reading the main story}

Nonfiction

\hypertarget{life-of-a-klansman-tells-ugly-truths-about-america-past-and-present}{%
\section{`Life of a Klansman' Tells Ugly Truths About America, Past and
Present}\label{life-of-a-klansman-tells-ugly-truths-about-america-past-and-present}}

\includegraphics{https://static01.nyt.com/images/2020/08/04/books/review/04Isaacson/04Isaacson-articleLarge.jpg?quality=75\&auto=webp\&disable=upscale}

Buy Book ▾

\begin{itemize}
\tightlist
\item
  \href{https://www.amazon.com/gp/search?index=books\&tag=NYTBSREV-20\&field-keywords=Life+of+a+Klansman\%3A+A+Family+History+in+White+Supremacy+Edward+Ball}{Amazon}
\item
  \href{https://du-gae-books-dot-nyt-du-prd.appspot.com/buy?title=Life+of+a+Klansman\%3A+A+Family+History+in+White+Supremacy\&author=Edward+Ball}{Apple
  Books}
\item
  \href{https://www.anrdoezrs.net/click-7990613-11819508?url=https\%3A\%2F\%2Fwww.barnesandnoble.com\%2Fw\%2F\%3Fean\%3D9780374186326}{Barnes
  and Noble}
\item
  \href{https://www.anrdoezrs.net/click-7990613-35140?url=https\%3A\%2F\%2Fwww.booksamillion.com\%2Fp\%2FLife\%2Bof\%2Ba\%2BKlansman\%253A\%2BA\%2BFamily\%2BHistory\%2Bin\%2BWhite\%2BSupremacy\%2FEdward\%2BBall\%2F9780374186326}{Books-A-Million}
\item
  \href{https://bookshop.org/a/3546/9780374186326}{Bookshop}
\item
  \href{https://www.indiebound.org/book/9780374186326?aff=NYT}{Indiebound}
\end{itemize}

When you purchase an independently reviewed book through our site, we
earn an affiliate commission.

By Walter Isaacson

\begin{itemize}
\item
  Aug. 4, 2020, 5:00 a.m. ET
\item
  \begin{itemize}
  \item
  \item
  \item
  \item
  \item
  \item
  \end{itemize}
\end{itemize}

\textbf{LIFE OF A KLANSMAN}\\
\textbf{A Family History in White Supremacy}\\
By Edward Ball

When his mother died in 2003, the writer Edward Ball went to New
Orleans, where her family had lived for generations, to bury her and
sort through her belongings. Among her papers were documents that had
been collected by her late aunt, including tales about the man who was
known in the family as ``our Klansman.''

Ball had already written, in 1998, a deeply reported National Book
Award-winning history,
``\href{https://www.nytimes.com/1998/03/01/books/skeletons-in-the-family-closet.html?searchResultPosition=1}{Slaves
in the Family},'' for which he tracked down descendants of those who had
once been enslaved by his South Carolina ancestors on his father's side.
In his new book, ``Life of a Klansman,'' he follows a similar course,
taking the reader along with him on a journey of discovery as he teases
out facts, engages in speculation and shares his emotions about the sad
saga of Constant Lecorgne, an unsuccessful carpenter and embittered
racist who was a great-great-grandfather on his mother's side.

The result is a haunting tapestry of interwoven stories that inform us
not just about our past but about the resentment-bred demons that are
all too present in our society today. ``This is a family story,'' he
writes. ``Yet it is not a family story wrapped in sugar, the way some
people like to serve them.'' The family is not just his, it's our
nation's.

\emph{{[} Read an excerpt from}
\href{https://www.nytimes.com/2020/08/04/books/review/life-of-a-klansman-by-edward-ball-an-excerpt.html}{\emph{``Life
of a Klansman.''}} \emph{{]}}

Lecorgne, born in 1832, was raised in a New Orleans that was, as it has
been throughout its history, very complex racially and ethnically. About
a quarter of the population were French-speaking whites, a quarter were
English-speaking whites, a quarter were free mixed-race Creoles and a
quarter were slaves. The Lecorgnes were in the first category, but they
rented a home from a free French-speaking woman of color.

Because he has few documents, Ball indulges in a lot of surmises and
speculations, perhaps a bit too many for my taste. He pictures the young
boy Lecorgne walking with his family the four blocks to Congo Square,
where the slaves were allowed to drum and dance on Sunday afternoons.
There is a sexual tension that the boy finds both attractive and
appalling. ``I think I can begin to see, in Congo Square, a script and a
stage, a place where Blackness and whiteness meet,'' Ball writes.
``Complications ensue. They move apart. Eventually the script calls for
a crescendo. Blackness and whiteness collide, and the ending, for our
Klansman, is an explosion.''

Lecorgne is the unsuccessful and unpopular middle child of a large
family. He tries to make a living as a carpenter, but he descends into
what is known in the local parlance as \emph{petits blancs}, the poor
working-class whites. Resentments accrue. When he marries, his wife's
family gives him a household slave as a dowry, but he has to sell her
for \$500 to afford a home.

The Civil War offers Lecorgne an outlet for his resentments and a chance
to finally earn a little respect from his family and neighbors. But even
there he fails. After joining one of Louisiana's militias as a captain,
he is demoted to a second lieutenant. On a train trip to Virginia he
gets into a melee and, along with much of his unit, is court-martialed.
At a public ceremony, he and his comrades have one-half of their scalps
shaved and are cashiered. Lecorgne heads back to New Orleans in
disgrace.

\includegraphics{https://static01.nyt.com/images/2020/03/03/books/review/Isaacson2/Isaacson2-articleLarge.jpg?quality=75\&auto=webp\&disable=upscale}

Under Reconstruction, the city becomes integrated. Blacks can vote,
testify against whites in court and sit where they want on the
streetcars; a few even attend integrated schools. Lecorgne's
neighborhood in uptown New Orleans, around where Napoleon Avenue meets
the river (which is where I grew up), becomes mixed, with Creoles,
Germans, Irish, Blacks and mulattoes all living on the same blocks. It's
nice to think what the city, and our nation, might have been had that
progression continued. But among the whites, especially the \emph{petits
blancs}, resentments built.

The clubhouses for resentful poor whites are the neighborhood
firehouses. Lecorgne joined one just off Napoleon Avenue, the Home Hook
\& Ladder Company, housed in a Romanesque building with a first-floor
facade clad in stone and a second in red brick. Its membership suddenly
swelled during Reconstruction to 85 men, far more than were necessary to
fight off the neighborhood's house fires. Instead, as Ball writes, ``the
firehouses play a big part in the tale of the Ku-klux,'' which is what
the loose-knit confederation of white supremacist organizations came to
be called.

Lecorgne was a minor player in this movement. But for that reason his
tale is valuable, both for understanding his times and for understanding
our own; he allows us a glimpse of who becomes one of the mass of
followers of racist movements, and why.

His one recorded inglorious moment came in early 1873. With Black
support, a Republican was elected governor, and the local white militias
took up arms to resist his rule. Lecorgne and a group of armed men
gathered with the goal of taking over their neighborhood police precinct
station, hoping it would spark a wider white uprising. Although the
newspapers referred to them as ``Ku-Kluxers,'' the rebel raiders most
likely did not wear robes and hoods. That practice was mainly for rural
marauders. They were successful, but the following night the police
staged a counterattack. As Lecorgne hid in a staircase, his cousin was
wounded and a friend was killed.

Lecorgne surrendered and was carried away to the city jail. In the
indictment, which misspelled his name, he is accused of treason and
violating federal law for having ``unlawfully maliciously and
traitorously conspired'' to attack state authorities. But a local judge
quickly dismissed all the charges. That low point was the high point of
his life.

Near the end of his book, Ball makes a fascinating digression. It
involves a prominent person of color who lived in New Orleans at the
same time as Lecorgne. Louis Charles Roudanez was a medical doctor,
trained in France and at Dartmouth, who published The New Orleans
Tribune, a daily newspaper for the Black community. An \emph{homme de
couleur libre}, Roudanez married a free woman of color. While
researching his own family, Ball decided to look for the descendants of
the Roudanez family.

He finds one of the physician-publisher's great-great-grandchildren,
named Mark Roudané, living in a leafy subdivision of St. Paul, Minn.
``He was raised white, and he appears white,'' Ball writes of Roudané.
``In middle age he learned that according to the one-drop rule of
blackness, he was not white.'' Roudané did not know the tale of his
father's ancestors, or even the Roudanez spelling of his family name,
until he stumbled across some family documents when he was 55. As
happened with Ball, the discovery of a bit of family history leads
Roudané on a quest. ``When my father died, in 2005, I was going through
his papers and throwing stuff away, and I found an unmarked binder,''
Roudané tells Ball. It contained papers showing how his father, who was
designated as ``colored'' on his birth certificate, had forsaken his
distinguished roots, changed the spelling of his name as a young man,
gone to Tulane by passing as white and then moved to the Midwest.
Despite this history, or perhaps because of it, he became a resentful
white racist. ``When it came to talking about Black people,'' Mark
Roudané told Ball, ``all this venom would come out. I thought, `Why is
my dad being ugly?' I didn't understand it.''

The interconnected strands of race and history give Ball's entrancing
stories a Faulknerian resonance. In Ball's retelling of his family saga,
the sins and stains of the past are still very much with us, not
something we can dismiss by blaming them on misguided ancestors who died
long ago. ``It is not a distortion to say that Constant's rampage 150
years ago helps, in some impossible-to-measure way, to clear space for
the authority and comfort of whites living now --- not just for me and
for his 50 or 60 descendants, but for whites in general,'' Ball writes.
``I am an heir to Constant's acts of terror. I do not deny it, and the
bitter truth makes me sick at the stomach.''

Advertisement

\protect\hyperlink{after-bottom}{Continue reading the main story}

\hypertarget{site-index}{%
\subsection{Site Index}\label{site-index}}

\hypertarget{site-information-navigation}{%
\subsection{Site Information
Navigation}\label{site-information-navigation}}

\begin{itemize}
\tightlist
\item
  \href{https://help.nytimes.com/hc/en-us/articles/115014792127-Copyright-notice}{©~2020~The
  New York Times Company}
\end{itemize}

\begin{itemize}
\tightlist
\item
  \href{https://www.nytco.com/}{NYTCo}
\item
  \href{https://help.nytimes.com/hc/en-us/articles/115015385887-Contact-Us}{Contact
  Us}
\item
  \href{https://www.nytco.com/careers/}{Work with us}
\item
  \href{https://nytmediakit.com/}{Advertise}
\item
  \href{http://www.tbrandstudio.com/}{T Brand Studio}
\item
  \href{https://www.nytimes.com/privacy/cookie-policy\#how-do-i-manage-trackers}{Your
  Ad Choices}
\item
  \href{https://www.nytimes.com/privacy}{Privacy}
\item
  \href{https://help.nytimes.com/hc/en-us/articles/115014893428-Terms-of-service}{Terms
  of Service}
\item
  \href{https://help.nytimes.com/hc/en-us/articles/115014893968-Terms-of-sale}{Terms
  of Sale}
\item
  \href{https://spiderbites.nytimes.com}{Site Map}
\item
  \href{https://help.nytimes.com/hc/en-us}{Help}
\item
  \href{https://www.nytimes.com/subscription?campaignId=37WXW}{Subscriptions}
\end{itemize}
