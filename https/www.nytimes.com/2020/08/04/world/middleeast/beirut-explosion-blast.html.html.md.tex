Sections

SEARCH

\protect\hyperlink{site-content}{Skip to
content}\protect\hyperlink{site-index}{Skip to site index}

\href{https://www.nytimes.com/section/world/middleeast}{Middle East}

\href{https://myaccount.nytimes.com/auth/login?response_type=cookie\&client_id=vi}{}

\href{https://www.nytimes.com/section/todayspaper}{Today's Paper}

\href{/section/world/middleeast}{Middle East}\textbar{}Huge Explosion
Rocks Beirut: Live Updates

\url{https://nyti.ms/33qudRi}

\begin{itemize}
\item
\item
\item
\item
\item
\item
\end{itemize}

Advertisement

\protect\hyperlink{after-top}{Continue reading the main story}

Supported by

\protect\hyperlink{after-sponsor}{Continue reading the main story}

LIVE UPDATES

Updated~

Aug. 4, 2020, 3:07 p.m. ET

Aug. 4, 2020, 3:07 p.m. ET

\hypertarget{huge-explosion-rocks-beirut-live-updates}{%
\section{Huge Explosion Rocks Beirut: Live
Updates}\label{huge-explosion-rocks-beirut-live-updates}}

The government had stored ``highly explosive materials'' at the blast
scene on the Lebanese capital's waterfront, a top military official
said. Hundreds of people were injured and the shock was felt across the
city.

\href{https://www.nytimes.com/by/ben-hubbard}{\includegraphics{https://static01.nyt.com/images/2018/10/10/multimedia/author-ben-hubbard/author-ben-hubbard-thumbLarge.png}}

By \href{https://www.nytimes.com/by/ben-hubbard}{Ben Hubbard}

Right Now

More than 400 people have turned up with injuries at just one of
Beirut's hospitals.

\hypertarget{heres-what-you-need-to-know}{%
\subsubsection{Here's what you need to
know:}\label{heres-what-you-need-to-know}}

\begin{itemize}
\tightlist
\item
  \protect\hyperlink{link-12ef1c10}{A smaller explosion was followed by
  a much larger one.}
\item
  \protect\hyperlink{link-4ec3be73}{``Explosive materials'' were stored
  at the blast site, and the disaster may have started with a fire at a
  warehouse, state-run media said.}
\item
  \protect\hyperlink{link-26e5e8b0}{It was not clear how many were hurt
  or killed.}
\item
  \protect\hyperlink{link-1202af77}{The explosion hit the waterfront,
  near several important buildings.}
\item
  \protect\hyperlink{link-3a2e264f}{The blast stirred memories of war in
  a city that had been relatively calm in recent years.}
\end{itemize}

\includegraphics{https://static01.nyt.com/images/2020/08/04/world/04lebanon-vidcover/04lebanon-vidcover-videoSixteenByNine3000.jpg}

\subsection{}

A smaller explosion was followed by a much larger one.

Two explosions shook Beirut, the second one with enough force to break
windows over a radius of miles, damaging and shaking buildings, wounding
hundreds of people and strewing debris over a wide area.

Videos posted online showed a shock wave erupting from the second
explosion, knocking people down and enveloping much of the center city
in a cloud of dust and smoke. Cars were overturned and streets were
blocked by debris, forcing many injured people to walk to hospitals.

Flames continued to rise from the rubble well after the explosions, and
a cloud of smoke, tinted pink in the sunset, rose thousands of feet into
the sky.

\hypertarget{-1}{%
\subsection{}\label{-1}}

``Explosive materials'' were stored at the blast site, and the disaster
may have started with a fire at a warehouse, state-run media said.

\includegraphics{https://static01.nyt.com/images/2020/08/04/world/04lebanon/merlin_175295589_6a1d5658-1abe-4cfc-971a-d8b74fe6b08b-articleLarge.jpg?quality=75\&auto=webp\&disable=upscale}

``Highly explosive materials,'' seized by the government years ago, were
stored where the explosions occurred, said Maj. Gen. Abbas Ibrahim, the
head of Lebanon's general security service, according to the National
News Agency.

General Ibrahim did not say what those materials were, but he warned
against getting ``ahead of the investigation'' and speculating about a
terrorist act.

At least one explosion, at about 6 p.m., stemmed from a fire at a
warehouse at Beirut's port,
\href{http://nna-leb.gov.lb/en/show-news/118492/Fire-breaks-out-in-warehouse-at-Port-of-Beirut-causes-major-explosion}{according
to Lebanon's National News Agency}.

There were local reports that the warehouse contained fireworks, and in
several videos posted online, colored flashes could be seen in the dark
smoke rising from the fire, just before the second explosion.

The governor of Beirut, Marwan Abboud, speaking on television, could not
say what had caused the explosion. Breaking into tears, he called it a
national catastrophe.

\hypertarget{-2}{%
\subsection{}\label{-2}}

It was not clear how many were hurt or killed.

Image

Evacuating the wounded from the scene of the explosion at the port in
Beirut on Tuesday.Credit...Anwar Amro/Agence France-Presse --- Getty
Images

Just one hospital, Rizk Hospital, said 400 people had gone there to be
treated for injuries suffered in the disaster, according to the National
News Agency, indicating how widespread the destruction was.

The secretary-general of the Kataeb political party, Nizar Najarian, was
killed in the blast, and among those injured was Kamal Hayek, the
chairman of the state-owned electricity company, who was in critical
condition, the news agency reported.

Videos of the aftermath posted online showed wounded people bleeding
amid the dust and rubble, and damage where flying debris had punched
holes in walls and furniture. On social media, people reported damage to
homes and cars far from the port.

The Lebanese Red Cross said that every available ambulance from North
Lebanon, Bekaa and South Lebanon was being dispatched to Beirut to help
patients.

At least one hospital was overwhelmed and was turning wounded people
away.

Public Health Minister Hamad Hassan announced that his ministry would
cover the costs of treating the wounded at hospitals, the National News
Agency reported. It said the decision covered both hospitals that have
contracts with the ministry as well as those that don't.

Prime Minister Hassan Diab announced that Wednesday would be a national
day of mourning, the National News Agency reported. The Lebanese
presidency said on Twitter that President Michel Aoun had instructed the
military to aid in the response, and called an emergency meeting of the
Supreme Defense Council on Tuesday evening.

\hypertarget{-3}{%
\subsection{}\label{-3}}

The explosion hit the waterfront, near several important buildings.

Image

Emergency workers and civilians at the site of the explosion in Beirut
on Tuesday.Credit...Anwar Amro/Agence France-Presse --- Getty Images

The explosions hit Beirut's northern, industrial waterfront, little more
than a mile away from the Grand Serail palace, where Lebanon's prime
minister is based. Many landmarks, including hospitals, mosques,
churches and universities are nearby.

They erupted next to a tall building called Beirut Port Silos, at or
near a structure identified on maps as a warehouse. Videos showed only
twisted metal and chunks of concrete where that warehouse had been, some
of it identifiable as the remains of trucks and shipping containers.

\hypertarget{-4}{%
\subsection{}\label{-4}}

The blast stirred memories of war in a city that had been relatively
calm in recent years.

Image

Evacuating wounded people along a road in Beirut.Credit...Hassan
Ammar/Associated Press

Beirut has suffered through a history of explosions --- car bombings,
shelling and airstrikes --- during a prolonged civil war and fighting
between Israel and the militant group Hezbollah.

But if the latest explosions were found to have been caused
intentionally, they would shatter a prolonged stretch of relative calm
in the Lebanese capital.

Less than a week ago, Israel said it had thwarted a raid by a
``terrorist squad'' from Hezbollah, the Shiite group that is part of
Lebanon's government, in a disputed border area. Israeli military
officials said there was an exchange of gunfire, which Hezbollah denied.

Israeli military officials say Hezbollah has planted many rockets in
southern Lebanon that could threaten northern Israel. But In recent
years, Hezbollah has refrained from killing Israelis while Israel has
largely avoided killing Hezbollah fighters in Syria, where they are
fighting on the Syrian government's side.

Both Israel and Hezbollah have sought to avoid a war that could
devastate Lebanon and Israel.

An Israeli intelligence official denied any Israeli involvement in the
explosion on Tuesday.

Image

The wreckage near the port in Beirut on Tuesday.Credit...Agence
France-Presse --- Getty Images

Nada Rashwan contributed reporting from Cairo, Maria Abi-Habib from Alan
Yuhas from Philadelphia, Adam Rasgon and Ronen Bergman from Tel Aviv,
Rick Gladstone from Eastham, Mass., and Richard Pérez-Peña from New
York.

Advertisement

\protect\hyperlink{after-bottom}{Continue reading the main story}

\hypertarget{site-index}{%
\subsection{Site Index}\label{site-index}}

\hypertarget{site-information-navigation}{%
\subsection{Site Information
Navigation}\label{site-information-navigation}}

\begin{itemize}
\tightlist
\item
  \href{https://help.nytimes.com/hc/en-us/articles/115014792127-Copyright-notice}{©~2020~The
  New York Times Company}
\end{itemize}

\begin{itemize}
\tightlist
\item
  \href{https://www.nytco.com/}{NYTCo}
\item
  \href{https://help.nytimes.com/hc/en-us/articles/115015385887-Contact-Us}{Contact
  Us}
\item
  \href{https://www.nytco.com/careers/}{Work with us}
\item
  \href{https://nytmediakit.com/}{Advertise}
\item
  \href{http://www.tbrandstudio.com/}{T Brand Studio}
\item
  \href{https://www.nytimes.com/privacy/cookie-policy\#how-do-i-manage-trackers}{Your
  Ad Choices}
\item
  \href{https://www.nytimes.com/privacy}{Privacy}
\item
  \href{https://help.nytimes.com/hc/en-us/articles/115014893428-Terms-of-service}{Terms
  of Service}
\item
  \href{https://help.nytimes.com/hc/en-us/articles/115014893968-Terms-of-sale}{Terms
  of Sale}
\item
  \href{https://spiderbites.nytimes.com}{Site Map}
\item
  \href{https://help.nytimes.com/hc/en-us}{Help}
\item
  \href{https://www.nytimes.com/subscription?campaignId=37WXW}{Subscriptions}
\end{itemize}
