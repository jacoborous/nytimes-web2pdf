Sections

SEARCH

\protect\hyperlink{site-content}{Skip to
content}\protect\hyperlink{site-index}{Skip to site index}

\href{https://www.nytimes.com/section/business}{Business}

\href{https://myaccount.nytimes.com/auth/login?response_type=cookie\&client_id=vi}{}

\href{https://www.nytimes.com/section/todayspaper}{Today's Paper}

\href{/section/business}{Business}\textbar{}The Tax Cut That Trump
Wants, but Few Others Do, Explained

\url{https://nyti.ms/3jAidSD}

\begin{itemize}
\item
\item
\item
\item
\item
\end{itemize}

\href{https://www.nytimes.com/news-event/coronavirus?action=click\&pgtype=Article\&state=default\&region=TOP_BANNER\&context=storylines_menu}{The
Coronavirus Outbreak}

\begin{itemize}
\tightlist
\item
  live\href{https://www.nytimes.com/2020/08/04/world/coronavirus-cases.html?action=click\&pgtype=Article\&state=default\&region=TOP_BANNER\&context=storylines_menu}{Latest
  Updates}
\item
  \href{https://www.nytimes.com/interactive/2020/us/coronavirus-us-cases.html?action=click\&pgtype=Article\&state=default\&region=TOP_BANNER\&context=storylines_menu}{Maps
  and Cases}
\item
  \href{https://www.nytimes.com/interactive/2020/science/coronavirus-vaccine-tracker.html?action=click\&pgtype=Article\&state=default\&region=TOP_BANNER\&context=storylines_menu}{Vaccine
  Tracker}
\item
  \href{https://www.nytimes.com/2020/08/02/us/covid-college-reopening.html?action=click\&pgtype=Article\&state=default\&region=TOP_BANNER\&context=storylines_menu}{College
  Reopening}
\item
  \href{https://www.nytimes.com/live/2020/08/04/business/stock-market-today-coronavirus?action=click\&pgtype=Article\&state=default\&region=TOP_BANNER\&context=storylines_menu}{Economy}
\end{itemize}

Advertisement

\protect\hyperlink{after-top}{Continue reading the main story}

Supported by

\protect\hyperlink{after-sponsor}{Continue reading the main story}

\hypertarget{the-tax-cut-that-trump-wants-but-few-others-do-explained}{%
\section{The Tax Cut That Trump Wants, but Few Others Do,
Explained}\label{the-tax-cut-that-trump-wants-but-few-others-do-explained}}

How a temporary reduction in payroll taxes became the president's go-to
proposal to stoke the economy.

\includegraphics{https://static01.nyt.com/images/2020/07/22/business/22DC-Virus-PayrollTax-01/merlin_174809538_a9a00342-4f69-47f8-afa5-c8083b696028-articleLarge.jpg?quality=75\&auto=webp\&disable=upscale}

\href{https://www.nytimes.com/by/jim-tankersley}{\includegraphics{https://static01.nyt.com/images/2018/10/19/multimedia/author-jim-tankersley/author-jim-tankersley-thumbLarge.png}}

By \href{https://www.nytimes.com/by/jim-tankersley}{Jim Tankersley}

\begin{itemize}
\item
  July 23, 2020
\item
  \begin{itemize}
  \item
  \item
  \item
  \item
  \item
  \end{itemize}
\end{itemize}

WASHINGTON --- President Trump badly wants to cut the payroll tax.
Hardly anyone in Congress shares his enthusiasm.

From the early days of the coronavirus pandemic sweeping the United
States, and plunging the economy into a sharp and brutal recession, Mr.
Trump has been pushing Congress to temporarily eliminate the taxes
American workers and their employers pay to help support Social Security
and Medicare. He
\href{https://twitter.com/realDonaldTrump/status/1237924658185469954}{pushed
such a cut on Twitter} in March, before lawmakers agreed on the first of
what would be several economic rescue packages passed this year:

\begin{quote}
Hoping to get the payroll tax cut approved by both Republicans and
Democrats, and please remember, very important for all countries \&
businesses to know that trade will in no way be affected by the 30-day
restriction on travel from Europe. The restriction stops people not
goods.

--- Donald J. Trump (@realDonaldTrump)
\href{https://twitter.com/realDonaldTrump/status/1237924658185469954?ref_src=twsrc\%5Etfw}{March
12, 2020}
\end{quote}

This week, as congressional Republicans prepared to introduce a bill
that would serve as their opening bid for negotiations over the next
major economic bill, Mr. Trump told reporters that a payroll tax cut
remained ``very important'' to the discussions.

``It's very good,'' Mr. Trump said. ``It's been proven to be successful.
It's a big saving for the people. It's a tremendous saving, and I think
it's an incentive for companies to hire their workers back and to keep
their workers.''

Most economists, even conservative ones, do not rank a payroll tax cut
anywhere close to the top of their list for best ways to support and
stimulate the American economy as it struggles to climb out of the
recession. They say it will cost a lot in lost tax revenues, while doing
little to induce hiring --- and excluding millions of unemployed workers
from its benefits. A broad cut would heavily benefit people who still
have jobs and are earning six-figure salaries, which is not the group
that is most in need of federal support right now. And it would cross
into the political danger zone around funding for a pair of safety net
programs that remain highly popular with the American public.

\hypertarget{latest-updates-economy}{%
\section{\texorpdfstring{\href{https://www.nytimes.com/live/2020/08/04/business/stock-market-today-coronavirus?action=click\&pgtype=Article\&state=default\&region=MAIN_CONTENT_1\&context=storylines_live_updates}{Latest
Updates:
Economy}}{Latest Updates: Economy}}\label{latest-updates-economy}}

\href{https://www.nytimes.com/live/2020/08/04/business/stock-market-today-coronavirus?action=click\&pgtype=Article\&state=default\&region=MAIN_CONTENT_1\&context=storylines_live_updates\#nbcuniversal-to-cut-about-10-percent-of-its-work-force}{37m
ago}

\href{https://www.nytimes.com/live/2020/08/04/business/stock-market-today-coronavirus?action=click\&pgtype=Article\&state=default\&region=MAIN_CONTENT_1\&context=storylines_live_updates\#nbcuniversal-to-cut-about-10-percent-of-its-work-force}{NBCUniversal
to cut about 10 percent of its work force.}

\href{https://www.nytimes.com/live/2020/08/04/business/stock-market-today-coronavirus?action=click\&pgtype=Article\&state=default\&region=MAIN_CONTENT_1\&context=storylines_live_updates\#loans-are-harder-to-get-even-as-interest-rates-are-low}{2h
ago}

\href{https://www.nytimes.com/live/2020/08/04/business/stock-market-today-coronavirus?action=click\&pgtype=Article\&state=default\&region=MAIN_CONTENT_1\&context=storylines_live_updates\#loans-are-harder-to-get-even-as-interest-rates-are-low}{Loans
are harder to get, even as interest rates are low.}

\href{https://www.nytimes.com/live/2020/08/04/business/stock-market-today-coronavirus?action=click\&pgtype=Article\&state=default\&region=MAIN_CONTENT_1\&context=storylines_live_updates\#black-owned-businesses-face-a-double-blow-as-the-pandemic-strikes-minority-communities}{4h
ago}

\href{https://www.nytimes.com/live/2020/08/04/business/stock-market-today-coronavirus?action=click\&pgtype=Article\&state=default\&region=MAIN_CONTENT_1\&context=storylines_live_updates\#black-owned-businesses-face-a-double-blow-as-the-pandemic-strikes-minority-communities}{Black-owned
businesses face a double blow as the pandemic strikes minority
communities.}

\href{https://www.nytimes.com/live/2020/08/04/business/stock-market-today-coronavirus?action=click\&pgtype=Article\&state=default\&region=MAIN_CONTENT_1\&context=storylines_live_updates}{See
more updates}

More live coverage:
\href{https://www.nytimes.com/2020/08/04/world/coronavirus-cases.html?action=click\&pgtype=Article\&state=default\&region=MAIN_CONTENT_1\&context=storylines_live_updates}{Global}

Bowing to political reality, the Treasury secretary, Steven Mnuchin,
said Thursday that the first draft of the next rescue package in the
Senate would not include a payroll tax cut. But Mr. Trump remains
enamored with the plan and is expected to continue lobbying for one,
particularly if he wins a second term.

Here's what the fuss is all about.

\hypertarget{its-a-tax-cut-for-workers-and-employers}{%
\subsection{It's a tax cut for workers and
employers}\label{its-a-tax-cut-for-workers-and-employers}}

Currently, the federal government imposes a 15.3 percent tax on workers'
wages, which is split evenly between employees and employers. Most of it
goes to fund Social Security. Only earnings below \$137,700 are subject
to the part of the tax that supports Social Security; all earnings are
subject to the part of the tax that funds Medicare. Mr. Trump has
proposed suspending the entirety of the payroll tax through the end of
the year.

Congress has already passed a bill this year that delays --- but does
not eliminate --- the employer side of those taxes, meaning companies
will not have to start paying their liabilities for this year until next
year. It is possible that lawmakers could vote to do the same with the
employee side in their next bill, but that would not be the true ``cut''
that Mr. Trump is interested in.

Actually suspending the taxes would cost the government about \$400
billion from August through the end of the year, according to estimates
from the Committee for a Responsible Federal Budget in Washington.

\hypertarget{trumps-favorite-economist-loves-it}{%
\subsection{Trump's favorite economist loves
it}\label{trumps-favorite-economist-loves-it}}

The biggest champion of the cut outside the White House has been Arthur
B. Laffer, the famed supply-side economist whom Mr. Trump
\href{https://www.nytimes.com/2019/06/19/us/politics/arthur-laffer-medal-of-freedom.html}{honored
with the Presidential Medal of Freedom} last year. Mr. Laffer touted the
cut and its benefits this week in a call with Mr. Trump and
congressional Republican leaders who had gathered in the Oval Office.

Mr. Laffer's acolytes, including conservative activists Stephen Moore
and Steve Forbes, have also pushed Mr. Trump to cut payroll taxes. Their
argument is that by reducing the cost of employing someone, and
increasing the amount of money workers take home, the cut will make both
hiring and job-seeking more attractive.

\hypertarget{few-other-economists-support-the-idea-nor-do-business-leaders-or-congressional-republicans}{%
\subsection{Few other economists support the idea, nor do business
leaders or congressional
Republicans}\label{few-other-economists-support-the-idea-nor-do-business-leaders-or-congressional-republicans}}

Other economists point out that the shift in incentives from a temporary
tax cut would be weak, at best. Employers would still have to factor in
the cost of paying the tax starting in January, which is when workers
would have to expect their take-home pay would shrink.

More important, cutting payroll taxes won't do much for laid-off workers
who have few prospects at a time when 30 million Americans are
unemployed. Economists have warned since March that such a move would
not help those workers.

``They were a bad idea then,'' Josh Bivens, director of research at the
liberal Economic Policy Institute, who
\href{https://www.epi.org/blog/employer-tax-credits-can-be-part-of-the-economic-response-to-covid-19-if-they-finance-direct-benefits-for-workers/}{first
wrote in opposition}to Mr. Trump's payroll tax proposals in the spring,
wrote in a direct message on Twitter this week, ``but, since then we've
lost 14 million-plus jobs, and so we now have 14 million fewer people
who would benefit now from a payroll tax cut than would've back in
March.'' It is, he said this week, ``a bad idea that has aged
terribly.''

Business groups also have shown little enthusiasm for the plan. ``We can
provide more targeted assistance both to employers, to help maintain
employment, but also to individuals who are unemployed,'' Neil Bradley,
the executive vice president and chief policy officer of the U.S.
Chamber of Commerce, told reporters last week. ``It is not an issue that
we heard from businesses or state and local chambers as a priority that
would help during this time.''

Mr. Trump had been pushing Republicans to include the tax cut in the
bill they will introduce to kick off the next round of negotiations with
Democrats. But his own party has balked.

Senator Charles E. Grassley of Iowa, the chairman of the Senate Finance
Committee, pushed back against the idea in a private Senate Republican
lunch attended by top administration officials on Tuesday, arguing that
direct payments to families, which would probably be sent out in
October, would prove more meaningful to individual voters.

Mr. Grassley expressed a similar sentiment to reporters this week. ``I
think that's going to do more economic good than if we dribble out \$30
every paycheck,'' he said. ``Because people are going to notice it, take
some action as a result.''

Emily Cochrane contributed reporting.

Advertisement

\protect\hyperlink{after-bottom}{Continue reading the main story}

\hypertarget{site-index}{%
\subsection{Site Index}\label{site-index}}

\hypertarget{site-information-navigation}{%
\subsection{Site Information
Navigation}\label{site-information-navigation}}

\begin{itemize}
\tightlist
\item
  \href{https://help.nytimes.com/hc/en-us/articles/115014792127-Copyright-notice}{©~2020~The
  New York Times Company}
\end{itemize}

\begin{itemize}
\tightlist
\item
  \href{https://www.nytco.com/}{NYTCo}
\item
  \href{https://help.nytimes.com/hc/en-us/articles/115015385887-Contact-Us}{Contact
  Us}
\item
  \href{https://www.nytco.com/careers/}{Work with us}
\item
  \href{https://nytmediakit.com/}{Advertise}
\item
  \href{http://www.tbrandstudio.com/}{T Brand Studio}
\item
  \href{https://www.nytimes.com/privacy/cookie-policy\#how-do-i-manage-trackers}{Your
  Ad Choices}
\item
  \href{https://www.nytimes.com/privacy}{Privacy}
\item
  \href{https://help.nytimes.com/hc/en-us/articles/115014893428-Terms-of-service}{Terms
  of Service}
\item
  \href{https://help.nytimes.com/hc/en-us/articles/115014893968-Terms-of-sale}{Terms
  of Sale}
\item
  \href{https://spiderbites.nytimes.com}{Site Map}
\item
  \href{https://help.nytimes.com/hc/en-us}{Help}
\item
  \href{https://www.nytimes.com/subscription?campaignId=37WXW}{Subscriptions}
\end{itemize}
