Sections

SEARCH

\protect\hyperlink{site-content}{Skip to
content}\protect\hyperlink{site-index}{Skip to site index}

\href{https://www.nytimes.com/section/nyregion}{New York}

\href{https://myaccount.nytimes.com/auth/login?response_type=cookie\&client_id=vi}{}

\href{https://www.nytimes.com/section/todayspaper}{Today's Paper}

\href{/section/nyregion}{New York}\textbar{}Blame Spreads for Nursing
Home Deaths Even as N.Y. Contains Virus

\url{https://nyti.ms/30FrAbi}

\begin{itemize}
\item
\item
\item
\item
\item
\item
\end{itemize}

\href{https://www.nytimes.com/news-event/coronavirus?action=click\&pgtype=Article\&state=default\&region=TOP_BANNER\&context=storylines_menu}{The
Coronavirus Outbreak}

\begin{itemize}
\tightlist
\item
  live\href{https://www.nytimes.com/2020/08/04/world/coronavirus-covid-19.html?action=click\&pgtype=Article\&state=default\&region=TOP_BANNER\&context=storylines_menu}{Latest
  Updates}
\item
  \href{https://www.nytimes.com/interactive/2020/us/coronavirus-us-cases.html?action=click\&pgtype=Article\&state=default\&region=TOP_BANNER\&context=storylines_menu}{Maps
  and Cases}
\item
  \href{https://www.nytimes.com/interactive/2020/science/coronavirus-vaccine-tracker.html?action=click\&pgtype=Article\&state=default\&region=TOP_BANNER\&context=storylines_menu}{Vaccine
  Tracker}
\item
  \href{https://www.nytimes.com/2020/08/02/us/covid-college-reopening.html?action=click\&pgtype=Article\&state=default\&region=TOP_BANNER\&context=storylines_menu}{College
  Reopening}
\item
  \href{https://www.nytimes.com/live/2020/08/03/business/stock-market-today-coronavirus?action=click\&pgtype=Article\&state=default\&region=TOP_BANNER\&context=storylines_menu}{Economy}
\end{itemize}

Advertisement

\protect\hyperlink{after-top}{Continue reading the main story}

Supported by

\protect\hyperlink{after-sponsor}{Continue reading the main story}

\hypertarget{blame-spreads-for-nursing-home-deaths-even-as-ny-contains-virus}{%
\section{Blame Spreads for Nursing Home Deaths Even as N.Y. Contains
Virus}\label{blame-spreads-for-nursing-home-deaths-even-as-ny-contains-virus}}

With more than 6,000 nursing home residents dying of the coronavirus, a
fight over whether relatives should be allowed to sue has erupted in
Albany.

\includegraphics{https://static01.nyt.com/images/2020/07/23/nyregion/23nyvirus-nursinghomes3/merlin_171748677_31d47a34-cda4-4db8-92e2-3dc17e0f1464-articleLarge.jpg?quality=75\&auto=webp\&disable=upscale}

\href{https://www.nytimes.com/by/jesse-mckinley}{\includegraphics{https://static01.nyt.com/images/2018/02/20/multimedia/author-jesse-mckinley/author-jesse-mckinley-thumbLarge.jpg}}\href{https://www.nytimes.com/by/luis-ferre-sadurni}{\includegraphics{https://static01.nyt.com/images/2018/06/22/multimedia/author-luis-ferre-sadurni/author-luis-ferre-sadurni-thumbLarge.png}}

By \href{https://www.nytimes.com/by/jesse-mckinley}{Jesse McKinley} and
\href{https://www.nytimes.com/by/luis-ferre-sadurni}{Luis Ferré-Sadurní}

\begin{itemize}
\item
  July 23, 2020
\item
  \begin{itemize}
  \item
  \item
  \item
  \item
  \item
  \item
  \end{itemize}
\end{itemize}

ALBANY --- As New York moves from coronavirus crisis to sustained
recovery, there remains a heartbreaking fact that some are trying to
explore and others seem to be trying to exploit:
\href{https://www.nytimes.com/interactive/2020/us/coronavirus-nursing-homes.html}{Nearly
6,500 people have died}of the virus in nursing homes and other long-term
facilities in the state.

Republicans in Washington and elsewhere have attacked Gov. Andrew M.
Cuomo for his role in, and response to, those deaths; Mr. Cuomo has
returned fire, accusing his foes of politicizing a human tragedy and
arguing that the blame for the number of deaths lay with infected health
care workers, not his own policies.

The death toll --- a figure that surpasses that of
\href{https://www.cdc.gov/covid-data-tracker/\#cases}{many states} ---
has also inspired questions from Mr. Cuomo's fellow Democrats, who rule
the State Legislature and have scheduled hearings on the issue next
month.

The tension and pain surrounding the topic have bled into the debate
over a related
\href{https://www.nysenate.gov/legislation/bills/2019/S8835}{bill} that
is expected to be passed on Thursday by the Legislature.

The initial goal of the bill's supporters was to void
\href{https://www.nytimes.com/2020/05/13/nyregion/nursing-homes-coronavirus-new-york.html}{a
last-minute provision}, buried into the state budget just before it was
passed in early April, that gave nursing homes and hospitals broad
immunity from lawsuits stemming from their failure to protect residents
from death or sickness caused by the coronavirus.

But after considerable pushback from the hospital and nursing home
industries, and legal questions about its scope, the legislation that's
being advanced is far weaker, with the immunity merely narrowed.

``This is just a first step,'' said the bill's lead sponsor, Assemblyman
Ron Kim, of Queens, where
\href{https://www.health.ny.gov/statistics/diseases/covid-19/fatalities_nursing_home_acf.pdf}{nearly
1,000 nursing homes residents died}. ``We're coming back after the
hearings to see how we can provide retroactive justice for anyone who
feels like they've been wronged.''

The fight over the bill highlights how fraught the issue has become for
Mr. Cuomo and other politicians, as well as nursing home executives, and
how sensitive all involved have been to suggestions that they are to
blame for the deaths.

Mr. Cuomo, a third-term Democrat, has pushed back aggressively on
assertions that his administration's directive of March 25, which
effectively ordered nursing homes to accept coronavirus-positive
patients from hospitals, led to more deaths.

In response to that criticism, the State Health Department released
\href{https://health.ny.gov/press/releases/2020/docs/nh_factors_report.pdf}{a
report} that essentially absolved the state of any blame; the report
concluded that ``most patients admitted to nursing homes from hospitals
were no longer contagious when admitted and therefore were not a source
of infection.''

\hypertarget{latest-updates-global-coronavirus-outbreak}{%
\section{\texorpdfstring{\href{https://www.nytimes.com/2020/08/04/world/coronavirus-covid-19.html?action=click\&pgtype=Article\&state=default\&region=MAIN_CONTENT_1\&context=storylines_live_updates}{Latest
Updates: Global Coronavirus
Outbreak}}{Latest Updates: Global Coronavirus Outbreak}}\label{latest-updates-global-coronavirus-outbreak}}

Updated 2020-08-04T10:03:05.885Z

\begin{itemize}
\tightlist
\item
  \href{https://www.nytimes.com/2020/08/04/world/coronavirus-covid-19.html?action=click\&pgtype=Article\&state=default\&region=MAIN_CONTENT_1\&context=storylines_live_updates\#link-6b644638}{`Long
  days, long nights': Washington prepares for a prolonged fight over
  virus relief.}
\item
  \href{https://www.nytimes.com/2020/08/04/world/coronavirus-covid-19.html?action=click\&pgtype=Article\&state=default\&region=MAIN_CONTENT_1\&context=storylines_live_updates\#link-7af9fca0}{Israel's
  rocky reopening of its schools may be a lesson for the U.S.}
\item
  \href{https://www.nytimes.com/2020/08/04/world/coronavirus-covid-19.html?action=click\&pgtype=Article\&state=default\&region=MAIN_CONTENT_1\&context=storylines_live_updates\#link-33bf9168}{Hurricane
  Isaias arrives in North Carolina as officials along the East Coast
  scramble.}
\end{itemize}

\href{https://www.nytimes.com/2020/08/04/world/coronavirus-covid-19.html?action=click\&pgtype=Article\&state=default\&region=MAIN_CONTENT_1\&context=storylines_live_updates}{See
more updates}

More live coverage:
\href{https://www.nytimes.com/live/2020/08/03/business/stock-market-today-coronavirus?action=click\&pgtype=Article\&state=default\&region=MAIN_CONTENT_1\&context=storylines_live_updates}{Markets}

The report said that the disease was spread by ``thousands of
employees'' who had the disease and did not know they were contagious.

\includegraphics{https://static01.nyt.com/images/2020/07/23/nyregion/23nyvirus-nursinghomes2/merlin_172110852_3a05975d-0ad7-4cfb-a817-3c4656801a6b-articleLarge.jpg?quality=75\&auto=webp\&disable=upscale}

Those conclusions did little to stem Republican criticism that the
governor was to blame for thousands of deaths, allegations carried on
Twitter and marked by hashtags like
\href{https://twitter.com/search?q=\%23KillerCuomo}{\#KillerCuomo}.

\href{https://republicans-oversight.house.gov/wp-content/uploads/2020/07/letter-to-Cuomo-about-NYSDOH-report-2.pdf}{In
a letter} sent to Mr. Cuomo in July, Representative Steve Scalise of
Louisiana, the Republican minority whip who is the ranking member of a
House subcommittee on the coronavirus crisis, said the state report was
filled with ``blame-shifting, name-calling and half-baked data
manipulations.''

Mr. Cuomo said his political enemies were spreading lies and
``ludicrous'' accusations.

``It was cheap, it was ugly, it was political, it was Fox News, it was
the haters and it was a lie,'' Mr. Cuomo said in
\href{https://soundcloud.com/nygovcuomo/governor-cuomo-is-a-guest-on-5}{a
July 10 radio interview}, a day after Mr. Scalise sent his letter.
``It's just a pure lie, not based on any fact, but they did it for
political expediency.''

For the families of people who died in nursing homes and assisted living
facilities in New York, the political friction overshadowed the
importance of seeking accountability for deaths that they believe were
preventable.

Many were outraged when the state included the immunity provision in the
budget. Advocates called it among the most restrictive protections
against lawsuits in the country.

``Having liability can cause a facility to be more diligent and prevent
incidents occurring that will cost them money,'' said Susan M. Dooha,
the executive director of the Center for Independence of the Disabled.
``The preventive power of liability has been muted.''

Under the budget provision, health care facilities and their employees
were protected from civil or criminal liability for the duration of the
coronavirus emergency, which Mr. Cuomo
\href{https://www.nytimes.com/2020/03/07/nyregion/coronavirus-new-york-queens.html?searchResultPosition=1}{declared
on March 7} and is still in force.

The legal immunity did not cover gross negligence or intentional
criminal misconduct, but would most likely cover a variety of other
scenarios, including harm that arose from a shortage in staffing or
protective equipment.

The provision was fought for and celebrated by industry lobbyists like
the Greater New York Hospital Association, which has close ties to
Governor Cuomo and has given hundreds of thousands of dollars to
Democratic campaign committees in Albany in recent months (as well as
lesser amounts to committees for Republicans, who sit in the minority in
both legislative chambers).

Under Mr. Kim's bill, that immunity would be modified to allow legal
action if it could be argued that a health care facility or health care
professional had failed to prevent a patient from contracting the
coronavirus, or had not tried to safeguard them from infection.

``We now know how to prevent and arrange for Covid,'' said Mr. Kim, who
is a Democrat. ``So we will be able to hold nursing homes accountable.''

Image

Assemblyman Ron Kim, right, at a nursing home in Queens, which has
suffered nearly 1,000 coronavirus-related deaths at nursing
homes.Credit...Hilary Swift for The New York Times

The bill will also specify that the immunity clause will ``only apply to
Covid-19 related care and treatment.'' That will restore a path for
medical malpractice suits unrelated to Covid-19, said Richard Gottfried,
the chairman of the Assembly Health Committee.

\href{https://www.nytimes.com/news-event/coronavirus?action=click\&pgtype=Article\&state=default\&region=MAIN_CONTENT_3\&context=storylines_faq}{}

\hypertarget{the-coronavirus-outbreak-}{%
\subsubsection{The Coronavirus Outbreak
›}\label{the-coronavirus-outbreak-}}

\hypertarget{frequently-asked-questions}{%
\paragraph{Frequently Asked
Questions}\label{frequently-asked-questions}}

Updated August 3, 2020

\begin{itemize}
\item ~
  \hypertarget{im-a-small-business-owner-can-i-get-relief}{%
  \paragraph{I'm a small-business owner. Can I get
  relief?}\label{im-a-small-business-owner-can-i-get-relief}}

  \begin{itemize}
  \tightlist
  \item
    The
    \href{https://www.nytimes.com/article/small-business-loans-stimulus-grants-freelancers-coronavirus.html?action=click\&pgtype=Article\&state=default\&region=MAIN_CONTENT_3\&context=storylines_faq}{stimulus
    bills enacted in March} offer help for the millions of American
    small businesses. Those eligible for aid are businesses and
    nonprofit organizations with fewer than 500 workers, including sole
    proprietorships, independent contractors and freelancers. Some
    larger companies in some industries are also eligible. The help
    being offered, which is being managed by the Small Business
    Administration, includes the Paycheck Protection Program and the
    Economic Injury Disaster Loan program. But lots of folks have
    \href{https://www.nytimes.com/interactive/2020/05/07/business/small-business-loans-coronavirus.html?action=click\&pgtype=Article\&state=default\&region=MAIN_CONTENT_3\&context=storylines_faq}{not
    yet seen payouts.} Even those who have received help are confused:
    The rules are draconian, and some are stuck sitting on
    \href{https://www.nytimes.com/2020/05/02/business/economy/loans-coronavirus-small-business.html?action=click\&pgtype=Article\&state=default\&region=MAIN_CONTENT_3\&context=storylines_faq}{money
    they don't know how to use.} Many small-business owners are getting
    less than they expected or
    \href{https://www.nytimes.com/2020/06/10/business/Small-business-loans-ppp.html?action=click\&pgtype=Article\&state=default\&region=MAIN_CONTENT_3\&context=storylines_faq}{not
    hearing anything at all.}
  \end{itemize}
\item ~
  \hypertarget{what-are-my-rights-if-i-am-worried-about-going-back-to-work}{%
  \paragraph{What are my rights if I am worried about going back to
  work?}\label{what-are-my-rights-if-i-am-worried-about-going-back-to-work}}

  \begin{itemize}
  \tightlist
  \item
    Employers have to provide
    \href{https://www.osha.gov/SLTC/covid-19/standards.html}{a safe
    workplace} with policies that protect everyone equally.
    \href{https://www.nytimes.com/article/coronavirus-money-unemployment.html?action=click\&pgtype=Article\&state=default\&region=MAIN_CONTENT_3\&context=storylines_faq}{And
    if one of your co-workers tests positive for the coronavirus, the
    C.D.C.} has said that
    \href{https://www.cdc.gov/coronavirus/2019-ncov/community/guidance-business-response.html}{employers
    should tell their employees} -\/- without giving you the sick
    employee's name -\/- that they may have been exposed to the virus.
  \end{itemize}
\item ~
  \hypertarget{should-i-refinance-my-mortgage}{%
  \paragraph{Should I refinance my
  mortgage?}\label{should-i-refinance-my-mortgage}}

  \begin{itemize}
  \tightlist
  \item
    \href{https://www.nytimes.com/article/coronavirus-money-unemployment.html?action=click\&pgtype=Article\&state=default\&region=MAIN_CONTENT_3\&context=storylines_faq}{It
    could be a good idea,} because mortgage rates have
    \href{https://www.nytimes.com/2020/07/16/business/mortgage-rates-below-3-percent.html?action=click\&pgtype=Article\&state=default\&region=MAIN_CONTENT_3\&context=storylines_faq}{never
    been lower.} Refinancing requests have pushed mortgage applications
    to some of the highest levels since 2008, so be prepared to get in
    line. But defaults are also up, so if you're thinking about buying a
    home, be aware that some lenders have tightened their standards.
  \end{itemize}
\item ~
  \hypertarget{what-is-school-going-to-look-like-in-september}{%
  \paragraph{What is school going to look like in
  September?}\label{what-is-school-going-to-look-like-in-september}}

  \begin{itemize}
  \tightlist
  \item
    It is unlikely that many schools will return to a normal schedule
    this fall, requiring the grind of
    \href{https://www.nytimes.com/2020/06/05/us/coronavirus-education-lost-learning.html?action=click\&pgtype=Article\&state=default\&region=MAIN_CONTENT_3\&context=storylines_faq}{online
    learning},
    \href{https://www.nytimes.com/2020/05/29/us/coronavirus-child-care-centers.html?action=click\&pgtype=Article\&state=default\&region=MAIN_CONTENT_3\&context=storylines_faq}{makeshift
    child care} and
    \href{https://www.nytimes.com/2020/06/03/business/economy/coronavirus-working-women.html?action=click\&pgtype=Article\&state=default\&region=MAIN_CONTENT_3\&context=storylines_faq}{stunted
    workdays} to continue. California's two largest public school
    districts --- Los Angeles and San Diego --- said on July 13, that
    \href{https://www.nytimes.com/2020/07/13/us/lausd-san-diego-school-reopening.html?action=click\&pgtype=Article\&state=default\&region=MAIN_CONTENT_3\&context=storylines_faq}{instruction
    will be remote-only in the fall}, citing concerns that surging
    coronavirus infections in their areas pose too dire a risk for
    students and teachers. Together, the two districts enroll some
    825,000 students. They are the largest in the country so far to
    abandon plans for even a partial physical return to classrooms when
    they reopen in August. For other districts, the solution won't be an
    all-or-nothing approach.
    \href{https://bioethics.jhu.edu/research-and-outreach/projects/eschool-initiative/school-policy-tracker/}{Many
    systems}, including the nation's largest, New York City, are
    devising
    \href{https://www.nytimes.com/2020/06/26/us/coronavirus-schools-reopen-fall.html?action=click\&pgtype=Article\&state=default\&region=MAIN_CONTENT_3\&context=storylines_faq}{hybrid
    plans} that involve spending some days in classrooms and other days
    online. There's no national policy on this yet, so check with your
    municipal school system regularly to see what is happening in your
    community.
  \end{itemize}
\item ~
  \hypertarget{is-the-coronavirus-airborne}{%
  \paragraph{Is the coronavirus
  airborne?}\label{is-the-coronavirus-airborne}}

  \begin{itemize}
  \tightlist
  \item
    The coronavirus
    \href{https://www.nytimes.com/2020/07/04/health/239-experts-with-one-big-claim-the-coronavirus-is-airborne.html?action=click\&pgtype=Article\&state=default\&region=MAIN_CONTENT_3\&context=storylines_faq}{can
    stay aloft for hours in tiny droplets in stagnant air}, infecting
    people as they inhale, mounting scientific evidence suggests. This
    risk is highest in crowded indoor spaces with poor ventilation, and
    may help explain super-spreading events reported in meatpacking
    plants, churches and restaurants.
    \href{https://www.nytimes.com/2020/07/06/health/coronavirus-airborne-aerosols.html?action=click\&pgtype=Article\&state=default\&region=MAIN_CONTENT_3\&context=storylines_faq}{It's
    unclear how often the virus is spread} via these tiny droplets, or
    aerosols, compared with larger droplets that are expelled when a
    sick person coughs or sneezes, or transmitted through contact with
    contaminated surfaces, said Linsey Marr, an aerosol expert at
    Virginia Tech. Aerosols are released even when a person without
    symptoms exhales, talks or sings, according to Dr. Marr and more
    than 200 other experts, who
    \href{https://academic.oup.com/cid/article/doi/10.1093/cid/ciaa939/5867798}{have
    outlined the evidence in an open letter to the World Health
    Organization}.
  \end{itemize}
\end{itemize}

But opponents quickly pointed to what they consider shortcomings of the
legislation, including a stipulation that the bill would only affect
future cases and would not be applied retroactively --- a clause that
they said would hurt those most affected.

``The overreliance, the overacceptance of the industry's lobbying
efforts, and the credulity that we give to the arguments they make is
what led to a large extent to tragedies for families across the state,''
said Richard Mollot, executive director of the Long Term Care Community
Coalition.

Nursing home facilities were encouraged to reach out to lawmakers to
voice their concerns, and the Greater New York Hospital Association sent
legislators a two-page memo on Tuesday outlining its opposition to any
legislative action meant to scale back immunity.

``New York hospitals are deeply committed to caring for New Yorkers to
the best of their abilities at all times, including in a possible
resurgence of Covid-19,'' the memo said. ``The Legislature should not
take actions to undercut or chill that commitment.''

Industry groups argued that repealing immunity could pose constitutional
issues and open the floodgates to a barrage of retroactive lawsuits
against health care facilities. They said the legislation did not take
into account the broad impact a resurgence of the virus could have on
the way hospitals and nursing homes care for all patients and residents,
not just those ill with the coronavirus.

A surge in cases, for example, could lead to a shortage of health care
workers and strain the supply of personal protective equipment, both of
which could affect the care provided to those not infected with the
virus, they said.

``If the pandemic comes back in full force like it was in April, the
pressures that come up at nursing homes don't affect only people with
Covid, they affect all the residents that you're serving,'' said Jim
Clyne, chief executive of
\href{https://www.leadingageny.org/}{LeadingAge New York}, a group that
represents nonprofit nursing homes.

Campaign finance records show that since mid-March, the Greater New York
Hospital Association has given more than \$300,000 to campaign
committees controlled by the Democrats --- who rule both chambers of the
Legislature --- and Republicans. That haul includes \$150,000 to the New
York State Democratic Assembly Democratic Committee and \$100,000 to
their Senate Democratic counterparts.

Both Senate and Assembly spokesmen noted, however, the hospital
association had opposed Mr. Kim's bill, which is sponsored by Senator
Luis Sepulveda in the Senate.

``Contributions do not influence our positions nor will they ever,''
said Michael Whyland, a spokesman for the Assembly.

On Wednesday, Mr. Cuomo said he could ``see the rationale'' for allowing
lawsuits on cases that are not coronavirus-related, but said he wanted
to review the entire bill before giving an opinion.

Mr. Kim said he hopes that the Legislature approves a victims
compensation fund at some later date for families of nursing home
victims, a sentiment shared by Mr. Gottfried, who said he still supports
a full repeal of the immunity provision.

But like determinations on what went wrong in the state's nursing homes,
Mr. Gottfried suggested that sometimes a full reckoning on an issue can
take time.

``Half a loaf is better than none,'' Mr. Gottfried said.

Advertisement

\protect\hyperlink{after-bottom}{Continue reading the main story}

\hypertarget{site-index}{%
\subsection{Site Index}\label{site-index}}

\hypertarget{site-information-navigation}{%
\subsection{Site Information
Navigation}\label{site-information-navigation}}

\begin{itemize}
\tightlist
\item
  \href{https://help.nytimes.com/hc/en-us/articles/115014792127-Copyright-notice}{©~2020~The
  New York Times Company}
\end{itemize}

\begin{itemize}
\tightlist
\item
  \href{https://www.nytco.com/}{NYTCo}
\item
  \href{https://help.nytimes.com/hc/en-us/articles/115015385887-Contact-Us}{Contact
  Us}
\item
  \href{https://www.nytco.com/careers/}{Work with us}
\item
  \href{https://nytmediakit.com/}{Advertise}
\item
  \href{http://www.tbrandstudio.com/}{T Brand Studio}
\item
  \href{https://www.nytimes.com/privacy/cookie-policy\#how-do-i-manage-trackers}{Your
  Ad Choices}
\item
  \href{https://www.nytimes.com/privacy}{Privacy}
\item
  \href{https://help.nytimes.com/hc/en-us/articles/115014893428-Terms-of-service}{Terms
  of Service}
\item
  \href{https://help.nytimes.com/hc/en-us/articles/115014893968-Terms-of-sale}{Terms
  of Sale}
\item
  \href{https://spiderbites.nytimes.com}{Site Map}
\item
  \href{https://help.nytimes.com/hc/en-us}{Help}
\item
  \href{https://www.nytimes.com/subscription?campaignId=37WXW}{Subscriptions}
\end{itemize}
