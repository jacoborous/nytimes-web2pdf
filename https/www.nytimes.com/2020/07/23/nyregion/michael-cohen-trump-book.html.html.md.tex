Sections

SEARCH

\protect\hyperlink{site-content}{Skip to
content}\protect\hyperlink{site-index}{Skip to site index}

\href{https://www.nytimes.com/section/nyregion}{New York}

\href{https://myaccount.nytimes.com/auth/login?response_type=cookie\&client_id=vi}{}

\href{https://www.nytimes.com/section/todayspaper}{Today's Paper}

\href{/section/nyregion}{New York}\textbar{}Judge Orders Cohen Released,
Citing `Retaliation' Over Tell-All Book

\url{https://nyti.ms/2ZU40II}

\begin{itemize}
\item
\item
\item
\item
\item
\item
\end{itemize}

Advertisement

\protect\hyperlink{after-top}{Continue reading the main story}

Supported by

\protect\hyperlink{after-sponsor}{Continue reading the main story}

\hypertarget{judge-orders-cohen-released-citing-retaliation-over-tell-all-book}{%
\section{Judge Orders Cohen Released, Citing `Retaliation' Over Tell-All
Book}\label{judge-orders-cohen-released-citing-retaliation-over-tell-all-book}}

A judge agreed that federal officials had returned Michael D. Cohen to
prison because he wanted to publish a book this fall about President
Trump.

\includegraphics{https://static01.nyt.com/images/2020/07/23/nyregion/23Cohen/merlin_174862884_19ebc8d2-78af-491c-8be3-1013fe8e4c3f-articleLarge.jpg?quality=75\&auto=webp\&disable=upscale}

By \href{https://www.nytimes.com/by/benjamin-weiser}{Benjamin Weiser}
and \href{https://www.nytimes.com/by/alan-feuer}{Alan Feuer}

\begin{itemize}
\item
  July 23, 2020
\item
  \begin{itemize}
  \item
  \item
  \item
  \item
  \item
  \item
  \end{itemize}
\end{itemize}

When Michael D. Cohen,
\href{https://www.nytimes.com/2020/07/23/us/politics/person-woman-man-camera-tv-trump.html}{President
Trump's} one-time lawyer and fixer, met with probation officers this
month to complete paperwork that would have let him serve the balance of
his prison term at home, he found a catch.

Mr. Cohen was already out on furlough because of the coronavirus. But to
remain at home, he was asked to sign a document that would have barred
him from publishing a book during the rest of his sentence. Mr. Cohen
balked because he was, in fact, writing a book --- a tell-all memoir
about his former boss, the president.

The officers sent him back to prison.

On Thursday, a federal judge ruled that the decision to return Mr. Cohen
to custody amounted to retaliation by the government and ordered him to
be released again into home confinement. Mr. Cohen is expected back in
his Manhattan apartment on Friday.

``I make the finding that the purpose of transferring Mr. Cohen from
furlough and home confinement to jail is retaliatory,'' the judge, Alvin
K. Hellerstein of Federal District Court in Manhattan, said in court.
``And it's retaliatory because of his desire to exercise his First
Amendment rights to publish a book and to discuss anything about the
book or anything else he wants on social media and with others.''

Judge Hellerstein's decision was a remarkable rebuke of prison and
probation officials and, by extension, the Trump administration. It
raised concerns that the authorities had used the penal system to
squelch the free speech rights of one of Mr. Trump's enemies in an
effort to protect the president.

Justin Long, a spokesman for the Bureau of Prisons, said that it was not
uncommon for prison officials to restrict inmates' contact with the
media. But he said that Mr. Cohen's refusal to agree to the media ban
``played no role whatsoever in the decision to remand him to secure
custody, nor did his intent to publish a book.''

``Any assertion that the decision to remand Michael Cohen to prison was
a retaliatory action is patently false,'' Mr. Long said.

The question of Mr. Cohen's release came before Judge Hellerstein after
Mr. Cohen
\href{https://www.nytimes.com/2020/07/21/nyregion/michael-cohen-trump-book.html?searchResultPosition=1}{sued
U.S. officials on Monday night}, claiming that the Trump administration
had sent him back to custody to prevent him from completing the book,
violating his freedom of speech.

In court papers, Mr. Cohen said the book would paint Mr. Trump as a
racist and offer revealing details about ``the president's behavior
behind closed doors.''

Mr. Cohen also pointed out that Mr. Trump and his supporters
\href{https://www.nytimes.com/2020/06/20/us/politics/john-bolton-book-ruling.html}{had
sought}
\href{https://www.nytimes.com/2020/07/13/us/politics/mary-trump-book.html}{to
derail} the publication of books written by John R. Bolton, the former
national security adviser, and Mr. Trump's niece, Mary L. Trump, whose
best-selling memoir
\href{https://www.nytimes.com/2020/07/08/books/review/mary-trump-book-takeaways.html}{laid
bare} a history of dysfunction in her family.

E. Danya Perry, one of Mr. Cohen's lawyers, called the judge's order ``a
victory for the First Amendment.''

The court hearing on Thursday was the latest chapter in a long-running
saga. Mr. Cohen, a legal bulldog who once bragged he would take a bullet
for Mr. Trump,
\href{https://slack-redir.net/link?url=https\%3A\%2F\%2Fwww.nytimes.com\%2F2018\%2F08\%2F21\%2Fnyregion\%2Fmichael-cohen-guilty-plea-trump-takeaways.html}{pleaded
guilty in 2018 to campaign finance violations and other crimes} and was
sentenced to three years in prison.

As he entered his plea, Mr. Cohen pointed the finger at the president,
telling the court that Mr. Trump had directed him during the 2016
election to arrange hush money payments to two women who claimed they
had had affairs with Mr. Trump. The president has denied those
allegations.

The provision that Mr. Cohen, 53, objected to would have barred him from
``engagement of any kind with the media, including print, TV, film,
books.'' It also sought to keep him from posting on social media,
according to a copy of the agreement attached to his lawsuit.

Judge Hellerstein, who was appointed to the bench in 1998 by President
Clinton, said these measures seemed to him to be highly unusual and
appeared to be directly related to Mr. Cohen's forthcoming book.

``In 21 years of being a judge and sentencing people and looking at the
terms and conditions of supervised release,'' he said, ``I have never
seen such a clause.''

Both in court papers and during a hearing on Thursday, the government
insisted that the probation officer in Mr. Cohen's case, Adam Pakula,
did not know about the book when he drafted the provisions. The
government has denied the document was ``a gag order'' or that it was
``custom-made'' for Mr. Cohen ``by high levels of the executive
branch.''

Mr. Pakula, in court papers, said he drafted the agreement ``without
input from the B.O.P. or anyone in the executive branch.''

In court, Judge Hellerstein seemed skeptical.

``Why would Pakula ask for something like this unless there was a
purpose to it, unless there was a retaliatory purpose saying, `You toe
the line about giving up your First Amendment rights or we will send you
to jail,''' the judge asked.

Judge Hellerstein also suggested that Mr. Pakula may have gotten some
``instruction'' about including the media ban in the agreement.

The government said in court papers earlier this week that the decision
to send Mr. Cohen back to prison had nothing to do with his book, but
had been made after he became ``combative'' while discussing the
agreement, behavior the officers found ``unacceptable.''

Judge Hellerstein said that such behavior seemed to him to be ``an
attorney's effort to negotiate an agreement, which is very common.''

In May, Mr. Cohen had been allowed to leave a minimum-security prison
camp in Otisville, N.Y., and go home as part of an effort by the Bureau
of Prisons to curb the spread of coronavirus in the federal prison
system.

Mr. Cohen's lawyers had argued that his health conditions, including
severe hypertension and a history of respiratory problems, put him at
risk if he remained in prison.

But on July 9, prison officials abruptly returned Mr. Cohen to
Otisville.

In his suit, Mr. Cohen claimed that he never hid the fact that he was
writing a book about Mr. Trump. He noted that he spent his mornings
working on the manuscript ``in plain sight'' in the prison's law
library, and said he also discussed the project openly with prison
officials, staff members and even other inmates.

According to the suit, the book will give a glimpse into Mr. Cohen's
``firsthand experiences with Mr. Trump'' and offer ``graphic details
about the president's behavior behind closed doors.''

``The narrative,'' the lawsuit says, ``describes pointedly certain
anti-Semitic remarks against prominent Jewish people and virulently
racist remarks against such Black leaders as President Barack Obama and
Nelson Mandela.''

The manuscript --- tentatively titled ``Disloyal: The True Story of
Michael Cohen, Former Personal Attorney to President Donald J. Trump''
--- is only the latest book to have emerged in recent weeks containing
detailed and critical revelations about the president's personal and
professional life.

Mr. Cohen's suit contends that Mr. Trump and his supporters have sought
to derail the publication of his book like the others.

In June, the Justice Department
\href{https://www.nytimes.com/2020/06/16/us/politics/john-bolton-book-publication.html}{asked
a judge to delay the release} of ``The Room Where It Happened,'' a
memoir by Mr. Bolton, the former national security adviser who, among
other things, confirmed accusations at the heart of the Democratic
impeachment case over the president's dealings with Ukraine. The judge
ultimately denied the request.

On the same day that Mr. Bolton's book was published, Mr. Trump's
younger brother, Robert S. Trump, filed
\href{https://www.nytimes.com/2020/06/23/us/politics/mary-trump-book-court.html}{a
suit seeking to stop} the publication of a family tell-all written by
their niece, Mary Trump.

After a few weeks of whirlwind litigation, the judge in that case sided
with Ms. Trump,
\href{https://www.nytimes.com/2020/07/07/us/politics/mary-trump-book.html}{allowing
her to publish her memoir}, which accused Mr. Trump of embracing
cheating ``as a way of life'' and of paying someone to take his college
entrance exams.

Advertisement

\protect\hyperlink{after-bottom}{Continue reading the main story}

\hypertarget{site-index}{%
\subsection{Site Index}\label{site-index}}

\hypertarget{site-information-navigation}{%
\subsection{Site Information
Navigation}\label{site-information-navigation}}

\begin{itemize}
\tightlist
\item
  \href{https://help.nytimes.com/hc/en-us/articles/115014792127-Copyright-notice}{©~2020~The
  New York Times Company}
\end{itemize}

\begin{itemize}
\tightlist
\item
  \href{https://www.nytco.com/}{NYTCo}
\item
  \href{https://help.nytimes.com/hc/en-us/articles/115015385887-Contact-Us}{Contact
  Us}
\item
  \href{https://www.nytco.com/careers/}{Work with us}
\item
  \href{https://nytmediakit.com/}{Advertise}
\item
  \href{http://www.tbrandstudio.com/}{T Brand Studio}
\item
  \href{https://www.nytimes.com/privacy/cookie-policy\#how-do-i-manage-trackers}{Your
  Ad Choices}
\item
  \href{https://www.nytimes.com/privacy}{Privacy}
\item
  \href{https://help.nytimes.com/hc/en-us/articles/115014893428-Terms-of-service}{Terms
  of Service}
\item
  \href{https://help.nytimes.com/hc/en-us/articles/115014893968-Terms-of-sale}{Terms
  of Sale}
\item
  \href{https://spiderbites.nytimes.com}{Site Map}
\item
  \href{https://help.nytimes.com/hc/en-us}{Help}
\item
  \href{https://www.nytimes.com/subscription?campaignId=37WXW}{Subscriptions}
\end{itemize}
