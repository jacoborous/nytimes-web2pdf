Sections

SEARCH

\protect\hyperlink{site-content}{Skip to
content}\protect\hyperlink{site-index}{Skip to site index}

\href{https://www.nytimes.com/section/travel}{Travel}

\href{https://myaccount.nytimes.com/auth/login?response_type=cookie\&client_id=vi}{}

\href{https://www.nytimes.com/section/todayspaper}{Today's Paper}

\href{/section/travel}{Travel}\textbar{}A High-Tech Array of Travel
Tools: `Smart' Health Cards, Temperature-Reading Glasses and More

\url{https://nyti.ms/2CEEvSN}

\begin{itemize}
\item
\item
\item
\item
\item
\end{itemize}

\href{https://www.nytimes.com/news-event/coronavirus?action=click\&pgtype=Article\&state=default\&region=TOP_BANNER\&context=storylines_menu}{The
Coronavirus Outbreak}

\begin{itemize}
\tightlist
\item
  live\href{https://www.nytimes.com/2020/08/01/world/coronavirus-covid-19.html?action=click\&pgtype=Article\&state=default\&region=TOP_BANNER\&context=storylines_menu}{Latest
  Updates}
\item
  \href{https://www.nytimes.com/interactive/2020/us/coronavirus-us-cases.html?action=click\&pgtype=Article\&state=default\&region=TOP_BANNER\&context=storylines_menu}{Maps
  and Cases}
\item
  \href{https://www.nytimes.com/interactive/2020/science/coronavirus-vaccine-tracker.html?action=click\&pgtype=Article\&state=default\&region=TOP_BANNER\&context=storylines_menu}{Vaccine
  Tracker}
\item
  \href{https://www.nytimes.com/interactive/2020/07/29/us/schools-reopening-coronavirus.html?action=click\&pgtype=Article\&state=default\&region=TOP_BANNER\&context=storylines_menu}{What
  School May Look Like}
\item
  \href{https://www.nytimes.com/live/2020/07/31/business/stock-market-today-coronavirus?action=click\&pgtype=Article\&state=default\&region=TOP_BANNER\&context=storylines_menu}{Economy}
\end{itemize}

Advertisement

\protect\hyperlink{after-top}{Continue reading the main story}

Supported by

\protect\hyperlink{after-sponsor}{Continue reading the main story}

\hypertarget{a-high-tech-array-of-travel-tools-smart-health-cards-temperature-reading-glasses-and-more}{%
\section{A High-Tech Array of Travel Tools: `Smart' Health Cards,
Temperature-Reading Glasses and
More}\label{a-high-tech-array-of-travel-tools-smart-health-cards-temperature-reading-glasses-and-more}}

Products that rely on artificial intelligence aim to make travel safer
during the pandemic. But hefty prices and privacy concerns are issues.

\includegraphics{https://static01.nyt.com/images/2020/08/01/travel/23aipassport/23aipassport-articleLarge.jpg?quality=75\&auto=webp\&disable=upscale}

By Debra Kamin

\begin{itemize}
\item
  Published July 23, 2020Updated July 24, 2020
\item
  \begin{itemize}
  \item
  \item
  \item
  \item
  \item
  \end{itemize}
\end{itemize}

Three weeks into Israel's lockdown to help prevent the spread of
coronavirus, Rafi Kaminer decided to get creative with his cabin fever.

The chief executive of \href{https://pangea-it.com/}{Pangea Group}, an
Israeli company that builds infrastructure for biometric identification
and digital analytics, Mr. Kaminer was used to flying abroad several
times a month. But with global air traffic reduced to a trickle and
borders sealed across continents, Mr. Kaminer found himself housebound
--- and itching for a solution. He began brainstorming with his brother,
Assaf Kaminer, an executive vice president at Pangea, and the two came
up with an idea: To get people flying again, invent a streamlined method
of determining that a traveler is free of Covid-19, resulting in a
document that could be presented at any airport in the world, encrypted
for security and customized for the unique testing regulations of each
port of entry.

So, tapping their co-workers and the power of artificial intelligence, a
Pangea team worked to invent it themselves.

The travel industry is well acquainted with artificial intelligence:
customer service chatbots, predictive search engines and automated
check-in services like self-service bag drops are becoming de rigueur.
But with coronavirus now ravaging the industry, programmers and digital
designers are seeing an opportunity to innovate.

\hypertarget{the-covid-19-pass-card}{%
\subsection{The Covid-19 Pass Card}\label{the-covid-19-pass-card}}

In June, Pangea announced its Covid-19 Pass Card. Unlike the documents
being considered in countries like Chile and Germany, which announce
that the holder has recovered from Covid-19, the Pass Card is more like
a digital passport with two parts: a biometric smart card and a
prediction engine that includes a web portal, accessible by computer or
smartphone.

It doesn't measure antibodies or offer proof of immune status. Rather,
the portal delivers customized testing requirements based on departure
and arrival cities, so the cardholders know if they need to get tested
for the virus before their flight or after they land, and for how many
days a test remains valid.

The smart card, which is encrypted and relies on facial and fingerprint
recognition, carries the cardholder's Covid-19 testing data, as well as
the traveler's medical profile and immunization records for other
ailments like yellow fever, measles and hepatitis.

Mr. Kaminer hopes that air passengers will be carrying Pangea's passport
within a matter of months. The company is awaiting approval from
Israel's Ministry of Health to grant the card to Israeli citizens, and
next they will push forward on discussions with port authority
representatives in a handful of U.S. cities, as well as in Johannesburg
and Addis Ababa.

``Corona is not going to leave us for at least the next 12 to 18 months.
So we need a solution,'' Mr. Kaminer said.

There are other applications, too: A universal encoded health-care card
could mean that emergency medical technicians could instantly know if an
unconscious heart attack victim was on blood pressure medication. It
could also mean that a patient entering a hospital not affiliated with
their medical group would still be able to offer doctors instant access
to their medical records.

\includegraphics{https://static01.nyt.com/images/2020/07/26/travel/26travel/oakImage-1595355248452-articleLarge.jpg?quality=75\&auto=webp\&disable=upscale}

\hypertarget{temperature-reading-glasses}{%
\subsection{Temperature-reading
glasses}\label{temperature-reading-glasses}}

The SARS crisis of 2002-2004, which helped drive the expansion of the
online shopping giants Alibaba and \href{http://JD.com}{JD.com} in
China, contributed to the global rise of e-commerce. The coronavirus may
very well do the same for travel innovations, paving the way for a new
ubiquity for artificial intelligence long after the pandemic is quelled.

While Pangea's data scientists were developing their biometric platform,
researchers in the Beijing and San Francisco offices of
\href{https://www.rokid.com/}{Rokid}, a technology company specializing
in robotics and A.I. development, began working on a prototype for
temperature-reading glasses.

\hypertarget{latest-updates-global-coronavirus-outbreak}{%
\section{\texorpdfstring{\href{https://www.nytimes.com/2020/08/01/world/coronavirus-covid-19.html?action=click\&pgtype=Article\&state=default\&region=MAIN_CONTENT_1\&context=storylines_live_updates}{Latest
Updates: Global Coronavirus
Outbreak}}{Latest Updates: Global Coronavirus Outbreak}}\label{latest-updates-global-coronavirus-outbreak}}

Updated 2020-08-01T17:33:09.863Z

\begin{itemize}
\tightlist
\item
  \href{https://www.nytimes.com/2020/08/01/world/coronavirus-covid-19.html?action=click\&pgtype=Article\&state=default\&region=MAIN_CONTENT_1\&context=storylines_live_updates\#link-2e88f231}{Top
  officials work to break impasse over jobless benefit that helped keep
  afloat millions of unemployed Americans.}
\item
  \href{https://www.nytimes.com/2020/08/01/world/coronavirus-covid-19.html?action=click\&pgtype=Article\&state=default\&region=MAIN_CONTENT_1\&context=storylines_live_updates\#link-8796723}{The
  virus picks up dangerous speed in the Midwest, and in areas that had
  seen success.}
\item
  \href{https://www.nytimes.com/2020/08/01/world/coronavirus-covid-19.html?action=click\&pgtype=Article\&state=default\&region=MAIN_CONTENT_1\&context=storylines_live_updates\#link-25930521}{Thousands
  in Berlin protest Germany's coronavirus measures.}
\end{itemize}

\href{https://www.nytimes.com/2020/08/01/world/coronavirus-covid-19.html?action=click\&pgtype=Article\&state=default\&region=MAIN_CONTENT_1\&context=storylines_live_updates}{See
more updates}

More live coverage:
\href{https://www.nytimes.com/live/2020/07/31/business/stock-market-today-coronavirus?action=click\&pgtype=Article\&state=default\&region=MAIN_CONTENT_1\&context=storylines_live_updates}{Markets}

They had the hardware on hand: The company had been producing Rokid
Glass, augmented-reality eyewear, since May 2019. But in March, Rokid
began exploring ways to allow wearers to know if they were coming close
to anyone with a fever. Their new
\href{https://www.rokid.com/en/rokid-glass-2/}{Rokid Glasses} aim to
kill two birds --- temperature detection and social distancing --- with
one pair of A.I.-powered spectacles.

The glasses use an infrared sensor and camera, allowing wearers to
essentially ``see'' the temperature of people around them. Liang Guan,
Rokid's U.S. director, said the glasses can currently measure up to 10
people's temperatures simultaneously.

The glasses went on the market this spring. The Dubai Transport Security
Department is a customer --- they've been using the glasses since April
for body-temperature detection in airports, on subways and in fire
stations. Singapore Mass Rapid Transit has also purchased them for the
same use, as well as Aeropuertos Argentina, one of the largest private
sector airport operators in the world, with 35 airports under its
management in South America.

The glasses
\href{https://www.amazon.com/Rokid-AI-Powered-Temperature-Measurement-Recognition/dp/B087X5VJQ2}{are
also available on Amazon}, at a hefty price tag of \$6,999.

In airports, on subways and in crowded public spaces, Rokid believes the
glasses will equip security officers with a critical tool for locating
people who could spread Covid-19. But there's a privacy issue at play:
Personal body temperature is private medical data, and the glasses allow
the wearer to access that data from anyone who crosses their line of
sight, with no opportunity for consent.

But, said Mr. Guan, ``We are going to live with Covid-19 probably longer
than anyone thought,'' and that, he said, will have an effect on
perceptions of privacy. ``In the future, the balance might be shifted
more to public safety. And I think by then, ordinary people might be
able to use these on the street.''

\hypertarget{a-tour-guide-that-speaks-17-languages}{%
\subsection{A tour guide that speaks 17
languages}\label{a-tour-guide-that-speaks-17-languages}}

Rokid's thermal detection glasses and Pangea's health passport join a
crowded sector of new high-tech tools developed for travel during the
pandemic.

Bespoke. Inc., an A.I. chatbot developer headquartered in Tokyo, in
February released \href{https://www.be-spoke.io/bebot/}{Bebot}, a
multilingual chatbot that offers travelers updated information about
coronavirus outbreaks, statistics and symptoms.

In January, Sitata, a travel app that monitors potential travel
disruptions, introduced a new pandemic-focused platform,
\href{https://www.covidchecker.com/en/}{Covid Checker,} to help
travelers track restrictions and take stock of risk.

And in Miami, developers at the upcoming
\href{https://www.legacymwc.com/}{Legacy Hotel and Residences}, a resort
and condominium complex anchored by an on-site medical center, are
banking both on an A.I.-controlled air filtration system and an
A.I.-powered medical diagnostics center to lure residents and guests
with the promise of health and safety.

Tour guides, too, are going artificial. Alex Bainbridge was at work on
an interactive tour guide to embed in a driverless car when the pandemic
hit. The chief executive of \href{https://www.autoura.com/}{Autoura},
which creates and delivers vehicle-based sightseeing experiences, knew
that while robotaxis and autonomous vehicles are in the works, they're
not ready for the market yet. But with just a bit of work, his guide,
\href{https://apps.apple.com/us/app/sahra/id1515905101}{SAHRA}
(Sightseeing Autonomous Hospitality Robot by Autoura), could be.

Powered by an app, SAHRA speaks 17 languages and asks her clients a
number of questions before creating a location-guided tour itinerary.
Although she currently only offers food tours, in a handful of cities,
including New York, London and Seville; Mr. Bainbridge says a wider
range of experiences and options for 25 cities are being developed.

\href{https://www.nytimes.com/news-event/coronavirus?action=click\&pgtype=Article\&state=default\&region=MAIN_CONTENT_3\&context=storylines_faq}{}

\hypertarget{the-coronavirus-outbreak-}{%
\subsubsection{The Coronavirus Outbreak
›}\label{the-coronavirus-outbreak-}}

\hypertarget{frequently-asked-questions}{%
\paragraph{Frequently Asked
Questions}\label{frequently-asked-questions}}

Updated July 27, 2020

\begin{itemize}
\item ~
  \hypertarget{should-i-refinance-my-mortgage}{%
  \paragraph{Should I refinance my
  mortgage?}\label{should-i-refinance-my-mortgage}}

  \begin{itemize}
  \tightlist
  \item
    \href{https://www.nytimes.com/article/coronavirus-money-unemployment.html?action=click\&pgtype=Article\&state=default\&region=MAIN_CONTENT_3\&context=storylines_faq}{It
    could be a good idea,} because mortgage rates have
    \href{https://www.nytimes.com/2020/07/16/business/mortgage-rates-below-3-percent.html?action=click\&pgtype=Article\&state=default\&region=MAIN_CONTENT_3\&context=storylines_faq}{never
    been lower.} Refinancing requests have pushed mortgage applications
    to some of the highest levels since 2008, so be prepared to get in
    line. But defaults are also up, so if you're thinking about buying a
    home, be aware that some lenders have tightened their standards.
  \end{itemize}
\item ~
  \hypertarget{what-is-school-going-to-look-like-in-september}{%
  \paragraph{What is school going to look like in
  September?}\label{what-is-school-going-to-look-like-in-september}}

  \begin{itemize}
  \tightlist
  \item
    It is unlikely that many schools will return to a normal schedule
    this fall, requiring the grind of
    \href{https://www.nytimes.com/2020/06/05/us/coronavirus-education-lost-learning.html?action=click\&pgtype=Article\&state=default\&region=MAIN_CONTENT_3\&context=storylines_faq}{online
    learning},
    \href{https://www.nytimes.com/2020/05/29/us/coronavirus-child-care-centers.html?action=click\&pgtype=Article\&state=default\&region=MAIN_CONTENT_3\&context=storylines_faq}{makeshift
    child care} and
    \href{https://www.nytimes.com/2020/06/03/business/economy/coronavirus-working-women.html?action=click\&pgtype=Article\&state=default\&region=MAIN_CONTENT_3\&context=storylines_faq}{stunted
    workdays} to continue. California's two largest public school
    districts --- Los Angeles and San Diego --- said on July 13, that
    \href{https://www.nytimes.com/2020/07/13/us/lausd-san-diego-school-reopening.html?action=click\&pgtype=Article\&state=default\&region=MAIN_CONTENT_3\&context=storylines_faq}{instruction
    will be remote-only in the fall}, citing concerns that surging
    coronavirus infections in their areas pose too dire a risk for
    students and teachers. Together, the two districts enroll some
    825,000 students. They are the largest in the country so far to
    abandon plans for even a partial physical return to classrooms when
    they reopen in August. For other districts, the solution won't be an
    all-or-nothing approach.
    \href{https://bioethics.jhu.edu/research-and-outreach/projects/eschool-initiative/school-policy-tracker/}{Many
    systems}, including the nation's largest, New York City, are
    devising
    \href{https://www.nytimes.com/2020/06/26/us/coronavirus-schools-reopen-fall.html?action=click\&pgtype=Article\&state=default\&region=MAIN_CONTENT_3\&context=storylines_faq}{hybrid
    plans} that involve spending some days in classrooms and other days
    online. There's no national policy on this yet, so check with your
    municipal school system regularly to see what is happening in your
    community.
  \end{itemize}
\item ~
  \hypertarget{is-the-coronavirus-airborne}{%
  \paragraph{Is the coronavirus
  airborne?}\label{is-the-coronavirus-airborne}}

  \begin{itemize}
  \tightlist
  \item
    The coronavirus
    \href{https://www.nytimes.com/2020/07/04/health/239-experts-with-one-big-claim-the-coronavirus-is-airborne.html?action=click\&pgtype=Article\&state=default\&region=MAIN_CONTENT_3\&context=storylines_faq}{can
    stay aloft for hours in tiny droplets in stagnant air}, infecting
    people as they inhale, mounting scientific evidence suggests. This
    risk is highest in crowded indoor spaces with poor ventilation, and
    may help explain super-spreading events reported in meatpacking
    plants, churches and restaurants.
    \href{https://www.nytimes.com/2020/07/06/health/coronavirus-airborne-aerosols.html?action=click\&pgtype=Article\&state=default\&region=MAIN_CONTENT_3\&context=storylines_faq}{It's
    unclear how often the virus is spread} via these tiny droplets, or
    aerosols, compared with larger droplets that are expelled when a
    sick person coughs or sneezes, or transmitted through contact with
    contaminated surfaces, said Linsey Marr, an aerosol expert at
    Virginia Tech. Aerosols are released even when a person without
    symptoms exhales, talks or sings, according to Dr. Marr and more
    than 200 other experts, who
    \href{https://academic.oup.com/cid/article/doi/10.1093/cid/ciaa939/5867798}{have
    outlined the evidence in an open letter to the World Health
    Organization}.
  \end{itemize}
\item ~
  \hypertarget{what-are-the-symptoms-of-coronavirus}{%
  \paragraph{What are the symptoms of
  coronavirus?}\label{what-are-the-symptoms-of-coronavirus}}

  \begin{itemize}
  \tightlist
  \item
    Common symptoms
    \href{https://www.nytimes.com/article/symptoms-coronavirus.html?action=click\&pgtype=Article\&state=default\&region=MAIN_CONTENT_3\&context=storylines_faq}{include
    fever, a dry cough, fatigue and difficulty breathing or shortness of
    breath.} Some of these symptoms overlap with those of the flu,
    making detection difficult, but runny noses and stuffy sinuses are
    less common.
    \href{https://www.nytimes.com/2020/04/27/health/coronavirus-symptoms-cdc.html?action=click\&pgtype=Article\&state=default\&region=MAIN_CONTENT_3\&context=storylines_faq}{The
    C.D.C. has also} added chills, muscle pain, sore throat, headache
    and a new loss of the sense of taste or smell as symptoms to look
    out for. Most people fall ill five to seven days after exposure, but
    symptoms may appear in as few as two days or as many as 14 days.
  \end{itemize}
\item ~
  \hypertarget{does-asymptomatic-transmission-of-covid-19-happen}{%
  \paragraph{Does asymptomatic transmission of Covid-19
  happen?}\label{does-asymptomatic-transmission-of-covid-19-happen}}

  \begin{itemize}
  \tightlist
  \item
    So far, the evidence seems to show it does. A widely cited
    \href{https://www.nature.com/articles/s41591-020-0869-5}{paper}
    published in April suggests that people are most infectious about
    two days before the onset of coronavirus symptoms and estimated that
    44 percent of new infections were a result of transmission from
    people who were not yet showing symptoms. Recently, a top expert at
    the World Health Organization stated that transmission of the
    coronavirus by people who did not have symptoms was ``very rare,''
    \href{https://www.nytimes.com/2020/06/09/world/coronavirus-updates.html?action=click\&pgtype=Article\&state=default\&region=MAIN_CONTENT_3\&context=storylines_faq\#link-1f302e21}{but
    she later walked back that statement.}
  \end{itemize}
\end{itemize}

While traditional city tours involve packed hop-on, hop-off buses or a
single guide shepherding a large group of strangers from location to
location, SAHRA's tours are physically distanced and tailored to
individuals or families. They are designed to be carried out on
bicycles, electric scooters or in private cars. SAHRA is part chatbot,
part interactive map, and the plan is to eventually embed the tours in
autonomous vehicles, which Mr. Bainbridge predicts will be commonplace
in the travel market by 2025.

The move toward A.I.-enhanced travel experiences, Mr. Bainbridge said,
is an egalitarian one.

``In the future, the definition of luxury will be having a human tour
guide,'' he said. ``We're not trying to recreate the human, we're not
even providing the same product that humans provide. It's a different
experience, at a completely different price tag, and we're not
disrupting the industry as much as transitioning the industry using
technology that already exists.''

\hypertarget{price-and-privacy}{%
\subsection{Price and privacy}\label{price-and-privacy}}

Yet researchers and sociologists say that as more such services enter
the market, they have the potential to amplify divisions in society. The
Pangea Pass Card costs about \$140. Rokid's temperature-reading glasses,
about \$7,000. Many people will not be able to take advantage of these
tools, said Deborah Raji, a technology fellow at the
\href{https://ainowinstitute.org/}{AI Now Institute} at New York
University.

``There's an inherent exclusion by giving some people the power to
access these tools over others,'' she said.

And then there's privacy. Pass cards that contain sensitive health data,
and glasses that reveal health information are powered by potent
technology, and there is the danger that the technology could fall into
the wrong hands.

Ms. Raji pointed to a 2019 example of a
\href{https://www.nytimes.com/2019/03/28/nyregion/rent-stabilized-buildings-facial-recognition.html}{Brooklyn
landlord looking to use facial recognition software} in a
rent-stabilized building to show how A.I. can quickly turn Orwellian.
When it comes to surrendering sensitive health data to a third party,
Ms. Raji added, companies that collect and store health data can
eventually be sold, with the data then harnessed for government
surveillance or as a risk-assessment metric by health insurers.

``There is always a risk of tying health data to an identity. Think
about who is more likely to get Covid-19 --- people of color, minorities
and people of lower socio-economic status. When you monitor people and
make judgments of how to behave toward them based on metrics of how sick
they are, the data is helping you understand how to avoid those people
rather than understand how to support them,'' she said.

``There's a lot of power when you're in control of that kind of
information, and if that power is improperly stewarded, it can be very
dangerous,'' she added.

But as global cases of Covid-19 continue to climb, A.I. could also serve
as an ounce of prevention when the next health crisis hits the travel
industry.

``If nine months ago we had the immunity passport, we could have
immediately analyzed where people from Wuhan were flying to, and known
to quarantine specific areas, not just in China but also in the places
that people from China went to,'' Mr. Kaminer said. ``A.I. will play a
major role in the management of future crises.''

\begin{center}\rule{0.5\linewidth}{\linethickness}\end{center}

\emph{\textbf{Follow New York Times Travel}}
\emph{on}\href{https://www.instagram.com/nytimestravel/}{\emph{Instagram}}\emph{,}\href{https://twitter.com/nytimestravel}{\emph{Twitter}}
\emph{and}\href{https://www.facebook.com/nytimestravel/}{\emph{Facebook}}\emph{.
And}\href{https://www.nytimes.com/newsletters/traveldispatch}{\emph{sign
up for our weekly Travel Dispatch newsletter}} \emph{to receive expert
tips on traveling smarter and inspiration for your next vacation.}

Advertisement

\protect\hyperlink{after-bottom}{Continue reading the main story}

\hypertarget{site-index}{%
\subsection{Site Index}\label{site-index}}

\hypertarget{site-information-navigation}{%
\subsection{Site Information
Navigation}\label{site-information-navigation}}

\begin{itemize}
\tightlist
\item
  \href{https://help.nytimes.com/hc/en-us/articles/115014792127-Copyright-notice}{©~2020~The
  New York Times Company}
\end{itemize}

\begin{itemize}
\tightlist
\item
  \href{https://www.nytco.com/}{NYTCo}
\item
  \href{https://help.nytimes.com/hc/en-us/articles/115015385887-Contact-Us}{Contact
  Us}
\item
  \href{https://www.nytco.com/careers/}{Work with us}
\item
  \href{https://nytmediakit.com/}{Advertise}
\item
  \href{http://www.tbrandstudio.com/}{T Brand Studio}
\item
  \href{https://www.nytimes.com/privacy/cookie-policy\#how-do-i-manage-trackers}{Your
  Ad Choices}
\item
  \href{https://www.nytimes.com/privacy}{Privacy}
\item
  \href{https://help.nytimes.com/hc/en-us/articles/115014893428-Terms-of-service}{Terms
  of Service}
\item
  \href{https://help.nytimes.com/hc/en-us/articles/115014893968-Terms-of-sale}{Terms
  of Sale}
\item
  \href{https://spiderbites.nytimes.com}{Site Map}
\item
  \href{https://help.nytimes.com/hc/en-us}{Help}
\item
  \href{https://www.nytimes.com/subscription?campaignId=37WXW}{Subscriptions}
\end{itemize}
