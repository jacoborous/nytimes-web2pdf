Sections

SEARCH

\protect\hyperlink{site-content}{Skip to
content}\protect\hyperlink{site-index}{Skip to site index}

\href{https://www.nytimes.com/section/obituaries}{Obituaries}

\href{https://myaccount.nytimes.com/auth/login?response_type=cookie\&client_id=vi}{}

\href{https://www.nytimes.com/section/todayspaper}{Today's Paper}

\href{/section/obituaries}{Obituaries}\textbar{}Overlooked No More:
Nancy Green, the `Real Aunt Jemima'

\url{https://nyti.ms/3jdOMWv}

\begin{itemize}
\item
\item
\item
\item
\item
\item
\end{itemize}

Advertisement

\protect\hyperlink{after-top}{Continue reading the main story}

Supported by

\protect\hyperlink{after-sponsor}{Continue reading the main story}

\hypertarget{overlooked-no-more-nancy-green-the-real-aunt-jemima}{%
\section{Overlooked No More: Nancy Green, the `Real Aunt
Jemima'}\label{overlooked-no-more-nancy-green-the-real-aunt-jemima}}

A nanny and cook, she played the part as the pancake flour company that
employed her perpetuated a racial stereotype. She died 97 years ago in
Chicago.

\includegraphics{https://static01.nyt.com/images/2020/07/20/obituaries/20overlooked-green1/17overlooked-green1-articleLarge.jpg?quality=75\&auto=webp\&disable=upscale}

\href{https://www.nytimes.com/by/sam-roberts}{\includegraphics{https://static01.nyt.com/images/2018/02/20/multimedia/author-sam-roberts/author-sam-roberts-thumbLarge.jpg}}

By \href{https://www.nytimes.com/by/sam-roberts}{Sam Roberts}

\begin{itemize}
\item
  Published July 17, 2020Updated July 18, 2020
\item
  \begin{itemize}
  \item
  \item
  \item
  \item
  \item
  \item
  \end{itemize}
\end{itemize}

\emph{Overlooked is a series of obituaries about remarkable people whose
deaths, beginning in 1851, went unreported in The Times.}

Nancy Green was standing under the South Side El on East 46th Street in
Chicago one Thursday in 1923 when she was struck and killed by a car
that had collided with a laundry truck and careened onto the sidewalk
where she was standing.

Not until the next Monday, though, when Joseph Gubbins, Cook County's
deputy coroner, was conducting an inquest into the accident, was Green
identified by her alter ego.

Her death became front page news across the nation (though not in The
New York Times) ---~and for good reason. For two decades she had
generated headlines --- also on front pages --- while on tour as one of
America's most enduring living trademarks: Aunt Jemima.

``I'se in town, honey,'' billboards and buttons featuring her likeness
proclaimed.

At stops along the way Green would flip flapjacks in a flour
barrel-shaped pavilion 16 feet in diameter while singing spirituals and
other obligatory tunes and waxing rhapsodic about antebellum plantation
servitude under benevolent white masters.

She had been recruited in 1890 as the original living incarnation of
Aunt Jemima and played the part into the first decade of the 20th
century, most famously at the World's Columbian Exposition in Chicago in
1892.

Aunt Jemima, the character, would outlast Green for another 97 years on
labels and boxes, until last month, when Quaker Oats, which bought the
brand in 1926 and which was acquired by PepsiCo in 2001,
\href{https://www.nytimes.com/2020/06/17/business/aunt-jemima-racial-stereotype.html}{announced
her retirement}, acknowledging that she had been ``based on a racial
stereotype.''

\includegraphics{https://static01.nyt.com/images/2020/07/17/obituaries/17overlooked-green2/17overlooked-green2-articleLarge.jpg?quality=75\&auto=webp\&disable=upscale}

The image on pancake boxes and syrup dispensers was originally inspired
by the song ``\href{https://www.youtube.com/watch?v=92EwphX3eJI}{Old
Aunt Jemima},'' which was written in 1875 by
\href{https://nkaa.uky.edu/nkaa/items/show/2433}{Billy Kersands}, a
Black comedian, and performed, often by white men, in minstrel shows. In
1889, inspired by one such performance, Chris Rutt, a former newspaper
reporter, and Charles Underwood, his partner in a milling company, which
they had bought that year, branded their self-rising pancake flour with
the Aunt Jemima name.

Better at promotion than profit-making, the partners sold their failing
company to the \href{https://chicagology.com/columbiaexpo/fair038/}{R.T.
Davis} Mill Company of St. Joseph, Mo., who promptly solicited his
salesmen to find a real-life Aunt Jemima.

It was Charles C. Jackson, a food wholesaler, who discovered Green, in
1890. She was a cook for the family of Charles M. Walker Jr., who would
become a Chicago alderman, corporation counsel and judge.

Most biographies say that Green was born into slavery on March 4, 1834,
in Mt. Sterling, Ky., in Montgomery County, east of Lexington, although
the 1900 census lists her year of birth as 1854. (Official birth
certificates for slaves were rarely filed.) She won her freedom and was
hired as a nanny and housekeeper by Walker's father, who transplanted
the family to Chicago.

Green helped care for Walker's sons, Charles and Samuel, and her
pancakes were said to be popular among the family's friends.

As Aunt Jemima, she proved to be a promotional bonanza for R.T. Davis at
the Columbian Exposition, which included an exhibit of a miniature West
African village whose natives were portrayed as primitive savages.

The Aunt Jemima mythology transported Green to a tiny cabin in
Louisiana, where she was the loyal cook for a Colonel Higbee, a
plantation owner on the Mississippi. When Union soldiers during the
Civil War threatened to rip off his mustache, the story went, she
diverted them with her pancakes long enough for the colonel to escape.
The troops were so smitten that they urged her to come north and share
her recipe.

This back story was created by James Webb Young, an advertising
executive, and the illustrator N.C. Wyeth (the father of the artist
\href{https://www.nytimes.com/2009/01/17/arts/design/17wyeth.html}{Andrew
Wyeth}). In promotional material, Aunt Jemima was called ``the cook
whose cabin became more famous than Uncle Tom's.''

``Those who knew her best,'' it went on, ``who knew her even from the
time when she first came up from her little cabin home, they found her
still the simple, earnest smiling mammy --- it was all the same to
her.''

In reality, ``this Aunt Jemima logo was an outgrowth of Old South
plantation nostalgia and romance,'' Riché Richardson, an associate
professor in the Africana Studies and Research Center at Cornell
University, wrote in
\href{https://www.nytimes.com/roomfordebate/2015/06/24/besides-the-confederate-flag-what-other-symbols-should-go/can-we-please-finally-get-rid-of-aunt-jemima}{The
New York Times} in 2015. It was an image, she said, ``grounded in an
idea about the `mammy,' a devoted and submissive servant who eagerly
nurtured the children of her white master and mistress while neglecting
her own.''

Green was said to have received a lifetime contract and made a fortune,
but it's more likely that she simply worked for the company (she
described herself in the 1910 census as a ``housekeeper'') while serving
as a missionary for the historic
\href{http://www.olivetbaptistchurchchicago.org/}{Olivet Baptist Church
in Chicago}.

In 1900, after 30 years of marriage, she was widowed. Green died from
her injuries in the car accident on Aug. 30, 1923, having outlived her
two children. She was believed to be 89. She was living with a
great-nephew and his wife at the time.

Image

Articles about Green appeared in newspapers across the nation, even
after her death. This one was from 1971.Credit...The Sacramento Bee, via
newspapers.com

Green was buried in an unmarked grave in Oakwood Cemetery in Chicago.
That same year, the United Daughters of the Confederacy nearly succeeded
in erecting a monument to ``faithful colored mammies.'' Legislation was
approved by the United States Senate, but did not make it past the
House.

For 15 years, Sherry Williams, the president of The
\href{https://bronzevillehistoricalsociety.wordpress.com/}{Bronzeville
Historical Society}, which preserves African-American culture in
Chicago, searched for a descendant of Green to grant permission to place
a headstone at her grave site.

This year, she reached a great-great-great nephew, Marcus Hayes, of
Huntsville, Ala., who heartily agreed. Hayes, who is studying to become
a pastor at Oakwood University, which is affiliated with the Seventh-day
Adventist Church, said in a phone interview that Green had used her
prominence to promote equality and would be disappointed that others had
profited from her talent while her family received no recognition or
compensation.

``She would want the real story to be told of her and the ladies that
came after her,'' Hayes said. ``Aunt Jemima is more than a character.
She is Nancy Green, and this is her recipe, and her legacy must be
told.''

Williams said she hoped to hold a ceremony for Green at the cemetery
later this year. She also made the case not to forget the Aunt Jemima
image.

``History does not simply disappear when you remove the Aunt Jemima
image and brand name,''
\href{https://bronzevillehistoricalsociety.wordpress.com/2020/06/23/aunt-jemima-removed-from-pancake-products-commentary-by-sherry-williams/}{Williams
wrote}.

``Aunt Jemima is representative of the countless Black women who were
and are the essential workers,'' she added. ``Nancy Green in particular
is the ideal woman to salute.''

Advertisement

\protect\hyperlink{after-bottom}{Continue reading the main story}

\hypertarget{site-index}{%
\subsection{Site Index}\label{site-index}}

\hypertarget{site-information-navigation}{%
\subsection{Site Information
Navigation}\label{site-information-navigation}}

\begin{itemize}
\tightlist
\item
  \href{https://help.nytimes.com/hc/en-us/articles/115014792127-Copyright-notice}{©~2020~The
  New York Times Company}
\end{itemize}

\begin{itemize}
\tightlist
\item
  \href{https://www.nytco.com/}{NYTCo}
\item
  \href{https://help.nytimes.com/hc/en-us/articles/115015385887-Contact-Us}{Contact
  Us}
\item
  \href{https://www.nytco.com/careers/}{Work with us}
\item
  \href{https://nytmediakit.com/}{Advertise}
\item
  \href{http://www.tbrandstudio.com/}{T Brand Studio}
\item
  \href{https://www.nytimes.com/privacy/cookie-policy\#how-do-i-manage-trackers}{Your
  Ad Choices}
\item
  \href{https://www.nytimes.com/privacy}{Privacy}
\item
  \href{https://help.nytimes.com/hc/en-us/articles/115014893428-Terms-of-service}{Terms
  of Service}
\item
  \href{https://help.nytimes.com/hc/en-us/articles/115014893968-Terms-of-sale}{Terms
  of Sale}
\item
  \href{https://spiderbites.nytimes.com}{Site Map}
\item
  \href{https://help.nytimes.com/hc/en-us}{Help}
\item
  \href{https://www.nytimes.com/subscription?campaignId=37WXW}{Subscriptions}
\end{itemize}
