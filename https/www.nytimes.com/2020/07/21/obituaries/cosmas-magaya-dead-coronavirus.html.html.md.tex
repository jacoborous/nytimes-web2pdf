Sections

SEARCH

\protect\hyperlink{site-content}{Skip to
content}\protect\hyperlink{site-index}{Skip to site index}

\href{https://www.nytimes.com/section/obituaries}{Obituaries}

\href{https://myaccount.nytimes.com/auth/login?response_type=cookie\&client_id=vi}{}

\href{https://www.nytimes.com/section/todayspaper}{Today's Paper}

\href{/section/obituaries}{Obituaries}\textbar{}Cosmas Magaya, Musician
and Teacher of African Traditions, Dies at 66

\url{https://nyti.ms/2ZKR47I}

\begin{itemize}
\item
\item
\item
\item
\item
\end{itemize}

\href{https://www.nytimes.com/news-event/coronavirus?action=click\&pgtype=Article\&state=default\&region=TOP_BANNER\&context=storylines_menu}{The
Coronavirus Outbreak}

\begin{itemize}
\tightlist
\item
  live\href{https://www.nytimes.com/2020/08/03/world/coronavirus-covid-19.html?action=click\&pgtype=Article\&state=default\&region=TOP_BANNER\&context=storylines_menu}{Latest
  Updates}
\item
  \href{https://www.nytimes.com/interactive/2020/us/coronavirus-us-cases.html?action=click\&pgtype=Article\&state=default\&region=TOP_BANNER\&context=storylines_menu}{Maps
  and Cases}
\item
  \href{https://www.nytimes.com/interactive/2020/science/coronavirus-vaccine-tracker.html?action=click\&pgtype=Article\&state=default\&region=TOP_BANNER\&context=storylines_menu}{Vaccine
  Tracker}
\item
  \href{https://www.nytimes.com/2020/08/02/us/covid-college-reopening.html?action=click\&pgtype=Article\&state=default\&region=TOP_BANNER\&context=storylines_menu}{College
  Reopening}
\item
  \href{https://www.nytimes.com/live/2020/08/03/business/stock-market-today-coronavirus?action=click\&pgtype=Article\&state=default\&region=TOP_BANNER\&context=storylines_menu}{Economy}
\end{itemize}

Advertisement

\protect\hyperlink{after-top}{Continue reading the main story}

Supported by

\protect\hyperlink{after-sponsor}{Continue reading the main story}

Those We've Lost

\hypertarget{cosmas-magaya-musician-and-teacher-of-african-traditions-dies-at-66}{%
\section{Cosmas Magaya, Musician and Teacher of African Traditions, Dies
at
66}\label{cosmas-magaya-musician-and-teacher-of-african-traditions-dies-at-66}}

A virtuoso of the mbira, a plucked instrument, he shared his knowledge
with the wider world to preserve wisdom handed down orally. He died of
Covid-19.

\includegraphics{https://static01.nyt.com/images/2020/07/23/obituaries/20Magaya/merlin_174766566_39bc3290-3818-4a8d-b313-37d3de5f159c-articleLarge.jpg?quality=75\&auto=webp\&disable=upscale}

\href{https://www.nytimes.com/by/jon-pareles}{\includegraphics{https://static01.nyt.com/images/2018/06/14/multimedia/author-jon-pareles/author-jon-pareles-thumbLarge.png}}

By \href{https://www.nytimes.com/by/jon-pareles}{Jon Pareles}

\begin{itemize}
\item
  Published July 21, 2020Updated July 22, 2020
\item
  \begin{itemize}
  \item
  \item
  \item
  \item
  \item
  \end{itemize}
\end{itemize}

\emph{This obituary is part of a series about people who have died in
the coronavirus pandemic. Read about others}
\href{https://www.nytimes.com/interactive/2020/obituaries/people-died-coronavirus-obituaries.html}{\emph{here}}\emph{.}

In his village in northern Zimbabwe, Cosmas Magaya played music to
summon ancestral spirits at traditional rituals of the Shona people:
ceremonies to request divine guidance, to ward off illness, or to call
for rain.

The same music made him an internationally acclaimed performer and
teacher of the plucked instrument called the mbira. He toured the world,
made recordings, taught at universities and was a fountainhead of Shona
cultural knowledge.

Mr. Magaya died of Covid-19 on July 10 in Harare, the capital, his
daughter Tsitsi Hantuba said. He was 66.

Mr. Magaya played the mbira dzavadzimu, the ``mbira of the ancestors'':
a wooden plank with tuned steel prongs and buzzing resonators attached.
He improvised around complex, age-old multilayered patterns, at once
\href{https://www.youtube.com/watch?\&v=TS5ASXu44bQ}{percussive and
meditative}.

In a long collaboration and friendship, Mr. Magaya helped Paul Berliner,
an ethnomusicologist and now a professor emeritus at Duke University, to
learn, analyze and transcribe the deep structures and improvisatory
extensions of Shona mbira music. They worked together to impart the
tradition to musicians worldwide.

In recent years, Mr. Magaya made annual visits to perform and teach in
the United States, and he had students in Europe
\href{https://www.youtube.com/watch?v=ZzOjBJadc_A}{and Africa} as well.

Some Zimbabweans thought he had revealed too much to outsiders. But, Ms.
Hantuba said, ``He was so passionate about mbira and traditional beliefs
that he wanted to share as much as he could.''

Mr. Berliner first studied with Mr. Magaya in 1971 in the country then
called Rhodesia, when the virtuosic 18-year-old Mr. Magaya was a
principal member of a renowned mbira group, Mhuri yekwaRwizi. Mr.
Berliner recorded the group's music on albums released for the Nonesuch
label's Explorer series --- ``The Soul of Mbira'' (1973) and ``Shona
Mbira Music'' (1977) --- that introduced many Westerners to traditional
mbira music. Mr. Berliner wrote books about the tradition, including the
\href{https://press.uchicago.edu/ucp/books/book/chicago/A/bo38181879.html}{``The
Art of Mbira: Musical Inheritance and Legacy''} (2020), guided by Mr.
Magaya's understanding of a profound, evolving cultural heritage.

In addition to Ms. Hantuba, Mr. Magaya is survived by his wife, Patricia
Nyamande; two other daughters, Matilda Magaya and Rutendo Magaya; a son,
Mudavanhu, who plays mbira in his father's style; and 11 grandchildren.
His first wife, Joyce Zinyengere, whom he married in 1976, died in 1999.

Cosmas Magaya was born on Oct. 5, 1953, in a rural area of the Mondoro
(now Mhondoro) district. His father, Joshua Magaya, was a farmer, a
traditional healer and a spirit medium. Cosmas began playing mbira when
he was 8, initially taught by a cousin, Ernest Chivhanga, who also built
mbiras. He was 12 when his cousin began taking him along to play at
ceremonies. The leader of Mhuri yekwaRwizi, the singer Hakurotwi Mude,
heard the teenage Mr. Magaya and invited him to join the group, and Mr.
Magaya moved to the capital.

During the protracted civil war that led to an independent Zimbabwe in
1980, Mr. Magaya moved to Bulawayo in the south. He taught mbira at the
Kwanongoma College of Music. But he mainly earned a living, from 1973 to
1997, as a depot manager for the government's Dairy Marketing Board.

Mr. Magaya rejoined Mhuri yekwaRwizi as the musical director for its
first international tours, in 1983 and 1985, in Europe. He
\href{https://www.nytimes.com/1999/11/09/arts/music-review-ancient-resonance-in-twinkling-syncopations.html}{later
toured internationally} on his own and with various ensembles. His
teaching in the United States included residencies at Middlebury College
and Duke, Brown and Stanford universities, and he regularly appeared at
the annual \href{https://zimfest.org/}{Zimbabwean Music Festival} in
Oregon.

He returned to live in northern Zimbabwe, where he raised corn and
cattle. In 2000 he became the program director for
\href{https://ancient-ways.org/regions/zimbabwe-nhimbe-for-progress/}{Nhimbe
for Progress}, a nonprofit organization devoted to health and education
in the region's impoverished villages, and in 2004 he succeeded his
father as village headman.

Mr. Magaya saw his collaboration with Mr. Berliner as a way of honoring
the memory of his ancestors and teachers. In ``The Art of Mbira,'' he
told Mr. Berliner, ``Once we've completed this study on behalf of our
late mbira-playing comrades --- leaving it for others who come behind us
--- I will know that if I die tomorrow, I can go to my grave
satisfied.''

\href{https://www.nytimes.com/interactive/2020/obituaries/people-died-coronavirus-obituaries.html?action=click\&pgtype=Article\&state=default\&region=BELOW_MAIN_CONTENT\&context=covid_obits_promo}{}

\hypertarget{those-weve-lost}{%
\section{Those We've Lost}\label{those-weve-lost}}

The coronavirus pandemic has taken an incalculable death toll. This
series is designed to put names and faces to the numbers.

Read more

\includegraphics{https://static01.nyt.com/images/2020/07/30/obituaries/30Pedro/30Pedro-square640.jpg}

\hypertarget{bernaldina-josuxe9-pedro}{%
\section{Bernaldina José Pedro}\label{bernaldina-josuxe9-pedro}}

d. Boa Vista, Brazil

Leader among the Indigenous Macuxi

\includegraphics{https://static01.nyt.com/images/2020/07/31/obituaries/31Swing/merlin_175167783_8913bc90-0d64-43f3-a655-1bb1bf1601c9-square640.jpg}

\hypertarget{john-eric-swing}{%
\section{John Eric Swing}\label{john-eric-swing}}

d. Fountain Valley, Calif.

Champion of Filipino-Americans

\includegraphics{https://static01.nyt.com/images/2020/07/27/obituaries/27Victor/merlin_175001436_38b11f8e-227a-4e2c-9821-7618af9b2524-square640.jpg}

\hypertarget{victor-victor}{%
\section{Victor Victor}\label{victor-victor}}

d. Santo Domingo, Dominican Republic

Beloved musician of the Dominican Republic

\includegraphics{https://static01.nyt.com/images/2020/07/31/obituaries/31Negron/merlin_175160169_516322ae-fd23-4969-b6b2-193ced371105-square640.jpg}

\hypertarget{dr-eddie-negruxf3n}{%
\section{Dr. Eddie Negrón}\label{dr-eddie-negruxf3n}}

d. Fort Walton Beach, Fla.

Internist on Florida's Emerald Coast

\includegraphics{https://static01.nyt.com/images/2020/07/30/obituaries/30Dobson/merlin_175115928_f6b9271c-8f05-4fe1-a38a-5ca4a58f8935-square640.jpg}

\hypertarget{dobby-dobson}{%
\section{Dobby Dobson}\label{dobby-dobson}}

d. Coral Springs, Fla.

Jamaican singer and songwriter

\includegraphics{https://static01.nyt.com/images/2020/08/01/obituaries/28Gonzalez/merlin_175002771_beb57888-3951-409a-ae13-03a94b2e962e-square640.jpg}

\hypertarget{waldemar-gonzalez}{%
\section{Waldemar Gonzalez}\label{waldemar-gonzalez}}

d. White Plains, N.Y.

Teacher and social worker

Advertisement

\protect\hyperlink{after-bottom}{Continue reading the main story}

\hypertarget{site-index}{%
\subsection{Site Index}\label{site-index}}

\hypertarget{site-information-navigation}{%
\subsection{Site Information
Navigation}\label{site-information-navigation}}

\begin{itemize}
\tightlist
\item
  \href{https://help.nytimes.com/hc/en-us/articles/115014792127-Copyright-notice}{©~2020~The
  New York Times Company}
\end{itemize}

\begin{itemize}
\tightlist
\item
  \href{https://www.nytco.com/}{NYTCo}
\item
  \href{https://help.nytimes.com/hc/en-us/articles/115015385887-Contact-Us}{Contact
  Us}
\item
  \href{https://www.nytco.com/careers/}{Work with us}
\item
  \href{https://nytmediakit.com/}{Advertise}
\item
  \href{http://www.tbrandstudio.com/}{T Brand Studio}
\item
  \href{https://www.nytimes.com/privacy/cookie-policy\#how-do-i-manage-trackers}{Your
  Ad Choices}
\item
  \href{https://www.nytimes.com/privacy}{Privacy}
\item
  \href{https://help.nytimes.com/hc/en-us/articles/115014893428-Terms-of-service}{Terms
  of Service}
\item
  \href{https://help.nytimes.com/hc/en-us/articles/115014893968-Terms-of-sale}{Terms
  of Sale}
\item
  \href{https://spiderbites.nytimes.com}{Site Map}
\item
  \href{https://help.nytimes.com/hc/en-us}{Help}
\item
  \href{https://www.nytimes.com/subscription?campaignId=37WXW}{Subscriptions}
\end{itemize}
