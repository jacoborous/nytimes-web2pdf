Sections

SEARCH

\protect\hyperlink{site-content}{Skip to
content}\protect\hyperlink{site-index}{Skip to site index}

\href{https://www.nytimes.com/section/us}{U.S.}

\href{https://myaccount.nytimes.com/auth/login?response_type=cookie\&client_id=vi}{}

\href{https://www.nytimes.com/section/todayspaper}{Today's Paper}

\href{/section/us}{U.S.}\textbar{}A Navy Veteran Had a Question for the
Feds in Portland. They Beat Him in Response.

\url{https://nyti.ms/3jmBR4t}

\begin{itemize}
\item
\item
\item
\item
\item
\item
\end{itemize}

\href{https://www.nytimes.com/news-event/george-floyd-protests-minneapolis-new-york-los-angeles?action=click\&pgtype=Article\&state=default\&region=TOP_BANNER\&context=storylines_menu}{Race
and America}

\begin{itemize}
\tightlist
\item
  \href{https://www.nytimes.com/2020/07/26/us/protests-portland-seattle-trump.html?action=click\&pgtype=Article\&state=default\&region=TOP_BANNER\&context=storylines_menu}{Protesters
  Return to Other Cities}
\item
  \href{https://www.nytimes.com/2020/07/24/us/portland-oregon-protests-white-race.html?action=click\&pgtype=Article\&state=default\&region=TOP_BANNER\&context=storylines_menu}{Portland
  at the Center}
\item
  \href{https://www.nytimes.com/2020/07/23/podcasts/the-daily/portland-protests.html?action=click\&pgtype=Article\&state=default\&region=TOP_BANNER\&context=storylines_menu}{Podcast:
  Showdown in Portland}
\item
  \href{https://www.nytimes.com/interactive/2020/07/16/us/black-lives-matter-protests-louisville-breonna-taylor.html?action=click\&pgtype=Article\&state=default\&region=TOP_BANNER\&context=storylines_menu}{45
  Days in Louisville}
\end{itemize}

Advertisement

\protect\hyperlink{after-top}{Continue reading the main story}

Supported by

\protect\hyperlink{after-sponsor}{Continue reading the main story}

\hypertarget{a-navy-veteran-had-a-question-for-the-feds-in-portland-they-beat-him-in-response}{%
\section{A Navy Veteran Had a Question for the Feds in Portland. They
Beat Him in
Response.}\label{a-navy-veteran-had-a-question-for-the-feds-in-portland-they-beat-him-in-response}}

The veteran said he wanted to ask the officers whether they felt their
actions violated the Constitution. Video shows them tear-gassing him and
smashing his hand with baton blows.

\includegraphics{https://static01.nyt.com/images/2020/07/20/us/20UNREST-PORTLAND-VETERAN/20UNREST-PORTLAND-VETERAN-videoSixteenByNineJumbo1600-v2.jpg}

\href{https://www.nytimes.com/by/john-ismay}{\includegraphics{https://static01.nyt.com/images/2018/07/12/multimedia/author-john-ismay/author-john-ismay-thumbLarge.png}}

By \href{https://www.nytimes.com/by/john-ismay}{John Ismay}

\begin{itemize}
\item
  Published July 20, 2020Updated July 22, 2020
\item
  \begin{itemize}
  \item
  \item
  \item
  \item
  \item
  \item
  \end{itemize}
\end{itemize}

Christopher J. David had largely ignored the protests in downtown
\href{https://www.nytimes.com/2020/07/21/us/portland-protests.html}{Portland},
Ore., but when he saw videos of unidentified federal agents grabbing
\href{https://www.nytimes.com/2020/07/21/us/portland-protests.html}{protesters}
off the street and throwing them into rented minivans, he felt compelled
to act.

Mr. David, a Navy veteran, said that federal agents' use of violent
tactics against
\href{https://www.nytimes.com/2020/07/22/us/portland-protests-courthouse.html}{protesters},
without the support of the mayor, the governor or local law enforcement,
was a violation of the oaths that agents take to support, uphold and
defend the Constitution.

And so, on Saturday, he took a bus downtown to ask the officers how they
squared their actions with that oath.

Instead of getting an answer, Mr. David was beaten with a baton by one
federal officer as another doused him with pepper spray,
\href{https://twitter.com/PDXzane/status/1284726088187310080}{according
to video footage of the encounter}. After he walked away from the
confrontation, Mr. David was taken to a nearby hospital, where a
specialist said his right hand was broken and would require surgery to
install pins, screws and plates. He declined pain medication.

``I wasn't even paying attention to the protests at all until the feds
came in,'' Mr. David said in an interview on Sunday night. ``That's when
I became aware.''

Protesters have been in the streets of Portland for more than 50
consecutive days, in response to the killing of George Floyd by the
police in Minneapolis. The arrival of federal officers in the city has
re-energized the demonstrations, which continued on Sunday night, with
tear gas once again deployed by U.S. agents.

Kenneth T. Cuccinelli, the acting deputy secretary for the Department of
Homeland Security, said on CNN that he was ``familiar with the video''
involving Mr. David and that ``maintaining an appropriate response is an
ongoing obligation.''

Mr. David, a graduate of the U.S. Naval Academy and a former varsity
wrestler who has lived in Portland since 2006, said he had attended only
one protest before --- a march for women's rights in Washington, D.C.,
in 1989. As a 53-year-old man with health concerns, he said that the
risk of the coronavirus was reason enough to stay away from downtown
Portland.

``It just didn't seem worth it to me at that point, but it reached that
threshold when I saw Pinochet-type behavior from our own government,''
he said, referring to the Chilean dictator.

With his mind made up, Mr. David grabbed a backpack with some essential
items --- migraine medicine, nicotine gum, his wallet and ID cards ---
and took a bus downtown, arriving near the Mark O. Hatfield U.S.
Courthouse about 8:15 p.m.

The courthouse has become
\href{https://www.nytimes.com/2020/07/17/us/portland-protests.html}{a
focus of protesters}, as well as the federal Homeland Security agents
who have been dispatched to protect it. But the response by those agents
in Portland has prompted a backlash over whether the officers are
exceeding their arrest authority and violating the rights of protesters
by detaining demonstrators in the area around the federal courthouse.

On Mr. David's backpack were patches commemorating his time as an
officer in the Navy's Civil Engineering Corps, serving with the
construction battalions --- the famed
\href{https://www.navy.mil/navydata/personnel/seabees/seabee1.html}{Seabees}.

He also wore a heather gray sweatshirt with the word ``Navy'' emblazoned
in blue across the top and a ball cap for the Academy's wrestling team.
He wanted the officers to know by sight that he was a veteran, and
someone they could talk to.

``I identified the hell out of myself for a reason, I want to give them
pause so we could talk,'' he said. ``So I wanted to go down there to
tell them that I believed they were not following their oath to the
Constitution. That was my goal.''

By 10:45 p.m., Mr. David was just about to leave to return home, he
said, when some protesters began removing fencing around the courthouse
and federal officers emerged. He made his way to a clutch of federal
agents.

A video shot by Zane Sparling, a reporter with The Portland Tribune,
captured what happened next. Officers in camouflage and gas masks beat
Mr. David with batons and blasted pepper spray at his face. The shaky
cellphone video shows him briefly shoving away the hand of the officer
with the spray-can before turning around, walking away and defiantly
throwing up the middle fingers of both hands. He turned and faced the
officers again, raising his middle fingers even higher --- though a
strike from a baton had just shattered his dominant right hand.

Internet users quickly called him ``Captain Portland'' for barely
flinching at the blows. Noting how at 6 feet 2 inches tall he towered
over the officers, some people compared him to the ``Game of Thrones''
character known as the Mountain.
\href{https://twitter.com/Tazerface16}{On Twitter}, he went from having
a handful of followers before the encounter to more than 60,000 on
Monday.

``My life has turned pretty dramatically weird,'' he said.

Sergio Olmos, Mike Baker and Zolan Kanno-Youngs contributed reporting.

Advertisement

\protect\hyperlink{after-bottom}{Continue reading the main story}

\hypertarget{site-index}{%
\subsection{Site Index}\label{site-index}}

\hypertarget{site-information-navigation}{%
\subsection{Site Information
Navigation}\label{site-information-navigation}}

\begin{itemize}
\tightlist
\item
  \href{https://help.nytimes.com/hc/en-us/articles/115014792127-Copyright-notice}{©~2020~The
  New York Times Company}
\end{itemize}

\begin{itemize}
\tightlist
\item
  \href{https://www.nytco.com/}{NYTCo}
\item
  \href{https://help.nytimes.com/hc/en-us/articles/115015385887-Contact-Us}{Contact
  Us}
\item
  \href{https://www.nytco.com/careers/}{Work with us}
\item
  \href{https://nytmediakit.com/}{Advertise}
\item
  \href{http://www.tbrandstudio.com/}{T Brand Studio}
\item
  \href{https://www.nytimes.com/privacy/cookie-policy\#how-do-i-manage-trackers}{Your
  Ad Choices}
\item
  \href{https://www.nytimes.com/privacy}{Privacy}
\item
  \href{https://help.nytimes.com/hc/en-us/articles/115014893428-Terms-of-service}{Terms
  of Service}
\item
  \href{https://help.nytimes.com/hc/en-us/articles/115014893968-Terms-of-sale}{Terms
  of Sale}
\item
  \href{https://spiderbites.nytimes.com}{Site Map}
\item
  \href{https://help.nytimes.com/hc/en-us}{Help}
\item
  \href{https://www.nytimes.com/subscription?campaignId=37WXW}{Subscriptions}
\end{itemize}
