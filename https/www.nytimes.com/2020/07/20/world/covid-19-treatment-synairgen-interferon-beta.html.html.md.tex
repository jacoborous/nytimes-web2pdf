Sections

SEARCH

\protect\hyperlink{site-content}{Skip to
content}\protect\hyperlink{site-index}{Skip to site index}

\href{https://www.nytimes.com/section/world}{World}

\href{https://myaccount.nytimes.com/auth/login?response_type=cookie\&client_id=vi}{}

\href{https://www.nytimes.com/section/todayspaper}{Today's Paper}

\href{/section/world}{World}\textbar{}New Treatment for Covid-19 Shows
Promise, but Scientists Urge Caution

\url{https://nyti.ms/2OGJ80X}

\begin{itemize}
\item
\item
\item
\item
\item
\end{itemize}

\href{https://www.nytimes.com/news-event/coronavirus?action=click\&pgtype=Article\&state=default\&region=TOP_BANNER\&context=storylines_menu}{The
Coronavirus Outbreak}

\begin{itemize}
\tightlist
\item
  live\href{https://www.nytimes.com/2020/08/04/world/coronavirus-cases.html?action=click\&pgtype=Article\&state=default\&region=TOP_BANNER\&context=storylines_menu}{Latest
  Updates}
\item
  \href{https://www.nytimes.com/interactive/2020/us/coronavirus-us-cases.html?action=click\&pgtype=Article\&state=default\&region=TOP_BANNER\&context=storylines_menu}{Maps
  and Cases}
\item
  \href{https://www.nytimes.com/interactive/2020/science/coronavirus-vaccine-tracker.html?action=click\&pgtype=Article\&state=default\&region=TOP_BANNER\&context=storylines_menu}{Vaccine
  Tracker}
\item
  \href{https://www.nytimes.com/2020/08/02/us/covid-college-reopening.html?action=click\&pgtype=Article\&state=default\&region=TOP_BANNER\&context=storylines_menu}{College
  Reopening}
\item
  \href{https://www.nytimes.com/live/2020/08/04/business/stock-market-today-coronavirus?action=click\&pgtype=Article\&state=default\&region=TOP_BANNER\&context=storylines_menu}{Economy}
\end{itemize}

Advertisement

\protect\hyperlink{after-top}{Continue reading the main story}

Supported by

\protect\hyperlink{after-sponsor}{Continue reading the main story}

\hypertarget{new-treatment-for-covid-19-shows-promise-but-scientists-urge-caution}{%
\section{New Treatment for Covid-19 Shows Promise, but Scientists Urge
Caution}\label{new-treatment-for-covid-19-shows-promise-but-scientists-urge-caution}}

A small study of an inhaled form of a commonly available drug,
interferon beta, suggests it could reduce the odds of patients becoming
severely ill.

\includegraphics{https://static01.nyt.com/images/2020/07/20/world/20virus-treatment/merlin_172215174_6283f11a-ac05-45df-a7d7-d7f06151224d-articleLarge.jpg?quality=75\&auto=webp\&disable=upscale}

\href{https://www.nytimes.com/by/benjamin-mueller}{\includegraphics{https://static01.nyt.com/images/2018/02/20/multimedia/author-benjamin-mueller/author-benjamin-mueller-thumbLarge.jpg}}

By \href{https://www.nytimes.com/by/benjamin-mueller}{Benjamin Mueller}

\begin{itemize}
\item
  July 20, 2020
\item
  \begin{itemize}
  \item
  \item
  \item
  \item
  \item
  \end{itemize}
\end{itemize}

LONDON --- A British drug company said Monday that an inhaled form of a
commonly used medicine could slash the odds of Covid-19 patients
becoming severely ill, a sliver of good news in the race to find
treatments that was met by scientists with equal measures of caution and
cheer.

The drug, based on interferon beta, a protein naturally produced by the
body to orchestrate its response to viruses, has become the focus of
intensifying efforts in Britain, China and the United States to treat
Covid-19 patients.

Scientists have found that the coronavirus attacks the body in part by
blocking its natural interferon response, disarming cells that would
otherwise be alerting neighboring cells to activate their own genes and
fortify themselves against the invading virus. In theory, administering
interferon to patients could invigorate its defenses in the early stages
of illness.

But giving patients interferon without eliciting serious side effects
has proved challenging. The symptoms of a seasonal flu, for example, are
largely produced by the mobilization of the body's interferon response,
scientists said.

The British drug company, Synairgen, tried to circumvent that problem by
developing an inhaled form of interferon that directly targets cells in
the lungs, rather than an injection, which can produce more intense side
effects. It conducted a small, double-blind trial on patients
hospitalized with Covid-19, the illness caused by the coronavirus, in
nine British hospitals.

The initial results,
\href{https://www.synairgen.com/wp-content/uploads/2020/07/200720-Synairgen-announces-positive-results-from-trial-of-SNG001-in-hospitalised-COVID-19-patients.pdf}{announced
in a brief news release} but not yet peer reviewed or published, were
promising: The inhaled form of interferon beta tested by Synairgen was
shown to reduce the odds of hospitalized patients becoming severely ill
--- needing ventilation, for example --- by 79 percent compared with
patients who received a placebo.

But the significance of the findings was seriously limited --- and, in
the view of some scientists, undercut --- by the small size of the
trial. It involved only 101 patients, Synairgen said, making it
difficult to know for certain how beneficial the drug was or how it
affected patients differently. The study could not rule out that the
drug was only marginally effective.

\hypertarget{latest-updates-global-coronavirus-outbreak}{%
\section{\texorpdfstring{\href{https://www.nytimes.com/2020/08/04/world/coronavirus-cases.html?action=click\&pgtype=Article\&state=default\&region=MAIN_CONTENT_1\&context=storylines_live_updates}{Latest
Updates: Global Coronavirus
Outbreak}}{Latest Updates: Global Coronavirus Outbreak}}\label{latest-updates-global-coronavirus-outbreak}}

Updated 2020-08-05T07:58:24.076Z

\begin{itemize}
\tightlist
\item
  \href{https://www.nytimes.com/2020/08/04/world/coronavirus-cases.html?action=click\&pgtype=Article\&state=default\&region=MAIN_CONTENT_1\&context=storylines_live_updates\#link-762df92}{As
  talks drag on, McConnell signals openness to jobless aid extension,
  and negotiators agree on a deadline.}
\item
  \href{https://www.nytimes.com/2020/08/04/world/coronavirus-cases.html?action=click\&pgtype=Article\&state=default\&region=MAIN_CONTENT_1\&context=storylines_live_updates\#link-1228a480}{Novavax
  sees encouraging results from two studies of its experimental
  vaccine.}
\item
  \href{https://www.nytimes.com/2020/08/04/world/coronavirus-cases.html?action=click\&pgtype=Article\&state=default\&region=MAIN_CONTENT_1\&context=storylines_live_updates\#link-794484ed}{Mississippians
  must now wear masks in public, governor says.}
\end{itemize}

\href{https://www.nytimes.com/2020/08/04/world/coronavirus-cases.html?action=click\&pgtype=Article\&state=default\&region=MAIN_CONTENT_1\&context=storylines_live_updates}{See
more updates}

More live coverage:
\href{https://www.nytimes.com/live/2020/08/04/business/stock-market-today-coronavirus?action=click\&pgtype=Article\&state=default\&region=MAIN_CONTENT_1\&context=storylines_live_updates}{Markets}

The company also fell short of the number of patients it originally said
in a
\href{https://clinicaltrials.gov/ct2/show/record/NCT04385095?term=interferon\&cond=covid-19\&draw=2\&view=record}{filing
on a U.S. government website} that it intended to enroll. That raised
concerns among scientists that the company had analyzed the results
earlier than it should have, once the treatment appeared to be
effective. They said a much larger randomized trial was needed before
they could assess the drug.

Scientists also noted that while injectable interferon has historically
been used to treat hepatitis infections, the inhaled form of treatment
was not yet licensed or widely available.

Still, amid a crisis caused by a disease with no known cure, the results
were tantalizing. If the finding is borne out, it may represent one of
the most significant breakthroughs to date in treating Covid-19
patients, virologists said.

``If there is the material to distribute it to the population, and you
could keep the price down, this could absolutely be a game changer,''
said Benjamin tenOever, a professor of microbiology at the Icahn School
of Medicine at Mount Sinai in New York. ``I don't doubt it will work. I
just don't know how feasible it is.''

Professor tenOever was a co-author of a
\href{https://www.cell.com/cell/fulltext/S0092-8674(20)30489-X}{study in
May in Cell}, a scientific journal, about how the virus blocks the
body's interferon response. He said evidence was piling up that
administering interferon could help limit the replication of the virus,
especially in the early stages of illness, fending the virus off for
long enough that a second set of genes could successfully eradicate it.

In hamsters, Professor tenOever said, there were signs that interferon
cleared the virus and blocked onward transmission.

\href{https://www.nytimes.com/interactive/2020/science/coronavirus-drugs-treatments.html}{}

\includegraphics{https://static01.nyt.com/images/2020/07/14/us/coronavirus-drugs-treatments-promo-1594761806092/coronavirus-drugs-treatments-promo-1594761806092-articleLarge-v12.png}

\hypertarget{coronavirus-drug-and-treatment-tracker}{%
\subsection{Coronavirus Drug and Treatment
Tracker}\label{coronavirus-drug-and-treatment-tracker}}

An updated list of potential treatments for Covid-19.

In China, the early results of a
\href{https://www.medrxiv.org/content/10.1101/2020.04.11.20061473v2}{study
among medical workers} also showed promise, concluding that interferon
nasal drops ``may effectively prevent Covid-19 in medical staff.'' The
study found that the drops ``have potential promise for protecting
susceptible healthy people during the coronavirus pandemic.''

Among the most pressing challenges for British and American researchers
studying interferon is the difficulty in recruiting patients in places
where caseloads have fallen. Professor tenOever said one clinical trial
planned at Mount Sinai had to be scuppered after coronavirus beds
emptied out.

Synairgen, whose share price soared after it announced the results on
Monday, is now struggling to recruit Covid-19 patients who are at home.

Stuart Neil, a professor of virology at King's College London, said
there had been fears early in the pandemic that giving interferon to
patients could worsen the over-aggressive immune response that was
itself sickening some of them. But more recent findings have indicated
that, in fact, infected patients mount a limited interferon response on
their own. That insight has laid the groundwork for studies like
Synairgen's.

\href{https://www.nytimes.com/news-event/coronavirus?action=click\&pgtype=Article\&state=default\&region=MAIN_CONTENT_3\&context=storylines_faq}{}

\hypertarget{the-coronavirus-outbreak-}{%
\subsubsection{The Coronavirus Outbreak
›}\label{the-coronavirus-outbreak-}}

\hypertarget{frequently-asked-questions}{%
\paragraph{Frequently Asked
Questions}\label{frequently-asked-questions}}

Updated August 4, 2020

\begin{itemize}
\item ~
  \hypertarget{i-have-antibodies-am-i-now-immune}{%
  \paragraph{I have antibodies. Am I now
  immune?}\label{i-have-antibodies-am-i-now-immune}}

  \begin{itemize}
  \tightlist
  \item
    As of right
    now,\href{https://www.nytimes.com/2020/07/22/health/covid-antibodies-herd-immunity.html?action=click\&pgtype=Article\&state=default\&region=MAIN_CONTENT_3\&context=storylines_faq}{that
    seems likely, for at least several months.} There have been
    frightening accounts of people suffering what seems to be a second
    bout of Covid-19. But experts say these patients may have a
    drawn-out course of infection, with the virus taking a slow toll
    weeks to months after initial exposure. People infected with the
    coronavirus typically
    \href{https://www.nature.com/articles/s41586-020-2456-9}{produce}
    immune molecules called antibodies, which are
    \href{https://www.nytimes.com/2020/05/07/health/coronavirus-antibody-prevalence.html?action=click\&pgtype=Article\&state=default\&region=MAIN_CONTENT_3\&context=storylines_faq}{protective
    proteins made in response to an
    infection}\href{https://www.nytimes.com/2020/05/07/health/coronavirus-antibody-prevalence.html?action=click\&pgtype=Article\&state=default\&region=MAIN_CONTENT_3\&context=storylines_faq}{.
    These antibodies may} last in the body
    \href{https://www.nature.com/articles/s41591-020-0965-6}{only two to
    three months}, which may seem worrisome, but that's perfectly normal
    after an acute infection subsides, said Dr. Michael Mina, an
    immunologist at Harvard University. It may be possible to get the
    coronavirus again, but it's highly unlikely that it would be
    possible in a short window of time from initial infection or make
    people sicker the second time.
  \end{itemize}
\item ~
  \hypertarget{im-a-small-business-owner-can-i-get-relief}{%
  \paragraph{I'm a small-business owner. Can I get
  relief?}\label{im-a-small-business-owner-can-i-get-relief}}

  \begin{itemize}
  \tightlist
  \item
    The
    \href{https://www.nytimes.com/article/small-business-loans-stimulus-grants-freelancers-coronavirus.html?action=click\&pgtype=Article\&state=default\&region=MAIN_CONTENT_3\&context=storylines_faq}{stimulus
    bills enacted in March} offer help for the millions of American
    small businesses. Those eligible for aid are businesses and
    nonprofit organizations with fewer than 500 workers, including sole
    proprietorships, independent contractors and freelancers. Some
    larger companies in some industries are also eligible. The help
    being offered, which is being managed by the Small Business
    Administration, includes the Paycheck Protection Program and the
    Economic Injury Disaster Loan program. But lots of folks have
    \href{https://www.nytimes.com/interactive/2020/05/07/business/small-business-loans-coronavirus.html?action=click\&pgtype=Article\&state=default\&region=MAIN_CONTENT_3\&context=storylines_faq}{not
    yet seen payouts.} Even those who have received help are confused:
    The rules are draconian, and some are stuck sitting on
    \href{https://www.nytimes.com/2020/05/02/business/economy/loans-coronavirus-small-business.html?action=click\&pgtype=Article\&state=default\&region=MAIN_CONTENT_3\&context=storylines_faq}{money
    they don't know how to use.} Many small-business owners are getting
    less than they expected or
    \href{https://www.nytimes.com/2020/06/10/business/Small-business-loans-ppp.html?action=click\&pgtype=Article\&state=default\&region=MAIN_CONTENT_3\&context=storylines_faq}{not
    hearing anything at all.}
  \end{itemize}
\item ~
  \hypertarget{what-are-my-rights-if-i-am-worried-about-going-back-to-work}{%
  \paragraph{What are my rights if I am worried about going back to
  work?}\label{what-are-my-rights-if-i-am-worried-about-going-back-to-work}}

  \begin{itemize}
  \tightlist
  \item
    Employers have to provide
    \href{https://www.osha.gov/SLTC/covid-19/standards.html}{a safe
    workplace} with policies that protect everyone equally.
    \href{https://www.nytimes.com/article/coronavirus-money-unemployment.html?action=click\&pgtype=Article\&state=default\&region=MAIN_CONTENT_3\&context=storylines_faq}{And
    if one of your co-workers tests positive for the coronavirus, the
    C.D.C.} has said that
    \href{https://www.cdc.gov/coronavirus/2019-ncov/community/guidance-business-response.html}{employers
    should tell their employees} -\/- without giving you the sick
    employee's name -\/- that they may have been exposed to the virus.
  \end{itemize}
\item ~
  \hypertarget{should-i-refinance-my-mortgage}{%
  \paragraph{Should I refinance my
  mortgage?}\label{should-i-refinance-my-mortgage}}

  \begin{itemize}
  \tightlist
  \item
    \href{https://www.nytimes.com/article/coronavirus-money-unemployment.html?action=click\&pgtype=Article\&state=default\&region=MAIN_CONTENT_3\&context=storylines_faq}{It
    could be a good idea,} because mortgage rates have
    \href{https://www.nytimes.com/2020/07/16/business/mortgage-rates-below-3-percent.html?action=click\&pgtype=Article\&state=default\&region=MAIN_CONTENT_3\&context=storylines_faq}{never
    been lower.} Refinancing requests have pushed mortgage applications
    to some of the highest levels since 2008, so be prepared to get in
    line. But defaults are also up, so if you're thinking about buying a
    home, be aware that some lenders have tightened their standards.
  \end{itemize}
\item ~
  \hypertarget{what-is-school-going-to-look-like-in-september}{%
  \paragraph{What is school going to look like in
  September?}\label{what-is-school-going-to-look-like-in-september}}

  \begin{itemize}
  \tightlist
  \item
    It is unlikely that many schools will return to a normal schedule
    this fall, requiring the grind of
    \href{https://www.nytimes.com/2020/06/05/us/coronavirus-education-lost-learning.html?action=click\&pgtype=Article\&state=default\&region=MAIN_CONTENT_3\&context=storylines_faq}{online
    learning},
    \href{https://www.nytimes.com/2020/05/29/us/coronavirus-child-care-centers.html?action=click\&pgtype=Article\&state=default\&region=MAIN_CONTENT_3\&context=storylines_faq}{makeshift
    child care} and
    \href{https://www.nytimes.com/2020/06/03/business/economy/coronavirus-working-women.html?action=click\&pgtype=Article\&state=default\&region=MAIN_CONTENT_3\&context=storylines_faq}{stunted
    workdays} to continue. California's two largest public school
    districts --- Los Angeles and San Diego --- said on July 13, that
    \href{https://www.nytimes.com/2020/07/13/us/lausd-san-diego-school-reopening.html?action=click\&pgtype=Article\&state=default\&region=MAIN_CONTENT_3\&context=storylines_faq}{instruction
    will be remote-only in the fall}, citing concerns that surging
    coronavirus infections in their areas pose too dire a risk for
    students and teachers. Together, the two districts enroll some
    825,000 students. They are the largest in the country so far to
    abandon plans for even a partial physical return to classrooms when
    they reopen in August. For other districts, the solution won't be an
    all-or-nothing approach.
    \href{https://bioethics.jhu.edu/research-and-outreach/projects/eschool-initiative/school-policy-tracker/}{Many
    systems}, including the nation's largest, New York City, are
    devising
    \href{https://www.nytimes.com/2020/06/26/us/coronavirus-schools-reopen-fall.html?action=click\&pgtype=Article\&state=default\&region=MAIN_CONTENT_3\&context=storylines_faq}{hybrid
    plans} that involve spending some days in classrooms and other days
    online. There's no national policy on this yet, so check with your
    municipal school system regularly to see what is happening in your
    community.
  \end{itemize}
\end{itemize}

``It's very exciting,'' Professor Neil said. ``By basically inhaling the
interferon into the site of infection, it looks like you're taking the
edge off the virus.''

Synairgen said that over the two-week treatment period, patients
receiving the interferon beta drug were twice as likely as patients who
received a placebo to recover to the point where they were no longer
limited by their illness. The company also said that breathlessness was
lower in patients receiving the drug.

But the sparsely detailed news release announcing the findings left many
questions unanswered. Among other concerns, scientists noted that the
company was mostly reporting outcomes over the course of a two-week
treatment period only. They warned that things could change later.

Synairgen, founded by researchers at the University of Southampton in
southern England, said that as a listed company, it was obligated by
stock market rules to report the early results of its trial.

The company said that it had originally developed the inhaled form of
interferon for patients who are especially susceptible to seasonal colds
and the flu, among them people with asthma and with chronic obstructive
pulmonary disease.

``We have recognized that as a broad spectrum antiviral, it could always
have a place if an unwanted, highly pathogenic virus emerged,'' Richard
Marsden, the chief executive of Synairgen, told reporters on Monday. He
said the company would work with regulators to make progress on the
development of the drug as quickly as possible.

Scientists believe the interferon drug will work most effectively on
patients who are not yet seriously ill. By contrast, another drug,
dexamethasone, has been shown to help more severely ill patients.

Dexamethasone is already being used at American and British hospitals to
treat coronavirus patients, doctors have said, and the World Health
Organization called for accelerating production to ensure an adequate
supply.

But the initial reports by scientists of the benefits of dexamethasone
also showed the hazards of conducting science by news release. When the
full study about dexamethasone was posted online --- after doctors had
already begun prescribing it --- it reported that while the drug seems
to help patients in dire condition, it might be risky for patients with
milder illness.

Advertisement

\protect\hyperlink{after-bottom}{Continue reading the main story}

\hypertarget{site-index}{%
\subsection{Site Index}\label{site-index}}

\hypertarget{site-information-navigation}{%
\subsection{Site Information
Navigation}\label{site-information-navigation}}

\begin{itemize}
\tightlist
\item
  \href{https://help.nytimes.com/hc/en-us/articles/115014792127-Copyright-notice}{©~2020~The
  New York Times Company}
\end{itemize}

\begin{itemize}
\tightlist
\item
  \href{https://www.nytco.com/}{NYTCo}
\item
  \href{https://help.nytimes.com/hc/en-us/articles/115015385887-Contact-Us}{Contact
  Us}
\item
  \href{https://www.nytco.com/careers/}{Work with us}
\item
  \href{https://nytmediakit.com/}{Advertise}
\item
  \href{http://www.tbrandstudio.com/}{T Brand Studio}
\item
  \href{https://www.nytimes.com/privacy/cookie-policy\#how-do-i-manage-trackers}{Your
  Ad Choices}
\item
  \href{https://www.nytimes.com/privacy}{Privacy}
\item
  \href{https://help.nytimes.com/hc/en-us/articles/115014893428-Terms-of-service}{Terms
  of Service}
\item
  \href{https://help.nytimes.com/hc/en-us/articles/115014893968-Terms-of-sale}{Terms
  of Sale}
\item
  \href{https://spiderbites.nytimes.com}{Site Map}
\item
  \href{https://help.nytimes.com/hc/en-us}{Help}
\item
  \href{https://www.nytimes.com/subscription?campaignId=37WXW}{Subscriptions}
\end{itemize}
