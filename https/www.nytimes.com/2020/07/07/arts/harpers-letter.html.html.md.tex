Sections

SEARCH

\protect\hyperlink{site-content}{Skip to
content}\protect\hyperlink{site-index}{Skip to site index}

\href{https://www.nytimes.com/section/arts}{Arts}

\href{https://myaccount.nytimes.com/auth/login?response_type=cookie\&client_id=vi}{}

\href{https://www.nytimes.com/section/todayspaper}{Today's Paper}

\href{/section/arts}{Arts}\textbar{}Artists and Writers Warn of an
`Intolerant Climate.' Reaction Is Swift.

\url{https://nyti.ms/2ZME5Bd}

\begin{itemize}
\item
\item
\item
\item
\item
\item
\end{itemize}

\href{https://www.nytimes.com/news-event/george-floyd-protests-minneapolis-new-york-los-angeles?action=click\&pgtype=Article\&state=default\&region=TOP_BANNER\&context=storylines_menu}{Race
and America}

\begin{itemize}
\tightlist
\item
  \href{https://www.nytimes.com/2020/07/26/us/protests-portland-seattle-trump.html?action=click\&pgtype=Article\&state=default\&region=TOP_BANNER\&context=storylines_menu}{Protesters
  Return to Other Cities}
\item
  \href{https://www.nytimes.com/2020/07/24/us/portland-oregon-protests-white-race.html?action=click\&pgtype=Article\&state=default\&region=TOP_BANNER\&context=storylines_menu}{Portland
  at the Center}
\item
  \href{https://www.nytimes.com/2020/07/23/podcasts/the-daily/portland-protests.html?action=click\&pgtype=Article\&state=default\&region=TOP_BANNER\&context=storylines_menu}{Podcast:
  Showdown in Portland}
\item
  \href{https://www.nytimes.com/interactive/2020/07/16/us/black-lives-matter-protests-louisville-breonna-taylor.html?action=click\&pgtype=Article\&state=default\&region=TOP_BANNER\&context=storylines_menu}{45
  Days in Louisville}
\end{itemize}

Advertisement

\protect\hyperlink{after-top}{Continue reading the main story}

Supported by

\protect\hyperlink{after-sponsor}{Continue reading the main story}

\hypertarget{artists-and-writers-warn-of-an-intolerant-climate-reaction-is-swift}{%
\section{Artists and Writers Warn of an `Intolerant Climate.' Reaction
Is
Swift.}\label{artists-and-writers-warn-of-an-intolerant-climate-reaction-is-swift}}

An open letter published by Harper's, signed by luminaries including
Margaret Atwood and Wynton Marsalis, argued for openness to ``opposing
views.'' The debate began immediately.

\includegraphics{https://static01.nyt.com/images/2020/07/07/arts/07HARPERS-LETTER-PROMO-GRID/07HARPERS-LETTER-PROMO-GRID-articleLarge.jpg?quality=75\&auto=webp\&disable=upscale}

\href{https://www.nytimes.com/by/jennifer-schuessler}{\includegraphics{https://static01.nyt.com/images/2018/02/16/multimedia/author-jennifer-schuessler/author-jennifer-schuessler-thumbLarge-v2.png}}\href{https://www.nytimes.com/by/elizabeth-a-harris}{\includegraphics{https://static01.nyt.com/images/2018/02/16/multimedia/author-elizabeth-a-harris/author-elizabeth-a-harris-thumbLarge.jpg}}

By \href{https://www.nytimes.com/by/jennifer-schuessler}{Jennifer
Schuessler} and
\href{https://www.nytimes.com/by/elizabeth-a-harris}{Elizabeth A.
Harris}

\begin{itemize}
\item
  July 7, 2020
\item
  \begin{itemize}
  \item
  \item
  \item
  \item
  \item
  \item
  \end{itemize}
\end{itemize}

The killing of George Floyd has brought an intense moment of racial
reckoning in the United States. As protests spread across the country,
they have been accompanied by open letters calling for ---~and promising
---~change at white-dominated institutions across the arts and academia.

But on Tuesday, a different type of letter appeared online.
\href{https://harpers.org/a-letter-on-justice-and-open-debate/}{Titled
``A Letter on Justice and Open Debate,'' and signed by 153 prominent
artists} and intellectuals, it began with an acknowledgment of
``powerful protests for racial and social justice'' before pivoting to a
warning against an ``intolerant climate'' engulfing the culture.

``The free exchange of information and ideas, the lifeblood of a liberal
society, is daily becoming more constricted,'' the letter declared,
citing ``an intolerance of opposing views, a vogue for public shaming
and ostracism and the tendency to dissolve complex policy issues in a
blinding moral certainty.''

``We refuse any false choice between justice and freedom, which cannot
exist without each other,'' it continues. ``As writers we need a culture
that leaves us room for experimentation, risk taking, and even
mistakes.''

The letter, which was published by Harper's Magazine and will also
appear in several leading international publications, surfaces a debate
that has been going on privately in newsrooms, universities and
publishing houses that have been navigating demands for diversity and
inclusion, while also asking which demands --- and the social media
dynamics that propel them --- go too far.

And on social media, the reaction was swift, with some heaping ridicule
on the letter's signatories --- who include cultural luminaries like
Margaret Atwood, Bill T. Jones and Wynton Marsalis, along with
journalists and academics --- for thin-skinnedness, privilege and, as
\href{https://twitter.com/dstfelix/status/1280500908611325958?s=20}{one
person} put it, fear of loss of ``relevance.''

``Okay, I did not sign THE LETTER when I was asked 9 days ago,'' Richard
Kim, the enterprise director of HuffPost, said on Twitter, ``because I
could see in 90 seconds that it was fatuous, self-important drivel that
would only troll the people it allegedly was trying to reach --- and I
said as much.''

The debate over diversity, free expression and the limits of acceptable
opinion is a long-burning one. But the letter, which was spearheaded by
the writer Thomas Chatterton Williams, began taking shape about a month
ago, as part of a long-running conversation about these issues with a
small group of writers including the historian David Greenberg, the
writer Mark Lilla and the journalists Robert Worth and George Packer.

Mr. Williams, a columnist for Harper's and contributing writer for The
New York Times Magazine, said that initially, there was concern over
timing.

``We didn't want to be seen as reacting to the protests we believe are
in response to egregious abuses by the police,'' he said. ``But for some
time, there's been a mood all of us have been quite concerned with.''

\includegraphics{https://static01.nyt.com/images/2020/07/07/arts/07harpers-letter2/merlin_174326472_8999c0fc-0cb2-4209-a99d-2756d10c75c2-articleLarge.jpg?quality=75\&auto=webp\&disable=upscale}

He said there wasn't one particular incident that provoked the letter.
But he did cite several recent ones, including the
\href{https://www.vulture.com/2020/06/national-book-critics-circle-resignations.html}{resignation
of more than half the board}of the National Book Critics Circle over its
statement supporting Black Lives Matter, a
\href{https://www.nytimes.com/2020/06/09/books/poetry-foundation-black-lives-matter.html}{similar
blowup} at the Poetry Foundation, and the case of
\href{https://nymag.com/intelligencer/2020/06/case-for-liberalism-tom-cotton-new-york-times-james-bennet.html}{David
Shor}, a data analyst at a consulting firm who was fired after he
\href{https://csdp.princeton.edu/news/wasow-research-widely-covered-media-how-1960s-black-protests-moved-elites-public-opinion-and}{tweeted
about academic research linking looting and vandalism} by protesters to
Richard Nixon's 1968 electoral victory.

Such incidents, Mr. Williams said, both fueled and echoed what he called
the far greater and more dangerous ``illiberalism'' of President Trump.

``Donald Trump is the Canceler in Chief,'' he said. ``But the correction
of Trump's abuses cannot become an overcorrection that stifles the
principles we believe in.''

Mr. Williams said the letter was very much a crowdsourced effort, with
about 20 people contributing language. Then it was circulated more
broadly for signatures, in what he describes as a process that was both
``organic'' and aimed at getting a group that was maximally diverse
politically, racially and otherwise.

``We're not just a bunch of old white guys sitting around writing this
letter,'' Mr. Williams, who is African-American, said. ``It includes
plenty of Black thinkers, Muslim thinkers, Jewish thinkers, people who
are trans and gay, old and young, right wing and left wing.''

``We believe these are values that are widespread and shared, and we
wanted the list to reflect that,'' he said.

Signatories include the leftist Noam Chomsky and the neoconservative
Francis Fukuyama. There are also figures associated with the traditional
defense of free speech, including Nadine Strossen, former president of
the American Civil Liberties Union, as well as some outspoken critics of
political correctness on campuses, including the linguist Steven Pinker
and the psychologist Jonathan Haidt.

The signers also include some figures who have lost positions amid
controversies, including Ian Buruma, the former editor of the New York
Review of Books, and Ronald S. Sullivan Jr., a Harvard Law School
professor who
\href{https://www.nytimes.com/2019/05/11/us/ronald-sullivan-harvard.html}{left
his position} as faculty dean of an undergraduate residence amid
protests over his legal defense of Harvey Weinstein.

There are also some leading Black intellectuals, including the historian
Nell Irvin Painter, the poets Reginald Dwayne Betts and Gregory Pardlo,
and the linguist John McWhorter. And there are a number of journalists,
including several opinion columnists for The New York Times.

Nicholas Lemann, a staff writer for The New Yorker and a former dean of
Columbia Journalism School, said that he rarely signs letters, but
thought this one was important.

``What concerns me is a sense that a lot of people out there seem to
think open argument over everything is an unhealthy thing,'' he said.
``I've spent my whole life having vigorous arguments with people I
disagree with, and don't want to think we are moving out of this
world.''

The principle of open argument, he added, becomes especially important
outside liberal-leaning enclaves, ``where people don't have the option
of shutting down these supposedly completely unacceptable views.''

Mr. Pardlo said that as somebody who has felt the ``chilling effect'' of
being the only person of color in predominantly white institutions, he
hoped the letter would spark conversation about those ``chilling forces,
no matter where they come from.''

He said he was surprised by some of the blowback to the letter.

``It seems some of the conversation has turned to who the signatories
are more than the content of the letter,'' he said.

There was particularly strong blowback over the inclusion of J.K.
Rowling, who has come under fierce criticism over a series of comments
widely seen as anti-transgender.

Emily VanDerWerff, a critic at large at Vox who is transgender, posted
on Twitter a letter she said she had sent to her editors, criticizing
the fact that the Vox writer Matthew Yglesias had signed the letter,
which she said was also signed by ``several prominent anti-trans
voices'' --- but noted that she was not calling for Mr. Yglesias to be
fired or reprimanded.

Doing so ``would only solidify, in his own mind, the belief that he is
being martyred,'' she wrote.

Mr. Yglesias declined to comment except to say that he has long admired
Ms. VanDerWerff's work and continued to ``respect her enormously.''

Amid the intense criticism, some signatories appeared to back away from
the letter. On Tuesday evening, the historian Kerri K. Greenidge
\href{https://twitter.com/GreenidgeKerri/status/1280608456152678406?s=20}{tweeted}``I
do not endorse this @Harpers letter,'' and said she was in touch with
the magazine about a retraction. (Giulia Melucci, a spokeswoman for
Harper's, said the magazine had fact-checked all signatures and that Dr.
Greenidge had signed off. But she said the magazine is ``respectfully
removing her name.'')

Another person who signed, who spoke on the condition of anonymity in an
effort to stay out of the growing storm, said she did not know who all
the other signatories were when she agreed to participate, and if she
had, she may not have signed. She also said that the letter, which was
about internet shaming, among other things, was now being used to shame
people on the internet.

But Mr. Betts, the director of the
\href{https://www.nytimes.com/2020/06/30/arts/mellon-foundation-elizabeth-alexander.html}{Million
Books Project}, a new effort aimed at getting book collections to more
than 1,000 prisons, was unfazed by the variety of signers.

``I'm rolling with people I wouldn't normally be in a room with,'' he
said. ``But you need to concede that what's in the letter is worthy of
some thought.''

He said that as someone who had spent more than eight years in prison
for a carjacking committed when he was a teenager, he was given pause by
what he called the unforgiving nature of the current moment. ``It's
antithetical to my notion of how we need to deal with problems in
society,'' he said.

He cited in particular the case of James Bennet,
\href{https://www.nytimes.com/2020/06/07/business/media/james-bennet-resigns-nytimes-op-ed.html}{who
resigned as the editorial page editor of The New York Times} following
an outcry over an Op-Ed by Senator Tom Cotton, Republican of Arkansas,
and cases of authors of young adult literature withdrawing books in the
face of criticism over cultural appropriation.

``You can criticize what people say, you can argue about platforms,''
Mr. Betts said. ``But it seems like some of the excesses of the moment
are leading people to be silenced in a new way.''

Eileen Murphy, a Times spokeswoman, declined to comment.

Mr. Williams said he was trying to think through how outrage over Mr.
Floyd's death had become so intertwined with calls for change ``at
organizations that don't have much to do with the situation George Floyd
found himself in.''

But to him, he said, the cause of the letter is clear: ``It's a defense
of people being able to speak and think freely without fear of
punishment or retribution, of the right to disagree and not fear for
your employment.''

Advertisement

\protect\hyperlink{after-bottom}{Continue reading the main story}

\hypertarget{site-index}{%
\subsection{Site Index}\label{site-index}}

\hypertarget{site-information-navigation}{%
\subsection{Site Information
Navigation}\label{site-information-navigation}}

\begin{itemize}
\tightlist
\item
  \href{https://help.nytimes.com/hc/en-us/articles/115014792127-Copyright-notice}{©~2020~The
  New York Times Company}
\end{itemize}

\begin{itemize}
\tightlist
\item
  \href{https://www.nytco.com/}{NYTCo}
\item
  \href{https://help.nytimes.com/hc/en-us/articles/115015385887-Contact-Us}{Contact
  Us}
\item
  \href{https://www.nytco.com/careers/}{Work with us}
\item
  \href{https://nytmediakit.com/}{Advertise}
\item
  \href{http://www.tbrandstudio.com/}{T Brand Studio}
\item
  \href{https://www.nytimes.com/privacy/cookie-policy\#how-do-i-manage-trackers}{Your
  Ad Choices}
\item
  \href{https://www.nytimes.com/privacy}{Privacy}
\item
  \href{https://help.nytimes.com/hc/en-us/articles/115014893428-Terms-of-service}{Terms
  of Service}
\item
  \href{https://help.nytimes.com/hc/en-us/articles/115014893968-Terms-of-sale}{Terms
  of Sale}
\item
  \href{https://spiderbites.nytimes.com}{Site Map}
\item
  \href{https://help.nytimes.com/hc/en-us}{Help}
\item
  \href{https://www.nytimes.com/subscription?campaignId=37WXW}{Subscriptions}
\end{itemize}
