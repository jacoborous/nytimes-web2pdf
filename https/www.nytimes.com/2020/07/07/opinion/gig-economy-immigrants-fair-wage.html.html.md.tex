Sections

SEARCH

\protect\hyperlink{site-content}{Skip to
content}\protect\hyperlink{site-index}{Skip to site index}

\href{https://myaccount.nytimes.com/auth/login?response_type=cookie\&client_id=vi}{}

\href{https://www.nytimes.com/section/todayspaper}{Today's Paper}

\href{/section/opinion}{Opinion}\textbar{}`When Someone Hires Me, They
Get the Boss Herself'

\href{https://nyti.ms/2VS92mD}{https://nyti.ms/2VS92mD}

\begin{itemize}
\item
\item
\item
\item
\item
\item
\end{itemize}

Advertisement

\protect\hyperlink{after-top}{Continue reading the main story}

\href{/section/opinion}{Opinion}

Supported by

\protect\hyperlink{after-sponsor}{Continue reading the main story}

Fixes

\hypertarget{when-someone-hires-me-they-get-the-boss-herself}{%
\section{`When Someone Hires Me, They Get the Boss
Herself'}\label{when-someone-hires-me-they-get-the-boss-herself}}

Can a sophisticated platform cooperative help minimize exploitive
working conditions in the gig economy, even during the pandemic?

By Michaela Haas

Dr. Haas is a journalist.

\begin{itemize}
\item
  July 7, 2020
\item
  \begin{itemize}
  \item
  \item
  \item
  \item
  \item
  \item
  \end{itemize}
\end{itemize}

\includegraphics{https://static01.nyt.com/images/2020/07/07/opinion/07FixesHass/07FixesHass-articleLarge.jpg?quality=75\&auto=webp\&disable=upscale}

Maria Carmen Tapia has learned a host of new skills in the last few
months. The 42-year-old housekeeper and all her colleagues at the
housekeeping app \href{https://www.upandgo.coop/}{Up \& Go} were trained
by Occupational Safety and Health Administration-authorized trainers,
learning to put on protective gear correctly and establish safety
protocols to keep themselves and their clients safe during the pandemic.

At first glance, Up \& Go resembles any other housekeeping app: When a
client requests a cleaning service in New York City, a few clicks bring
a trained housekeeper like Ms. Tapia to the door with her scrubber,
broom and eco-sprays.

But Up \& Go is different from most gig economy platforms. Ms. Tapia is
not an underpaid worker eking out a meager living in an expensive city,
but an owner of her own enterprise. ``When someone hires me, they get
the boss herself,'' she said.

Up \& Go is a new platform model for cooperatives. In 2017, the
nonprofit Center for Family Life, a neighborhood-based family and social
services organization in Sunset Park, Brooklyn, invited Ms. Tapia to
join. Now she is one of 51 housekeepers, all Latin American immigrants,
who own the platform. ``We want to bring fair work to the workers by
bringing the co-ops into the 21st century,'' said Sylvia Morse, the
project's coordinator.

The owner-workers share offices, customer service representatives and
the app. They talk every two weeks to answer questions like: Should we
include more housekeepers and expand the cooperative? During New York
City's stay-at-home order, they had to suspend residential cleanings,
which is the core of their business model. But Ms. Tapia and her
colleagues made up for some of the losses by fulfilling Up \& Go's
contracts for commercial cleanings, and reopened for domestic cleaning
on July 1.

The Center for Family Life has worked in Sunset Park since 1978. It
offers counseling, child care and senior care --- and for the last 12
years it has tried to raise workers' wages by helping them form
cooperatives. Instead of the \$11 to \$12 an hour Ms. Tapia was earning
when she found clients through fliers, she now usually makes \$25 an
hour with Up \& Go.

For a \$135 fee she does a five-hour deep cleaning of a one-bedroom
apartment: She scrubs the bathroom, polishes the floors. ``If a house
requires more work, we raise the fee,'' Ms. Tapia said. A mother of two,
she has worked as a housekeeper and babysitter since she emigrated from
Ecuador with her mother 21 years ago. ``I was barely making minimum
wage,'' she said. ``My life has changed drastically.''

Cooperatives have been booming in
\href{https://www1.nyc.gov/nycbusiness/article/worker-cooperatives}{New
York}. Landscapers, nurses and painters increasingly organize themselves
in co-ops. But Up \& Go is unique in combining the traditional co-op
structure with the online platform and app technology of the gig
economy. It tries to pick the best of two worlds: For clients, the ease
of an app to book help quickly. For workers, the security of regular
jobs plus the pride of being an entrepreneur.

The workers agreed on safety protocols as soon as the threat of Covid-19
became clear. ``In this stressful time,'' Ms. Morse said, ``it makes a
big difference that workers have the space to share, establish best
practices.'' To make up for the loss of income, the center did extra
fund-raising, extended the hours of its food pantry and connected those
who struggle to pay their bills with emergency funds.

``Early on, the big challenge was to make sure the co-op members had the
right information, took precautions to protect themselves and their
customers,'' Ms. Morse said. ``This is where we really saw the power of
the co-op. We had these systems in place. Everything was grounded in the
worker's experience, rather than `we need to make as much money as
possible; let's send people out even if they don't have adequate
protective gear.''' So far none of Up \& Go's workers has contracted
Covid-19.

The gig economy, with platforms like Uber, Handy and TaskRabbit, seemed
to offer opportunities for freelancers and unskilled workers when it
took off after the 2008 recession. But economists soon
\href{https://www.nytimes.com/2017/04/10/opinion/the-gig-economys-false-promise.html?searchResultPosition=1}{warned
against pitfalls}, and the pandemic has
\href{https://www.nytimes.com/2020/03/18/technology/gig-economy-pandemic.html}{disproportionately
affected gig economy workers}. As independent contractors, they do not
qualify for basic protections like minimum wages. The pandemic has put
many of them out of work with no safety net or forced some to consider
working
\href{https://www.nytimes.com/2020/03/30/business/economy/coronavirus-instacart-amazon.html}{without
adequate protection}.

By contrast, the Up \& Go worker-owners distribute the existing jobs
fairly among themselves, and every worker has agreed to stay home should
they develop any symptoms. ``A big difference is that we know our rights
better,'' said Ms. Tapia. ``We are not only workers but entrepreneurs.
That changed who I am and how I see myself. I am constantly learning new
things --- how to run a business, how to handle a democratic
decision-making process, what good management is.'' She also appreciates
that the co-op members decide collectively whether to approve new
members and that if she falls ill, ``another worker will keep my spot
open.''

The 51 Up \& Go worker-owners, only two of whom are men, speak very
little English. Ms. Tapia knows enough English to arrange basic
housekeeping but leaves more complex conversations and conflict
resolution to professional customer representatives. ``It is exactly
this clientele we want to serve with Up \& Go,'' Ms. Morse emphasized.
``Historically, this type of work has been done by immigrants and
predominantly by women. Racism and sexism have devalued this work.''

Development costs are the biggest hurdle in copying the Up \& Go model.
Barclays and \href{https://www.robinhood.org/}{Robin Hood}, an
anti-poverty organization in New York City, funded the start-up with
\$500,000 and invested again this year. Tech experts from
\href{https://colab.coop/}{CoLab Cooperative} improved the app. This
financial and logistical support allows the Up \& Go owners to keep 95
percent of their income; 5 percent goes to support the infrastructure.

``Who owns the technology? Who decides how it is designed? How does it
influence the quality of life for workers in the gig economy?'' Ms.
Morse said. ``When we look at the change in the working world, these
questions are important.''

This model will work long-term only if enough consumers are willing to
pay a higher fee to be certain that their housekeepers are legally
documented and fairly paid. ``I've seen \$30 online specials for an
initial cleaning,'' Ms. Morse said. ``Up \& Go just can't do that.''

Instead, Up \& Go offers quality and trust. ``It's a matter of trust to
let someone into your home,'' she said. ``With Up \& Go, you know that
the person has not just downloaded an app but is a co-owner who went
through a training process. The sense of accountability and safety is
huge.''

To be sure, while its website describes the multiple precautions the
worker-owners take to avoid spreading or contracting Covid-19, it is
also frank with prospective customers that there is always a risk of
contagion. So Up \& Go, like many service enterprises, requires
customers to sign a liability waiver before hiring their cleaners.

Before the pandemic, Ms. Morse said, Up \& Go was likely to turn a
profit this year and expand. Now she hopes that commercial cleanings
will keep the start-up afloat and that other cities and industries will
copy the model. She found it especially encouraging that nearly 50
percent of its residential clients continue to pay for the services to
support the workers even if they could not receive their services during
the lockdown.

Michaela Haas, Ph.D, is a journalist and the author, most recently, of
``Bouncing Forward: The Art \& Science of Cultivating Resilience.''

\emph{To receive email alerts for Fixes columns, sign up}
\href{http://eepurl.com/ABIxL}{\emph{here.}}

\emph{The Times is committed to publishing}
\href{https://www.nytimes.com/2019/01/31/opinion/letters/letters-to-editor-new-york-times-women.html}{\emph{a
diversity of letters}} \emph{to the editor. We'd like to hear what you
think about this or any of our articles. Here are some}
\href{https://help.nytimes.com/hc/en-us/articles/115014925288-How-to-submit-a-letter-to-the-editor}{\emph{tips}}\emph{.
And here's our email:}
\href{mailto:letters@nytimes.com}{\emph{letters@nytimes.com}}\emph{.}

\emph{Follow The New York Times Opinion section on}
\href{https://www.facebook.com/nytopinion}{\emph{Facebook}}\emph{,}
\href{http://twitter.com/NYTOpinion}{\emph{Twitter (@NYTopinion)}}
\emph{and}
\href{https://www.instagram.com/nytopinion/}{\emph{Instagram}}\emph{.}

Advertisement

\protect\hyperlink{after-bottom}{Continue reading the main story}

\hypertarget{site-index}{%
\subsection{Site Index}\label{site-index}}

\hypertarget{site-information-navigation}{%
\subsection{Site Information
Navigation}\label{site-information-navigation}}

\begin{itemize}
\tightlist
\item
  \href{https://help.nytimes.com/hc/en-us/articles/115014792127-Copyright-notice}{©~2020~The
  New York Times Company}
\end{itemize}

\begin{itemize}
\tightlist
\item
  \href{https://www.nytco.com/}{NYTCo}
\item
  \href{https://help.nytimes.com/hc/en-us/articles/115015385887-Contact-Us}{Contact
  Us}
\item
  \href{https://www.nytco.com/careers/}{Work with us}
\item
  \href{https://nytmediakit.com/}{Advertise}
\item
  \href{http://www.tbrandstudio.com/}{T Brand Studio}
\item
  \href{https://www.nytimes.com/privacy/cookie-policy\#how-do-i-manage-trackers}{Your
  Ad Choices}
\item
  \href{https://www.nytimes.com/privacy}{Privacy}
\item
  \href{https://help.nytimes.com/hc/en-us/articles/115014893428-Terms-of-service}{Terms
  of Service}
\item
  \href{https://help.nytimes.com/hc/en-us/articles/115014893968-Terms-of-sale}{Terms
  of Sale}
\item
  \href{https://spiderbites.nytimes.com}{Site Map}
\item
  \href{https://help.nytimes.com/hc/en-us}{Help}
\item
  \href{https://www.nytimes.com/subscription?campaignId=37WXW}{Subscriptions}
\end{itemize}
