Sections

SEARCH

\protect\hyperlink{site-content}{Skip to
content}\protect\hyperlink{site-index}{Skip to site index}

\href{https://www.nytimes.com/section/politics}{Politics}

\href{https://myaccount.nytimes.com/auth/login?response_type=cookie\&client_id=vi}{}

\href{https://www.nytimes.com/section/todayspaper}{Today's Paper}

\href{/section/politics}{Politics}\textbar{}Trump Floats an Election
Delay, and Republicans Shoot It Down

\url{https://nyti.ms/3gfV5ad}

\begin{itemize}
\item
\item
\item
\item
\item
\end{itemize}

\begin{itemize}
\item
  \href{https://www.nytimes.com/interactive/2020/08/04/us/elections/results-arizona-kansas-michigan-missouri-primaries.html?action=click\&pgtype=Article\&state=default\&region=TOP_BANNER\&context=storylines_menu}{Latest
  Results}
\item
  \href{https://www.nytimes.com/article/biden-vice-president-2020.html?action=click\&pgtype=Article\&state=default\&region=TOP_BANNER\&context=storylines_menu}{Biden's
  V.P. Search}
\item
  \href{https://www.nytimes.com/interactive/2020/07/24/us/politics/trump-biden-campaign-donors.html?action=click\&pgtype=Article\&state=default\&region=TOP_BANNER\&context=storylines_menu}{Map
  of Donations}
\item
  \href{https://www.nytimes.com/interactive/2020/us/elections/delegate-count-primary-results.html?action=click\&pgtype=Article\&state=default\&region=TOP_BANNER\&context=storylines_menu}{Delegate
  Count}
\item
  \href{https://www.nytimes.com/interactive/2019/us/politics/2020-presidential-candidates.html?action=click\&pgtype=Article\&state=default\&region=TOP_BANNER\&context=storylines_menu}{The
  Candidates}
\item
  \href{https://www.nytimes.com/newsletters/politics?action=click\&pgtype=Article\&state=default\&region=TOP_BANNER\&context=storylines_menu}{Politics
  Newsletter}
\end{itemize}

Advertisement

\protect\hyperlink{after-top}{Continue reading the main story}

Supported by

\protect\hyperlink{after-sponsor}{Continue reading the main story}

\hypertarget{trump-floats-an-election-delay-and-republicans-shoot-it-down}{%
\section{Trump Floats an Election Delay, and Republicans Shoot It
Down}\label{trump-floats-an-election-delay-and-republicans-shoot-it-down}}

The president's suggestion that the Nov. 3 vote could be delayed ---
something he cannot do on his own --- drew unusually firm Republican
resistance and signaled worry about his re-election bid.

\includegraphics{https://static01.nyt.com/images/2020/07/30/us/politics/30trump-election1/merlin_175126236_6c5d937a-ab38-460a-87aa-892bfd358495-articleLarge.jpg?quality=75\&auto=webp\&disable=upscale}

\href{https://www.nytimes.com/by/maggie-haberman}{\includegraphics{https://static01.nyt.com/images/2018/07/12/multimedia/author-maggie-haberman/author-maggie-haberman-thumbLarge.png}}\href{https://www.nytimes.com/by/jonathan-martin}{\includegraphics{https://static01.nyt.com/images/2018/11/06/multimedia/author-jonathan-martin/author-jonathan-martin-thumbLarge.png}}\href{https://www.nytimes.com/by/reid-j-epstein}{\includegraphics{https://static01.nyt.com/images/2019/06/25/reader-center/author-reid-epstein/9e877853d8234217b58e5762253aa771-thumbLarge.png}}

By \href{https://www.nytimes.com/by/maggie-haberman}{Maggie Haberman},
\href{https://www.nytimes.com/by/jonathan-martin}{Jonathan Martin} and
\href{https://www.nytimes.com/by/reid-j-epstein}{Reid J. Epstein}

\begin{itemize}
\item
  Published July 30, 2020Updated Aug. 4, 2020
\item
  \begin{itemize}
  \item
  \item
  \item
  \item
  \item
  \end{itemize}
\end{itemize}

Facing disastrous economic news and rising coronavirus deaths,
\href{https://www.nytimes.com/interactive/2020/us/elections/donald-trump.html}{President
Trump} on Thursday floated
\href{https://www.nytimes.com/2020/08/04/podcasts/the-daily/mail-in-voting-president-trump.html?action=click\&module=Briefings\&pgtype=Homepage}{delaying
the Nov. 3 election}, a suggestion that lacks legal authority and could
undermine confidence in an election that polls show him on course to
lose.

Republican leaders in Congress, who often claim not to have seen Mr.
Trump's outlandish statements and tweets and who infrequently challenge
him in public, promptly and vocally condemned any notion that the
election would be moved.

It was a moment of striking political isolation for the president, as
Republicans felt no need to defend him, Democrats condemned him, and
three former presidents gathered in a rare moment together, paying
tribute at
\href{https://www.nytimes.com/2020/07/30/us/john-lewis-live-funeral.html}{the
funeral of Representative John Lewis} of Georgia.

Mr. Trump is facing about as dire a run-up to a presidential election as
any incumbent could imagine: the
\href{https://www.nytimes.com/live/2020/07/30/business/stock-market-today-coronavirus}{worst
quarter} in the economy on record, an unceasing health crisis, protests
nationwide and a country paralyzed by the lack of a financial recovery
plan with no solution in sight --- all compounded by his own inability
to curtail his behavior.

His remarks on Twitter about the election delay --- which he linked to
his baseless claims about the potential for mail-in voter fraud --- were
one of the few clear signs that the president now realizes how deep a
hole he has dug for himself in his re-election effort. Aides have
described him as pained by the widespread rejection he is seeing in
public opinion polls, even as he continues with self-sabotaging behavior
rather than taking steps that might help him, like getting involved in
negotiations for a deal on Capitol Hill to lift the economy.

``With Universal Mail-In Voting (not Absentee Voting, which is good),
2020 will be the most INACCURATE \& FRAUDULENT Election in history,''
Mr. Trump
\href{https://twitter.com/realDonaldTrump/status/1288818160389558273?s=20}{wrote}.
``It will be a great embarrassment to the USA. Delay the Election until
people can properly, securely and safely vote???''

Mr. Trump later pinned the tweet at the top of his Twitter feed,
ensuring people would continue to see it. Hours later, despite warnings
from his campaign officials that delays are likely in tabulating results
on Nov. 3, Mr. Trump said in a separate
\href{https://twitter.com/realDonaldTrump/status/1288933078287745024?s=20}{tweet},
``Must know Election results on the night of the Election, not days,
months or even years later!''

That second statement reflects a concern that Democrats have given voice
to --- that Mr. Trump will try to focus on the same-day voting tallies
to claim victory, even when the full results may be unknown for days.

At a late-afternoon briefing with reporters, Mr. Trump defended the
initial tweet, saying that he feared delays in counting votes. But he
declined to elaborate on whether he was seriously proposing moving the
election.

Mr. Trump posted the first tweet shortly after the Commerce Department
announced that the gross domestic product for the second quarter of the
year had fallen precipitously by 9.5 percent, reflecting the widespread
shutdown of businesses beginning in March to combat the spread of the
coronavirus.

\hypertarget{latest-updates-2020-election}{%
\section{\texorpdfstring{\href{https://www.nytimes.com/2020/08/04/us/elections/primary-election-michigan-arizona-kansas.html?action=click\&pgtype=Article\&state=default\&region=MAIN_CONTENT_1\&context=storylines_live_updates}{Latest
Updates: 2020
Election}}{Latest Updates: 2020 Election}}\label{latest-updates-2020-election}}

Updated 2020-08-05T03:23:56.561Z

\begin{itemize}
\tightlist
\item
  \href{https://www.nytimes.com/2020/08/04/us/elections/primary-election-michigan-arizona-kansas.html?action=click\&pgtype=Article\&state=default\&region=MAIN_CONTENT_1\&context=storylines_live_updates\#link-3924dd44}{Two
  G.O.P. Senate primaries offer --- what else? --- a test of loyalty to
  Trump.}
\item
  \href{https://www.nytimes.com/2020/08/04/us/elections/primary-election-michigan-arizona-kansas.html?action=click\&pgtype=Article\&state=default\&region=MAIN_CONTENT_1\&context=storylines_live_updates\#link-62a8e06b}{The
  military-style uniforms of federal agents who responded to the unrest
  in Portland will be replaced.}
\item
  \href{https://www.nytimes.com/2020/08/04/us/elections/primary-election-michigan-arizona-kansas.html?action=click\&pgtype=Article\&state=default\&region=MAIN_CONTENT_1\&context=storylines_live_updates\#link-32b39e33}{President
  Trump is suddenly a big supporter of mail-in voting --- in Florida.}
\end{itemize}

\href{https://www.nytimes.com/2020/08/04/us/elections/primary-election-michigan-arizona-kansas.html?action=click\&pgtype=Article\&state=default\&region=MAIN_CONTENT_1\&context=storylines_live_updates}{See
more updates}

Mr. Trump, who often tests the boundaries of his authority, has
increasingly used public comments to lay groundwork for arguing that the
election results are illegitimate if he loses. Though he does not have
the constitutional authority to unilaterally change the date of the
election, his tweet prompted a now-familiar round of assertions about
what his true intention was with his statement.

With Mr. Trump, that is frequently a guessing game. The president has
often posted remarks on Twitter that are aimed at sparking a reaction
from people. At other times, he posts in reaction to what he sees on
cable news shows. And sometimes he tries to change what those shows are
focusing on with his tweets, offering a diversion.

Whatever his motivation on Thursday, senior Republicans and an array of
senators wanted no part of it, diverging from their standard practice of
walking on eggshells after a Trump eruption.

``Never in the history of the federal elections have we not held an
election, and we should go forward,'' said Representative Kevin McCarthy
of California, the House minority leader and an enthusiastic supporter
of Mr. Trump's, adding that he understood ``the president's concern
about mail-in voting.''

\includegraphics{https://static01.nyt.com/images/2020/07/30/us/politics/30trump-election3/merlin_175125528_d9a5161a-4545-4d12-b133-08d3febe0a9d-articleLarge.jpg?quality=75\&auto=webp\&disable=upscale}

Senator Mitch McConnell, the majority leader, echoed Mr. McCarthy,
saying ``we'll find a way'' to hold the election on Nov. 3.

Senators Ted Cruz and Marco Rubio, rivals for the 2016 Republican
presidential nomination who have since become staunch Trump supporters,
both dismissed the idea that the date for the election could change.
Senator Lindsey Graham, Mr. Trump's foremost public defender in the
Senate, said there would be a secure vote in November. And officials in
key swing states showed little interest in engaging on the topic.

``We're going to have an election, it's going to be legitimate, it's
going to be credible, it's going to be the same as it's always been,''
Mr. Rubio told reporters at the Capitol in Washington.

Mr. Cruz agreed. ``I think election fraud is a serious problem,'' he
said. ``But, no, we should not delay the election.''

People close to Mr. Trump said that the president has at times discussed
with associates whether the election can be delayed, and has been told
definitively that only an amendment to the Constitution could change the
date. But his tweet was discomfiting to most of his aides, who tried to
clean up his statement later by contending that he had been referring to
the possibility that the outcome won't be known until weeks after the
election.

This is not the first time that Mr. Trump has raised the idea of
thwarting rules or laws that he finds objectionable, and he often fails
to follow through. He has repeatedly hurled threats, whether it is
defunding universities or blocking federal aid to states, the substance
of which he has no intent, or capacity, to fulfill.

The president, who did not serve in government before being elected to
the highest office in the country, has never fully absorbed what powers
he does and does not have, or how to wield his authority. What Mr. Trump
has always been mindful of, dating to his time as a real estate
developer, is the danger of being labeled a failure.

So in response to his weakened standing in the presidential race, Mr.
Trump has been reaching for arguments to explain his difficulties this
year, repeatedly noting how the virus undermined the booming economy for
which he claims credit.

In this vein, any uncertainty about the balloting offers him an opening
to raise questions about the legitimacy of his loss, regardless of
whether he challenges the results.

Trump-weary Republicans may make that a more difficult task, however.

Representative Liz Cheney, Republican of Wyoming, a sometime critic of
the president who is eyeing the top ranks of the House leadership, said:
``We are not moving the date of the election. The resistance to this
idea among Republicans is overwhelming.''

Scott Jennings, a Republican strategist and an adviser to Mr. McConnell,
called Mr. Trump's statement ``unfocused,'' and ``insecure,'' saying it
``separates him from his own party and most of mainstream political
thought at a time when he needs to be fully focused on coronavirus, the
economy, and defining Biden as out of the mainstream.''

``Republicans,'' Mr. Jennings added, ``have reacted correctly by
rejecting the notion of delay.''

To Mr. Jennings and other Republican strategists, Mr. Trump is playing
with fire by suggesting to his supporters that mail voting can't be
trusted, given that it may be the best option for some people in an era
in which almost every activity has been changed to combat the virus's
spread. Making Republican voters distrust mail voting could negatively
affect not just Mr. Trump, but a host of down-ballot candidates.

''The reality is,'' Mr. Jennings said, ``he needs every Republican vote
there is, and he needs them any way he can get them, no matter how they
are cast.''

The president has repeatedly railed against mail voting, creating
outlandish scenarios of ballot theft to undermine confidence in the
practice.

Image

Mr. Trump has frequently broken with presidential precedent in doubting
the legitimacy of elections.~Credit...Doug Mills/The New York Times

\includegraphics{https://static01.nyt.com/images/2017/01/29/podcasts/the-daily-album-art/the-daily-album-art-articleInline-v2.jpg?quality=75\&auto=webp\&disable=upscale}

\hypertarget{listen-to-the-daily-is-the-us-ready-to-vote-by-mail}{%
\subsubsection{Listen to `The Daily': Is the U.S. Ready to Vote by
Mail?}\label{listen-to-the-daily-is-the-us-ready-to-vote-by-mail}}

What we've learned from the very different experiences of two states,
Georgia and Montana.

transcript

Back to The Daily

bars

0:00/25:27

-25:27

transcript

\hypertarget{listen-to-the-daily-is-the-us-ready-to-vote-by-mail-1}{%
\subsection{Listen to `The Daily': Is the U.S. Ready to Vote by
Mail?}\label{listen-to-the-daily-is-the-us-ready-to-vote-by-mail-1}}

\hypertarget{hosted-by-michael-barbaro-produced-by-luke-vander-ploeg-and-rachel-quester-with-help-from-michael-simon-johnson-and-edited-by-lisa-chow}{%
\subsubsection{Hosted by Michael Barbaro; produced by Luke Vander Ploeg
and Rachel Quester; with help from Michael Simon Johnson; and edited by
Lisa
Chow}\label{hosted-by-michael-barbaro-produced-by-luke-vander-ploeg-and-rachel-quester-with-help-from-michael-simon-johnson-and-edited-by-lisa-chow}}

\hypertarget{what-weve-learned-from-the-very-different-experiences-of-two-states-georgia-and-montana}{%
\paragraph{What we've learned from the very different experiences of two
states, Georgia and
Montana.}\label{what-weve-learned-from-the-very-different-experiences-of-two-states-georgia-and-montana}}

\begin{itemize}
\item
  michael barbaro\\
  From The New York Times, I'm Michael Barbaro. This is ``The Daily.''
\item
  {[}music{]}\\
  Today: The United States is preparing to hold its first socially
  distant presidential election in history. Will it actually work? My
  colleague, Reid Epstein, on what we've learned from the very different
  experience of two states.

  It's Tuesday, August 4.

  So Reid, it is precisely three months to the day until Election Day.
  And from what you can tell, given the state of the pandemic, will
  voting on November 3 at all resemble the way we have cast ballots in
  the past?
\item
  reid epstein\\
  For a lot of us, no.

  You know, the idea that people have that they wake up the morning of
  Election Day and go to the middle school or a church or something
  nearby and wait in line for a few minutes and vote, for a lot of us
  that's going to be replaced with opening your mailbox and seeing a
  ballot that comes from your local board of elections. And sitting down
  at the kitchen table and figuring out sort of who's running for what
  and looking for what they stand for, and filling in the ovals as you
  see fit. And then when you're done with that, you'll either put it
  back in the mailbox or you'll drop it off in a ballot collection box.
  And that will be how people vote. And it will take place over a matter
  of weeks, and not just crammed in on Election Day.
\item
  michael barbaro\\
  And just to be clear, what you're describing is mail-in voting. And I
  want to make sure I understand exactly what that is, because I think
  for a lot of people there is a familiarity with absentee voting. And
  is that different?
\item
  reid epstein\\
  I mean, it's basically the same. A lot of us have received an absentee
  ballot if we're traveling for work or at school and not at home. But
  mail-in voting is on a mass scale of absentee voting that we're going
  to see for this election.
\item
  michael barbaro\\
  Right.

  And so what has been our experience in the United States with mail-in
  voting up to this point?
\item
  reid epstein\\
  It's very different depending on where you live. There are five states
  that have been conducting their elections entirely by mail ---
  Washington, Oregon, Hawaii, Utah and Colorado, where they send ballots
  to every registered voter and people return them either through the
  mail or in drop boxes. And so if you live in one of those places, this
  year's election will look pretty normal to you. You'll get your
  ballot. If you live in a Washington state you'll get a little booklet,
  which will be 80 to 100 pages with testimonials from everyone who's
  running for every office on the ballot --- dogcatcher to president of
  the United States --- that you have a chance to read and study. And
  you'll fill it out at your leisure, and as long as you have it
  returned by whatever the local deadline is, your vote will count.
\item
  michael barbaro\\
  And by all accounts, how well does mail-in voting work in those five
  states?
\item
  reid epstein\\
  It works great by almost all accounts from anybody you talk to in
  those states, from both parties. It has increased turnout,
  particularly among demographics of people who are least likely to vote
  otherwise. I had a conversation last week with the Republican state
  party chairman in Utah, who said that Utahns have been voting by mail
  for over a decade, and that they have effective safeguards and
  procedures in place to keep the voting on the up and up. And that they
  don't expect any problems there with mail voting this fall, because
  they haven't had any in the past.
\item
  michael barbaro\\
  So mail-in voting is smooth in the states that have it. And of course,
  many other states have some level of experience with absentee ballots,
  which would seem to suggest there's a pretty straightforward path to
  taking mail-in voting nationwide for the entire country in 2020.
\item
  reid epstein\\
  You might think that, but it's been proven to not be that simple. And
  the best example of where things have gotten messy was in Georgia.
  Georgia's presidential primary was supposed to be back in March. And a
  couple weeks before it was held, the secretary of state, a Republican
  named Brad Raffensperger, postponed it to May. And then a couple of
  weeks before the May primary, everything was postponed to June,
  because they didn't quite know what to do with the pandemic. And after
  the second postponement, it was clear that they needed to do something
  to alleviate what they expected to be a crush of people voting on
  their primary day.

  And so they sent absentee ballot application forms to nearly 7 million
  active registered voters in Georgia in an effort to get people to vote
  by mail and not come into either early voting centers or Election Day
  voting sites.
\item
  michael barbaro\\
  So 7 million people got applications for ballots, but not ballots
  themselves?
\item
  reid epstein\\
  That's right.
\item
  michael barbaro\\
  So how many people ultimately sought a ballot with these applications
  in Georgia?
\item
  reid epstein\\
  About a million and a half people returned the ballot applications
  from the state to their county board of elections. And almost all of
  those people received a ballot, but there were still tens of thousands
  of people who requested ballots and didn't receive them. Either it
  didn't show up, or it showed up after the election. Or it showed up a
  malfunctioning ballot or a bad return envelope, or some way that it
  couldn't be returned properly. One example of that is Stacey Abrams,
  who was the Democratic party nominee for governor in 2018.
\item
  archived recording (stacey abrams)\\
  I applied for my absentee ballot in May when there was some delay in
  receiving it. But when it finally arrived ---
\end{itemize}

reid epstein

Received a ballot.

\begin{itemize}
\tightlist
\item
  archived recording (stacey abrams)\\
  --- I filled out the ballot, completed it and got ready to put it in
  the return envelope.
\end{itemize}

reid epstein

And the return envelope was sealed shut. And she talked about using an
iron to try to unseal it.

\begin{itemize}
\tightlist
\item
  archived recording (stacey abrams)\\
  And I attempted to steam it open, because I've watched lots of mystery
  shows. It did not work.
\end{itemize}

michael barbaro

Wow.

reid epstein

But was unsuccessful, and so she went and stood in line at an early
voting site that Georgia had kept open for people to vote and was able
to cast her ballot.

\begin{itemize}
\tightlist
\item
  archived recording (stacey abrams)\\
  Luckily for me, I live in an area where the lines were not terribly
  long. But all I can think about are the people who did not receive
  their ballots, who were forced instead into hours-long lines.
\end{itemize}

reid epstein

Jon Ossoff, who is the Democratic nominee for Senate from Georgia ---

\begin{itemize}
\tightlist
\item
  archived recording (jon ossoff)\\
  My wife Alisha and I applied for our absentee ballots a month ago. We
  wrote the county. We sent letters to the county. We sounded the alarm
  publicly. We never got our ballots, and I'm on the ballot.
\end{itemize}

reid epstein

--- waited five hours in line at a community swimming pool.

\begin{itemize}
\tightlist
\item
  archived recording (jon ossoff)\\
  We waited five hours to do so.
\end{itemize}

reid epstein

Because he never received his absentee ballot.

\begin{itemize}
\tightlist
\item
  archived recording (jon ossoff)\\
  And we are strong, so we can do that. But think of the seniors.
\end{itemize}

reid epstein

And so after he waits in line for five hours and votes, he gets home.
And what's come in the mail after a month? Is his absentee ballot. And
then once Election Day came in Georgia ---

\begin{itemize}
\tightlist
\item
  archived recording\\
  Now to the Primary Day chaos --- hundreds of voters waiting in
  hours-long lines.
\end{itemize}

reid epstein

You had limits of how many people could be inside those rooms at
churches and schools, because of the pandemic.

\begin{itemize}
\tightlist
\item
  archived recording\\
  Voters told us they waited for as long as four hours to get inside to
  vote.
\end{itemize}

reid epstein

So you had images of people waiting five, six hours in line outside the
buildings in the summer Georgia heat.

\begin{itemize}
\tightlist
\item
  archived recording\\
  We saw two voters turn around and drive off because the line was so
  long.
\end{itemize}

reid epstein

And not indoors waiting to vote inside.

\begin{itemize}
\item
  archived recording (voter)\\
  Once I got inside, I think the most frustrating part was that several
  of the machines were broken. It seemed like maybe half of the machines
  were down.
\item
  archived recording (poll worker)\\
  Equipment was delivered late. Some of it's not charged up. The
  check-in folks aren't really clear about what to do.
\end{itemize}

michael barbaro

Why do you think things went so badly in Georgia? Both the mail-in and
the in-person voting?

reid epstein

The bottom line, really, is that it's really hard to run either a mail
election or an in-person election. And doing them both at the same time,
without really the resources to do either, is a recipe for disaster. The
elections administrator for Fulton County, which includes Atlanta in
Georgia, the night of the primary, said during a Zoom press conference
---

\begin{itemize}
\tightlist
\item
  archived recording (election administrator)\\
  The absentee-by-mail process, it was an election in and of itself.
\end{itemize}

reid epstein

He was asked to run two elections at once.

\begin{itemize}
\tightlist
\item
  archived recording (election administrator)\\
  We became an absentee-by-mail state. We still had to do our full
  complement of Election Day infrastructure. And it stretched us.
\end{itemize}

reid epstein

And that he didn't have the resources or the capacity to do that. And
that was precisely what led to all of the problems that people
experienced on their primary day.

\begin{itemize}
\tightlist
\item
  archived recording (election administrator)\\
  We ran into a lot of challenges this time. So I think we are going to
  learn from this and move forward.
\end{itemize}

michael barbaro

So Georgia seems almost like a case study of how not to hold an election
in the middle of a pandemic.

reid epstein

Yeah. I mean, they kind of did all elements of what you would think of
an election, plus the mail-in voting, but didn't really do any of them
well. And didn't really devote adequate resources to any part of it. And
in the end, the result was about as bad an experience as you can get.

{[}music{]}

michael barbaro

We'll be right back.

Reid, did any state get a pandemic primary right over the past couple of
weeks and months? Were there any anti-Georgias?

reid epstein

I mean, the best example is probably Montana. The governor there, Steve
Bullock, by executive order gave the state's 56 counties the authority
to mail ballots to all of their registered voters. And every registered
voter in the state got a ballot in the mail.

michael barbaro

And just to be clear, they got a ballot, not a ballot application?

reid epstein

They got a ballot, and not the application. So they skipped the step
that Georgia and other states required. And so voters there got the
ballots in the mail. They filled them out. And as long as they were
returned before the deadline, the ballots counted. And in the end,
Montana had the highest voter turnout of any state in this year's
presidential primary.

michael barbaro

So what happened with in-person voting in Montana?

reid epstein

They kept some of it, but not in a form that would be really
recognizable in a pre-pandemic world. There were boards of elections'
offices that were open for people who needed to change their
registration or cast a ballot, but for the vast majority of voters in
Montana the election took place through the mail.

michael barbaro

Reid, I'm mindful that as smooth as this all seemed to go in Montana,
Montana is not necessarily representative of the whole country, right?
It has a pretty small population.

reid epstein

That's right. There's about the same amount of people in the state of
Montana as there are in Fulton County in Georgia, in the state's biggest
county. And so it is a much more challenging situation to run an
election with a lot more people than it is in a small state.

michael barbaro

So how useful is Montana as a model?

reid epstein

I mean, what Montana shows is that if you give people ballots and don't
make them jump through hoops to get to them, you're more likely to have
more of them return them and vote.

michael barbaro

Meaning just mailing them the actual ballots, and not making them apply
and then get the ballots the way Georgia did?

reid epstein

That's right.

michael barbaro

So kind of in summary, mail-in voting is complicated, especially
complicated in big states. But it can work when local authorities get
ballots into people's hands. In other words, this can work under ideal
circumstances around the country?

reid epstein

It can work when there is a concerted effort by local elections
officials to make it work.

\begin{itemize}
\tightlist
\item
  archived recording (donald trump)\\
  If we went to mail-in balloting, our election all over the world would
  look as a total joke. It would be a total joke.
\end{itemize}

reid epstein

What's going on in this country now is you have the president who is
making a concerted effort to make it not work.

\begin{itemize}
\tightlist
\item
  archived recording (donald trump)\\
  And they're in all the mailboxes. And kids go and they raid the
  mailboxes, and they hand them to people that are signing the ballots
  down the end of the street, which is happening. They grab the ballots.
  You don't think that happens?
\end{itemize}

reid epstein

He regularly tweets and speaks about how mail voting is sort of fraud
filled, and a recipe for disaster, and a reason Republicans won't win.

\begin{itemize}
\tightlist
\item
  archived recording (donald trump)\\
  In some cases, they won't sell them, like, to a Republican community,
  a conservative community. They don't happen to send the ballots to
  those communities. And there's no way of checking. No. You have to go
  and you have to vote.
\end{itemize}

reid epstein

And so it hasn't taken long for this idea to take hold among Republican
voters, particularly base Republican voters, that there is something
inherently wrong with voting by mail.

michael barbaro

And is there any documented evidence of that? That mail-in voting is
somehow more open to fraud?

reid epstein

No, there's not. There have been a couple of isolated instances of
what's called ``ballot harvesting.'' We saw it in a congressional race
in North Carolina a couple of years ago from the Republican side. But
most of what you hear when people describe fraud in the vote-by-mail
system are either old wives' tales or fantastic stories of an apartment
building with 30 ballots in the mailbox. Things that have been chain
letters or Facebook memes that aren't necessarily grounded in any real
evidence.

michael barbaro

So how do you explain why the president is claiming that this is a
problematic mode of voting? What accounts for that?

reid epstein

I mean, there is a widespread belief among Republican voters and some
Republican elected officials that allowing more people to vote will be
advantageous to Democrats. And so taken in that context, it makes sense
that he would resist some of the push toward mail-in voting, because it
does make it easier for a lot more people to vote.

michael barbaro

Reid, is there a version of this where the president seems to be
discouraging mail-in voting, but mail-in voting becomes the predominant
way that voting occurs in this fall's election? And therefore, wouldn't
that mean that the president was telling his own voters, don't trust the
most important way of voting for my re-election, for his re-election?

reid epstein

There's a lot of concern about that happening already. You know, you see
in states that aren't entirely vote-by-mail, but where it's a
predominant way of voting --- like Florida and Arizona, that have large
populations of older Republican voters --- that Republican officials in
those states have been trying to push a message that's expressly counter
to what the president has been saying about voting by mail. That we've
seen in Utah, where the Republican state chairman told me that their
system is not like what happens in other states, and that President
Trump, in describing fraud elsewhere, must be talking about somewhere
else other than Utah.

michael barbaro

So in other words, these officials are trying to tell their voters trust
the system, even if the president doesn't tell you to trust it.

reid epstein

Trust the system here, because the system where you live is functional,
and the president is talking about other places. But that sort of nuance
is lost on a lot of voters who, you know, hear what the president says
and are generally disinclined to believe people who disagree with him,
particularly within their own party, because of the way that the party
and the president have functioned over the last four years.

michael barbaro

So couldn't the president be actually hurting his own re-election
chances by doing this?

reid epstein

He's certainly hurting the enthusiasm for voting by mail among his
Republican supporters.

michael barbaro

So Reid, back to this moment --- three months to the day that we will
have the presidential election. Where would you say the whole country
is? Are most states looking like Georgia? You know, this is going to be
a mess? Or are a lot of states looking like Montana in terms of their
preparations for this election? Which is to say, they can pull it off.

reid epstein

I think we have more states that at the moment are looking like Georgia
than Montana. Of the 45 states that have some sort of in-person voting,
almost all of them are going to open polling places for people to show
up at. But as we've seen, it's really hard to do two elections at once.
And it's complex. It involves a lot of resources. And most of these
states haven't done a big general election where most of their voters
cast ballots in the mail. It's going to be new for the voters. It's
going to be new for the elections administrators. And it's going to be
new for the people that count the ballots on election night.

michael barbaro

And Reid, what are the likely consequences of so many states trying this
for the first time?

reid epstein

I mean, the first obvious consequence is on November 3, in the evening
when we're used to turning on television and the internet and seeing the
results of the elections come in, we may not have them. And what
President Trump and Joe Biden do about declaring victory or not. Trump
has said the election should be called on election night, no matter
what. But we know that there will be millions and millions of ballots
that aren't counted on election night, whether it's because states have
voting rules that say your ballot counts as long as it's postmarked by
Election Day, or because it takes some of these states longer to count
an absentee ballot that came in the mail than it would have taken them
to count a ballot that was marked and delivered into a machine at a
polling place. And all of that is going to have a real significant
impact on what we know about the presidential contest, the Senate races
and races all down the ballot.

michael barbaro

And of course, that's just when we know the results. If this election is
as messy and complicated as everything you're saying suggests it might
be, I wonder if that raises a much bigger question? Which is how much do
you think the country will trust the outcome of the elections on
November 3 if voting ends up feeling kind of haphazard?

reid epstein

I mean, a lot of that will depend on people's experiences heading into
the election. Whether a lot of people have problems voting by mail, or
don't even try to vote by mail, or have to wait in line for hours at
either early voting centers or on Election Day. You know, a lot of the
feelings about this was going to be dictated frankly by what the
president says and does. You know, if the president wakes up on the
morning of November 4 and says, I don't care that there are more votes
left to be counted, I won. Then you can sort of imagine how that's going
to play out.

Or if there's a state, like take Arizona, where it appears that
President Trump has won on election night, but four days later they've
received more votes and it flips to Joe Biden and that's the deciding
state, there's innumerable chaos that could happen from that. And so it
doesn't take much to make this system look like it's chaotic, even
though in a lot of times it's either sort of how the system is designed
to work, because mail-in-voting just does take longer. Even though it's
a more efficient way of getting more people to vote, it can take longer
to tabulate the results from that.

michael barbaro

So you're basically telling us to be patient. And Americans are not
patient.

reid epstein

They're not. You know, I mean if you want to not stress out about the
results of the election, buy a plane ticket to some place without
internet access or cell signal on November 2 and come back a week and a
half later.

michael barbaro

Or three weeks or four weeks.

reid epstein

Go to the Galapagos Islands or somewhere, and come back in a couple of
weeks. And by then, you'll probably know who won.

michael barbaro

{[}LAUGHS{]} Well, Reid, thank you very much.

reid epstein

Thank you, Michael.

michael barbaro

On Monday, the Democratic governor of Nevada, Steve Sisolak, signed a
law that would require sending ballots to every registered voter in the
state, saying that it would ensure, quote, ``the safest, most accessible
election possible under these unprecedented circumstances.'' In
response, President Trump threatened legal action to try to block the
measure.

\begin{itemize}
\tightlist
\item
  archived recording (donald trump)\\
  Well, there's never been a push like this for mail-in ballots, and if
  you look at ---
\end{itemize}

michael barbaro

During a news conference at the White House, Trump was asked whether he
would consider restricting mail-in voting nationwide through an
executive order.

\begin{itemize}
\tightlist
\item
  archived recording (donald trump)\\
  Universal mail-in ballots is going to be a great embarrassment to our
  country. I have the right to do it. We haven't gotten there yet, but
  we'll see what happens. We will be suing in Nevada.
\end{itemize}

michael barbaro

We'll be right back.

{[}music{]}

Here's what else you need to know today. On Monday, the Manhattan
District Attorney, Cy Vance, suggested that his investigation into
President Trump's finances and company is much broader than previously
known. A new court filing indicates that the investigation, which was
thought to focus on hush money payments made to women during the 2016
campaign, may actually be examining a range of possible criminal
activity, including potential bank and insurance fraud. The Supreme
Court sided with Vance in a major ruling last month, saying that the
president lacked the legal basis to block the district attorney from
seeking years of his tax records as part of the investigation.

And, The Times reports that nearly 3,000 small businesses in New York
City have permanently closed because of the pandemic --- more than in
any U.S. city. A new report estimates that one third of the city's small
businesses, or about 75,000 of them, may never reopen because of
lockdowns and lost revenue.

That's it for ``The Daily.'' I'm Michael Barbaro. See you tomorrow.

Even for Mr. Trump,
\href{https://twitter.com/realDonaldTrump/status/1288818160389558273?s=20}{suggesting
a delay in the election} was an extraordinary breach of presidential
decorum that will increase the chances that he and his core supporters
don't accept the legitimacy of the election should he lose to former
Vice President
\href{https://www.nytimes.com/interactive/2020/us/elections/joe-biden.html}{Joseph
R. Biden Jr.} Mr. Trump's comments about the election looked all the
more discordant coming just hours before the funeral for Mr. Lewis, a
Democrat who as a young man was beaten and jailed as he advocated voting
rights.

Without mentioning his successor by name, former President Barack Obama
used his eulogy of Mr. Lewis to rebuke Mr. Trump.

Speaking from the pulpit of Atlanta's Ebenezer Baptist Church, where the
Rev. Dr. Martin Luther King Jr. was reared and eventually preached, Mr.
Obama lashed Mr. Trump for ``even undermining the Postal Service in the
run-up to an election that's going to be dependent on mail-in ballots so
people don't get sick.''

For all the eye-rolling dismissals among Republicans, Mr. Trump's
remarks irritated and embarrassed his allies --- and represented the
latest illustration of how he is not only complicating his own campaign
but also compounding his party's challenge this fall.

Already burdened with an administration that only briefly attempted a
full-scale response to a public health crisis that has
\href{https://www.nytimes.com/interactive/2020/us/coronavirus-us-cases.html}{sickened
millions} of Americans and killed over 150,000 while ravaging the
economy, Republicans on the ballot are increasingly being undermined by
Mr. Trump's response to his misfortune.

Just this week, after he finally bowed to pressure to urge people to
take virus safety measures, the president lamented how unpopular he is
compared with his high-profile medical advisers.

And then he
\href{https://www.nytimes.com/2020/07/28/technology/virus-video-trump.html}{publicized
an online video} promoting an unproven virus treatment from a doctor who
has previously opined on alien DNA and the impact of having sex with
demons in one's dreams.

His growing desperation to close the gap with Mr. Biden has also caused
headaches for Republicans because he has increasingly employed
race-baiting language that few in the party care to defend.

``I am happy to inform all of the people living their Suburban Lifestyle
Dream that you will no longer be bothered or financially hurt by having
low income housing built in your neighborhood,''
\href{https://www.nytimes.com/2020/07/29/us/politics/trump-suburbs-housing-white-voters.html}{he
tweeted on Wednesday}.

Luke Broadwater, Emily Cochrane and Matt Stevens contributed reporting.

\hypertarget{our-2020-election-guide}{%
\section{Our 2020 Election Guide}\label{our-2020-election-guide}}

Updated Aug. 4, 2020

\begin{itemize}
\item
  \begin{center}\rule{0.5\linewidth}{\linethickness}\end{center}

  \hypertarget{the-latest}{%
  \subsection{The Latest}\label{the-latest}}

  \begin{itemize}
  \tightlist
  \item
    Kris Kobach, a polarizing figure in Kansas politics,
    \href{https://www.nytimes.com/2020/08/04/us/politics/kobach-tlaib.html?action=click\&pgtype=Article\&state=default\&region=BELOW_MAIN_CONTENT\&context=storylines_guide}{lost
    the Senate primary there}, relieving G.O.P. officials who feared he
    could jeopardize a safe seat.
  \end{itemize}
\item
  \begin{center}\rule{0.5\linewidth}{\linethickness}\end{center}

  \hypertarget{bidens-vp-search}{%
  \subsection{Biden's V.P. Search}\label{bidens-vp-search}}

  \begin{itemize}
  \tightlist
  \item
    \href{https://www.nytimes.com/article/biden-vice-president-2020.html?action=click\&pgtype=Article\&state=default\&region=BELOW_MAIN_CONTENT\&context=storylines_guide}{Here
    are 13 women} who have been under consideration to be Joe Biden's
    running mate, and why each might be chosen --- and might not be.
  \end{itemize}
\item
  \begin{center}\rule{0.5\linewidth}{\linethickness}\end{center}

  \hypertarget{keep-up-with-our-coverage}{%
  \subsection{Keep Up With Our
  Coverage}\label{keep-up-with-our-coverage}}

  \begin{itemize}
  \tightlist
  \item
    Get an
    \href{https://www.nytimes.com/newsletters/politics?action=click\&pgtype=Article\&state=default\&region=BELOW_MAIN_CONTENT\&context=storylines_guide}{email}
    recapping the day's news
  \end{itemize}

  \begin{itemize}
  \tightlist
  \item
    Download our mobile app on
    \href{https://apps.apple.com/us/app/nytimes/id284862083?ls=1\&mat_click_id=5c79ae7455014fd1bd66b5610c05b8f2-20191112-16948\&referrer=mat_click_id\%3D5c79ae7455014fd1bd66b5610c05b8f2-20191112-16948\%26link_click_id\%3D722930677036718082}{iOS}
    and
    \href{http://a.localytics.com/android?id=com.nytimes.android\&referrer=utm_source\%3Dother_nyt_mobile_web\%26utm_medium\%3DWeb\%2520page\%26utm_term\%3DGeneral\%2520Mobile\%2520Page\%26utm_campaign\%3DNYT\%2520Mobile\%2520General\%2520Page}{Android}
    and turn on Breaking News and Politics alerts
  \end{itemize}
\end{itemize}

Advertisement

\protect\hyperlink{after-bottom}{Continue reading the main story}

\hypertarget{site-index}{%
\subsection{Site Index}\label{site-index}}

\hypertarget{site-information-navigation}{%
\subsection{Site Information
Navigation}\label{site-information-navigation}}

\begin{itemize}
\tightlist
\item
  \href{https://help.nytimes.com/hc/en-us/articles/115014792127-Copyright-notice}{©~2020~The
  New York Times Company}
\end{itemize}

\begin{itemize}
\tightlist
\item
  \href{https://www.nytco.com/}{NYTCo}
\item
  \href{https://help.nytimes.com/hc/en-us/articles/115015385887-Contact-Us}{Contact
  Us}
\item
  \href{https://www.nytco.com/careers/}{Work with us}
\item
  \href{https://nytmediakit.com/}{Advertise}
\item
  \href{http://www.tbrandstudio.com/}{T Brand Studio}
\item
  \href{https://www.nytimes.com/privacy/cookie-policy\#how-do-i-manage-trackers}{Your
  Ad Choices}
\item
  \href{https://www.nytimes.com/privacy}{Privacy}
\item
  \href{https://help.nytimes.com/hc/en-us/articles/115014893428-Terms-of-service}{Terms
  of Service}
\item
  \href{https://help.nytimes.com/hc/en-us/articles/115014893968-Terms-of-sale}{Terms
  of Sale}
\item
  \href{https://spiderbites.nytimes.com}{Site Map}
\item
  \href{https://help.nytimes.com/hc/en-us}{Help}
\item
  \href{https://www.nytimes.com/subscription?campaignId=37WXW}{Subscriptions}
\end{itemize}
