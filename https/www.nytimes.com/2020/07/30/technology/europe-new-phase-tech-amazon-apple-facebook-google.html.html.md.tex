\href{/section/technology}{Technology}\textbar{}`This Is a New Phase':
Europe Shifts Tactics to Limit Tech's Power

\url{https://nyti.ms/2D1uiQy}

\begin{itemize}
\item
\item
\item
\item
\item
\end{itemize}

\includegraphics{https://static01.nyt.com/images/2020/07/24/business/00eutech/00eutech-superJumbo.jpg}

Credit...Matt Chase

Sections

\protect\hyperlink{site-content}{Skip to
content}\protect\hyperlink{site-index}{Skip to site index}

\hypertarget{this-is-a-new-phase-europe-shifts-tactics-to-limit-techs-power}{%
\section{`This Is a New Phase': Europe Shifts Tactics to Limit Tech's
Power}\label{this-is-a-new-phase-europe-shifts-tactics-to-limit-techs-power}}

The region's lawmakers and regulators are taking direct aim at Amazon,
Facebook, Google and Apple in a series of proposed laws.

Credit...Matt Chase

Supported by

\protect\hyperlink{after-sponsor}{Continue reading the main story}

By \href{https://www.nytimes.com/by/adam-satariano}{Adam Satariano}

\begin{itemize}
\item
  July 30, 2020
\item
  \begin{itemize}
  \item
  \item
  \item
  \item
  \item
  \end{itemize}
\end{itemize}

LONDON --- European Union leaders are pursuing a new law to make it
illegal for Amazon and Apple to give their own products preferential
treatment over those of rivals that are sold on their online stores.

In Britain, officials are drawing up a law to force Facebook to make its
services work more easily with rival social networks, and to push Google
to share some search data with smaller competitors.

And in Germany, authorities are debating a rule that would let
regulators essentially halt certain business practices at the tech
companies during an antitrust investigation.

Europe's lawmakers and regulators have shifted to a new stage in their
battle to limit the power of the world's biggest tech companies. The
region has long been at the forefront of using existing antitrust laws
and
\href{https://www.nytimes.com/2018/07/18/technology/google-eu-android-fine.html}{levying
multibillion dollar penalties} against the tech giants, but officials
now say that those tactics have not gone far enough in altering the
behavior of Apple, Amazon, Google and Facebook. So they are drafting at
least half a dozen new laws and regulations to aim at the heart of how
those tech companies' businesses work.

Europe has embarked on its legal blitz just as the United States has
started flexing its own tech regulatory muscles. On Wednesday, the chief
executives of Amazon, Apple, Google and Facebook were grilled by
lawmakers in a
\href{https://www.nytimes.com/2020/07/28/technology/amazon-apple-facebook-google-antitrust-hearing.html?action=click\&module=Top\%20Stories\&pgtype=Homepage}{congressional
hearing to scrutinize their power}. All
\href{https://www.nytimes.com/live/2020/07/29/technology/tech-ceos-hearing-testimony}{defended
themselves against criticism} from Democrats about anticompetitive
business practices and accusations from Republicans that they were
muzzling conservative voices. On Thursday, all four companies
\href{https://www.nytimes.com/live/2020/07/30/business/stock-market-today-coronavirus/amazons-earnings-double-as-sales-surge}{showed
their financial muscle} by reporting
\href{https://www.nytimes.com/live/2020/07/30/business/stock-market-today-coronavirus/apple-blows-past-expectations-with-surging-sales-and-profits}{billions
of dollars in profits} and
\href{https://www.nytimes.com/live/2020/07/30/business/stock-market-today-coronavirus/facebook-nearly-doubles-its-profit-but-warns-of-fallout-from-ad-boycotts}{surging
revenue}.

The momentum in the United States is set to grow. The Justice Department
is expected to announce
\href{https://www.nytimes.com/2020/05/15/technology/google-antitrust-investigation.html}{an
antitrust case against Google} in the coming weeks. The Federal Trade
Commission and state attorneys general are
also\href{https://www.nytimes.com/2020/07/17/technology/ftc-facebook-investigation.html}{investigating
Facebook}, Apple and Amazon for potential anti-competitive behavior.

Those actions, coupled with the efforts in Europe, represent a double
whammy for the tech giants. If the proposed laws in Europe are enacted,
the policies could lead to a major overhaul of the region's digital
economy, where there are more than 500 million consumers, by regulating
the tech companies more like traditional industries such as
telecommunications and finance.

``This is a new phase,'' Margrethe Vestager, the European Commission
executive vice president who is
\href{https://www.nytimes.com/2019/11/19/technology/tech-regulator-europe.html}{leading
the effort in Brussels to write new laws}, said in an interview.

Ms. Vestager said the proposed laws would lower hurdles to force the
tech companies to change and even restrict them from moving into new
product areas. ``At stake is whether or not these markets will be open
and contestable and innovative, or if they will just be governed by
these walled gardens of de facto monopolies,'' she said.

\includegraphics{https://static01.nyt.com/images/2020/07/23/business/23eutech2/merlin_164270304_20037baf-ea1f-42b3-80ce-3cb40b5b5915-articleLarge.jpg?quality=75\&auto=webp\&disable=upscale}

European officials are working on the new laws against Big Tech
alongside more traditional tactics such as antitrust investigations.
European Union officials are investigating whether Apple's App Store
policies are anti-competitive, and are
\href{https://www.nytimes.com/2020/06/11/technology/amazon-antitrust-european-union.html}{preparing
charges against Amazon} for abusing its e-commerce dominance to box out
smaller rivals. The European Union is also reviewing Google's purchase
of the wearables maker Fitbit, while
\href{https://www.gov.uk/cma-cases/facebook-inc-giphy-inc-merger-inquiry}{Britain
opened an inquiry} in June into Facebook's acquisition of Giphy, a GIF
company.

Google, Facebook, Apple and Amazon are closely monitoring Europe's
proposals. While the companies have publicly said they want to work with
the region's lawmakers and regulators, their lobbying groups have argued
that Europe's aggressive actions are partially an effort to protect
homegrown industries.

``Popular tech services are increasingly being developed outside of the
E.U.,'' said Christian Borggreen, vice president of the Computer and
Communications Industry Association, an industry group in Brussels.
``The E.U. should strive to become a leader in tech innovation, not just
in tech regulation.''

Amazon, Facebook, Apple and Google declined to comment.

For years, Europe set the standard in tech regulation --- only to find
that its efforts did not make much of a dent as the tech behemoths
continued to grow.

Consider that the European Commission found Google guilty of antitrust
violations three times from 2017 to 2019, resulting in fines of roughly
8.25 billion euros, or about \$9.7 billion at current conversion rates.
But the cases each took several years to complete, giving
\href{https://www.nytimes.com/2019/11/11/business/europe-technology-antitrust-regulation.html}{Google
ample time to secure its dominance} in online advertising, smartphone
software and internet search. The monetary penalties, which are small
for a company with more than \$160 billion in annual revenue, remain
tied up in court appeals.

Other legal efforts, such as Europe's landmark privacy law called
\href{https://www.nytimes.com/2018/05/24/technology/europe-gdpr-privacy.html}{the
General Data Protection Regulation}, were aimed at many industries and
were not just aimed at the tech companies. Since G.D.P.R. was enacted in
2018, it has been
\href{https://www.nytimes.com/2020/04/27/technology/GDPR-privacy-law-europe.html}{faulted
for lack of enforcement}.

So over the past year, European regulators and lawmakers began a
concerted effort to draw up new laws that specifically homed in on the
tech companies' businesses.

Much of the energy came from officials in Brussels, where European Union
leaders set policies for the 27-nation bloc. In December, Ms. Vestager,
who had already spent five years as the world's top tech industry
watchdog, began a new five-year term leading digital policy and
antitrust oversight. She and her colleagues vowed to take an
\href{https://www.google.com/search?q=nytimes+vestager+satariano\&oq=nytimes+vestager+satariano\&aqs=chrome..69i57j69i64j69i61.5096j0j7\&sourceid=chrome\&ie=UTF-8}{even
harder line}.

They proposed new rules to make it easier for regulators to begin
investigations against the tech companies. One proposed law, the Digital
Services Act, would draw more business boundaries for search engines,
marketplaces, social networks and app stores. Policymakers are debating
barring Amazon, Apple and others from giving their products preferential
treatment in their digital stores. Ms. Vestager said there was broad
political support for the ideas, which could become law by next year.

Among European countries, Britain has become particularly active in
moving to rein in the tech giants. Lawmakers are debating the creation
of a regulator
to\href{https://www.gov.uk/government/news/new-regime-needed-to-take-on-tech-giants}{focus
on the largest tech companies}, holding them to new codes of conduct so
they do not use exploitative or exclusionary business practices.

Image

Andrea Coscelli, head of the Competition and Markets Authority.
Britain's antitrust agency recently published a report accusing Google
and Facebook of anticompetitive behavior in online ads.Credit...CMA

``We have crossed a line,'' said Andrea Coscelli, the head of Britain's
antitrust agency, the Competition and Markets Authority, which published
a
\href{https://www.gov.uk/cma-cases/online-platforms-and-digital-advertising-market-study}{400-plus-page
report} this month accusing Google and Facebook of anticompetitive
behavior in online advertising. ``Something needs to happen sooner
rather than later, and it needs to be done in an intelligent way.''

Mr. Coscelli said the lack of specific tech regulation reminded him of
the lax oversight of banks before the 2008 financial crisis. Regulators
should treat the tech giants more like formerly state-owned enterprises
such as British Telecom and Deutsche Telekom, he said. Starting in the
1980s, those companies were often blocked from practices like bundling
new services at reduced prices, or moving into product areas where new
companies were emerging. Europe is now considered among the world's most
competitive wireless markets.

In Germany, authorities said they were debating rules to restrict how
the tech companies use their dominance in one area to enter new markets.
In recent years, Apple has leveraged its strength in smartphones and
tablets to subsidize its entrance into the video-streaming market,
Google has used its search engine to offer travel services and Facebook
has offered new e-commerce services off its base of social networking.

``If a platform is so big and if a platform has such a powerful
position, there are opportunities to abuse this power,'' said Andreas
Mundt, Germany's top antitrust regulator.

In France, policymakers are debating a new law that would censor hate
speech online, making Facebook, YouTube and Twitter legally liable for
content posted by users, though the proposal
\href{https://www.nytimes.com/2020/06/18/world/europe/france-internet-hate-speech-regulation.html}{is
already facing legal challenges}. Germany has adopted a similar
proposal, and Britain and the European Union are considering such
measures as well. France is also leading an effort with Italy and the
European Union to force the tech companies to pay more taxes.

Many hurdles remain before the proposals become law. Some question
whether the regulations would be effective, particularly if they take
years to enact. Others said any laws could be watered down during the
legislative process as companies pour money into lobbying, or that in a
rush to get something done, flawed policy will be put into place.

``There is a desire to `go further,' but European regulators are
struggling to define the specific problems they want to fix,'' said Joe
McNamee, a veteran internet policy consultant in Brussels, who is
particularly concerned about new online censorship rules. ``Badly
designed measures are unlikely to achieve their goals at the same time
as creating collateral damage.''

William E. Kovacic, a professor specializing in antitrust law at George
Washington University, said that even if many of the proposals did not
become law, the increased scrutiny alone would lead the tech companies
to change behavior.

``It's like the policeman at your elbow,'' he said.

Advertisement

\protect\hyperlink{after-bottom}{Continue reading the main story}

\hypertarget{site-index}{%
\subsection{Site Index}\label{site-index}}

\hypertarget{site-information-navigation}{%
\subsection{Site Information
Navigation}\label{site-information-navigation}}

\begin{itemize}
\tightlist
\item
  \href{https://help.nytimes.com/hc/en-us/articles/115014792127-Copyright-notice}{©~2020~The
  New York Times Company}
\end{itemize}

\begin{itemize}
\tightlist
\item
  \href{https://www.nytco.com/}{NYTCo}
\item
  \href{https://help.nytimes.com/hc/en-us/articles/115015385887-Contact-Us}{Contact
  Us}
\item
  \href{https://www.nytco.com/careers/}{Work with us}
\item
  \href{https://nytmediakit.com/}{Advertise}
\item
  \href{http://www.tbrandstudio.com/}{T Brand Studio}
\item
  \href{https://www.nytimes.com/privacy/cookie-policy\#how-do-i-manage-trackers}{Your
  Ad Choices}
\item
  \href{https://www.nytimes.com/privacy}{Privacy}
\item
  \href{https://help.nytimes.com/hc/en-us/articles/115014893428-Terms-of-service}{Terms
  of Service}
\item
  \href{https://help.nytimes.com/hc/en-us/articles/115014893968-Terms-of-sale}{Terms
  of Sale}
\item
  \href{https://spiderbites.nytimes.com}{Site Map}
\item
  \href{https://help.nytimes.com/hc/en-us}{Help}
\item
  \href{https://www.nytimes.com/subscription?campaignId=37WXW}{Subscriptions}
\end{itemize}
