Sections

SEARCH

\protect\hyperlink{site-content}{Skip to
content}\protect\hyperlink{site-index}{Skip to site index}

\href{https://www.nytimes.com/section/health}{Health}

\href{https://myaccount.nytimes.com/auth/login?response_type=cookie\&client_id=vi}{}

\href{https://www.nytimes.com/section/todayspaper}{Today's Paper}

\href{/section/health}{Health}\textbar{}Children May Carry Coronavirus
at High Levels, Study Finds

\url{https://nyti.ms/3f9oeCp}

\begin{itemize}
\item
\item
\item
\item
\item
\item
\end{itemize}

\href{https://www.nytimes.com/news-event/coronavirus?action=click\&pgtype=Article\&state=default\&region=TOP_BANNER\&context=storylines_menu}{The
Coronavirus Outbreak}

\begin{itemize}
\tightlist
\item
  live\href{https://www.nytimes.com/2020/08/01/world/coronavirus-covid-19.html?action=click\&pgtype=Article\&state=default\&region=TOP_BANNER\&context=storylines_menu}{Latest
  Updates}
\item
  \href{https://www.nytimes.com/interactive/2020/us/coronavirus-us-cases.html?action=click\&pgtype=Article\&state=default\&region=TOP_BANNER\&context=storylines_menu}{Maps
  and Cases}
\item
  \href{https://www.nytimes.com/interactive/2020/science/coronavirus-vaccine-tracker.html?action=click\&pgtype=Article\&state=default\&region=TOP_BANNER\&context=storylines_menu}{Vaccine
  Tracker}
\item
  \href{https://www.nytimes.com/interactive/2020/07/29/us/schools-reopening-coronavirus.html?action=click\&pgtype=Article\&state=default\&region=TOP_BANNER\&context=storylines_menu}{What
  School May Look Like}
\item
  \href{https://www.nytimes.com/live/2020/07/31/business/stock-market-today-coronavirus?action=click\&pgtype=Article\&state=default\&region=TOP_BANNER\&context=storylines_menu}{Economy}
\end{itemize}

Advertisement

\protect\hyperlink{after-top}{Continue reading the main story}

Supported by

\protect\hyperlink{after-sponsor}{Continue reading the main story}

\hypertarget{children-may-carry-coronavirus-at-high-levels-study-finds}{%
\section{Children May Carry Coronavirus at High Levels, Study
Finds}\label{children-may-carry-coronavirus-at-high-levels-study-finds}}

The research does not prove that infected children are contagious, but
it should influence the debate about reopening schools, some experts
said.

\includegraphics{https://static01.nyt.com/images/2020/08/01/science/30VIRUS-CHILDREN1/30VIRUS-CHILDREN1-articleLarge.jpg?quality=75\&auto=webp\&disable=upscale}

By \href{https://www.nytimes.com/by/apoorva-mandavilli}{Apoorva
Mandavilli}

\begin{itemize}
\item
  July 30, 2020
\item
  \begin{itemize}
  \item
  \item
  \item
  \item
  \item
  \item
  \end{itemize}
\end{itemize}

\href{https://www.nytimes.com/es/2020/07/31/espanol/ciencia-y-tecnologia/ninos-contagio-coronavirus.html}{Leer
en español}

It has been a comforting refrain in the national conversation about
reopening schools: Young children are mostly spared by the coronavirus
and don't seem to spread it to others, at least not very often.

But on Thursday, a study introduced an unwelcome wrinkle into this
smooth narrative.

Infected children
\href{https://jamanetwork.com/journals/jamapediatrics/fullarticle/2768952}{have
at least as much of the coronavirus in their noses and throats} as
infected adults, according to the research. Indeed, children younger
than age 5 may host up to 100 times as much of the virus in the upper
respiratory tract as adults, the authors found.

That measurement does not necessarily prove children are passing the
virus to others. Still, the findings should influence the debate over
reopening schools, several experts said.

``The school situation is so complicated --- there are many nuances
beyond just the scientific one,'' said Dr. Taylor Heald-Sargent, a
pediatric infectious diseases expert at the Ann and Robert H. Lurie
Children's Hospital of Chicago, who led the study, published in JAMA
Pediatrics.

``But one takeaway from this is that we can't assume that just because
kids aren't getting sick, or very sick, that they don't have the
virus.''

The study is not without caveats: It was small, and did not specify the
participants' race or sex, or whether they had underlying conditions.
The tests looked for viral RNA, genetic pieces of the coronavirus,
rather than the live virus itself. (Its genetic material is RNA, not
DNA.)

Still, experts were alarmed to learn that young children may carry
significant amounts of the coronavirus.

``I've heard lots of people saying, `Well, kids aren't susceptible, kids
don't get infected.' And this clearly shows that's not true,'' said
Stacey Schultz-Cherry, a virologist at St. Jude Children's Research
Hospital.

``I think this is an important, really important, first step in
understanding the role that kids are playing in transmission.''

Jason Kindrachuk, a virologist at the University of Manitoba, said:
``Now that we're rolling into the end of July and looking at trying to
open up schools the next month, this really needs to be considered.''

\hypertarget{latest-updates-global-coronavirus-outbreak}{%
\section{\texorpdfstring{\href{https://www.nytimes.com/2020/08/01/world/coronavirus-covid-19.html?action=click\&pgtype=Article\&state=default\&region=MAIN_CONTENT_1\&context=storylines_live_updates}{Latest
Updates: Global Coronavirus
Outbreak}}{Latest Updates: Global Coronavirus Outbreak}}\label{latest-updates-global-coronavirus-outbreak}}

Updated 2020-08-02T06:58:18.835Z

\begin{itemize}
\tightlist
\item
  \href{https://www.nytimes.com/2020/08/01/world/coronavirus-covid-19.html?action=click\&pgtype=Article\&state=default\&region=MAIN_CONTENT_1\&context=storylines_live_updates\#link-34047410}{The
  U.S. reels as July cases more than double the total of any other
  month.}
\item
  \href{https://www.nytimes.com/2020/08/01/world/coronavirus-covid-19.html?action=click\&pgtype=Article\&state=default\&region=MAIN_CONTENT_1\&context=storylines_live_updates\#link-780ec966}{Top
  U.S. officials work to break an impasse over the federal jobless
  benefit.}
\item
  \href{https://www.nytimes.com/2020/08/01/world/coronavirus-covid-19.html?action=click\&pgtype=Article\&state=default\&region=MAIN_CONTENT_1\&context=storylines_live_updates\#link-2bc8948}{Its
  outbreak untamed, Melbourne goes into even greater lockdown.}
\end{itemize}

\href{https://www.nytimes.com/2020/08/01/world/coronavirus-covid-19.html?action=click\&pgtype=Article\&state=default\&region=MAIN_CONTENT_1\&context=storylines_live_updates}{See
more updates}

More live coverage:
\href{https://www.nytimes.com/live/2020/07/31/business/stock-market-today-coronavirus?action=click\&pgtype=Article\&state=default\&region=MAIN_CONTENT_1\&context=storylines_live_updates}{Markets}

The standard diagnostic test amplifies the virus's genetic material in
cycles, with the signal growing brighter each round. The more virus
present in the swab initially, the fewer cycles needed for a clear
result.

Dr. Heald-Sargent, who has a research interest in coronaviruses, began
noticing that children's tests were coming back with low ``cycle
thresholds,'' or C.T.s, suggesting that their samples were teeming with
the virus.

Intrigued, she called the hospital lab on a Sunday and asked to look
back at test results for the past several weeks. ``It wasn't even
something we had set out to look for,'' she said.

She and her colleagues analyzed samples collected with nasopharyngeal
swabs between March 23 and April 27 at drive-through testing sites in
Chicago and from people who came to the hospital for any reason,
including symptoms of Covid-19.

They looked at swabs taken from 145 people: 46 children younger than age
5; 51 children aged 5 to 17; and 48 adults aged 18 to 65. To forestall
criticisms that really ill children would be expected to have a lot of
the virus, the team excluded children who needed oxygen support. Most of
the children in the study reported only a fever or cough, Dr.
Heald-Sargent said.

To compare the groups fairly, the team included only children and adults
who had mild to moderate symptoms and for whom they had information
about when symptoms began. Dr. Heald-Sargent left out people who didn't
have symptoms and who did not remember when they had started to feel
ill, as well as those who had symptoms for more than a week before the
testing.

The results confirmed Dr. Heald-Sargent's hunch: Older children and
adults had similar C.T.s, with a median of about 11 and ranging up to
17. But children younger than age 5 had significantly lower C.T.s of
about 6.5. The upper limit of the range in these children was a C.T. of
12, however --- still comparable to those of older children and adults.

``It definitely shows that kids do have levels of virus similar to and
maybe even higher than adults,'' Dr. Heald-Sargent said. ``It wouldn't
be surprising if they were able to shed'' the virus and spread it to
others.

\includegraphics{https://static01.nyt.com/images/2020/07/30/science/30VIRUS-CHILDREN2/merlin_171979788_859ad6e5-7a26-4f54-ab55-3acbf62e70b6-articleLarge.jpg?quality=75\&auto=webp\&disable=upscale}

The results are consistent with those from a German
\href{https://www.nytimes.com/2020/05/05/health/coronavirus-children-transmission-school.html}{study
of 47 infected children} between the ages 1 and 11, which showed that
children who did not have symptoms had viral loads as high as adults',
or higher. And a recent study from France found that asymptomatic
children had
\href{https://academic.oup.com/cid/article/doi/10.1093/cid/ciaa1044/5876373}{C.T.
values similar} to those of children with symptoms.

C.T. values are a reasonable proxy for the amount of coronavirus
present, said Dr. Kindrachuk, who relied on this metric during the Ebola
outbreaks in West Africa.

\href{https://www.nytimes.com/news-event/coronavirus?action=click\&pgtype=Article\&state=default\&region=MAIN_CONTENT_3\&context=storylines_faq}{}

\hypertarget{the-coronavirus-outbreak-}{%
\subsubsection{The Coronavirus Outbreak
›}\label{the-coronavirus-outbreak-}}

\hypertarget{frequently-asked-questions}{%
\paragraph{Frequently Asked
Questions}\label{frequently-asked-questions}}

Updated July 27, 2020

\begin{itemize}
\item ~
  \hypertarget{should-i-refinance-my-mortgage}{%
  \paragraph{Should I refinance my
  mortgage?}\label{should-i-refinance-my-mortgage}}

  \begin{itemize}
  \tightlist
  \item
    \href{https://www.nytimes.com/article/coronavirus-money-unemployment.html?action=click\&pgtype=Article\&state=default\&region=MAIN_CONTENT_3\&context=storylines_faq}{It
    could be a good idea,} because mortgage rates have
    \href{https://www.nytimes.com/2020/07/16/business/mortgage-rates-below-3-percent.html?action=click\&pgtype=Article\&state=default\&region=MAIN_CONTENT_3\&context=storylines_faq}{never
    been lower.} Refinancing requests have pushed mortgage applications
    to some of the highest levels since 2008, so be prepared to get in
    line. But defaults are also up, so if you're thinking about buying a
    home, be aware that some lenders have tightened their standards.
  \end{itemize}
\item ~
  \hypertarget{what-is-school-going-to-look-like-in-september}{%
  \paragraph{What is school going to look like in
  September?}\label{what-is-school-going-to-look-like-in-september}}

  \begin{itemize}
  \tightlist
  \item
    It is unlikely that many schools will return to a normal schedule
    this fall, requiring the grind of
    \href{https://www.nytimes.com/2020/06/05/us/coronavirus-education-lost-learning.html?action=click\&pgtype=Article\&state=default\&region=MAIN_CONTENT_3\&context=storylines_faq}{online
    learning},
    \href{https://www.nytimes.com/2020/05/29/us/coronavirus-child-care-centers.html?action=click\&pgtype=Article\&state=default\&region=MAIN_CONTENT_3\&context=storylines_faq}{makeshift
    child care} and
    \href{https://www.nytimes.com/2020/06/03/business/economy/coronavirus-working-women.html?action=click\&pgtype=Article\&state=default\&region=MAIN_CONTENT_3\&context=storylines_faq}{stunted
    workdays} to continue. California's two largest public school
    districts --- Los Angeles and San Diego --- said on July 13, that
    \href{https://www.nytimes.com/2020/07/13/us/lausd-san-diego-school-reopening.html?action=click\&pgtype=Article\&state=default\&region=MAIN_CONTENT_3\&context=storylines_faq}{instruction
    will be remote-only in the fall}, citing concerns that surging
    coronavirus infections in their areas pose too dire a risk for
    students and teachers. Together, the two districts enroll some
    825,000 students. They are the largest in the country so far to
    abandon plans for even a partial physical return to classrooms when
    they reopen in August. For other districts, the solution won't be an
    all-or-nothing approach.
    \href{https://bioethics.jhu.edu/research-and-outreach/projects/eschool-initiative/school-policy-tracker/}{Many
    systems}, including the nation's largest, New York City, are
    devising
    \href{https://www.nytimes.com/2020/06/26/us/coronavirus-schools-reopen-fall.html?action=click\&pgtype=Article\&state=default\&region=MAIN_CONTENT_3\&context=storylines_faq}{hybrid
    plans} that involve spending some days in classrooms and other days
    online. There's no national policy on this yet, so check with your
    municipal school system regularly to see what is happening in your
    community.
  \end{itemize}
\item ~
  \hypertarget{is-the-coronavirus-airborne}{%
  \paragraph{Is the coronavirus
  airborne?}\label{is-the-coronavirus-airborne}}

  \begin{itemize}
  \tightlist
  \item
    The coronavirus
    \href{https://www.nytimes.com/2020/07/04/health/239-experts-with-one-big-claim-the-coronavirus-is-airborne.html?action=click\&pgtype=Article\&state=default\&region=MAIN_CONTENT_3\&context=storylines_faq}{can
    stay aloft for hours in tiny droplets in stagnant air}, infecting
    people as they inhale, mounting scientific evidence suggests. This
    risk is highest in crowded indoor spaces with poor ventilation, and
    may help explain super-spreading events reported in meatpacking
    plants, churches and restaurants.
    \href{https://www.nytimes.com/2020/07/06/health/coronavirus-airborne-aerosols.html?action=click\&pgtype=Article\&state=default\&region=MAIN_CONTENT_3\&context=storylines_faq}{It's
    unclear how often the virus is spread} via these tiny droplets, or
    aerosols, compared with larger droplets that are expelled when a
    sick person coughs or sneezes, or transmitted through contact with
    contaminated surfaces, said Linsey Marr, an aerosol expert at
    Virginia Tech. Aerosols are released even when a person without
    symptoms exhales, talks or sings, according to Dr. Marr and more
    than 200 other experts, who
    \href{https://academic.oup.com/cid/article/doi/10.1093/cid/ciaa939/5867798}{have
    outlined the evidence in an open letter to the World Health
    Organization}.
  \end{itemize}
\item ~
  \hypertarget{what-are-the-symptoms-of-coronavirus}{%
  \paragraph{What are the symptoms of
  coronavirus?}\label{what-are-the-symptoms-of-coronavirus}}

  \begin{itemize}
  \tightlist
  \item
    Common symptoms
    \href{https://www.nytimes.com/article/symptoms-coronavirus.html?action=click\&pgtype=Article\&state=default\&region=MAIN_CONTENT_3\&context=storylines_faq}{include
    fever, a dry cough, fatigue and difficulty breathing or shortness of
    breath.} Some of these symptoms overlap with those of the flu,
    making detection difficult, but runny noses and stuffy sinuses are
    less common.
    \href{https://www.nytimes.com/2020/04/27/health/coronavirus-symptoms-cdc.html?action=click\&pgtype=Article\&state=default\&region=MAIN_CONTENT_3\&context=storylines_faq}{The
    C.D.C. has also} added chills, muscle pain, sore throat, headache
    and a new loss of the sense of taste or smell as symptoms to look
    out for. Most people fall ill five to seven days after exposure, but
    symptoms may appear in as few as two days or as many as 14 days.
  \end{itemize}
\item ~
  \hypertarget{does-asymptomatic-transmission-of-covid-19-happen}{%
  \paragraph{Does asymptomatic transmission of Covid-19
  happen?}\label{does-asymptomatic-transmission-of-covid-19-happen}}

  \begin{itemize}
  \tightlist
  \item
    So far, the evidence seems to show it does. A widely cited
    \href{https://www.nature.com/articles/s41591-020-0869-5}{paper}
    published in April suggests that people are most infectious about
    two days before the onset of coronavirus symptoms and estimated that
    44 percent of new infections were a result of transmission from
    people who were not yet showing symptoms. Recently, a top expert at
    the World Health Organization stated that transmission of the
    coronavirus by people who did not have symptoms was ``very rare,''
    \href{https://www.nytimes.com/2020/06/09/world/coronavirus-updates.html?action=click\&pgtype=Article\&state=default\&region=MAIN_CONTENT_3\&context=storylines_faq\#link-1f302e21}{but
    she later walked back that statement.}
  \end{itemize}
\end{itemize}

Still, he and others said that ideally researchers would grow infectious
virus from samples, rather than test only for the virus's RNA.

``I suspect that it probably will translate into meaning that there is
more actual virus there as well, but we can't say that without seeing
the data,'' said Juliet Morrison, a virologist at the University of
California, Riverside.

Some RNA viruses multiply quickly and are prone to genetic errors that
render the virus incapable of infecting cells. Some RNA detected in
children may represent these ``defective'' viruses: ``We need to
understand how much of that is actually infectious virus,'' Dr.
Schultz-Cherry said.

(The researchers said they did not have access to the type of
high-security lab required to grow infectious coronavirus, but other
teams \href{https://pubmed.ncbi.nlm.nih.gov/32603290/}{have cultivated
virus} from children's samples.)

The experts all emphasized that the findings at least indicate that
children can be infected. Those who harbor a lot of virus may spread it
to others in their households, or to teachers and other school staff
members when schools reopen.

Many school districts are planning to protect students and staff members
by implementing physical distancing, cloth face coverings and hand
hygiene. But it's unclear how well staff members and teachers can keep
young children from getting too close to others, Dr. Kindrachuk said.

``Frankly, I just haven't seen a lot of discussion about how that aspect
is going to be controlled,'' he said.

\textbf{\emph{{[}}\href{http://on.fb.me/1paTQ1h}{\emph{Like the Science
Times page on Facebook.}}} ****** \emph{\textbar{} Sign up for the}
\textbf{\href{http://nyti.ms/1MbHaRU}{\emph{Science Times
newsletter.}}\emph{{]}}}

Observations from schools in several countries have suggested that, at
least in places with mild outbreaks and preventive measures in place,
children do not seem to spread the coronavirus to others efficiently.

Strong immune responses in children could limit both how much virus they
can spread to others and for how long. The children's overall health,
underlying conditions such as obesity or diabetes, and sex may also
influence the ability to transmit the virus.

Some experts have suggested that children
\href{https://www.nytimes.com/2020/06/30/us/coronavirus-schools-reopening-guidelines-aap.html}{may
transmit less virus} because of their smaller lung capacity, height or
other physical aspects.

Dr. Morrison dismissed those suggestions. The virus is shed from the
upper respiratory tract, not the lungs, she noted.

``We are going to be reopening day care and elementary schools,'' she
said. If these results hold up, ``then yeah, I'd be worried.''

Advertisement

\protect\hyperlink{after-bottom}{Continue reading the main story}

\hypertarget{site-index}{%
\subsection{Site Index}\label{site-index}}

\hypertarget{site-information-navigation}{%
\subsection{Site Information
Navigation}\label{site-information-navigation}}

\begin{itemize}
\tightlist
\item
  \href{https://help.nytimes.com/hc/en-us/articles/115014792127-Copyright-notice}{©~2020~The
  New York Times Company}
\end{itemize}

\begin{itemize}
\tightlist
\item
  \href{https://www.nytco.com/}{NYTCo}
\item
  \href{https://help.nytimes.com/hc/en-us/articles/115015385887-Contact-Us}{Contact
  Us}
\item
  \href{https://www.nytco.com/careers/}{Work with us}
\item
  \href{https://nytmediakit.com/}{Advertise}
\item
  \href{http://www.tbrandstudio.com/}{T Brand Studio}
\item
  \href{https://www.nytimes.com/privacy/cookie-policy\#how-do-i-manage-trackers}{Your
  Ad Choices}
\item
  \href{https://www.nytimes.com/privacy}{Privacy}
\item
  \href{https://help.nytimes.com/hc/en-us/articles/115014893428-Terms-of-service}{Terms
  of Service}
\item
  \href{https://help.nytimes.com/hc/en-us/articles/115014893968-Terms-of-sale}{Terms
  of Sale}
\item
  \href{https://spiderbites.nytimes.com}{Site Map}
\item
  \href{https://help.nytimes.com/hc/en-us}{Help}
\item
  \href{https://www.nytimes.com/subscription?campaignId=37WXW}{Subscriptions}
\end{itemize}
