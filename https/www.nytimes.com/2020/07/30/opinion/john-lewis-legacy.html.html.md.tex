Sections

SEARCH

\protect\hyperlink{site-content}{Skip to
content}\protect\hyperlink{site-index}{Skip to site index}

\href{https://myaccount.nytimes.com/auth/login?response_type=cookie\&client_id=vi}{}

\href{https://www.nytimes.com/section/todayspaper}{Today's Paper}

\href{/section/opinion}{Opinion}\textbar{}John Lewis Believed America
Would Survive Trump

\url{https://nyti.ms/2DiXWjZ}

\begin{itemize}
\item
\item
\item
\item
\item
\item
\end{itemize}

Advertisement

\protect\hyperlink{after-top}{Continue reading the main story}

\href{/section/opinion}{Opinion}

Supported by

\protect\hyperlink{after-sponsor}{Continue reading the main story}

\hypertarget{john-lewis-believed-america-would-survive-trump}{%
\section{John Lewis Believed America Would Survive
Trump}\label{john-lewis-believed-america-would-survive-trump}}

He told us to keep the faith. It's not easy.

\href{https://www.nytimes.com/by/michelle-goldberg}{\includegraphics{https://static01.nyt.com/images/2018/04/02/opinion/michelle-goldberg/michelle-goldberg-thumbLarge.png}}

By \href{https://www.nytimes.com/by/michelle-goldberg}{Michelle
Goldberg}

Opinion Columnist

\begin{itemize}
\item
  July 30, 2020
\item
  \begin{itemize}
  \item
  \item
  \item
  \item
  \item
  \item
  \end{itemize}
\end{itemize}

\includegraphics{https://static01.nyt.com/images/2020/07/30/opinion/30lewis1/merlin_128348066_40469054-04f5-42bc-bfec-2928f224ab45-articleLarge.jpg?quality=75\&auto=webp\&disable=upscale}

In June 2017, I ran into John Lewis outside of Atlanta, where he was
campaigning for his former intern Jon Ossoff in the special election for
Georgia's Sixth Congressional District. I asked him something I asked
everyone in those days, when the horror of this administration was still
fresh: How confident was he that America would recover from Donald
Trump?

``We will get there,'' Lewis said. ``We will survive. We will survive.''
During the civil rights movement, he said, there were people ``who said
that we wouldn't get a Civil Rights Act when we were marching from
Selma. We wouldn't get a Voting Rights Act. We wouldn't get a Fair
Housing Act. But we never gave up, we never gave in. We kept the
faith.''

There was something saintly about Lewis, whose funeral was held on
Thursday. What's striking in accounts of his youthful encounters with
snarling, murderous white supremacy is not just his courage, but also
his calm and otherworldly clarity.

The historian Taylor Branch described a 1961 debate within part of the
civil rights movement about whether to keep up demonstrations in
Nashville in the face of escalating white violence. ``Whenever asked a
question, he ignored the fine points of whatever theory was being put
forward and said simply, `We're gonna march tonight,''' Branch wrote of
Lewis.

A prominent white clergyman named Will D. Campbell lost his temper,
accusing Lewis of the sin of pride. ``Lewis smiled warmly at Campbell,
as though taking pity on him,'' wrote Branch. ```OK, I'm a sinner,' he
replied softy. `We're gonna march.''' Lewis's persistence won, the march
went on and he was arrested for the fourth of at least 45 times.

Lewis, the best of this country, had seen the worst of it and still had
faith. ``Ordinary people with extraordinary vision can redeem the soul
of America by getting in what I call good trouble, necessary trouble,''
he
\href{https://www.nytimes.com/2020/07/30/opinion/john-lewis-civil-rights-america.html}{wrote
in an essay} for The Times to publish on the day of his funeral.

Lately I've struggled to hold on to hope of redemption. On the night
Trump was elected, I felt as if the ground was crumbling beneath my
feet, and yet looking back I was naïve about how bad things could get.

Mass death from a pandemic running unchecked, without even the pretense
that the federal government will help us. A Congress that's allowing
emergency aid to the unemployed to lapse during an economic crisis as
bad or worse than the Great Depression. Unidentified men in camouflage
beating protesters in the streets. Public education near collapse. A
president musing about postponing the election, and his thuggish
secretary of state backing him by suggesting that's
\href{https://twitter.com/atrupar/status/1288847087950495749?s=20}{a
live possibility}.

My life is far easier than that of most people in this country, but
since March of this despicable year I've felt dread from the moment I
open my eyes in the morning until I fall asleep at night. Sometimes in
the evening my son cries and says he hates being a child in the time of
coronavirus, and I try not to cry too.

If this president makes good on his threats to undermine an election
he's likely to lose, many of us will be called to pour into the streets
and face the brutality of Trump's goons. This thought makes me feel
ground down and frightened, not brave and defiant. In middle age I've
started to envy those like Lewis who are able to believe in God.

But something I take from reading about the lives of civil rights heroes
is that confidence didn't always precede action. Sometimes it was
action's result. Branch wrote of how, the first time Lewis was arrested,
``a lifetime of absorbed taboos against any kind of trouble with the law
quickened into terror.'' But on the ride to jail, ``dread gave way to an
exhilaration unlike any he had ever known.''

Lewis would later
\href{https://twitter.com/repjohnlewis/status/489776381363236864?s=20}{describe
it} as a sense of liberation, of crossing over. He and his fellow
activists showed that hope is as much a practice as a feeling.

At Lewis's funeral, Barack Obama eulogized him from Martin Luther King
Jr.'s former pulpit. Our last real president was blunt about the
parallels between the current regime and the villains of the civil
rights era: ``George Wallace may be gone, but we can witness our federal
government sending agents to use tear gas and batons against peaceful
demonstrators,'' he said.

Lewis, said Obama, ``devoted his time on this earth fighting the very
attacks on democracy, and what's best in America, that we're seeing
circulate right now.''

Lewis's movement defeated men like Wallace in one generation, then saw a
man like Wallace replace the first Black president in the next. ``In
spite of it all, we must be hopeful, we must be optimistic, we must
never get lost in a sea of despair,'' Lewis told me three years ago. He
wasn't just describing a disposition. He was describing a discipline.

\emph{The Times is committed to publishing}
\href{https://www.nytimes.com/2019/01/31/opinion/letters/letters-to-editor-new-york-times-women.html}{\emph{a
diversity of letters}} \emph{to the editor. We'd like to hear what you
think about this or any of our articles. Here are some}
\href{https://help.nytimes.com/hc/en-us/articles/115014925288-How-to-submit-a-letter-to-the-editor}{\emph{tips}}\emph{.
And here's our email:}
\href{mailto:letters@nytimes.com}{\emph{letters@nytimes.com}}\emph{.}

\emph{Follow The New York Times Opinion section on}
\href{https://www.facebook.com/nytopinion}{\emph{Facebook}}\emph{,}
\href{http://twitter.com/NYTOpinion}{\emph{Twitter (@NYTopinion)}}
\emph{and}
\href{https://www.instagram.com/nytopinion/}{\emph{Instagram}}\emph{.}

Advertisement

\protect\hyperlink{after-bottom}{Continue reading the main story}

\hypertarget{site-index}{%
\subsection{Site Index}\label{site-index}}

\hypertarget{site-information-navigation}{%
\subsection{Site Information
Navigation}\label{site-information-navigation}}

\begin{itemize}
\tightlist
\item
  \href{https://help.nytimes.com/hc/en-us/articles/115014792127-Copyright-notice}{©~2020~The
  New York Times Company}
\end{itemize}

\begin{itemize}
\tightlist
\item
  \href{https://www.nytco.com/}{NYTCo}
\item
  \href{https://help.nytimes.com/hc/en-us/articles/115015385887-Contact-Us}{Contact
  Us}
\item
  \href{https://www.nytco.com/careers/}{Work with us}
\item
  \href{https://nytmediakit.com/}{Advertise}
\item
  \href{http://www.tbrandstudio.com/}{T Brand Studio}
\item
  \href{https://www.nytimes.com/privacy/cookie-policy\#how-do-i-manage-trackers}{Your
  Ad Choices}
\item
  \href{https://www.nytimes.com/privacy}{Privacy}
\item
  \href{https://help.nytimes.com/hc/en-us/articles/115014893428-Terms-of-service}{Terms
  of Service}
\item
  \href{https://help.nytimes.com/hc/en-us/articles/115014893968-Terms-of-sale}{Terms
  of Sale}
\item
  \href{https://spiderbites.nytimes.com}{Site Map}
\item
  \href{https://help.nytimes.com/hc/en-us}{Help}
\item
  \href{https://www.nytimes.com/subscription?campaignId=37WXW}{Subscriptions}
\end{itemize}
