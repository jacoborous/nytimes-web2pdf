Sections

SEARCH

\protect\hyperlink{site-content}{Skip to
content}\protect\hyperlink{site-index}{Skip to site index}

\href{https://myaccount.nytimes.com/auth/login?response_type=cookie\&client_id=vi}{}

\href{https://www.nytimes.com/section/todayspaper}{Today's Paper}

\href{/section/opinion}{Opinion}\textbar{}The Nightmare on Pennsylvania
Avenue

\url{https://nyti.ms/2PaCYXg}

\begin{itemize}
\item
\item
\item
\item
\item
\item
\end{itemize}

Advertisement

\protect\hyperlink{after-top}{Continue reading the main story}

\href{/section/opinion}{Opinion}

Supported by

\protect\hyperlink{after-sponsor}{Continue reading the main story}

\hypertarget{the-nightmare-on-pennsylvania-avenue}{%
\section{The Nightmare on Pennsylvania
Avenue}\label{the-nightmare-on-pennsylvania-avenue}}

Trump is the kind of boss who can't do the job --- and won't go away.

\href{https://www.nytimes.com/by/paul-krugman}{\includegraphics{https://static01.nyt.com/images/2018/04/02/opinion/paul-krugman/paul-krugman-thumbLarge.png}}

By \href{https://www.nytimes.com/by/paul-krugman}{Paul Krugman}

Opinion Columnist

\begin{itemize}
\item
  July 30, 2020
\item
  \begin{itemize}
  \item
  \item
  \item
  \item
  \item
  \item
  \end{itemize}
\end{itemize}

\includegraphics{https://static01.nyt.com/images/2020/07/30/opinion/30krugmanWeb/merlin_175068288_6fc4088e-0dfb-4ea9-b02b-1b5f2149e907-articleLarge.jpg?quality=75\&auto=webp\&disable=upscale}

Every worker's nightmare is the horrible boss --- everyone knows at
least one --- who is utterly incompetent yet refuses to step aside. Such
bosses have the reverse Midas touch --- everything they handle turns to
crud --- but they'll pull out every stop, violate every norm, to stay in
that corner office. And they damage, sometimes destroy, the institutions
they're supposed to lead.

Donald Trump is, of course, one of those bosses. Unfortunately, he's not
just a bad business executive. He is, God help us, the president. And
the institution he may destroy is the United States of America.

Has any previous president failed his big test as thoroughly as Trump
has these past few months? He rejected the advice of health experts and
pushed for a rapid economic reopening, hoping for a boom leading into
the election. He ridiculed and belittled measures that would have helped
slow the spread of the coronavirus, including wearing face masks and
practicing social distancing, turning what should have been common sense
into a front in the culture war.

The result has been disaster both epidemiological and economic.

Over the past week the U.S.
\href{https://ourworldindata.org/coronavirus-data-explorer?zoomToSelection=true\&deathsMetric=true\&interval=smoothed\&smoothing=7\&country=USA~DEU\&pickerMetric=location\&pickerSort=asc}{death
toll} from Covid-19 averaged more than 1,000 people a day, compared with
just four --- four! --- per day in Germany. Vice President Mike Pence's
mid-June
\href{https://www.wsj.com/articles/there-isnt-a-coronavirus-second-wave-11592327890}{declaration}
that ``There isn't a coronavirus `second wave''' felt like whistling in
the dark even at the time; now it feels like a sick joke.

And all those extra deaths don't seem to have bought us anything in
terms of economic performance. America's economic contraction in the
first half of 2020 was
\href{https://www.wsj.com/articles/germanys-economy-suffers-biggest-contraction-on-record-but-green-shoots-emerge-11596101866}{almost
identical} to the contraction in Germany, despite our far higher death
toll. And while life in Germany has in many ways returned to normal, a
\href{https://twitter.com/ernietedeschi/status/1286740199796596743}{variety}
of
\href{https://twitter.com/ernietedeschi/status/1286741059435995136}{indicators}
\href{https://twitter.com/dandolfa/status/1287907164955303936}{suggest}
that after two months of rapid job growth, the U.S. recovery is stalling
in the face of a resurgent pandemic.

Wait, it gets worse. Trump, his officials and their allies in the Senate
have been totally committed to the idea that the U.S. economy will
experience a stunningly rapid recovery despite the wave of new
infections and deaths. They bought into that view so completely that
they seem incapable of taking on board the overwhelming evidence that it
isn't happening.

Just a few days ago Larry Kudlow, Trump's top economist, insisted that a
so-called V-shaped recovery was
\href{https://www.foxbusiness.com/economy/kudlow-maintains-v-shaped-economic-recovery-still-intact-despite-coronavirus-resurgence}{still
on track} and that ``unemployment claims and continuing claims are
falling rapidly.'' In fact,
\href{https://fred.stlouisfed.org/graph/fredgraph.png?g=tzhR}{both are
rising}.

But because the Trump team insisted that a roaring recovery was coming,
and refused to notice that it wasn't happening, we've now stumbled into
a completely gratuitous economic crisis.

Thanks to Republican inaction, millions of unemployed workers have seen
their last checks from the Pandemic Unemployment Compensation program,
which was meant to sustain them through a coronavirus-ravaged economy;
the virus is still raging, but their life support has been cut off.

So Trump has completely botched his job, bringing unnecessary pain to
millions of Americans and unnecessary death to thousands. He may not
care, but voters do. So he should be trying to turn things around, if
only as a matter of political and personal self-interest.

But here's the thing: Even if Trump were the kind of guy who could learn
from his mistakes, it's too late. If we had found ourselves in our
current situation a year ago, there might still have been time for Trump
to get the virus under control and turn the economy around. But the
election is just around the corner.

Suppose that the numbers on deaths and jobs were to get somewhat better
over the next three months. How much would that improve voters' views of
the denier in chief? How much credence would the public give, even to
genuinely good news, after the false dawn this past spring? At this
point Trump is simply a failed president, and everyone except his
die-hard supporters knows it.

But as I said at the beginning, Trump is one of those nightmare bosses
who can't do the job but won't step aside.

So of course he's now talking about
\href{https://www.washingtonpost.com/politics/trump-floats-idea-of-delaying-the-november-election-as-he-ramps-up-attacks-on-voting-by-mail/2020/07/30/15fe7ac6-d264-11ea-9038-af089b63ac21_story.html?hpid=hp_hp-top-table-main_trumpelection-920am\%3Ahomepage\%2Fstory-ans}{delaying
the election}. This was predictable; indeed,
\href{https://www.usatoday.com/story/news/politics/2020/04/24/joe-biden-says-trump-try-kick-back-election-november/3018352001/}{Joe
Biden} predicted it months ago, amid much mockery from pundits (none of
whom, I predict, will apologize).

Now, Trump can't do that. There will be an election on Nov. 3. But what
Trump can do, if he loses, is claim that the election was stolen, that
there were millions of fraudulent votes, that the results aren't
legitimate. Hey, he did that after losing the popular vote in 2016, even
though he won the Electoral College.

Such antics almost surely wouldn't let him stay in the White House,
although the process of getting him out may be \ldots{} interesting. But
they could produce a lot of chaos and quite possibly some violence
across the nation. And anyone who doesn't think disgruntled Trump
supporters would try to sabotage a Biden administration --- including
its efforts to deal with the pandemic --- hasn't been paying attention.

This is what happens when you put a horrible boss in charge of running
the country. And nobody can say when, if ever, the damage will be
repaired.

\emph{The Times is committed to publishing}
\href{https://www.nytimes.com/2019/01/31/opinion/letters/letters-to-editor-new-york-times-women.html}{\emph{a
diversity of letters}} \emph{to the editor. We'd like to hear what you
think about this or any of our articles. Here are some}
\href{https://help.nytimes.com/hc/en-us/articles/115014925288-How-to-submit-a-letter-to-the-editor}{\emph{tips}}\emph{.
And here's our email:}
\href{mailto:letters@nytimes.com}{\emph{letters@nytimes.com}}\emph{.}

\emph{Follow The New York Times Opinion section on}
\href{https://www.facebook.com/nytopinion}{\emph{Facebook}}\emph{,}
\href{http://twitter.com/NYTOpinion}{\emph{Twitter (@NYTopinion)}}
\emph{and}
\href{https://www.instagram.com/nytopinion/}{\emph{Instagram}}\emph{.}

Advertisement

\protect\hyperlink{after-bottom}{Continue reading the main story}

\hypertarget{site-index}{%
\subsection{Site Index}\label{site-index}}

\hypertarget{site-information-navigation}{%
\subsection{Site Information
Navigation}\label{site-information-navigation}}

\begin{itemize}
\tightlist
\item
  \href{https://help.nytimes.com/hc/en-us/articles/115014792127-Copyright-notice}{©~2020~The
  New York Times Company}
\end{itemize}

\begin{itemize}
\tightlist
\item
  \href{https://www.nytco.com/}{NYTCo}
\item
  \href{https://help.nytimes.com/hc/en-us/articles/115015385887-Contact-Us}{Contact
  Us}
\item
  \href{https://www.nytco.com/careers/}{Work with us}
\item
  \href{https://nytmediakit.com/}{Advertise}
\item
  \href{http://www.tbrandstudio.com/}{T Brand Studio}
\item
  \href{https://www.nytimes.com/privacy/cookie-policy\#how-do-i-manage-trackers}{Your
  Ad Choices}
\item
  \href{https://www.nytimes.com/privacy}{Privacy}
\item
  \href{https://help.nytimes.com/hc/en-us/articles/115014893428-Terms-of-service}{Terms
  of Service}
\item
  \href{https://help.nytimes.com/hc/en-us/articles/115014893968-Terms-of-sale}{Terms
  of Sale}
\item
  \href{https://spiderbites.nytimes.com}{Site Map}
\item
  \href{https://help.nytimes.com/hc/en-us}{Help}
\item
  \href{https://www.nytimes.com/subscription?campaignId=37WXW}{Subscriptions}
\end{itemize}
