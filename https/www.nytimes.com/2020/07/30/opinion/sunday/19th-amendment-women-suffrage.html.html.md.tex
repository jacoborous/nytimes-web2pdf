Sections

SEARCH

\protect\hyperlink{site-content}{Skip to
content}\protect\hyperlink{site-index}{Skip to site index}

\href{https://www.nytimes.com/section/opinion/sunday}{Sunday Review}

\href{https://myaccount.nytimes.com/auth/login?response_type=cookie\&client_id=vi}{}

\href{https://www.nytimes.com/section/todayspaper}{Today's Paper}

\href{/section/opinion/sunday}{Sunday Review}\textbar{}100 Years of
Voting Hasn't Done What We Thought It Would

\href{https://nyti.ms/3gmkL55}{https://nyti.ms/3gmkL55}

\begin{itemize}
\item
\item
\item
\item
\item
\item
\end{itemize}

Advertisement

\protect\hyperlink{after-top}{Continue reading the main story}

\href{/section/opinion}{Opinion}

Supported by

\protect\hyperlink{after-sponsor}{Continue reading the main story}

\hypertarget{100-years-of-voting-hasnt-done-what-we-thought-it-would}{%
\section{100 Years of Voting Hasn't Done What We Thought It
Would}\label{100-years-of-voting-hasnt-done-what-we-thought-it-would}}

The unfinished business of the women's vote.

\href{https://www.nytimes.com/by/gail-collins}{\includegraphics{https://static01.nyt.com/images/2018/04/03/opinion/gail-collins/gail-collins-thumbLarge.png}}

By \href{https://www.nytimes.com/by/gail-collins}{Gail Collins}

Opinion Columnist

\begin{itemize}
\item
  July 30, 2020
\item
  \begin{itemize}
  \item
  \item
  \item
  \item
  \item
  \item
  \end{itemize}
\end{itemize}

\includegraphics{https://static01.nyt.com/images/2020/08/02/opinion/30collins1/merlin_171985329_3e27fc41-8ac6-430c-828b-4b9070d1293c-articleLarge.jpg?quality=75\&auto=webp\&disable=upscale}

What better way to celebrate the 100th anniversary of women's suffrage
than by discussing the way it turned out to be a big flop?

The great champions of the 19th Amendment thought that when America's
women got the right to vote, they'd immediately start to change the
nation. Promote women's issues, like better health care and education.
Refocus politics from special interests to the general good.

Then in 1920, for the first time, they went to polls across the nation
with their husbands, brothers, fathers and sons and elected ---
President Warren Harding.

In 1921, Congress, with a wary eye on the newly enfranchised sex, passed
the Sheppard-Towner Maternity and Infancy Protection Act. It was a
modest effort to improve health care for the poor by training nurses,
licensing midwives and establishing clinics for young mothers and their
babies.

The physicians' associations saw it as government-subsidized competition
--- socialized medicine! --- and hated it. During debate on the bill,
one opponent claimed the sponsors were pandering to busybody old maids
who were always pushing do-gooder causes.

``Old maids are voting now,'' a colleague reminded him.

But the doctors kept complaining, and as time passed, politicians began
to notice that they weren't hearing much from the new female electorate.
In 1929, the act was repealed.

The Sheppard-Towner debacle was one of the best examples of how the
effects of women's suffrage turned out to be more complicated than its
champions had imagined. Everything worked great when it came to the
title cause of giving women the right to vote. But the leaders of the
movement had expected to use the ballot to transform the nation. For a
very long time, nothing happened.

Well, except for Prohibition. Banning the sale of liquor was one cause
that really did bring the women together. Most of them didn't drink, but
their husbands did. The upper-class men retired to the study or a club
after dinner to sip some liquor and have fun talking among themselves.
Poor men went off to a saloon to get soused, spending the family's
much-needed cash.

\includegraphics{https://static01.nyt.com/images/2020/07/30/opinion/30collins4/30collins4-articleLarge.jpg?quality=75\&auto=webp\&disable=upscale}

Many American girls grew up believing that virtually every social evil
came from alcohol.
\href{https://francesperkinscenter.org/life-new/}{Frances Perkins}, the
New Deal secretary of labor, recalled that she was raised to believe
that poverty was just a result of drinking --- and laziness.

Once Congress approved the 19th Amendment, the liquor lobbyists
stampeded to the state legislatures to try to stop ratification. They
won enough battles to leave suffragists one state short of victory and
only Tennessee left to vote. All eyes turned to Nashville.

The State Senate voted yes while virtually everybody in the capitol was
getting
\href{https://www.nytimes.com/2018/03/05/opinion/women-votes-feminism-alcohol.html}{swacked
on the lobbyists' free samples}. Then it all came down to the House of
Representatives, where the ``no'' group had a one-man majority. On Aug.
18, 1920, a 24-year-old suffrage opponent named Harry Burn got up and
reported to his colleagues that he'd gotten a letter from his mother
telling him to ``be a good boy'' and help the women's cause.

``I know that a mother's advice is always the safest for a boy to
follow,'' he told his colleagues. And he switched his vote. Suffrage
ruled.

That was a great culmination, and much more fun to report than the slog
that preceded it. We will refrain from revisiting what suffragists
counted as 480 campaigns to get state legislatures to submit the issue
to the voters.

Some fights had been much, much easier than others. Lawmakers in Wyoming
had eagerly voted for the franchise in 1869, hoping it might be a draw
for a territory in which men outnumbered women six to one. ``We now
expect quite an immigration of ladies to Wyoming,'' said The Cheyenne
Leader hopefully after the legislature voted for women's suffrage, as
well as women's property rights and equal pay for women schoolteachers.

(There was nothing like being a rare commodity to raise the bar on
women's opportunities. Back when the first male colonists were settling
into the New World, they wrote back advertising for female émigrés,
promising they would find a husband in a snap, as long as they were
``but civil and under 50 years of age.'')

Wherever suffrage arrived, there were lots of women who resisted the
idea of getting involved. Election Day was, in many neighborhoods, a
rowdy time when political parties tried to encourage voter turnout with
--- yes! --- free liquor. ``Saloons, marching, drinking all day ---
voting was seen as a very masculine act,'' says Debbie Walsh of the
\href{https://cawp.rutgers.edu/}{Center for American Women and Politics}
at Rutgers University.

Theodore Roosevelt told a crowd of suffrage supporters he was the only
person in his family who agreed with their agenda, and urged them to
``go and convert my wife and daughters.'' His young niece Eleanor was
among the unenthusiastic.

I don't have to tell you that things changed. Women went to the polls
more and more with every generation. But politicians still presumed that
they'd vote with their menfolk unless something very unusual cropped up.

When Woodrow Wilson was up for re-election in 1916 his handlers did
worry about the ``women's vote'' in the states where they already had
the franchise. The president's wife had died during his first term of
office and Wilson rather quickly picked up with Edith Galt, the widow of
a prominent Washington jeweler. They wanted to marry right away, but
Wilson's aides were afraid of how the news might affect the female
electorate. In the end, the answer was: not much.

Perhaps voters didn't hear the gossip in political circles about what
was said to be a hot and heavy premarital affair. (The political
columnist Murray Kempton told me he heard a joke when he was a boy in
the 1920s, in which when the president proposed, Mrs. Galt was so
excited she fell out of bed. ``I think my sainted mother told me that
one,'' Kempton recalled.) After the Wilson engagement became official,
The Washington Post printed a social note containing one of the most
famous typos in American history: ``The President gave himself up for
the time being to entering his fiancée.''

Image

A photo illustration by the American Press Association in 1915 reflected
public interest in the relationship between Edith Galt and President
Woodrow Wilson.Credit...Library of Congress

OK, that's just an interesting diversion. But Wilson won, and the
conviction that women were mainly just duplicating the votes of their
husbands or fathers held sway.

You have to wonder, as the years went on, how many husbands were
actually reflecting their wives' opinions when they went to the polls.
The balance of power within families has shifted dramatically over the
last 50 years, mainly because of money. The transformation began when
the country's post-World War II economic boom hit the killer recession
of the 1970s, and everyone began to realize that a whole lot of the
families of the future would not be able to afford a middle-class
lifestyle unless the wives kept working.

The women's movement combined with the hard facts of the economy created
a world in which almost no one envisioned young women with a distinctly
different wage-earning future from men. I'll never forget a visit I made
to a community college in Connecticut, back around 1980. I was invited
for some reason to speak to a class of young men, and I asked them to
describe for me their ideal mate. There were a few polite murmurs about
a good sense of humor and fine moral character --- then someone called
out, ``And a good earner!'' I cannot tell you how enthusiastic the room
became over the ``good earner'' qualification.

It took professional politicians quite a while to notice there was a
change going on. Then in 1980, when Ronald Reagan defeated
then-President Jimmy Carter, it became clear the country had moved on to
a whole new political wave.
\href{http://cawp.rutgers.edu/sites/default/files/resources/ggpresvote.pdf}{Analysis
of the final tallies} showed that both sexes favored Reagan, but the
women split very narrowly while the men went Republican 55 percent to 36
percent. The gender gap was born, and it really turned into a canyon in
1996, when Bill Clinton won the women's vote by a wide margin, while men
narrowly favored Bob Dole.

These days, women go to the polls more faithfully than men, and they are
more likely to vote Democratic. That doesn't mean they always win. In
2000, women favored Al Gore for president over George W. Bush, 54-44
percent, while the men went for Bush, 54-43. In 2016, the male voters
gave us Donald Trump in an election where the gender gap yawned at 11
points.

But the power is there. Black women, who've fought dual battles against
racism and sexism to exercise their right to vote, knocked the socks off
Democratic organizers in Alabama in 2017 when they gave long-shot Senate
candidate Doug Jones
\href{https://thehill.com/homenews/campaign/364665-exit-polls-98-percent-of-black-women-voted-for-jones}{98
percent of their vote} and a victory over Republican
former-judge-and-pursuer-of-teenage-girls Roy Moore.

Image

Democratic congresswomen dressed in white in honor of suffragists at the
State of the Union address in 2019.Credit...Sarah Silbiger/The New York
Times

If 1920s heroines like Susan B. Anthony and Ida B. Wells were around
now, they'd be setting their targets way higher than the voting booth.
We live in an era that's beginning to find women running for office
almost as normal as Mom having a job outside the home. Nearly
\href{https://cawp.rutgers.edu/current-numbers}{a quarter of our current
Congress is female}, and the pace is picking up all the time. I still
remember in 2001 when Hillary Clinton was sworn in to the Senate and my
young niece innocently asked my sister if men were allowed to be in the
Senate, too. Susan B. Anthony would have fainted with happiness.

Women who tearily discovered in 2016 that they weren't going to be able
to introduce their daughters to the first woman president have mostly
gotten over it. If everything we think we know about the current
presidential race is reasonably true --- and nothing crazy happens over
the rest of the campaign --- next January the country will have a female
vice president, a woman who the voters trusted as second in command to
78-year-old Joe Biden.

``Women's issues'' --- like guaranteed quality health care for all and
reproductive freedom --- may still not have universal political support.
But they're now political goals for a vast swath of the voting public,
both male and female. And maybe it won't be too long before someone's
little niece in the future innocently asks her mother whether men are
allowed to be president, too.

\emph{The Times is committed to publishing}
\href{https://www.nytimes.com/2019/01/31/opinion/letters/letters-to-editor-new-york-times-women.html}{\emph{a
diversity of letters}} \emph{to the editor. We'd like to hear what you
think about this or any of our articles. Here are some}
\href{https://help.nytimes.com/hc/en-us/articles/115014925288-How-to-submit-a-letter-to-the-editor}{\emph{tips}}\emph{.
And here's our email:}
\href{mailto:letters@nytimes.com}{\emph{letters@nytimes.com}}\emph{.}

\emph{Follow The New York Times Opinion section on}
\href{https://www.facebook.com/nytopinion}{\emph{Facebook}}\emph{,}
\href{http://twitter.com/NYTOpinion}{\emph{Twitter (@NYTopinion)}}
\emph{and}
\href{https://www.instagram.com/nytopinion/}{\emph{Instagram}}\emph{.}

Advertisement

\protect\hyperlink{after-bottom}{Continue reading the main story}

\hypertarget{site-index}{%
\subsection{Site Index}\label{site-index}}

\hypertarget{site-information-navigation}{%
\subsection{Site Information
Navigation}\label{site-information-navigation}}

\begin{itemize}
\tightlist
\item
  \href{https://help.nytimes.com/hc/en-us/articles/115014792127-Copyright-notice}{©~2020~The
  New York Times Company}
\end{itemize}

\begin{itemize}
\tightlist
\item
  \href{https://www.nytco.com/}{NYTCo}
\item
  \href{https://help.nytimes.com/hc/en-us/articles/115015385887-Contact-Us}{Contact
  Us}
\item
  \href{https://www.nytco.com/careers/}{Work with us}
\item
  \href{https://nytmediakit.com/}{Advertise}
\item
  \href{http://www.tbrandstudio.com/}{T Brand Studio}
\item
  \href{https://www.nytimes.com/privacy/cookie-policy\#how-do-i-manage-trackers}{Your
  Ad Choices}
\item
  \href{https://www.nytimes.com/privacy}{Privacy}
\item
  \href{https://help.nytimes.com/hc/en-us/articles/115014893428-Terms-of-service}{Terms
  of Service}
\item
  \href{https://help.nytimes.com/hc/en-us/articles/115014893968-Terms-of-sale}{Terms
  of Sale}
\item
  \href{https://spiderbites.nytimes.com}{Site Map}
\item
  \href{https://help.nytimes.com/hc/en-us}{Help}
\item
  \href{https://www.nytimes.com/subscription?campaignId=37WXW}{Subscriptions}
\end{itemize}
