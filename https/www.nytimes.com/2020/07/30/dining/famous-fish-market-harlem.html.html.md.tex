Sections

SEARCH

\protect\hyperlink{site-content}{Skip to
content}\protect\hyperlink{site-index}{Skip to site index}

\href{https://www.nytimes.com/section/food}{Food}

\href{https://myaccount.nytimes.com/auth/login?response_type=cookie\&client_id=vi}{}

\href{https://www.nytimes.com/section/todayspaper}{Today's Paper}

\href{/section/food}{Food}\textbar{}A Harlem Restaurant That's Withstood
Gentrification, a Pandemic and Time

\url{https://nyti.ms/33g3BT3}

\begin{itemize}
\item
\item
\item
\item
\item
\item
\end{itemize}

Advertisement

\protect\hyperlink{after-top}{Continue reading the main story}

Supported by

\protect\hyperlink{after-sponsor}{Continue reading the main story}

\hypertarget{a-harlem-restaurant-thats-withstood-gentrification-a-pandemic-and-time}{%
\section{A Harlem Restaurant That's Withstood Gentrification, a Pandemic
and
Time}\label{a-harlem-restaurant-thats-withstood-gentrification-a-pandemic-and-time}}

Long lines are still forming at Famous Fish Market, a Black-owned
business that's been in the same family for nearly 50 years.

\includegraphics{https://static01.nyt.com/images/2020/08/05/dining/05fishmarket1/05fishmarket1-articleLarge.jpg?quality=75\&auto=webp\&disable=upscale}

By Kayla Stewart

\begin{itemize}
\item
  July 30, 2020
\item
  \begin{itemize}
  \item
  \item
  \item
  \item
  \item
  \item
  \end{itemize}
\end{itemize}

At 145th Street and St. Nicholas Avenue in Harlem, a line begins outside
a modest storefront near the subway station and extends down the avenue
for about 10 hours straight.

The line is overwhelmingly filled with Black people, elderly and young,
parents and children, singles and couples, who come for the fried
whiting, shrimp and clams --- as customers have been doing for 46 years.

\href{http://www.famousfishmarket.us/}{Famous Fish Market} has been a
staple of its historic neighborhood since 1974, and today the
restaurant's family- and Black-owned roots are rare in a place that has
been deeply affected over the years by poverty, gentrification and, most
recently, the pandemic, as well as the Black Lives Matter protests.

But customers still show up, masks on, phones out, ready for what
sometimes can be an hourlong wait for fresh seafood and French fries.
Nearby fish markets have tried to achieve the popularity of Famous Fish
Market, but have yet to draw the same long lines.

The owner, Sterling Eric Strickland, has kept his family's business
going even through difficult circumstances for 22 years, working
alongside his wife, Viola, and his daughter, Erica.

``I treat my customers like I like to be treated, and so far that method
has worked for us,'' said Mr. Strickland, who goes by Eric.

The story of Famous Fish Market begins with Eloise Cherry, Mr.
Strickland's aunt. In the 1950s, Ms. Cherry, who is now 87, moved to New
York from Mount Airy, N.C., with her husband, Al. She initially worked
as a beautician, and he worked as a barber.

Mr. Cherry, who had been a cook in the U.S. Navy, taught her his
grandmother's recipe for what is now the beloved ---~and
secret~---~seasoning for the restaurant's batter. Ms. Cherry knew the
recipe was good, and she had the foresight to try building a business on
it, opening the fish market in the same space it occupies now. She
turned the business over to Mr. Strickland in 1998.

\includegraphics{https://static01.nyt.com/images/2020/08/05/dining/05fishmarket4/05fishmarket4-articleLarge.jpg?quality=75\&auto=webp\&disable=upscale}

The way Famous Fish Market operates has shifted because of Covid-19. The
restaurant has cut its hours, and only one customer is allowed inside at
a time, as opposed to the numerous people who used to pack in. And there
have been challenges. Like many
\href{https://www.nytimes.com/2020/05/18/business/minority-businesses-coronavirus-loans.html}{Black-owned
businesses}, this one was twice denied a Paycheck Protection Program
loan before it received a bank loan to sustain the business through the
crisis.

The batter recipe has been adapted, with one version used for the shrimp
and one for the whiting, two of the most popular options on the menu.
The kitchen is small, and Mr. Strickland insists on keeping the stoves
clean and changing the cooking oil every day. ``People can taste that
freshness,'' he said.

Orders are prepared in front of each customer, so patrons can watch
their shrimp, clams or whiting sizzle in that fresh oil while fries cook
alongside. Servers quickly place the food in a brown serving basket, and
offer tartar sauce, hot sauce and two slices of warm bread, while
customers bathe their seafood in a layer of salt-and-pepper seasoning
mix.

Once they are back at home, or standing just outside the restaurant if
desire gets to them first, customers get to enjoy the comforts of a fish
fry in the middle of the city.

It's a ritual with religious connections: Some Christians eat fish on
Fridays, forgoing meat. Mr. Strickland said Fridays are often busy at
the market.

``I'm a country boy, and I'm a country Black boy,'' said Mr. Strickland,
who is also from Mount Airy. ``I know what good food is.''

Image

The fish on the menu is fried whiting, made using a secret recipe from
Mr. Strickland's family.Credit...Lelanie Foster for The New York Times

Image

Fried shrimp is another house specialty.Credit...Lelanie Foster for The
New York Times

But it's more than good --- it's rich, indulgent and historic. In
essence, it's Black. Because it's packaged to go in a plastic bag, it
can be easy to forget that this food is not just regular takeout, but
something far more personal: a generational and carefully kept family
recipe.

The whiting is flaky and fresh; the shrimp is plump and tenderly fried;
the clams are crispy and comforting. (Mr. Strickland says it's all about
keeping the batter fluffy.) The fries, often given little attention at
other places, are crisp on the outside, with a soft interior. They break
apart gracefully at every bite.

Amber Jarvis arrived 30 minutes before the market opened on a recent
rainy Tuesday so she could be first in line.

``I've been coming here for two years,'' she said. ``My brother, who
told me about it, comes in from Queens, and I come in from Brooklyn. You
can't find prime whiting in too many places, and the whole combo is just
delicious.''

Image

Fish orders come with two pieces of bread, which customers often use to
make a sandwich.Credit...Lelanie Foster for The New York Times

Stewart Green takes the train in from Brooklyn regularly, a ride that is
often an hour on the subway or longer, to grab a whiting fish sandwich.

``I like to support my own,'' he said of Famous Fish's Black ownership,
which is unusual in the restaurant business. ``And you just can't go
wrong with a fish sandwich. I grew up in Harlem, and have always enjoyed
the freshness of the food. When I want to treat myself, I come here.''

Keeping the business going in an ever-changing New York hasn't always
been easy, especially given Manhattan's rising costs, Mr. Strickland
said. He has two stoves in the kitchen; he is hoping to find the time
and money to install a third.

Mr. Strickland, who just celebrated his 69th birthday, is navigating the
health problems that often come with aging. For about 10 years, he and
his wife have split their time between North Carolina and Harlem.

Mr. Strickland says he knows that eventually he'll need to turn the
business over to his daughter --- but he is not quite ready for that.
Until then, he and his family can be found behind the counter in Harlem.

``I take pride in what I do, and the neighborhood appreciates it,'' he
said. ``People know us, and they remember us. It's been good to see
during this time.''

Famous Fish Market, 684 St Nicholas Avenue, 212-491-8323;
\href{http://www.famousfishmarket.us/}{famousfishmarket.us}

\emph{Follow} \href{https://twitter.com/nytfood}{\emph{NYT Food on
Twitter}} \emph{and}
\href{https://www.instagram.com/nytcooking/}{\emph{NYT Cooking on
Instagram}}\emph{,}
\href{https://www.facebook.com/nytcooking/}{\emph{Facebook}}\emph{,}
\href{https://www.youtube.com/nytcooking}{\emph{YouTube}} \emph{and}
\href{https://www.pinterest.com/nytcooking/}{\emph{Pinterest}}\emph{.}
\href{https://www.nytimes.com/newsletters/cooking}{\emph{Get regular
updates from NYT Cooking, with recipe suggestions, cooking tips and
shopping advice}}\emph{.}

Advertisement

\protect\hyperlink{after-bottom}{Continue reading the main story}

\hypertarget{site-index}{%
\subsection{Site Index}\label{site-index}}

\hypertarget{site-information-navigation}{%
\subsection{Site Information
Navigation}\label{site-information-navigation}}

\begin{itemize}
\tightlist
\item
  \href{https://help.nytimes.com/hc/en-us/articles/115014792127-Copyright-notice}{©~2020~The
  New York Times Company}
\end{itemize}

\begin{itemize}
\tightlist
\item
  \href{https://www.nytco.com/}{NYTCo}
\item
  \href{https://help.nytimes.com/hc/en-us/articles/115015385887-Contact-Us}{Contact
  Us}
\item
  \href{https://www.nytco.com/careers/}{Work with us}
\item
  \href{https://nytmediakit.com/}{Advertise}
\item
  \href{http://www.tbrandstudio.com/}{T Brand Studio}
\item
  \href{https://www.nytimes.com/privacy/cookie-policy\#how-do-i-manage-trackers}{Your
  Ad Choices}
\item
  \href{https://www.nytimes.com/privacy}{Privacy}
\item
  \href{https://help.nytimes.com/hc/en-us/articles/115014893428-Terms-of-service}{Terms
  of Service}
\item
  \href{https://help.nytimes.com/hc/en-us/articles/115014893968-Terms-of-sale}{Terms
  of Sale}
\item
  \href{https://spiderbites.nytimes.com}{Site Map}
\item
  \href{https://help.nytimes.com/hc/en-us}{Help}
\item
  \href{https://www.nytimes.com/subscription?campaignId=37WXW}{Subscriptions}
\end{itemize}
