Sections

SEARCH

\protect\hyperlink{site-content}{Skip to
content}\protect\hyperlink{site-index}{Skip to site index}

\href{https://www.nytimes.com/section/style/self-care/}{Self-Care}

\href{https://myaccount.nytimes.com/auth/login?response_type=cookie\&client_id=vi}{}

\href{https://www.nytimes.com/section/todayspaper}{Today's Paper}

\href{/section/style/self-care/}{Self-Care}\textbar{}Self-Care for Black
Journalists

\url{https://nyti.ms/3frJl47}

\begin{itemize}
\item
\item
\item
\item
\item
\item
\end{itemize}

\href{https://www.nytimes.com/news-event/george-floyd-protests-minneapolis-new-york-los-angeles?action=click\&pgtype=Article\&state=default\&region=TOP_BANNER\&context=storylines_menu}{Race
and America}

\begin{itemize}
\tightlist
\item
  \href{https://www.nytimes.com/interactive/2020/07/03/us/george-floyd-protests-crowd-size.html?action=click\&pgtype=Article\&state=default\&region=TOP_BANNER\&context=storylines_menu}{Black
  Lives Matter Movement}
\item
  \href{https://www.nytimes.com/interactive/2020/06/28/us/i-cant-breathe-police-arrest.html?action=click\&pgtype=Article\&state=default\&region=TOP_BANNER\&context=storylines_menu}{History
  of `I Can't Breathe'}
\item
  \href{https://www.nytimes.com/interactive/2020/06/10/upshot/black-lives-matter-attitudes.html?action=click\&pgtype=Article\&state=default\&region=TOP_BANNER\&context=storylines_menu}{How
  Public Opinion Shifted}
\item
  \href{https://www.nytimes.com/interactive/2020/07/16/us/black-lives-matter-protests-louisville-breonna-taylor.html?action=click\&pgtype=Article\&state=default\&region=TOP_BANNER\&context=storylines_menu}{45
  Days in Louisville}
\end{itemize}

Advertisement

\protect\hyperlink{after-top}{Continue reading the main story}

Supported by

\protect\hyperlink{after-sponsor}{Continue reading the main story}

\hypertarget{self-care-for-black-journalists}{%
\section{Self-Care for Black
Journalists}\label{self-care-for-black-journalists}}

In a news cycle filled with tragedy, much of it disproportionately
affecting people of color, Black reporters and editors are reimagining
coping strategies.

\includegraphics{https://static01.nyt.com/images/2020/07/09/fashion/09BLACK-JOURNALISTS-Natelege-Whaley/merlin_174323043_57123aa6-e6ba-46e0-9178-6e1334e3e7b4-articleLarge.jpg?quality=75\&auto=webp\&disable=upscale}

By Patrice Peck

\begin{itemize}
\item
  Published July 14, 2020Updated July 15, 2020
\item
  \begin{itemize}
  \item
  \item
  \item
  \item
  \item
  \item
  \end{itemize}
\end{itemize}

Heat flushed Natelegé Whaley's body as she wrote a news article about
the shooting by police that
\href{https://www.nytimes.com/article/breonna-taylor-police.html}{killed
Breonna Taylor}. Ms. Whaley, a journalist, figured she was tired. Then
came the mental fog, digestive issues and blurred vision. When these
seemingly separate issues snowballed into a panic attack and a trip to
the emergency room in late May, Ms. Whaley, 31, who lives in Brooklyn,
connected the dots.

``I'm writing about the suffering of someone who looks like me,'' she
said. ``We just keep going and going and going and going because we feel
like that's what we must do, and that's not healthy.''

The news today is filled with grief, especially for Black journalists
reporting on violence against Black people, socioeconomic disparities
underscored by the coronavirus pandemic and racism in the workplace. The
situation is complicated by the fact that often they are doing this work
for publications where most of the staff is white.

``Black journalists, like nurses or psychotherapists or anyone else who
regularly hears or views trauma narratives, may experience vicarious
trauma, or distress that stems from repeated exposure to the trauma of
others,'' said Robin D. Stone, a licensed mental health counselor
specializing in trauma-informed treatment. ``They may feel especially
vulnerable that the person on the respirator or in the violent video
could be them or someone they love.'' (Ms. Stone knows the world of
reporting intimately: For more than 20 years, she was a journalist,
including a stint at The New York Times.)

The conditions can be particularly challenging for freelancers, who
cannot rely on a biweekly paycheck or corporate health insurance, Ms.
Whaley said.

``Even though you're getting paid, it really puts us in a vulnerable
position while the company profits off the work that requires immense
emotional and mental labor,'' she said. ``Yes, the stories need to be
told. But no one is really thinking about whether Black freelancers have
the resources they need to stay sane during this time.''

Image

Julia Craven, a reporter for Slate.

\hypertarget{addressing-trauma}{%
\subsection{Addressing Trauma}\label{addressing-trauma}}

Black Americans are underrepresented in American newsrooms; a Pew
Research Center survey of data from 2013 to 2017 found that only 7
percent of newsroom employees are Black. (At The New York Times, 9
percent of newsroom employees are Black.) Often Black journalists are
called upon to report and write specifically about issues within their
own community, which may involve viewing imagery that depicts violence,
hatred and death.

Many of them have
\href{https://www.huffpost.com/entry/black-journalists-media-reckoning-coronavirus-protests_n_5f0886d5c5b67a80bc06c683?guccounter=1\&guce_referrer=aHR0cHM6Ly90LmNvL1llVXE1a2dZU3A_YW1wPTE\&guce_referrer_sig=AQAAAI-nT8EfHpllwL0NGE4Zp1KVkb-NaJhDUuKhopPe-9QqpGJCUC1plFEkJU9HV1twl47KO7XZ4lFAMlimSMiCkj4VykYT1hwmBCiQrAX0C6YK8rthWkoOvToTXia8FxjwdzPSs6aeGEKhtO-ABcsLUrgk7hjEdw774SNGdW_ax_ba}{begun
speaking out}about the importance of prioritizing mental health care and
wellness. In the absence of employer-sponsored insurance and mental
health services for freelancers, and in light of recent discussions on
workplace burnout, many Black journalists are rethinking the work that
is required to report on atrocities in Black communities.

``I feel like we're still figuring it out,'' Ms. Whaley said. ``We're
just starting to have these open conversations about mental health
because in the past Black journalists were just supposed to be happy
just to be in this space, especially if you have a job at a major
publication. It's like everyone thinks that you've made it.''

Inundated with a nonstop stream of race-related news, today's Black
journalists are adopting a mix of traditional and informal practices to
better care for and protect their own mental health and wellness.

``All Black Americans have some degree of PTSD,'' said Dr. Monnica
Williams, a clinical psychologist and expert in race-based stress and
trauma. In the case of Black journalists, Dr. Williams referred to
studies of being ``repeatedly exposed to details of traumatic
experiences in your line of work.''

``Being a journalist is not any different because you're being
constantly exposed to these gruesome details of horrific instances of
racism,'' she said. ``So it's just the same.''

To better manage on a day-to-day basis, Dr. Williams recommended a
``toolbox of coping strategies'' that includes seeking social support
within one's communities, briefly limiting one's exposure to cues of
racism, engaging with religious or spiritual practices, seeking
distraction from cues of racism, and participating in restful and
relaxing activities.

And how does one determine if and when they should take a break? Dr.
Williams pointed to several examples of racial stress and trauma
interfering with one's daily functions, like being depressed or anxious
for most of the day, or having trouble sleeping.

``When you see it affecting your quality of life, that's a pretty good
sign you should write stories about puppies or something,'' she said.

\includegraphics{https://static01.nyt.com/images/2020/07/09/fashion/09BLACK-JOURNALISTS-Clydeen-McDonald/merlin_174323028_6169af4e-dd82-4271-9aaf-0c6e9761a7e2-articleLarge.jpg?quality=75\&auto=webp\&disable=upscale}

But sometimes taking a break from writing means cutting off one's main
source of income, especially without the support of paid sick days or
paid time off. Ms. Whaley proposed offering Black journalists
fully-funded sabbaticals every to rest, recover and reset: ``That would
be a real reparation to me because we need it.''

``I want all Black journalists to know you deserve so much better,'' she
continued. ``And I deserved better than what I gave myself and what this
industry has given me.''

\hypertarget{coping-mechanisms}{%
\subsection{Coping Mechanisms}\label{coping-mechanisms}}

Julia Craven, 27, a reporter for Slate in Washington, D.C., has been
reporting exclusively on racism since graduating from the University of
North Carolina at Chapel Hill in 2014. Two years into her career, the
news cycle flooded with reports of hate crimes and white supremacist
ideologies, fueled in part by the 2016 presidential election. There were
also numerous stories of Black individuals who had been killed in police
custody. Ms. Craven felt she could barely tread water. In each killed
person, she would catch a glimpse of her loved ones: her brother, her
boyfriend, her best friend, her sister and sometimes even herself.

``Everything seemed like it was constantly happening, so I went back
into therapy,'' Ms. Craven said. ``I knew that I needed to develop some
sort of self care system because if my mental health ain't on point,
then I can't do my job.''

More recently, at her therapist's suggestion, Ms. Craven has made a
concerted effort to limit her exposure to the news on the weekends. The
move has given her the space and time to focus on herself on days off
and be more present when she is at work, particularly at a demanding
time.

Dr. Williams said she frequently advises her Black clients, friends and
even acquaintances to unplug from social media to recover from stress
and recommends they not watch videos of Black people being harmed. ``I
don't think journalists need to see these videos unless your job is to
write a detailed account of how the person died, second by second,'' she
said.

But therapy and an escape from the news is a luxury to many, especially
uninsured and freelance journalists, like Ms. Whaley. She said she tried
seeing a therapist, but her funds were limited given her uneven
employment and the cost of an in-network therapist through her health
insurance.

``I couldn't afford the therapy because I'm a freelancer and not a
full-time staff writer with benefits,'' she said. ``But then I need to
go to therapy to help cope with my freelance career.''

Image

Nsikan Akpan, a science editor at National Geographic.

After Ms. Whaley's panic attack, though, she pulled back from an
assignment for her own well-being, and an editor sent her a link to the
\href{https://www.gofundme.com/f/black-journalists-therapy-relief-fund}{Black
Journalists Therapy Relief Fund}.

Sonia Weiser, 28, a white freelance writer based in Manhattan, started
the relief fund through a GoFundMe page after witnessing an outpouring
of
\href{https://twitter.com/weischoice/status/1270755036294000640}{calls}
for Black writers to cover racial violence, as well as the protests
galvanized by the killing of George Floyd, often for relatively little
compensation.

``It just felt rude and disrespectful to put the onus on Black
journalists, especially when so much of the trauma incurred in the
industry is because of white employers,'' she said.

After she created the fund, people donated to meet the \$20,000 goal and
raised \$32,000 within 48 hours. Ms. Weiser has since raised over
\$70,000, and has
\href{https://iwmf.submittable.com/submit/25d0d67f-9c98-4813-9259-7d80bca55195/joint-application-form-for-iwmf-u-s-journalism-emergency-fund-and-black-journali}{partnered}
with the International Women's Media Foundation for additional support.
They have provided microgrants to 84 applicants (the majority of whom
don't have health insurance that covers mental health expenses),
matching nearly every person's desired amount up to \$2,000.

As one of the fund recipients, Ms. Whaley has received enough money to
see a therapist --- a Black woman, which was Ms. Whaley's preference ---
twice a week for the next four to six months. (She sought more
affordable psychotherapy sessions through Open Path Collective, a
nonprofit organization providing affordable, in-office and online
psychotherapy services ranging from \$30 to \$80 per session.)

``I was able to take a deep breath after that,'' Ms. Whaley said.

While Ms. Stone recommends a therapist if trauma or stress-related
symptoms are interfering with a person's work or home life, she said she
also encourages Black journalists to cultivate a world outside of work
and to seek support through communities of peers with whom they can
share their experiences and find common ground and validation.

Clydeen McDonald, 33, a freelance journalist from Trinidad and Tobago,
said he had been despondent over his work, especially after two
high-profile, historically white-staffed national publications passed on
his pitches about coronavirus-related news in the Caribbean region.

``Sometimes I find myself thinking, `Did I get rejected because it was
not the right pitch or not professional enough?''' he said.

As a journalist, Mr. McDonald said he feels pressure to make sure people
from his home country and other Caribbean nations see themselves in
timely, in-depth news beyond hurricane coverage. Otherwise, he has not
done his job, he said.

To alleviate stress, Mr. McDonald, who has been living most recently in
Ho Chi Minh City, Vietnam, has weekly phone conversations with his
mother and younger sister --- or as Dr. Williams put it, he has found
social support within his own community of friends and family.

Nsikan Akpan, 34, a science editor at National Geographic in Washington,
D.C., unwinds by going on physically demanding bike rides and speaking
with his fiancé, his friends from college and other loved ones, as well
as self-prescribing a sort of musical therapy that involves listening to
throwback Kanye West albums for the sake of nostalgia.

Image

Charlie Brinkhurt-Cuff, the head of editorial at gal-dem magazine, a
British publication.

``Something about some of the tracks on his early albums just really
speak to me,'' Mr. Akpan said. The song ``Hey Mama,'' a tribute to Mr.
West's mother from his 2005 album ``Late Registration,'' is one he has
listened to repeatedly. ``I think that really stuck out to me,
especially because George Floyd was calling out for his mom at the end,
and my dad died last year so my mom and I got closer through that,'' he
said. ``I've definitely been thinking about her a lot.''

Charlie Brinkhurst-Cuff, 27, the head of editorial at
\href{https://gal-dem.com/}{gal-dem magazine}, a British publication
that centers perspectives of women and nonbinary people of color, said
for a while, she chose to prioritize her work ``ahead of personal
concerns.'' She noticed her stress levels were at an all-time high,
underscored by a monthlong eye twitch. She felt anguish about her
decision, recently, to take a week off.

``We've been doing this work, this anti-racist kind of reporting, for
years now,'' she said. ``I've never seen this level of interest in what
we do and how we do it. It's been intense. It's been very draining. This
period of time and this increased level of interest won't last, and I
kind of want to make the most out of it while people care.''

Advertisement

\protect\hyperlink{after-bottom}{Continue reading the main story}

\hypertarget{site-index}{%
\subsection{Site Index}\label{site-index}}

\hypertarget{site-information-navigation}{%
\subsection{Site Information
Navigation}\label{site-information-navigation}}

\begin{itemize}
\tightlist
\item
  \href{https://help.nytimes.com/hc/en-us/articles/115014792127-Copyright-notice}{©~2020~The
  New York Times Company}
\end{itemize}

\begin{itemize}
\tightlist
\item
  \href{https://www.nytco.com/}{NYTCo}
\item
  \href{https://help.nytimes.com/hc/en-us/articles/115015385887-Contact-Us}{Contact
  Us}
\item
  \href{https://www.nytco.com/careers/}{Work with us}
\item
  \href{https://nytmediakit.com/}{Advertise}
\item
  \href{http://www.tbrandstudio.com/}{T Brand Studio}
\item
  \href{https://www.nytimes.com/privacy/cookie-policy\#how-do-i-manage-trackers}{Your
  Ad Choices}
\item
  \href{https://www.nytimes.com/privacy}{Privacy}
\item
  \href{https://help.nytimes.com/hc/en-us/articles/115014893428-Terms-of-service}{Terms
  of Service}
\item
  \href{https://help.nytimes.com/hc/en-us/articles/115014893968-Terms-of-sale}{Terms
  of Sale}
\item
  \href{https://spiderbites.nytimes.com}{Site Map}
\item
  \href{https://help.nytimes.com/hc/en-us}{Help}
\item
  \href{https://www.nytimes.com/subscription?campaignId=37WXW}{Subscriptions}
\end{itemize}
