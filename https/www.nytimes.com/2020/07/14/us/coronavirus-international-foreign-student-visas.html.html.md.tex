Sections

SEARCH

\protect\hyperlink{site-content}{Skip to
content}\protect\hyperlink{site-index}{Skip to site index}

\href{https://www.nytimes.com/section/us}{U.S.}

\href{https://myaccount.nytimes.com/auth/login?response_type=cookie\&client_id=vi}{}

\href{https://www.nytimes.com/section/todayspaper}{Today's Paper}

\href{/section/us}{U.S.}\textbar{}U.S. Rescinds Plan to Strip Visas From
International Students in Online Classes

\url{https://nyti.ms/390knpR}

\begin{itemize}
\item
\item
\item
\item
\item
\end{itemize}

\href{https://www.nytimes.com/news-event/coronavirus?action=click\&pgtype=Article\&state=default\&region=TOP_BANNER\&context=storylines_menu}{The
Coronavirus Outbreak}

\begin{itemize}
\tightlist
\item
  live\href{https://www.nytimes.com/2020/08/02/world/coronavirus-updates.html?action=click\&pgtype=Article\&state=default\&region=TOP_BANNER\&context=storylines_menu}{Latest
  Updates}
\item
  \href{https://www.nytimes.com/interactive/2020/us/coronavirus-us-cases.html?action=click\&pgtype=Article\&state=default\&region=TOP_BANNER\&context=storylines_menu}{Maps
  and Cases}
\item
  \href{https://www.nytimes.com/interactive/2020/science/coronavirus-vaccine-tracker.html?action=click\&pgtype=Article\&state=default\&region=TOP_BANNER\&context=storylines_menu}{Vaccine
  Tracker}
\item
  \href{https://www.nytimes.com/interactive/2020/07/29/us/schools-reopening-coronavirus.html?action=click\&pgtype=Article\&state=default\&region=TOP_BANNER\&context=storylines_menu}{What
  School May Look Like}
\item
  \href{https://www.nytimes.com/live/2020/07/31/business/stock-market-today-coronavirus?action=click\&pgtype=Article\&state=default\&region=TOP_BANNER\&context=storylines_menu}{Economy}
\end{itemize}

Advertisement

\protect\hyperlink{after-top}{Continue reading the main story}

Supported by

\protect\hyperlink{after-sponsor}{Continue reading the main story}

\hypertarget{us-rescinds-plan-to-strip-visas-from-international-students-in-online-classes}{%
\section{U.S. Rescinds Plan to Strip Visas From International Students
in Online
Classes}\label{us-rescinds-plan-to-strip-visas-from-international-students-in-online-classes}}

The Trump administration said it would no longer require foreign
students to attend in-person classes during the coronavirus pandemic in
order to remain in the country.

\includegraphics{https://static01.nyt.com/images/2020/07/14/us/14virus-studentvisas/merlin_174372168_1fa304c3-de76-4e2c-9429-dfe78651a38a-articleLarge.jpg?quality=75\&auto=webp\&disable=upscale}

\href{https://www.nytimes.com/by/miriam-jordan/}{\includegraphics{https://static01.nyt.com/images/2018/02/16/multimedia/author-miriam-jordan/author-miriam-jordan-thumbLarge-v2.png}}\href{https://www.nytimes.com/by/anemona-hartocollis}{\includegraphics{https://static01.nyt.com/images/2018/06/13/multimedia/author-anemona-hartocollis/author-anemona-hartocollis-thumbLarge-v3.jpg}}

By \href{https://www.nytimes.com/by/miriam-jordan/}{Miriam Jordan} and
\href{https://www.nytimes.com/by/anemona-hartocollis}{Anemona
Hartocollis}

\begin{itemize}
\item
  Published July 14, 2020Updated July 16, 2020
\item
  \begin{itemize}
  \item
  \item
  \item
  \item
  \item
  \end{itemize}
\end{itemize}

In a rare and swift immigration policy reversal, the Trump
administration on Tuesday bowed to snowballing opposition from
universities, Silicon Valley and 20 states and abandoned a plan to strip
international college students of their visas if they did not attend at
least some classes in person.

The policy, which would have subjected foreign students to deportation
if they did not show up for class on campus, had thrown the higher
education world into turmoil at a time when universities are grappling
with whether to reopen campuses during the coronavirus pandemic.

The loss of international students could have cost universities millions
of dollars in tuition and jeopardized the ability of U.S. companies to
hire the highly skilled workers who often start their careers with an
American education.

Two days after the policy was announced on July 6, Harvard and the
Massachusetts Institute of Technology filed the first of a litany of
lawsuits seeking to block it.

On Tuesday, minutes before a federal judge in Boston was to hear
arguments on their challenge, the judge, Allison D. Burroughs, announced
that the administration had agreed to rescind the policy and allow
international students to remain in the country even if they are taking
all their classes online.

The government has argued that the requirement that students take at
least one in-person class was actually more lenient than the rule that
had been in effect for close to 20 years which required foreign students
to take most of their classes in person.

But that rule was temporarily suspended on March 13, when Mr. Trump
declared a national emergency and campuses across the country began
shutting down, with classes moving online. On July 6, the government
made its announcement that foreign students could not remain in the
United States if their studies were entirely online.

``If they're not going to be a student or they're going to be 100
percent online, then they don't have a basis to be here,'' Kenneth T.
Cuccinelli II, the acting deputy secretary of the Department of Homeland
Security, said in an interview on CNN after the policy was announced.
``They should go home, and then they can return when the school opens.''

Each year, about 1 million international students enroll in American
universities. They contribute \$41 billion to the economy annually and
support more than 458,000 jobs.

\hypertarget{latest-updates-global-coronavirus-outbreak}{%
\section{\texorpdfstring{\href{https://www.nytimes.com/2020/08/01/world/coronavirus-covid-19.html?action=click\&pgtype=Article\&state=default\&region=MAIN_CONTENT_1\&context=storylines_live_updates}{Latest
Updates: Global Coronavirus
Outbreak}}{Latest Updates: Global Coronavirus Outbreak}}\label{latest-updates-global-coronavirus-outbreak}}

Updated 2020-08-02T17:52:35.962Z

\begin{itemize}
\tightlist
\item
  \href{https://www.nytimes.com/2020/08/01/world/coronavirus-covid-19.html?action=click\&pgtype=Article\&state=default\&region=MAIN_CONTENT_1\&context=storylines_live_updates\#link-34047410}{The
  U.S. reels as July cases more than double the total of any other
  month.}
\item
  \href{https://www.nytimes.com/2020/08/01/world/coronavirus-covid-19.html?action=click\&pgtype=Article\&state=default\&region=MAIN_CONTENT_1\&context=storylines_live_updates\#link-780ec966}{Top
  U.S. officials work to break an impasse over the federal jobless
  benefit.}
\item
  \href{https://www.nytimes.com/2020/08/01/world/coronavirus-covid-19.html?action=click\&pgtype=Article\&state=default\&region=MAIN_CONTENT_1\&context=storylines_live_updates\#link-2bc8948}{Its
  outbreak untamed, Melbourne goes into even greater lockdown.}
\end{itemize}

\href{https://www.nytimes.com/2020/08/01/world/coronavirus-covid-19.html?action=click\&pgtype=Article\&state=default\&region=MAIN_CONTENT_1\&context=storylines_live_updates}{See
more updates}

More live coverage:
\href{https://www.nytimes.com/live/2020/07/31/business/stock-market-today-coronavirus?action=click\&pgtype=Article\&state=default\&region=MAIN_CONTENT_1\&context=storylines_live_updates}{Markets}

In addition to Harvard and M.I.T., the attorneys general of 20 states,
including Massachusetts and California, also sued, charging that the
policy was reckless, cruel and senseless. Scores of universities threw
their support behind the litigation, along with organizations
representing international students.

The pressure grew on Monday, when more than a dozen technology
companies, including Google, Facebook and Twitter, also came out in
support of the lawsuit, arguing that the policy would harm their
businesses. Then on Tuesday, 15 Republican members of Congress signed a
letter urging the Trump administration to restore its previous policy on
international students.

Representative Rodney Davis, an Illinois Republican who had organized
the letter, applauded the Trump administration's ``right decision'' to
cancel its plan. ``These hardworking students are the best and brightest
from their countries, and they help our communities grow both culturally
and economically,'' he said.

The policy about foreign students was one of a number of measures
President Trump has taken to advance his agenda on immigration, using
the coronavirus pandemic and the need to
\href{https://www.nytimes.com/2020/05/03/us/coronavirus-immigration-stephen-miller-public-health.html}{protect
the country from health threats} as justification.

The administration has also stopped processing green cards for
applicants abroad; closed the southwestern border to nonessential
travel, allowing only Americans and legal permanent residents to enter
while barring asylum seekers; and banned the entry of thousands of
foreigners on work visas.

The universities said in their court challenge to the latest policy that
``by all appearances,'' the government's attempt to force international
students to study on campus had also been a political move, calculated
to advance the Trump administration's agenda to force universities to
reopen their gates with in-person classes.

They said forcing students to return to their home countries would in
many cases separate them from families in the United States, returning
them to places where they no longer had a home. In some cases, they
would be living under the thumb of repressive regimes, which policed or
restricted internet access.

Many international students from Asian countries would have to contend
with a time difference that would mean taking classes between 2 a.m. and
7 a.m., if they had internet access.

Foreign students attending elementary, middle and high schools on visas
would have also had to depart the country if their classes went 100
percent online.

In court filings, universities said that some arriving students already
had been barred from entering the country by immigration officials at
airports who told them that their institutions were going online.

After the announcement on Tuesday that the policy had been rescinded,
university officials praised the decision and warned that they would be
prepared to go back to court should the administration make any further
moves to restrict the ability of international students to study online
when necessary.

``This is a significant victory,'' Harvard's president, Lawrence S.
Bacow, said in a statement. ``The directive had disrupted all of
American higher education. I have heard from countless international
students who said that the July 6 directive had put them at serious
risk. These students --- our students --- can now rest easier and focus
on their education, which is all they ever wanted to do.''

L. Rafael Reif, president of M.I.T., said the swift opposition was
evidence of ``the important role international students play in our
education, research and innovation enterprises here in the United
States. These students make us stronger, and we hurt ourselves when we
alienate them.''

The University of Southern California, among the universities with the
largest population of international students, said in a statement that
it was ``thrilled that the government backed down and rescinded its rule
that would have revoked visas for international students.''

\href{https://www.nytimes.com/news-event/coronavirus?action=click\&pgtype=Article\&state=default\&region=MAIN_CONTENT_3\&context=storylines_faq}{}

\hypertarget{the-coronavirus-outbreak-}{%
\subsubsection{The Coronavirus Outbreak
›}\label{the-coronavirus-outbreak-}}

\hypertarget{frequently-asked-questions}{%
\paragraph{Frequently Asked
Questions}\label{frequently-asked-questions}}

Updated July 27, 2020

\begin{itemize}
\item ~
  \hypertarget{should-i-refinance-my-mortgage}{%
  \paragraph{Should I refinance my
  mortgage?}\label{should-i-refinance-my-mortgage}}

  \begin{itemize}
  \tightlist
  \item
    \href{https://www.nytimes.com/article/coronavirus-money-unemployment.html?action=click\&pgtype=Article\&state=default\&region=MAIN_CONTENT_3\&context=storylines_faq}{It
    could be a good idea,} because mortgage rates have
    \href{https://www.nytimes.com/2020/07/16/business/mortgage-rates-below-3-percent.html?action=click\&pgtype=Article\&state=default\&region=MAIN_CONTENT_3\&context=storylines_faq}{never
    been lower.} Refinancing requests have pushed mortgage applications
    to some of the highest levels since 2008, so be prepared to get in
    line. But defaults are also up, so if you're thinking about buying a
    home, be aware that some lenders have tightened their standards.
  \end{itemize}
\item ~
  \hypertarget{what-is-school-going-to-look-like-in-september}{%
  \paragraph{What is school going to look like in
  September?}\label{what-is-school-going-to-look-like-in-september}}

  \begin{itemize}
  \tightlist
  \item
    It is unlikely that many schools will return to a normal schedule
    this fall, requiring the grind of
    \href{https://www.nytimes.com/2020/06/05/us/coronavirus-education-lost-learning.html?action=click\&pgtype=Article\&state=default\&region=MAIN_CONTENT_3\&context=storylines_faq}{online
    learning},
    \href{https://www.nytimes.com/2020/05/29/us/coronavirus-child-care-centers.html?action=click\&pgtype=Article\&state=default\&region=MAIN_CONTENT_3\&context=storylines_faq}{makeshift
    child care} and
    \href{https://www.nytimes.com/2020/06/03/business/economy/coronavirus-working-women.html?action=click\&pgtype=Article\&state=default\&region=MAIN_CONTENT_3\&context=storylines_faq}{stunted
    workdays} to continue. California's two largest public school
    districts --- Los Angeles and San Diego --- said on July 13, that
    \href{https://www.nytimes.com/2020/07/13/us/lausd-san-diego-school-reopening.html?action=click\&pgtype=Article\&state=default\&region=MAIN_CONTENT_3\&context=storylines_faq}{instruction
    will be remote-only in the fall}, citing concerns that surging
    coronavirus infections in their areas pose too dire a risk for
    students and teachers. Together, the two districts enroll some
    825,000 students. They are the largest in the country so far to
    abandon plans for even a partial physical return to classrooms when
    they reopen in August. For other districts, the solution won't be an
    all-or-nothing approach.
    \href{https://bioethics.jhu.edu/research-and-outreach/projects/eschool-initiative/school-policy-tracker/}{Many
    systems}, including the nation's largest, New York City, are
    devising
    \href{https://www.nytimes.com/2020/06/26/us/coronavirus-schools-reopen-fall.html?action=click\&pgtype=Article\&state=default\&region=MAIN_CONTENT_3\&context=storylines_faq}{hybrid
    plans} that involve spending some days in classrooms and other days
    online. There's no national policy on this yet, so check with your
    municipal school system regularly to see what is happening in your
    community.
  \end{itemize}
\item ~
  \hypertarget{is-the-coronavirus-airborne}{%
  \paragraph{Is the coronavirus
  airborne?}\label{is-the-coronavirus-airborne}}

  \begin{itemize}
  \tightlist
  \item
    The coronavirus
    \href{https://www.nytimes.com/2020/07/04/health/239-experts-with-one-big-claim-the-coronavirus-is-airborne.html?action=click\&pgtype=Article\&state=default\&region=MAIN_CONTENT_3\&context=storylines_faq}{can
    stay aloft for hours in tiny droplets in stagnant air}, infecting
    people as they inhale, mounting scientific evidence suggests. This
    risk is highest in crowded indoor spaces with poor ventilation, and
    may help explain super-spreading events reported in meatpacking
    plants, churches and restaurants.
    \href{https://www.nytimes.com/2020/07/06/health/coronavirus-airborne-aerosols.html?action=click\&pgtype=Article\&state=default\&region=MAIN_CONTENT_3\&context=storylines_faq}{It's
    unclear how often the virus is spread} via these tiny droplets, or
    aerosols, compared with larger droplets that are expelled when a
    sick person coughs or sneezes, or transmitted through contact with
    contaminated surfaces, said Linsey Marr, an aerosol expert at
    Virginia Tech. Aerosols are released even when a person without
    symptoms exhales, talks or sings, according to Dr. Marr and more
    than 200 other experts, who
    \href{https://academic.oup.com/cid/article/doi/10.1093/cid/ciaa939/5867798}{have
    outlined the evidence in an open letter to the World Health
    Organization}.
  \end{itemize}
\item ~
  \hypertarget{what-are-the-symptoms-of-coronavirus}{%
  \paragraph{What are the symptoms of
  coronavirus?}\label{what-are-the-symptoms-of-coronavirus}}

  \begin{itemize}
  \tightlist
  \item
    Common symptoms
    \href{https://www.nytimes.com/article/symptoms-coronavirus.html?action=click\&pgtype=Article\&state=default\&region=MAIN_CONTENT_3\&context=storylines_faq}{include
    fever, a dry cough, fatigue and difficulty breathing or shortness of
    breath.} Some of these symptoms overlap with those of the flu,
    making detection difficult, but runny noses and stuffy sinuses are
    less common.
    \href{https://www.nytimes.com/2020/04/27/health/coronavirus-symptoms-cdc.html?action=click\&pgtype=Article\&state=default\&region=MAIN_CONTENT_3\&context=storylines_faq}{The
    C.D.C. has also} added chills, muscle pain, sore throat, headache
    and a new loss of the sense of taste or smell as symptoms to look
    out for. Most people fall ill five to seven days after exposure, but
    symptoms may appear in as few as two days or as many as 14 days.
  \end{itemize}
\item ~
  \hypertarget{does-asymptomatic-transmission-of-covid-19-happen}{%
  \paragraph{Does asymptomatic transmission of Covid-19
  happen?}\label{does-asymptomatic-transmission-of-covid-19-happen}}

  \begin{itemize}
  \tightlist
  \item
    So far, the evidence seems to show it does. A widely cited
    \href{https://www.nature.com/articles/s41591-020-0869-5}{paper}
    published in April suggests that people are most infectious about
    two days before the onset of coronavirus symptoms and estimated that
    44 percent of new infections were a result of transmission from
    people who were not yet showing symptoms. Recently, a top expert at
    the World Health Organization stated that transmission of the
    coronavirus by people who did not have symptoms was ``very rare,''
    \href{https://www.nytimes.com/2020/06/09/world/coronavirus-updates.html?action=click\&pgtype=Article\&state=default\&region=MAIN_CONTENT_3\&context=storylines_faq\#link-1f302e21}{but
    she later walked back that statement.}
  \end{itemize}
\end{itemize}

U.S.C., which led a coalition of 20 universities and colleges on the
West Coast that sued the government, said its international students
``deserve the right to continue their education without risk of
deportation.''

After the announcement of the now-rescinded policy this month,
international students had begun scrambling to figure out next steps.
Some said they booked flights home. Others held out hope that their
colleges might add classes with an in-person component to enable them to
remain in the United States.

\includegraphics{https://static01.nyt.com/images/2020/07/14/us/14virus-studentvisas02/14virus-studentvisas02-articleLarge.jpg?quality=75\&auto=webp\&disable=upscale}

Willow Cai, a rising junior at U.S.C. who is from China, said she had
rushed last week to register for a golf class to ensure she had an
in-person class. After hearing the administration had rescinded the new
policy, she expressed relief and outrage at the government's initiative.

``I'm like really emotional right now. The July 6 directive should never
have happened,'' said Ms. Cai, 20, a cinema major. She also said she
planned to drop the golf class.

``All we want is to continue our education in peace during a global
pandemic,'' she said on Tuesday in Los Angeles. ``It seems this
administration has no concern for international students beyond our
wallets.''

Alexander Auster, 22, a second-year student at George Washington
University Law School from Berlin, had already signed a lease on an
apartment for the next academic year when the policy was announced. He
had arrived in the United States when he was 17 to join the university's
varsity swim team.

``It feels really good now,'' he said after hearing that the government
had walked back its plan. ``Now I am hopeful and optimistic. I thought I
had to go home.''

An analysis of the government's earlier order by Moody's Analytics found
that it could have had a serious economic impact. According to NAFSA:
Association of International Educators, three jobs are created for every
seven international students. In the 2019-20 school year there were over
1 million international students enrolled in the United States, which
would translate to almost half a million jobs. ``ICE's policy would put
many of those positions at risk,'' Moody's Analytics said.

A third of all international students are from China, where travel
restrictions could make it hard for them to return, Moody's said.

College towns are heavily dependent on international students for their
economic well-being, and the Partnership for a New American Economy and
the American Enterprise Institute estimate that every H-1B visa, a
foreign worker visa, leads to two jobs being created.

Advertisement

\protect\hyperlink{after-bottom}{Continue reading the main story}

\hypertarget{site-index}{%
\subsection{Site Index}\label{site-index}}

\hypertarget{site-information-navigation}{%
\subsection{Site Information
Navigation}\label{site-information-navigation}}

\begin{itemize}
\tightlist
\item
  \href{https://help.nytimes.com/hc/en-us/articles/115014792127-Copyright-notice}{©~2020~The
  New York Times Company}
\end{itemize}

\begin{itemize}
\tightlist
\item
  \href{https://www.nytco.com/}{NYTCo}
\item
  \href{https://help.nytimes.com/hc/en-us/articles/115015385887-Contact-Us}{Contact
  Us}
\item
  \href{https://www.nytco.com/careers/}{Work with us}
\item
  \href{https://nytmediakit.com/}{Advertise}
\item
  \href{http://www.tbrandstudio.com/}{T Brand Studio}
\item
  \href{https://www.nytimes.com/privacy/cookie-policy\#how-do-i-manage-trackers}{Your
  Ad Choices}
\item
  \href{https://www.nytimes.com/privacy}{Privacy}
\item
  \href{https://help.nytimes.com/hc/en-us/articles/115014893428-Terms-of-service}{Terms
  of Service}
\item
  \href{https://help.nytimes.com/hc/en-us/articles/115014893968-Terms-of-sale}{Terms
  of Sale}
\item
  \href{https://spiderbites.nytimes.com}{Site Map}
\item
  \href{https://help.nytimes.com/hc/en-us}{Help}
\item
  \href{https://www.nytimes.com/subscription?campaignId=37WXW}{Subscriptions}
\end{itemize}
