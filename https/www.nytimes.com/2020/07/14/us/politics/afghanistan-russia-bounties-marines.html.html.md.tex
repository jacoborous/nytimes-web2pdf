Sections

SEARCH

\protect\hyperlink{site-content}{Skip to
content}\protect\hyperlink{site-index}{Skip to site index}

\href{https://www.nytimes.com/section/politics}{Politics}

\href{https://myaccount.nytimes.com/auth/login?response_type=cookie\&client_id=vi}{}

\href{https://www.nytimes.com/section/todayspaper}{Today's Paper}

\href{/section/politics}{Politics}\textbar{}Three Marines, Now Focus of
Russian Bounties Investigation, Show the Costs of an Endless War

\url{https://nyti.ms/2CzsFsR}

\begin{itemize}
\item
\item
\item
\item
\item
\end{itemize}

Advertisement

\protect\hyperlink{after-top}{Continue reading the main story}

Supported by

\protect\hyperlink{after-sponsor}{Continue reading the main story}

\hypertarget{three-marines-now-focus-of-russian-bounties-investigation-show-the-costs-of-an-endless-war}{%
\section{Three Marines, Now Focus of Russian Bounties Investigation,
Show the Costs of an Endless
War}\label{three-marines-now-focus-of-russian-bounties-investigation-show-the-costs-of-an-endless-war}}

The U.S. mission in Afghanistan is now described as training, advising
and assisting Afghan troops. But American forces are still patrolling
areas that are as deadly as they were in 2001.

\includegraphics{https://static01.nyt.com/images/2020/07/10/us/politics/00dc-marines1/00dc-marines1-articleLarge-v2.jpg?quality=75\&auto=webp\&disable=upscale}

By \href{https://www.nytimes.com/by/helene-cooper}{Helene Cooper},
\href{https://www.nytimes.com/by/jennifer-steinhauer}{Jennifer
Steinhauer},
\href{https://www.nytimes.com/by/thomas-gibbons-neff}{Thomas
Gibbons-Neff} and \href{https://www.nytimes.com/by/eric-schmitt}{Eric
Schmitt}

\begin{itemize}
\item
  Published July 14, 2020Updated July 29, 2020
\item
  \begin{itemize}
  \item
  \item
  \item
  \item
  \item
  \end{itemize}
\end{itemize}

The American military convoy was almost back to Bagram Air Base in
Afghanistan when a vehicle, laden with explosives, careened into it and
detonated.

The powerful blast blew up a heavily armored troop carrier, engulfing it
in flames. Marines poured out of the other vehicles in the convoy as
they battled desperately to save the occupants of the burning carrier,
including a Marine reservist --- a New York City firefighter --- who had
once rescued a woman from a burning high-rise apartment.

But all of the American ingenuity that had gone into armoring military
vehicles was not enough to stave off the horrendous damage caused by the
blast in April 2019. The firefighter, Staff Sgt. Christopher K.A.
Slutman, 43, did not beat the fire this time. Two other Marines, also
reservists, were also killed, casualties of a two-decade war that has
relentlessly continued to exact its toll on American troops.

Now those three Marines are at the center of the latest iteration of the
continuing saga of President Trump and Russia.

American intelligence agencies are investigating whether that car bomb
was detonated at the behest of a
\href{https://www.nytimes.com/2020/06/26/us/politics/russia-afghanistan-bounties.html?searchResultPosition=1}{Russian
military agency paying bounties to Afghan militia} groups for killing
American troops. Such a possibility, if true, would be a staggering
repudiation of Mr. Trump's yearslong embrace of President Vladimir V.
Putin of Russia. Thus far, there is no conclusive evidence linking the
deaths to any kind of Russian bounty.

Perhaps even more significant is that it has taken the debate over
possible
\href{https://www.nytimes.com/2020/07/29/us/politics/trump-putin-bounties.html}{Russian
bounties} to bring what happened to Sergeant Slutman and the two other
Marines --- Sgt. Robert A. Hendriks, 25, and Staff Sgt. Benjamin S.
Hines, 31 --- to the forefront of American consciousness.

Eighteen years of war have left most Americans largely oblivious to the
troops still deploying to Afghanistan. They are still suiting up in
camouflage, still saying goodbye to loved ones and still boarding
flights from Fayetteville, N.C., to Newark to Frankfurt to Kuwait to
Kabul before the final few miles to Bagram.

The mission of the American endeavor in Afghanistan has changed from a
war on terrorism to one meant to ``train, advise and assist'' Afghan
troops. But American forces are still going on patrols in remote areas
of the country. The mountain ridges, village roads and poppy fields are
as deadly as they were in 2001.

Americans are still fighting, and they are still dying. In 2019, the
three Marines were among 22 Americans lost in a long-forgotten war.

\includegraphics{https://static01.nyt.com/images/2020/07/10/us/politics/00dc-marines2/merlin_153255333_efd33248-a671-4734-a3cd-c1c4b4fd7e02-articleLarge.jpg?quality=75\&auto=webp\&disable=upscale}

Interviews with family, friends, service members and military officials
paint an all-too-familiar story of the three: young men who, after the
Sept. 11, 2001, attacks, sought to serve their country.

They were everyone's neighbors, and they came from hometowns in the
Mid-Atlantic and the Northeast. Sergeant Hines was a ``Star Wars'' buff.
Sergeant Slutman doted on his three daughters. Sergeant Hendriks played
lacrosse, was a unionized construction worker and led the way for his
brother to join the Marines as well.

They died two weeks before their unit was scheduled to go home --- and
while American and Taliban negotiators wrestled over details of a
proposed peace deal.

For those who loved them, the deaths of the three were excruciating.
Recalling the details again, a little more than a year later, is like
experiencing it twice, friends and family members said.

The cascade of news articles that followed the disclosure of
intelligence reports regarding the Russian bounties has been like
``pouring salt on the wound,'' said Jason Rina, a friend of Sergeant
Slutman's, who spent five years working with him at the Cross Bronx
Expressway, the name of their New York firehouse.

Erik Hendriks, Sergeant Hendriks's father, had largely avoided the news
in the past year, but he said the Russian bounty story scraped his pain
raw. ``If it does come out as true,'' he said in an interview,
``obviously the heartache would be terrible.''

\hypertarget{their-deployment-begins}{%
\subsection{Their Deployment Begins}\label{their-deployment-begins}}

The three Marines arrived in Afghanistan in October 2018 to find a home
base that existed under frequent rocket attack. Bagram Air Base, about
25 miles north of Kabul, the capital, is a huge complex that has served
as a command center for the American-led NATO mission in Afghanistan for
almost as long as there has been an American-led NATO mission in
Afghanistan.

Before that, the base was fought over by the Northern Alliance and the
Taliban during Afghanistan's brutal civil war. Before that, it was home
to the Soviet Union's 40th Army, the same base where the Kremlin first
positioned elite troops in the summer of 1979 as they prepared for their
own long war that ended in humiliation and defeat.

One of the first things that NATO troops did after the 2001 invasion was
to secure Bagram. By the time the Marines arrived, the base was home to
thousands of rotating NATO troops, civilian contractors and airmen.

There were office buildings, barracks and enormous dining facilities.
There was a Pizza Hut, a coffee cafe and a military shopping facility
with hair products and a chance to win a Harley-Davidson motorcycle
propped up in the entryway.

Image

Staff Sgt. Benjamin S. Hines, second from left; Sergeant Slutman, rear
center, in sunglasses; and Sgt. Robert A.~Hendriks, right, were killed
last year in a car bomb attack.Credit...via the Hendriks Family

But the commercial normalcy belies the fact that Bagram remains in a war
zone. Flak jackets are mandatory in some areas, and the sounds of
rockets and shelling from Taliban and other insurgents are the
background music of the airfield, almost as common as the planes taking
off and landing. Multiple suicide bombers have killed dozens of people
both outside and inside the base over the years. The near constancy of
the attacks at Bagram make them seem almost routine, according to
soldiers and Marines who deploy there.

The three Marine reservists were to go on patrols alongside troops from
the former Soviet republic of Georgia. By all accounts, it was a quiet
six-month deployment.

Sergeant Hendriks, a tall, mustachioed turret gunner, was frequently
spotted on base lifting weights in his fatigues and olive T-shirt with
the words ``All It Takes Is All You Got'' on the back. He needed to keep
in shape. Getting in and out of the armored vehicles involved a lot of
contortions because of his height.

He grew up in Locust Valley, N.Y., and was beguiled from an early age by
the idea of military service. A beloved uncle and grandfather were
veterans, and when one of his best friends joined the Marines, he
decided to follow, his father said.

``He came to me in high school in his senior year and asked me to go
sign for him,'' Mr. Hendriks said. ``I said: `Robbie think about it. You
don't know if you want to have a girlfriend, you don't know about your
future.'''

But his son, stern and straightforward, liked the steady and reliable
nature of the military and thought it was a good fit for his orderly
personality. ``He was a perfectionist,'' his father recalled. ``He knew
details. If you gave him a pencil and said, `Here, put this somewhere,'
he could find it two years later. If you told Robbie, `Do this for the
next 10 hours,' he would do it.''

After graduating in 2012 from Locust Valley High School, Sergeant
Hendriks joined the Marine Corps Reserves, assigned to the Site Support
Second Battalion, 25th Marine Regiment, in Garden City, N.Y. He was only
17. The next year, his younger brother Joseph Hendriks followed suit.

Image

President Trump and President Ashraf Ghani of Afghanistan speaking to
troops in November at Bagram Air Base. The base is home to thousands of
rotating NATO troops, civilian contractors and airmen.Credit...Erin
Schaff/The New York Times

Sergeant Hines, a former football player who was engaged to be married,
had always been enthusiastic about physical training, said Lt. Col. Joe
Innerst, a retired Marine who started an R.O.T.C. program in the
Dallastown Area School District in Pennsylvania when Sergeant Hines was
a high school senior.

He loved to rewatch ``Star Wars'' movies, his friends said, and
considered the series the ideal representation of good versus evil. He
also was notoriously late, operating on what his family and friends
jokingly referred to as ``Ben Time.'' He was close friends with the
third Marine, Sergeant Slutman, despite their 12-year age difference. He
joined other Marines who jokingly called Sergeant Slutman ``Old Man,''
friends said.

Sergeant Slutman, who had been a Marine for 14 years by the time he was
deployed at Bagram, arrived at the airfield already well tested and
respected after years of fighting fires across New York City. He had
received medals for bravery in 2014 after rescuing an unconscious woman
from a burning building in the South Bronx. He was constantly
challenging other Marines who called him old to try to keep up with him,
his friends said. But above all, they said, he adored his three
daughters.

\hypertarget{attack-on-their-convoy}{%
\subsection{Attack on Their Convoy}\label{attack-on-their-convoy}}

That October, a car bomb exploded near a Czech convoy outside Bagram,
wounding five Czech troops. But Thanksgiving and Christmas came and went
uneventfully for the Marines. In February, the new acting defense
secretary, Patrick M. Shanahan, showed up in an all-black outfit,
prompting the American news media to call him ``Dr. Evil.''

A month later, Sergeant Hendriks celebrated his 25th birthday.

The three Marines and the rest of their unit were going out almost
daily. Sleep, eat, patrol. The routine has been repeated many thousands
of times by 775,000 American troops in Afghanistan for more than 18
years. Sometimes the Marines were on foot, and sometimes they were in
convoys as they patrolled the Bagram perimeter.

The Marines knew that every time they left the airfield, their chances
of getting hit by an improvised roadside explosive or a car bomb went up
exponentially. The road between Kabul and Bagram was especially
dangerous, which is why visiting American dignitaries took helicopters
between the locations.

On April 8 --- less than two weeks before the Marines were due to return
to the United States --- the three set out on another patrol. Sergeants
Slutman, Hendriks and Hines were in the same vehicle.

Just before 4 p.m., as the convoy was approaching an intersection near
Bagram, a vehicle slammed into the three Marines' personnel carrier and
detonated. The Taliban quickly claimed responsibility.

Image

Sergeant Slutman's wife, Shannon, and their daughters at his funeral in
New York.Credit...Benjamin Norman for The New York Times

That night, 7,000 miles away in York, Pa., two Marines in dress uniform
arrived on the doorstep of Fletcher Slutman, Sergeant Slutman's father.
``Would you like to invite your wife in?'' one asked him, after they had
settled around the kitchen table. Mr. Slutman shook his head no.

But the Marine nodded his head yes.

Three nights later, on Thursday, April 11, a military plane landed at
Dover Air Force Base in Delaware with the remains of the three Marines.

Sergeant Hendriks's brother Joseph, who had followed his sibling into
the Marines and later followed him on a deployment to Afghanistan,
accompanied the remains of his brother, who was promoted posthumously to
sergeant, back to Dover.

Thirteen days after that, on April 24, friends and family of Sergeant
Hines gathered for a memorial service. They talked ``Star Wars.'' Then
they played an excerpt from ``The Imperial March,'' the horn-heavy theme
made famous as an anthem to Darth Vader.

It was beloved by Sergeant Hines, his friends said.

The investigation into the deaths of the three Marines continues.
Although Mr. Trump has dismissed the suspected Russian payments as
``fake news,'' Congress has begun hearings into the matter. Gen. Mark A.
Milley, the chairman of the Joint Chiefs of Staff, said that while the
government so far lacks proof that any Russian bounties caused specific
military casualties, ``we are still looking.''

``We're not done,'' General Milley
\href{https://www.nytimes.com/2020/07/09/us/politics/congress-russian-bounties.html}{told
a House committee} last week. ``We're going to run this thing to
ground.''

Sharon Otterman contributed reporting.

Advertisement

\protect\hyperlink{after-bottom}{Continue reading the main story}

\hypertarget{site-index}{%
\subsection{Site Index}\label{site-index}}

\hypertarget{site-information-navigation}{%
\subsection{Site Information
Navigation}\label{site-information-navigation}}

\begin{itemize}
\tightlist
\item
  \href{https://help.nytimes.com/hc/en-us/articles/115014792127-Copyright-notice}{©~2020~The
  New York Times Company}
\end{itemize}

\begin{itemize}
\tightlist
\item
  \href{https://www.nytco.com/}{NYTCo}
\item
  \href{https://help.nytimes.com/hc/en-us/articles/115015385887-Contact-Us}{Contact
  Us}
\item
  \href{https://www.nytco.com/careers/}{Work with us}
\item
  \href{https://nytmediakit.com/}{Advertise}
\item
  \href{http://www.tbrandstudio.com/}{T Brand Studio}
\item
  \href{https://www.nytimes.com/privacy/cookie-policy\#how-do-i-manage-trackers}{Your
  Ad Choices}
\item
  \href{https://www.nytimes.com/privacy}{Privacy}
\item
  \href{https://help.nytimes.com/hc/en-us/articles/115014893428-Terms-of-service}{Terms
  of Service}
\item
  \href{https://help.nytimes.com/hc/en-us/articles/115014893968-Terms-of-sale}{Terms
  of Sale}
\item
  \href{https://spiderbites.nytimes.com}{Site Map}
\item
  \href{https://help.nytimes.com/hc/en-us}{Help}
\item
  \href{https://www.nytimes.com/subscription?campaignId=37WXW}{Subscriptions}
\end{itemize}
