Sections

SEARCH

\protect\hyperlink{site-content}{Skip to
content}\protect\hyperlink{site-index}{Skip to site index}

\href{https://www.nytimes.com/section/books}{Books}

\href{https://myaccount.nytimes.com/auth/login?response_type=cookie\&client_id=vi}{}

\href{https://www.nytimes.com/section/todayspaper}{Today's Paper}

\href{/section/books}{Books}\textbar{}Yes, Fake News Is a Problem. But
There's a Real News Problem, Too.

\url{https://nyti.ms/2WUogI0}

\begin{itemize}
\item
\item
\item
\item
\item
\end{itemize}

Advertisement

\protect\hyperlink{after-top}{Continue reading the main story}

Supported by

\protect\hyperlink{after-sponsor}{Continue reading the main story}

\href{/column/books-of-the-times}{Books of The Times}

\hypertarget{yes-fake-news-is-a-problem-but-theres-a-real-news-problem-too}{%
\section{Yes, Fake News Is a Problem. But There's a Real News Problem,
Too.}\label{yes-fake-news-is-a-problem-but-theres-a-real-news-problem-too}}

By \href{https://www.nytimes.com/by/jennifer-szalai}{Jennifer Szalai}

\begin{itemize}
\item
  July 26, 2020
\item
  \begin{itemize}
  \item
  \item
  \item
  \item
  \item
  \end{itemize}
\end{itemize}

\includegraphics{https://static01.nyt.com/images/2020/07/30/books/29BOOKSULLIVAN1/29BOOKSULLIVAN1-articleLarge.png?quality=75\&auto=webp\&disable=upscale}

Buy Book ▾

\begin{itemize}
\tightlist
\item
  \href{https://www.amazon.com/gp/search?index=books\&tag=NYTBSREV-20\&field-keywords=Ghosting+the+News+Margaret+Sullivan}{Amazon}
\item
  \href{https://du-gae-books-dot-nyt-du-prd.appspot.com/buy?title=Ghosting+the+News\&author=Margaret+Sullivan}{Apple
  Books}
\item
  \href{https://www.anrdoezrs.net/click-7990613-11819508?url=https\%3A\%2F\%2Fwww.barnesandnoble.com\%2Fw\%2F\%3Fean\%3D9781733623780}{Barnes
  and Noble}
\item
  \href{https://www.anrdoezrs.net/click-7990613-35140?url=https\%3A\%2F\%2Fwww.booksamillion.com\%2Fp\%2FGhosting\%2Bthe\%2BNews\%2FMargaret\%2BSullivan\%2F9781733623780}{Books-A-Million}
\item
  \href{https://bookshop.org/a/3546/9781733623780}{Bookshop}
\item
  \href{https://www.indiebound.org/book/9781733623780?aff=NYT}{Indiebound}
\end{itemize}

When you purchase an independently reviewed book through our site, we
earn an affiliate commission.

What do you call it when a hedge fund buys a local newspaper and
squeezes it for revenue, laying off editors and reporters and selling
off the paper's downtown headquarters for conversion into
\href{https://www.nytimes.com/2020/02/11/business/newspaper-building-redevelopment.html}{luxury
condos} or
\href{https://www.nytimes.com/2020/07/10/us/alden-global-capital-pottstown-mercury.html}{a
boutique hotel}?

The devastation has become common enough that some observers have
resorted to shorthand for what collectively amounts to an
extinction-level event. One former editor calls it a ``harvesting
strategy''; Margaret Sullivan, in her new book, ``Ghosting the News,''
calls it ``strip-mining.'' Like the climate emergency that Sullivan
mentions by way of comparison, the decimation of local news yields two
phenomena that happen to feed off each other: The far-reaching effects
are cataclysmic, and it's hard to convince a significant number of
people that they ought to care.

``Disinformation'' and ``fake news'' bring to mind scheming operatives,
Russian troll farms and noisy propaganda; stories about them are
titillating enough to garner plenty of attention. But what Sullivan
writes about is a ``real-news problem'' --- the shuttering of more than
2,000 American newspapers since 2004, and the creation of ``news
deserts,'' or entire counties with no local news outlets at all.

She begins her book with the example of a 2019 story from The Buffalo
News about a suburban police chief who received an unexplained \$100,000
payout when he abruptly retired. The article didn't win any awards or
even appear on the front page, Sullivan writes. ``It merely was the kind
of day-in-and-day-out local reporting that makes secretive town
officials unhappy.''

``Merely'' and ``day-in-and-day-out''; Sullivan also describes the
article as ``routine-enough fare.'' ``Ghosting the News'' is a brisk and
pointed tribute to painstaking, ordinary and valuable work. As the media
columnist for The Washington Post and the former public editor for The
New York Times, Sullivan has spent most of the past decade writing for a
national audience, but for 32 years before that she worked at The
Buffalo News, starting as a summer intern and eventually becoming the
newspaper's editor.

Image

Margaret Sullivan, author of ``Ghosting the News.''Credit...Michael
Benabib

Sullivan recalls the flush days when the paper boasted a newsroom fully
staffed by journalists who could combine their calling with a career.
Then came the internet, which siphoned off attention and revenue; after
that, the deluge of the 2008 financial crisis, which swept away the
vestiges of print advertising. Sullivan cut the payroll of the paper by
offering buyouts. She got rid of the full-time art critic and eliminated
the Sunday magazine --- ``a particularly wrenching decision because my
then-husband was the magazine's editor.''

The Buffalo News was owned by Warren Buffett's Berkshire Hathaway until
the beginning of this year, when Buffett declared it was time for him to
leave the newspaper industry and sold his portfolio of 31 dailies and 49
weeklies. Buffett said he believes in the importance of journalism, but
he doesn't consider himself a philanthropist. He got into the business
because it made money, with fat profit margins in the good years
reaching 30 percent. When he bought The Buffalo News in 1977, he decided
that the city could sustain only one daily, and he knocked out the
competition until his was the last paper standing. A monopoly newspaper
was like an unregulated toll bridge: With a loyal and captive market, he
could raise rates whenever he wanted.

Advertisers may have been peddling baubles or junk food, but their cash
funded serious journalism --- the kind that could afford to send a
reporter to, say, every municipal board meeting. ``People knew that,''
the former editor of the once mighty Youngstown Vindicator told
Sullivan, ``and they behaved.'' This watchdog function had tangible
benefits for subscribers and nonsubscribers alike. ``When local
reporting waned,'' Sullivan writes, ``municipal borrowing costs went
up.'' Local news outlets provide the due diligence that bondholders
often count on. Without the specter of a public shaming, corruption is
freer to flourish.

Sullivan surveys the alternative models that have sprung up in response
to journalism's ecosystem collapse. There's the nonprofit reporting
outfit ProPublica, and a ``news brigade'' of volunteer journalists in
Michigan. Sullivan's own employer was acquired by Jeff Bezos in 2013 for
\href{https://www.nytimes.com/2019/02/11/business/media/washington-post-jeff-bezos.html}{\$250
million}. ``Jeff Bezos has not attempted to influence coverage at The
Washington Post,'' she writes, though billionaire owners aren't always
so hands-off. The casino magnate Sheldon Adelson bought the
well-respected Review-Journal in Las Vegas, which was known for its
investigative pieces on the casino industry, and leaned on its staff to
produce puff pieces about his properties instead. Adelson turned the
watchdog into a lap dog.

The situation is so dire, Sullivan says, that she entertains what was
once unthinkable --- the possibility of government-subsidized
journalistic outlets. She calls the argument for government help ``not
unreasonable,'' even if she hasn't been entirely convinced yet. Her
attempts to strike a hopeful note can sound unsatisfying because of how
problematic all the solutions are. Nonprofit start-ups have the benefit
of being ``nimbler,'' Sullivan says, though what does nimbler often mean
in practice? A non-unionized newsroom staffed by 24-year-olds who can be
paid junior-level salaries and, unlike veteran journalists three decades
older, wouldn't necessarily be ruined by a layoff?

Sullivan is left to highlight the essential work that local reporters
do, emphasizing how The Palm Beach Post and The Miami Herald continued
to pursue the story of Jeffrey Epstein's sex trafficking long after
others had decided that the abuse scandal had ``gone stale.'' More
recently, local journalists recorded the influx of
\href{https://www.opb.org/news/article/federal-law-enforcement-unmarked-vehicles-portland-protesters/}{unidentified
federal troops} into Portland, Ore., where they were seizing and
detaining people without telling them why or what was happening to them;
the example was too late to be included in Sullivan's book, and it only
goes to show how critical and relentless the need is for reporters on
the ground.

``Ghosting the News'' concludes with a soaring quote from the Italian
theorist Antonio Gramsci about ``pessimism of the intellect and optimism
of the will,'' but the local reporter in Sullivan follows it up with a
more immediate analogy: Even if no one seems to be coming to the rescue
while your house is on fire, you still have to ``get out your garden
hose and bucket, and keep acting as if the fire trucks are on the way.''

Advertisement

\protect\hyperlink{after-bottom}{Continue reading the main story}

\hypertarget{site-index}{%
\subsection{Site Index}\label{site-index}}

\hypertarget{site-information-navigation}{%
\subsection{Site Information
Navigation}\label{site-information-navigation}}

\begin{itemize}
\tightlist
\item
  \href{https://help.nytimes.com/hc/en-us/articles/115014792127-Copyright-notice}{©~2020~The
  New York Times Company}
\end{itemize}

\begin{itemize}
\tightlist
\item
  \href{https://www.nytco.com/}{NYTCo}
\item
  \href{https://help.nytimes.com/hc/en-us/articles/115015385887-Contact-Us}{Contact
  Us}
\item
  \href{https://www.nytco.com/careers/}{Work with us}
\item
  \href{https://nytmediakit.com/}{Advertise}
\item
  \href{http://www.tbrandstudio.com/}{T Brand Studio}
\item
  \href{https://www.nytimes.com/privacy/cookie-policy\#how-do-i-manage-trackers}{Your
  Ad Choices}
\item
  \href{https://www.nytimes.com/privacy}{Privacy}
\item
  \href{https://help.nytimes.com/hc/en-us/articles/115014893428-Terms-of-service}{Terms
  of Service}
\item
  \href{https://help.nytimes.com/hc/en-us/articles/115014893968-Terms-of-sale}{Terms
  of Sale}
\item
  \href{https://spiderbites.nytimes.com}{Site Map}
\item
  \href{https://help.nytimes.com/hc/en-us}{Help}
\item
  \href{https://www.nytimes.com/subscription?campaignId=37WXW}{Subscriptions}
\end{itemize}
