Sections

SEARCH

\protect\hyperlink{site-content}{Skip to
content}\protect\hyperlink{site-index}{Skip to site index}

\href{https://www.nytimes.com/section/politics}{Politics}

\href{https://myaccount.nytimes.com/auth/login?response_type=cookie\&client_id=vi}{}

\href{https://www.nytimes.com/section/todayspaper}{Today's Paper}

\href{/section/politics}{Politics}\textbar{}Army Officer Who Clashed
With Trump Over Impeachment Is Set to Retire

\url{https://nyti.ms/2O5yzo7}

\begin{itemize}
\item
\item
\item
\item
\item
\item
\end{itemize}

Advertisement

\protect\hyperlink{after-top}{Continue reading the main story}

Supported by

\protect\hyperlink{after-sponsor}{Continue reading the main story}

\hypertarget{army-officer-who-clashed-with-trump-over-impeachment-is-set-to-retire}{%
\section{Army Officer Who Clashed With Trump Over Impeachment Is Set to
Retire}\label{army-officer-who-clashed-with-trump-over-impeachment-is-set-to-retire}}

The White House had made clear to Pentagon officials that President
Trump did not want to see Lt. Col. Alexander S. Vindman promoted.

\includegraphics{https://static01.nyt.com/images/2020/07/08/us/politics/08dc-vindman/merlin_164668554_d6f4f652-946a-41d8-ad40-19ee641327a5-articleLarge.jpg?quality=75\&auto=webp\&disable=upscale}

\href{https://www.nytimes.com/by/eric-schmitt}{\includegraphics{https://static01.nyt.com/images/2018/06/12/multimedia/author-eric-schmitt/author-eric-schmitt-thumbLarge-v2.png}}\href{https://www.nytimes.com/by/helene-cooper}{\includegraphics{https://static01.nyt.com/images/2018/08/24/multimedia/author-helene-cooper/author-helene-cooper-thumbLarge.png}}

By \href{https://www.nytimes.com/by/eric-schmitt}{Eric Schmitt} and
\href{https://www.nytimes.com/by/helene-cooper}{Helene Cooper}

\begin{itemize}
\item
  July 8, 2020
\item
  \begin{itemize}
  \item
  \item
  \item
  \item
  \item
  \item
  \end{itemize}
\end{itemize}

WASHINGTON --- An Army officer who was a prominent witness during the
impeachment inquiry into President Trump last year said on Wednesday
that he had decided to retire after what his lawyer called a campaign of
White House intimidation and retaliation.

The incident is the latest in what Pentagon and congressional officials
say could be another flash point between the president and the military.

The witness, Lt. Col. Alexander S. Vindman, a decorated Iraq war veteran
who served on the staff of the White House National Security Council, is
among scores of officers who have been picked to be promoted to full
colonel this year. Typically, such promotions are backed by Army and
Pentagon officials before moving to the White House for final approval,
and then to the Senate for a confirmation vote.

But the White House had made clear to officials in the Pentagon's office
of personnel and readiness, which handles such matters, that Mr. Trump
did not want to see Colonel Vindman promoted, officials said.

Mr. Trump's allies at the White House asked Pentagon officials to find
instances of misconduct by Colonel Vindman that would justify blocking
his promotion, administration officials said on Wednesday.

On multiple occasions, including this week, the White House pressed the
Pentagon to seek witnesses who would come forward and say that Colonel
Vindman acted improperly, the officials said.

But Defense Secretary Mark T. Esper and Army Secretary Ryan McCarthy
have been unable to produce such evidence, largely because it does not
exist, according to one administration official who spoke on the
condition of anonymity.

With that hurdle cleared, Mr. Esper on Monday approved the promotion
list, including Colonel Vindman, and it was expected to be delivered to
the White House by Friday, a second administration official said.

Senior Army leaders were caught off guard by Colonel Vindman's decision
on Wednesday. Mr. McCarthy was expected to have a general officer
contact Colonel Vindman to discuss his options, an administration
official said.

But people familiar with Colonel Vindman's decision said he felt
increasingly pessimistic that he had a meaningful future in the Army. He
announced his decision in a short Twitter message on Wednesday morning.

``Today I officially requested retirement from the US Army, an
organization I love,'' he said. ``My family and I look forward to the
next chapter of our lives.''

Colonel Vindman's lawyer, David Pressman, said in a statement that the
officer was the victim of campaign of ``bullying'' and ``intimidation''
by the White House.

``Through a campaign of bullying, intimidation and retaliation, the
president of the United States attempted to force LTC Vindman to choose:
Between adhering to the law or pleasing a president,'' Mr. Pressman
said. ``Between honoring his oath or protecting his career. Between
protecting his promotion or the promotion of his fellow soldiers.''

Mr. Pressman added, ``Vindman did what the law compelled him to do; and
for that he was bullied by the president and his proxies.''

The White House declined to comment.

In his role as a Ukraine expert on the National Security Council staff,
Colonel Vindman was on
\href{https://www.nytimes.com/interactive/2019/09/25/us/politics/trump-ukraine-transcript.html}{Mr.
Trump's phone call on July 25} with Ukraine's president that later was a
central element of the impeachment inquiry. Colonel Vindman testified in
the House impeachment hearings that it was
``\href{https://www.nytimes.com/2019/11/19/us/politics/impeachment-hearings.html}{improper
for the president}'' to coerce a foreign country to investigate a
political opponent.

Hours before Colonel Vindman
\href{https://www.nytimes.com/2020/02/07/us/politics/alexander-vindman-gordon-sondland-fired.html}{was
marched out of the White House in February} by security guards, Mr.
Trump foreshadowed his fate when asked if he would be pushed out.
``Well, I'm not happy with him,'' the president told reporters. ``You
think I'm supposed to be happy with him? I'm not.''

A person familiar with Colonel Vindman's decision said he decided to
retire after more than 21 years in the Army when it became apparent he
would not be able to serve in a useful capacity in his area of
specialty, Eurasia affairs. He had been scheduled to start a term at the
Army War College later this summer.

Colonel Vindman's retirement, which still must be approved by the Army,
comes despite promises from Mr. Esper and other senior military leaders
to protect from retribution members of the armed services who return to
military duties after serving tours at the White House.

Senator Tammy Duckworth, Democrat of Illinois, said last week that she
would block Senate confirmation of 1,123 military personnel promotions
until she received assurances that Colonel Vindman's promotion would not
be blocked.

``Lt. Col. Vindman's decision to retire puts the spotlight on Secretary
of Defense Mark Esper's failure to protect a decorated combat veteran
against a vindictive commander in chief,'' Ms. Duckworth said in a
statement on Wednesday.

Advertisement

\protect\hyperlink{after-bottom}{Continue reading the main story}

\hypertarget{site-index}{%
\subsection{Site Index}\label{site-index}}

\hypertarget{site-information-navigation}{%
\subsection{Site Information
Navigation}\label{site-information-navigation}}

\begin{itemize}
\tightlist
\item
  \href{https://help.nytimes.com/hc/en-us/articles/115014792127-Copyright-notice}{©~2020~The
  New York Times Company}
\end{itemize}

\begin{itemize}
\tightlist
\item
  \href{https://www.nytco.com/}{NYTCo}
\item
  \href{https://help.nytimes.com/hc/en-us/articles/115015385887-Contact-Us}{Contact
  Us}
\item
  \href{https://www.nytco.com/careers/}{Work with us}
\item
  \href{https://nytmediakit.com/}{Advertise}
\item
  \href{http://www.tbrandstudio.com/}{T Brand Studio}
\item
  \href{https://www.nytimes.com/privacy/cookie-policy\#how-do-i-manage-trackers}{Your
  Ad Choices}
\item
  \href{https://www.nytimes.com/privacy}{Privacy}
\item
  \href{https://help.nytimes.com/hc/en-us/articles/115014893428-Terms-of-service}{Terms
  of Service}
\item
  \href{https://help.nytimes.com/hc/en-us/articles/115014893968-Terms-of-sale}{Terms
  of Sale}
\item
  \href{https://spiderbites.nytimes.com}{Site Map}
\item
  \href{https://help.nytimes.com/hc/en-us}{Help}
\item
  \href{https://www.nytimes.com/subscription?campaignId=37WXW}{Subscriptions}
\end{itemize}
