Sections

SEARCH

\protect\hyperlink{site-content}{Skip to
content}\protect\hyperlink{site-index}{Skip to site index}

\href{https://www.nytimes.com/section/politics}{Politics}

\href{https://myaccount.nytimes.com/auth/login?response_type=cookie\&client_id=vi}{}

\href{https://www.nytimes.com/section/todayspaper}{Today's Paper}

\href{/section/politics}{Politics}\textbar{}Like Father, Like Son:
President Trump Lets Others Mourn

\url{https://nyti.ms/305Sr12}

\begin{itemize}
\item
\item
\item
\item
\item
\item
\end{itemize}

\href{https://www.nytimes.com/news-event/george-floyd-protests-minneapolis-new-york-los-angeles?action=click\&pgtype=Article\&state=default\&region=TOP_BANNER\&context=storylines_menu}{Race
and America}

\begin{itemize}
\tightlist
\item
  \href{https://www.nytimes.com/2020/07/26/us/protests-portland-seattle-trump.html?action=click\&pgtype=Article\&state=default\&region=TOP_BANNER\&context=storylines_menu}{Protesters
  Return to Other Cities}
\item
  \href{https://www.nytimes.com/2020/07/24/us/portland-oregon-protests-white-race.html?action=click\&pgtype=Article\&state=default\&region=TOP_BANNER\&context=storylines_menu}{Portland
  at the Center}
\item
  \href{https://www.nytimes.com/2020/07/23/podcasts/the-daily/portland-protests.html?action=click\&pgtype=Article\&state=default\&region=TOP_BANNER\&context=storylines_menu}{Podcast:
  Showdown in Portland}
\item
  \href{https://www.nytimes.com/interactive/2020/07/16/us/black-lives-matter-protests-louisville-breonna-taylor.html?action=click\&pgtype=Article\&state=default\&region=TOP_BANNER\&context=storylines_menu}{45
  Days in Louisville}
\end{itemize}

Advertisement

\protect\hyperlink{after-top}{Continue reading the main story}

Supported by

\protect\hyperlink{after-sponsor}{Continue reading the main story}

\hypertarget{like-father-like-son-president-trump-lets-others-mourn}{%
\section{Like Father, Like Son: President Trump Lets Others
Mourn}\label{like-father-like-son-president-trump-lets-others-mourn}}

Whether he is dealing with the loss of a family member or the deaths of
nearly 150,000 Americans in a surging pandemic, President Trump almost
never displays empathy in public. He learned it from his father.

\includegraphics{https://static01.nyt.com/images/2020/07/28/us/politics/28dc-fredtrump1/merlin_97610656_78a7876d-c54b-430a-a599-163bcf354205-articleLarge.jpg?quality=75\&auto=webp\&disable=upscale}

\href{https://www.nytimes.com/by/annie-karni}{\includegraphics{https://static01.nyt.com/images/2019/02/05/multimedia/author-annie-karni/author-annie-karni-thumbLarge.png}}\href{https://www.nytimes.com/by/katie-rogers}{\includegraphics{https://static01.nyt.com/images/2018/06/12/multimedia/author-katie-rogers/author-katie-rogers-thumbLarge-v2.png}}

By \href{https://www.nytimes.com/by/annie-karni}{Annie Karni} and
\href{https://www.nytimes.com/by/katie-rogers}{Katie Rogers}

\begin{itemize}
\item
  Published July 28, 2020Updated July 29, 2020
\item
  \begin{itemize}
  \item
  \item
  \item
  \item
  \item
  \item
  \end{itemize}
\end{itemize}

\href{https://www.nytimes.com/es/2020/07/31/espanol/estados-unidos/fred-trump-donald-trump.html}{Leer
en español}

WASHINGTON --- The Marble Collegiate Church on Fifth Avenue in Manhattan
was packed with developers, politicians and New York celebrities, more
than 600 in all, for the funeral of Fred C. Trump, the builder whose
no-frills brick rental towers transformed Brooklyn and Queens.

Three of his four living children, who had grown up listening to the
sermons of the church's most famous minister, Norman Vincent Peale,
offered loving eulogies to their father. Then it was Donald Trump's
turn.

He began by talking about himself.

He had learned of his father's death, he told the crowd that day in June
1999, just moments after reading a front-page New York Times article
about his biggest development to date, Trump Place.

``Donald started his eulogy by saying, `I was having the greatest year
of my business career, and I was sitting having breakfast thinking of
how well things were going for me,''' when he learned of his father's
death, said Alan Marcus, a former public relations consultant for the
Trump Organization. ``Donald's eulogy was all about Donald, and
everybody in Vincent Peale's church knew it.''

Gwenda Blair, a Trump family biographer, also attended the funeral. She,
too, could not help but take note of the eulogy, which she described in
her book ``The Trumps.''

``Was it surprising?'' Ms. Blair said in an interview. ``No. Was it
stunning? Yes.''

\includegraphics{https://static01.nyt.com/images/2020/07/29/us/politics/29dc-fredtrump-print1/28dc-fredtrump-articleLarge.jpg?quality=75\&auto=webp\&disable=upscale}

Whether he is dealing with the loss of a family member, the deaths of
nearly 150,000 Americans in a surging pandemic, more than 30 million
people out of work or the racial unrest brought on by the killings of
African-Americans by white police officers, President Trump almost never
shows empathy in public.
\href{https://www.nytimes.com/2020/07/07/nyregion/mary-trump-book.html}{A
book published this summer} by his niece, Mary L. Trump, has focused
renewed attention on this trait.

Mr. Trump has held no national day of mourning for victims of the virus.
He has surrounded himself at Rose Garden events with business executives
pushing to reopen the economy rather than families who have lost jobs or
loved ones. In grim speeches over the Fourth of July weekend, he angrily
denounced what he branded as the ``new far-left fascism'' and never once
mentioned George Floyd, the Black man whose death in police custody has
set off worldwide protests over racial injustice.

There are many reasons --- denial and disorganization among them ---
that Mr. Trump's handling of the virus has led to catastrophic and
overlapping crises in the United States. But even Republicans say one
primary cause is the president's failure to put himself in the shoes of
others and harness their pain. His unwillingness, or inability, to
comfort an anxious nation has appalled critics, stunned allies and
aggravated White House staff members, who remain perplexed why this most
basic part of presidential leadership eludes him.

``His style as a leader is having to be a tough guy,'' Representative
Peter T. King of New York, one of the president's allies, said in an
interview. ``You can't show any type of weakness. He doesn't want to
show that this is getting the best of him.''

Mr. Trump has exhibited this behavior all his life, friends and family
members say. He learned it, they say, at home, particularly from his
father, a disciplinarian who spent hundreds of millions of dollars
financing his son's career and taught him to either dominate or submit.
In Fred Trump's world, showing sadness or hurt was a sign of weakness.

``The only thing that Trump ever cared about was he had this thing:
`I've got to win. Teach me how to win,''' George White, a former
classmate of Mr. Trump's at the New York Military Academy who spent
years around both father and son, said in an interview.

Recalling Fred's hard-driving influence, Mr. White said that Mr. Trump's
former school mentor, a World War II combat veteran named Theodore
Dobias, once told him that ``he had never seen a cadet whose father was
harder on him than his father was on Donald Trump.'' Fred Trump would
visit nearly every weekend to keep watch over his son, Mr. White said.

Mr. Trump's father is still part of his life, said Andrew Stein, a
former Manhattan borough president who has known the president for
decades and has met regularly with him at the White House. Mr. Trump, he
said, has often pointed up to the ceiling and referred to his father
when they have been alone in the Oval Office. ``He'll look up to heaven,
and say, `Fred, can you believe this?''' Mr. Stein said.

This article is based on interviews with more than 20 of Mr. Trump's
friends, political allies, administration members, family members, and
current and former employees.

Fred Trump's domineering relationship with his children, and how that
shaped his second son, is now the central animating force of the
best-selling ``Too Much and Never Enough: How My Family Created the
World's Most Dangerous Man,'' by Ms. Trump, a clinical psychologist and
Mr. Trump's only niece.

``Acknowledging the victims of Covid-19 would be to associate himself
with their weakness, a trait his father taught him to despise,'' Ms.
Trump wrote.

Robert Trump, the president's younger brother --- who along with Mr.
Trump tried to stop publication of the book --- disputed that
characterization. In a statement for this article, he said he knew ``how
selfless my father was and Donald is, much more so than anyone would
ever realize.''

\hypertarget{dominate-or-submit}{%
\subsection{Dominate or submit}\label{dominate-or-submit}}

Born in 1946 into the optimism and energy of postwar America, Mr. Trump
grew up in a red-brick, white-columned McMansion built by his father in
what was then a gated, nearly all-white community in Queens. He was, by
his own admission in his autobiography ``The Art of the Deal,'' a
difficult, tempestuous child. A favorite activity was testing other
people, from children in his neighborhood to figures of authority.
Neighbors
\href{https://www.washingtonpost.com/lifestyle/style/young-donald-trump-military-school/2016/06/22/f0b3b164-317c-11e6-8758-d58e76e11b12_story.html}{once
caught him throwing rocks} over a fence at a young child in a playpen.

``Even in elementary school, I was a very assertive, aggressive kid,''
Mr. Trump wrote.

The household was strict. Fred Trump was ``stiff and formal,'' said a
neighbor, Annamaria Forcier, and was focused on work and money. (His
father, Frederick Trump, had made a fortune in the Gold Rush before
dying of the Spanish flu in 1918.)

The president's mother, Mary Anne MacLeod Trump, was a fisherman's
daughter from a Scottish village in the Outer Hebrides who arrived in
New York in 1930 at the age of 18. Mary Anne found a job as a maid at
the home of Andrew Carnegie's widow, according to census records that
the journalist Nina Burleigh unearthed for her book ``Golden Handcuffs:
The Secret History of Trump's Women.'' The home is now the Cooper Hewitt
Museum in Manhattan.

Mrs. Trump's brush with society engendered the outsider's love of
ceremony and pomp shared by her son. In her book, Ms. Burleigh wrote
that ``Mary's airs were the antithesis'' of her husband's Germanic
tendencies. Her sense of humor could often be turned back on Donald
Trump, one of the president's children said.

Image

A yearbook photo of Mr. Trump from his time at the New York Military
Academy, where he attended junior high school.Credit...Fred R. Conrad
for The New York Times

But Fred Trump ran the show, and the children learned to be stoic in the
face of loss, even when their mother fell seriously ill with
peritonitis, an inflammation of the tissue lining the abdominal cavity,
and faced a lengthy hospitalization and lingering illness after the
birth of her fifth and last child.

``My father came home and told me she wasn't expected to live, but I
should go to school and he'd call me if anything changed,'' Maryanne
Trump Barry, one of his daughters, said in an interview with Ms. Blair.
``That's right, go to school as usual.''

In Mary Trump's view, Donald Trump --- who was two and a half years old
at the time --- suffered harm the year his mother was sick. ``Donald's
needs, which had been met inconsistently before his mother's illness,
were barely met at all by his father,'' Ms. Trump wrote. ``That Fred
would, by default, become the primary source of Donald's solace when he
was much more likely to be a source of fear or rejection put Donald into
an intolerable position: total dependence on a caregiver who was also
likely to be a source of his terror.''

As a result, she wrote, he ``suffered deprivations that would scar him
for life.''

Fred Trump Jr., the second born and the first son, was pushed hard by
his father as the presumed heir to the family business. But Fred Jr.
never took to real estate and died alone in the hospital in 1981 after a
long struggle with alcoholism. He was 42. According to Ms. Trump, his
daughter, Donald Trump went to the movies that night and Fred Trump Sr.
did not visit him.

The family rarely talked about Fred Jr.'s death, but in a 1990 interview
in Playboy, Donald Trump spent a few moments reflecting on it. ``I saw
people really taking advantage of Fred and the lesson I learned was
always to keep up my guard 100 percent, whereas he didn't,'' Mr. Trump
said. ``He didn't feel that there was really reason for that, which is a
fatal mistake in life. People are too trusting. I'm a very untrusting
guy.''

\hypertarget{he-doesnt-have-time-to-have-empathy}{%
\subsection{`He doesn't have time to have
empathy'}\label{he-doesnt-have-time-to-have-empathy}}

Dan P. McAdams, a professor of psychology and human development at
Northwestern who has written about Mr. Trump, said in an interview that
from childhood on, Mr. Trump --- with the help of his father ---
conditioned himself to approach life as a series of battles to be won.

``He doesn't have time to have empathy for anybody because the world is
out to get him,'' Mr. McAdams said.

After his brother's death, Mr. Trump became the heir, and over the next
decades he and his father were close partners in the schemes and tax
evasions that were part of the family business. They talked almost every
day and spent time together on weekends.

``I was never intimidated by my father, the way most people were,'' Mr.
Trump wrote in his autobiography. ``I stood up to him, and he respected
that. We had a relationship that was almost businesslike.''

Like his father, Mr. Trump moved on in the face of loss. At the Trump
Organization he was not a boss who reached out to express condolences.
``One of his bankers had died and somebody in this small circle said,
`Donald, don't you think you should call the family?''' recalled Mr.
Marcus, the former Trump Organization public relations consultant. ``He
said: `Why? He's dead.'''

Mr. Trump's coldness in the face of illness shocked even some of his
closest associates. After Roy Cohn, Mr. Trump's longtime personal
lawyer, learned he had AIDS in the 1980s, Mr. Trump abruptly cut off
contact with him --- a dramatic shift from the connected relationship
they had enjoyed for years, which associates recalled involved talking
on the phone at least five times a day.

``He discards people who are no longer useful, and it doesn't matter
what renders the person no longer useful,'' said Michael D'Antonio, a
Trump biographer. ``If you are disgraced, or you're dying, or deceased,
you no longer exist to him.'' Mr. D'Antonio recalled Mr. Trump telling
him that he had given Mr. Cohn a place to stay at the end of his life.
``Donald thought providing him with something of material worth was
adequate,'' he said.

Image

The Trump Taj Mahal casino in Atlantic City in 1990. Mr. Trump seemed to
prioritize his businesses over empathy toward others.Credit...Tony
Ward/Mirrorpix, via Getty Images

In 1989 a helicopter flying from New York to Atlantic City crashed and
\href{https://www.nytimes.com/1989/10/11/nyregion/copter-crash-kills-3-aides-of-trump.html}{killed
three top executives at Mr. Trump's Atlantic City casinos}. Mr. Trump
infamously used the tragedy to his own advantage, planting stories in
local newspapers that he had been scheduled to board the aircraft until
the last minute and had narrowly escaped death himself. In a later book
he admitted he had never been scheduled to fly on the helicopter at all.

Jack O'Donnell, who was the president of the Trump Plaza Hotel and
Casino at the time and wrote a scathing book about Mr. Trump, said Mr.
Trump processed the deaths mostly as a meteoric hit to his business.

But the night of the crash, Mr. O'Donnell recalled, Mr. Trump did
something unusual for him.

``I didn't think he was capable of it,'' Mr. O'Donnell said. ``But he
flew down to Atlantic City and he personally went to the homes of the
widows and spent time with them.'' Months later, however, ``he blamed
those same guys for issues he created,'' Mr. O'Donnell said. ``It was
why I finally left him, in a huge argument.''

A little more than a decade later, when Mr. Trump's mother was seriously
ill, he had to be reminded by his siblings to peel away from work and
visit her at the hospital, Mr. Marcus recalled. She died at the age of
88 in 2000, only a year after her husband.

\hypertarget{a-great-day-for-everybody}{%
\subsection{`A great day for
everybody'}\label{a-great-day-for-everybody}}

Image

Mr. Trump has used White House events to meet with business leaders,
rather than mourn victims of the coronavirus pandemic.Credit...Doug
Mills/The New York Times

In response to this article, Hogan Gidley, a former White House
spokesman who has since transitioned over to the campaign, **** said the
president did show empathy. He sent three news clippings as evidence,
which all generated positive coverage for Mr. Trump.

\href{https://www.jta.org/1988/07/20/archive/orthodox-child-with-rare-ailment-is-rescued-aboard-tycoons-jet}{One
from 1988} recounted how Mr. Trump donated the use of his private jet to
fly a sick child to New York for treatment for a rare medical problem.
Another detailed an effort by Mr. Trump in 1986 to help a widow raise
money to
\href{https://apnews.com/24c831825e0dab47d51d8d25bffe45f5}{cover her
mortgage payments}. The third covered Mr. Trump's
\href{https://www.aol.com/2013/11/08/trump-gift-barton-buffalo/}{\$10,000
donation} in 2013 to a bus driver who saved a woman from jumping off a
bridge.

Last month in the Rose Garden as Mr. Trump highlighted a dip in the
unemployment rate,
\href{https://www.nytimes.com/2020/06/05/us/politics/trump-jobs-report-george-floyd.html}{he
invoked Mr. Floyd}.

``Hopefully, George is looking down right now and saying this is a great
thing that's happening for our country,'' he said. ``A great day for
him, a great day for everybody.''

For Mr. Marcus, the former publicist who had attended Fred Trump Sr.'s
funeral 21 years earlier, the president's words brought on a sense of
déjà vu. ``It had some parallels with the eulogy he delivered for his
father,'' Mr. Marcus said. Once again, ``it was all about him.''

Advertisement

\protect\hyperlink{after-bottom}{Continue reading the main story}

\hypertarget{site-index}{%
\subsection{Site Index}\label{site-index}}

\hypertarget{site-information-navigation}{%
\subsection{Site Information
Navigation}\label{site-information-navigation}}

\begin{itemize}
\tightlist
\item
  \href{https://help.nytimes.com/hc/en-us/articles/115014792127-Copyright-notice}{©~2020~The
  New York Times Company}
\end{itemize}

\begin{itemize}
\tightlist
\item
  \href{https://www.nytco.com/}{NYTCo}
\item
  \href{https://help.nytimes.com/hc/en-us/articles/115015385887-Contact-Us}{Contact
  Us}
\item
  \href{https://www.nytco.com/careers/}{Work with us}
\item
  \href{https://nytmediakit.com/}{Advertise}
\item
  \href{http://www.tbrandstudio.com/}{T Brand Studio}
\item
  \href{https://www.nytimes.com/privacy/cookie-policy\#how-do-i-manage-trackers}{Your
  Ad Choices}
\item
  \href{https://www.nytimes.com/privacy}{Privacy}
\item
  \href{https://help.nytimes.com/hc/en-us/articles/115014893428-Terms-of-service}{Terms
  of Service}
\item
  \href{https://help.nytimes.com/hc/en-us/articles/115014893968-Terms-of-sale}{Terms
  of Sale}
\item
  \href{https://spiderbites.nytimes.com}{Site Map}
\item
  \href{https://help.nytimes.com/hc/en-us}{Help}
\item
  \href{https://www.nytimes.com/subscription?campaignId=37WXW}{Subscriptions}
\end{itemize}
