Sections

SEARCH

\protect\hyperlink{site-content}{Skip to
content}\protect\hyperlink{site-index}{Skip to site index}

\href{https://myaccount.nytimes.com/auth/login?response_type=cookie\&client_id=vi}{}

\href{https://www.nytimes.com/section/todayspaper}{Today's Paper}

\href{/section/opinion}{Opinion}\textbar{}If Our Masks Could Speak

\url{https://nyti.ms/2X7Offr}

\begin{itemize}
\item
\item
\item
\item
\item
\item
\end{itemize}

Advertisement

\protect\hyperlink{after-top}{Continue reading the main story}

\href{/section/opinion}{Opinion}

Supported by

\protect\hyperlink{after-sponsor}{Continue reading the main story}

\hypertarget{if-our-masks-could-speak}{%
\section{If Our Masks Could Speak}\label{if-our-masks-could-speak}}

Something that's supposed to cover our mouths speaks volumes about how
crazy some people have gotten.

\href{https://www.nytimes.com/by/thomas-l-friedman}{\includegraphics{https://static01.nyt.com/images/2018/04/02/opinion/thomas-l-friedman/thomas-l-friedman-thumbLarge.png}}

By \href{https://www.nytimes.com/by/thomas-l-friedman}{Thomas L.
Friedman}

Opinion Columnist

\begin{itemize}
\item
  July 28, 2020
\item
  \begin{itemize}
  \item
  \item
  \item
  \item
  \item
  \item
  \end{itemize}
\end{itemize}

\includegraphics{https://static01.nyt.com/images/2020/07/28/opinion/28friedmanWeb/28friedmanWeb-articleLarge.jpg?quality=75\&auto=webp\&disable=upscale}

\href{https://www.nytimes.com/es/2020/07/30/espanol/opinion/usar-cubrebocas-politica.html}{Leer
en español}

When people ask me about my mood these days, I tell them that I feel
like I'm a reporter for The Pompeii Daily News in A.D. 79, and I'm
sitting in the foothills of Mount Vesuvius and someone just walked up
and asked, ``Hey, do you feel a rumbling?''

Do I ever.

The summer of 2020 could be remembered as one of those truly important
dates in American history. Everywhere you turn you see parents who don't
know where or if their kids will go to school this fall, renters who
don't know when or if they will be evicted, unemployed who don't know
what if any safety net Congress will put under them, businesses that
don't know how or if they can hold on another day --- and none of us who
know whether we'll be able to vote in November.

That is a lot of hot, molten anxiety building up beneath our economy,
society, schools and city streets --- just waiting to blow the top off
our country --- because we have so failed at managing the coronavirus.
We have 25 percent of all recorded infections in the world, and we're
only four percent of the world's population. In the ultimate irony,
Vietnam, which has a little less than one-third of our population but
has reported only 416 cases and no deaths, is feeling sorry for us.

How did we get so inept?

If, God forbid, America were buried under lava the way Pompeii was and
future archaeologists were to come along and dig it out, I have no doubt
that the artifact they'd dust off and hold up first to answer that huge
question would be a simple item that costs pennies to make and is so
easy to wear: the face mask.

For something that's supposed to cover our mouths it speaks volumes
about how crazy some have gotten. Specifically, that face mask tells how
the world's richest and most scientifically advanced country generated a
cadre of leaders and citizens who made wearing a covering over their
nose and mouth to prevent the spread of a contagion into a
freedom-of-speech issue and cultural marker --- something no other
country in the world did.

There is nothing more demoralizing than this, nothing that set us back
in the fight against Covid-19 further and faster. A society that can
politicize something as simple as a face mask in a pandemic can
politicize anything, can make anything a wedge issue --- physics,
gravity, rainfall, you name it. And a society that politicizes
everything will never realize its full potential in good times or
prevent the worst in bad times.

And that's where we are now. When you compare the sacrifices ---
including the ultimate sacrifice --- that the Greatest Generation of
Americans made to defend their fellow citizens from the scourge of
Nazism with how little some members of today's generations will
sacrifice to defend their fellow Americans from the scourge of Covid-19
--- by just wearing a face mask --- it leaves you speechless.

It's inexcusable. Resisting wearing a mask in a pandemic is nothing more
than selfish, libertarian nonsense masquerading as a comic-book defense
of freedom: ``Don't tread on me, but I can breathe on you.''

And yet for months our president and vice president, and most Republican
governors and their followers, equated resisting mask-wearing with
resisting an infringement on personal freedom, rather than the most
effective and cheapest thing we could do to limit the spread of the
virus and get back to work and our kids back to school.

President Trump's resistance to masks actually had nothing to do with
ideology. It was just his primitive opposition to anything that would
highlight the true health crisis we were in and that therefore might
hurt his re-election.

But Vice President Mike Pence --- always happy to put lipstick on
Trump's piggishness --- dressed up his crude mask-resistance in elegant
constitutional garb. When asked by a reporter at Trump's Tulsa rally a
few weeks ago why the president appeared unconcerned about the absence
of masks and social distancing at his event, Pence solemnly intoned: ``I
want to remind you again, freedom of speech and the right to peaceably
assemble is in the Constitution of the U.S. Even in a health crisis, the
American people don't forfeit our
\href{https://www.esquire.com/news-politics/politics/a32984272/mike-pence-masks-social-distancing-trump-rallies/}{constitutional
rights}.''

What a fraud.

As John Finn, professor emeritus of government at Wesleyan University,
\href{https://theconversation.com/the-constitution-doesnt-have-a-problem-with-mask-mandates-142335}{writing
on TheConversation.com}, noted, ``There are two reasons why mask
mandates don't violate the First Amendment. First, a mask doesn't keep
you from expressing yourself. \ldots{} Additionally, the First
Amendment, like all liberties ensured by the Constitution, is not
absolute. All constitutional rights are subject to the government's
authority to protect the health, safety and welfare of the community.''

A
\href{https://www.bcg.com/publications/2020/why-its-not-too-late-to-contain-the-virus}{study
by the Boston Consulting Group} on which countries not only flattened
the curve of the coronavirus but ``crushed it,'' found that the key to
reopening economies while also containing virus transmission was
``physical distancing, frequent hand-washing and the widespread use of
masks'' --- and the fact that these governments developed detailed
guidelines for all three when it came to workplace settings, schools and
public transportation.

But our future archaeologists would also be right to focus on face
masks, because the early intense resistance by pro-Trump Republican
leaders and faithful to wearing them was the distilled essence of how
far off track today's G.O.P. and its enabling media ecosystem have
drifted. In that sense it was yet another stark reminder that we can't
be at our best as a country --- as we need to be most in a pandemic ---
without a principled conservative party, grounded in science, not just
cultural markers and mindless, kneejerk libertarianism.

We have a way to go.
\href{https://www.forbes.com/sites/jackbrewster/2020/07/24/19-states-still-dont-mandate-masks-18-are-run-by-republican-governors/\#37bbd2e16243}{Forbes
reported last week} that ``of the 19 states that have yet to issue a
mask mandate, 18 are run by Republican governors.''

But let's give a small shout-out to Republican governors Larry Hogan of
Maryland, Mike DeWine of Ohio, Eric Holcomb of Indiana and Kay Ivey of
Alabama, who have been or become pro-mask. It is not only good for their
states' physical health but also the country's political health.

Wearing a mask in this pandemic is a sign of respect for your fellow
citizens and neighbors --- no matter what their race, creed or political
affiliation. Wearing a mask says: ``I'm not just concerned about myself.
I'm concerned about you, too. We are all part of the same community, the
same country and the same struggle to stay healthy.''

A different president would have been urging every American, from the
start of this pandemic, to don a red, white and blue mask. He would have
used such a mask to do double duty --- crush Covid-19 and bring us
together for the long march needed to do so.

As I said, a different president.

\emph{The Times is committed to publishing}
\href{https://www.nytimes.com/2019/01/31/opinion/letters/letters-to-editor-new-york-times-women.html}{\emph{a
diversity of letters}} \emph{to the editor. We'd like to hear what you
think about this or any of our articles. Here are some}
\href{https://help.nytimes.com/hc/en-us/articles/115014925288-How-to-submit-a-letter-to-the-editor}{\emph{tips}}\emph{.
And here's our email:}
\href{mailto:letters@nytimes.com}{\emph{letters@nytimes.com}}\emph{.}

\emph{Follow The New York Times Opinion section on}
\href{https://www.facebook.com/nytopinion}{\emph{Facebook}}\emph{,}
\href{http://twitter.com/NYTOpinion}{\emph{Twitter (@NYTopinion)}}
\emph{and}
\href{https://www.instagram.com/nytopinion/}{\emph{Instagram}}\emph{.}

Advertisement

\protect\hyperlink{after-bottom}{Continue reading the main story}

\hypertarget{site-index}{%
\subsection{Site Index}\label{site-index}}

\hypertarget{site-information-navigation}{%
\subsection{Site Information
Navigation}\label{site-information-navigation}}

\begin{itemize}
\tightlist
\item
  \href{https://help.nytimes.com/hc/en-us/articles/115014792127-Copyright-notice}{©~2020~The
  New York Times Company}
\end{itemize}

\begin{itemize}
\tightlist
\item
  \href{https://www.nytco.com/}{NYTCo}
\item
  \href{https://help.nytimes.com/hc/en-us/articles/115015385887-Contact-Us}{Contact
  Us}
\item
  \href{https://www.nytco.com/careers/}{Work with us}
\item
  \href{https://nytmediakit.com/}{Advertise}
\item
  \href{http://www.tbrandstudio.com/}{T Brand Studio}
\item
  \href{https://www.nytimes.com/privacy/cookie-policy\#how-do-i-manage-trackers}{Your
  Ad Choices}
\item
  \href{https://www.nytimes.com/privacy}{Privacy}
\item
  \href{https://help.nytimes.com/hc/en-us/articles/115014893428-Terms-of-service}{Terms
  of Service}
\item
  \href{https://help.nytimes.com/hc/en-us/articles/115014893968-Terms-of-sale}{Terms
  of Sale}
\item
  \href{https://spiderbites.nytimes.com}{Site Map}
\item
  \href{https://help.nytimes.com/hc/en-us}{Help}
\item
  \href{https://www.nytimes.com/subscription?campaignId=37WXW}{Subscriptions}
\end{itemize}
