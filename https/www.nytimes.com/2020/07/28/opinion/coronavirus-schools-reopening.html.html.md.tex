Sections

SEARCH

\protect\hyperlink{site-content}{Skip to
content}\protect\hyperlink{site-index}{Skip to site index}

\href{https://myaccount.nytimes.com/auth/login?response_type=cookie\&client_id=vi}{}

\href{https://www.nytimes.com/section/todayspaper}{Today's Paper}

\href{/section/opinion}{Opinion}\textbar{}What Will Schools Do When a
Teacher Gets Covid-19?

\url{https://nyti.ms/2X4ng4d}

\begin{itemize}
\item
\item
\item
\item
\item
\item
\end{itemize}

Advertisement

\protect\hyperlink{after-top}{Continue reading the main story}

\href{/section/opinion}{Opinion}

Supported by

\protect\hyperlink{after-sponsor}{Continue reading the main story}

\hypertarget{what-will-schools-do-when-a-teacher-gets-covid-19}{%
\section{What Will Schools Do When a Teacher Gets
Covid-19?}\label{what-will-schools-do-when-a-teacher-gets-covid-19}}

Cases are inevitable. Schools need to plan now.

By Emily Oster

Dr. Oster is the author of ``Cribsheet: A Data-Driven Guide to Better,
More Relaxed Parenting, From Birth to Preschool.''

\begin{itemize}
\item
  July 28, 2020
\item
  \begin{itemize}
  \item
  \item
  \item
  \item
  \item
  \item
  \end{itemize}
\end{itemize}

\includegraphics{https://static01.nyt.com/images/2020/07/28/opinion/28oster/28oster-articleLarge.jpg?quality=75\&auto=webp\&disable=upscale}

The logistics of reopening schools are daunting. Plans are full of
details about which days kids will be eligible for, and pages and pages
on preventing students and staffs from getting sick. What kind of limits
will be placed on class sizes? What kind of cleaning? Will there be
symptom checks or temperature screens? Masks for everyone or just
adults?

These plans are important and necessary. But there is an issue that we
aren't talking enough about: What happens when there is a Covid-19 case
in a school? The Centers for Disease Control and Prevention released
\href{https://www.cdc.gov/coronavirus/2019-ncov/community/schools-childcare/prepare-safe-return.html}{its
first guidelines} on this topic last week, a long-overdue step toward
getting schools to take this question seriously.

The instinct, I think, is to say we are working to make sure that
doesn't happen, and of course that is the goal. But that goal is
unrealistic. Even if schools are successful at ensuring there is no
Covid-19 \emph{spread} in schools at all, there will still be cases
arising from the community.

When we look at data from places with open schools ---
\href{https://www.bloomberg.com/news/articles/2020-07-19/covid-s-spread-in-schools-is-questioned-in-latest-nordic-study}{Sweden,
for example} --- they are encouraging in showing that teaching is not a
high-risk job. But that means that teachers are infected at the same
rate as the rest of the community. Put bluntly: If 5 percent of adults
in a community have Covid-19, we expect 5 percent of school employees to
have it even if they are at no greater risk. This problem is largest in
places that currently have high community spread, but it is a concern
virtually anywhere.

Bottom line: When schools open, there will be cases. It is necessary to
have a concrete plan for what will happen when this occurs.

It is worth pausing for a moment on why there is a reluctance to discuss
this. In my view, it is because those who want to open are afraid that
if they acknowledge there will be cases in schools, those who oppose
opening will use that to argue schools are unsafe. Indeed, there
\href{https://www.facebook.com/refusetoreturn/}{are movements in
California} and elsewhere saying that teachers should not return to the
classroom until there are \emph{no} new Covid-19 cases in the school
community for 14 days. This is effectively a mandate to not open at all,
possibly ever.

However, this concern should lead us to \emph{more} transparency rather
than less. Is it really better to trick people into opening, only to
face panic and anger when there is a case? If we face the reality now,
we are better able to prepare both emotionally and practically for what
is inevitable.

Once you acknowledge the reality of cases in schools, it is clear that
schools need a plan. The first part of this plan should recognize that
schools should not open in person until cases of the virus in the
surrounding areas are low. Putting a precise number on this is
difficult, but at a minimum places that have locked down except for
essential services should not open schools.

But for areas with low incidence, you still need a plan. And this plan
needs at least two parts.

First, there needs to be what I'd call a micro plan: What happens when a
single student or teacher in a classroom tests positive? Of course the
affected person will need to remain home until cleared for a return to
school. But what about the rest of the classroom, the rest of the floor,
the rest of the school?

\href{https://www.cdc.gov/coronavirus/2019-ncov/community/schools-childcare/schools.html\#anchor_1589932092921}{C.D.C.
guidelines} are fairly clear on what to do with the sick individual and
what type of cleaning should be done. The
\href{http://cdc.gov/coronavirus/2019-ncov/community/schools-childcare/prepare-safe-return.html}{guidance
on the overall school approach} is less specific. It suggests schools
probably do not need to close for a single case, but beyond that, it
pushes the decision largely onto schools and local health departments.
It suggests a host of factors to consider --- community transmission
levels, contact levels and so on --- but does not draw any bright lines.
Even the suggestion of not closing after a single case is not
definitive.

Schools are left to choose their own approaches. One extreme is to
basically do nothing --- just tell the sick student or teacher to stay
home. The other extreme is to shut down the school for each case. If a
school plans to do the latter, it may as well not open at all. There is
an intermediate option: Close the classroom for a few days, clean it and
reopen.

It isn't obvious to me what the optimal micro plan is, although I'd be
inclined to a middle road where the infected person is out of school and
the rest of the class is encouraged to check for symptoms closely.

The school also needs a macro plan. Let's say you will keep the school
open even if there are some cases: Is there a point where an outbreak is
large enough that you \emph{would} close the school? Again, guidelines
are vague on this. The C.D.C. doesn't make any concrete statements.

We might look to places that have had open schools for evidence of what
worked. Many European countries have opened schools, largely
successfully. They did so taking various approaches to closures. In
Germany, classmates and teachers (but not the rest of the school)
\href{https://www.sciencemag.org/news/2020/07/school-openings-across-globe-suggest-ways-keep-coronavirus-bay-despite-outbreaks}{were
isolated} for two weeks after a reported case. Taiwan, apparently,
planned to close schools for two or more cases but as of early this
month had yet to face that. Israel, which has had probably the most
fraught reopening, closed schools for every case. This has resulted in a
very large number of school closures.

The evidence from other countries suggests schools could take a variety
of approaches. My view is that the most important thing is that they are
explicit about which approach they will take. Not just in broad strokes,
but in detail. ** And more than that, in advance.

For parents, knowing the chance that your school will be shut down at a
moment's notice is key in the decision about whether your children
return or not. I am as eager as anyone to have my kids back in school,
but if the school will shut down for two weeks after each case, I may
prefer to embrace the inevitable and plan for it rather than whiplash
back and forth. This planning could involve identifying backup care,
talking to other parents about how to maintain social time during a
school closing or even deciding that we should opt for an entirely
online experience from the start.

Schools have a similar incentive. The more shutdown you plan for, the
more robust the online learning plan needs to be.

Schools need to face reality now, make a plan and then stick to it.

\href{https://www.brown.edu/research/projects/oster/}{Emily Oster}, an
economics professor at Brown, is the author of ``Expecting Better'' and
``Cribsheet: A Data-Driven Guide to Better, More Relaxed Parenting, From
Birth to Preschool.''

\emph{The Times is committed to publishing}
\href{https://www.nytimes.com/2019/01/31/opinion/letters/letters-to-editor-new-york-times-women.html}{\emph{a
diversity of letters}} \emph{to the editor. We'd like to hear what you
think about this or any of our articles. Here are some}
\href{https://help.nytimes.com/hc/en-us/articles/115014925288-How-to-submit-a-letter-to-the-editor}{\emph{tips}}\emph{.
And here's our email:}
\href{mailto:letters@nytimes.com}{\emph{letters@nytimes.com}}\emph{.}

\emph{Follow The New York Times Opinion section on}
\href{https://www.facebook.com/nytopinion}{\emph{Facebook}}\emph{,}
\href{http://twitter.com/NYTOpinion}{\emph{Twitter (@NYTopinion)}}
\emph{and}
\href{https://www.instagram.com/nytopinion/}{\emph{Instagram}}\emph{.}

Advertisement

\protect\hyperlink{after-bottom}{Continue reading the main story}

\hypertarget{site-index}{%
\subsection{Site Index}\label{site-index}}

\hypertarget{site-information-navigation}{%
\subsection{Site Information
Navigation}\label{site-information-navigation}}

\begin{itemize}
\tightlist
\item
  \href{https://help.nytimes.com/hc/en-us/articles/115014792127-Copyright-notice}{©~2020~The
  New York Times Company}
\end{itemize}

\begin{itemize}
\tightlist
\item
  \href{https://www.nytco.com/}{NYTCo}
\item
  \href{https://help.nytimes.com/hc/en-us/articles/115015385887-Contact-Us}{Contact
  Us}
\item
  \href{https://www.nytco.com/careers/}{Work with us}
\item
  \href{https://nytmediakit.com/}{Advertise}
\item
  \href{http://www.tbrandstudio.com/}{T Brand Studio}
\item
  \href{https://www.nytimes.com/privacy/cookie-policy\#how-do-i-manage-trackers}{Your
  Ad Choices}
\item
  \href{https://www.nytimes.com/privacy}{Privacy}
\item
  \href{https://help.nytimes.com/hc/en-us/articles/115014893428-Terms-of-service}{Terms
  of Service}
\item
  \href{https://help.nytimes.com/hc/en-us/articles/115014893968-Terms-of-sale}{Terms
  of Sale}
\item
  \href{https://spiderbites.nytimes.com}{Site Map}
\item
  \href{https://help.nytimes.com/hc/en-us}{Help}
\item
  \href{https://www.nytimes.com/subscription?campaignId=37WXW}{Subscriptions}
\end{itemize}
