Sections

SEARCH

\protect\hyperlink{site-content}{Skip to
content}\protect\hyperlink{site-index}{Skip to site index}

\href{https://www.nytimes.com/section/opinion/sunday}{Sunday Review}

\href{https://myaccount.nytimes.com/auth/login?response_type=cookie\&client_id=vi}{}

\href{https://www.nytimes.com/section/todayspaper}{Today's Paper}

\href{/section/opinion/sunday}{Sunday Review}\textbar{}Donald Trump Is
the Best Ever President in the History of the Cosmos

\url{https://nyti.ms/30Kpsil}

\begin{itemize}
\item
\item
\item
\item
\item
\item
\end{itemize}

Advertisement

\protect\hyperlink{after-top}{Continue reading the main story}

\href{/section/opinion}{Opinion}

Supported by

\protect\hyperlink{after-sponsor}{Continue reading the main story}

\hypertarget{donald-trump-is-the-best-ever-president-in-the-history-of-the-cosmos}{%
\section{Donald Trump Is the Best Ever President in the History of the
Cosmos}\label{donald-trump-is-the-best-ever-president-in-the-history-of-the-cosmos}}

That's no more fantastical than the rest of his re-election campaign.

\href{https://www.nytimes.com/by/frank-bruni}{\includegraphics{https://static01.nyt.com/images/2018/04/03/opinion/frank-bruni/frank-bruni-thumbLarge.png}}

By \href{https://www.nytimes.com/by/frank-bruni}{Frank Bruni}

Opinion Columnist

\begin{itemize}
\item
  July 25, 2020
\item
  \begin{itemize}
  \item
  \item
  \item
  \item
  \item
  \item
  \end{itemize}
\end{itemize}

\includegraphics{https://static01.nyt.com/images/2020/07/26/opinion/sunday/26bruni/26bruni-articleLarge.jpg?quality=75\&auto=webp\&disable=upscale}

\hypertarget{listen-to-this-op-ed}{%
\subsubsection{Listen to This Op-Ed}\label{listen-to-this-op-ed}}

Audio Recording by Audm

\emph{To hear more audio stories from publishers like The New York
Times,
download}\href{https://www.audm.com/?utm_source=nytmag\&utm_medium=embed\&utm_campaign=left_behind_draper}{**}\href{https://www.audm.com/?utm_source=nytopinion\&utm_medium=embed\&utm_campaign=trump_history_cosmos}{\emph{Audm
for iPhone or Android}}\emph{.}

It's no longer interesting, or particularly newsworthy, to point out
that Donald Trump lies. It stopped being interesting a long time ago. He
lied en route to the presidency. He lied about the crowd at his
inauguration. His speech itself was one big lie. And the falsehoods only
metastasized from there.

Why? We've covered that, too, most recently in all the chatter about
``Too Much and Never Enough,'' by Mary Trump, who is not only his niece
but also a clinical psychologist. He lies because he grew up among
liars. He lies because hyperbole and hooey buoy his fragile ego. He lies
because he is practiced at it, is habituated to it and never seems to
pay much of a price for it.

What intrigues me is that last part: the impunity. I want to understand
how he has gotten away with all of the lying, because I'm desperate to
know whether he'll continue to.

That's the question at the heart of his re-election bid, because his
strategy isn't really ``law and order'' or racism or a demonization of
liberals as monument-phobic wackadoodles or a
\href{https://www.nytimes.com/2020/05/17/opinion/trump-biden-age.html}{diminution
of Joe Biden} as a doddering wreck. All of those gambits are there, but
they spring from and burble back to a larger, overarching scheme. His
strategy is fiction. His strategy is lies.

Can he sell enough Americans on the make-believe that he really cares
about the quality of life in cities and is dispatching federal officers
as a constructive measure rather than a provocative one, in a flash of
empathy versus a fit of vanity? He gave himself away a few days ago when
\href{https://www.whitehouse.gov/briefings-statements/remarks-president-trump-operation-legend-combatting-violent-crime-american-cities/}{he
punctuated a mention} of ``the wonderful people of Chicago'' with the
needless notation that it's ``a city I know very well.'' Everything
Trump says is self-referential, and everything he does is
self-reverential.

Can he feed voters the fantasy that his actions in the infancy of this
pandemic saved lives and that our country's world-leading death toll and
un-flattened curve are more figment than fact or at least more fluke
than indictment? Can he convincingly
\href{https://www.newsweek.com/donald-trump-mask-timeline-avoid-patriotic-1519300}{don
the mask} of a longtime evangelist for masks?

His recent
\href{https://nymag.com/intelligencer/2020/07/trump-interview-chris-wallace-cognitive-dementia-fox-news.html}{interview
with Chris Wallace} of Fox News was a trial run of this and \ldots{}
wow. Up was down. Black was white. A superficial check of
\href{https://www.washingtonpost.com/politics/trump-bragging-cognitive-test-dementia/2020/07/22/6578e826-cb65-11ea-91f1-28aca4d833a0_story.html}{his
cognitive coherence} was a profound spelunking of his cerebral glory.

He claimed that Joe Biden had pledged to defund --- no, \emph{abolish}
--- the police, when Biden had
\href{https://www.factcheck.org/2020/07/trumps-false-recurring-claim-about-bidens-stance-on-police/}{done
nothing of the kind}. He boasted that America's management of this
pandemic made us ``the envy of the world,'' when in fact we're so
densely diseased that we're
\href{https://www.nytimes.com/article/eu-travel-ban-explained-usa.html}{barred
from entering most of Europe}. Oh, and he's cruising toward four more
years: All of those pollsters who predict otherwise are incompetent
fabulists. (Talk about projection.)

Then there are the Trump campaign's ads, which are ``Veep''-grade
caricatures of the usual fakery, not to mention paragons of incompetence
in their own regard. One that appeared on Facebook in early July said,
``WE WILL PROTECT THIS'' --- just like that, in URGENT CAPITAL LETTERS
--- beneath a picture of a statue of Jesus. But Trump won't be
protecting that statue, because, as eagle-eyed observers noticed,
\href{https://www.thedailybeast.com/trump-camp-vows-to-protect-brazils-most-iconic-statue-from-left-wing-mobs}{it
was the Christ the Redeemer monument} that looms over Rio de Janeiro.

Another Facebook ad a few weeks later comprised two side-by-side
pictures. Under an image of Trump were the words ``Public Safety.''
Under a separate image, of a police officer crumpled on the ground amid
protesters, were ``Chaos \& Violence.''

Scary! But, again, foreign. The scene wasn't Portland or Minneapolis or
Washington or Chicago circa 2020, although that was the obvious
suggestion. The picture, it turns out, was taken in Ukraine. \emph{Six
years ago}. For a more complete and very funny deconstruction of its
infelicity, \href{https://thebulwark.com/trumps-new-ad-is-amazing/}{read
Jonathan Last's riff in The Bulwark}.

The Trump campaign's television commercials, meanwhile, have
\href{https://www.nytimes.com/2020/07/21/us/politics/trump-portland-federal-agents.html}{painted
a dystopia} of rampant criminality in Democratic-controlled metropolises
where the police no longer function or exist. One
\href{https://www.nytimes.com/2020/07/21/us/politics/trump-portland-federal-agents.html}{shows
an elderly woman being attacked} by a burglar as she listens to a 911
recording that tells her to ``leave a message.''

If this is Trump's tenor in July, just imagine October. By the time he's
done,
\href{https://www.cnn.com/2018/11/01/politics/willie-horton-ad-1988-explainer-trnd/index.html}{Willie
Horton} will look like Peter Pan.

It's beyond ludicrous. But is it too much? I once would have answered an
emphatic yes. Now I just don't know.

Every president's election illuminates the moment in which it occurs,
and Trump's told us something important --- and terrifying --- about our
relationship with the truth. He relied like no candidate before him on a
new infrastructure of misinformation and disinformation, tweeting toward
Bethlehem while his allies made Mark Zuckerberg their stooge. If you're
peddling fiction, Twitter and Facebook are the right bazaars.

But they're hardly the only ones. The web (how aptly named) has fostered
the proliferation of ``news'' sites with partisan and micro-partisan
agendas. They amount to
\href{https://www.nytimes.com/2018/10/30/opinion/internet-violence-hate-prejudice.html}{flourishing
ecosystems for alternate realities}. Many Americans believe that Trump
is an underappreciated martyr because they marinate in selective,
manipulated and outright fraudulent factoids. And Trump and his minions
have really figured out how to slather on the marinade.

When Robert Mueller released the conclusions of his investigation into
the Trump campaign's ties to Russia, everyone focused on its second
section, about Trump, when the first was at least as important. It
documented the
\href{https://www.pbs.org/newshour/show/inside-the-mueller-report-a-sophisticated-russian-interference-campaign}{extent
and ingenuity of Russia's attempts to pervert the election}. But even
many of the people who paid it heed missed the point, which wasn't
Russia's nefariousness. It was the process's corruptibility. It was the
power of lies in a world gone digital.

As for the power of a liar, well, that's what Trump is testing. He got
away with
\href{https://www.nytimes.com/2016/07/17/us/politics/donald-trump-business.html}{lies
in his business career} because he chose professional avenues paved with
deception and crowded with con men. Plus he had --- and still has --- a
special talent for treating drivel as gospel, as long as it's tumbling
from his lips. That's the great advantage of the truly amoral: They're
liberated from any tug of conscience, so there's no suspicious hesitancy
in their words, no revelatory panic in their eyes. Damn the verities and
full steam ahead.

He got away with lies in 2016 because of social media, because show
business and politics had finally fused to the point where one was
indistinguishable from the other, and because many Americans had grown
so skeptical of traditional candidates that an utterly untraditional one
seemed more trustworthy on some level. Trump was the diet that hadn't
yet failed them. They were ready to believe.

But to believe now is to ignore the receipts. About 150,000 Americans
have died from Covid-19. Tens of millions have tumbled into financial
ruin or are on the precipice of it. Racial tensions are at a palpable
boil. And Trump keeps having to double back to correct his predictions
and retrace his missteps. Charlotte, Jacksonville, Charlotte: I've lost
track of
\href{https://www.cnn.com/2020/07/23/politics/rnc-jacksonville/index.html}{where
the Republicans are convening next month} and of who's on board, though
I remain primed for Trump's remarks. He alone can fictionalize it.

From now until Nov. 3, Trump will take the grand inventions that attend
any presidential candidate's campaign to a newly grandiose level,
signaled by
\href{https://www.whitehouse.gov/briefings-statements/remarks-president-trump-press-briefing-072220/}{his
insistence a few days ago} that he'd ``done more for Black Americans
than anybody with the possible exception of Abraham Lincoln.'' I love
that ``possible.'' Trump, Lincoln: It's a jump ball, really.

So while this election is indeed a contest between two men with two
visions, it's also something else. It's the tallest tale Trump has ever
scaled, the greatest story ever told. It's a referendum on the reaches
of his persuasion. It's a judgment of the depths of Americans'
gullibility.

Have we cut the cord to reality? Then Trump has a chance. And America
hasn't a prayer.

\emph{I invite you to sign up for my free}
\href{https://www.nytimes.com/newsletters/frank-bruni}{\emph{weekly
email newsletter}}\emph{. You can follow me on Twitter
(}\href{https://twitter.com/FrankBruni}{\emph{@FrankBruni}}\emph{).}

\emph{Listen to}
\href{https://www.nytimes.com/column/the-argument}{\emph{``The
Argument'' podcast}} \emph{every Thursday morning, with Ross Douthat,
Michelle Goldberg and me.}

Advertisement

\protect\hyperlink{after-bottom}{Continue reading the main story}

\hypertarget{site-index}{%
\subsection{Site Index}\label{site-index}}

\hypertarget{site-information-navigation}{%
\subsection{Site Information
Navigation}\label{site-information-navigation}}

\begin{itemize}
\tightlist
\item
  \href{https://help.nytimes.com/hc/en-us/articles/115014792127-Copyright-notice}{©~2020~The
  New York Times Company}
\end{itemize}

\begin{itemize}
\tightlist
\item
  \href{https://www.nytco.com/}{NYTCo}
\item
  \href{https://help.nytimes.com/hc/en-us/articles/115015385887-Contact-Us}{Contact
  Us}
\item
  \href{https://www.nytco.com/careers/}{Work with us}
\item
  \href{https://nytmediakit.com/}{Advertise}
\item
  \href{http://www.tbrandstudio.com/}{T Brand Studio}
\item
  \href{https://www.nytimes.com/privacy/cookie-policy\#how-do-i-manage-trackers}{Your
  Ad Choices}
\item
  \href{https://www.nytimes.com/privacy}{Privacy}
\item
  \href{https://help.nytimes.com/hc/en-us/articles/115014893428-Terms-of-service}{Terms
  of Service}
\item
  \href{https://help.nytimes.com/hc/en-us/articles/115014893968-Terms-of-sale}{Terms
  of Sale}
\item
  \href{https://spiderbites.nytimes.com}{Site Map}
\item
  \href{https://help.nytimes.com/hc/en-us}{Help}
\item
  \href{https://www.nytimes.com/subscription?campaignId=37WXW}{Subscriptions}
\end{itemize}
