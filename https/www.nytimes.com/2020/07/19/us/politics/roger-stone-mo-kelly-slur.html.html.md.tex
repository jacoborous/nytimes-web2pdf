Sections

SEARCH

\protect\hyperlink{site-content}{Skip to
content}\protect\hyperlink{site-index}{Skip to site index}

\href{https://www.nytimes.com/section/politics}{Politics}

\href{https://myaccount.nytimes.com/auth/login?response_type=cookie\&client_id=vi}{}

\href{https://www.nytimes.com/section/todayspaper}{Today's Paper}

\href{/section/politics}{Politics}\textbar{}Roger Stone Denies Using
Racial Slur on Radio Show

\url{https://nyti.ms/3h76DfY}

\begin{itemize}
\item
\item
\item
\item
\item
\end{itemize}

Advertisement

\protect\hyperlink{after-top}{Continue reading the main story}

Supported by

\protect\hyperlink{after-sponsor}{Continue reading the main story}

\hypertarget{roger-stone-denies-using-racial-slur-on-radio-show}{%
\section{Roger Stone Denies Using Racial Slur on Radio
Show}\label{roger-stone-denies-using-racial-slur-on-radio-show}}

Though the audio suggests otherwise, Mr. Stone said he did not use a
slur in referring to his interviewer, who is Black. He also contended
that the word was not offensive.

\includegraphics{https://static01.nyt.com/images/2020/08/17/us/politics/17xp-stone-radio/merlin_174523263_7eab793e-8087-4ae0-a2c7-cc19582c5424-articleLarge.jpg?quality=75\&auto=webp\&disable=upscale}

By \href{https://www.nytimes.com/by/aimee-ortiz}{Aimee Ortiz} and
\href{https://www.nytimes.com/by/marie-fazio}{Marie Fazio}

\begin{itemize}
\item
  July 19, 2020
\item
  \begin{itemize}
  \item
  \item
  \item
  \item
  \item
  \end{itemize}
\end{itemize}

Roger Stone, the political operative who was spared a prison sentence
this month by his friend President Trump, denied on Sunday that he had
uttered a racial slur on a radio show the night before, calling the
accusation a ``smear'' while also contending the word was not offensive.

During a
\href{https://www.spreaker.com/user/kfiam640/roger-stone-podcast-2020}{live
interview} on ``The Mo'Kelly Show'' on Saturday night, the host, Morris
W. O'Kelly, who is Black, questioned the role that Mr. Stone's
relationship and proximity to the president played in the commutation of
his sentence.

Mr. O'Kelly said: ``There are thousands of people treated unfairly
daily. How your number just happened to come up in the lottery --- I am
guessing it was more than just luck, Roger, right?''

Mr. Stone, who was speaking by phone, responded by muttering words that
sounded like ``arguing with this Negro''; the beginning of his sentence
was hard to hear. It sounded as if Mr. Stone was not speaking directly
into the phone, but rather to himself or to someone in the room with
him.

When Mr. O'Kelly asked him to repeat what he said, Mr. Stone let out a
sigh, then remained silent for almost 40 seconds. Acting as if the
connection had been severed, Mr. Stone vehemently denied that he used
the slur.

``I did not --- you're out of your mind,'' Mr. Stone told the host.

\includegraphics{https://static01.nyt.com/images/2020/07/18/multimedia/mo-kelly-philly-radio/mo-kelly-philly-radio-articleLarge.jpg?quality=75\&auto=webp\&disable=upscale}

On Sunday, in a statement sent by text message to The New York Times,
Mr. Stone at various points appeared to acknowledge the slur had been
used, blamed technical difficulties on the show's part, denied he said
the word and then argued it was not offensive.

``Somebody can very clearly be heard using the alleged epitaph after he
cut my sound feed off three times,'' he wrote in the text message,
apparently meaning ``epithet.''

Saying he supported affirmative action and opposed the war on drugs, Mr.
Stone wrote that he was not racist and that the episode was ``a smear
designed to boost'' Mr. O'Kelly's ratings.

He also wrote: ``Mr. O'Kelly needs a good peroxide cleaning of the wax
in his ears because at no time did I call him a negro. That said, Mr.
O'Kelly needs to spend a little more time studying black history and
institutions.'' The word, he continued, ``is far from a slur.''

The word was commonly used to refer to Black Americans through part of
the 1960s, but for decades it has been considered offensive. Mr. Stone
cited the championing of the term by the sociologist and N.A.A.C.P.
co-founder W.E.B. DuBois, and the continued use of the word by the
United Negro College Fund.

Mr. DuBois died in 1963. The charity retains its name but has
\href{https://www.nytimes.com/2008/01/17/business/media/17adco.html}{rebranded
itself as U.N.C.F. to de-emphasize the word}.

Asked to respond to Mr. Stone's statement on Sunday, Mr. O'Kelly said
there had been no technical issues, and that the audio was clear.

``I wish Mr. Stone would choose one explanation and stick with it,'' Mr.
O'Kelly said. ``The audio has not changed, neither should his
explanation.''

\href{https://www.mediamatters.org/donald-trump/trump-ally-roger-stones-scrubbed-tweets-stupid-negro-fat-negro-muff-diver-elitist-cnt}{Mr.
Stone has been accused} of using this kind of language in the past,
according to Media Matters for America, a liberal-leaning media
watchdog, which noted in 2016 that Mr. Stone had scrubbed his Twitter
account of inappropriate posts.

On July 10, days before Mr. Stone was set to report to prison, Mr. Trump
commuted
\href{https://www.nytimes.com/2020/07/10/us/politics/trump-roger-stone-clemency.html}{Mr.
Stone's sentence}. Mr. Stone had been sentenced to a 40-month term for
\href{https://www.nytimes.com/2019/11/15/us/politics/roger-stone-trial-guilty.html}{seven
felony crimes} relating to obstruction of a congressional investigation
into Mr. Trump's 2016 campaign and possible ties to Russia.

``The Mo'Kelly Show'' is broadcast on Saturday and Sunday nights on
KFI-AM (640) in Los Angeles and on iHeartRadio. The interview on
Saturday was the second time Mr. O'Kelly had spoken with Mr. Stone.

Mr. O'Kelly told The Times that he had not invited Mr. Stone on the show
to provoke or goad him. He said he was ``disappointed and dismayed that,
in 2020, that's where we are.''

``It's the diet version of the N-word, but as an African-American man,
it's something I deal with pretty frequently,'' Mr. O'Kelly said. ``If
there's a takeaway from the conversation, it is that Roger Stone gave an
unvarnished look into what is in the heart of many Americans today.''

Advertisement

\protect\hyperlink{after-bottom}{Continue reading the main story}

\hypertarget{site-index}{%
\subsection{Site Index}\label{site-index}}

\hypertarget{site-information-navigation}{%
\subsection{Site Information
Navigation}\label{site-information-navigation}}

\begin{itemize}
\tightlist
\item
  \href{https://help.nytimes.com/hc/en-us/articles/115014792127-Copyright-notice}{©~2020~The
  New York Times Company}
\end{itemize}

\begin{itemize}
\tightlist
\item
  \href{https://www.nytco.com/}{NYTCo}
\item
  \href{https://help.nytimes.com/hc/en-us/articles/115015385887-Contact-Us}{Contact
  Us}
\item
  \href{https://www.nytco.com/careers/}{Work with us}
\item
  \href{https://nytmediakit.com/}{Advertise}
\item
  \href{http://www.tbrandstudio.com/}{T Brand Studio}
\item
  \href{https://www.nytimes.com/privacy/cookie-policy\#how-do-i-manage-trackers}{Your
  Ad Choices}
\item
  \href{https://www.nytimes.com/privacy}{Privacy}
\item
  \href{https://help.nytimes.com/hc/en-us/articles/115014893428-Terms-of-service}{Terms
  of Service}
\item
  \href{https://help.nytimes.com/hc/en-us/articles/115014893968-Terms-of-sale}{Terms
  of Sale}
\item
  \href{https://spiderbites.nytimes.com}{Site Map}
\item
  \href{https://help.nytimes.com/hc/en-us}{Help}
\item
  \href{https://www.nytimes.com/subscription?campaignId=37WXW}{Subscriptions}
\end{itemize}
