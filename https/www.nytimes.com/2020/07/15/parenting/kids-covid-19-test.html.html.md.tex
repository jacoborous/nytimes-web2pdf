Sections

SEARCH

\protect\hyperlink{site-content}{Skip to
content}\protect\hyperlink{site-index}{Skip to site index}

\href{https://www.nytimes.com/section/parenting}{Parenting}

\href{https://myaccount.nytimes.com/auth/login?response_type=cookie\&client_id=vi}{}

\href{https://www.nytimes.com/section/todayspaper}{Today's Paper}

\href{/section/parenting}{Parenting}\textbar{}Coronavirus Tests Can Be
Scary for Kids. Here's How to Make Them Easier.

\url{https://nyti.ms/3h0g9RP}

\begin{itemize}
\item
\item
\item
\item
\item
\end{itemize}

\href{https://www.nytimes.com/news-event/coronavirus?action=click\&pgtype=Article\&state=default\&region=TOP_BANNER\&context=storylines_menu}{The
Coronavirus Outbreak}

\begin{itemize}
\tightlist
\item
  live\href{https://www.nytimes.com/2020/08/01/world/coronavirus-covid-19.html?action=click\&pgtype=Article\&state=default\&region=TOP_BANNER\&context=storylines_menu}{Latest
  Updates}
\item
  \href{https://www.nytimes.com/interactive/2020/us/coronavirus-us-cases.html?action=click\&pgtype=Article\&state=default\&region=TOP_BANNER\&context=storylines_menu}{Maps
  and Cases}
\item
  \href{https://www.nytimes.com/interactive/2020/science/coronavirus-vaccine-tracker.html?action=click\&pgtype=Article\&state=default\&region=TOP_BANNER\&context=storylines_menu}{Vaccine
  Tracker}
\item
  \href{https://www.nytimes.com/interactive/2020/07/29/us/schools-reopening-coronavirus.html?action=click\&pgtype=Article\&state=default\&region=TOP_BANNER\&context=storylines_menu}{What
  School May Look Like}
\item
  \href{https://www.nytimes.com/live/2020/07/31/business/stock-market-today-coronavirus?action=click\&pgtype=Article\&state=default\&region=TOP_BANNER\&context=storylines_menu}{Economy}
\end{itemize}

Advertisement

\protect\hyperlink{after-top}{Continue reading the main story}

Supported by

\protect\hyperlink{after-sponsor}{Continue reading the main story}

\hypertarget{coronavirus-tests-can-be-scary-for-kids-heres-how-to-make-them-easier}{%
\section{Coronavirus Tests Can Be Scary for Kids. Here's How to Make
Them
Easier.}\label{coronavirus-tests-can-be-scary-for-kids-heres-how-to-make-them-easier}}

A nasal swab is invasive and uncomfortable for anyone. For children,
knowing what to expect can help ease the anxiety.

\includegraphics{https://static01.nyt.com/images/2020/07/14/multimedia/14parenting-medical-tests/merlin_172021248_7fd69a07-a257-4e57-a492-db047f5db08f-articleLarge.jpg?quality=75\&auto=webp\&disable=upscale}

By Holly Rosenkrantz

\begin{itemize}
\item
  Published July 15, 2020Updated July 17, 2020
\item
  \begin{itemize}
  \item
  \item
  \item
  \item
  \item
  \end{itemize}
\end{itemize}

Cynthia Jaeger was packing up the family minivan for the two-hour trip
from her house in Richmond, Va., to Washington, D.C., when her
8-year-old daughter began having an anxiety attack. Zoey, Jaeger's
middle child, has a rare neurological condition, and she needed to go to
Children's National Hospital for chemotherapy and intensive cognitive
testing.

A visit to the hospital didn't freak Zoey out --- she was a sturdy
patient who was accustomed to medical procedures. But after Jaeger told
her daughter she had to get a coronavirus nasal test when they got to
the hospital, Zoey refused to get in the minivan. Jaeger coaxed her into
going, and ultimately had to restrain her daughter in her lap while a
nurse reached into the van and swabbed Zoey's nose.

``It was such a nightmare. I was crying, Zoey was crying. It was just a
mess, there's got to be a better way,'' Jaeger said. ``She was
traumatized.''

Testing children for coronavirus is becoming more prevalent as states
reopen, posing challenges for parents, caregivers and medical providers
in administering the test. Most hospitals now require patients to get a
coronavirus nasal swab before any surgeries and procedures. Some summer
camps and athletic programs are mandating children get tested. And day
care centers and schools are considering imposing testing requirements
for the fall.

``With more availability of testing, with schools looking at different
models for opening, these tests are going to be a very common experience
for children, like getting a vaccination,'' said Kim Stephens, president
of the Association of Child Life Professionals, an organization that
represents workers who help kids handle the anxiety associated with
hospital visits.

\hypertarget{latest-updates-global-coronavirus-outbreak}{%
\section{\texorpdfstring{\href{https://www.nytimes.com/2020/08/01/world/coronavirus-covid-19.html?action=click\&pgtype=Article\&state=default\&region=MAIN_CONTENT_1\&context=storylines_live_updates}{Latest
Updates: Global Coronavirus
Outbreak}}{Latest Updates: Global Coronavirus Outbreak}}\label{latest-updates-global-coronavirus-outbreak}}

Updated 2020-08-02T10:04:29.623Z

\begin{itemize}
\tightlist
\item
  \href{https://www.nytimes.com/2020/08/01/world/coronavirus-covid-19.html?action=click\&pgtype=Article\&state=default\&region=MAIN_CONTENT_1\&context=storylines_live_updates\#link-34047410}{The
  U.S. reels as July cases more than double the total of any other
  month.}
\item
  \href{https://www.nytimes.com/2020/08/01/world/coronavirus-covid-19.html?action=click\&pgtype=Article\&state=default\&region=MAIN_CONTENT_1\&context=storylines_live_updates\#link-780ec966}{Top
  U.S. officials work to break an impasse over the federal jobless
  benefit.}
\item
  \href{https://www.nytimes.com/2020/08/01/world/coronavirus-covid-19.html?action=click\&pgtype=Article\&state=default\&region=MAIN_CONTENT_1\&context=storylines_live_updates\#link-2bc8948}{Its
  outbreak untamed, Melbourne goes into even greater lockdown.}
\end{itemize}

\href{https://www.nytimes.com/2020/08/01/world/coronavirus-covid-19.html?action=click\&pgtype=Article\&state=default\&region=MAIN_CONTENT_1\&context=storylines_live_updates}{See
more updates}

More live coverage:
\href{https://www.nytimes.com/live/2020/07/31/business/stock-market-today-coronavirus?action=click\&pgtype=Article\&state=default\&region=MAIN_CONTENT_1\&context=storylines_live_updates}{Markets}

While many medical procedures --- like dental work, X-rays and annual
vaccinations --- often cause kids distress, experts say the coronavirus
nasal swab test is a perfect tripwire for inducing anxiety in children.
That is partly because of the novel coronavirus itself. Many kids
recognize that the virus is formidable; it has upended the world around
them for the past several months and unleashed a steady drumbeat of
scary news.

And the test is very uncomfortable. Nurses take a cotton swab and
capture cells far inside the nose, which lasts a few seconds and often
hurts. When children wiggle their bodies or shake their heads, they will
likely experience more significant pain. Some facilities require the
test to be done in a drive-up setting to minimize person-to-person
contact. In those scenarios, the test is often done while a child is
sitting in the back seat of a car, and conducted by a nurse who is not
only cocooned in protective gear but also reaching through a half-open
rear window.

``Nurses are rightly concerned with physical safety during tests and
procedures, but we also have to worry about emotional safety,'' Stephens
said. ``A lot of little minor insults can add up to long-term emotional
harm that makes people afraid to go to the doctor, and affects overall
medical compliance.''

\href{https://www.wjgnet.com/2219-2808/full/v5/i2/143.htm}{Research
shows that} nerve-racking medical tests and interventions can affect
children psychologically. For example,
\href{https://www.cmaj.ca/content/182/18/E843.short}{a study} found that
10 percent of adults abstain from medical procedures involving needles,
such as vaccinations, because of poor experiences they had as a child.

Even though the nasal exam is new --- and scary --- for some kids,
parents and caregivers can help by telling kids exactly what to expect,
and leveling with them that the test is unpleasant.

``It's important to be open and honest and share the truth about the
test,'' Stephens said. ``It's normal for a child to feel anxious and
worried. But in the end, it's an opportunity to build trust.''

For younger kids, there are
\href{https://www.mayoclinic.org/patient-education?VID=VID-20483010}{videos}
that can demystify the test, and Stephens said it helps to describe the
process in familiar terms for kids.

``We talk to them about a soft cotton Q-tip that is going to grab some
of their boogers,'' she said. She also recommended using play time to
act out the procedure.

\href{https://www.nytimes.com/news-event/coronavirus?action=click\&pgtype=Article\&state=default\&region=MAIN_CONTENT_3\&context=storylines_faq}{}

\hypertarget{the-coronavirus-outbreak-}{%
\subsubsection{The Coronavirus Outbreak
›}\label{the-coronavirus-outbreak-}}

\hypertarget{frequently-asked-questions}{%
\paragraph{Frequently Asked
Questions}\label{frequently-asked-questions}}

Updated July 27, 2020

\begin{itemize}
\item ~
  \hypertarget{should-i-refinance-my-mortgage}{%
  \paragraph{Should I refinance my
  mortgage?}\label{should-i-refinance-my-mortgage}}

  \begin{itemize}
  \tightlist
  \item
    \href{https://www.nytimes.com/article/coronavirus-money-unemployment.html?action=click\&pgtype=Article\&state=default\&region=MAIN_CONTENT_3\&context=storylines_faq}{It
    could be a good idea,} because mortgage rates have
    \href{https://www.nytimes.com/2020/07/16/business/mortgage-rates-below-3-percent.html?action=click\&pgtype=Article\&state=default\&region=MAIN_CONTENT_3\&context=storylines_faq}{never
    been lower.} Refinancing requests have pushed mortgage applications
    to some of the highest levels since 2008, so be prepared to get in
    line. But defaults are also up, so if you're thinking about buying a
    home, be aware that some lenders have tightened their standards.
  \end{itemize}
\item ~
  \hypertarget{what-is-school-going-to-look-like-in-september}{%
  \paragraph{What is school going to look like in
  September?}\label{what-is-school-going-to-look-like-in-september}}

  \begin{itemize}
  \tightlist
  \item
    It is unlikely that many schools will return to a normal schedule
    this fall, requiring the grind of
    \href{https://www.nytimes.com/2020/06/05/us/coronavirus-education-lost-learning.html?action=click\&pgtype=Article\&state=default\&region=MAIN_CONTENT_3\&context=storylines_faq}{online
    learning},
    \href{https://www.nytimes.com/2020/05/29/us/coronavirus-child-care-centers.html?action=click\&pgtype=Article\&state=default\&region=MAIN_CONTENT_3\&context=storylines_faq}{makeshift
    child care} and
    \href{https://www.nytimes.com/2020/06/03/business/economy/coronavirus-working-women.html?action=click\&pgtype=Article\&state=default\&region=MAIN_CONTENT_3\&context=storylines_faq}{stunted
    workdays} to continue. California's two largest public school
    districts --- Los Angeles and San Diego --- said on July 13, that
    \href{https://www.nytimes.com/2020/07/13/us/lausd-san-diego-school-reopening.html?action=click\&pgtype=Article\&state=default\&region=MAIN_CONTENT_3\&context=storylines_faq}{instruction
    will be remote-only in the fall}, citing concerns that surging
    coronavirus infections in their areas pose too dire a risk for
    students and teachers. Together, the two districts enroll some
    825,000 students. They are the largest in the country so far to
    abandon plans for even a partial physical return to classrooms when
    they reopen in August. For other districts, the solution won't be an
    all-or-nothing approach.
    \href{https://bioethics.jhu.edu/research-and-outreach/projects/eschool-initiative/school-policy-tracker/}{Many
    systems}, including the nation's largest, New York City, are
    devising
    \href{https://www.nytimes.com/2020/06/26/us/coronavirus-schools-reopen-fall.html?action=click\&pgtype=Article\&state=default\&region=MAIN_CONTENT_3\&context=storylines_faq}{hybrid
    plans} that involve spending some days in classrooms and other days
    online. There's no national policy on this yet, so check with your
    municipal school system regularly to see what is happening in your
    community.
  \end{itemize}
\item ~
  \hypertarget{is-the-coronavirus-airborne}{%
  \paragraph{Is the coronavirus
  airborne?}\label{is-the-coronavirus-airborne}}

  \begin{itemize}
  \tightlist
  \item
    The coronavirus
    \href{https://www.nytimes.com/2020/07/04/health/239-experts-with-one-big-claim-the-coronavirus-is-airborne.html?action=click\&pgtype=Article\&state=default\&region=MAIN_CONTENT_3\&context=storylines_faq}{can
    stay aloft for hours in tiny droplets in stagnant air}, infecting
    people as they inhale, mounting scientific evidence suggests. This
    risk is highest in crowded indoor spaces with poor ventilation, and
    may help explain super-spreading events reported in meatpacking
    plants, churches and restaurants.
    \href{https://www.nytimes.com/2020/07/06/health/coronavirus-airborne-aerosols.html?action=click\&pgtype=Article\&state=default\&region=MAIN_CONTENT_3\&context=storylines_faq}{It's
    unclear how often the virus is spread} via these tiny droplets, or
    aerosols, compared with larger droplets that are expelled when a
    sick person coughs or sneezes, or transmitted through contact with
    contaminated surfaces, said Linsey Marr, an aerosol expert at
    Virginia Tech. Aerosols are released even when a person without
    symptoms exhales, talks or sings, according to Dr. Marr and more
    than 200 other experts, who
    \href{https://academic.oup.com/cid/article/doi/10.1093/cid/ciaa939/5867798}{have
    outlined the evidence in an open letter to the World Health
    Organization}.
  \end{itemize}
\item ~
  \hypertarget{what-are-the-symptoms-of-coronavirus}{%
  \paragraph{What are the symptoms of
  coronavirus?}\label{what-are-the-symptoms-of-coronavirus}}

  \begin{itemize}
  \tightlist
  \item
    Common symptoms
    \href{https://www.nytimes.com/article/symptoms-coronavirus.html?action=click\&pgtype=Article\&state=default\&region=MAIN_CONTENT_3\&context=storylines_faq}{include
    fever, a dry cough, fatigue and difficulty breathing or shortness of
    breath.} Some of these symptoms overlap with those of the flu,
    making detection difficult, but runny noses and stuffy sinuses are
    less common.
    \href{https://www.nytimes.com/2020/04/27/health/coronavirus-symptoms-cdc.html?action=click\&pgtype=Article\&state=default\&region=MAIN_CONTENT_3\&context=storylines_faq}{The
    C.D.C. has also} added chills, muscle pain, sore throat, headache
    and a new loss of the sense of taste or smell as symptoms to look
    out for. Most people fall ill five to seven days after exposure, but
    symptoms may appear in as few as two days or as many as 14 days.
  \end{itemize}
\item ~
  \hypertarget{does-asymptomatic-transmission-of-covid-19-happen}{%
  \paragraph{Does asymptomatic transmission of Covid-19
  happen?}\label{does-asymptomatic-transmission-of-covid-19-happen}}

  \begin{itemize}
  \tightlist
  \item
    So far, the evidence seems to show it does. A widely cited
    \href{https://www.nature.com/articles/s41591-020-0869-5}{paper}
    published in April suggests that people are most infectious about
    two days before the onset of coronavirus symptoms and estimated that
    44 percent of new infections were a result of transmission from
    people who were not yet showing symptoms. Recently, a top expert at
    the World Health Organization stated that transmission of the
    coronavirus by people who did not have symptoms was ``very rare,''
    \href{https://www.nytimes.com/2020/06/09/world/coronavirus-updates.html?action=click\&pgtype=Article\&state=default\&region=MAIN_CONTENT_3\&context=storylines_faq\#link-1f302e21}{but
    she later walked back that statement.}
  \end{itemize}
\end{itemize}

Because keeping your body motionless makes the test more bearable, the
Children's Hospital of Philadelphia
\href{https://www.chop.edu/health-resources/preparing-your-child-drive-thru-covid-19-testing}{instructs}
parents to tell kids to ``try to hold your head still like a soldier''
or ``let's pretend we've been frozen like Elsa.'' It also recommends
validating a child's feelings by saying, ``It's OK to feel upset about
this.'' And it endorses bringing a favorite comfort item to the testing
site like a blanket or a stuffed animal.

``Even for kids who have excellent coping skills, they are having a very
hard time with this test,'' said Jennifer Rodemeyer, a child life
specialist at the Mayo Clinic in Rochester, Minn. ``We have found
anything that comes in deeply through the nose is very challenging, and
there is no topical anesthetic we can use to make it more comfortable.''

Karen Turner is a patient advocate who works with children with
developmental disabilities at Massachusetts General Hospital in Boston.
Many of the kids she sees visit the hospital regularly and must get the
coronavirus test numerous times. She said some of the proven strategies
that help kids with autism cope with fight-or-flight situations can also
help kids cope with the test.

She suggests putting children in a weighted vest during a swab to
provide deep pressure, which is shown to be calming. Another method
often used to help autistic children deal with anxiety-producing events
is to create a booklet known as a
\href{https://www.massgeneral.org/assets/MGH/pdf/children/lurie-center-nose-test-social-story.pdf}{social
story} that illustrates the situation. Turner says showing a child a
social story that
\href{https://www.healthymindcentre.com.au/covid19story}{depicts all the
steps} of the Covid-19 test can defuse some of the fear.

Meanwhile, at the Mayo Clinic, nurses have found that parents should
keep children from lying on their backs when they get the test since it
is natural to feel a loss of control in that position. ``Sitting up is
best,'' Rodemeyer advised.

``Obviously, we would love it if there was a less invasive way to do
this, but for now, the nasal swab is the best and most accurate way to
do this testing,'' said Rodemeyer. ``Children are going to have to go
out into the world, go to the hospital, and take this test. Covid-19
isn't going away.''

Advertisement

\protect\hyperlink{after-bottom}{Continue reading the main story}

\hypertarget{site-index}{%
\subsection{Site Index}\label{site-index}}

\hypertarget{site-information-navigation}{%
\subsection{Site Information
Navigation}\label{site-information-navigation}}

\begin{itemize}
\tightlist
\item
  \href{https://help.nytimes.com/hc/en-us/articles/115014792127-Copyright-notice}{©~2020~The
  New York Times Company}
\end{itemize}

\begin{itemize}
\tightlist
\item
  \href{https://www.nytco.com/}{NYTCo}
\item
  \href{https://help.nytimes.com/hc/en-us/articles/115015385887-Contact-Us}{Contact
  Us}
\item
  \href{https://www.nytco.com/careers/}{Work with us}
\item
  \href{https://nytmediakit.com/}{Advertise}
\item
  \href{http://www.tbrandstudio.com/}{T Brand Studio}
\item
  \href{https://www.nytimes.com/privacy/cookie-policy\#how-do-i-manage-trackers}{Your
  Ad Choices}
\item
  \href{https://www.nytimes.com/privacy}{Privacy}
\item
  \href{https://help.nytimes.com/hc/en-us/articles/115014893428-Terms-of-service}{Terms
  of Service}
\item
  \href{https://help.nytimes.com/hc/en-us/articles/115014893968-Terms-of-sale}{Terms
  of Sale}
\item
  \href{https://spiderbites.nytimes.com}{Site Map}
\item
  \href{https://help.nytimes.com/hc/en-us}{Help}
\item
  \href{https://www.nytimes.com/subscription?campaignId=37WXW}{Subscriptions}
\end{itemize}
