Sections

SEARCH

\protect\hyperlink{site-content}{Skip to
content}\protect\hyperlink{site-index}{Skip to site index}

\href{https://www.nytimes.com/section/politics}{Politics}

\href{https://myaccount.nytimes.com/auth/login?response_type=cookie\&client_id=vi}{}

\href{https://www.nytimes.com/section/todayspaper}{Today's Paper}

\href{/section/politics}{Politics}\textbar{}Trump Commutes Sentence of
Roger Stone in Case He Long Denounced

\url{https://nyti.ms/3iQV6TH}

\begin{itemize}
\item
\item
\item
\item
\item
\item
\end{itemize}

Advertisement

\protect\hyperlink{after-top}{Continue reading the main story}

Supported by

\protect\hyperlink{after-sponsor}{Continue reading the main story}

\hypertarget{trump-commutes-sentence-of-roger-stone-in-case-he-long-denounced}{%
\section{Trump Commutes Sentence of Roger Stone in Case He Long
Denounced}\label{trump-commutes-sentence-of-roger-stone-in-case-he-long-denounced}}

The president's friend had been convicted of impeding a congressional
inquiry that threatened Mr. Trump.

\includegraphics{https://static01.nyt.com/images/2020/07/09/us/politics/00dc-stone-pardon/merlin_169227537_d884bc30-9755-4994-a6f2-89236e944527-articleLarge.jpg?quality=75\&auto=webp\&disable=upscale}

\href{https://www.nytimes.com/by/peter-baker}{\includegraphics{https://static01.nyt.com/images/2018/06/13/multimedia/peter-baker/peter-baker-thumbLarge-v2.png}}\href{https://www.nytimes.com/by/maggie-haberman}{\includegraphics{https://static01.nyt.com/images/2018/07/12/multimedia/author-maggie-haberman/author-maggie-haberman-thumbLarge.png}}\href{https://www.nytimes.com/by/sharon-lafraniere}{\includegraphics{https://static01.nyt.com/images/2018/07/12/multimedia/author-sharon-lafraniere/author-sharon-lafraniere-thumbLarge.png}}

By \href{https://www.nytimes.com/by/peter-baker}{Peter Baker},
\href{https://www.nytimes.com/by/maggie-haberman}{Maggie Haberman} and
\href{https://www.nytimes.com/by/sharon-lafraniere}{Sharon LaFraniere}

\begin{itemize}
\item
  July 10, 2020
\item
  \begin{itemize}
  \item
  \item
  \item
  \item
  \item
  \item
  \end{itemize}
\end{itemize}

WASHINGTON --- President Trump
\href{https://www.nytimes.com/article/trump-pardons-commutations.html}{commuted
the sentence} of his longtime friend
\href{https://www.nytimes.com/2020/07/19/us/politics/roger-stone-mo-kelly-slur.html}{Roger
J. Stone Jr.} on seven felony crimes on Friday, using the power of his
office to spare a former campaign adviser days before Mr. Stone was to
report to a federal prison to serve a 40-month term.

In
\href{https://www.whitehouse.gov/briefings-statements/statement-press-secretary-regarding-executive-grant-clemency-roger-stone-jr/}{a
lengthy written statement} punctuated by the sort of inflammatory
language and angry grievances characteristic of the president's Twitter
feed, the White House denounced the ``overzealous prosecutors'' who
convicted Mr. Stone on ``process-based charges'' stemming from the
``witch hunts'' and ``Russia hoax'' investigation.

The statement did not assert that Mr. Stone was innocent of the false
statements and obstruction counts, only that he should not have been
pursued because prosecutors ultimately filed no charges of an underlying
conspiracy between Mr. Trump's campaign and Russia. ``Roger Stone has
already suffered greatly,'' it said. ``He was treated very unfairly, as
were many others in this case. Roger Stone is now a free man!''

The commutation, announced late on a Friday, when potentially damaging
news is often released, was the latest action by the Trump
administration upending the justice system to help the president's
convicted friends. The Justice Department moved in May
\href{https://www.nytimes.com/2020/05/07/us/politics/michael-flynn-case-dropped.html}{to
dismiss its own criminal case} against Mr. Trump's former national
security adviser Michael T. Flynn, who had pleaded guilty to lying to
the F.B.I. And last month,
\href{https://www.nytimes.com/2020/06/20/nyregion/trump-geoffrey-berman-fired-sdny.html}{Mr.
Trump fired Geoffrey S. Berman}, the United States attorney whose office
prosecuted Michael D. Cohen, the president's former personal lawyer, and
has been investigating Rudolph W. Giuliani, another of his lawyers.

Democrats quickly condemned the president's decision, characterizing it
as an abuse of the rule of law. ``With this commutation, Trump makes
clear that there are two systems of justice in America: one for his
criminal friends, and one for everyone else,'' said Representative Adam
B. Schiff, Democrat of California and a leader of the drive to impeach
Mr. Trump last year for pressuring Ukraine to incriminate his domestic
rivals.

Two House committee chairmen quickly announced that they would
investigate the circumstances of the commutation, suggesting that it was
a reward for Mr. Stone's silence protecting the president. ``No other
president has exercised the clemency power for such a patently personal
and self-serving purpose,'' said a statement issued by Representatives
Jerrold Nadler and Carolyn B. Maloney, both New York Democrats.

Senator Mitt Romney, Republican of Utah, called the commutation
\href{https://twitter.com/MittRomney/status/1281937795616067586}{``Unprecedented,
historic corruption''} in a tweet.

Mr. Stone, 67, a longtime Republican operative,
\href{https://www.nytimes.com/2019/11/15/us/politics/roger-stone-trial-guilty.html}{was
convicted} of obstructing a congressional investigation into Mr. Trump's
2016 campaign and possible ties to Russia. Prosecutors convinced jurors
that he lied under oath, withheld a trove of documents and threatened an
associate with harm if he cooperated with congressional investigators.
Mr. Stone maintained his innocence and claimed prosecutors wanted him to
offer information about Mr. Trump that he said did not exist.

As his time to report to prison neared, Mr. Stone openly lobbied for
clemency, maintaining that he could die in prison and emphasizing that
he had stayed loyal to the president rather than help investigators.

``He knows I was under enormous pressure to turn on him,''
\href{https://twitter.com/howardfineman/status/1281681337351626752}{Mr.
Stone told the journalist Howard Fineman} on Friday shortly before the
announcement. ``It would have eased my situation considerably. But I
didn't.''

In an interview with Fox News this week, he characterized himself as
collateral damage in the quest to target Mr. Trump. ``He is aware that
the people trying to destroy Michael Flynn, now trying to destroy me,
are the people trying to destroy him,'' Mr. Stone said.

In an appearance on Fox last month, Mr. Stone even suggested that he
wanted to help Mr. Trump win in November, saying his biggest fear of
going to prison other than his health was ``that I may not be free to do
everything within my power to re-elect this president.''

While it was not clear when the two last spoke before the decision, Mr.
Trump called Mr. Stone on Friday to deliver the news of his clemency
personally, according to an official briefed on the conversation.

The president has used his power to
\href{https://www.nytimes.com/2020/02/18/us/politics/trump-pardon-blagojevich-debartolo.html}{issue
pardons or commutations} to a variety of political allies, supporters or
people with connections to his own circle, like the former New York
police commissioner
\href{https://www.nytimes.com/2020/02/19/us/politics/trump-pardons.html}{Bernard
B. Kerik}, the financier
\href{https://www.nytimes.com/2020/02/18/business/michael-milken-case-lessons.html}{Michael
R. Milken} and former Gov.
\href{https://www.nytimes.com/2020/02/19/us/rod-blagojevich-chicago.html}{Rod
R. Blagojevich} of Illinois.

But Mr. Stone is the first figure directly connected to the president's
campaign to benefit from his clemency power. While Mr. Trump has
publicly dangled pardons for associates targeted by investigators, that
was a line he had been wary of crossing until now amid warnings from
advisers concerned about the possible political damage.

The debate over clemency for Mr. Stone has raged within the White House
for months. Among those who advocated on behalf of it from outside the
building were Tucker Carlson, the influential Fox News anchor, and
Representative Matt Gaetz, Republican of Florida, according to people
familiar with the discussions.

Within the White House, Mr. Stone had few allies. Many Trump aides who
knew him from the campaign did not like him, were envious of his long
relationship with Mr. Trump or thought clemency would be bad politics.

Mark Meadows, the White House chief of staff, expressed concern about
potential political consequences, according to two people familiar with
the discussions, although he has left people with different impressions
about where he stands. The same is true of Jared Kushner, Mr. Trump's
son-in-law and senior adviser, who has been involved in most of the
clemency discussions throughout the past three years.

Pat A. Cipollone, the White House counsel, was concerned about
intervening on Mr. Stone's behalf, according to the people close to the
discussions. One of the few within the White House who backed clemency
was Larry Kudlow, the president's top economic adviser and an old friend
of Mr. Stone's. Mr. Kudlow spends more time with Mr. Trump than many
other advisers.

``Mr. Stone is incredibly honored that President Trump used his awesome
and unique power under the Constitution of the United States for this
act of mercy,'' Grant Smith, Mr. Stone's lawyer, said after the
announcement. ``Mr. and Mrs. Stone appreciate all the consideration the
president gave to this matter.''

Mr. Stone has been one of the most flamboyant rogues in American
politics for decades, maintaining a wardrobe of more than 100 suits,
bleaching his hair, posing for photographs half-naked and cheerfully
engaging in dirty tricks that others would disavow. He made political
contributions to a Republican challenger to President Richard M. Nixon
in 1972 under the name of the Young Socialist Alliance and hired an
operative to try to infiltrate the campaign of George McGovern, the
Democratic candidate.

He was accused of leaving a threatening, profanity-laced voice mail
message for the father of Gov. Eliot Spitzer of New York, resulting in
Mr. Stone's resignation. But he later got his revenge on Mr. Spitzer by
claiming credit for spreading the rumor that the governor wore black
dress socks during sexual escapades with prostitutes.

An unapologetic admirer of Mr. Nixon who even had the disgraced
president's face tattooed on his back, Mr. Stone also worked for other
major Republican candidates, including President Ronald Reagan, Gov.
Thomas H. Kean of New Jersey and Senator Bob Dole, the party's 1996
nominee for president.

Mr. Stone's history of scandals and dirty tricks did not trouble Mr.
Trump. Mr. Stone is not only Mr. Trump's longest-serving political
adviser, but has been integral to most of his political activities over
the past three decades.

He was there when the celebrity real estate developer first wrote ``The
Art of the Deal'' in 1987 and a makeshift effort in New Hampshire was
made to draft Mr. Trump to run for president. He helped organize Mr.
Trump's speech to the Conservative Political Action Conference in 2011
when he declared himself against abortion rights. And he helped map out
the first days of Mr. Trump's 2016 campaign before quitting over its
direction. The falling out was sour, part of a roller-coaster,
feud-and-friends relationship between the two men that played out over
the years. At one point, Mr. Trump called Mr. Stone a ``stone-cold
loser,'' and aides later said the president viewed him as a
self-promoter.

But after Mr. Stone was indicted, the president repeatedly assailed the
prosecutors, judge and even jury forewoman, hinting at a possible
pardon. ``Roger Stone and everybody has to be treated fairly,'' Mr.
Trump said after the sentencing in February. ``This has not been a fair
process.''

Mr. Stone was sentenced against a backdrop of upheaval at the Justice
Department not seen for decades. Four career prosecutors recommended
that he be sentenced to seven to nine years in prison, citing advisory
sentencing guidelines. After Mr. Trump attacked the recommendation on
Twitter,
\href{https://www.nytimes.com/2020/02/11/us/politics/roger-stone-sentencing.html}{Attorney
General William P. Barr overruled it}. Mr. Trump then publicly applauded
him for doing so, even though the attorney general said he made the
decision on his own and criticized the president on television for
undercutting his credibility.

The prosecutors
\href{https://www.nytimes.com/2020/02/11/us/politics/roger-stone-sentencing.html}{withdrew
from the case} in protest, and one quit the department entirely. At Mr.
Stone's sentencing hearing, Judge Amy Berman Jackson of the United
States District Court for the District of Columbia called the situation
``unprecedented.'' Without naming him,
\href{https://www.nytimes.com/2020/02/20/us/roger-stone-40-months-sentencing-verdict.html}{she
suggested} that the president had tried to influence the course of
justice by publicly attacking her, the jurors and the Justice Department
lawyers.

``The dismay and disgust at any attempt to interfere with the efforts of
prosecutors and members of the judiciary to fulfill their duty should
transcend party,'' she said.

\href{https://www.google.com/amp/s/abcnews.go.com/amp/Politics/attorney-general-william-barr-defends-justice-department-claims/story\%3fid=71680447}{In
an interview with ABC News} this week, Mr. Barr defended both the
original prosecution of Mr. Stone as well as his own intervention to
reduce the punishment, saying the case itself was ``righteous'' but the
sentencing recommendation ``excessive.''

Mr. Stone, who lives in Florida, had been ordered earlier to report to
the Bureau of Prisons by June 30 to begin serving his sentence. He
sought a two-month delay, citing the coronavirus pandemic sweeping
through federal prisons, but Judge Jackson
\href{https://www.nytimes.com/2020/06/26/us/politics/roger-stone-prison.html}{granted
him only a two-week reprieve}, noting that the prison he was to report
to was ``unaffected'' by the outbreak.

Two other former aides to Mr. Trump who were convicted of federal crimes
were released from prison to serve out their sentences under home arrest
because of the pandemic.

Mr. Cohen, who broke with Mr. Trump and publicly accused him of vast
wrongdoing,
\href{https://www.nytimes.com/2020/05/20/nyregion/michael-cohen-coronavirus-prison-release.html}{was
released} from a federal prison camp in May, but
\href{https://www.nytimes.com/2020/07/09/nyregion/michael-cohen-arrested.html}{taken
back into custody} this week after refusing to agree to terms of his
home confinement that would have forced him to scrub a tell-all book he
planned to publish in September. He was serving a
\href{https://www.nytimes.com/2018/12/12/nyregion/michael-cohen-sentence-trump.html}{three-year
sentence for} campaign finance violations and other crimes related to a
scheme to pay hush money during the 2016 race to two women who said they
had affairs with Mr. Trump, which the president has denied.

Paul Manafort, the president's former campaign chairman,
\href{https://www.nytimes.com/2020/05/13/us/politics/paul-manafort-released-coronavirus.html}{was
also released} in May from a central Pennsylvania prison, where he was
serving a seven-and-a-half year sentence for bank and tax fraud. He is
now confined at home.

Mr. Trump has suggested he might issue clemency to other associates, and
the extended statement released by the White House on Friday night
sought to make the case that the Russia investigation was so
illegitimate that charges resulting from it were themselves invalid.

``These charges were the product of recklessness borne of frustration
and malice,'' the statement said. ``This is why the out-of-control
Mueller prosecutors, desperate for splashy headlines to compensate for a
failed investigation, set their sights on Mr. Stone.''

Peter Baker and Sharon LaFraniere reported from Washington, and Maggie
Haberman from New York.

Advertisement

\protect\hyperlink{after-bottom}{Continue reading the main story}

\hypertarget{site-index}{%
\subsection{Site Index}\label{site-index}}

\hypertarget{site-information-navigation}{%
\subsection{Site Information
Navigation}\label{site-information-navigation}}

\begin{itemize}
\tightlist
\item
  \href{https://help.nytimes.com/hc/en-us/articles/115014792127-Copyright-notice}{©~2020~The
  New York Times Company}
\end{itemize}

\begin{itemize}
\tightlist
\item
  \href{https://www.nytco.com/}{NYTCo}
\item
  \href{https://help.nytimes.com/hc/en-us/articles/115015385887-Contact-Us}{Contact
  Us}
\item
  \href{https://www.nytco.com/careers/}{Work with us}
\item
  \href{https://nytmediakit.com/}{Advertise}
\item
  \href{http://www.tbrandstudio.com/}{T Brand Studio}
\item
  \href{https://www.nytimes.com/privacy/cookie-policy\#how-do-i-manage-trackers}{Your
  Ad Choices}
\item
  \href{https://www.nytimes.com/privacy}{Privacy}
\item
  \href{https://help.nytimes.com/hc/en-us/articles/115014893428-Terms-of-service}{Terms
  of Service}
\item
  \href{https://help.nytimes.com/hc/en-us/articles/115014893968-Terms-of-sale}{Terms
  of Sale}
\item
  \href{https://spiderbites.nytimes.com}{Site Map}
\item
  \href{https://help.nytimes.com/hc/en-us}{Help}
\item
  \href{https://www.nytimes.com/subscription?campaignId=37WXW}{Subscriptions}
\end{itemize}
