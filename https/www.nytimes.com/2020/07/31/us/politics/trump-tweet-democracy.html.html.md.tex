Sections

SEARCH

\protect\hyperlink{site-content}{Skip to
content}\protect\hyperlink{site-index}{Skip to site index}

\href{https://www.nytimes.com/section/politics}{Politics}

\href{https://myaccount.nytimes.com/auth/login?response_type=cookie\&client_id=vi}{}

\href{https://www.nytimes.com/section/todayspaper}{Today's Paper}

\href{/section/politics}{Politics}\textbar{}More Than Just a Tweet:
Trump's Campaign to Undercut Democracy

\href{https://nyti.ms/3jZcMgf}{https://nyti.ms/3jZcMgf}

\begin{itemize}
\item
\item
\item
\item
\item
\item
\end{itemize}

\begin{itemize}
\item
  \href{https://www.nytimes.com/2020/08/07/us/elections/biden-vs-trump.html?action=click\&pgtype=Article\&state=default\&region=TOP_BANNER\&context=storylines_menu}{Election
  Updates}
\item
  \href{https://www.nytimes.com/interactive/2020/08/06/us/elections/results-tennessee-primary-elections.html?action=click\&pgtype=Article\&state=default\&region=TOP_BANNER\&context=storylines_menu}{Tennessee
  Results}
\item
  \href{https://www.nytimes.com/article/biden-vice-president-2020.html?action=click\&pgtype=Article\&state=default\&region=TOP_BANNER\&context=storylines_menu}{Biden's
  V.P. Search}
\item
  \href{https://www.nytimes.com/interactive/2019/us/politics/2020-presidential-candidates.html?action=click\&pgtype=Article\&state=default\&region=TOP_BANNER\&context=storylines_menu}{The
  Candidates}
\item
  \href{https://www.nytimes.com/newsletters/politics?action=click\&pgtype=Article\&state=default\&region=TOP_BANNER\&context=storylines_menu}{Politics
  Newsletter}
\end{itemize}

Advertisement

\protect\hyperlink{after-top}{Continue reading the main story}

Supported by

\protect\hyperlink{after-sponsor}{Continue reading the main story}

News Analysis

\hypertarget{more-than-just-a-tweet-trumps-campaign-to-undercut-democracy}{%
\section{More Than Just a Tweet: Trump's Campaign to Undercut
Democracy}\label{more-than-just-a-tweet-trumps-campaign-to-undercut-democracy}}

Floating the idea of delaying the election was the latest step in the
president's running effort to discredit the election, risking long-term
damage to public trust in the system.

\includegraphics{https://static01.nyt.com/images/2020/07/31/us/31dc-trump-democracy/merlin_175135440_b6103548-fa0a-42e6-b8b4-8c47aa78ae3b-articleLarge.jpg?quality=75\&auto=webp\&disable=upscale}

\href{https://www.nytimes.com/by/peter-baker}{\includegraphics{https://static01.nyt.com/images/2018/06/13/multimedia/peter-baker/peter-baker-thumbLarge-v2.png}}

By \href{https://www.nytimes.com/by/peter-baker}{Peter Baker}

\begin{itemize}
\item
  July 31, 2020
\item
  \begin{itemize}
  \item
  \item
  \item
  \item
  \item
  \item
  \end{itemize}
\end{itemize}

Nothing in the Constitution gives President Trump
\href{https://www.nytimes.com/2020/07/30/us/politics/trump-postpone-election.html?action=click\&module=Top\%20Stories\&pgtype=Homepage}{the
power to delay the November election}, and even fellow
\href{https://www.nytimes.com/2020/07/30/us/politics/trump-delay-2020-election.html?action=click\&module=Top\%20Stories\&pgtype=Homepage}{Republicans
dismissed it out of hand} when he broached it on Thursday. But that was
not the point. With a possible defeat looming, the point was to tell
Americans that they should not trust their own democracy.

The idea of putting off the vote was the culmination of months of
discrediting an election that polls suggest Mr. Trump is currently
losing by a wide margin. He has repeatedly predicted
\href{https://twitter.com/realDonaldTrump/status/1256366878873792512}{``RIGGED
ELECTIONS''} and a
\href{https://twitter.com/realDonaldTrump/status/1265255835124539392?s=20}{``substantially
fraudulent''} vote and
\href{https://www.nytimes.com/2020/06/23/us/politics/trump-arizona.html}{``the
most corrupt election in the history of our country,''} all based on
false, unfounded or exaggerated claims.

It is the kind of language resonant of conspiracy theorists, cranks and
defeated candidates, not an incumbent living in the White House. Never
before has a sitting president of the United States sought to undermine
public faith in the election system the way Mr. Trump has. He has
\href{https://www.nytimes.com/aponline/2020/07/19/us/politics/ap-us-trump.html}{refused
to commit to respecting the results} and, even after his election-delay
trial balloon was panned by Republican allies, he raised the specter on
Thursday evening of months of lawsuits challenging the outcome.

Mr. Trump has put on the line not merely the outcome of this fall's
contest but the credibility of the system as a whole, according to even
scholars and operatives normally sympathetic to the president. Just
floating the possibility of postponing a presidential election, an idea
anathema in America and reminiscent of authoritarian countries without
the rule of law, risks eroding the most important ingredient in a
democracy --- the belief by most Americans that, whatever its manifest
flaws, the election result will be fundamentally fair.

``It undermines the faith of the public in our electoral process,'' said
\href{https://www.nytimes.com/2019/12/04/us/politics/jonathan-turley.html}{Jonathan
Turley}, a George Washington University law professor who testified on
Mr. Trump's side last year during the House impeachment hearings. ``Any
constitutional system is ultimately held together by a leap of faith.
Citizens must trust the process if you want them to yield to it. What
the president is doing is seeding distrust about the legitimacy of even
the holding of the election.''

\href{https://www.nytimes.com/2019/12/04/us/politics/michael-gerhardt.html}{Michael
J. Gerhardt}, a constitutional scholar at the University of North
Carolina who testified on the other side in those hearings, said Mr.
Trump's statements were part of a pattern of disdain for the norms that
have defined the United States for generations.

``In the long term, I think there's going to be a lot of institutional
damage,'' he said, ``and the rule of law is going to be undermined to a
very large extent.''

Even some of Mr. Trump's own current and former advisers see his attacks
on the election system as a reflection of fear that he may lose and as a
transparent effort to create a narrative to explain that away. Sam
Nunberg, an adviser on Mr. Trump's 2016 campaign, said the president was
``trying to get ahead of a potential loss'' by blaming it on external
factors like the coronavirus.

``What President Trump does not seem to understand is that unlike past
experiences where he was able to frame a defeat as a win, there is no
spin for losing a re-election as an incumbent president and taking down
the Republican Party with him,'' Mr. Nunberg said. ``Despite what he may
believe, even the overwhelming majority of the president's supporters
are not interested in this claptrap.''

He added: ``Republican voters and conservative media will ultimately
feel that if you cannot beat Joe Biden, you do not deserve another
term.''

As recently as April, a Republican National Committee official said
former Vice President Joseph R. Biden Jr. was
\href{https://twitter.com/SteveGuest/status/1253796935011651587}{``off
his rocker''} to suggest that Mr. Trump
\href{https://www.nytimes.com/2020/04/24/us/politics/joseph-biden-trump-election.html?smid=tw-share}{might
seek to ``kick back the election somehow.''} But in fact, Mr. Trump has
a long history of sowing doubt in election results that do not go the
way he wants them to go.

When he appeared to be losing to Hillary Clinton in 2016, he repeatedly
suggested that the election was being rigged and
\href{https://www.nytimes.com/2016/10/20/us/politics/presidential-debate.html}{would
not commit to accepting the results} --- until he won, that is. And even
after winning the Electoral College, he insisted that
\href{https://www.nytimes.com/2016/11/27/us/politics/trump-adviser-steps-up-searing-attack-on-romney.html}{he
had actually won the popular vote}, too, because three million illegal
immigrants had supposedly voted for Mrs. Clinton, a claim seemingly made
up out of thin air and one for which
\href{https://www.nytimes.com/2018/01/03/us/politics/trump-voter-fraud-commission.html}{his
own commission found no evidence}.

In 2020 alone, Mr. Trump has already made public comments, posted
Twitter messages or reposted others suggesting election fraud 91 times,
according to data
\href{https://www.newyorker.com/news/letter-from-trumps-washington/trump-is-the-election-crisis-he-is-warning-about}{compiled
for The New Yorker} by Factba.se, a service that collects and analyzes
data on his presidency. Going back to 2012,
\href{https://factba.se/}{Factba.se} counted 713 instances when Mr.
Trump cited vote fraud, spiking especially in 2016 and 2018 before
elections in which he had a stake.

Some of Mr. Trump's allies have said that he has justifiable reasons to
raise concerns about widespread mail-in voting being employed in light
of the coronavirus pandemic, even though there is a long history of its
use without evidence of widespread fraud. And they accuse the Democrats
of being the ones unwilling to accept election results when they lose,
pointing to the yearslong effort to investigate Russian interference in
the 2016 campaign and any ties to Mr. Trump's organization.

In
\href{https://www.cbsnews.com/video/hillary-clinton-trump-knows-hes-an-illegitimate-president/\#x}{an
interview last year with CBS News}, Mrs. Clinton made clear that she
considered Mr. Trump's election shady. ``I believe he knows he's an
illegitimate president,'' she said.

She is hardly the only defeated candidate to see injustice in her loss.
Going back to the early days of the republic, questions have been raised
about the legitimacy of presidential victories from those on the losing
side.

Andrew Jackson, who won the popular vote and had the most Electoral
College votes in 1824 but not a majority in a four-way race, ended up
losing to John Quincy Adams when the House decided the matter. Jackson
spent the next four years accusing Adams of a making a ``corrupt
bargain'' to secure the support of the third-place candidate, Henry
Clay, in exchange for appointment as secretary of state. Jackson got his
revenge by beating Adams in an 1828 rematch.

Likewise, Democrats complained when Rutherford B. Hayes won in a
disputed election in 1876, calling him Rutherfraud B. Hayes and His
Fraudulency. Republicans suspected that John F. Kennedy beat Richard M.
Nixon in 1960 thanks to vote fraud in Texas and Illinois, and many
Democrats never accepted George W. Bush's victory over Al Gore in 2000
after the Florida recount was halted by the Supreme Court.

But the complaints do not typically come from the Oval Office,
especially before an election has even been held. And no sitting
president has made a serious effort to delay his own re-election, not
even Abraham Lincoln in 1864 during the Civil War or Franklin D.
Roosevelt in 1944 during World War II. Elections were held as scheduled
during the pandemics of 1918 and 1968, as well.

Ronald C. White, a prominent Lincoln biographer, noted that the 16th
president did not try to postpone the election even though he thought he
was likely to lose. Instead, he made it possible for soldiers in the
field to cast their ballots, recognizing that they might support their
former general, George B. McClellan, who was his Democratic challenger.

``Even as the pandemic, economic collapse and racial protests have Trump
calling himself a wartime president, the real wartime president,
Lincoln, determined that the election of 1864 must go forward as a sign
that the Union would go forward,'' Mr. White said.

Jill Lepore, a Harvard University professor and the author of
\href{https://www.nytimes.com/2018/09/14/books/review/jill-lepore-these-truths.html}{``These
Truths: A History of the United States,''} said presidents bear a
responsibility to foster faith in democracy.

``Far from undermining public confidence in the democracy over which he
presides, it is the obligation of every president to cultivate that
confidence by guaranteeing voting rights, by condemning foreign
interference in American political campaigns, by promoting free, safe
and secure elections, and by abiding by their outcome,'' she said.

Mr. Trump has for years been drawn to leaders of other countries who did
not share that view, especially autocrats like Presidents Vladimir V.
Putin of Russia, Recep Tayyip Erdogan of Turkey and Xi Jinping of China.
He has expressed admiration for their leadership and envy that in their
systems they can be decisive without bureaucratic or political
impediments while avoiding criticism of their crackdowns on internal
dissent.

For Americans who have made it their mission to encourage free and fair
elections in countries like those and elsewhere, Mr. Trump's suggestion
to delay the November vote and his drumbeat of criticism leading up to
it sounded like what they confront abroad, not at home.

``I have never seen such an effort to sow distrust in our elections,''
said Michael J. Abramowitz, the president of Freedom House, a
nonpartisan organization that promotes democracy around the world. ``We
are used to seeing this kind of behavior from authoritarians around the
globe, but it is particularly disturbing coming from the president of
the United States.''

\hypertarget{our-2020-election-guide}{%
\section{Our 2020 Election Guide}\label{our-2020-election-guide}}

Updated Aug. 7, 2020

\begin{itemize}
\item
  \begin{center}\rule{0.5\linewidth}{\linethickness}\end{center}

  \hypertarget{the-latest}{%
  \subsection{The Latest}\label{the-latest}}

  \begin{itemize}
  \tightlist
  \item
    \href{https://www.nytimes.com/2020/08/07/us/politics/russia-china-trump-biden-election-interference.html?action=click\&pgtype=Article\&state=default\&region=BELOW_MAIN_CONTENT\&context=storylines_guide}{Russia
    is using a range of techniques to denigrate Joe Biden}, American
    intelligence officials said, declaring that Moscow continues to try
    to interfere in the 2020 campaign to help President Trump.
  \end{itemize}
\item
  \begin{center}\rule{0.5\linewidth}{\linethickness}\end{center}

  \hypertarget{bidens-vp-search}{%
  \subsection{Biden's V.P. Search}\label{bidens-vp-search}}

  \begin{itemize}
  \tightlist
  \item
    \href{https://www.nytimes.com/article/biden-vice-president-2020.html?action=click\&pgtype=Article\&state=default\&region=BELOW_MAIN_CONTENT\&context=storylines_guide}{Here
    are 13 women} who have been under consideration to be Joe Biden's
    running mate, and why each might be chosen --- and might not be.
  \end{itemize}
\item
  \begin{center}\rule{0.5\linewidth}{\linethickness}\end{center}

  \hypertarget{keep-up-with-our-coverage}{%
  \subsection{Keep Up With Our
  Coverage}\label{keep-up-with-our-coverage}}

  \begin{itemize}
  \tightlist
  \item
    Get an
    \href{https://www.nytimes.com/newsletters/politics?action=click\&pgtype=Article\&state=default\&region=BELOW_MAIN_CONTENT\&context=storylines_guide}{email}
    recapping the day's news
  \end{itemize}

  \begin{itemize}
  \tightlist
  \item
    Download our mobile app on
    \href{https://apps.apple.com/us/app/nytimes/id284862083?ls=1\&mat_click_id=5c79ae7455014fd1bd66b5610c05b8f2-20191112-16948\&referrer=mat_click_id\%3D5c79ae7455014fd1bd66b5610c05b8f2-20191112-16948\%26link_click_id\%3D722930677036718082}{iOS}
    and
    \href{http://a.localytics.com/android?id=com.nytimes.android\&referrer=utm_source\%3Dother_nyt_mobile_web\%26utm_medium\%3DWeb\%2520page\%26utm_term\%3DGeneral\%2520Mobile\%2520Page\%26utm_campaign\%3DNYT\%2520Mobile\%2520General\%2520Page}{Android}
    and turn on Breaking News and Politics alerts
  \end{itemize}
\end{itemize}

Advertisement

\protect\hyperlink{after-bottom}{Continue reading the main story}

\hypertarget{site-index}{%
\subsection{Site Index}\label{site-index}}

\hypertarget{site-information-navigation}{%
\subsection{Site Information
Navigation}\label{site-information-navigation}}

\begin{itemize}
\tightlist
\item
  \href{https://help.nytimes.com/hc/en-us/articles/115014792127-Copyright-notice}{©~2020~The
  New York Times Company}
\end{itemize}

\begin{itemize}
\tightlist
\item
  \href{https://www.nytco.com/}{NYTCo}
\item
  \href{https://help.nytimes.com/hc/en-us/articles/115015385887-Contact-Us}{Contact
  Us}
\item
  \href{https://www.nytco.com/careers/}{Work with us}
\item
  \href{https://nytmediakit.com/}{Advertise}
\item
  \href{http://www.tbrandstudio.com/}{T Brand Studio}
\item
  \href{https://www.nytimes.com/privacy/cookie-policy\#how-do-i-manage-trackers}{Your
  Ad Choices}
\item
  \href{https://www.nytimes.com/privacy}{Privacy}
\item
  \href{https://help.nytimes.com/hc/en-us/articles/115014893428-Terms-of-service}{Terms
  of Service}
\item
  \href{https://help.nytimes.com/hc/en-us/articles/115014893968-Terms-of-sale}{Terms
  of Sale}
\item
  \href{https://spiderbites.nytimes.com}{Site Map}
\item
  \href{https://help.nytimes.com/hc/en-us}{Help}
\item
  \href{https://www.nytimes.com/subscription?campaignId=37WXW}{Subscriptions}
\end{itemize}
