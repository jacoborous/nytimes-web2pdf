Sections

SEARCH

\protect\hyperlink{site-content}{Skip to
content}\protect\hyperlink{site-index}{Skip to site index}

\href{https://www.nytimes.com/section/science}{Science}

\href{https://myaccount.nytimes.com/auth/login?response_type=cookie\&client_id=vi}{}

\href{https://www.nytimes.com/section/todayspaper}{Today's Paper}

\href{/section/science}{Science}\textbar{}Lizard Popsicles, Anyone?

\url{https://nyti.ms/3fnVlmp}

\begin{itemize}
\item
\item
\item
\item
\item
\end{itemize}

Advertisement

\protect\hyperlink{after-top}{Continue reading the main story}

Supported by

\protect\hyperlink{after-sponsor}{Continue reading the main story}

\hypertarget{lizard-popsicles-anyone}{%
\section{Lizard Popsicles, Anyone?}\label{lizard-popsicles-anyone}}

The coolest reptiles on the planet occasionally freeze solid.

\includegraphics{https://static01.nyt.com/images/2020/08/04/science/00SCI-LIZARDPOPSICLES1/merlin_173962470_d8c1a6b6-b309-4002-88ac-c64c519216f9-articleLarge.jpg?quality=75\&auto=webp\&disable=upscale}

By Joshua Rapp Learn

\begin{itemize}
\item
  July 31, 2020
\item
  \begin{itemize}
  \item
  \item
  \item
  \item
  \item
  \end{itemize}
\end{itemize}

Stephan Halloy was conducting surveys on plants and wildlife on the high
plateaus around San Miguel de Tucumán in northwestern Argentina in the
1970s when he first encountered lizard Popsicles.

The mountains around the Argentine city climb rapidly to elevations of
13,000 to 16,400 feet, packing a multitude of ecological niches into a
relatively small area. The plateaus at the top can be hot in the
afternoon, but quickly drop to below freezing at night --- not exactly
the type of place you would expect to find a lizard.

Nonetheless, Dr. Halloy, now a senior adviser with the New Zealand
Ministry for Primary Industries, caught a few and placed them in a box
outside his tent overnight. ``When I opened the box the next morning
they were hard as wood --- you couldn't bend them,'' Dr. Halloy recalled
recently. ``They looked absolutely dead.''

But once the sun came up, the lizards began to thaw and were soon
running around in the box just like normal.

``Obviously I found that very surprising,'' Dr. Halloy said.

In the 1990s, Robert Espinoza, a biologist at California State
University, Northridge, heard this story from Dr. Halloy, and he has
been studying lizard Popsicles ever since. The lizards belong to the
genus Liolaemus, and research by Dr. Espinoza and his colleagues has
revealed that the lizards are indisputably the coolest on the planet.
Whereas tropical lizards like iguanas
\href{https://www.nytimes.com/2020/01/21/us/frozen-iguanas-miami-weather.html}{fall
from trees} when it gets cold, Liolaemus can supercool their bodies,
tolerate freezing and live farther south and at higher elevations than
any other known lizard species.

``They're real record holders,'' Dr. Espinoza said.

Liolaemus species have been found on the island of Tierra del Fuego, at
the southern end of the Americas, and one researcher has even heard
stories of them walking on Perito Moreno, a glacier in Patagonia. Most
Liolaemus are found in Argentina and Chile, although some are found as
far north as Peru. Containing more than 272 documented species,
Liolaemus is the second-largest genus among all mammals, birds and
reptiles, after only anoles, another type of lizard.

\hypertarget{supercool}{%
\subsection{Supercool}\label{supercool}}

Dr. Espinoza is still investigating how these lizards survive such cold
climates. In one experiment, his team fitted models of lizards, made of
hollow copper, with temperature loggers and placed them at one area at
13,369 feet in Salta Province. The models recorded temperatures as low
as minus 11.2 degrees Fahrenheit on the surface and 15.8 Fahrenheit
underground. (The lizards usually spend the night in burrows.)

The team then tested the cold adaptations of six species from varying
elevations. They found that some could survive cooling as low as 21.2
degrees Fahrenheit, although Dr. Espinoza suspects that wild lizards can
withstand colder temperatures. Liolaemus huasihuasicus, the species that
Dr. Halloy initially encountered, lives on a mountain about 1,640 feet
higher than the highest species Dr. Espinoza looked at --- a presumably
colder area.

Dr. Halloy noted in a 1989
\href{https://scholarspace.manoa.hawaii.edu/bitstream/10125/1214/v43n2-170-184.pdf}{publication}
that Liolaemus huasihuasicus could survive freezing at 14 degrees
Fahrenheit, but only when at an elevation of 13,944 feet; the lizards
died when cooled to 26.6 degrees Fahrenheit at tests conducted at 1,476
feet.

\includegraphics{https://static01.nyt.com/images/2020/08/04/science/00SCI-LIZARD-POPSICLES2/merlin_174042471_52013239-e01e-4753-aaf2-9537d01a55ee-articleLarge.jpg?quality=75\&auto=webp\&disable=upscale}

Dr. Espinoza and his co-authors found that Liolaemus lizards have
adapted abilities to deal with the cold through three mechanisms. Some
lizards avoid extreme cold by going underground. Others use a process of
supercooling; by staying completely still, they can allow their bodies
to drop below freezing without actually freezing solid. Finally, some
can also tolerate full-body freezing for short periods of time. Dr.
Espinoza said that some Liolaemus species likely made use of more than
one mechanism, depending on the conditions.

The strategy of full-body freezing is likely similar to that seen in
North American wood frogs, which stay frozen over winter thanks to an
antifreeze-like glucose solution that protects the cells; Dr. Espinoza
still needs to investigate this hypothesis to be sure. The world's
southernmost gecko, Darwin's marked gecko, another Argentine lizard that
Dr. Espinoza has
\href{https://www.sciencedirect.com/science/article/abs/pii/S030645651300020X}{studied},
most likely adopts the supercooling strategy.

\hypertarget{lizards-of-many-colors}{%
\subsection{Lizards of many colors}\label{lizards-of-many-colors}}

The reason Liolaemus lizards can withstand such cold temperatures and
high elevations may also explain why there are so many of the lizards.
Whereas there were only about 50 described to science when Dr. Halloy
worked on them in the late 1970s, there are now 272 species.

Dr. Espinoza and others have
\href{https://bioone.org/journals/Herpetologica/volume-64/issue-4/08-022R1.1/Two-New-Species-of-span-classgenus-speciesLiolaemus-span-Iguania/10.1655/08-022R1.1.short}{discovered}
a number of species, and his occasional co-author Fernando Lobo, a
zoologist at the National University of Salta in Argentina, has
\href{https://bioone.org/journals/Journal-of-Herpetology/volume-44/issue-2/08-334.1/Two-New-Species-of-Lizards-of-the-Liolaemus-montanus-Group/10.1670/08-334.1.short}{discovered}
30 or more species of Liolaemus and its close cousin, the genus
Phymaturus*.* In one case, Dr. Lobo discovered a species under his tent,
in cloudy, frozen weather in the Argentine province of Santa Cruz near
the Chilean border.

``They didn't look like any of the others,'' Dr. Lobo said. ``We
suspected they were a new one. We've had that excitement dozens of times
in these 25 years.''

At the current rate of discovery, Liolaemus will likely become the most
numerous genus of living mammals, reptiles and birds in coming years.

\hypertarget{a-species-pump}{%
\subsection{`A species pump'}\label{a-species-pump}}

The large number of Liolaemus species may be related to the mountainous
region where they live, Dr. Espinoza said. The Andes are relatively
young --- about the same evolutionary age as the lizards. He believes
that as the Andes pushed out of Earth's crust, the genus splintered into
myriad ecological niches that eventually resulted in new species.

``The Andes are just kind of a species pump creating all these new
types,'' Dr. Espinoza said.

Most of the lizards are similar in size, but they differ greatly in
color and even in diet and birth strategies. Dr. Lobo related a story
about an expedition in Argentina's Jujuy Province. During their work, a
local woman appeared from a small sheep- and llama-herding village in
the mountains.

``She told us very clearly which one was which species with their Indian
names, and said `That one lays eggs and that one gives birth,''' Dr.
Lobo said.

Image

L. magellanicus, encountered in El Chalten near Mount Fitz Roy in
Argentina by the author.Credit...Joshua Rapp Learn

Dr. Espinoza said that half the lizards give birth to live young,
perhaps because laying eggs in cold temperatures is likely not a recipe
for success for some species. In 2016 he also
\href{https://bioone.org/journals/Copeia/volume-104/issue-2/CH-15-381/The-First-Parthenogenetic-Pleurodont-Iguanian--A-New-All-female/10.1643/CH-15-381.short}{described}
a new species, Liolaemus ** parthenos, in which the females reproduce
through virgin birth, without fertilization from a male.

Melisa Olave, a researcher with Argentina's National Scientific and
Technical Research Council, who heard about the lizards on Perito
Moreno, conducted a recent
\href{https://onlinelibrary.wiley.com/doi/full/10.1111/jbi.13807?campaign=woletoc}{study}
showing that the rise of the Andes may not be the only factor driving
Liolaemus evolution and diversity. Liolaemus species have very low
extinction rates relative to other lizards. Their variation in habitat
use, generalist approach to diet --- some species are herbivorous while
others are omnivorous or carnivorous --- and different forms of
reproduction may be critical to explaining Liolaemus species' richness
and survival. She said that being a generalist is typically considered
advantageous, because it is easier to find suitable habitats in the
highly varied landscapes of South America's southern cone.

In other words, the high diversity of Liolaemus may be more a product of
low extinction rate than of habitat splintering.

Dr. Espinoza agreed that species persistence over time could be a
contributing factor in species richness, but he also believes that alone
cannot explain the diversity.

In any case, Dr. Olave shares the general sense of wonder infecting many
of the researchers who have worked with these lizards.

``Liolaemus species have an extraordinary ability to survive through
time,'' she said.

\textbf{\emph{{[}}\href{http://on.fb.me/1paTQ1h}{\emph{Like the Science
Times page on Facebook.}}} ****** \emph{\textbar{} Sign up for the}
\textbf{\href{http://nyti.ms/1MbHaRU}{\emph{Science Times
newsletter.}}\emph{{]}}}

Advertisement

\protect\hyperlink{after-bottom}{Continue reading the main story}

\hypertarget{site-index}{%
\subsection{Site Index}\label{site-index}}

\hypertarget{site-information-navigation}{%
\subsection{Site Information
Navigation}\label{site-information-navigation}}

\begin{itemize}
\tightlist
\item
  \href{https://help.nytimes.com/hc/en-us/articles/115014792127-Copyright-notice}{©~2020~The
  New York Times Company}
\end{itemize}

\begin{itemize}
\tightlist
\item
  \href{https://www.nytco.com/}{NYTCo}
\item
  \href{https://help.nytimes.com/hc/en-us/articles/115015385887-Contact-Us}{Contact
  Us}
\item
  \href{https://www.nytco.com/careers/}{Work with us}
\item
  \href{https://nytmediakit.com/}{Advertise}
\item
  \href{http://www.tbrandstudio.com/}{T Brand Studio}
\item
  \href{https://www.nytimes.com/privacy/cookie-policy\#how-do-i-manage-trackers}{Your
  Ad Choices}
\item
  \href{https://www.nytimes.com/privacy}{Privacy}
\item
  \href{https://help.nytimes.com/hc/en-us/articles/115014893428-Terms-of-service}{Terms
  of Service}
\item
  \href{https://help.nytimes.com/hc/en-us/articles/115014893968-Terms-of-sale}{Terms
  of Sale}
\item
  \href{https://spiderbites.nytimes.com}{Site Map}
\item
  \href{https://help.nytimes.com/hc/en-us}{Help}
\item
  \href{https://www.nytimes.com/subscription?campaignId=37WXW}{Subscriptions}
\end{itemize}
