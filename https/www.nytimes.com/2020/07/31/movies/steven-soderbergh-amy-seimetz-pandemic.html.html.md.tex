Sections

SEARCH

\protect\hyperlink{site-content}{Skip to
content}\protect\hyperlink{site-index}{Skip to site index}

\href{https://www.nytimes.com/section/movies}{Movies}

\href{https://myaccount.nytimes.com/auth/login?response_type=cookie\&client_id=vi}{}

\href{https://www.nytimes.com/section/todayspaper}{Today's Paper}

\href{/section/movies}{Movies}\textbar{}Steven Soderbergh and Amy
Seimetz Made the Pandemic Movies of the Moment

\url{https://nyti.ms/3jSIfkt}

\begin{itemize}
\item
\item
\item
\item
\item
\end{itemize}

\href{https://www.nytimes.com/spotlight/at-home?action=click\&pgtype=Article\&state=default\&region=TOP_BANNER\&context=at_home_menu}{At
Home}

\begin{itemize}
\tightlist
\item
  \href{https://www.nytimes.com/2020/07/28/books/time-for-a-literary-road-trip.html?action=click\&pgtype=Article\&state=default\&region=TOP_BANNER\&context=at_home_menu}{Take:
  A Literary Road Trip}
\item
  \href{https://www.nytimes.com/2020/07/29/magazine/bored-with-your-home-cooking-some-smoky-eggplant-will-fix-that.html?action=click\&pgtype=Article\&state=default\&region=TOP_BANNER\&context=at_home_menu}{Cook:
  Smoky Eggplant}
\item
  \href{https://www.nytimes.com/2020/07/27/travel/moose-michigan-isle-royale.html?action=click\&pgtype=Article\&state=default\&region=TOP_BANNER\&context=at_home_menu}{Look
  Out: For Moose}
\item
  \href{https://www.nytimes.com/interactive/2020/at-home/even-more-reporters-editors-diaries-lists-recommendations.html?action=click\&pgtype=Article\&state=default\&region=TOP_BANNER\&context=at_home_menu}{Explore:
  Reporters' Obsessions}
\end{itemize}

Advertisement

\protect\hyperlink{after-top}{Continue reading the main story}

Supported by

\protect\hyperlink{after-sponsor}{Continue reading the main story}

\hypertarget{steven-soderbergh-and-amy-seimetz-made-the-pandemic-movies-of-the-moment}{%
\section{Steven Soderbergh and Amy Seimetz Made the Pandemic Movies of
the
Moment}\label{steven-soderbergh-and-amy-seimetz-made-the-pandemic-movies-of-the-moment}}

His 2011 ``Contagion'' and her new thriller ``She Dies Tomorrow'' have
added resonance now. ``Why is this kind of imagery so compelling?''
Soderbergh asked.

\includegraphics{https://static01.nyt.com/images/2020/08/03/arts/00SODERBERGH-COMBO/00SODERBERGH-COMBO-articleLarge-v2.jpg?quality=75\&auto=webp\&disable=upscale}

\href{https://www.nytimes.com/by/kyle-buchanan}{\includegraphics{https://static01.nyt.com/images/2019/06/20/reader-center/kyle-buchanan-now/kyle-buchanan-now-thumbLarge-v2.png}}

By \href{https://www.nytimes.com/by/kyle-buchanan}{Kyle Buchanan}

\begin{itemize}
\item
  July 31, 2020
\item
  \begin{itemize}
  \item
  \item
  \item
  \item
  \item
  \end{itemize}
\end{itemize}

The directors Steven Soderbergh and Amy Seimetz had prepared for a
significant spring. Her new film
\href{https://www.youtube.com/watch?v=hcMFjCPkP3M}{``She Dies
Tomorrow''} was intended to debut at the South by Southwest film
festival in March, after which she'd fly to Detroit to act in
Soderbergh's crime drama ``Kill Switch.''

Those plans were made pre-pandemic, of course. As the spread of the
coronavirus in the United States reached alarming levels,
\href{https://www.nytimes.com/2020/03/06/arts/music/sxsw-cancelled.html}{South
by Southwest} was canceled, and ``Kill Switch'' was halted two weeks
before shooting. Hollywood had come to a standstill.

``I knew nothing was going back to normal anytime soon,'' Seimetz said.
``It was an interesting process to watch everyone face the facts.''

But a funny thing has happened to Soderbergh and Seimetz in lockdown, as
two of their movies have found new resonance during the Covid-19 era.
``Contagion,'' Soderbergh's 2011 pandemic thriller starring Matt Damon,
\href{https://www.nytimes.com/2020/03/04/business/media/coronavirus-contagion-movie.html}{rocketed
up the iTunes rental charts} in March, while ``She Dies Tomorrow,'' out
Friday in drive-ins and next week on digital, offers a more subjective
take on going viral: An anxious young woman (Kate Lyn Sheil) is
convinced she will die the next day, and anyone she tells is soon
gripped by the same fearful prophecy.

\includegraphics{https://static01.nyt.com/images/2020/07/31/arts/31soderbergh-seimetz2/merlin_175157076_1b5835fd-2211-4f14-bf0f-afa397072df5-articleLarge.jpg?quality=75\&auto=webp\&disable=upscale}

```Contagion' is relentlessly objective in its style and its formal
structure whereas Amy's film, by design, is this sort of fever dream in
both its style and its storytelling,'' Soderbergh told me this week
during a pandemic-focused Zoom call with Seimetz. ``It's interesting to
me as an example of how you can give artists the same central idea and
they will go off on two completely different tangents just because of
who they are.''

These are edited excerpts from our conversation.

\textbf{What were the first few months of lockdown like for both of
you?}

\textbf{STEVEN SODERBERGH} I knew in January from talking to my friends
in the world of epidemiology that this was serious. I would call them
and say, ``So what do you think?'' and their entire quote was ``It's
going to be bad.'' But you're trying to balance these very conflicting,
primal reactions to what's happening with the virus and your own career
advancement, so it's a really strange collision of your civic duty and
your ego. I had moments of trying to check myself.

\textbf{AMY SEIMETZ} Another thing that's evolved is my conversations
with executives. At the beginning of this, they were like, ``We're not
in the office anymore, so we're just going to call you all the time and
ask when things will be ready.'' It's been interesting to see how those
calls all dropped off in silence, which I prefer because it allows me to
do the work. The other side product of this is that I have an entire
freezer full of vegetable stock from the beginning of quarantine. Like,
huge lifetime supplies of lentils.

\textbf{What did you make of all the people who were drawn to
``Contagion'' during the early days of the pandemic?}

\textbf{SODERBERGH} It does pose a larger question about why we've had
this attraction to disaster movies. Why is this kind of imagery, this
spectacle of destruction, so compelling to us? Is it pure fantasy, or is
it something darker that's wound into us that we don't fully understand?

\textbf{SEIMETZ} I have a theory about that, because I actually
witnessed myself doing this. When the pandemic first started, in order
to not feel anxious, I was binge-watching completely mindless crap like
\href{https://www.netflix.com/title/80241027}{``Too Hot to Handle''} and
\href{https://www.netflix.com/title/80996601}{``Love Is Blind,''} and I
was like, ``Why do I feel so ill after? I'm just trying to take my mind
off things.'' And then I was like, ``I'll watch
\href{https://www.netflix.com/title/80998491}{`After Life,'} with Ricky
Gervais,'' and I was just sobbing the entire time, but I felt so much
better!

I needed to feel those emotions, like loss and sadness and fear. I think
suppressing them sort of makes you more anxious, so there's a cathartic
element to watching something like ``Contagion,'' which I found
strangely comforting.

Image

Matt Damon in ``Contagion,'' the 2011 Soderbergh film that drew new
audiences at the start of the pandemic.Credit...Claudette Barius/Warner
Bros.

\textbf{Steven, I found}
\textbf{\href{https://www.cidrap.umn.edu/news-perspective/2011/09/contagion-portrays-extreme-not-impossible-scenario}{this
take}} \textbf{on ``Contagion'' published back in 2011: ``It's one of
the most accurate movies I have seen on infectious disease outbreaks of
any type \ldots{} very dramatic, tense, exciting.'' Do you know who said
that about the film?}

\textbf{SODERBERGH} No, who?

\textbf{It was Dr. Anthony Fauci.}

\textbf{SODERBERGH} Oh wow! That's nice. We tried to be really rigorous
about the science, obviously, and I think I can defend most of that. The
biggest conceit that we indulged in was how quickly the vaccine was
found --- we compressed that timeline greatly, especially given what was
technologically possible then.

\textbf{Is there anything happening now that you didn't foresee when
making the film?}

\textbf{SODERBERGH} What I couldn't have predicted was the fracturing of
society that it would generate, and all of the things it would expose
when the tide goes out, so to speak. I didn't anticipate that it would
reveal so starkly the sort of economic disparity that we're aware of
intellectually but that a lot of us are able to insulate ourselves from
being directly affected by. Now, nobody escapes this. There are very few
people whose lives will not be completely altered by Covid.

The other thing we're all dealing with, that the movie doesn't address
because of its focus, is the general psychological effect on the public
because of an event like this. A cure, a vaccine, mitigating therapies
--- all that stuff is hugely important, but there's going to be an
incredible psychological toll that we're going to have to figure out how
to address. It's not like we can just turn a switch and have it be like
it never happened.

\textbf{Amy's film is more about that psychological toll, and how
quickly anxiety can become contagious itself.}

\textbf{SEIMETZ} The tricky thing about anxiety is sharing that you have
it can make other people anxious, and there's a feeling that you're
burdening them by doing so. Your anxiety then becomes their anxiety, in
a way that's very literal in this movie. It's happened with the news
cycle, too: I found myself becoming completely addicted to the news,
getting anxious from it, and then compulsively watching it more. So it's
also about news cycles spreading panic and the addiction to panic.

Image

Tunde Adebimpe in a scene from ``She Dies Tomorrow.''Credit...Jay
Keitel/Neon

\textbf{That reminds me of the Jude Law character from ``Contagion,''
who capitalized on the country's panic to hawk a fake miracle cure. I've
seen people reference that character when President Trump touts the
unproven hydroxychloroquine as a cure for the coronavirus.}

\textbf{SODERBERGH} It was amusing to me that at one point, there was a
suggestion from outside the creative team that we cut that character out
of the film. We'd have these test screenings and people would hate him!
The cards would come back and I'd say, ``I know! He's supposed to be
polarizing.'' But we felt pretty confident that the issues brought up by
Jude Law's character in this film would be very central to the narrative
when this thing does happen.

He's also not wrong all the time, like with his rant on the park bench
where he describes how they're rushing the trials for the vaccine and
how the pharmaceutical companies are going to be the ones who benefit.
Look, I'm obviously pro-vaccine, but when you're talking about putting
something into the bodies of everyone on the planet, that's a very, very
serious thing. You could have a side effect that goes down to a decimal
point you can barely see, but if you're going to give it to everyone,
that can still be tens of millions of people that have a negative
reaction. In everybody's rush to get to the other end of this, we really
do have to be careful here.

Image

Jude Law as a conspiracy theorist in ``Contagion.''Credit...Claudette
Barius/Warner Bros.

\textbf{Steven, you're heading a Directors Guild committee to figure out
how to get Hollywood back to work safely. What are the problems you're
facing?}

\textbf{SODERBERGH} I think the biggest issue now is because of the
resurgence {[}of the virus{]}, how do we get access to the resources and
the personnel that we need to run these protocols to keep a set safe?
It's one thing to do one or two projects and see how it goes, but
there's a movement in the last two or three weeks to get lots of
productions back up and running at the same time. That's going to be
tricky.

\textbf{With baseball, they got it back up and running but there's
already been}
\textbf{\href{https://www.nytimes.com/2020/07/27/sports/baseball/marlins-game-canceled.html}{a
pretty significant outbreak}. Could Hollywood face the same risk?}

\textbf{SODERBERGH} Having spent a lot of the weekend very happily
watching baseball, I was not happy about the Marlins, but I think that's
a much more difficult situation than we're confronting because of the
nature of the game and the fact that they're traveling all over the
place. We have an ability on a project to control how we move, where we
move, how many people come with us --- it's something that can be
manipulated to keep people safe.

I think if we can withstand the economic surcharge that's going to come
with keeping a project safe --- which I estimate is between 15 to 20
percent of the budget, depending on the project --- and if we can scale
this quickly enough, then I know we can keep people safe. If you follow
these protocols we're about to finish up with, I feel pretty confident
saying that you're not going to get sick at work. If you got sick on one
of our projects, it was during the 12 to 14 hours when I didn't have you
and I couldn't control your behavior. That's going to be the trick, is
all of this downtime when you don't know what people are up to.

\textbf{But what happens if people} \emph{\textbf{do}} \textbf{get sick
in that downtime and then come to set?}

\textbf{SODERBERGH} Look, it's complex, but Joel Coen is shooting
``Macbeth'' in L.A. right now, and there's a crew member who's been
{[}keeping{]} a pretty detailed diary. And it seems to be working!
They're using the rapid testing, which isn't as accurate as the
full-blown nasal
\href{https://www.nytimes.com/2020/07/06/health/fast-coronavirus-tests.html}{PCR
test}, but they're making up for that by testing a lot, eight times for
every five-day workweek. That's a good approach.

\textbf{Amy, is the surcharge Steven mentioned going to limit the amount
of independent films that can be made over the next year?}

\textbf{SEIMETZ} I think there's going to be a conversation with unions
to ease up on some of the crewing mandates, because you can't really
shoot with a larger crew when you don't have enough of a budget for
those protocols. From talking to other filmmakers, they're thinking
about small crews and small casts and shooting outside, so there's ways
to do it. With ``She Dies Tomorrow,'' the {[}Directors Guild{]} was very
gracious in allowing me to have a pared-down crew of about six people
--- we were pretty much following protocol {[}long before there was a{]}
protocol.

\textbf{What about bigger films? How will ``Kill Switch'' change when
you resume shooting that?}

\textbf{SODERBERGH} I'll tell you in eight weeks. A lot of this is all
abstract until you get on set and actually see how this stuff works, and
I intend to be very public in my experience of making that movie in
order to educate people. I'm sure I'm going to learn a lot, and I'm sure
a lot of the assumptions that we're making will turn out to need
adjustment. This is a living thing, and it's going to have to evolve,
but in what way won't be clear until we get out there.

Advertisement

\protect\hyperlink{after-bottom}{Continue reading the main story}

\hypertarget{site-index}{%
\subsection{Site Index}\label{site-index}}

\hypertarget{site-information-navigation}{%
\subsection{Site Information
Navigation}\label{site-information-navigation}}

\begin{itemize}
\tightlist
\item
  \href{https://help.nytimes.com/hc/en-us/articles/115014792127-Copyright-notice}{©~2020~The
  New York Times Company}
\end{itemize}

\begin{itemize}
\tightlist
\item
  \href{https://www.nytco.com/}{NYTCo}
\item
  \href{https://help.nytimes.com/hc/en-us/articles/115015385887-Contact-Us}{Contact
  Us}
\item
  \href{https://www.nytco.com/careers/}{Work with us}
\item
  \href{https://nytmediakit.com/}{Advertise}
\item
  \href{http://www.tbrandstudio.com/}{T Brand Studio}
\item
  \href{https://www.nytimes.com/privacy/cookie-policy\#how-do-i-manage-trackers}{Your
  Ad Choices}
\item
  \href{https://www.nytimes.com/privacy}{Privacy}
\item
  \href{https://help.nytimes.com/hc/en-us/articles/115014893428-Terms-of-service}{Terms
  of Service}
\item
  \href{https://help.nytimes.com/hc/en-us/articles/115014893968-Terms-of-sale}{Terms
  of Sale}
\item
  \href{https://spiderbites.nytimes.com}{Site Map}
\item
  \href{https://help.nytimes.com/hc/en-us}{Help}
\item
  \href{https://www.nytimes.com/subscription?campaignId=37WXW}{Subscriptions}
\end{itemize}
