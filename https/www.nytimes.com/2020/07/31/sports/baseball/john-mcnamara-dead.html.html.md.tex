Sections

SEARCH

\protect\hyperlink{site-content}{Skip to
content}\protect\hyperlink{site-index}{Skip to site index}

\href{https://www.nytimes.com/section/sports/baseball}{Baseball}

\href{https://myaccount.nytimes.com/auth/login?response_type=cookie\&client_id=vi}{}

\href{https://www.nytimes.com/section/todayspaper}{Today's Paper}

\href{/section/sports/baseball}{Baseball}\textbar{}John McNamara, Red
Sox Skipper in '86 Series Loss, Dies at 88

\url{https://nyti.ms/2Pfflgh}

\begin{itemize}
\item
\item
\item
\item
\item
\end{itemize}

Advertisement

\protect\hyperlink{after-top}{Continue reading the main story}

Supported by

\protect\hyperlink{after-sponsor}{Continue reading the main story}

\hypertarget{john-mcnamara-red-sox-skipper-in-86-series-loss-dies-at-88}{%
\section{John McNamara, Red Sox Skipper in '86 Series Loss, Dies at
88}\label{john-mcnamara-red-sox-skipper-in-86-series-loss-dies-at-88}}

He led six big league teams with some success, but he's best remembered
for questionable managerial moves in a crushing (for Boston) sixth game
against the Mets.

\includegraphics{https://static01.nyt.com/images/2020/08/01/obituaries/01McNamara-obit1/merlin_175081713_f9bbd914-b8a7-4c5f-9062-8d7cea3f591e-articleLarge.jpg?quality=75\&auto=webp\&disable=upscale}

\href{https://www.nytimes.com/by/richard-sandomir}{\includegraphics{https://static01.nyt.com/images/2018/12/10/multimedia/author-richard-sandomir/author-richard-sandomir-thumbLarge.png}}

By \href{https://www.nytimes.com/by/richard-sandomir}{Richard Sandomir}

\begin{itemize}
\item
  July 31, 2020
\item
  \begin{itemize}
  \item
  \item
  \item
  \item
  \item
  \end{itemize}
\end{itemize}

\href{https://www.baseball-reference.com/managers/mcnamjo99.shtml}{John
McNamara}, who managed the Boston Red Sox to within one out of a
\href{https://www.baseball-reference.com/postseason/1986_WS.shtml}{World
Series championship against the Mets in 1986} --- and whose strategy in
the critical sixth game has been questioned ever since --- died on
Tuesday at his home in Brentwood, Tenn., a Nashville suburb. He was 88.

His wife, Ellen McNamara, who confirmed the death, said the cause had
not been determined.

\href{https://sabr.org/bioproj/person/john-mcnamara/}{McNamara}was hired
by the Red Sox in 1985 --- it was his fifth Major League managerial job
--- and the next season guided them to a 95-66 record and the American
League pennant. Then, with the Red Sox leading the Mets three games to
two in the World Series, McNamara's moves became, at the time, the
latest woeful chapter in the saga of a team that had not won a
championship since 1918.

With the Red Sox leading Game 6, 3-2, McNamara removed his ace pitcher,
Roger Clemens, after seven strong innings and replaced him with Calvin
Schiraldi, who let the Mets tie the game in the eighth. McNamara then
kept Schiraldi in the game until the 10th, when he gave up three hits.

McNamara might have replaced his hobbling first baseman,
\href{https://www.nytimes.com/2019/05/27/obituaries/bill-buckner-all-star-shadowed-by-world-series-error-dies-at-69.html}{Bill
Buckner}, late in the game with Dave Stapleton, a better fielder, as he
had done in Boston's three victories in the series. But he did not.

The Red Sox went ahead, 5-3, in the 10th. But the Mets famously won in
the bottom of the inning with three runs on a single off Schiraldi; a
wild pitch by Bob Stanley, who had relieved Schiraldi; and a ground ball
hit by Mookie Wilson that skittered between Buckner's legs, scoring Ray
Knight.

\includegraphics{https://static01.nyt.com/images/2020/08/01/obituaries/01McNamara-obit2/merlin_175081536_15add0e0-8616-41da-9e65-87f3c9d40a8b-articleLarge.jpg?quality=75\&auto=webp\&disable=upscale}

McNamara insisted that Clemens had asked to be taken out of the game
after the seventh inning. In ``1986: A Postseason Remembered,''
\href{https://www.mlb.com/news/network-recalls-1986-postseason/c-25882718\#:~:text=9\%20at\%209\%20p.m.\%20ET,-November\%203\%2C\%202011\&text=After\%20one\%20of\%20the\%20most,9\%20at\%209\%20p.m.\%20ET.}{an
MLB Network documentary} from 2011, he recalled waiting on the dugout
steps as Clemens walked off the field.

``And he came down the steps and he said,
\href{https://www.nytimes.com/2011/11/08/sports/baseball/former-red-sox-manager-john-mcnamara-recalls-final-out-that-wasnt-to-be.html}{`That's
all I can pitch.'}Quote unquote,'' McNamara said. He was incredulous, he
said, but then Clemens showed him a paper cut on his middle finger.

Clemens acknowledged in the documentary that he had had blood on his
finger but denied that he had asked to be taken out.

McNamara countered, ``That is not the truth, and I don't lie.''

McNamara never second-guessed himself for keeping Buckner in the game,
saying that Buckner, not Stapleton, was his best first baseman.

``Stapleton's nickname was Shakey,'' he said in the documentary. ``And
you know what that implies.''

The Red Sox jumped to a 3-0 lead in Game 7, but the Mets rebounded to
win, 8-5, and take the Series. The Red Sox would not win a World Series
until 2004.

McNamara found some relief from his heartbreak a few days later, when he
was voted the 1986 American League manager of the year.

Image

McNamara during the fateful Game 6 of the 1986 series at Shea Stadium in
New York. On base for the Mets was Ray Knight. The Boston players were
Spike Owen (5), Marty Barrett (17), Calvin Schiraldi (right) and Bill
Buckner (behind McNamara). Buckner died last year at 69.Credit...Ruby
Washington/The New York Times

John Francis McNamara was born on June 4, 1932, in Sacramento. His
father, John, was a railroad worker from Ireland. His mother, Josephine
(Lane) McNamara, worked for the state of California after her husband
died in 1944. Young John played baseball and basketball in high school
and signed with the St. Louis Cardinals in 1951 for a \$12,000 bonus
(the equivalent of about \$120,000 today).

A catcher, he played 14 seasons in the minor leagues but never made it
to the majors. ``I could catch and throw with anybody, but I knew I
wasn't going to make the big leagues,'' he told The Hartford Courant in
1985.

His managerial career began in 1959 in Lewiston, Idaho, whose minor
league club became a low-level farm team of the Kansas City Athletics
the next season. McNamara eventually managed at higher tiers in the A's
organization, nurturing future major leaguers like Reggie Jackson.

When Jackson was inducted into the Baseball Hall of Fame in 1993, he
recalled the decency that McNamara had shown him in 1967 as the manager
of the Birmingham A's in Alabama, where Jackson, as a Black man,
continued to face discrimination.

\href{https://www.baltimoresun.com/news/bs-xpm-1993-08-02-1993214108-story.html}{``He
wouldn't allow the team to eat in a restaurant where I wasn't welcome,''
Jackson said.}

McNamara joined the A's as a coach in 1968, the team's first year in
Oakland, Calif., and replaced Hank Bauer as manager late in the
following season. McNamara himself was fired after the 1970 season.

Through the '70s and early '80s he managed the San Diego Padres, the
Cincinnati Reds (leading them to the National League Championship
Series, which they lost to Pittsburgh) and the California Angels. He
left the Angels after the 1984 season to take over the Red Sox from
\href{https://www.nytimes.com/2010/07/22/sports/baseball/22Houk.html\#:~:text=Ralph\%20Houk\%2C\%20Yankees\%20Manager\%2C\%20Dies\%20at\%2090,-By\%20Richard\%20Goldstein\&text=Ralph\%20Houk\%2C\%20a\%20third\%2Dstring,He\%20was\%2090.\&text=\%E2\%80\%9CI'm\%20Ralph\%20Houk.\%E2\%80\%9D}{Ralph
Houk}, who had retired.

The 1986 season was the high point of McNamara's time in Boston. The Red
Sox slumped to fifth place in the American League East in 1987, and he
was fired the next season during the All-Star break. He then managed the
Cleveland Indians in 1990 and through part of the 1991 season before
being dismissed by them as well.

He next worked in the Angels organization for five years before briefly
taking over as interim manager in 1996 --- his last assignment as a
skipper. He retired with a career record
of\href{https://www.baseball-reference.com/managers/mcnamjo99.shtml}{1,160
wins and 1,233 losses.}

Soon after the 1996
season\href{https://www.upi.com/Archives/1996/10/17/McNamara-grandsons-killed-by-dad/8841845524800/}{,
two of McNamara's grandson}s, ages 6 and 4, were killed by their father,
McNamara's son-in-law, who then killed himself.

McNamara married Ellen Goode in 1984. His previous marriage had ended in
divorce. In addition to his wife, he is survived by his daughters, Peggy
McNamara and Susan Salsbery, and a son, Michael --- all from his first
marriage --- as well as eight grandchildren and a great-grandson.

Image

McNamara after Gary Carter of the Mets hit a home run in the eighth
inning of Game 4 of the 1986 World Series in Boston. The Mets won that
game.Credit...Paul Benoit/Associated Press

After the Mets celebrated their crushing victory over the Red Sox in
Game 6 of the 1986 series, a disappointed McNamara was asked by
reporters about his team's long history of not having won a title in 68
years.

``I don't know anything about history,'' he replied, his voice toneless
and his expression described as a clenched fist, ``and don't tell me
anything about that choke crap."

Advertisement

\protect\hyperlink{after-bottom}{Continue reading the main story}

\hypertarget{site-index}{%
\subsection{Site Index}\label{site-index}}

\hypertarget{site-information-navigation}{%
\subsection{Site Information
Navigation}\label{site-information-navigation}}

\begin{itemize}
\tightlist
\item
  \href{https://help.nytimes.com/hc/en-us/articles/115014792127-Copyright-notice}{©~2020~The
  New York Times Company}
\end{itemize}

\begin{itemize}
\tightlist
\item
  \href{https://www.nytco.com/}{NYTCo}
\item
  \href{https://help.nytimes.com/hc/en-us/articles/115015385887-Contact-Us}{Contact
  Us}
\item
  \href{https://www.nytco.com/careers/}{Work with us}
\item
  \href{https://nytmediakit.com/}{Advertise}
\item
  \href{http://www.tbrandstudio.com/}{T Brand Studio}
\item
  \href{https://www.nytimes.com/privacy/cookie-policy\#how-do-i-manage-trackers}{Your
  Ad Choices}
\item
  \href{https://www.nytimes.com/privacy}{Privacy}
\item
  \href{https://help.nytimes.com/hc/en-us/articles/115014893428-Terms-of-service}{Terms
  of Service}
\item
  \href{https://help.nytimes.com/hc/en-us/articles/115014893968-Terms-of-sale}{Terms
  of Sale}
\item
  \href{https://spiderbites.nytimes.com}{Site Map}
\item
  \href{https://help.nytimes.com/hc/en-us}{Help}
\item
  \href{https://www.nytimes.com/subscription?campaignId=37WXW}{Subscriptions}
\end{itemize}
