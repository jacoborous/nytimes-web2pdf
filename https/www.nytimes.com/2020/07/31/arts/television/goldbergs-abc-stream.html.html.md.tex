Sections

SEARCH

\protect\hyperlink{site-content}{Skip to
content}\protect\hyperlink{site-index}{Skip to site index}

\href{https://www.nytimes.com/section/arts/television}{Television}

\href{https://myaccount.nytimes.com/auth/login?response_type=cookie\&client_id=vi}{}

\href{https://www.nytimes.com/section/todayspaper}{Today's Paper}

\href{/section/arts/television}{Television}\textbar{}Comfort Viewing:
Why I Still Love `The Goldbergs'

\url{https://nyti.ms/3k1Tirp}

\begin{itemize}
\item
\item
\item
\item
\item
\end{itemize}

Advertisement

\protect\hyperlink{after-top}{Continue reading the main story}

Supported by

\protect\hyperlink{after-sponsor}{Continue reading the main story}

\hypertarget{comfort-viewing-why-i-still-love-the-goldbergs}{%
\section{Comfort Viewing: Why I Still Love `The
Goldbergs'}\label{comfort-viewing-why-i-still-love-the-goldbergs}}

The period sitcom about a Jewish family in the '80s has for seven
seasons been a weekly gift of old-fashioned zingers.

\includegraphics{https://static01.nyt.com/images/2020/07/31/arts/31comfort-goldbergs1/merlin_82047917_75c6c2ae-273e-478f-a891-7d18fa33a1b0-articleLarge.jpg?quality=75\&auto=webp\&disable=upscale}

By Noel Murray

\begin{itemize}
\item
  Published July 31, 2020Updated Aug. 1, 2020, 10:22 a.m. ET
\item
  \begin{itemize}
  \item
  \item
  \item
  \item
  \item
  \end{itemize}
\end{itemize}

In the TV business, hit shows inevitably inspire imitators, but ABC's
programming department maybe went a bit too crazy at the copy machine
after ``Modern Family'' became a smash in the fall of 2009. Before long,
the network had added ``The Middle,'' ``The Neighbors,'' ``Trophy
Wife,'' ``Fresh off the Boat,'' ``black-ish,'' ``The Real O'Neals,''
``Speechless'' and ``The Kids Are Alright'' --- all sitcoms about the
complications of domestic life, from different perspectives. Black?
Asian-American? Catholic? Gay? Disabled? Stuck in the '70s? From outer
space? For much of the 2010s, ABC gave multiple kinds of families a
prime-time platform.

After seven seasons and counting, ``The Goldbergs'' (available to stream
\href{https://www.hulu.com/series/the-goldbergs-a43a85fb-d4c8-4d11-9c28-070153643bec}{on
Hulu} with recent episodes \href{https://abc.com/shows/the-goldbergs}{on
ABC's website}) is one of the last of the shows from this era still
airing. Next to ``Modern Family'' and ``The Middle,'' it's the one
that's run the longest, perhaps because it's a distillation of the best
parts of the ABC family sitcom formula, combining heartfelt sentiment
and cranked-up farce.

Most of the shows in that big 2010s ABC wave had one unique hook. ``The
Goldbergs'' has two: It's about a Jewish family, and it's set in the
1980s. Based on the creator Adam F. Goldberg's own childhood in the
Philadelphia suburb Jenkintown, the series is narrated by the comedian
Patton Oswalt as the grown-up Adam, who tells semi-true stories from an
era he calls ``1980-something.'' Some episodes are about life in the
time of John Hughes movies and video arcades. Others are about living in
a household with a smothering mother, an emotionally distant father and
bickering siblings.

When ``The Goldbergs'' debuted in 2013, some early reviews dismissed the
series as too loud, too busy and too beholden to nostalgia. But to me
it's a finely tuned comic contraption, with an ace multigenerational
cast. The great George Segal plays the family's easygoing playboy
grandfather, while the improv comedy veterans Wendi McLendon-Covey
(``Reno 911!'') and Jeff Garlin (``Curb Your Enthusiasm'') play the
parents, Beverly and Murray. Three impressive young actors (Sean
Giambrone as Adam, Troy Gentile as Barry and Hayley Orrantia as Erica)
each bring an outsize, offbeat energy to the brash, neurotic Goldberg
kids.

That cast --- even more than the '80s references --- is what really
makes ``The Goldbergs'' special. They work together like gears in a
cuckoo clock.

\includegraphics{https://static01.nyt.com/images/2020/07/31/arts/31comfort-goldbergs2/merlin_82340468_229d786d-228a-462e-b586-43f172dfb251-articleLarge.jpg?quality=75\&auto=webp\&disable=upscale}

\hypertarget{what-the-skeptics-miss}{%
\subsection{What the Skeptics Miss}\label{what-the-skeptics-miss}}

Humor is subjective, granted. But as someone who appreciates a sturdily
crafted, snappily delivered joke, I've always found ``The Goldbergs'' to
be one of TV's most consistently funny sitcoms. When the cast really
gets revved up --- aided by editing that cuts quickly between their
broad gestures and the punchy dialogue --- it's like watching skilled
acrobats flipping wildly through the air and landing gracefully.

In one episode this past season, the family gathered for a ``game
night'' that devolved quickly into an argument over what to play,
accompanied by a rehashing of past grievances that kept getting more and
more preposterous. The one-liners flew by at a dizzying pace for three
solid minutes. (Reminded he once angrily kicked an Operation board so
hard he needed an actual operation, the bullheaded Barry cheerfully
noted, ``They sewed my toe back on, and now I have to think extra-hard
when I want to wiggle it!'')

Even mediocre ``Goldbergs'' episodes usually feature one or two scenes
at the level of the game night repartee. The show has been a weekly gift
for fans of old-fashioned zingers.

Image

Hayley Orrantia, left, is one of several young actors who bring an
outsize, offbeat energy to the neurotic Goldberg children.~Credit...ABC

\hypertarget{what-i-overlook}{%
\subsection{What I Overlook}\label{what-i-overlook}}

One of the biggest knocks against ``The Goldbergs'' is that its version
of ``the '80s'' is more a vague concept than the authentic portrait of a
historical decade. Adam F. Goldberg and his writers consciously choose
to skip back in forth in the cultural timeline while telling an
otherwise chronological story about three Jenkintown teens. One week the
characters will be obsessed with the 1985 movie ``Commando,'' starring
Arnold Schwarzenegger. A few weeks later they'll all be talking about
``Dead Poets Society,'' from 1989. It's a mess.

This lack of interest in actual dates does matter. Goldberg's writers
have shown a genuine affection and understanding for '80s pop culture,
but they've unintentionally been disrespecting that culture's larger
narrative. Music, TV, movies and games evolve with their times, affected
by one another and by sociopolitical trends. ``The Goldbergs'' doesn't
properly represent any of those particulars. It's a homogenized blend of
all of its influences, much sweeter and smoother than the times that
produced them.

But ultimately, the vagueness works for this show. ``The Goldbergs'' is
meant to be a little 22-minute shot of joy and warmth. And it's pretty
potent.

Image

When ``The Goldbergs'' debuted in 2013, some reviews dismissed the
series as too loud, too busy and too beholden to nostalgia. To the
writer, it's a finely tuned comic contraption.Credit...Tony Rivetti/ABC

\hypertarget{how-long-can-the-show-go-on}{%
\subsection{How Long Can the Show Go
On?}\label{how-long-can-the-show-go-on}}

It's somewhat surprising that ``The Goldbergs'' keeps getting renewed.
It has never finished a season in the Nielsen Top 20. While the series
is available in full on Hulu (and airs nearly every day in syndication),
it rarely gets talked about on Twitter or turned into memes.

And frankly, a year ago I might've said ``The Goldbergs'' had run its
course. How many times can Adam get obsessed with some '80s movie? How
often can Beverly promise to give her kids more space? Do we need to
hear Oswalt end yet another episode with a maudlin, ``That's the thing
about family \ldots{}''?

But even though the actors playing the younger Goldbergs stopped looking
like fresh-faced teenagers years ago, I'm not ready for them or anyone
else in the cast to disappear from my TV. They're uniquely talented and
funny, and I'll miss them when they're gone.

And if I'm being honest, seeing them is a welcome reminder of the days
when ABC aired a half-dozen ``Goldbergs''-like sitcoms each week. It
makes me feel nostalgic --- not for the 1980s but for the 2010s.

Advertisement

\protect\hyperlink{after-bottom}{Continue reading the main story}

\hypertarget{site-index}{%
\subsection{Site Index}\label{site-index}}

\hypertarget{site-information-navigation}{%
\subsection{Site Information
Navigation}\label{site-information-navigation}}

\begin{itemize}
\tightlist
\item
  \href{https://help.nytimes.com/hc/en-us/articles/115014792127-Copyright-notice}{©~2020~The
  New York Times Company}
\end{itemize}

\begin{itemize}
\tightlist
\item
  \href{https://www.nytco.com/}{NYTCo}
\item
  \href{https://help.nytimes.com/hc/en-us/articles/115015385887-Contact-Us}{Contact
  Us}
\item
  \href{https://www.nytco.com/careers/}{Work with us}
\item
  \href{https://nytmediakit.com/}{Advertise}
\item
  \href{http://www.tbrandstudio.com/}{T Brand Studio}
\item
  \href{https://www.nytimes.com/privacy/cookie-policy\#how-do-i-manage-trackers}{Your
  Ad Choices}
\item
  \href{https://www.nytimes.com/privacy}{Privacy}
\item
  \href{https://help.nytimes.com/hc/en-us/articles/115014893428-Terms-of-service}{Terms
  of Service}
\item
  \href{https://help.nytimes.com/hc/en-us/articles/115014893968-Terms-of-sale}{Terms
  of Sale}
\item
  \href{https://spiderbites.nytimes.com}{Site Map}
\item
  \href{https://help.nytimes.com/hc/en-us}{Help}
\item
  \href{https://www.nytimes.com/subscription?campaignId=37WXW}{Subscriptions}
\end{itemize}
