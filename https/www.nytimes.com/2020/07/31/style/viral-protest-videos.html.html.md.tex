Sections

SEARCH

\protect\hyperlink{site-content}{Skip to
content}\protect\hyperlink{site-index}{Skip to site index}

\href{https://www.nytimes.com/section/style}{Style}

\href{https://myaccount.nytimes.com/auth/login?response_type=cookie\&client_id=vi}{}

\href{https://www.nytimes.com/section/todayspaper}{Today's Paper}

\href{/section/style}{Style}\textbar{}Why Protest Tactics Spread Like
Memes

\url{https://nyti.ms/3jWsXLt}

\begin{itemize}
\item
\item
\item
\item
\item
\end{itemize}

\href{https://www.nytimes.com/news-event/george-floyd-protests-minneapolis-new-york-los-angeles?action=click\&pgtype=Article\&state=default\&region=TOP_BANNER\&context=storylines_menu}{Race
and America}

\begin{itemize}
\tightlist
\item
  \href{https://www.nytimes.com/2020/07/26/us/protests-portland-seattle-trump.html?action=click\&pgtype=Article\&state=default\&region=TOP_BANNER\&context=storylines_menu}{Protesters
  Return to Other Cities}
\item
  \href{https://www.nytimes.com/2020/07/24/us/portland-oregon-protests-white-race.html?action=click\&pgtype=Article\&state=default\&region=TOP_BANNER\&context=storylines_menu}{Portland
  at the Center}
\item
  \href{https://www.nytimes.com/2020/07/23/podcasts/the-daily/portland-protests.html?action=click\&pgtype=Article\&state=default\&region=TOP_BANNER\&context=storylines_menu}{Podcast:
  Showdown in Portland}
\item
  \href{https://www.nytimes.com/interactive/2020/07/16/us/black-lives-matter-protests-louisville-breonna-taylor.html?action=click\&pgtype=Article\&state=default\&region=TOP_BANNER\&context=storylines_menu}{45
  Days in Louisville}
\end{itemize}

Advertisement

\protect\hyperlink{after-top}{Continue reading the main story}

Supported by

\protect\hyperlink{after-sponsor}{Continue reading the main story}

\hypertarget{why-protest-tactics-spread-like-memes}{%
\section{Why Protest Tactics Spread Like
Memes}\label{why-protest-tactics-spread-like-memes}}

When items like umbrellas and leaf blowers are subverted into objects of
resistance, they become very shareable.

\includegraphics{https://static01.nyt.com/images/2020/08/03/fashion/03_Umbrella_Thumb/03_Umbrella_Thumb-superJumbo.jpg}

July 31, 2020

By Tracy Ma

With \href{https://www.nytimes.com/by/natalie-shutler}{Natalie Shutler}

Written by \href{https://www.nytimes.com/by/jonah-engel-bromwich}{Jonah
Engel Bromwich}

Video by \href{https://www.nytimes.com/by/shane-oneill}{Shane O'Neill}

A video frame captured in Hong Kong in August 2019 shows a group of
pro-democracy protesters, smoke pluming toward them, racing to place an
\href{https://slate.com/news-and-politics/2019/06/hong-kong-tear-gas-water-bottle.html}{orange
traffic cone over a tear-gas canister}. A video taken nine months later
and 7,000 miles away, at a Black Lives Matter protest in Minneapolis,
shows another small group using the same maneuver. Two moments, two
continents, two cone placers, their postures nearly identical.

Image

Hong Kong, August 2019.Credit...Getty Images

Image

Washington, D.C., May 2020.Credit...Getty Images

Image

Hong Kong, August 2019.Credit...Vincent Thian/Associated Press

Image

Portland, Ore., July 2020.Credit...David Swanson/EPA, via Shutterstock

Images of protest spread on social media reveal many other matching
moments from opposite sides of the world, and they often feature
everyday objects wielded ingeniously.

Leaf blowers are used to diffuse clouds of tear gas; hockey sticks and
tennis rackets are brandished to bat canisters back toward authorities;
high-power laser pointers are used to thwart surveillance cameras; and
plywood, boogie boards, umbrellas and more have served as shields to
protect protesters from projectiles and create barricades.

An Xiao Mina, a researcher at the Berkman Klein Center for Internet and
Society at Harvard University, has studied these echoes. In the summer
of 2014, when the
\href{https://www.nytimes.com/2019/08/30/world/asia/hong-kong-protests.html}{Umbrella
Movement} in Hong Kong and the
\href{https://www.nytimes.com/2016/08/23/us/how-blacklivesmatter-came-to-define-a-movement.html}{Black
Lives Matter protests} in the United States that followed the
\href{https://www.nytimes.com/interactive/2014/08/13/us/ferguson-missouri-town-under-siege-after-police-shooting.html}{police
killing of Michael Brown} were taking place, she noted that the
protesters spoke a common language, even sharing the same hand gesture
characterized by the chant ``Hands up, don't shoot.''

Occasionally, there was even direct acknowledgment between the disparate
groups, ``as when Ferguson protesters donned umbrellas against the rain
and cheekily thanked protesters in Hong Kong for the idea,'' Ms. Mina
wrote in her 2018 book, ``Memes to Movements.''

\includegraphics{https://static01.nyt.com/images/2020/08/04/fashion/04_LeafBlower_Thumb/04_LeafBlower_Thumb-superJumbo.jpg}

But often, she noted, the images' similarity was unwitting. In their
spread, their simultaneity and their indirect influence on each other,
the protest videos had all the characteristics of memes, those units of
culture and behavior that spread rapidly online. The same cultural
transfer that gives us uncanny
\href{https://www.nytimes.com/2020/07/14/style/what-is-the-cake-meme.html}{cake-slicing
memes} and
\href{https://www.nytimes.com/2018/08/23/style/shiggy-challenges-inmyfeelings.html}{viral
challenges} also advances the language of protest.

``We live in this world of attention dynamics so it makes sense that
tactics start to converge,'' Ms. Mina said. She called the images'
tendency to build on each other ``memetic piggybacking,'' and noted that
everyday items that are subverted into objects of protest are
``inherently charismatic.''

Image

Hong Kong, November 2019.Credit...Getty Images

Image

Portland, Ore., July 2020.Credit...Marcio Jose Sanchez/Associated Press

Franklin López, a founder and former member of
\href{https://sub.media/}{Sub.media}, an anarchist video collective that
has filmed dozens of protests, said that ``videos shared through social
media and mainstream media reports become rough `how-to guides' on
protest tactics.''

``You see peeps in Hong Kong using umbrellas as countersurveillance
tools and folks over here will say, `hey, brilliant idea!' and you'll
see umbrellas at the next militant protests,'' he said.

Of course, it's not just social media mimicry. Ms. Mina pointed out that
``activists from around the world do actively learn from each other and
exchange tactical tips.''

\includegraphics{https://static01.nyt.com/images/2020/08/06/fashion/06_Laser_B_Thumb/06_Laser_B_Thumb-superJumbo.jpg}

On the topic of direct communication between groups in Hong Kong and the
United States, Mr. López said: ``Texts outlining not only tactics and
strategies but reports of what worked and what didn't are shared and
translated, but also talked about in in-person events, film screenings
and internet talks.''

In June, for example, Lausan, a group that formed during the Hong Kong
protests that seeks to connect leftist movements in various countries,
was a host
of\href{https://lausan.hk/2020/notes-from-black-liberation-and-hong-kong/}{a
webinar.}It provided a forum for Hong Kong and American activists to
share strategies.

Katharin Tai, a doctoral candidate in political science at M.I.T. who
studies Chinese foreign policy and the intersection of international
politics and the internet, separated information sharing between Hong
Kong and the United States into two categories.

One was group-to-group sharing of tactics between the sets of
protesters, though she noted that because both protest efforts were
non-hierarchical, they were not necessarily organized from above.

The second, she said, included the translation of helpful graphics and
information --- say, which sort of gas masks best protect against tear
gas --- that are then posted online. ``That's the less organized way,
where they're just kind of pushing it out into the ether,'' she said.

Image

Hong Kong, September 2019.Credit...Getty Images

Image

Portland, Ore., July 2020.Credit...Associated Press

The social internet has sped up a long history of direct and indirect
dialogue between protest movements around the world.

Mark Bray, an organizer of Occupy Wall Street and a lecturer at Rutgers
University, said that sharing or imitating protest strategies and
tactics is ``as old as protest strategies and tactics are,'' but that
social media ``has exposed people to more different tactics.''

``In that sense, like all kinds of new communications technologies, it
has shortened the perceived distance between movements around the
world,'' said Mr. Bray, who is the author of ``Antifa: The Anti-Fascist
Handbook,'' a history of that movement.

\includegraphics{https://static01.nyt.com/images/2020/08/01/reader-center/01_Barricades_Thumb/01_Barricades_Thumb-superJumbo-v2.jpg}

Anastasia Veneti, who teaches at Bournemouth University in England and
specializes in media coverage of protest movements, said that
photographs and video that have been produced and circulated by the
protesters ``have influenced professional photographers who have begun
to produce similar images.''

``With this global wave of post-2010 activism, we've seen that this
paradigm or media framing has started to change and to a great extent,
this change is to be credited to the fact that protesters themselves are
better organized thanks to the use of new media technologies,'' she
said.

Matching protest images are not only found between Hong Kong and the
United States. They crop up in Mexico and Greece, Kurdistan and
Catalonia.

Image

Nantes, France, June 2016.Credit...Stephane Mahe/Reuters

Image

Notre-Dame-des-Landes, France, April 2018.Credit...Getty Images

Image

Hong Kong, August 2019.Credit...Tyrone Siu/Reuters

Image

Hong Kong, November 2019.Credit...Fazry Ismail/EPA, via Shutterstock

Image

Beirut, Lebanon, June 2020.Credit...Bilal Hussein/Associated Press

Image

Santiago, Chile, January 2020.Credit...Getty Images

But Hong Kong does play a central role in the activist imagination,
scholars and activists said, thanks both to the tactical ingenuity of
protesters there, as well as Western media's willingness to cover
pro-democracy demonstrations extensively.

Gabriella Coleman, a professor at McGill University who studies digital
activism, noted that even nonpolitical publications were moved to cover
the Hong Kong protests. ``Because Hong Kong is seen as a Western-style
democracy that's being eaten up by its authoritarian parent, there's no
controversy in reporting on it,'' she said.

Asked whether Hong Kong loomed particularly large in the eyes of
experienced protesters, Mr. López answered emphatically: ``Hell yeah!''
He called the protests in Hong Kong ``epic.''

``More than anything the discipline, organization and persistence of
these folks has been awe inspiring,'' Mr. López said, adding that the
people of Hong Kong ``are showing us what is possible.''

Image

Hong Kong, December 2019.Credit...Danish Siddiqui/Reuters

Image

Portland, Ore., July 2020.Credit...Caitlin Ochs/Reuters

Advertisement

\protect\hyperlink{after-bottom}{Continue reading the main story}

\hypertarget{site-index}{%
\subsection{Site Index}\label{site-index}}

\hypertarget{site-information-navigation}{%
\subsection{Site Information
Navigation}\label{site-information-navigation}}

\begin{itemize}
\tightlist
\item
  \href{https://help.nytimes.com/hc/en-us/articles/115014792127-Copyright-notice}{©~2020~The
  New York Times Company}
\end{itemize}

\begin{itemize}
\tightlist
\item
  \href{https://www.nytco.com/}{NYTCo}
\item
  \href{https://help.nytimes.com/hc/en-us/articles/115015385887-Contact-Us}{Contact
  Us}
\item
  \href{https://www.nytco.com/careers/}{Work with us}
\item
  \href{https://nytmediakit.com/}{Advertise}
\item
  \href{http://www.tbrandstudio.com/}{T Brand Studio}
\item
  \href{https://www.nytimes.com/privacy/cookie-policy\#how-do-i-manage-trackers}{Your
  Ad Choices}
\item
  \href{https://www.nytimes.com/privacy}{Privacy}
\item
  \href{https://help.nytimes.com/hc/en-us/articles/115014893428-Terms-of-service}{Terms
  of Service}
\item
  \href{https://help.nytimes.com/hc/en-us/articles/115014893968-Terms-of-sale}{Terms
  of Sale}
\item
  \href{https://spiderbites.nytimes.com}{Site Map}
\item
  \href{https://help.nytimes.com/hc/en-us}{Help}
\item
  \href{https://www.nytimes.com/subscription?campaignId=37WXW}{Subscriptions}
\end{itemize}
