Sections

SEARCH

\protect\hyperlink{site-content}{Skip to
content}\protect\hyperlink{site-index}{Skip to site index}

\href{https://www.nytimes.com/section/world/africa}{Africa}

\href{https://myaccount.nytimes.com/auth/login?response_type=cookie\&client_id=vi}{}

\href{https://www.nytimes.com/section/todayspaper}{Today's Paper}

\href{/section/world/africa}{Africa}\textbar{}For Senegal's Biggest
Holiday, a Shortage of the All-Important Sheep

\url{https://nyti.ms/30T2nds}

\begin{itemize}
\item
\item
\item
\item
\item
\item
\end{itemize}

\href{https://www.nytimes.com/news-event/coronavirus?action=click\&pgtype=Article\&state=default\&region=TOP_BANNER\&context=storylines_menu}{The
Coronavirus Outbreak}

\begin{itemize}
\tightlist
\item
  live\href{https://www.nytimes.com/2020/08/04/world/coronavirus-cases.html?action=click\&pgtype=Article\&state=default\&region=TOP_BANNER\&context=storylines_menu}{Latest
  Updates}
\item
  \href{https://www.nytimes.com/interactive/2020/us/coronavirus-us-cases.html?action=click\&pgtype=Article\&state=default\&region=TOP_BANNER\&context=storylines_menu}{Maps
  and Cases}
\item
  \href{https://www.nytimes.com/interactive/2020/science/coronavirus-vaccine-tracker.html?action=click\&pgtype=Article\&state=default\&region=TOP_BANNER\&context=storylines_menu}{Vaccine
  Tracker}
\item
  \href{https://www.nytimes.com/2020/08/02/us/covid-college-reopening.html?action=click\&pgtype=Article\&state=default\&region=TOP_BANNER\&context=storylines_menu}{College
  Reopening}
\item
  \href{https://www.nytimes.com/live/2020/08/04/business/stock-market-today-coronavirus?action=click\&pgtype=Article\&state=default\&region=TOP_BANNER\&context=storylines_menu}{Economy}
\end{itemize}

Advertisement

\protect\hyperlink{after-top}{Continue reading the main story}

Supported by

\protect\hyperlink{after-sponsor}{Continue reading the main story}

Senegal Dispatch

\hypertarget{for-senegals-biggest-holiday-a-shortage-of-the-all-important-sheep}{%
\section{For Senegal's Biggest Holiday, a Shortage of the All-Important
Sheep}\label{for-senegals-biggest-holiday-a-shortage-of-the-all-important-sheep}}

Properly celebrating Tabaski, as Eid al-Adha is known in Senegal,
requires a sacrificial sheep. Coronavirus restrictions have made the
animals more expensive, putting them out of reach of many.

\includegraphics{https://static01.nyt.com/images/2020/07/28/world/28Senegal-Sheep-Dispatch/28Senegal-Sheep-Dispatch-articleLarge.jpg?quality=75\&auto=webp\&disable=upscale}

By \href{https://www.nytimes.com/by/ruth-maclean}{Ruth Maclean}

\begin{itemize}
\item
  July 29, 2020
\item
  \begin{itemize}
  \item
  \item
  \item
  \item
  \item
  \item
  \end{itemize}
\end{itemize}

MISSIRAH, Senegal --- The couscous was all ready for the lunchtime crowd
in Yassin Dicko's restaurant near one of the biggest sheep markets in
Senegal.

But except for family members, the place was empty. Ms. Dicko's usual
customers, shepherds from Mauritania, had not shown up. She looked
outside to the vast holding area where just a few meager herds of sheep
trotted around, their bleating oddly human.

While the corrals in Missirah, in central Senegal, are usually packed
with sheep at this time of year, hardly any salesmen had shown up.
Almost no buyers either.

``This is a real crisis,'' she said.

It was 13 days before Tabaski, the Senegalese version of Eid al-Adha,
the biggest religious celebration of the year in the country, which is
about 95 percent Muslim.

No sheep for the Tabaski feast for a Muslim Senegalese family is like no
presents at Christmas for a Christian one, and in the two weeks before
the holiday, which takes place this year on July 31, there is usually a
shopping rush for the animals.

But monthslong government-imposed measures to contain the coronavirus
--- borders closed, markets shuttered and travel severely restricted ---
have been financially devastating for many people in Senegal, putting a
purchase of great cultural and social importance beyond the reach of
many this year.

\includegraphics{https://static01.nyt.com/images/2020/07/28/world/28Senegal-Sheep-Dispatch-02/28Senegal-Sheep-Dispatch-02-articleLarge.jpg?quality=75\&auto=webp\&disable=upscale}

Even in ordinary years, the sheep is a major purchase for many families,
who take out high-interest
\href{https://www.boaniger.com/particuliers/prets/pret-tabaski/}{loans}
to pay for it and the other food and new clothes that are typically an
obligatory part of the celebration.

Owners guard their sheep purchases closely, sometimes even sleeping in
the same room with the animal, to avoid falling victim to a theft. While
most families buy their sheep in the month before Tabaski, some buy them
a year or more in advance and do the fattening up themselves.

This year, aware that citizens would be worried about catching the virus
at in-person markets, the livestock ministry set up a Tinder-like
digital matchmaking site, where sellers could post appealing photos of
their sheep.

\hypertarget{latest-updates-global-coronavirus-outbreak}{%
\section{\texorpdfstring{\href{https://www.nytimes.com/2020/08/04/world/coronavirus-cases.html?action=click\&pgtype=Article\&state=default\&region=MAIN_CONTENT_1\&context=storylines_live_updates}{Latest
Updates: Global Coronavirus
Outbreak}}{Latest Updates: Global Coronavirus Outbreak}}\label{latest-updates-global-coronavirus-outbreak}}

Updated 2020-08-05T07:58:24.076Z

\begin{itemize}
\tightlist
\item
  \href{https://www.nytimes.com/2020/08/04/world/coronavirus-cases.html?action=click\&pgtype=Article\&state=default\&region=MAIN_CONTENT_1\&context=storylines_live_updates\#link-762df92}{As
  talks drag on, McConnell signals openness to jobless aid extension,
  and negotiators agree on a deadline.}
\item
  \href{https://www.nytimes.com/2020/08/04/world/coronavirus-cases.html?action=click\&pgtype=Article\&state=default\&region=MAIN_CONTENT_1\&context=storylines_live_updates\#link-1228a480}{Novavax
  sees encouraging results from two studies of its experimental
  vaccine.}
\item
  \href{https://www.nytimes.com/2020/08/04/world/coronavirus-cases.html?action=click\&pgtype=Article\&state=default\&region=MAIN_CONTENT_1\&context=storylines_live_updates\#link-794484ed}{Mississippians
  must now wear masks in public, governor says.}
\end{itemize}

\href{https://www.nytimes.com/2020/08/04/world/coronavirus-cases.html?action=click\&pgtype=Article\&state=default\&region=MAIN_CONTENT_1\&context=storylines_live_updates}{See
more updates}

More live coverage:
\href{https://www.nytimes.com/live/2020/08/04/business/stock-market-today-coronavirus?action=click\&pgtype=Article\&state=default\&region=MAIN_CONTENT_1\&context=storylines_live_updates}{Markets}

Buyers were able to swipe left or right through hundreds of sheep
profiles on Sama Xaru Tabaski --- or ``My Tabaski Sheep'' in Wolof, the
most widely spoken of Senegal's many languages --- and then arrange a
deal for the animal that catches their fancy, skipping hours of risky
face-to-face haggling.

But the service has received meager traffic so far, and even if a
would-be buyer can find a sheep, it may not be affordable.

A \$140 sheep last year now costs over \$170, with the hike
attributable, according to industry players, to the ripple effects of
the coronavirus restrictions.

Around half of Senegal's Tabaski sheep come from neighboring Mali and
Mauritania, and until late June, all the herds were completely stuck on
the other side of the border, except those smuggled across.

Image

A truck full of sheep crossing over the Mali border into Senegal.~Their
shepherds ride above them in hammocks.Credit...Ricci Shryock for The New
York Times

Now sheep are welcome, but even though the bovid border has been
reopened, the expected surge of sheep has not materialized.

One reason is that sheep now must be transported only in trucks, and
many drivers are charging twice as much as usual to move the animals.

On a recent Saturday, shepherds, swinging their sticks, yipped and
whooped their skinny charges into the back of a truck in Kidira, a town
in Senegal on the border with Mali where the sheep had been let out to
graze. They were traveling to Dakar with the animals, riding in hammocks
strung from truck beams above their undulant flocks.

Watching the shepherds and their charges scurry aboard was Alassane
Ndongo, president of the local herders' association.

While three shepherds can ride in each truck, Mr. Ndongo said that
owners don't because they generally travel alongside by car, and cars
are banned from crossing the border.

So many owners were simply keeping their flocks home, another reason for
the shortages.

In addition, Mr. Ndongo added, in typical years, many herds would cross
the border on foot. But that requires setting off months in advance, and
in 2020, that start date for the journey was just when governments were
ordering everyone indoors.

Image

Fatou Diallo, Oumou-Khairy Diallo and Sadio Diallo at a bus station in
Tambacounda, Senegal. The cousins had just come back from a wedding in
Dakar.Credit...Ricci Shryock for The New York Times

While the restrictions imposed to combat the coronavirus have been
financially devastating for many, Senegal has not been as hard hit as
many countries by Covid-19.
\href{https://www.nytimes.com/interactive/2020/world/coronavirus-maps.html}{Fewer
than}10,000 people have had it, of whom 194 have died, according to
official figures.

In Dakar, Senegal's capital, where the streets every year run red with
sheep's blood on Tabaski morning, Abou Gallo Thiello Kane runs the
swankiest sheep mall in town, a spacious, manicured tent full of
magnificent beasts.

A famous \href{https://www.youtube.com/watch?v=SfeOEF8zvcs}{mbalax
dancer} and comedian as well as sheep merchant to the stars --- local
celebrities and government ministers are said to be among his clients
--- Mr. Kane's 20 years in the business have taught him what prospective
buyers look for.

\href{https://www.nytimes.com/news-event/coronavirus?action=click\&pgtype=Article\&state=default\&region=MAIN_CONTENT_3\&context=storylines_faq}{}

\hypertarget{the-coronavirus-outbreak-}{%
\subsubsection{The Coronavirus Outbreak
›}\label{the-coronavirus-outbreak-}}

\hypertarget{frequently-asked-questions}{%
\paragraph{Frequently Asked
Questions}\label{frequently-asked-questions}}

Updated August 4, 2020

\begin{itemize}
\item ~
  \hypertarget{i-have-antibodies-am-i-now-immune}{%
  \paragraph{I have antibodies. Am I now
  immune?}\label{i-have-antibodies-am-i-now-immune}}

  \begin{itemize}
  \tightlist
  \item
    As of right
    now,\href{https://www.nytimes.com/2020/07/22/health/covid-antibodies-herd-immunity.html?action=click\&pgtype=Article\&state=default\&region=MAIN_CONTENT_3\&context=storylines_faq}{that
    seems likely, for at least several months.} There have been
    frightening accounts of people suffering what seems to be a second
    bout of Covid-19. But experts say these patients may have a
    drawn-out course of infection, with the virus taking a slow toll
    weeks to months after initial exposure. People infected with the
    coronavirus typically
    \href{https://www.nature.com/articles/s41586-020-2456-9}{produce}
    immune molecules called antibodies, which are
    \href{https://www.nytimes.com/2020/05/07/health/coronavirus-antibody-prevalence.html?action=click\&pgtype=Article\&state=default\&region=MAIN_CONTENT_3\&context=storylines_faq}{protective
    proteins made in response to an
    infection}\href{https://www.nytimes.com/2020/05/07/health/coronavirus-antibody-prevalence.html?action=click\&pgtype=Article\&state=default\&region=MAIN_CONTENT_3\&context=storylines_faq}{.
    These antibodies may} last in the body
    \href{https://www.nature.com/articles/s41591-020-0965-6}{only two to
    three months}, which may seem worrisome, but that's perfectly normal
    after an acute infection subsides, said Dr. Michael Mina, an
    immunologist at Harvard University. It may be possible to get the
    coronavirus again, but it's highly unlikely that it would be
    possible in a short window of time from initial infection or make
    people sicker the second time.
  \end{itemize}
\item ~
  \hypertarget{im-a-small-business-owner-can-i-get-relief}{%
  \paragraph{I'm a small-business owner. Can I get
  relief?}\label{im-a-small-business-owner-can-i-get-relief}}

  \begin{itemize}
  \tightlist
  \item
    The
    \href{https://www.nytimes.com/article/small-business-loans-stimulus-grants-freelancers-coronavirus.html?action=click\&pgtype=Article\&state=default\&region=MAIN_CONTENT_3\&context=storylines_faq}{stimulus
    bills enacted in March} offer help for the millions of American
    small businesses. Those eligible for aid are businesses and
    nonprofit organizations with fewer than 500 workers, including sole
    proprietorships, independent contractors and freelancers. Some
    larger companies in some industries are also eligible. The help
    being offered, which is being managed by the Small Business
    Administration, includes the Paycheck Protection Program and the
    Economic Injury Disaster Loan program. But lots of folks have
    \href{https://www.nytimes.com/interactive/2020/05/07/business/small-business-loans-coronavirus.html?action=click\&pgtype=Article\&state=default\&region=MAIN_CONTENT_3\&context=storylines_faq}{not
    yet seen payouts.} Even those who have received help are confused:
    The rules are draconian, and some are stuck sitting on
    \href{https://www.nytimes.com/2020/05/02/business/economy/loans-coronavirus-small-business.html?action=click\&pgtype=Article\&state=default\&region=MAIN_CONTENT_3\&context=storylines_faq}{money
    they don't know how to use.} Many small-business owners are getting
    less than they expected or
    \href{https://www.nytimes.com/2020/06/10/business/Small-business-loans-ppp.html?action=click\&pgtype=Article\&state=default\&region=MAIN_CONTENT_3\&context=storylines_faq}{not
    hearing anything at all.}
  \end{itemize}
\item ~
  \hypertarget{what-are-my-rights-if-i-am-worried-about-going-back-to-work}{%
  \paragraph{What are my rights if I am worried about going back to
  work?}\label{what-are-my-rights-if-i-am-worried-about-going-back-to-work}}

  \begin{itemize}
  \tightlist
  \item
    Employers have to provide
    \href{https://www.osha.gov/SLTC/covid-19/standards.html}{a safe
    workplace} with policies that protect everyone equally.
    \href{https://www.nytimes.com/article/coronavirus-money-unemployment.html?action=click\&pgtype=Article\&state=default\&region=MAIN_CONTENT_3\&context=storylines_faq}{And
    if one of your co-workers tests positive for the coronavirus, the
    C.D.C.} has said that
    \href{https://www.cdc.gov/coronavirus/2019-ncov/community/guidance-business-response.html}{employers
    should tell their employees} -\/- without giving you the sick
    employee's name -\/- that they may have been exposed to the virus.
  \end{itemize}
\item ~
  \hypertarget{should-i-refinance-my-mortgage}{%
  \paragraph{Should I refinance my
  mortgage?}\label{should-i-refinance-my-mortgage}}

  \begin{itemize}
  \tightlist
  \item
    \href{https://www.nytimes.com/article/coronavirus-money-unemployment.html?action=click\&pgtype=Article\&state=default\&region=MAIN_CONTENT_3\&context=storylines_faq}{It
    could be a good idea,} because mortgage rates have
    \href{https://www.nytimes.com/2020/07/16/business/mortgage-rates-below-3-percent.html?action=click\&pgtype=Article\&state=default\&region=MAIN_CONTENT_3\&context=storylines_faq}{never
    been lower.} Refinancing requests have pushed mortgage applications
    to some of the highest levels since 2008, so be prepared to get in
    line. But defaults are also up, so if you're thinking about buying a
    home, be aware that some lenders have tightened their standards.
  \end{itemize}
\item ~
  \hypertarget{what-is-school-going-to-look-like-in-september}{%
  \paragraph{What is school going to look like in
  September?}\label{what-is-school-going-to-look-like-in-september}}

  \begin{itemize}
  \tightlist
  \item
    It is unlikely that many schools will return to a normal schedule
    this fall, requiring the grind of
    \href{https://www.nytimes.com/2020/06/05/us/coronavirus-education-lost-learning.html?action=click\&pgtype=Article\&state=default\&region=MAIN_CONTENT_3\&context=storylines_faq}{online
    learning},
    \href{https://www.nytimes.com/2020/05/29/us/coronavirus-child-care-centers.html?action=click\&pgtype=Article\&state=default\&region=MAIN_CONTENT_3\&context=storylines_faq}{makeshift
    child care} and
    \href{https://www.nytimes.com/2020/06/03/business/economy/coronavirus-working-women.html?action=click\&pgtype=Article\&state=default\&region=MAIN_CONTENT_3\&context=storylines_faq}{stunted
    workdays} to continue. California's two largest public school
    districts --- Los Angeles and San Diego --- said on July 13, that
    \href{https://www.nytimes.com/2020/07/13/us/lausd-san-diego-school-reopening.html?action=click\&pgtype=Article\&state=default\&region=MAIN_CONTENT_3\&context=storylines_faq}{instruction
    will be remote-only in the fall}, citing concerns that surging
    coronavirus infections in their areas pose too dire a risk for
    students and teachers. Together, the two districts enroll some
    825,000 students. They are the largest in the country so far to
    abandon plans for even a partial physical return to classrooms when
    they reopen in August. For other districts, the solution won't be an
    all-or-nothing approach.
    \href{https://bioethics.jhu.edu/research-and-outreach/projects/eschool-initiative/school-policy-tracker/}{Many
    systems}, including the nation's largest, New York City, are
    devising
    \href{https://www.nytimes.com/2020/06/26/us/coronavirus-schools-reopen-fall.html?action=click\&pgtype=Article\&state=default\&region=MAIN_CONTENT_3\&context=storylines_faq}{hybrid
    plans} that involve spending some days in classrooms and other days
    online. There's no national policy on this yet, so check with your
    municipal school system regularly to see what is happening in your
    community.
  \end{itemize}
\end{itemize}

``Women usually want a fat sheep, and a young one, because they're
thinking about how it will taste,'' he said from behind a mask, as
assistants hauled mangers up and down his ovine kingdom. ``Men, on the
other hand, buy for aesthetics, and for love.''

Professional butchers are kept busy in Dakar for those who are too
squeamish to slit the animal's throat themselves or for those who have
grown too close to their woolly companion.

``Sheep for us are like dogs for Europeans,'' said Mr. Kane, who is also
president of the National Federation of Sheep Industry Actors. ``They're
good company. They're useful, and they're friendly.''

Image

Abou Ka gathering his sheep at the market in Missirah.Credit...Ricci
Shryock for The New York Times

The sheep's role in the Tabaski celebration is far more central than
just providing a meal. Eid al-Adha honors the story of Ibrahim, whom God
asked to sacrifice his cherished son, Ismail, but then told him at the
last minute he could swap in a ram.

``God did not tell Ibrahim to kill the sheep, or to eat it,'' Mr. Kane
said. ``He said it should be a sacrifice. So you have to choose the
sheep you love.''

Most of Mr. Kane's sheep are less lovable than imposing: They are
Ladoums, an enormous and majestic crossbreed much prized by the
Senegalese, for whom the mere mention of a Ladoum can elicit a wistful
sigh.

Mr. Kane's best Ladoums sell for up to \$3,500, and as I spoke with
their owner, the regal creatures seemed to know it. One tossed his long
tail; another stamped groomed hooves in freshly sprinkled sand. A third,
his ears poking through striped, curling horns, looked up with a snooty
expression as we walked past.

They needn't be so conceited. This year, even elite buyers are thin on
the ground, Mr. Kane said.

The reduced incomes brought by the pandemic shutdown has pushed some
Senegalese from comfortable to struggling.

Ms. Dicko, the restaurant owner, said she was worrying about her friends
and neighbors, those who used to make \$9 a day and now could only make
\$4. For them, she said, a sheep was out of the question, and tough days
were ahead.

``There will be a lot of hardship,'' she said.

Ousmane Balde contributed reporting.

Advertisement

\protect\hyperlink{after-bottom}{Continue reading the main story}

\hypertarget{site-index}{%
\subsection{Site Index}\label{site-index}}

\hypertarget{site-information-navigation}{%
\subsection{Site Information
Navigation}\label{site-information-navigation}}

\begin{itemize}
\tightlist
\item
  \href{https://help.nytimes.com/hc/en-us/articles/115014792127-Copyright-notice}{©~2020~The
  New York Times Company}
\end{itemize}

\begin{itemize}
\tightlist
\item
  \href{https://www.nytco.com/}{NYTCo}
\item
  \href{https://help.nytimes.com/hc/en-us/articles/115015385887-Contact-Us}{Contact
  Us}
\item
  \href{https://www.nytco.com/careers/}{Work with us}
\item
  \href{https://nytmediakit.com/}{Advertise}
\item
  \href{http://www.tbrandstudio.com/}{T Brand Studio}
\item
  \href{https://www.nytimes.com/privacy/cookie-policy\#how-do-i-manage-trackers}{Your
  Ad Choices}
\item
  \href{https://www.nytimes.com/privacy}{Privacy}
\item
  \href{https://help.nytimes.com/hc/en-us/articles/115014893428-Terms-of-service}{Terms
  of Service}
\item
  \href{https://help.nytimes.com/hc/en-us/articles/115014893968-Terms-of-sale}{Terms
  of Sale}
\item
  \href{https://spiderbites.nytimes.com}{Site Map}
\item
  \href{https://help.nytimes.com/hc/en-us}{Help}
\item
  \href{https://www.nytimes.com/subscription?campaignId=37WXW}{Subscriptions}
\end{itemize}
