Sections

SEARCH

\protect\hyperlink{site-content}{Skip to
content}\protect\hyperlink{site-index}{Skip to site index}

\href{https://www.nytimes.com/section/world/africa}{Africa}

\href{https://myaccount.nytimes.com/auth/login?response_type=cookie\&client_id=vi}{}

\href{https://www.nytimes.com/section/todayspaper}{Today's Paper}

\href{/section/world/africa}{Africa}\textbar{}For Senegal's Biggest
Holiday, a Shortage of the All-Important Sheep

\url{https://nyti.ms/30T2nds}

\begin{itemize}
\item
\item
\item
\item
\item
\item
\end{itemize}

\href{https://www.nytimes.com/news-event/coronavirus?action=click\&pgtype=Article\&state=default\&region=TOP_BANNER\&context=storylines_menu}{The
Coronavirus Outbreak}

\begin{itemize}
\tightlist
\item
  live\href{https://www.nytimes.com/2020/08/01/world/coronavirus-covid-19.html?action=click\&pgtype=Article\&state=default\&region=TOP_BANNER\&context=storylines_menu}{Latest
  Updates}
\item
  \href{https://www.nytimes.com/interactive/2020/us/coronavirus-us-cases.html?action=click\&pgtype=Article\&state=default\&region=TOP_BANNER\&context=storylines_menu}{Maps
  and Cases}
\item
  \href{https://www.nytimes.com/interactive/2020/science/coronavirus-vaccine-tracker.html?action=click\&pgtype=Article\&state=default\&region=TOP_BANNER\&context=storylines_menu}{Vaccine
  Tracker}
\item
  \href{https://www.nytimes.com/interactive/2020/07/29/us/schools-reopening-coronavirus.html?action=click\&pgtype=Article\&state=default\&region=TOP_BANNER\&context=storylines_menu}{What
  School May Look Like}
\item
  \href{https://www.nytimes.com/live/2020/07/31/business/stock-market-today-coronavirus?action=click\&pgtype=Article\&state=default\&region=TOP_BANNER\&context=storylines_menu}{Economy}
\end{itemize}

Advertisement

\protect\hyperlink{after-top}{Continue reading the main story}

Supported by

\protect\hyperlink{after-sponsor}{Continue reading the main story}

Senegal Dispatch

\hypertarget{for-senegals-biggest-holiday-a-shortage-of-the-all-important-sheep}{%
\section{For Senegal's Biggest Holiday, a Shortage of the All-Important
Sheep}\label{for-senegals-biggest-holiday-a-shortage-of-the-all-important-sheep}}

Properly celebrating Tabaski, as Eid al-Adha is known in Senegal,
requires a sacrificial sheep. Coronavirus restrictions have made the
animals more expensive, putting them out of reach of many.

\includegraphics{https://static01.nyt.com/images/2020/07/28/world/28Senegal-Sheep-Dispatch/28Senegal-Sheep-Dispatch-articleLarge.jpg?quality=75\&auto=webp\&disable=upscale}

By \href{https://www.nytimes.com/by/ruth-maclean}{Ruth Maclean}

\begin{itemize}
\item
  July 29, 2020
\item
  \begin{itemize}
  \item
  \item
  \item
  \item
  \item
  \item
  \end{itemize}
\end{itemize}

MISSIRAH, Senegal --- The couscous was all ready for the lunchtime crowd
in Yassin Dicko's restaurant near one of the biggest sheep markets in
Senegal.

But except for family members, the place was empty. Ms. Dicko's usual
customers, shepherds from Mauritania, had not shown up. She looked
outside to the vast holding area where just a few meager herds of sheep
trotted around, their bleating oddly human.

While the corrals in Missirah, in central Senegal, are usually packed
with sheep at this time of year, hardly any salesmen had shown up.
Almost no buyers either.

``This is a real crisis,'' she said.

It was 13 days before Tabaski, the Senegalese version of Eid al-Adha,
the biggest religious celebration of the year in the country, which is
about 95 percent Muslim.

No sheep for the Tabaski feast for a Muslim Senegalese family is like no
presents at Christmas for a Christian one, and in the two weeks before
the holiday, which takes place this year on July 31, there is usually a
shopping rush for the animals.

But monthslong government-imposed measures to contain the coronavirus
--- borders closed, markets shuttered and travel severely restricted ---
have been financially devastating for many people in Senegal, putting a
purchase of great cultural and social importance beyond the reach of
many this year.

\includegraphics{https://static01.nyt.com/images/2020/07/28/world/28Senegal-Sheep-Dispatch-02/28Senegal-Sheep-Dispatch-02-articleLarge.jpg?quality=75\&auto=webp\&disable=upscale}

Even in ordinary years, the sheep is a major purchase for many families,
who take out high-interest
\href{https://www.boaniger.com/particuliers/prets/pret-tabaski/}{loans}
to pay for it and the other food and new clothes that are typically an
obligatory part of the celebration.

Owners guard their sheep purchases closely, sometimes even sleeping in
the same room with the animal, to avoid falling victim to a theft. While
most families buy their sheep in the month before Tabaski, some buy them
a year or more in advance and do the fattening up themselves.

This year, aware that citizens would be worried about catching the virus
at in-person markets, the livestock ministry set up a Tinder-like
digital matchmaking site, where sellers could post appealing photos of
their sheep.

\hypertarget{latest-updates-global-coronavirus-outbreak}{%
\section{\texorpdfstring{\href{https://www.nytimes.com/2020/08/01/world/coronavirus-covid-19.html?action=click\&pgtype=Article\&state=default\&region=MAIN_CONTENT_1\&context=storylines_live_updates}{Latest
Updates: Global Coronavirus
Outbreak}}{Latest Updates: Global Coronavirus Outbreak}}\label{latest-updates-global-coronavirus-outbreak}}

Updated 2020-08-01T18:42:36.154Z

\begin{itemize}
\tightlist
\item
  \href{https://www.nytimes.com/2020/08/01/world/coronavirus-covid-19.html?action=click\&pgtype=Article\&state=default\&region=MAIN_CONTENT_1\&context=storylines_live_updates\#link-3ac56579}{Top
  officials work to break impasse over jobless benefit.}
\item
  \href{https://www.nytimes.com/2020/08/01/world/coronavirus-covid-19.html?action=click\&pgtype=Article\&state=default\&region=MAIN_CONTENT_1\&context=storylines_live_updates\#link-8796723}{The
  virus picks up dangerous speed in the Midwest, and in areas that had
  seen success.}
\item
  \href{https://www.nytimes.com/2020/08/01/world/coronavirus-covid-19.html?action=click\&pgtype=Article\&state=default\&region=MAIN_CONTENT_1\&context=storylines_live_updates\#link-25930521}{Thousands
  in Berlin protest Germany's coronavirus measures.}
\end{itemize}

\href{https://www.nytimes.com/2020/08/01/world/coronavirus-covid-19.html?action=click\&pgtype=Article\&state=default\&region=MAIN_CONTENT_1\&context=storylines_live_updates}{See
more updates}

More live coverage:
\href{https://www.nytimes.com/live/2020/07/31/business/stock-market-today-coronavirus?action=click\&pgtype=Article\&state=default\&region=MAIN_CONTENT_1\&context=storylines_live_updates}{Markets}

Buyers were able to swipe left or right through hundreds of sheep
profiles on Sama Xaru Tabaski --- or ``My Tabaski Sheep'' in Wolof, the
most widely spoken of Senegal's many languages --- and then arrange a
deal for the animal that catches their fancy, skipping hours of risky
face-to-face haggling.

But the service has received meager traffic so far, and even if a
would-be buyer can find a sheep, it may not be affordable.

A \$140 sheep last year now costs over \$170, with the hike
attributable, according to industry players, to the ripple effects of
the coronavirus restrictions.

Around half of Senegal's Tabaski sheep come from neighboring Mali and
Mauritania, and until late June, all the herds were completely stuck on
the other side of the border, except those smuggled across.

Image

A truck full of sheep crossing over the Mali border into Senegal.~Their
shepherds ride above them in hammocks.Credit...Ricci Shryock for The New
York Times

Now sheep are welcome, but even though the bovid border has been
reopened, the expected surge of sheep has not materialized.

One reason is that sheep now must be transported only in trucks, and
many drivers are charging twice as much as usual to move the animals.

On a recent Saturday, shepherds, swinging their sticks, yipped and
whooped their skinny charges into the back of a truck in Kidira, a town
in Senegal on the border with Mali where the sheep had been let out to
graze. They were traveling to Dakar with the animals, riding in hammocks
strung from truck beams above their undulant flocks.

Watching the shepherds and their charges scurry aboard was Alassane
Ndongo, president of the local herders' association.

While three shepherds can ride in each truck, Mr. Ndongo said that
owners don't because they generally travel alongside by car, and cars
are banned from crossing the border.

So many owners were simply keeping their flocks home, another reason for
the shortages.

In addition, Mr. Ndongo added, in typical years, many herds would cross
the border on foot. But that requires setting off months in advance, and
in 2020, that start date for the journey was just when governments were
ordering everyone indoors.

Image

Fatou Diallo, Oumou-Khairy Diallo and Sadio Diallo at a bus station in
Tambacounda, Senegal. The cousins had just come back from a wedding in
Dakar.Credit...Ricci Shryock for The New York Times

While the restrictions imposed to combat the coronavirus have been
financially devastating for many, Senegal has not been as hard hit as
many countries by Covid-19.
\href{https://www.nytimes.com/interactive/2020/world/coronavirus-maps.html}{Fewer
than}10,000 people have had it, of whom 194 have died, according to
official figures.

In Dakar, Senegal's capital, where the streets every year run red with
sheep's blood on Tabaski morning, Abou Gallo Thiello Kane runs the
swankiest sheep mall in town, a spacious, manicured tent full of
magnificent beasts.

A famous \href{https://www.youtube.com/watch?v=SfeOEF8zvcs}{mbalax
dancer} and comedian as well as sheep merchant to the stars --- local
celebrities and government ministers are said to be among his clients
--- Mr. Kane's 20 years in the business have taught him what prospective
buyers look for.

\href{https://www.nytimes.com/news-event/coronavirus?action=click\&pgtype=Article\&state=default\&region=MAIN_CONTENT_3\&context=storylines_faq}{}

\hypertarget{the-coronavirus-outbreak-}{%
\subsubsection{The Coronavirus Outbreak
›}\label{the-coronavirus-outbreak-}}

\hypertarget{frequently-asked-questions}{%
\paragraph{Frequently Asked
Questions}\label{frequently-asked-questions}}

Updated July 27, 2020

\begin{itemize}
\item ~
  \hypertarget{should-i-refinance-my-mortgage}{%
  \paragraph{Should I refinance my
  mortgage?}\label{should-i-refinance-my-mortgage}}

  \begin{itemize}
  \tightlist
  \item
    \href{https://www.nytimes.com/article/coronavirus-money-unemployment.html?action=click\&pgtype=Article\&state=default\&region=MAIN_CONTENT_3\&context=storylines_faq}{It
    could be a good idea,} because mortgage rates have
    \href{https://www.nytimes.com/2020/07/16/business/mortgage-rates-below-3-percent.html?action=click\&pgtype=Article\&state=default\&region=MAIN_CONTENT_3\&context=storylines_faq}{never
    been lower.} Refinancing requests have pushed mortgage applications
    to some of the highest levels since 2008, so be prepared to get in
    line. But defaults are also up, so if you're thinking about buying a
    home, be aware that some lenders have tightened their standards.
  \end{itemize}
\item ~
  \hypertarget{what-is-school-going-to-look-like-in-september}{%
  \paragraph{What is school going to look like in
  September?}\label{what-is-school-going-to-look-like-in-september}}

  \begin{itemize}
  \tightlist
  \item
    It is unlikely that many schools will return to a normal schedule
    this fall, requiring the grind of
    \href{https://www.nytimes.com/2020/06/05/us/coronavirus-education-lost-learning.html?action=click\&pgtype=Article\&state=default\&region=MAIN_CONTENT_3\&context=storylines_faq}{online
    learning},
    \href{https://www.nytimes.com/2020/05/29/us/coronavirus-child-care-centers.html?action=click\&pgtype=Article\&state=default\&region=MAIN_CONTENT_3\&context=storylines_faq}{makeshift
    child care} and
    \href{https://www.nytimes.com/2020/06/03/business/economy/coronavirus-working-women.html?action=click\&pgtype=Article\&state=default\&region=MAIN_CONTENT_3\&context=storylines_faq}{stunted
    workdays} to continue. California's two largest public school
    districts --- Los Angeles and San Diego --- said on July 13, that
    \href{https://www.nytimes.com/2020/07/13/us/lausd-san-diego-school-reopening.html?action=click\&pgtype=Article\&state=default\&region=MAIN_CONTENT_3\&context=storylines_faq}{instruction
    will be remote-only in the fall}, citing concerns that surging
    coronavirus infections in their areas pose too dire a risk for
    students and teachers. Together, the two districts enroll some
    825,000 students. They are the largest in the country so far to
    abandon plans for even a partial physical return to classrooms when
    they reopen in August. For other districts, the solution won't be an
    all-or-nothing approach.
    \href{https://bioethics.jhu.edu/research-and-outreach/projects/eschool-initiative/school-policy-tracker/}{Many
    systems}, including the nation's largest, New York City, are
    devising
    \href{https://www.nytimes.com/2020/06/26/us/coronavirus-schools-reopen-fall.html?action=click\&pgtype=Article\&state=default\&region=MAIN_CONTENT_3\&context=storylines_faq}{hybrid
    plans} that involve spending some days in classrooms and other days
    online. There's no national policy on this yet, so check with your
    municipal school system regularly to see what is happening in your
    community.
  \end{itemize}
\item ~
  \hypertarget{is-the-coronavirus-airborne}{%
  \paragraph{Is the coronavirus
  airborne?}\label{is-the-coronavirus-airborne}}

  \begin{itemize}
  \tightlist
  \item
    The coronavirus
    \href{https://www.nytimes.com/2020/07/04/health/239-experts-with-one-big-claim-the-coronavirus-is-airborne.html?action=click\&pgtype=Article\&state=default\&region=MAIN_CONTENT_3\&context=storylines_faq}{can
    stay aloft for hours in tiny droplets in stagnant air}, infecting
    people as they inhale, mounting scientific evidence suggests. This
    risk is highest in crowded indoor spaces with poor ventilation, and
    may help explain super-spreading events reported in meatpacking
    plants, churches and restaurants.
    \href{https://www.nytimes.com/2020/07/06/health/coronavirus-airborne-aerosols.html?action=click\&pgtype=Article\&state=default\&region=MAIN_CONTENT_3\&context=storylines_faq}{It's
    unclear how often the virus is spread} via these tiny droplets, or
    aerosols, compared with larger droplets that are expelled when a
    sick person coughs or sneezes, or transmitted through contact with
    contaminated surfaces, said Linsey Marr, an aerosol expert at
    Virginia Tech. Aerosols are released even when a person without
    symptoms exhales, talks or sings, according to Dr. Marr and more
    than 200 other experts, who
    \href{https://academic.oup.com/cid/article/doi/10.1093/cid/ciaa939/5867798}{have
    outlined the evidence in an open letter to the World Health
    Organization}.
  \end{itemize}
\item ~
  \hypertarget{what-are-the-symptoms-of-coronavirus}{%
  \paragraph{What are the symptoms of
  coronavirus?}\label{what-are-the-symptoms-of-coronavirus}}

  \begin{itemize}
  \tightlist
  \item
    Common symptoms
    \href{https://www.nytimes.com/article/symptoms-coronavirus.html?action=click\&pgtype=Article\&state=default\&region=MAIN_CONTENT_3\&context=storylines_faq}{include
    fever, a dry cough, fatigue and difficulty breathing or shortness of
    breath.} Some of these symptoms overlap with those of the flu,
    making detection difficult, but runny noses and stuffy sinuses are
    less common.
    \href{https://www.nytimes.com/2020/04/27/health/coronavirus-symptoms-cdc.html?action=click\&pgtype=Article\&state=default\&region=MAIN_CONTENT_3\&context=storylines_faq}{The
    C.D.C. has also} added chills, muscle pain, sore throat, headache
    and a new loss of the sense of taste or smell as symptoms to look
    out for. Most people fall ill five to seven days after exposure, but
    symptoms may appear in as few as two days or as many as 14 days.
  \end{itemize}
\item ~
  \hypertarget{does-asymptomatic-transmission-of-covid-19-happen}{%
  \paragraph{Does asymptomatic transmission of Covid-19
  happen?}\label{does-asymptomatic-transmission-of-covid-19-happen}}

  \begin{itemize}
  \tightlist
  \item
    So far, the evidence seems to show it does. A widely cited
    \href{https://www.nature.com/articles/s41591-020-0869-5}{paper}
    published in April suggests that people are most infectious about
    two days before the onset of coronavirus symptoms and estimated that
    44 percent of new infections were a result of transmission from
    people who were not yet showing symptoms. Recently, a top expert at
    the World Health Organization stated that transmission of the
    coronavirus by people who did not have symptoms was ``very rare,''
    \href{https://www.nytimes.com/2020/06/09/world/coronavirus-updates.html?action=click\&pgtype=Article\&state=default\&region=MAIN_CONTENT_3\&context=storylines_faq\#link-1f302e21}{but
    she later walked back that statement.}
  \end{itemize}
\end{itemize}

``Women usually want a fat sheep, and a young one, because they're
thinking about how it will taste,'' he said from behind a mask, as
assistants hauled mangers up and down his ovine kingdom. ``Men, on the
other hand, buy for aesthetics, and for love.''

Professional butchers are kept busy in Dakar for those who are too
squeamish to slit the animal's throat themselves or for those who have
grown too close to their woolly companion.

``Sheep for us are like dogs for Europeans,'' said Mr. Kane, who is also
president of the National Federation of Sheep Industry Actors. ``They're
good company. They're useful, and they're friendly.''

Image

Abou Ka gathering his sheep at the market in Missirah.Credit...Ricci
Shryock for The New York Times

The sheep's role in the Tabaski celebration is far more central than
just providing a meal. Eid al-Adha honors the story of Ibrahim, whom God
asked to sacrifice his cherished son, Ismail, but then told him at the
last minute he could swap in a ram.

``God did not tell Ibrahim to kill the sheep, or to eat it,'' Mr. Kane
said. ``He said it should be a sacrifice. So you have to choose the
sheep you love.''

Most of Mr. Kane's sheep are less lovable than imposing: They are
Ladoums, an enormous and majestic crossbreed much prized by the
Senegalese, for whom the mere mention of a Ladoum can elicit a wistful
sigh.

Mr. Kane's best Ladoums sell for up to \$3,500, and as I spoke with
their owner, the regal creatures seemed to know it. One tossed his long
tail; another stamped groomed hooves in freshly sprinkled sand. A third,
his ears poking through striped, curling horns, looked up with a snooty
expression as we walked past.

They needn't be so conceited. This year, even elite buyers are thin on
the ground, Mr. Kane said.

The reduced incomes brought by the pandemic shutdown has pushed some
Senegalese from comfortable to struggling.

Ms. Dicko, the restaurant owner, said she was worrying about her friends
and neighbors, those who used to make \$9 a day and now could only make
\$4. For them, she said, a sheep was out of the question, and tough days
were ahead.

``There will be a lot of hardship,'' she said.

Ousmane Balde contributed reporting.

Advertisement

\protect\hyperlink{after-bottom}{Continue reading the main story}

\hypertarget{site-index}{%
\subsection{Site Index}\label{site-index}}

\hypertarget{site-information-navigation}{%
\subsection{Site Information
Navigation}\label{site-information-navigation}}

\begin{itemize}
\tightlist
\item
  \href{https://help.nytimes.com/hc/en-us/articles/115014792127-Copyright-notice}{©~2020~The
  New York Times Company}
\end{itemize}

\begin{itemize}
\tightlist
\item
  \href{https://www.nytco.com/}{NYTCo}
\item
  \href{https://help.nytimes.com/hc/en-us/articles/115015385887-Contact-Us}{Contact
  Us}
\item
  \href{https://www.nytco.com/careers/}{Work with us}
\item
  \href{https://nytmediakit.com/}{Advertise}
\item
  \href{http://www.tbrandstudio.com/}{T Brand Studio}
\item
  \href{https://www.nytimes.com/privacy/cookie-policy\#how-do-i-manage-trackers}{Your
  Ad Choices}
\item
  \href{https://www.nytimes.com/privacy}{Privacy}
\item
  \href{https://help.nytimes.com/hc/en-us/articles/115014893428-Terms-of-service}{Terms
  of Service}
\item
  \href{https://help.nytimes.com/hc/en-us/articles/115014893968-Terms-of-sale}{Terms
  of Sale}
\item
  \href{https://spiderbites.nytimes.com}{Site Map}
\item
  \href{https://help.nytimes.com/hc/en-us}{Help}
\item
  \href{https://www.nytimes.com/subscription?campaignId=37WXW}{Subscriptions}
\end{itemize}
