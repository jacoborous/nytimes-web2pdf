Sections

SEARCH

\protect\hyperlink{site-content}{Skip to
content}\protect\hyperlink{site-index}{Skip to site index}

\href{https://www.nytimes.com/section/style}{Style}

\href{https://myaccount.nytimes.com/auth/login?response_type=cookie\&client_id=vi}{}

\href{https://www.nytimes.com/section/todayspaper}{Today's Paper}

\href{/section/style}{Style}\textbar{}Titans of Tech Testify in Their
Trust-Me Suits

\url{https://nyti.ms/307J3Km}

\begin{itemize}
\item
\item
\item
\item
\item
\end{itemize}

Advertisement

\protect\hyperlink{after-top}{Continue reading the main story}

Supported by

\protect\hyperlink{after-sponsor}{Continue reading the main story}

Critic's Notebook

\hypertarget{titans-of-tech-testify-in-their-trust-me-suits}{%
\section{Titans of Tech Testify in Their Trust-Me
Suits}\label{titans-of-tech-testify-in-their-trust-me-suits}}

It's a tongue twister, but also a strategy. Jeff Bezos, Mark Zuckerberg,
Tim Cook and Sundar Pichai appeared before Congress and dressed the
part.

\includegraphics{https://static01.nyt.com/images/2020/07/30/fashion/30ZOOMSUITS-COMBO/29ZOOMSUITS-COMBO-articleLarge.jpg?quality=75\&auto=webp\&disable=upscale}

\href{https://www.nytimes.com/by/vanessa-friedman}{\includegraphics{https://static01.nyt.com/images/2018/06/12/multimedia/vanessa-friedman/vanessa-friedman-thumbLarge.png}}

By \href{https://www.nytimes.com/by/vanessa-friedman}{Vanessa Friedman}

\begin{itemize}
\item
  July 29, 2020
\item
  \begin{itemize}
  \item
  \item
  \item
  \item
  \item
  \end{itemize}
\end{itemize}

They didn't look like titans. They didn't look like masters of the
universe. They didn't look like ``emperors of the online economy,'' as
Representative David Cicilline, chairman of the antitrust subcommittee
of the House Judiciary Committee and Democrat from Rhode Island, called
them. Nor did they look like ``cyber barons,'' as Representative Jamie
Raskin, a Democrat from Maryland, said.

``They'' --- the four chief executives of Big Tech: Jeff Bezos of
Amazon; Mark Zuckerberg of Facebook; Tim Cook of Apple; and Sundar
Pichai of Alphabet, the parent company of Google, all witnesses in a
congressional hearing on their market dominance --- didn't even look all
that big.

In fact, beamed in from their offices on the West Coast because of
concerns about the coronavirus, facing down the mask-clad members of
Congress who were socially distanced from one another on the
wood-paneled stage of the hearing room in the Rayburn House Office
Building, the four men looked more like four guys dressed up in their
first graduation suits --- serious, sincere, a little uncomfortable ---
than the four horsemen of the digital apocalypse whose planetary power
was a threat to one and all.

Which was the point, of course. Mr. Bezos and co. were wearing costumes
donned with purpose. They knew they were going to be looked at and
judged, not just by the men and women in the room, but also by the
viewing public. The hearing had been billed as potentially as
significant as the Big Tobacco hearings that changed the cigarette
industry. There was a lot at stake.

And though the whole idea of having to testify via video rather than in
person seemed at first to be a disadvantage, it may have worked in their
favor, serving to shrink their presence to human size and allowing them
all sorts of opportunities for scene setting and character building.

\includegraphics{https://static01.nyt.com/images/2020/08/20/fashion/29ZOOMSUITS-zuckerberg/29ZOOMSUITS-zuckerberg-articleLarge-v2.jpg?quality=75\&auto=webp\&disable=upscale}

So Mr. Zuckerberg, framed against a white background that resembled barn
siding, didn't wear his
\href{https://runway.blogs.nytimes.com/2014/11/12/mark-zuckerberg-adopts-obamas-approach-to-dressing/}{signature
Brunello Cucinelli gray T-shirt} and black hoodie, but rather a blue
suit and a blue-and-white checked tie that was slightly askew, as if he
had stuck one finger inside so he could take a deep breath.

Image

Tim Cook of Apple.

Mr. Cook, most often a model of
\href{https://www.nytimes.com/2014/10/16/fashion/for-tim-cook-of-apple-the-fashion-of-no-fashion.html}{normcore
fashion choices}, opted for a dark gray suit and a light gray tie ---
the same gray as his glasses frames --- the knot listing just to one
side. A long Zen planter's worth of verdant greenery was arrayed behind
him; beside him was a sustaining mug of tea.

Image

Sundar Pichai of Google.

Mr. Pichai, whose style
\href{https://www.buzzfeednews.com/article/mathonan/searching-for-google-ceo-sundar-pichai-the-most-powerful-tec\#.pp4M4prlM}{BuzzFeed}
once described as ``Banana Republic dad,'' also appeared in a subtly
patterned gray tie, though his echoed the patterned artwork on the wall
behind him and perfectly matched his gray suit.

Which matched his hair and beard, which matched the gray pottery on the
cabinet behind him, out of which bloomed a healthily lush green plant,
one part of an artistic and minimal still life. He sat mostly with his
hands clasped on the desk in front of him, radiating a sort of
beneficent calm.

Image

Jeff Bezos of Amazon.

And Mr. Bezos, in his first appearance before Congress, eschewed his
newly favored
\href{https://www.nytimes.com/2018/08/02/style/jeff-bezos-style-icon.html}{leather
jackets}, open-necked shirts and buff-body tailoring for a simple dark
suit and tie, which stood out against his homey light wood shelving,
scattered with vases, a few select books and other decorative objects,
while he sustained himself with occasional snacks kept just offscreen.

Snacks! So what if he's the richest man in the world. He's really just
like you and me.

Not for Mr. Bezos or his compatriots the overblown tie knots of the
European business moguls or the brassiness of the double-breasted; not
for them the look-at-me power red of the president. Not for them the
perfectly made-to-measure style, where not a single wrinkle mars the
glossy expanse of fabric, which practically sings, ``money is no
object.''

Not for them, even, the flourishes of a pocket hankie or a lapel pin.
The reference was, if anything, the men in the gray flannel suits of
yore.

That they wore such suits at all was a nod to the mores of Washington,
because in their natural environment, denizens of the digital world
often see the garments as sartorial shackles that reflect old ways of
thinking. But these witnesses' outfits, in their straightforward
styling, both separated the chief executives from their traditional
camouflage, thus making them seem less subversively Other, and conveyed
respect for the office before which the men were appearing.

All of the committee members wore conservative suits or jackets too,
save Greg Steube, the Republican congressman of Florida, who modeled
many shades of grape, and Jim Jordan, the Republican congressman from
Ohio, who has made appearing in shirt sleeves a trademark. The better,
\href{https://soundcloud.com/user-671880160/jim-jordan-talks-oversight-in-the-minority-why-he-doesnt-wear-a-jacket}{he
has said}, to feel ready to rumble.

Indeed, judging by his undone cuffs, already turned back and poised to
be rolled up, as well as his combative language (and tendency to keep
forgetting to put his mask on), he was really ready this time. Though
his opponents, sitting quietly in their on-screen version of ``Hollywood
Squares,'' the Capitol version, seemed to be playing a different part.

After all, if you are trying to convince a group of lawmakers that the
words they keep using to describe you --- ``dominant,'' ``power,''
``billions,'' ``trillions'' --- are not nearly the whole story, you
don't want to just tell the story of how you are the embodiment of the
American success story; of your humble beginnings and crazy dreams. You
don't want to just give voice to your concern for customers and users
and small businesses. You want to channel Clark Kent of Smallville,
rather than Superman.

Well, they wouldn't look very good in tights, anyway.

Advertisement

\protect\hyperlink{after-bottom}{Continue reading the main story}

\hypertarget{site-index}{%
\subsection{Site Index}\label{site-index}}

\hypertarget{site-information-navigation}{%
\subsection{Site Information
Navigation}\label{site-information-navigation}}

\begin{itemize}
\tightlist
\item
  \href{https://help.nytimes.com/hc/en-us/articles/115014792127-Copyright-notice}{©~2020~The
  New York Times Company}
\end{itemize}

\begin{itemize}
\tightlist
\item
  \href{https://www.nytco.com/}{NYTCo}
\item
  \href{https://help.nytimes.com/hc/en-us/articles/115015385887-Contact-Us}{Contact
  Us}
\item
  \href{https://www.nytco.com/careers/}{Work with us}
\item
  \href{https://nytmediakit.com/}{Advertise}
\item
  \href{http://www.tbrandstudio.com/}{T Brand Studio}
\item
  \href{https://www.nytimes.com/privacy/cookie-policy\#how-do-i-manage-trackers}{Your
  Ad Choices}
\item
  \href{https://www.nytimes.com/privacy}{Privacy}
\item
  \href{https://help.nytimes.com/hc/en-us/articles/115014893428-Terms-of-service}{Terms
  of Service}
\item
  \href{https://help.nytimes.com/hc/en-us/articles/115014893968-Terms-of-sale}{Terms
  of Sale}
\item
  \href{https://spiderbites.nytimes.com}{Site Map}
\item
  \href{https://help.nytimes.com/hc/en-us}{Help}
\item
  \href{https://www.nytimes.com/subscription?campaignId=37WXW}{Subscriptions}
\end{itemize}
