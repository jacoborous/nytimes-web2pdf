Sections

SEARCH

\protect\hyperlink{site-content}{Skip to
content}\protect\hyperlink{site-index}{Skip to site index}

\href{https://www.nytimes.com/section/us}{U.S.}

\href{https://myaccount.nytimes.com/auth/login?response_type=cookie\&client_id=vi}{}

\href{https://www.nytimes.com/section/todayspaper}{Today's Paper}

\href{/section/us}{U.S.}\textbar{}Robert E. Lee High in Virginia Will Be
Renamed for John Lewis, District Says

\url{https://nyti.ms/2CBdbF7}

\begin{itemize}
\item
\item
\item
\item
\item
\end{itemize}

Advertisement

\protect\hyperlink{after-top}{Continue reading the main story}

Supported by

\protect\hyperlink{after-sponsor}{Continue reading the main story}

\hypertarget{robert-e-lee-high-in-virginia-will-be-renamed-for-john-lewis-district-says}{%
\section{Robert E. Lee High in Virginia Will Be Renamed for John Lewis,
District
Says}\label{robert-e-lee-high-in-virginia-will-be-renamed-for-john-lewis-district-says}}

Mr. Lewis, the civil rights giant who died last week, beat out a list
that included Barack Obama and Cesar Chavez to have the high school, in
Fairfax County, named after him.

\includegraphics{https://static01.nyt.com/images/2020/07/24/multimedia/24xp-lewis-school-pix1/24xp-lewis-school-pix1-articleLarge.jpg?quality=75\&auto=webp\&disable=upscale}

\href{https://www.nytimes.com/by/sandra-e-garcia}{\includegraphics{https://static01.nyt.com/images/2020/07/10/reader-center/author-sandra-e-garcia/author-sandra-e-garcia-thumbLarge.png}}

By \href{https://www.nytimes.com/by/sandra-e-garcia}{Sandra E. Garcia}

\begin{itemize}
\item
  Published July 24, 2020Updated July 26, 2020
\item
  \begin{itemize}
  \item
  \item
  \item
  \item
  \item
  \end{itemize}
\end{itemize}

A Virginia school district announced on Thursday that it would rename
Robert E. Lee High School in Springfield for
\href{https://www.nytimes.com/2020/07/25/us/photos-john-lewis-memorial.html}{John
Lewis, the Georgia congressman} and civil rights giant who died last
week.

The name change, which is expected to go into effect in September, has
been in the works for over a year and a half, according to Tamara
Derenak Kaufax, a school board member. Mr. Lewis, who was called ``the
conscience of the Congress'' by his colleagues, had been on a short list
of names since March that also included former President Barack Obama
and Cesar Chavez, the farmworker organizer, she added.

``We thought, `Does the Confederacy represent who we are?''' Ms. Derenak
Kaufax said.

The school district, which is also named after Lee, a Confederate
general, did not have policies and regulations in place for a name
change, Ms. Derenak Kaufax said. The district also wanted to have a
robust conversation about the possible new school name with the
community and students, she said.

\href{https://www.fcps.edu/school-board/kimberly-boateng}{Kimberly
Boateng}, 17, last year's student body representative on the Fairfax
County School Board, said she had rallied her class and lobbied her
peers for the name change.

``Initially, the name didn't really affect me,'' said Ms. Boateng, who
will be a senior this fall at the future John R. Lewis High School.
``Most people don't call it the Robert E. Lee School, we just called it
Lee because it was embarrassing. We always had to explain that we aren't
the name.''

The board's new student representative,
\href{https://www.fcps.edu/staff/nathan-onibudo}{Nathan Onibudo}, 17,
said that he was happy the long process of renaming the school was
finally over and that he could be proud of the man his school is named
after.

``The name on the school building is something each student had to walk
under, and you want that name to be someone they can aspire to be,'' Mr.
Onibudo said. ``That person should be someone that a student can
embody.''

Mr. Lewis's lifetime of activism began when he was a student,
demonstrating against voter disenfranchisement and Jim Crow laws. At 21,
Mr. Lewis was one of the original
\href{https://www.nytimes.com/2020/07/18/us/politics/freedom-riders-john-lewis-work.html}{13
Freedom Riders} who traveled across the South to protest segregation.
When he was 23, Mr. Lewis was the sixth person to speak at the 1963
\href{https://archive.nytimes.com/www.nytimes.com/interactive/2013/08/24/us/march-on-washington-original-coverage.html}{March
on Washington}, where the Rev. Dr. Martin Luther King Jr. gave his
landmark ``I Have A Dream'' speech. Mr. Lewis, a Democrat, was elected
to Congress in 1986 and represented Georgia's Fifth District until his
death on July 17.

Ms. Derenak Kaufax said she felt that Mr. Lewis's legacy aligned much
more closely with the school's values than Lee's did. She called Mr.
Lewis an icon for starting ``good trouble'' and for making people
understand his ideas in such a way that they could eventually come
together. That was a legacy that the school needed and wanted, she said.

``I wanted a name where everybody who walked through those doors felt
safe and supported,'' Ms. Derenak Kaufax added, pointing to the
diversity of the school, which is about 85 percent nonwhite. ``I really
believe that a school has to make the students and staff feel
comfortable and the community proud, and none of those things existed
with the name Robert E. Lee.''

Ms. Boateng said she had been eager to graduate next June but for now is
more looking forward to simply walking into her school again.

She has imagined returning to school and seeing Mr. Lewis's name on the
front of the building. ``The feeling of knowing I go to John R. Lewis
High is amazing,'' she said.

Advertisement

\protect\hyperlink{after-bottom}{Continue reading the main story}

\hypertarget{site-index}{%
\subsection{Site Index}\label{site-index}}

\hypertarget{site-information-navigation}{%
\subsection{Site Information
Navigation}\label{site-information-navigation}}

\begin{itemize}
\tightlist
\item
  \href{https://help.nytimes.com/hc/en-us/articles/115014792127-Copyright-notice}{©~2020~The
  New York Times Company}
\end{itemize}

\begin{itemize}
\tightlist
\item
  \href{https://www.nytco.com/}{NYTCo}
\item
  \href{https://help.nytimes.com/hc/en-us/articles/115015385887-Contact-Us}{Contact
  Us}
\item
  \href{https://www.nytco.com/careers/}{Work with us}
\item
  \href{https://nytmediakit.com/}{Advertise}
\item
  \href{http://www.tbrandstudio.com/}{T Brand Studio}
\item
  \href{https://www.nytimes.com/privacy/cookie-policy\#how-do-i-manage-trackers}{Your
  Ad Choices}
\item
  \href{https://www.nytimes.com/privacy}{Privacy}
\item
  \href{https://help.nytimes.com/hc/en-us/articles/115014893428-Terms-of-service}{Terms
  of Service}
\item
  \href{https://help.nytimes.com/hc/en-us/articles/115014893968-Terms-of-sale}{Terms
  of Sale}
\item
  \href{https://spiderbites.nytimes.com}{Site Map}
\item
  \href{https://help.nytimes.com/hc/en-us}{Help}
\item
  \href{https://www.nytimes.com/subscription?campaignId=37WXW}{Subscriptions}
\end{itemize}
