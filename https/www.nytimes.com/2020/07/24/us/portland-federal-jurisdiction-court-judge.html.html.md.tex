Sections

SEARCH

\protect\hyperlink{site-content}{Skip to
content}\protect\hyperlink{site-index}{Skip to site index}

\href{https://www.nytimes.com/section/us}{U.S.}

\href{https://myaccount.nytimes.com/auth/login?response_type=cookie\&client_id=vi}{}

\href{https://www.nytimes.com/section/todayspaper}{Today's Paper}

\href{/section/us}{U.S.}\textbar{}Judge Rejects Challenge to Federal
Agents Targeting Portland Protesters

\url{https://nyti.ms/2WVCbh5}

\begin{itemize}
\item
\item
\item
\item
\item
\end{itemize}

Advertisement

\protect\hyperlink{after-top}{Continue reading the main story}

Supported by

\protect\hyperlink{after-sponsor}{Continue reading the main story}

\hypertarget{judge-rejects-challenge-to-federal-agents-targeting-portland-protesters}{%
\section{Judge Rejects Challenge to Federal Agents Targeting Portland
Protesters}\label{judge-rejects-challenge-to-federal-agents-targeting-portland-protesters}}

Oregon's attorney general had sued to prevent federal agents from
patrolling the city in unmarked vehicles and detaining protesters.

\includegraphics{https://static01.nyt.com/images/2020/07/17/autossell/portland-v1-2/portland-v1-2-videoSixteenByNineJumbo1600.jpg}

\href{https://www.nytimes.com/by/thomas-fuller}{\includegraphics{https://static01.nyt.com/images/2018/06/12/multimedia/author-thomas-fuller/author-thomas-fuller-thumbLarge.png}}

By \href{https://www.nytimes.com/by/thomas-fuller}{Thomas Fuller}

\begin{itemize}
\item
  July 24, 2020
\item
  \begin{itemize}
  \item
  \item
  \item
  \item
  \item
  \end{itemize}
\end{itemize}

PORTLAND, Ore. --- A federal judge in Portland on Friday rejected
Oregon's legal bid to restrict the operations of federal agents who have
been battling protesters nightly in the city.

The lawsuit by Oregon's attorney general, Ellen Rosenblum, argued that
the operations of federal authorities, who were using unmarked vehicles
to drive around downtown
\href{https://www.nytimes.com/article/portland-protests-explained-protesters.html}{Portland}
and detain protesters, resembled abductions. It called on the court to
order the agents to stop arresting individuals without probable cause
and to clearly identify themselves and their agency before detaining or
arresting ``any person off the streets in Oregon.''

In his ruling, Judge Michael W. Mosman of the U.S. District Court in
Portland said the attorney general's office did not have standing to
bring the case because it had not shown that the issue was ``an interest
that is specific to the state itself.''

``I find the State of Oregon lacks standing here and therefore deny its
request for a temporary restraining order,'' the judge wrote in his
ruling.

``I am quite disappointed,'' Ms. Rosenblum said in an interview. ``If I
don't have standing, I'm not quite sure who does.''

A number of other lawsuits have been filed by private parties against
the presence of the federal agents, and Ms. Rosenblum said she hoped
they would be more successful.

``Every American needs to be concerned about what's happening in
Portland,'' Ms. Rosenblum said. ``It could be happening in your city
next.''

``There's no reason for this kind of secret police tactics,'' she said.

The state's case cited the detention of Mark Pettibone, who said he was
confronted on July 15 by armed men dressed in camouflage who took him
off the street and pushed him into a van. He was driven to a building,
placed into a cell and read his Miranda rights. But the state's
complaint said he was not told why he was arrested, nor provided with a
lawyer. ``He alleges that he was released without any paperwork,
citation, or record of his arrest,'' the complaint said.

It said that people picked up by unidentified federal agents could fear
that they were being abducted.

``Ordinarily, a person exercising his right to walk through the streets
of Portland who is confronted by anonymous men in military-type fatigues
and ordered into an unmarked van can reasonably assume that he is being
kidnapped and is the victim of a crime,'' the complaint said.

The lawsuit said federal agents were violating the First, Fourth and
Fifth Amendments to the Constitution by denying the right to peacefully
protest, failing to provide due process and conducting unreasonable
searches and seizures.

Advertisement

\protect\hyperlink{after-bottom}{Continue reading the main story}

\hypertarget{site-index}{%
\subsection{Site Index}\label{site-index}}

\hypertarget{site-information-navigation}{%
\subsection{Site Information
Navigation}\label{site-information-navigation}}

\begin{itemize}
\tightlist
\item
  \href{https://help.nytimes.com/hc/en-us/articles/115014792127-Copyright-notice}{©~2020~The
  New York Times Company}
\end{itemize}

\begin{itemize}
\tightlist
\item
  \href{https://www.nytco.com/}{NYTCo}
\item
  \href{https://help.nytimes.com/hc/en-us/articles/115015385887-Contact-Us}{Contact
  Us}
\item
  \href{https://www.nytco.com/careers/}{Work with us}
\item
  \href{https://nytmediakit.com/}{Advertise}
\item
  \href{http://www.tbrandstudio.com/}{T Brand Studio}
\item
  \href{https://www.nytimes.com/privacy/cookie-policy\#how-do-i-manage-trackers}{Your
  Ad Choices}
\item
  \href{https://www.nytimes.com/privacy}{Privacy}
\item
  \href{https://help.nytimes.com/hc/en-us/articles/115014893428-Terms-of-service}{Terms
  of Service}
\item
  \href{https://help.nytimes.com/hc/en-us/articles/115014893968-Terms-of-sale}{Terms
  of Sale}
\item
  \href{https://spiderbites.nytimes.com}{Site Map}
\item
  \href{https://help.nytimes.com/hc/en-us}{Help}
\item
  \href{https://www.nytimes.com/subscription?campaignId=37WXW}{Subscriptions}
\end{itemize}
