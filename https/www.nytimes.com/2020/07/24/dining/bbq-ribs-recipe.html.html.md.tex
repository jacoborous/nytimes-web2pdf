Sections

SEARCH

\protect\hyperlink{site-content}{Skip to
content}\protect\hyperlink{site-index}{Skip to site index}

\href{https://www.nytimes.com/section/food}{Food}

\href{https://myaccount.nytimes.com/auth/login?response_type=cookie\&client_id=vi}{}

\href{https://www.nytimes.com/section/todayspaper}{Today's Paper}

\href{/section/food}{Food}\textbar{}6 Steps to the Best Barbecued Ribs

\url{https://nyti.ms/2WUm17H}

\begin{itemize}
\item
\item
\item
\item
\item
\item
\end{itemize}

\href{https://www.nytimes.com/spotlight/at-home?action=click\&pgtype=Article\&state=default\&region=TOP_BANNER\&context=at_home_menu}{At
Home}

\begin{itemize}
\tightlist
\item
  \href{https://www.nytimes.com/2020/07/28/books/time-for-a-literary-road-trip.html?action=click\&pgtype=Article\&state=default\&region=TOP_BANNER\&context=at_home_menu}{Take:
  A Literary Road Trip}
\item
  \href{https://www.nytimes.com/2020/07/29/magazine/bored-with-your-home-cooking-some-smoky-eggplant-will-fix-that.html?action=click\&pgtype=Article\&state=default\&region=TOP_BANNER\&context=at_home_menu}{Cook:
  Smoky Eggplant}
\item
  \href{https://www.nytimes.com/2020/07/27/travel/moose-michigan-isle-royale.html?action=click\&pgtype=Article\&state=default\&region=TOP_BANNER\&context=at_home_menu}{Look
  Out: For Moose}
\item
  \href{https://www.nytimes.com/interactive/2020/at-home/even-more-reporters-editors-diaries-lists-recommendations.html?action=click\&pgtype=Article\&state=default\&region=TOP_BANNER\&context=at_home_menu}{Explore:
  Reporters' Obsessions}
\end{itemize}

Advertisement

\protect\hyperlink{after-top}{Continue reading the main story}

Supported by

\protect\hyperlink{after-sponsor}{Continue reading the main story}

\hypertarget{6-steps-to-the-best-barbecued-ribs}{%
\section{6 Steps to the Best Barbecued
Ribs}\label{6-steps-to-the-best-barbecued-ribs}}

A backyard grill can easily produce the spicy, smoky slabs that for many
are barbecue's ultimate prize.

\includegraphics{https://static01.nyt.com/images/2020/07/29/dining/29Ribs1/29Ribs1-articleLarge-v2.jpg?quality=75\&auto=webp\&disable=upscale}

By Steven Raichlen

\begin{itemize}
\item
  Published July 24, 2020Updated July 28, 2020
\item
  \begin{itemize}
  \item
  \item
  \item
  \item
  \item
  \item
  \end{itemize}
\end{itemize}

Let Texans brag about brisket and Carolinians extol pulled pork
shoulder. For the rest of us, the ultimate emblem of barbecue --- and
test of a grill master's mettle --- is ribs.

Picture meaty slabs stung with spice, bronzed with smoke and slathered
with sticky sweet barbecue sauce. The meat is tender, but not too
tender, with a profound pork flavor enhanced by the pit master's art.

You may have thought such alchemy possible only at the best barbecue
joints. But great ribs are surprisingly easy to make at home, which is
good news at a time when eating out can be fraught.

Ribs are barbecue at its most primal and unadorned; indeed, that's the
crux of their appeal. It's conceivable that you might eat
\href{https://www.nytimes.com/2019/05/24/dining/smoked-brisket.html}{brisket}
with a knife and fork, or more likely between two slices of white bread.
Pork shoulder comes either shredded or chopped, and you always eat it on
a bun.

Ribs, on the other hand, demand to be devoured caveman-style, ripped
apart with bare hands and gnawed right off the bone.

I won't say that cooking them is quick --- if you want speed, grill a
steak. But if ribs take two to three hours in all, the actual prep can
be done in 30 minutes. True, the process as I've laid it out here
requires a homemade spice rub and barbecue sauce, and turning your grill
into a smoker. But the results are achievable by all, and eminently
worth the effort.

Just follow these six steps, and the recipe I've provided. No special
equipment is needed beyond the grill you probably already have in your
backyard.

\hypertarget{1-choose-the-right-ribs}{%
\subsection{1. Choose the right ribs.}\label{1-choose-the-right-ribs}}

\includegraphics{https://static01.nyt.com/images/2020/07/29/dining/29Ribs2/merlin_107503807_074ed3d2-a1ba-4533-b668-2fb016b8cf90-articleLarge.jpg?quality=75\&auto=webp\&disable=upscale}

The pig supplies four types of ribs: baby backs (sometimes called top
loin ribs), spareribs, rib tips and country-style ribs. You want to use
the baby backs, which are cut from high on the hog (quite literally, as
they abut the backbone). Baby backs have the most generous marbling and
the tenderest meat, which makes them relatively quick to cook --- and a
natural for newcomers. When possible, buy ribs from a heritage pork
breed, like Berkshire (sometimes called Kurobuta) or Mangalitsa. They
cost more, but their intense porky flavor justifies the price.

\hypertarget{2-layer-the-flavors}{%
\subsection{2. Layer the flavors.}\label{2-layer-the-flavors}}

Image

A dry rub, a barbecue sauce and an initial layer of mustard help to
layer flavor onto the ribs.Credit...Tara Donne for The New York Times.
Food Stylist: Chris Lanier.

One of the secrets of great ribs --- indeed, great barbecue in general
--- is a process that creates layers of flavors. I start with a slather,
like Dijon mustard, that I brush on both sides of each rack of ribs.
Next, I apply a rub --- in the recipe here, a fragrant amalgam of chile
powder, brown sugar, salt and pepper, with celery seed added for spice.
The third layer comes from apple cider, which you spritz on halfway
through cooking. (This also helps keep the ribs moist.) The fourth layer
--- the varnish, as it were --- takes the form of a chipotle bourbon
barbecue sauce, which you sear into the meat over a hot fire, creating a
glossy finish. The crowning touch is a light, fresh sprinkle of rub
added right before serving the ribs to bring attention back to the
spice.

\hypertarget{3-grill-over-indirect-heat}{%
\subsection{3. Grill over indirect
heat.}\label{3-grill-over-indirect-heat}}

Image

Indirect heat keeps the ribs from overcooking.Credit...Tara Donne for
The New York Times. Food Stylist: Chris Lanier.

Most professional pit masters cook ribs low and slow in a smoker. You're
going to use a hotter and faster method called indirect grilling. In
short, you cook the ribs next to, not directly over, the fire, with the
grill lid closed and hardwood added to produce wood smoke.

To set up a charcoal grill for indirect grilling, light the coals, then
pour or rake them into two mounds at opposite sides of the grill. Place
a foil pan in the center to catch the dripping rib fat. The ribs go onto
the grate over this drip pan, away from the heat.

To set up a two-burner gas grill for indirect grilling, light one side
and cook the ribs on the unlit side. On a three-burner gas grill, light
the outside or front and rear burners, and cook the ribs over the unlit
burner in the center. On a four- to six-burner gas grill, light the
outside burners and, again, cook the ribs in the center.

On a kamado-style grill, insert the heat diffuser, a ceramic plate that
separates the food from the fire. Pellet grills, by their very design,
grill indirectly, so no special setup is needed. Note that with all
these types of grills, the lid must be closed.

If cooking four or more racks of ribs, you may want to invest in a rib
rack, which holds the slabs vertically, allowing you to fit four racks
of ribs in the space two slabs would take lying flat.

\hypertarget{4-apply-the-smoke}{%
\subsection{4. Apply the smoke.}\label{4-apply-the-smoke}}

Image

Smoke imparts the flavor that many American barbecue fans
crave.Credit...Tara Donne for The New York Times. Food Stylist: Chris
Lanier.

Wood smoke has been called the umami of barbecue. It is certainly
barbecue's soul. While you can make delectable baby back ribs without
wood smoke, as the French and Brazilians do, they won't taste like
American barbecue. So which wood to use? Debate rages in barbecue
circles over the superiority of apple versus cherry, hickory versus
mesquite, or whether to employ a combination of several woods. Mesquite
lends the strongest flavor, but any hardwood chunk or chip will deliver
the requisite smokiness. I smoked the ribs in my recipe with cherry
wood, simply because I had it on hand.

To smoke ribs on a charcoal grill, add hardwood chunks or soaked,
drained hardwood chips to the embers. (Soaking helps slow the rate of
combustion, so the chips smolder and smoke before they catch fire.) In a
kamado, intersperse the chunks or chips with the charcoal. A pellet
grill has the wood, and smoke, built into the pellets. (Note: On a
pellet grill you get more smoke flavor at lower temperatures, so
lengthen the cooking time accordingly.)

It's harder, but not impossible, to smoke on a gas grill. If your grill
has a smoker box with a dedicated burner, add chunks or chips there. If
not, place a few hardwood chunks under the grate, directly over the
burners. Or make smoking pouches: Wrap soaked, drained wood chips in
heavy-duty foil to form a flat pillow shape. Poke holes at one-inch
intervals in the top with the tip of a meat thermometer, to release the
smoke. Position the resulting pouches (two for the ribs here) under the
grate, directly over the burners. Run the grill on high until you see
smoke, then dial the temperature back to 300 degrees.

What about smoking ribs indoors? Desperate times like these call for
desperate measures. I know it smacks of heresy, but you can achieve a
reasonable approximation of barbecued ribs in the oven. Cook them on a
rack in a roasting pan at 300 degrees. To add a smoke flavor, mix
one-half teaspoon liquid smoke with four tablespoons melted butter, and
brush this mixture on both sides of the ribs a few times during the last
hour of cooking.

\hypertarget{5-sizzle-the-sauce}{%
\subsection{5. Sizzle the sauce.}\label{5-sizzle-the-sauce}}

Image

The sauce is seared into the meat in the last few minutes of
grilling.Credit...Tara Donne for The New York Times. Food Stylist: Chris
Lanier.

While barbecue sauce isn't mandatory, for most Americans, ribs just
don't taste complete without it.
\href{https://cooking.nytimes.com/recipes/1021242-spice-rubbed-baby-back-ribs-with-chipotle-bourbon-barbecue-sauce}{This
version} calls for one of my favorites --- a sweet, smoky blend of
molasses, brown sugar and ketchup, with bourbon for kick and chipotle
chiles to crank up the heat. Here, too, the sauce goes on in layers ---
first brushed on and roasted into the ribs during the last 20 minutes of
cooking, then applied again and seared into the meat over high heat, and
finally served with the ribs for spooning or dipping.

A crucial factor is the sizzle, which involves directly grilling the
ribs (move them right over the fire or lit burners) for the final four
minutes or so --- long enough to caramelize the brown sugar and
molasses, and sear the sauce into the meat.

Take care to avoid the cardinal sin of applying the sauce too early,
when the ribs first go on the grill. That's what grown-ups did when I
was young, and the sugar in the sauce invariably burned long before the
meat was cooked. For many years, I thought barbecue was supposed to
taste burned.

\hypertarget{6-know-when-your-ribs-are-done}{%
\subsection{6. Know when your ribs are
done.}\label{6-know-when-your-ribs-are-done}}

Image

When the rib bones are exposed by a quarter- to a half-inch, the ribs
are done.Credit...Tara Donne for The New York Times. Food Stylist: Chris
Lanier.

Pork ribs come with their own version of the pop-up thermometer that
signals that a turkey is done: The meat shrinks back from the ends of
the bones. When you see a quarter- to a half-inch of clean bone at the
end of each rib, it is ready. You should be able to pull the individual
ribs apart with your fingers. The meat should resist, but just a little.

Many Americans are accustomed to meat that falls off the bones --- a
style that may have begun with the doleful practice of boiling or
steaming ribs before grilling. The notion of soft (dare I say mushy?)
ribs took root in our collective culinary consciousness.

Root out that idea now. Yes, ribs should be tender, but they should
definitely retain a little chew. That's why we have teeth.

Recipe:
\textbf{\href{https://cooking.nytimes.com/recipes/1021242-spice-rubbed-baby-back-ribs-with-chipotle-bourbon-barbecue-sauce}{Spice-Rubbed
Baby Back Ribs With Chipotle-Bourbon Barbecue Sauce}}

\emph{Follow} \href{https://twitter.com/nytfood}{\emph{NYT Food on
Twitter}} \emph{and}
\href{https://www.instagram.com/nytcooking/}{\emph{NYT Cooking on
Instagram}}\emph{,}
\href{https://www.facebook.com/nytcooking/}{\emph{Facebook}}\emph{,}
\href{https://www.youtube.com/nytcooking}{\emph{YouTube}} \emph{and}
\href{https://www.pinterest.com/nytcooking/}{\emph{Pinterest}}\emph{.}
\href{https://www.nytimes.com/newsletters/cooking}{\emph{Get regular
updates from NYT Cooking, with recipe suggestions, cooking tips and
shopping advice}}\emph{.}

Advertisement

\protect\hyperlink{after-bottom}{Continue reading the main story}

\hypertarget{site-index}{%
\subsection{Site Index}\label{site-index}}

\hypertarget{site-information-navigation}{%
\subsection{Site Information
Navigation}\label{site-information-navigation}}

\begin{itemize}
\tightlist
\item
  \href{https://help.nytimes.com/hc/en-us/articles/115014792127-Copyright-notice}{©~2020~The
  New York Times Company}
\end{itemize}

\begin{itemize}
\tightlist
\item
  \href{https://www.nytco.com/}{NYTCo}
\item
  \href{https://help.nytimes.com/hc/en-us/articles/115015385887-Contact-Us}{Contact
  Us}
\item
  \href{https://www.nytco.com/careers/}{Work with us}
\item
  \href{https://nytmediakit.com/}{Advertise}
\item
  \href{http://www.tbrandstudio.com/}{T Brand Studio}
\item
  \href{https://www.nytimes.com/privacy/cookie-policy\#how-do-i-manage-trackers}{Your
  Ad Choices}
\item
  \href{https://www.nytimes.com/privacy}{Privacy}
\item
  \href{https://help.nytimes.com/hc/en-us/articles/115014893428-Terms-of-service}{Terms
  of Service}
\item
  \href{https://help.nytimes.com/hc/en-us/articles/115014893968-Terms-of-sale}{Terms
  of Sale}
\item
  \href{https://spiderbites.nytimes.com}{Site Map}
\item
  \href{https://help.nytimes.com/hc/en-us}{Help}
\item
  \href{https://www.nytimes.com/subscription?campaignId=37WXW}{Subscriptions}
\end{itemize}
