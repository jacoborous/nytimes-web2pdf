Sections

SEARCH

\protect\hyperlink{site-content}{Skip to
content}\protect\hyperlink{site-index}{Skip to site index}

\href{/section/world/europe}{Europe}\textbar{}As Neo-Nazis Seed Military
Ranks, Germany Confronts `an Enemy Within'

\url{https://nyti.ms/38qjxTh}

\begin{itemize}
\item
\item
\item
\item
\item
\item
\end{itemize}

\includegraphics{https://static01.nyt.com/images/2020/07/05/world/00germany-ksk1/merlin_174034917_e33227ac-794e-4500-83e4-fc410dee5818-articleLarge.jpg?quality=75\&auto=webp\&disable=upscale}

\hypertarget{as-neo-nazis-seed-military-ranks-germany-confronts-an-enemy-within}{%
\section{As Neo-Nazis Seed Military Ranks, Germany Confronts `an Enemy
Within'}\label{as-neo-nazis-seed-military-ranks-germany-confronts-an-enemy-within}}

After plastic explosives and Nazi memorabilia were found at an elite
soldier's home, Germany worries about a problem of far-right
infiltration at the heart of its democracy.

Shooting drills at the base of the KSK, the military special forces, in
Calw, Germany.Credit...Laetitia Vancon for The New York Times

Supported by

\protect\hyperlink{after-sponsor}{Continue reading the main story}

\href{https://www.nytimes.com/by/katrin-bennhold}{\includegraphics{https://static01.nyt.com/images/2018/07/13/multimedia/author-katrin-bennhold/author-katrin-bennhold-thumbLarge.png}}

By \href{https://www.nytimes.com/by/katrin-bennhold}{Katrin Bennhold}

\begin{itemize}
\item
  Published July 3, 2020Updated July 10, 2020
\item
  \begin{itemize}
  \item
  \item
  \item
  \item
  \item
  \item
  \end{itemize}
\end{itemize}

\href{https://www.nytimes.com/es/2020/07/10/espanol/mundo/alemania-ksk-neonazi.html}{Leer
en español}

CALW, Germany --- As Germany emerged from its coronavirus lockdown in
May, police commandos pulled up outside a rural property owned by a
sergeant major in the special forces, the country's most highly trained
and secretive military unit.

They brought a digger.

The sergeant major's nickname was Little Sheep. He was suspected of
being a neo-Nazi. Buried in the garden, the police found two kilograms
of PETN plastic explosives, a detonator, a fuse, an AK-47, a silencer,
two knives, a crossbow and thousands of rounds of ammunition, much of it
believed to have been stolen from the German military.

They also found an SS songbook, 14 editions of a magazine for former
members of the Waffen SS and a host of other Nazi memorabilia.

``He had a plan,'' said Eva Högl, Germany's parliamentary commissioner
for the armed forces. ``And he is not the only one.''

Germany has a problem. For years, politicians and security chiefs
rejected the notion of any far-right infiltration of the security
services, speaking only of ``individual cases.'' The idea of networks
was dismissed. The superiors of those exposed as extremists were
protected. Guns and ammunition disappeared from military stockpiles with
no real investigation.

The government is now waking up. Cases of far-right extremists in the
military and the police, some hoarding weapons and explosives, have
multiplied alarmingly. The nation's top intelligence officials and
senior military commanders are moving to confront an issue that has
become too dangerous to ignore.

The problem has deepened with the emergence of the Alternative for
Germany party, or AfD, which
\href{https://www.nytimes.com/2019/10/26/world/europe/afd-election-east-germany-hoecke.html}{legitimized
a far-right ideology} that used the arrival of more than a million
migrants in 2015 --- and more recently the
\href{https://www.nytimes.com/2020/05/18/world/europe/coronavirus-germany-far-right.html}{coronavirus
pandemic} --- to engender a sense of impending crisis.

\includegraphics{https://static01.nyt.com/images/2020/06/29/world/00germany-ksk2/merlin_174042501_09225840-1504-43db-87ec-5e49ad58a061-articleLarge.jpg?quality=75\&auto=webp\&disable=upscale}

Most concerning to the authorities is that the extremists appear to be
concentrated in the military unit that is supposed to be the most elite
and dedicated to the German state, the special forces, known by their
German acronym, the KSK.

This week, Germany's defense minister, Annegret Kramp-Karrenbauer, took
the drastic step of
\href{https://www.nytimes.com/2020/07/01/world/europe/german-special-forces-far-right.html}{disbanding
a fighting company} in the KSK considered infested with extremists.
Little Sheep, the sergeant major whose weapons stash was uncovered in
May, was a member.

Some 48,000 rounds of ammunition and 62 kilograms, or about 137 pounds,
of explosives have disappeared from the KSK altogether, she said.

Germany's military counterintelligence agency is now investigating more
than 600 soldiers for far-right extremism, out of 184,000 in the
military. Some 20 of them are in the KSK, a proportion that is five
times higher than in other units.

But the German authorities are concerned that the problem may be far
larger and that other security institutions have been infiltrated as
well. Over the past 13 months, far-right terrorists have
\href{https://www.nytimes.com/2019/06/17/world/europe/germany-terrorism-walter-lubcke.html}{assassinated
a politician},
\href{https://www.nytimes.com/2019/10/10/world/europe/germany-synagogue-attack.html?searchResultPosition=10}{attacked
a synagogue} and
\href{https://www.nytimes.com/2020/02/20/world/europe/germany-hanau-shisha-bar-shooting.html?searchResultPosition=28}{shot
dead nine immigrants} and German descendants of immigrants.

Thomas Haldenwang, president of Germany's domestic intelligence agency,
has identified far-right extremism and terrorism as the
``\href{https://www.nytimes.com/2020/02/21/world/europe/germany-shooting-terrorism.html?searchResultPosition=25}{biggest
danger} to German democracy today.''

In interviews I conducted over the course of the year with military and
intelligence officials, and avowed far-right members themselves, they
described nationwide networks of current and former soldiers and police
officers with ties to the far right.

In many cases, soldiers have used the networks to prepare for when they
predict Germany's democratic order will collapse. They call it Day X.
Officials worry it is really a pretext for inciting terrorist acts, or
worse, a putsch.

``For far-right extremists, the preparation of Day X and its
precipitation blend into one another,'' Martina Renner, a lawmaker on
the homeland security committee of the German Parliament, told me.

The ties, officials say, sometimes reach deep into old neo-Nazi networks
and the more polished intellectual scene of the so-called
\href{https://www.nytimes.com/2018/12/27/world/europe/germany-far-right-generation-identity.html}{New
Right}. Extremists are hoarding weapons, maintaining safe houses, and in
some cases keeping lists of political enemies.

This week yet another case emerged, of a reservist, now suspended, who
kept a list with cellphone numbers and addresses of 17 prominent
politicians, who have been alerted. The case led to at least nine other
raids across the country on Friday.

Some German news media have referred to a
\href{https://taz.de/Rechtes-Netzwerk-in-der-Bundeswehr/!5548926/}{``shadow
army,''} drawing parallels to the 1920s, when nationalist cells within
the military hoarded arms, plotted coups and conspired to overthrow
democracy.

Most officials still reject this analogy. But the striking lack of
understanding of the numbers involved, even at the highest levels of the
government, has contributed to a deep unease.

``Once they really started looking, they found a lot of cases,'' said
Konstantin von Notz, deputy president of the intelligence oversight
committee in the German Parliament. ``When you have hundreds of
individual cases it begins to look like we have a structural problem. It
is extremely worrying.''

Mr. von Notz pointed out that Brendan Tarrant, who massacred 51 Muslim
worshipers last year at two mosques in Christchurch, New Zealand, had
traveled Europe a year earlier and included an ominous line in his
manifesto.

Image

A memorial to victims of the shooting at two mosques last year in
Christchurch, New Zealand, that killed 51 worshipers. The gunman had
written of nationalist infiltration of European armed
forces.Credit...Adam Dean for The New York Times

``I would estimate the number of soldiers in European armed forces that
also belong to nationalist groups to number in the hundreds of
thousands, with just as many employed in law enforcement positions,''
Mr. Tarrant had written.

Investigators, Mr. von Notz said, ``should take these words seriously.''

But investigating the problem is itself fraught: Even the military
counterintelligence agency, charged with monitoring extremism inside the
armed forces, may be infiltrated.

A high-ranking investigator in the extremism unit was suspended in June
after sharing confidential material from the May raid with a contact in
the KSK, who in turn passed it on to at least eight other soldiers,
tipping them off that the agency might turn its attention to them next.

``If the very people who are meant to protect our democracy are plotting
against it, we have a big problem,'' said Stephan Kramer, president of
the domestic intelligence agency in the state of Thuringia. ``How do you
find them?''

``These are battle-hardened men who know how to evade surveillance
because they are trained in conducting surveillance themselves,'' he
added.

``What we are dealing with is an enemy within.''

\hypertarget{inside-the-shoot-house}{%
\subsection{Inside the `Shoot House'}\label{inside-the-shoot-house}}

The air inside the ``shoot house'' smelled acrid, so many live rounds
had been fired.

I was standing in the shooting range on the outskirts of the sleepy
German town of Calw, in the Black Forest region, having been invited
early this year for a rare visit inside the KSK's base, the most heavily
guarded in the country.

A camouflaged soldier with a G36 assault rifle crouched along a broken
door frame. Two shadows popped up. The soldier fired four times ---
head, torso, head, torso --- then went on to systematically eliminate
two dozen other ``enemies.'' He did not miss once.

Image

Targets at the ``shoot house'' on the KSK base.Credit...Laetitia Vancon
for The New York Times

The KSK are Germany's answer to the Navy Seals. But these days their
commander, Gen. Markus Kreitmayr, an affable Bavarian who has done tours
in Bosnia, Kosovo and Afghanistan, is a man divided between his loyalty
to them and recognizing that he has a serious problem on his hands.

The general was late for our interview. He had just spent four hours
questioning a member of his unit about a party where half a dozen KSK
soldiers were reported to have flashed Hitler salutes.

``I can't explain why there are allegedly so many cases of `far-right
extremism' in the military,'' he said. The KSK is ``clearly more
affected than others, that appears to be a fact.''

It was never easy to be a soldier in postwar Germany. Given its Nazi
history and the destruction it foisted on Europe in World War II, the
country maintains a conflicted relationship to its military.

For decades, Germany tried to forge a force that represented a
democratic society and its values. But in 2011 it
\href{https://www.nytimes.com/2011/07/01/world/europe/01germany.html}{abolished
conscription} and moved to a volunteer force. As a result, the military
increasingly reflects not the broad society, but a narrower slice of it.

General Kreitmayr said that ``a big percentage'' of his soldiers are
eastern Germans, a region where the AfD does disproportionately well.
Roughly half the men on the list of KSK members suspected of being
far-right extremists are also from the east, he added.

Image

Gen. Markus Kreithmayr, at rear, has called the current crisis in the
KSK unit ``the most difficult phase in its history.''Credit...Laetitia
Vancon for The New York Times

The general has called the current crisis in the unit ``the most
difficult phase in its history.''

In our interview, he said that he could not rule out a significant
degree of infiltration from the far right. ``I don't know if there is a
shadow army in Germany,'' he told me.

``But I am worried,'' he said, ``and not just as the commander of the
KSK, but as a citizen --- that in the end something like that does exist
and that maybe our people are part of it.''

Officials talk of a perceptible shift ``in values'' among new recruits.
In conversations, the soldiers themselves, who could not be identified
under the unit's guidelines, said that if there was a tipping point in
the unit, it came with the migrant crisis of 2015.

As hundreds of thousands of asylum seekers from Syria and Afghanistan
were making their way to Germany, the mood on the base was anxious, they
recalled.

``We are soldiers who are charged with defending this country and then
they just opened the borders, no control,'' one officer recalled. ``We
were at the limit.''

It was in this atmosphere that a 30-year-old KSK soldier from Halle, in
eastern Germany, set up a Telegram chat network for soldiers, police
officers and others united in their belief that the migrants would
destroy the country.

His name was André Schmitt. But he goes by the nickname Hannibal.

\hypertarget{hannibals-network}{%
\subsection{Hannibal's Network}\label{hannibals-network}}

In a house in rural western Germany, behind a curtain of iron chains and
past the crossbow in the hall, a dungeonlike room bathed in purple light
opens into a bar area. An oversized image of a naked woman dominates the
back wall.

It was there that I met Mr. Schmitt early this year. He gave permission
for his name to be used, but did not want the location disclosed or any
photographs.

He left active service last September after stolen training grenades
were found at a building belonging to his parents. But, he says, he
still has his network: ``Special forces, intelligence, business
executives, Freemasons,'' he said. They meet here regularly. The house,
he says, is owned by a wealthy supporter.

Image

A tactical defense training workshop in March in the state of North
Rhine-Westphalia organized by Uniter,~a private network for
security-related personnel.Credit...Laetitia Vancon for The New York
Times

``The forces are like a big family,'' Mr. Schmitt told me, ``everyone
knows each other.''

When he set up his Telegram chats in 2015, he did so geographically ---
north, south, east, west --- just like the German military. In parallel,
he ran a group called Uniter, an organization for security-related
professionals that provides social benefits but also paramilitary
training.

Several former members of his chats are now under investigation by
prosecutors for plotting terrorism. Some were ordering body bags. One
faces trial.

Mr. Schmitt's situation is more complex. He acknowledged serving as an
informer on the KSK for the military counterintelligence agency in
mid-2017, when he met regularly with a liaison officer. Today the
military is paying for him to get a business degree.

He himself was never named a suspect. German officials denied that they
protected him. But this week the domestic intelligence agency announced
that it was placing his current network, Uniter, under surveillance.

The authorities first stumbled onto his chats in 2017 while
investigating a soldier in the network who was suspected of organizing a
terror plot.

Investigators are now looking into whether the chats and Uniter were the
early skeleton of a nationwide far-right network that has infiltrated
state institutions. As yet, they cannot say. The New York Times obtained
police statements by Mr. Schmitt and others in his network related to
the 2017 case.

Initially, Mr. Schmitt and other members say, the chats were about
sharing information, much of it about the supposed threats posed by
migrants, which Mr. Schmitt admitted to the police he had inflated to
``motivate'' people.

Image

A refugee family waited to board a train to Germany at the Keleti train
station in Budapest in 2015. Europe experienced a huge influx of people
fleeing conflict in Syria and Afghanistan that year.Credit...Mauricio
Lima for The New York Times

``It was about internal unrest because of sleeper cells and worldwide
extremist groups, gang formations, terrorist threats,'' Mr. Schmitt told
the police.

The chats were popular among KSK soldiers. Mr. Schmitt said he counted
69 of his comrades in the network in 2015.

A fellow KSK soldier, identified by investigators as Robert P., but
known as Petrus, who ran two of the chats, told the police two years
later that it might have been more than twice that: ``I have to say,
presumably half the unit was in there.''

Soon the chats morphed from a platform for sharing information to one
dedicated to preparing for Day X. Sipping mineral water, Mr. Schmitt
described this as ``war gaming.'' He portrayed a Europe under threat
from gangs, Islamists and Antifa. He called them ``enemy troops on our
ground.''

His network helped members get ready to respond to what he portrayed as
an inevitable conflict, sometimes acting on their own.

``Day X is personal,'' he said. ``For one guy it's this day, for another
guy it's another day.''

``It's the day you activate your plans,'' he said.

Chat members met in person, worked out what provisions and weapons to
stockpile, and where to keep safe houses. Dozens were identified. One
was the military base in Calw itself. They practiced how to recognize
each other, using military code, at ``pickup points'' where members
could gather on Day X.

Image

The town of Calw in the Black Forest region. All KSK soldiers are
stationed at a base outside the town.Credit...Laetitia Vancon for The
New York Times

The sense of urgency grew.

On March 21, 2016, a chat member, identified only as Matze, wrote about
a pickup point near Nuremberg. There were, he wrote, ``sufficient
weapons and ammo present to battle one's way on.''

Later that year, Mr. Schmitt sent a message to others in the chat
network. In the previous 18 months, he wrote, they had gathered ``2,000
like-minded people'' in Germany and abroad.

When I met him, Mr. Schmitt called it ``a global like-minded
brotherhood.''

He denies ever planning to bring about Day X, but he is still convinced
that it will come, maybe sooner rather than later with the pandemic.

``We know thanks to our sources in the banks and in the intelligence
services that at the latest by the end of September the big economic
crash will come,'' he said in a follow-up phone call this week.

``There will be insolvencies and mass unemployment,'' he prophesied.
``People will take to the street.''

\hypertarget{pig-heads-and-hitler-salutes}{%
\subsection{Pig Heads and Hitler
Salutes}\label{pig-heads-and-hitler-salutes}}

One night in 2017, Little Sheep, the sergeant major whose weapons stash
was uncovered in May, was among about 70 KSK soldiers of Second Company
who had gathered at a military shooting range.

Investigators have identified him only as Philipp Sch. He and the others
had organized a special leaving party for a lieutenant colonel, a man
celebrated as a war hero for shooting his way out of an ambush in
Afghanistan while carrying one of his men.

The colonel, an imposing man covered in Cyrillic tattoos who enjoys
cage-fighting in his spare time, had to complete an obstacle course. It
involved hacking apart tree trunks and throwing severed pig heads.

As a prize, his men had flown in a woman. But the colonel ended up dead
drunk. The woman, rather than being his trophy, went to the police.

Standing by the fire with a handful of soldiers, she had witnessed them
singing neo-Nazi lyrics and raising their right arm. One man stood out
for his enthusiasm, she recalled in a
\href{https://daserste.ndr.de/panorama/archiv/2017/Hitlergruss-Ermittlungen-gegen-Kompaniechef,bundeswehr1738.html}{televised
report} by the public broadcaster ARD. She called him the ``Nazi
grandpa.''

Image

The Zeppelin Bar at the KSK base in Calw. As home to the special forces
unit, the base is the most heavily guarded in Germany.Credit...Laetitia
Vancon for The New York Times

Though just 45, ``the Nazi grandpa'' was Little Sheep, who had joined
the KSK in 2001.

In the three years since the party, the military counterintelligence
service kept an eye on the sergeant major. But that did not stop the KSK
from promoting him to the highest possible noncommissioned officer rank.

The handling of the case fit a pattern, soldiers and officials say.

In June, a KSK soldier addressed a 12-page letter to the defense
minister, pleading for an investigation into what he described as a
``toxic culture of acceptance'' and ``culture of fear'' inside the unit.
Tips about extremist comrades were ``collectively ignored or even
tolerated.'' One of his instructors had likened the KSK to the Waffen
SS, the soldier wrote.

The instructor, a lieutenant colonel, was himself on the radar for
far-right leanings since 2007, when he wrote a threatening email to
another soldier. ``You are being watched, no, not by impotent
instrumentalized agencies, but by officers of a new generation, who will
act when the times demand it,'' it read. ``Long live the holy Germany.''

The KSK commander at the time did not suspend the lieutenant. He merely
disciplined him. I asked General Kreitmayr, who took over command in
2018, about the case.

``Look, today in the year 2020, with all the knowledge that we have, we
look at the email from 2007 and say, `It's obvious,''' he told me.

``But at that time we only thought: Man, what's wrong with him? He
should pull himself together.''

\hypertarget{the-hallway-of-history}{%
\subsection{The Hallway of History}\label{the-hallway-of-history}}

The back door of the main building on the base in Calw leads into a long
corridor known as the ``hallway of history,'' a collection of
memorabilia gathered over the KSK's nearly 25 years that includes a
stuffed German shepherd, Kato, who parachuted from 30,000 feet with a
commando team.

Image

A hallway at the base in Calw displays memorabilia from the KSK's nearly
25-year history.Credit...Laetitia Vancon for The New York Times

Conspicuously missing is any mention of a disgraced former KSK
commander, Gen. Reinhard Günzel, who was dismissed after he wrote a 2003
letter in support of an anti-Semitic speech by a conservative lawmaker.

General Günzel subsequently published a book called ``Secret Warriors.''
In it, he placed the KSK in the tradition of a notorious special forces
unit under the Nazis that committed numerous war crimes, including
massacres of Jews. He has been a popular speaker at far-right events.

``What you basically have is one of the founding commanders of the KSK
becoming a prominent ideologue of the New Right,'' said Christian
Weissgerber, a former soldier who has
\href{https://www.ofv.ch/sachbuch/detail/mein-vaterland-warum-ich-ein-neonazi-war/103760/}{written
a book} about his own experience of being a neo-Nazi in the military.

\href{https://www.nytimes.com/2018/12/27/world/europe/germany-far-right-generation-identity.html}{The
New Right}, which encompasses youth activists, intellectuals and the
AfD, worries General Kreitmayr. \href{https://www.martinhohmann.de/}{The
lawmaker} whose anti-Semitic comments led to General Günzel's firing all
those years ago now sits in the German Parliament for the AfD.

``You have leading representatives of political parties like the AfD,
who say things that not only make you sick but that are clearly
far-right, radical ideology,'' General Kreitmayr said.

Soldiers were not immune to this cultural shift in the country, he said.
Just recently a fellow general had become a mayoral candidate for the
AfD. Several former soldiers represent the party in Parliament.

Image

The Reichstag in Berlin, the home of the German Parliament. Several
former soldiers represent the far-right AfD party in
Parliament.Credit...Emile Ducke for The New York Times

Down the hill from the shoot house is the Green Saloon, a cross between
a boardroom and a bar. It is dominated by a vast oil painting depicting
KSK soldiers and their German shepherd successfully attacking a Taliban
hide-out.

It is a scene familiar to several soldiers who had gathered the day I
was there. But the soldiers I spoke with questioned the strategy behind
a war that has run for two decades with few concrete results, except an
increase in migration at home.

``My girls asked me: `Why do you have to go to Afghanistan when there
are children from the Kunduz in our class?''' recounted one officer. ``I
did not have an answer.''

When he took a delegation of KSK soldiers to meet with political parties
in Parliament, he asked them the same question. ``They did not have an
answer, either,'' he said.

Only one lawmaker made a clear statement, he said. He was from the AfD.
``He said we should have left a long time ago,'' the officer recalled.

Image

Material from Afghanistan exhibited at the Calw base.Credit...Laetitia
Vancon for The New York Times

\emph{Christopher F. Schuetze contributed reporting.}

Advertisement

\protect\hyperlink{after-bottom}{Continue reading the main story}

\hypertarget{site-index}{%
\subsection{Site Index}\label{site-index}}

\hypertarget{site-information-navigation}{%
\subsection{Site Information
Navigation}\label{site-information-navigation}}

\begin{itemize}
\tightlist
\item
  \href{https://help.nytimes.com/hc/en-us/articles/115014792127-Copyright-notice}{©~2020~The
  New York Times Company}
\end{itemize}

\begin{itemize}
\tightlist
\item
  \href{https://www.nytco.com/}{NYTCo}
\item
  \href{https://help.nytimes.com/hc/en-us/articles/115015385887-Contact-Us}{Contact
  Us}
\item
  \href{https://www.nytco.com/careers/}{Work with us}
\item
  \href{https://nytmediakit.com/}{Advertise}
\item
  \href{http://www.tbrandstudio.com/}{T Brand Studio}
\item
  \href{https://www.nytimes.com/privacy/cookie-policy\#how-do-i-manage-trackers}{Your
  Ad Choices}
\item
  \href{https://www.nytimes.com/privacy}{Privacy}
\item
  \href{https://help.nytimes.com/hc/en-us/articles/115014893428-Terms-of-service}{Terms
  of Service}
\item
  \href{https://help.nytimes.com/hc/en-us/articles/115014893968-Terms-of-sale}{Terms
  of Sale}
\item
  \href{https://spiderbites.nytimes.com}{Site Map}
\item
  \href{https://help.nytimes.com/hc/en-us}{Help}
\item
  \href{https://www.nytimes.com/subscription?campaignId=37WXW}{Subscriptions}
\end{itemize}
