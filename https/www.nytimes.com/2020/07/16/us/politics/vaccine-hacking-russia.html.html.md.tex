Sections

SEARCH

\protect\hyperlink{site-content}{Skip to
content}\protect\hyperlink{site-index}{Skip to site index}

\href{https://www.nytimes.com/section/politics}{Politics}

\href{https://myaccount.nytimes.com/auth/login?response_type=cookie\&client_id=vi}{}

\href{https://www.nytimes.com/section/todayspaper}{Today's Paper}

\href{/section/politics}{Politics}\textbar{}Russia Is Trying to Steal
Virus Vaccine Data, Western Nations Say

\url{https://nyti.ms/2Wsbe4n}

\begin{itemize}
\item
\item
\item
\item
\item
\end{itemize}

\href{https://www.nytimes.com/news-event/coronavirus?action=click\&pgtype=Article\&state=default\&region=TOP_BANNER\&context=storylines_menu}{The
Coronavirus Outbreak}

\begin{itemize}
\tightlist
\item
  live\href{https://www.nytimes.com/2020/08/02/world/coronavirus-updates.html?action=click\&pgtype=Article\&state=default\&region=TOP_BANNER\&context=storylines_menu}{Latest
  Updates}
\item
  \href{https://www.nytimes.com/interactive/2020/us/coronavirus-us-cases.html?action=click\&pgtype=Article\&state=default\&region=TOP_BANNER\&context=storylines_menu}{Maps
  and Cases}
\item
  \href{https://www.nytimes.com/interactive/2020/science/coronavirus-vaccine-tracker.html?action=click\&pgtype=Article\&state=default\&region=TOP_BANNER\&context=storylines_menu}{Vaccine
  Tracker}
\item
  \href{https://www.nytimes.com/interactive/2020/07/29/us/schools-reopening-coronavirus.html?action=click\&pgtype=Article\&state=default\&region=TOP_BANNER\&context=storylines_menu}{What
  School May Look Like}
\item
  \href{https://www.nytimes.com/live/2020/07/31/business/stock-market-today-coronavirus?action=click\&pgtype=Article\&state=default\&region=TOP_BANNER\&context=storylines_menu}{Economy}
\end{itemize}

Advertisement

\protect\hyperlink{after-top}{Continue reading the main story}

Supported by

\protect\hyperlink{after-sponsor}{Continue reading the main story}

\hypertarget{russia-is-trying-to-steal-virus-vaccine-data-western-nations-say}{%
\section{Russia Is Trying to Steal Virus Vaccine Data, Western Nations
Say}\label{russia-is-trying-to-steal-virus-vaccine-data-western-nations-say}}

The hackers have been targeting British, Canadian and American
organizations racing to create coronavirus vaccines.

\includegraphics{https://static01.nyt.com/images/2020/07/16/us/politics/16dc-intel/merlin_173444058_24505b78-df60-451b-8bfe-bd08c11f5724-articleLarge.jpg?quality=75\&auto=webp\&disable=upscale}

\href{https://www.nytimes.com/by/julian-e-barnes}{\includegraphics{https://static01.nyt.com/images/2019/12/13/reader-center/author-julian-barnes/author-julian-barnes-thumbLarge.png}}

By \href{https://www.nytimes.com/by/julian-e-barnes}{Julian E. Barnes}

\begin{itemize}
\item
  Published July 16, 2020Updated July 28, 2020
\item
  \begin{itemize}
  \item
  \item
  \item
  \item
  \item
  \end{itemize}
\end{itemize}

WASHINGTON ---
\href{https://www.nytimes.com/2020/07/28/us/politics/russia-disinformation-coronavirus.html}{Russian}
hackers are attempting to steal
\href{https://www.nytimes.com/2020/07/28/us/politics/russia-disinformation-coronavirus.html}{coronavirus
vaccine} research, the American, British and Canadian governments said
Thursday, accusing the Kremlin of opening a new front in its spy battles
with the West amid the worldwide competition to contain the pandemic.

The National Security Agency said that a hacking group implicated in the
2016 break-ins into Democratic Party servers has been trying to steal
intelligence on
\href{https://www.nytimes.com/interactive/2020/science/coronavirus-vaccine-tracker.html}{vaccines
from universities, companies and other health care organizations}. The
group, associated with Russian intelligence and known as both APT29 and
Cozy Bear, has sought to exploit the chaos created by the
\href{https://www.nytimes.com/news-event/coronavirus}{coronavirus
pandemic}, officials said.

American intelligence officials said the Russians were aiming to steal
research to develop their own vaccine more quickly, not to sabotage
other countries' efforts. There was likely little immediate damage to
global public health, cybersecurity experts said.

The Russian espionage nevertheless signals a new kind of competition
between Moscow and Washington akin to Cold War spies stealing
technological secrets during the space race generations ago.

The Russian hackers have targeted British, Canadian and American
organizations using malware and sending fraudulent emails to try to
trick their employees into turning over passwords and other security
credentials, all in an effort to gain access to the vaccine research as
well as information about medical supply chains.

The accusations against Russia were also the latest example of an
increasing willingness in recent months by the United States and its
closest intelligence allies to publicly accuse foreign adversaries of
breaches and cyberattacks. The American government has previously warned
about efforts
\href{https://www.nytimes.com/2020/05/10/us/politics/coronavirus-china-cyber-hacking.html}{by
China and Iran} to steal vaccine research.

Attributing such attacks, however, is imprecise, an ambiguity that
Moscow takes advantage of in denying responsibility, as it did Thursday.

Still, government officials, as well as outside experts, expressed
strong confidence that Cozy Bear, controlled by Russia's elite S.V.R.
intelligence agency, was responsible for the attempted intrusions into
the virus vaccine research.

``We condemn these despicable attacks against those doing vital work to
combat the coronavirus pandemic,'' said Paul Chichester, the director of
operations for Britain's National Cyber Security Center.

The head of the center, Ciaran Martin,
\href{https://www.nbcnews.com/news/world/russia-attempting-steal-coronavirus-vaccine-research-u-s-u-k-n1234021}{told
NBC News} that the cyberattacks were first detected in February and that
no evidence had emerged that data was stolen.

\hypertarget{latest-updates-global-coronavirus-outbreak}{%
\section{\texorpdfstring{\href{https://www.nytimes.com/2020/08/01/world/coronavirus-covid-19.html?action=click\&pgtype=Article\&state=default\&region=MAIN_CONTENT_1\&context=storylines_live_updates}{Latest
Updates: Global Coronavirus
Outbreak}}{Latest Updates: Global Coronavirus Outbreak}}\label{latest-updates-global-coronavirus-outbreak}}

Updated 2020-08-02T17:52:35.962Z

\begin{itemize}
\tightlist
\item
  \href{https://www.nytimes.com/2020/08/01/world/coronavirus-covid-19.html?action=click\&pgtype=Article\&state=default\&region=MAIN_CONTENT_1\&context=storylines_live_updates\#link-34047410}{The
  U.S. reels as July cases more than double the total of any other
  month.}
\item
  \href{https://www.nytimes.com/2020/08/01/world/coronavirus-covid-19.html?action=click\&pgtype=Article\&state=default\&region=MAIN_CONTENT_1\&context=storylines_live_updates\#link-780ec966}{Top
  U.S. officials work to break an impasse over the federal jobless
  benefit.}
\item
  \href{https://www.nytimes.com/2020/08/01/world/coronavirus-covid-19.html?action=click\&pgtype=Article\&state=default\&region=MAIN_CONTENT_1\&context=storylines_live_updates\#link-2bc8948}{Its
  outbreak untamed, Melbourne goes into even greater lockdown.}
\end{itemize}

\href{https://www.nytimes.com/2020/08/01/world/coronavirus-covid-19.html?action=click\&pgtype=Article\&state=default\&region=MAIN_CONTENT_1\&context=storylines_live_updates}{See
more updates}

More live coverage:
\href{https://www.nytimes.com/live/2020/07/31/business/stock-market-today-coronavirus?action=click\&pgtype=Article\&state=default\&region=MAIN_CONTENT_1\&context=storylines_live_updates}{Markets}

Government officials would not identify victims of the hackings. But the
primary target of the attacks appeared to be Oxford University in
Britain and the British-Swedish pharmaceutical company AstraZeneca,
which have been jointly working on a vaccine, said Robert Hannigan, the
former head of G.C.H.Q., the British intelligence agency.

Oxford scientists said on Thursday that they had noticed a surprising
resemblance between their vaccine approach and the work that Russian
scientists had reported.

Though Russia could be seeking to steal the vaccine data to bolster its
own research, it could also be trying to avoid relying on Western
countries for any eventual coronavirus vaccine.

While AstraZeneca has announced it will make the Oxford vaccine
available at cost, governments and philanthropies have paid huge sums to
the company to secure their place in line, even without any guarantee it
will work. The United States has said it will pay up to
\href{https://www.nytimes.com/2020/05/21/health/coronavirus-vaccine-astrazeneca.html}{\$1.2
billion to AstraZeneca} to fund a clinical trial and secure 300 million
doses. Russia could find itself near the back of the line if the vaccine
proves successful.

``Russia clearly doesn't want to disrupt vaccine production, but they
don't want to be dependent on the U.S. or the U.K. for production and
discovery of the vaccine,'' said Mr. Hannigan, now an executive at the
BlueVoyant cybersecurity firm. ``It not impossible to think Kremlin
pride is such that they don't want that to happen.''

An intense international race is underway to develop a vaccine for the
coronavirus that has already
\href{https://www.nytimes.com/interactive/2020/world/coronavirus-maps.html}{killed
580,000 people} and upended daily life around the world. More than
\href{https://www.nytimes.com/interactive/2020/science/coronavirus-vaccine-tracker.html}{155
vaccines} are under development, including 23 being tested on humans.

Some vaccines work by altering another common virus to mimic the
coronavirus to prompt an immune response without making people sick. The
research by Oxford and AstraZeneca is based on one such pathogen, a
chimpanzee adenovirus. Russia's Ministry of Health is trying to use two
other adenoviruses but is not as far along in its testing as the Oxford
researchers are.

Some officials suggested the Russian attacks have not been hugely
successful but were widespread enough to warrant a coordinated
international warning.

Across the globe, intelligence services have stepped up their focus on
information surrounding the virus. The F.B.I. director, Christopher A.
Wray, accused China last week of
\href{https://www.fbi.gov/news/speeches/the-threat-posed-by-the-chinese-government-and-the-chinese-communist-party-to-the-economic-and-national-security-of-the-united-states}{``working
to compromise American health care organizations''} conducting Covid-19
research.

``Russia is not alone,'' said John Hultquist, the senior director of
intelligence analysis at FireEye, a Silicon Valley cybersecurity firm.
``A lot of people are in this game even if they haven't been called out
yet. The whole pandemic is absolutely riddled with spies.''

Chinese government hackers have long focused on stealing intellectual
property and technology. Russia has aimed much of its recent
cyberespionage, like election interference, at weakening geopolitical
rivals and strengthening its hand.

``China is more well known for theft through hacking than Russia, which
is of course better now for using hacks for disruption and chaos,'' said
\href{https://securingdemocracy.gmfus.org/author/laura-rosenberger/}{Laura
Rosenberger}, a former Obama administration official who now leads the
\href{https://securingdemocracy.gmfus.org/}{Alliance for Securing
Democracy.} ``But there's no question that whoever gets to a vaccine
first thinks they will have geopolitical advantage, and that's something
I'd expect Russia to want.''

Still, a Russian intrusion could inadvertently damage some vaccine data
and additional security protocols to protect from future cyberattacks
could impose a burden on researchers. Private firms are more at risk
than the public, said Mike Chapple, a former National Security Agency
computer scientist who teaches cybersecurity at the University of Notre
Dame.

``The potential harm here is limited to commercial harm, to companies
that are devoting a lot of their own resources into developing a vaccine
in hopes it will be financially rewarding down the road,'' he said.

The Kremlin mocked the announcements Thursday, and Russian officials
said they did not know who could have hacked the companies or research
centers in Britain. One Russian official said the accusation was an
attempt to discredit Moscow's own work on a vaccine.

\href{https://www.nytimes.com/news-event/coronavirus?action=click\&pgtype=Article\&state=default\&region=MAIN_CONTENT_3\&context=storylines_faq}{}

\hypertarget{the-coronavirus-outbreak-}{%
\subsubsection{The Coronavirus Outbreak
›}\label{the-coronavirus-outbreak-}}

\hypertarget{frequently-asked-questions}{%
\paragraph{Frequently Asked
Questions}\label{frequently-asked-questions}}

Updated July 27, 2020

\begin{itemize}
\item ~
  \hypertarget{should-i-refinance-my-mortgage}{%
  \paragraph{Should I refinance my
  mortgage?}\label{should-i-refinance-my-mortgage}}

  \begin{itemize}
  \tightlist
  \item
    \href{https://www.nytimes.com/article/coronavirus-money-unemployment.html?action=click\&pgtype=Article\&state=default\&region=MAIN_CONTENT_3\&context=storylines_faq}{It
    could be a good idea,} because mortgage rates have
    \href{https://www.nytimes.com/2020/07/16/business/mortgage-rates-below-3-percent.html?action=click\&pgtype=Article\&state=default\&region=MAIN_CONTENT_3\&context=storylines_faq}{never
    been lower.} Refinancing requests have pushed mortgage applications
    to some of the highest levels since 2008, so be prepared to get in
    line. But defaults are also up, so if you're thinking about buying a
    home, be aware that some lenders have tightened their standards.
  \end{itemize}
\item ~
  \hypertarget{what-is-school-going-to-look-like-in-september}{%
  \paragraph{What is school going to look like in
  September?}\label{what-is-school-going-to-look-like-in-september}}

  \begin{itemize}
  \tightlist
  \item
    It is unlikely that many schools will return to a normal schedule
    this fall, requiring the grind of
    \href{https://www.nytimes.com/2020/06/05/us/coronavirus-education-lost-learning.html?action=click\&pgtype=Article\&state=default\&region=MAIN_CONTENT_3\&context=storylines_faq}{online
    learning},
    \href{https://www.nytimes.com/2020/05/29/us/coronavirus-child-care-centers.html?action=click\&pgtype=Article\&state=default\&region=MAIN_CONTENT_3\&context=storylines_faq}{makeshift
    child care} and
    \href{https://www.nytimes.com/2020/06/03/business/economy/coronavirus-working-women.html?action=click\&pgtype=Article\&state=default\&region=MAIN_CONTENT_3\&context=storylines_faq}{stunted
    workdays} to continue. California's two largest public school
    districts --- Los Angeles and San Diego --- said on July 13, that
    \href{https://www.nytimes.com/2020/07/13/us/lausd-san-diego-school-reopening.html?action=click\&pgtype=Article\&state=default\&region=MAIN_CONTENT_3\&context=storylines_faq}{instruction
    will be remote-only in the fall}, citing concerns that surging
    coronavirus infections in their areas pose too dire a risk for
    students and teachers. Together, the two districts enroll some
    825,000 students. They are the largest in the country so far to
    abandon plans for even a partial physical return to classrooms when
    they reopen in August. For other districts, the solution won't be an
    all-or-nothing approach.
    \href{https://bioethics.jhu.edu/research-and-outreach/projects/eschool-initiative/school-policy-tracker/}{Many
    systems}, including the nation's largest, New York City, are
    devising
    \href{https://www.nytimes.com/2020/06/26/us/coronavirus-schools-reopen-fall.html?action=click\&pgtype=Article\&state=default\&region=MAIN_CONTENT_3\&context=storylines_faq}{hybrid
    plans} that involve spending some days in classrooms and other days
    online. There's no national policy on this yet, so check with your
    municipal school system regularly to see what is happening in your
    community.
  \end{itemize}
\item ~
  \hypertarget{is-the-coronavirus-airborne}{%
  \paragraph{Is the coronavirus
  airborne?}\label{is-the-coronavirus-airborne}}

  \begin{itemize}
  \tightlist
  \item
    The coronavirus
    \href{https://www.nytimes.com/2020/07/04/health/239-experts-with-one-big-claim-the-coronavirus-is-airborne.html?action=click\&pgtype=Article\&state=default\&region=MAIN_CONTENT_3\&context=storylines_faq}{can
    stay aloft for hours in tiny droplets in stagnant air}, infecting
    people as they inhale, mounting scientific evidence suggests. This
    risk is highest in crowded indoor spaces with poor ventilation, and
    may help explain super-spreading events reported in meatpacking
    plants, churches and restaurants.
    \href{https://www.nytimes.com/2020/07/06/health/coronavirus-airborne-aerosols.html?action=click\&pgtype=Article\&state=default\&region=MAIN_CONTENT_3\&context=storylines_faq}{It's
    unclear how often the virus is spread} via these tiny droplets, or
    aerosols, compared with larger droplets that are expelled when a
    sick person coughs or sneezes, or transmitted through contact with
    contaminated surfaces, said Linsey Marr, an aerosol expert at
    Virginia Tech. Aerosols are released even when a person without
    symptoms exhales, talks or sings, according to Dr. Marr and more
    than 200 other experts, who
    \href{https://academic.oup.com/cid/article/doi/10.1093/cid/ciaa939/5867798}{have
    outlined the evidence in an open letter to the World Health
    Organization}.
  \end{itemize}
\item ~
  \hypertarget{what-are-the-symptoms-of-coronavirus}{%
  \paragraph{What are the symptoms of
  coronavirus?}\label{what-are-the-symptoms-of-coronavirus}}

  \begin{itemize}
  \tightlist
  \item
    Common symptoms
    \href{https://www.nytimes.com/article/symptoms-coronavirus.html?action=click\&pgtype=Article\&state=default\&region=MAIN_CONTENT_3\&context=storylines_faq}{include
    fever, a dry cough, fatigue and difficulty breathing or shortness of
    breath.} Some of these symptoms overlap with those of the flu,
    making detection difficult, but runny noses and stuffy sinuses are
    less common.
    \href{https://www.nytimes.com/2020/04/27/health/coronavirus-symptoms-cdc.html?action=click\&pgtype=Article\&state=default\&region=MAIN_CONTENT_3\&context=storylines_faq}{The
    C.D.C. has also} added chills, muscle pain, sore throat, headache
    and a new loss of the sense of taste or smell as symptoms to look
    out for. Most people fall ill five to seven days after exposure, but
    symptoms may appear in as few as two days or as many as 14 days.
  \end{itemize}
\item ~
  \hypertarget{does-asymptomatic-transmission-of-covid-19-happen}{%
  \paragraph{Does asymptomatic transmission of Covid-19
  happen?}\label{does-asymptomatic-transmission-of-covid-19-happen}}

  \begin{itemize}
  \tightlist
  \item
    So far, the evidence seems to show it does. A widely cited
    \href{https://www.nature.com/articles/s41591-020-0869-5}{paper}
    published in April suggests that people are most infectious about
    two days before the onset of coronavirus symptoms and estimated that
    44 percent of new infections were a result of transmission from
    people who were not yet showing symptoms. Recently, a top expert at
    the World Health Organization stated that transmission of the
    coronavirus by people who did not have symptoms was ``very rare,''
    \href{https://www.nytimes.com/2020/06/09/world/coronavirus-updates.html?action=click\&pgtype=Article\&state=default\&region=MAIN_CONTENT_3\&context=storylines_faq\#link-1f302e21}{but
    she later walked back that statement.}
  \end{itemize}
\end{itemize}

Dmitri S. Peskov, the spokesman for President Vladimir V. Putin of
Russia, told reporters that the accusations were unacceptable. ``Russia
has nothing to do with these attempts,'' he said.

Cozy Bear is one of the highest-profile, and most successful, hacking
groups associated with the Russian government. It was implicated
alongside the group Fancy Bear in the
\href{https://www.nytimes.com/2018/07/13/us/politics/mueller-indictment-russian-intelligence-hacking.html}{2016
hacking of the Democratic National Committee}. Though Cozy Bear is
believed
\href{https://www.nytimes.com/2019/01/18/technology/dnc-russian-hacking.html}{to
have breached the committee's computers}, it played no known role in
releasing stolen Democratic emails.

Cozy Bear ``has a long history of targeting governmental, diplomatic,
think tank, health care and energy organizations for intelligence gain,
so we encourage everyone to take this threat seriously,'' said Anne
Neuberger, the National Security Agency's cybersecurity director.

The malware used by Cozy Bear to steal the vaccine research included
code known as ``WellMess'' and ``WellMail.'' The Russian group has not
previously used that malware, according to British officials.

But American experts say the tactics used in trying to obtain access to
the vaccine data bear all the hallmarks of Russian intelligence
officials. And American officials said they were confident in
attributing the attacks to the Russian hacking group.

The American, British and Canadian governments said Cozy Bear used
\href{https://media.defense.gov/2020/Jul/16/2002457639/-1/-1/0/NCSC_APT29_ADVISORY-QUAD-OFFICIAL-20200709-1810.PDF}{recently
publicized weak spots in computer networks} to get a foothold. If
organizations do not immediately patch a vulnerability that a software
company has identified, their networks can be exposed to hacks.

Once Cozy Bear hackers exploit those gaps to gain entry to a computer
system, they create legitimate credentials to maintain access even after
the hole is patched.

While the various Russian hacking groups often share similar targets,
they are run by different intelligence agencies for different purposes.

Hackers with Cozy Bear are after information but do not generally
release it publicly, according to government and outside experts. Fancy
Bear, which works for Russian military intelligence and is also known as
APT28, will often publicize the information it steals.

Cozy Bear's ties are to the S.V.R., the Russian equivalent of the
C.I.A., according to current and former officials. Unlike other Russian
hackers, Cozy Bears operations are sophisticated, stealthy and hard to
detect.

``Their job is quiet, old-fashioned intelligence collection,'' said Mr.
Hultquist, the cybersecurity analyst.

Reporting was contributed by Nicole Perlroth from San Francisco, David
D. Kirkpatrick and Stephen Castle from London, Andrew Higgins from
Moscow, and Charlie Savage from Washington.

Advertisement

\protect\hyperlink{after-bottom}{Continue reading the main story}

\hypertarget{site-index}{%
\subsection{Site Index}\label{site-index}}

\hypertarget{site-information-navigation}{%
\subsection{Site Information
Navigation}\label{site-information-navigation}}

\begin{itemize}
\tightlist
\item
  \href{https://help.nytimes.com/hc/en-us/articles/115014792127-Copyright-notice}{©~2020~The
  New York Times Company}
\end{itemize}

\begin{itemize}
\tightlist
\item
  \href{https://www.nytco.com/}{NYTCo}
\item
  \href{https://help.nytimes.com/hc/en-us/articles/115015385887-Contact-Us}{Contact
  Us}
\item
  \href{https://www.nytco.com/careers/}{Work with us}
\item
  \href{https://nytmediakit.com/}{Advertise}
\item
  \href{http://www.tbrandstudio.com/}{T Brand Studio}
\item
  \href{https://www.nytimes.com/privacy/cookie-policy\#how-do-i-manage-trackers}{Your
  Ad Choices}
\item
  \href{https://www.nytimes.com/privacy}{Privacy}
\item
  \href{https://help.nytimes.com/hc/en-us/articles/115014893428-Terms-of-service}{Terms
  of Service}
\item
  \href{https://help.nytimes.com/hc/en-us/articles/115014893968-Terms-of-sale}{Terms
  of Sale}
\item
  \href{https://spiderbites.nytimes.com}{Site Map}
\item
  \href{https://help.nytimes.com/hc/en-us}{Help}
\item
  \href{https://www.nytimes.com/subscription?campaignId=37WXW}{Subscriptions}
\end{itemize}
