Colin Powell Still Wants Answers

\href{https://nyti.ms/3h8iBpJ}{https://nyti.ms/3h8iBpJ}

\begin{itemize}
\item
\item
\item
\item
\item
\item
\end{itemize}

\includegraphics{https://static01.nyt.com/images/2020/07/19/magazine/19mag-iraq-3/19mag-iraq-3-articleLarge-v2.jpg?quality=75\&auto=webp\&disable=upscale}

Sections

\protect\hyperlink{site-content}{Skip to
content}\protect\hyperlink{site-index}{Skip to site index}

The Great ReadFeature

\hypertarget{colin-powell-still-wants-answers}{%
\section{Colin Powell Still Wants
Answers}\label{colin-powell-still-wants-answers}}

In 2003, he made the case for invading Iraq to halt its weapons
programs. The analysts who provided the intelligence now say it was
doubted inside the C.I.A. at the time.

Colin Powell in Virginia this month. ``I knew I didn't have any
choice,'' he said. ``He's the president.''Credit...Gabriella Demczuk for
The New York Times

Supported by

\protect\hyperlink{after-sponsor}{Continue reading the main story}

By \href{https://www.nytimes.com/by/robert-draper}{Robert Draper}

\begin{itemize}
\item
  Published July 16, 2020Updated July 17, 2020
\item
  \begin{itemize}
  \item
  \item
  \item
  \item
  \item
  \item
  \end{itemize}
\end{itemize}

Early one morning in August 2002, Jack Straw, the British foreign
minister at the time, drove with a small entourage to a beach house in
East Hampton on Long Island. The house belonged to the billionaire
Ronald Lauder, who for most of August was hosting his good friend and
Straw's American counterpart, Colin Powell.

The foreign minister and the secretary of state had become
extraordinarily close over the previous year. Powell's customary 11 p.m.
calls to the Straw household had prompted Straw's wife to refer to him
as ``the other man in my life.'' The August meeting at the Lauder
residence, Powell would later say, was an attempt to answer a question:
``Could we both stop a war?''

For nearly a year --- since just a few days after the Sept. 11 attacks
--- Powell had watched as the idea of invading Iraq, once the
preoccupation of a handful of die-hards in other corners of the Bush
administration, took on increasingly undeniable momentum. Powell thought
such an invasion would be disastrous --- and yet the prospect had for
months seemed so preposterous to Powell and his deputies at the State
Department that he assumed it would burn out of its own accord.

But by that August, it had become evident to Powell that he was not
winning the argument. On Monday, Aug. 5, a couple of weeks before the
meeting in East Hampton, he and Condoleezza Rice, President George W.
Bush's national security adviser, joined Bush for dinner at the White
House residence. For two hours, Rice said little while Powell proceeded
to do what no one else in the Bush administration had done or would do:
tell the president to his face that things in Iraq could go horribly
wrong. ``If you break it, you own it,'' he
\href{https://www.nytimes.com/2004/04/17/world/the-struggle-for-iraq-policy-wary-powell-said-to-have-warned-bush-on-war.html}{famously
told Bush}. ``This will become your first term.''

As they sat on the veranda of the beach house, Powell recounted the
dinner meeting to Straw. ``I told him, `Removing Saddam is the easy
part,''' he said. `` `You'll be the proud owner of 25 million Iraqis in
18 fractious provinces.''' They talked for three hours. Powell spoke
ruefully of Vice President Dick Cheney and Defense Secretary Donald
Rumsfeld --- men he had known for years, both of whom had changed, he
told Straw, and not for the better.

Straw listened sympathetically. He shared Powell's views on the folly of
invasion. His own boss, Prime Minister Tony Blair, professed a
commitment to regime change in Iraq, but one that was orderly and
supported by other countries in the West as well as in the Arab world.
Such a coalition, achieved through the passage of a United Nations
resolution, might persuade Saddam Hussein to comply with weapons
inspectors and avoid military confrontation. But Blair's attempts to
deliver this message to Bush were not getting through, in part because
the prime minister was not terribly forceful in delivering it. Straw was
plainly frustrated with Blair, who he feared was becoming Bush's
enabler. Powell pressed him to keep trying. ``You've got to get Tony to
convince the president to go to the U.N.,'' he said.

The day after he returned to London, Straw warned Blair that he should
not dismiss the prospect of Bush's unilaterally taking his country to
war. ``You have to take this seriously,'' the foreign minister said,
``because there are contrary voices. Cheney and Rumsfeld are in a
different place. We haven't landed this yet.'' Powell was Blair's ally
in this cause, but Straw could see that the secretary of state was only
a single voice in Bush's ear, and not necessarily the one that counted.

As it turned out, the secretary's voice was the most prescient in the
Bush administration. And yet Powell's ``you break it, you own it''
warning to the president would be overshadowed by the fact that he was
also the war's most effective salesman. The sale had been made in a
\href{https://www.nytimes.com/2003/02/05/international/middleeast/with-audio-tapes-and-images-powell-makes-case-to-un.html}{speech
before the United Nations} on Feb. 5, 2003: a methodical recitation of
the American intelligence agencies' findings on Iraq's weapons program
demonstrating the urgency of putting an end to it, by invasion if
necessary. It was precisely the secretary of state's skepticism about
the wisdom of war that made him the Bush White House's indispensable
pitchman for it. Alone among the president's war council, the four-star
general was seen by Republicans and Democrats, the news media and the
public as a figure of unassailable credibility. If Powell said Hussein
presented an immediate danger to the United States, then surely it was
so.

The speech remains one of the most indelible public moments of the Bush
presidency. By the time Powell resigned from his post, his performance
that morning before the U.N. Security Council had come to symbolize the
tragic recklessness of Bush's decision to go to war. Iraq, it was by
then widely understood, had played no role in the Sept. 11 attacks, nor
did it possess weapons of mass destruction. Nearly all the intelligence
Powell presented to the world in his speech turned out to be false.

\includegraphics{https://static01.nyt.com/images/2020/07/19/magazine/19mag-Iraq-1/19mag-Iraq-1-articleLarge.jpg?quality=75\&auto=webp\&disable=upscale}

\textbf{With the benefit} of 15 years of hindsight, it's possible to see
Powell's U.N. speech as a signal event in the broader story of American
governance. It is Exhibit A for the argument that would help propel
Donald Trump to the White House in 2016 --- that the U.S. government was
not on the level, that the ``establishment'' figures of both parties
were at once fools and manipulators. In June, when Powell told CNN that
\href{https://www.nytimes.com/2020/06/06/us/politics/trump-biden-republicans-voters.html}{he
would be voting for Joe Biden} in November, Trump shot back on Twitter:
``Didn't Powell say that Iraq had `weapons of mass destruction?' They
didn't, but off we went to WAR!''

Because of its long shadow, the U.N. speech invites one of the Bush
presidency's most poignant what-ifs. What if that same voice that
publicly proclaimed the necessity of invading Iraq had instead told Bush
privately that it was not merely an invitation to unintended
consequences but a mistake, as he personally believed it to be? What if
he had said no to Bush when he asked him to speak before the U.N.?
Powell would almost certainly have been obligated to resign, and many if
not all of his top staff members involved in the Iraq issue would also
have quit; several had already considered doing so the previous summer.

If the State Department's top team had emptied out their desks, what
would Powell's close friend Straw have done? ``If Powell had decided to
resign in advance of the Iraq war,'' Straw told me, ``I would almost
certainly have done so, too.'' Blair's support in the Labour Party would
have cratered --- and had Blair withdrawn his support for war under
pressure from Parliament or simply failed to win an authorization vote,
the narrative of collapsed momentum would have dominated the news
coverage for weeks. Doubters in the upper ranks of the American military
--- there were several --- would have been empowered to speak out;
intelligence would have been re-examined; Democrats, now liberated from
the political pressures of the midterm elections, would most likely have
joined the chorus.

This domino effect required a first move by Bush's secretary of state.
``But I knew I didn't have any choice,'' Powell told me. ``What choice
did I have? He's the president.''

``I'm sort of not the resigning type,'' Straw said. ``Nor is Powell. And
that's the problem.''

In August 2018, in the course of researching a book on the lead-up to
the Iraq war, I went to see Powell at the office in Alexandria, Va.,
that he has maintained since leaving the Bush administration in early
2005. Powell, who is now 83, is as proud and blunt-speaking as he was
during his career in public service. Over the course of our two hourlong
conversations, he made clear that he was all too aware of the lonely
turf he was destined to occupy in history.

It was not the turf that anyone, least of all Powell himself, would have
imagined for him in 2001. He entered the Bush administration as a
four-star general of immense popularity and political influence. He left
it four years later, discarded by Bush in favor of a more like-minded
chief diplomat, Condoleezza Rice. He mournfully predicted to others that
his obituary's first paragraph would include his authorship of the U.N.
speech.

In the decade and a half since then, Powell's world and Bush's have
intersected only at the margins. The secretary takes pains not to speak
ill of the president he once served, even when he announced in 2008 that
he would be supporting Barack Obama as Bush's successor. He was on hand
for the opening of Bush's presidential library in 2013. But he has not
attended the administration's annual alumni gatherings, and since
leaving office he has refused to defend Bush's legacy-defining decision
to invade Iraq.

On the one other occasion I interviewed Powell, while gathering material
for a book about Bush's presidency in 2006, he was wary and did not wish
to speak on the record; it was a time of chaos in Iraq, and of
score-settling memoirs in Washington. A dozen years later, however, that
caginess had mostly fallen away. Some of the core mysteries that still
hung over the most consequential American foreign-policy decision in a
half-century, I found, remained mysteries even to Powell. At one point
during our first conversation in 2018, he paraphrased a line about
Iraq's supposed weapons of mass destruction from the intelligence
assessment that had informed his U.N. speech, which intelligence
officials had assured him was rock solid: `` `We judge that they have
100 to 500 metric tons of chemical weapons, all produced within the last
year.' How could they have known that?'' he said with caustic disbelief.

I told Powell I intended to track down the authors of that assessment.
Smirking, he replied, ``You might tell them I'm curious about it.''

Not long after meeting Powell, I did manage to speak to several analysts
who helped produce the classified assessment of Iraq's supposed weapons
program and who had not previously talked with reporters. In fact, I
learned, there was exactly zero proof that Hussein had a
chemical-weapons stockpile. The C.I.A. analysts knew only that he
\emph{once} had such a stockpile, before the 1991 Persian Gulf war, and
that it was thought to be as much as 500 metric tons before the weapons
were destroyed. The analysts had noted what seemed to be recent
suspicious movement of vehicles around suspected chemical-weapons
plants. There also seemed to be signs --- though again, no hard proof
--- that Iraq had an active biological-weapons program, so, they
reasoned, the country was probably manufacturing chemical weapons as
well. This was, I learned, typical of the prewar intelligence estimates:
They amounted to semi-educated guesses built on previous and
seldom-challenged guesses that always assumed the worst and imagined
deceptiveness in everything the Iraqi regime did. The analysts knew not
to present these judgments as facts. But that distinction had become
lost by the time Powell spoke before the U.N.

Moreover, a circular reasoning guided the intelligence community's
prewar estimates. As an intelligence official --- one of many who spoke
to me on the condition of anonymity --- said: ``We knew where we were
headed, and that was war. Which ironically made it that much more
difficult to change the analytic line that we'd stuck with for 10 years.
For 10 years, it was our pretty strong judgment that Saddam had chemical
capability.'' Whether or not this was still true, ``with American
soldiers about to go in, we weren't going to change our mind and say,
`Never mind.'''

\textbf{``I am capable} of self-pity,'' Powell wrote in ``My American
Journey,'' his 1995 memoir. ``But not for long.'' In his ascent to
chairman of the Joint Chiefs of Staff under President George H.W. Bush,
the Harlem-born son of Jamaican immigrants had prevailed over racism,
hard-ass generals in the Army and right-wingers who found him
insufficiently hawkish. His appointment by Bush and Cheney, then the
secretary of defense, also turned out to be a stroke of political
genius. During the gulf war, his poise and resolve on television rallied
Americans leery of foreign entanglements after the horror of Vietnam. It
was thoroughly unsurprising when Bush's son appointed Powell his
secretary of state.

But their relationship was fraught from the start. Bush was not at all
like his father, whom Powell had greatly admired. The new president was
far more conservative, far less reverential of international alliances.
Bush also understood the power that Powell's popularity conferred on
him, and he knew that Powell, who had once considered and decided
against running for president, could change his mind anytime he wished.

And when it came to policy in the Middle East, Powell was not where the
rest of Bush's team was. He was, as a top National Security Council
staff member who respected Powell would recall, ``more of a dissident,
who,'' as the administration drifted steadily toward confrontation with
Hussein, ``would say, `I'm fighting a rear-guard action against these
{[}expletive{]} crazies.'''

Recalling the chaotic days after the Sept. 11 attacks, Powell told me,
``The American people wanted somebody killed.'' Bush himself confessed
to a gathering of religious leaders in the Oval Office on the afternoon
of Sept. 20, ``I'm having difficulty controlling my blood lust.'' For
Powell, it was plain at the time that the ``somebody'' deserving to be
killed was Osama bin Laden, along with his network and the Taliban
government in Afghanistan that had given him haven. When Bush and the
rest of his senior foreign-policy team gathered at Camp David four days
after the attacks, Powell argued that the world would support such a
mission --- but that a global coalition would fall apart if the U.S.
began attacking other countries.

Rumsfeld archly replied: ``Then maybe it's not a coalition worth
having.'' Rumsfeld argued that a ``global war on terror'' should in fact
be global. This was not an academic argument. A number of voices inside
the administration had for years before the Sept. 11 attacks viewed
Hussein as a principal sponsor of radical Palestinian groups and now
maintained that any counterterrorism effort worth its salt necessarily
encompassed Iraq. These figures were concentrated in Rumsfeld's Pentagon
and in Cheney's office. They included Rumsfeld's deputy, Paul Wolfowitz;
the under secretary of defense for policy, Douglas Feith; Scooter Libby,
Cheney's chief of staff; and Cheney himself.

At Camp David, Wolfowitz went so far as to argue that Hussein was most
likely behind the Sept. 11 attacks. Iraq was ``the head of the snake,''
he contended, and should be America's primary target. Powell thought the
deputy secretary of defense's logic was absurd. But, he noted, Bush did
not dismiss it outright, saying instead, ``OK, we'll leave Iraq for
later.''

Bush was true to his word. On Oct. 7, the president announced the
beginning of Operation Enduring Freedom, a military attack on Al Qaeda
and the Taliban. His administration's policy focused on Afghanistan
throughout the final months of 2001. But while spending Thanksgiving
with Army troops at Fort Campbell in Kentucky, the president proclaimed,
``Afghanistan is just the beginning of the war on terror.''

A month later, Bush was briefed by Gen. Tommy Franks of U.S. Central
Command on a possible plan for invading Iraq. And a month after that, on
Jan. 29, 2002, the president delivered his
\href{https://www.nytimes.com/2002/01/30/us/state-union-president-bush-s-state-union-address-congress-nation.html}{State
of the Union address} branding Iraq, Iran and North Korea the Axis of
Evil. ``Iraq,'' he told Congress, ``continues to flaunt its hostility
towards America and to support terror.''

Throughout early 2002, the Iraq debate played out largely in the
National Security Council cabinet-level meetings known as the Principals
Committee. Powell advocated the approach championed by Blair and Straw:
have Bush go to the U.N. and press for a resolution condemning Hussein.
Rumsfeld was adamant that the United States should not be slowed down by
coalition-building. The interagency gatherings often descended into
face-to-face bickering between the two sides, quarrels that spilled over
into bureaucratic back alleys. Skilled infighter though he was, Powell
was plainly frustrated by what one Principals Committee attendee
described as ``Don's style, this Socratic asking of questions rather
than tell you where he stood.''

Rumsfeld was not Powell's only rival in the room. Cheney had a history
with both men. He owed his career to Rumsfeld, whose coattails had
carried him from the Office of Economic Opportunity to the Ford White
House in 1974. And as the elder Bush's defense secretary, Cheney watched
attentively as his Joint Chiefs chairman hoovered up publicity. That had
been useful during the gulf war, up to a point. But Powell had also
offered unsolicited policymaking advice to the White House and
off-the-cuff troop-downsizing estimates to the press. Cheney --- a
figure of legendary discretion whose Secret Service code name at one
time was Back Seat --- had come to believe that Colin Powell was playing
for Colin Powell.

In the Principals Committee meetings, men who had known one another for
decades could no longer disguise their ill feelings. At the beginning of
one meeting, Richard Armitage, Powell's deputy secretary, genially
offered the vice president some coffee. Cheney smiled. ``Rich,''
Armitage recalled him replying, ``if you gave it to me, I'd have to have
a taster.''

As one of Powell's subordinates put it: ``The secretary and Armitage
thought we could get by with a rope-a-dope approach: Let's play along.
Let them hang themselves. Because this idea is so cockamamie, it'll
never happen.'' Of Hussein, ``Powell kept saying, `He's a bad guy in a
box, so let's keep building the box,''' another one of his deputies
recalled. ``And he hoped that over time, the president might say: `Ah,
OK, I get it. The box is good.'''

But by the summer of 2002, this argument was clearly losing ground. One
morning that summer, Powell's under secretary of state for political
affairs, Marc Grossman, called Libby's deputy, Eric Edelman. The two had
traveled in the same foreign-policy circles for decades, but their
collegiality had begun to fray over Iraq. So Edelman was surprised when
Grossman said, ``I'd like to meet with you on some kind of neutral
territory.'' They chose the coffee shop in the basement of the Corcoran
Gallery.

Once they were seated, Grossman got right to the point. ``Eric,''
Edelman recalled him asking, ``has the president already decided to go
to war, and we're just in this interagency circle jerk?''

``I don't think the president has decided to go to war,'' Edelman
replied. ``But I do think the president has decided the problem Saddam
presents can't just drag on forever.''

Just hours before Powell joined Bush for dinner on Aug. 5, General
Franks briefed Bush on what would become the final war plan for invading
Iraq. Still, Powell could see that his grim prophecy to the president
--- ``this will become your first term'' --- registered. ``What should I
do?'' Bush asked.

Go to the United Nations, Powell advised him. After all, Hussein had
violated numerous U.N. resolutions regarding his weapons program,
aggression toward Kuwait and treatment of his own people. The U.N. was
the aggrieved party. But if he were to do so, Powell added, there was a
chance that Hussein would surrender his weapons. Bush would have to
accept a changed regime as a substitute for regime change.

It was arguably the most important message that Bush would hear from any
of his subordinates in his entire presidency --- and, in what Powell
left out of the message, the most important missed opportunity. When
Bush asked, ``What should I do?'' his secretary of state could have
spoken his mind and said, ``Don't invade Iraq.'' But he didn't.

\textbf{Perhaps the most} tireless lobbyist for invasion in 2002 was a
smooth-talking Iraqi expatriate named
\href{https://www.nytimes.com/2015/11/04/world/middleeast/ahmad-chalabi-iraq-dead.html}{Ahmad
Chalabi}. The leader of the Iraqi National Congress, an aspiring
government in exile, Chalabi had for years been feeding sympathetic
policymakers and journalists a utopian vision of what a post-Hussein
democratic Iraq might look like. On the veranda in East Hampton, Powell
complained to Straw that Wolfowitz, Feith, Cheney and Libby were
hopelessly smitten with Chalabi. ``You wouldn't believe how much this
guy is shaping our policy,'' he told Straw.

Chalabi had also been vigorously disseminating intelligence seeking to
tie Hussein to Al Qaeda. Cheney, Libby, Wolfowitz and Feith found his
evidence on this subject to be persuasive. By contrast, Powell's team
found it highly unlikely that Hussein would consort with Islamic
terrorists who despised the secular Iraqi regime.

George Tenet, the director of the C.I.A., broadly agreed with Powell on
the administration hawks' intelligence --- so it was at first glance
mystifying that the U.S. intelligence community, by the summer of 2002,
was providing the most convincing arguments for going to war. Tenet had
by then come to believe that Bush's mind was made up about overthrowing
Hussein, even as the president continued to maintain otherwise. Some who
worked with Tenet --- a Clinton holdover whose agency's work was
repeatedly criticized by Rumsfeld and others --- thought he fretted that
the White House would come to see him as unhelpful and proceed to
disregard the C.I.A.'s assessments altogether. ``Here we had this
precious access,'' recalled one of Tenet's senior analysts, ``and he
didn't want to blow it.''

Sometime in May 2002, Bush received a Presidential Daily Briefing from
the C.I.A. that included perhaps the most alarming intelligence about
Iraq that he had yet heard. National Security Agency intercepts had
picked up communications between an Iraqi general and an Iraqi
procurement agent who was based in Australia. The general had directed
the procurement agent to buy equipment for Iraq's unmanned aerial
vehicles program. In the spring of 2002, the agent had given an
Australian equipment distributor his shopping list. Among the items was
Garmin GPS software that included maps of major American cities.

Alarmed, the distributor contacted the authorities. This P.D.B.
presented Bush with the first intelligence appearing to confirm his
nightmare scenario: Hussein intended to attack the United States. This
marked a turning point for Bush, according to one of his senior
advisers. ``We get this report about, They've bought this software
that's supposed to be mapping the United States. He's hearing this
intel, and the diplomacy is going nowhere. And so I think that's when he
really starts thinking, I've got to get something done in Iraq.''

As it happened, there was a more innocent explanation for the mapping
software. Two C.I.A. analysts and an Australian intelligence officer
eventually brought the Iraqi procurement agent in for questioning and
confronted him about the American maps. The Iraqi was stunned. He said
it was the Garmin hardware he had been interested in. The only reason he
bought the mapping software, he said, was because he thought the
hardware wouldn't work without it. The presentation on the vendor's web
page seemed to confirm this account.

But this revelation, like others tempering the most dire view of Iraq's
capabilities, was swept aside by the self-compounding momentum toward
war. In a
\href{https://www.nytimes.com/2002/10/08/us/threats-responses-president-s-speech-bush-sees-urgent-duty-pre-empt-attack-iraq.html}{speech
in Cincinnati} in October 2002, Bush likened America's confrontation
with Hussein to World War II --- an indicator that the president could
not foresee a diplomatic outcome.

In early December, word reached the C.I.A. that the White House wanted
it to prepare an oral presentation on Iraq's weapons program that would
feature an ``Adlai Stevenson moment'' --- referring to the 1962 episode
in which the U.S. ambassador to the U.N. presented open-and-shut
photographic evidence of Soviet ballistic-missile installations in Cuba.
The timing of the request seemed odd, given that Hans Blix, the U.N.'s
chief weapons inspector, and his team were already in Iraq and would
presumably be furnishing on-the-ground visual proof of Hussein's
arsenal, if it existed, any day now. The fact that such a presentation
was being ordered up was tantamount to a White House vote of no
confidence in Blix.

The presentation was referred to internally at the C.I.A. as the Case.
That Tenet did not resist the request suggested that the agency had
crossed a red line. ``The first thing they teach you in C.I.A. 101 is
you don't help them make the case,'' said an agency official who was
involved in the project. ``But we were all infected in the case for
war.''

Image

Credit...Photo illustration by Joan Wong

\textbf{The task of supervising} the intelligence on Iraq's weapons
program fell largely to Tenet's deputy director, John McLaughlin.
McLaughlin was a beloved figure among the agency's analysts. As measured
and even-tempered as Tenet was mercurial, he wore natty suspenders but
was otherwise a by-the-book professional who pored over classified
documents with a ruler, sliding it slowly downward line by line. He
enjoyed performing sleight-of-hand coin tricks, which earned him the
code name Merlin from the C.I.A. security detail.

McLaughlin met with the agency's analytical team headed by Bob Walpole,
the national intelligence officer for strategic programs. The deputy
director told the analysts that the White House had asked for their best
story on Iraq. The analysts sent up what visuals they had.

McLaughlin reviewed them with astonishment. ``This is all there is?'' he
asked when they convened again. He also asked them, ``Do we have any
slam-dunk evidence of W.M.D.?''

Larry Fox, a senior chemical-weapons analyst, did not watch basketball.
He asked McLaughlin what ``slam dunk'' meant.

``Like a smoking gun,'' the deputy director explained. ``Undeniable.
Caught red-handed.''

``Ah,'' Fox said. ``Well, no. We don't have any.''

For the next two weeks, several analysts fine-tuned the presentation. On
Friday afternoon, Dec. 20, McLaughlin stood in Rumsfeld's conference
room at the Pentagon before a group that included Wolfowitz, Feith and
Franks and recited the Case. Rumsfeld and his team were polite but
visibly unimpressed. They asked few questions.

The following morning, McLaughlin and his colleagues were sent to the
Oval Office for a repeat performance, accompanied by Tenet, for a
gathering that included Bush, Cheney, Rice and Libby. ``This is a rough
draft --- it's still in development,'' McLaughlin began. For the next 20
or so minutes, McLaughlin spoke almost entirely uninterrupted. It was a
smoother performance than his briefing the day before at the Pentagon.
Bush and the others listened intently. But a thick silence settled in
after he finished. ``Again, this is a first draft,'' Tenet assured the
president.

``Nice try,'' the president said to McLaughlin. He did not appear to
mean it sarcastically. Bush expressed his concern clearly, according to
notes taken by an attendee: ``Look, in about five weeks I may have to
ask the fathers and mothers of America to send their sons and daughters
off to war. This has to be well developed.'' He emphasized the need to
make the case to ``the average citizen. So it needs to be more
convincing. Probably needs some better examples.''

It was clear to everyone in the room that Bush had already made up his
mind about the Iraqi threat. The only question to him was whether the
C.I.A. had what it took to convince the public that the threat justified
war. ``Maybe have a lawyer look at how to lay out the structure of the
argument,'' Bush continued. ``Maybe someone with Madison Avenue
experience should look at the presentation.'' He added, ``And it needs
to tie all this into terrorism, for the domestic audience.''

The president asked Tenet whether his agency could build a more
convincing case. It was to that question --- not, as often reported, a
question relating to whether Hussein posed a threat --- that the C.I.A.
director infamously replied: ``Slam dunk.''

McLaughlin tried again. He instructed Bob Walpole to make the Case more
persuasive. ``Give me everything you've got,'' Walpole in turn told his
weapons team, according to one of the analysts. ``Never mind sourcing or
other problems.'' He wanted the kitchen sink.

On Dec. 28, Walpole and McLaughlin went to the White House to discuss
the Case with Rice. Just a couple of minutes into his summary, Rice
stopped him. ``Bob?'' she said with evident concern. ``If these are just
assertions, we need to know this now.''

``They're analytical assessments,'' Walpole replied. ``The agencies have
attached confidence levels to them.''

Rice studied her copy, frowning. ``What's `high confidence'?'' she
asked. ``About 90 percent?''

``About that,'' he said.

The national security adviser gaped at Walpole and McLaughlin. ``Well,''
she finally said, ``that's a heck of a lot lower than what the P.D.B.s
are saying!''

The chemical and biological weapons cases were based on inference,
Walpole conceded. The nuclear case, he said, was ``the weakest.'' Rice
turned to McLaughlin. ``You have gotten the president way out on a limb
on this,'' she said. Walpole --- who personally thought that invading
Iraq made absolutely no sense --- nonetheless could see that the
administration wouldn't be satisfied with a case that was built only on
deceptions and shady activity. He wrote to his analysts, ``We must make
a public case that `Iraq HAS WMDs.'''

Unknown to Walpole's team, a parallel process was underway in the Office
of the Vice President. Immediately after the Dec. 21 meeting in the Oval
Office, Cheney had said privately to Bush, ``You know, Scooter's already
been working on something we could use.'' Two days later, Libby called
Edelman, his deputy, and told him about McLaughlin's weak presentation.
``The president doesn't think it's nearly persuasive enough,'' Cheney's
chief of staff said. ``And so they've given O.V.P. the assignment of
redoing that.''

The next morning, Cheney's staff got to work on their alternative
presentation. John Hannah, Cheney's assistant for national security
affairs, was tasked with the section on biological, chemical and nuclear
weapons. Libby had instructed his Middle East specialist to put every
damning bit of raw intelligence he could find into his brief. The burden
would then be on the C.I.A. analysts to argue why any of it should be
thrown out.

On Saturday, Jan. 25, Libby gave a preview of the new presentation in
the Situation Room. The audience included Rice, Wolfowitz, Armitage and
Stephen Hadley, the deputy national security adviser. More notable, the
political side of the White House --- including Karl Rove, Bush's
longtime adviser, and Dan Bartlett, his communications director --- was
now hearing the intelligence case against Hussein for the very first
time.

Wolfowitz thought Cheney's chief of staff had done a great job. Rove
found much to admire about it as well. Because many in the group were
communications specialists, the conversation quickly moved on from the
intelligence to the matter of its delivery. ``I recall the general sense
was, Who would be the best person to make this case at the U.N.?'' Rove
told me. ``And the obvious answer was Colin Powell, chief diplomat.''

\textbf{``Are you with} me on this?'' Bush asked Powell. The two were
alone in the Oval Office on Jan. 13, 2003. ``I think I have to do this.
I want you with me.''

Powell had cautioned Bush a few months earlier about the consequences of
invading Iraq, and he had gone further in private conversations with
others, saying he thought the idea of going to war was foolish on its
face. But the secretary of state had never expressed this outright
opposition to the president. And although Powell would not admit it,
Bush's request that he be the one to make the case against Hussein to
the U.N. was enormously flattering.

Even Cheney had explicitly acknowledged that Powell was the right man
for the job. As the secretary told one of his top aides: ``The vice
president said to me: `You're the most popular man in America. Do
something with that popularity.''' But, Powell added to his aide, he
wasn't sure he could say no to Bush anyway. ``There's only so many times
I can go toe to toe with the V.P.,'' he said. ``The more I think about
it, the more I realize it's important to keep the job.''

Once the decision was made that Powell would deliver the U.N. speech, he
was handed the text that Libby's team had prepared. Powell viewed the
document suspiciously. Among the first things he noticed about Libby's
text were the lurid intimations about Hussein's supposed ties with bin
Laden's organization. ``You guys really believe all this
{[}expletive{]}?'' he scoffed to one of Cheney's deputies. After first
scrapping the entire section dealing with Iraq's alleged ties to Al
Qaeda, the secretary tasked Carl Ford, the director of the State
Department's Bureau of Intelligence and Research (I.N.R.), with
reviewing the speech's claims on biological, chemical and nuclear
weapons.

Ford's staff worked overnight. Their memo of objections to Hannah's
weapons section on Jan. 31 came to six single-spaced pages and cited at
least 38 items that were deemed either ``weak'' or ``unsubstantiated.''
The I.N.R. analysts warned that Iraq's alleged chemical-weapon
decontamination trucks could simply be water trucks. Libby's team had
claimed that a shipment of aluminum tubes that the C.I.A. had
intercepted on its way to Iraq in 2001 was intended for use in
uranium-enrichment centrifuges (a claim that was
\href{https://www.nytimes.com/2002/09/08/world/threats-responses-iraqis-us-says-hussein-intensifies-quest-for-bomb-parts.html}{leaked
to The New York Times}). The I.N.R. analysts maintained that the tubes
were for rocket launchers. Three of the critique's most common phrases
were ``plausibility open to question,'' ``highly questionable'' and
``draft states it as fact.''

Meanwhile, Powell's chief of staff, Col. Lawrence Wilkerson, was also
hashing out the text on weapons with Hannah. The sources in the text
weren't footnoted, and Wilkerson grimaced as he watched Hannah fumble
through his binders. After one query, Hannah produced a New York Times
article as his source. Between I.N.R.'s factual objections and Hannah's
halting command of the material, Powell was fast losing faith in the
work by Libby's team. He instructed Wilkerson to start from scratch.

It was George Tenet who came to the rescue, Powell later said. Tenet
suggested that he base the new speech on the National Intelligence
Estimate relating to Iraq's weapons capability that had been thrown
together in less than three weeks the previous September. It was, after
all, the consensus product of the American intelligence community. What
could go wrong?

For the next three days, Powell, dressed in jeans, sat in Tenet's
conference room on the seventh floor in C.I.A. headquarters with his
speechwriting team. Line by line, data point by data point, the
secretary read out the text and then asked: ``Does that sound right?
What's the source on this? Opposition? Kurdish? Asylum seeker? Can we
trust him?'' If the answer did not suit him, Powell's reply would be:
``I'm not comfortable with that. Throw it on the floor.''

To the outside observer, the process seemed methodical and professional.
Dan Bartlett dropped by the C.I.A. over the weekend. ``Everybody's in
the room,'' Bartlett recalled. ``He's got their undivided attention.
This is going to be done right. I left thinking, OK, I feel good about
this.''

Powell had reason to feel sanguine about the process as well. Tenet was
there, along with McLaughlin and the aluminum tube he had taken to
carrying as a prop, which at one point he rolled across the
conference-room table. Whenever Powell seemed concerned about a
particular claim, Tenet's staff would usher in what seemed to be the
proper analyst to affirm the source's validity.

What Powell did not know was that there were other C.I.A. officials not
present in the conference room who seriously doubted much of the
National Intelligence Estimate's contents. This was particularly evident
on the subject of Hussein's biological-weapons capabilities. Some of the
most arresting visuals in the Case --- the only ones that seemed to
catch the attention of the Pentagon officials during McLaughlin's early
rehearsal of the C.I.A.'s presentation --- were photographs of a vehicle
believed to be an Iraqi mobile biological-weapons lab. Its description
had been supplied by a former Iraqi chemical engineer code-named
Curveball, who had made his way to Germany in 1999, seeking asylum and
in exchange offering spectacular details about Iraq's weapons program.
``The really strong stuff was Curveball,'' remembered Bill McLaughlin, a
C.I.A. military analyst (and no relation to John McLaughlin) who was in
the conference room on Saturday, Feb. 3. ``It was the kind of
specificity we needed to show. It was the centerpiece of the
discussion.''

But Curveball's claims to have been part of a mobile biological-weapons
program had also polarized the agency. The American intelligence
community still did not have access to the source himself. ``We don't
have a case officer in touch with this guy,'' Tenet had once muttered to
his staff. Though many analysts at the C.I.A. considered the Iraqi
engineer credible, the agency's Directorate of Operations officers, who
dealt firsthand with informants, believed they knew a liar when they saw
one. In Curveball, they saw a
\href{https://www.nytimes.com/2004/07/11/world/the-reach-of-war-conclusions-powell-s-solid-cia-tips-were-soft-committee-says.html}{liar}.

In December, John McLaughlin asked his executive assistant, Stephen
Slick, to (as Slick would put it) ``get to the bottom of a disagreement
within the building about the veracity of one human source.'' Tyler
Drumheller, the chief of the directorate's European division, instructed
Margaret Henoch, the division's chief of the group of countries that
included Germany, to ``look into Curveball.'' Referring to the
directorate's deputy director, Jim Pavitt, he added, ``Pavitt wants him
to be vetted, because apparently we're going to use him to justify going
into Iraq.''

Henoch's staff's discussions with German intelligence agents led them to
conclude that Curveball was not on the level. On Dec. 19, Henoch argued
this point to Slick. To a chief biological-weapons analyst in the room
who had fervently believed Curveball's claims, Henoch said: ``You guys
are trained to write papers. You write to prove a thesis, rather than
evaluating the information. And I think that's what you've done here.''

Henoch was overruled; a day later, Slick issued his opinion that the
intelligence community had conducted an ``exhaustive review'' of
Curveball and ``judged him credible.'' But Slick later acknowledged that
there was ``not much more'' to the biological-weapons case than
Curveball.

When another C.I.A. analyst expressed concern about Curveball to a
deputy on the weapons of mass destruction task force, the deputy's email
response began, ``Let's keep in mind the fact that this war's going to
happen regardless of what Curveball said or didn't say, and that the
Powers That Be probably aren't terribly interested in whether Curveball
knows what he's talking about.'' Pavitt, too, conveyed to a colleague
that war was inevitable and that those against it could ``tap dance nude
on Pennsylvania Avenue and it would make no difference.''

McLaughlin would later insist that he was unaware that doubts had been
expressed about Curveball's veracity. Still, before Powell was to
deliver his U.N. speech, the deputy director instructed Slick to check
on Curveball's ``current status/whereabouts.'' Slick's memo to
Drumheller on Feb. 3 said, ``A great deal of effort is being expended to
vet the intelligence that underlies SecState's upcoming U.N.
presentation.''

But the memo made no mention of a cable that had been sent to the
agency's headquarters a week before by the C.I.A.'s chief of station in
Berlin, Joe Wippl. The German intelligence agency handling Curveball
``has not been able to verify his reporting,'' Wippl warned. He added:
``The source himself is problematical. Defer to headquarters, but to use
information from another liaison service's source whose information
cannot be verified on such an important, key topic should take the most
serious consideration.''

Powell knew nothing about these serious concerns. The C.I.A.'s
dissenters were not in the room during the secretary's U.N. speech
preparation --- and Curveball's intelligence was the room's star
attraction. ``George was on the team, and that itself is an issue,''
Wippl would later reflect. ``It was, `Hey, guys, we're going to war ---
and we'll find this stuff anyway once we're there.' It's something that,
in retrospect, kind of makes you sick.''

\textbf{On the evening} of Feb. 4 at U.N. headquarters, Powell went over
his speech one final time. He asked Tenet if he felt comfortable with
the facts marshaled in the speech. The C.I.A. director said that he did.
``Good,'' Powell said. ``Because I want you sitting right behind me when
I give it tomorrow morning.'' Tenet was reluctant --- he was aware that
his appearing with the secretary would give the appearance that the
C.I.A. was putting its seal of approval on administration policy --- but
he was way past the point of protesting.

At 10:30 the following morning, Powell addressed the international body.
For the next 76 minutes, he laid out the U.S. government's case against
Hussein. ``My colleagues, every statement I make today is backed up by
sources, solid sources,'' Powell said in his calm, sonorous baritone.
``These are not assertions. What we're giving you are facts and
conclusions based on solid intelligence.''

The story Powell told marked a departure from the Bush administration's
evocations of madness, evil and mushroom clouds. It was an
investigator's meticulous brief of institutionalized deception and
murderous intent. Powell spoke of a key source, ``an eyewitness, an
Iraqi chemical engineer,'' who happened to be watching the speech at
home with his wife in Erlangen, Germany. He spoke of one of Curveball's
confirming sources, ``an Iraqi major'' --- surprising a Defense
Intelligence Agency staff member watching the speech who, months
earlier, had interviewed the major and determined him to be a
fabricator.

He spoke of decontamination trucks at chemical-weapons factories, to the
consternation of the chemical-weapons analyst Larry Fox, who had
repeatedly warned that the speech was making too much out of what might
well be innocuous vehicles but had been repeatedly overruled by his
superiors. And he spoke of aluminum tubes that ``most experts think''
were to be used for uranium enrichment --- ignoring his department's own
experts, including the I.N.R.'s director, Carl Ford, who became
heartsick watching Powell on TV and informed the secretary three months
later that he was resigning.

In the audience in the Security Council chamber was a young U.N. weapons
inspector named Dawson Cagle, who had recently returned from Baghdad.
Sitting next to Cagle was one of Hans Blix's senior munitions experts,
who had also just returned from Iraq's capital. The expert's mouth
opened when Powell displayed photographs of trucks moving into a
suspected weapons of mass destruction bunker hours before an inspection
team was due to visit, followed by a photo of the inspectors filing
through a now-empty bunker. ``I'm in that photo,'' the munitions expert
whispered to Cagle. ``I went into that bunker that those trucks pulled
up to. There was a three-inch layer of pigeon dung covering everything.
And a layer of dust on top of that. There's no way someone came in and
cleaned that place out. No way they could've faked that.''

But back at the White House, Bush watched Powell's speech in the small
dining room connected to the Oval Office, visibly pleased. On Capitol
Hill, at a Democratic Senate caucus meeting after the U.N. speech, Tom
Daschle, the majority leader, told his colleagues that he was now
``really convinced'' that Hussein had weapons of mass destruction. To
the caucus, he said: ``You may not trust Dick Cheney. But do you not
trust Colin Powell?''

Advertisement

\protect\hyperlink{after-bottom}{Continue reading the main story}

\hypertarget{site-index}{%
\subsection{Site Index}\label{site-index}}

\hypertarget{site-information-navigation}{%
\subsection{Site Information
Navigation}\label{site-information-navigation}}

\begin{itemize}
\tightlist
\item
  \href{https://help.nytimes.com/hc/en-us/articles/115014792127-Copyright-notice}{©~2020~The
  New York Times Company}
\end{itemize}

\begin{itemize}
\tightlist
\item
  \href{https://www.nytco.com/}{NYTCo}
\item
  \href{https://help.nytimes.com/hc/en-us/articles/115015385887-Contact-Us}{Contact
  Us}
\item
  \href{https://www.nytco.com/careers/}{Work with us}
\item
  \href{https://nytmediakit.com/}{Advertise}
\item
  \href{http://www.tbrandstudio.com/}{T Brand Studio}
\item
  \href{https://www.nytimes.com/privacy/cookie-policy\#how-do-i-manage-trackers}{Your
  Ad Choices}
\item
  \href{https://www.nytimes.com/privacy}{Privacy}
\item
  \href{https://help.nytimes.com/hc/en-us/articles/115014893428-Terms-of-service}{Terms
  of Service}
\item
  \href{https://help.nytimes.com/hc/en-us/articles/115014893968-Terms-of-sale}{Terms
  of Sale}
\item
  \href{https://spiderbites.nytimes.com}{Site Map}
\item
  \href{https://help.nytimes.com/hc/en-us}{Help}
\item
  \href{https://www.nytimes.com/subscription?campaignId=37WXW}{Subscriptions}
\end{itemize}
