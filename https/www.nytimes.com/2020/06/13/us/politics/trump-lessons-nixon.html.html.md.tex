Sections

SEARCH

\protect\hyperlink{site-content}{Skip to
content}\protect\hyperlink{site-index}{Skip to site index}

\href{https://www.nytimes.com/section/politics}{Politics}

\href{https://myaccount.nytimes.com/auth/login?response_type=cookie\&client_id=vi}{}

\href{https://www.nytimes.com/section/todayspaper}{Today's Paper}

\href{/section/politics}{Politics}\textbar{}Trump's Lessons From Nixon
Missed One Important Thing

\url{https://nyti.ms/30zpF9M}

\begin{itemize}
\item
\item
\item
\item
\item
\end{itemize}

\begin{itemize}
\item
  \href{https://www.nytimes.com/2020/07/31/us/elections/biden-vs-trump.html?action=click\&pgtype=Article\&state=default\&region=TOP_BANNER\&context=storylines_menu}{Election
  Updates}
\item
  \href{https://www.nytimes.com/article/biden-vice-president-2020.html?action=click\&pgtype=Article\&state=default\&region=TOP_BANNER\&context=storylines_menu}{Biden's
  V.P. Search}
\item
  \href{https://www.nytimes.com/interactive/2020/07/24/us/politics/trump-biden-campaign-donors.html?action=click\&pgtype=Article\&state=default\&region=TOP_BANNER\&context=storylines_menu}{Map
  of Donations}
\item
  \href{https://www.nytimes.com/interactive/2020/us/elections/delegate-count-primary-results.html?action=click\&pgtype=Article\&state=default\&region=TOP_BANNER\&context=storylines_menu}{Delegate
  Count}
\item
  \href{https://www.nytimes.com/interactive/2019/us/politics/2020-presidential-candidates.html?action=click\&pgtype=Article\&state=default\&region=TOP_BANNER\&context=storylines_menu}{The
  Candidates}
\item
  \href{https://www.nytimes.com/newsletters/politics?action=click\&pgtype=Article\&state=default\&region=TOP_BANNER\&context=storylines_menu}{Politics
  Newsletter}
\end{itemize}

Advertisement

\protect\hyperlink{after-top}{Continue reading the main story}

Supported by

\protect\hyperlink{after-sponsor}{Continue reading the main story}

News Analysis

\hypertarget{trumps-lessons-from-nixon-missed-one-important-thing}{%
\section{Trump's Lessons From Nixon Missed One Important
Thing}\label{trumps-lessons-from-nixon-missed-one-important-thing}}

Both men have faced impeachment proceedings, and have spoken of the
``silent majority'' and law and order. But Nixon paid more attention to
what people --- especially suburban voters --- thought.

\includegraphics{https://static01.nyt.com/images/2020/06/14/us/politics/00trump-nixon1/merlin_173448474_fce098b2-18b7-4cea-9b59-3e7d736653ee-articleLarge.jpg?quality=75\&auto=webp\&disable=upscale}

\href{https://www.nytimes.com/by/sarah-lyall}{\includegraphics{https://static01.nyt.com/images/2018/02/20/multimedia/author-sarah-lyall/author-sarah-lyall-thumbLarge.jpg}}\href{https://www.nytimes.com/by/jeremy-w-peters}{\includegraphics{https://static01.nyt.com/images/2018/11/06/multimedia/author-jeremy-w-peters/author-jeremy-w-peters-thumbLarge.png}}

By \href{https://www.nytimes.com/by/sarah-lyall}{Sarah Lyall} and
\href{https://www.nytimes.com/by/jeremy-w-peters}{Jeremy W. Peters}

\begin{itemize}
\item
  Published June 13, 2020Updated June 14, 2020
\item
  \begin{itemize}
  \item
  \item
  \item
  \item
  \item
  \end{itemize}
\end{itemize}

``I learned a lot from Richard Nixon,'' President Trump
\href{https://thehill.com/homenews/administration/496758-trump-says-he-learned-a-lot-from-nixon-dont-fire-people}{declared}
recently, speaking of the only U.S. president ever to resign in
disgrace. ``I study history.''

It was a bold assertion from Mr. Trump, not least because he and Nixon
both share the dubious distinction of facing impeachment after being
accused of abusing the power of the presidency. But if the president has
indeed studied the Nixon years --- a period characterized by widespread
social unrest that has parallels in the turbulence of today --- it is
not clear, historians say, whether he understands what lessons to draw
from them.

Mr. Trump's walkabout outside the White House earlier this month as
demonstrations swirled around him invited a direct comparison with Nixon
--- because Nixon made a similar trip. It was May 9, 1970, and it felt
like the country was on fire. Violence was erupting on college campuses
over the bombing of Cambodia. Tens of thousands of people were gathering
on the National Mall to protest the war in Vietnam and the killing of
four students by the Ohio National Guard at Kent State University. The
White House was fortified with extra troops.

Wracked by doubt and self-flagellation, unable to sleep, Nixon slipped
out of the building just after 4:35 a.m. with a handful of aides and
Secret Service agents and traveled to the Lincoln Memorial. There, he
tried to explain his Vietnam policy to a group of student demonstrators.

``I know probably most of you think I'm an S.O.B.,'' he
\href{https://www.youtube.com/watch?v=opK4XeVEE1Y}{told them}. ``But I
want you to know that I understand just how you feel.''

At times, Mr. Trump seems to be borrowing from a playbook that is a
half-century old, without seeing how profoundly the country has changed.

He is betting on the resonance of a message that served Republicans well
for decades, when dog whistles about crime and lawlessness were
effective at stoking the anxieties of white suburban voters. But that
messaging may be less effective at a time of growing awareness of racial
injustice, especially among educated suburban voters who lean Republican
but are put off by Mr. Trump's tendency to foment division and inflame
racial tension.

\includegraphics{https://static01.nyt.com/images/2020/06/14/us/politics/00trump-nixon2/merlin_173442975_ead098c1-3afc-4aaa-91ba-b15a6a273bdd-articleLarge.jpg?quality=75\&auto=webp\&disable=upscale}

One clear way to see what Mr. Trump has in fact not learned from Nixon
is to look closely at those two encounters 50 years apart.

Mr. Trump's photo op began with Nixon on his mind. Just before he
marched across Lafayette Square, his path cleared by law enforcement who
violently dispersed peaceful protesters, he declared himself ``your
president of law and order.'' It was a conspicuous appropriation of the
catchphrase Nixon deployed to sell himself as the candidate for
Americans weary of the tumult of the 1960s. Then when Mr. Trump reached
St. John's Church, he held a Bible aloft for the cameras.

\hypertarget{latest-updates-2020-election}{%
\section{\texorpdfstring{\href{https://www.nytimes.com/2020/07/31/us/elections/biden-vs-trump.html?action=click\&pgtype=Article\&state=default\&region=MAIN_CONTENT_1\&context=storylines_live_updates}{Latest
Updates: 2020
Election}}{Latest Updates: 2020 Election}}\label{latest-updates-2020-election}}

Updated 2020-08-01T01:26:45.732Z

\begin{itemize}
\tightlist
\item
  \href{https://www.nytimes.com/2020/07/31/us/elections/biden-vs-trump.html?action=click\&pgtype=Article\&state=default\&region=MAIN_CONTENT_1\&context=storylines_live_updates\#link-29fdff45}{Kamala
  Harris, a top vice-presidential contender, confronts double
  standards.}
\item
  \href{https://www.nytimes.com/2020/07/31/us/elections/biden-vs-trump.html?action=click\&pgtype=Article\&state=default\&region=MAIN_CONTENT_1\&context=storylines_live_updates\#link-13ec3d9c}{Karen
  Bass and Susan Rice are rising on Biden's vice-presidential
  shortlist.}
\item
  \href{https://www.nytimes.com/2020/07/31/us/elections/biden-vs-trump.html?action=click\&pgtype=Article\&state=default\&region=MAIN_CONTENT_1\&context=storylines_live_updates\#link-49e9a016}{Trump
  says Russian bounties to kill U.S. troops `never took place.'}
\end{itemize}

\href{https://www.nytimes.com/2020/07/31/us/elections/biden-vs-trump.html?action=click\&pgtype=Article\&state=default\&region=MAIN_CONTENT_1\&context=storylines_live_updates}{See
more updates}

But there are plenty of reasons that messaging might be a harder sell
today.

``The world has moved on,'' said Rick Perlstein, author of the book
``Nixonland.''

``Maybe the last laugh is on Donald Trump,'' he continued, ``the guy who
had signs at his rallies saying `silent majority' and who uses phrases
like `law and order,' and thinks he can run the same kind of script in a
different act.''

Right now there appears to be no ``silent majority'' --- at least in the
sense that Nixon meant, when an actual majority of Americans resented
the more vocal, left-leaning protest movements of the day.

Polls today show strong support for the demonstrations sparked by the
killing of George Floyd, who died in police custody after a Minneapolis
police officer held his knee on Mr. Floyd's neck for nearly 9 minutes. A
Monmouth University
\href{https://www.monmouth.edu/polling-institute/reports/monmouthpoll_US_060220/}{poll}
released this month found that 57 percent of Americans thought that the
anger that set off the current protests was ``fully justified.'' And 76
percent said that racial and ethnic discrimination is a big problem. A
separate PBS/NPR/Marist College
\href{http://maristpoll.marist.edu/wp-content/uploads/2020/06/NPR_PBS-NewsHour_Marist-Poll_USA-NOS-and-Tables_2006041039.pdf}{poll}
found that 62 percent of Americans believed the protests were mostly
legitimate.

``Inconceivable in 1968,'' Mr. Perlstein added, when more Americans were
on the side of the police.

And the protesters across the country --- and the world --- seem to
represent a far larger segment of society than those in the Nixon era.
``You look at the protests and that was a far more representative
cross-section of America out on the streets peacefully protesting who
felt moved to do something,'' former President Barack Obama
\href{https://www.cnbc.com/2020/06/03/barack-obama-says-protests-across-the-country-arent-like-the-1968-riots-which-some-think-helped-elect-nixon.html}{said}
recently.

Image

The public approved of the forceful tactics used by the police against
protesters at the 1968 Democratic National Convention in
Chicago.Credit...Barton Silverman/The New York Times

Beverly Gage, professor of American history at Yale University, said
Nixon's impromptu outing to meet protesters face-to-face was a
spontaneous expression of the kind of inner turmoil that Mr. Trump does
not seem to share. ``It was an anguished Nixon, and in many ways it was
a very human moment,'' she said.

The similarities between the circumstances the two faced as president
seem almost eerie. So can their similarities of character. These include
a hatred of the news media; a sense of grievance at the enemies, real or
imagined, they believe stand in the way of re-election; and a desire to
present themselves as law-and-order bulwarks against the forces of
chaos.

Soon after delivering his famous speech in the fall of 1969 in which he
beseeched the ``great silent majority'' of Americans for patience as he
dealt with the war in Vietnam, Nixon declared war on the press. He
dispatched his vice president, Spiro Agnew, to deliver a series of
broadside attacks on the major newspapers and networks for what he saw
as overly critical coverage of him.

Mr. Trump likewise dislikes any news outlet he considers critical of
him. But he is in the difficult position, critics say, of attempting to
promote himself as a law-and-order candidate when the failings of law
and order are being exposed on his watch.

Image

Nixon campaigned in Chicago the week after the Democratic
convention.Credit...Associated Press

``The city is burning, and Trump is Nero,'' said Timothy Naftali, who
teaches history at New York University and is a former director of the
Richard Nixon Presidential Library.

Nixon was able to capitalize on law-and-order sentiment during the 1968
election, benefiting from the presence of the segregationist George
Wallace on the right to present himself to voters as the candidate of
mainstream stability, while also representing order to those who saw
lawlessness all around them. Public opinion was on his side. After
Chicago police officers brutalized a group of demonstrators protesting
outside the 1968 Democratic National Convention, Gallup
\href{https://news.gallup.com/poll/8053/gallup-brain-war-peace-protests.aspx}{reported}
that 56 percent of Americans said they approved of the way law
enforcement handled the matter.

Patrick J. Buchanan, a Nixon speechwriter, recalled how divergent views
of the events --- one in the media that sympathized with the protesters;
another in the heartland that supported efforts to quell the unrest ---
helped elect Nixon. ``The press was all in on a `police riot,' while
Middle America supported the Chicago cops, as I urged Nixon to do,'' Mr.
Buchanan said. Nixon then campaigned in the streets of Chicago to
underscore his tough stance.

But there was more to him than that. Although Nixon said and did
horrible things in private --- speaking disparagingly of members of
minority groups, never mind orchestrating a criminal conspiracy to win
re-election --- he ``believed that the presidency was a dignified office
and there were things he did not want to be publicly associated with,''
Professor Naftali said.

In contrast, he added, ``Donald Trump doesn't believe in the concept of
being on your best behavior'' and seems to believe that the office is an
extension of himself. ``Nixon tied himself in knots to do things
secretly. Trump just does them in the open,'' Professor Naftali said.

The two men met a handful of times. According to their mutual friend,
Roger Stone, Nixon was immediately impressed. After Mr. Trump appeared
on the ``Phil Donahue Show'' in the 1980s, the former president wrote
Mr. Trump saying that his wife, Pat, was especially blown away. ``She
predicts whenever you decide to run for office you will be a winner!''
Mr. Stone recalled in his book ``The Making of the President 2016.''

As traumatic as Watergate and Nixon's disgrace and resignation proved to
be for America --- and as ugly as Nixon's programs to infiltrate and
engage in surveillance against his enemies were --- historians credit
him for his understanding of governance and his many accomplishments in
office. These include the opening of China, the signing of the first
SALT arms-limitation treaty with the Soviet Union, the creation of the
Environmental Protection Agency and the signing into law of the
Endangered Species Act.

``With Trump you get all the dark side of Nixon and none of the good,''
said John A. Farrell, author of the 2017 biography ``Richard Nixon: The
Life.'' ``There's not one record of accomplishment to take to the voters
--- no foreign policy triumph or domestic accomplishment.''

This makes re-election trickier as Mr. Trump faces multiple crises and
can no longer point to the brightest spot of his presidency: the
once-strong economy.

``If you're going to be a president who runs on and profits politically
from dividing the country and making Americans hate each other,'' Mr.
Farrell added, ``you better have a set of accomplishments as a
counterweight --- or else history will not be kind to you.''

History, of course, has been generally unkind to Nixon. But even he
managed to pull the country together as the election approached, in a
way that seems all but impossible in the singularly difficult year of
2020.

In 1972, Nixon won re-election with one of the biggest landslides ever,
carrying 49 states.

\hypertarget{our-2020-election-guide}{%
\section{Our 2020 Election Guide}\label{our-2020-election-guide}}

Updated July 31, 2020

\begin{itemize}
\item
  \begin{center}\rule{0.5\linewidth}{\linethickness}\end{center}

  \hypertarget{the-latest}{%
  \subsection{The Latest}\label{the-latest}}

  \begin{itemize}
  \tightlist
  \item
    President Trump's assault on the Postal Service is intersecting with
    his attacks on mail-in voting.
    \href{https://www.nytimes.com/2020/07/31/us/politics/trump-usps-mail-delays.html?action=click\&pgtype=Article\&state=default\&region=BELOW_MAIN_CONTENT\&context=storylines_guide}{Voting
    rights groups say it is a recipe for disaster.}
  \end{itemize}
\item
  \begin{center}\rule{0.5\linewidth}{\linethickness}\end{center}

  \hypertarget{bidens-vp-search}{%
  \subsection{Biden's V.P. Search}\label{bidens-vp-search}}

  \begin{itemize}
  \tightlist
  \item
    \href{https://www.nytimes.com/article/biden-vice-president-2020.html?action=click\&pgtype=Article\&state=default\&region=BELOW_MAIN_CONTENT\&context=storylines_guide}{Here
    are 13 women} who have been under consideration to be Joe Biden's
    running mate, and why each might be chosen --- and might not be.
  \end{itemize}
\item
  \begin{center}\rule{0.5\linewidth}{\linethickness}\end{center}

  \hypertarget{keep-up-with-our-coverage}{%
  \subsection{Keep Up With Our
  Coverage}\label{keep-up-with-our-coverage}}

  \begin{itemize}
  \tightlist
  \item
    Get an
    \href{https://www.nytimes.com/newsletters/politics?action=click\&pgtype=Article\&state=default\&region=BELOW_MAIN_CONTENT\&context=storylines_guide}{email}
    recapping the day's news
  \end{itemize}

  \begin{itemize}
  \tightlist
  \item
    Download our mobile app on
    \href{https://apps.apple.com/us/app/nytimes/id284862083?ls=1\&mat_click_id=5c79ae7455014fd1bd66b5610c05b8f2-20191112-16948\&referrer=mat_click_id\%3D5c79ae7455014fd1bd66b5610c05b8f2-20191112-16948\%26link_click_id\%3D722930677036718082}{iOS}
    and
    \href{http://a.localytics.com/android?id=com.nytimes.android\&referrer=utm_source\%3Dother_nyt_mobile_web\%26utm_medium\%3DWeb\%2520page\%26utm_term\%3DGeneral\%2520Mobile\%2520Page\%26utm_campaign\%3DNYT\%2520Mobile\%2520General\%2520Page}{Android}
    and turn on Breaking News and Politics alerts
  \end{itemize}
\end{itemize}

Advertisement

\protect\hyperlink{after-bottom}{Continue reading the main story}

\hypertarget{site-index}{%
\subsection{Site Index}\label{site-index}}

\hypertarget{site-information-navigation}{%
\subsection{Site Information
Navigation}\label{site-information-navigation}}

\begin{itemize}
\tightlist
\item
  \href{https://help.nytimes.com/hc/en-us/articles/115014792127-Copyright-notice}{©~2020~The
  New York Times Company}
\end{itemize}

\begin{itemize}
\tightlist
\item
  \href{https://www.nytco.com/}{NYTCo}
\item
  \href{https://help.nytimes.com/hc/en-us/articles/115015385887-Contact-Us}{Contact
  Us}
\item
  \href{https://www.nytco.com/careers/}{Work with us}
\item
  \href{https://nytmediakit.com/}{Advertise}
\item
  \href{http://www.tbrandstudio.com/}{T Brand Studio}
\item
  \href{https://www.nytimes.com/privacy/cookie-policy\#how-do-i-manage-trackers}{Your
  Ad Choices}
\item
  \href{https://www.nytimes.com/privacy}{Privacy}
\item
  \href{https://help.nytimes.com/hc/en-us/articles/115014893428-Terms-of-service}{Terms
  of Service}
\item
  \href{https://help.nytimes.com/hc/en-us/articles/115014893968-Terms-of-sale}{Terms
  of Sale}
\item
  \href{https://spiderbites.nytimes.com}{Site Map}
\item
  \href{https://help.nytimes.com/hc/en-us}{Help}
\item
  \href{https://www.nytimes.com/subscription?campaignId=37WXW}{Subscriptions}
\end{itemize}
