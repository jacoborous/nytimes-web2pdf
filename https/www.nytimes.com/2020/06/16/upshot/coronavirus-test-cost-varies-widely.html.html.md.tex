Sections

SEARCH

\protect\hyperlink{site-content}{Skip to
content}\protect\hyperlink{site-index}{Skip to site index}

\href{https://myaccount.nytimes.com/auth/login?response_type=cookie\&client_id=vi}{}

\href{https://www.nytimes.com/section/todayspaper}{Today's Paper}

\href{/section/upshot}{The Upshot}\textbar{}Most Coronavirus Tests Cost
About \$100. Why Did One Cost \$2,315?

\url{https://nyti.ms/3hArglB}

\begin{itemize}
\item
\item
\item
\item
\item
\item
\end{itemize}

\href{https://www.nytimes.com/news-event/coronavirus?action=click\&pgtype=Article\&state=default\&region=TOP_BANNER\&context=storylines_menu}{The
Coronavirus Outbreak}

\begin{itemize}
\tightlist
\item
  live\href{https://www.nytimes.com/2020/08/04/world/coronavirus-cases.html?action=click\&pgtype=Article\&state=default\&region=TOP_BANNER\&context=storylines_menu}{Latest
  Updates}
\item
  \href{https://www.nytimes.com/interactive/2020/us/coronavirus-us-cases.html?action=click\&pgtype=Article\&state=default\&region=TOP_BANNER\&context=storylines_menu}{Maps
  and Cases}
\item
  \href{https://www.nytimes.com/interactive/2020/science/coronavirus-vaccine-tracker.html?action=click\&pgtype=Article\&state=default\&region=TOP_BANNER\&context=storylines_menu}{Vaccine
  Tracker}
\item
  \href{https://www.nytimes.com/2020/08/02/us/covid-college-reopening.html?action=click\&pgtype=Article\&state=default\&region=TOP_BANNER\&context=storylines_menu}{College
  Reopening}
\item
  \href{https://www.nytimes.com/live/2020/08/04/business/stock-market-today-coronavirus?action=click\&pgtype=Article\&state=default\&region=TOP_BANNER\&context=storylines_menu}{Economy}
\end{itemize}

Advertisement

\protect\hyperlink{after-top}{Continue reading the main story}

Upshot

Supported by

\protect\hyperlink{after-sponsor}{Continue reading the main story}

\hypertarget{most-coronavirus-tests-cost-about-100-why-did-one-cost-2315}{%
\section{Most Coronavirus Tests Cost About \$100. Why Did One Cost
\$2,315?}\label{most-coronavirus-tests-cost-about-100-why-did-one-cost-2315}}

U.S. health care prices are unregulated, opaque and unpredictable. When
Congress required insurers to cover Covid-19 testing, a few providers
decided to take advantage.

By \href{https://www.nytimes.com/by/sarah-kliff}{Sarah Kliff}

\begin{itemize}
\item
  June 16, 2020
\item
  \begin{itemize}
  \item
  \item
  \item
  \item
  \item
  \item
  \end{itemize}
\end{itemize}

\includegraphics{https://static01.nyt.com/images/2020/06/12/upshot/up-coronavirus-test/merlin_173471394_2d24e97c-c589-4d62-ba16-4a68917e15f5-articleLarge.jpg?quality=75\&auto=webp\&disable=upscale}

In a one-story brick building in suburban Dallas, between a dentist
office and a family medicine clinic, is a medical laboratory that has
run some of the most expensive
\href{https://www.nytimes.com/2020/06/25/upshot/virus-testing-shortfall-arizona.html}{coronavirus
tests} in America.

Insurers have paid Gibson Diagnostic Labs as much as \$2,315 for
individual coronavirus tests. In a couple of cases, the price rose as
high as \$6,946 when the lab said it mistakenly charged patients three
times the base rate.

The company has no special or different technology from, say, major
diagnostic labs that charge \$100. It is one of a small number of
medical labs, hospitals and emergency rooms taking advantage of the way
Congress has designed compensation for coronavirus tests and treatment.

``We've seen a small number of laboratories that are charging egregious
prices for Covid-19 tests,'' said Angie Meoli, a senior vice president
at Aetna, one of the insurers required to cover testing costs.

How can a simple coronavirus test cost \$100 in one lab and 2,200
percent more in another? It comes back to a fundamental fact about the
American health care system: The government does not regulate health
care prices.

This tends to have two major outcomes that health policy experts have
seen before, and are seeing again with coronavirus testing.

The first is high prices over all. Most medical care in the United
States costs
\href{https://healthcostinstitute.org/hcci-research/international-comparisons-of-health-care-prices-2017-ifhp-survey}{double
or triple}what it would in a peer country. An appendectomy, for example,
costs \$3,050 in Britain and \$6,710 in New Zealand, two countries that
regulate health prices. In the United States, the average price is
\$13,020.

The second outcome is huge price variation, as each doctor's office and
hospital sets its own charges for care. One
\href{https://jamanetwork.com/journals/jamainternalmedicine/fullarticle/1151669}{2012
study}found that hospitals in California charge between \$1,529 and
\$182,955 for uncomplicated appendectomies.

``It's not unheard-of that one hospital can charge 100 times the price
of another for the same thing,'' said Dr. Renee Hsia, a professor at the
University of California, San Francisco, and an author of the
appendectomy study. ``There is no other market I can think of where that
happens except health care.''

There is
\href{https://journalofethics.ama-assn.org/article/complex-relationship-between-cost-and-quality-us-health-care/2014-02}{little
evidence} that higher prices correlate with better care. What's
different about the more expensive providers is that they've set higher
prices for their services.

\hypertarget{latest-updates-global-coronavirus-outbreak}{%
\section{\texorpdfstring{\href{https://www.nytimes.com/2020/08/04/world/coronavirus-cases.html?action=click\&pgtype=Article\&state=default\&region=MAIN_CONTENT_1\&context=storylines_live_updates}{Latest
Updates: Global Coronavirus
Outbreak}}{Latest Updates: Global Coronavirus Outbreak}}\label{latest-updates-global-coronavirus-outbreak}}

Updated 2020-08-05T07:58:24.076Z

\begin{itemize}
\tightlist
\item
  \href{https://www.nytimes.com/2020/08/04/world/coronavirus-cases.html?action=click\&pgtype=Article\&state=default\&region=MAIN_CONTENT_1\&context=storylines_live_updates\#link-762df92}{As
  talks drag on, McConnell signals openness to jobless aid extension,
  and negotiators agree on a deadline.}
\item
  \href{https://www.nytimes.com/2020/08/04/world/coronavirus-cases.html?action=click\&pgtype=Article\&state=default\&region=MAIN_CONTENT_1\&context=storylines_live_updates\#link-1228a480}{Novavax
  sees encouraging results from two studies of its experimental
  vaccine.}
\item
  \href{https://www.nytimes.com/2020/08/04/world/coronavirus-cases.html?action=click\&pgtype=Article\&state=default\&region=MAIN_CONTENT_1\&context=storylines_live_updates\#link-794484ed}{Mississippians
  must now wear masks in public, governor says.}
\end{itemize}

\href{https://www.nytimes.com/2020/08/04/world/coronavirus-cases.html?action=click\&pgtype=Article\&state=default\&region=MAIN_CONTENT_1\&context=storylines_live_updates}{See
more updates}

More live coverage:
\href{https://www.nytimes.com/live/2020/08/04/business/stock-market-today-coronavirus?action=click\&pgtype=Article\&state=default\&region=MAIN_CONTENT_1\&context=storylines_live_updates}{Markets}

Patients are, in the short run, somewhat protected from big coronavirus
testing bills. The federal government
\href{https://www.hrsa.gov/CovidUninsuredClaim}{set aside \$1 billion}
to pick up the tab for uninsured Americans who get tested. For the
insured, federal laws require that health plans cover the full costs of
coronavirus testing without applying a deductible or co-payment.

But American patients will eventually bear the costs of these expensive
tests in the form of higher insurance premiums. In some cases, they are
paying for additional tests, for flu and other respiratory diseases,
that doctors tack onto coronavirus orders. Those charges are not exempt
from co-payments and can fall into a patient's deductible.

Those kinds of bills could make patients wary of seeking care or testing
in the future, which could enable the further spread of coronavirus. In
an April poll, the Kaiser Family Foundation
\href{https://www.kff.org/coronavirus-covid-19/report/kff-health-tracking-poll-early-april-2020/}{found}that
most Americans were worried they wouldn't be able to afford coronavirus
testing or treatment if they needed it.

Redacted medical bills and explanation-of-benefit documents provided by
health insurers, coupled with bills that New York Times readers
\href{https://twitter.com/sarahkliff/status/1270342169539366914}{have
shared}, show the huge price variation in coronavirus tests. In Texas
alone, the charge for a test can range from \$27 to the \$2,315 that
Gibson Diagnostic has charged.

Image

Signs alerted drivers to free coronavirus testing at a community center
in Tupelo, Miss., last week.Credit...Thomas Wells/The Northeast
Mississippi Daily Journal, via Associated Press

Some patients are billed nothing at all for testing at public sites,
where local government agencies pick up the tab. It's hard to know the
true range of what health providers charge and what insurers pay,
because both parties typically keep that information
\href{https://www.nytimes.com/2019/12/04/health/hospitals-trump-prices-transparency.html}{secret}.

Health care providers testing for coronavirus also have additional
protections if they want to charge high prices. The recent CARES Act
requires that insurers cover the full cost of coronavirus testing, with
no co-pays or deductibles applied to the patient. The health plans must
also pay an out-of-network doctor or lab its full charge so long as the
provider posts that ``cash price'' online.

That latter provision is meant to prevent a practice known as ``balance
billing'': when an insurer pays an out-of-network doctor something less
than the full charge, and the doctor bills the patient for the
remainder.

Health policy experts
\href{https://www.brookings.edu/blog/usc-brookings-schaeffer-on-health-policy/2020/04/09/how-the-cares-act-affects-covid-19-test-pricing/\#:~:text=Loren\%20Adler,-Associate\%20Director\%2C\%20USC\&text=Tucked\%20in\%20the\%20Coronavirus\%20Aid,The\%20first\%20of\%20these\%20(Sec.}{worry}
that the policy unintentionally gives some providers the green light to
set exceptionally high charges, knowing that insurers are legally
required to pay.

``If you are an out-of-network lab, you can name your price,'' said
Loren Adler, an associate director at the U.S.C.-Brookings Schaeffer
Initiative for Health Policy. ``I could say it's \$50,000, and you are
required to pay me that amount.''

No health care provider has been quite that bold in its coronavirus
testing prices; most have kept their charges relatively modest.

Many health care providers have settled on test prices of \$50 to \$200.
Medicare initially paid heath providers \$51.31 for coronavirus tests
but
\href{https://www.cms.gov/newsroom/press-releases/cms-increases-medicare-payment-high-production-coronavirus-lab-tests-0}{bumped}
reimbursements up to \$100 in mid-April. LabCorp, one of the country's
largest diagnostic testing firms,
\href{https://www.labcorp.com/coronavirus-disease-covid-19/health-plan-information}{bills
insurers} \$100 for its tests.

A few health providers have set their prices significantly higher. A
chain of emergency rooms in Texas and Oklahoma, for example, has
regularly charged patients \$500 to \$990 for coronavirus tests. A small
hospital in Colorado and a laboratory in New Jersey have also come to
insurers' attention for their especially high bills.

Multiple insurers identified Texas as the state where they've received
the highest proportion of expensive tests. Blue Cross and Blue Shield of
Texas has received more than 600 out-of-network bills for coronavirus
tests that are over \$500, with an average charge of \$1,114.

Gibson Diagnostic Labs' website advertises ``Covid-19 testing for your
patients with results in just 24 to 48 hours.'' The website states the
``cash price'' for a coronavirus test as \$150, which is what they bill
the government for uninsured patients' tests. The billed charges for
insured patients were many multiples higher.

\href{https://www.nytimes.com/news-event/coronavirus?action=click\&pgtype=Article\&state=default\&region=MAIN_CONTENT_3\&context=storylines_faq}{}

\hypertarget{the-coronavirus-outbreak-}{%
\subsubsection{The Coronavirus Outbreak
›}\label{the-coronavirus-outbreak-}}

\hypertarget{frequently-asked-questions}{%
\paragraph{Frequently Asked
Questions}\label{frequently-asked-questions}}

Updated August 4, 2020

\begin{itemize}
\item ~
  \hypertarget{i-have-antibodies-am-i-now-immune}{%
  \paragraph{I have antibodies. Am I now
  immune?}\label{i-have-antibodies-am-i-now-immune}}

  \begin{itemize}
  \tightlist
  \item
    As of right
    now,\href{https://www.nytimes.com/2020/07/22/health/covid-antibodies-herd-immunity.html?action=click\&pgtype=Article\&state=default\&region=MAIN_CONTENT_3\&context=storylines_faq}{that
    seems likely, for at least several months.} There have been
    frightening accounts of people suffering what seems to be a second
    bout of Covid-19. But experts say these patients may have a
    drawn-out course of infection, with the virus taking a slow toll
    weeks to months after initial exposure. People infected with the
    coronavirus typically
    \href{https://www.nature.com/articles/s41586-020-2456-9}{produce}
    immune molecules called antibodies, which are
    \href{https://www.nytimes.com/2020/05/07/health/coronavirus-antibody-prevalence.html?action=click\&pgtype=Article\&state=default\&region=MAIN_CONTENT_3\&context=storylines_faq}{protective
    proteins made in response to an
    infection}\href{https://www.nytimes.com/2020/05/07/health/coronavirus-antibody-prevalence.html?action=click\&pgtype=Article\&state=default\&region=MAIN_CONTENT_3\&context=storylines_faq}{.
    These antibodies may} last in the body
    \href{https://www.nature.com/articles/s41591-020-0965-6}{only two to
    three months}, which may seem worrisome, but that's perfectly normal
    after an acute infection subsides, said Dr. Michael Mina, an
    immunologist at Harvard University. It may be possible to get the
    coronavirus again, but it's highly unlikely that it would be
    possible in a short window of time from initial infection or make
    people sicker the second time.
  \end{itemize}
\item ~
  \hypertarget{im-a-small-business-owner-can-i-get-relief}{%
  \paragraph{I'm a small-business owner. Can I get
  relief?}\label{im-a-small-business-owner-can-i-get-relief}}

  \begin{itemize}
  \tightlist
  \item
    The
    \href{https://www.nytimes.com/article/small-business-loans-stimulus-grants-freelancers-coronavirus.html?action=click\&pgtype=Article\&state=default\&region=MAIN_CONTENT_3\&context=storylines_faq}{stimulus
    bills enacted in March} offer help for the millions of American
    small businesses. Those eligible for aid are businesses and
    nonprofit organizations with fewer than 500 workers, including sole
    proprietorships, independent contractors and freelancers. Some
    larger companies in some industries are also eligible. The help
    being offered, which is being managed by the Small Business
    Administration, includes the Paycheck Protection Program and the
    Economic Injury Disaster Loan program. But lots of folks have
    \href{https://www.nytimes.com/interactive/2020/05/07/business/small-business-loans-coronavirus.html?action=click\&pgtype=Article\&state=default\&region=MAIN_CONTENT_3\&context=storylines_faq}{not
    yet seen payouts.} Even those who have received help are confused:
    The rules are draconian, and some are stuck sitting on
    \href{https://www.nytimes.com/2020/05/02/business/economy/loans-coronavirus-small-business.html?action=click\&pgtype=Article\&state=default\&region=MAIN_CONTENT_3\&context=storylines_faq}{money
    they don't know how to use.} Many small-business owners are getting
    less than they expected or
    \href{https://www.nytimes.com/2020/06/10/business/Small-business-loans-ppp.html?action=click\&pgtype=Article\&state=default\&region=MAIN_CONTENT_3\&context=storylines_faq}{not
    hearing anything at all.}
  \end{itemize}
\item ~
  \hypertarget{what-are-my-rights-if-i-am-worried-about-going-back-to-work}{%
  \paragraph{What are my rights if I am worried about going back to
  work?}\label{what-are-my-rights-if-i-am-worried-about-going-back-to-work}}

  \begin{itemize}
  \tightlist
  \item
    Employers have to provide
    \href{https://www.osha.gov/SLTC/covid-19/standards.html}{a safe
    workplace} with policies that protect everyone equally.
    \href{https://www.nytimes.com/article/coronavirus-money-unemployment.html?action=click\&pgtype=Article\&state=default\&region=MAIN_CONTENT_3\&context=storylines_faq}{And
    if one of your co-workers tests positive for the coronavirus, the
    C.D.C.} has said that
    \href{https://www.cdc.gov/coronavirus/2019-ncov/community/guidance-business-response.html}{employers
    should tell their employees} -\/- without giving you the sick
    employee's name -\/- that they may have been exposed to the virus.
  \end{itemize}
\item ~
  \hypertarget{should-i-refinance-my-mortgage}{%
  \paragraph{Should I refinance my
  mortgage?}\label{should-i-refinance-my-mortgage}}

  \begin{itemize}
  \tightlist
  \item
    \href{https://www.nytimes.com/article/coronavirus-money-unemployment.html?action=click\&pgtype=Article\&state=default\&region=MAIN_CONTENT_3\&context=storylines_faq}{It
    could be a good idea,} because mortgage rates have
    \href{https://www.nytimes.com/2020/07/16/business/mortgage-rates-below-3-percent.html?action=click\&pgtype=Article\&state=default\&region=MAIN_CONTENT_3\&context=storylines_faq}{never
    been lower.} Refinancing requests have pushed mortgage applications
    to some of the highest levels since 2008, so be prepared to get in
    line. But defaults are also up, so if you're thinking about buying a
    home, be aware that some lenders have tightened their standards.
  \end{itemize}
\item ~
  \hypertarget{what-is-school-going-to-look-like-in-september}{%
  \paragraph{What is school going to look like in
  September?}\label{what-is-school-going-to-look-like-in-september}}

  \begin{itemize}
  \tightlist
  \item
    It is unlikely that many schools will return to a normal schedule
    this fall, requiring the grind of
    \href{https://www.nytimes.com/2020/06/05/us/coronavirus-education-lost-learning.html?action=click\&pgtype=Article\&state=default\&region=MAIN_CONTENT_3\&context=storylines_faq}{online
    learning},
    \href{https://www.nytimes.com/2020/05/29/us/coronavirus-child-care-centers.html?action=click\&pgtype=Article\&state=default\&region=MAIN_CONTENT_3\&context=storylines_faq}{makeshift
    child care} and
    \href{https://www.nytimes.com/2020/06/03/business/economy/coronavirus-working-women.html?action=click\&pgtype=Article\&state=default\&region=MAIN_CONTENT_3\&context=storylines_faq}{stunted
    workdays} to continue. California's two largest public school
    districts --- Los Angeles and San Diego --- said on July 13, that
    \href{https://www.nytimes.com/2020/07/13/us/lausd-san-diego-school-reopening.html?action=click\&pgtype=Article\&state=default\&region=MAIN_CONTENT_3\&context=storylines_faq}{instruction
    will be remote-only in the fall}, citing concerns that surging
    coronavirus infections in their areas pose too dire a risk for
    students and teachers. Together, the two districts enroll some
    825,000 students. They are the largest in the country so far to
    abandon plans for even a partial physical return to classrooms when
    they reopen in August. For other districts, the solution won't be an
    all-or-nothing approach.
    \href{https://bioethics.jhu.edu/research-and-outreach/projects/eschool-initiative/school-policy-tracker/}{Many
    systems}, including the nation's largest, New York City, are
    devising
    \href{https://www.nytimes.com/2020/06/26/us/coronavirus-schools-reopen-fall.html?action=click\&pgtype=Article\&state=default\&region=MAIN_CONTENT_3\&context=storylines_faq}{hybrid
    plans} that involve spending some days in classrooms and other days
    online. There's no national policy on this yet, so check with your
    municipal school system regularly to see what is happening in your
    community.
  \end{itemize}
\end{itemize}

Three large health insurers independently identified Gibson Diagnostic,
which is in Irving, Texas, as the source of their highest-priced tests
received during the pandemic.

One national health plan was surprised to notice testing for sexually
transmitted diseases tacked onto some of the coronavirus bills that ran
through Gibson Diagnostic.

In a statement last week, the company said the \$2,315 price was the
result of ``human error'' that occurred when a billing department
employee entered the wrong price into an internal system. It billed 117
tests at that price, and had 23 of the claims paid in full. Some
insurers paid partial reimbursements or sent back no money at all.

The company said one insurance plan flagged the high price in mid-April,
which led it to reduce the price to \$500. The new charge was still 500
percent of the Medicare rate and \$350 higher than the online cash
price. The company declined to comment on how it settled on the new
price and why it differed from the one posted on its website.

Gibson Diagnostic also said that it had recently reversed a few of its
\$2,315 charges and, after an inquiry from The Times, would reverse the
rest of those bills within 24 hours.

Other laboratory owners questioned why even \$500 would be necessary to
run a relatively simple test. A data set of 29,160 coronavirus test
bills provided by Castlight Health, a firm that assists companies with
health benefits, found that 87 percent cost \$100 or less.

The American Clinical Laboratory Association estimates that its members,
which have run a collective 11 million coronavirus tests, charge between
\$95 and \$209.

``I don't believe it's commercially reasonable,'' said Peter Gudaitis,
who runs Aculabs in New Jersey, a member of the association.

Gibson Diagnostic may have come to a similar conclusion: This week, the
company reached out to The Times to say it would once again lower its
price. Now, the lab charges \$300 per coronavirus test.

The high prices have frustrated state insurance regulators, who lack
authority to tamp down what health care providers charge. ``We see these
infrequently, but they are infuriating when they do occur,'' said Mike
Rhoads, a deputy commissioner of consumer services at the Oklahoma
Insurance Department. ``There are free testing sites in our state. This
does not need to happen.''

He has encouraged the administrators of health plans he regulates to
contact their members of Congress, to urge refinements to the CARES Act
that would help bring prices down.

Some members of Congress say they are looking into the issue,
particularly those who recently worked on a bipartisan effort
\href{https://www.nytimes.com/2019/12/17/upshot/surprise-billing-democrats-2020.html}{to
outlaw surprise medical bills} (that effort has been sidelined since the
arrival of the pandemic). Legislators say they are still researching the
issue, and no action is currently planned.

``We've got no regulatory authority over the health care providers,''
Mr. Rhoads said. ``There is not much we can do. We hope that somebody
puts some pressure on these out-of-network providers to stop doing this,
particularly during this period of time.''

Advertisement

\protect\hyperlink{after-bottom}{Continue reading the main story}

\hypertarget{site-index}{%
\subsection{Site Index}\label{site-index}}

\hypertarget{site-information-navigation}{%
\subsection{Site Information
Navigation}\label{site-information-navigation}}

\begin{itemize}
\tightlist
\item
  \href{https://help.nytimes.com/hc/en-us/articles/115014792127-Copyright-notice}{©~2020~The
  New York Times Company}
\end{itemize}

\begin{itemize}
\tightlist
\item
  \href{https://www.nytco.com/}{NYTCo}
\item
  \href{https://help.nytimes.com/hc/en-us/articles/115015385887-Contact-Us}{Contact
  Us}
\item
  \href{https://www.nytco.com/careers/}{Work with us}
\item
  \href{https://nytmediakit.com/}{Advertise}
\item
  \href{http://www.tbrandstudio.com/}{T Brand Studio}
\item
  \href{https://www.nytimes.com/privacy/cookie-policy\#how-do-i-manage-trackers}{Your
  Ad Choices}
\item
  \href{https://www.nytimes.com/privacy}{Privacy}
\item
  \href{https://help.nytimes.com/hc/en-us/articles/115014893428-Terms-of-service}{Terms
  of Service}
\item
  \href{https://help.nytimes.com/hc/en-us/articles/115014893968-Terms-of-sale}{Terms
  of Sale}
\item
  \href{https://spiderbites.nytimes.com}{Site Map}
\item
  \href{https://help.nytimes.com/hc/en-us}{Help}
\item
  \href{https://www.nytimes.com/subscription?campaignId=37WXW}{Subscriptions}
\end{itemize}
