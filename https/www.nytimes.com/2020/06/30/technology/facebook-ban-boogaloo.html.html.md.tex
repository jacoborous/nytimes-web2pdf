Sections

SEARCH

\protect\hyperlink{site-content}{Skip to
content}\protect\hyperlink{site-index}{Skip to site index}

\href{https://www.nytimes.com/section/technology}{Technology}

\href{https://myaccount.nytimes.com/auth/login?response_type=cookie\&client_id=vi}{}

\href{https://www.nytimes.com/section/todayspaper}{Today's Paper}

\href{/section/technology}{Technology}\textbar{}Facebook Bans Network
With `Boogaloo' Ties

\url{https://nyti.ms/2YNMjKd}

\begin{itemize}
\item
\item
\item
\item
\item
\end{itemize}

Advertisement

\protect\hyperlink{after-top}{Continue reading the main story}

Supported by

\protect\hyperlink{after-sponsor}{Continue reading the main story}

\hypertarget{facebook-bans-network-with-boogaloo-ties}{%
\section{Facebook Bans Network With `Boogaloo'
Ties}\label{facebook-bans-network-with-boogaloo-ties}}

The social network said it was designating the antigovernment movement
as a dangerous organization.

\includegraphics{https://static01.nyt.com/images/2020/06/30/business/30BOOGALOO/merlin_171708825_712949e3-47ef-4ed9-8983-a7e76a212b55-articleLarge.jpg?quality=75\&auto=webp\&disable=upscale}

By \href{https://www.nytimes.com/by/davey-alba}{Davey Alba}

\begin{itemize}
\item
  June 30, 2020
\item
  \begin{itemize}
  \item
  \item
  \item
  \item
  \item
  \end{itemize}
\end{itemize}

Facebook said on Tuesday that it took down a network of accounts, groups
and pages
\href{https://about.fb.com/news/2020/06/banning-a-violent-network-in-the-us/}{connected
to an antigovernment movement} in the United States that encourages
violence.

People and groups associated with the decentralized movement, called
boogaloo, will be banned from Facebook and Instagram, which Facebook
also owns, the company said. Facebook said it had removed 220 Facebook
accounts, 95 Instagram accounts, 28 pages and 106 groups as a result of
the decision. It is also designating boogaloo as a dangerous
organization on the social network, meaning it shares the same
classification as terrorist activity, organized hate and large-scale
criminal organizations on Facebook.

As a result, Facebook said it would ban people and organizations linked
to boogaloo, and remove content that praises, supports and represents
the movement.

The boogaloo network promoted ``violence against civilians, law
enforcement, and government officials and institutions,'' the company
wrote in a blog post. ``Members of this network seek to recruit others
within the broader boogaloo movement, sharing the same content online
and adopting the same offline appearance as others in the movement to do
so.''

The decision is the latest in a flurry of recent moves by tech companies
to tighten the speech allowed on their popular services and more
aggressively police extreme movements. The issue has become more
pronounced in recent weeks after the
\href{https://www.nytimes.com/2020/05/31/us/george-floyd-investigation.html}{death
of George Floyd}, a Black man in Minneapolis who was killed in police
custody last month. The killing set off major protests across the
country demanding changes to police departments and the treatment of
Black people more broadly.

On Monday,
\href{https://www.nytimes.com/2020/06/29/technology/reddit-hate-speech.html}{Reddit
said} it was banning roughly 2,000 communities from across the political
spectrum that attacked people or regularly engaged in hate speech,
including ``r/The\_Donald,'' a community devoted to President Trump.
YouTube said it barred six channels for violating its policies,
including those of two prominent white supremacists, David Duke and
Richard Spencer.

Facebook's changes have so far largely focused on the boogaloo movement
and white supremacy hate groups.
\href{https://www.reuters.com/article/us-facebook-boogaloo/facebook-moves-to-limit-spread-of-boogaloo-groups-after-charges-idUSKBN23C011}{In
May}, Facebook said it updated its policies to ban the use of
``boogaloo'' and related terms when used in posts that contain
depictions of armed violence. The company said it had identified and
removed over 800 posts tied to boogaloo over the past two months because
they defied its
\href{https://www.facebook.com/communitystandards/credible_violence}{Violence
and Incitement policy}, and that it did not recommend pages and groups
referencing the movement to others on the social network. This month,
the company said that it had
\href{https://www.nytimes.com/aponline/2020/06/17/business/ap-us-america-protests-facebook-hate-groups.html}{removed}
two networks of accounts connected to white supremacy groups that
encouraged real-world violence.

Followers of the boogaloo movement seek to exploit public unrest to
incite a race war that will bring about a new government. Its adherents
are usually staunch defenders of the Second Amendment, and some use Nazi
iconography and its extremist symbols, according to organizations that
track hate groups.

``Boogaloo'' is a pop culture reference derived from a 1984 movie called
``Breakin' 2: Electric Boogaloo'' that became a cult classic. Online, it
has been connected to what some consider sarcastic and humorous memes,
as well as with occasional physical violence and militaristic shows of
force.

In June, the Federal Bureau of Investigation
\href{https://www.nytimes.com/2020/06/11/us/antifa-protests-george-floyd.html}{arrested}
three men in Nevada who called themselves members of the boogaloo
movement, accusing them of trying to incite violence at an anti-police
protest in Las Vegas. In May, police officers in Denver seized three
assault rifles, magazines, several bulletproof vests and other military
equipment from the car trunk of a self-identified boogaloo follower who
was headed to a Black Lives Matter protest --- and had previously
live-streamed his support for armed confrontations with the police.

In addition to the boogaloo network, Facebook said it would also remove
400 public and private groups and more than a hundred pages that also
violate its
\href{https://www.facebook.com/communitystandards/dangerous_individuals_organizations}{Dangerous
Individuals and Organizations policy}. Alex Stamos, director of the
Stanford Internet Observatory and the former chief security officer at
Facebook, said the company's dangerous organizations policy came out of
the fight to kick the terrorist group ISIS off social media.

Facebook said it would continue to identify and remove attempts by
members of the boogaloo movement to return to the social network.

Graham Brookie, director of the Atlantic Council's Digital Forensic
Research Lab, which studies disinformation, applauded Facebook's
crackdown on Tuesday.

``The Dangerous Individuals policy at Facebook mirrors the language of
law enforcement, and meets a high threshold of online harms that lead to
direct action in the real world,'' Mr. Brookie said. ``Limiting the
online conversation that leads to that action is a good thing and a
public safety issue.''

Emerson Brooking, a resident fellow at the Atlantic Council's Digital
Forensic Research Lab, said that deciding which posts linked to the
boogaloo movement could stay up and what should be taken down had always
been ``a content moderation nightmare'' for social networks.

``Many adherents can claim, truthfully, that they do not engage in
violence or advocate for white nationalism,'' he said. ``As a result, it
has evaded content moderation policies for several months.'' With its
announcement, he said, Facebook demonstrated an understanding of how
harmful the boogaloo movement was.

But Mr. Stamos said the decentralized nature of the movement and its
tendency to use irony and euphemism in posts could make continued
enforcement difficult.

``Deciding who is actually a boogaloo member now that they are motivated
to obfuscate their allegiances will be a huge, ongoing challenge,'' Mr.
Stamos said.

Advertisement

\protect\hyperlink{after-bottom}{Continue reading the main story}

\hypertarget{site-index}{%
\subsection{Site Index}\label{site-index}}

\hypertarget{site-information-navigation}{%
\subsection{Site Information
Navigation}\label{site-information-navigation}}

\begin{itemize}
\tightlist
\item
  \href{https://help.nytimes.com/hc/en-us/articles/115014792127-Copyright-notice}{©~2020~The
  New York Times Company}
\end{itemize}

\begin{itemize}
\tightlist
\item
  \href{https://www.nytco.com/}{NYTCo}
\item
  \href{https://help.nytimes.com/hc/en-us/articles/115015385887-Contact-Us}{Contact
  Us}
\item
  \href{https://www.nytco.com/careers/}{Work with us}
\item
  \href{https://nytmediakit.com/}{Advertise}
\item
  \href{http://www.tbrandstudio.com/}{T Brand Studio}
\item
  \href{https://www.nytimes.com/privacy/cookie-policy\#how-do-i-manage-trackers}{Your
  Ad Choices}
\item
  \href{https://www.nytimes.com/privacy}{Privacy}
\item
  \href{https://help.nytimes.com/hc/en-us/articles/115014893428-Terms-of-service}{Terms
  of Service}
\item
  \href{https://help.nytimes.com/hc/en-us/articles/115014893968-Terms-of-sale}{Terms
  of Sale}
\item
  \href{https://spiderbites.nytimes.com}{Site Map}
\item
  \href{https://help.nytimes.com/hc/en-us}{Help}
\item
  \href{https://www.nytimes.com/subscription?campaignId=37WXW}{Subscriptions}
\end{itemize}
