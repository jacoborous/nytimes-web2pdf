Sections

SEARCH

\protect\hyperlink{site-content}{Skip to
content}\protect\hyperlink{site-index}{Skip to site index}

\href{https://www.nytimes.com/section/technology/personaltech}{Personal
Tech}

\href{https://myaccount.nytimes.com/auth/login?response_type=cookie\&client_id=vi}{}

\href{https://www.nytimes.com/section/todayspaper}{Today's Paper}

\href{/section/technology/personaltech}{Personal Tech}\textbar{}E-Bikes
Are Having Their Moment. They Deserve It.

\url{https://nyti.ms/3eLA2Lr}

\begin{itemize}
\item
\item
\item
\item
\item
\end{itemize}

\href{https://www.nytimes.com/spotlight/at-home?action=click\&pgtype=Article\&state=default\&region=TOP_BANNER\&context=at_home_menu}{At
Home}

\begin{itemize}
\tightlist
\item
  \href{https://www.nytimes.com/2020/07/28/books/time-for-a-literary-road-trip.html?action=click\&pgtype=Article\&state=default\&region=TOP_BANNER\&context=at_home_menu}{Take:
  A Literary Road Trip}
\item
  \href{https://www.nytimes.com/2020/07/29/magazine/bored-with-your-home-cooking-some-smoky-eggplant-will-fix-that.html?action=click\&pgtype=Article\&state=default\&region=TOP_BANNER\&context=at_home_menu}{Cook:
  Smoky Eggplant}
\item
  \href{https://www.nytimes.com/2020/07/27/travel/moose-michigan-isle-royale.html?action=click\&pgtype=Article\&state=default\&region=TOP_BANNER\&context=at_home_menu}{Look
  Out: For Moose}
\item
  \href{https://www.nytimes.com/interactive/2020/at-home/even-more-reporters-editors-diaries-lists-recommendations.html?action=click\&pgtype=Article\&state=default\&region=TOP_BANNER\&context=at_home_menu}{Explore:
  Reporters' Obsessions}
\end{itemize}

Advertisement

\protect\hyperlink{after-top}{Continue reading the main story}

Supported by

\protect\hyperlink{after-sponsor}{Continue reading the main story}

Tech Fix

\hypertarget{e-bikes-are-having-their-moment-they-deserve-it}{%
\section{E-Bikes Are Having Their Moment. They Deserve
It.}\label{e-bikes-are-having-their-moment-they-deserve-it}}

The benefits of owning a battery-powered two-wheeler far outweigh the
downsides, especially in a pandemic.

\includegraphics{https://static01.nyt.com/images/2020/06/03/business/03techfix1/merlin_173044968_b11a15e2-c026-42bf-b05b-38a6d8a3782a-articleLarge.jpg?quality=75\&auto=webp\&disable=upscale}

\href{https://www.nytimes.com/by/brian-x-chen}{\includegraphics{https://static01.nyt.com/images/2018/02/16/multimedia/author-brian-x-chen/author-brian-x-chen-thumbLarge.jpg}}

By \href{https://www.nytimes.com/by/brian-x-chen}{Brian X. Chen}

\begin{itemize}
\item
  June 3, 2020
\item
  \begin{itemize}
  \item
  \item
  \item
  \item
  \item
  \end{itemize}
\end{itemize}

Many of us are entering a new stage of pandemic grief: adaptation. We
are asking ourselves: How do we live with this new reality?

For many Americans, part of the solution has been to buy an electric
bike. The battery-powered two-wheelers have become a compelling
alternative for
\href{https://www.nytimes.com/2020/05/28/health/cdc-coronavirus-offices.html?action=click\&module=Top\%20Stories\&pgtype=Homepage}{commuters
who are being discouraged} from taking public transportation and Ubers.
For others, the bikes provide much-needed fresh air after months of
confinement.

So it's no surprise that e-bikes are now as difficult to buy as a bottle
of hand sanitizer was a few weeks ago. In March,
\href{https://www.nytimes.com/2020/05/18/nyregion/bike-shortage-coronavirus.html}{sales
of e-bikes jumped 85 percent} from a year earlier, according to the NPD
Group, a research firm. Amazon, Walmart and Specialized are sold out of
most models. Even smaller brands like Ride1Up and VanMoof have waiting
lists.

That's a remarkable shift. For many years, e-bikes carried the stigma of
being vehicles for lazy pedalers and seniors. The bikes draw power from
a battery and motor to make pedaling significantly easier. You can also
accelerate with the press of a button, transforming cycling from a
strenuous exercise into a joy ride.

``I was convinced that e-bikes would completely change cities all over
the world in the next 10 years, but it seems like because of this
crisis, suddenly it's all happening in the next three or four months,''
said Taco Carlier, the chief executive of VanMoof, which is based in
Amsterdam.

If you are contemplating an e-bike purchase, there are trade-offs to
consider. For one, the battery packs and motors add bulk. For another,
these ostentatious bikes may lure thieves.

To find out what you get for your money, I tested two different e-bikes
on the streets and steep hills of San Francisco over the last two weeks.
Both can be ordered online:
\href{https://www.vanmoof.com/en-US}{VanMoof's \$1,998 S3}, an
internet-connected smart bike, and
\href{https://ride1up.com/product/700-series/}{Ride1Up's \$1,495 700
Series}, which is more like a normal bicycle with a battery and motor.

After the tests, I'm totally sold. E-bikes, I concluded, are for people
who want to get around quickly with minimal effort --- and that's a
large portion of the population. Here's what you need to know.

\hypertarget{comparing-the-e-bikes}{%
\subsection{Comparing the e-bikes.}\label{comparing-the-e-bikes}}

\includegraphics{https://static01.nyt.com/images/2020/06/03/business/03techfix2/merlin_173044839_217a78b8-1b5d-41a2-9a95-0f9a3a39095d-articleLarge.jpg?quality=75\&auto=webp\&disable=upscale}

E-bikes come in many forms and with various features. They also range
widely in price: Some cost a few hundred dollars, while others cost tens
of thousands of dollars. In general, though, e-bikes fall into two
camps:

\begin{itemize}
\item
  \textbf{E-bikes with pedal assistance.} These use a motor system and
  sensors to detect how fast or hard you are pedaling and determine how
  much power to provide. So if you are pedaling hard or slow up a hill,
  the motor will use more power to assist you. Well-known brands include
  \href{https://www.trekbikes.com/us/en_US/ebike_faq/}{Trek},
  \href{https://www.specialized.com/nz/en/stories/turbo-faq}{Specialized}
  and \href{https://www.fujibikes.com/usa/bikes/electric/}{Fuji}.
\item
  \textbf{E-bikes with a throttle.} These work like the twist throttle
  on motorcycles and mo-peds. To accelerate, you press a trigger or
  twist a handlebar. Many modern e-bikes with a throttle also have pedal
  assist. Brands include
  \href{https://www.radpowerbikes.com/products/radwagon-electric-cargo-bike?variant=32100542283872\#specs}{Rad
  Power}, \href{https://lunacycle.com/}{Luna Cycle} and
  \href{https://www.aventon.com/products/aventon-pace-500-complete-bike}{Aventon}.
\end{itemize}

VanMoof's S3, which was released in late April, is a pedal-assist
e-bike. Instead of a throttle, it has a Turbo Boost button on the right
handlebar, which immediately gives a jolt of power. It has a top speed
of about 20 miles per hour and can travel about 90 miles on a full
charge.

VanMoof e-bikes are known for their antitheft security. Kicking a button
on the rear brake activates an electronic lock, which makes the rear
wheel unmovable. Trying to pick up the locked bike triggers a loud
alarm. In addition, the bike includes a cellular connection to help you
find it if it's stolen, using VanMoof's smartphone app.

Ride1Up's 700 series has both a throttle and pedal assistance. On the
left handlebar is a small screen with buttons to let you select the
pedal-assist level; on the right handle bar is a gear shifter. With a
larger, faster motor than the VanMoof, the Ride1Up has a top speed of 28
m.p.h. and can travel about 50 miles on a full charge.

\hypertarget{testing-testing}{%
\subsection{Testing, testing.}\label{testing-testing}}

Image

Ride1Up's control panel offers nine pedal-assist levels.Credit...Jim
Wilson/The New York Times

For two weeks, I alternated between riding the VanMoof and the Ride1Up.
I found you get what you pay for: While \$1,500 buys you a nice e-bike
that takes time to get used to, like the Ride1Up, an additional \$500
secures you a VanMoof, a smarter bike that is extremely simple to use.

The VanMoof's motor system made pedaling feel more natural and smooth,
like riding a normal bicycle but with a bit of oomph. The motor was also
very quiet, and at points I forgot I was riding an e-bike. In areas
where pedaling was more challenging, like hills, a press of the Turbo
Boost button provided an extra push.

The Ride1Up bike was less intuitive. The control panel on the handlebar
lets you choose from nine pedal-assist levels. Level 3 felt sufficient
for getting me around the streets, but Level 5 felt better for getting
up hills. Sometimes, when trying to pedal from a stop, I forgot to lower
the pedal assist from Level 5, which caused the bike to jerk forward.
That was a bit scary.

Ride1Up offers a
\href{https://www.youtube.com/watch?v=12ifbdLegvw}{YouTube tutorial on
advanced settings} for people to adjust the power of each pedal-assist
level. Eventually, I reduced the power output for Levels 4 and 5, which
made pedaling smoother.

As for the Ride1Up's throttle, which is a trigger on the left handlebar,
it was nice to have the option to accelerate without pedaling when I was
getting exhausted. It did feel like cheating, though.

\hypertarget{the-downsides}{%
\subsection{The downsides.}\label{the-downsides}}

Image

The VanMoof weighs about 41 pounds. And the Ride1Up? About
55.Credit...Jim Wilson/The New York Times

Testing the two e-bikes underlined some of their trade-offs.

\begin{itemize}
\item
  \textbf{E-bikes are heavy.} The VanMoof weighs about 41 pounds and the
  Ride1Up about 55 pounds --- more than double the average road bike,
  which weighs about 20 pounds. You probably won't want an e-bike if
  you'd have to regularly carry it up many flights of stairs.
\item
  \textbf{Maintenance may be tricky.} VanMoof and Ride1Up said their
  bikes were designed to be user-serviceable, and any local bike
  mechanic should also be able to service minor parts, like brake pads.

  But with e-bikes in general, you may need to seek help from the maker
  if something major goes wrong with proprietary electronic components.
  It's a safer bet to buy your e-bike from a local store that can
  service it.
\item
  \textbf{They may attract burglars.} Parking the VanMoof made me
  anxious. Whenever I was locking it up, it got lots of attention from
  passers-by --- it looks like an elegantly designed tech product.

  A VanMoof spokesman said that up to 20 of its bikes are reported
  stolen each month worldwide, and that 70 percent are found within two
  weeks. So make sure to have renters or home insurance that covers the
  theft of e-bikes. (VanMoof offers its own three-year insurance for
  \$340.)
\item
  \textbf{Batteries are expensive.} Like smartphones, e-bikes use
  consumable batteries that eventually need to be replaced. With regular
  riding, the batteries for the VanMoof and the Ride1Up may deplete in
  three to five years. Replacements cost roughly \$350.
\end{itemize}

\hypertarget{but-the-pros-outweigh-the-cons}{%
\subsection{But the pros outweigh the
cons.}\label{but-the-pros-outweigh-the-cons}}

Despite some misgivings, my experience with e-bikes made me realize the
benefits are far greater than the downsides.

Most important, e-bikes kept me out of my car. Whenever I had a reason
to go outside --- like making a trip to the grocery store or dropping
off baked goods at a friend's --- I preferred riding an e-bike.

This will become increasingly important in the coming months. As
businesses reopen, the
\href{https://www.nytimes.com/2020/05/28/health/cdc-coronavirus-offices.html}{Centers
for Disease Control and Prevention has advised commuters to drive in
cars alone}. An e-bike may become crucial for squeezing through
nightmare traffic.

There's another benefit, which is important in hard times: E-bikes bring
joy. I'm no fan of cycling in San Francisco, but on an e-bike, I saw
more of the outdoors than I normally would, while keeping a safe
distance from people. That beat bingeing on Netflix.

So I'll probably buy an e-bike soon, even if it means getting on a
waiting list. I figure we could all use a little more joy.

Advertisement

\protect\hyperlink{after-bottom}{Continue reading the main story}

\hypertarget{site-index}{%
\subsection{Site Index}\label{site-index}}

\hypertarget{site-information-navigation}{%
\subsection{Site Information
Navigation}\label{site-information-navigation}}

\begin{itemize}
\tightlist
\item
  \href{https://help.nytimes.com/hc/en-us/articles/115014792127-Copyright-notice}{©~2020~The
  New York Times Company}
\end{itemize}

\begin{itemize}
\tightlist
\item
  \href{https://www.nytco.com/}{NYTCo}
\item
  \href{https://help.nytimes.com/hc/en-us/articles/115015385887-Contact-Us}{Contact
  Us}
\item
  \href{https://www.nytco.com/careers/}{Work with us}
\item
  \href{https://nytmediakit.com/}{Advertise}
\item
  \href{http://www.tbrandstudio.com/}{T Brand Studio}
\item
  \href{https://www.nytimes.com/privacy/cookie-policy\#how-do-i-manage-trackers}{Your
  Ad Choices}
\item
  \href{https://www.nytimes.com/privacy}{Privacy}
\item
  \href{https://help.nytimes.com/hc/en-us/articles/115014893428-Terms-of-service}{Terms
  of Service}
\item
  \href{https://help.nytimes.com/hc/en-us/articles/115014893968-Terms-of-sale}{Terms
  of Sale}
\item
  \href{https://spiderbites.nytimes.com}{Site Map}
\item
  \href{https://help.nytimes.com/hc/en-us}{Help}
\item
  \href{https://www.nytimes.com/subscription?campaignId=37WXW}{Subscriptions}
\end{itemize}
