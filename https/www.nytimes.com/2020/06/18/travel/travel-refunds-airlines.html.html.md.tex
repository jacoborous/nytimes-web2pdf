Sections

SEARCH

\protect\hyperlink{site-content}{Skip to
content}\protect\hyperlink{site-index}{Skip to site index}

\href{https://www.nytimes.com/section/travel}{Travel}

\href{https://myaccount.nytimes.com/auth/login?response_type=cookie\&client_id=vi}{}

\href{https://www.nytimes.com/section/todayspaper}{Today's Paper}

\href{/section/travel}{Travel}\textbar{}Help! It's Been Months. I'm
Still in a Travel Mess.

\url{https://nyti.ms/2YIWQVM}

\begin{itemize}
\item
\item
\item
\item
\item
\end{itemize}

\href{https://www.nytimes.com/news-event/coronavirus?action=click\&pgtype=Article\&state=default\&region=TOP_BANNER\&context=storylines_menu}{The
Coronavirus Outbreak}

\begin{itemize}
\tightlist
\item
  live\href{https://www.nytimes.com/2020/08/01/world/coronavirus-covid-19.html?action=click\&pgtype=Article\&state=default\&region=TOP_BANNER\&context=storylines_menu}{Latest
  Updates}
\item
  \href{https://www.nytimes.com/interactive/2020/us/coronavirus-us-cases.html?action=click\&pgtype=Article\&state=default\&region=TOP_BANNER\&context=storylines_menu}{Maps
  and Cases}
\item
  \href{https://www.nytimes.com/interactive/2020/science/coronavirus-vaccine-tracker.html?action=click\&pgtype=Article\&state=default\&region=TOP_BANNER\&context=storylines_menu}{Vaccine
  Tracker}
\item
  \href{https://www.nytimes.com/interactive/2020/07/29/us/schools-reopening-coronavirus.html?action=click\&pgtype=Article\&state=default\&region=TOP_BANNER\&context=storylines_menu}{What
  School May Look Like}
\item
  \href{https://www.nytimes.com/live/2020/07/31/business/stock-market-today-coronavirus?action=click\&pgtype=Article\&state=default\&region=TOP_BANNER\&context=storylines_menu}{Economy}
\end{itemize}

Advertisement

\protect\hyperlink{after-top}{Continue reading the main story}

Supported by

\protect\hyperlink{after-sponsor}{Continue reading the main story}

Tripped Up

\hypertarget{help-its-been-months-im-still-in-a-travel-mess}{%
\section{Help! It's Been Months. I'm Still in a Travel
Mess.}\label{help-its-been-months-im-still-in-a-travel-mess}}

Unclear policies, confusing customer-service protocols and
not-yet-fulfilled refunds continue to be an issue. Our columnist sees
what she can do.

\includegraphics{https://static01.nyt.com/images/2020/06/18/travel/18tripped-up-rev/18tripped-up-rev-articleLarge.jpg?quality=75\&auto=webp\&disable=upscale}

By Sarah Firshein

\begin{itemize}
\item
  June 18, 2020
\item
  \begin{itemize}
  \item
  \item
  \item
  \item
  \item
  \end{itemize}
\end{itemize}

I've received thousands of emails from Times readers since the
coronavirus put the brakes on travel in mid-March. And if there's one
thing that's obvious, it's that the aftershocks of canceled trips
persist. Unclear policies, confusing customer-service protocols and
not-yet-fulfilled refunds continue to be an issue --- months later, and
even as the world reopens.

This week, I dipped into my email inbox, grab-bag style, for three
questions to tackle in brief. Have your own? Ask me at
\href{mailto:travel@nytimes.com}{\nolinkurl{travel@nytimes.com}}.

\hypertarget{dear-tripped-up}{%
\subsubsection{\texorpdfstring{\textbf{Dear Tripped
Up,}}{Dear Tripped Up,}}\label{dear-tripped-up}}

In early February, I used Orbitz to book a round-trip Qantas Airways
flight from San Francisco to Melbourne, scheduled to depart in April.
The flight was canceled. Who should one lobby for refunds in this case:
the online travel agency or the airline? Meredith

\hypertarget{hi-meredith}{%
\subsubsection{Hi Meredith,}\label{hi-meredith}}

I've gotten a lot of questions about this issue. Many readers report
being herded like cattle from an online travel agency to an airline, and
back --- then around again.

Always start with the company that sold you the ticket. O.T.A.s like
Orbitz can change most reservations (save for certain types on some
low-cost carriers), so you're well within reason to let the customer
service representative do your bidding with the airline.

\hypertarget{latest-updates-global-coronavirus-outbreak}{%
\section{\texorpdfstring{\href{https://www.nytimes.com/2020/08/01/world/coronavirus-covid-19.html?action=click\&pgtype=Article\&state=default\&region=MAIN_CONTENT_1\&context=storylines_live_updates}{Latest
Updates: Global Coronavirus
Outbreak}}{Latest Updates: Global Coronavirus Outbreak}}\label{latest-updates-global-coronavirus-outbreak}}

Updated 2020-08-02T07:33:01.580Z

\begin{itemize}
\tightlist
\item
  \href{https://www.nytimes.com/2020/08/01/world/coronavirus-covid-19.html?action=click\&pgtype=Article\&state=default\&region=MAIN_CONTENT_1\&context=storylines_live_updates\#link-34047410}{The
  U.S. reels as July cases more than double the total of any other
  month.}
\item
  \href{https://www.nytimes.com/2020/08/01/world/coronavirus-covid-19.html?action=click\&pgtype=Article\&state=default\&region=MAIN_CONTENT_1\&context=storylines_live_updates\#link-780ec966}{Top
  U.S. officials work to break an impasse over the federal jobless
  benefit.}
\item
  \href{https://www.nytimes.com/2020/08/01/world/coronavirus-covid-19.html?action=click\&pgtype=Article\&state=default\&region=MAIN_CONTENT_1\&context=storylines_live_updates\#link-2bc8948}{Its
  outbreak untamed, Melbourne goes into even greater lockdown.}
\end{itemize}

\href{https://www.nytimes.com/2020/08/01/world/coronavirus-covid-19.html?action=click\&pgtype=Article\&state=default\&region=MAIN_CONTENT_1\&context=storylines_live_updates}{See
more updates}

More live coverage:
\href{https://www.nytimes.com/live/2020/07/31/business/stock-market-today-coronavirus?action=click\&pgtype=Article\&state=default\&region=MAIN_CONTENT_1\&context=storylines_live_updates}{Markets}

I've
\href{https://www.nytimes.com/2020/05/01/travel/trip-refund-airlines.html}{said
it before} and I'll say it again: When a flight to, from or within the
United States is canceled because of the pandemic, you're entitled to a
cash refund. And if you're not getting that as an initial option from
sites like Orbitz, it's because they can only process refunds after
they've been granted that right by the end provider; here, the airline.
That's why
\href{https://www.nytimes.com/2020/04/03/travel/coronavirus-refund-travel-ota.html?smtyp=cur\&smid=tw-nytimestravelhttps://www.nytimes.com/2020/04/03/travel/coronavirus-refund-travel-ota.html}{getting
refunds from}O.T.A.s can be complicated, especially during this global
crisis. These agencies are squished between an unprecedented number of
consumer requests and airlines that stall or stonewall --- or perhaps
never pick up the phone. If you're hitting roadblocks like these, ask
that your case be escalated (Orbitz and other O.T.A.s have a hierarchy
of urgency). Don't be shy about following up.

That persistence seemed to work for you, even without my help: By the
time I emailed Orbitz, a spokeswoman confirmed that the refund was
already in process. Yes, it's now June, and although call volumes in
March and April have leveled off for most agencies by now, the current
environment is still far from normal --- patience is key.

\hypertarget{dear-tripped-up-1}{%
\subsubsection{\texorpdfstring{\textbf{Dear Tripped
Up,}}{Dear Tripped Up,}}\label{dear-tripped-up-1}}

I have a flight to France coming up next month. Delta rebooked me on a
different flight on my original departure date, which seemed fine --- I
understand flexibility is key nowadays. But when I hopped online to
double-check my options, I didn't see the new flight on Delta's
schedule. In fact, it didn't appear until a few days later. What's the
story? Sabine

\hypertarget{hi-sabine}{%
\subsubsection{Hi Sabine,}\label{hi-sabine}}

Although airline technology has improved in recent years, the pandemic
has placed extraordinary stress on the system.

Even as the world begins to open up, flight schedules will be slimmer
than normal this summer. Delta, like all airlines, has been compressing
its schedule into fewer flights, so it's possible that the flight you
were rebooked on was kept ``off-sale'' to give already-booked passengers
dibs before being made available to new passengers. It's also possible
that the new flight was full for a period of time, during which time it
wasn't appearing for sale online.

There are complexities to publishing and adjusting airline schedules,
for sure, but there is one upside to a system that's so in flux: mistake
fares, or bargain-basement price glitches, that eagle-eyed consumers can
score before they're corrected. There are several websites that comb for
these deals, including \href{https://www.secretflying.com/}{Secret
Flying,} \href{https://scottscheapflights.com/}{Scott's Cheap Flights}
and \href{https://www.airfarewatchdog.com/}{Airfarewatchdog.}

\hypertarget{dear-tripped-up-2}{%
\subsubsection{\texorpdfstring{\textbf{Dear Tripped
Up,}}{Dear Tripped Up,}}\label{dear-tripped-up-2}}

I paid a \$125 ``miles redeposit fee'' when I canceled my United
Airlines reward flight a couple of months ago. How can I get that money
back? Jen

\hypertarget{hi-jen}{%
\subsubsection{Hi Jen,}\label{hi-jen}}

Save for the famously flexible Southwest, most airlines have long
imposed some sort of fee --- usually between \$75 and \$150 --- when
passengers change or cancel tickets booked with miles. These so-called
``restocking fees'' tend to lessen or disappear with elite status.

\href{https://www.nytimes.com/news-event/coronavirus?action=click\&pgtype=Article\&state=default\&region=MAIN_CONTENT_3\&context=storylines_faq}{}

\hypertarget{the-coronavirus-outbreak-}{%
\subsubsection{The Coronavirus Outbreak
›}\label{the-coronavirus-outbreak-}}

\hypertarget{frequently-asked-questions}{%
\paragraph{Frequently Asked
Questions}\label{frequently-asked-questions}}

Updated July 27, 2020

\begin{itemize}
\item ~
  \hypertarget{should-i-refinance-my-mortgage}{%
  \paragraph{Should I refinance my
  mortgage?}\label{should-i-refinance-my-mortgage}}

  \begin{itemize}
  \tightlist
  \item
    \href{https://www.nytimes.com/article/coronavirus-money-unemployment.html?action=click\&pgtype=Article\&state=default\&region=MAIN_CONTENT_3\&context=storylines_faq}{It
    could be a good idea,} because mortgage rates have
    \href{https://www.nytimes.com/2020/07/16/business/mortgage-rates-below-3-percent.html?action=click\&pgtype=Article\&state=default\&region=MAIN_CONTENT_3\&context=storylines_faq}{never
    been lower.} Refinancing requests have pushed mortgage applications
    to some of the highest levels since 2008, so be prepared to get in
    line. But defaults are also up, so if you're thinking about buying a
    home, be aware that some lenders have tightened their standards.
  \end{itemize}
\item ~
  \hypertarget{what-is-school-going-to-look-like-in-september}{%
  \paragraph{What is school going to look like in
  September?}\label{what-is-school-going-to-look-like-in-september}}

  \begin{itemize}
  \tightlist
  \item
    It is unlikely that many schools will return to a normal schedule
    this fall, requiring the grind of
    \href{https://www.nytimes.com/2020/06/05/us/coronavirus-education-lost-learning.html?action=click\&pgtype=Article\&state=default\&region=MAIN_CONTENT_3\&context=storylines_faq}{online
    learning},
    \href{https://www.nytimes.com/2020/05/29/us/coronavirus-child-care-centers.html?action=click\&pgtype=Article\&state=default\&region=MAIN_CONTENT_3\&context=storylines_faq}{makeshift
    child care} and
    \href{https://www.nytimes.com/2020/06/03/business/economy/coronavirus-working-women.html?action=click\&pgtype=Article\&state=default\&region=MAIN_CONTENT_3\&context=storylines_faq}{stunted
    workdays} to continue. California's two largest public school
    districts --- Los Angeles and San Diego --- said on July 13, that
    \href{https://www.nytimes.com/2020/07/13/us/lausd-san-diego-school-reopening.html?action=click\&pgtype=Article\&state=default\&region=MAIN_CONTENT_3\&context=storylines_faq}{instruction
    will be remote-only in the fall}, citing concerns that surging
    coronavirus infections in their areas pose too dire a risk for
    students and teachers. Together, the two districts enroll some
    825,000 students. They are the largest in the country so far to
    abandon plans for even a partial physical return to classrooms when
    they reopen in August. For other districts, the solution won't be an
    all-or-nothing approach.
    \href{https://bioethics.jhu.edu/research-and-outreach/projects/eschool-initiative/school-policy-tracker/}{Many
    systems}, including the nation's largest, New York City, are
    devising
    \href{https://www.nytimes.com/2020/06/26/us/coronavirus-schools-reopen-fall.html?action=click\&pgtype=Article\&state=default\&region=MAIN_CONTENT_3\&context=storylines_faq}{hybrid
    plans} that involve spending some days in classrooms and other days
    online. There's no national policy on this yet, so check with your
    municipal school system regularly to see what is happening in your
    community.
  \end{itemize}
\item ~
  \hypertarget{is-the-coronavirus-airborne}{%
  \paragraph{Is the coronavirus
  airborne?}\label{is-the-coronavirus-airborne}}

  \begin{itemize}
  \tightlist
  \item
    The coronavirus
    \href{https://www.nytimes.com/2020/07/04/health/239-experts-with-one-big-claim-the-coronavirus-is-airborne.html?action=click\&pgtype=Article\&state=default\&region=MAIN_CONTENT_3\&context=storylines_faq}{can
    stay aloft for hours in tiny droplets in stagnant air}, infecting
    people as they inhale, mounting scientific evidence suggests. This
    risk is highest in crowded indoor spaces with poor ventilation, and
    may help explain super-spreading events reported in meatpacking
    plants, churches and restaurants.
    \href{https://www.nytimes.com/2020/07/06/health/coronavirus-airborne-aerosols.html?action=click\&pgtype=Article\&state=default\&region=MAIN_CONTENT_3\&context=storylines_faq}{It's
    unclear how often the virus is spread} via these tiny droplets, or
    aerosols, compared with larger droplets that are expelled when a
    sick person coughs or sneezes, or transmitted through contact with
    contaminated surfaces, said Linsey Marr, an aerosol expert at
    Virginia Tech. Aerosols are released even when a person without
    symptoms exhales, talks or sings, according to Dr. Marr and more
    than 200 other experts, who
    \href{https://academic.oup.com/cid/article/doi/10.1093/cid/ciaa939/5867798}{have
    outlined the evidence in an open letter to the World Health
    Organization}.
  \end{itemize}
\item ~
  \hypertarget{what-are-the-symptoms-of-coronavirus}{%
  \paragraph{What are the symptoms of
  coronavirus?}\label{what-are-the-symptoms-of-coronavirus}}

  \begin{itemize}
  \tightlist
  \item
    Common symptoms
    \href{https://www.nytimes.com/article/symptoms-coronavirus.html?action=click\&pgtype=Article\&state=default\&region=MAIN_CONTENT_3\&context=storylines_faq}{include
    fever, a dry cough, fatigue and difficulty breathing or shortness of
    breath.} Some of these symptoms overlap with those of the flu,
    making detection difficult, but runny noses and stuffy sinuses are
    less common.
    \href{https://www.nytimes.com/2020/04/27/health/coronavirus-symptoms-cdc.html?action=click\&pgtype=Article\&state=default\&region=MAIN_CONTENT_3\&context=storylines_faq}{The
    C.D.C. has also} added chills, muscle pain, sore throat, headache
    and a new loss of the sense of taste or smell as symptoms to look
    out for. Most people fall ill five to seven days after exposure, but
    symptoms may appear in as few as two days or as many as 14 days.
  \end{itemize}
\item ~
  \hypertarget{does-asymptomatic-transmission-of-covid-19-happen}{%
  \paragraph{Does asymptomatic transmission of Covid-19
  happen?}\label{does-asymptomatic-transmission-of-covid-19-happen}}

  \begin{itemize}
  \tightlist
  \item
    So far, the evidence seems to show it does. A widely cited
    \href{https://www.nature.com/articles/s41591-020-0869-5}{paper}
    published in April suggests that people are most infectious about
    two days before the onset of coronavirus symptoms and estimated that
    44 percent of new infections were a result of transmission from
    people who were not yet showing symptoms. Recently, a top expert at
    the World Health Organization stated that transmission of the
    coronavirus by people who did not have symptoms was ``very rare,''
    \href{https://www.nytimes.com/2020/06/09/world/coronavirus-updates.html?action=click\&pgtype=Article\&state=default\&region=MAIN_CONTENT_3\&context=storylines_faq\#link-1f302e21}{but
    she later walked back that statement.}
  \end{itemize}
\end{itemize}

Although commonplace in normal times, these fees felt particularly
out-of-step at the start of the pandemic. Facing increased pressure in
March, many carriers waived them; United took a bit longer than some
others to reverse its policy but eventually fell in line. On June 1, the
airline announced that it will continue to waive redeposit fees on all
award tickets with 2020 departure dates.

Refunds and waivers
\href{https://www.nytimes.com/2020/04/11/travel/coronavirus-travel-trip-refunds.html}{have
been a moving target} since the pandemic started, and if you got
ensnared in an unfavorable policy before it changed for the better, you
have plenty of company. If you're committed to getting that restocking
fee reversed, your best bet is to call the airline --- a United
spokeswoman wouldn't commit to this as an across-the-board option, but
she said the company will address the issue on a case-by-case basis.

\begin{center}\rule{0.5\linewidth}{\linethickness}\end{center}

\href{https://twitter.com/sfirshein?lang=en}{Sarah Firshein} is a
Brooklyn-based writer. If you need advice about a best-laid travel plan
that went awry, \textbf{\href{mailto:travel@nytimes.com}{send an email
to travel@nytimes.com}.}

\begin{center}\rule{0.5\linewidth}{\linethickness}\end{center}

\textbf{LET'S GO, SAFELY.} \emph{Discover more Travel coverage by
following us on}
\href{https://twitter.com/nytimestravel}{\emph{Twitter}} \emph{and}
\href{https://www.facebook.com/nytimestravel/}{\emph{Facebook}}\emph{.
And}
\href{https://www.nytimes.com/newsletters/traveldispatch?action=click\&module=inline\&pgtype=Article}{\emph{sign
up for our}} **
\href{https://www.nytimes.com/newsletters/traveldispatch}{\emph{Travel
Dispatch newsletter}}\emph{: Each week you'll receive tips on traveling
smarter, stories on hot destinations and access to photos from all over
the world.}

Advertisement

\protect\hyperlink{after-bottom}{Continue reading the main story}

\hypertarget{site-index}{%
\subsection{Site Index}\label{site-index}}

\hypertarget{site-information-navigation}{%
\subsection{Site Information
Navigation}\label{site-information-navigation}}

\begin{itemize}
\tightlist
\item
  \href{https://help.nytimes.com/hc/en-us/articles/115014792127-Copyright-notice}{©~2020~The
  New York Times Company}
\end{itemize}

\begin{itemize}
\tightlist
\item
  \href{https://www.nytco.com/}{NYTCo}
\item
  \href{https://help.nytimes.com/hc/en-us/articles/115015385887-Contact-Us}{Contact
  Us}
\item
  \href{https://www.nytco.com/careers/}{Work with us}
\item
  \href{https://nytmediakit.com/}{Advertise}
\item
  \href{http://www.tbrandstudio.com/}{T Brand Studio}
\item
  \href{https://www.nytimes.com/privacy/cookie-policy\#how-do-i-manage-trackers}{Your
  Ad Choices}
\item
  \href{https://www.nytimes.com/privacy}{Privacy}
\item
  \href{https://help.nytimes.com/hc/en-us/articles/115014893428-Terms-of-service}{Terms
  of Service}
\item
  \href{https://help.nytimes.com/hc/en-us/articles/115014893968-Terms-of-sale}{Terms
  of Sale}
\item
  \href{https://spiderbites.nytimes.com}{Site Map}
\item
  \href{https://help.nytimes.com/hc/en-us}{Help}
\item
  \href{https://www.nytimes.com/subscription?campaignId=37WXW}{Subscriptions}
\end{itemize}
