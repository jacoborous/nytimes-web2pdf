Sections

SEARCH

\protect\hyperlink{site-content}{Skip to
content}\protect\hyperlink{site-index}{Skip to site index}

\href{https://www.nytimes.com/section/arts/design}{Art \& Design}

\href{https://myaccount.nytimes.com/auth/login?response_type=cookie\&client_id=vi}{}

\href{https://www.nytimes.com/section/todayspaper}{Today's Paper}

\href{/section/arts/design}{Art \& Design}\textbar{}Digital Field Trips:
Museum Adventures Abound for Kids

\url{https://nyti.ms/3efxqWf}

\begin{itemize}
\item
\item
\item
\item
\item
\item
\end{itemize}

\href{https://www.nytimes.com/spotlight/at-home?action=click\&pgtype=Article\&state=default\&region=TOP_BANNER\&context=at_home_menu}{At
Home}

\begin{itemize}
\tightlist
\item
  \href{https://www.nytimes.com/2020/07/28/books/time-for-a-literary-road-trip.html?action=click\&pgtype=Article\&state=default\&region=TOP_BANNER\&context=at_home_menu}{Take:
  A Literary Road Trip}
\item
  \href{https://www.nytimes.com/2020/07/29/magazine/bored-with-your-home-cooking-some-smoky-eggplant-will-fix-that.html?action=click\&pgtype=Article\&state=default\&region=TOP_BANNER\&context=at_home_menu}{Cook:
  Smoky Eggplant}
\item
  \href{https://www.nytimes.com/2020/07/27/travel/moose-michigan-isle-royale.html?action=click\&pgtype=Article\&state=default\&region=TOP_BANNER\&context=at_home_menu}{Look
  Out: For Moose}
\item
  \href{https://www.nytimes.com/interactive/2020/at-home/even-more-reporters-editors-diaries-lists-recommendations.html?action=click\&pgtype=Article\&state=default\&region=TOP_BANNER\&context=at_home_menu}{Explore:
  Reporters' Obsessions}
\end{itemize}

Advertisement

\protect\hyperlink{after-top}{Continue reading the main story}

Supported by

\protect\hyperlink{after-sponsor}{Continue reading the main story}

Summer Guide for Families

\hypertarget{digital-field-trips-museum-adventures-abound-for-kids}{%
\section{Digital Field Trips: Museum Adventures Abound for
Kids}\label{digital-field-trips-museum-adventures-abound-for-kids}}

Cultural institutions are finding creative ways to engage young visitors
virtually this summer, and many of the offerings are free.

\includegraphics{https://static01.nyt.com/images/2020/06/19/arts/00kids-museums-carle/merlin_173591583_24d13241-5363-46ca-9d16-1c206bcb1e7b-articleLarge.jpg?quality=75\&auto=webp\&disable=upscale}

By \href{https://www.nytimes.com/by/laurel-graeber}{Laurel Graeber}

\begin{itemize}
\item
  Published June 18, 2020Updated June 22, 2020
\item
  \begin{itemize}
  \item
  \item
  \item
  \item
  \item
  \item
  \end{itemize}
\end{itemize}

Museums have become extraordinarily creative in throwing open their
virtual doors to young people still on lockdown. Educators are providing
at-home opportunities to emulate renowned artists, go on odysseys to the
stars, collaboratively create a picture book on women's history and even
chill out with a skink. Here's a selection of offerings, many of them
free.

\hypertarget{childrens-museums}{%
\subsection{Children's Museums}\label{childrens-museums}}

\includegraphics{https://static01.nyt.com/images/2020/06/19/arts/00kids-museums-Childrens/merlin_173583141_8b007823-1ac6-4aa3-8710-31a72474c212-articleLarge.jpg?quality=75\&auto=webp\&disable=upscale}

Almost as soon as quarantine began, the
\textbf{\href{https://cmom.org/}{Children's Museum of Manhattan}}
instituted \href{https://athome.cmom.org/}{CMOM at Home}, a daily series
with themes --- from Magical Monday to Surprise Sunday --- and related
projects and videos. The over 80 selections now online include
instructions for doing
\href{https://athome.cmom.org/join-the-dinosaur-march-with-laurie-berkner/}{a
dinosaur march} with the musician Laurie Berkner and
\href{https://athome.cmom.org/ways-to-say-hello-with-mumu-fresh-and-callaloo-kids/}{saying
hello in multiple languages} with the organization
\href{http://www.callalookids.com/}{Callaloo Kids}.

``We're showing the world to children every day --- different ways of
cooking or dancing or talking,'' said Leslie Bushara, the museum's
deputy director for education and exhibitions. The CMOM at Home on June
28 will celebrate Pride with the band Queer Kids and a rainbow-wand art
project. A later episode will focus on the artist
\href{https://www.youtube.com/watch?v=A90E3QuF0Hw}{Delano Dunn}. ``It
will be an activity where children can explore racial identity,'' Ms.
Bushara said.

\textbf{\href{https://www.sugarhillmuseum.org/}{The Sugar Hill
Children's Museum of Art \& Storytelling}} also plans to raise young
people's social consciousness. On July 28,
\href{https://www.nytimes.com/2018/10/25/arts/design/sugar-hill-museum-art-to-new-yorks-youngest.html}{this
Manhattan institution} will commemorate the anniversary of one of the
earliest civil rights marches,
\href{https://naacp.org/silent-protest-parade-centennial/}{the Silent
Protest Parade of 1917} (in response to deadly attacks on black
residents by white mobs in Illinois), with the virtual ``Children's Art
Exhibition for Justice.'' The event will incorporate a video history of
the march, works by children, spoken-word pieces and art collaborations
by preteens, teenagers and the artist Dionis Ortiz.

Parents can find more ways to help their children understand the efforts
to end inequality in the
\href{https://cmany.org/blog/view/online-resources-families-regarding-racial-injustice/}{Online
Resources for Families Regarding Racial Injustice}, posted by the
\textbf{\href{https://cmany.org/}{Children's Museum of the Arts}}. This
New York museum, which still has spaces in its weeklong
\href{https://cmany.org/classes-and-activities/art-colony/summer-103-charlton/}{Online
Summer Art Colony Camps}, also offers free video art lessons, craft
projects and storybook readings. Next Thursday it will host a
\href{https://cmany.org/classes-and-activities/events/rico-gatson-virtual-studio-tour/}{low-cost
virtual tour} of the artist Rico Gatson's studio.

And don't forget web pages of activities like
\href{https://www.chicagochildrensmuseum.org/recipes-for-play-at-home}{Recipes
for Play at Home}, from the
\textbf{\href{https://www.chicagochildrensmuseum.org/}{Chicago
Children's Museum}};
\href{https://sichildrensmuseum.org/at-home-with-sicm/}{At Home With
SICM}, from the \textbf{\href{https://sichildrensmuseum.org/}{Staten
Island Children's Museum}}; and
\href{https://www.brooklynkids.org/bcm-and-you/}{BCM + You}, from the
\textbf{\href{https://www.brooklynkids.org/}{Brooklyn Children's
Museum}}.

\hypertarget{art-museums}{%
\subsection{Art Museums}\label{art-museums}}

Image

An installation view of ``Liberty (Liberté),'' by Puppies Puppies. The
performance piece, featured in the 2017 Whitney Biennial, is among the
works children can engage with in the Whitney Kids Art
Challenge.Credit...Matthew Carasella

However old the art fan, virtual galleries await. **** On June 29,
\textbf{\href{https://www.moma.org/visit/families}{the Museum of Modern
Art}} will initiate the Samuel and Ronnie Heyman Family Art Lab at Home,
a series of creative videos and prompts based on MoMA's collection. The
\textbf{\href{https://whitney.org/}{Whitney Museum of American Art}}'s
\href{https://whitney.org/families/kids-art-challenge}{Kids Art
Challenge}, which began in April, continues to add six projects every
two weeks: Click on a Whitney work and discover ways to explore its
themes.

``We tried for a mix of old favorites, like
\href{https://whitney.org/families/kids-art-challenge/alexander-calder}{Calder's
`Circus},' as well as newer works,'' like ``Liberty (Liberté),'' a
Statue of Liberty riff by the performance artist
\href{https://www.artspace.com/magazine/interviews_features/material-art-fair-2016/puppies-puppies-interview-53446}{Puppies
Puppies}, said Heather Maxson, the Whitney's director of school, youth
and family programs. The museum also offers
\href{https://whitney.org/whitney-from-home/hopper-coloring}{downloadable
images} of Edward Hopper's art to color; his ``Early Sunday Morning''
(1930) has inspired both
\href{https://whitney.org/families/kids-art-challenge/edward-hopper}{an
art challenge} and a coloring page.

On July 6, Ms. Maxson will introduce Whitney Summer Studio, a six-week
program of free 40-minute Zoom art classes, with a family session on
Saturdays. ``We're going to work on ways to connect families that are
separated by distance,'' she said, ``so you can work with your grandma
in Florida.''

\textbf{\href{https://www.guggenheim.org/}{The Guggenheim Museum}}
created its interactive
\href{https://www.guggenheim.org/event/guggenheim-family-tours-at-home}{Family
Tours at Home}, on select summer Saturdays, with a similar goal in mind.
It has also turned
``\href{https://www.guggenheim.org/exhibition/a-year-with-children-2020}{A
Year With Children 2020},'' its annual show of New York City student
artwork, into an e-book.

Little bookworms will especially appreciate virtual visits to the
\textbf{\href{https://www.carlemuseum.org/}{Eric Carle Museum of Picture
Book Art}} in Amherst, Mass. Fans of Carle's titles like ``The Very
Hungry Caterpillar'' will devour this museum's offerings, among them
\href{https://sway.office.com/4gm5EH94jhEswJ7z}{a virtual exhibition}
featuring 21 children's book illustrators.

Finicky adolescents will find programs, too, including
\href{https://www.metmuseum.org/events/programs/teens/teen-studio}{collage
workshops} next month at the
\textbf{\href{https://www.metmuseum.org/}{Metropolitan Museum of Art}},
followed by a
\href{https://www.metmuseum.org/events/programs/teens/career-labs}{Career
Lab}. And the \textbf{\href{http://www.movingimage.us/}{Museum of the
Moving Image}} in Queens offers a summer-long bonanza: media camps, Town
Halls for Teen Media Makers and a
\href{http://www.movingimage.us/education/teencouncil}{Teen Film
Festival}. Both tweens and teens will enjoy the museum's continuing
series ``Jim Henson's World,'' which presents
\href{http://www.movingimage.us/visit/calendar/2020/06/20/detail/jim-hensons-world-new-visions-of-puppets-on-screen}{an
online conversation with four puppeteer-filmmakers} on Saturday.

\hypertarget{history-and-culture-museums}{%
\subsection{History and Culture
Museums}\label{history-and-culture-museums}}

Image

The Museum of Jewish Heritage --- a Living Memorial to the Holocaust has
online materials that include the HBO documentary ``The Number on
Great-Grandpa's Arm,''~about the relationship between a child and a
Holocaust survivor.~Credit...HBO

The Smithsonian's \textbf{\href{https://nmaahc.si.edu/}{National Museum
of African American History and Culture}}, in Washington, and its
\textbf{\href{https://americanindian.si.edu/}{National Museum of the
American Indian}}, in Manhattan and Washington, offer multimedia digital
materials illuminating the country's reckoning with racism. The
African-American
\href{https://www.nytimes.com/2020/06/11/arts/design/museums-protests-race-smithsonian.html}{museum's
new web portal},
\href{https://nmaahc.si.edu/learn/talking-about-race}{Talking About
Race}, is especially helpful in starting difficult conversations. ``We
talk a lot about the danger of silence,'' said Candra Flanagan, the
museum's director of teaching and learning. ``When we're silent, it
forces kids to figure things out on their own.''

This museum's online programs include a
\href{https://nmaahc.si.edu/event/juneteenth-celebration-resilience?trumbaEmbed=view\%3Devent\%26eventid\%3D145373621}{Juneteenth
celebration}, marking the end of slavery in the United States, on Friday
and Saturday, and a
\href{https://nmaahc.si.edu/event/nhd-nmaahc-virtual-student-documentary-showcase?trumbaEmbed=view\%3Devent\%26eventid\%3D143246785}{virtual
student documentary showcase}, which runs through Wednesday. From Aug. 3
to 7, the museum will host
\href{https://nmaahc.si.edu/learn/students/young-historians-institute-virtual-remix}{Young
Historians Institute: The Virtual Remix}. A selective program for high
school students that requires applications and tuition, it will explore
the African-American experience in the Revolutionary era.

In New York, \textbf{\href{https://mjhnyc.org/}{the Museum of Jewish
Heritage --- a Living Memorial to the Holocaust}} presents its own
online resources to introduce children gently to a devastating history.
One highlight, the HBO documentary
``\href{https://www.nytimes.com/2018/04/19/arts/events-for-children-in-nyc-this-week.html}{The
Number on Great-Grandpa's Arm},'' offers an uplifting look at a
relationship between a child and a Holocaust survivor.

More virtual avenues beckon at the
\textbf{\href{https://www.nyhistory.org/}{New-York Historical Society}},
whose free
\href{https://www.nyhistory.org/childrens-museum/family-programs?reading-history-home-high-five-glenn-burke/june/28/2020}{Reading
Into History @ Home} book club on June 28 hosts Phil Bildner, author of
``A High Five for Glenn Burke,'' about the first professional baseball
player to come out as gay. The society, which will continue its Living
History Zoom sessions with costumed interpreters, also offers
\href{https://www.nyhistory.org/childrens-museum/family-programs?camp-history-home-women-march/july/20/2020}{Camp
History @ Home} from July 20 to Aug. 13. It will center on the
suffragist exhibition
``\href{https://www.nyhistory.org/exhibitions/women-march}{Women
March}'' and the book ``Little Leaders: Bold Women in Black History,''
by Vashti Harrison. Participants will collaborate on a picture e-book of
women's rights activists (and receive a physical copy later).

\textbf{\href{https://www.tenement.org/}{The Tenement Museum}} ****
offers **** another window onto New York history. Known for using actors
to portray real immigrants on its historical properties, it is
\href{https://www.tenement.org/events/}{continuing the practice
virtually}. Twice this summer, children can visit 1916 with
\href{https://www.tenement.org/events/virtual-family-event-meet-victoria-confino-june-24/}{Victoria
Confino}, a Sephardic Jewish teenager who immigrated from Kastoria, a
city that today is part of Greece. Another program,
\href{https://www.tenement.org/events/virtual-family-event-building-a-community-july-8/}{Building
a Community}, uses video and oral history to introduce the Puerto Rican
\href{https://www.tenement.org/events/virtual-family-event-building-a-community-july-8/}{Saez
Velez family} in the 1950s and later.

\hypertarget{science-museums}{%
\subsection{Science Museums}\label{science-museums}}

Image

Green and orange Sharpshooters included in the Natural History Museum of
Los Angeles County's online exhibition ``Spiky, Hairy, Shiny: Insects of
L.A.''Credit...Lisa Gonzalez/BioSCAN, via Natural History Museums of Los
Angeles County

Image

A Torymid wasp is also included in the online exhibition, which has
360-degree views.Credit...Lisa Gonzalez/BioSCAN, via Natural History
Museums of Los Angeles County

``I truly believe one silver lining that will come out of this crisis
will be an entire generation of children with an increased interest in
science and innovation,'' Crystal Bowyer, the president and chief
executive of the new science-oriented
\textbf{\href{https://nationalchildrensmuseum.org/}{National Children's
Museum}} in Washington, said about the Covid-19 pandemic in an email.
``Children are home right now thinking about what they can do.''

This museum has just started STEAM Daydream, a monthly podcast whose
first episode,
``\href{https://nationalchildrensmuseum.org/podcast/}{Health Science
Heroes},'' focuses on global disease and the anxiety it causes.
\href{https://www.exploratorium.edu/learn}{Viruses and Us}, from the
\textbf{\href{https://www.exploratorium.edu/}{Exploratorium}} in San
Francisco, is a compilation of online videos and activities. The
\textbf{\href{https://nysci.org/}{New York Hall of Science}}'s **** many
web resources include **** a
\href{https://nysci.org/home/science-behind-coronavirus/}{virtual
coronavirus exhibition} in English and Spanish, as well as
``\href{https://nysci.org/school/resources/transmissions-gone-viral/}{Transmissions:
Gone Viral},'' an engrossing interactive graphic novel inspired by the
1999 West Nile outbreak.

Although not virus-related, the
``\href{https://lsc.org/education/lsc-in-the-house}{Live From Surgery}''
Facebook streams from the \textbf{\href{https://lsc.org/}{Liberty
Science Center}} in Jersey City are just as compelling. Those who aren't
squeamish can check the center's website for archived videos of a heart
transplant and a robotic procedure on a kidney.

Have outer space or wildlife enthusiasts at home? The
\textbf{\href{https://www.intrepidmuseum.org/}{Intrepid Sea, Air \&
Space Museum}} presents Virtual Astronomy Live every month --- with
opportunities to meet astronauts --- and multiple aviation- and
space-themed programs. At
the\href{https://www.amnh.org/}{}\textbf{\href{https://www.amnh.org/}{American
Museum of Natural History}}, ****
\href{https://www.amnh.org/explore}{virtual adventures} include live
YouTube watch parties like
\href{https://www.youtube.com/watch?v=lUCC6ae0XZA\&feature=youtu.be}{Field
Trip: Mapping the Universe}, on Friday, and an
\href{https://www.amnh.org/calendar/in-the-field-bats}{exploration of
bat biodiversity} on June 26. The museum's website and app for children,
\href{https://www.amnh.org/explore/ology}{OLogy}, also has enough games,
projects and videos to keep the young and the restless busy all summer.
Check out its zoology section to
\href{https://www.amnh.org/explore/ology/zoology/you-are-the-queen}{play
the role of a queen wasp} or learn what
\href{https://www.amnh.org/explore/ology/zoology/what-s-this-life-at-the-limits}{a
tardigrade} is.

Speaking of funky creatures, on Monday the
\textbf{\href{https://nhm.org/}{Natural History Museum of Los Angeles
County}} will begin offering a 360-degree tour of its online show
``\href{https://nhm.org/spiky-hairy-shiny-insects-la}{Spiky, Hairy,
Shiny: Insects of L.A}.,'' whose bugs appear in colorful close-ups. And
don't miss the museum's ``\href{https://nhmlac.org/connects}{Walk on the
Wild Side'' videos}, in which children can
\href{https://nhmlac.org/stories/chill-out-tallulah-skink}{meet that
skink}. Her name is Tallulah, and she's surprisingly sociable.

Advertisement

\protect\hyperlink{after-bottom}{Continue reading the main story}

\hypertarget{site-index}{%
\subsection{Site Index}\label{site-index}}

\hypertarget{site-information-navigation}{%
\subsection{Site Information
Navigation}\label{site-information-navigation}}

\begin{itemize}
\tightlist
\item
  \href{https://help.nytimes.com/hc/en-us/articles/115014792127-Copyright-notice}{©~2020~The
  New York Times Company}
\end{itemize}

\begin{itemize}
\tightlist
\item
  \href{https://www.nytco.com/}{NYTCo}
\item
  \href{https://help.nytimes.com/hc/en-us/articles/115015385887-Contact-Us}{Contact
  Us}
\item
  \href{https://www.nytco.com/careers/}{Work with us}
\item
  \href{https://nytmediakit.com/}{Advertise}
\item
  \href{http://www.tbrandstudio.com/}{T Brand Studio}
\item
  \href{https://www.nytimes.com/privacy/cookie-policy\#how-do-i-manage-trackers}{Your
  Ad Choices}
\item
  \href{https://www.nytimes.com/privacy}{Privacy}
\item
  \href{https://help.nytimes.com/hc/en-us/articles/115014893428-Terms-of-service}{Terms
  of Service}
\item
  \href{https://help.nytimes.com/hc/en-us/articles/115014893968-Terms-of-sale}{Terms
  of Sale}
\item
  \href{https://spiderbites.nytimes.com}{Site Map}
\item
  \href{https://help.nytimes.com/hc/en-us}{Help}
\item
  \href{https://www.nytimes.com/subscription?campaignId=37WXW}{Subscriptions}
\end{itemize}
