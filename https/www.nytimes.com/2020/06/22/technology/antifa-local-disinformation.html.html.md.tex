Sections

SEARCH

\protect\hyperlink{site-content}{Skip to
content}\protect\hyperlink{site-index}{Skip to site index}

\href{https://www.nytimes.com/section/technology}{Technology}

\href{https://myaccount.nytimes.com/auth/login?response_type=cookie\&client_id=vi}{}

\href{https://www.nytimes.com/section/todayspaper}{Today's Paper}

\href{/section/technology}{Technology}\textbar{}41 Cities, Many Sources:
How False Antifa Rumors Spread Locally

\url{https://nyti.ms/3eqEX4y}

\begin{itemize}
\item
\item
\item
\item
\item
\item
\end{itemize}

\href{https://www.nytimes.com/news-event/george-floyd-protests-minneapolis-new-york-los-angeles?action=click\&pgtype=Article\&state=default\&region=TOP_BANNER\&context=storylines_menu}{Race
and America}

\begin{itemize}
\tightlist
\item
  \href{https://www.nytimes.com/2020/07/26/us/protests-portland-seattle-trump.html?action=click\&pgtype=Article\&state=default\&region=TOP_BANNER\&context=storylines_menu}{Protesters
  Return to Other Cities}
\item
  \href{https://www.nytimes.com/2020/07/24/us/portland-oregon-protests-white-race.html?action=click\&pgtype=Article\&state=default\&region=TOP_BANNER\&context=storylines_menu}{Portland
  at the Center}
\item
  \href{https://www.nytimes.com/2020/07/23/podcasts/the-daily/portland-protests.html?action=click\&pgtype=Article\&state=default\&region=TOP_BANNER\&context=storylines_menu}{Podcast:
  Showdown in Portland}
\item
  \href{https://www.nytimes.com/interactive/2020/07/16/us/black-lives-matter-protests-louisville-breonna-taylor.html?action=click\&pgtype=Article\&state=default\&region=TOP_BANNER\&context=storylines_menu}{45
  Days in Louisville}
\end{itemize}

Advertisement

\protect\hyperlink{after-top}{Continue reading the main story}

Supported by

\protect\hyperlink{after-sponsor}{Continue reading the main story}

\hypertarget{41-cities-many-sources-how-false-antifa-rumors-spread-locally}{%
\section{41 Cities, Many Sources: How False Antifa Rumors Spread
Locally}\label{41-cities-many-sources-how-false-antifa-rumors-spread-locally}}

Claims about the involvement of anti-fascist activists in protests of
racism show the many ways false information spreads inside communities
online.

\includegraphics{https://static01.nyt.com/images/2020/06/18/business/00antifa-misinfo/merlin_173055924_e85d731a-1438-4cc3-8ae2-b8b68f63614d-articleLarge.jpg?quality=75\&auto=webp\&disable=upscale}

By \href{https://www.nytimes.com/by/davey-alba}{Davey Alba} and Ben
Decker

\begin{itemize}
\item
  June 22, 2020
\item
  \begin{itemize}
  \item
  \item
  \item
  \item
  \item
  \item
  \end{itemize}
\end{itemize}

In recent weeks, as demonstrations against racism spread across the
country, residents in at least 41 U.S. cities and towns became alarmed
by rumors that the loose collective of anti-fascist activists known as
antifa was headed to their area, according to an analysis by The New
York Times. In many cases, they contacted their local law enforcement
for help.

In each case, it was for
\href{https://www.nytimes.com/2020/06/11/us/antifa-protests-george-floyd.html}{a
threat that never appeared}.

President Trump has spread some unfounded rumors about antifa to a
national audience --- including his accusation, without evidence, that a
75-year-old Buffalo protester who was hospitalized after being knocked
down by a police officer could be
``\href{https://www.nytimes.com/2020/06/09/nyregion/who-is-martin-gugino-buffalo-police.html}{an
antifa provocateur.}''

But on the local level, the source of the false information has usually
been more subtle, and shows the complexity of stunting misinformation
online. The bad information often first appears in a Twitter or Facebook
post, or a YouTube video there. It is then shared on online spaces like
local Facebook groups, the neighborhood social networking app Nextdoor
and community texting networks. These posts can fall under the radar of
the tech companies and online fact checkers.

``The dynamic is tricky because many times these local groups don't have
much prior awareness of the body of conspiratorial content surrounding
some of these topics,'' said Renée DiResta, a disinformation researcher
at the Stanford Internet Observatory. ``The first thing they see is a
trusted fellow community member giving them a warning.''

Here are four ways that antifa falsehoods spread in local communities.

\hypertarget{after-one-tweet-dozens-of-calls-to-the-police}{%
\subsection{After One Tweet, Dozens of Calls to the
Police}\label{after-one-tweet-dozens-of-calls-to-the-police}}

\includegraphics{https://static01.nyt.com/images/2020/06/18/business/00antifa-misinfo-siouxfalls/merlin_173672286_90c2f87a-e961-46f5-a89c-faf76a3d993f-articleLarge.jpg?quality=75\&auto=webp\&disable=upscale}

On the last weekend in May, the police in Sioux Falls, S.D., decided to
investigate whether busloads of antifa protesters were headed to town.
It shows what can happen from a single tweet.

They were responding to a rumor spreading quickly among residents
online, and first posted to Twitter by the local Chamber of Commerce.

``We're being told that buses are en route from Fargo for today's march
downtown\ldots{},'' the group posted on Twitter. ``Please bring in any
furniture, signs, etc. that could be possibly thrown through windows.''

The tweet was later deleted, but not before the rumor spread verbatim on
Facebook, where it was even translated into Spanish. On Facebook,
screenshots of the tweet and other posts about the group's message
collected more than 4,600 likes and shares according to CrowdTangle, a
Facebook-owned tool that analyzes interactions across social media.

These included shares by the Facebook pages of three local news outlets
with a combined reach of 36,238 followers, and two posts in
Spanish-speaking local Facebook groups, which reached 2,611 followers.

Twitter said it had taken down ``hundreds of groups'' under its violent
extremist group policy and ``continues to enforce our policies against
hateful conduct every day across the world.'' Facebook said its
fact-checking partners rate many false claims about the protests,
including about antifa.

The rumor led dozens of people to reach out to the local police that
Sunday, according to Sam Clemens, the public information officer at the
Sioux Falls Police Department.

``But on the day of the protests, we didn't have any evidence of any
buses coming from out of town carrying people,'' Mr. Clemens said. The
vast majority of protesters were local residents, he said.

The Greater Sioux Falls Chamber of Commerce said it had gotten the
information from sources it knew and believed to be credible.

``We received information that led us to believe there was a cause for
concern. As such, we wanted to encourage local business owners to take
responsible, precautionary steps for their businesses,'' said Jeff
Griffin, the group's president. ``We removed the post when we realized
it was contributing to a different message that we did not intend.''

\hypertarget{from-youtube-to-infowars}{%
\subsection{From YouTube to Infowars}\label{from-youtube-to-infowars}}

Image

A screenshot from a YouTube video purporting to show Yucaipa, Calif.,
residents preparing for ``potential antifia looting ahead of a planned
BLM protest.''

A false rumor about antifa protesters in Yucaipa, Calif., a city about
70 miles from Los Angeles, started with one viral YouTube video about
the city. Before long, it had even reached a national audience.

A YouTube video posted on June 2, featuring scenes of men in masks and
holding guns, purportedly residents of the city preparing for
``potential antifa looting ahead of a planned BLM protest,'' **** has
**** collected 17,200 views in the days since. Facebook posts of photos
claiming to show the Yucaipa residents defending their town were posted
at least 587 times in Facebook groups, and amassed over 24,000 likes and
shares, according to the Times analysis. They were shared in pro-Trump
and far-right Facebook groups, as well as other local community groups.

Farshad Shadloo, a YouTube spokesman, said that, like Facebook, the
video service uses fact-checking panels to flag false information, and
that the company aims to promote videos from authoritative sources about
the protests.

On the same day, the conservative commentator and former Fox News host
Todd Starnes published a blog post titled, ``TOWN FIGHTS antifa: `They
Just Beat the Ever-Loving Snot Out of Them.''' It collected over 48,000
likes and shares, and reached three million followers on Facebook.

A day later, the conspiracy website Infowars posted an article about the
false narrative, which spread it further among followers of conspiracy
groups and several Facebook groups dedicated to praising Mr. Trump.

A representative for Mr. Starnes said he was unavailable to respond.

The Yucaipa Police Department
\href{https://twitter.com/YucaipaPD/status/1267979797059133440}{confirmed}
on Twitter that it had responded to reports of fights in public on June
1, but did not mention the involvement of antifa. A public information
officer for the department pointed to a
\href{https://www.youtube.com/watch?v=KeSiWIbNXEY\&feature=youtu.be}{YouTube
video} posted last week, in which a Yucaipa police lieutenant, Julie
Brumm-Landen, said the city had not experienced looting or destruction
from protests of racism.

``The information about antifa or planned criminal activity in Yucaipa
is nothing more than internet speculation and false rumors,'' Lt.
Brumm-Landen added. ``Any peaceful protests that takes place will have
the full support and protection of the Yucaipa Police Department.'' That
video was viewed just 100 times.

\hypertarget{warnings-from-a-congressional-candidate}{%
\subsection{Warnings From a Congressional
Candidate}\label{warnings-from-a-congressional-candidate}}

Image

A congressional candidate over 2,000 miles away from Yucaipa started to
spread a similar message. The episode highlights how even when a tech
company removes bad local information, it can happen too late.

Marjorie Taylor Greene, a Republican in northwest Georgia and a
professed member of the fringe conspiracy theory group QAnon, tweeted an
ad for her House campaign showing her holding an AR-15-style rifle and
threatening antifa
\href{https://www.nytimes.com/2020/06/11/us/antifa-protests-george-floyd.html}{activists}.
``You won't burn our churches, loot our businesses or destroy our
homes,'' she said in the ad. It was retweeted 20,000 times.

That same campaign ad was removed from Facebook two days later --- but
not before it racked up over 1.2 million views. According to the social
network, the video violated the company's policies against promoting the
use of firearms.

``We removed it because it advocates the use of deadly weapons against a
clearly defined group of people, which violates our policies against
inciting violence,'' said Andrea Vallone, a Facebook spokeswoman.

No group of antifa activists arrived in Georgia. But that didn't seem to
hurt Ms. Greene's political campaign. One week after her ad posted, she
finished first in her primary, winning 41 percent of the vote in the
strongly Republican 14th Congressional District, and has a strong chance
of winning a runoff vote in August.

Ms. Greene, who has a history of
\href{https://www.nytimes.com/2020/06/17/us/marjorie-taylor-greene-georgia.html}{making
offensive remarks} about blacks, Jews and Muslims, appears to have no
remorse about spreading unfounded rumors of antifa coming to town.

``I'm sick and tired of watching establishment Republicans play defense
while the Fake News Media cheers on antifa terrorists, B.L.M. rioters
and the woke cancel culture as they burn our cities, loot our
businesses, vandalize our memorials and divide our nation,'' Ms. Greene
said in an emailed statement.

\hypertarget{facebook-group-posts-then-text-messages}{%
\subsection{Facebook Group Posts, Then Text
Messages}\label{facebook-group-posts-then-text-messages}}

Image

In late May to early June, there was a rumor that ``two bus loads of
antifa'' were heading to Locust, N.C., about 25 miles east of Charlotte.
The rumor was shared in text messages among people in the area --- far
out of sight of any fact-checking organization.

On June 1, the rumor surfaced in Facebook groups with names like
DeplorablePride.org and Albemarle News and Weather.

That same evening, the police in Locust
\href{https://www.facebook.com/971105932951713/photos/a.1627165354012431/3050055735056712/?type=3\&__xts__\%5B0\%5D=68.ARA_CCbgmvw1vn-Rpj-uzp23zApNlmhYrmV7kAtkCrWO5lfHA5skNnUYW2fj_5wf6r8kxnpw_dwH0ucrGRhE-2HEtaM9AyVNqOLSMcNBInwixXXKwUdH2idPkUXkBB1zLlC-XbjjgGtx0GcK98L5LviWf207V-RyJDtJoCrK_mr6p2b1n-xKWBX0d-3DeX4fw3mbnXtcq1l_EixqJKZ-fp0Ellypj2SkkfshKk53Qu6G81p6t1sn9WMXz9tYxCvp7RxCHyK9rt5gTs7UQVRw3SdA0ijGo-roueC8w7bWpwFYl88ZINOTTmfR0H5HrEWgqF5pUJZOtraFuQOlZPf0KK6tTw\&__tn__=-R}{posted
a screenshot} of a text that had been circulating in the community over
the weekend. The text falsely claimed that police officers had been
knocking on doors to warn that ``a black organization is bringing 2 bus
loads of people to walmart in locust with intentions on looting and
burning down the suburbs.'' The post, on Facebook, assured residents
that the Police Department had not been spreading the rumor.

Jeffrey Shew, the assistant chief of police, said all the residents who
reached out to the department to report the buses ``had no direct
knowledge'' of violent protesters coming to town. He said they were only
sharing what they had seen on social media. By midnight on June 1, Mr.
Shew said, it was clear that the rumors were untrue.

``No protests, groups looking to protest or groups looking to riot
occurred,'' he said.

On June 2, the police posted another message on Facebook
\href{https://www.facebook.com/permalink.php?story_fbid=3051048361624116\&id=971105932951713\&__xts__\%5B0\%5D=68.ARDOEJsYdf3LZ57Mi_zK-u33tmrlZaw_x9KMCh5_NzQXPBb1nuyBH6YVwtdTbLc8YVnQGAkpd25IIKuPhL2W93GP3WSmtImjE-1swyUdUauEgSINmBVt4xHFgS4JWicuyCMHHwGWzfbB46xQBUCDSjvs3vChZs9S6WW5U0zrf_BHsPe-X0MAcnm0mPxIoOMRB3Uw21GLz0B2PN-leSVawzpWWdAV11WD5kmnT5IMPIE-wbI0FDSmdEVilltF7ZwEcUSwuss1EryU3EwDjblvWXFrTxLYJP-6szS5Q431m1JNc8Cu7P0Pc8s_mu6ba17ruAe8dpWtD16Yejh9g9Bf_w\&__tn__=-R}{emphasizing
that the rumors had no substance}. It exemplified that often, community
members themselves are the ones on the front lines of debunking false
rumors.

``We had absolutely zero confirmed credible information related to these
activities however out of an abundance of caution we did arrange or
stage extra resources and officers in Locust in the event there was any
legitimacy to the posts,'' the post by the Locust Police Department
read. ``Now in the morning after, we can 100\% confirm there was zero
truth to any of the posts that we observed.''

Posts containing the original rumor reached 27,855 followers on
Facebook, according to the Times analysis. The police's posts reached
2,966 followers on Facebook.

Advertisement

\protect\hyperlink{after-bottom}{Continue reading the main story}

\hypertarget{site-index}{%
\subsection{Site Index}\label{site-index}}

\hypertarget{site-information-navigation}{%
\subsection{Site Information
Navigation}\label{site-information-navigation}}

\begin{itemize}
\tightlist
\item
  \href{https://help.nytimes.com/hc/en-us/articles/115014792127-Copyright-notice}{©~2020~The
  New York Times Company}
\end{itemize}

\begin{itemize}
\tightlist
\item
  \href{https://www.nytco.com/}{NYTCo}
\item
  \href{https://help.nytimes.com/hc/en-us/articles/115015385887-Contact-Us}{Contact
  Us}
\item
  \href{https://www.nytco.com/careers/}{Work with us}
\item
  \href{https://nytmediakit.com/}{Advertise}
\item
  \href{http://www.tbrandstudio.com/}{T Brand Studio}
\item
  \href{https://www.nytimes.com/privacy/cookie-policy\#how-do-i-manage-trackers}{Your
  Ad Choices}
\item
  \href{https://www.nytimes.com/privacy}{Privacy}
\item
  \href{https://help.nytimes.com/hc/en-us/articles/115014893428-Terms-of-service}{Terms
  of Service}
\item
  \href{https://help.nytimes.com/hc/en-us/articles/115014893968-Terms-of-sale}{Terms
  of Sale}
\item
  \href{https://spiderbites.nytimes.com}{Site Map}
\item
  \href{https://help.nytimes.com/hc/en-us}{Help}
\item
  \href{https://www.nytimes.com/subscription?campaignId=37WXW}{Subscriptions}
\end{itemize}
