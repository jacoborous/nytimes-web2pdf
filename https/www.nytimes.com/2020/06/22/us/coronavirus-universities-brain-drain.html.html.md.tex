Sections

SEARCH

\protect\hyperlink{site-content}{Skip to
content}\protect\hyperlink{site-index}{Skip to site index}

\href{/section/us}{U.S.}\textbar{}Why Some State Universities Are Seeing
an Influx

\url{https://nyti.ms/3fJm2Cr}

\begin{itemize}
\item
\item
\item
\item
\item
\item
\end{itemize}

\href{https://www.nytimes.com/news-event/coronavirus?action=click\&pgtype=Article\&state=default\&region=TOP_BANNER\&context=storylines_menu}{The
Coronavirus Outbreak}

\begin{itemize}
\tightlist
\item
  live\href{https://www.nytimes.com/2020/08/02/world/coronavirus-updates.html?action=click\&pgtype=Article\&state=default\&region=TOP_BANNER\&context=storylines_menu}{Latest
  Updates}
\item
  \href{https://www.nytimes.com/interactive/2020/us/coronavirus-us-cases.html?action=click\&pgtype=Article\&state=default\&region=TOP_BANNER\&context=storylines_menu}{Maps
  and Cases}
\item
  \href{https://www.nytimes.com/interactive/2020/science/coronavirus-vaccine-tracker.html?action=click\&pgtype=Article\&state=default\&region=TOP_BANNER\&context=storylines_menu}{Vaccine
  Tracker}
\item
  \href{https://www.nytimes.com/interactive/2020/07/29/us/schools-reopening-coronavirus.html?action=click\&pgtype=Article\&state=default\&region=TOP_BANNER\&context=storylines_menu}{What
  School May Look Like}
\item
  \href{https://www.nytimes.com/live/2020/07/31/business/stock-market-today-coronavirus?action=click\&pgtype=Article\&state=default\&region=TOP_BANNER\&context=storylines_menu}{Economy}
\end{itemize}

\includegraphics{https://static01.nyt.com/images/2020/06/19/us/00VIRUS-WVU-campus/merlin_173623710_1be70d9c-4fdc-41e0-a5eb-0cd4d3e5ddfb-articleLarge.jpg?quality=75\&auto=webp\&disable=upscale}

\hypertarget{why-some-state-universities-are-seeing-an-influx}{%
\section{Why Some State Universities Are Seeing an
Influx}\label{why-some-state-universities-are-seeing-an-influx}}

The pandemic is giving a new competitive edge to states that have long
seen their top students lured away by elite schools.

The campus of West Virginia University in Morgantown,
W.Va.Credit...Rebecca Kiger for The New York Times

Supported by

\protect\hyperlink{after-sponsor}{Continue reading the main story}

\href{https://www.nytimes.com/by/anemona-hartocollis}{\includegraphics{https://static01.nyt.com/images/2018/06/13/multimedia/author-anemona-hartocollis/author-anemona-hartocollis-thumbLarge-v3.jpg}}

By \href{https://www.nytimes.com/by/anemona-hartocollis}{Anemona
Hartocollis}

\begin{itemize}
\item
  Published June 22, 2020Updated June 24, 2020
\item
  \begin{itemize}
  \item
  \item
  \item
  \item
  \item
  \item
  \end{itemize}
\end{itemize}

MORGANTOWN, W.Va. --- On a crisp day in November, three young men
gathered at the tiny municipal airport in Morgantown, just minutes from
West Virginia University's campus. One paced nervously, having never
flown before. Another was dressed as the university's mascot, wearing
buckskin and toting a rifle. The third was reading a book, Ralph
Ellison's ``Invisible Man.''

The three men, all W.V.U. students, were boarding a private plane for a
recruiting trip to McDowell County, the remote heart of Appalachia, with
Gordon Gee, the university's president. Their job was to persuade some
of the state's most promising high school students to stay home for
college, like they did, and not be tempted by the glamour and mystique
of elite schools in faraway states.

``I've joked about the fact that I want to build a wall around West
Virginia and keep all the kids here,'' Dr. Gee said, chuckling. ``A
state can't flourish that can't keep its young people there.''

State-funded universities have always striven to keep their states'
brightest students at home, knowing that many of those who leave their
communities will never return. Now, as the pandemic erodes the economy
and civil unrest sweeps the country, colleges are seeing renewed success
in their efforts to reverse years of brain drain, with students
responding to a new focus on basics, like family and community, over
prestige.

\includegraphics{https://static01.nyt.com/images/2020/06/19/us/00VIRUS-WVU-gee/merlin_173623233_e659138d-8e0c-4634-828b-f99d4af0b02d-articleLarge.jpg?quality=75\&auto=webp\&disable=upscale}

New Jersey, a densely populated state in a region with many college
options, has been a big exporter of college students. So this spring, 10
public college and university presidents dreamed up the New Jersey
Scholar Corps, their version of a pandemic Peace Corps. Their goal was
to convince New Jersey students studying in other states to return, by
offering expedited application review and volunteer opportunities.

At one of the 10, Montclair State University, 16 students applied to
transfer back from out of state, and half have accepted offers of
admission, with others in the works. Over all, the in-state acceptance
rate at Montclair State is up almost 2 percent over last year.

``We are at the moment when we can get the attention of families who
historically overlooked their in-state opportunities, and perhaps begin
to change the mind-set,'' said Joseph A. Brennan, vice president of
communications and marketing.

Since the pandemic began, the University of Kansas has been getting more
transfer students from other four-year institutions. ``In many
instances, those are students from Kansas who went away to institutions
who then are coming back to Kansas,'' Matt Melvin, vice provost for
enrollment management, said. ``We always see some of that, but it seems
more pronounced because of the pandemic.''

\hypertarget{latest-updates-global-coronavirus-outbreak}{%
\section{\texorpdfstring{\href{https://www.nytimes.com/2020/08/01/world/coronavirus-covid-19.html?action=click\&pgtype=Article\&state=default\&region=MAIN_CONTENT_1\&context=storylines_live_updates}{Latest
Updates: Global Coronavirus
Outbreak}}{Latest Updates: Global Coronavirus Outbreak}}\label{latest-updates-global-coronavirus-outbreak}}

Updated 2020-08-02T17:52:35.962Z

\begin{itemize}
\tightlist
\item
  \href{https://www.nytimes.com/2020/08/01/world/coronavirus-covid-19.html?action=click\&pgtype=Article\&state=default\&region=MAIN_CONTENT_1\&context=storylines_live_updates\#link-34047410}{The
  U.S. reels as July cases more than double the total of any other
  month.}
\item
  \href{https://www.nytimes.com/2020/08/01/world/coronavirus-covid-19.html?action=click\&pgtype=Article\&state=default\&region=MAIN_CONTENT_1\&context=storylines_live_updates\#link-780ec966}{Top
  U.S. officials work to break an impasse over the federal jobless
  benefit.}
\item
  \href{https://www.nytimes.com/2020/08/01/world/coronavirus-covid-19.html?action=click\&pgtype=Article\&state=default\&region=MAIN_CONTENT_1\&context=storylines_live_updates\#link-2bc8948}{Its
  outbreak untamed, Melbourne goes into even greater lockdown.}
\end{itemize}

\href{https://www.nytimes.com/2020/08/01/world/coronavirus-covid-19.html?action=click\&pgtype=Article\&state=default\&region=MAIN_CONTENT_1\&context=storylines_live_updates}{See
more updates}

More live coverage:
\href{https://www.nytimes.com/live/2020/07/31/business/stock-market-today-coronavirus?action=click\&pgtype=Article\&state=default\&region=MAIN_CONTENT_1\&context=storylines_live_updates}{Markets}

In order to accommodate those transfers, ``who were really in panic mode
because their world was turned upside down,'' the university has quietly
and informally made its deposit and scholarship deadlines more flexible,
Mr. Melvin said.

The flip side, he said, is that some of the university's out-of-state
students are returning to their home states. But while the balance is in
flux, he hopes that Kansas will see a net gain because it is perceived
as a safe place, where the incidence of infection and death from the
virus has been relatively low.

What is more, he said, most of the university's out-of-state students
live within a five to eight-hour drive. ``Even if we have a second wave,
God help us, many of our students, even though they are from out of
state, can get home,'' he said.

Public universities like Kansas and West Virginia have long struggled to
compete against prestigious private schools that heavily recruit
students from rural states to increase their geographic, political and
socioeconomic diversity.

Historically,
\href{https://www.nytimes.com/2018/10/15/us/harvard-affirmative-action-trial-asian-americans.html}{Harvard
has lowered} its minimum standardized test scores for some students
recruited from what it calls ``sparse country'' --- 20 largely rural
states like Montana, South Dakota, Alabama and, yes, West Virginia,
where few students tend to apply to elite universities.

Dr. Gee said he knew that for students focused on marquee names, W.V.U.
might not be the first choice. But in his recruiting trips all over the
state on the private plane leased by the university --- he tries to hit
all 55 counties once a year --- he frames it as the right choice if
students want to serve their communities.

``I want all of you to stay and never cross that border, and I want our
young people to stay,'' he told the mayor, a pastor and other pillars of
the community who gathered for a reception in his honor in McDowell
County on the trip in November. ``We are everyone's university.''

Sometimes his pitch succeeds, Dr. Gee said, as with the young man who
called him from Boston Logan International Airport a few years ago to
say that he did not feel like he belonged when he visited Harvard.

Often it does not.

Image

Public universities like W.V.U. have long struggled to compete for their
states' best students.Credit...Rebecca Kiger for The New York Times

Image

A helicopter departs Ruby Memorial Hospital, part of W.V.U.'s Health
Science Center in Morgantown.Credit...Rebecca Kiger for The New York
Times

But Dr. Gee has leaned into his usual outreach during the pandemic. ``I
have called every valedictorian in the state, and I am calling every
student body president right now,'' he said last month. ``In a small
state, I can do that.''

Each week, he has also been calling about 50 students who put down
deposits for the fall, just to see if they had any concerns about
attending.

The pandemic, he says, has played to West Virginia's strengths. ``No one
can say a pandemic is healthy,'' he said. ``But the short- and long-term
trends for a place like West Virginia are going to prove very positive.
We have a big academic medical center right in the middle of campus. So
we can take care of those kids.''

Emilie Charles lives in Huntington, W.Va., an industrial city on the
Ohio River. She graduated this spring from the regional high school
where, as a soccer player and aspiring pre-med student, she was being
courted by schools from California to Florida.

Notre Dame, in Indiana, sent a handwritten card, telling her why the
school liked her, and put her in touch with students who could tell her
what a great place it was. But those students seemed so different from
her that they could not, she said, tell her ``what it would be like for
me, specifically.''

Pepperdine, in Malibu, Calif., sent her a note quoting a line about her
devotion to Christian values, from her personal essay, but she worried
about going all the way to the West Coast and finding that the
dormitories were closed because of the virus.

The unstable economy made her parents, both doctors, feel more
vulnerable, and the feeling rubbed off on her. Then she won a
full-tuition scholarship to West Virginia, and Dr. Gee personally called
to congratulate her. ``I thought that was really cool,'' she said. ``I
thought I would hate staying home, but now that I actually have, it's
home.''

\href{https://www.nytimes.com/news-event/coronavirus?action=click\&pgtype=Article\&state=default\&region=MAIN_CONTENT_3\&context=storylines_faq}{}

\hypertarget{the-coronavirus-outbreak-}{%
\subsubsection{The Coronavirus Outbreak
›}\label{the-coronavirus-outbreak-}}

\hypertarget{frequently-asked-questions}{%
\paragraph{Frequently Asked
Questions}\label{frequently-asked-questions}}

Updated July 27, 2020

\begin{itemize}
\item ~
  \hypertarget{should-i-refinance-my-mortgage}{%
  \paragraph{Should I refinance my
  mortgage?}\label{should-i-refinance-my-mortgage}}

  \begin{itemize}
  \tightlist
  \item
    \href{https://www.nytimes.com/article/coronavirus-money-unemployment.html?action=click\&pgtype=Article\&state=default\&region=MAIN_CONTENT_3\&context=storylines_faq}{It
    could be a good idea,} because mortgage rates have
    \href{https://www.nytimes.com/2020/07/16/business/mortgage-rates-below-3-percent.html?action=click\&pgtype=Article\&state=default\&region=MAIN_CONTENT_3\&context=storylines_faq}{never
    been lower.} Refinancing requests have pushed mortgage applications
    to some of the highest levels since 2008, so be prepared to get in
    line. But defaults are also up, so if you're thinking about buying a
    home, be aware that some lenders have tightened their standards.
  \end{itemize}
\item ~
  \hypertarget{what-is-school-going-to-look-like-in-september}{%
  \paragraph{What is school going to look like in
  September?}\label{what-is-school-going-to-look-like-in-september}}

  \begin{itemize}
  \tightlist
  \item
    It is unlikely that many schools will return to a normal schedule
    this fall, requiring the grind of
    \href{https://www.nytimes.com/2020/06/05/us/coronavirus-education-lost-learning.html?action=click\&pgtype=Article\&state=default\&region=MAIN_CONTENT_3\&context=storylines_faq}{online
    learning},
    \href{https://www.nytimes.com/2020/05/29/us/coronavirus-child-care-centers.html?action=click\&pgtype=Article\&state=default\&region=MAIN_CONTENT_3\&context=storylines_faq}{makeshift
    child care} and
    \href{https://www.nytimes.com/2020/06/03/business/economy/coronavirus-working-women.html?action=click\&pgtype=Article\&state=default\&region=MAIN_CONTENT_3\&context=storylines_faq}{stunted
    workdays} to continue. California's two largest public school
    districts --- Los Angeles and San Diego --- said on July 13, that
    \href{https://www.nytimes.com/2020/07/13/us/lausd-san-diego-school-reopening.html?action=click\&pgtype=Article\&state=default\&region=MAIN_CONTENT_3\&context=storylines_faq}{instruction
    will be remote-only in the fall}, citing concerns that surging
    coronavirus infections in their areas pose too dire a risk for
    students and teachers. Together, the two districts enroll some
    825,000 students. They are the largest in the country so far to
    abandon plans for even a partial physical return to classrooms when
    they reopen in August. For other districts, the solution won't be an
    all-or-nothing approach.
    \href{https://bioethics.jhu.edu/research-and-outreach/projects/eschool-initiative/school-policy-tracker/}{Many
    systems}, including the nation's largest, New York City, are
    devising
    \href{https://www.nytimes.com/2020/06/26/us/coronavirus-schools-reopen-fall.html?action=click\&pgtype=Article\&state=default\&region=MAIN_CONTENT_3\&context=storylines_faq}{hybrid
    plans} that involve spending some days in classrooms and other days
    online. There's no national policy on this yet, so check with your
    municipal school system regularly to see what is happening in your
    community.
  \end{itemize}
\item ~
  \hypertarget{is-the-coronavirus-airborne}{%
  \paragraph{Is the coronavirus
  airborne?}\label{is-the-coronavirus-airborne}}

  \begin{itemize}
  \tightlist
  \item
    The coronavirus
    \href{https://www.nytimes.com/2020/07/04/health/239-experts-with-one-big-claim-the-coronavirus-is-airborne.html?action=click\&pgtype=Article\&state=default\&region=MAIN_CONTENT_3\&context=storylines_faq}{can
    stay aloft for hours in tiny droplets in stagnant air}, infecting
    people as they inhale, mounting scientific evidence suggests. This
    risk is highest in crowded indoor spaces with poor ventilation, and
    may help explain super-spreading events reported in meatpacking
    plants, churches and restaurants.
    \href{https://www.nytimes.com/2020/07/06/health/coronavirus-airborne-aerosols.html?action=click\&pgtype=Article\&state=default\&region=MAIN_CONTENT_3\&context=storylines_faq}{It's
    unclear how often the virus is spread} via these tiny droplets, or
    aerosols, compared with larger droplets that are expelled when a
    sick person coughs or sneezes, or transmitted through contact with
    contaminated surfaces, said Linsey Marr, an aerosol expert at
    Virginia Tech. Aerosols are released even when a person without
    symptoms exhales, talks or sings, according to Dr. Marr and more
    than 200 other experts, who
    \href{https://academic.oup.com/cid/article/doi/10.1093/cid/ciaa939/5867798}{have
    outlined the evidence in an open letter to the World Health
    Organization}.
  \end{itemize}
\item ~
  \hypertarget{what-are-the-symptoms-of-coronavirus}{%
  \paragraph{What are the symptoms of
  coronavirus?}\label{what-are-the-symptoms-of-coronavirus}}

  \begin{itemize}
  \tightlist
  \item
    Common symptoms
    \href{https://www.nytimes.com/article/symptoms-coronavirus.html?action=click\&pgtype=Article\&state=default\&region=MAIN_CONTENT_3\&context=storylines_faq}{include
    fever, a dry cough, fatigue and difficulty breathing or shortness of
    breath.} Some of these symptoms overlap with those of the flu,
    making detection difficult, but runny noses and stuffy sinuses are
    less common.
    \href{https://www.nytimes.com/2020/04/27/health/coronavirus-symptoms-cdc.html?action=click\&pgtype=Article\&state=default\&region=MAIN_CONTENT_3\&context=storylines_faq}{The
    C.D.C. has also} added chills, muscle pain, sore throat, headache
    and a new loss of the sense of taste or smell as symptoms to look
    out for. Most people fall ill five to seven days after exposure, but
    symptoms may appear in as few as two days or as many as 14 days.
  \end{itemize}
\item ~
  \hypertarget{does-asymptomatic-transmission-of-covid-19-happen}{%
  \paragraph{Does asymptomatic transmission of Covid-19
  happen?}\label{does-asymptomatic-transmission-of-covid-19-happen}}

  \begin{itemize}
  \tightlist
  \item
    So far, the evidence seems to show it does. A widely cited
    \href{https://www.nature.com/articles/s41591-020-0869-5}{paper}
    published in April suggests that people are most infectious about
    two days before the onset of coronavirus symptoms and estimated that
    44 percent of new infections were a result of transmission from
    people who were not yet showing symptoms. Recently, a top expert at
    the World Health Organization stated that transmission of the
    coronavirus by people who did not have symptoms was ``very rare,''
    \href{https://www.nytimes.com/2020/06/09/world/coronavirus-updates.html?action=click\&pgtype=Article\&state=default\&region=MAIN_CONTENT_3\&context=storylines_faq\#link-1f302e21}{but
    she later walked back that statement.}
  \end{itemize}
\end{itemize}

Family migration out of the state has left West Virginia with a
shrinking number of students in kindergarten through 12th grade. The
state also has a tradition of working-class students graduating high
school and going straight into decent-paying jobs in the oil, gas and
coal industries, said George Zimmerman, head of admissions and
recruitment at W.V.U. Convincing some students that higher education is
worth it has been a long-term challenge.

``We're trying to inject a college-going culture,'' Mr. Zimmerman said.

That was part of Dr. Gee's mission when his plane landed in Beckley, in
the southwestern part of the state, last fall. The seven passengers
switched to a van for the hour-plus drive on winding mountain roads to
Welch, the county seat.

The mountain scenery was breathtaking, but cell reception was often
blocked in the steep hollers, making Google Maps useless, so an
experienced aide navigated from memory. ``This is Trump country and coal
country,'' Dr. Gee remarked, looking out at the hillsides where houses
seemed to cling by their fingernails.

Brice Shumate, the student who had never flown on a plane before, was a
native son; he grew up 10 minutes away from Welch. He was excited to be
headed to his former high school, where his younger sister was still a
student.

Apart from a cousin who attended W.V.U. for about half a semester, Mr.
Shumate said, he was the first in his family to go to college. His
father started working in the coal mines at age 19, like most of the men
in his family. ``I've got a lot of respect for the work, but it wasn't
what I wanted to do with my life. I didn't want to destroy my body to
make a living.''

David Laub, the book reader, grew up in Martinsburg, an easy drive to
Washington and more cosmopolitan than Welch. He was raised to be a
Jehovah's Witness missionary, but rebelled. He received a full
scholarship to West Virginia. He had considered Duke, he said, but it
was too expensive.

Fears that the rising cost of college
\href{https://www.nytimes.com/2020/04/15/us/coronavirus-colleges-universities-admissions.html}{will
keep students away} from higher education have risen with the pandemic.
A recent survey of college and university presidents by the American
Council on Education, a trade group, found that maintaining fall or
summer enrollment was their biggest concern, followed by their long-term
financial viability.

For state residents, tuition and fees at W.V.U. for 2019-20 were about
\$9,000 a year, plus \$10,000 in room and board. By comparison, Yale
University estimates the cost of attendance at \$78,725 for 2020-21.

Image

Georgia Beatty of Weirton, W.Va., said she gave up a spot at New York
University in favor of West Virginia mainly because of the price
difference.Credit...Rebecca Kiger for The New York Times

The price difference was a big draw for Juliet Wanosky, who grew up in
Parkersburg, ``an everybody-knows-everybody kind of town,'' she said,
and was valedictorian of her class this year. Her father is a chemical
engineer, her mother a substitute school secretary. She toured M.I.T.,
Carnegie Mellon and Harvard before the high sticker prices scared her
away from even applying.

She wishes she had applied, just to see if she could get in, not because
she would have gone there. ``The coronavirus pandemic has definitely
made me more confident in my decision to go to W.V.U.,'' she said.

The trend showed up in a survey W.V.U. took of incoming students in May.
It asked their parents: ``Has the Covid-19 crisis led you to believe
it's important that your student choose a school that is closer to
home?'' It found that 39 percent of those from West Virginia said yes,
compared with just 17 percent of parents from the surrounding states of
Ohio, Maryland, Pennsylvania and New Jersey.

Some families do not want their children leaving the state or going to
schools with liberal reputations because they worry it will change them.

Georgia Beatty said she gave up a spot at New York University in favor
of West Virginia, where she is currently a senior, mainly because of the
price difference. Now she is determined to broaden her opportunities by
leaving the state for graduate school.

But she has butted heads with her grandfather, a retired police officer,
who believes that universities radicalize students, and that going out
of state will make it worse, especially in this protest era.

``There's always been a distrust in my family of higher education,'' she
said. ``I'm sort of the black sheep.''

Advertisement

\protect\hyperlink{after-bottom}{Continue reading the main story}

\hypertarget{site-index}{%
\subsection{Site Index}\label{site-index}}

\hypertarget{site-information-navigation}{%
\subsection{Site Information
Navigation}\label{site-information-navigation}}

\begin{itemize}
\tightlist
\item
  \href{https://help.nytimes.com/hc/en-us/articles/115014792127-Copyright-notice}{©~2020~The
  New York Times Company}
\end{itemize}

\begin{itemize}
\tightlist
\item
  \href{https://www.nytco.com/}{NYTCo}
\item
  \href{https://help.nytimes.com/hc/en-us/articles/115015385887-Contact-Us}{Contact
  Us}
\item
  \href{https://www.nytco.com/careers/}{Work with us}
\item
  \href{https://nytmediakit.com/}{Advertise}
\item
  \href{http://www.tbrandstudio.com/}{T Brand Studio}
\item
  \href{https://www.nytimes.com/privacy/cookie-policy\#how-do-i-manage-trackers}{Your
  Ad Choices}
\item
  \href{https://www.nytimes.com/privacy}{Privacy}
\item
  \href{https://help.nytimes.com/hc/en-us/articles/115014893428-Terms-of-service}{Terms
  of Service}
\item
  \href{https://help.nytimes.com/hc/en-us/articles/115014893968-Terms-of-sale}{Terms
  of Sale}
\item
  \href{https://spiderbites.nytimes.com}{Site Map}
\item
  \href{https://help.nytimes.com/hc/en-us}{Help}
\item
  \href{https://www.nytimes.com/subscription?campaignId=37WXW}{Subscriptions}
\end{itemize}
