Sections

SEARCH

\protect\hyperlink{site-content}{Skip to
content}\protect\hyperlink{site-index}{Skip to site index}

\href{https://www.nytimes.com/section/nyregion}{New York}

\href{https://myaccount.nytimes.com/auth/login?response_type=cookie\&client_id=vi}{}

\href{https://www.nytimes.com/section/todayspaper}{Today's Paper}

\href{/section/nyregion}{New York}\textbar{}A Moratorium on Evictions
Ends, Leaving Thousands of Tenants Fearful

\url{https://nyti.ms/2zTlmeq}

\begin{itemize}
\item
\item
\item
\item
\item
\item
\end{itemize}

\href{https://www.nytimes.com/news-event/coronavirus?action=click\&pgtype=Article\&state=default\&region=TOP_BANNER\&context=storylines_menu}{The
Coronavirus Outbreak}

\begin{itemize}
\tightlist
\item
  live\href{https://www.nytimes.com/2020/08/03/world/coronavirus-covid-19.html?action=click\&pgtype=Article\&state=default\&region=TOP_BANNER\&context=storylines_menu}{Latest
  Updates}
\item
  \href{https://www.nytimes.com/interactive/2020/us/coronavirus-us-cases.html?action=click\&pgtype=Article\&state=default\&region=TOP_BANNER\&context=storylines_menu}{Maps
  and Cases}
\item
  \href{https://www.nytimes.com/interactive/2020/science/coronavirus-vaccine-tracker.html?action=click\&pgtype=Article\&state=default\&region=TOP_BANNER\&context=storylines_menu}{Vaccine
  Tracker}
\item
  \href{https://www.nytimes.com/2020/08/02/us/covid-college-reopening.html?action=click\&pgtype=Article\&state=default\&region=TOP_BANNER\&context=storylines_menu}{College
  Reopening}
\item
  \href{https://www.nytimes.com/live/2020/08/03/business/stock-market-today-coronavirus?action=click\&pgtype=Article\&state=default\&region=TOP_BANNER\&context=storylines_menu}{Economy}
\end{itemize}

Advertisement

\protect\hyperlink{after-top}{Continue reading the main story}

Supported by

\protect\hyperlink{after-sponsor}{Continue reading the main story}

\hypertarget{a-moratorium-on-evictions-ends-leaving-thousands-of-tenants-fearful}{%
\section{A Moratorium on Evictions Ends, Leaving Thousands of Tenants
Fearful}\label{a-moratorium-on-evictions-ends-leaving-thousands-of-tenants-fearful}}

Eviction cases are expected to soar in New York City as housing courts
reopen and landlords seek to recoup income lost during the pandemic.

\includegraphics{https://static01.nyt.com/images/2020/06/23/nyregion/23nyvirus-evict1/merlin_172244208_28138155-a697-4240-9042-c5993e4e5ce3-articleLarge.jpg?quality=75\&auto=webp\&disable=upscale}

\href{https://www.nytimes.com/by/matthew-haag}{\includegraphics{https://static01.nyt.com/images/2018/06/14/multimedia/author-matthew-haag/author-matthew-haag-thumbLarge.jpg}}

By \href{https://www.nytimes.com/by/matthew-haag}{Matthew Haag}

\begin{itemize}
\item
  June 22, 2020
\item
  \begin{itemize}
  \item
  \item
  \item
  \item
  \item
  \item
  \end{itemize}
\end{itemize}

A
\href{https://www.nytimes.com/2020/07/23/business/evictions-moratorium-cares-act.html}{moratorium
on evictions} that New York State imposed during the coronavirus
pandemic expired over the weekend, raising fears that tens of thousands
of residents struggling in the worst economic collapse since the Great
Depression will be called into housing courts, which reopened on Monday.

Housing rights groups estimate that in the coming days, 50,000 to 60,000
cases could be filed in New York City's housing courts. In addition,
thousands of cases that were already in progress but were paused in
March can now resume.

The number of eviction cases expected to be filed reflects the typical
caseload in a three-month period, which was the length of the
moratorium. But it does not take into account the fallout from the more
than one million city residents who have lost their jobs or were
furloughed in recent months and whose federal stimulus payments of an
extra \$600 per week will soon run out, housing advocates say.

A second order issued by the state that shields tenants directly
affected by the pandemic expires in late August and could produce an
even bigger wave of eviction cases.

``All levels of government have to realize that they cannot let tens of
thousands of people end up in homeless shelters,'' said Edward
Josephson, the director of litigation and housing at Legal Services NYC.
``It's the most dire thing that we have ever seen.''

But many landlords say they, too, are facing financial calamity, with
the loss of rental income leaving them unable to pay their own bills,
including mortgages, and invest in building upkeep.

``It is clear that the economic impacts of the Covid-19 pandemic are
nowhere near an end,'' said Jay Martin, the executive director of the
Community Housing Improvement Program, or CHIP, which represents about
4,000 property owners. ``There are thousands of tenants and building
owners who need help now.''

As housing courts nationwide begin to reopen and federal stimulus checks
are about to end, eviction cases are expected to soar. A
\href{https://www.amherstcapital.com/documents/20649/0/Amherst+Market+Commentary+-+May+2020+Issue/b6520038-308f-4708-8ed5-3b678d8ed560}{recent
report by Amherst}, an analytics and data real estate firm, found that
up to 28 million renters are at risk of eviction.

In the days leading up to the first moratorium deadline,
\href{https://www.nysenate.gov/sites/default/files/article/attachment/2020.06.17_-_letter_from_nys_legislators_to_gov._cuomo_re_eviction_moratorium_and_court_closures.pdf}{dozens
of members of the New York State Legislature}, as well as many housing
groups, urged Gov. Andrew M. Cuomo to extend universal protection to all
tenants, even in cases not directly caused by the pandemic.

They also expressed concern that housing courts would reopen physically
on Monday, placing tenants and others at risk of contracting and
spreading the virus.

\hypertarget{latest-updates-global-coronavirus-outbreak}{%
\section{\texorpdfstring{\href{https://www.nytimes.com/2020/08/03/world/coronavirus-covid-19.html?action=click\&pgtype=Article\&state=default\&region=MAIN_CONTENT_1\&context=storylines_live_updates}{Latest
Updates: Global Coronavirus
Outbreak}}{Latest Updates: Global Coronavirus Outbreak}}\label{latest-updates-global-coronavirus-outbreak}}

Updated 2020-08-04T06:59:59.801Z

\begin{itemize}
\tightlist
\item
  \href{https://www.nytimes.com/2020/08/03/world/coronavirus-covid-19.html?action=click\&pgtype=Article\&state=default\&region=MAIN_CONTENT_1\&context=storylines_live_updates\#link-4547638f}{Fauci
  defends Birx after she is criticized by Trump.}
\item
  \href{https://www.nytimes.com/2020/08/03/world/coronavirus-covid-19.html?action=click\&pgtype=Article\&state=default\&region=MAIN_CONTENT_1\&context=storylines_live_updates\#link-15e7f995}{Trump
  derides Democrats as lawmakers and administration officials try to
  break stimulus impasse.}
\item
  \href{https://www.nytimes.com/2020/08/03/world/coronavirus-covid-19.html?action=click\&pgtype=Article\&state=default\&region=MAIN_CONTENT_1\&context=storylines_live_updates\#link-e5a2cda}{The
  deadline for 2020 census counting has been moved up by a month.}
\end{itemize}

\href{https://www.nytimes.com/2020/08/03/world/coronavirus-covid-19.html?action=click\&pgtype=Article\&state=default\&region=MAIN_CONTENT_1\&context=storylines_live_updates}{See
more updates}

More live coverage:
\href{https://www.nytimes.com/live/2020/08/03/business/stock-market-today-coronavirus?action=click\&pgtype=Article\&state=default\&region=MAIN_CONTENT_1\&context=storylines_live_updates}{Markets}

But the state's chief administrative judge, Lawrence K. Marks, decided
against that, citing public health concerns. But case filings can be
sent online or through the mail, and hearings will be held virtually.

Susanna Blankley, the coalition coordinator for the Right to Counsel NYC
Coalition, said it was ``unconscionable'' for housing courts to restart
at all.

``In what world is it good to evict people in the middle of a
pandemic?'' Ms. Blankley said. ``Who are you opening for? It has to be
for the landlords.''

Even though the courthouses were closed on Monday, people protested the
virtual reopening outside the Brooklyn location, holding signs that
read, ``EVICTION FREE NYC.''

The past three months have been extremely difficult not only for
tenants, but also for smaller landlords.

About 25 percent of renters have not paid rent in May, April and June,
according to a survey by CHIP. About 20 percent of the landlords
represented by the group said they were concerned about losing their
properties.

Lincoln Eccles, who owns a 14-unit apartment building in Crown Heights,
Brooklyn, said the closing of housing courts in March delayed two cases
he had against separate tenants who have not paid rent in years.
Together, the tenants owe tens of thousands of dollars in rent, he said.

\includegraphics{https://static01.nyt.com/images/2020/06/23/nyregion/23nyvirus-evict2/23nyvirus-evict2-articleLarge.jpg?quality=75\&auto=webp\&disable=upscale}

Before the pandemic, Mr. Eccles said he expected to resolve those cases
by July or August. With the courts' closing and their typical snail's
pace even as they reopen, he does not expect a resolution for many more
months.

He said he collected full rent payments from only nine of his 14 units
this month; some of the tenants have not paid because of the pandemic,
he said.

\href{https://www.nytimes.com/news-event/coronavirus?action=click\&pgtype=Article\&state=default\&region=MAIN_CONTENT_3\&context=storylines_faq}{}

\hypertarget{the-coronavirus-outbreak-}{%
\subsubsection{The Coronavirus Outbreak
›}\label{the-coronavirus-outbreak-}}

\hypertarget{frequently-asked-questions}{%
\paragraph{Frequently Asked
Questions}\label{frequently-asked-questions}}

Updated August 3, 2020

\begin{itemize}
\item ~
  \hypertarget{im-a-small-business-owner-can-i-get-relief}{%
  \paragraph{I'm a small-business owner. Can I get
  relief?}\label{im-a-small-business-owner-can-i-get-relief}}

  \begin{itemize}
  \tightlist
  \item
    The
    \href{https://www.nytimes.com/article/small-business-loans-stimulus-grants-freelancers-coronavirus.html?action=click\&pgtype=Article\&state=default\&region=MAIN_CONTENT_3\&context=storylines_faq}{stimulus
    bills enacted in March} offer help for the millions of American
    small businesses. Those eligible for aid are businesses and
    nonprofit organizations with fewer than 500 workers, including sole
    proprietorships, independent contractors and freelancers. Some
    larger companies in some industries are also eligible. The help
    being offered, which is being managed by the Small Business
    Administration, includes the Paycheck Protection Program and the
    Economic Injury Disaster Loan program. But lots of folks have
    \href{https://www.nytimes.com/interactive/2020/05/07/business/small-business-loans-coronavirus.html?action=click\&pgtype=Article\&state=default\&region=MAIN_CONTENT_3\&context=storylines_faq}{not
    yet seen payouts.} Even those who have received help are confused:
    The rules are draconian, and some are stuck sitting on
    \href{https://www.nytimes.com/2020/05/02/business/economy/loans-coronavirus-small-business.html?action=click\&pgtype=Article\&state=default\&region=MAIN_CONTENT_3\&context=storylines_faq}{money
    they don't know how to use.} Many small-business owners are getting
    less than they expected or
    \href{https://www.nytimes.com/2020/06/10/business/Small-business-loans-ppp.html?action=click\&pgtype=Article\&state=default\&region=MAIN_CONTENT_3\&context=storylines_faq}{not
    hearing anything at all.}
  \end{itemize}
\item ~
  \hypertarget{what-are-my-rights-if-i-am-worried-about-going-back-to-work}{%
  \paragraph{What are my rights if I am worried about going back to
  work?}\label{what-are-my-rights-if-i-am-worried-about-going-back-to-work}}

  \begin{itemize}
  \tightlist
  \item
    Employers have to provide
    \href{https://www.osha.gov/SLTC/covid-19/standards.html}{a safe
    workplace} with policies that protect everyone equally.
    \href{https://www.nytimes.com/article/coronavirus-money-unemployment.html?action=click\&pgtype=Article\&state=default\&region=MAIN_CONTENT_3\&context=storylines_faq}{And
    if one of your co-workers tests positive for the coronavirus, the
    C.D.C.} has said that
    \href{https://www.cdc.gov/coronavirus/2019-ncov/community/guidance-business-response.html}{employers
    should tell their employees} -\/- without giving you the sick
    employee's name -\/- that they may have been exposed to the virus.
  \end{itemize}
\item ~
  \hypertarget{should-i-refinance-my-mortgage}{%
  \paragraph{Should I refinance my
  mortgage?}\label{should-i-refinance-my-mortgage}}

  \begin{itemize}
  \tightlist
  \item
    \href{https://www.nytimes.com/article/coronavirus-money-unemployment.html?action=click\&pgtype=Article\&state=default\&region=MAIN_CONTENT_3\&context=storylines_faq}{It
    could be a good idea,} because mortgage rates have
    \href{https://www.nytimes.com/2020/07/16/business/mortgage-rates-below-3-percent.html?action=click\&pgtype=Article\&state=default\&region=MAIN_CONTENT_3\&context=storylines_faq}{never
    been lower.} Refinancing requests have pushed mortgage applications
    to some of the highest levels since 2008, so be prepared to get in
    line. But defaults are also up, so if you're thinking about buying a
    home, be aware that some lenders have tightened their standards.
  \end{itemize}
\item ~
  \hypertarget{what-is-school-going-to-look-like-in-september}{%
  \paragraph{What is school going to look like in
  September?}\label{what-is-school-going-to-look-like-in-september}}

  \begin{itemize}
  \tightlist
  \item
    It is unlikely that many schools will return to a normal schedule
    this fall, requiring the grind of
    \href{https://www.nytimes.com/2020/06/05/us/coronavirus-education-lost-learning.html?action=click\&pgtype=Article\&state=default\&region=MAIN_CONTENT_3\&context=storylines_faq}{online
    learning},
    \href{https://www.nytimes.com/2020/05/29/us/coronavirus-child-care-centers.html?action=click\&pgtype=Article\&state=default\&region=MAIN_CONTENT_3\&context=storylines_faq}{makeshift
    child care} and
    \href{https://www.nytimes.com/2020/06/03/business/economy/coronavirus-working-women.html?action=click\&pgtype=Article\&state=default\&region=MAIN_CONTENT_3\&context=storylines_faq}{stunted
    workdays} to continue. California's two largest public school
    districts --- Los Angeles and San Diego --- said on July 13, that
    \href{https://www.nytimes.com/2020/07/13/us/lausd-san-diego-school-reopening.html?action=click\&pgtype=Article\&state=default\&region=MAIN_CONTENT_3\&context=storylines_faq}{instruction
    will be remote-only in the fall}, citing concerns that surging
    coronavirus infections in their areas pose too dire a risk for
    students and teachers. Together, the two districts enroll some
    825,000 students. They are the largest in the country so far to
    abandon plans for even a partial physical return to classrooms when
    they reopen in August. For other districts, the solution won't be an
    all-or-nothing approach.
    \href{https://bioethics.jhu.edu/research-and-outreach/projects/eschool-initiative/school-policy-tracker/}{Many
    systems}, including the nation's largest, New York City, are
    devising
    \href{https://www.nytimes.com/2020/06/26/us/coronavirus-schools-reopen-fall.html?action=click\&pgtype=Article\&state=default\&region=MAIN_CONTENT_3\&context=storylines_faq}{hybrid
    plans} that involve spending some days in classrooms and other days
    online. There's no national policy on this yet, so check with your
    municipal school system regularly to see what is happening in your
    community.
  \end{itemize}
\item ~
  \hypertarget{is-the-coronavirus-airborne}{%
  \paragraph{Is the coronavirus
  airborne?}\label{is-the-coronavirus-airborne}}

  \begin{itemize}
  \tightlist
  \item
    The coronavirus
    \href{https://www.nytimes.com/2020/07/04/health/239-experts-with-one-big-claim-the-coronavirus-is-airborne.html?action=click\&pgtype=Article\&state=default\&region=MAIN_CONTENT_3\&context=storylines_faq}{can
    stay aloft for hours in tiny droplets in stagnant air}, infecting
    people as they inhale, mounting scientific evidence suggests. This
    risk is highest in crowded indoor spaces with poor ventilation, and
    may help explain super-spreading events reported in meatpacking
    plants, churches and restaurants.
    \href{https://www.nytimes.com/2020/07/06/health/coronavirus-airborne-aerosols.html?action=click\&pgtype=Article\&state=default\&region=MAIN_CONTENT_3\&context=storylines_faq}{It's
    unclear how often the virus is spread} via these tiny droplets, or
    aerosols, compared with larger droplets that are expelled when a
    sick person coughs or sneezes, or transmitted through contact with
    contaminated surfaces, said Linsey Marr, an aerosol expert at
    Virginia Tech. Aerosols are released even when a person without
    symptoms exhales, talks or sings, according to Dr. Marr and more
    than 200 other experts, who
    \href{https://academic.oup.com/cid/article/doi/10.1093/cid/ciaa939/5867798}{have
    outlined the evidence in an open letter to the World Health
    Organization}.
  \end{itemize}
\end{itemize}

``If it's a choice between me being solvent or the tenant staying in
place, I have no choice but being solvent,'' said Mr. Eccles, who said
he was operating in the red this year.

He is still paying off a \$9,000 repair on the building's boiler in
February.

``Contrary to what the tenant advocates say, most of us landlords ---
the real landlords --- going to court is the last option,'' he said. ``I
don't know of anyone who would knowingly evict someone who had a
financial loss because of Covid.''

Nonetheless, eviction is an option that landlords do exercise. There
were 37,000 residential evictions in 2018 and 2019 combined,
\href{https://www1.nyc.gov/assets/hra/downloads/pdf/services/civiljustice/OCJ_Annual_Report_2019.pdf}{according
to the city}, though the annual number has steadily declined in recent
years. There were nearly 3,000 residential evictions this year before
the moratorium took place.

Deborah Metts is worried she will be added to that list.

She lost her marketing executive job in March and could not afford the
next month's rent on her Harlem apartment. Since then, Ms. Metts has
helped start a tenant association in her building, organized a rent
strike among some tenants and asked the property owner to give residents
a break on rent. (The company has not responded, she said.)

``The last few months for me have been scary and really eye-opening,''
Ms. Metts, 37, said. ``The minute you go from privilege, in a sense of
still being employed, to having nothing --- things change very
quickly.''

She said she was more concerned that her neighbors could face eviction
proceedings and have no place to go if they lost their homes.

Of the ZIP codes in New York City with the most evictions in 2018 and
2019, people of color make up more than 96 percent of the population in
those areas, according to census data. All of them were in the Bronx or
Brooklyn.

``When we talk about evictions, the vast majority of those people will
be black and brown people,'' Ms. Metts said. ``If society cares about
social justice, one of the places they need to put their money where
their mouth is is in eviction prevention.''

About 2.3 million tenants in New York City recently received slight
relief on their future rent. The city's Rent Guidelines Board, which
sets rates for rent-regulated apartments,
\href{https://www.nytimes.com/2020/06/17/nyregion/nyc-rent-guidelines-board-freeze.html}{froze
those rents for a year}; it was only the third time the board had not
approved a yearly increase in its 50-year history. The freeze does not
affect market-based rents.

Across the country, including in New York, tenants have rallied around a
broad \#CancelRent campaign, criticizing eviction moratoriums and rent
freezes as inadequate during the economic turmoil.

For the past several months, MD Khan, a cabdriver, has been receiving
letters from his landlord titled ``Notice of failure to pay rent.'' Mr.
Khan, 57, said he stopped driving in mid-March because he worried about
contracting the virus.

Still not working, he has not been able to afford his \$1,175 rent on
his studio apartment in Jamaica, Queens, since March. He said he
believed the letters were the first step in his landlord eventually
taking him to housing court.

``I don't have a job,'' Mr. Khan said, ``and I don't have income.''

Advertisement

\protect\hyperlink{after-bottom}{Continue reading the main story}

\hypertarget{site-index}{%
\subsection{Site Index}\label{site-index}}

\hypertarget{site-information-navigation}{%
\subsection{Site Information
Navigation}\label{site-information-navigation}}

\begin{itemize}
\tightlist
\item
  \href{https://help.nytimes.com/hc/en-us/articles/115014792127-Copyright-notice}{©~2020~The
  New York Times Company}
\end{itemize}

\begin{itemize}
\tightlist
\item
  \href{https://www.nytco.com/}{NYTCo}
\item
  \href{https://help.nytimes.com/hc/en-us/articles/115015385887-Contact-Us}{Contact
  Us}
\item
  \href{https://www.nytco.com/careers/}{Work with us}
\item
  \href{https://nytmediakit.com/}{Advertise}
\item
  \href{http://www.tbrandstudio.com/}{T Brand Studio}
\item
  \href{https://www.nytimes.com/privacy/cookie-policy\#how-do-i-manage-trackers}{Your
  Ad Choices}
\item
  \href{https://www.nytimes.com/privacy}{Privacy}
\item
  \href{https://help.nytimes.com/hc/en-us/articles/115014893428-Terms-of-service}{Terms
  of Service}
\item
  \href{https://help.nytimes.com/hc/en-us/articles/115014893968-Terms-of-sale}{Terms
  of Sale}
\item
  \href{https://spiderbites.nytimes.com}{Site Map}
\item
  \href{https://help.nytimes.com/hc/en-us}{Help}
\item
  \href{https://www.nytimes.com/subscription?campaignId=37WXW}{Subscriptions}
\end{itemize}
