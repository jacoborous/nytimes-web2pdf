Sections

SEARCH

\protect\hyperlink{site-content}{Skip to
content}\protect\hyperlink{site-index}{Skip to site index}

\href{https://www.nytimes.com/section/health}{Health}

\href{https://myaccount.nytimes.com/auth/login?response_type=cookie\&client_id=vi}{}

\href{https://www.nytimes.com/section/todayspaper}{Today's Paper}

\href{/section/health}{Health}\textbar{}Federal Agency Halts Studies of
Hydroxychloroquine, Drug Trump Promoted

\url{https://nyti.ms/311PqQ4}

\begin{itemize}
\item
\item
\item
\item
\item
\end{itemize}

\href{https://www.nytimes.com/news-event/coronavirus?action=click\&pgtype=Article\&state=default\&region=TOP_BANNER\&context=storylines_menu}{The
Coronavirus Outbreak}

\begin{itemize}
\tightlist
\item
  live\href{https://www.nytimes.com/2020/08/02/world/coronavirus-updates.html?action=click\&pgtype=Article\&state=default\&region=TOP_BANNER\&context=storylines_menu}{Latest
  Updates}
\item
  \href{https://www.nytimes.com/interactive/2020/us/coronavirus-us-cases.html?action=click\&pgtype=Article\&state=default\&region=TOP_BANNER\&context=storylines_menu}{Maps
  and Cases}
\item
  \href{https://www.nytimes.com/interactive/2020/science/coronavirus-vaccine-tracker.html?action=click\&pgtype=Article\&state=default\&region=TOP_BANNER\&context=storylines_menu}{Vaccine
  Tracker}
\item
  \href{https://www.nytimes.com/interactive/2020/07/29/us/schools-reopening-coronavirus.html?action=click\&pgtype=Article\&state=default\&region=TOP_BANNER\&context=storylines_menu}{What
  School May Look Like}
\item
  \href{https://www.nytimes.com/live/2020/07/31/business/stock-market-today-coronavirus?action=click\&pgtype=Article\&state=default\&region=TOP_BANNER\&context=storylines_menu}{Economy}
\end{itemize}

Advertisement

\protect\hyperlink{after-top}{Continue reading the main story}

Supported by

\protect\hyperlink{after-sponsor}{Continue reading the main story}

\hypertarget{federal-agency-halts-studies-of-hydroxychloroquine-drug-trump-promoted}{%
\section{Federal Agency Halts Studies of Hydroxychloroquine, Drug Trump
Promoted}\label{federal-agency-halts-studies-of-hydroxychloroquine-drug-trump-promoted}}

The National Institutes of Health decided to stop one trial because the
drug was unlikely to benefit patients, and another because not enough
people enrolled.

\includegraphics{https://static01.nyt.com/images/2020/06/20/science/20virus-drug/merlin_173191026_480392c1-e26f-4edc-918d-ffd2bb29e81b-articleLarge.jpg?quality=75\&auto=webp\&disable=upscale}

By \href{https://www.nytimes.com/by/katie-thomas}{Katie Thomas}

\begin{itemize}
\item
  June 20, 2020
\item
  \begin{itemize}
  \item
  \item
  \item
  \item
  \item
  \end{itemize}
\end{itemize}

The National Institutes of Health said Saturday that it had stopped two
clinical trials of hydroxychloroquine, the malaria drug that President
Trump promoted to treat and prevent the coronavirus, one because the
drug was unlikely to be effective and the other because not enough
patients signed up to participate.

The agency halted a trial that had aimed to enroll more than 500
patients after an independent oversight board determined that the drug
did not appear to benefit hospitalized patients. The same day, the
N.I.H. said it had closed another trial --- of hydroxychloroquine and
the antibiotic azithromycin --- because only about 20 patients had
enrolled in the planned study of 2,000 people.

The two trials the N.I.H. shut down represent the latest evidence that
hydroxychloroquine has not lived up to its early promise of fighting the
coronavirus.

\href{https://www.nih.gov/news-events/news-releases/nih-halts-clinical-trial-hydroxychloroquine}{The
N.I.H. said Saturday} that an independent oversight board that monitors
safety met late Friday to discuss the 500-patient trial and ``determined
that while there was no harm, the study drug was very unlikely to be
beneficial to hospitalized patients with Covid-19,'' the disease caused
by the virus.

``In effect, the drug didn't work,'' said Dr. William Schaffner, a
professor of infectious diseases at Vanderbilt University Medical Center
who was not involved in the research. ``I think we can put this drug
aside and now devote our attention to other potential treatments.''

\hypertarget{latest-updates-global-coronavirus-outbreak}{%
\section{\texorpdfstring{\href{https://www.nytimes.com/2020/08/01/world/coronavirus-covid-19.html?action=click\&pgtype=Article\&state=default\&region=MAIN_CONTENT_1\&context=storylines_live_updates}{Latest
Updates: Global Coronavirus
Outbreak}}{Latest Updates: Global Coronavirus Outbreak}}\label{latest-updates-global-coronavirus-outbreak}}

Updated 2020-08-02T17:52:35.962Z

\begin{itemize}
\tightlist
\item
  \href{https://www.nytimes.com/2020/08/01/world/coronavirus-covid-19.html?action=click\&pgtype=Article\&state=default\&region=MAIN_CONTENT_1\&context=storylines_live_updates\#link-34047410}{The
  U.S. reels as July cases more than double the total of any other
  month.}
\item
  \href{https://www.nytimes.com/2020/08/01/world/coronavirus-covid-19.html?action=click\&pgtype=Article\&state=default\&region=MAIN_CONTENT_1\&context=storylines_live_updates\#link-780ec966}{Top
  U.S. officials work to break an impasse over the federal jobless
  benefit.}
\item
  \href{https://www.nytimes.com/2020/08/01/world/coronavirus-covid-19.html?action=click\&pgtype=Article\&state=default\&region=MAIN_CONTENT_1\&context=storylines_live_updates\#link-2bc8948}{Its
  outbreak untamed, Melbourne goes into even greater lockdown.}
\end{itemize}

\href{https://www.nytimes.com/2020/08/01/world/coronavirus-covid-19.html?action=click\&pgtype=Article\&state=default\&region=MAIN_CONTENT_1\&context=storylines_live_updates}{See
more updates}

More live coverage:
\href{https://www.nytimes.com/live/2020/07/31/business/stock-market-today-coronavirus?action=click\&pgtype=Article\&state=default\&region=MAIN_CONTENT_1\&context=storylines_live_updates}{Markets}

Mr. Trump had called the drug a ``game changer'' and
\href{https://www.nytimes.com/2020/05/18/us/politics/trump-hydroxychloroquine-covid-coronavirus.html}{took
it himself} in hopes of protecting himself from infection with the
coronavirus. Drugmakers donated millions of doses to the federal
stockpile, which distributed them to hospitals around the country, where
doctors with
\href{https://www.nytimes.com/2020/04/17/health/trump-hydroxychloroquine-coronavirus.html}{few
other options} administered the drug to severely ill patients.

On Monday, the Food and Drug Administration
\href{https://www.nytimes.com/2020/06/15/health/fda-hydroxychloroquine-malaria.html}{revoked
the emergency authorization} it had granted to hospitals to give
hydroxychloroquine and a related drug, chloroquine, to patients. The
agency said that the drugs were unlikely to be effective and could carry
potential risks.

While the N.I.H. said it did not identify safety concerns as part of its
review of the 500-patient trial, others have concluded that
hydroxychloroquine carries potential risks. This spring, the F.D.A.
issued a
\href{https://www.nytimes.com/2020/04/24/health/fda-hydroxychloroquine-coronavirus.html}{warning
that the drugs could cause dangerous heart arrhythmias in Covid-19
patients.}

The planned trial, which was being run by the National Heart, Lung, and
Blood Institute, a division of the N.I.H., had enrolled more than 470
patients when the study was stopped. The study sought to learn whether
the drug benefited hospitalized patients and those who visited the
emergency room, as well as people who were likely to be admitted to the
hospital. It was one of several placebo-controlled studies that had been
organized to test the drug after a series of small, poorly controlled
trials showed early signs of a benefit.

The other trial that was halted, involving hydroxychloroquine and
azithromycin, was testing
\href{https://www.nih.gov/news-events/news-releases/nih-begins-clinical-trial-hydroxychloroquine-azithromycin-treat-covid-19}{whether
the drugs}, given together earlier in the disease, could prevent
hospitalization and death from Covid-19. That study, led by the National
Institute of Allergy and Infectious Diseases, an arm of the N.I.H.,
``sought to fill this knowledge gap by testing it in a randomized,
placebo-controlled trial --- considered the gold standard for
determining whether an intervention can benefit patients,''
\href{https://www.niaid.nih.gov/news-events/bulletin-nih-clinical-trial-evaluating-hydroxychloroquine-and-azithromycin-covid-19}{the
agency said in a statement}.

The agency said it did not find any safety problems, but concluded that
the recent decision by the F.D.A. to revoke its emergency authorization
of hydroxychloroquine ``could further dampen enthusiasm for enrollment
in studies evaluating these drugs.''

Several other large trials of hydroxychloroquine have been stopped or
have not shown the drug to be effective.

``No surprise,'' said Dr. David R. Boulware, an infectious disease
specialist at the University of Minnesota, who is studying
hydroxychloroquine as a preventive and an early treatment against the
coronavirus. He said the evidence is increasingly clear that the drug
does not work in hospitalized patients. ``Whether there is any benefit
for pre-exposure prophylaxis or early treatment is as yet unknown, but
the role of hydroxychloroquine --- if any --- appears increasingly
doubtful.''

\href{https://www.nytimes.com/news-event/coronavirus?action=click\&pgtype=Article\&state=default\&region=MAIN_CONTENT_3\&context=storylines_faq}{}

\hypertarget{the-coronavirus-outbreak-}{%
\subsubsection{The Coronavirus Outbreak
›}\label{the-coronavirus-outbreak-}}

\hypertarget{frequently-asked-questions}{%
\paragraph{Frequently Asked
Questions}\label{frequently-asked-questions}}

Updated July 27, 2020

\begin{itemize}
\item ~
  \hypertarget{should-i-refinance-my-mortgage}{%
  \paragraph{Should I refinance my
  mortgage?}\label{should-i-refinance-my-mortgage}}

  \begin{itemize}
  \tightlist
  \item
    \href{https://www.nytimes.com/article/coronavirus-money-unemployment.html?action=click\&pgtype=Article\&state=default\&region=MAIN_CONTENT_3\&context=storylines_faq}{It
    could be a good idea,} because mortgage rates have
    \href{https://www.nytimes.com/2020/07/16/business/mortgage-rates-below-3-percent.html?action=click\&pgtype=Article\&state=default\&region=MAIN_CONTENT_3\&context=storylines_faq}{never
    been lower.} Refinancing requests have pushed mortgage applications
    to some of the highest levels since 2008, so be prepared to get in
    line. But defaults are also up, so if you're thinking about buying a
    home, be aware that some lenders have tightened their standards.
  \end{itemize}
\item ~
  \hypertarget{what-is-school-going-to-look-like-in-september}{%
  \paragraph{What is school going to look like in
  September?}\label{what-is-school-going-to-look-like-in-september}}

  \begin{itemize}
  \tightlist
  \item
    It is unlikely that many schools will return to a normal schedule
    this fall, requiring the grind of
    \href{https://www.nytimes.com/2020/06/05/us/coronavirus-education-lost-learning.html?action=click\&pgtype=Article\&state=default\&region=MAIN_CONTENT_3\&context=storylines_faq}{online
    learning},
    \href{https://www.nytimes.com/2020/05/29/us/coronavirus-child-care-centers.html?action=click\&pgtype=Article\&state=default\&region=MAIN_CONTENT_3\&context=storylines_faq}{makeshift
    child care} and
    \href{https://www.nytimes.com/2020/06/03/business/economy/coronavirus-working-women.html?action=click\&pgtype=Article\&state=default\&region=MAIN_CONTENT_3\&context=storylines_faq}{stunted
    workdays} to continue. California's two largest public school
    districts --- Los Angeles and San Diego --- said on July 13, that
    \href{https://www.nytimes.com/2020/07/13/us/lausd-san-diego-school-reopening.html?action=click\&pgtype=Article\&state=default\&region=MAIN_CONTENT_3\&context=storylines_faq}{instruction
    will be remote-only in the fall}, citing concerns that surging
    coronavirus infections in their areas pose too dire a risk for
    students and teachers. Together, the two districts enroll some
    825,000 students. They are the largest in the country so far to
    abandon plans for even a partial physical return to classrooms when
    they reopen in August. For other districts, the solution won't be an
    all-or-nothing approach.
    \href{https://bioethics.jhu.edu/research-and-outreach/projects/eschool-initiative/school-policy-tracker/}{Many
    systems}, including the nation's largest, New York City, are
    devising
    \href{https://www.nytimes.com/2020/06/26/us/coronavirus-schools-reopen-fall.html?action=click\&pgtype=Article\&state=default\&region=MAIN_CONTENT_3\&context=storylines_faq}{hybrid
    plans} that involve spending some days in classrooms and other days
    online. There's no national policy on this yet, so check with your
    municipal school system regularly to see what is happening in your
    community.
  \end{itemize}
\item ~
  \hypertarget{is-the-coronavirus-airborne}{%
  \paragraph{Is the coronavirus
  airborne?}\label{is-the-coronavirus-airborne}}

  \begin{itemize}
  \tightlist
  \item
    The coronavirus
    \href{https://www.nytimes.com/2020/07/04/health/239-experts-with-one-big-claim-the-coronavirus-is-airborne.html?action=click\&pgtype=Article\&state=default\&region=MAIN_CONTENT_3\&context=storylines_faq}{can
    stay aloft for hours in tiny droplets in stagnant air}, infecting
    people as they inhale, mounting scientific evidence suggests. This
    risk is highest in crowded indoor spaces with poor ventilation, and
    may help explain super-spreading events reported in meatpacking
    plants, churches and restaurants.
    \href{https://www.nytimes.com/2020/07/06/health/coronavirus-airborne-aerosols.html?action=click\&pgtype=Article\&state=default\&region=MAIN_CONTENT_3\&context=storylines_faq}{It's
    unclear how often the virus is spread} via these tiny droplets, or
    aerosols, compared with larger droplets that are expelled when a
    sick person coughs or sneezes, or transmitted through contact with
    contaminated surfaces, said Linsey Marr, an aerosol expert at
    Virginia Tech. Aerosols are released even when a person without
    symptoms exhales, talks or sings, according to Dr. Marr and more
    than 200 other experts, who
    \href{https://academic.oup.com/cid/article/doi/10.1093/cid/ciaa939/5867798}{have
    outlined the evidence in an open letter to the World Health
    Organization}.
  \end{itemize}
\item ~
  \hypertarget{what-are-the-symptoms-of-coronavirus}{%
  \paragraph{What are the symptoms of
  coronavirus?}\label{what-are-the-symptoms-of-coronavirus}}

  \begin{itemize}
  \tightlist
  \item
    Common symptoms
    \href{https://www.nytimes.com/article/symptoms-coronavirus.html?action=click\&pgtype=Article\&state=default\&region=MAIN_CONTENT_3\&context=storylines_faq}{include
    fever, a dry cough, fatigue and difficulty breathing or shortness of
    breath.} Some of these symptoms overlap with those of the flu,
    making detection difficult, but runny noses and stuffy sinuses are
    less common.
    \href{https://www.nytimes.com/2020/04/27/health/coronavirus-symptoms-cdc.html?action=click\&pgtype=Article\&state=default\&region=MAIN_CONTENT_3\&context=storylines_faq}{The
    C.D.C. has also} added chills, muscle pain, sore throat, headache
    and a new loss of the sense of taste or smell as symptoms to look
    out for. Most people fall ill five to seven days after exposure, but
    symptoms may appear in as few as two days or as many as 14 days.
  \end{itemize}
\item ~
  \hypertarget{does-asymptomatic-transmission-of-covid-19-happen}{%
  \paragraph{Does asymptomatic transmission of Covid-19
  happen?}\label{does-asymptomatic-transmission-of-covid-19-happen}}

  \begin{itemize}
  \tightlist
  \item
    So far, the evidence seems to show it does. A widely cited
    \href{https://www.nature.com/articles/s41591-020-0869-5}{paper}
    published in April suggests that people are most infectious about
    two days before the onset of coronavirus symptoms and estimated that
    44 percent of new infections were a result of transmission from
    people who were not yet showing symptoms. Recently, a top expert at
    the World Health Organization stated that transmission of the
    coronavirus by people who did not have symptoms was ``very rare,''
    \href{https://www.nytimes.com/2020/06/09/world/coronavirus-updates.html?action=click\&pgtype=Article\&state=default\&region=MAIN_CONTENT_3\&context=storylines_faq\#link-1f302e21}{but
    she later walked back that statement.}
  \end{itemize}
\end{itemize}

On Wednesday, the
\href{https://www.who.int/emergencies/diseases/novel-coronavirus-2019/global-research-on-novel-coronavirus-2019-ncov/solidarity-clinical-trial-for-covid-19-treatments}{World
Health Organization said} it was stopping the hydroxychloroquine arm of
a large clinical trial that is testing several treatments against the
virus because evidence ---
\href{https://www.recoverytrial.net/news/statement-from-the-chief-investigators-of-the-randomised-evaluation-of-covid-19-therapy-recovery-trial-on-hydroxychloroquine-5-june-2020-no-clinical-benefit-from-use-of-hydroxychloroquine-in-hospitalised-patients-with-covid-19}{including
results from a large study} in the United Kingdom --- showed it did not
reduce mortality rates of hospitalized patients.

And on Friday, the Swiss drugmaker
\href{https://www.nytimes.com/reuters/2020/06/19/world/europe/19reuters-health-coronavirus-novartis-hydroxychloroquine.html}{Novartis
said it was halting its clinical trial} because it could not recruit
enough patients to sign up. Only a handful enrolled even though the
company had planned for a study involving 440 people.

Other researchers are still studying whether hydroxychloroquine could be
used earlier in the course of the illness, or to prevent people from
getting infected from the virus. While some of those studies are still
underway,
\href{https://www.nytimes.com/2020/06/03/health/hydroxychloroquine-coronavirus-trump.html}{one
such trial --- led by Dr. Boulware --- concluded earlier this month}
that hydroxychloroquine was not effective in preventing infections after
someone had been exposed.

That trial included 821 people and looked at whether those who were
exposed to the virus, such as health care workers or family members of
people who had been infected, were less or more likely to be infected
themselves if they were given the drug. It found that there was not a
meaningful difference.

\textbf{\emph{{[}}\href{http://on.fb.me/1paTQ1h}{\emph{Like the Science
Times page on Facebook.}}} ****** \emph{\textbar{} Sign up for the}
\textbf{\href{http://nyti.ms/1MbHaRU}{\emph{Science Times
newsletter.}}\emph{{]}}}

Advertisement

\protect\hyperlink{after-bottom}{Continue reading the main story}

\hypertarget{site-index}{%
\subsection{Site Index}\label{site-index}}

\hypertarget{site-information-navigation}{%
\subsection{Site Information
Navigation}\label{site-information-navigation}}

\begin{itemize}
\tightlist
\item
  \href{https://help.nytimes.com/hc/en-us/articles/115014792127-Copyright-notice}{©~2020~The
  New York Times Company}
\end{itemize}

\begin{itemize}
\tightlist
\item
  \href{https://www.nytco.com/}{NYTCo}
\item
  \href{https://help.nytimes.com/hc/en-us/articles/115015385887-Contact-Us}{Contact
  Us}
\item
  \href{https://www.nytco.com/careers/}{Work with us}
\item
  \href{https://nytmediakit.com/}{Advertise}
\item
  \href{http://www.tbrandstudio.com/}{T Brand Studio}
\item
  \href{https://www.nytimes.com/privacy/cookie-policy\#how-do-i-manage-trackers}{Your
  Ad Choices}
\item
  \href{https://www.nytimes.com/privacy}{Privacy}
\item
  \href{https://help.nytimes.com/hc/en-us/articles/115014893428-Terms-of-service}{Terms
  of Service}
\item
  \href{https://help.nytimes.com/hc/en-us/articles/115014893968-Terms-of-sale}{Terms
  of Sale}
\item
  \href{https://spiderbites.nytimes.com}{Site Map}
\item
  \href{https://help.nytimes.com/hc/en-us}{Help}
\item
  \href{https://www.nytimes.com/subscription?campaignId=37WXW}{Subscriptions}
\end{itemize}
