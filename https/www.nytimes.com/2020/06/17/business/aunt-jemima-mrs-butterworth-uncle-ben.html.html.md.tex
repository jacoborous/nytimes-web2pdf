Sections

SEARCH

\protect\hyperlink{site-content}{Skip to
content}\protect\hyperlink{site-index}{Skip to site index}

\href{https://www.nytimes.com/section/business}{Business}

\href{https://myaccount.nytimes.com/auth/login?response_type=cookie\&client_id=vi}{}

\href{https://www.nytimes.com/section/todayspaper}{Today's Paper}

\href{/section/business}{Business}\textbar{}After Aunt Jemima, Reviews
Underway for Uncle Ben, Mrs. Butterworth and Cream of Wheat

\url{https://nyti.ms/3hHiQcu}

\begin{itemize}
\item
\item
\item
\item
\item
\end{itemize}

\href{https://www.nytimes.com/news-event/george-floyd-protests-minneapolis-new-york-los-angeles?action=click\&pgtype=Article\&state=default\&region=TOP_BANNER\&context=storylines_menu}{Race
and America}

\begin{itemize}
\tightlist
\item
  \href{https://www.nytimes.com/2020/07/26/us/protests-portland-seattle-trump.html?action=click\&pgtype=Article\&state=default\&region=TOP_BANNER\&context=storylines_menu}{Protesters
  Return to Other Cities}
\item
  \href{https://www.nytimes.com/2020/07/24/us/portland-oregon-protests-white-race.html?action=click\&pgtype=Article\&state=default\&region=TOP_BANNER\&context=storylines_menu}{Portland
  at the Center}
\item
  \href{https://www.nytimes.com/2020/07/23/podcasts/the-daily/portland-protests.html?action=click\&pgtype=Article\&state=default\&region=TOP_BANNER\&context=storylines_menu}{Podcast:
  Showdown in Portland}
\item
  \href{https://www.nytimes.com/interactive/2020/07/16/us/black-lives-matter-protests-louisville-breonna-taylor.html?action=click\&pgtype=Article\&state=default\&region=TOP_BANNER\&context=storylines_menu}{45
  Days in Louisville}
\end{itemize}

Advertisement

\protect\hyperlink{after-top}{Continue reading the main story}

Supported by

\protect\hyperlink{after-sponsor}{Continue reading the main story}

\hypertarget{after-aunt-jemima-reviews-underway-for-uncle-ben-mrs-butterworth-and-cream-of-wheat}{%
\section{After Aunt Jemima, Reviews Underway for Uncle Ben, Mrs.
Butterworth and Cream of
Wheat}\label{after-aunt-jemima-reviews-underway-for-uncle-ben-mrs-butterworth-and-cream-of-wheat}}

People have long complained about the use of African-American
stereotypes in marketing. The Aunt Jemima decision has prompted more
companies to take action.

\includegraphics{https://static01.nyt.com/images/2020/06/17/multimedia/17xp-unrest-brands1/merlin_173622012_b49d8e8d-9599-419e-b4bf-9ff1701bd2ee-articleLarge.jpg?quality=75\&auto=webp\&disable=upscale}

By \href{https://www.nytimes.com/by/maria-cramer}{Maria Cramer}

\begin{itemize}
\item
  June 17, 2020
\item
  \begin{itemize}
  \item
  \item
  \item
  \item
  \item
  \end{itemize}
\end{itemize}

\href{https://www.nytimes.com/es/2020/06/18/espanol/negocios/aunt-jemima-racismo-estereotipos.html}{Leer
en español}

Within hours of
\href{https://www.nytimes.com/2020/06/17/business/aunt-jemima-racial-stereotype.html}{the
announcement that Aunt Jemima was being retired} from store shelves, at
least three more food companies rushed to respond to complaints about
other brands that have been criticized for using racial stereotypes.

Mars Food, the owner of the brand Uncle Ben's rice, which features an
older black man smiling on the box, said on Wednesday afternoon that it
would ``evolve'' the brand as protests over racism and police brutality
across the country continue.

``We recognize that now is the right time to evolve the Uncle Ben's
brand, including its visual brand identity, which we will do,'' said
Caroline Sherman, a spokeswoman for Mars. The company does not know the
nature of the changes, or the timing, she said, ``but we are evaluating
all possibilities.''

Shortly after that announcement, ConAgra Brands, the maker of Mrs.
Butterworth's pancake syrup,
\href{https://www.conagrabrands.com/news-room/news-conagra-brands-announces-mrs-butterworths-brand-review-prn-122733}{released
a statement} saying the company had begun a ``complete brand and package
review.''

Critics have long associated the shape of the Mrs. Butterworth's bottle
with the mammy, a caricature of black women as subservient to white
people.

And later on Wednesday, B\&G Foods Inc., the parent company of Cream of
Wheat, announced that it too was conducting a review of its packaging.

The porridge box, which depicts a beaming black man in a white chef's
uniform, has not been altered much since its debut in the late 19th
century. The character was named ``Rastus,'' a pejorative term for black
men, and he was once depicted as a barely literate cook who did not know
what vitamins were.

``We understand there are concerns regarding the chef image,'' the
company said in its statement, ``and we are committed to evaluating our
packaging and will proactively take steps to ensure that we and our
brands do not inadvertently contribute to systemic racism.''

\includegraphics{https://static01.nyt.com/images/2020/06/17/multimedia/17xp-unrest-brands2/merlin_152462895_c6bd053b-361b-4246-9f44-1e8b2e01cca2-articleLarge.jpg?quality=75\&auto=webp\&disable=upscale}

\hypertarget{complaints-about-the-images-go-back-many-years}{%
\subsection{Complaints about the images go back many
years.}\label{complaints-about-the-images-go-back-many-years}}

The recent widespread anti-racism protests have renewed the focus on
companies that for decades have used racial images to sell rice,
breakfast foods, dairy products and snacks, among other products and
services.

They have fielded complaints about these depictions before, and they
have sometimes made adjustments. In 2007, Uncle Ben, whose face has
appeared on the box of rice since the 1940s, was promoted from a servant
to chairman of the board.

But Kevin D. Thomas, a professor of multicultural branding in the Race,
Ethnic and Indigenous Studies Program at Marquette University, said he
hoped the current push for change would finally lead to a substantial
overhaul in the marketing world.

``I'm hoping this is a reckoning and we're going to start seeing
something that becomes pervasive,'' Professor Thomas said.

For decades, many have expressed concerns about the matronly shape of
the Mrs. Butterworth's container.

``I think the key issue with Mrs. Butterworth is her physical shape,
which strongly resembles the mammy caricature,'' Professor Thomas said.
``So while she's been personified as an elderly white woman, mainly
through \href{https://www.youtube.com/watch?v=z45ys7oJuCA}{vocal
affect}, her physique and style of dress bear a striking resemblance to
that of the mammy.''

In its statement, ConAgra Brands said Mrs. Butterworth was ``intended to
evoke the images of a loving grandmother.'' But the company said it
wanted to stand in solidarity with ``our black and brown communities,
and we can see that our packaging may be interpreted in a way that is
wholly inconsistent with our values.''

The images of placid, smiling African-Americans on commercial products
were often created during times of racial upheaval, Professor Thomas
said.

Characters like Aunt Jemima, who was first depicted as a mammy, followed
Reconstruction when white people were scared of what it meant to live
alongside newly freed slaves, he said.

``There was a lot of angst around that. There was terror and a sense of
what does this mean for white supremacy?'' he said.

Professor Thomas suggested that the advertisers were trying to market
products around those fears: ``Can we assuage some of that to get back
to those quote-un-quote calmer days when we had the slave in the kitchen
taking care of our kids?''

Another intent of stereotypes in marketing is to make some goods seem
more exotic, experts say.

Chiquita Banana's ambassador is Miss Chiquita, who carries a basket of
fruit on her head and wears a tight, stereotypical Latin dance costume
made up of ruffles.

Image

Chiquita Banana's ambassador, Miss Chiquita, is meant to portray
something exotic, but that can have the effect of marginalizing people,
one expert said.Credit...Marco Ugarte/Associated Press

``Chiquita Banana has that sort of alluring representation that is meant
to give people this vision of something that is exotic and other,'' said
Rebecca Hains, a professor of media and communication at Salem State
University in Massachusetts. ``But othering people is really
problematic. It marginalizes people and suggests that they're not
important or equal to the majority.''

This is not only a phenomenon in the United States. In 2009, a young
Inuit
woman\href{https://www.odt.co.nz/news/national/eskimo-makers-defy-racism-claims}{publicly
denounced} Pascall, a candy manufacturer in Australia and New Zealand,
for appropriating her culture to sell its ``Eskimo'' marshmallows and
other sweets. The company refused to change the name.

\hypertarget{have-companies-dropped-or-modified-brands-in-the-past}{%
\subsection{Have companies dropped or modified brands in the
past?}\label{have-companies-dropped-or-modified-brands-in-the-past}}

Indeed.

The
\href{https://www.nytimes.com/2020/04/17/business/land-o-lakes-butter.html}{Native
American woman who once adorned packages of Land O'Lakes} cheese and
butter was removed this year.

Beth Ford, the Land O'Lakes chief executive, said in February that it
was time the company recognize the need for ``packaging that reflects
the foundation and heart of our company culture.''

In 1967, Frito-Lay introduced the
\href{https://www.youtube.com/watch?v=fOUilxJWm24}{``Frito Bandito,''} a
gun-toting Mexican who spoke with a thick accent and threatened to steal
chips from kids.

Mexican-American advocacy groups denounced the character and demanded
the company stop using it to sell chips. Frito responded by making the
Frito Bandito less unkempt. His beard was shaved and his gold tooth was
removed, but the character did not fully disappear until around 1971.

In the 1950s, the Sambo's chain began opening pancake restaurants by the
hundreds across the United States. The founders, Sam Battistone Sr. and
Newell Bohnett, said the restaurant's name was based on the first
letters of their names. But the name was long reviled as racist, and in
many towns the restaurant
\href{https://www.nytimes.com/1978/11/18/archives/sambos-under-fire-over-name-sambos-is-under-attack-over-name.html}{rebranded
itself} as ``The Jolly Tiger'' under local pressure.

Last week, the last Sambo's in the United States, located in Santa
Barbara, Calif., where the chain started, decided to finally change the
name. For now, workers have covered the sign outside with a peace
symbol, an ampersand and the word ``love.''

\href{https://www.change.org/p/sambos-is-a-racial-slur-help-me-change-the-name-of-the-restaurant}{Customers
had circulated a petition} this month seeking the change, and the owners
agreed it was time.

Image

Sambo's pancake restaurant in 2012.Credit...Steve Hamblin/Alamy

``Our family has looked into our hearts and realize that we must be
sensitive when others whom we respect make a strong appeal,''
\href{https://www.facebook.com/sambosrestaurant/?rf=102187006489323}{the
restaurant owners said on Facebook}.

``We are starting over and will try again until we get it done,'' the
owners said. ``Let's continue to pull together as a community and be
better for this moment in history.''

Sheelagh McNeill contributed research and Neil Vigdor contributed
reporting**.**

Advertisement

\protect\hyperlink{after-bottom}{Continue reading the main story}

\hypertarget{site-index}{%
\subsection{Site Index}\label{site-index}}

\hypertarget{site-information-navigation}{%
\subsection{Site Information
Navigation}\label{site-information-navigation}}

\begin{itemize}
\tightlist
\item
  \href{https://help.nytimes.com/hc/en-us/articles/115014792127-Copyright-notice}{©~2020~The
  New York Times Company}
\end{itemize}

\begin{itemize}
\tightlist
\item
  \href{https://www.nytco.com/}{NYTCo}
\item
  \href{https://help.nytimes.com/hc/en-us/articles/115015385887-Contact-Us}{Contact
  Us}
\item
  \href{https://www.nytco.com/careers/}{Work with us}
\item
  \href{https://nytmediakit.com/}{Advertise}
\item
  \href{http://www.tbrandstudio.com/}{T Brand Studio}
\item
  \href{https://www.nytimes.com/privacy/cookie-policy\#how-do-i-manage-trackers}{Your
  Ad Choices}
\item
  \href{https://www.nytimes.com/privacy}{Privacy}
\item
  \href{https://help.nytimes.com/hc/en-us/articles/115014893428-Terms-of-service}{Terms
  of Service}
\item
  \href{https://help.nytimes.com/hc/en-us/articles/115014893968-Terms-of-sale}{Terms
  of Sale}
\item
  \href{https://spiderbites.nytimes.com}{Site Map}
\item
  \href{https://help.nytimes.com/hc/en-us}{Help}
\item
  \href{https://www.nytimes.com/subscription?campaignId=37WXW}{Subscriptions}
\end{itemize}
