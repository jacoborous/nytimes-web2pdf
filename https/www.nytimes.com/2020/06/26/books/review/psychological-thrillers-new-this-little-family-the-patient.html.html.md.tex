Sections

SEARCH

\protect\hyperlink{site-content}{Skip to
content}\protect\hyperlink{site-index}{Skip to site index}

\href{https://www.nytimes.com/section/books/review}{Book Review}

\href{https://myaccount.nytimes.com/auth/login?response_type=cookie\&client_id=vi}{}

\href{https://www.nytimes.com/section/todayspaper}{Today's Paper}

\href{/section/books/review}{Book Review}\textbar{}Psychological
Thrillers That Will Mess With Your Head

\url{https://nyti.ms/2NwUrrZ}

\begin{itemize}
\item
\item
\item
\item
\item
\end{itemize}

Advertisement

\protect\hyperlink{after-top}{Continue reading the main story}

Supported by

\protect\hyperlink{after-sponsor}{Continue reading the main story}

FIction

\hypertarget{psychological-thrillers-that-will-mess-with-your-head}{%
\section{Psychological Thrillers That Will Mess With Your
Head}\label{psychological-thrillers-that-will-mess-with-your-head}}

\includegraphics{https://static01.nyt.com/images/2020/06/28/books/review/28Lyall-SUB01/28Lyall-SUB01-articleLarge.jpg?quality=75\&auto=webp\&disable=upscale}

By \href{https://www.nytimes.com/by/sarah-lyall}{Sarah Lyall}

\begin{itemize}
\item
  June 26, 2020
\item
  \begin{itemize}
  \item
  \item
  \item
  \item
  \item
  \end{itemize}
\end{itemize}

Summer is full upon us, and what we need most are thrillers to take us
out of our own heads. Along comes \textbf{THIS LITTLE FAMILY (Other
Press, 264 pp., paper, \$15.99),} a short, sharp debut by the French
author Inès Bayard, which has murder in its first paragraph. Marie, a
young Parisian mother, laces a lovingly prepared meal with poison and
feeds it to her husband and her toddler before eating it herself. The
worst part, perhaps, is that one of them survives.

How did she arrive at this terrible moment? The rest of the story
explains what happened to make a lovely Frenchwoman with everything to
live for descend into murderous despair. It is also a cry of anguish and
fury about how easily men get away with committing sexual violence, and
how devastating the consequences are for women.

Bayard is a muscular, propulsive writer, and the novel, elegantly
translated by Adriana Hunter, feels urgent, immediate. You can't stop
reading, even as you want to look away.

As is so often the case, there is a dead body --- Sara Morgan, slashed
to death by her boyfriend, Blake Campbell --- at the center of Nicola
Maye Goldberg's fascinating \textbf{NOTHING CAN HURT YOU (Bloomsbury,
225 pp., \$26).} But the book is less about the murder itself than about
its aftermath, the long tendrils of guilt, sadness, anger and confusion
that stretch out from a single act, wrapping themselves around everyone
they touch.

Each chapter focuses on a person affected by the killing, sometimes
peripherally. There is a young woman in a Florida rehab facility who
meets Blake years after the murder. (He got off by pleading temporary
insanity.) There is the prosecutor in the case, whose private burden is
a sister who never recovered from being raped 30 years earlier by three
teenage boys.

There is Sara's niece, a first grader whose tooth-related obsession will
unsettle readers who remember Gillian Flynn's ``Sharp Objects.'' There
is a girl Sara babysat for; there are her parents; there is the troubled
woman who discovered her body --- each gets a moment. The last chapter
is reserved for Sara, and Goldberg lovingly describes her not at the
moment of death but at an earlier time, when she was young and brave and
full of life.

Imagine a mental patient so malevolent, so diabolical, that anyone who
comes into contact with him --- psychiatrists, nurses, orderlies, other
patients --- is driven insane. Some of them commit suicide. That is the
delightfully bonkers premise of Jasper DeWitt's \textbf{THE PATIENT
(Houghton Mifflin Harcourt, 210 pp., \$23),} set in a forbidding state
psychiatric institution in Connecticut.

Into this fraught atmosphere waltzes Parker, a full-of-himself young
doctor eager to make his mark. What a fun challenge, he thinks: an
untreatable patient, locked up for more than two decades, who is evil
incarnate! Clearly he is the person to crack this case. Ha, ha is the
correct response to this delusionary view. But it's not until Parker
meets Joe, the patient in question, that the reader begins to get an
inkling of what he is up against.

Is Parker insane? Is Joe? How about the other doctors? And what are we
to make of the noise emanating from Joe's room? ``It didn't sound human
at all,'' DeWitt writes. ``Instead, what emerged from that room was a
sepulchral, moist, hacking chuckle that sounded like it came from a
rotting throat.''

Debra Jo Immergut's stunning \textbf{YOU AGAIN (Ecco/HarperCollins, 265
pp., \$27.99)} feels eerily relevant, perfect for this time of deep
uncertainty and rapidly shifting news. It is dreamlike and immersive,
like falling into someone else's alternative reality.

It is 2015 in New York City, and 46-year-old Abby Willard, who works for
a pharmaceutical company, looks out her taxi window to see herself,
walking down the street at the age of 22. Her inclination to write it
off as a momentary illusion becomes harder to sustain when she keeps
running into the same girl at all her old city haunts, doing the things
she once did.

Back then she was wild, in love with a photographer who was addicted to
heroin, preparing a portfolio of paintings for her art school
application. What happened to her life as an artist? How did she get
here?

There is more to disturb. One of Abby's teenage sons seems to have
joined an Antifa group, at least as it operates in Brooklyn --- people
who dress in black and hatch plans to overthrow the rich. Her marriage
buckles. She starts to suffer from headaches, confusion, a sense of
temporal porousness. There's so much she wants to tell her younger self.

It turns out that there are a few things her younger self wants to tell
her, too.

Advertisement

\protect\hyperlink{after-bottom}{Continue reading the main story}

\hypertarget{site-index}{%
\subsection{Site Index}\label{site-index}}

\hypertarget{site-information-navigation}{%
\subsection{Site Information
Navigation}\label{site-information-navigation}}

\begin{itemize}
\tightlist
\item
  \href{https://help.nytimes.com/hc/en-us/articles/115014792127-Copyright-notice}{©~2020~The
  New York Times Company}
\end{itemize}

\begin{itemize}
\tightlist
\item
  \href{https://www.nytco.com/}{NYTCo}
\item
  \href{https://help.nytimes.com/hc/en-us/articles/115015385887-Contact-Us}{Contact
  Us}
\item
  \href{https://www.nytco.com/careers/}{Work with us}
\item
  \href{https://nytmediakit.com/}{Advertise}
\item
  \href{http://www.tbrandstudio.com/}{T Brand Studio}
\item
  \href{https://www.nytimes.com/privacy/cookie-policy\#how-do-i-manage-trackers}{Your
  Ad Choices}
\item
  \href{https://www.nytimes.com/privacy}{Privacy}
\item
  \href{https://help.nytimes.com/hc/en-us/articles/115014893428-Terms-of-service}{Terms
  of Service}
\item
  \href{https://help.nytimes.com/hc/en-us/articles/115014893968-Terms-of-sale}{Terms
  of Sale}
\item
  \href{https://spiderbites.nytimes.com}{Site Map}
\item
  \href{https://help.nytimes.com/hc/en-us}{Help}
\item
  \href{https://www.nytimes.com/subscription?campaignId=37WXW}{Subscriptions}
\end{itemize}
