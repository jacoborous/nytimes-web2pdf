Sections

SEARCH

\protect\hyperlink{site-content}{Skip to
content}\protect\hyperlink{site-index}{Skip to site index}

\href{https://myaccount.nytimes.com/auth/login?response_type=cookie\&client_id=vi}{}

\href{https://www.nytimes.com/section/todayspaper}{Today's Paper}

\href{/section/opinion}{Opinion}\textbar{}Last Testament of Maurice the
Rooster

\href{https://nyti.ms/2NzGd9W}{https://nyti.ms/2NzGd9W}

\begin{itemize}
\item
\item
\item
\item
\item
\item
\end{itemize}

Advertisement

\protect\hyperlink{after-top}{Continue reading the main story}

\href{/section/opinion}{Opinion}

Supported by

\protect\hyperlink{after-sponsor}{Continue reading the main story}

\hypertarget{last-testament-of-maurice-the-rooster}{%
\section{Last Testament of Maurice the
Rooster}\label{last-testament-of-maurice-the-rooster}}

Cultivate your garden. That never disappoints.

\href{https://www.nytimes.com/by/roger-cohen}{\includegraphics{https://static01.nyt.com/images/2014/11/01/opinion/cohen-circular/cohen-circular-thumbLarge-v6.png}}

By \href{https://www.nytimes.com/by/roger-cohen}{Roger Cohen}

Opinion Columnist

\begin{itemize}
\item
  June 26, 2020
\item
  \begin{itemize}
  \item
  \item
  \item
  \item
  \item
  \item
  \end{itemize}
\end{itemize}

\includegraphics{https://static01.nyt.com/images/2020/06/26/opinion/00cohen1/merlin_156502959_553d0150-41dc-43c6-963a-5e82a1edb23f-articleLarge.jpg?quality=75\&auto=webp\&disable=upscale}

Meanwhile, in other news, Maurice,
\href{https://www.nytimes.com/2019/06/23/world/europe/france-rural-urban-rooster.html}{the
most famous rooster in France}, is dead.

I know, there's been a lot to think about. Keeping six feet apart,
losing jobs, living in rectangular Zoom boxes, learning new unhappy
forms of greeting, dealing with bored children, making payroll, getting
used to the deprivations of a virtual life. It's not been easy to
separate the wheat from the chaff, as Maurice might have put it.

The crowing coq from Oléron, a small island off France's western coast,
became a national hero last year when he and his owner were sued by
second-home neighbors who wanted Maurice removed for making too much
noise and waking them up on their vacation.

A great French fight pitting rural tradition and terroir (that ineffable
mix of soil, sun and moisture that define a place and a person's
immemorial connection to it) against tourism and modernity was engaged.

This was a case of deep France versus globalization, heritage versus
holidays, the rooted chicken owner versus the rootless urban dweller, a
parable of our times. A cockerel in a culture war is a formidable thing.

About
\href{https://www.thelocal.fr/20190905/maurice-the-cockerel-to-learn-his-fate-in-row-over-noise-in-rural-france}{140,000
people signed a petition} supporting a rooster's right to make a noise.
(The crowing Gallic coq is of course an eternal symbol of France.) Last
September,
\href{https://www.nytimes.com/2019/09/05/world/europe/france-maurice-rooster.html}{a
judge ruled in Maurice's favor} and his lawyer, Julien Papineau,
pronounced a great truth: ``This rooster was not being unbearable. He
was just being himself.''

\includegraphics{https://static01.nyt.com/images/2020/06/27/opinion/27cohen-tall/merlin_156502533_14714212-4edc-4bf5-bd5b-c8d4f8e5c7e0-articleLarge.jpg?quality=75\&auto=webp\&disable=upscale}

Now Maurice is no more. Perhaps the stress got to him. Corinne Fesseau,
his owner, announced last week that he had died in May of coryza --- a
respiratory infection common to chickens --- and she had buried him in
her garden. She waited to divulge the news because France was in crisis
and ``Covid-19 was more important than my cockerel.''

Maurice, whom my colleague Adam Nossiter memorably described as ``a
cantankerous fowl with a magnificent puffed-out coat,'' was 6 years old.
Fesseau offered this epitaph: ``Maurice was an emblem, a symbol of rural
life and a hero.''

She did not allude to Maurice's last will and testament, but a neighbor
in St.-Pierre-d'Oléron, where the rooster lived and died, sent it along
to me:

\begin{quote}
I am not a hero. That's an overused word. I spoke my own truth. I did
what came naturally to me. Many things change but the essential things
do not.

The sun sets. The sun rises. Shaking my wattles, raising my head, I had
to greet the morning. I could never resist, and why should I have? I had
to crow. This was my particular joy, my particular thing. Each of us has
one. Honor it.

I am sorry to have caused a fuss. I never wanted to annoy anyone. Those
neighbors from Limoges, with their busy city lives, I know they wanted
their peace. They had been saving for their summer vacation. Perhaps
what they missed is that a sound, like my crowing or a ship's foghorn or
a train whistle, may form part of the peace of a place.

A little more patience, a little less agitation, never did any harm. I
never went anywhere, and I was happy. There's more to a coop than meets
the eye. There's more to any place if you look long enough.

I was content to have three hens as companions. They kept me busy.
Contentment, for me, was being attuned to the rhythms and cycles of
life. The chicken and the egg.

This is a strange season to be ending my days on this small planet.
Human beings, so restless, seem fearful. I hear there is a virus. I am
not sure exactly what the virus is. I think the virus is many things. It
always lurks, and it will pass, and some other scourge will appear. Keep
your eye on the sunrise.

My countrymen are angry. What else is new? It's always too much or too
little in France but, my God, what a country of boundless pleasures!
Bastille Day is coming along. Off with their heads, out with the old, in
with the new! We French are revolution specialists. The world needs a
good revolution now and then.

Even if everything changes so that everything can stay the same.
Cultivate your garden. That never disappoints.

I will miss Corinne. I will miss strutting about. I will miss puffing
out my plumage and making heads turn (yes, I admit it, I noticed that).
I will miss emptying my lungs in the dawn, such a perfect feeling. I
will miss the little familiar sounds that offer comfort.

I bequeath the 1,000 euros the judge awarded me to the establishment of
an online (yes!) audio museum of rural sounds. Lest this hectic world
forget.

May peace spread across the earth, but please do not confuse peace with
silence.

Maurice the Rooster
\end{quote}

We live in earnest, sensitive and literal times, so I had better specify
that I made that up. There's a lot to be said for make-believe.
Especially when you are living in a socially distanced box.

\emph{The Times is committed to publishing}
\href{https://www.nytimes.com/2019/01/31/opinion/letters/letters-to-editor-new-york-times-women.html}{\emph{a
diversity of letters}} \emph{to the editor. We'd like to hear what you
think about this or any of our articles. Here are some}
\href{https://help.nytimes.com/hc/en-us/articles/115014925288-How-to-submit-a-letter-to-the-editor}{\emph{tips}}\emph{.
And here's our email:}
\href{mailto:letters@nytimes.com}{\emph{letters@nytimes.com}}\emph{.}

\emph{Follow The New York Times Opinion section on}
\href{https://www.facebook.com/nytopinion}{\emph{Facebook}}\emph{,}
\href{http://twitter.com/NYTOpinion}{\emph{Twitter (@NYTopinion)}}
\emph{and}
\href{https://www.instagram.com/nytopinion/}{\emph{Instagram}}\emph{.}

Advertisement

\protect\hyperlink{after-bottom}{Continue reading the main story}

\hypertarget{site-index}{%
\subsection{Site Index}\label{site-index}}

\hypertarget{site-information-navigation}{%
\subsection{Site Information
Navigation}\label{site-information-navigation}}

\begin{itemize}
\tightlist
\item
  \href{https://help.nytimes.com/hc/en-us/articles/115014792127-Copyright-notice}{©~2020~The
  New York Times Company}
\end{itemize}

\begin{itemize}
\tightlist
\item
  \href{https://www.nytco.com/}{NYTCo}
\item
  \href{https://help.nytimes.com/hc/en-us/articles/115015385887-Contact-Us}{Contact
  Us}
\item
  \href{https://www.nytco.com/careers/}{Work with us}
\item
  \href{https://nytmediakit.com/}{Advertise}
\item
  \href{http://www.tbrandstudio.com/}{T Brand Studio}
\item
  \href{https://www.nytimes.com/privacy/cookie-policy\#how-do-i-manage-trackers}{Your
  Ad Choices}
\item
  \href{https://www.nytimes.com/privacy}{Privacy}
\item
  \href{https://help.nytimes.com/hc/en-us/articles/115014893428-Terms-of-service}{Terms
  of Service}
\item
  \href{https://help.nytimes.com/hc/en-us/articles/115014893968-Terms-of-sale}{Terms
  of Sale}
\item
  \href{https://spiderbites.nytimes.com}{Site Map}
\item
  \href{https://help.nytimes.com/hc/en-us}{Help}
\item
  \href{https://www.nytimes.com/subscription?campaignId=37WXW}{Subscriptions}
\end{itemize}
