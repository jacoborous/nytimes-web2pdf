Sections

SEARCH

\protect\hyperlink{site-content}{Skip to
content}\protect\hyperlink{site-index}{Skip to site index}

\href{https://www.nytimes.com/section/world/middleeast}{Middle East}

\href{https://myaccount.nytimes.com/auth/login?response_type=cookie\&client_id=vi}{}

\href{https://www.nytimes.com/section/todayspaper}{Today's Paper}

\href{/section/world/middleeast}{Middle East}\textbar{}Lebanon's
Currency Plunges, and Protesters Surge Into Streets

\url{https://nyti.ms/2XRkhgs}

\begin{itemize}
\item
\item
\item
\item
\item
\item
\end{itemize}

Advertisement

\protect\hyperlink{after-top}{Continue reading the main story}

Supported by

\protect\hyperlink{after-sponsor}{Continue reading the main story}

\hypertarget{lebanons-currency-plunges-and-protesters-surge-into-streets}{%
\section{Lebanon's Currency Plunges, and Protesters Surge Into
Streets}\label{lebanons-currency-plunges-and-protesters-surge-into-streets}}

Demonstrations broke out across the country after the pound sank to a
new low against the U.S. dollar, obliterating the purchasing power of
many Lebanese.

\includegraphics{https://static01.nyt.com/images/2020/06/11/world/11lebanon/merlin_173448639_1c867651-e747-4e64-ad2b-545214213fbc-articleLarge.jpg?quality=75\&auto=webp\&disable=upscale}

By \href{https://www.nytimes.com/by/ben-hubbard}{Ben Hubbard} and
\href{https://www.nytimes.com/by/hwaida-saad}{Hwaida Saad}

\begin{itemize}
\item
  June 11, 2020
\item
  \begin{itemize}
  \item
  \item
  \item
  \item
  \item
  \item
  \end{itemize}
\end{itemize}

BEIRUT, Lebanon --- A new wave of anti-government protests erupted
across Lebanon on Thursday with people blocking roads, burning tires and
chanting against the political elite amid a deepening economic crisis.

The protests, in a number of cities and in multiple parts of the
capital, Beirut, did not appear to be coordinated, but broke out after
the Lebanese pound sank to a new low against the U.S. dollar,
obliterating the purchasing power of many Lebanese.

Prime Minister Hassan Diab called for an emergency cabinet meeting on
Friday to address the crisis.

Lebanon, a sectarian democracy with 5.5 million people, has been mired
in intertwined political and economic crises since protesters took to
the streets last fall to denounce the country's leaders for decades of
\href{https://www.nytimes.com/2019/12/03/world/middleeast/lebanon-protests-corruption.html}{mismanagement
and corruption}.

Those protests forced the
\href{https://www.nytimes.com/2019/12/19/world/middleeast/lebanon-prime-minister-hassan-diab.html}{resignation}
of Prime Minister Saad Hariri in October, but petered out in March amid
a government-imposed lockdown aimed at preventing the spread of the
coronavirus.

The lockdown accelerated the country's economic decline. Businesses have
closed and unemployment has spiked as the government has cascaded toward
insolvency. In March, it failed to make a \$1.2 billion payment for
foreign bonds, the first such default in Lebanon's history.

A
\href{https://www.nytimes.com/2019/12/19/world/middleeast/lebanon-prime-minister-hassan-diab.html}{new
government led by Mr. Diab}, who was sworn in in December, is in talks
with the International Monetary Fund over a multibillion-dollar aid
package, but there are no signs of an imminent agreement.

Much of the public anger has focused on the banks, which have imposed
tight restrictions on dollar withdrawals, and on the collapse of the
Lebanese pound, which the government had kept pegged at 1,500 to the
dollar for decades, permitting Lebanese to use the two currencies
interchangeably.

But the pound's value has been dropping on the black market for months,
and on Thursday reached a new low: more than 5,000 to the dollar. That
is a 70-percent drop in value since October.

Thursday's protests appeared to be a spontaneous burst of anger from
citizens who have watched the government repeatedly fail to carry out
reforms while the value of their salaries and savings has dropped.

Hundreds of protesters gathered in central Beirut, blocking a main
thoroughfare, lighting bonfires and chanting against sectarianism,
according to videos posted on social media. Other protests broke out in
the southern cities of Sidon and Nabatiyeh and on the main highway
running along the Mediterranean coast.

Protesters in the northern city of Tripoli lit a branch of Lebanon's
Central Bank on fire and clashed with security forces who tried to
disperse them with tear gas, according to the state-run National News
Agency.

Advertisement

\protect\hyperlink{after-bottom}{Continue reading the main story}

\hypertarget{site-index}{%
\subsection{Site Index}\label{site-index}}

\hypertarget{site-information-navigation}{%
\subsection{Site Information
Navigation}\label{site-information-navigation}}

\begin{itemize}
\tightlist
\item
  \href{https://help.nytimes.com/hc/en-us/articles/115014792127-Copyright-notice}{©~2020~The
  New York Times Company}
\end{itemize}

\begin{itemize}
\tightlist
\item
  \href{https://www.nytco.com/}{NYTCo}
\item
  \href{https://help.nytimes.com/hc/en-us/articles/115015385887-Contact-Us}{Contact
  Us}
\item
  \href{https://www.nytco.com/careers/}{Work with us}
\item
  \href{https://nytmediakit.com/}{Advertise}
\item
  \href{http://www.tbrandstudio.com/}{T Brand Studio}
\item
  \href{https://www.nytimes.com/privacy/cookie-policy\#how-do-i-manage-trackers}{Your
  Ad Choices}
\item
  \href{https://www.nytimes.com/privacy}{Privacy}
\item
  \href{https://help.nytimes.com/hc/en-us/articles/115014893428-Terms-of-service}{Terms
  of Service}
\item
  \href{https://help.nytimes.com/hc/en-us/articles/115014893968-Terms-of-sale}{Terms
  of Sale}
\item
  \href{https://spiderbites.nytimes.com}{Site Map}
\item
  \href{https://help.nytimes.com/hc/en-us}{Help}
\item
  \href{https://www.nytimes.com/subscription?campaignId=37WXW}{Subscriptions}
\end{itemize}
