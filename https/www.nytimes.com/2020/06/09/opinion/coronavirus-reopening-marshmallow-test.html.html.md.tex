Sections

SEARCH

\protect\hyperlink{site-content}{Skip to
content}\protect\hyperlink{site-index}{Skip to site index}

\href{https://myaccount.nytimes.com/auth/login?response_type=cookie\&client_id=vi}{}

\href{https://www.nytimes.com/section/todayspaper}{Today's Paper}

\href{/section/opinion}{Opinion}\textbar{}America Fails the Marshmallow
Test

\url{https://nyti.ms/2UsT5CJ}

\begin{itemize}
\item
\item
\item
\item
\item
\item
\end{itemize}

Advertisement

\protect\hyperlink{after-top}{Continue reading the main story}

\href{/section/opinion}{Opinion}

Supported by

\protect\hyperlink{after-sponsor}{Continue reading the main story}

\hypertarget{america-fails-the-marshmallow-test}{%
\section{America Fails the Marshmallow
Test}\label{america-fails-the-marshmallow-test}}

We lack the will to beat Covid-19.

\href{https://www.nytimes.com/by/paul-krugman}{\includegraphics{https://static01.nyt.com/images/2018/04/02/opinion/paul-krugman/paul-krugman-thumbLarge.png}}

By \href{https://www.nytimes.com/by/paul-krugman}{Paul Krugman}

Opinion Columnist

\begin{itemize}
\item
  June 9, 2020
\item
  \begin{itemize}
  \item
  \item
  \item
  \item
  \item
  \item
  \end{itemize}
\end{itemize}

\includegraphics{https://static01.nyt.com/images/2020/06/09/opinion/09krugman-newseltter/merlin_173233989_8864c8e5-b50b-461e-9611-ecfc680893c7-articleLarge.jpg?quality=75\&auto=webp\&disable=upscale}

\href{https://cn.nytimes.com/opinion/20200610/coronavirus-reopening-marshmallow-test/}{阅读简体中文版}\href{https://cn.nytimes.com/opinion/20200610/coronavirus-reopening-marshmallow-test/zh-hant/}{閱讀繁體中文版}\href{https://www.nytimes.com/es/2020/06/11/espanol/opinion/coronavirus-economia-krugman.html}{Leer
en español}

\emph{This article is part of Paul Krugman's free newsletter. You can}
\href{https://www.nytimes.com/newsletters/paul-krugman}{\emph{sign up
here}} \emph{to receive it every Tuesday.}

The marshmallow test is a famous psychological experiment that tests
children's willingness to delay gratification. Children are offered a
marshmallow, but told that they can have a second marshmallow if they're
willing to wait 15 minutes before eating the first one. Claims that
children with the willpower to hold out do much better in life
\href{https://www.theatlantic.com/family/archive/2018/06/marshmallow-test/561779/}{haven't
held up well}, but the experiment is still a useful metaphor for many
choices in life, both by individuals and by larger groups.

One way to think about the Covid-19 pandemic is that it poses a kind of
marshmallow test for society.

At this point, there have been enough international success stories in
dealing with the coronavirus to leave us with a clear sense of what
beating the pandemic takes. First, you have to impose strict social
distancing long enough to reduce the number of infected people to a
small fraction of the population. Then you have to implement a regime of
testing, tracing and isolating: quickly identifying any new outbreak,
finding everyone exposed and quarantining them until the danger is past.

This strategy is workable. South Korea has done it.
\href{https://www.nytimes.com/2020/06/08/world/australia/new-zealand-coronavirus-ardern.html}{New
Zealand} has done it.

But you have to be strict and you have to be patient, staying the course
until the pandemic is over, not giving in to the temptation to return to
normal life while the virus is still widespread. So it is, as I said, a
kind of marshmallow test.

And America is failing that test.

New U.S. cases and deaths have
\href{https://www.nytimes.com/interactive/2020/us/coronavirus-us-cases.html}{declined}
since early April, but that's almost entirely because the greater New
York area, after a horrific outbreak, has achieved huge progress. In
many parts of the country --- including our most populous states,
California, Texas, and Florida --- the disease is still spreading.
Overall, new cases are plateauing and may be starting to rise. Yet state
governments are moving to reopen anyway.

This is a very different story from what's happening in
\href{https://www.bloomberg.com/graphics/2020-coronavirus-cases-world-map/?srnd=premium\&sref=qzusa8bC\#global-new-cases}{other
advanced countries}, even hard-hit nations like Italy and Spain, where
new cases have fallen dramatically. It now looks likely that by late
summer we'll be the only major wealthy nation where large numbers of
people are still dying from Covid-19.

Why are we failing the test? It's easy to blame Donald Trump, a
man-child who would surely gobble down that first marshmallow, then try
to steal marshmallows from other kids. But America's impatience, its
unwillingness to do what it takes to deal with a threat that can't be
beaten with threats of violence, runs much deeper than one man.

It doesn't help that Republicans are ideologically opposed to government
safety-net programs, which are what make the economic consequences of
social distancing tolerable; as I explain in
\href{https://www.nytimes.com/2020/06/08/opinion/coronavirus-jobs-report.html?action=click\&module=Opinion\&pgtype=Homepage}{my
recent column}, they seem determined to let crucial emergency relief
expire far too soon. Nor does it help that even low-cost measures to
limit the spread of Covid-19, above all wearing face masks (which mainly
protect other people), have been caught up in our culture wars.

America in 2020, it seems, is too disunited, with too many people in the
grip of ideology and partisanship, to deal effectively with a pandemic.
We have the knowledge, we have the resources, but we don't have the
will.

\begin{center}\rule{0.5\linewidth}{\linethickness}\end{center}

\textbf{Paul Krugman's Newsletter:} \emph{Get a better understanding of
the economy --- and an even deeper look at what's on Paul's mind.}
\href{https://www.nytimes.com/newsletters/paul-krugman}{\emph{Sign up
here}}\emph{.}

\emph{The Times is committed to publishing}
\href{https://www.nytimes.com/2019/01/31/opinion/letters/letters-to-editor-new-york-times-women.html}{\emph{a
diversity of letters}} \emph{to the editor. We'd like to hear what you
think about this or any of our articles. Here are some}
\href{https://help.nytimes.com/hc/en-us/articles/115014925288-How-to-submit-a-letter-to-the-editor}{\emph{tips}}\emph{.
And here's our email:}
\href{mailto:letters@nytimes.com}{\emph{letters@nytimes.com}}\emph{.}

\emph{Follow The New York Times Opinion section on}
\href{https://www.facebook.com/nytopinion}{\emph{Facebook}}\emph{,}
\href{http://twitter.com/NYTOpinion}{\emph{Twitter (@NYTopinion)}}
\emph{and}
\href{https://www.instagram.com/nytopinion/}{\emph{Instagram}}\emph{.}

Advertisement

\protect\hyperlink{after-bottom}{Continue reading the main story}

\hypertarget{site-index}{%
\subsection{Site Index}\label{site-index}}

\hypertarget{site-information-navigation}{%
\subsection{Site Information
Navigation}\label{site-information-navigation}}

\begin{itemize}
\tightlist
\item
  \href{https://help.nytimes.com/hc/en-us/articles/115014792127-Copyright-notice}{©~2020~The
  New York Times Company}
\end{itemize}

\begin{itemize}
\tightlist
\item
  \href{https://www.nytco.com/}{NYTCo}
\item
  \href{https://help.nytimes.com/hc/en-us/articles/115015385887-Contact-Us}{Contact
  Us}
\item
  \href{https://www.nytco.com/careers/}{Work with us}
\item
  \href{https://nytmediakit.com/}{Advertise}
\item
  \href{http://www.tbrandstudio.com/}{T Brand Studio}
\item
  \href{https://www.nytimes.com/privacy/cookie-policy\#how-do-i-manage-trackers}{Your
  Ad Choices}
\item
  \href{https://www.nytimes.com/privacy}{Privacy}
\item
  \href{https://help.nytimes.com/hc/en-us/articles/115014893428-Terms-of-service}{Terms
  of Service}
\item
  \href{https://help.nytimes.com/hc/en-us/articles/115014893968-Terms-of-sale}{Terms
  of Sale}
\item
  \href{https://spiderbites.nytimes.com}{Site Map}
\item
  \href{https://help.nytimes.com/hc/en-us}{Help}
\item
  \href{https://www.nytimes.com/subscription?campaignId=37WXW}{Subscriptions}
\end{itemize}
