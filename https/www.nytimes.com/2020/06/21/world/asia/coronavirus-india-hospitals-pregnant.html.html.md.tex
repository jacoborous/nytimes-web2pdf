Sections

SEARCH

\protect\hyperlink{site-content}{Skip to
content}\protect\hyperlink{site-index}{Skip to site index}

\href{https://www.nytimes.com/section/world/asia}{Asia Pacific}

\href{https://myaccount.nytimes.com/auth/login?response_type=cookie\&client_id=vi}{}

\href{https://www.nytimes.com/section/todayspaper}{Today's Paper}

\href{/section/world/asia}{Asia Pacific}\textbar{}8 Hospitals in 15
Hours: A Pregnant Woman's Crisis in the Pandemic

\url{https://nyti.ms/3fJmIHV}

\begin{itemize}
\item
\item
\item
\item
\item
\item
\end{itemize}

\href{https://www.nytimes.com/news-event/coronavirus?action=click\&pgtype=Article\&state=default\&region=TOP_BANNER\&context=storylines_menu}{The
Coronavirus Outbreak}

\begin{itemize}
\tightlist
\item
  live\href{https://www.nytimes.com/2020/08/02/world/coronavirus-updates.html?action=click\&pgtype=Article\&state=default\&region=TOP_BANNER\&context=storylines_menu}{Latest
  Updates}
\item
  \href{https://www.nytimes.com/interactive/2020/us/coronavirus-us-cases.html?action=click\&pgtype=Article\&state=default\&region=TOP_BANNER\&context=storylines_menu}{Maps
  and Cases}
\item
  \href{https://www.nytimes.com/interactive/2020/science/coronavirus-vaccine-tracker.html?action=click\&pgtype=Article\&state=default\&region=TOP_BANNER\&context=storylines_menu}{Vaccine
  Tracker}
\item
  \href{https://www.nytimes.com/interactive/2020/07/29/us/schools-reopening-coronavirus.html?action=click\&pgtype=Article\&state=default\&region=TOP_BANNER\&context=storylines_menu}{What
  School May Look Like}
\item
  \href{https://www.nytimes.com/live/2020/07/31/business/stock-market-today-coronavirus?action=click\&pgtype=Article\&state=default\&region=TOP_BANNER\&context=storylines_menu}{Economy}
\end{itemize}

Advertisement

\protect\hyperlink{after-top}{Continue reading the main story}

Supported by

\protect\hyperlink{after-sponsor}{Continue reading the main story}

\hypertarget{8-hospitals-in-15-hours-a-pregnant-womans-crisis-in-the-pandemic}{%
\section{8 Hospitals in 15 Hours: A Pregnant Woman's Crisis in the
Pandemic}\label{8-hospitals-in-15-hours-a-pregnant-womans-crisis-in-the-pandemic}}

Her baby was coming, and her complications were growing more dangerous.
But nowhere would take her --- an increasingly common story as India's
health care system buckles under pressure.

\includegraphics{https://static01.nyt.com/images/2020/06/19/world/00virus-india-hospital001sub/00virus-india-hospital001sub-articleLarge.jpg?quality=75\&auto=webp\&disable=upscale}

\href{https://www.nytimes.com/by/jeffrey-gettleman}{\includegraphics{https://static01.nyt.com/images/2018/10/10/multimedia/author-jeffrey-gettleman/author-jeffrey-gettleman-thumbLarge.png}}\href{https://www.nytimes.com/by/suhasini-raj}{\includegraphics{https://static01.nyt.com/images/2019/11/22/reader-center/author-Suhasini-Raj/author-Suhasini-Raj-thumbLarge.png}}

By \href{https://www.nytimes.com/by/jeffrey-gettleman}{Jeffrey
Gettleman} and \href{https://www.nytimes.com/by/suhasini-raj}{Suhasini
Raj}

\begin{itemize}
\item
  Published June 21, 2020Updated July 12, 2020
\item
  \begin{itemize}
  \item
  \item
  \item
  \item
  \item
  \item
  \end{itemize}
\end{itemize}

NEW DELHI --- Neelam Kumari Gautam woke up at 5 a.m. with shooting labor
pains. Her husband put her gently in the back of a rickshaw and motored
with her to a hospital. Then another. Then another. Her pain was so
intense she could barely breathe, but none would take her.

``Why are the doctors not taking me in?'' she asked her husband,
Bijendra Singh, over and over again. ``What's the matter? I will die.''

Mr. Singh began to panic. He knew what he was up against. As
\href{https://www.nytimes.com/2020/07/02/world/asia/india-coronavirus-wedding-groom.html}{India's
coronavirus crisis} has accelerated --- India is now reporting more
infections a day than any other nation except the United States or
Brazil --- the country's already strained and underfunded health care
system has begun to buckle.

A database of recent deaths reveals that scores of people have died in
the streets or in the back of ambulances, denied critical care. Ms.
Gautam's odyssey through eight different hospitals in 15 hours in
India's biggest metropolitan area serves as a devastating window into
what is really happening here.

Indian government rules explicitly call for emergency services to be
rendered, but still people in desperate need of treatment keep getting
turned away, especially in New Delhi, the capital. Infections are rising
quickly, Delhi's hospitals are overloaded and many health care workers
are afraid of treating new patients in case they have coronavirus, which
has killed more than 13,000 people in India.

``There is currently little or no chance of admission to hospitals for
people with Covid-19, but also for people with other intensive care
needs,'' read a warning just issued by the German Embassy in New Delhi.

\includegraphics{https://static01.nyt.com/images/2020/06/21/world/21virus-india-hospital-new2/merlin_172881807_9d126e5c-2cbc-42c2-a013-911d848f7c3a-articleLarge.jpg?quality=75\&auto=webp\&disable=upscale}

After watching reports on Indian television showing dead bodies in the
lobby of a government hospital and crying patients being ignored, a
panel of judges on India's Supreme Court said,
\href{https://indianexpress.com/article/india/coronavirus-dead-bodies-delhi-supreme-court-notice-centre-6455474/}{``The
situation in Delhi is horrendous, horrific and pathetic.''}

The bigger picture is that Prime Minister Narendra Modi's government is
struggling with overlapping crises. Last week,
\href{https://www.nytimes.com/2020/06/16/world/asia/indian-china-border-clash.html}{Chinese
troops beat 20 Indian soldiers to death} along their disputed border in
the Himalayas, triggering the most dangerous showdown between the two
nuclear powers in decades.

At the same time, India's economy is nose-diving, and the coronavirus
pandemic has cost this country more than 100 million jobs. Desperate to
turn the economy around, Mr. Modi has rejected health experts' counsel
to put the country back under lockdown, saying that India must ``unlock,
unlock, unlock.''

\hypertarget{latest-updates-global-coronavirus-outbreak}{%
\section{\texorpdfstring{\href{https://www.nytimes.com/2020/08/01/world/coronavirus-covid-19.html?action=click\&pgtype=Article\&state=default\&region=MAIN_CONTENT_1\&context=storylines_live_updates}{Latest
Updates: Global Coronavirus
Outbreak}}{Latest Updates: Global Coronavirus Outbreak}}\label{latest-updates-global-coronavirus-outbreak}}

Updated 2020-08-02T17:52:35.962Z

\begin{itemize}
\tightlist
\item
  \href{https://www.nytimes.com/2020/08/01/world/coronavirus-covid-19.html?action=click\&pgtype=Article\&state=default\&region=MAIN_CONTENT_1\&context=storylines_live_updates\#link-34047410}{The
  U.S. reels as July cases more than double the total of any other
  month.}
\item
  \href{https://www.nytimes.com/2020/08/01/world/coronavirus-covid-19.html?action=click\&pgtype=Article\&state=default\&region=MAIN_CONTENT_1\&context=storylines_live_updates\#link-780ec966}{Top
  U.S. officials work to break an impasse over the federal jobless
  benefit.}
\item
  \href{https://www.nytimes.com/2020/08/01/world/coronavirus-covid-19.html?action=click\&pgtype=Article\&state=default\&region=MAIN_CONTENT_1\&context=storylines_live_updates\#link-2bc8948}{Its
  outbreak untamed, Melbourne goes into even greater lockdown.}
\end{itemize}

\href{https://www.nytimes.com/2020/08/01/world/coronavirus-covid-19.html?action=click\&pgtype=Article\&state=default\&region=MAIN_CONTENT_1\&context=storylines_live_updates}{See
more updates}

More live coverage:
\href{https://www.nytimes.com/live/2020/07/31/business/stock-market-today-coronavirus?action=click\&pgtype=Article\&state=default\&region=MAIN_CONTENT_1\&context=storylines_live_updates}{Markets}

When things got better, Mr. Singh and his wife had hoped to buy an
apartment in a gated community in Noida, a satellite city of New Delhi
crammed with tall glass buildings, many malls and many hospitals. She
worked on an assembly line producing electrical wire. He serviced
machines at a printing press.

Together, Mr. Singh, 31, and Ms. Gautam, 30, earned a respectable
\$8,000 a year, putting them solidly in India's rising middle class.
``Two wheels of a well-oiled machine, making our family go around,'' Mr.
Singh said.

Their son, Rudraksh, turned 6 just before their new baby was due.

Image

A family photo from 2018 of Bijendra Singh and Neelam Kumari Gautam with
their son, Rudraksh.Credit...Courtesy the family of Bijendra Singh

As Ms. Gautam entered her ninth month, she ran into some health problems
and spent five days, in late May and early June, in the hospital for
pregnancy-related high blood pressure, bleeding and possibly typhoid.

On June 5, as she began to go into labor, the first hospital they tried
was the ESIC Model Hospital, a sprawling government facility in Noida.
Mr. Singh said that the first thing the doctor said to her was, ``I'll
slap you if you take off your mask.''

They were shocked, Mr. Singh said. But Ms. Gautam was having trouble
breathing. They didn't argue. She begged for oxygen, which the hospital
had, along with ventilators. But instead of helping, the doctor told
them to go to another government hospital, on the other side of town.
There, too, she was turned away.

An administrator at the first hospital declined to comment, and a doctor
at the second hospital said Ms. Gautam needed intensive care, which the
hospital couldn't provide.

Even before Covid-19 arrived, Indian hospitals were beleaguered.
\href{https://www.livemint.com/news/india/india-s-economy-needs-big-dose-of-health-spending-11586365603651.html}{The
Indian government spends less than 2,000 rupees} (about \$26) per person
per year on health care. The hope during India's lockdown, which began
in late March but was mostly lifted by early June, was that the
restrictions would slow the spread of the virus and give cities time to
scale up hospital capacity before the worst hit.

That didn't happen, or not nearly enough, and Delhi now finds itself
thousands of beds short --- the central government just repurposed
\href{https://www.cbsnews.com/news/coronavirus-in-india-delhi-cases-surge-train-cars-and-hotels-converted-emergency-hospital-2020-06-17/}{hundreds
of railway cars to be used as sick bays}. And still there is great
confusion about admitting patients who don't have coronavirus.

Image

In New Delhi, a banquet hall usually used for weddings has been
converted to a makeshift coronavirus hospital.Credit...Manish
Swarup/Associated Press

Some hospitals say they need to test every patient before treating them.
Others simply perform a quick temperature check.

Several Indian doctors said that private, profit-driven hospitals were
terrified to accept new patients --- particularly those with breathing
problems --- because they didn't want to risk getting shut down, which
has happened to private clinics if one of their patients tested positive
for the coronavirus.

``Government policy has created this chaos,'' said Rajesh Kumar
Prajapati, an orthopedic surgeon and former medical school professor.

As complaints began to pile up that hospitals were turning away sick
people, the powerful home ministry issued
\href{https://www.mha.gov.in/sites/default/files/PR_HSLettertoStatesUTsonMovementofMedicalProfeesionals_11052020.pdf}{a
directive} re-emphasizing that all hospitals should remain open for
``all patients, Covid and non-Covid emergencies.''

But clearly not everyone has been listening. A 13-year-old boy in Agra
died of a stomach ailment in April after being turned away from six
hospitals, his distraught family said. Another boy, in Punjab, with an
obstructed airway, was rejected from seven hospitals and died in the
arms of a family friend, who happened to be a doctor but was not given
any help to save him.

``This is inhuman,'' said Dheeraj Singh, the doctor in Punjab.

Image

Indian workers preparing railway coaches to be used as sick bays in New
Delhi last week.Credit...Raj K Raj/Hindustan Times, via Getty Images

Cases like these caught the eye of Thejesh G. N., an electronics
engineer in India's tech hub, Bangalore. He helped build a database
tracking publicly reported deaths and found that at least 63 people have
died in recent weeks from being denied critical care. Most health
experts believe that the number is far higher.

The third hospital that Ms. Gautam went to, Shivalik Hospital, was the
one that had treated her for her prenatal troubles. This time, doctors
gave her a little oxygen, but Mr. Singh said they feared she might have
coronavirus and abruptly ordered her to leave.

\href{https://www.nytimes.com/news-event/coronavirus?action=click\&pgtype=Article\&state=default\&region=MAIN_CONTENT_3\&context=storylines_faq}{}

\hypertarget{the-coronavirus-outbreak-}{%
\subsubsection{The Coronavirus Outbreak
›}\label{the-coronavirus-outbreak-}}

\hypertarget{frequently-asked-questions}{%
\paragraph{Frequently Asked
Questions}\label{frequently-asked-questions}}

Updated July 27, 2020

\begin{itemize}
\item ~
  \hypertarget{should-i-refinance-my-mortgage}{%
  \paragraph{Should I refinance my
  mortgage?}\label{should-i-refinance-my-mortgage}}

  \begin{itemize}
  \tightlist
  \item
    \href{https://www.nytimes.com/article/coronavirus-money-unemployment.html?action=click\&pgtype=Article\&state=default\&region=MAIN_CONTENT_3\&context=storylines_faq}{It
    could be a good idea,} because mortgage rates have
    \href{https://www.nytimes.com/2020/07/16/business/mortgage-rates-below-3-percent.html?action=click\&pgtype=Article\&state=default\&region=MAIN_CONTENT_3\&context=storylines_faq}{never
    been lower.} Refinancing requests have pushed mortgage applications
    to some of the highest levels since 2008, so be prepared to get in
    line. But defaults are also up, so if you're thinking about buying a
    home, be aware that some lenders have tightened their standards.
  \end{itemize}
\item ~
  \hypertarget{what-is-school-going-to-look-like-in-september}{%
  \paragraph{What is school going to look like in
  September?}\label{what-is-school-going-to-look-like-in-september}}

  \begin{itemize}
  \tightlist
  \item
    It is unlikely that many schools will return to a normal schedule
    this fall, requiring the grind of
    \href{https://www.nytimes.com/2020/06/05/us/coronavirus-education-lost-learning.html?action=click\&pgtype=Article\&state=default\&region=MAIN_CONTENT_3\&context=storylines_faq}{online
    learning},
    \href{https://www.nytimes.com/2020/05/29/us/coronavirus-child-care-centers.html?action=click\&pgtype=Article\&state=default\&region=MAIN_CONTENT_3\&context=storylines_faq}{makeshift
    child care} and
    \href{https://www.nytimes.com/2020/06/03/business/economy/coronavirus-working-women.html?action=click\&pgtype=Article\&state=default\&region=MAIN_CONTENT_3\&context=storylines_faq}{stunted
    workdays} to continue. California's two largest public school
    districts --- Los Angeles and San Diego --- said on July 13, that
    \href{https://www.nytimes.com/2020/07/13/us/lausd-san-diego-school-reopening.html?action=click\&pgtype=Article\&state=default\&region=MAIN_CONTENT_3\&context=storylines_faq}{instruction
    will be remote-only in the fall}, citing concerns that surging
    coronavirus infections in their areas pose too dire a risk for
    students and teachers. Together, the two districts enroll some
    825,000 students. They are the largest in the country so far to
    abandon plans for even a partial physical return to classrooms when
    they reopen in August. For other districts, the solution won't be an
    all-or-nothing approach.
    \href{https://bioethics.jhu.edu/research-and-outreach/projects/eschool-initiative/school-policy-tracker/}{Many
    systems}, including the nation's largest, New York City, are
    devising
    \href{https://www.nytimes.com/2020/06/26/us/coronavirus-schools-reopen-fall.html?action=click\&pgtype=Article\&state=default\&region=MAIN_CONTENT_3\&context=storylines_faq}{hybrid
    plans} that involve spending some days in classrooms and other days
    online. There's no national policy on this yet, so check with your
    municipal school system regularly to see what is happening in your
    community.
  \end{itemize}
\item ~
  \hypertarget{is-the-coronavirus-airborne}{%
  \paragraph{Is the coronavirus
  airborne?}\label{is-the-coronavirus-airborne}}

  \begin{itemize}
  \tightlist
  \item
    The coronavirus
    \href{https://www.nytimes.com/2020/07/04/health/239-experts-with-one-big-claim-the-coronavirus-is-airborne.html?action=click\&pgtype=Article\&state=default\&region=MAIN_CONTENT_3\&context=storylines_faq}{can
    stay aloft for hours in tiny droplets in stagnant air}, infecting
    people as they inhale, mounting scientific evidence suggests. This
    risk is highest in crowded indoor spaces with poor ventilation, and
    may help explain super-spreading events reported in meatpacking
    plants, churches and restaurants.
    \href{https://www.nytimes.com/2020/07/06/health/coronavirus-airborne-aerosols.html?action=click\&pgtype=Article\&state=default\&region=MAIN_CONTENT_3\&context=storylines_faq}{It's
    unclear how often the virus is spread} via these tiny droplets, or
    aerosols, compared with larger droplets that are expelled when a
    sick person coughs or sneezes, or transmitted through contact with
    contaminated surfaces, said Linsey Marr, an aerosol expert at
    Virginia Tech. Aerosols are released even when a person without
    symptoms exhales, talks or sings, according to Dr. Marr and more
    than 200 other experts, who
    \href{https://academic.oup.com/cid/article/doi/10.1093/cid/ciaa939/5867798}{have
    outlined the evidence in an open letter to the World Health
    Organization}.
  \end{itemize}
\item ~
  \hypertarget{what-are-the-symptoms-of-coronavirus}{%
  \paragraph{What are the symptoms of
  coronavirus?}\label{what-are-the-symptoms-of-coronavirus}}

  \begin{itemize}
  \tightlist
  \item
    Common symptoms
    \href{https://www.nytimes.com/article/symptoms-coronavirus.html?action=click\&pgtype=Article\&state=default\&region=MAIN_CONTENT_3\&context=storylines_faq}{include
    fever, a dry cough, fatigue and difficulty breathing or shortness of
    breath.} Some of these symptoms overlap with those of the flu,
    making detection difficult, but runny noses and stuffy sinuses are
    less common.
    \href{https://www.nytimes.com/2020/04/27/health/coronavirus-symptoms-cdc.html?action=click\&pgtype=Article\&state=default\&region=MAIN_CONTENT_3\&context=storylines_faq}{The
    C.D.C. has also} added chills, muscle pain, sore throat, headache
    and a new loss of the sense of taste or smell as symptoms to look
    out for. Most people fall ill five to seven days after exposure, but
    symptoms may appear in as few as two days or as many as 14 days.
  \end{itemize}
\item ~
  \hypertarget{does-asymptomatic-transmission-of-covid-19-happen}{%
  \paragraph{Does asymptomatic transmission of Covid-19
  happen?}\label{does-asymptomatic-transmission-of-covid-19-happen}}

  \begin{itemize}
  \tightlist
  \item
    So far, the evidence seems to show it does. A widely cited
    \href{https://www.nature.com/articles/s41591-020-0869-5}{paper}
    published in April suggests that people are most infectious about
    two days before the onset of coronavirus symptoms and estimated that
    44 percent of new infections were a result of transmission from
    people who were not yet showing symptoms. Recently, a top expert at
    the World Health Organization stated that transmission of the
    coronavirus by people who did not have symptoms was ``very rare,''
    \href{https://www.nytimes.com/2020/06/09/world/coronavirus-updates.html?action=click\&pgtype=Article\&state=default\&region=MAIN_CONTENT_3\&context=storylines_faq\#link-1f302e21}{but
    she later walked back that statement.}
  \end{itemize}
\end{itemize}

``We are a small mother and child hospital,'' said the hospital's
director, Ravi Mohta. ``We did what we could.''

The couple hobbled back to the rickshaw. Ms. Gautam was fading. She
stopped talking and began heavily sweating. She clung to her husband's
hand.

It wasn't simply that the doctors couldn't help her, Mr. Singh said. It
was as if they didn't want to help her.

``They didn't care if she was dead or alive,'' he said.

At a fourth hospital, a branch of
\href{https://www.fortishealthcare.com/}{Fortis}, an Indian health care
giant, Mr. Singh pleaded for a ventilator. Mr. Singh said the doctor's
response was: ``She's going to die. Take her wherever you wish.''

In a statement, the hospital said that it had no space for her, tried to
stabilize her and then offered to take her by ambulance to another
hospital. Mr. Singh said that the hospital's efforts were cursory and
that there was no offer of an ambulance.

Image

Treating a patient suffering from coronavirus in the emergency ward of a
hospital in New Delhi this month. The home ministry issued a directive
that all hospitals should be open for ``all patients, Covid and
non-Covid emergencies.''Credit...Atish Patel/Agence France-Presse ---
Getty Images

They tried three other hospitals, hurrying from one to the other, losing
precious time. When all refused, Mr. Singh called the police.

He said that two officers met him at the entrance of the Government
Institute of Medical Sciences, a large public hospital, and tried to
persuade the doctors to admit his wife. But the doctors wouldn't listen
to the police officers, either.

Administrators at that hospital declined to comment.

After that failed, they raced in an ambulance to Max Super Specialty
Hospital in Ghaziabad, more than 25 miles away. It was now late
afternoon, still bright, around 100 degrees outside. More than eight
hours had passed since Ms. Gautam and her husband had set off from their
home, eager to meet their new baby soon.

But the Max hospital --- their eighth that day --- gave them the same
heartbreaking answer: no beds.

Ms. Gautam closed her eyes and whispered: ``Save me.''

Mr. Singh told the ambulance to rush back to the Government Institute of
Medical Sciences.

He hunched in the back, leaning over his wife, pleading with her not to
give up. He looked down at her face. She reached up and clutched his
shirt. Her hands tightly clenched the fabric.

Image

New Delhi this month.Credit...Xavier Galiana/Agence France-Presse ---
Getty Images

As they finally pulled into the hospital, she stopped breathing. Her
neck slumped. Mr. Singh jumped out of the ambulance, grabbed a
wheelchair and frantically wheeled her into the emergency room.

At 8:05 p.m., after 8 different hospitals and 15 hours, Neelam Kumari
Gautam was pronounced dead. The baby also died.

A preliminary government investigation said: ``Hospital administration
and staff have been found guilty of carelessness.''

She has not been the only pregnant woman to die in labor after being
turned away. The same thing happened to
\href{https://www.thehansindia.com/telangana/telangana-denial-of-medical-aid-to-a-pregnant-woman-high-court-seeks-details-on-criminal-cases-filed-against-doctors-624837?infinitescroll=1https://timesofindia.indiatimes.com/city/hyderabad/6-women-docs-held-liable-for-death-of-gadwal-mother-baby/articleshow/76016968.cms}{a
young mother in Hyderabad} and another in Kashmir. In that case, the
family said, the hospital staff were so uncaring that they didn't even
help with an ambulance to take the body home. The woman's family had to
wheel her body down the road, in a stretcher, for several miles.

As the authorities consider criminal charges in Ms. Gautam's case, her
husband spends his days at home looking after his son, Rudraksh. The boy
asked him to throw away all of his mother's clothes.

``They remind me of her,'' he said.

The spark has gone out of Rudraksh's eye, Mr. Singh said.

A few days ago, he told his dad that when he grows up, he wants to be a
doctor, so ``I can make dead people come alive.''

Advertisement

\protect\hyperlink{after-bottom}{Continue reading the main story}

\hypertarget{site-index}{%
\subsection{Site Index}\label{site-index}}

\hypertarget{site-information-navigation}{%
\subsection{Site Information
Navigation}\label{site-information-navigation}}

\begin{itemize}
\tightlist
\item
  \href{https://help.nytimes.com/hc/en-us/articles/115014792127-Copyright-notice}{©~2020~The
  New York Times Company}
\end{itemize}

\begin{itemize}
\tightlist
\item
  \href{https://www.nytco.com/}{NYTCo}
\item
  \href{https://help.nytimes.com/hc/en-us/articles/115015385887-Contact-Us}{Contact
  Us}
\item
  \href{https://www.nytco.com/careers/}{Work with us}
\item
  \href{https://nytmediakit.com/}{Advertise}
\item
  \href{http://www.tbrandstudio.com/}{T Brand Studio}
\item
  \href{https://www.nytimes.com/privacy/cookie-policy\#how-do-i-manage-trackers}{Your
  Ad Choices}
\item
  \href{https://www.nytimes.com/privacy}{Privacy}
\item
  \href{https://help.nytimes.com/hc/en-us/articles/115014893428-Terms-of-service}{Terms
  of Service}
\item
  \href{https://help.nytimes.com/hc/en-us/articles/115014893968-Terms-of-sale}{Terms
  of Sale}
\item
  \href{https://spiderbites.nytimes.com}{Site Map}
\item
  \href{https://help.nytimes.com/hc/en-us}{Help}
\item
  \href{https://www.nytimes.com/subscription?campaignId=37WXW}{Subscriptions}
\end{itemize}
