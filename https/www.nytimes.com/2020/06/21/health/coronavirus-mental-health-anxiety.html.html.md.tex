Sections

SEARCH

\protect\hyperlink{site-content}{Skip to
content}\protect\hyperlink{site-index}{Skip to site index}

\href{https://www.nytimes.com/section/health}{Health}

\href{https://myaccount.nytimes.com/auth/login?response_type=cookie\&client_id=vi}{}

\href{https://www.nytimes.com/section/todayspaper}{Today's Paper}

\href{/section/health}{Health}\textbar{}The Pandemic's Mental Toll: More
Ripple Than Tsunami

\url{https://nyti.ms/3fKpEE6}

\begin{itemize}
\item
\item
\item
\item
\item
\item
\end{itemize}

\href{https://www.nytimes.com/news-event/coronavirus?action=click\&pgtype=Article\&state=default\&region=TOP_BANNER\&context=storylines_menu}{The
Coronavirus Outbreak}

\begin{itemize}
\tightlist
\item
  live\href{https://www.nytimes.com/2020/08/01/world/coronavirus-covid-19.html?action=click\&pgtype=Article\&state=default\&region=TOP_BANNER\&context=storylines_menu}{Latest
  Updates}
\item
  \href{https://www.nytimes.com/interactive/2020/us/coronavirus-us-cases.html?action=click\&pgtype=Article\&state=default\&region=TOP_BANNER\&context=storylines_menu}{Maps
  and Cases}
\item
  \href{https://www.nytimes.com/interactive/2020/science/coronavirus-vaccine-tracker.html?action=click\&pgtype=Article\&state=default\&region=TOP_BANNER\&context=storylines_menu}{Vaccine
  Tracker}
\item
  \href{https://www.nytimes.com/interactive/2020/07/29/us/schools-reopening-coronavirus.html?action=click\&pgtype=Article\&state=default\&region=TOP_BANNER\&context=storylines_menu}{What
  School May Look Like}
\item
  \href{https://www.nytimes.com/live/2020/07/31/business/stock-market-today-coronavirus?action=click\&pgtype=Article\&state=default\&region=TOP_BANNER\&context=storylines_menu}{Economy}
\end{itemize}

Advertisement

\protect\hyperlink{after-top}{Continue reading the main story}

Supported by

\protect\hyperlink{after-sponsor}{Continue reading the main story}

Mind

\hypertarget{the-pandemics-mental-toll-more-ripple-than-tsunami}{%
\section{The Pandemic's Mental Toll: More Ripple Than
Tsunami}\label{the-pandemics-mental-toll-more-ripple-than-tsunami}}

Some health officials have forecast a steep rise in new mental health
disorders. But the impact isn't likely to last.

\includegraphics{https://static01.nyt.com/images/2020/06/23/science/00SCI-VIRUS-MENTAL1/00SCI-VIRUS-MENTAL1-articleLarge.jpg?quality=75\&auto=webp\&disable=upscale}

\href{https://www.nytimes.com/by/benedict-carey}{\includegraphics{https://static01.nyt.com/images/2018/02/16/multimedia/author-benedict-carey/author-benedict-carey-thumbLarge.jpg}}

By \href{https://www.nytimes.com/by/benedict-carey}{Benedict Carey}

\begin{itemize}
\item
  Published June 21, 2020Updated June 22, 2020
\item
  \begin{itemize}
  \item
  \item
  \item
  \item
  \item
  \item
  \end{itemize}
\end{itemize}

The psychological fallout from the coronavirus pandemic has yet to fully
show itself, but some experts have forecast a tsunami of new disorders,
and news accounts have
\href{https://www.washingtonpost.com/health/2020/05/04/mental-health-coronavirus}{amplified
that message}.

The World Health Organization
\href{https://www.who.int/news-room/detail/14-05-2020-substantial-investment-needed-to-avert-mental-health-crisis}{warned
in May} of ``a massive increase in mental health conditions in the
coming months,'' wrought by anxiety and isolation. Digital platforms
such as Crisis Text Line and Talkspace regularly reported spikes in
activity through the spring. And more than half of American adults said
the pandemic had worsened their mental health, according to a recent
survey by the Kaiser Family Foundation.

But this wave of new mental problems is still well offshore, and it
could yet prove to be a mirage. Psychiatrists and therapists who work
with people in the wake of earthquakes, hurricanes and other disasters
noted that surges in anxiety and helplessness were natural reactions
that seldom become traumatic or chronic. Surveys that ask people about
their emotions are poor predictors of lasting distress, and the
prevalence of severe mental disorders, like schizophrenia and bipolar
disorder, are stable and very unlikely to have changed because of the
pandemic. Most people living with these conditions needed continuous
care before the virus took hold and will still need it when outbreaks
are contained.

``In most disasters, the vast majority of people do well,'' said Dr.
Steven Southwick, a professor of psychiatry at Yale who has worked with
survivors after numerous cataclysms, including mass shootings. ``Very
few people understand how resilient they really are until faced with
extraordinary circumstances. In fact, one of our first jobs in these
situations is to call attention to just that.''

Prescription trends provide little evidence of an explosion in mental
disorders in recent months. In March, at the height of the epidemic in
many regions, prescriptions for anti-anxiety drugs such as Xanax and
Klonopin were up by 15 percent over February; antidepressants were up by
14 percent, and sleeping pills by 5 percent, according to data provided
by OptumRx, the pharmacy benefit arm of UnitedHealth Group.

\hypertarget{latest-updates-global-coronavirus-outbreak}{%
\section{\texorpdfstring{\href{https://www.nytimes.com/2020/08/01/world/coronavirus-covid-19.html?action=click\&pgtype=Article\&state=default\&region=MAIN_CONTENT_1\&context=storylines_live_updates}{Latest
Updates: Global Coronavirus
Outbreak}}{Latest Updates: Global Coronavirus Outbreak}}\label{latest-updates-global-coronavirus-outbreak}}

Updated 2020-08-02T07:42:09.613Z

\begin{itemize}
\tightlist
\item
  \href{https://www.nytimes.com/2020/08/01/world/coronavirus-covid-19.html?action=click\&pgtype=Article\&state=default\&region=MAIN_CONTENT_1\&context=storylines_live_updates\#link-34047410}{The
  U.S. reels as July cases more than double the total of any other
  month.}
\item
  \href{https://www.nytimes.com/2020/08/01/world/coronavirus-covid-19.html?action=click\&pgtype=Article\&state=default\&region=MAIN_CONTENT_1\&context=storylines_live_updates\#link-780ec966}{Top
  U.S. officials work to break an impasse over the federal jobless
  benefit.}
\item
  \href{https://www.nytimes.com/2020/08/01/world/coronavirus-covid-19.html?action=click\&pgtype=Article\&state=default\&region=MAIN_CONTENT_1\&context=storylines_live_updates\#link-2bc8948}{Its
  outbreak untamed, Melbourne goes into even greater lockdown.}
\end{itemize}

\href{https://www.nytimes.com/2020/08/01/world/coronavirus-covid-19.html?action=click\&pgtype=Article\&state=default\&region=MAIN_CONTENT_1\&context=storylines_live_updates}{See
more updates}

More live coverage:
\href{https://www.nytimes.com/live/2020/07/31/business/stock-market-today-coronavirus?action=click\&pgtype=Article\&state=default\&region=MAIN_CONTENT_1\&context=storylines_live_updates}{Markets}

But those rates began to decline in early April. And the total
prescriptions in that month were 8.7 million for anxiety drugs and 27.4
million for antidepressants --- very close to their usual averages for
April, according to data supplied by IQVIA, a health care analytics
firm. Prescriptions for other categories of psychiatric drugs, like
anti-psychosis medications, remained at average monthly levels through
March and April.

``Modest transient rises in the use of antidepressant and anxiety
medications allay concerns over the pandemic having driven steep
increases in common mood and anxiety disorders,'' Dr. Mark Olfson, a
professor of psychiatry at Columbia, said in an email. The March bump in
prescriptions for anxiety drugs in particular could partly reflect
people stockpiling medications that they were already taking, or
increasing their dosage, he said.

\includegraphics{https://static01.nyt.com/images/2020/06/23/science/00SCI-VIRUS-MENTAL2/00SCI-VIRUS-MENTAL2-articleLarge.jpg?quality=75\&auto=webp\&disable=upscale}

In the wake of the Sept. 11 terror attacks, many health officials also
were concerned about a wave of new mental disorders that might overwhelm
the system.
\href{https://ajp.psychiatryonline.org/doi/pdf/10.1176/appi.ajp.161.8.1377}{In
a 2004 study,} researchers dug into prescription data for the month
before and after the terrorist attack, comparing prescription rates
across a range of psychiatric drugs. ``The acute shock and fear of the
events of September 11 were not accompanied by a commensurate increase
in the use of psychotropic medications,'' they concluded, except for a
modest increase in New York City.

The evidence from recent surveys asking people about their emotions
during the pandemic is not convincing one way or the other either,
experts said. One reason is that these surveys often do not make
distinctions between people in the thick of the action --- front-line
workers, in this case --- and everyone else. Millions of Americans have
been juggling Zoom cocktail hours with Netflix binges: a time-management
challenge, perhaps, but not one that has been linked to prolonged
trauma.

Moreover, psychological distress usually takes time to consolidate into
the kind of persistent condition that drives people to seek treatment,
revealing a diagnosable psychiatric disorder. Generalized anxiety
disorder, for instance, is defined in part by excessive anxiety for at
least six months. Post-traumatic stress requires, first, experiencing a
life-threatening event, either personally; through a loved one; or up
close, like witnessing deaths in an intensive care unit. Nightmares and
other reverberations of the trauma are common, but these typically must
persist for at least three months to qualify for the full diagnosis of a
chronic condition.

``There are a number of surveys out there, and I think they are all
useful, to some extent,'' said Emma Beth McGinty, an associate professor
in the Johns Hopkins Bloomberg School of Public Health. ``But they're
using a mishmash of measures of symptoms of depression and anxiety, and
not a validated psychiatric instrument,'' or questionnaire.

The
\href{https://jamanetwork.com/journals/jama/fullarticle/2766941?guestAccessKey=8b48ddf2-8b20-4aaf-8d21-1b201bc8f567\&utm_source=For_The_Media\&utm_medium=referral\&utm_campaign=ftm_links\&utm_content=tfl\&utm_term=060320}{best
American survey to date}, posted early this month by JAMA and led by Dr.
McGinty, administered a standard,
\href{https://www.hcp.med.harvard.edu/ncs/ftpdir/k6/Self\%20admin_K6.pdf}{widely
studied psychiatric questionnaire} online to a nationally representative
sample of 1,468 adults. It found that 14 percent of people had high
levels of psychological distress, compared with an average of 4 percent
during the pre-Covid era. It found little difference in respondents'
feelings of loneliness, compared to averages before the pandemic.

\href{https://www.nytimes.com/news-event/coronavirus?action=click\&pgtype=Article\&state=default\&region=MAIN_CONTENT_3\&context=storylines_faq}{}

\hypertarget{the-coronavirus-outbreak-}{%
\subsubsection{The Coronavirus Outbreak
›}\label{the-coronavirus-outbreak-}}

\hypertarget{frequently-asked-questions}{%
\paragraph{Frequently Asked
Questions}\label{frequently-asked-questions}}

Updated July 27, 2020

\begin{itemize}
\item ~
  \hypertarget{should-i-refinance-my-mortgage}{%
  \paragraph{Should I refinance my
  mortgage?}\label{should-i-refinance-my-mortgage}}

  \begin{itemize}
  \tightlist
  \item
    \href{https://www.nytimes.com/article/coronavirus-money-unemployment.html?action=click\&pgtype=Article\&state=default\&region=MAIN_CONTENT_3\&context=storylines_faq}{It
    could be a good idea,} because mortgage rates have
    \href{https://www.nytimes.com/2020/07/16/business/mortgage-rates-below-3-percent.html?action=click\&pgtype=Article\&state=default\&region=MAIN_CONTENT_3\&context=storylines_faq}{never
    been lower.} Refinancing requests have pushed mortgage applications
    to some of the highest levels since 2008, so be prepared to get in
    line. But defaults are also up, so if you're thinking about buying a
    home, be aware that some lenders have tightened their standards.
  \end{itemize}
\item ~
  \hypertarget{what-is-school-going-to-look-like-in-september}{%
  \paragraph{What is school going to look like in
  September?}\label{what-is-school-going-to-look-like-in-september}}

  \begin{itemize}
  \tightlist
  \item
    It is unlikely that many schools will return to a normal schedule
    this fall, requiring the grind of
    \href{https://www.nytimes.com/2020/06/05/us/coronavirus-education-lost-learning.html?action=click\&pgtype=Article\&state=default\&region=MAIN_CONTENT_3\&context=storylines_faq}{online
    learning},
    \href{https://www.nytimes.com/2020/05/29/us/coronavirus-child-care-centers.html?action=click\&pgtype=Article\&state=default\&region=MAIN_CONTENT_3\&context=storylines_faq}{makeshift
    child care} and
    \href{https://www.nytimes.com/2020/06/03/business/economy/coronavirus-working-women.html?action=click\&pgtype=Article\&state=default\&region=MAIN_CONTENT_3\&context=storylines_faq}{stunted
    workdays} to continue. California's two largest public school
    districts --- Los Angeles and San Diego --- said on July 13, that
    \href{https://www.nytimes.com/2020/07/13/us/lausd-san-diego-school-reopening.html?action=click\&pgtype=Article\&state=default\&region=MAIN_CONTENT_3\&context=storylines_faq}{instruction
    will be remote-only in the fall}, citing concerns that surging
    coronavirus infections in their areas pose too dire a risk for
    students and teachers. Together, the two districts enroll some
    825,000 students. They are the largest in the country so far to
    abandon plans for even a partial physical return to classrooms when
    they reopen in August. For other districts, the solution won't be an
    all-or-nothing approach.
    \href{https://bioethics.jhu.edu/research-and-outreach/projects/eschool-initiative/school-policy-tracker/}{Many
    systems}, including the nation's largest, New York City, are
    devising
    \href{https://www.nytimes.com/2020/06/26/us/coronavirus-schools-reopen-fall.html?action=click\&pgtype=Article\&state=default\&region=MAIN_CONTENT_3\&context=storylines_faq}{hybrid
    plans} that involve spending some days in classrooms and other days
    online. There's no national policy on this yet, so check with your
    municipal school system regularly to see what is happening in your
    community.
  \end{itemize}
\item ~
  \hypertarget{is-the-coronavirus-airborne}{%
  \paragraph{Is the coronavirus
  airborne?}\label{is-the-coronavirus-airborne}}

  \begin{itemize}
  \tightlist
  \item
    The coronavirus
    \href{https://www.nytimes.com/2020/07/04/health/239-experts-with-one-big-claim-the-coronavirus-is-airborne.html?action=click\&pgtype=Article\&state=default\&region=MAIN_CONTENT_3\&context=storylines_faq}{can
    stay aloft for hours in tiny droplets in stagnant air}, infecting
    people as they inhale, mounting scientific evidence suggests. This
    risk is highest in crowded indoor spaces with poor ventilation, and
    may help explain super-spreading events reported in meatpacking
    plants, churches and restaurants.
    \href{https://www.nytimes.com/2020/07/06/health/coronavirus-airborne-aerosols.html?action=click\&pgtype=Article\&state=default\&region=MAIN_CONTENT_3\&context=storylines_faq}{It's
    unclear how often the virus is spread} via these tiny droplets, or
    aerosols, compared with larger droplets that are expelled when a
    sick person coughs or sneezes, or transmitted through contact with
    contaminated surfaces, said Linsey Marr, an aerosol expert at
    Virginia Tech. Aerosols are released even when a person without
    symptoms exhales, talks or sings, according to Dr. Marr and more
    than 200 other experts, who
    \href{https://academic.oup.com/cid/article/doi/10.1093/cid/ciaa939/5867798}{have
    outlined the evidence in an open letter to the World Health
    Organization}.
  \end{itemize}
\item ~
  \hypertarget{what-are-the-symptoms-of-coronavirus}{%
  \paragraph{What are the symptoms of
  coronavirus?}\label{what-are-the-symptoms-of-coronavirus}}

  \begin{itemize}
  \tightlist
  \item
    Common symptoms
    \href{https://www.nytimes.com/article/symptoms-coronavirus.html?action=click\&pgtype=Article\&state=default\&region=MAIN_CONTENT_3\&context=storylines_faq}{include
    fever, a dry cough, fatigue and difficulty breathing or shortness of
    breath.} Some of these symptoms overlap with those of the flu,
    making detection difficult, but runny noses and stuffy sinuses are
    less common.
    \href{https://www.nytimes.com/2020/04/27/health/coronavirus-symptoms-cdc.html?action=click\&pgtype=Article\&state=default\&region=MAIN_CONTENT_3\&context=storylines_faq}{The
    C.D.C. has also} added chills, muscle pain, sore throat, headache
    and a new loss of the sense of taste or smell as symptoms to look
    out for. Most people fall ill five to seven days after exposure, but
    symptoms may appear in as few as two days or as many as 14 days.
  \end{itemize}
\item ~
  \hypertarget{does-asymptomatic-transmission-of-covid-19-happen}{%
  \paragraph{Does asymptomatic transmission of Covid-19
  happen?}\label{does-asymptomatic-transmission-of-covid-19-happen}}

  \begin{itemize}
  \tightlist
  \item
    So far, the evidence seems to show it does. A widely cited
    \href{https://www.nature.com/articles/s41591-020-0869-5}{paper}
    published in April suggests that people are most infectious about
    two days before the onset of coronavirus symptoms and estimated that
    44 percent of new infections were a result of transmission from
    people who were not yet showing symptoms. Recently, a top expert at
    the World Health Organization stated that transmission of the
    coronavirus by people who did not have symptoms was ``very rare,''
    \href{https://www.nytimes.com/2020/06/09/world/coronavirus-updates.html?action=click\&pgtype=Article\&state=default\&region=MAIN_CONTENT_3\&context=storylines_faq\#link-1f302e21}{but
    she later walked back that statement.}
  \end{itemize}
\end{itemize}

``The longer people experience these levels of psychological distress,
the more likely they are to present with a diagnosis that would benefit
from treatment,'' Dr. McGinty said in a phone interview. ``But the
question of whether that's really going to happen is an open one. We did
this in early April, right as the shutdown and stay-at-home orders were
implemented, when people were experiencing all this for the first time.
One might hypothesize that the stress has eased, we've gotten more used
to this and the world has opened up a bit.''

Dr. McGinty and her collaborators plan to conduct another such survey
later this summer, she said, and possibly one in the fall, to see
whether levels of psychological distress change as the epidemic changes
shape through the year.

The fear of infection and disruptions caused by the coronavirus, without
question, have intensified the distress of many individuals, especially
those who have lost regular access to care as a result, or who had
pre-existing dread of infections --- from obsessive-compulsive disorder,
for example.

``In my mind, since this started, there's nowhere that's clean,'' said
Naomi, a doctor in New York City who has a diagnosis of O.C.D.; she
asked that her last name be omitted for privacy. ``It was like a
complete meltdown for me, because being a doctor, I know so much about
pathogens and what can so easily happen.''

But when it comes to collective trauma of the chronic, disabling kind,
many experts remain skeptical. Studies done in the wake of hurricanes,
earthquakes and floods find that no more than 10 percent of people
develop such prolonged reactions --- and those are the people directly
and intimately hit by the destruction. The other 90 percent pick up the
pieces, and in time the nightmares and surges of panic recede.

Living through a pandemic is nothing like surviving a natural
catastrophe such as those: it's less visible, less predictable, a
creeping threat rather than flying debris --- a marathon,
psychologically, rather than a sprint to safety. A wave of new mental
health disorders may indeed be on the way, especially if Covid-19 cases
explode again late in the year, or the economic downturn deepens.

But the evidence so far says nothing persuasive about whether it will be
a tsunami or a ripple.

\textbf{\emph{{[}}\href{http://on.fb.me/1paTQ1h}{\emph{Like the Science
Times page on Facebook.}}} ****** \emph{\textbar{} Sign up for the}
\textbf{\href{http://nyti.ms/1MbHaRU}{\emph{Science Times
newsletter.}}\emph{{]}}}

Advertisement

\protect\hyperlink{after-bottom}{Continue reading the main story}

\hypertarget{site-index}{%
\subsection{Site Index}\label{site-index}}

\hypertarget{site-information-navigation}{%
\subsection{Site Information
Navigation}\label{site-information-navigation}}

\begin{itemize}
\tightlist
\item
  \href{https://help.nytimes.com/hc/en-us/articles/115014792127-Copyright-notice}{©~2020~The
  New York Times Company}
\end{itemize}

\begin{itemize}
\tightlist
\item
  \href{https://www.nytco.com/}{NYTCo}
\item
  \href{https://help.nytimes.com/hc/en-us/articles/115015385887-Contact-Us}{Contact
  Us}
\item
  \href{https://www.nytco.com/careers/}{Work with us}
\item
  \href{https://nytmediakit.com/}{Advertise}
\item
  \href{http://www.tbrandstudio.com/}{T Brand Studio}
\item
  \href{https://www.nytimes.com/privacy/cookie-policy\#how-do-i-manage-trackers}{Your
  Ad Choices}
\item
  \href{https://www.nytimes.com/privacy}{Privacy}
\item
  \href{https://help.nytimes.com/hc/en-us/articles/115014893428-Terms-of-service}{Terms
  of Service}
\item
  \href{https://help.nytimes.com/hc/en-us/articles/115014893968-Terms-of-sale}{Terms
  of Sale}
\item
  \href{https://spiderbites.nytimes.com}{Site Map}
\item
  \href{https://help.nytimes.com/hc/en-us}{Help}
\item
  \href{https://www.nytimes.com/subscription?campaignId=37WXW}{Subscriptions}
\end{itemize}
