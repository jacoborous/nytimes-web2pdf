Sections

SEARCH

\protect\hyperlink{site-content}{Skip to
content}\protect\hyperlink{site-index}{Skip to site index}

\href{https://www.nytimes.com/section/books}{Books}

\href{https://myaccount.nytimes.com/auth/login?response_type=cookie\&client_id=vi}{}

\href{https://www.nytimes.com/section/todayspaper}{Today's Paper}

\href{/section/books}{Books}\textbar{}Charles Webb, Elusive Author of
`The Graduate,' Dies at 81

\url{https://nyti.ms/2NB5yAh}

\begin{itemize}
\item
\item
\item
\item
\item
\end{itemize}

Advertisement

\protect\hyperlink{after-top}{Continue reading the main story}

Supported by

\protect\hyperlink{after-sponsor}{Continue reading the main story}

\hypertarget{charles-webb-elusive-author-of-the-graduate-dies-at-81}{%
\section{Charles Webb, Elusive Author of `The Graduate,' Dies at
81}\label{charles-webb-elusive-author-of-the-graduate-dies-at-81}}

His novel was turned into an era-defining movie, but he was never
comfortable with its success, and he chose to live in poverty.

\includegraphics{https://static01.nyt.com/images/2020/06/29/obituaries/26Webb1/26Webb1-articleLarge.jpg?quality=75\&auto=webp\&disable=upscale}

\href{https://www.nytimes.com/by/john-leland}{\includegraphics{https://static01.nyt.com/images/2018/02/20/multimedia/author-john-leland/author-john-leland-thumbLarge.jpg}}

By \href{https://www.nytimes.com/by/john-leland}{John Leland}

\begin{itemize}
\item
  June 28, 2020
\item
  \begin{itemize}
  \item
  \item
  \item
  \item
  \item
  \end{itemize}
\end{itemize}

Charles Webb, who wrote the 1963 novel ``The Graduate,'' the basis for
the hit 1967 film, and then spent decades running from its success, died
on June 16 in East Sussex, England. He was 81.

A spokesman for his son John confirmed the death, in a hospital, but did
not specify the cause.

Mr. Webb's novel, written shortly after college and based largely on his
relationship with his wife, Eve Rudd, was made into an era-defining
film, directed by Mike Nichols and starring Dustin Hoffman and Anne
Bancroft, that gave voice to a generation's youthful rejection of
materialism. Mr. Webb and his wife, both born into privilege, carried
that rejection well beyond youth, choosing to live in poverty and giving
away whatever money came their way, even as the movie's acclaim
continued to follow them.

``My whole life has been measured by it,'' he told the British newspaper
The Telegraph in 2007, when the couple were living in a drab hotel room
paid for by British social services.

Mr. Webb published eight books, including a sequel to ``The Graduate,''
``Home School'' (2007), in which the main characters, Benjamin and
Elaine, are grown up and teaching their children themselves. He agreed
to publish it only to pay off a 30,000-pound debt, said Jack Malvern, a
Times of London reporter who was friendly with Mr. Webb and helped with
that deal.

``He had a very odd relationship with money,'' said Caroline Dawnay, who
was briefly Mr. Webb's agent in the early 2000s when his novel ``New
Cardiff'' was made into the 2003 movie
\href{https://www.youtube.com/watch?v=HExTPHMoxr4}{``Hope Springs,''}
starring Colin Firth. ``He never wanted any. He had an anarchist view of
the relationship between humanity and money.''

He gave away homes, paintings, his inheritance, even his royalties from
``The Graduate,'' which became a million-seller after the movie's
success, to the benefit of the Anti-Defamation League. He awarded his
10,000-pound payout from ``Hope Springs'' as a prize to a performance
artist named Dan Shelton, who had mailed himself to the Tate Modern in a
cardboard box.

At his second wedding to Ms. Rudd --- they married in 1962, then
divorced in 1981 to protest the institution of marriage, then remarried
around 2001 for immigration purposes --- he did not give his bride a
ring, because he disapproved of jewelry. Ms. Dawnay, the only witness
save two strangers pulled in off the street, recalled that the couple
walked nine miles to the registry office for the ceremony, wearing the
only clothes they owned.

Lots of people momentarily embrace the idea of leaving the rat race,
like the characters in ``The Graduate.'' Mr. Webb and Ms. Rudd did it,
with all the consequences it entailed. If they regretted the choice,
they did not say so.

``When you run out of money it's a purifying experience,'' Mr. Webb told
The Times of London after the couple moved to England. ``It focuses the
mind like nothing else.''

\includegraphics{https://static01.nyt.com/images/2020/06/26/obituaries/00Webb3/merlin_81144752_b5fdc2cb-cdaa-401b-b33e-3c7f074ca6be-articleLarge.jpg?quality=75\&auto=webp\&disable=upscale}

Charles Richard Webb was born on June 9, 1939, in San Francisco, and
grew up in Pasadena, Calif. His father, Dr. Richard Webb, was a heart
specialist, part of a wealthy social circle like the one Charles would
skewer in ``The Graduate.'' (Charles described his relationship with his
father as ``reasonably bad.'') His mother, Janet Farrington Webb, was,
he said, a socialite and an avid reader from whom he ``was always
looking for crumbs of approval.'' He said ``The Graduate'' was an
attempt to win her favor; it went decidedly wrong.

A younger brother, Sidney Farrington Webb, became a doctor in Las
Cruces, N.M.

Charles went to boarding school and then to Williams College in
Massachusetts, where he earned a degree in American history and
literature in 1961. He said his schools had been ``chosen'' for him ``on
the basis of how it looked.'' A mediocre student, he nonetheless managed
to win a two-year writing fellowship, which he used to write ``The
Graduate.''

While at Williams, he met Ms. Rudd, a Bennington College student. She
was a former debutante from a family of teachers with a bohemian streak
--- her brother was the avant-garde jazz trombonist
\href{https://www.nytimes.com/2017/12/26/obituaries/roswell-rudd-82-trombonist-with-a-wide-open-approach-is-dead.html}{Roswell
Rudd} --- and they both rejected the bourgeois worlds of their families.
Their first date, they told interviewers, was in a cemetery.

Their romance, and her mother's disapproval of him, became the basis for
``The Graduate.'' The inspiration for the character Mrs. Robinson, who
seduces young Benjamin, may have come from one of his parents' friends,
whom he accidentally saw naked.

Reviewing the book in The Times, Orville Prescott called it a
``fictional failure'' but favorably compared its protagonist to Holden
Caulfield of ``The Catcher in the Rye.''

With its mumbling ennui and conversations that do not connect, the novel
captured the moment just before the repressed Eisenhower era blossomed
into the Technicolor 1960s. The characters are not idealistic; they're
groping for ideals, their flight from their parents' values and
lifestyles more solitary than collective. In the last pages, Benjamin
and Elaine are alone on a bus, shaken, heading into a future that is
opaque to them. Hello darkness, my old friend.

Image

Mr. Webb's romance with Eve Rudd, and her mother's disapproval of him,
became the basis for ``The Graduate.''

So began the iconoclastic journey of Charles and Eve, who later adopted
the single name Fred, in solidarity with a self-help group for men with
low self-esteem. Despite her parents' intervention the couple married,
then later sold their wedding gifts back to the guests and donated the
money to charity.

``Their wedding was a total contradiction to the way they ended up
living,'' Priscilla Rudd Wolf, Eve's sister, said in an email. ``It was
a big wedding; my sister wore a white bridal gown; I was maid of honor.
It was in the Salisbury School Chapel, where my parents taught, and the
whole town was there.'' She added: ``They seemed like a typical
all-American couple off to a typical all-American life. But that wasn't
to be.''

Shedding their possessions became a full-time mission. They gave away a
California bungalow, the first of three houses they would jettison,
saying that owning things oppressed them.

Mr. Webb declined his inheritance from his father's family but was
unable to decline the money from his mother's; so they gave that away,
along with artwork by Andy Warhol, Roy Lichtenstein and Robert
Rauschenberg.

As the 1960s bloomed, the couple underwent gestalt therapy. Fred, a
painter, hosted a one-woman show in the nude as a feminist statement.
She shaved her head --- in order, she said, to shed the oppressive
demands of feminine adornment.

They moved to California and then back east to a dilapidated house in
Hastings-on-Hudson, N.Y., in Westchester County, and had two sons, John
and David.

Mr. Webb followed ``The Graduate'' with ``Love, Roger'' (1969) and ``The
Marriage of a Young Stockbroker'' (1970), which Lawrence Turman, who
produced ``The Graduate,'' turned into a movie starring Richard
Benjamin. It fizzled. Critics compared his later books unfavorably with
his debut.

He refused to do book signings, Ms. Dawnay said, viewing them as ``a sin
against decency.''

In the late 1970s the couple moved back to the West Coast and took their
sons out of school, choosing to home-school them, which was not
sanctioned at the time. So the family moved around, at one point living
in a Volkswagen bus, driving from one campground to another. In a 1992
interview with The Washington Post, John Webb called that part of his
education
``\href{https://www.washingtonpost.com/archive/lifestyle/1992/12/20/the-dropout/b90ccd01-4bab-48b7-9801-77092aa1c943/?itid=lk_inline_manual_28}{unschooling}.''

Charles Webb worked menial jobs: clerk at a Kmart, itinerant farmworker,
house cleaner. The couple were caretakers at a nudist colony in New
Jersey, earning \$198 a week.

Mr. Webb complained about being tied to ``The Graduate,'' but in the
early 1990s he wrote a sequel, ``Gwen,'' narrated by Benjamin and
Elaine's daughter. Benjamin works at a Kmart and as a janitor at his old
school, finding liberation in giving up his material trappings to serve
others.

``Gwen'' was never published; Mr. Webb went nearly 25 years between
books before ``New Cardiff,'' in 2001.

By then the couple were living in England --- they had moved there, he
said, so he could try writing an English character --- and their sons
were grown.

Ms. Dawnay, who visited the couple in Brighton, said they lived with
almost no furniture and only one change of clothes. Though ``New
Cardiff'' was warmly received, it did not revive Mr. Webb's career, nor
did the ``Graduate'' sequel he finally did publish, ``Home School.''

Fred, Mr. Webb's wife, died in 2019, Mr. Malvern said, leaving him quite
alone, although he is survived by his sons --- David, a performance
artist who once cooked a copy of ``The Graduate'' and ate it with
cranberry sauce, and
\href{https://ihsmarkit.com/experts/webb-john.html}{John,} a director at
the consulting and research firm IHS Markit --- and his brother. Mr.
Malvern said he did not know whether Mr. Webb had still been writing.

Mr. Webb's death brings to a close a decades-long experiment that was
less a retreat than an attempt to change the terms of engagement between
artists and the world.

As he once told The Boston Globe, ``The public's praise of creative
people is a mask --- a mask for jealousy or hatred.'' By the couple's
various renunciations, he said, ``We hope to make the point that the
creative process is really a defense mechanism on the part of artists
--- that creativity is not a romantic notion.''

Advertisement

\protect\hyperlink{after-bottom}{Continue reading the main story}

\hypertarget{site-index}{%
\subsection{Site Index}\label{site-index}}

\hypertarget{site-information-navigation}{%
\subsection{Site Information
Navigation}\label{site-information-navigation}}

\begin{itemize}
\tightlist
\item
  \href{https://help.nytimes.com/hc/en-us/articles/115014792127-Copyright-notice}{©~2020~The
  New York Times Company}
\end{itemize}

\begin{itemize}
\tightlist
\item
  \href{https://www.nytco.com/}{NYTCo}
\item
  \href{https://help.nytimes.com/hc/en-us/articles/115015385887-Contact-Us}{Contact
  Us}
\item
  \href{https://www.nytco.com/careers/}{Work with us}
\item
  \href{https://nytmediakit.com/}{Advertise}
\item
  \href{http://www.tbrandstudio.com/}{T Brand Studio}
\item
  \href{https://www.nytimes.com/privacy/cookie-policy\#how-do-i-manage-trackers}{Your
  Ad Choices}
\item
  \href{https://www.nytimes.com/privacy}{Privacy}
\item
  \href{https://help.nytimes.com/hc/en-us/articles/115014893428-Terms-of-service}{Terms
  of Service}
\item
  \href{https://help.nytimes.com/hc/en-us/articles/115014893968-Terms-of-sale}{Terms
  of Sale}
\item
  \href{https://spiderbites.nytimes.com}{Site Map}
\item
  \href{https://help.nytimes.com/hc/en-us}{Help}
\item
  \href{https://www.nytimes.com/subscription?campaignId=37WXW}{Subscriptions}
\end{itemize}
