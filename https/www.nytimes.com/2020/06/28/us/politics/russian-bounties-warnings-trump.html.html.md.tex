Sections

SEARCH

\protect\hyperlink{site-content}{Skip to
content}\protect\hyperlink{site-index}{Skip to site index}

\href{https://www.nytimes.com/section/politics}{Politics}

\href{https://myaccount.nytimes.com/auth/login?response_type=cookie\&client_id=vi}{}

\href{https://www.nytimes.com/section/todayspaper}{Today's Paper}

\href{/section/politics}{Politics}\textbar{}Spies and Commandos Warned
Months Ago of Russian Bounties on U.S. Troops

\url{https://nyti.ms/2Vqy2kE}

\begin{itemize}
\item
\item
\item
\item
\item
\item
\end{itemize}

Advertisement

\protect\hyperlink{after-top}{Continue reading the main story}

Supported by

\protect\hyperlink{after-sponsor}{Continue reading the main story}

\hypertarget{spies-and-commandos-warned-months-ago-of-russian-bounties-on-us-troops}{%
\section{Spies and Commandos Warned Months Ago of Russian Bounties on
U.S.
Troops}\label{spies-and-commandos-warned-months-ago-of-russian-bounties-on-us-troops}}

The recovery of large amounts of American cash at a Taliban outpost in
Afghanistan helped tip off U.S. officials. It is believed that at least
one U.S. troop death was the result of the bounties.

\includegraphics{https://static01.nyt.com/images/2020/06/28/us/politics/28dc-intel-pix/merlin_153270840_2aa28df4-33dc-45a7-989e-d0f11d063775-articleLarge.jpg?quality=75\&auto=webp\&disable=upscale}

\href{https://www.nytimes.com/by/eric-schmitt}{\includegraphics{https://static01.nyt.com/images/2018/06/12/multimedia/author-eric-schmitt/author-eric-schmitt-thumbLarge-v2.png}}\href{https://www.nytimes.com/by/adam-goldman}{\includegraphics{https://static01.nyt.com/images/2018/07/12/multimedia/author-adam-goldman/author-adam-goldman-thumbLarge.png}}\href{https://www.nytimes.com/by/nicholas-fandos}{\includegraphics{https://static01.nyt.com/images/2018/11/06/multimedia/author-nicholas-fandos/author-nicholas-fandos-thumbLarge-v2.png}}

By \href{https://www.nytimes.com/by/eric-schmitt}{Eric Schmitt},
\href{https://www.nytimes.com/by/adam-goldman}{Adam Goldman} and
\href{https://www.nytimes.com/by/nicholas-fandos}{Nicholas Fandos}

\begin{itemize}
\item
  Published June 28, 2020Updated July 29, 2020
\item
  \begin{itemize}
  \item
  \item
  \item
  \item
  \item
  \item
  \end{itemize}
\end{itemize}

WASHINGTON --- United States intelligence officers and Special
Operations forces in Afghanistan alerted their superiors as early as
January to a suspected
\href{https://www.nytimes.com/2020/07/29/us/politics/trump-putin-bounties.html}{Russian
plot to pay bounties} to the
\href{https://www.nytimes.com/2020/06/30/us/politics/russian-bounties-afghanistan-intelligence.html}{Taliban}
to kill
\href{https://www.nytimes.com/2020/06/30/us/politics/russian-bounties-afghanistan-intelligence.html}{American
troops} in Afghanistan, according to officials briefed on the matter.
They believed at least one U.S. troop death was the result of the
bounties, two of the officials said.

The crucial information that led the spies and commandos to focus on the
bounties included the recovery of a large amount of American cash from a
raid on a
\href{https://www.nytimes.com/2020/07/01/world/asia/afghan-russia-bounty-middleman.html}{Taliban}
outpost that prompted suspicions. Interrogations of captured militants
and criminals played a central role in making the intelligence community
confident in its assessment that the
\href{https://www.nytimes.com/2020/06/29/us/politics/trump-russia-plot-afghanistan.html}{Russians}
had offered and paid bounties in 2019, another official has said.

Armed with this information, military and intelligence officials have
been reviewing American and other coalition combat casualties over the
past 18 months to determine whether any were victims of the plot. Four
Americans were killed in combat in early 2020, but the Taliban have not
attacked American positions since
\href{https://www.nytimes.com/2020/02/29/world/asia/us-taliban-deal.html}{a
February agreement} to end the long-running war in Afghanistan.

The details added to the picture of the classified intelligence
assessment, which The New York Times reported Friday
\href{https://www.nytimes.com/2020/06/26/us/politics/russia-afghanistan-bounties.html}{has
been under discussion} inside the Trump administration since at least
March, and emerged as the White House confronted a growing chorus of
criticism on Sunday over its apparent failure to authorize a response to
Russia.

Mr. Trump defended himself by denying
\href{https://www.nytimes.com/2020/06/26/us/politics/russia-afghanistan-bounties.html}{the
Times report} that he had been briefed on the intelligence, expanding on
a similar White House rebuttal a day earlier. But leading congressional
Democrats and some Republicans demanded a response to Russia that,
according to officials, the administration has yet to authorize.

The president ``needs to immediately expose and handle this, and stop
Russia's shadow war,'' Representative Adam Kinzinger, Republican of
Illinois and a member of the House Foreign Affairs Committee,
\href{https://twitter.com/RepKinzinger/status/1276647856133332994}{wrote
on Twitter}.

Appearing on the ABC program ``This Week,'' Speaker Nancy Pelosi said
she had not been briefed on the intelligence assessment and had asked
for an immediate report to Congress. She accused Mr. Trump of wanting
``to ignore'' any charges against Russia.

``Russia has never gotten over the humiliation they suffered in
Afghanistan, and now they are taking it out on us, our troops,'' she
said of the Soviet Union's bloody war there in the 1980s. ``This is
totally outrageous. You would think that the minute the president heard
of it, he would want to know more instead of denying that he knew
anything.''

Spokespeople for the C.I.A., the director of national intelligence and
the Pentagon declined to comment on the new findings. A National
Security Council spokesman, John L. Ullyot, said in a statement on
Sunday night, ``The veracity of the underlying allegations continues to
be evaluated.''

Mr. Trump
\href{https://twitter.com/realDonaldTrump/status/1277431695248183298?s=20}{said
Sunday night on Twitter} that ``Intel just reported to me that they did
not find this info credible, and therefore did not report it to me or
@VP.'' One senior administration official offered a similar explanation,
saying that Mr. Trump was not briefed because the intelligence agencies
had come to no consensus on the findings.

But another official said there was broad agreement that the
intelligence assessment was accurate, with some complexities because
different aspects of the intelligence --- including interrogations and
surveillance data --- resulted in some differences among agencies in how
much confidence to put in each type.

Though the White House press secretary, Kayleigh McEnany, claimed on
Saturday that Mr. Trump had not been briefed about the intelligence
report, one American official had told The Times that the report was
briefed to the highest levels of the White House. Another said it was
included in the President's Daily Brief, a compendium of foreign policy
and national security intelligence compiled for Mr. Trump to read.

Ms. McEnany did not challenge The Times's reporting on the existence of
the intelligence assessment, a National Security Council interagency
meeting about it in late March and the White House's inaction. Multiple
other news organizations also subsequently reported on the assessment,
and
\href{https://www.washingtonpost.com/national-security/russian-bounties-to-taliban-linked-militants-resulted-in-deaths-of-us-troops-according-to-intelligence-assessments/2020/06/28/74ffaec2-b96a-11ea-80b9-40ece9a701dc_story.html}{The
Washington Post first reported} on Sunday that the bounties were
believed to have resulted in the death of at least one American service
member.

The officials briefed on the matter said that the assessment had been
treated as a closely held secret but that the administration expanded
briefings about it over the last week --- including sharing information
about it with the British government, whose forces were among those said
to have been targeted.

Republicans in Congress demanded more information from the Trump
administration about what happened and how the White House planned to
respond.

Representative Liz Cheney of Wyoming, the third-ranking House
Republican, said in
\href{https://twitter.com/Liz_Cheney/status/1277200077724037122}{a
Twitter post} on Sunday: ``If reporting about Russian bounties on U.S.
forces is true, the White House must explain: 1. Why weren't the
president or vice president briefed? Was the info in the PDB? 2. Who did
know and when? 3. What has been done in response to protect our forces
\& hold Putin accountable?''

Multiple Republicans retweeted Ms. Cheney's post. Representative Daniel
Crenshaw, Republican of Texas and a former member of the Navy SEALs,
amplified her message,
\href{https://twitter.com/RepDanCrenshaw/status/1277258533776568321}{tweeting},
``We need answers.''

In a statement in response to questions, Senator Mitch McConnell of
Kentucky, the majority leader, said he had long warned about Russia's
work to undermine American interests in the Middle East and southwest
Asia and noted that he
\href{https://www.nytimes.com/2019/01/31/us/politics/senate-vote-syria-afghanistan.html}{wrote
an amendment} last year rebuking Mr. Trump's withdrawal of forces from
Syria and Afghanistan.

``The United States needs to prioritize defense resources, maintain a
sufficient regional military presence and continue to impose serious
consequences on those who threaten us and our allies --- like our
strikes in Syria and Afghanistan against ISIS, the Taliban and Russian
mercenary forces that threatened our partners,'' Mr. McConnell said.

Aides for other top Republicans either declined to comment or did not
respond to requests for comment on Sunday, including Representative
Kevin McCarthy of California, the top House Republican; Senator Marco
Rubio of Florida, the acting chairman of the Senate Intelligence
Committee; and Senator Jim Risch of Idaho, the chairman of the Senate
Foreign Relations~Committee.

In addition to saying he was never ``briefed or told'' about the
intelligence report --- a formulation that went beyond the White House
denial of any formal briefing --- Mr. Trump
\href{https://mobile.twitter.com/realDonaldTrump/status/1277202159109537793}{also
cast doubt} on the assessment's credibility, which statements from his
subordinates had not.

Specifically, he described the intelligence report as being about
``so-called attacks on our troops in Afghanistan by Russians''; the
report described bounties paid to Taliban militants by Russian military
intelligence officers, not direct attacks. Mr. Trump also suggested that
the developments could be a ``hoax'' and questioned whether The Times's
sources --- government officials who spoke on the condition of anonymity
--- existed.

Mr. Trump then pivoted to attack former Vice President
\href{https://www.nytimes.com/interactive/2020/us/elections/joe-biden.html}{Joseph
R. Biden Jr.}, who
\href{https://www.nytimes.com/2020/06/27/us/politics/trump-russia-bounties-afghanistan.html}{criticized
the president} on Saturday for failing to punish Russia for offering
bounties to the Taliban, as well as Mr. Biden's son, Hunter, who is the
target of unsubstantiated claims that he helped a Ukrainian energy firm
curry favor with the Obama administration when his father was vice
president.

``Nobody's been tougher on Russia than the Trump Administration,''
\href{https://www.twitter.com/realDonaldTrump/status/1277202162070753280}{Mr.
Trump tweeted}. ``With Corrupt Joe Biden \& Obama, Russia had a field
day, taking over important parts of Ukraine --- Where's Hunter?''

American officials said the Russian plot to pay bounties to Taliban
fighters came into focus over the past several months after intelligence
analysts and Special Operations forces put together key pieces of
evidence.

One official said the seizure of a large amount of American cash at one
Taliban site got ``everybody's attention'' in Afghanistan. It was not
clear when the money was recovered.

Two officials said the information about the bounty hunting was ``well
known'' among the intelligence community in Afghanistan, including the
C.I.A.'s chief of station and other top officials there, like the
military commandos hunting the Taliban. The information was distributed
in intelligence reports and highlighted in some of them.

The assessment was compiled and sent up the chain of command to senior
military and intelligence officials, eventually landing at the highest
levels of the White House. The Security Council meeting in March came at
a delicate time, as the
\href{https://www.nytimes.com/news-event/coronavirus}{coronavirus
pandemic} was becoming a crisis and prompting shutdowns around the
country.

A former American official said the national security adviser, Robert C.
O'Brien, and the president's chief of staff, Mark Meadows, would have
been involved in any decision to brief Mr. Trump on Russia's activities,
as would have the intelligence analyst who briefs the president.. The
director of the C.I.A., Gina Haspel, might have also weighed in, the
former official said.

Ms. McEnany cited those three senior officials in her statement saying
the president had not been briefed.

National security officials have tracked Russia's relationship with the
Taliban for years and determined that Moscow has provided financial and
material support to senior and regional Taliban leaders.

While Russia has at times cooperated with the United States and appeared
interested in Afghan stability, it often seems to work at crosscurrents
with its own national interest if the result is damage to American
national interests, said a former senior Trump White House official, who
spoke on the condition of anonymity to discuss sensitive security
assessments.

Revenge is also a factor in Russia's support for the Taliban, the
official said. Russia has been keen to even the scales after
\href{https://www.nytimes.com/2018/05/24/world/middleeast/american-commandos-russian-mercenaries-syria.html}{a
bloody confrontation in 2018 in Syria}, when a massive U.S.
counterattack killed hundreds of Syrian forces along with Russian
mercenaries nominally supported by the Kremlin.

``They are keeping a score sheet, and they want to punish us for that
incident,'' the official said.

Both
\href{https://twitter.com/RusEmbUSA/status/1276998092261339137}{Russia}
and the Taliban have denied the American intelligence assessment.

Ms. Pelosi said that if the president had not, in fact, been briefed,
then the country should be concerned that his administration was afraid
to share with him information regarding Russia.

Ms. Pelosi said that the episode underscored Mr. Trump's accommodating
stance toward Russia and that with him, ``all roads lead to Putin.''

``This is as bad as it gets, and yet the president will not confront the
Russians on this score, denies being briefed,'' she said. ``Whether he
is or not, his administration knows, and some of our allies who work
with us in Afghanistan have been briefed and accept this report.''

John R. Bolton, Mr. Trump's former national security adviser, said on
``This Week'' that he was not aware of the intelligence assessment, but
he questioned Mr. Trump's response on Twitter.

``What would motivate the president to do that, because it looks bad if
Russians are paying to kill Americans and we're not doing anything about
it?'' Mr. Bolton said. ``The presidential reaction is to say: `It's not
my responsibility. Nobody told me about it.' And therefore to duck any
complaints that he hasn't acted effectively.''

Mr. Bolton said this summed up Mr. Trump's decision-making on national
security issues. ``It's just unconnected to the reality he's dealing
with.''

Reporting was contributed by Julian E. Barnes, Charlie Savage, Thomas
Gibbons-Neff, Michael Schwirtz and Michael D. Shear.

Advertisement

\protect\hyperlink{after-bottom}{Continue reading the main story}

\hypertarget{site-index}{%
\subsection{Site Index}\label{site-index}}

\hypertarget{site-information-navigation}{%
\subsection{Site Information
Navigation}\label{site-information-navigation}}

\begin{itemize}
\tightlist
\item
  \href{https://help.nytimes.com/hc/en-us/articles/115014792127-Copyright-notice}{©~2020~The
  New York Times Company}
\end{itemize}

\begin{itemize}
\tightlist
\item
  \href{https://www.nytco.com/}{NYTCo}
\item
  \href{https://help.nytimes.com/hc/en-us/articles/115015385887-Contact-Us}{Contact
  Us}
\item
  \href{https://www.nytco.com/careers/}{Work with us}
\item
  \href{https://nytmediakit.com/}{Advertise}
\item
  \href{http://www.tbrandstudio.com/}{T Brand Studio}
\item
  \href{https://www.nytimes.com/privacy/cookie-policy\#how-do-i-manage-trackers}{Your
  Ad Choices}
\item
  \href{https://www.nytimes.com/privacy}{Privacy}
\item
  \href{https://help.nytimes.com/hc/en-us/articles/115014893428-Terms-of-service}{Terms
  of Service}
\item
  \href{https://help.nytimes.com/hc/en-us/articles/115014893968-Terms-of-sale}{Terms
  of Sale}
\item
  \href{https://spiderbites.nytimes.com}{Site Map}
\item
  \href{https://help.nytimes.com/hc/en-us}{Help}
\item
  \href{https://www.nytimes.com/subscription?campaignId=37WXW}{Subscriptions}
\end{itemize}
