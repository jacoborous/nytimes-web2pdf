Sections

SEARCH

\protect\hyperlink{site-content}{Skip to
content}\protect\hyperlink{site-index}{Skip to site index}

\href{https://www.nytimes.com/section/politics}{Politics}

\href{https://myaccount.nytimes.com/auth/login?response_type=cookie\&client_id=vi}{}

\href{https://www.nytimes.com/section/todayspaper}{Today's Paper}

\href{/section/politics}{Politics}\textbar{}U.S. Charges Chinese
Military Officers in 2017 Equifax Hacking

\href{https://nyti.ms/2UQvm0j}{https://nyti.ms/2UQvm0j}

\begin{itemize}
\item
\item
\item
\item
\item
\item
\end{itemize}

Advertisement

\protect\hyperlink{after-top}{Continue reading the main story}

Supported by

\protect\hyperlink{after-sponsor}{Continue reading the main story}

\hypertarget{us-charges-chinese-military-officers-in-2017-equifax-hacking}{%
\section{U.S. Charges Chinese Military Officers in 2017 Equifax
Hacking}\label{us-charges-chinese-military-officers-in-2017-equifax-hacking}}

The indictment suggests that the breach was part of a series of thefts
by China to use the data to target American officials.

\includegraphics{https://static01.nyt.com/images/2020/02/10/us/politics/10dc-cyber/10dc-cyber-articleLarge.jpg?quality=75\&auto=webp\&disable=upscale}

\href{https://www.nytimes.com/by/katie-benner}{\includegraphics{https://static01.nyt.com/images/2018/02/16/multimedia/author-katie-benner/author-katie-benner-thumbLarge-v2.png}}

By \href{https://www.nytimes.com/by/katie-benner}{Katie Benner}

\begin{itemize}
\item
  Published Feb. 10, 2020Updated May 7, 2020
\item
  \begin{itemize}
  \item
  \item
  \item
  \item
  \item
  \item
  \end{itemize}
\end{itemize}

\href{https://cn.nytimes.com/usa/20200211/equifax-hack-china/}{阅读简体中文版}\href{https://cn.nytimes.com/usa/20200211/equifax-hack-china/zh-hant/}{閱讀繁體中文版}

WASHINGTON --- Four members of China's military
\href{https://www.justice.gov/opa/press-release/file/1246891/download}{were
charged} on Monday with hacking into Equifax, one of the nation's
largest credit reporting agencies, and stealing trade secrets and the
personal data of about 145 million Americans in 2017.

The charges underscored China's quest to obtain Americans' data and its
willingness to flout a
\href{https://www.nytimes.com/2015/09/26/world/asia/xi-jinping-white-house.html}{2015
agreement with the United States} to refrain from hacking and
\href{https://www.nytimes.com/2020/05/07/world/asia/china-hacking-military-aria.html}{cyberattacks},
all in an effort to expand economic power and influence.

The indictment suggests the hack was part of a series of major data
thefts organized by the People's Liberation Army and Chinese
intelligence agencies. China can use caches of personal information and
combine them with artificial intelligence to better target American
intelligence officers and other officials, Attorney General William P.
Barr said.

``This was a deliberate and sweeping intrusion into the private
information of the American people,'' he said.

\includegraphics{https://static01.nyt.com/images/2020/02/10/us/politics/10dc-barr/10dc-barr-videoSixteenByNine3000.jpg}

The information stolen from Equifax, which is based in Atlanta, could
reveal whether any American officials are under financial stress and
thus susceptible to bribery or blackmail.

Though not as large as other major breaches,
\href{https://www.nytimes.com/2017/09/07/business/equifax-cyberattack.html}{the
attack on Equifax} was far more severe. Hackers stole names, birth dates
and Social Security numbers of nearly half of all Americans --- data
that can be used to access information like medical histories and bank
accounts.

``This kind of attack on American industry is of a piece with other
Chinese illegal acquisitions of sensitive personal data,'' Mr. Barr said
at a news conference announcing the charges, citing China's theft of
records in recent years from the government's
\href{https://www.nytimes.com/2015/08/01/world/asia/us-decides-to-retaliate-against-chinas-hacking.html}{Office
of Personnel Management},
\href{https://www.nytimes.com/2019/01/04/us/politics/marriott-hack-passports.html}{Marriott
International} and
\href{https://www.nytimes.com/2019/05/09/technology/anthem-hack-indicted-breach.html}{the
insurance company Anthem}.

The biggest of those breaches was the theft in 2015 of roughly 22
million security clearance files from the government personnel office,
which keeps track of federal employees and contractors.

It quickly became clear that the data was of significant value to the
Chinese government: American officials with security clearances ---
including some of the most senior members of the government --- had to
reveal foreign contacts, relationships including extramarital affairs,
health histories and information about their children and other family
members.

The breach was so severe that the C.I.A. had to cancel assignments for
undercover officers planning to go to China; though the agency did not
submit its employees' information to the personnel office, those
individuals were often undercover as State Department or other
government officials.

Then it got worse. Hacks into Anthem's database and Starwood hotels ---
later taken over by Marriott --- appeared to be orchestrated by the same
or related Chinese groups. The United States assessed that China was
building a vast database of who worked with whom in national security
jobs, where they traveled and what their health histories were,
according to American officials.

Over time, China can use the data sets to improve its artificial
intelligence capabilities to the point where it can predict which
Americans will be primed for future grooming and recruitment, John C.
Demers, the assistant attorney general for national security at the
Justice Department, said in an interview.

The charges were only the second time that the Justice Department has
indicted Chinese military officers on suspicions of hacking. In 2014,
five Chinese military officers were indicted in data thefts from a labor
union, critical infrastructure and companies including U.S. Steel.

The Justice Department rarely secures indictments against members of
foreign militaries or intelligence services, in part to avoid
retaliation against American troops and spies, but Mr. Barr said it has
made exceptions for state-sponsored actors who hacked into American
networks to steal intellectual property or interfere in United States
elections.

In 2015, President Barack Obama and President Xi Jinping of China agreed
to rein in economically motivated cyberattacks in order to cooperate
with requests to investigate cybercrimes and to avoid targeting critical
infrastructure in each other's countries.

While Justice Department officials do not believe economic espionage was
the primary goal of the Equifax hacking, Mr. Demers said the attack
could be seen as a violation of the spirit of that deal.

``China sees economic interests and intelligence interests as one and
the same,'' he said. ``Commercial benefits are national security
benefits in China.''

The indictment shows that in addition to signing treaties and adopting
certain conventions, the United States must also be willing to publicly
identify and indict state actors in criminal cases, said Megan Brown,
the leader of the cyber and privacy practice at the law firm Wiley Rein.

``This is how we will drive international norms: by indicting people,
not solely by negotiating treaties and adopting conventions,'' she said.

The nine-count indictment accused the Chinese military of hacking into
Equifax's computer networks, maintaining unauthorized access to them and
stealing sensitive, personally identifiable information about Americans.

Months before the attack, the government warned Equifax that its network
contained a vulnerability, but the company did not patch it, according
to government documents. The hacking was ``entirely preventable,'' a
congressional study concluded in 2018.

The defendants --- Wu Zhiyong, Wang Qian, Xu Ke and Liu Lei, all members
of the People's Liberation Army --- exploited that weakness in May 2017
to break into the network, conduct weeks of surveillance and steal
Equifax employee login credentials before filching trade secrets and
data. They masked their activity by using encrypted communications and
routing their internet traffic through 34 servers in nearly 20
countries, including Switzerland and Singapore, according to
prosecutors.

For the most part, they managed to erase their tracks inside of the
Equifax network. But investigators eventually traced their activity to
two China-based servers that connected directly to Equifax.

Investigators identified the four indicted officers by reviewing
forensic data, analyzing the malware used in the attack and establishing
a digital footprint that linked them to the intrusion, David Bowdich,
the deputy director of the F.B.I., said at the news conference.

In the months after Equifax was hacked, security researchers concluded
that criminals, not state actors, had siphoned information over a few
months after gaining access to the network. That alone was enough to
force the resignation of the company's chief executive.

But that explanation appeared increasingly suspect over time because the
Equifax data --- like the information gleaned from the Office of
Personnel Management --- did not appear broadly for sale on the
so-called dark web, where illicitly obtained information is often sold
for use in cybercrime.

Law enforcement officials have not yet found evidence that the Chinese
government has used the data from the Equifax hacking, Mr. Bowdich said.

The company reiterated on Monday the difficulty of warding off
state-sponsored attacks. Companies often fall back on that explanation;
Senator Mark Warner of Virginia, the top Democrat on the Senate
Intelligence Committee, pushed back after the indictment was made
public.

``A company in the business of collecting and retaining massive amounts
of Americans' sensitive personal information must act with the utmost
care --- and face any consequences that arise from that failure,'' he
said in a statement.

The hackers' encryption of their operations inside Equifax's networks is
a common technique and has raised new questions about why such sensitive
data in American databases is not legally required to be encrypted,
experts noted. Many companies have resisted such regulation, in part
because encrypted data can be harder for them to search.

China has ``pioneered an expansive approach to stealing innovation,''
Christopher A. Wray, the director of the F.B.I., said last week at a
conference on the threats posed by China.

He said China was racing to obtain information about sectors as diverse
as agriculture and medicine to advance its economy, using a mix of legal
means like company acquisitions and illicit acts like spying and
cyberattacks.

``They've shown that they're willing to steal their way up the economic
ladder at our expense,'' Mr. Wray said.

The outcry from consumers and lawmakers after the Equifax breach and the
company's clumsy response was strong: Its executives were chastised, and
Equifax eventually
\href{https://www.nytimes.com/2019/07/22/business/equifax-settlement.html}{settled
with regulators} for up to \$700 million.

But of the 147 million consumers affected, only a little more than 10
percent had
\href{https://www.nytimes.com/2020/01/22/business/equifax-breach-settlement.html}{filed
for some type of compensation} as of Dec. 1.

Of those, more than 4.5 million filed claims for a cash payment of up to
\$125, one of the settlement options. But the company had set aside only
\$31 million for that option, which amounts to less than \$7 a person.

While the thefts present a national security risk, Americans have
``almost become as a country immune to these breaches,'' Mr. Bowdich
said.

``You hear about it in the news and you think, `Well there goes my
credit card number, my Social Security number, my bank account
information,' and you sign up for another year of free credit card
monitoring information,'' he said. ``We cannot think like that in this
country.''

David E. Sanger contributed reporting from Washington, Nicole Perlroth
from San Francisco and Tara Siegel Bernard from New York.

Advertisement

\protect\hyperlink{after-bottom}{Continue reading the main story}

\hypertarget{site-index}{%
\subsection{Site Index}\label{site-index}}

\hypertarget{site-information-navigation}{%
\subsection{Site Information
Navigation}\label{site-information-navigation}}

\begin{itemize}
\tightlist
\item
  \href{https://help.nytimes.com/hc/en-us/articles/115014792127-Copyright-notice}{©~2020~The
  New York Times Company}
\end{itemize}

\begin{itemize}
\tightlist
\item
  \href{https://www.nytco.com/}{NYTCo}
\item
  \href{https://help.nytimes.com/hc/en-us/articles/115015385887-Contact-Us}{Contact
  Us}
\item
  \href{https://www.nytco.com/careers/}{Work with us}
\item
  \href{https://nytmediakit.com/}{Advertise}
\item
  \href{http://www.tbrandstudio.com/}{T Brand Studio}
\item
  \href{https://www.nytimes.com/privacy/cookie-policy\#how-do-i-manage-trackers}{Your
  Ad Choices}
\item
  \href{https://www.nytimes.com/privacy}{Privacy}
\item
  \href{https://help.nytimes.com/hc/en-us/articles/115014893428-Terms-of-service}{Terms
  of Service}
\item
  \href{https://help.nytimes.com/hc/en-us/articles/115014893968-Terms-of-sale}{Terms
  of Sale}
\item
  \href{https://spiderbites.nytimes.com}{Site Map}
\item
  \href{https://help.nytimes.com/hc/en-us}{Help}
\item
  \href{https://www.nytimes.com/subscription?campaignId=37WXW}{Subscriptions}
\end{itemize}
