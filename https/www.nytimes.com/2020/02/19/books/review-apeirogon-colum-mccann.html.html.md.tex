Sections

SEARCH

\protect\hyperlink{site-content}{Skip to
content}\protect\hyperlink{site-index}{Skip to site index}

\href{https://www.nytimes.com/section/books}{Books}

\href{https://myaccount.nytimes.com/auth/login?response_type=cookie\&client_id=vi}{}

\href{https://www.nytimes.com/section/todayspaper}{Today's Paper}

\href{/section/books}{Books}\textbar{}Colum McCann's New Novel Makes a
Good-Intentioned Collage Out of Real Tragedy

\url{https://nyti.ms/38C1s3I}

\begin{itemize}
\item
\item
\item
\item
\item
\item
\end{itemize}

Advertisement

\protect\hyperlink{after-top}{Continue reading the main story}

Supported by

\protect\hyperlink{after-sponsor}{Continue reading the main story}

\href{/column/books-of-the-times}{Books of The Times}

\hypertarget{colum-mccanns-new-novel-makes-a-good-intentioned-collage-out-of-real-tragedy}{%
\section{Colum McCann's New Novel Makes a Good-Intentioned Collage Out
of Real
Tragedy}\label{colum-mccanns-new-novel-makes-a-good-intentioned-collage-out-of-real-tragedy}}

By \href{https://www.nytimes.com/by/dwight-garner}{Dwight Garner}

\begin{itemize}
\item
  Feb. 19, 2020
\item
  \begin{itemize}
  \item
  \item
  \item
  \item
  \item
  \item
  \end{itemize}
\end{itemize}

\includegraphics{https://static01.nyt.com/images/2020/02/20/books/19bookmccann1/19bookmccann1-articleLarge.jpg?quality=75\&auto=webp\&disable=upscale}

Buy Book ▾

\begin{itemize}
\tightlist
\item
  \href{https://www.amazon.com/gp/search?index=books\&tag=NYTBSREV-20\&field-keywords=Apeirogon+Colum+McCann}{Amazon}
\item
  \href{https://du-gae-books-dot-nyt-du-prd.appspot.com/buy?title=Apeirogon\&author=Colum+McCann}{Apple
  Books}
\item
  \href{https://www.anrdoezrs.net/click-7990613-11819508?url=https\%3A\%2F\%2Fwww.barnesandnoble.com\%2Fw\%2F\%3Fean\%3D9781400069606}{Barnes
  and Noble}
\item
  \href{https://www.anrdoezrs.net/click-7990613-35140?url=https\%3A\%2F\%2Fwww.booksamillion.com\%2Fp\%2FApeirogon\%2FColum\%2BMcCann\%2F9781400069606}{Books-A-Million}
\item
  \href{https://bookshop.org/a/3546/9781400069606}{Bookshop}
\item
  \href{https://www.indiebound.org/book/9781400069606?aff=NYT}{Indiebound}
\end{itemize}

When you purchase an independently reviewed book through our site, we
earn an affiliate commission.

Colum McCann's new novel, ``Apeirogon,'' is based on an uplifting true
story. It's about two fathers --- Rami Elhanan, an Israeli, and Bassam
Aramin, a Palestinian --- who each lost a young daughter to senseless
violence. They have become friends and work together, through an
organization called Combatants for Peace, to bring the opposing sides
together.

McCann takes their story and drops it to the ground, where it shatters.
To read ``Apeirogon'' is to watch him pick up the shards. As befits a
writer who ruminates about the nature of storytelling, there are 1,001
of these shards, each numbered, in a homage to ``One Thousand and One
Nights,'' the collection of Middle Eastern folk tales.

This is an early warning sign. It is possible to admire ``One Thousand
and One Nights'' while having learned through hard experience that a
writer who derives too much inspiration from it is generally one to
avoid, unless what's desired is a self-inflicted intellectual
glitter-bombing. The Time magazine film critic Stephanie Zacharek,
writing on Twitter, recently called ``storytelling'' a ``jazz-hands
word.'' ``Apeirogon'' is a jazz-hands novel.

In many of McCann's 1,001 shards, we follow Rami and Bassam in something
like real time. They attend meetings, give lectures, worry about
crossing border checkpoints. In others, we flash back to earlier points
in their lives. From multiple angles, we witness the events that led to
their daughters being killed.

Rami and Bassam no longer want to silo their suffering. They want for
their grandchildren what the narrator of the Israeli writer David
Grossman's exquisite novel
\href{https://www.nytimes.com/1989/04/04/books/books-of-the-times-wrestling-with-the-beast-of-the-holocaust.html}{``See
Under: Love''} wanted, ``to live in this world from birth to death and
know nothing of war.'' You sense you would like Rami and Bassam if you
got to know them.

But we are not allowed to settle into the texture and nuance of their
experience. We're evicted from the narrative on almost every page so
that McCann can tweezer in arty and only vaguely relevant facts about
birds, or about John Cage's music, or about the Dead Sea Scrolls or the
derivation of the word ``dextrose.''

He's going for a collage effect. He'd like to chime with Shirley
MacLaine in ``The Apartment,'' who looks into a broken mirror and says
to Jack Lemmon: ``I like it that way. Makes me look the way I feel.''

McCann's shards are set apart from one another on the page by a
thumb's-width of white space. They are tiles without grout. This
trending method of organizing a novel (for example,
\href{https://www.nytimes.com/2020/01/31/books/review-weather-jenny-offill.html}{Jenny
Offill}) is not yet insufferable but has a trap door: It exaggerates a
writer's weaknesses.

Image

Colum McCann, whose new novel is ``Apeirogon.''Credit...Elizabeth Eagle

Offill's weakness is that, despite her comic tone, her wit can fizzle.
McCann's is that, even in his best novels, such as the National Book
Award-winning
\href{https://www.nytimes.com/2009/08/02/books/review/Mahler-t.html}{``Let
the Great World Spin''} (2009), his work can be humorless and
self-important.

``Apeirogon'' --- the title refers to a shape with a limitless number of
sides --- is so solemn, so certain of its own goodness and moral value,
that it tips almost instantly over into camp, into corn. It's as if the
author were gunning for the Paulo Coelho Chair in Maudlin Schlock.

In an author's note at the front of this novel, McCann writes: ``We live
our lives, suggested Rilke, in widening circles that reach out across
the entire expanse.'' McCann has a gift for quoting others at their most
flatulent: ``The only interesting thing is to live, said Mitterand'';
``Hertzl wrote: If you divide death by life, you will find a circle.''

For added effect, McCann repeats lines he likes, such as the Mitterand
quote, again and again. Often these phrases are his own, such as ``the
rim of a tightening lung.''

McCann's numbered shards --- they ascend from 1 to 500, then turn and
begin to descend --- offer structure in place of content. ``Apeirogon''
is not a meal but a table littered with ingredients: a paw of garlic, a
frozen lamb shank, two potatoes, a big knob of celeriac, three peas.

When you insist on a lot of white space between paragraphs and sometimes
between single sentences, and if your work is humid, the effect can
unintentionally verge on the amusing. Each sentence has an
apricot-colored scarf tied around its neck. And it's as if the reader
has been given 10 seconds and a bong hit between each one; time to
squint and nod and say, ``So true.''

A sampling of those sentences: ``He had learned that the cure for fate
was patience''; ``In a letter to Rami, Bassam wrote that one of the
principal qualities of pain is that it demands to be defeated first,
then understood''; ``The geese are said to bring news of the dead to the
heavens.''

As with Bob Geldof speaking of Ethiopian famine at the time of Live Aid,
McCann's ardor is unmistakable. He is a well-meaning man. But his
analysis of the predicaments that face the Middle East is not raw or
original or sophisticated. His message is optimistic --- we need to talk
more, and understand each other's humanity --- and banal. ``Apeirogon''
is like a political memoir that bangs on about the importance of
bipartisanship as if the senator had, just this morning, arrived at the
idea.

Great writing, Walt Whitman wrote, is composed of words that are
``whirled like chain-shot rocks.'' Enough rocks have been whirled in the
Middle East. And this novel is only tossing around pillows.

Advertisement

\protect\hyperlink{after-bottom}{Continue reading the main story}

\hypertarget{site-index}{%
\subsection{Site Index}\label{site-index}}

\hypertarget{site-information-navigation}{%
\subsection{Site Information
Navigation}\label{site-information-navigation}}

\begin{itemize}
\tightlist
\item
  \href{https://help.nytimes.com/hc/en-us/articles/115014792127-Copyright-notice}{©~2020~The
  New York Times Company}
\end{itemize}

\begin{itemize}
\tightlist
\item
  \href{https://www.nytco.com/}{NYTCo}
\item
  \href{https://help.nytimes.com/hc/en-us/articles/115015385887-Contact-Us}{Contact
  Us}
\item
  \href{https://www.nytco.com/careers/}{Work with us}
\item
  \href{https://nytmediakit.com/}{Advertise}
\item
  \href{http://www.tbrandstudio.com/}{T Brand Studio}
\item
  \href{https://www.nytimes.com/privacy/cookie-policy\#how-do-i-manage-trackers}{Your
  Ad Choices}
\item
  \href{https://www.nytimes.com/privacy}{Privacy}
\item
  \href{https://help.nytimes.com/hc/en-us/articles/115014893428-Terms-of-service}{Terms
  of Service}
\item
  \href{https://help.nytimes.com/hc/en-us/articles/115014893968-Terms-of-sale}{Terms
  of Sale}
\item
  \href{https://spiderbites.nytimes.com}{Site Map}
\item
  \href{https://help.nytimes.com/hc/en-us}{Help}
\item
  \href{https://www.nytimes.com/subscription?campaignId=37WXW}{Subscriptions}
\end{itemize}
