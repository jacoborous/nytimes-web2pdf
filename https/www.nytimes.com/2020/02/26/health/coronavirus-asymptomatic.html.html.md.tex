Sections

SEARCH

\protect\hyperlink{site-content}{Skip to
content}\protect\hyperlink{site-index}{Skip to site index}

\href{https://www.nytimes.com/section/health}{Health}

\href{https://myaccount.nytimes.com/auth/login?response_type=cookie\&client_id=vi}{}

\href{https://www.nytimes.com/section/todayspaper}{Today's Paper}

\href{/section/health}{Health}\textbar{}They Were Infected With the
Coronavirus. They Never Showed Signs.

\url{https://nyti.ms/32w2L27}

\begin{itemize}
\item
\item
\item
\item
\item
\item
\end{itemize}

\href{https://www.nytimes.com/news-event/coronavirus?action=click\&pgtype=Article\&state=default\&region=TOP_BANNER\&context=storylines_menu}{The
Coronavirus Outbreak}

\begin{itemize}
\tightlist
\item
  live\href{https://www.nytimes.com/2020/08/02/world/coronavirus-updates.html?action=click\&pgtype=Article\&state=default\&region=TOP_BANNER\&context=storylines_menu}{Latest
  Updates}
\item
  \href{https://www.nytimes.com/interactive/2020/us/coronavirus-us-cases.html?action=click\&pgtype=Article\&state=default\&region=TOP_BANNER\&context=storylines_menu}{Maps
  and Cases}
\item
  \href{https://www.nytimes.com/interactive/2020/science/coronavirus-vaccine-tracker.html?action=click\&pgtype=Article\&state=default\&region=TOP_BANNER\&context=storylines_menu}{Vaccine
  Tracker}
\item
  \href{https://www.nytimes.com/interactive/2020/07/29/us/schools-reopening-coronavirus.html?action=click\&pgtype=Article\&state=default\&region=TOP_BANNER\&context=storylines_menu}{What
  School May Look Like}
\item
  \href{https://www.nytimes.com/live/2020/07/31/business/stock-market-today-coronavirus?action=click\&pgtype=Article\&state=default\&region=TOP_BANNER\&context=storylines_menu}{Economy}
\end{itemize}

Advertisement

\protect\hyperlink{after-top}{Continue reading the main story}

Supported by

\protect\hyperlink{after-sponsor}{Continue reading the main story}

\hypertarget{they-were-infected-with-the-coronavirus-they-never-showed-signs}{%
\section{They Were Infected With the Coronavirus. They Never Showed
Signs.}\label{they-were-infected-with-the-coronavirus-they-never-showed-signs}}

Even asymptomatic people who are infected may be able to spread the
virus. But people without symptoms are rarely tested.

\includegraphics{https://static01.nyt.com/images/2020/02/26/science/26VIRUS-ASYMPTOMATIC1/26VIRUS-ASYMPTOMATIC1-articleLarge.jpg?quality=75\&auto=webp\&disable=upscale}

\href{https://www.nytimes.com/by/roni-caryn-rabin}{\includegraphics{https://static01.nyt.com/images/2018/02/20/multimedia/author-roni-caryn-rabin/author-roni-caryn-rabin-thumbLarge-v3.png}}

By \href{https://www.nytimes.com/by/roni-caryn-rabin}{Roni Caryn Rabin}

\begin{itemize}
\item
  Published Feb. 26, 2020Updated March 6, 2020
\item
  \begin{itemize}
  \item
  \item
  \item
  \item
  \item
  \item
  \end{itemize}
\end{itemize}

\href{https://cn.nytimes.com/health/20200227/coronavirus-asymptomatic/}{阅读简体中文版}\href{https://cn.nytimes.com/health/20200227/coronavirus-asymptomatic/zh-hant/}{閱讀繁體中文版}

In Anyang, China, five members of a family came down with the
coronavirus after hosting a guest from Wuhan in early January.
\href{https://jamanetwork.com/journals/jama/fullarticle/2762028}{But the
visitor, a 20-year-old woman, never}got sick herself.

Some individuals who are infected with the coronavirus can spread it
even though they have no symptoms, studies have shown.

Asymptomatic carriers are a well-known phenomenon. But the coronavirus
is a new pathogen, and these cases may complicate scientific efforts to
detect cases and to curb transmission.

``I don't think there's any question that someone who is without
symptoms and carrying the virus can transmit the virus to somebody
else,'' said Dr. Anthony Fauci, director of the National Institute of
Allergy and Infectious Diseases.

``The question is, how prevalent is that phenomenon? Is that becoming an
important driver of the outbreaks, or is it an unusual occurrence?''

When asymptomatic carriers are important factors in an outbreak, he
said, ``you are going to put greater emphasis and burden on testing
people.''

At the moment, the
\href{https://www.cdc.gov/coronavirus/2019-ncov/hcp/clinical-criteria.html}{Centers
for Disease Control and Prevention allows for testing only symptomatic
people who traveled to China recently or those who have had contact}with
someone who tested positive for coronavirus. (Officials have said the
criteria may be re-evaluated.)

\hypertarget{latest-updates-global-coronavirus-outbreak}{%
\section{\texorpdfstring{\href{https://www.nytimes.com/2020/08/01/world/coronavirus-covid-19.html?action=click\&pgtype=Article\&state=default\&region=MAIN_CONTENT_1\&context=storylines_live_updates}{Latest
Updates: Global Coronavirus
Outbreak}}{Latest Updates: Global Coronavirus Outbreak}}\label{latest-updates-global-coronavirus-outbreak}}

Updated 2020-08-02T17:52:35.962Z

\begin{itemize}
\tightlist
\item
  \href{https://www.nytimes.com/2020/08/01/world/coronavirus-covid-19.html?action=click\&pgtype=Article\&state=default\&region=MAIN_CONTENT_1\&context=storylines_live_updates\#link-34047410}{The
  U.S. reels as July cases more than double the total of any other
  month.}
\item
  \href{https://www.nytimes.com/2020/08/01/world/coronavirus-covid-19.html?action=click\&pgtype=Article\&state=default\&region=MAIN_CONTENT_1\&context=storylines_live_updates\#link-780ec966}{Top
  U.S. officials work to break an impasse over the federal jobless
  benefit.}
\item
  \href{https://www.nytimes.com/2020/08/01/world/coronavirus-covid-19.html?action=click\&pgtype=Article\&state=default\&region=MAIN_CONTENT_1\&context=storylines_live_updates\#link-2bc8948}{Its
  outbreak untamed, Melbourne goes into even greater lockdown.}
\end{itemize}

\href{https://www.nytimes.com/2020/08/01/world/coronavirus-covid-19.html?action=click\&pgtype=Article\&state=default\&region=MAIN_CONTENT_1\&context=storylines_live_updates}{See
more updates}

More live coverage:
\href{https://www.nytimes.com/live/2020/07/31/business/stock-market-today-coronavirus?action=click\&pgtype=Article\&state=default\&region=MAIN_CONTENT_1\&context=storylines_live_updates}{Markets}

``We could be missing a great number of cases that don't fit into those
criteria,'' said Dr. Michael Osterholm, director of the Center for
Infectious Disease Research and Policy at the University of Minnesota.

``I suspect there are a number of additional cases in this country that
are transmitting this virus, just like we're seeing in other countries.
Absence of evidence is not evidence of absence.''

People who are infected but asymptomatic can spread disease efficiently.
They are hardy and mobile. They have no reason to avoid crowds or
kissing. They don't know they are sick, and no one else does.

These individuals are also hard to detect, suggesting that the current
policies to try to contain the spread of the virus may not be adequate.
Simply screening international travelers with symptoms of illness ---
and explicitly precluding tests of patients without a known link to
China --- may mean new cases are missed.

In February, Germany flew 126 people home from the Wuhan area. Ten
passengers were segregated from the others, because they didn't feel
well or thought they had been exposed to the coronavirus. But everyone
was offered testing.

The 10 isolated patients tested negative, but two people --- who felt
fine --- surprised scientists by testing positive. They were
hospitalized, monitored and tested repeatedly.

\href{https://www.nejm.org/doi/full/10.1056/NEJMc2001899}{While one
developed a mild rash and slightly sore throat, neither}became ill.

\includegraphics{https://static01.nyt.com/images/2020/02/26/science/26VIRUS-ASYMPTOMATIC2/26VIRUS-ASYMPTOMATIC2-articleLarge.jpg?quality=75\&auto=webp\&disable=upscale}

\href{https://www.cdc.gov/coronavirus/2019-ncov/cases-in-us.html}{There
have been 59 confirmed coronavirus cases so far} in the United States,
but little testing has occurred for a country of this size. The C.D.C.
has run only 445 tests, not counting tests on people who were
repatriated.

Most of the confirmed cases are passengers repatriated from the Diamond
Princess cruise ship. The C.D.C. reported on Wednesday that two more
passengers under quarantine have become ill.

Federal health officials warned on Tuesday that hospitals, schools and
businesses needed to start preparing for outbreaks in the United States.
Containment strategies may have to expand to include steps like closing
schools, ordering people to work from home, and restricting public
gatherings.

The secretary of health and human services, Alex M. Azar II, said he was
alarmed by the infections occurring in some parts of the world that have
no clear link to confirmed cases.

Until now, the vast majority of infections and deaths have been in
China, where the coronavirus originated in Wuhan before spreading to
about 40 other nations.

\href{https://www.nytimes.com/news-event/coronavirus?action=click\&pgtype=Article\&state=default\&region=MAIN_CONTENT_3\&context=storylines_faq}{}

\hypertarget{the-coronavirus-outbreak-}{%
\subsubsection{The Coronavirus Outbreak
›}\label{the-coronavirus-outbreak-}}

\hypertarget{frequently-asked-questions}{%
\paragraph{Frequently Asked
Questions}\label{frequently-asked-questions}}

Updated July 27, 2020

\begin{itemize}
\item ~
  \hypertarget{should-i-refinance-my-mortgage}{%
  \paragraph{Should I refinance my
  mortgage?}\label{should-i-refinance-my-mortgage}}

  \begin{itemize}
  \tightlist
  \item
    \href{https://www.nytimes.com/article/coronavirus-money-unemployment.html?action=click\&pgtype=Article\&state=default\&region=MAIN_CONTENT_3\&context=storylines_faq}{It
    could be a good idea,} because mortgage rates have
    \href{https://www.nytimes.com/2020/07/16/business/mortgage-rates-below-3-percent.html?action=click\&pgtype=Article\&state=default\&region=MAIN_CONTENT_3\&context=storylines_faq}{never
    been lower.} Refinancing requests have pushed mortgage applications
    to some of the highest levels since 2008, so be prepared to get in
    line. But defaults are also up, so if you're thinking about buying a
    home, be aware that some lenders have tightened their standards.
  \end{itemize}
\item ~
  \hypertarget{what-is-school-going-to-look-like-in-september}{%
  \paragraph{What is school going to look like in
  September?}\label{what-is-school-going-to-look-like-in-september}}

  \begin{itemize}
  \tightlist
  \item
    It is unlikely that many schools will return to a normal schedule
    this fall, requiring the grind of
    \href{https://www.nytimes.com/2020/06/05/us/coronavirus-education-lost-learning.html?action=click\&pgtype=Article\&state=default\&region=MAIN_CONTENT_3\&context=storylines_faq}{online
    learning},
    \href{https://www.nytimes.com/2020/05/29/us/coronavirus-child-care-centers.html?action=click\&pgtype=Article\&state=default\&region=MAIN_CONTENT_3\&context=storylines_faq}{makeshift
    child care} and
    \href{https://www.nytimes.com/2020/06/03/business/economy/coronavirus-working-women.html?action=click\&pgtype=Article\&state=default\&region=MAIN_CONTENT_3\&context=storylines_faq}{stunted
    workdays} to continue. California's two largest public school
    districts --- Los Angeles and San Diego --- said on July 13, that
    \href{https://www.nytimes.com/2020/07/13/us/lausd-san-diego-school-reopening.html?action=click\&pgtype=Article\&state=default\&region=MAIN_CONTENT_3\&context=storylines_faq}{instruction
    will be remote-only in the fall}, citing concerns that surging
    coronavirus infections in their areas pose too dire a risk for
    students and teachers. Together, the two districts enroll some
    825,000 students. They are the largest in the country so far to
    abandon plans for even a partial physical return to classrooms when
    they reopen in August. For other districts, the solution won't be an
    all-or-nothing approach.
    \href{https://bioethics.jhu.edu/research-and-outreach/projects/eschool-initiative/school-policy-tracker/}{Many
    systems}, including the nation's largest, New York City, are
    devising
    \href{https://www.nytimes.com/2020/06/26/us/coronavirus-schools-reopen-fall.html?action=click\&pgtype=Article\&state=default\&region=MAIN_CONTENT_3\&context=storylines_faq}{hybrid
    plans} that involve spending some days in classrooms and other days
    online. There's no national policy on this yet, so check with your
    municipal school system regularly to see what is happening in your
    community.
  \end{itemize}
\item ~
  \hypertarget{is-the-coronavirus-airborne}{%
  \paragraph{Is the coronavirus
  airborne?}\label{is-the-coronavirus-airborne}}

  \begin{itemize}
  \tightlist
  \item
    The coronavirus
    \href{https://www.nytimes.com/2020/07/04/health/239-experts-with-one-big-claim-the-coronavirus-is-airborne.html?action=click\&pgtype=Article\&state=default\&region=MAIN_CONTENT_3\&context=storylines_faq}{can
    stay aloft for hours in tiny droplets in stagnant air}, infecting
    people as they inhale, mounting scientific evidence suggests. This
    risk is highest in crowded indoor spaces with poor ventilation, and
    may help explain super-spreading events reported in meatpacking
    plants, churches and restaurants.
    \href{https://www.nytimes.com/2020/07/06/health/coronavirus-airborne-aerosols.html?action=click\&pgtype=Article\&state=default\&region=MAIN_CONTENT_3\&context=storylines_faq}{It's
    unclear how often the virus is spread} via these tiny droplets, or
    aerosols, compared with larger droplets that are expelled when a
    sick person coughs or sneezes, or transmitted through contact with
    contaminated surfaces, said Linsey Marr, an aerosol expert at
    Virginia Tech. Aerosols are released even when a person without
    symptoms exhales, talks or sings, according to Dr. Marr and more
    than 200 other experts, who
    \href{https://academic.oup.com/cid/article/doi/10.1093/cid/ciaa939/5867798}{have
    outlined the evidence in an open letter to the World Health
    Organization}.
  \end{itemize}
\item ~
  \hypertarget{what-are-the-symptoms-of-coronavirus}{%
  \paragraph{What are the symptoms of
  coronavirus?}\label{what-are-the-symptoms-of-coronavirus}}

  \begin{itemize}
  \tightlist
  \item
    Common symptoms
    \href{https://www.nytimes.com/article/symptoms-coronavirus.html?action=click\&pgtype=Article\&state=default\&region=MAIN_CONTENT_3\&context=storylines_faq}{include
    fever, a dry cough, fatigue and difficulty breathing or shortness of
    breath.} Some of these symptoms overlap with those of the flu,
    making detection difficult, but runny noses and stuffy sinuses are
    less common.
    \href{https://www.nytimes.com/2020/04/27/health/coronavirus-symptoms-cdc.html?action=click\&pgtype=Article\&state=default\&region=MAIN_CONTENT_3\&context=storylines_faq}{The
    C.D.C. has also} added chills, muscle pain, sore throat, headache
    and a new loss of the sense of taste or smell as symptoms to look
    out for. Most people fall ill five to seven days after exposure, but
    symptoms may appear in as few as two days or as many as 14 days.
  \end{itemize}
\item ~
  \hypertarget{does-asymptomatic-transmission-of-covid-19-happen}{%
  \paragraph{Does asymptomatic transmission of Covid-19
  happen?}\label{does-asymptomatic-transmission-of-covid-19-happen}}

  \begin{itemize}
  \tightlist
  \item
    So far, the evidence seems to show it does. A widely cited
    \href{https://www.nature.com/articles/s41591-020-0869-5}{paper}
    published in April suggests that people are most infectious about
    two days before the onset of coronavirus symptoms and estimated that
    44 percent of new infections were a result of transmission from
    people who were not yet showing symptoms. Recently, a top expert at
    the World Health Organization stated that transmission of the
    coronavirus by people who did not have symptoms was ``very rare,''
    \href{https://www.nytimes.com/2020/06/09/world/coronavirus-updates.html?action=click\&pgtype=Article\&state=default\&region=MAIN_CONTENT_3\&context=storylines_faq\#link-1f302e21}{but
    she later walked back that statement.}
  \end{itemize}
\end{itemize}

So far, at least 81,109 people have been infected and at least 2,718
have died.

But other countries may not have confirmed cases because they haven't
tested very many people or don't have the resources to run tests.

Some public health experts fear stealth transmissions may already be
occurring in communities in the United States. But if sick individuals
have no direct link to China, they will not be eligible for testing, so
they will not be detected. That may help spread the disease.

``To our knowledge there is no sustained transmission in this country at
this point unless it is under the radar,'' Dr. Fauci said.

In Italy, health officials in some regions have taken a different
approach.

After 10 deaths attributed to the new coronavirus, health officials
started aggressive and widespread testing in some regions. They turned
up hundreds of other infections, including many in people who did not
display any symptoms.

Quarantines have been imposed on at least 10 towns, and the movement of
tens of thousands of people has been limited. There have been no deaths
attributed to the coronavirus in the United States.

\textbf{\emph{{[}}\href{http://on.fb.me/1paTQ1h}{\emph{Like the Science
Times page on Facebook.}}} ****** \emph{\textbar{} Sign up for the}
\textbf{\href{http://nyti.ms/1MbHaRU}{\emph{Science Times
newsletter.}}\emph{{]}}}

Earlier reports about asymptomatic transmission --- including a
published report
about\href{https://www.nejm.org/doi/full/10.1056/NEJMc2001468}{a Chinese
woman who visited Germany}for a few days in January, infecting several
colleagues there and not realizing she was ill until she returned home
--- have been criticized.

A follow-up report said the woman had vague symptoms, like fatigue,
though not the kind of symptoms typically associated with the
coronavirus.

If it is true that asymptomatic or minimally symptomatic people can
transmit the disease frequently and efficiently, testing may need to be
broadened, experts said.

``This implies we may need many more tests that can be used out in the
field, at the point of care,'' said Dr. Judith N. Wasserheit,
co-director of the University of Washington MetaCenter for Pandemic
Preparedness and Global Health Security. ``We're still learning about
the biology of this virus and how it causes disease.''

Dr. Sandra Ciesek, of the Institute of Medical Virology at University
Hospital Frankfurt, who was one of the authors of a letter in The New
England Journal of Medicine that described the German patients who did
not become ill, said the problem was that ``normally, you don't screen
asymptomatic healthy people for the virus because it's too expensive.''

``This shows we might have more infected people already all over the
world than we expect,'' she said.

Advertisement

\protect\hyperlink{after-bottom}{Continue reading the main story}

\hypertarget{site-index}{%
\subsection{Site Index}\label{site-index}}

\hypertarget{site-information-navigation}{%
\subsection{Site Information
Navigation}\label{site-information-navigation}}

\begin{itemize}
\tightlist
\item
  \href{https://help.nytimes.com/hc/en-us/articles/115014792127-Copyright-notice}{©~2020~The
  New York Times Company}
\end{itemize}

\begin{itemize}
\tightlist
\item
  \href{https://www.nytco.com/}{NYTCo}
\item
  \href{https://help.nytimes.com/hc/en-us/articles/115015385887-Contact-Us}{Contact
  Us}
\item
  \href{https://www.nytco.com/careers/}{Work with us}
\item
  \href{https://nytmediakit.com/}{Advertise}
\item
  \href{http://www.tbrandstudio.com/}{T Brand Studio}
\item
  \href{https://www.nytimes.com/privacy/cookie-policy\#how-do-i-manage-trackers}{Your
  Ad Choices}
\item
  \href{https://www.nytimes.com/privacy}{Privacy}
\item
  \href{https://help.nytimes.com/hc/en-us/articles/115014893428-Terms-of-service}{Terms
  of Service}
\item
  \href{https://help.nytimes.com/hc/en-us/articles/115014893968-Terms-of-sale}{Terms
  of Sale}
\item
  \href{https://spiderbites.nytimes.com}{Site Map}
\item
  \href{https://help.nytimes.com/hc/en-us}{Help}
\item
  \href{https://www.nytimes.com/subscription?campaignId=37WXW}{Subscriptions}
\end{itemize}
