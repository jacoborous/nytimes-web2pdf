Sections

SEARCH

\protect\hyperlink{site-content}{Skip to
content}\protect\hyperlink{site-index}{Skip to site index}

\href{https://www.nytimes.com/section/world/africa}{Africa}

\href{https://myaccount.nytimes.com/auth/login?response_type=cookie\&client_id=vi}{}

\href{https://www.nytimes.com/section/todayspaper}{Today's Paper}

\href{/section/world/africa}{Africa}\textbar{}Africa, Intertwined With
China, Fears Coronavirus Outbreak

\url{https://nyti.ms/2vOzB1F}

\begin{itemize}
\item
\item
\item
\item
\item
\end{itemize}

\href{https://www.nytimes.com/news-event/coronavirus?action=click\&pgtype=Article\&state=default\&region=TOP_BANNER\&context=storylines_menu}{The
Coronavirus Outbreak}

\begin{itemize}
\tightlist
\item
  live\href{https://www.nytimes.com/2020/08/01/world/coronavirus-covid-19.html?action=click\&pgtype=Article\&state=default\&region=TOP_BANNER\&context=storylines_menu}{Latest
  Updates}
\item
  \href{https://www.nytimes.com/interactive/2020/us/coronavirus-us-cases.html?action=click\&pgtype=Article\&state=default\&region=TOP_BANNER\&context=storylines_menu}{Maps
  and Cases}
\item
  \href{https://www.nytimes.com/interactive/2020/science/coronavirus-vaccine-tracker.html?action=click\&pgtype=Article\&state=default\&region=TOP_BANNER\&context=storylines_menu}{Vaccine
  Tracker}
\item
  \href{https://www.nytimes.com/interactive/2020/07/29/us/schools-reopening-coronavirus.html?action=click\&pgtype=Article\&state=default\&region=TOP_BANNER\&context=storylines_menu}{What
  School May Look Like}
\item
  \href{https://www.nytimes.com/live/2020/07/31/business/stock-market-today-coronavirus?action=click\&pgtype=Article\&state=default\&region=TOP_BANNER\&context=storylines_menu}{Economy}
\end{itemize}

Advertisement

\protect\hyperlink{after-top}{Continue reading the main story}

Supported by

\protect\hyperlink{after-sponsor}{Continue reading the main story}

\hypertarget{africa-intertwined-with-china-fears-coronavirus-outbreak}{%
\section{Africa, Intertwined With China, Fears Coronavirus
Outbreak}\label{africa-intertwined-with-china-fears-coronavirus-outbreak}}

There are no confirmed cases of coronavirus in Africa yet, but with
steady traffic to and from China, experts worry that the epidemic could
overrun already-strained health systems.

\includegraphics{https://static01.nyt.com/images/2020/02/05/world/Virus-Africa01/merlin_168179964_ac4aceb3-6e7d-40ae-a61a-31752ac15dd4-articleLarge.jpg?quality=75\&auto=webp\&disable=upscale}

By Simon Marks and
\href{https://www.nytimes.com/by/abdi-latif-dahir}{Abdi Latif Dahir}

\begin{itemize}
\item
  Feb. 6, 2020
\item
  \begin{itemize}
  \item
  \item
  \item
  \item
  \item
  \end{itemize}
\end{itemize}

ADDIS ABABA, Ethiopia --- As the 9 a.m. flight from Dubai arrived at the
international airport in Ethiopia's capital on Wednesday, four
government health specialists in face masks and protective glasses
worked their way to the line of incoming passengers to check their
passports.

The health specialists found what they were looking for --- two
passengers who had just returned from China --- and pulled them aside to
check their temperatures to see if they might have been infected by the
coronavirus.

The two passengers had normal temperatures, so were allowed to continue
on their way, a government policy that does not account for an
incubation period that is up to 14 days.

As the coronavirus wreaks havoc in China and has spread to countries
around the world, experts are increasingly concerned that Africa is
particularly vulnerable.

The continent's health system is already fragile. It has few facilities
even to test for the virus. Its doctors are already straining to contain
deadly outbreaks of other diseases, like malaria, measles and Ebola.

And on top of that, Africa has large numbers of Chinese workers, many
now returning to the continent after visits to China for the Lunar New
Year. Meanwhile, some of the 81,000 African students who have been
studying in China are now heading home. But while more and more
countries tighten their controls over travel with China, Ethiopia has
kept the door open and the planes flying.

If the coronavirus hits Africa, said Dr. John Nkengasong, director of
Africa Centers for Disease Control and Prevention in Addis Ababa, ``it
will be massive.''

\includegraphics{https://static01.nyt.com/images/2020/02/05/world/Virus-Africa02/merlin_168166587_135f9a94-6760-42b1-93a1-6baadb224e6b-articleLarge.jpg?quality=75\&auto=webp\&disable=upscale}

There have been 32 suspected cases of coronavirus in Africa, but none
tested positive for the virus, according to the Africa C.D.C. But until
this week, only two countries on the continent --- South Africa and
Senegal --- had laboratories capable of testing for the coronavirus.

Most hospitals on the continent, other than large ones in capitals or
regional seats, do not have the intensive care units that patients
diagnosed with the coronavirus might require, experts say.

\hypertarget{latest-updates-global-coronavirus-outbreak}{%
\section{\texorpdfstring{\href{https://www.nytimes.com/2020/08/01/world/coronavirus-covid-19.html?action=click\&pgtype=Article\&state=default\&region=MAIN_CONTENT_1\&context=storylines_live_updates}{Latest
Updates: Global Coronavirus
Outbreak}}{Latest Updates: Global Coronavirus Outbreak}}\label{latest-updates-global-coronavirus-outbreak}}

Updated 2020-08-02T07:42:09.613Z

\begin{itemize}
\tightlist
\item
  \href{https://www.nytimes.com/2020/08/01/world/coronavirus-covid-19.html?action=click\&pgtype=Article\&state=default\&region=MAIN_CONTENT_1\&context=storylines_live_updates\#link-34047410}{The
  U.S. reels as July cases more than double the total of any other
  month.}
\item
  \href{https://www.nytimes.com/2020/08/01/world/coronavirus-covid-19.html?action=click\&pgtype=Article\&state=default\&region=MAIN_CONTENT_1\&context=storylines_live_updates\#link-780ec966}{Top
  U.S. officials work to break an impasse over the federal jobless
  benefit.}
\item
  \href{https://www.nytimes.com/2020/08/01/world/coronavirus-covid-19.html?action=click\&pgtype=Article\&state=default\&region=MAIN_CONTENT_1\&context=storylines_live_updates\#link-2bc8948}{Its
  outbreak untamed, Melbourne goes into even greater lockdown.}
\end{itemize}

\href{https://www.nytimes.com/2020/08/01/world/coronavirus-covid-19.html?action=click\&pgtype=Article\&state=default\&region=MAIN_CONTENT_1\&context=storylines_live_updates}{See
more updates}

More live coverage:
\href{https://www.nytimes.com/live/2020/07/31/business/stock-market-today-coronavirus?action=click\&pgtype=Article\&state=default\&region=MAIN_CONTENT_1\&context=storylines_live_updates}{Markets}

``If this happens in Africa it will be a huge struggle because the
health services are quite overstretched dealing with ongoing diseases
like malaria and measles and the current Ebola outbreak,'' said Michel
Yao, the World Health Organization's Emergency Operations Program
Manager for Africa.

Africa was largely spared in 2002 and 2003 when the SARS virus, which
also originated in China, spread around the world, killing nearly 800
people and infecting more than 8,000, mostly in China and Hong Kong.
Africa reported only one case, in South Africa.

But the risk is far greater now, experts say. China and Africa have
become intertwined in the last two decades as China has expanded its
political, economic, and military ties to Africa,
\href{https://www.nytimes.com/2019/01/13/world/africa/china-loans-africa-usa.html}{funding
large infrastructure projects} and pledging tens of billions of dollars
in investments and loans.

Chinese citizens have flocked to Africa, working in industries ranging
from manufacturing and technology to health care and construction.
Estimates of how many Chinese are now living in Africa range
\href{http://www.sais-cari.org/data-chinese-workers-in-africa}{from
about 200,000} to as many
\href{https://www.migrationpolicy.org/article/african-countries-relax-short-term-visa-policies-chinese}{as
two million}.

Air travel between China and Africa has
\href{https://qz.com/africa/1675287/china-to-africa-flights-jumped-630-in-the-past-nine-years/}{increased
exponentially} in the last decade alone, from one flight a day to an
average of eight direct flights.

Image

Chinese and Ugandan workers at the Isimba Hydro Power Project in eastern
Uganda in 2018. Chinese citizens have flocked to Africa, working in
industries ranging from manufacturing and technology to health care and
construction.Credit...Joao Silva/The New York Times

Ethiopian Airlines, Africa's biggest and most profitable carrier, is the
main gateway between China and Africa, shuttling up to 1,500 passengers
each day between Addis Ababa and China on
\href{https://twitter.com/flyethiopian/status/1068381553192050691}{dozens
of weekly flights}. The airline
\href{http://www.xinhuanet.com/english/2018-11/28/c_137637633.htm}{has a
center} in the Addis Ababa airport to help Chinese travelers easily
process their visas to dozens of African states. The Ethiopian airport
itself was
\href{https://qz.com/africa/1535255/ethiopias-addis-ababa-bole-expanded-airport-triples-size/}{built
in part with funding from China}.

The Ethiopian carrier has continued operating its China routes while
many other international airlines --- including African carriers like
\href{https://twitter.com/KenyaAirways/status/1223119066937692162}{Kenya
Airways},
\href{https://twitter.com/EGYPTAIR/status/1222876029875838977}{Egypt
Air}, and
\href{https://twitter.com/RAM_Maroc/status/1222991000605536270}{Morocco's
Royal Air} --- have suspended flights to China.

The Ethiopian airline and government have now come under criticism in
the midst of the coronavirus outbreak over speculation that China
pressured the Ethiopian government not to halt the flights.

The office of the prime minister and Ethiopian Airlines both declined to
comment, referring questions to the Ethiopian Public Health Institute,
which is dealing with Ethiopia's response to the coronavirus.

Lia Tadesse, Ethiopia's state minister of health, said the decision to
continue flights to China came not from her ministry but from ``a higher
government level.'' But she denied there was any pressure from China.

Yet travelers between China and Africa can also arrive through hubs in
other continents, providing further pathways for the virus.

So African countries and global health organizations are now scrambling
to ramp up the capacity to cope with the epidemic.

Ethiopia has built isolation units at the airport in Addis Ababa, and
designated specific intensive care units in hospitals, said Ms. Tadesse,
Ethiopia's health minister.

The W.H.O. said this week that four more countries --- Ghana,
Madagascar, Nigeria and Sierra Leone --- can now conduct the tests.
Ethiopia says it will have testing capabilities by the end of the week.
But many other African countries will still have to send test kits
elsewhere, delaying any response.

Image

Staff members working in a secure laboratory at the Pasteur Institute in
Dakar, Senegal, one of the few labs in Africa capable of testing for the
new virus.Credit...Seyllou/Agence France-Presse --- Getty Images

``Unfortunately, many disease surveillance systems throughout African
countries are weak and most of the continent lacks diagnostic
capability,'' said Dr. Ngozi Erondu, associate fellow in the Global
Health Program at Chatham House, an international affairs research group
in London. ``Identifying most cases and controlling the outbreak could
be difficult, especially in the poorest and most resource-constrained
countries.''

\href{https://www.nytimes.com/news-event/coronavirus?action=click\&pgtype=Article\&state=default\&region=MAIN_CONTENT_3\&context=storylines_faq}{}

\hypertarget{the-coronavirus-outbreak-}{%
\subsubsection{The Coronavirus Outbreak
›}\label{the-coronavirus-outbreak-}}

\hypertarget{frequently-asked-questions}{%
\paragraph{Frequently Asked
Questions}\label{frequently-asked-questions}}

Updated July 27, 2020

\begin{itemize}
\item ~
  \hypertarget{should-i-refinance-my-mortgage}{%
  \paragraph{Should I refinance my
  mortgage?}\label{should-i-refinance-my-mortgage}}

  \begin{itemize}
  \tightlist
  \item
    \href{https://www.nytimes.com/article/coronavirus-money-unemployment.html?action=click\&pgtype=Article\&state=default\&region=MAIN_CONTENT_3\&context=storylines_faq}{It
    could be a good idea,} because mortgage rates have
    \href{https://www.nytimes.com/2020/07/16/business/mortgage-rates-below-3-percent.html?action=click\&pgtype=Article\&state=default\&region=MAIN_CONTENT_3\&context=storylines_faq}{never
    been lower.} Refinancing requests have pushed mortgage applications
    to some of the highest levels since 2008, so be prepared to get in
    line. But defaults are also up, so if you're thinking about buying a
    home, be aware that some lenders have tightened their standards.
  \end{itemize}
\item ~
  \hypertarget{what-is-school-going-to-look-like-in-september}{%
  \paragraph{What is school going to look like in
  September?}\label{what-is-school-going-to-look-like-in-september}}

  \begin{itemize}
  \tightlist
  \item
    It is unlikely that many schools will return to a normal schedule
    this fall, requiring the grind of
    \href{https://www.nytimes.com/2020/06/05/us/coronavirus-education-lost-learning.html?action=click\&pgtype=Article\&state=default\&region=MAIN_CONTENT_3\&context=storylines_faq}{online
    learning},
    \href{https://www.nytimes.com/2020/05/29/us/coronavirus-child-care-centers.html?action=click\&pgtype=Article\&state=default\&region=MAIN_CONTENT_3\&context=storylines_faq}{makeshift
    child care} and
    \href{https://www.nytimes.com/2020/06/03/business/economy/coronavirus-working-women.html?action=click\&pgtype=Article\&state=default\&region=MAIN_CONTENT_3\&context=storylines_faq}{stunted
    workdays} to continue. California's two largest public school
    districts --- Los Angeles and San Diego --- said on July 13, that
    \href{https://www.nytimes.com/2020/07/13/us/lausd-san-diego-school-reopening.html?action=click\&pgtype=Article\&state=default\&region=MAIN_CONTENT_3\&context=storylines_faq}{instruction
    will be remote-only in the fall}, citing concerns that surging
    coronavirus infections in their areas pose too dire a risk for
    students and teachers. Together, the two districts enroll some
    825,000 students. They are the largest in the country so far to
    abandon plans for even a partial physical return to classrooms when
    they reopen in August. For other districts, the solution won't be an
    all-or-nothing approach.
    \href{https://bioethics.jhu.edu/research-and-outreach/projects/eschool-initiative/school-policy-tracker/}{Many
    systems}, including the nation's largest, New York City, are
    devising
    \href{https://www.nytimes.com/2020/06/26/us/coronavirus-schools-reopen-fall.html?action=click\&pgtype=Article\&state=default\&region=MAIN_CONTENT_3\&context=storylines_faq}{hybrid
    plans} that involve spending some days in classrooms and other days
    online. There's no national policy on this yet, so check with your
    municipal school system regularly to see what is happening in your
    community.
  \end{itemize}
\item ~
  \hypertarget{is-the-coronavirus-airborne}{%
  \paragraph{Is the coronavirus
  airborne?}\label{is-the-coronavirus-airborne}}

  \begin{itemize}
  \tightlist
  \item
    The coronavirus
    \href{https://www.nytimes.com/2020/07/04/health/239-experts-with-one-big-claim-the-coronavirus-is-airborne.html?action=click\&pgtype=Article\&state=default\&region=MAIN_CONTENT_3\&context=storylines_faq}{can
    stay aloft for hours in tiny droplets in stagnant air}, infecting
    people as they inhale, mounting scientific evidence suggests. This
    risk is highest in crowded indoor spaces with poor ventilation, and
    may help explain super-spreading events reported in meatpacking
    plants, churches and restaurants.
    \href{https://www.nytimes.com/2020/07/06/health/coronavirus-airborne-aerosols.html?action=click\&pgtype=Article\&state=default\&region=MAIN_CONTENT_3\&context=storylines_faq}{It's
    unclear how often the virus is spread} via these tiny droplets, or
    aerosols, compared with larger droplets that are expelled when a
    sick person coughs or sneezes, or transmitted through contact with
    contaminated surfaces, said Linsey Marr, an aerosol expert at
    Virginia Tech. Aerosols are released even when a person without
    symptoms exhales, talks or sings, according to Dr. Marr and more
    than 200 other experts, who
    \href{https://academic.oup.com/cid/article/doi/10.1093/cid/ciaa939/5867798}{have
    outlined the evidence in an open letter to the World Health
    Organization}.
  \end{itemize}
\item ~
  \hypertarget{what-are-the-symptoms-of-coronavirus}{%
  \paragraph{What are the symptoms of
  coronavirus?}\label{what-are-the-symptoms-of-coronavirus}}

  \begin{itemize}
  \tightlist
  \item
    Common symptoms
    \href{https://www.nytimes.com/article/symptoms-coronavirus.html?action=click\&pgtype=Article\&state=default\&region=MAIN_CONTENT_3\&context=storylines_faq}{include
    fever, a dry cough, fatigue and difficulty breathing or shortness of
    breath.} Some of these symptoms overlap with those of the flu,
    making detection difficult, but runny noses and stuffy sinuses are
    less common.
    \href{https://www.nytimes.com/2020/04/27/health/coronavirus-symptoms-cdc.html?action=click\&pgtype=Article\&state=default\&region=MAIN_CONTENT_3\&context=storylines_faq}{The
    C.D.C. has also} added chills, muscle pain, sore throat, headache
    and a new loss of the sense of taste or smell as symptoms to look
    out for. Most people fall ill five to seven days after exposure, but
    symptoms may appear in as few as two days or as many as 14 days.
  \end{itemize}
\item ~
  \hypertarget{does-asymptomatic-transmission-of-covid-19-happen}{%
  \paragraph{Does asymptomatic transmission of Covid-19
  happen?}\label{does-asymptomatic-transmission-of-covid-19-happen}}

  \begin{itemize}
  \tightlist
  \item
    So far, the evidence seems to show it does. A widely cited
    \href{https://www.nature.com/articles/s41591-020-0869-5}{paper}
    published in April suggests that people are most infectious about
    two days before the onset of coronavirus symptoms and estimated that
    44 percent of new infections were a result of transmission from
    people who were not yet showing symptoms. Recently, a top expert at
    the World Health Organization stated that transmission of the
    coronavirus by people who did not have symptoms was ``very rare,''
    \href{https://www.nytimes.com/2020/06/09/world/coronavirus-updates.html?action=click\&pgtype=Article\&state=default\&region=MAIN_CONTENT_3\&context=storylines_faq\#link-1f302e21}{but
    she later walked back that statement.}
  \end{itemize}
\end{itemize}

The World Health Organization
\href{https://www.afro.who.int/news/who-ramps-preparedness-novel-coronavirus-african-region}{is
stepping up aid} to 13 African countries that have direct links or a
high volume of travel to China, working to improve early detection of
cases and speed samples to labs that can do the tests. The agency has
said it will need \$675 million through April, primarily to assist poor
countries in Africa and Asia with weak public health systems. The Bill
and Melinda Gates Foundation on Wednesday committed \$100 million to
fight the virus, partly for at-risk populations in Africa.

With no cases yet confirmed in Africa itself, many Africans have focused
their concern on the students who are living in China and may have been
exposed.

About 4,600 African students and citizens live in the epidemic's center,
Hubei province, whose capital, Wuhan, is where the coronavirus first
emerged, according to Development Reimagined, a consulting organization
with headquarters in Beijing.

In all, more than 81,000 Africans ****
\href{http://en.moe.gov.cn/news/press_releases/201904/t20190418_378586.html}{were
studying in mainland China} **** in 2018, lured by generous
scholarships, affordable tuition and the
\href{https://qz.com/africa/1119447/china-is-training-africas-next-generation-of-leaders/}{hope
of becoming a bridge connecting their nations and an ascendant China}.

The first African to be diagnosed with coronavirus is a 21-year-old
Cameroonian student studying at Yangtze University in Hubei province.

The lockdown in Wuhan is taking a toll on students like Abdikadir
Mohamed, a 23-year-old from Somalia who has been studying petroleum
engineering at the China University of Geosciences in Wuhan.

Image

An empty roadway in Wuhan, the central Chinese city where the
coronavirus first appeared.~Credit...Getty Images

Mr. Mohamed, who the Somali government said is one of 50 Somalis living
in Wuhan, said that for nearly 20 hours a day, he doesn't leave his
one-bedroom apartment.

The experience is like being ``stuck in a bad dream,'' he said in a
telephone call on Tuesday.

While other African countries, including Morocco, Mauritius and Egypt,
have evacuated their citizens from China, Mr. Mohamed said that the
Somali students' pleas to their government to evacuate them have been
futile.

``Our families are worried,'' Mr. Mohamed said. ``They are calling us
every minute. If you don't pick the phone at once, they panic.''

Simon Marks reported from Addis Ababa, and Abdi Latif Dahir from
Nairobi, Kenya. Lynsey Chutel contributed reporting from Johannesburg,
Ruth Maclean from Dakar, and Donald G. McNeil Jr. from New York.

Advertisement

\protect\hyperlink{after-bottom}{Continue reading the main story}

\hypertarget{site-index}{%
\subsection{Site Index}\label{site-index}}

\hypertarget{site-information-navigation}{%
\subsection{Site Information
Navigation}\label{site-information-navigation}}

\begin{itemize}
\tightlist
\item
  \href{https://help.nytimes.com/hc/en-us/articles/115014792127-Copyright-notice}{©~2020~The
  New York Times Company}
\end{itemize}

\begin{itemize}
\tightlist
\item
  \href{https://www.nytco.com/}{NYTCo}
\item
  \href{https://help.nytimes.com/hc/en-us/articles/115015385887-Contact-Us}{Contact
  Us}
\item
  \href{https://www.nytco.com/careers/}{Work with us}
\item
  \href{https://nytmediakit.com/}{Advertise}
\item
  \href{http://www.tbrandstudio.com/}{T Brand Studio}
\item
  \href{https://www.nytimes.com/privacy/cookie-policy\#how-do-i-manage-trackers}{Your
  Ad Choices}
\item
  \href{https://www.nytimes.com/privacy}{Privacy}
\item
  \href{https://help.nytimes.com/hc/en-us/articles/115014893428-Terms-of-service}{Terms
  of Service}
\item
  \href{https://help.nytimes.com/hc/en-us/articles/115014893968-Terms-of-sale}{Terms
  of Sale}
\item
  \href{https://spiderbites.nytimes.com}{Site Map}
\item
  \href{https://help.nytimes.com/hc/en-us}{Help}
\item
  \href{https://www.nytimes.com/subscription?campaignId=37WXW}{Subscriptions}
\end{itemize}
