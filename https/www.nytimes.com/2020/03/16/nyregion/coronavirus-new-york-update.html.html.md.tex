Sections

SEARCH

\protect\hyperlink{site-content}{Skip to
content}\protect\hyperlink{site-index}{Skip to site index}

\href{https://www.nytimes.com/section/nyregion}{New York}

\href{https://myaccount.nytimes.com/auth/login?response_type=cookie\&client_id=vi}{}

\href{https://www.nytimes.com/section/todayspaper}{Today's Paper}

\href{/section/nyregion}{New York}\textbar{}Rush for Jobless Benefits
Crashes New York State Website

\url{https://nyti.ms/2WsBUTF}

\begin{itemize}
\item
\item
\item
\item
\item
\item
\end{itemize}

\hypertarget{the-coronavirus-outbreak}{%
\subsubsection{\texorpdfstring{\href{https://www.nytimes.com/news-event/coronavirus?name=styln-coronavirus-national\&region=TOP_BANNER\&variant=undefined\&block=storyline_menu_recirc\&action=click\&pgtype=Article\&impression_id=aba58f50-e0e8-11ea-9c1d-631a2322174c}{The
Coronavirus
Outbreak}}{The Coronavirus Outbreak}}\label{the-coronavirus-outbreak}}

\begin{itemize}
\tightlist
\item
  live\href{https://www.nytimes.com/2020/08/17/world/coronavirus-covid.html?name=styln-coronavirus-national\&region=TOP_BANNER\&variant=undefined\&block=storyline_menu_recirc\&action=click\&pgtype=Article\&impression_id=aba5b660-e0e8-11ea-9c1d-631a2322174c}{Latest
  Updates}
\item
  \href{https://www.nytimes.com/interactive/2020/us/coronavirus-us-cases.html?name=styln-coronavirus-national\&region=TOP_BANNER\&variant=undefined\&block=storyline_menu_recirc\&action=click\&pgtype=Article\&impression_id=aba5b661-e0e8-11ea-9c1d-631a2322174c}{Maps
  and Cases}
\item
  \href{https://www.nytimes.com/interactive/2020/science/coronavirus-vaccine-tracker.html?name=styln-coronavirus-national\&region=TOP_BANNER\&variant=undefined\&block=storyline_menu_recirc\&action=click\&pgtype=Article\&impression_id=aba5b662-e0e8-11ea-9c1d-631a2322174c}{Vaccine
  Tracker}
\item
  \href{https://www.nytimes.com/2020/08/17/us/k-12-schools-reopening.html?name=styln-coronavirus-national\&region=TOP_BANNER\&variant=undefined\&block=storyline_menu_recirc\&action=click\&pgtype=Article\&impression_id=aba5b663-e0e8-11ea-9c1d-631a2322174c}{State
  of Play for K-12}
\item
  \href{https://www.nytimes.com/live/2020/08/17/business/stock-market-today-coronavirus?name=styln-coronavirus-national\&region=TOP_BANNER\&variant=undefined\&block=storyline_menu_recirc\&action=click\&pgtype=Article\&impression_id=aba5b664-e0e8-11ea-9c1d-631a2322174c}{Markets
  \& Economy}
\end{itemize}

Advertisement

\protect\hyperlink{after-top}{Continue reading the main story}

Supported by

\protect\hyperlink{after-sponsor}{Continue reading the main story}

\hypertarget{rush-for-jobless-benefits-crashes-new-york-state-website}{%
\section{Rush for Jobless Benefits Crashes New York State
Website}\label{rush-for-jobless-benefits-crashes-new-york-state-website}}

Mayor Bill de Blasio said that New York City was taking steps to add
hospital beds as the coronavirus outbreak continued to spread.

\begin{itemize}
\item
  Published March 16, 2020Updated March 17, 2020
\item
  \begin{itemize}
  \item
  \item
  \item
  \item
  \item
  \item
  \end{itemize}
\end{itemize}

\emph{{[}This briefing has ended. For the latest updates on the
coronavirus outbreak in the New York area,}
\href{https://www.nytimes.com/2020/03/17/nyregion/coronavirus-new-york-update.html}{\emph{read
Tuesday's live coverage}}\emph{.{]}}

\hypertarget{heres-what-you-need-to-know}{%
\subsubsection{Here's what you need to
know:}\label{heres-what-you-need-to-know}}

\begin{itemize}
\tightlist
\item
  \protect\hyperlink{link-13ae310a}{New York, New Jersey and Connecticut
  ban gatherings of 50 or more.}
\item
  \protect\hyperlink{link-1c0b3898}{Unemployment claims are `comparable
  to post 9/11,' state says.}
\item
  \protect\hyperlink{link-6934947d}{New York City is adding 8,200
  hospital beds.}
\item
  \protect\hyperlink{link-1faf6e73}{Cases continue to climb across the
  region.}
\item
  \protect\hyperlink{link-749c668e}{New York City schools are closed.
  Online lessons start next week.}
\end{itemize}

\includegraphics{https://static01.nyt.com/images/2020/03/16/nyregion/16nyvirus-briefingNEW15/16nyvirus-briefingNEW15-articleLarge.jpg?quality=75\&auto=webp\&disable=upscale}

\hypertarget{new-york-new-jersey-and-connecticut-ban-gatherings-of-50-or-more}{%
\subsection{New York, New Jersey and Connecticut ban gatherings of 50 or
more.}\label{new-york-new-jersey-and-connecticut-ban-gatherings-of-50-or-more}}

The governors of New York, New Jersey and Connecticut announced broad
restrictions on public life on Monday, with gatherings of more than 50
people banned in all three states and many nonessential businesses
ordered closed.

All schools in New York State were also closed for at least two weeks,
Gov. Andrew M. Cuomo announced. The move came after
\href{https://www.nytimes.com/2020/03/16/nyregion/nyc-schools-closed-coronavirus.html}{New
York City's public school system}, the nation's largest, shut down for
at least five weeks.

Casinos, gyms and movie theaters in the three states must close by 8
p.m. Monday, Mr. Cuomo of New York said on a joint call with Govs.
Philip D. Murphy of New Jersey and Ned Lamont of Connecticut.

Bars and restaurants will be limited to takeout and delivery, Mr. Cuomo
said. Groceries, gas stations, pharmacies and some other businesses can
stay open. Mr. Cuomo said he was also encouraging other businesses to
close at 8 p.m.

Some of the region's most famous landmarks, including the Statue of
Liberty, Ellis Island and the Empire State Building, also shut down.

Mr. Lamont said on Monday night that the Danbury Hospital near the New
York border was at capacity and that 200 nurses there could not report
to work because they might have had contact with patients infected with
the coronavirus.

The governor said there were not enough tests to determine whether the
nurses had the virus.

``If I could test those nurses, I could potentially get them back into
the game,'' he said on MSNBC.

Earlier on Monday, Mr. Murphy had asked New Jersey residents to stay at
home from 8 p.m. until 5 a.m. every day for now, a day after suggesting
that he was considering a mandatory curfew.

``We are strongly asking, pleading with folks, to stay home,'' Mr.
Murphy said on Monday night. Asked how long the recommendation would be
in effect, he said: ``It certainly is at least weeks, and it may be many
months.''

\hypertarget{latest-updates-the-coronavirus-outbreak}{%
\section{\texorpdfstring{\href{https://www.nytimes.com/2020/08/17/world/coronavirus-covid.html?action=click\&pgtype=Article\&state=default\&region=MAIN_CONTENT_1\&context=storylines_live_updates}{Latest
Updates: The Coronavirus
Outbreak}}{Latest Updates: The Coronavirus Outbreak}}\label{latest-updates-the-coronavirus-outbreak}}

Updated 2020-08-17T23:52:32.519Z

\begin{itemize}
\tightlist
\item
  \href{https://www.nytimes.com/2020/08/17/world/coronavirus-covid.html?action=click\&pgtype=Article\&state=default\&region=MAIN_CONTENT_1\&context=storylines_live_updates\#link-6cb7525}{Education
  officials in the U.S. grapple with coronavirus fears, outbreaks and
  protests.}
\item
  \href{https://www.nytimes.com/2020/08/17/world/coronavirus-covid.html?action=click\&pgtype=Article\&state=default\&region=MAIN_CONTENT_1\&context=storylines_live_updates\#link-23d8429b}{Some
  doctors say people they treat are more inclined to believe social
  media posts than medical professionals.}
\item
  \href{https://www.nytimes.com/2020/08/17/world/coronavirus-covid.html?action=click\&pgtype=Article\&state=default\&region=MAIN_CONTENT_1\&context=storylines_live_updates\#link-21a159a0}{The
  U.S. postmaster general will testify before a House panel next week.}
\end{itemize}

\href{https://www.nytimes.com/2020/08/17/world/coronavirus-covid.html?action=click\&pgtype=Article\&state=default\&region=MAIN_CONTENT_1\&context=storylines_live_updates}{See
more updates}

More live coverage:
\href{https://www.nytimes.com/live/2020/08/17/business/stock-market-today-coronavirus?action=click\&pgtype=Article\&state=default\&region=MAIN_CONTENT_1\&context=storylines_live_updates}{Markets}

Like Gov. Cuomo, Mr. Murphy has activated the National Guard, whose
members may be asked to work at testing sites, prepare hospitals for
coronavirus patients, control traffic and deliver food to students who
can no longer obtain their meals at school.

\hypertarget{unemployment-claims-are-comparable-to-post-911-state-says}{%
\subsection{Unemployment claims are `comparable to post 9/11,' state
says.}\label{unemployment-claims-are-comparable-to-post-911-state-says}}

The sudden flood of laid-off workers seeking unemployment benefits
swamped New York's Labor Department on Monday.

After Mr. Cuomo waived the usual seven-day waiting period to apply,
workers who had been let go over the weekend immediately tried to
replace some of their lost income.

Within hours, frustrated applicants were complaining on social media
about not being able to apply online. Some said that the state's system
was crashing throughout the day.

Madeleine Witenberg tried to help her partner, Irene Leon, who had just
been laid off from her full-time job as a bartender and server at a
restaurant in Brooklyn's Dumbo neighborhood. Using two different
computers, the women tried unsuccessfully three times to navigate the
state's system, Ms. Leon said.

``It just cut me off as soon as I was making progress,'' she said.

Late in the day, the department acknowledged the problem.

``Today we experienced a massive increase in the volume of unemployment
insurance claims which slowed down the server,'' it said in a statement.
``It is currently being addressed.''

A department spokeswoman, Deanna Cohen, said the agency had received
8,758 calls by noon, more than triple what it got last Monday.

``We are seeing a spike in volume that is comparable to post 9/11 but
make no mistake, anyone entitled to these benefits is going to receive
them in a timely manner,'' Ms. Cohen said.

\hypertarget{new-york-city-is-adding-8200-hospital-beds}{%
\subsection{New York City is adding 8,200 hospital
beds.}\label{new-york-city-is-adding-8200-hospital-beds}}

Image

Lenox Hill Hospital on the Upper East Side.Credit...Gabby Jones for The
New York Times

Mr. de Blasio announced on Monday that New York City was rushing to add
more hospital beds in the next few weeks.

By canceling elective surgeries and dismissing patients from hospitals
more quickly, the city can free up about 7,000 patient beds in existing
hospitals, the mayor said. Another 1,200 to 1,300 beds could be added by
taking over unused space in existing hospitals and converting a newly
built nursing home in Brooklyn that is not yet occupied.

The mayor appealed for federal assistance in helping to staff the new
beds. He said he would like to bring health care workers from other
states that have not been hit as hard by the pandemic and to use
military doctors and nurses.

``These are battlefield-level conditions,'' he said.

With help from the Federal Emergency Management Agency, the city also
announced plans to open five drive-through testing sites. Details on
their locations were not released. (New Jersey officials said the state
would open drive-though test centers at Bergen County Community College
and the PNC Bank Arts Center.)

\hypertarget{cases-continue-to-climb-across-the-region}{%
\subsection{Cases continue to climb across the
region.}\label{cases-continue-to-climb-across-the-region}}

As of Monday, New York State had 950 confirmed coronavirus cases,
officials said, up from 729 on Sunday. Nine people have died from the
virus, including seven in New York City; 158 people have been
hospitalized. In a bright note, the City Council speaker, Corey Johnson,
said 16 people who had been hospitalized had been discharged.

Cases in New Jersey nearly doubled in a day, to 178 on Monday, up from
98 on Sunday. Three people in the state died after testing positive for
the virus. Connecticut reported 41 confirmed cases on Monday, up from 26
on Sunday. No coronavirus death had been reported in the state.

The largest concentration of cases in New York State is in New York
City, where 463 people had tested positive. There were 220 confirmed
cases in Westchester County. On Long Island, there were 109 confirmed
cases in Nassau County and 63 in Suffolk County.

\hypertarget{new-york-city-schools-are-closed-online-lessons-start-next-week}{%
\subsection{New York City schools are closed. Online lessons start next
week.}\label{new-york-city-schools-are-closed-online-lessons-start-next-week}}

Image

Charles Deberry, principal at P.S. 76 A. Phillip Randolph School in
Harlem, surveyed the empty courtyard area on Monday after the decision
to close New York City's public school system.Credit...Brittainy
Newman/The New York Times

The decision to close New York City's 1.1-million-student public school
system came after days of mounting pressure on Mr. de Blasio. He had
\href{https://www.nytimes.com/2020/03/16/nyregion/nyc-schools-closed-coronavirus.html}{vowed
to keep schools open as long as possible because so many working
families depend on them} not just for education but for child care and
meals.

The city plans to restart classes on Monday, March 23, with instruction
offered online only. Students who do not have computers at home will be
lent laptops, the mayor said, and the city will help students get
internet access.

``These children need you,'' Mr. de Blasio said in an appeal to the
city's teachers.

The mayor said he hoped to reopen the full school system April 20 but he
cautioned that schools might stay shut through the end of the academic
year.

In New Jersey, where most schools were already closed, Mr. Murphy said a
statewide shutdown was ``imminent.''

In Connecticut,
\href{https://twitter.com/govnedlamont/status/1239315132435619843?s=21}{Mr.
Lamont said on Sunday} that all public schools in the state would shut
down on Tuesday and stay closed until at least March 31.

\hypertarget{new-york-is-preparing-for-online-classes-tell-us-how-its-going}{%
\subsection{New York is preparing for online classes. Tell us how it's
going.}\label{new-york-is-preparing-for-online-classes-tell-us-how-its-going}}

The New York Times is looking for New York City teachers to tell us
about the switch to remote learning. We want to hear about lesson plans,
what you're learning from colleagues during training and how you're
planning to check on students that need the most support.

\href{https://www.nytimes.com/news-event/coronavirus?action=click\&pgtype=Article\&state=default\&region=MAIN_CONTENT_3\&context=storylines_faq}{}

\hypertarget{the-coronavirus-outbreak-}{%
\subsubsection{The Coronavirus Outbreak
›}\label{the-coronavirus-outbreak-}}

\hypertarget{frequently-asked-questions}{%
\paragraph{Frequently Asked
Questions}\label{frequently-asked-questions}}

Updated August 17, 2020

\begin{itemize}
\item ~
  \hypertarget{why-does-standing-six-feet-away-from-others-help}{%
  \paragraph{Why does standing six feet away from others
  help?}\label{why-does-standing-six-feet-away-from-others-help}}

  \begin{itemize}
  \tightlist
  \item
    The coronavirus spreads primarily through droplets from your mouth
    and nose, especially when you cough or sneeze. The C.D.C., one of
    the organizations using that measure,
    \href{https://www.nytimes.com/2020/04/14/health/coronavirus-six-feet.html?action=click\&pgtype=Article\&state=default\&region=MAIN_CONTENT_3\&context=storylines_faq}{bases
    its recommendation of six feet} on the idea that most large droplets
    that people expel when they cough or sneeze will fall to the ground
    within six feet. But six feet has never been a magic number that
    guarantees complete protection. Sneezes, for instance, can launch
    droplets a lot farther than six feet,
    \href{https://jamanetwork.com/journals/jama/fullarticle/2763852}{according
    to a recent study}. It's a rule of thumb: You should be safest
    standing six feet apart outside, especially when it's windy. But
    keep a mask on at all times, even when you think you're far enough
    apart.
  \end{itemize}
\item ~
  \hypertarget{i-have-antibodies-am-i-now-immune}{%
  \paragraph{I have antibodies. Am I now
  immune?}\label{i-have-antibodies-am-i-now-immune}}

  \begin{itemize}
  \tightlist
  \item
    As of right
    now,\href{https://www.nytimes.com/2020/07/22/health/covid-antibodies-herd-immunity.html?action=click\&pgtype=Article\&state=default\&region=MAIN_CONTENT_3\&context=storylines_faq}{that
    seems likely, for at least several months.} There have been
    frightening accounts of people suffering what seems to be a second
    bout of Covid-19. But experts say these patients may have a
    drawn-out course of infection, with the virus taking a slow toll
    weeks to months after initial exposure. People infected with the
    coronavirus typically
    \href{https://www.nature.com/articles/s41586-020-2456-9}{produce}
    immune molecules called antibodies, which are
    \href{https://www.nytimes.com/2020/05/07/health/coronavirus-antibody-prevalence.html?action=click\&pgtype=Article\&state=default\&region=MAIN_CONTENT_3\&context=storylines_faq}{protective
    proteins made in response to an
    infection}\href{https://www.nytimes.com/2020/05/07/health/coronavirus-antibody-prevalence.html?action=click\&pgtype=Article\&state=default\&region=MAIN_CONTENT_3\&context=storylines_faq}{.
    These antibodies may} last in the body
    \href{https://www.nature.com/articles/s41591-020-0965-6}{only two to
    three months}, which may seem worrisome, but that's perfectly normal
    after an acute infection subsides, said Dr. Michael Mina, an
    immunologist at Harvard University. It may be possible to get the
    coronavirus again, but it's highly unlikely that it would be
    possible in a short window of time from initial infection or make
    people sicker the second time.
  \end{itemize}
\item ~
  \hypertarget{im-a-small-business-owner-can-i-get-relief}{%
  \paragraph{I'm a small-business owner. Can I get
  relief?}\label{im-a-small-business-owner-can-i-get-relief}}

  \begin{itemize}
  \tightlist
  \item
    The
    \href{https://www.nytimes.com/article/small-business-loans-stimulus-grants-freelancers-coronavirus.html?action=click\&pgtype=Article\&state=default\&region=MAIN_CONTENT_3\&context=storylines_faq}{stimulus
    bills enacted in March} offer help for the millions of American
    small businesses. Those eligible for aid are businesses and
    nonprofit organizations with fewer than 500 workers, including sole
    proprietorships, independent contractors and freelancers. Some
    larger companies in some industries are also eligible. The help
    being offered, which is being managed by the Small Business
    Administration, includes the Paycheck Protection Program and the
    Economic Injury Disaster Loan program. But lots of folks have
    \href{https://www.nytimes.com/interactive/2020/05/07/business/small-business-loans-coronavirus.html?action=click\&pgtype=Article\&state=default\&region=MAIN_CONTENT_3\&context=storylines_faq}{not
    yet seen payouts.} Even those who have received help are confused:
    The rules are draconian, and some are stuck sitting on
    \href{https://www.nytimes.com/2020/05/02/business/economy/loans-coronavirus-small-business.html?action=click\&pgtype=Article\&state=default\&region=MAIN_CONTENT_3\&context=storylines_faq}{money
    they don't know how to use.} Many small-business owners are getting
    less than they expected or
    \href{https://www.nytimes.com/2020/06/10/business/Small-business-loans-ppp.html?action=click\&pgtype=Article\&state=default\&region=MAIN_CONTENT_3\&context=storylines_faq}{not
    hearing anything at all.}
  \end{itemize}
\item ~
  \hypertarget{what-are-my-rights-if-i-am-worried-about-going-back-to-work}{%
  \paragraph{What are my rights if I am worried about going back to
  work?}\label{what-are-my-rights-if-i-am-worried-about-going-back-to-work}}

  \begin{itemize}
  \tightlist
  \item
    Employers have to provide
    \href{https://www.osha.gov/SLTC/covid-19/standards.html}{a safe
    workplace} with policies that protect everyone equally.
    \href{https://www.nytimes.com/article/coronavirus-money-unemployment.html?action=click\&pgtype=Article\&state=default\&region=MAIN_CONTENT_3\&context=storylines_faq}{And
    if one of your co-workers tests positive for the coronavirus, the
    C.D.C.} has said that
    \href{https://www.cdc.gov/coronavirus/2019-ncov/community/guidance-business-response.html}{employers
    should tell their employees} -\/- without giving you the sick
    employee's name -\/- that they may have been exposed to the virus.
  \end{itemize}
\item ~
  \hypertarget{what-is-school-going-to-look-like-in-september}{%
  \paragraph{What is school going to look like in
  September?}\label{what-is-school-going-to-look-like-in-september}}

  \begin{itemize}
  \tightlist
  \item
    It is unlikely that many schools will return to a normal schedule
    this fall, requiring the grind of
    \href{https://www.nytimes.com/2020/06/05/us/coronavirus-education-lost-learning.html?action=click\&pgtype=Article\&state=default\&region=MAIN_CONTENT_3\&context=storylines_faq}{online
    learning},
    \href{https://www.nytimes.com/2020/05/29/us/coronavirus-child-care-centers.html?action=click\&pgtype=Article\&state=default\&region=MAIN_CONTENT_3\&context=storylines_faq}{makeshift
    child care} and
    \href{https://www.nytimes.com/2020/06/03/business/economy/coronavirus-working-women.html?action=click\&pgtype=Article\&state=default\&region=MAIN_CONTENT_3\&context=storylines_faq}{stunted
    workdays} to continue. California's two largest public school
    districts --- Los Angeles and San Diego --- said on July 13, that
    \href{https://www.nytimes.com/2020/07/13/us/lausd-san-diego-school-reopening.html?action=click\&pgtype=Article\&state=default\&region=MAIN_CONTENT_3\&context=storylines_faq}{instruction
    will be remote-only in the fall}, citing concerns that surging
    coronavirus infections in their areas pose too dire a risk for
    students and teachers. Together, the two districts enroll some
    825,000 students. They are the largest in the country so far to
    abandon plans for even a partial physical return to classrooms when
    they reopen in August. For other districts, the solution won't be an
    all-or-nothing approach.
    \href{https://bioethics.jhu.edu/research-and-outreach/projects/eschool-initiative/school-policy-tracker/}{Many
    systems}, including the nation's largest, New York City, are
    devising
    \href{https://www.nytimes.com/2020/06/26/us/coronavirus-schools-reopen-fall.html?action=click\&pgtype=Article\&state=default\&region=MAIN_CONTENT_3\&context=storylines_faq}{hybrid
    plans} that involve spending some days in classrooms and other days
    online. There's no national policy on this yet, so check with your
    municipal school system regularly to see what is happening in your
    community.
  \end{itemize}
\end{itemize}

If you can, send us a screenshot of your lesson, or a photo of your home
classroom setup. Your name and comments may be published, but your
contact information will not. A reporter or editor may follow up with
you.

\subsection{}

\hypertarget{fallout-is-immediate-from-school-and-restaurant-closings}{%
\subsection{Fallout is immediate from school and restaurant
closings.}\label{fallout-is-immediate-from-school-and-restaurant-closings}}

Image

Maria Cardenas, center, paid for groceries while her daughter, Ingrid
Lozano, 10, waited at La Boina Roja Meat in Jackson Heights, Queens, on
Monday.~Credit...Desiree Rios for The New York Times

Sandra Martinez and her daughter, Nicole, wore face masks on Monday and
headed to a grocery store in Jackson Heights, Queens, to pick up canned
food and toilet paper.

Nicole, 11, would normally be at her middle school while her mother
worked as a waitress at a Colombian restaurant. But
\href{https://www.nytimes.com/2020/03/15/nyregion/nyc-schools-closed.html}{the
closing of New York City's public school system} had forced them to
overhaul their routine.

``I'm worried about the bills, the car, the rent,'' said Ms. Martinez,
42, who will be out of work and unpaid for an indefinite period as
\href{https://www.nytimes.com/2020/03/15/nyregion/coronavirus-nyc-shutdown.html}{restaurants
and bars shift to delivery only}.

Families across New York City were scrambling for resources and child
care as the threat of the coronavirus prompted a school shutdown
\href{https://www.nytimes.com/2020/03/16/nyregion/nyc-schools-closed-coronavirus.html}{that}put
a heavy strain on parents. (About 14,000 New York City students picked
up free meals at their schools on Monday, schools chancellor Richard A.
Carranza said, a tiny fraction of the roughly 750,000 children who
qualify for free or reduced price meals. Mr. Carranza said he expected
that number to grow.)

The challenge facing Ms. Martinez showed not just the impact of the
schools being closed, but also the affect of
\href{https://www.nytimes.com/2020/03/16/nyregion/nyc-closing-bars-restaurants-coronavirus.html}{the
virus's spread on New York City's restaurant industry.}

``We're completely lost,'' said Odalys Rivera, pouring coffee at a new
taqueria, Cena, that opened in Brooklyn's Windsor Terrace last year, the
dream of her brother and her cousin, the owners.

The shutdown promised to hurt everyone from owners and celebrity chefs
to waiters, waitresses, bar-backs and busboys.

``We have never experienced something like this,'' said Daniel Boulud,
the chef and restaurateur who owns
\href{https://www.nytimes.com/2013/07/24/dining/reviews/restaurant-review-daniel-on-the-upper-east-side.html}{Daniel}.

\hypertarget{the-outbreak-could-lead-to-widespread-job-losses-and-business-failures}{%
\subsection{The outbreak could lead to widespread job losses and
business
failures.}\label{the-outbreak-could-lead-to-widespread-job-losses-and-business-failures}}

Image

The Barclays Center subway station, normally Brooklyn's busiest, was
deserted Monday morning.Credit...John Taggart for The New York Times

New York City faces
\href{https://www.nytimes.com/2020/03/16/nyregion/Coronavirus-nyc-economy-.html}{the
prospect of sweeping job losses and business failures}, with theaters,
bars, restaurants and tourism all closing down or heavily restricted.

James Parrott, director of economic and fiscal policies at the Center
for New York City Affairs at the New School, said the city could lose up
to 500,000 tourism jobs, with lost wages amounting to \$1 billion a
month.

Scott M. Stringer, the New York City comptroller, estimated that the
latest restrictions ``could conservatively cost the city \$3.2 billion
in lost tax revenues over the next six months.''

``We're facing the possibility of a prolonged recession,'' he said.

\hypertarget{jail-visits-and-court-cases-are-postponed-and-the-city-wont-enforce-restrictions-on-e-bikes}{%
\subsection{Jail visits and court cases are postponed, and the city
won't enforce restrictions on
e-bikes.}\label{jail-visits-and-court-cases-are-postponed-and-the-city-wont-enforce-restrictions-on-e-bikes}}

The mayor said on Monday that the Police Department would not enforce a
law barring the use of electric bikes during the coronavirus crisis. The
move would help those who use such bike to make food deliveries. He also
said his administration was considering suspending alternate-side
parking regulations.

In addition, New York's courts postponed many criminal cases
indefinitely and will stop performing all but their most essential
functions. Eviction proceedings have also been suspended.

People charged with felonies who are out on bail will have their cases
adjourned ``until further notice,'' the state's chief administrative
judge, Lawrence K. Marks, wrote.

Visits to city jails will be suspended starting on Wednesday. The city
said it would increase access to phones and postal service to help
detainees stay in contact, and officials said they might also establish
a ``televisit'' system.

One agency whose operations did not appear to have been affected by the
outbreak was the U.S. Immigration and Customs Enforcement, or ICE, whose
agents have continued to arrest immigrants around New York City over the
past 10 days.

The arrests have come in places with growing numbers of coronavirus
cases like New Rochelle and Suffolk County, alarming advocates and
lawyers who believe they could endanger vulnerable people who are
already in custody.

ICE did not immediately respond to a request for comment.

The New York Immigrant Family Unity Project, which provides legal
representation to poor people facing deportation, said it had filed
requests seeking the release of more than two dozen older and medically
vulnerable clients.

Jonah Engel Bromwich, Annie Correal. Michael Gold, Matthew Haag, Patrick
McGeehan, Jesse McKinley, Andy Newman, Edgar Sandoval, Eliza Shapiro,
Liam Stack, Tracey Tully and Benjamin Weiser contributed reporting.

Advertisement

\protect\hyperlink{after-bottom}{Continue reading the main story}

\hypertarget{site-index}{%
\subsection{Site Index}\label{site-index}}

\hypertarget{site-information-navigation}{%
\subsection{Site Information
Navigation}\label{site-information-navigation}}

\begin{itemize}
\tightlist
\item
  \href{https://help.nytimes.com/hc/en-us/articles/115014792127-Copyright-notice}{©~2020~The
  New York Times Company}
\end{itemize}

\begin{itemize}
\tightlist
\item
  \href{https://www.nytco.com/}{NYTCo}
\item
  \href{https://help.nytimes.com/hc/en-us/articles/115015385887-Contact-Us}{Contact
  Us}
\item
  \href{https://www.nytco.com/careers/}{Work with us}
\item
  \href{https://nytmediakit.com/}{Advertise}
\item
  \href{http://www.tbrandstudio.com/}{T Brand Studio}
\item
  \href{https://www.nytimes.com/privacy/cookie-policy\#how-do-i-manage-trackers}{Your
  Ad Choices}
\item
  \href{https://www.nytimes.com/privacy}{Privacy}
\item
  \href{https://help.nytimes.com/hc/en-us/articles/115014893428-Terms-of-service}{Terms
  of Service}
\item
  \href{https://help.nytimes.com/hc/en-us/articles/115014893968-Terms-of-sale}{Terms
  of Sale}
\item
  \href{https://spiderbites.nytimes.com}{Site Map}
\item
  \href{https://help.nytimes.com/hc/en-us}{Help}
\item
  \href{https://www.nytimes.com/subscription?campaignId=37WXW}{Subscriptions}
\end{itemize}
