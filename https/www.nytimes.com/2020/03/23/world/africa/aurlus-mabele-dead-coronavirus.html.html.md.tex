Sections

SEARCH

\protect\hyperlink{site-content}{Skip to
content}\protect\hyperlink{site-index}{Skip to site index}

\href{https://www.nytimes.com/section/world/africa}{Africa}

\href{https://myaccount.nytimes.com/auth/login?response_type=cookie\&client_id=vi}{}

\href{https://www.nytimes.com/section/todayspaper}{Today's Paper}

\href{/section/world/africa}{Africa}\textbar{}Aurlus Mabele, Congolese
King of Soukous Music, Dies at 66

\url{https://nyti.ms/2WCe8EB}

\begin{itemize}
\item
\item
\item
\item
\item
\item
\end{itemize}

\href{https://www.nytimes.com/news-event/coronavirus?action=click\&pgtype=Article\&state=default\&region=TOP_BANNER\&context=storylines_menu}{The
Coronavirus Outbreak}

\begin{itemize}
\tightlist
\item
  live\href{https://www.nytimes.com/2020/08/03/world/coronavirus-covid-19.html?action=click\&pgtype=Article\&state=default\&region=TOP_BANNER\&context=storylines_menu}{Latest
  Updates}
\item
  \href{https://www.nytimes.com/interactive/2020/us/coronavirus-us-cases.html?action=click\&pgtype=Article\&state=default\&region=TOP_BANNER\&context=storylines_menu}{Maps
  and Cases}
\item
  \href{https://www.nytimes.com/interactive/2020/science/coronavirus-vaccine-tracker.html?action=click\&pgtype=Article\&state=default\&region=TOP_BANNER\&context=storylines_menu}{Vaccine
  Tracker}
\item
  \href{https://www.nytimes.com/2020/08/02/us/covid-college-reopening.html?action=click\&pgtype=Article\&state=default\&region=TOP_BANNER\&context=storylines_menu}{College
  Reopening}
\item
  \href{https://www.nytimes.com/live/2020/08/03/business/stock-market-today-coronavirus?action=click\&pgtype=Article\&state=default\&region=TOP_BANNER\&context=storylines_menu}{Economy}
\end{itemize}

Advertisement

\protect\hyperlink{after-top}{Continue reading the main story}

Supported by

\protect\hyperlink{after-sponsor}{Continue reading the main story}

THose we've lost

\hypertarget{aurlus-mabele-congolese-king-of-soukous-music-dies-at-66}{%
\section{Aurlus Mabele, Congolese King of Soukous Music, Dies at
66}\label{aurlus-mabele-congolese-king-of-soukous-music-dies-at-66}}

His up-tempo hits and high-wattage performances were highlighted by
spectacular dance moves. He contracted the coronavirus and died in
Paris.

\includegraphics{https://static01.nyt.com/images/2020/03/24/obituaries/23Mabele1/23Mabele1-articleLarge.jpg?quality=75\&auto=webp\&disable=upscale}

By \href{https://www.nytimes.com/by/abdi-latif-dahir}{Abdi Latif Dahir}

\begin{itemize}
\item
  Published March 23, 2020Updated April 16, 2020
\item
  \begin{itemize}
  \item
  \item
  \item
  \item
  \item
  \item
  \end{itemize}
\end{itemize}

\emph{This obituary is part of a series about people who have died in
the coronavirus pandemic. Read about others}
\href{https://www.nytimes.com/series/people-who-have-died-of-the-coronavirus}{\emph{here}}\emph{.}

Aurlus Mabele, the Congolese singer who was called ``the king of
soukous,'' the energetic dance hall music that blends traditional
African and Caribbean rhythms with pop and soul, died on Thursday in
Paris. He was 66.

His death, at a hospital,
\href{https://www.facebook.com/lizamonetofficial/posts/1566001873562459}{was
confirmed} by his daughter, the singer Liza Monet, who said her father
had contracted the coronavirus. He had had a stroke a few years ago and
had been in fragile health.

The coronavirus pandemic has continued to surge in France, with more
than 16,000 cases and almost 700 deaths as of Monday.

Mr. Mabele rose to fame across Africa in the 1970s and '80s with his
up-tempo hits and high-wattage performances highlighted by spectacular
dance moves. In his early 20s he founded the musical group Les Ndimbola
Lokole in Brazzaville, the capital of the Republic of Congo, gaining
popularity with recordings of songs like ``Waka Waka'' and ``Zebola.''

After moving to France in the 1980s, he helped start the band Loketo,
which means ``hips'' in Lingala, the language spoken in parts of both
the Republic of Congo and the Democratic Republic of Congo. As the
group's lead singer, Mr. Mabele worked alongside the renowned guitarist
\href{http://africanmusic.org/artists/diblo.html}{Diblo Dibala}.

The band thrived on developing and playing
\href{https://www.last.fm/tag/soukous}{soukous}, a modern variation of
the Congolese rumba music. The word soukous is derived from the French
word ``secouer,'' which means ``to shake,'' and as Mr. Mabele's band
Loketo gained fame, the genre took hold in dance halls around the world,
including in France.

Before breaking up in the 1990s, the band recorded bouncy songs like
\href{https://www.youtube.com/watch?v=ug4lzZNUo1M}{``Extra Ball,''}
``Douce Isabelle'' and
\href{https://www.dailymotion.com/video/x1ju7ip}{``Choc a Distance''}
and sold millions of albums worldwide. The group toured Africa, Europe,
the Caribbean and the United States.

Performing in Lower Manhattan at the club S.O.B.'s (for Sound of Brazil)
in 1989, Loketo ``did what it does best: packed the dance floor,'' Peter
Watrous wrote
\href{https://www.nytimes.com/1989/10/06/arts/review-music-congo-s-beat-lifts-a-crowd-to-its-feet-and-the-floor.html}{in
his review} in The New York Times.

``And while the show had its visual side --- two women came out and
invited audience members to bump and grind onstage with them --- it was
the intense interlocking of instruments, feeding off Diblo's guitar
figures, that kept the music effective,'' Mr. Watrous added. ``Like a
mosaic, each little part contributed to a bright, gleaming whole that
added up to a wicked dance machine.''

Mr. Mabele was born Aurélien Miatsonama on Oct. 24, 1953, in
Brazzaville. In addition to Ms. Monet (who was born Alexandra Marie),
his survivors include 12 other children.

His death drew messages of condolence from around the world. Mav
Cacharel, a member of Loketo,
\href{https://www.facebook.com/mavcacharel.fr/posts/3200452843298452?_rdc=1\&_rdr}{said
on Facebook}, ``May the peace and protection of the Lord remain in us.''

\href{https://www.nytimes.com/interactive/2020/obituaries/people-died-coronavirus-obituaries.html?action=click\&pgtype=Article\&state=default\&region=BELOW_MAIN_CONTENT\&context=covid_obits_promo}{}

\hypertarget{those-weve-lost}{%
\section{Those We've Lost}\label{those-weve-lost}}

The coronavirus pandemic has taken an incalculable death toll. This
series is designed to put names and faces to the numbers.

Read more

\includegraphics{https://static01.nyt.com/images/2020/07/30/obituaries/30Pedro/30Pedro-square640.jpg}

\hypertarget{bernaldina-josuxe9-pedro}{%
\section{Bernaldina José Pedro}\label{bernaldina-josuxe9-pedro}}

d. Boa Vista, Brazil

Leader among the Indigenous Macuxi

\includegraphics{https://static01.nyt.com/images/2020/07/31/obituaries/31Swing/merlin_175167783_8913bc90-0d64-43f3-a655-1bb1bf1601c9-square640.jpg}

\hypertarget{john-eric-swing}{%
\section{John Eric Swing}\label{john-eric-swing}}

d. Fountain Valley, Calif.

Champion of Filipino-Americans

\includegraphics{https://static01.nyt.com/images/2020/07/27/obituaries/27Victor/merlin_175001436_38b11f8e-227a-4e2c-9821-7618af9b2524-square640.jpg}

\hypertarget{victor-victor}{%
\section{Victor Victor}\label{victor-victor}}

d. Santo Domingo, Dominican Republic

Beloved musician of the Dominican Republic

\includegraphics{https://static01.nyt.com/images/2020/07/31/obituaries/31Negron/merlin_175160169_516322ae-fd23-4969-b6b2-193ced371105-square640.jpg}

\hypertarget{dr-eddie-negruxf3n}{%
\section{Dr. Eddie Negrón}\label{dr-eddie-negruxf3n}}

d. Fort Walton Beach, Fla.

Internist on Florida's Emerald Coast

\includegraphics{https://static01.nyt.com/images/2020/07/30/obituaries/30Dobson/merlin_175115928_f6b9271c-8f05-4fe1-a38a-5ca4a58f8935-square640.jpg}

\hypertarget{dobby-dobson}{%
\section{Dobby Dobson}\label{dobby-dobson}}

d. Coral Springs, Fla.

Jamaican singer and songwriter

\includegraphics{https://static01.nyt.com/images/2020/08/01/obituaries/28Gonzalez/merlin_175002771_beb57888-3951-409a-ae13-03a94b2e962e-square640.jpg}

\hypertarget{waldemar-gonzalez}{%
\section{Waldemar Gonzalez}\label{waldemar-gonzalez}}

d. White Plains, N.Y.

Teacher and social worker

Advertisement

\protect\hyperlink{after-bottom}{Continue reading the main story}

\hypertarget{site-index}{%
\subsection{Site Index}\label{site-index}}

\hypertarget{site-information-navigation}{%
\subsection{Site Information
Navigation}\label{site-information-navigation}}

\begin{itemize}
\tightlist
\item
  \href{https://help.nytimes.com/hc/en-us/articles/115014792127-Copyright-notice}{©~2020~The
  New York Times Company}
\end{itemize}

\begin{itemize}
\tightlist
\item
  \href{https://www.nytco.com/}{NYTCo}
\item
  \href{https://help.nytimes.com/hc/en-us/articles/115015385887-Contact-Us}{Contact
  Us}
\item
  \href{https://www.nytco.com/careers/}{Work with us}
\item
  \href{https://nytmediakit.com/}{Advertise}
\item
  \href{http://www.tbrandstudio.com/}{T Brand Studio}
\item
  \href{https://www.nytimes.com/privacy/cookie-policy\#how-do-i-manage-trackers}{Your
  Ad Choices}
\item
  \href{https://www.nytimes.com/privacy}{Privacy}
\item
  \href{https://help.nytimes.com/hc/en-us/articles/115014893428-Terms-of-service}{Terms
  of Service}
\item
  \href{https://help.nytimes.com/hc/en-us/articles/115014893968-Terms-of-sale}{Terms
  of Sale}
\item
  \href{https://spiderbites.nytimes.com}{Site Map}
\item
  \href{https://help.nytimes.com/hc/en-us}{Help}
\item
  \href{https://www.nytimes.com/subscription?campaignId=37WXW}{Subscriptions}
\end{itemize}
