Sections

SEARCH

\protect\hyperlink{site-content}{Skip to
content}\protect\hyperlink{site-index}{Skip to site index}

\href{https://www.nytimes.com/section/world/africa}{Africa}

\href{https://myaccount.nytimes.com/auth/login?response_type=cookie\&client_id=vi}{}

\href{https://www.nytimes.com/section/todayspaper}{Today's Paper}

\href{/section/world/africa}{Africa}\textbar{}Africa Braces for
Coronavirus, but Slowly

\url{https://nyti.ms/38ZsWzH}

\begin{itemize}
\item
\item
\item
\item
\item
\end{itemize}

\href{https://www.nytimes.com/news-event/coronavirus?action=click\&pgtype=Article\&state=default\&region=TOP_BANNER\&context=storylines_menu}{The
Coronavirus Outbreak}

\begin{itemize}
\tightlist
\item
  live\href{https://www.nytimes.com/2020/08/01/world/coronavirus-covid-19.html?action=click\&pgtype=Article\&state=default\&region=TOP_BANNER\&context=storylines_menu}{Latest
  Updates}
\item
  \href{https://www.nytimes.com/interactive/2020/us/coronavirus-us-cases.html?action=click\&pgtype=Article\&state=default\&region=TOP_BANNER\&context=storylines_menu}{Maps
  and Cases}
\item
  \href{https://www.nytimes.com/interactive/2020/science/coronavirus-vaccine-tracker.html?action=click\&pgtype=Article\&state=default\&region=TOP_BANNER\&context=storylines_menu}{Vaccine
  Tracker}
\item
  \href{https://www.nytimes.com/interactive/2020/07/29/us/schools-reopening-coronavirus.html?action=click\&pgtype=Article\&state=default\&region=TOP_BANNER\&context=storylines_menu}{What
  School May Look Like}
\item
  \href{https://www.nytimes.com/live/2020/07/31/business/stock-market-today-coronavirus?action=click\&pgtype=Article\&state=default\&region=TOP_BANNER\&context=storylines_menu}{Economy}
\end{itemize}

Advertisement

\protect\hyperlink{after-top}{Continue reading the main story}

Supported by

\protect\hyperlink{after-sponsor}{Continue reading the main story}

\hypertarget{africa-braces-for-coronavirus-but-slowly}{%
\section{Africa Braces for Coronavirus, but
Slowly}\label{africa-braces-for-coronavirus-but-slowly}}

The virus has not yet hit most of the African continent hard, and
neither has social distancing.

\includegraphics{https://static01.nyt.com/images/2020/03/17/world/17virus-africa1/merlin_170580978_2c2ca1f8-d819-4d37-b119-d3d6627e1494-articleLarge.jpg?quality=75\&auto=webp\&disable=upscale}

By \href{https://www.nytimes.com/by/ruth-maclean}{Ruth Maclean}

\begin{itemize}
\item
  Published March 17, 2020Updated June 29, 2020
\item
  \begin{itemize}
  \item
  \item
  \item
  \item
  \item
  \end{itemize}
\end{itemize}

OUAGADOUGOU, Burkina Faso --- Commuters on bicycles and motorbikes wove
through traffic in this West African city wearing face masks to protect
their lungs --- but not against coronavirus. They were protecting
themselves from the fine dust blowing in from the Sahara.

Widespread panic over the coronavirus has not yet arrived in this
country and many others in Africa, even as the pandemic has swept across
China, and now Europe and the United States.

Sub-Saharan
\href{https://www.nytimes.com/2020/06/29/world/africa/Africa-middle-class-coronavirus.html}{Africa}
has not been hit as hard or as early by
\href{https://www.nytimes.com/2020/05/17/world/africa/coronavirus-kano-nigeria-hotspot.html}{coronavirus},
despite
\href{https://www.nytimes.com/2020/02/06/world/africa/africa-coronavirus-china.html}{predictions
by many experts}who had warned that the high traffic between the
continent and China, where the outbreak started, would set off the
\href{https://www.nytimes.com/2020/06/29/world/africa/Africa-middle-class-coronavirus.html}{infection
in Africa}. Instead, it has been
\href{https://www.afro.who.int/news/more-15-countries-africa-report-covid-19-cases}{mostly
people coming from Europe} and North America who have carried the virus
to Africa.

The first two cases in Burkina Faso were a husband-and-wife team of
megachurch pastors, local celebrities, who contracted the virus after
attending a Lenten prayer conference in France. Of the 20 cases now
confirmed in Burkina Faso, two are members of the couple's megachurch
--- and both are from France.

Several African nations, including Uganda, Ghana, Kenya, South Sudan and
South Africa --- the sub-Saharan country with the most cases ---
recently imposed travel bans on swathes of Europe and on the United
States, countries that for years have set strict limits on Africans
entering their borders.

\includegraphics{https://static01.nyt.com/images/2020/03/17/world/17virus-africa5/merlin_170581350_42edcc33-719d-4250-9a9b-aa604b17028e-articleLarge.jpg?quality=75\&auto=webp\&disable=upscale}

But some experts said that people across the continent had yet to take
the threat of coronavirus seriously enough, even though African
presidents have begun announcing strict measures to try to prevent its
spread.

``That's the danger I'm worried about. We can't wait for a repeat of
what happened in China,'' said Oyewale Tomori, a professor of virology
and former president of the Nigerian Academy of Science.

As the number of cases on the continent has slowly climbed, reaching
\href{https://twitter.com/WHOAFRO}{more than 410 across 30 countries} on
Tuesday, some African leaders tried to prepare their countries. Senegal
banned public gatherings, including religious ones. South Africa
declared a national disaster and closed half its borders. Libya closed
its airspace.

\hypertarget{latest-updates-global-coronavirus-outbreak}{%
\section{\texorpdfstring{\href{https://www.nytimes.com/2020/08/01/world/coronavirus-covid-19.html?action=click\&pgtype=Article\&state=default\&region=MAIN_CONTENT_1\&context=storylines_live_updates}{Latest
Updates: Global Coronavirus
Outbreak}}{Latest Updates: Global Coronavirus Outbreak}}\label{latest-updates-global-coronavirus-outbreak}}

Updated 2020-08-02T07:42:09.613Z

\begin{itemize}
\tightlist
\item
  \href{https://www.nytimes.com/2020/08/01/world/coronavirus-covid-19.html?action=click\&pgtype=Article\&state=default\&region=MAIN_CONTENT_1\&context=storylines_live_updates\#link-34047410}{The
  U.S. reels as July cases more than double the total of any other
  month.}
\item
  \href{https://www.nytimes.com/2020/08/01/world/coronavirus-covid-19.html?action=click\&pgtype=Article\&state=default\&region=MAIN_CONTENT_1\&context=storylines_live_updates\#link-780ec966}{Top
  U.S. officials work to break an impasse over the federal jobless
  benefit.}
\item
  \href{https://www.nytimes.com/2020/08/01/world/coronavirus-covid-19.html?action=click\&pgtype=Article\&state=default\&region=MAIN_CONTENT_1\&context=storylines_live_updates\#link-2bc8948}{Its
  outbreak untamed, Melbourne goes into even greater lockdown.}
\end{itemize}

\href{https://www.nytimes.com/2020/08/01/world/coronavirus-covid-19.html?action=click\&pgtype=Article\&state=default\&region=MAIN_CONTENT_1\&context=storylines_live_updates}{See
more updates}

More live coverage:
\href{https://www.nytimes.com/live/2020/07/31/business/stock-market-today-coronavirus?action=click\&pgtype=Article\&state=default\&region=MAIN_CONTENT_1\&context=storylines_live_updates}{Markets}

President Paul Kagame of Rwanda and Prime Minister Abiy Ahmed of
Ethiopia
\href{https://twitter.com/PaulKagame/status/1239263206691999748?s=20}{posted
videos} on social media of themselves washing their hands.

In Burkina Faso, the government has closed schools and universities and
banned public gatherings, but has enforced the measure haphazardly, and
did not apply it to religious meetings.

The celebrity pastors, Mamadou and Hortense Karambiri, and their health
were the talk of the mango tree-shaded drinking spots that dot
Ouagadougou. The couple leads a church of 12,000 members, and had held a
service before coming down with symptoms. Their church, Bethel Israel
Tabernacle, canceled its Sunday services.

But the country is not in panic mode. Not yet.

Image

A staff member at an Assemblies of God evangelical church administered
hand sanitizer at an entrance.Credit...Finbarr O'Reilly for The New York
Times

More than 5,000 people gathered for Friday prayers at Ouagadougou's
Grand Mosque, where men in face masks and latex gloves pumped
disinfectant and soap into the hands of attendees.

On the weekend, thousands of men and women put on their Sunday best,
climbed aboard their motorcycles, and zoomed to church.

Worshipers arriving at the central Assemblies of God church put down
their tambourines to have their hands spritzed with sanitizer. The air
conditioning was turned off and the windows opened. Communion was
canceled.

``Don't give in to panic, don't give in to fear,'' Rev. Jean-Baptiste
Rouamba told his congregation, after a special announcement about hand
washing and coughing into elbows. ``Fear is another kind of sickness.''

After services, he said in an interview that he would cancel worship
services if the government ordered it. But he said that his services had
been more popular than ever since the outbreak, attracting up to 2,000
people.

If things got worse, he said, he would hold two Sunday services instead
of one so that people could sit with a seat between them.

``If it got the Karambiri couple, nobody is safe,'' he said, putting his
hand on the shoulder of a young congregant who wore an airline eye mask
over his mouth.

Image

Men attending Friday prayers at the Grand Mosque last
week.Credit...Finbarr O'Reilly for The New York Times

Experts have varying explanations for why coronavirus has not yet hit
Africa hard:
\href{https://papers.ssrn.com/sol3/papers.cfm?abstract_id=3551767}{some
say} it is slower to spread in warmer weather, though this
\href{https://www.nytimes.com/reuters/2020/03/16/world/asia/16reuters-health-coronavirus-southeast-asia.html}{is
disputed}; others that the continent's
\href{https://www.theafricareport.com/24160/covid-19-is-only-slowly-reaching-africa-thats-no-surprise/}{relatively
limited international links} have slowed it down.

It has not gone unnoticed on the continent that the preponderance of
cases originated from Europe and the United States. Last week, after
Kenya announced that the country's first case of coronavirus was a woman
who had traveled from the United States through London to Nairobi,
rumors circulated on social media that Africans are immune to the virus.

``I would like to disabuse that notion,'' said Mutahi Kagwe, Kenya's
minister of health,
\href{https://www.msn.com/en-nz/news/world/kenya-confirms-first-case-of-coronavirus-in-east-africa/ar-BB118ONC?ocid=st2}{at
a news conference}. ``The lady is an African, like you and I.''

\href{https://www.nytimes.com/news-event/coronavirus?action=click\&pgtype=Article\&state=default\&region=MAIN_CONTENT_3\&context=storylines_faq}{}

\hypertarget{the-coronavirus-outbreak-}{%
\subsubsection{The Coronavirus Outbreak
›}\label{the-coronavirus-outbreak-}}

\hypertarget{frequently-asked-questions}{%
\paragraph{Frequently Asked
Questions}\label{frequently-asked-questions}}

Updated July 27, 2020

\begin{itemize}
\item ~
  \hypertarget{should-i-refinance-my-mortgage}{%
  \paragraph{Should I refinance my
  mortgage?}\label{should-i-refinance-my-mortgage}}

  \begin{itemize}
  \tightlist
  \item
    \href{https://www.nytimes.com/article/coronavirus-money-unemployment.html?action=click\&pgtype=Article\&state=default\&region=MAIN_CONTENT_3\&context=storylines_faq}{It
    could be a good idea,} because mortgage rates have
    \href{https://www.nytimes.com/2020/07/16/business/mortgage-rates-below-3-percent.html?action=click\&pgtype=Article\&state=default\&region=MAIN_CONTENT_3\&context=storylines_faq}{never
    been lower.} Refinancing requests have pushed mortgage applications
    to some of the highest levels since 2008, so be prepared to get in
    line. But defaults are also up, so if you're thinking about buying a
    home, be aware that some lenders have tightened their standards.
  \end{itemize}
\item ~
  \hypertarget{what-is-school-going-to-look-like-in-september}{%
  \paragraph{What is school going to look like in
  September?}\label{what-is-school-going-to-look-like-in-september}}

  \begin{itemize}
  \tightlist
  \item
    It is unlikely that many schools will return to a normal schedule
    this fall, requiring the grind of
    \href{https://www.nytimes.com/2020/06/05/us/coronavirus-education-lost-learning.html?action=click\&pgtype=Article\&state=default\&region=MAIN_CONTENT_3\&context=storylines_faq}{online
    learning},
    \href{https://www.nytimes.com/2020/05/29/us/coronavirus-child-care-centers.html?action=click\&pgtype=Article\&state=default\&region=MAIN_CONTENT_3\&context=storylines_faq}{makeshift
    child care} and
    \href{https://www.nytimes.com/2020/06/03/business/economy/coronavirus-working-women.html?action=click\&pgtype=Article\&state=default\&region=MAIN_CONTENT_3\&context=storylines_faq}{stunted
    workdays} to continue. California's two largest public school
    districts --- Los Angeles and San Diego --- said on July 13, that
    \href{https://www.nytimes.com/2020/07/13/us/lausd-san-diego-school-reopening.html?action=click\&pgtype=Article\&state=default\&region=MAIN_CONTENT_3\&context=storylines_faq}{instruction
    will be remote-only in the fall}, citing concerns that surging
    coronavirus infections in their areas pose too dire a risk for
    students and teachers. Together, the two districts enroll some
    825,000 students. They are the largest in the country so far to
    abandon plans for even a partial physical return to classrooms when
    they reopen in August. For other districts, the solution won't be an
    all-or-nothing approach.
    \href{https://bioethics.jhu.edu/research-and-outreach/projects/eschool-initiative/school-policy-tracker/}{Many
    systems}, including the nation's largest, New York City, are
    devising
    \href{https://www.nytimes.com/2020/06/26/us/coronavirus-schools-reopen-fall.html?action=click\&pgtype=Article\&state=default\&region=MAIN_CONTENT_3\&context=storylines_faq}{hybrid
    plans} that involve spending some days in classrooms and other days
    online. There's no national policy on this yet, so check with your
    municipal school system regularly to see what is happening in your
    community.
  \end{itemize}
\item ~
  \hypertarget{is-the-coronavirus-airborne}{%
  \paragraph{Is the coronavirus
  airborne?}\label{is-the-coronavirus-airborne}}

  \begin{itemize}
  \tightlist
  \item
    The coronavirus
    \href{https://www.nytimes.com/2020/07/04/health/239-experts-with-one-big-claim-the-coronavirus-is-airborne.html?action=click\&pgtype=Article\&state=default\&region=MAIN_CONTENT_3\&context=storylines_faq}{can
    stay aloft for hours in tiny droplets in stagnant air}, infecting
    people as they inhale, mounting scientific evidence suggests. This
    risk is highest in crowded indoor spaces with poor ventilation, and
    may help explain super-spreading events reported in meatpacking
    plants, churches and restaurants.
    \href{https://www.nytimes.com/2020/07/06/health/coronavirus-airborne-aerosols.html?action=click\&pgtype=Article\&state=default\&region=MAIN_CONTENT_3\&context=storylines_faq}{It's
    unclear how often the virus is spread} via these tiny droplets, or
    aerosols, compared with larger droplets that are expelled when a
    sick person coughs or sneezes, or transmitted through contact with
    contaminated surfaces, said Linsey Marr, an aerosol expert at
    Virginia Tech. Aerosols are released even when a person without
    symptoms exhales, talks or sings, according to Dr. Marr and more
    than 200 other experts, who
    \href{https://academic.oup.com/cid/article/doi/10.1093/cid/ciaa939/5867798}{have
    outlined the evidence in an open letter to the World Health
    Organization}.
  \end{itemize}
\item ~
  \hypertarget{what-are-the-symptoms-of-coronavirus}{%
  \paragraph{What are the symptoms of
  coronavirus?}\label{what-are-the-symptoms-of-coronavirus}}

  \begin{itemize}
  \tightlist
  \item
    Common symptoms
    \href{https://www.nytimes.com/article/symptoms-coronavirus.html?action=click\&pgtype=Article\&state=default\&region=MAIN_CONTENT_3\&context=storylines_faq}{include
    fever, a dry cough, fatigue and difficulty breathing or shortness of
    breath.} Some of these symptoms overlap with those of the flu,
    making detection difficult, but runny noses and stuffy sinuses are
    less common.
    \href{https://www.nytimes.com/2020/04/27/health/coronavirus-symptoms-cdc.html?action=click\&pgtype=Article\&state=default\&region=MAIN_CONTENT_3\&context=storylines_faq}{The
    C.D.C. has also} added chills, muscle pain, sore throat, headache
    and a new loss of the sense of taste or smell as symptoms to look
    out for. Most people fall ill five to seven days after exposure, but
    symptoms may appear in as few as two days or as many as 14 days.
  \end{itemize}
\item ~
  \hypertarget{does-asymptomatic-transmission-of-covid-19-happen}{%
  \paragraph{Does asymptomatic transmission of Covid-19
  happen?}\label{does-asymptomatic-transmission-of-covid-19-happen}}

  \begin{itemize}
  \tightlist
  \item
    So far, the evidence seems to show it does. A widely cited
    \href{https://www.nature.com/articles/s41591-020-0869-5}{paper}
    published in April suggests that people are most infectious about
    two days before the onset of coronavirus symptoms and estimated that
    44 percent of new infections were a result of transmission from
    people who were not yet showing symptoms. Recently, a top expert at
    the World Health Organization stated that transmission of the
    coronavirus by people who did not have symptoms was ``very rare,''
    \href{https://www.nytimes.com/2020/06/09/world/coronavirus-updates.html?action=click\&pgtype=Article\&state=default\&region=MAIN_CONTENT_3\&context=storylines_faq\#link-1f302e21}{but
    she later walked back that statement.}
  \end{itemize}
\end{itemize}

Some warn that if and when the virus gets into crowded cities like
Kinshasa, Lagos and Addis Ababa, the results will be disastrous.

Many African countries set up public health institutions in the wake of
the Ebola outbreak that began in West Africa in 2013, and the African
Union established the Africa Centers for Disease Control and Prevention,
which coordinates the fight against outbreaks.

``The Ebola outbreak was a wake up call for the entire continent that
our public health systems and health systems as a whole were weak,''
said Dr. John Nkengasong, director of the Africa C.D.C.

However, the continent's public health systems have never been
well-funded, and experts warned that this vulnerability, along with
crowded conditions and poor sanitation in cities, and the unpredictable
movement of populations, could make outbreaks impossible to control.

``I don't believe, if we have a large influx of people with the virus,
we can cope,'' said Dr. Tomori.

Nevertheless, in Ouagadougou in recent days, life continued almost as
normal. Photographers jostled a stream of wedding parties into position
in front of Ouagadougou's fanciest swimming pool. Strawberry hawkers
elbowed each other to sell their wares at car windows.

Image

Children using hand sanitizer before entering the Assemblies of God
evangelical church.Credit...Finbarr O'Reilly for The New York Times

More than 500 men gathered on Saturday in Samandin, a neighborhood in
the capital city, for the inauguration of a new local crime-fighting
group. Seeking shade in the 104 degree heat, they sat close together on
plastic chairs under tarpaulins for more than three hours. There were no
hand washing facilities, hand sanitizer or disposable masks to be seen.

Presiding over the ceremony was the Malgré-Naaba of Samandin, a
traditional chief, who gave up his personal name when he assumed the
role.

``I think we can manage it if we practice the correct behavior,'' he
said. He then warmly shook the hands of a dozen supplicants.

The Malgré-Naaba acknowledged that the government had forbidden such
gatherings, but said the event was an exceptional case.

``This was planned in advance,'' he said. ``But coronavirus --- that was
not.''

Abdi Latif Dahir contributed reporting from Nairobi, Kenya, and Kampala,
Uganda, Simon Marks from Addis Ababa, Ethiopia, and Finbarr O'Reilly
from Ouagadougou.

Advertisement

\protect\hyperlink{after-bottom}{Continue reading the main story}

\hypertarget{site-index}{%
\subsection{Site Index}\label{site-index}}

\hypertarget{site-information-navigation}{%
\subsection{Site Information
Navigation}\label{site-information-navigation}}

\begin{itemize}
\tightlist
\item
  \href{https://help.nytimes.com/hc/en-us/articles/115014792127-Copyright-notice}{©~2020~The
  New York Times Company}
\end{itemize}

\begin{itemize}
\tightlist
\item
  \href{https://www.nytco.com/}{NYTCo}
\item
  \href{https://help.nytimes.com/hc/en-us/articles/115015385887-Contact-Us}{Contact
  Us}
\item
  \href{https://www.nytco.com/careers/}{Work with us}
\item
  \href{https://nytmediakit.com/}{Advertise}
\item
  \href{http://www.tbrandstudio.com/}{T Brand Studio}
\item
  \href{https://www.nytimes.com/privacy/cookie-policy\#how-do-i-manage-trackers}{Your
  Ad Choices}
\item
  \href{https://www.nytimes.com/privacy}{Privacy}
\item
  \href{https://help.nytimes.com/hc/en-us/articles/115014893428-Terms-of-service}{Terms
  of Service}
\item
  \href{https://help.nytimes.com/hc/en-us/articles/115014893968-Terms-of-sale}{Terms
  of Sale}
\item
  \href{https://spiderbites.nytimes.com}{Site Map}
\item
  \href{https://help.nytimes.com/hc/en-us}{Help}
\item
  \href{https://www.nytimes.com/subscription?campaignId=37WXW}{Subscriptions}
\end{itemize}
