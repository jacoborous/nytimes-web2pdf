Sections

SEARCH

\protect\hyperlink{site-content}{Skip to
content}\protect\hyperlink{site-index}{Skip to site index}

\href{https://www.nytimes.com/section/us}{U.S.}

\href{https://myaccount.nytimes.com/auth/login?response_type=cookie\&client_id=vi}{}

\href{https://www.nytimes.com/section/todayspaper}{Today's Paper}

\href{/section/us}{U.S.}\textbar{}Days After a Funeral in a Georgia
Town, Coronavirus `Hit Like a Bomb'

\url{https://nyti.ms/2WX2F2r}

\begin{itemize}
\item
\item
\item
\item
\item
\item
\end{itemize}

\href{https://www.nytimes.com/news-event/coronavirus?action=click\&pgtype=Article\&state=default\&region=TOP_BANNER\&context=storylines_menu}{The
Coronavirus Outbreak}

\begin{itemize}
\tightlist
\item
  live\href{https://www.nytimes.com/2020/08/02/world/coronavirus-updates.html?action=click\&pgtype=Article\&state=default\&region=TOP_BANNER\&context=storylines_menu}{Latest
  Updates}
\item
  \href{https://www.nytimes.com/interactive/2020/us/coronavirus-us-cases.html?action=click\&pgtype=Article\&state=default\&region=TOP_BANNER\&context=storylines_menu}{Maps
  and Cases}
\item
  \href{https://www.nytimes.com/interactive/2020/science/coronavirus-vaccine-tracker.html?action=click\&pgtype=Article\&state=default\&region=TOP_BANNER\&context=storylines_menu}{Vaccine
  Tracker}
\item
  \href{https://www.nytimes.com/interactive/2020/07/29/us/schools-reopening-coronavirus.html?action=click\&pgtype=Article\&state=default\&region=TOP_BANNER\&context=storylines_menu}{What
  School May Look Like}
\item
  \href{https://www.nytimes.com/live/2020/07/31/business/stock-market-today-coronavirus?action=click\&pgtype=Article\&state=default\&region=TOP_BANNER\&context=storylines_menu}{Economy}
\end{itemize}

Advertisement

\protect\hyperlink{after-top}{Continue reading the main story}

Supported by

\protect\hyperlink{after-sponsor}{Continue reading the main story}

\hypertarget{days-after-a-funeral-in-a-georgia-town-coronavirus-hit-like-a-bomb}{%
\section{Days After a Funeral in a Georgia Town, Coronavirus `Hit Like a
Bomb'}\label{days-after-a-funeral-in-a-georgia-town-coronavirus-hit-like-a-bomb}}

A mourner came to Albany, Ga., to attend the funeral of a retired
janitor. After a pause while the infections incubated, the virus swept
through the community.

\includegraphics{https://static01.nyt.com/images/2020/03/31/us/31virus-cluster1/merlin_171074628_d4995a88-09fa-450a-950d-cba33cd4de98-articleLarge.jpg?quality=75\&auto=webp\&disable=upscale}

\href{https://www.nytimes.com/by/ellen-barry}{\includegraphics{https://static01.nyt.com/images/2018/10/08/multimedia/author-ellen-barry/author-ellen-barry-thumbLarge.png}}

By \href{https://www.nytimes.com/by/ellen-barry}{Ellen Barry}

\begin{itemize}
\item
  March 30, 2020
\item
  \begin{itemize}
  \item
  \item
  \item
  \item
  \item
  \item
  \end{itemize}
\end{itemize}

It was an old-fashioned Southern funeral.

There was a repast table crammed with casseroles, Brunswick stew, fried
chicken and key lime cake. Andrew Jerome Mitchell, a retired janitor,
was one of 10 siblings. They told stories, debated for the umpteenth
time how he got the nickname Doorface.

People wiped tears away, and embraced, and blew their noses, and belted
out hymns. They laughed, remembering. It was a big gathering, with
upward of 200 mourners overflowing the memorial chapel, so people had to
stand outside.

Dorothy Johnson has gone over the scene in her mind over the last month,
asking herself who it was who brought the virus to her brother's
funeral.

``We don't know who the person was,'' she said. ``It would help me to
know.''

During the weeks that followed, illnesses linked to the coronavirus have
torn through her hometown, Albany, Ga., with about two dozen relatives
falling ill, including six of her siblings. Ms. Johnson herself was
released from an isolation ward to the news that her daughter, Tonya,
was in grave condition, her heart rate dropping.

Like the Biogen conference in Boston and
\href{https://www.nytimes.com/2020/03/23/us/coronavirus-westport-connecticut-party-zero.html}{a
40th birthday party in Westport, Conn.}, the funeral of Andrew Jerome
Mitchell on Feb. 29 will be recorded as what epidemiologists call a
``super-spreading event,'' in which a small number of people propagate a
huge number of infections.

This rural county in southwest Georgia, 40 miles from the nearest
interstate, now has one of the most intense clusters of the coronavirus
in the country.

With a population of only 90,000, Dougherty
County\href{http://www.southwestgeorgiapublichealth.org/?fbclid=IwAR11j4IASewgcRU4ogqaQecmBtM0HgsDwLIXnthrf9NMJzt1Q7MxOvyawgg}{has
registered 24 deaths}, far more than any other county in the state, with
six more possible coronavirus deaths under investigation, according to
Michael L. Fowler, the local coroner. Ninety percent of the people who
died were African-American, he said.

The region's hospitals are overloaded with sick and dying patients,
having registered nearly 600 positive cases. Last week, Gov. Brian Kemp
dispatched the National Guard to help stage additional intensive care
beds and relieve exhausted doctors and nurses.

Ms. Johnson said that she assumed one of the guests had brought the
virus to her brother's funeral, where ``you hug and you kiss and you
embrace.'' But she had no more information than that.

``Really, there is no face to what is going on in Albany,'' she said.

Whether the initial carrier --- the whodunit of infectious disease ---
matters at all depends on whom you ask. But the timing does matter. For
10 days the virus spread, invisibly, and no one knew it was there. By
the time stringent social distancing was introduced, on March 22, it was
everywhere.

``We're not blaming that one visitor, but potentially a community is one
person away from something like this exploding,'' said Scott Steiner,
the chief executive of Phoebe Putney Health System, which has taken the
brunt of the surge.

``If you get early delivery of it, it shows you what can happen,'' he
said. ``Had that person come in, had a barbecue, visited family and went
home, that would have been a different story.''

\includegraphics{https://static01.nyt.com/images/2020/03/31/us/31virus-clusters2/merlin_171050442_777fde62-8759-46df-840b-805287a956de-articleLarge.jpg?quality=75\&auto=webp\&disable=upscale}

\hypertarget{a-big-southern-funeral}{%
\subsection{A big Southern funeral}\label{a-big-southern-funeral}}

Mr. Mitchell died suddenly. Emell Murray, Mr. Mitchell's companion of 20
years, found him in the living room of their home on the morning of Feb.
24, said her daughter, Alice Bell. There was no autopsy, but it appeared
to be natural causes, she said, possibly a heart attack.

``He had been up all night,'' said Ms. Johnson, his sister. ``When she
woke up to get the baby ready for school, she found him face down on the
floor.''

\hypertarget{latest-updates-global-coronavirus-outbreak}{%
\section{\texorpdfstring{\href{https://www.nytimes.com/2020/08/01/world/coronavirus-covid-19.html?action=click\&pgtype=Article\&state=default\&region=MAIN_CONTENT_1\&context=storylines_live_updates}{Latest
Updates: Global Coronavirus
Outbreak}}{Latest Updates: Global Coronavirus Outbreak}}\label{latest-updates-global-coronavirus-outbreak}}

Updated 2020-08-02T17:52:35.962Z

\begin{itemize}
\tightlist
\item
  \href{https://www.nytimes.com/2020/08/01/world/coronavirus-covid-19.html?action=click\&pgtype=Article\&state=default\&region=MAIN_CONTENT_1\&context=storylines_live_updates\#link-34047410}{The
  U.S. reels as July cases more than double the total of any other
  month.}
\item
  \href{https://www.nytimes.com/2020/08/01/world/coronavirus-covid-19.html?action=click\&pgtype=Article\&state=default\&region=MAIN_CONTENT_1\&context=storylines_live_updates\#link-780ec966}{Top
  U.S. officials work to break an impasse over the federal jobless
  benefit.}
\item
  \href{https://www.nytimes.com/2020/08/01/world/coronavirus-covid-19.html?action=click\&pgtype=Article\&state=default\&region=MAIN_CONTENT_1\&context=storylines_live_updates\#link-2bc8948}{Its
  outbreak untamed, Melbourne goes into even greater lockdown.}
\end{itemize}

\href{https://www.nytimes.com/2020/08/01/world/coronavirus-covid-19.html?action=click\&pgtype=Article\&state=default\&region=MAIN_CONTENT_1\&context=storylines_live_updates}{See
more updates}

More live coverage:
\href{https://www.nytimes.com/live/2020/07/31/business/stock-market-today-coronavirus?action=click\&pgtype=Article\&state=default\&region=MAIN_CONTENT_1\&context=storylines_live_updates}{Markets}

Dougherty County, dominated by cotton plantations in the 19th century,
routinely
\href{https://www.countyhealthrankings.org/sites/default/files/media/document/CHR2020_GA_v2.pdf}{ranks
near the bottom of Georgia's 159 counties in terms of most health
outcomes}, with high rates of diabetes and lung disease, and its first
coronavirus cases did not stand out to doctors as something unusual.

The night of the funeral, a 67-year-old man who had come to Albany to
attend was admitted to Phoebe Putney Memorial Hospital, complaining of
shortness of breath, Mr. Steiner said.

The man had chronic lung disease, and no history of travel that would
suggest exposure to the coronavirus, and he was not put in isolation,
Mr. Steiner said. Staff members figured that he had just run out of
oxygen.

The man spent the next week in the hospital, attended by at least 50
employees, then was transferred on March 7 back to the Atlanta area,
where he was tested for the coronavirus. Not until March 10 did the
Albany hospital learn he had tested positive, Mr. Steiner said. He died
on March 12, the state's first coronavirus death.

By then, the infection was quietly spreading through town. Mr.
Mitchell's longtime companion, Ms. Murray, 75, found herself racked with
chills and fever, Ms. Bell, her daughter, said. She was told she had a
urinary tract infection and admitted to an ordinary ward, where she was
visited by three of her sisters, Ms. Bell said. All three have since
become sick with the coronavirus, she said. One of them has died.

Image

A normally busy strip in downtown Albany.Credit...Audra Melton for The
New York Times

\hypertarget{it-hit-like-a-bomb}{%
\subsection{`It hit like a bomb'}\label{it-hit-like-a-bomb}}

On March 10, word reached Albany that the Phoebe Putney patient had
tested positive for the virus. A few days of relative quiet followed,
and then, in the words of Mr. Fowler, the coroner, ``it hit like a
bomb.''

``Some of them might have went to the funeral,'' Mr. Fowler said. ``Some
may have been family members of people at the funeral. Every day after
that, someone was dying.''

The six-month stockpile of protective equipment that the hospital had
prepared was gone, Mr. Steiner said, in seven days.

At first the doctors and nurses just tried to take in what they were
seeing: a series of people --- including young people in relatively good
health --- showing up with a cough and fever.

Then, alarmingly,
\href{https://abcnews.go.com/US/video-diaries-health-care-workers-document-fight-lives/story?id=69833398}{their
need for oxygen would sharply increase,} and they would go into
full-blown respiratory failure, their lungs filling with fluid, said Dr.
Enrique Lopez, 41, a surgical intensivist, who specializes in treating
the critically ill.

``All the units were full, all of them, and there would be days when we
would be intubating five people in a row, back to back, room after room
after room,'' he said. ``It was one of the times in my career I truly
felt overwhelmed.''

The cases arrived in great waves, overwhelming each new effort to add
beds.

The 14 medical intensive care unit beds were filled within two days of
the first wave of coronavirus patients; they converted 12 cardiac I.C.U.
beds, but those, too, were filled two days later; 12 beds in the
surgical I.C.U. were filled three days after that, Mr. Steiner said.

For a few days, the hospital was so short of staff members that
employees who had tested positive but did not yet have symptoms were
asked to work.

``If I had 1,000 nurses sitting at home, and could send the ones testing
positive out, I would, but we don't have that, and nobody has that,''
Mr. Steiner said. ``You get to the point where you say, `If I don't have
the staff, I can't care for the patients.'''

State directives changed last week, mandating a weeklong quarantine for
health care workers who test positive.

Dr. Lopez, the surgeon, avoided contact with his family for two weeks,
for fear of infecting them.

``I'm sleeping in the garage in one of our closets,'' he said. ``I park
the truck, strip down in the garage, wash myself off, my wife puts out a
plate of food for me, I eat the food, and then I go back to the
garage.''

Image

The Rev. Daniel Simmons, pastor of Mt. Zion Baptist Church in Albany,
questioned whether two funerals were in fact the sole source of the
county's infections.Credit...Audra Melton for The New York Times

\hypertarget{detective-work}{%
\subsection{Detective work}\label{detective-work}}

The funerals in Albany --- of Mr. Mitchell, and then of a man named
Johnny Carter, held at the funeral home a week later --- quickly emerged
as a source of infection.

Of the first 23 patients to test positive at Phoebe Putney, all had
attended at least one of the two funerals, Mr. Steiner said. That was
easy to figure out.

\href{https://www.nytimes.com/news-event/coronavirus?action=click\&pgtype=Article\&state=default\&region=MAIN_CONTENT_3\&context=storylines_faq}{}

\hypertarget{the-coronavirus-outbreak-}{%
\subsubsection{The Coronavirus Outbreak
›}\label{the-coronavirus-outbreak-}}

\hypertarget{frequently-asked-questions}{%
\paragraph{Frequently Asked
Questions}\label{frequently-asked-questions}}

Updated July 27, 2020

\begin{itemize}
\item ~
  \hypertarget{should-i-refinance-my-mortgage}{%
  \paragraph{Should I refinance my
  mortgage?}\label{should-i-refinance-my-mortgage}}

  \begin{itemize}
  \tightlist
  \item
    \href{https://www.nytimes.com/article/coronavirus-money-unemployment.html?action=click\&pgtype=Article\&state=default\&region=MAIN_CONTENT_3\&context=storylines_faq}{It
    could be a good idea,} because mortgage rates have
    \href{https://www.nytimes.com/2020/07/16/business/mortgage-rates-below-3-percent.html?action=click\&pgtype=Article\&state=default\&region=MAIN_CONTENT_3\&context=storylines_faq}{never
    been lower.} Refinancing requests have pushed mortgage applications
    to some of the highest levels since 2008, so be prepared to get in
    line. But defaults are also up, so if you're thinking about buying a
    home, be aware that some lenders have tightened their standards.
  \end{itemize}
\item ~
  \hypertarget{what-is-school-going-to-look-like-in-september}{%
  \paragraph{What is school going to look like in
  September?}\label{what-is-school-going-to-look-like-in-september}}

  \begin{itemize}
  \tightlist
  \item
    It is unlikely that many schools will return to a normal schedule
    this fall, requiring the grind of
    \href{https://www.nytimes.com/2020/06/05/us/coronavirus-education-lost-learning.html?action=click\&pgtype=Article\&state=default\&region=MAIN_CONTENT_3\&context=storylines_faq}{online
    learning},
    \href{https://www.nytimes.com/2020/05/29/us/coronavirus-child-care-centers.html?action=click\&pgtype=Article\&state=default\&region=MAIN_CONTENT_3\&context=storylines_faq}{makeshift
    child care} and
    \href{https://www.nytimes.com/2020/06/03/business/economy/coronavirus-working-women.html?action=click\&pgtype=Article\&state=default\&region=MAIN_CONTENT_3\&context=storylines_faq}{stunted
    workdays} to continue. California's two largest public school
    districts --- Los Angeles and San Diego --- said on July 13, that
    \href{https://www.nytimes.com/2020/07/13/us/lausd-san-diego-school-reopening.html?action=click\&pgtype=Article\&state=default\&region=MAIN_CONTENT_3\&context=storylines_faq}{instruction
    will be remote-only in the fall}, citing concerns that surging
    coronavirus infections in their areas pose too dire a risk for
    students and teachers. Together, the two districts enroll some
    825,000 students. They are the largest in the country so far to
    abandon plans for even a partial physical return to classrooms when
    they reopen in August. For other districts, the solution won't be an
    all-or-nothing approach.
    \href{https://bioethics.jhu.edu/research-and-outreach/projects/eschool-initiative/school-policy-tracker/}{Many
    systems}, including the nation's largest, New York City, are
    devising
    \href{https://www.nytimes.com/2020/06/26/us/coronavirus-schools-reopen-fall.html?action=click\&pgtype=Article\&state=default\&region=MAIN_CONTENT_3\&context=storylines_faq}{hybrid
    plans} that involve spending some days in classrooms and other days
    online. There's no national policy on this yet, so check with your
    municipal school system regularly to see what is happening in your
    community.
  \end{itemize}
\item ~
  \hypertarget{is-the-coronavirus-airborne}{%
  \paragraph{Is the coronavirus
  airborne?}\label{is-the-coronavirus-airborne}}

  \begin{itemize}
  \tightlist
  \item
    The coronavirus
    \href{https://www.nytimes.com/2020/07/04/health/239-experts-with-one-big-claim-the-coronavirus-is-airborne.html?action=click\&pgtype=Article\&state=default\&region=MAIN_CONTENT_3\&context=storylines_faq}{can
    stay aloft for hours in tiny droplets in stagnant air}, infecting
    people as they inhale, mounting scientific evidence suggests. This
    risk is highest in crowded indoor spaces with poor ventilation, and
    may help explain super-spreading events reported in meatpacking
    plants, churches and restaurants.
    \href{https://www.nytimes.com/2020/07/06/health/coronavirus-airborne-aerosols.html?action=click\&pgtype=Article\&state=default\&region=MAIN_CONTENT_3\&context=storylines_faq}{It's
    unclear how often the virus is spread} via these tiny droplets, or
    aerosols, compared with larger droplets that are expelled when a
    sick person coughs or sneezes, or transmitted through contact with
    contaminated surfaces, said Linsey Marr, an aerosol expert at
    Virginia Tech. Aerosols are released even when a person without
    symptoms exhales, talks or sings, according to Dr. Marr and more
    than 200 other experts, who
    \href{https://academic.oup.com/cid/article/doi/10.1093/cid/ciaa939/5867798}{have
    outlined the evidence in an open letter to the World Health
    Organization}.
  \end{itemize}
\item ~
  \hypertarget{what-are-the-symptoms-of-coronavirus}{%
  \paragraph{What are the symptoms of
  coronavirus?}\label{what-are-the-symptoms-of-coronavirus}}

  \begin{itemize}
  \tightlist
  \item
    Common symptoms
    \href{https://www.nytimes.com/article/symptoms-coronavirus.html?action=click\&pgtype=Article\&state=default\&region=MAIN_CONTENT_3\&context=storylines_faq}{include
    fever, a dry cough, fatigue and difficulty breathing or shortness of
    breath.} Some of these symptoms overlap with those of the flu,
    making detection difficult, but runny noses and stuffy sinuses are
    less common.
    \href{https://www.nytimes.com/2020/04/27/health/coronavirus-symptoms-cdc.html?action=click\&pgtype=Article\&state=default\&region=MAIN_CONTENT_3\&context=storylines_faq}{The
    C.D.C. has also} added chills, muscle pain, sore throat, headache
    and a new loss of the sense of taste or smell as symptoms to look
    out for. Most people fall ill five to seven days after exposure, but
    symptoms may appear in as few as two days or as many as 14 days.
  \end{itemize}
\item ~
  \hypertarget{does-asymptomatic-transmission-of-covid-19-happen}{%
  \paragraph{Does asymptomatic transmission of Covid-19
  happen?}\label{does-asymptomatic-transmission-of-covid-19-happen}}

  \begin{itemize}
  \tightlist
  \item
    So far, the evidence seems to show it does. A widely cited
    \href{https://www.nature.com/articles/s41591-020-0869-5}{paper}
    published in April suggests that people are most infectious about
    two days before the onset of coronavirus symptoms and estimated that
    44 percent of new infections were a result of transmission from
    people who were not yet showing symptoms. Recently, a top expert at
    the World Health Organization stated that transmission of the
    coronavirus by people who did not have symptoms was ``very rare,''
    \href{https://www.nytimes.com/2020/06/09/world/coronavirus-updates.html?action=click\&pgtype=Article\&state=default\&region=MAIN_CONTENT_3\&context=storylines_faq\#link-1f302e21}{but
    she later walked back that statement.}
  \end{itemize}
\end{itemize}

``This wasn't like a team of scientists in a bunch of suits,'' said
Chris J. Cohilas, chairman of the Dougherty County Board of
Commissioners. ``We're a big small town where everybody knows everybody.
We know who is in our hospital, and we know who went to what funeral.''

Word went out ``so quickly and so aggressively'' that those who attended
either of the funerals should get tested, Mr. Cohilas said. But not
quickly enough to prevent an infected person from serving as a juror
\href{https://www.albanyherald.com/news/jurors-finish-hearing-evidence-in-august-downtown-albany-slaying/article_84f921e6-63d9-11ea-a8fa-7b42ab2ccd4f.html}{in
a high-profile murder trial} that ended on March 12. That set off a new
set of infections in the sheriff's office and the courthouse, he said.

The warnings drove a wedge between people in Albany, said the Rev.
Daniel Simmons, the senior pastor of Albany's Mt. Zion Baptist Church,
who, like others interviewed, said he questioned whether the funerals
were in fact the sole source of the infections.

``It created fear: Who will be at the gathering that I'm going to on
Sunday, that funeral, or that wedding? Do I go? Do I not go?'' said Mr.
Simmons, whose church was not connected to either funeral. ``People
began to say, were you at the funeral? That became a question.''

The city's churches, he said, began to feel unfairly singled out.

``That is the focus: the church, the church,'' he said. ``It has done
damage because there is stigma. There is almost this wall of hostility
that has been raised between certain parts of the community and the
church.''

Ms. Johnson, whose family hosted the Feb. 29 funeral, said the
speculation had been painful.

``I have family members angry because people are saying that my brother
was the culprit,'' she said. ``He's a dead person. He's not even
breathing. But they're angry because the rumor mill is saying that he
was the spreader of the virus.''

Image

Dorothy Johnson's daughter, Tonya M. Thomas, in a photo provided by her
family.

\hypertarget{a-scramble-for-care}{%
\subsection{A scramble for care}\label{a-scramble-for-care}}

By last week, the question of how the virus had entered the county had
been eclipsed by the number of people sick and dying. Mr. Mitchell's
companion, Ms. Murray, has been hospitalized and discharged twice, the
last time on March 24, against her daughter's protests.

``I begged them not to let her come home, but they did it anyway,'' Ms.
Bell said. ``They brought her into this house like a sack of potatoes,
on a stretcher.''

Ms. Bell, 49, said she did not have the strength to turn her mother over
in bed, and had called repeatedly to ask for help.

``I'm begging for help,'' she said. ``I'm here with two kids, and I
don't know if I have been exposed to it.''

She feels, she said, ``as if they sent my mother home to die.''

Phoebe Putney is diverting patients to other Georgia hospitals at a rate
it has never before approached, transferring 40 in a recent 72-hour
period, Mr. Steiner said. But he denied that any gravely ill patients
have been sent home.

``Anybody we've discharged has been discharged appropriately,
clinically,'' he said.

Mr. Simmons said that many families are struggling to care for the sick
at home, and that for some, a sense of panic has begun to set in.

``Part of the control in life is thinking, if you needed help, you've
got somewhere to go,'' he said. ``When that is taken off the table, all
sense of control is gone, and hope starts fading.''

He read aloud text messages he received over the weekend. ``Please
continue to pray,'' one said. ``My mother, my grandmother and my
grandfather have been admitted to the ER with coronavirus symptoms.''

Then, later, ``My mother has died.''

For Ms. Johnson, only one person mattered last week.

Her daughter, Tonya M. Thomas, was all she thought of while she was in
the hospital. The illness had hit them almost simultaneously, but
unaccountably, her 51-year-old daughter was the worst hit, with double
pneumonia.

``I was trying to feel better so I could come up here and take care of
my daughter,'' said Ms. Johnson, an oncology nurse. ``I felt like if I
hadn't been in the hospital I could have advocated for her.''

She arrived in time, at 5:45 on Friday afternoon, to be with Ms. Thomas
as she died. She called her ``a beautiful spirit,'' her family's center.
Her best friend.

She unplugged her daughter's ventilators and removed the IV tubes from
her body.

Ms. Thomas's husband, son and sister were in the room.

``It just hurts so bad, I just don't understand it,'' Ms. Johnson said.
``We came together at a funeral of someone we love, and everyone came up
and got sick.''

Her daughter's funeral will be at the graveside, with no more than 10
people present, in accordance with social distancing regulation.

Advertisement

\protect\hyperlink{after-bottom}{Continue reading the main story}

\hypertarget{site-index}{%
\subsection{Site Index}\label{site-index}}

\hypertarget{site-information-navigation}{%
\subsection{Site Information
Navigation}\label{site-information-navigation}}

\begin{itemize}
\tightlist
\item
  \href{https://help.nytimes.com/hc/en-us/articles/115014792127-Copyright-notice}{©~2020~The
  New York Times Company}
\end{itemize}

\begin{itemize}
\tightlist
\item
  \href{https://www.nytco.com/}{NYTCo}
\item
  \href{https://help.nytimes.com/hc/en-us/articles/115015385887-Contact-Us}{Contact
  Us}
\item
  \href{https://www.nytco.com/careers/}{Work with us}
\item
  \href{https://nytmediakit.com/}{Advertise}
\item
  \href{http://www.tbrandstudio.com/}{T Brand Studio}
\item
  \href{https://www.nytimes.com/privacy/cookie-policy\#how-do-i-manage-trackers}{Your
  Ad Choices}
\item
  \href{https://www.nytimes.com/privacy}{Privacy}
\item
  \href{https://help.nytimes.com/hc/en-us/articles/115014893428-Terms-of-service}{Terms
  of Service}
\item
  \href{https://help.nytimes.com/hc/en-us/articles/115014893968-Terms-of-sale}{Terms
  of Sale}
\item
  \href{https://spiderbites.nytimes.com}{Site Map}
\item
  \href{https://help.nytimes.com/hc/en-us}{Help}
\item
  \href{https://www.nytimes.com/subscription?campaignId=37WXW}{Subscriptions}
\end{itemize}
