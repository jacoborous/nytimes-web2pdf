Sections

SEARCH

\protect\hyperlink{site-content}{Skip to
content}\protect\hyperlink{site-index}{Skip to site index}

\href{https://www.nytimes.com/section/movies}{Movies}

\href{https://myaccount.nytimes.com/auth/login?response_type=cookie\&client_id=vi}{}

\href{https://www.nytimes.com/section/todayspaper}{Today's Paper}

\href{/section/movies}{Movies}\textbar{}How `Honeyland' Became an Oscar
Game Changer

\href{https://nyti.ms/2GmysAo}{https://nyti.ms/2GmysAo}

\begin{itemize}
\item
\item
\item
\item
\item
\end{itemize}

Advertisement

\protect\hyperlink{after-top}{Continue reading the main story}

Supported by

\protect\hyperlink{after-sponsor}{Continue reading the main story}

\hypertarget{how-honeyland-became-an-oscar-game-changer}{%
\section{How `Honeyland' Became an Oscar Game
Changer}\label{how-honeyland-became-an-oscar-game-changer}}

The movie from North Macedonia is the first to be nominated for both
best documentary and best international feature.

\includegraphics{https://static01.nyt.com/images/2020/01/26/arts/26honeyland-lead/merlin_158307861_30bb4da4-e3bf-4bd1-99f3-ee416967bb6d-articleLarge.jpg?quality=75\&auto=webp\&disable=upscale}

\href{https://www.nytimes.com/by/sara-aridi}{\includegraphics{https://static01.nyt.com/images/2019/04/30/multimedia/author-sara-aridi/author-sara-aridi-thumbLarge.png}}

By \href{https://www.nytimes.com/by/sara-aridi}{Sara Aridi}

\begin{itemize}
\item
  Published Jan. 24, 2020Updated Jan. 27, 2020
\item
  \begin{itemize}
  \item
  \item
  \item
  \item
  \item
  \end{itemize}
\end{itemize}

The Oscar race is in full swing, and all eyes are on the big contenders:
``Joker''! ``Marriage Story''! ``Once Upon a Time \ldots{} in
Hollywood''!

Yet one little-known movie amid the star-studded titles has quietly
broken ground.

\href{https://www.nytimes.com/2019/07/25/movies/honeyland-review.html}{``Honeyland''}
is the first film to be nominated for best documentary and best
international feature (the category formerly known as best
foreign-language film). It follows Hatidze Muratova, a middle-aged
beekeeper whose peaceful life in the Macedonian countryside is disrupted
when a chaotic family moves in next door.

The movie premiered at the
\href{https://www.sundance.org/blogs/news/2019-sundance-film-festival-awards-announced}{Sundance
Film Festival} last year and came out on top with three awards,
including the grand jury prize for documentary in the world cinema
showcase.

It went on to win accolades at smaller festivals across the globe and is
still riding high. It has a 99 percent fresh rating on
\href{https://www.rottentomatoes.com/m/honeyland}{Rotten Tomatoes}, and,
in December, The New York Times critic A.O. Scott named it
\href{https://www.nytimes.com/2019/12/04/movies/best-films.html}{the
best movie of 2019}.

The film, he wrote, ``is nothing less than a found epic, a real-life
environmental allegory and, not least, a stinging comedy about the
age-old problem of inconsiderate neighbors.''

``Honeyland'' is the underdog in the international feature category.
It's the feature-length debut of the directors Tamara Kotevska and
Ljubomir Stefanov, and it's competing with two much-talked-about titles
by veteran filmmakers: Bong Joon Ho's comedy thriller,
\href{https://www.nytimes.com/2019/10/10/movies/parasite-review.html}{``Parasite,''}
and Pedro Almodóvar's drama,
\href{https://www.nytimes.com/2019/10/03/movies/pain-and-glory-review.html}{``Pain
and Glory.''}

In the documentary category, it's up against
\href{https://www.nytimes.com/2019/08/20/movies/american-factory-review.html}{``American
Factory,''} the first Netflix film from Barack and Michelle Obama's
production company.

So why has a movie about a poor woman in an isolated village from
little-known filmmakers resonated with viewers around the world?

At the outset, ``Honeyland'' captures Muratova going about her daily
life. We see her singing to her bees; selling her honey in Skopje, the
Macedonian capital; and caring for her ailing, octogenarian mother, who
is half-blind and hard of hearing.

Then we meet the new neighbors: Hussein Sam, his wife, their seven
children and their cattle and chickens. Where Muratova is calm and
cheerful, Sam's family is raucous and ill-tempered (not to mention
foul-mouthed). Their differences become problematic when Sam takes a
stab at beekeeping and breaks Muratova's golden rule: Leave half of the
honey for the bees.

Sam may come off as the villain --- his ways threaten the fate of both
Muratova and her bees. Then again, he's merely a father trying to
provide for his family and satisfy an impatient buyer. His predicament,
the directors said, is just one element that makes ``Honeyland'' a
universal story.

``The film works like a mirror,'' Kotevska said in a phone interview.
``Some people recognize themselves in Hatidze. Some recognize themselves
in the other family.''

Their quarrel propels the narrative forward. Then there are touching
moments between Muratova and her bedridden mother, who is acutely aware
of her daughter's heavy load. The filmmakers also capture the growing
bond between Muratova and one of Sam's sons, who often escapes into her
quiet world after shouting matches with his father.

The result is a nuanced tale that touches on loneliness, capitalism and
a dying way of life. Most of all, Stefanov said, he and Kotevska wanted
to show how greed functions on ``a very basic level'' --- in this case,
on a remote patch of land inhabited by a handful of people.

Critics have been singing the film's praises.
\href{https://www.latimes.com/entertainment-arts/movies/story/2019-07-25/honeyland-review-documentary-beekeeper}{The
Los Angeles Times} wrote that few documentaries ``have offered such an
intimately infuriating, methodically detailed allegory of the earth's
wonders being ravaged by the consequences of human greed.''

\href{https://www.hollywoodreporter.com/review/honeyland-review-1177112}{The
Hollywood Reporter wrote}: ``The chronicle that Stefanov and Kotevska
have distilled abounds in moments of unguarded discovery --- moments
that can be tender, humorous, rackety or serene.''

Still, Stefanov and Kotevska had no idea they would be heading to the
Oscars. ``After Sundance, it was clear that the film is good and people
love it,'' Stefanov said. ``But we didn't expect two nominations.''

They didn't even expect to tell the story to begin with.

The directors stumbled upon Muratova's beehives while doing research for
an environmental documentary. After meeting her, they were intrigued by
her beekeeping traditions, which go back generations.

They went on to shoot more than 400 hours of footage over the course of
three years, working in rough conditions. Muratova lived in a small,
ramshackle hut with no electricity. Stefanov and Kotevska would visit
for a few days at a time and sleep in tents. Their only plan was to wait
for compelling shots.

The movie's Oscar nomination for best international feature, Kotevska
said, is proof that fiction and nonfiction work should not be judged
separately. (Whether a documentary will ever be
\href{https://www.indiewire.com/2020/01/honeyland-oscar-noms-documentaries-academy-awards-1202202814/}{nominated
for best picture} is a different story.)

``Our understanding of film is that it shouldn't have boundaries,''
Kotevska said. ``Good storytelling is good storytelling.''

Advertisement

\protect\hyperlink{after-bottom}{Continue reading the main story}

\hypertarget{site-index}{%
\subsection{Site Index}\label{site-index}}

\hypertarget{site-information-navigation}{%
\subsection{Site Information
Navigation}\label{site-information-navigation}}

\begin{itemize}
\tightlist
\item
  \href{https://help.nytimes.com/hc/en-us/articles/115014792127-Copyright-notice}{©~2020~The
  New York Times Company}
\end{itemize}

\begin{itemize}
\tightlist
\item
  \href{https://www.nytco.com/}{NYTCo}
\item
  \href{https://help.nytimes.com/hc/en-us/articles/115015385887-Contact-Us}{Contact
  Us}
\item
  \href{https://www.nytco.com/careers/}{Work with us}
\item
  \href{https://nytmediakit.com/}{Advertise}
\item
  \href{http://www.tbrandstudio.com/}{T Brand Studio}
\item
  \href{https://www.nytimes.com/privacy/cookie-policy\#how-do-i-manage-trackers}{Your
  Ad Choices}
\item
  \href{https://www.nytimes.com/privacy}{Privacy}
\item
  \href{https://help.nytimes.com/hc/en-us/articles/115014893428-Terms-of-service}{Terms
  of Service}
\item
  \href{https://help.nytimes.com/hc/en-us/articles/115014893968-Terms-of-sale}{Terms
  of Sale}
\item
  \href{https://spiderbites.nytimes.com}{Site Map}
\item
  \href{https://help.nytimes.com/hc/en-us}{Help}
\item
  \href{https://www.nytimes.com/subscription?campaignId=37WXW}{Subscriptions}
\end{itemize}
