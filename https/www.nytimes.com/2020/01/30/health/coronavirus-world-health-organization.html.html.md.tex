Sections

SEARCH

\protect\hyperlink{site-content}{Skip to
content}\protect\hyperlink{site-index}{Skip to site index}

\href{https://www.nytimes.com/section/health}{Health}

\href{https://myaccount.nytimes.com/auth/login?response_type=cookie\&client_id=vi}{}

\href{https://www.nytimes.com/section/todayspaper}{Today's Paper}

\href{/section/health}{Health}\textbar{}W.H.O. Declares Global Emergency
as Wuhan Coronavirus Spreads

\url{https://nyti.ms/2RER70M}

\begin{itemize}
\item
\item
\item
\item
\item
\item
\end{itemize}

\href{https://www.nytimes.com/news-event/coronavirus?action=click\&pgtype=Article\&state=default\&region=TOP_BANNER\&context=storylines_menu}{The
Coronavirus Outbreak}

\begin{itemize}
\tightlist
\item
  live\href{https://www.nytimes.com/2020/08/02/world/coronavirus-updates.html?action=click\&pgtype=Article\&state=default\&region=TOP_BANNER\&context=storylines_menu}{Latest
  Updates}
\item
  \href{https://www.nytimes.com/interactive/2020/us/coronavirus-us-cases.html?action=click\&pgtype=Article\&state=default\&region=TOP_BANNER\&context=storylines_menu}{Maps
  and Cases}
\item
  \href{https://www.nytimes.com/interactive/2020/science/coronavirus-vaccine-tracker.html?action=click\&pgtype=Article\&state=default\&region=TOP_BANNER\&context=storylines_menu}{Vaccine
  Tracker}
\item
  \href{https://www.nytimes.com/interactive/2020/07/29/us/schools-reopening-coronavirus.html?action=click\&pgtype=Article\&state=default\&region=TOP_BANNER\&context=storylines_menu}{What
  School May Look Like}
\item
  \href{https://www.nytimes.com/live/2020/07/31/business/stock-market-today-coronavirus?action=click\&pgtype=Article\&state=default\&region=TOP_BANNER\&context=storylines_menu}{Economy}
\end{itemize}

Advertisement

\protect\hyperlink{after-top}{Continue reading the main story}

Supported by

\protect\hyperlink{after-sponsor}{Continue reading the main story}

\hypertarget{who-declares-global-emergency-as-wuhan-coronavirus-spreads}{%
\section{W.H.O. Declares Global Emergency as Wuhan Coronavirus
Spreads}\label{who-declares-global-emergency-as-wuhan-coronavirus-spreads}}

The announcement came as nearly 10,000 cases have been reported
worldwide.

\includegraphics{https://static01.nyt.com/images/2020/01/30/world/30who-decision01/30who-decision01-articleLarge.jpg?quality=75\&auto=webp\&disable=upscale}

\href{https://www.nytimes.com/by/sui-lee-wee}{\includegraphics{https://static01.nyt.com/images/2018/07/13/multimedia/author-sui-lee-wee/author-sui-lee-wee-thumbLarge.png}}\href{https://www.nytimes.com/by/donald-g-mcneil-jr}{\includegraphics{https://static01.nyt.com/images/2018/06/13/multimedia/author-donald-g-mcneil-jr/author-donald-g-mcneil-jr-thumbLarge-v4.png}}\href{https://www.nytimes.com/by/javier-c-hernandez}{\includegraphics{https://static01.nyt.com/images/2018/11/22/multimedia/author-javier-c-hernandez/author-javier-c-hernandez-thumbLarge-v3.png}}

By \href{https://www.nytimes.com/by/sui-lee-wee}{Sui-Lee Wee},
\href{https://www.nytimes.com/by/donald-g-mcneil-jr}{Donald G. McNeil
Jr.} and \href{https://www.nytimes.com/by/javier-c-hernandez}{Javier C.
Hernández}

\begin{itemize}
\item
  Published Jan. 30, 2020Updated April 16, 2020
\item
  \begin{itemize}
  \item
  \item
  \item
  \item
  \item
  \item
  \end{itemize}
\end{itemize}

The World Health Organization declared a global health emergency on
Thursday as the
\href{https://www.nytimes.com/2020/04/16/health/WHO-Trump-coronavirus.html}{coronavirus}
outbreak spread well beyond China, where it emerged last month.

The move reversed the organization's decision just a week ago to
\href{https://www.nytimes.com/2020/01/23/health/china-virus-who-emergency.html}{hold
off such a declaration}. Since then, there have been thousands of new
cases in China and clear evidence of human-to-human transmission
\href{https://www.nytimes.com/2020/01/29/health/china-coronavirus-outbreak.html}{in
several other countries}, including the United States.

All of which warranted a reconsideration by the W.H.O.'s emergency
committee, officials said.

The declaration ``is not a vote of no confidence in China,'' said Tedros
Adhanom Ghebreyesus, the W.H.O.'s director-general. ``On the contrary,
the
\href{https://www.nytimes.com/2020/04/16/health/WHO-Trump-coronavirus.html}{W.H.O.}
continues to have confidence in China's capacity to control the
outbreak.''

The declaration comes now, he said, because of fears that the
coronavirus may reach countries with weak health care systems, where it
could run amok, potentially infecting millions of people and killing
thousands.

Following the declaration, the State Department on Thursday night
\href{https://travel.state.gov/content/travel/en/traveladvisories/traveladvisories/china-travel-advisory.html}{warned
travelers to avoid China altogether}.

China's Foreign Ministry spokeswoman, Hua Chunying, said that the
country ``is fully confident and capable of winning the battle against
this epidemic.''

In a statement on the ministry's website, she added that China was
willing to continue to work with the W.H.O. and other countries to
safeguard public health.

The W.H.O.'s declaration --- officially called a ``public health
emergency of international concern'' --- does not have the force of law.

The agency is governed by an annual convocation of the health ministers
of all U.N. countries, and its role is only to offer advice. Governments
then make their own decisions about how they protect themselves.

\hypertarget{latest-updates-global-coronavirus-outbreak}{%
\section{\texorpdfstring{\href{https://www.nytimes.com/2020/08/01/world/coronavirus-covid-19.html?action=click\&pgtype=Article\&state=default\&region=MAIN_CONTENT_1\&context=storylines_live_updates}{Latest
Updates: Global Coronavirus
Outbreak}}{Latest Updates: Global Coronavirus Outbreak}}\label{latest-updates-global-coronavirus-outbreak}}

Updated 2020-08-02T17:52:35.962Z

\begin{itemize}
\tightlist
\item
  \href{https://www.nytimes.com/2020/08/01/world/coronavirus-covid-19.html?action=click\&pgtype=Article\&state=default\&region=MAIN_CONTENT_1\&context=storylines_live_updates\#link-34047410}{The
  U.S. reels as July cases more than double the total of any other
  month.}
\item
  \href{https://www.nytimes.com/2020/08/01/world/coronavirus-covid-19.html?action=click\&pgtype=Article\&state=default\&region=MAIN_CONTENT_1\&context=storylines_live_updates\#link-780ec966}{Top
  U.S. officials work to break an impasse over the federal jobless
  benefit.}
\item
  \href{https://www.nytimes.com/2020/08/01/world/coronavirus-covid-19.html?action=click\&pgtype=Article\&state=default\&region=MAIN_CONTENT_1\&context=storylines_live_updates\#link-2bc8948}{Its
  outbreak untamed, Melbourne goes into even greater lockdown.}
\end{itemize}

\href{https://www.nytimes.com/2020/08/01/world/coronavirus-covid-19.html?action=click\&pgtype=Article\&state=default\&region=MAIN_CONTENT_1\&context=storylines_live_updates}{See
more updates}

More live coverage:
\href{https://www.nytimes.com/live/2020/07/31/business/stock-market-today-coronavirus?action=click\&pgtype=Article\&state=default\&region=MAIN_CONTENT_1\&context=storylines_live_updates}{Markets}

States of emergency are ``merely guidance,'' said Dr. David L. Heymann,
a former W.H.O. assistant director-general who now analyzes the work of
the agency's emergency committee. Governments and even private companies
``may or may not follow it.''

Nonetheless, emergency declarations signal that the world's top health
advisory body thinks the situation is grave. Many scientific experts
welcomed the decision.

The public health emergency ``allows them to further lean into the role
of global leadership for governments and the private sector,'' said Dr.
Thomas R. Frieden, the former director of the Centers for Disease
Control and Prevention and a veteran of several global health
emergencies.

The first goal, he said, should be to understand more about how the
virus is spreading --- whether mostly in hospitals and clinics, what
ages and sexes or professions are most affected, how sick they become
and what risk factors are dangerous.

But Amir Attaran, a professor of law and epidemiology at the University
of Ottawa and a frequent W.H.O. critic, called the declaration
``inexcusably late.'' The committee's reasoning that it lacked enough
scientific evidence to declare an emergency last week was
``balderdash,'' he added.

``W.H.O. is paralyzed for the same political reasons that ruined its
scientific judgment in SARS, Ebola and Zika,'' he said. ``Borders are
closed, aircraft grounded and ships anchored as W.H.O. mutely dithers
over whether or not to declare an emergency.''

``Events have comprehensively overtaken them, proving their uselessness
yet again,'' he added.

\includegraphics{https://static01.nyt.com/images/2020/01/31/multimedia/china-wuhan-coronavirus-maps-promo/china-wuhan-coronavirus-maps-promo-articleLarge-v52.png?quality=75\&auto=webp\&disable=upscale}

Declaring emergencies is always a hard decision, Dr. Tedros said on
Wednesday. Border closings and flight cancellations may cause hardships
for millions of healthy people near the epicenter, and enormous economic
disruption.

In the worst cases, supplies of
\href{https://www.businessinsider.com/no-food-crowded-hospitals-wuhan-first-week-in-coronavirus-quarantine-2020-1}{food
and medicine can run short} and panic can spread, threatening to do more
damage than the disease.

Experts at the W.H.O. have lavishly and repeatedly praised China's
response as remarkably aggressive.

The country is building two hospitals, in just two weeks, to house
coronavirus patients. Chinese scientists deposited the genetic signature
of the coronavirus in public databases, greatly speeding the development
of diagnostic tests and, potentially, vaccines.

Chinese authorities cordoned off the major cities at the outbreak's
epicenter, Hubei Province,
\href{https://thehill.com/changing-america/well-being/longevity/479733-more-than-30-million-on-lockdown-amid-coronavirus}{stranding
more than 50 million people} at the height of the Lunar New Year
holidays --- a measure that few other countries could have undertaken.

Whether that massive cordon will prove effective remains to be seen.
Five million people were able to leave Wuhan, the city where the
outbreak began, before its train and bus stations and airports were
closed, the mayor said. There were soon outbreaks across the country.

Dr. Tedros praised the Chinese government, saying it ``is setting a new
standard for outbreak response.'' Other countries should be grateful
that only 98 of the nearly 10,000 cases confirmed so far have occurred
outside China's borders, he said.

Despite the emergency declaration, and despite the State Department's
urging Americans to stay away from China, the W.H.O. opposes
restrictions on travel to China or on trade with it.

Measures the agency considers unwarranted include border closures, visa
restrictions and the quarantining of apparently healthy visitors from
the affected regions, said the chairman of the agency's emergency
committee, Dr. Didier Houssin, an adviser to France's national health
security agency.

Many such measures have already been taken against China by other
countries, and Dr. Houssin said the W.H.O. would question the scientific
rationales behind them.

\href{https://www.nytimes.com/news-event/coronavirus?action=click\&pgtype=Article\&state=default\&region=MAIN_CONTENT_3\&context=storylines_faq}{}

\hypertarget{the-coronavirus-outbreak-}{%
\subsubsection{The Coronavirus Outbreak
›}\label{the-coronavirus-outbreak-}}

\hypertarget{frequently-asked-questions}{%
\paragraph{Frequently Asked
Questions}\label{frequently-asked-questions}}

Updated July 27, 2020

\begin{itemize}
\item ~
  \hypertarget{should-i-refinance-my-mortgage}{%
  \paragraph{Should I refinance my
  mortgage?}\label{should-i-refinance-my-mortgage}}

  \begin{itemize}
  \tightlist
  \item
    \href{https://www.nytimes.com/article/coronavirus-money-unemployment.html?action=click\&pgtype=Article\&state=default\&region=MAIN_CONTENT_3\&context=storylines_faq}{It
    could be a good idea,} because mortgage rates have
    \href{https://www.nytimes.com/2020/07/16/business/mortgage-rates-below-3-percent.html?action=click\&pgtype=Article\&state=default\&region=MAIN_CONTENT_3\&context=storylines_faq}{never
    been lower.} Refinancing requests have pushed mortgage applications
    to some of the highest levels since 2008, so be prepared to get in
    line. But defaults are also up, so if you're thinking about buying a
    home, be aware that some lenders have tightened their standards.
  \end{itemize}
\item ~
  \hypertarget{what-is-school-going-to-look-like-in-september}{%
  \paragraph{What is school going to look like in
  September?}\label{what-is-school-going-to-look-like-in-september}}

  \begin{itemize}
  \tightlist
  \item
    It is unlikely that many schools will return to a normal schedule
    this fall, requiring the grind of
    \href{https://www.nytimes.com/2020/06/05/us/coronavirus-education-lost-learning.html?action=click\&pgtype=Article\&state=default\&region=MAIN_CONTENT_3\&context=storylines_faq}{online
    learning},
    \href{https://www.nytimes.com/2020/05/29/us/coronavirus-child-care-centers.html?action=click\&pgtype=Article\&state=default\&region=MAIN_CONTENT_3\&context=storylines_faq}{makeshift
    child care} and
    \href{https://www.nytimes.com/2020/06/03/business/economy/coronavirus-working-women.html?action=click\&pgtype=Article\&state=default\&region=MAIN_CONTENT_3\&context=storylines_faq}{stunted
    workdays} to continue. California's two largest public school
    districts --- Los Angeles and San Diego --- said on July 13, that
    \href{https://www.nytimes.com/2020/07/13/us/lausd-san-diego-school-reopening.html?action=click\&pgtype=Article\&state=default\&region=MAIN_CONTENT_3\&context=storylines_faq}{instruction
    will be remote-only in the fall}, citing concerns that surging
    coronavirus infections in their areas pose too dire a risk for
    students and teachers. Together, the two districts enroll some
    825,000 students. They are the largest in the country so far to
    abandon plans for even a partial physical return to classrooms when
    they reopen in August. For other districts, the solution won't be an
    all-or-nothing approach.
    \href{https://bioethics.jhu.edu/research-and-outreach/projects/eschool-initiative/school-policy-tracker/}{Many
    systems}, including the nation's largest, New York City, are
    devising
    \href{https://www.nytimes.com/2020/06/26/us/coronavirus-schools-reopen-fall.html?action=click\&pgtype=Article\&state=default\&region=MAIN_CONTENT_3\&context=storylines_faq}{hybrid
    plans} that involve spending some days in classrooms and other days
    online. There's no national policy on this yet, so check with your
    municipal school system regularly to see what is happening in your
    community.
  \end{itemize}
\item ~
  \hypertarget{is-the-coronavirus-airborne}{%
  \paragraph{Is the coronavirus
  airborne?}\label{is-the-coronavirus-airborne}}

  \begin{itemize}
  \tightlist
  \item
    The coronavirus
    \href{https://www.nytimes.com/2020/07/04/health/239-experts-with-one-big-claim-the-coronavirus-is-airborne.html?action=click\&pgtype=Article\&state=default\&region=MAIN_CONTENT_3\&context=storylines_faq}{can
    stay aloft for hours in tiny droplets in stagnant air}, infecting
    people as they inhale, mounting scientific evidence suggests. This
    risk is highest in crowded indoor spaces with poor ventilation, and
    may help explain super-spreading events reported in meatpacking
    plants, churches and restaurants.
    \href{https://www.nytimes.com/2020/07/06/health/coronavirus-airborne-aerosols.html?action=click\&pgtype=Article\&state=default\&region=MAIN_CONTENT_3\&context=storylines_faq}{It's
    unclear how often the virus is spread} via these tiny droplets, or
    aerosols, compared with larger droplets that are expelled when a
    sick person coughs or sneezes, or transmitted through contact with
    contaminated surfaces, said Linsey Marr, an aerosol expert at
    Virginia Tech. Aerosols are released even when a person without
    symptoms exhales, talks or sings, according to Dr. Marr and more
    than 200 other experts, who
    \href{https://academic.oup.com/cid/article/doi/10.1093/cid/ciaa939/5867798}{have
    outlined the evidence in an open letter to the World Health
    Organization}.
  \end{itemize}
\item ~
  \hypertarget{what-are-the-symptoms-of-coronavirus}{%
  \paragraph{What are the symptoms of
  coronavirus?}\label{what-are-the-symptoms-of-coronavirus}}

  \begin{itemize}
  \tightlist
  \item
    Common symptoms
    \href{https://www.nytimes.com/article/symptoms-coronavirus.html?action=click\&pgtype=Article\&state=default\&region=MAIN_CONTENT_3\&context=storylines_faq}{include
    fever, a dry cough, fatigue and difficulty breathing or shortness of
    breath.} Some of these symptoms overlap with those of the flu,
    making detection difficult, but runny noses and stuffy sinuses are
    less common.
    \href{https://www.nytimes.com/2020/04/27/health/coronavirus-symptoms-cdc.html?action=click\&pgtype=Article\&state=default\&region=MAIN_CONTENT_3\&context=storylines_faq}{The
    C.D.C. has also} added chills, muscle pain, sore throat, headache
    and a new loss of the sense of taste or smell as symptoms to look
    out for. Most people fall ill five to seven days after exposure, but
    symptoms may appear in as few as two days or as many as 14 days.
  \end{itemize}
\item ~
  \hypertarget{does-asymptomatic-transmission-of-covid-19-happen}{%
  \paragraph{Does asymptomatic transmission of Covid-19
  happen?}\label{does-asymptomatic-transmission-of-covid-19-happen}}

  \begin{itemize}
  \tightlist
  \item
    So far, the evidence seems to show it does. A widely cited
    \href{https://www.nature.com/articles/s41591-020-0869-5}{paper}
    published in April suggests that people are most infectious about
    two days before the onset of coronavirus symptoms and estimated that
    44 percent of new infections were a result of transmission from
    people who were not yet showing symptoms. Recently, a top expert at
    the World Health Organization stated that transmission of the
    coronavirus by people who did not have symptoms was ``very rare,''
    \href{https://www.nytimes.com/2020/06/09/world/coronavirus-updates.html?action=click\&pgtype=Article\&state=default\&region=MAIN_CONTENT_3\&context=storylines_faq\#link-1f302e21}{but
    she later walked back that statement.}
  \end{itemize}
\end{itemize}

Neither he nor Dr. Tedros questioned the decision by the United States
and other countries to evacuate their citizens from China, and they said
the cancellation of flights by airlines was justified if the real reason
was that they had no passengers.

Dr. Tedros, who met with President Xi Jinping in Beijing on Tuesday,
said he was struck by how much Mr. Xi knew about the outbreak and by the
fact that Ma Xiaowei, director of China's National Health Commission,
was on the ground in Wuhan leading the response.

That is not say there haven't been missteps. The W.H.O. last week
described its risk assessment for the outbreak as ``moderate,'' when it
should have said ``high.'' The error was corrected in a footnote to the
agency's report; Dr. Tedros described it on Twitter as a ``human
error.''

Experts in the United States have complained of spotty epidemiological
information from China.

Also, the W.H.O. cannot share information with Taiwan, which now has
eight coronavirus patients, because Taiwan is not a member of the United
Nations.

The agency ``doesn't want to upset its major stakeholders,'' said
Charles Clift, a senior consulting fellow at the Chatham House, an
international affairs research group in London. ``China carries the
political clout that other countries don't.''

And the outbreak seems to accelerating, along with its consequences.
China said on Friday that another 43 people had died from the disease,
bringing the total to 213. No deaths have yet occurred outside China.

On Thursday, Russia closed much of its 2,600-mile border with China and
stopped all train service between the countries, except for a regular
train between Moscow and Beijing. Some airlines, including British
Airways, have stopped flying there; others are greatly reducing their
service.

Within China, some medical experts have questioned their country's
response, arguing that local officials should have imposed stricter
travel restrictions before the virus spilled out of Wuhan. The country
has now confirmed cases in every province and region.

Residents have complained that local authorities kept mum about the
outbreak's severity --- initially insisting that there was no evidence
of person-to-person transmission outside Wuhan --- but admitted the
truth after press reports in Hong Kong.

As the dimensions of the outbreak became clear, the mayor of Wuhan on
Monday offered to resign.

A W.H.O. delegation was allowed to visit Wuhan for just one day. Dr.
Gauden Galea, the organization's representative in Beijing, said the
visit was not intended ``to pass judgment.''

``Everything is being done with a sense of intensity, and to our
assessment, good practice,'' he added. ``I don't want to be an
apologist. You have to understand the large scale and the
comprehensiveness of the operation.''

Following the trip, China agreed to permit international experts
coordinated by the W.H.O. into the country to work with Chinese
scientists on containing the epidemic. The C.D.C. is assembling a team
to join them.

The W.H.O. has made just five emergency declarations since its power to
do so was established in 2005: for the pandemic influenza in 2009; a
polio resurgence in 2014; the Ebola epidemic in West Africa that year;
the Zika virus outbreak in 2016; and an Ebola outbreak in the Democratic
Republic of Congo last year.

Dr. Peter Piot, director of the London School of Hygiene and Tropical
Medicine and one of the discoverers of the Ebola virus and the presence
of AIDS in Africa, agreed with the W.H.O.'s emergency declaration but
felt the process was flawed.

``It is time for the W.H.O. to change its all-or-nothing, binary
approach'' to declaring an emergency, Dr. Piot said. ``In every
emergency, there is a spectrum of alert levels.''

Dr. Tedros expressed the same frustration at a news conference on
Wednesday, suggesting that the agency might need to switch to a
graduated ``green-yellow-red" system.

Austin Ramzy contributed reporting from Hong Kong, and Chris Horton from
Taipei. Elsie Chen contributed research.

Advertisement

\protect\hyperlink{after-bottom}{Continue reading the main story}

\hypertarget{site-index}{%
\subsection{Site Index}\label{site-index}}

\hypertarget{site-information-navigation}{%
\subsection{Site Information
Navigation}\label{site-information-navigation}}

\begin{itemize}
\tightlist
\item
  \href{https://help.nytimes.com/hc/en-us/articles/115014792127-Copyright-notice}{©~2020~The
  New York Times Company}
\end{itemize}

\begin{itemize}
\tightlist
\item
  \href{https://www.nytco.com/}{NYTCo}
\item
  \href{https://help.nytimes.com/hc/en-us/articles/115015385887-Contact-Us}{Contact
  Us}
\item
  \href{https://www.nytco.com/careers/}{Work with us}
\item
  \href{https://nytmediakit.com/}{Advertise}
\item
  \href{http://www.tbrandstudio.com/}{T Brand Studio}
\item
  \href{https://www.nytimes.com/privacy/cookie-policy\#how-do-i-manage-trackers}{Your
  Ad Choices}
\item
  \href{https://www.nytimes.com/privacy}{Privacy}
\item
  \href{https://help.nytimes.com/hc/en-us/articles/115014893428-Terms-of-service}{Terms
  of Service}
\item
  \href{https://help.nytimes.com/hc/en-us/articles/115014893968-Terms-of-sale}{Terms
  of Sale}
\item
  \href{https://spiderbites.nytimes.com}{Site Map}
\item
  \href{https://help.nytimes.com/hc/en-us}{Help}
\item
  \href{https://www.nytimes.com/subscription?campaignId=37WXW}{Subscriptions}
\end{itemize}
