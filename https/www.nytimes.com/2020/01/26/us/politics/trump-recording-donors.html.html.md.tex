Sections

SEARCH

\protect\hyperlink{site-content}{Skip to
content}\protect\hyperlink{site-index}{Skip to site index}

\href{https://www.nytimes.com/section/politics}{Politics}

\href{https://myaccount.nytimes.com/auth/login?response_type=cookie\&client_id=vi}{}

\href{https://www.nytimes.com/section/todayspaper}{Today's Paper}

\href{/section/politics}{Politics}\textbar{}Recording Shows That the
Swamp Has Not Been Drained

\url{https://nyti.ms/38IdtEy}

\begin{itemize}
\item
\item
\item
\item
\item
\end{itemize}

\begin{itemize}
\item
  \href{https://www.nytimes.com/2020/07/31/us/elections/biden-vs-trump.html?action=click\&pgtype=Article\&state=default\&region=TOP_BANNER\&context=storylines_menu}{Election
  Updates}
\item
  \href{https://www.nytimes.com/article/biden-vice-president-2020.html?action=click\&pgtype=Article\&state=default\&region=TOP_BANNER\&context=storylines_menu}{Biden's
  V.P. Search}
\item
  \href{https://www.nytimes.com/interactive/2020/07/24/us/politics/trump-biden-campaign-donors.html?action=click\&pgtype=Article\&state=default\&region=TOP_BANNER\&context=storylines_menu}{Map
  of Donations}
\item
  \href{https://www.nytimes.com/interactive/2020/us/elections/delegate-count-primary-results.html?action=click\&pgtype=Article\&state=default\&region=TOP_BANNER\&context=storylines_menu}{Delegate
  Count}
\item
  \href{https://www.nytimes.com/interactive/2019/us/politics/2020-presidential-candidates.html?action=click\&pgtype=Article\&state=default\&region=TOP_BANNER\&context=storylines_menu}{The
  Candidates}
\item
  \href{https://www.nytimes.com/newsletters/politics?action=click\&pgtype=Article\&state=default\&region=TOP_BANNER\&context=storylines_menu}{Politics
  Newsletter}
\end{itemize}

Advertisement

\protect\hyperlink{after-top}{Continue reading the main story}

Supported by

\protect\hyperlink{after-sponsor}{Continue reading the main story}

News Analysis

\hypertarget{recording-shows-that-the-swamp-has-not-been-drained}{%
\section{Recording Shows That the Swamp Has Not Been
Drained}\label{recording-shows-that-the-swamp-has-not-been-drained}}

The recording of a dinner President Trump held with big-dollar donors in
2018 demonstrates just how alive and well special-interest access and
influence remains in Washington.

\includegraphics{https://static01.nyt.com/images/2020/01/26/us/politics/26dc-lobby1/26dc-lobby1-articleLarge.jpg?quality=75\&auto=webp\&disable=upscale}

\href{https://www.nytimes.com/by/kenneth-p-vogel}{\includegraphics{https://static01.nyt.com/images/2018/02/20/multimedia/author-kenneth-p-vogel/author-kenneth-p-vogel-thumbLarge-v3.png}}\href{https://www.nytimes.com/by/eric-lipton}{\includegraphics{https://static01.nyt.com/images/2018/12/06/multimedia/author-eric-lipton/author-eric-lipton-thumbLarge.png}}

By \href{https://www.nytimes.com/by/kenneth-p-vogel}{Kenneth P. Vogel}
and \href{https://www.nytimes.com/by/eric-lipton}{Eric Lipton}

\begin{itemize}
\item
  Published Jan. 26, 2020Updated July 6, 2020
\item
  \begin{itemize}
  \item
  \item
  \item
  \item
  \item
  \end{itemize}
\end{itemize}

WASHINGTON --- It became such a central slogan of
\href{https://www.nytimes.com/2020/07/06/us/politics/trump-lobbyists-swamp-campaign.html}{Donald
J. Trump's} 2016 campaign that at rallies his supporters would chant the
three words representing his pledge to take on big donors and special
interests: ``Drain the swamp.''

But as President Trump ramps up his 2020 re-election bid, it is clear
that he has tolerated if not fostered a
\href{https://www.nytimes.com/2020/07/06/us/politics/trump-lobbyists-swamp-campaign.html}{swamp}
of his own in Washington, granting up-close access to deep-pocketed
supporters and interest groups willing to write six- and seven-figure
checks to his political operation. Some have used the opportunity to
plead their cases directly to him.

The latest evidence came over the weekend, with the
\href{https://www.nytimes.com/2020/01/25/us/politics/trump-ukraine-donors.html}{release
of a secret recording} of an April 2018 dinner for major donors and
prospective donors to a super PAC supporting Mr. Trump.

While news of the recording primarily focused on Mr. Trump's
\href{https://www.nytimes.com/2020/01/24/us/politics/trump-recording-yovanovitch.html}{call
for the removal of Marie L. Yovanovitch} as ambassador to Ukraine after
a donor claimed she had disparaged the president, the recording revealed
that Mr. Trump engaged in policy discussions with many other donors
pushing their own agendas.

There was the New York real estate developer whose company's project in
South Korea was proposed to Mr. Trump as a possible site for his summit
with Kim Jong-un, the leader of North Korea.

There was the Canadian steel magnate who pushed the president to further
limit steel imports to the United States, and whose companies donated
\$1.75 million to the super PAC.

Other attendees discussed government policies that could benefit their
businesses, including building a highway for self-driving trucks and
regulations that would help make trucks powered by gas compressors to be
more competitive with electric-powered vehicles.

The recording is a glimpse into a broader pattern in which the
administration appointed industry lobbyists to key policymaking jobs,
heeded the deregulatory wishes of big corporations and granted regular
access to donors and influential political supporters. Some of the
policies sought by the donors at the 2018 dinner have been subsequently
introduced in Congress; it is unclear in those cases whether the
president or the White House intervened.

\includegraphics{https://static01.nyt.com/images/2020/01/28/us/politics/trump-dinner/trump-dinner-videoSixteenByNineJumbo1600.jpg}

In other cases, Mr. Trump has directly championed the businesses of some
of his biggest donors, as he did in the weeks after his inauguration
when he
\href{https://features.propublica.org/trump-inc-podcast/sheldon-adelson-casino-magnate-trump-macau-and-japan/}{reportedly
discussed} with Prime Minister Shinzo Abe of Japan an effort by the
casino magnate Sheldon Adelson to build a casino there.

Mr. Trump's assiduous courtship of major donors closely mirrors behavior
for which he chastised his opponents in 2016, when he cast himself as a
billionaire whose ability to finance his own campaign would ensure that
he was not beholden to financial backers.

\hypertarget{latest-updates-2020-election}{%
\section{\texorpdfstring{\href{https://www.nytimes.com/2020/07/31/us/elections/biden-vs-trump.html?action=click\&pgtype=Article\&state=default\&region=MAIN_CONTENT_1\&context=storylines_live_updates}{Latest
Updates: 2020
Election}}{Latest Updates: 2020 Election}}\label{latest-updates-2020-election}}

Updated 2020-08-01T01:26:45.732Z

\begin{itemize}
\tightlist
\item
  \href{https://www.nytimes.com/2020/07/31/us/elections/biden-vs-trump.html?action=click\&pgtype=Article\&state=default\&region=MAIN_CONTENT_1\&context=storylines_live_updates\#link-29fdff45}{Kamala
  Harris, a top vice-presidential contender, confronts double
  standards.}
\item
  \href{https://www.nytimes.com/2020/07/31/us/elections/biden-vs-trump.html?action=click\&pgtype=Article\&state=default\&region=MAIN_CONTENT_1\&context=storylines_live_updates\#link-13ec3d9c}{Karen
  Bass and Susan Rice are rising on Biden's vice-presidential
  shortlist.}
\item
  \href{https://www.nytimes.com/2020/07/31/us/elections/biden-vs-trump.html?action=click\&pgtype=Article\&state=default\&region=MAIN_CONTENT_1\&context=storylines_live_updates\#link-49e9a016}{Trump
  says Russian bounties to kill U.S. troops `never took place.'}
\end{itemize}

\href{https://www.nytimes.com/2020/07/31/us/elections/biden-vs-trump.html?action=click\&pgtype=Article\&state=default\&region=MAIN_CONTENT_1\&context=storylines_live_updates}{See
more updates}

In the months after starting his presidential campaign, Mr. Trump
branded his Republican rivals, as well as his eventual Democratic
challenger, Hillary Clinton, as
``\href{https://twitter.com/realDonaldTrump/status/627841345789558788}{puppets}''
of major donors who funded their campaigns and supportive super PACs.

In one characteristic broadside at his rivals in late 2015, he assailed
Jeb Bush, the former Florida governor, and Senator Marco Rubio, also of
Florida, both of whom were seeking the Republican nomination at the
time, for their embrace of super PACs funded by major donors.

``And you look at Hillary --- let's go to the other side --- they have
super PACs, where they control the candidate just like you control a
puppet,''
\href{https://www.politico.com/story/2015/11/donald-trump-sheldon-adelson-paul-singer-koch-brothers-215540}{Mr.
Trump said}. ``We don't want anybody to form super PACs for me. We sent
legal notices: `Please give all the money back.' We don't want it.''

It was not long before Mr. Trump reversed himself.

His campaign began aggressively courting donations to supplement the
personal money he was spending on his 2016 bid, and his team eventually
blessed the formation of a super PAC that solicited large checks from
major donors to
\href{https://www.nytimes.com/2016/06/08/us/politics/super-pac-supporting-donald-trump-makes-first-ad-buy-for-general-election.html}{air
ads attacking Mrs. Clinton}.

Once elected, Mr. Trump's team signaled that he did not intend to spend
his own money on his re-election. His allies formed a pair of political
groups using variations of the name America First that could accept
unlimited donations. He began appearing at events for donors, the most
generous of whom were
\href{https://www.politico.com/story/2017/06/14/trump-donors-comey-testimony-239570}{invited
to the White House} for briefings with top administration officials.

He has attended many donor gatherings and fund-raisers that have been
held at the Trump International Hotel in Washington, including the
dinner that was the subject of the recording released over the weekend.
Held in a private suite on April 30, 2018, it was for donors and
prospective donors to America First Action, a super PAC that has raised
nearly \$50 million to support Mr. Trump and allied candidates.

The recording, which includes video at times, shows Mr. Trump entering
the suite and posing for some photographs before joining donors in a
dining room with 16 plush chairs around an ornately set table accented
with floral arrangements.

Mr. Trump updated the donors on some of the most pressing issues facing
his administration, including its ongoing negotiations with China over
trade and North Korea over nuclear weapons. He seemed to encourage the
donors to share their concerns.

\includegraphics{https://static01.nyt.com/images/2020/01/26/us/politics/26dc-lobby3/26dc-lobby3-articleLarge.jpg?quality=75\&auto=webp\&disable=upscale}

Mr. Trump mentioned to the donors that his administration had selected a
date and a location for his first meeting with Mr. Kim, which
\href{https://www.nytimes.com/2018/06/12/world/asia/north-korea-summit.html}{would
be held in Singapore} in the weeks after the dinner. One of the dinner
attendees suggested a different site for the summit: a so-called smart
development outside Seoul, South Korea, called
\href{https://www.wsj.com/articles/developer-feuds-with-korean-partner-over-busted-smart-city-11560261729}{Songdo},
featuring a convention center, apartments and a golf course designed by
the golfer Jack Nicklaus.

A leading stakeholder in the development was a company run by Stanley C.
Gale, a donor to Republican campaigns and committees who attended the
dinner, according to people familiar with the event. It was also
attended by the golfer's grandson and namesake, Jack Nicklaus III, who
works for Mr. Gale's company, according to a LinkedIn profile. Mr. Gale
did not respond to a request for comment.

During the discussion, Mr. Trump told the guests, ``You know that Kim
Jong-un is a great golfer.'' His remark prompted laughter and led
another guest to suggest that Mr. Kim's scores were recorded as all
holes-in-one in his authoritarian country.

Another guest was
\href{https://www.nytimes.com/2019/05/20/us/politics/hes-one-of-the-biggest-backers-of-trumps-push-to-protect-american-steel-and-hes-canadian.html}{Barry
Zekelman}, a Canadian citizen who owned a United States-based steel-tube
manufacturing company that
\href{https://www.nytimes.com/2019/05/20/us/politics/hes-one-of-the-biggest-backers-of-trumps-push-to-protect-american-steel-and-hes-canadian.html}{donated
\$1.75 million to America First Action}, avoiding running afoul of a ban
on foreign donations in American politics. He used the dinner to push
the president on two challenges facing his company: cheap steel tube
imports from Asia and new federal rules that made it harder to find
truck drivers.

He urged Mr. Trump to go further in his effort to limit steel imports to
the United States and questioned the rules intended to prevent fatal
truck accidents by using electronic monitoring systems to limit the
hours drivers could be on the road.

``Say someone is half an hour from home on their long-haul truck ---
they literally have to pull over on the side of the road and stop,'' Mr.
Zekelman said. ``They can't go home.''

Mr. Trump did not seem to be aware of the new federal rules that
required those monitoring systems.

``They have a method that you shut down a truck?'' Mr. Trump said, after
Mr. Zekelman questioned the effect the new rules had on his ability to
move the steel pipe he manufactured. ``Wow.''

Image

Barry Zekelman, left, spoke to an employee at the Atlas Tube steel plant
in Harrow, Ontario, in April.Credit...Mark Felix for The New York Times

Since that dinner,
\href{https://www.congress.gov/bill/116th-congress/house-bill/1697?q=\%7B\%22search\%22\%3A\%5B\%22S.+1463\%22\%5D\%7D\&s=1\&r=51}{legislation
has been introduced} in the House with the cosponsorship of 12
Republicans, including the brother of Vice President Mike Pence, to
allow smaller trucking companies to get exemptions from the rule.

Legislation has also been introduced to help natural gas vehicles
compete with electric ones. It was
\href{https://www.prnewswire.com/news-releases/vng-applauds-legislation-ushering-in-new-era-for-light-duty-natural-gas-vehicles-300683483.html}{applauded
by an Ohio company} that makes gas compressors, Ariel Corporation. One
of its executives, Thomas Rastin, was on the invitation list for the
April dinner. He and a woman resembling his wife, Karen Buchwald Wright,
who owns Ariel Corporation, are briefly visible in the video of the
event. Together, the couple have donated a combined \$875,000 to America
First Action. He did not respond to questions about whether he was the
voice on the recording urging the president to take steps to help the
industry.

Another invitee was Wayne Hoovestol, who owns trucking companies in the
Midwest, including one that works with the United States Postal Service.
On the recording, a male voice says he runs a company that does business
with the Postal Service and urges Mr. Trump to consider supporting the
construction of a 500-mile section of highway to be used exclusively by
self-driving trucks.

Paying truck drivers, the voice said, was one of his company's biggest
costs.

``All the technology is there, right now,'' he said. ``It is absolutely
safe.''

A limited liability company that shared an address and personnel with
one of Mr. Hoovestol's companies donated \$250,000 to America First
Action on the day of the dinner.

Mr. Hoovestol did not respond to a request for comment.

The recording was made by a dinner attendee, Igor Fruman, and was
released by the lawyer for another, Lev Parnas, an associate of Mr.
Fruman.

The two, both Soviet-born American businessmen, would go on to play
central roles in the pressure campaign against Ukraine that led to Mr.
Trump's impeachment.

During the dinner, Mr. Parnas and Mr. Fruman discussed with Mr. Trump a
natural gas venture they were pursuing in Ukraine. Mr. Parnas also asked
the president to consider changing banking regulations to aid another
business venture they would soon pursue: a
\href{https://www.nytimes.com/2019/10/23/us/kukushkin-giuliani-russia-cannabis-marijuana.html}{plan
to win marijuana retail licenses} in Nevada and elsewhere.

The month after the dinner, they
\href{http://docquery.fec.gov/cgi-bin/fecimg/?201807159115673127}{donated
\$325,000 to America First Action} through Global Energy Producers, a
company they had recently formed to pursue energy deals.

The men have since been indicted on campaign finance charges related to
their business ventures and have pleaded not guilty.

Ben Protess contributed reporting from New York.

\hypertarget{our-2020-election-guide}{%
\section{Our 2020 Election Guide}\label{our-2020-election-guide}}

Updated July 31, 2020

\begin{itemize}
\item
  \begin{center}\rule{0.5\linewidth}{\linethickness}\end{center}

  \hypertarget{the-latest}{%
  \subsection{The Latest}\label{the-latest}}

  \begin{itemize}
  \tightlist
  \item
    President Trump's assault on the Postal Service is intersecting with
    his attacks on mail-in voting.
    \href{https://www.nytimes.com/2020/07/31/us/politics/trump-usps-mail-delays.html?action=click\&pgtype=Article\&state=default\&region=BELOW_MAIN_CONTENT\&context=storylines_guide}{Voting
    rights groups say it is a recipe for disaster.}
  \end{itemize}
\item
  \begin{center}\rule{0.5\linewidth}{\linethickness}\end{center}

  \hypertarget{bidens-vp-search}{%
  \subsection{Biden's V.P. Search}\label{bidens-vp-search}}

  \begin{itemize}
  \tightlist
  \item
    \href{https://www.nytimes.com/article/biden-vice-president-2020.html?action=click\&pgtype=Article\&state=default\&region=BELOW_MAIN_CONTENT\&context=storylines_guide}{Here
    are 13 women} who have been under consideration to be Joe Biden's
    running mate, and why each might be chosen --- and might not be.
  \end{itemize}
\item
  \begin{center}\rule{0.5\linewidth}{\linethickness}\end{center}

  \hypertarget{keep-up-with-our-coverage}{%
  \subsection{Keep Up With Our
  Coverage}\label{keep-up-with-our-coverage}}

  \begin{itemize}
  \tightlist
  \item
    Get an
    \href{https://www.nytimes.com/newsletters/politics?action=click\&pgtype=Article\&state=default\&region=BELOW_MAIN_CONTENT\&context=storylines_guide}{email}
    recapping the day's news
  \end{itemize}

  \begin{itemize}
  \tightlist
  \item
    Download our mobile app on
    \href{https://apps.apple.com/us/app/nytimes/id284862083?ls=1\&mat_click_id=5c79ae7455014fd1bd66b5610c05b8f2-20191112-16948\&referrer=mat_click_id\%3D5c79ae7455014fd1bd66b5610c05b8f2-20191112-16948\%26link_click_id\%3D722930677036718082}{iOS}
    and
    \href{http://a.localytics.com/android?id=com.nytimes.android\&referrer=utm_source\%3Dother_nyt_mobile_web\%26utm_medium\%3DWeb\%2520page\%26utm_term\%3DGeneral\%2520Mobile\%2520Page\%26utm_campaign\%3DNYT\%2520Mobile\%2520General\%2520Page}{Android}
    and turn on Breaking News and Politics alerts
  \end{itemize}
\end{itemize}

Advertisement

\protect\hyperlink{after-bottom}{Continue reading the main story}

\hypertarget{site-index}{%
\subsection{Site Index}\label{site-index}}

\hypertarget{site-information-navigation}{%
\subsection{Site Information
Navigation}\label{site-information-navigation}}

\begin{itemize}
\tightlist
\item
  \href{https://help.nytimes.com/hc/en-us/articles/115014792127-Copyright-notice}{©~2020~The
  New York Times Company}
\end{itemize}

\begin{itemize}
\tightlist
\item
  \href{https://www.nytco.com/}{NYTCo}
\item
  \href{https://help.nytimes.com/hc/en-us/articles/115015385887-Contact-Us}{Contact
  Us}
\item
  \href{https://www.nytco.com/careers/}{Work with us}
\item
  \href{https://nytmediakit.com/}{Advertise}
\item
  \href{http://www.tbrandstudio.com/}{T Brand Studio}
\item
  \href{https://www.nytimes.com/privacy/cookie-policy\#how-do-i-manage-trackers}{Your
  Ad Choices}
\item
  \href{https://www.nytimes.com/privacy}{Privacy}
\item
  \href{https://help.nytimes.com/hc/en-us/articles/115014893428-Terms-of-service}{Terms
  of Service}
\item
  \href{https://help.nytimes.com/hc/en-us/articles/115014893968-Terms-of-sale}{Terms
  of Sale}
\item
  \href{https://spiderbites.nytimes.com}{Site Map}
\item
  \href{https://help.nytimes.com/hc/en-us}{Help}
\item
  \href{https://www.nytimes.com/subscription?campaignId=37WXW}{Subscriptions}
\end{itemize}
