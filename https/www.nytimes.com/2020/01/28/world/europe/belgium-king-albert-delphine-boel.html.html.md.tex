Sections

SEARCH

\protect\hyperlink{site-content}{Skip to
content}\protect\hyperlink{site-index}{Skip to site index}

\href{https://www.nytimes.com/section/world/europe}{Europe}

\href{https://myaccount.nytimes.com/auth/login?response_type=cookie\&client_id=vi}{}

\href{https://www.nytimes.com/section/todayspaper}{Today's Paper}

\href{/section/world/europe}{Europe}\textbar{}Ex-King of Belgium
Acknowledges a Long-Dismissed Daughter

\url{https://nyti.ms/2uAjJiN}

\begin{itemize}
\item
\item
\item
\item
\item
\end{itemize}

Advertisement

\protect\hyperlink{after-top}{Continue reading the main story}

Supported by

\protect\hyperlink{after-sponsor}{Continue reading the main story}

\hypertarget{ex-king-of-belgium-acknowledges-a-long-dismissed-daughter}{%
\section{Ex-King of Belgium Acknowledges a Long-Dismissed
Daughter}\label{ex-king-of-belgium-acknowledges-a-long-dismissed-daughter}}

After a court-ordered DNA test, the 85-year-old King Albert II, who
abdicated in 2013, conceded that he was the biological father of the
artist Delphine Boël, 51.

\includegraphics{https://static01.nyt.com/images/2020/01/28/world/28belgium-king02/merlin_165178536_56c39095-a981-4e1c-800a-f2854cca27f5-articleLarge.jpg?quality=75\&auto=webp\&disable=upscale}

\href{https://www.nytimes.com/by/aurelien-breeden}{\includegraphics{https://static01.nyt.com/images/2017/03/14/world/aurelien-breeden/aurelien-breeden-thumbLarge-v2.png}}

By \href{https://www.nytimes.com/by/aurelien-breeden}{Aurelien Breeden}

\begin{itemize}
\item
  Jan. 28, 2020
\item
  \begin{itemize}
  \item
  \item
  \item
  \item
  \item
  \end{itemize}
\end{itemize}

King Albert II, the former Belgian monarch, conceded this week that DNA
tests showed he was the biological father of Delphine Boël, an artist
who claimed to be his daughter from an extramarital affair, an
extraordinary admission after years of lawsuits that
\href{https://www.nytimes.com/2013/07/20/world/europe/belgium-also-awaiting-possible-news-of-a-new-royal.html}{exposed
the royal family to unusual levels of scrutiny}.

King Albert's lawyers
\href{https://www.lesoir.be/275765/article/2020-01-27/le-roi-albert-ii-reconnait-etre-le-pere-biologique-de-delphine-boel-le}{said
in a statement on Monday} that the 85-year-old former monarch, who
initially refused to comply with
\href{https://www.nytimes.com/2018/11/05/world/europe/belgium-king-paternity.html}{court-ordered
DNA tests} before finally
\href{https://www.nytimes.com/2019/05/21/world/europe/belgium-king-albert-dna-test.html}{submitting
last year}, had ``taken note'' of the results.

The lawyers added that there were ``legal arguments and objections''
establishing that ``a legal paternity is not necessarily the reflection
of a biological paternity,'' but that the king had decided ``not to
raise them and to put an end to this difficult procedure, in honor and
dignity.''

Alain Berenboom, one of the king's lawyers, said on Tuesday that the
Court of Appeal in Brussels would hear the case for a final time in June
and that the legal proceedings would come to a close shortly afterward.

Ms. Boël, 51, is \href{https://www.delphineboel.com/}{a Belgian visual
artist} who has claimed for years that she was conceived during an
affair in the 1960s between her mother, Baroness Sybille de Selys
Longchamps, and Albert, who was then a prince and married to Paola Ruffo
di Calabria, an Italian princess. Prince Albert and his wife already had
three children.

Rumors of an illegitimate child were first alluded to in a 1999 book
about Queen Paola by a journalist, Mario Danneels, and Ms. Boël made her
first public claim that King Albert was her father in a 2005 interview.

While King Albert has never explicitly denied Ms. Boël's claim, he
refused to acknowledge it, and Ms. Boël filed a lawsuit seeking
recognition as his biological daughter in 2013. King Albert ceded the
throne to his eldest son that year, ending his immunity from
prosecution.

\includegraphics{https://static01.nyt.com/images/2020/01/28/world/28belgium-king03/merlin_146870391_4384cfa2-85a7-4370-a46f-3f189770b24a-articleLarge.jpg?quality=75\&auto=webp\&disable=upscale}

Belgium is a constitutional monarchy, where most of the governing power
rests with the Parliament, much like the systems in other European
nations such as Britain, Spain or the Netherlands.

Marc Uyttendaele, Ms. Boël's lawyer, said on Tuesday that she was
``relieved'' to be ``considered as a legitimate child'' because it would
``put an end to the social exclusion that she was subjected to and will
prevent her children from having to bear this burden.''

But Mr. Uyttendaele added that she had been ``hurt'' by the ``coldness''
of the king's statement.

In it, the king's lawyers said that he had ``never been involved with
any family, social or educative decision whatsoever regarding Madame
Delphine Boël'' and that it was Ms. Boël who, ``40 years later,'' had
decided to ``change families'' through a ``long'' and ``painful''
lawsuit.

``The attitude taken by Albert II yesterday does not foretell the
opening of a dialogue between his daughter and him,'' Mr. Uyttendaele
said. ``For her part, she is open to this dialogue, but without much
hope.''

King Albert II was a second son and, as such, was not expected to
succeed to the throne. Instead, he enjoyed a lavish lifestyle of parties
and travel, and when Ms. Boël was born in 1968, the prince privately
recognized her as his daughter and cared for her, according to her
lawyer.

But Albert's elder brother, King Baudouin, died suddenly of heart
failure in 1993, leaving no children, meaning Albert became king.
According to Ms. Boël, that is when her biological father, seeking to
avoid a scandal, cut all ties.

Ms. Boël will not enter the royal line of succession, but she will be in
line to inherit some part of the king's private fortune. Her lawyer,
however, stressed that she had filed the suit for emotional reasons, not
financial ones, noting that Jacques Boël, who raised Ms. Boël as his
only daughter, was a wealthy man.

``Her motives were therefore in no way profit-seeking --- quite the
opposite,'' Mr. Uyttendaele said.

Advertisement

\protect\hyperlink{after-bottom}{Continue reading the main story}

\hypertarget{site-index}{%
\subsection{Site Index}\label{site-index}}

\hypertarget{site-information-navigation}{%
\subsection{Site Information
Navigation}\label{site-information-navigation}}

\begin{itemize}
\tightlist
\item
  \href{https://help.nytimes.com/hc/en-us/articles/115014792127-Copyright-notice}{©~2020~The
  New York Times Company}
\end{itemize}

\begin{itemize}
\tightlist
\item
  \href{https://www.nytco.com/}{NYTCo}
\item
  \href{https://help.nytimes.com/hc/en-us/articles/115015385887-Contact-Us}{Contact
  Us}
\item
  \href{https://www.nytco.com/careers/}{Work with us}
\item
  \href{https://nytmediakit.com/}{Advertise}
\item
  \href{http://www.tbrandstudio.com/}{T Brand Studio}
\item
  \href{https://www.nytimes.com/privacy/cookie-policy\#how-do-i-manage-trackers}{Your
  Ad Choices}
\item
  \href{https://www.nytimes.com/privacy}{Privacy}
\item
  \href{https://help.nytimes.com/hc/en-us/articles/115014893428-Terms-of-service}{Terms
  of Service}
\item
  \href{https://help.nytimes.com/hc/en-us/articles/115014893968-Terms-of-sale}{Terms
  of Sale}
\item
  \href{https://spiderbites.nytimes.com}{Site Map}
\item
  \href{https://help.nytimes.com/hc/en-us}{Help}
\item
  \href{https://www.nytimes.com/subscription?campaignId=37WXW}{Subscriptions}
\end{itemize}
