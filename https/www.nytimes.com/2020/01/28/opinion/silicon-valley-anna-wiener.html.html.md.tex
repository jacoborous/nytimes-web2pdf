Sections

SEARCH

\protect\hyperlink{site-content}{Skip to
content}\protect\hyperlink{site-index}{Skip to site index}

\href{https://myaccount.nytimes.com/auth/login?response_type=cookie\&client_id=vi}{}

\href{https://www.nytimes.com/section/todayspaper}{Today's Paper}

\href{/section/opinion}{Opinion}\textbar{}The Flawed Humanity of Silicon
Valley

\url{https://nyti.ms/2vuAlsM}

\begin{itemize}
\item
\item
\item
\item
\item
\end{itemize}

Advertisement

\protect\hyperlink{after-top}{Continue reading the main story}

\href{/section/opinion}{Opinion}

Supported by

\protect\hyperlink{after-sponsor}{Continue reading the main story}

\hypertarget{the-flawed-humanity-of-silicon-valley}{%
\section{The Flawed Humanity of Silicon
Valley}\label{the-flawed-humanity-of-silicon-valley}}

Behind the scenes of the surveillance economy.

\href{https://www.nytimes.com/by/charlie-warzel}{\includegraphics{https://static01.nyt.com/images/2019/03/15/opinion/charlie-warzel/charlie-warzel-thumbLarge-v3.png}}

By \href{https://www.nytimes.com/by/charlie-warzel}{Charlie Warzel}

Mr. Warzel is an opinion writer at large.

\begin{itemize}
\item
  Jan. 28, 2020
\item
  \begin{itemize}
  \item
  \item
  \item
  \item
  \item
  \end{itemize}
\end{itemize}

\includegraphics{https://static01.nyt.com/images/2020/01/28/opinion/28warzelWeb/28warzelWeb-articleLarge.jpg?quality=75\&auto=webp\&disable=upscale}

Every week brings a fresh hell in the tech world. As news of the latest
scandals pile up over weeks, months and eventually years, narratives
switch. Friendly tech companies become ``Big Tech.'' The narrative is
flattened. The tech giants become monolithic and their employees become
caricatures --- often of villains.

The truth is always messier, more interesting and more human. It is a
central tension animating Anna Wiener's excellent memoir,
``\href{https://us.macmillan.com/books/9780374719760}{Uncanny Valley.}''
The book traces Ms. Wiener's navigating the tech world as a start-up
employee in the mid 2010s --- what might be thought of as the last years
before Silicon Valley's fall from darling status. Ms. Wiener said she
was drawn into the tech world by its propulsive qualities. Graduating
into a recession and spending her early 20s in publishing, tech offered
opportunities: jobs, the seductive feeling of creating something and, of
course, the money was good.

But what makes ``Uncanny Valley'' so valuable is the way it humanizes
the tech industry without letting it off the hook. The book allows us to
see the way that flawed technology is made and marketed: not by
villains, but by blind spots, uncritical thinking and armies of
ambivalent people coming into work each day trying their best --- all
while, sometimes unwittingly, laying the foundation of the surveillance
economy.

From a privacy standpoint, ``Uncanny Valley'' is helpful in
understanding what it's like being on the other end of the torrent of
information that streams from our devices each minute. Early on, Ms.
Wiener recounts working for a successful data analytics company and the
gold rush toward big data, noting that ``not everyone knew what they
needed from big data, but everyone knew that they needed it.''

When confronted with the mass of information her company collected, Ms.
Wiener describes feeling uncomfortable with the ``God Mode'' view that
granted employees full access to user data. ``This was a privileged
vantage point from which to observe the tech industry, and we tried not
to talk about it,'' she writes. This, she notes, becomes a pattern. When
Edward Snowden blew the whistle on the
\href{https://www.washingtonpost.com/news/wonk/wp/2013/06/12/heres-everything-we-know-about-prism-to-date/}{National
Security Agency's Prism program in 2013}, employees at her own data
company never discussed the news.

What she describes is a familiar dissociation for anyone who spends time
interrogating tech companies on their privacy policies. Her company
simply didn't consider itself part of the surveillance economy:

\begin{quote}
``We weren't thinking about our role in facilitating and normalizing the
creation of unregulated, privately held databases on human behavior. We
were just allowing product managers to run better A/B tests. We were
just helping developers make better apps. It was all so simple: people
loved our product and leveraged it to improve their own products, so
that people would love them, too. There was nothing nefarious about it.
Besides, if we didn't do it, someone else would. We were far from the
only third-party analytics tool on the market. The sole moral quandary
in our space that we acknowledged outright was the question of whether
or not to sell data to advertisers. This was something we did not do,
and we were righteous about it. We were just a neutral platform, a
conduit. If anyone raised concerns about the information our users were
collecting, or the potential for abuse of our product, the solutions
manager would try to bring us back to earth by reminding us that we
weren't data brokers. We did not build cross-platform profiles. We
didn't involve third parties. Users might not know they were being
tracked, but that was between them and our customer companies.''
\end{quote}

They were, in other words, just doing their jobs.

Ms. Wiener frequently returns to this reticence to question the product,
the end goals of the technology and the Silicon Valley ethos as a whole.

At her next job working on the terms of service team for a large open
source code platform, she reveals how the evolution of the internet
pushed her and her co-workers into becoming ``reluctant content
moderators.'' Soon it became her team's job to fashion a balance between
preserving free speech on her platform and protecting it from trolls and
neo-Nazis:

\begin{quote}
``We wanted to tread lightly: core participants in the open-source
software community were sensitive to corporate oversight, and we didn't
want to undercut anyone's techno-utopianism by becoming an overreaching
arm of the company-state. We wanted to be on the side of human rights,
free speech and free expression, creativity and equality. At the same
time, it was an international platform, and who among us could have
articulated a coherent stance on international human rights?''
\end{quote}

As a journalist who has covered content moderation issues for the better
part of a decade, the perspective is somewhat clarifying. Decisions that
feel ad hoc or made by one or two people in the belly of a large company
often are. What looks from the outside like a conspiracy or nefarious
techno-authoritarianism is often just confusion caused by poor
management, poor communication and dizzying growth. ``Most of the
company did not seem aware of how common it was for our tools to be
abused,'' Ms. Wiener writes of her group of de facto moderators. ``They
did not even seem to know that our team existed. It wasn't their fault
--- we were easy to miss. There were four of us for the platform's nine
million users.''

In this instance, ``Uncanny Valley'' shows how the internet can thrust
ordinary people into extraordinary positions of power --- usually
without qualifications or a how-to guide. This is not to say that the
book excuses any of the industry's reckless behavior. Like a good travel
writer, Ms. Wiener positions herself as an insider-outsider,
``participating in something bigger than myself and still feeling apart
from it.'' And she is sufficiently critical of her and her peers'
participation in the industry. She writes that she would ``wonder
whether the N.S.A. whistle-blower had been the first moral test for my
generation of entrepreneurs and tech workers, and we had blown it,'' she
writes at one point near the end of the memoir.

Ms. Wiener's memoir comes at a point where the backlash against Silicon
Valley is strong enough to have earned its own name. Narratives have
hardened and aggrieved tech employees are adopting a
\href{https://www.buzzfeednews.com/article/charliewarzel/facebooks-tensions-zuckerberg-sandberg}{``bunker
mentality.''} As
\href{https://themargins.substack.com/p/facebooks-pr-feels-broken}{Ranjan
Roy of the newsletter Margins wrote} recently of Facebook, ``the rank
and file are seeing that they are the villains, and will increasingly
become so.'' As so much of the reporting shows, the increased scrutiny
and criticism of the techlash is important and almost all is warranted.
Big Tech has amassed wild, unregulated power that has grown unchecked.

Still, it's easy to get conspiratorial and to fall comfortably into
black and white notions of good versus evil. ``Uncanny Valley'' is a
reminder that the reality is far more muddled but no less damning. Our
dystopia isn't just the product of mustache-twirling billionaires drunk
with power and fueled by greed --- though it is that, too, sometimes.
It's also the result of uncritical thinking, blind spots caused by an
overwhelmingly white male work force and a pathological reluctance to
ask the bigger question: Where is this all going? What am I building?

\hypertarget{what-im-reading}{%
\subsection{What I'm Reading:}\label{what-im-reading}}

\href{https://www.washingtonpost.com/technology/2020/01/28/off-facebook-activity-page/}{Facebook
will now show you exactly how it stalks you --- even when you're not
using Facebook.}

\href{https://www.eff.org/deeplinks/2020/01/ring-doorbell-app-packed-third-party-trackers}{Ring
doorbell app packed with third-party trackers.}

\href{https://t.co/kXGegjaSPp?amp=1}{Leaked documents expose the
secretive market for your web browsing data.}

\href{https://www.technologyreview.com/f/615098/facial-recognition-clearview-ai-epic-privacy-moratorium-surveillance/}{40
groups have called for a U.S. moratorium on facial recognition
technology}

\emph{Like other media companies, The Times collects data on its
visitors when they read stories like this one. For more detail please
see}
\href{https://help.nytimes.com/hc/en-us/articles/115014892108-Privacy-policy?module=inline}{\emph{our
privacy policy}} \emph{and}
\href{https://www.nytimes.com/2019/04/10/opinion/sulzberger-new-york-times-privacy.html?rref=collection\%2Fspotlightcollection\%2Fprivacy-project-does-privacy-matter\&action=click\&contentCollection=opinion\&region=stream\&module=stream_unit\&version=latest\&contentPlacement=8\&pgtype=collection}{\emph{our
publisher's description}} \emph{of The Times's practices and continued
steps to increase transparency and protections.}

\emph{Follow}
\href{https://twitter.com/privacyproject}{\emph{@privacyproject}}
\emph{on Twitter and The New York Times Opinion Section on}
\href{https://www.facebook.com/nytopinion}{\emph{Facebook}}
\emph{and}\href{https://www.instagram.com/nytopinion/}{\emph{Instagram}}\emph{.}

\hypertarget{glossary-replacer}{%
\subsection{glossary replacer}\label{glossary-replacer}}

Advertisement

\protect\hyperlink{after-bottom}{Continue reading the main story}

\hypertarget{site-index}{%
\subsection{Site Index}\label{site-index}}

\hypertarget{site-information-navigation}{%
\subsection{Site Information
Navigation}\label{site-information-navigation}}

\begin{itemize}
\tightlist
\item
  \href{https://help.nytimes.com/hc/en-us/articles/115014792127-Copyright-notice}{©~2020~The
  New York Times Company}
\end{itemize}

\begin{itemize}
\tightlist
\item
  \href{https://www.nytco.com/}{NYTCo}
\item
  \href{https://help.nytimes.com/hc/en-us/articles/115015385887-Contact-Us}{Contact
  Us}
\item
  \href{https://www.nytco.com/careers/}{Work with us}
\item
  \href{https://nytmediakit.com/}{Advertise}
\item
  \href{http://www.tbrandstudio.com/}{T Brand Studio}
\item
  \href{https://www.nytimes.com/privacy/cookie-policy\#how-do-i-manage-trackers}{Your
  Ad Choices}
\item
  \href{https://www.nytimes.com/privacy}{Privacy}
\item
  \href{https://help.nytimes.com/hc/en-us/articles/115014893428-Terms-of-service}{Terms
  of Service}
\item
  \href{https://help.nytimes.com/hc/en-us/articles/115014893968-Terms-of-sale}{Terms
  of Sale}
\item
  \href{https://spiderbites.nytimes.com}{Site Map}
\item
  \href{https://help.nytimes.com/hc/en-us}{Help}
\item
  \href{https://www.nytimes.com/subscription?campaignId=37WXW}{Subscriptions}
\end{itemize}
