Sections

SEARCH

\protect\hyperlink{site-content}{Skip to
content}\protect\hyperlink{site-index}{Skip to site index}

\href{https://myaccount.nytimes.com/auth/login?response_type=cookie\&client_id=vi}{}

\href{https://www.nytimes.com/section/todayspaper}{Today's Paper}

\href{/section/opinion}{Opinion}\textbar{}Weaponizing Truth Against
Opioids

\href{https://nyti.ms/2Ry9W5K}{https://nyti.ms/2Ry9W5K}

\begin{itemize}
\item
\item
\item
\item
\item
\item
\end{itemize}

Advertisement

\protect\hyperlink{after-top}{Continue reading the main story}

\href{/section/opinion}{Opinion}

Supported by

\protect\hyperlink{after-sponsor}{Continue reading the main story}

\hypertarget{weaponizing-truth-against-opioids}{%
\section{Weaponizing Truth Against
Opioids}\label{weaponizing-truth-against-opioids}}

A decades-old campaign that knew how to talk to teenagers persuaded
millions of them to never start smoking. Can it now persuade them never
to misuse opioids?

\includegraphics{https://static01.nyt.com/images/2019/02/13/opinion/tina-rosenberg/tina-rosenberg-thumbLarge-v2.png}

By Tina Rosenberg

Ms. Rosenberg is a co-founder of the
\href{http://solutionsjournalism.org}{Solutions Journalism Networ}k,
which supports rigorous reporting about responses to social problems.

\begin{itemize}
\item
  Jan. 28, 2020
\item
  \begin{itemize}
  \item
  \item
  \item
  \item
  \item
  \item
  \end{itemize}
\end{itemize}

\includegraphics{https://static01.nyt.com/images/2020/01/28/opinion/28fixesrosenberg1/merlin_167934957_232a7c38-3181-4248-a8f3-def39e368882-articleLarge.jpg?quality=75\&auto=webp\&disable=upscale}

This ad depicts a true story. It begins: A young man walks around his
car, which is up on a jack. He says in voice-over: ``I got some Oxy
after I hurt my neck. First I took them to feel better. Then I kept
taking them. I didn't know they'd be this addictive. I didn't know how
far I'd go to get more.''

Here's how far he'd go: He lies down under the car. Then he
\href{https://www.thetruth.com/o/articles/joe}{kicks out the jack}.

You see the car fall, hear a crunch. ``Joe S. from Maine broke his back
to get more prescription opioids,'' the screen says. And then a
voice-over: ``Opioid dependence can happen after just five days. Know
the truth, spread the truth.''

A similar ad shows a woman taking off her seatbelt and deliberately
crashing her car. In another, a man breaks his hand with a hammer. In
another, a man slams his arm in a door. Together, those comprised the
first national ad campaign in America aimed at preventing opioid misuse.

They came from the \href{http://www.truthinitiative.org}{Truth
Initiative}, the organization that was behind truth ads that helped
bring about one of the most important public health victories in
American history: In 1996,
\href{https://www.washingtonpost.com/archive/politics/1997/11/02/officials-seek-a-path-to-cut-into-haze-of-youth-smoking/bdfdf1a0-4c3e-433c-844c-8eca9b746253/}{34
percent of high school seniors} had smoked a cigarette in the previous
month --- the same as in 1975. But by 2019, less than 6 percent smoked.

Now truth ads are being focusing on opioid abuse. There's a
relationship: More than 70 percent of opioid users
\href{https://www.ncbi.nlm.nih.gov/pmc/articles/PMC4990064/}{also
smoke}, and there is evidence that
\href{https://www.sciencedirect.com/science/article/pii/S0091743519303214}{smoking
can prime the brain for other addictions}.

\includegraphics{https://static01.nyt.com/images/2020/01/28/opinion/28fixesrosenberg2/merlin_167934969_4ed0a9df-ece2-4121-be45-6ade80f7db47-articleLarge.jpg?quality=75\&auto=webp\&disable=upscale}

Can the strategy work again? Let's look at what the group did right in
its antismoking campaign.

Before the truth ads, antismoking ads for teenagers were created by
public health experts who, not surprisingly, emphasized that cigarettes
can kill you.

That threat works well to encourage adults to quit smoking. But it
doesn't prevent teens from starting. For many adolescents, the danger is
a lure; they smoke to rebel against preachy adults. To them, the
consequences are decades away, and teens are immortal.

In the 1990s, Florida and California hired advertising agencies that
didn't know about public health, but did know about selling to teens.
They created ads to redirect teenage rebellion against the manipulations
of the tobacco industry. Some examples:

\begin{itemize}
\tightlist
\item
  One California advertisement portrayed tobacco executives in a
  smoke-filled room, cackling maniacally as they plot to replace 1,100
  customers who quit smoking every day --- ``Actually, technically, they
  die,'' one says.
\end{itemize}

\begin{itemize}
\tightlist
\item
  Another ad showed rappers attacking Big Tobacco for targeting
  African-Americans with menthol cigarettes. It ended, ``We used to pick
  it; now they want us to smoke it.''
\end{itemize}

\begin{itemize}
\tightlist
\item
  In Florida, teenagers created ads for a campaign they named ``truth.''
  One showed girls making prank phone calls to tobacco ad executives.
  ``What is the lucky part about Lucky Strike cigarettes?'' a girl asks.
  ``It is because" (pause) I might live?''
\end{itemize}

That was in 1998. Over the next two years, teen smoking in Florida fell
by 17.5 percent.

Florida was one of the first states to sue tobacco companies. When a
nationwide settlement was reached, part of the money created a national
truth campaign. It copied Florida's strategies and hired its people. And
it enjoyed the same success.

Of course, truth ads were only one part of this victory. States and
cities raised their cigarette taxes. Indoor smoking vanished. But the
ads were crucial.

(This success is now threatened by an explosion in vaping --- also known
as ``juuling,'' after the biggest brand, Juul, whose use among teenagers
and young adults
\href{https://truthinitiative.org/press/press-release/new-truth-initiative-study-finds-juul-use-doubled-one-year-tobacco-and-nicotine}{doubled
from 2018 to 2019}. Young people who vape are four times more likely to
try cigarettes than those who don't. Counting vaping, teen tobacco use
now is the highest it has been in decades.)

Those truth ads still target adolescent smoking. New teenagers are
minted every year, so the campaign never ends. It also makes
``\href{https://www.thetruth.com/articles/videos/ditch-juul}{Ditch
Juul'' ads} that show teens creatively demolishing their vaping
equipment.

All this is useful for anti-opioid campaigns. ``There's a lot of
crossover in terms of strategy for reaching your target audience of
youth,'' said Matthew Farrelly, senior director of the Center for Health
Policy Science and Tobacco Research at RTI International in North
Carolina. ``The variety of tactics for getting in front of teens is
nearly identical.''

Surveys about the truth ads show that 80 percent of teenagers recognize
the brand. And the organization knows what works. A
\href{https://www.ncbi.nlm.nih.gov/pubmed/25372236}{review of published
studies} in 2014 concluded, ``Youth are more likely to recall and think
about advertising that includes personal testimonials; a surprising
narrative; and intense images, sound and editing.'' That's Joe S. from
Maine, all right.

Anti-opioid ads can use the antismoking template, but the message must
be different. ``Nobody needs education on the fact that tobacco is
really bad for you,'' said Robin Koval, president and chief executive of
the Truth Initiative. ``With opioids, we found out how little people
know.''

Adolescents know about the misery of addiction --- and addiction and
overdose at 20 are scary in a way lung cancer at 60 is not. ``But there
is a lack of understanding and awareness of how easily you can become
addicted,'' Ms. Koval said.

Some adolescents start using opioids because their friends share. For
others, the beginning is pain medication. Getting a prescription in high
school after knee surgery or wisdom teeth removal is associated with a
\href{https://pediatrics.aappublications.org/content/136/5/e1169}{33
percent higher chance of misusing opioids} later in life.

A
\href{https://www.cdc.gov/mmwr/volumes/66/wr/mm6610a1.htm\#F1_down}{study
published by the Centers for Disease Control} and Prevention concluded,
``The probability of long-term opioid use increases sharply in the first
days of therapy'' --- with a spike after five days. So every truth ad
ends with: ``Opioid dependence can happen after just five days.''

``The hardest thing to go up against is nobody believes they're going to
become addicted to opioids,'' Ms. Koval said. ``We had to reverse this
notion of otherness. And we have to combat the stigma that says: This is
a moral failing and therefore I can't believe it would happen to me or
anybody we know.''

The Joe S. ad is part of a campaign that began in July 2018, in
collaboration with the White House Office of National Drug Control
Policy and the Ad Council. Facebook, Google, Turner, Amazon and Vice
donated advertising time or space. (The Google News Initiative is a
funder of the Solutions Journalism Network, my employer.)

Image

"Treatment Box" captures 26-year-old Rebekkah's experience with opioid
addiction and her journey to recovery. The Emmy Award-winning video
brought people face-to-face with opioid addiction and treatment through
a multiscreen installation that showed how opioid addiction can happen
to anyone in as little as five days.Credit...Truth Initiative

Last year, a truth video titled ``Treatment Box: Rebekkah's Story,'' won
an Emmy. In it, Rebekkah, then 26, started using opioids at 14 after a
cheerleading injury. The ad showed her detoxing --- minute by minute.
The video was projected into a clear glass bedroom-size box in Times
Square. Passers-by watched Rebekkah's pain.

A new ad series, ``\href{https://www.thetruth.com/o/opioids}{Best Day},"
has a relatable ``it could be you'' message, said Margie Skeer,
associate professor at Tufts University School of Medicine. It's
graduation day, or signing day for college sports, or a basketball
awards assembly --- but the happy teens explain that on this, their best
day, they see their futures as bleak: that the pain of adult stress or a
sports injury will lead them to opioids, then addiction, and finally to
overdosing.
\href{https://www.thetruth.com/o/articles/videos/signing-day}{``My mom
will bury me with my cleats,''} one girl says.

It's too soon to know if these ads reduce youth opioid misuse. A test of
the Joe S.-style ads found that people who saw them were significantly
more likely to agree that someone like them could become dependent on
prescription opioids.

Image

``Best Day,'' the most recent installment of The Truth About Opioids,
illustrates how a young person's shining moment can turn out to be the
start of a very different story. In the episode ``Senior Night,'' a high
school basketball player, on the greatest night of his young career,
previews seriously injuring~ his knee in college, becoming addicted to
opioids, then turning to heroin and dying from an
overdose.Credit...Truth Initiative

``It's maybe not quite as elaborate and creative and multifaceted as the
truth campaign,'' said Dr. Farrelly. ``But if their goal is to raise
awareness of risk, it might not be a bad strategy.''

But that's not the same as avoiding addiction. The gripping Joe S. ads
produce horror and fear. ``Fear works in the moment,'' Dr. Skeer said.
``But when people are scared but don't know what to do next, it can
become paralyzing.'' Other public health researchers also said that the
ads needed to answer the question: ``O.K., so what \emph{should} I do?''

The truth campaign responded that opioids were complicated, and that
covering various scenarios in the ad would have been confusing. Instead,
it seeks to drive young people to the Truth Initiative website, where
they can get advice. The first ads should have a clear focus on
``telling stories that felt authentic and true in the voices of young
people,'' Ms. Koval said. ``Helping them to understand this can happen
to you.''

Tina Rosenberg won a Pulitzer Prize for her book
``\href{http://www.randomhouse.com/catalog/display.pperl?isbn=9780679744993}{The
Haunted Land: Facing Europe's Ghosts After Communism}.'' She is a former
editorial writer for The Times, a vice president of the Solutions
Journalism Network, and the author, most recently, of
``\href{http://books.wwnorton.com/books/Join-the-Club}{Join the Club:
How Peer Pressure Can Transform the World}'' and the World War II spy
story e-book
\href{https://www.goodreads.com/book/show/16124470-d-for-deception}{``D
for Deception.''}

\emph{To receive email alerts for Fixes columns, sign up}
\href{http://eepurl.com/ABIxL}{\emph{here.}}

\emph{The Times is committed to publishing}
\href{https://www.nytimes.com/2019/01/31/opinion/letters/letters-to-editor-new-york-times-women.html}{\emph{a
diversity of letters}} \emph{to the editor. We'd like to hear what you
think about this or any of our articles. Here are some}
\href{https://help.nytimes.com/hc/en-us/articles/115014925288-How-to-submit-a-letter-to-the-editor}{\emph{tips}}\emph{.
And here's our email:}
\href{mailto:letters@nytimes.com}{\emph{letters@nytimes.com}}\emph{.}

\emph{Follow The New York Times Opinion section on}
\href{https://www.facebook.com/nytopinion}{\emph{Facebook}}\emph{,}
\href{http://twitter.com/NYTOpinion}{\emph{Twitter (@NYTopinion)}}
\emph{and}
\href{https://www.instagram.com/nytopinion/}{\emph{Instagram}}\emph{.}

Advertisement

\protect\hyperlink{after-bottom}{Continue reading the main story}

\hypertarget{site-index}{%
\subsection{Site Index}\label{site-index}}

\hypertarget{site-information-navigation}{%
\subsection{Site Information
Navigation}\label{site-information-navigation}}

\begin{itemize}
\tightlist
\item
  \href{https://help.nytimes.com/hc/en-us/articles/115014792127-Copyright-notice}{©~2020~The
  New York Times Company}
\end{itemize}

\begin{itemize}
\tightlist
\item
  \href{https://www.nytco.com/}{NYTCo}
\item
  \href{https://help.nytimes.com/hc/en-us/articles/115015385887-Contact-Us}{Contact
  Us}
\item
  \href{https://www.nytco.com/careers/}{Work with us}
\item
  \href{https://nytmediakit.com/}{Advertise}
\item
  \href{http://www.tbrandstudio.com/}{T Brand Studio}
\item
  \href{https://www.nytimes.com/privacy/cookie-policy\#how-do-i-manage-trackers}{Your
  Ad Choices}
\item
  \href{https://www.nytimes.com/privacy}{Privacy}
\item
  \href{https://help.nytimes.com/hc/en-us/articles/115014893428-Terms-of-service}{Terms
  of Service}
\item
  \href{https://help.nytimes.com/hc/en-us/articles/115014893968-Terms-of-sale}{Terms
  of Sale}
\item
  \href{https://spiderbites.nytimes.com}{Site Map}
\item
  \href{https://help.nytimes.com/hc/en-us}{Help}
\item
  \href{https://www.nytimes.com/subscription?campaignId=37WXW}{Subscriptions}
\end{itemize}
