Sections

SEARCH

\protect\hyperlink{site-content}{Skip to
content}\protect\hyperlink{site-index}{Skip to site index}

\href{https://www.nytimes.com/section/smarter-living}{Smarter Living}

\href{https://myaccount.nytimes.com/auth/login?response_type=cookie\&client_id=vi}{}

\href{https://www.nytimes.com/section/todayspaper}{Today's Paper}

\href{/section/smarter-living}{Smarter Living}\textbar{}You're Tracked
Everywhere You Go Online. Use This Guide to Fight Back.

\url{https://nyti.ms/2pQsvas}

\begin{itemize}
\item
\item
\item
\item
\item
\end{itemize}

Advertisement

\protect\hyperlink{after-top}{Continue reading the main story}

Supported by

\protect\hyperlink{after-sponsor}{Continue reading the main story}

\hypertarget{youre-tracked-everywhere-you-go-online-use-this-guide-to-fight-back}{%
\section{You're Tracked Everywhere You Go Online. Use This Guide to
Fight
Back.}\label{youre-tracked-everywhere-you-go-online-use-this-guide-to-fight-back}}

You can't stop all of it, but you don't have to give up.

\includegraphics{https://static01.nyt.com/images/2019/12/02/smarter-living/02sl_newsletter-digital/25sl_newsletter-articleLarge.jpg?quality=75\&auto=webp\&disable=upscale}

\href{https://www.nytimes.com/by/tim-herrera}{\includegraphics{https://static01.nyt.com/images/2018/12/07/multimedia/author-tim-herrera/author-tim-herrera-thumbLarge.png}}

By \href{https://www.nytimes.com/by/tim-herrera}{Tim Herrera}

\begin{itemize}
\item
  Published Nov. 24, 2019Updated Nov. 29, 2019
\item
  \begin{itemize}
  \item
  \item
  \item
  \item
  \item
  \end{itemize}
\end{itemize}

\emph{Welcome to the Smarter Living newsletter! Every Monday, S.L.
editor} \href{https://twitter.com/timherrera}{\emph{Tim Herrera}}
\emph{emails readers with tips and advice for living a better, more
fulfilling life.}
\href{https://www.nytimes.com/newsletters/smarter-living?module=inline}{\emph{Sign
up here}} \emph{to get it in your inbox.}

\href{https://twitter.com/TimHerrera/status/1195067771463360512}{Here
are some mildly terrifying things} I learned when I recently did an
online privacy checkup: Google was sharing my creditworthiness with
third parties. If you want Target to
\href{https://twitter.com/TimHerrera/status/1195075728859156481}{stop
sharing your information} with marketers, you have to call them. And, my
favorite: If you would like Hearst, the publishing giant, to stop
sharing your physical mailing address with third parties, you have to
\href{https://twitter.com/TimHerrera/status/1195070845967851521}{mail a
physical letter} with your request to the company's lawyers.

Cool cool cool.

I was inspired by
\href{https://www.nytimes.com/2019/11/04/business/secret-consumer-score-access.html}{this
story} my colleague
\href{https://www.nytimes.com/by/kashmir-hill}{Kashmir Hill} wrote this
month about the company Sift, which collects your consumer data and
analyzes then scores your transactions.

``As consumers, we all have `secret scores': hidden ratings that
determine how long each of us waits on hold when calling a business,
whether we can return items at a store, and what type of service we
receive,'' Ms. Hill wrote. ``A low score sends you to the back of the
queue; high scores get you elite treatment.'' (If you're interested, you
can request your own secret dossier by emailing
\href{mailto:privacy@sift.com}{\nolinkurl{privacy@sift.com}}, though the
company is backed up because of the ``recent press coverage.'' It took
them two weeks to respond to my request.)

\hypertarget{surprised-we-were-too}{%
\subsection{Surprised? We were, too}\label{surprised-we-were-too}}

Like many people, I was a little stunned at the intimate level of data
that was being collected. Ms. Hill was, too.

``I know that we are tracked in surprising ways, and have reported on
those surprising ways extensively, but even I was shocked to get a
400-page file on myself back from a company I'd never heard of,'' she
told me. ``It was bizarre to see what I had ordered from an Indian
restaurant three years ago in the report and disturbing to find all the
private Airbnb messages that I had sent to hosts. I didn't think any
company beyond Airbnb would have that data.''

It's no secret that we're being tracked everywhere online. We all know
this; every one of us has a story about an alarmingly specific ad
appearing on Facebook, or a directly targeted Amazon promo following us
around the internet. But as internet-connect devices become more
prevalent in our everyday lives --- think smart TVs, smart speakers and
smart refrigerators, for example --- and as our reliance on smartphones
increases, we're just creating so much more data than we used to, said
Bennett Cyphers, a staff technologist at the
\href{https://www.eff.org/}{Electronic Frontier Foundation}, a nonprofit
digital rights organization that advocates for consumer online privacy.

``There are just more streams of data out there to be aggregated and
tied to profiles and sold,'' Mr. Cyphers said. ``Because people don't
realize that their car is collecting data about their location and
sending it off to some server somewhere, they're less likely to think
about that, and companies are less likely to be held accountable for
that kind of thing.''

He added: ``Information is being shared completely haphazardly, and
there's no accountability at any stage, especially in America.''

\hypertarget{who-cares-i-have-nothing-to-hide}{%
\subsection{Who cares? I have nothing to
hide}\label{who-cares-i-have-nothing-to-hide}}

We've all heard that one before.

``The only people I've heard say, `Who cares?' are people who don't
understand the scope of the problem,'' Mr. Cyphers said.

``A lot of the tracking systems out there make it easier for law
enforcement to gather data without warrants,'' he said. ``A lot of
trackers sell data directly to law enforcement and to Immigrations and
Customs Enforcement. I think the bottom line is that it's creepy at
best. It enables manipulative advertising and political messaging in
ways that make it a lot easier for the messengers to be unaccountable.
It enables discriminatory advertising without a lot of accountability,
and in the worst cases it can put real people in real danger.''

\emph{{[}Like what you're reading?}
\href{https://www.nytimes.com/newsletters/smarter-living?module=inline}{\emph{Sign
up here}} \emph{for the Smarter Living newsletter to get stories like
this (and much more!) delivered straight to your inbox every Monday
morning.{]}}

Still, there are signs that things could be improving, if slowly. The
\href{https://www.nytimes.com/2018/04/04/us/politics/cambridge-analytica-scandal-fallout.html}{Cambridge
Analytica scandal}, Mr. Cyphers said, ``dredged up the worst parts of
the industry into the press and popular knowledge,'' which in some ways
forced companies and lawmakers to acknowledge the issue. Sweeping
changes, such as the
\href{https://www.nytimes.com/2018/06/28/technology/california-online-privacy-law.html?module=inline}{California
Consumer Privacy Act} and Europe's
\href{https://www.nytimes.com/2018/05/24/technology/europe-gdpr-privacy.html}{GDPR},
have led the way in giving internet users new rights and protections,
and Mr. Cyphers said that ``popular awareness and the techlash has
opened up room for real regulation.''

But we're a long way from a privacy utopia.

``As long as you can make a buck and what you're doing isn't illegal,''
Mr. Cyphers said, ``someone's going to do it.''

\hypertarget{what-can-i-do}{%
\subsection{What can I do?}\label{what-can-i-do}}

First, be more cautious of the information you voluntary hand over.

``Don't hand over data unless you have to!'' Ms. Hill said. ``If a store
asks for your email address or ZIP code, say no. When Facebook asks you
to upload your contact book, don't do it. If you're buying some
sensitive product (prenatal vitamins, medication), don't use your store
loyalty card and use cash.''

Added Mr. Cyphers: ``Think hard before you enter your email into a form
online about why the company actually needs your email and what they
might do with it. You can lie. It's not illegal to put a fake email, or
a fake phone number or a fake name in the vast majority of services you
sign up for,'' he said. ``There's no reason they need it, there's no
reason you have to give it to them.''

Beyond being more wary of handing out your data, there are some things
you can do about the data that is already out there --- and now we come
back to that privacy checkup I mentioned earlier.

One of the best resources for opting out of advertiser tracking is the
website \href{https://simpleoptout.com/}{simpleoptout.com}, which
provides links to the opt-out pages for some of the most popular
destinations online --- places that are definitely tracking you as you
read this.

Some of the major ones you should opt out of right now include:

\begin{itemize}
\item
  \href{https://simpleoptout.com/\#optoutprescreen.com}{OptOutPrescreen.com}
\item
  \href{https://simpleoptout.com/\#paypal}{PayPal}
\item
  \href{https://simpleoptout.com/\#spokeo}{Spokeo}
\item
  \href{https://simpleoptout.com/\#visa}{Visa}
\item
  \href{https://simpleoptout.com/\#lexis-nexis}{Lexis-Nexis}
\item
  \href{https://simpleoptout.com/\#whitepages.com}{WhitePages.com}
\item
  \href{https://simpleoptout.com/\#apple}{Apple}
\item
  \href{https://simpleoptout.com/\#yahoo/oath}{Yahoo}
\item
  \href{https://simpleoptout.com/\#amazon.com}{Amazon}
\item
  \href{https://simpleoptout.com/\#microsoft}{Microsoft}
\item
  \href{https://simpleoptout.com/\#verizon}{Verizon}
\item
  \href{https://simpleoptout.com/\#at\&t}{AT\&T}
\item
  \href{https://simpleoptout.com/\#chase}{Chase}
\item
  \href{https://simpleoptout.com/\#youtube}{YouTube}
\item
  \href{https://simpleoptout.com/\#comcast-xfinity}{Comcast Xfinity}
\item
  \href{https://simpleoptout.com/\#google}{Google} (This one will take a
  while, it's a labyrinth of menus and settings.)
\item
  \href{https://simpleoptout.com/\#bank-of-america}{Bank of America}
\end{itemize}

Additionally, do a checkup of how social media sites are using your
data:

\begin{itemize}
\item
  \href{https://simpleoptout.com/\#facebook}{Facebook}
\item
  \href{https://simpleoptout.com/\#twitter}{Twitter}
\item
  \href{https://simpleoptout.com/\#linkedin}{LinkedIn}
\item
  \href{https://simpleoptout.com/\#reddit}{Reddit}
\end{itemize}

There are also extensions you can install to your web browser to prevent
some online tracking. The \href{https://www.eff.org/}{EFF} has built two
tools you should install ---
\href{https://www.eff.org/privacybadger}{Privacy Badger} and
\href{https://www.eff.org/https-everywhere}{HTTPS Everywhere} --- and
also recommends the extension
\href{https://github.com/gorhill/uBlock\#ublock-origin}{uBlock Origin}.
The EFF also has \href{https://ssd.eff.org/}{this guide to Surveillance
Self-Defense}, which has an extensive library of guides to protecting
yourself online. And as for your general browsing, think about using
\href{https://www.mozilla.org/en-US/firefox/new/}{Mozilla's Firefox} if
you don't already;
\href{https://www.washingtonpost.com/technology/2019/06/21/google-chrome-has-become-surveillance-software-its-time-switch/}{this
story from The Washington Post} will tell you why.

Your smartphone is a whole other game of cat and mouse, but there are a
few basic things everyone should do.
\href{https://www.usatoday.com/story/tech/columnist/komando/2019/02/14/your-smartphone-tracking-you-how-stop-sharing-data-ads/2839642002/}{This
guide from USA Today is a perfect place to start}, whether you have an
iPhone or Android device.

Phew! It's a lot, I know, and unfortunately we're only scratching the
surface; protecting your privacy is a never-ending process that requires
constant vigilance. But each of these steps is worth the time
investment, and perhaps the most important thing to keep in mind: Don't
let yourself be lulled into a false sense of security.

``I wish I could say it has changed my behavior,'' Ms. Hill said when I
asked her if reporting her story on Sift has changed her online
behavior, ``but what's become clearer and clearer to me in reporting on
privacy over the last decade is that you can't completely stop the data
collection (unless you go live in a dark cave sans power).''

She added: ``But at the end of the day, there is little we can do as
individuals; there's really a need for change on a more systemic level
to give us more control over our data.''

I want to hear about your experiences as you do a checkup on your
digital privacy. Tell me on Twiter
\href{https://twitter.com/timherrera}{@timherrera}.

Thanks, have a great week!

--- Tim

\begin{center}\rule{0.5\linewidth}{\linethickness}\end{center}

\hypertarget{tip-of-the-week}{%
\subsubsection{Tip of the Week}\label{tip-of-the-week}}

\emph{This week I've invited the writer}
\href{https://twitter.com/OczyKate}{\emph{Kate Oczypok}} \emph{to teach
how to make the perfect table centerpiece.}

With the holiday season now upon us, I'm starting to think about what
I'd like my holiday table to look like. And if there's one part of a
festive table that makes for great conversation, it's a gorgeous
centerpiece.

If you're fretting about the last piece of the puzzle for your holiday
dinner, here are three ideas for the perfect centerpiece this season.

\textbf{Use your space wisely}

If you live in a small apartment, chances are you're not going to go for
a big, cascading centerpiece. Wirecutter, a New York Times company that
reviews and recommends products, suggests using
\href{https://thewirecutter.com/lists/how-to-set-the-table-for-your-first-dinner-party/}{tea
lights in votives} to set the mood; they won't drip wax on the table and
since they're shorter candles, guests can see one another across the
table. Also, the centerpiece won't look as if it's overpowering your
entire apartment.

\textbf{Think outside the box}

When our dog passed away last spring, my boyfriend and I were
overwhelmed by the generosity of friends and family. Our loved ones had
sent us five bouquets of flowers and we didn't know what to do with all
of them. Since Easter was just days away and we had friends coming for
dinner, we put a leaf on our dining table and lined the center of it
with the bouquets. It was a beautiful tribute to our dear old Moe.

\textbf{Go for the unusual}

If what you choose to put at the center of your table is unusual,
chances are it will be a great icebreaker, especially for guests who
don't know one another.
This\href{https://www.thesill.com/products/air-plant-assortment-1?variant=30392047239273}{Air
Plant Trio} is small and just oddly shaped enough to encourage some fun
dinner talk. They come with eight different choices for planter color,
too, and they look
great\href{https://thewirecutter.com/gifts/best-hostess-gifts/}{grouped
together} on a holiday table.

Advertisement

\protect\hyperlink{after-bottom}{Continue reading the main story}

\hypertarget{site-index}{%
\subsection{Site Index}\label{site-index}}

\hypertarget{site-information-navigation}{%
\subsection{Site Information
Navigation}\label{site-information-navigation}}

\begin{itemize}
\tightlist
\item
  \href{https://help.nytimes.com/hc/en-us/articles/115014792127-Copyright-notice}{©~2020~The
  New York Times Company}
\end{itemize}

\begin{itemize}
\tightlist
\item
  \href{https://www.nytco.com/}{NYTCo}
\item
  \href{https://help.nytimes.com/hc/en-us/articles/115015385887-Contact-Us}{Contact
  Us}
\item
  \href{https://www.nytco.com/careers/}{Work with us}
\item
  \href{https://nytmediakit.com/}{Advertise}
\item
  \href{http://www.tbrandstudio.com/}{T Brand Studio}
\item
  \href{https://www.nytimes.com/privacy/cookie-policy\#how-do-i-manage-trackers}{Your
  Ad Choices}
\item
  \href{https://www.nytimes.com/privacy}{Privacy}
\item
  \href{https://help.nytimes.com/hc/en-us/articles/115014893428-Terms-of-service}{Terms
  of Service}
\item
  \href{https://help.nytimes.com/hc/en-us/articles/115014893968-Terms-of-sale}{Terms
  of Sale}
\item
  \href{https://spiderbites.nytimes.com}{Site Map}
\item
  \href{https://help.nytimes.com/hc/en-us}{Help}
\item
  \href{https://www.nytimes.com/subscription?campaignId=37WXW}{Subscriptions}
\end{itemize}
