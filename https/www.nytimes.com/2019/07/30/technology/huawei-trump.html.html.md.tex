Sections

SEARCH

\protect\hyperlink{site-content}{Skip to
content}\protect\hyperlink{site-index}{Skip to site index}

\href{https://www.nytimes.com/section/technology}{Technology}

\href{https://myaccount.nytimes.com/auth/login?response_type=cookie\&client_id=vi}{}

\href{https://www.nytimes.com/section/todayspaper}{Today's Paper}

\href{/section/technology}{Technology}\textbar{}Huawei's Sales Jump
Despite Trump's Blacklisting

\url{https://nyti.ms/2Zn5TL4}

\begin{itemize}
\item
\item
\item
\item
\item
\end{itemize}

Advertisement

\protect\hyperlink{after-top}{Continue reading the main story}

Supported by

\protect\hyperlink{after-sponsor}{Continue reading the main story}

\hypertarget{huaweis-sales-jump-despite-trumps-blacklisting}{%
\section{Huawei's Sales Jump Despite Trump's
Blacklisting}\label{huaweis-sales-jump-despite-trumps-blacklisting}}

\includegraphics{https://static01.nyt.com/images/2019/07/30/business/30huawei/merlin_155583492_a7fdea77-1f55-4ccd-852e-ee11c2e1dff7-articleLarge.jpg?quality=75\&auto=webp\&disable=upscale}

By \href{https://www.nytimes.com/by/raymond-zhong}{Raymond Zhong}

\begin{itemize}
\item
  July 30, 2019
\item
  \begin{itemize}
  \item
  \item
  \item
  \item
  \item
  \end{itemize}
\end{itemize}

\href{https://cn.nytimes.com/technology/20190731/huawei-sales-trump-blacklist/}{阅读简体中文版}\href{https://cn.nytimes.com/technology/20190731/huawei-sales-trump-blacklist/zh-hant/}{閱讀繁體中文版}

BEIJING --- A little over two months after Huawei's chief executive
began comparing his embattled company to
\href{https://www.nytimes.com/2019/06/17/technology/huawei-trump.html}{a
bullet-riddled fighter plane}, the Chinese tech giant said its sales for
January through June grew by nearly a quarter from a year earlier, a
sign that the Trump administration's clampdown has hardly brought the
company crashing to the ground.

``Neither production nor shipment has been interrupted, not for one
single day,'' Liang Hua, the chairman of Huawei's board of directors,
said on Tuesday at the company's headquarters in the southern city of
Shenzhen. ``No matter how many difficulties we might face, we remain
confident in the company's future development.''

Still, Huawei's troubles with Washington have not left it unscathed. Its
smartphone sales outside China plummeted after the Trump administration
restricted the company's access to American technology in May, though
Mr. Liang said that they had since recovered somewhat.

Revenue for the first half of the year came in at \$58 billion. But
second-quarter sales grew by less than 13 percent from a year earlier,
compared with 39 percent growth in the first quarter. Huawei did not
indicate this in its news conference on Tuesday, highlighting only the
combined figures for the first half of the year. Because it is not a
publicly traded company, it can be selective about which financial
results to release.

Mr. Liang also warned that the second half of the year might be more
challenging than the first.

Huawei's future has been uncertain ever since Washington began
ratcheting up efforts to undermine the company, saying that its products
are dangerously susceptible to influence and disruption by the Chinese
government. Huawei rejects the insinuations.

Its fate is now entangled with talks between the United States and China
to end their yearlong tariff war. As those negotiations have swung
between optimistic highs and gloomy lows, so has the outlook for
Huawei's business.

After American officials spent months warning the world about the risks
of using Huawei's equipment to build next-generation wireless networks,
the Commerce Department
\href{https://www.nytimes.com/2019/05/16/technology/huawei-ban-president-trump.html}{took
direct aim at the company's operations} in May by putting it on an
export blacklist. This meant that American companies like Qualcomm and
Intel would need special permission to sell Huawei the microchips and
other specialized components that go into its products.

American tech suppliers swiftly halted shipments to Huawei in response.
The Chinese company's founder and chief executive, Ren Zhengfei,
predicted in June that revenue this year would be around
\href{https://www.nytimes.com/2019/06/17/technology/huawei-trump.html}{\$30
billion less} than previously forecast. That gap alone represents more
revenue than Ericsson, one of Huawei's main rivals in telecom equipment,
took in all year in 2018.

Before long, though, some American tech companies decided that they
could
\href{https://www.nytimes.com/2019/06/25/technology/huawei-trump-ban-technology.html}{resume
selling certain items} to Huawei despite the blacklisting. Their lawyers
determined that Washington's restrictions did not apply to hardware that
was manufactured outside the United States and that did not contain much
in the way of sensitive American components and technology.

``Once we determined that we could continue to resume most of those
products, it really got back to normal pretty quickly,'' Dave Pahl, the
head of investor relations at the chip maker Texas Instruments, said
during a conference call with analysts last week.

Then, after meeting with China's top leader, Xi Jinping, in Japan last
month, President Trump said the administration would permit
\href{https://www.nytimes.com/2019/06/29/world/asia/g20-trump-xi-trade-talks.html}{more
American sales to Huawei}, in a gesture meant to smooth the path toward
a trade deal.

Huawei cannot breathe easy, however. The Commerce Department still has
not indicated exactly how it will decide who gets licenses to sell to
the company. American trade negotiators were in Shanghai on Tuesday to
\href{https://www.nytimes.com/2019/07/29/us/politics/trade-china-trump.html}{try
once more to piece together an accord}.

Some of Huawei's American partners have already started making
potentially hard-to-reverse changes to their operations in response to
the blacklisting. The electronics manufacturer Flex said last week that
it had scaled down its activities in China for supplying Huawei.

At the news conference on Tuesday, Mr. Liang said that Huawei's ability
to supply equipment for next-generation 5G wireless networks had not
been affected by the Commerce Department's restrictions. The company had
made adequate preparations in advance, he said.

In fact, he said, Huawei has signed 11 additional 5G contracts since the
May blacklisting, bringing its total to 50, across 30 countries.

Mr. Liang also said Huawei's smartphone sales outside China had
recovered to some degree since the blacklisting. In the immediate wake
of the Commerce Department's action,
\href{https://www.nytimes.com/2019/05/24/business/huawei-us-china-europe-.html}{many
phone buyers in Europe and other places} shied away from Huawei handsets
out of concern that the devices would not be able to run Google's
Android operating system and other Google apps.

Huawei's handset sales have been surging in China, where many Google
services are blocked anyway. But for the rest of the world, it remains
unclear whether the Silicon Valley giant will be given permission to
serve Huawei smartphone users. Mr. Liang acknowledged that Huawei still
had work to do to ensure the health of its consumer device business,
which accounted for over half the company's revenue in the first six
months of the year.

``If the U.S. allows us to use Android, then using Android and its
ecosystem is always our preferred choice,'' Mr. Liang said. If not, he
said, Huawei is capable of developing and deploying its own replacement
operating system, although he declined to give details about the
company's efforts in this area.

For many years, Huawei's chief executive, Mr. Ren, was a reclusive
figure who almost never spoke to reporters. Of late, he has been a
near-constant presence in the global news media, sounding exuberant
about his company's prospects for surviving Mr. Trump's onslaught.

Speaking this month with a group of Italian journalists, Mr. Ren boasted
that Huawei had already patched up 70 to 80 percent of what he called
the ``bullet holes'' in its products --- the ways in which they rely on
American-sourced parts or technology.

By the end of the year, he predicted, Huawei will have filled more than
90 percent of these holes.

``We can stand on our own right now,'' Mr. Ren said. ``We don't need to
depend on the U.S. to continue serving our customers. The more advanced
a system is, the more capable we are of standing on our own.''

Advertisement

\protect\hyperlink{after-bottom}{Continue reading the main story}

\hypertarget{site-index}{%
\subsection{Site Index}\label{site-index}}

\hypertarget{site-information-navigation}{%
\subsection{Site Information
Navigation}\label{site-information-navigation}}

\begin{itemize}
\tightlist
\item
  \href{https://help.nytimes.com/hc/en-us/articles/115014792127-Copyright-notice}{©~2020~The
  New York Times Company}
\end{itemize}

\begin{itemize}
\tightlist
\item
  \href{https://www.nytco.com/}{NYTCo}
\item
  \href{https://help.nytimes.com/hc/en-us/articles/115015385887-Contact-Us}{Contact
  Us}
\item
  \href{https://www.nytco.com/careers/}{Work with us}
\item
  \href{https://nytmediakit.com/}{Advertise}
\item
  \href{http://www.tbrandstudio.com/}{T Brand Studio}
\item
  \href{https://www.nytimes.com/privacy/cookie-policy\#how-do-i-manage-trackers}{Your
  Ad Choices}
\item
  \href{https://www.nytimes.com/privacy}{Privacy}
\item
  \href{https://help.nytimes.com/hc/en-us/articles/115014893428-Terms-of-service}{Terms
  of Service}
\item
  \href{https://help.nytimes.com/hc/en-us/articles/115014893968-Terms-of-sale}{Terms
  of Sale}
\item
  \href{https://spiderbites.nytimes.com}{Site Map}
\item
  \href{https://help.nytimes.com/hc/en-us}{Help}
\item
  \href{https://www.nytimes.com/subscription?campaignId=37WXW}{Subscriptions}
\end{itemize}
