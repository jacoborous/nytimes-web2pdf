Sections

SEARCH

\protect\hyperlink{site-content}{Skip to
content}\protect\hyperlink{site-index}{Skip to site index}

\href{https://www.nytimes.com/section/movies}{Movies}

\href{https://myaccount.nytimes.com/auth/login?response_type=cookie\&client_id=vi}{}

\href{https://www.nytimes.com/section/todayspaper}{Today's Paper}

\href{/section/movies}{Movies}\textbar{}`The Lion King' Review: The Art
of Herding Digital Cats

\url{https://nyti.ms/2LIiCnM}

\begin{itemize}
\item
\item
\item
\item
\item
\item
\end{itemize}

Advertisement

\protect\hyperlink{after-top}{Continue reading the main story}

Supported by

\protect\hyperlink{after-sponsor}{Continue reading the main story}

\hypertarget{the-lion-king-review-the-art-of-herding-digital-cats}{%
\section{`The Lion King' Review: The Art of Herding Digital
Cats}\label{the-lion-king-review-the-art-of-herding-digital-cats}}

Beyoncé, Donald Glover and Seth Rogen are some of the famous voices in a
super-realistic version of the Disney and Broadway favorite.

\includegraphics{https://static01.nyt.com/images/2019/07/11/arts/11lionking1/merlin_157773891_b39eb435-6bb4-4c1f-8a26-0fae0eafe8a7-articleLarge.jpg?quality=75\&auto=webp\&disable=upscale}

\href{https://www.nytimes.com/by/a-o--scott}{\includegraphics{https://static01.nyt.com/images/2018/02/20/multimedia/author-a-o-scott/author-a-o-scott-thumbLarge.jpg}}

By \href{https://www.nytimes.com/by/a-o--scott}{A.O. Scott}

\begin{itemize}
\item
  July 11, 2019
\item
  \begin{itemize}
  \item
  \item
  \item
  \item
  \item
  \item
  \end{itemize}
\end{itemize}

\begin{itemize}
\tightlist
\item
  The Lion King\\
  Directed by Jon Favreau Animation, Adventure, Drama, Family, Musical
  PG 1h 58m
\end{itemize}

\href{https://www.imdb.com/showtimes/title/tt6105098?ref_=ref_ext_NYT}{Find
Tickets}

When you purchase a ticket for an independently reviewed film through
our site, we earn an affiliate commission.

Watching the newest version of ``The Lion King'' --- a big-screen
celebrity-voiced musical trying its best to look like a television
nature documentary --- I recalled a line from
\href{https://www.goodreads.com/book/show/137365.The_Studio}{John
Gregory
Dunne's}\href{https://www.goodreads.com/book/show/137365.The_Studio}{1969}\href{https://www.goodreads.com/book/show/137365.The_Studio}{book
``The Studio''}that may be my all-time favorite sentence in the annals
of movie writing. ``Six months were devoted to teaching Chee Chee the
Chimpanzee how to cook bacon and eggs,'' Dunne wrote, referring to a
character in \href{https://www.youtube.com/watch?v=49CmzIRX1k0}{``Doctor
Dolittle,''} one of many real animals cast in that big-budget,
family-friendly musical spectacle.

Dunne was pounding the pavement on the 20th Century Fox lot at a time of
political tension and social fracture, when the Hollywood studios seemed
to be facing an existential crisis. Pretty much like now, in other
words, except that the money and ingenuity those studios used to spend
on things like teaching chimps to make breakfast now go toward turning
lines of code into fur and sinew. This is undoubtedly an ethical
improvement, much as Chee Chee may have enjoyed hanging out with Rex
Harrison. The hope behind this ``Lion King,'' opening July 19, is that
advancing digital technology will also enhance the luster of the
moviegoing experience.

It does and it doesn't. There are a great many impressive moments in
this film, and a few that might elicit a gasp of amazement or an
appreciative burst of laughter from even a jaded viewer. For example:
When Pumbaa, the flatulent warthog voiced by Seth Rogen, absent-mindedly
scratches his left ear with his hind leg, I confess that I nearly wept.
Not because the scene was especially touching or sad, but because of the
sheer extravagant craft that had clearly gone into rendering those two
seconds of reflexive animal behavior. I was nearly as moved by the
efforts of a dung beetle to propel a ball of scat across a patch of
desert. The digital artisans responsible for these images didn't
necessarily have to do it all with such fanatical care, and the fact
that they did is surely worthy of admiration.

So if a movie could be judged solely on technique, ``The Lion King''
might qualify as a great one. And it kind of wants to be judged that way
--- for its technical skin rather than its dramatic soul. The opening
sequence (it doesn't seem right to call it a ``shot'') fools the eye in
subtle and brazen ways. You might think there are real creatures mixed
into the computer-generated menagerie (there aren't), but at the same
time the flights of animal choreography lie beyond the skill of any
trainer. Then the music starts, and it's ``The Circle of Life'' and baby
Simba is cute enough to make all the trolls on Twitter go awwwww.

The mixture of coziness and, well, awe in that opening number is as
on-brand as anything Disney has done since --- I guess since a few weeks
ago, when it released ``Toy Story 4.'' Once the voices start up, we are
in a comfortable and familiar pop-cultural space, even if the talking
beasts don't look much like cartoons. (A possible exception might be
Zazu, the hornbill who sounds like John Oliver. Or maybe it's just that
John Oliver looks like a hornbill.) The antelopes lope. The elephants
lumber. The graceful lions bask in their languorous power, lolling and
growling and setting up the parameters of the story.

\includegraphics{https://static01.nyt.com/images/2019/07/11/arts/11lionking2/merlin_157773888_a468974c-65a7-43da-9c8d-a5458d0cacb9-articleLarge.jpg?quality=75\&auto=webp\&disable=upscale}

Not many surprises there. Simba's father, Mufasa (James Earl Jones),
rules the savanna with a gentle paw and a loyal queen, Sarabi (Alfre
Woodard). Mufasa's brother, Scar (Chiwetel Ejiofor), is the snake in
this garden, scheming first to kill Simba, the rightful heir to the
throne, and then to get rid of Mufasa. Scar has the help of Shenzi
(Florence Kasumba) and her army of hyenas, whose closed-up, predatory
faces are genuinely scary, especially when they appear for the first
time.

Small children may have some trouble at that point, and also with Scar's
ruthless political machinations, which are pretty murdery for Disney.
But it's likely that much of the audience, young and old, will have some
familiarity with the narrative, whether from the
\href{https://www.nytimes.com/1994/06/15/movies/review-film-the-hero-within-the-child-within.html}{1994
animated feature} or from the
\href{https://www.nytimes.com/1997/11/14/movies/theater-review-cub-comes-of-age-a-twice-told-cosmic-tale.html}{long-running,
much-loved Broadway show}. ``The Lion King'' currently under review
isn't meant to replace or outdo either of those, but rather to multiply
revenue streams and use a beloved property to show off some new tricks.
A lot of people will go, expecting to like what they see, and for the
most part they won't be disappointed.

I said earlier that the movie, which was directed by Jon Favreau and
written by Jeff Nathanson, looks like a nature documentary. But it plays
more like an especially glitzy presentation reel at a trade convention,
with popular songs and high-end talent pushing an exciting new product
that nobody is sure quite how to use. Simba and his best friend, Nala,
voiced as cubs by JD McCrary and Shahadi Wright Joseph, grow up into
Donald Glover and Beyoncé, and when they get going on ``Can You Feel the
Love Tonight \ldots{}''

It's O.K. When Pumbaa and his pal Timon the meerkat show up --- I'm not
going to stir up trouble by saying which one might be the other's
sidekick --- we get a brisk vaudevillian double act from Rogen and Billy
Eichner. That's O.K. too. But of all the second-golden-age Disney
animated features, the original ``Lion King'' is the most Shakespearean,
as well as being the most ideologically coherent Hollywood defense of
monarchy until
\href{https://www.nytimes.com/2018/02/06/movies/black-panther-review-movie.html}{``Black
Panther.''} The grandeur and intimacy, the earthy humor and heavenly
songs have given it gravity and staying power.

Those are somehow missing here. The songs don't have the pop or the
splendor. The terror and wonder of the intra-pride battles are muted.
There is a lot of professionalism but not much heart. It may be that the
realism of the animals makes it hard to connect with them as characters,
undermining the inspired anthropomorphism that has been the most
enduring source of Disney magic.

Real lions don't sing --- not even like Beyoncé --- and don't actually
govern other creatures. The closer the movie gets to nature in its look,
the more blatant, intrusive and purposeless its artifice seems. It might
have worked better without songs or dialogue: surely the Disney wizards
could have figured out how to spin an epic tale of royal succession and
self-discovery through purely visual means. Or else someone could have
spent a few months teaching the digital Pumbaa to whip up a nice tofu
scramble.

\textbf{The Lion King}

Rated PG. Not too red in tooth and claw. Running time: 1 hour 58
minutes.

Advertisement

\protect\hyperlink{after-bottom}{Continue reading the main story}

\hypertarget{site-index}{%
\subsection{Site Index}\label{site-index}}

\hypertarget{site-information-navigation}{%
\subsection{Site Information
Navigation}\label{site-information-navigation}}

\begin{itemize}
\tightlist
\item
  \href{https://help.nytimes.com/hc/en-us/articles/115014792127-Copyright-notice}{©~2020~The
  New York Times Company}
\end{itemize}

\begin{itemize}
\tightlist
\item
  \href{https://www.nytco.com/}{NYTCo}
\item
  \href{https://help.nytimes.com/hc/en-us/articles/115015385887-Contact-Us}{Contact
  Us}
\item
  \href{https://www.nytco.com/careers/}{Work with us}
\item
  \href{https://nytmediakit.com/}{Advertise}
\item
  \href{http://www.tbrandstudio.com/}{T Brand Studio}
\item
  \href{https://www.nytimes.com/privacy/cookie-policy\#how-do-i-manage-trackers}{Your
  Ad Choices}
\item
  \href{https://www.nytimes.com/privacy}{Privacy}
\item
  \href{https://help.nytimes.com/hc/en-us/articles/115014893428-Terms-of-service}{Terms
  of Service}
\item
  \href{https://help.nytimes.com/hc/en-us/articles/115014893968-Terms-of-sale}{Terms
  of Sale}
\item
  \href{https://spiderbites.nytimes.com}{Site Map}
\item
  \href{https://help.nytimes.com/hc/en-us}{Help}
\item
  \href{https://www.nytimes.com/subscription?campaignId=37WXW}{Subscriptions}
\end{itemize}
