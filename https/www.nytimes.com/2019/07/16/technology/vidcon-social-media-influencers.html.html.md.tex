Sections

SEARCH

\protect\hyperlink{site-content}{Skip to
content}\protect\hyperlink{site-index}{Skip to site index}

\href{https://www.nytimes.com/section/technology}{Technology}

\href{https://myaccount.nytimes.com/auth/login?response_type=cookie\&client_id=vi}{}

\href{https://www.nytimes.com/section/todayspaper}{Today's Paper}

\href{/section/technology}{Technology}\textbar{}Don't Scoff at
Influencers. They're Taking Over the World.

\href{https://nyti.ms/2GeQeWO}{https://nyti.ms/2GeQeWO}

\begin{itemize}
\item
\item
\item
\item
\item
\item
\end{itemize}

Advertisement

\protect\hyperlink{after-top}{Continue reading the main story}

Supported by

\protect\hyperlink{after-sponsor}{Continue reading the main story}

The Shift

\hypertarget{dont-scoff-at-influencers-theyre-taking-over-the-world}{%
\section{Don't Scoff at Influencers. They're Taking Over the
World.}\label{dont-scoff-at-influencers-theyre-taking-over-the-world}}

As social media expands its cultural dominance, the people who can steer
the online conversation will have an upper hand.

\includegraphics{https://static01.nyt.com/images/2019/07/16/business/16roose2/merlin_157990155_ae5a0bab-fb2f-490c-9288-7825ae752695-articleLarge.jpg?quality=75\&auto=webp\&disable=upscale}

\href{https://www.nytimes.com/by/kevin-roose}{\includegraphics{https://static01.nyt.com/images/2018/02/20/multimedia/author-kevin-roose/author-kevin-roose-thumbLarge.jpg}}

By \href{https://www.nytimes.com/by/kevin-roose}{Kevin Roose}

\begin{itemize}
\item
  July 16, 2019
\item
  \begin{itemize}
  \item
  \item
  \item
  \item
  \item
  \item
  \end{itemize}
\end{itemize}

\href{https://cn.nytimes.com/technology/20190718/vidcon-social-media-influencers/}{阅读简体中文版}\href{https://cn.nytimes.com/technology/20190718/vidcon-social-media-influencers/zh-hant/}{閱讀繁體中文版}\href{https://www.nytimes.com/es/2019/07/19/influencers-influentes-youtubers-instagram}{Leer
en español}

ANAHEIM, Calif. --- When the first TikTok star is elected president, I
hope she will save some room in her cabinet for older and more
conventional bureaucrats, even if they don't have millions of followers,
great hair or amazing dance moves.

I say ``when,'' not ``if,'' because I just spent three days at VidCon,
the annual social media convention in Anaheim, hanging out with a few
thousand current and future internet celebrities. And it's increasingly
obvious to me that the teenagers and 20-somethings who have mastered
these platforms --- and who are often dismissed as shallow, preening
narcissists by adults who don't know any better --- are going to
dominate not just internet culture or the entertainment industry but
society as a whole.

On the surface, this can be a terrifying proposition. One day at VidCon,
I hung out with a crew of teenage Instagram stars, who seemed to spend
most of their time filming ``collabs'' with other creators and
complimenting one another on their ``drip,'' influencer-speak for
clothes and accessories. (In their case, head-to-toe Gucci and
Balenciaga outfits with diamond necklaces and designer sneakers.)
Another day, I
\href{https://twitter.com/kevinroose/status/1149852860022050817}{witnessed}
an awkward dance battle between two budding TikTok influencers, neither
of whom could have been older than 10. (Adults who are just catching up:
\href{https://www.nytimes.com/2019/03/10/style/what-is-tik-tok.html}{TikTok
is a short-form video app} owned by the Chinese internet company
Bytedance.)

But if you can look past the silliness and status-seeking, many people
at VidCon are hard at work. Being an influencer can be an
\href{https://www.nytimes.com/2019/07/09/style/emma-chamberlain-youtube.html}{exhausting,
burnout-inducing job}, and the people who are good at it have typically
spent years working their way up the ladder. Many social media
influencers are essentially one-person start-ups, and the best ones can
spot trends, experiment relentlessly with new formats and platforms,
build an authentic connection with an audience, pay close attention to
their channel analytics, and figure out how to distinguish themselves in
a crowded media environment --- all while churning out a constant stream
of new content.

Not all influencers are brilliant polymaths, of course. Some of them
have succeeded by virtue of being conventionally attractive, or good at
video games, or in possession of some other surface-level attribute.
Others have made their names with dubious stunts and
\href{https://www.nytimes.com/interactive/2019/06/08/technology/youtube-radical.html}{extreme
political commentary}.

But as social media expands its cultural dominance, the people who can
steer the online conversation will have an upper hand in whatever niche
they occupy --- whether that's media, politics, business or some other
field.

``The way to think of influencers or creators is as entrepreneurs,''
said Chris Stokel-Walker, the author of ``YouTubers.'' ``These people
are setting up businesses, hiring staff, managing budgets. These are
massively transferable skills.''

Just look at Representative Alexandria Ocasio-Cortez, the New York
Democrat who has become a powerful force in Congress by pairing her
policy agenda with an intuitive understanding of what works online. Or
look at what's happening in Brazil, where
\href{https://www.buzzfeednews.com/article/ryanhatesthis/brazils-congressional-youtubers}{YouTubers
are winning political elections} by mobilizing their online fan bases.

In the business world, influencer culture is already an established
force. A generation of direct-to-consumer brands that were built using
the tools and tactics of social media has skyrocketed to success ---
like Glossier, the influencer-beloved beauty company that recently
raised \$100 million at a valuation of more than \$1 billion, or Away,
the luggage start-up whose ubiquitous Instagram ads helped it reach a
valuation of \$1.4 billion. Many social media stars strike endorsement
deals with major brands, in addition to earning money through
advertising and merchandise sales. And even executives in sleepy,
old-line industries now hire ``personal branding consultants'' to help
increase their online followings.

Natalie Alzate, a YouTuber with more than 10 million subscribers who
goes by \href{https://www.youtube.com/user/nataliesoutlet}{Natalies
Outlet,} is an example of the wave of influencers who treated their
online brand-building as a business rather than a fun hobby. Four years
ago, when Ms. Alzate first came to VidCon, she was a marketing student
with fewer than 7,000 subscribers. She decided to study her favorite
YouTubers, watch how they made their videos and then test videos in
multiple genres, seeing which ones performed best on her channel.

\includegraphics{https://static01.nyt.com/images/2019/07/16/business/16roose1/merlin_157990158_e884758e-6684-49ef-af87-678ad8e9f95c-articleLarge.jpg?quality=75\&auto=webp\&disable=upscale}

``I grew up watching people, like Michelle Phan, that were building
legacies out of, honestly, just being really relatable online,'' Ms.
Alzate said. ``It was always an aspiration.''

Eventually, she hit on formats --- like beauty tips and lifehacks ---
that reliably performed well, and she was off to the races. Today, she
is a full-time YouTuber with a small staff, a production studio and the
kind of fame she once coveted.

In truth, influencers have been running the world for years. We just
haven't called them that. Instead, we called them ``movie stars'' or
``talk-radio hosts'' or ``Davos attendees.'' The ability to stay
relevant and attract attention to your work has always been critical.
And who, aside from perhaps President Trump, is better at getting
attention than a YouTube star?

VidCon, which started 10 years ago as a meet-and-greet event for popular
YouTubers, is a perfect place to observe influencers in their natural
habitat. And many of them were here to promote their channels, to
network with other creators and to make strides toward the dream of
internet fame.

Sometimes, that meant appearing in photos and videos with more popular
influencers in an attempt to increase their own following, a practice
known in influencer circles as ``clout chasing.'' Other times, it meant
going to panels with titles like ``Curating Your Personal Brand'' and
``How to Go Viral and Build an Audience.'' For VidCon's featured
creators, the super-famous ones with millions of followers, it can mean
spending the day at a meet-and-greet with fans before going out to
V.I.P. parties at night.

Not all of the young people I met at VidCon will spend their whole lives
pursuing internet fame. Some of them will grow up, go off to college and
wind up becoming doctors, lawyers or accountants. Some will fizzle out
and be replaced by a younger generation of internet stars.

But the lessons they learned from performing on YouTube, Instagram, and
TikTok will stick with them, regardless of where they end up. Just as
the 20th century groomed a generation of children steeped in the ethos
of TV culture, the 21st century will produce a generation of business
moguls, politicians and media figures who grew up chasing clout online
and understand how to operate the levers of the attention economy.

``In the early days, it felt like this was a sub-niche of youth
culture,'' Beau Bryant, the general manager of talent at Fullscreen, a
management agency for digital creators, told me at VidCon. He gestured
around at a room filled with influencers sitting on velvet couches. Some
were taking selfies and editing their Instagram stories. Others were
holding business meetings about partnerships and sponsored content
deals.

``Now, it just feels like this is what youth culture is,'' Mr. Bryant
said.

In other words, influencers are the future. Dismiss them at your peril.

Advertisement

\protect\hyperlink{after-bottom}{Continue reading the main story}

\hypertarget{site-index}{%
\subsection{Site Index}\label{site-index}}

\hypertarget{site-information-navigation}{%
\subsection{Site Information
Navigation}\label{site-information-navigation}}

\begin{itemize}
\tightlist
\item
  \href{https://help.nytimes.com/hc/en-us/articles/115014792127-Copyright-notice}{©~2020~The
  New York Times Company}
\end{itemize}

\begin{itemize}
\tightlist
\item
  \href{https://www.nytco.com/}{NYTCo}
\item
  \href{https://help.nytimes.com/hc/en-us/articles/115015385887-Contact-Us}{Contact
  Us}
\item
  \href{https://www.nytco.com/careers/}{Work with us}
\item
  \href{https://nytmediakit.com/}{Advertise}
\item
  \href{http://www.tbrandstudio.com/}{T Brand Studio}
\item
  \href{https://www.nytimes.com/privacy/cookie-policy\#how-do-i-manage-trackers}{Your
  Ad Choices}
\item
  \href{https://www.nytimes.com/privacy}{Privacy}
\item
  \href{https://help.nytimes.com/hc/en-us/articles/115014893428-Terms-of-service}{Terms
  of Service}
\item
  \href{https://help.nytimes.com/hc/en-us/articles/115014893968-Terms-of-sale}{Terms
  of Sale}
\item
  \href{https://spiderbites.nytimes.com}{Site Map}
\item
  \href{https://help.nytimes.com/hc/en-us}{Help}
\item
  \href{https://www.nytimes.com/subscription?campaignId=37WXW}{Subscriptions}
\end{itemize}
