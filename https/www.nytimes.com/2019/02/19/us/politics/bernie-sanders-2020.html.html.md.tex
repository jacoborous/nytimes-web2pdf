Sections

SEARCH

\protect\hyperlink{site-content}{Skip to
content}\protect\hyperlink{site-index}{Skip to site index}

\href{https://www.nytimes.com/section/politics}{Politics}

\href{https://myaccount.nytimes.com/auth/login?response_type=cookie\&client_id=vi}{}

\href{https://www.nytimes.com/section/todayspaper}{Today's Paper}

\href{/section/politics}{Politics}\textbar{}Bernie Sanders, Once the
Progressive Outlier, Joins a Crowded Presidential Field

\url{https://nyti.ms/2Ejjrze}

\begin{itemize}
\item
\item
\item
\item
\item
\item
\end{itemize}

\begin{itemize}
\item
  \href{https://www.nytimes.com/2020/07/31/us/elections/biden-vs-trump.html?action=click\&pgtype=Article\&state=default\&region=TOP_BANNER\&context=storylines_menu}{Election
  Updates}
\item
  \href{https://www.nytimes.com/article/biden-vice-president-2020.html?action=click\&pgtype=Article\&state=default\&region=TOP_BANNER\&context=storylines_menu}{Biden's
  V.P. Search}
\item
  \href{https://www.nytimes.com/interactive/2020/07/24/us/politics/trump-biden-campaign-donors.html?action=click\&pgtype=Article\&state=default\&region=TOP_BANNER\&context=storylines_menu}{Map
  of Donations}
\item
  \href{https://www.nytimes.com/interactive/2020/us/elections/delegate-count-primary-results.html?action=click\&pgtype=Article\&state=default\&region=TOP_BANNER\&context=storylines_menu}{Delegate
  Count}
\item
  \href{https://www.nytimes.com/interactive/2019/us/politics/2020-presidential-candidates.html?action=click\&pgtype=Article\&state=default\&region=TOP_BANNER\&context=storylines_menu}{The
  Candidates}
\item
  \href{https://www.nytimes.com/newsletters/politics?action=click\&pgtype=Article\&state=default\&region=TOP_BANNER\&context=storylines_menu}{Politics
  Newsletter}
\end{itemize}

Advertisement

\protect\hyperlink{after-top}{Continue reading the main story}

Supported by

\protect\hyperlink{after-sponsor}{Continue reading the main story}

\hypertarget{bernie-sanders-once-the-progressive-outlier-joins-a-crowded-presidential-field}{%
\section{Bernie Sanders, Once the Progressive Outlier, Joins a Crowded
Presidential
Field}\label{bernie-sanders-once-the-progressive-outlier-joins-a-crowded-presidential-field}}

\includegraphics{https://static01.nyt.com/images/2019/01/23/us/politics/xxsandersrunningHFO/xxsandersrunningHFO-videoSixteenByNine3000.jpg}

By \href{https://www.nytimes.com/by/sydney-ember}{Sydney Ember}

\begin{itemize}
\item
  Feb. 19, 2019
\item
  \begin{itemize}
  \item
  \item
  \item
  \item
  \item
  \item
  \end{itemize}
\end{itemize}

Senator Bernie Sanders, the Vermont independent and 2016 Democratic
primary runner-up whose populist agenda has helped push the party to the
left, embarked on Tuesday on a second run for president, in a bid that
will test
\href{https://www.nytimes.com/2019/02/19/us/politics/on-politics-bernie-2020.html}{whether
he retains his anti-establishment appeal or loses ground to newer faces
who have adopted many of his ideas.}

A professed democratic socialist whose calls for
``\href{https://www.nytimes.com/2019/02/02/us/politics/medicare-for-all-2020.html}{Medicare
for all},'' a \$15 minimum wage and tuition-free public colleges have
become pillars of the party's left wing, Mr. Sanders joins the race at a
time when Republicans are trying to define the Democratic field and its
ideas as out of the political mainstream. In Mr. Sanders, who has not
joined the Democratic Party, Republicans have an easy target to try to
make the face of the opposition.

But Mr. Sanders, 77, starts with stronger support from small-dollar
donors and liberal voters than most other candidates. And he is among
the best-known Democrats in a crowded field, as well as one of the most
outspoken against President Trump, whom he has repeatedly called a
``pathological liar'' and a ``racist.''

``During our 2016 campaign, when we brought forth our progressive
agenda, we were told that our ideas were `radical' and `extreme,''' Mr.
Sanders said on Tuesday in an email to supporters. ``Three years have
come and gone. And, as result of millions of Americans standing up and
fighting back, all of these policies and more are now supported by a
majority of Americans.''

\emph{{[}Update:}
\href{https://www.nytimes.com/2019/04/14/us/politics/bernie-sanders-2020-candidates.html?action=click\&module=Intentional\&pgtype=Article}{\emph{Bernie
Sanders accuses liberal think tank of smearing progressive
candidates}}\emph{.{]}}

This time around, Mr. Sanders, enters the race at a far different
electoral moment. Much of his populist agenda has been embraced by other
Democrats, at a time when many voters are eager to elevate female and
nonwhite standard bearers. He will no longer have the Clinton dynasty as
a foil; instead, his competition will include progressives like Senator
Elizabeth Warren of Massachusetts, who has broadly supported many of the
same economic positions for years.

And he will face far more scrutiny than three years ago, when much of
the news media and political class treated him as more of an outlier
than as a genuine challenger for the nomination. Already, he has had to
\href{https://www.nytimes.com/2019/01/02/us/politics/bernie-sanders-campaign-sexism.html}{quell
the unease about his campaign's treatment of women} that has been
disclosed in recent weeks and prompted two public apologies. His
stumbles on issues of race and identity continue to concern activists
who fear he has learned little from his previous White House bid.

Whether Mr. Sanders can break through in a crowded field of diverse
candidates, many of whom champion the progressive message he made
popular, could go a long way in determining the direction of the
Democratic Party in the age of Trump.

``A lot of people still believe that he is the one who can take Trump
out,'' said Yvette Simpson, chief executive of the political group
Democracy for America. The question now, she said, is ``how does he
distinguish himself in that bigger field?''

\href{https://www.nytimes.com/interactive/2019/02/14/us/politics/2020-democratic-candidates-president.html}{}

\includegraphics{https://static01.nyt.com/images/2019/02/13/us/2020-democratic-candidates-president-promo-1550099345708/2020-democratic-candidates-president-promo-1550099345708-articleLarge-v15.png}

\hypertarget{how-the-democratic-presidential-field-has-narrowed}{%
\subsection{How the Democratic Presidential Field Has
Narrowed}\label{how-the-democratic-presidential-field-has-narrowed}}

See how the current Democratic presidential field compares with past
cycles.

Republicans have seized on Mr. Sanders's entrance, eager to ascribe the
socialist label to all Democrats. Soon after his announcement, the Trump
re-election campaign issued a statement denouncing ``every'' Democratic
candidate for ``embracing his brand of socialism.'' The president said
pointedly that Mr. Sanders ``missed his time.''

In an interview on ``CBS This Morning,'' Mr. Sanders did not shy away
from calling himself a democratic socialist. Mr. Trump, Mr. Sanders
said, is ``going to say, `Bernie Sanders wants the United States to
become Venezuela.'''

\hypertarget{latest-updates-2020-election}{%
\section{\texorpdfstring{\href{https://www.nytimes.com/2020/07/31/us/elections/biden-vs-trump.html?action=click\&pgtype=Article\&state=default\&region=MAIN_CONTENT_1\&context=storylines_live_updates}{Latest
Updates: 2020
Election}}{Latest Updates: 2020 Election}}\label{latest-updates-2020-election}}

Updated 2020-08-01T01:26:45.732Z

\begin{itemize}
\tightlist
\item
  \href{https://www.nytimes.com/2020/07/31/us/elections/biden-vs-trump.html?action=click\&pgtype=Article\&state=default\&region=MAIN_CONTENT_1\&context=storylines_live_updates\#link-29fdff45}{Kamala
  Harris, a top vice-presidential contender, confronts double
  standards.}
\item
  \href{https://www.nytimes.com/2020/07/31/us/elections/biden-vs-trump.html?action=click\&pgtype=Article\&state=default\&region=MAIN_CONTENT_1\&context=storylines_live_updates\#link-13ec3d9c}{Karen
  Bass and Susan Rice are rising on Biden's vice-presidential
  shortlist.}
\item
  \href{https://www.nytimes.com/2020/07/31/us/elections/biden-vs-trump.html?action=click\&pgtype=Article\&state=default\&region=MAIN_CONTENT_1\&context=storylines_live_updates\#link-49e9a016}{Trump
  says Russian bounties to kill U.S. troops `never took place.'}
\end{itemize}

\href{https://www.nytimes.com/2020/07/31/us/elections/biden-vs-trump.html?action=click\&pgtype=Article\&state=default\&region=MAIN_CONTENT_1\&context=storylines_live_updates}{See
more updates}

``Bernie Sanders does not want to have the United States become the
horrific economic situation that unfortunately exists in Venezuela right
now,'' he said. ``What Bernie Sanders wants is to learn from countries
around the world why other countries are doing a better job of dealing
with income and wealth inequality than we are.''

Mr. Sanders will start with several advantages, including the foundation
of a 50-state organization;
\href{https://www.nytimes.com/2019/02/09/us/politics/2020-democrats-campaign-funding.html}{a
massive lead among low-dollar donors} that is roughly equivalent to the
donor base of all the other Democratic hopefuls combined; and a cache of
fervent, unwavering supporters who spent the day exulting in his
decision to run. A coveted speaker, he can still electrify crowds in a
way few politicians can. He enjoys wide name recognition, and several
early polls had him running second behind former Vice President Joseph
R. Biden Jr.

The strength of that donor list was quickly apparent: In the 12 hours
after Mr. Sanders formally joined the race, his campaign said it had
pulled in more than \$4 million.

{[}\href{https://www.nytimes.com/interactive/2019/us/politics/2020-presidential-candidates.html?action=click\&module=Intentional\&pgtype=Article}{\emph{Check
out the 2020 Democratic field with our candidate tracker.}}{]}

And while rising stars like
\href{https://www.nytimes.com/2019/01/13/nyregion/ocasio-cortez-democrats-congress.html}{Alexandria
Ocasio-Cortez} and Ayanna Pressley have siphoned some of his authority
over the party's progressive wing, Mr. Sanders still claims to have
spawned a ``political revolution'' that, true revolution or not, has
ignited a generation of young, socialist-leaning voters and reshaped the
Democratic Party.

He is also partly responsible for the party's decision last year to
\href{https://www.nytimes.com/2018/08/25/us/politics/superdelegates-democrats-dnc.html}{overhaul
its presidential nomination process}, including sharply reducing the
influence of superdelegates and increasing the transparency around
debates --- factors he felt greatly favored Hillary Clinton in 2016.

Asked what would be different in 2020, Mr. Sanders replied bluntly:
``We're going to win.''

``Bottom line,'' he said, ``it is absolutely imperative that Donald
Trump be defeated.'' Though he had harsh words for the president, he
said he was fond of
\href{https://www.nytimes.com/interactive/2019/us/politics/2020-presidential-candidates.html}{the
five other senators who were running} for the Democratic nomination.
``They are in some cases my friends,'' he said.

\includegraphics{https://static01.nyt.com/images/2019/03/18/us/politics/xxvid-bernie-2020/xxvid-bernie-2020-videoSixteenByNine3000-v4.jpg}

Mr. Sanders did not immediately announce where he would campaign first.
Faiz Shakir, the national political director at the A.C.L.U. and a
former adviser to Senator Harry Reid of Nevada, the former majority
leader, will serve as campaign manager.

With his booming voice and familiar wide-armed grip at the lectern, Mr.
Sanders has long positioned himself as a champion of the working class
and a passionate opponent of Wall Street and the moneyed elite. His
remarks often include diatribes against the millionaires and
billionaires --- one of his most common refrains is that the ``three
wealthiest people in America own more wealth than the bottom 50
percent'' --- as well as denunciations of ``super PACs'' and the
influence of big money on politics. In particular, he has sharply
criticized Amazon and Walmart over their wages and treatment of workers.

{[}\emph{Make sense of the people, issues and ideas}
\href{https://www.nytimes.com/newsletters/politics?smid=rd?action=click\&module=Intentional\&pgtype=Article}{\emph{shaping
American politics with our newsletter.}}{]}

In his email to supporters, as well as
\href{https://twitter.com/BernieSanders/status/1097828878310096901}{a
campaign announcement video}, Mr. Sanders laid out a litany of policy
issues, familiar to anyone who has followed him: universal health care,
tuition-free public college, women's reproductive rights, lower
prescription drug prices, criminal justice reform.

``Our campaign is about taking on the powerful special interests that
dominate our economic and political life,'' he said.

While some presidential candidates have avoided direct broadsides
against President Trump, Mr. Sanders, ever combative, addressed his
potential opponent head on.

``We are running against a president who is a pathological liar, a
fraud, a racist, a sexist, a xenophobe and someone who is undermining
American democracy as he leads us in an authoritarian direction,'' he
said.

Born in Brooklyn, with an accent to match, Mr. Sanders ran
unsuccessfully in the 1970s for governor and United States senator in
Vermont before being elected mayor of Burlington in 1981. For 16 years,
he served as the only congressman in the state before he was elected to
the Senate in 2006.

Mr. Sanders has been a
\href{https://www.nytimes.com/2016/03/15/us/politics/bernie-sanders-amendments.html}{modest
legislator} and something of a lone wolf in Washington, promoting
largely the same legislative agenda since his early days as a mayor. He
voted against the Iraq War and, in 2008, was one of roughly two dozen
senators to vote against the \$700 billion bailout of big banks.

\includegraphics{https://static01.nyt.com/images/2019/02/20/us/20sandersruns-HFO2/20sandersruns-HFO2-articleLarge.jpg?quality=75\&auto=webp\&disable=upscale}

While often viewed as a pesky left-wing gadfly, he is also known to
reach across the aisle, working on legislation with Senator Charles E.
Grassley of Iowa and Senator John McCain of Arizona, both Republicans.
He has
\href{https://www.washingtonpost.com/news/post-politics/wp/2016/02/12/1994-crime-bill-haunts-clinton-and-sanders-as-criminal-justice-reform-rises-to-top-in-democratic-contest/?utm_term=.93a0a5d9f1ce}{rationalized
voting} for the 1994 crime bill, now heavily criticized for some of its
draconian provisions, by saying he had favored progressive parts of the
bill, including the Violence Against Women Act, while strongly opposing
measures that would lead to mass incarceration.

Mr. Sanders is the longest-serving independent in congressional history,
a point of pride for him but one of consternation and annoyance for some
Democrats who are quick to suggest he does not have the party's
interests at heart. Some Democrats blame him for Mrs. Clinton's loss in
2016, saying his anti-establishment rhetoric during his campaign
inflamed divisions in the party that proved insurmountable.

One 2016 issue that has resurfaced is his record on gun control,
Democratic strategists have said, given the intensity of the debate
around gun violence following recent mass shootings. In 2005, Mr.
Sanders voted for a law that granted immunity to gun manufacturers and
dealers from most liability lawsuits. Mr. Sanders has also been
criticized for support he received
\href{https://www.washingtonpost.com/politics/how-the-nra-helped-put-bernie-sanders-in-congress/2015/07/19/ed1be26c-2bfe-11e5-bd33-395c05608059_story.html?utm_term=.36a779048ad8}{from
the N.R.A.} when he was running for Congress in 1990, in part because he
vowed not to support a bill that mandated a waiting period for handgun
sales.

Though his message is well worn, Mr. Sanders has indicated that he is
trying to remedy weaknesses from his first presidential campaign,
including his
\href{https://www.nytimes.com/2018/04/05/us/politics/bernie-sanders-obama-mississippi.html}{lack
of support from black voters}. In recent months, he has made a series of
trips to the South, where in 2016 he drew less than 20 percent of the
black vote. During the Martin Luther King Jr. holiday,
\href{https://www.nytimes.com/2019/01/22/us/politics/south-carolina-democrats-president.html}{he
made a swing through South Carolina} --- where black voters made up
about 60 percent of the Democratic primary vote in 2016 --- that
included addressing supporters and students and speaking with lawmakers.

He has also tried to shore up his foreign policy credentials, becoming a
critic of the United States support of Saudi Arabia's war in Yemen. Late
last year, the Senate
\href{https://www.nytimes.com/2018/12/13/us/politics/yemen-saudi-war-pompeo-mattis.html}{passed
a resolution}, which Mr. Sanders helped introduce, to end American
military assistance for the kingdom's war there.

But he also continues to draw ire from critics who say that the way he
talks about race and identity is out of step with the calls for
diversity and change within the party.

Almost immediately after making his announcement, Mr. Sanders was
attacked for his response to a question on Vermont Public Radio about
whether he thought he best represented the current Democratic Party.

``We have got to look at candidates, you know, not by the color of their
skin, not by their sexual orientation or their gender and not by their
age,''
\href{https://www.vpr.org/post/hes-2020-bernie-sanders-running-president-again\#stream/0}{Mr.
Sanders said}. ``I think we have got to try to move us toward a
nondiscriminatory society which looks at people based on their
abilities, based on what they stand for.''

Whether Mr. Sanders can secure the Democratic nomination may depend on
his ability to change his timeworn message --- and adapt to the demands
and yearnings of a party that he helped shape.

``Before 2016, nobody was really talking about universal health care and
universal education as much as he was; he really brought that into the
mainstream for a lot of people,'' said Dave Degner, the chair of the
Tama County Democratic Party in Iowa. ``If he comes out saying the same
things that he did before, it's not new anymore.''

\hypertarget{our-2020-election-guide}{%
\section{Our 2020 Election Guide}\label{our-2020-election-guide}}

Updated July 31, 2020

\begin{itemize}
\item
  \begin{center}\rule{0.5\linewidth}{\linethickness}\end{center}

  \hypertarget{the-latest}{%
  \subsection{The Latest}\label{the-latest}}

  \begin{itemize}
  \tightlist
  \item
    President Trump's assault on the Postal Service is intersecting with
    his attacks on mail-in voting.
    \href{https://www.nytimes.com/2020/07/31/us/politics/trump-usps-mail-delays.html?action=click\&pgtype=Article\&state=default\&region=BELOW_MAIN_CONTENT\&context=storylines_guide}{Voting
    rights groups say it is a recipe for disaster.}
  \end{itemize}
\item
  \begin{center}\rule{0.5\linewidth}{\linethickness}\end{center}

  \hypertarget{bidens-vp-search}{%
  \subsection{Biden's V.P. Search}\label{bidens-vp-search}}

  \begin{itemize}
  \tightlist
  \item
    \href{https://www.nytimes.com/article/biden-vice-president-2020.html?action=click\&pgtype=Article\&state=default\&region=BELOW_MAIN_CONTENT\&context=storylines_guide}{Here
    are 13 women} who have been under consideration to be Joe Biden's
    running mate, and why each might be chosen --- and might not be.
  \end{itemize}
\item
  \begin{center}\rule{0.5\linewidth}{\linethickness}\end{center}

  \hypertarget{keep-up-with-our-coverage}{%
  \subsection{Keep Up With Our
  Coverage}\label{keep-up-with-our-coverage}}

  \begin{itemize}
  \tightlist
  \item
    Get an
    \href{https://www.nytimes.com/newsletters/politics?action=click\&pgtype=Article\&state=default\&region=BELOW_MAIN_CONTENT\&context=storylines_guide}{email}
    recapping the day's news
  \end{itemize}

  \begin{itemize}
  \tightlist
  \item
    Download our mobile app on
    \href{https://apps.apple.com/us/app/nytimes/id284862083?ls=1\&mat_click_id=5c79ae7455014fd1bd66b5610c05b8f2-20191112-16948\&referrer=mat_click_id\%3D5c79ae7455014fd1bd66b5610c05b8f2-20191112-16948\%26link_click_id\%3D722930677036718082}{iOS}
    and
    \href{http://a.localytics.com/android?id=com.nytimes.android\&referrer=utm_source\%3Dother_nyt_mobile_web\%26utm_medium\%3DWeb\%2520page\%26utm_term\%3DGeneral\%2520Mobile\%2520Page\%26utm_campaign\%3DNYT\%2520Mobile\%2520General\%2520Page}{Android}
    and turn on Breaking News and Politics alerts
  \end{itemize}
\end{itemize}

Advertisement

\protect\hyperlink{after-bottom}{Continue reading the main story}

\hypertarget{site-index}{%
\subsection{Site Index}\label{site-index}}

\hypertarget{site-information-navigation}{%
\subsection{Site Information
Navigation}\label{site-information-navigation}}

\begin{itemize}
\tightlist
\item
  \href{https://help.nytimes.com/hc/en-us/articles/115014792127-Copyright-notice}{©~2020~The
  New York Times Company}
\end{itemize}

\begin{itemize}
\tightlist
\item
  \href{https://www.nytco.com/}{NYTCo}
\item
  \href{https://help.nytimes.com/hc/en-us/articles/115015385887-Contact-Us}{Contact
  Us}
\item
  \href{https://www.nytco.com/careers/}{Work with us}
\item
  \href{https://nytmediakit.com/}{Advertise}
\item
  \href{http://www.tbrandstudio.com/}{T Brand Studio}
\item
  \href{https://www.nytimes.com/privacy/cookie-policy\#how-do-i-manage-trackers}{Your
  Ad Choices}
\item
  \href{https://www.nytimes.com/privacy}{Privacy}
\item
  \href{https://help.nytimes.com/hc/en-us/articles/115014893428-Terms-of-service}{Terms
  of Service}
\item
  \href{https://help.nytimes.com/hc/en-us/articles/115014893968-Terms-of-sale}{Terms
  of Sale}
\item
  \href{https://spiderbites.nytimes.com}{Site Map}
\item
  \href{https://help.nytimes.com/hc/en-us}{Help}
\item
  \href{https://www.nytimes.com/subscription?campaignId=37WXW}{Subscriptions}
\end{itemize}
