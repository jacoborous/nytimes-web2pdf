Sections

SEARCH

\protect\hyperlink{site-content}{Skip to
content}\protect\hyperlink{site-index}{Skip to site index}

\href{https://www.nytimes.com/section/nyregion}{New York}

\href{https://myaccount.nytimes.com/auth/login?response_type=cookie\&client_id=vi}{}

\href{https://www.nytimes.com/section/todayspaper}{Today's Paper}

\href{/section/nyregion}{New York}\textbar{}Amazon Pulls Out of Planned
New York City Headquarters

\url{https://nyti.ms/2UYaeSh}

\begin{itemize}
\item
\item
\item
\item
\item
\item
\end{itemize}

Advertisement

\protect\hyperlink{after-top}{Continue reading the main story}

Supported by

\protect\hyperlink{after-sponsor}{Continue reading the main story}

\hypertarget{amazon-pulls-out-of-planned-new-york-city-headquarters}{%
\section{Amazon Pulls Out of Planned New York City
Headquarters}\label{amazon-pulls-out-of-planned-new-york-city-headquarters}}

\includegraphics{https://static01.nyt.com/images/2019/02/14/nyregion/14NYAMAZON2/merlin_146867472_15c39bb2-67e7-40b9-9ae6-ef4693dd2ce9-videoSixteenByNine3000.jpg}

By \href{https://www.nytimes.com/by/j-david-goodman}{J. David Goodman}

\begin{itemize}
\item
  Feb. 14, 2019
\item
  \begin{itemize}
  \item
  \item
  \item
  \item
  \item
  \item
  \end{itemize}
\end{itemize}

\emph{{[}What you need to know to start the day:}
\href{https://www.nytimes.com/newsletters/newyorktoday?module=inline}{\emph{Get
New York Today in your inbox.}}\emph{{]}}

Amazon on Thursday canceled its plans to build an expansive corporate
campus in New York City after facing an unexpectedly fierce backlash
from lawmakers, progressive activists and union leaders, who contended
that a tech giant did not deserve nearly \$3 billion in government
incentives.

The decision was an abrupt turnabout by Amazon after a much-publicized
search for a second headquarters, which had ended with its announcement
in November that it would open two new sites --- one in Queens, with
more than 25,000 jobs, and another in Virginia.

Amazon's retreat was a blow to Gov. Andrew M. Cuomo and Mayor Bill de
Blasio, damaging their effort to further diversify the city's economy by
making it an inviting location for the technology industry.

The agreement to lure Amazon to Long Island City, Queens, had stirred
intense debate in New York about the use of public subsidies to entice
wealthy companies, the rising cost of living in gentrifying
neighborhoods, and the city's very identity.

``A number of state and local politicians have made it clear that they
oppose our presence and will not work with us to build the type of
relationships that are required to go forward,'' Amazon said in a
statement.

The company made its decision late Wednesday, after growing increasingly
concerned that the backlash in New York showed no sign of abating and
was tarnishing its image beyond the city, according to two people with
knowledge of the discussions inside the company.

In recent days, Mr. de Blasio had tried to reach Jeff Bezos, Amazon's
chief executive, according to one official. But Mr. Bezos did not speak
with him, nor with Mr. Cuomo.

\emph{{[}What you need to know to start the day:}
\href{https://www.nytimes.com/newsletters/newyorktoday?module=inline}{\emph{Get
New York Today in your inbox}}\emph{.{]}}

The company's decision was at least a short-term win for insurgent
progressive politicians led by Representative Alexandria Ocasio-Cortez,
whose upset victory last year occurred in the western corner of Queens
where Amazon had planned its site.

Her race galvanized the party's left flank, which mobilized against the
deal, helped swing New York's Legislature into Democratic hands, and
struck fear in the hearts of some local politicians.

On Thursday, Ms. Ocasio-Cortez seemed to revel in Amazon's decision,
writing on Twitter that ``anything is possible.''

\includegraphics{https://static01.nyt.com/images/2019/02/14/nyregion/14nyamazon/14nyamazon-articleLarge-v2.jpg?quality=75\&auto=webp\&disable=upscale}

Not only progressive activists took issue with the Amazon deal: Michael
R. Bloomberg, who championed New York City as a technology hub while
mayor, questioned the incentive package earlier this month.

The company also had its supporters --- in the city's business
community, among some unions and within nearby public housing, where
some residents were hopeful that the project would bring jobs. A pair of
polls showed broad support around the city and state.

But in the end, it was not enough to persuade the company to ride out
the torrent of negative attention.

Amazon did not inform the governor or the mayor of its decision to pull
out until Thursday morning, shortly before the company posted its
announcement online.

Mr. Cuomo and Mr. de Blasio reacted in starkly different ways. The
governor blamed the newly emboldened Democrats who now control the State
Senate for derailing the project.

``A small group of politicians put their own narrow political interests
above their community --- which poll after poll showed overwhelmingly
supported bringing Amazon to Long Island City --- the state's economic
future and the best interests of the people of this state,'' the
governor said in a statement.

For his part, Mr. de Blasio turned on the company after having
steadfastly backed the deal.

``We gave Amazon the opportunity to be a good neighbor and do business
in the greatest city in the world,'' Mr. de Blasio said. ``Instead of
working with the community, Amazon threw away that opportunity.''

The mayor and the governor, who only rarely find common cause, met
Monday in Albany and discussed how to save the deal, which had appeared
increasingly imperiled, according to a person familiar with the
conversations. After the meeting, Mr. de Blasio spoke to a senior Amazon
executive by phone and was told that the company remained committed to
New York, the person said.

Both the mayor's and the governor's offices reassured Amazon executives
that, despite the vocal criticism, the deal they had negotiated would be
approved. But the company appeared upset at even a moderate level of
resistance, said the person, who, like many of the people describing
private conversations at the company and with elected officials, did so
on the condition of anonymity.

A decisive moment appeared to come when the Senate Democrats selected
Senator Michael Gianaris of Queens for a state board with the power to
veto the deal. Mr. Gianaris had once supported the efforts to bring
Amazon to New York, but became a vocal critic after learning the details
of the plan.

Image

Gov. Andrew M. Cuomo, center, and Mayor Bill de Blasio, second from
right, had been forcefully defending the deal they
negotiated.Credit...Chang W. Lee/The New York Times

Over time, opposition to Amazon had spread from the specifics of the
deal to the company's corporate practices. Elected officials and
activists in New York drew attention to the company's anti-union stance
and its work with federal immigration officials --- positions unpopular
with Democratic leaders across the country.

Amazon executives felt they had been open to answering questions,
submitting to two City Council hearings, with another planned for later
this month. They had begun working on a hiring plan, people with
knowledge of the planning said, and were encouraged by public support in
two polls of voters, conducted by Quinnipiac University and Siena
College. While the subsidies were less popular, the deal to bring
Amazon, and tens of thousands of jobs, was welcomed by a variety of
groups.

On Wednesday, Mr. Cuomo had even brokered a meeting between Amazon
executives and union leaders who had been resistant to the deal ---
including from the Retail, Wholesale and Department Store union and the
Teamsters.

``Amazon and the governor and everybody agreed yesterday on a way to
move forward,'' said Stuart Appelbaum of the retail union, who was part
of the meeting. ``Shame on them. The arrogance of saying `do it my way
or not at all.'''

Some unions supported the deal, and even those opposed had appeared
willing to work with Amazon if the company agreed to not work against
the unionization of its employees in New York. An Amazon representative,
during one council hearing, pointedly said the company would not agree
to such terms.

\emph{{[}Column:}
\href{https://www.nytimes.com/2019/02/14/business/amazon-new-york-city.html}{\emph{For
once, Amazon loses a popularity contest}}\emph{.{]}}

Kathryn S. Wylde, the chief executive of the Partnership for New York
City, an influential business group, said the reception Amazon had
received sent a ``pretty bad message to the job creators'' of the city
and the world.

``How can anyone be surprised?'' Ms. Wylde said. ``We competed
successfully, made a deal and spent the last three months trashing our
new partner.''

When Amazon announced plans for a second headquarters in September 2017,
it promised 50,000 high-paying jobs and billions in investment for a
community that would be coequal to its home in Seattle. The voracious
company, whose ambitions outgrew the number of people it could hire in
the Pacific Northwest, set off a nationwide frenzy, with more than 200
cities making bids.

Amazon decided last fall that no one city could provide the number of
tech workers it needed and split the headquarters in two.

The company has long been willing to take short-term pain in exchange
for maintaining long-term leverage. In Seattle, Amazon's relationship
with officials soured as it grew to become the city's dominant employer.
Last year, when the Seattle City Council proposed taxing large employers
to pay for homeless services and affordable housing, Amazon took a rare
public stance and threatened to halt its expansion. In the end, the city
retreated and got rid of even a pared back version of the tax it had
adopted.

Image

A view of the Pepsi-Cola sign in Long Island City. A campus for as many
as 40,000 Amazon workers was planned for the
neighborhood.Credit...Christopher Lee for The New York Times

The Seattle relationship looms over Amazon's growth plans. As much as
Amazon wanted New York's talent, it was not worth facing years of
opposition on broad swaths of issues.

Instead, Amazon will grow across its tech hubs, which include large
outposts in cities like Boston, Austin, Tex., and Vancouver, British
Columbia, as well as smaller ones in Pittsburgh and Detroit. It will
lose the value it has said it finds in having employees in a centralized
corporate campus, but will maintain flexibility to grow where and when
it wants.

Even in New York, where Amazon already has 5,000 workers, about half at
a distribution center on Staten Island, the company still plans to add
more jobs, particularly in advertising, fashion and web services.

Mr. Gianaris said the collapse of the deal in Queens revealed the
company's unwillingness to work with the community it had wanted to
join.

``Like a petulant child, Amazon insists on getting its way or takes its
ball and leaves,'' said Mr. Gianaris, whose district includes Long
Island City. ``The only thing that happened here is that a community
that was going to be profoundly affected by their presence started
asking questions.

``Even by their own words,'' he added, pointing to the company's
statement on the pullout, ``Amazon admits they will grow their presence
in New York without their promised subsidies. So what was all this
really about?''

While small protests greeted the company after its initial announcement
in November, the first inkling that opposition had taken hold among the
city's Democratic politicians came during a hostile City Council hearing
the next month. Protesters filled the seats, unfurled banners and
chanted against the company. Not a single council member spoke up in
defense of the deal or the company.

Company executives fared no better at their second appearance, in
January, though supporters, lobbyists and consultants were better
prepared. Unions supporting the deal, including the powerful 32BJ
Service Employees International Union and the Building and Construction
Trades Council of Greater New York, staged a rally outside City Hall
immediately after one held by opponents.

\emph{{[}Amazon}
\href{https://www.nytimes.com/2019/02/14/business/economy/amazon-union-cuomo.html}{\emph{had
a `productive meeting'}} \emph{with union leaders, just a day before
they pulled out.{]}}

Still, the company did not hire a single New Yorker as an employee to
represent it in discussions with local groups. Its main representatives
traveled between Washington and Manhattan, and only one had moved into
an apartment to work with community members and foster support.

Gianna Cerbone, who owns a restaurant several blocks from what would
have been the main Amazon campus, said the demise of the deal was a
major blow to people who need jobs and local businesses that would have
benefited.

``I'm really upset because I don't think they realized what they did,''
she said of the elected officials who had opposed the plan. ``And
they're proud of it? They think they did something lovely? They wanted
the political gain, they should have done it in a different way. They
get put into office for us, not to work for themselves.''

Advertisement

\protect\hyperlink{after-bottom}{Continue reading the main story}

\hypertarget{site-index}{%
\subsection{Site Index}\label{site-index}}

\hypertarget{site-information-navigation}{%
\subsection{Site Information
Navigation}\label{site-information-navigation}}

\begin{itemize}
\tightlist
\item
  \href{https://help.nytimes.com/hc/en-us/articles/115014792127-Copyright-notice}{©~2020~The
  New York Times Company}
\end{itemize}

\begin{itemize}
\tightlist
\item
  \href{https://www.nytco.com/}{NYTCo}
\item
  \href{https://help.nytimes.com/hc/en-us/articles/115015385887-Contact-Us}{Contact
  Us}
\item
  \href{https://www.nytco.com/careers/}{Work with us}
\item
  \href{https://nytmediakit.com/}{Advertise}
\item
  \href{http://www.tbrandstudio.com/}{T Brand Studio}
\item
  \href{https://www.nytimes.com/privacy/cookie-policy\#how-do-i-manage-trackers}{Your
  Ad Choices}
\item
  \href{https://www.nytimes.com/privacy}{Privacy}
\item
  \href{https://help.nytimes.com/hc/en-us/articles/115014893428-Terms-of-service}{Terms
  of Service}
\item
  \href{https://help.nytimes.com/hc/en-us/articles/115014893968-Terms-of-sale}{Terms
  of Sale}
\item
  \href{https://spiderbites.nytimes.com}{Site Map}
\item
  \href{https://help.nytimes.com/hc/en-us}{Help}
\item
  \href{https://www.nytimes.com/subscription?campaignId=37WXW}{Subscriptions}
\end{itemize}
