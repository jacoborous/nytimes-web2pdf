Sections

SEARCH

\protect\hyperlink{site-content}{Skip to
content}\protect\hyperlink{site-index}{Skip to site index}

\href{https://www.nytimes.com/section/politics}{Politics}

\href{https://myaccount.nytimes.com/auth/login?response_type=cookie\&client_id=vi}{}

\href{https://www.nytimes.com/section/todayspaper}{Today's Paper}

\href{/section/politics}{Politics}\textbar{}Kamala Harris and Cory
Booker, Courting Black Support, Pitch Differing Economic Plans

\url{https://nyti.ms/2J7nfGs}

\begin{itemize}
\item
\item
\item
\item
\item
\item
\end{itemize}

\begin{itemize}
\item
  \href{https://www.nytimes.com/2020/07/31/us/elections/biden-vs-trump.html?action=click\&pgtype=Article\&state=default\&region=TOP_BANNER\&context=storylines_menu}{Election
  Updates}
\item
  \href{https://www.nytimes.com/article/biden-vice-president-2020.html?action=click\&pgtype=Article\&state=default\&region=TOP_BANNER\&context=storylines_menu}{Biden's
  V.P. Search}
\item
  \href{https://www.nytimes.com/interactive/2020/07/24/us/politics/trump-biden-campaign-donors.html?action=click\&pgtype=Article\&state=default\&region=TOP_BANNER\&context=storylines_menu}{Map
  of Donations}
\item
  \href{https://www.nytimes.com/interactive/2020/us/elections/delegate-count-primary-results.html?action=click\&pgtype=Article\&state=default\&region=TOP_BANNER\&context=storylines_menu}{Delegate
  Count}
\item
  \href{https://www.nytimes.com/interactive/2019/us/politics/2020-presidential-candidates.html?action=click\&pgtype=Article\&state=default\&region=TOP_BANNER\&context=storylines_menu}{The
  Candidates}
\item
  \href{https://www.nytimes.com/newsletters/politics?action=click\&pgtype=Article\&state=default\&region=TOP_BANNER\&context=storylines_menu}{Politics
  Newsletter}
\end{itemize}

Advertisement

\protect\hyperlink{after-top}{Continue reading the main story}

Supported by

\protect\hyperlink{after-sponsor}{Continue reading the main story}

\hypertarget{kamala-harris-and-cory-booker-courting-black-support-pitch-differing-economic-plans}{%
\section{Kamala Harris and Cory Booker, Courting Black Support, Pitch
Differing Economic
Plans}\label{kamala-harris-and-cory-booker-courting-black-support-pitch-differing-economic-plans}}

\includegraphics{https://static01.nyt.com/images/2019/04/24/us/politics/00harrisbooker1/merlin_150537420_34f83f06-9bc5-4fd9-a3eb-8f9f699a52cf-articleLarge.jpg?quality=75\&auto=webp\&disable=upscale}

By \href{https://www.nytimes.com/by/nick-corasaniti}{Nick Corasaniti}
and \href{https://www.nytimes.com/by/astead-w-herndon}{Astead W.
Herndon}

\begin{itemize}
\item
  May 1, 2019
\item
  \begin{itemize}
  \item
  \item
  \item
  \item
  \item
  \item
  \end{itemize}
\end{itemize}

LONDONDERRY, N.H. --- On his third trip to New Hampshire recently,
Senator Cory Booker had a new and very specific focus: his
\href{https://www.nytimes.com/2019/04/06/us/politics/cory-booker-2020-baby-bonds.html?searchResultPosition=1}{``baby
bonds'' proposal to combat income inequality} by giving every child a
government-funded savings account.

His plan would eliminate the racial wealth gap, Mr. Booker promised
voters, and ``create a fairer playing field where everybody has a stake
in this economy.'' He has since embarked on a 15-day national tour where
he touted numerous economic policies, including a broad rewriting and
expansion of earned-income tax credits, a plan his campaign
\href{https://www.nbcnews.com/politics/2020-election/cory-booker-unveils-plan-cut-taxes-half-country-n994616}{is
calling the ``Rise Credit.''}

One of Mr. Booker's main rivals, Senator Kamala Harris, has been touting
her own plan to close the wealth gap, which she calls the ``LIFT Act.''
It would provide a refundable tax credit worth up to \$6,000 for
households.

As income inequality becomes an increasingly central issue in the
Democratic primary, the plans offered by Mr. Booker and Ms. Harris are a
microcosm of the larger competition between two Democrats who have long
been vying for similar sources of support. As the leading
African-American candidates in the race, they are waging a primary
within the primary for the right to claim a valuable prize: recognition
as the consensus choice of black voters, donors and community leaders in
a crowded Democratic field.

\emph{{[}}\href{https://www.nytimes.com/newsletters/politics?smid=rd\%3Faction\%3Dclick\&module=inline\&pgtype=Article}{\emph{Sign
up for our politics newsletter and join our conversation about the 2020
presidential race.}}\emph{{]}}

``I think they'll both have strengths in the African-American community,
but to get the numbers that you need to get in a crowded field, they'll
both need to have crossover appeal,'' said Stephen Benjamin, the mayor
of Columbia, S.C. Mr. Benjamin has engaged with both candidates, but is
waiting until it gets closer to his state's primary in February 2020 to
make an endorsement.

``So, yes, it's a collision course between the two African-American
candidates, but they must also carve out their own lane,'' he said.

Former Vice President Joseph R. Biden Jr. has tried to position himself
as the continuation of President Barack Obama's legacy since entering
the race last week, and enjoys deep relationships with black leaders in
the Democratic Party. In several national polls, which can provide an
early reading of the electorate, Mr. Biden leads among all demographics,
including black voters. Ms. Harris has slipped to single digits in
recent weeks after a strong opening month, and Mr. Booker continues to
languish.

For Ms. Harris and Mr. Booker, the path to the 2020 presidential race
began more than a decade ago. At the Democratic National Convention in
2008, the two barrier-breaking black officials were featured during a
luncheon for rising stars in a party that was already being reshaped in
Mr. Obama's image. Ms. Harris at the time was the San Francisco district
attorney; Mr. Booker was in his first term as mayor of Newark.

Ms. Harris's election to the Senate in 2016 raised her national profile
and established her as a likely presidential contender. Mr. Booker won
his Senate seat in 2013, cementing his national ascendence.

Since then they have crossed paths repeatedly as they jockey for support
from influential black leaders, and for top billing at a similar set of
events. They have both shown up, for instance, at a 2017 weekend in the
Hamptons for the Black Economic Alliance; at a race and criminal justice
panel held by BET earlier this year; and for the N.A.A.C.P. Image Awards
in March, where they presented together.

\hypertarget{latest-updates-2020-election}{%
\section{\texorpdfstring{\href{https://www.nytimes.com/2020/07/31/us/elections/biden-vs-trump.html?action=click\&pgtype=Article\&state=default\&region=MAIN_CONTENT_1\&context=storylines_live_updates}{Latest
Updates: 2020
Election}}{Latest Updates: 2020 Election}}\label{latest-updates-2020-election}}

Updated 2020-08-01T01:26:45.732Z

\begin{itemize}
\tightlist
\item
  \href{https://www.nytimes.com/2020/07/31/us/elections/biden-vs-trump.html?action=click\&pgtype=Article\&state=default\&region=MAIN_CONTENT_1\&context=storylines_live_updates\#link-29fdff45}{Kamala
  Harris, a top vice-presidential contender, confronts double
  standards.}
\item
  \href{https://www.nytimes.com/2020/07/31/us/elections/biden-vs-trump.html?action=click\&pgtype=Article\&state=default\&region=MAIN_CONTENT_1\&context=storylines_live_updates\#link-13ec3d9c}{Karen
  Bass and Susan Rice are rising on Biden's vice-presidential
  shortlist.}
\item
  \href{https://www.nytimes.com/2020/07/31/us/elections/biden-vs-trump.html?action=click\&pgtype=Article\&state=default\&region=MAIN_CONTENT_1\&context=storylines_live_updates\#link-49e9a016}{Trump
  says Russian bounties to kill U.S. troops `never took place.'}
\end{itemize}

\href{https://www.nytimes.com/2020/07/31/us/elections/biden-vs-trump.html?action=click\&pgtype=Article\&state=default\&region=MAIN_CONTENT_1\&context=storylines_live_updates}{See
more updates}

Both senators have faced skepticism from the party's left at times
during their career --- Mr. Booker for his ties to Wall Street, his
record
\href{https://www.nytimes.com/2019/03/27/us/politics/cory-booker-2020-criminal-justice.html}{on
policing while mayor}, and his embrace of charter schools, and Ms.
Harris for her mixed record on criminal justice reform. And both have
structured their presidential campaigns around an economic policy
proposal aimed at reducing the wealth gap, while carefully avoiding the
anti-Wall Street rhetoric associated with Senators Bernie Sanders of
Vermont and Elizabeth Warren of Massachusetts.

\includegraphics{https://static01.nyt.com/images/2019/04/24/us/politics/00harrisbooker2/merlin_153885786_b089fe21-f9c9-4d8b-9242-695bbf9cfdfe-articleLarge.jpg?quality=75\&auto=webp\&disable=upscale}

\emph{{[}}\href{https://www.nytimes.com/interactive/2019/us/politics/2020-presidential-candidates.html?action=click\&module=inline\&pgtype=Article}{\emph{Keep
up with the 2020 Democratic field with our tracker.}}\emph{{]}}

Mr. Booker's ``baby bonds'' proposal calls for every child born in the
United States to be given a \$1,000 savings account that the government
would fund annually on a tiered basis, depending on family income. The
lump sum is presented when the child turns 18 and can only be used for
education, investing in a business or buying a home.

Conservatives and some economic experts say the proposal is a fantasy
that could never be realized; Hillary Clinton
\href{http://politicalticker.blogs.cnn.com/2007/10/05/giuliani-attacks-clinton-over-5000-baby-bond/}{was
heavily criticized by Republican}s when she proposed a similar idea in
2007 and quickly backed off the proposal.

But many on the left say it represents the kind of bold policy approach
they are seeking, and experts like
\href{https://www.urban.org/author/kilolo-kijakazi}{Kilolo Kijakazi}, a
fellow at the Urban Institute, have said it could directly narrow the
racial wealth gap.

Ms. Harris's policy, called LIFT the Middle Class Act, would provide
lower-income families up to \$500 in monthly payments, on top of
existing tax credits and public benefits. It has been praised as a rare
tangible benefit to the working poor, the kind seldom seen in
Washington.

But some progressives view it as not bold enough --- calling it an
expansion of the earned-income tax credit by another name --- and say it
will not lift up the neediest Americans. Ms. Harris has also attempted
to use the policy as a catchall description of her economic vision,
reframing questions about the racial wealth gap and reparations, for
instance, into a broader discussion of racial disparity.

Recently, some scholars who previously praised the policy have made sure
to clarify it is not a panacea.

The LIFT Act ``won't do much for wealth concentration, and it won't do
much for altering the position of people with the lower end of the
wealth distribution,'' said Sandy Darity, a Duke University professor
who is a leading scholar on reparations and the racial wealth gap.

The reactions to the two economic plans underscore not only the
lingering skepticism about Ms. Harris's progressive bona fides, but also
the divergent strategies she and Mr. Booker are using to present their
campaigns to voters.

In private conversations, Ms. Harris's allies acknowledge they are
making a different bet than Mr. Booker and other rivals like Ms. Warren
and Mr. Sanders. Ms. Harris has leapt into the top tier of candidates
---
\href{https://www.nytimes.com/interactive/2019/us/politics/campaign-finance-2020-fundraising.html}{she
was second to Mr. Sanders in first-quarter fund-raising} --- with a less
radical policy agenda. Her campaign's first policy rollout was a
\href{https://www.nytimes.com/2019/03/26/us/politics/kamala-harris-teacher-pay.html}{federal
investment in teacher pay}, an almost universally supported idea among
her Democratic colleagues.

Rather than trying to appease the party's left wing with policies that
focus on large-scale wealth redistribution and structural change, Ms.
Harris has staked her bet on an incrementalism more reminiscent of Mr.
Obama or Mrs. Clinton.

Felicia Wong, the president of the Roosevelt Institute, a liberal think
tank, who has tried to push candidates to embrace bolder economic
policies, said Mr. Booker has gone to greater lengths to match his
policies with the progressive moment.

``It's notable that Senator Booker, for the last year, has focused on
issues at the top of the economy, like corporate concentration of power,
and he's connected that with people's daily lives,'' Ms. Wong said. ``In
contrast, Senator Harris's policy pieces, while maybe good individually,
don't yet hang together.''

Image

Mr. Booker's ``baby bonds'' proposal calls for every child born in the
United States to be given a \$1,000 savings account that the government
would fund annually on a tiered basis, depending on family
income.Credit...Elizabeth Frantz for The New York Times

Though he leans heavily into bolder policies, Mr. Booker has recently
attempted to distance himself from unyielding candidates on the party's
left flank, like Mr. Sanders. ``A real progressive movement does not
hold progress for communities like mine hostage today for promises that
perfection will come tomorrow,'' he said recently at his
\href{https://www.nytimes.com/2019/04/13/us/politics/cory-booker-newark.html}{hometown
kickoff rally}.

Still, there is no shortage of reasons Ms. Harris's bet on pragmatic
liberalism may work, and she currently enjoys a significant edge on Mr.
Booker in early indicators such as fund-raising, polling and national
campaign apparatus.

In South Carolina, though Mr. Booker snagged the first endorsement from
a state legislator, Ms. Harris recently earned the support of Bakari
Sellers, a former state representative and one of the state's most
sought-after surrogates. In announcing his endorsement, Mr. Sellers
praised Ms. Harris's teacher pay plan in particular.

``These issues hit home for me, and Kamala has repeatedly offered clear
solutions for each one, proving there is no problem or person too small
to be heard,'' Mr. Sellers said.

Both campaigns view the early South Carolina primary as essential to
their prospects. Other contenders, particularly Mr. Biden, enjoy deep
relationships among the black political class there. But Mr. Booker and
Ms. Harris have already invested heavily in the state, in an attempt to
secure staff and early endorsements.

Ms. Harris's campaign has also raised eyebrows for a reluctance to
engage presidential forums related to black voters.

Ms. Harris, for instance, has yet to agree to attend the Black Economic
Alliance's presidential forum on black wealth in South Carolina, though
Mr. Booker and other presidential contenders have said yes. She The
People, the advocacy group for women of color that
\href{https://www.nytimes.com/2019/04/24/us/politics/she-the-people-forum-2020-women.html}{hosted
a presidential forum in Houston last week}, announced a lineup that
originally included almost every top presidential candidate besides Ms.
Harris. ``You have to ask her campaign,'' said the group's founder,
Aimee Allison, when asked about the absence.

Ms. Harris later reversed her decision and participated in the event.

Mr. Booker, who has spent years cultivating support from influential
black leaders like the Rev. Al Sharpton, has also been reaching out to
black pastors and the faith community. In South Carolina he has visited
six churches since announcing his candidacy, and he spoke at the
National Baptist Convention winter meeting last year.

Both campaigns are also counting on deep relationships within the
Congressional Black Caucus, where they are the only two members who are
senators. Though Mr. Booker has been a more frequent presence at the
group's Wednesday lunches, according to two former congressional aides,
both candidates have been active in the caucus, and are counting on
eventual support. So far, only members from each candidate's home state
have offered direct endorsements.

In South Carolina, the choice facing black voters was evident at a town
hall held by Mr. Booker in Denmark, a small rural town not often visited
by candidates.

Benjamin Jones, 69, said he liked both candidates and was looking to
hear about policies that would address racial inequalities, particularly
in criminal justice reform.

``I'm still shopping around,'' Mr. Jones said after the two-plus hour
speech, panel and town-hall event. ``But, I think a lot of black people
want to know, like Janet Jackson says, `What have you done for me
lately?'''

\hypertarget{our-2020-election-guide}{%
\section{Our 2020 Election Guide}\label{our-2020-election-guide}}

Updated July 31, 2020

\begin{itemize}
\item
  \begin{center}\rule{0.5\linewidth}{\linethickness}\end{center}

  \hypertarget{the-latest}{%
  \subsection{The Latest}\label{the-latest}}

  \begin{itemize}
  \tightlist
  \item
    President Trump's assault on the Postal Service is intersecting with
    his attacks on mail-in voting.
    \href{https://www.nytimes.com/2020/07/31/us/politics/trump-usps-mail-delays.html?action=click\&pgtype=Article\&state=default\&region=BELOW_MAIN_CONTENT\&context=storylines_guide}{Voting
    rights groups say it is a recipe for disaster.}
  \end{itemize}
\item
  \begin{center}\rule{0.5\linewidth}{\linethickness}\end{center}

  \hypertarget{bidens-vp-search}{%
  \subsection{Biden's V.P. Search}\label{bidens-vp-search}}

  \begin{itemize}
  \tightlist
  \item
    \href{https://www.nytimes.com/article/biden-vice-president-2020.html?action=click\&pgtype=Article\&state=default\&region=BELOW_MAIN_CONTENT\&context=storylines_guide}{Here
    are 13 women} who have been under consideration to be Joe Biden's
    running mate, and why each might be chosen --- and might not be.
  \end{itemize}
\item
  \begin{center}\rule{0.5\linewidth}{\linethickness}\end{center}

  \hypertarget{keep-up-with-our-coverage}{%
  \subsection{Keep Up With Our
  Coverage}\label{keep-up-with-our-coverage}}

  \begin{itemize}
  \tightlist
  \item
    Get an
    \href{https://www.nytimes.com/newsletters/politics?action=click\&pgtype=Article\&state=default\&region=BELOW_MAIN_CONTENT\&context=storylines_guide}{email}
    recapping the day's news
  \end{itemize}

  \begin{itemize}
  \tightlist
  \item
    Download our mobile app on
    \href{https://apps.apple.com/us/app/nytimes/id284862083?ls=1\&mat_click_id=5c79ae7455014fd1bd66b5610c05b8f2-20191112-16948\&referrer=mat_click_id\%3D5c79ae7455014fd1bd66b5610c05b8f2-20191112-16948\%26link_click_id\%3D722930677036718082}{iOS}
    and
    \href{http://a.localytics.com/android?id=com.nytimes.android\&referrer=utm_source\%3Dother_nyt_mobile_web\%26utm_medium\%3DWeb\%2520page\%26utm_term\%3DGeneral\%2520Mobile\%2520Page\%26utm_campaign\%3DNYT\%2520Mobile\%2520General\%2520Page}{Android}
    and turn on Breaking News and Politics alerts
  \end{itemize}
\end{itemize}

Advertisement

\protect\hyperlink{after-bottom}{Continue reading the main story}

\hypertarget{site-index}{%
\subsection{Site Index}\label{site-index}}

\hypertarget{site-information-navigation}{%
\subsection{Site Information
Navigation}\label{site-information-navigation}}

\begin{itemize}
\tightlist
\item
  \href{https://help.nytimes.com/hc/en-us/articles/115014792127-Copyright-notice}{©~2020~The
  New York Times Company}
\end{itemize}

\begin{itemize}
\tightlist
\item
  \href{https://www.nytco.com/}{NYTCo}
\item
  \href{https://help.nytimes.com/hc/en-us/articles/115015385887-Contact-Us}{Contact
  Us}
\item
  \href{https://www.nytco.com/careers/}{Work with us}
\item
  \href{https://nytmediakit.com/}{Advertise}
\item
  \href{http://www.tbrandstudio.com/}{T Brand Studio}
\item
  \href{https://www.nytimes.com/privacy/cookie-policy\#how-do-i-manage-trackers}{Your
  Ad Choices}
\item
  \href{https://www.nytimes.com/privacy}{Privacy}
\item
  \href{https://help.nytimes.com/hc/en-us/articles/115014893428-Terms-of-service}{Terms
  of Service}
\item
  \href{https://help.nytimes.com/hc/en-us/articles/115014893968-Terms-of-sale}{Terms
  of Sale}
\item
  \href{https://spiderbites.nytimes.com}{Site Map}
\item
  \href{https://help.nytimes.com/hc/en-us}{Help}
\item
  \href{https://www.nytimes.com/subscription?campaignId=37WXW}{Subscriptions}
\end{itemize}
