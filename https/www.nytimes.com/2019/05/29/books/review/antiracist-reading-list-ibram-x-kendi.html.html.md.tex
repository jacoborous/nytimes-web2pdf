Sections

SEARCH

\protect\hyperlink{site-content}{Skip to
content}\protect\hyperlink{site-index}{Skip to site index}

\href{https://www.nytimes.com/section/books/review}{Book Review}

\href{https://myaccount.nytimes.com/auth/login?response_type=cookie\&client_id=vi}{}

\href{https://www.nytimes.com/section/todayspaper}{Today's Paper}

\href{/section/books/review}{Book Review}\textbar{}An Antiracist Reading
List

\url{https://nyti.ms/2WdkrPQ}

\begin{itemize}
\item
\item
\item
\item
\item
\end{itemize}

\href{https://www.nytimes.com/spotlight/at-home?action=click\&pgtype=Article\&state=default\&region=TOP_BANNER\&context=at_home_menu}{At
Home}

\begin{itemize}
\tightlist
\item
  \href{https://www.nytimes.com/2020/08/03/well/family/the-benefits-of-talking-to-strangers.html?action=click\&pgtype=Article\&state=default\&region=TOP_BANNER\&context=at_home_menu}{Talk:
  To Strangers}
\item
  \href{https://www.nytimes.com/2020/08/01/at-home/coronavirus-make-pizza-on-a-grill.html?action=click\&pgtype=Article\&state=default\&region=TOP_BANNER\&context=at_home_menu}{Make:
  Grilled Pizza}
\item
  \href{https://www.nytimes.com/2020/07/31/arts/television/goldbergs-abc-stream.html?action=click\&pgtype=Article\&state=default\&region=TOP_BANNER\&context=at_home_menu}{Watch:
  'The Goldbergs'}
\item
  \href{https://www.nytimes.com/interactive/2020/at-home/even-more-reporters-editors-diaries-lists-recommendations.html?action=click\&pgtype=Article\&state=default\&region=TOP_BANNER\&context=at_home_menu}{Explore:
  Reporters' Google Docs}
\end{itemize}

Advertisement

\protect\hyperlink{after-top}{Continue reading the main story}

Supported by

\protect\hyperlink{after-sponsor}{Continue reading the main story}

Further Reading

\hypertarget{an-antiracist-reading-list}{%
\section{An Antiracist Reading List}\label{an-antiracist-reading-list}}

Ibram X. Kendi on books to help America transcend its racist heritage.

\includegraphics{https://static01.nyt.com/images/2019/06/02/books/review/02Kendi/02Kendi-articleLarge.jpg?quality=75\&auto=webp\&disable=upscale}

By Ibram X. Kendi

\begin{itemize}
\item
  May 29, 2019
\item
  \begin{itemize}
  \item
  \item
  \item
  \item
  \item
  \end{itemize}
\end{itemize}

No one becomes ``not racist,'' despite a tendency by Americans to
identify themselves that way. We can only strive to be ``antiracist'' on
a daily basis, to continually rededicate ourselves to the lifelong task
of overcoming our country's racist heritage.

We learn early the racist notion that white people have more because
they are more; that people of color have less because they are less. I
had internalized this worldview by my high school graduation, seeing
myself and my race as less than other people and blaming other blacks
for racial inequities.

To build a nation of equal opportunity for everyone, we need to
dismantle this spurious legacy of our common upbringing. One of the best
ways to do this is by reading books. Not books that reinforce old ideas
about who we think we are, what we think America is, what we think
racism is. Instead, we need to read books that are difficult or
unorthodox, that don't go down easily. Books that force us to confront
our self-serving beliefs and make us aware that ``I'm not racist'' is a
slogan of denial.

The reading list below is composed of just such books --- a combination
of classics, relatively obscure works and a few of recent vintage. Think
of it as a stepladder to antiracism, each step addressing a different
stage of the journey toward destroying racism's insidious hold on all of
us.

\emph{{[}}
\href{https://www.nytimes.com/article/books-race-america.html}{\emph{We
asked novelists, historians, poets, comedians and activists about the
books that deepened their awareness of racism in America}}\emph{. {]}}

Biology

\hypertarget{fatal-invention}{%
\subsection{FATAL INVENTION}\label{fatal-invention}}

How Science, Politics, and Big Business Re-create Race in the
Twenty-First Century

\hypertarget{by-dorothy-roberts}{%
\subsubsection{\texorpdfstring{\textbf{By Dorothy
Roberts}}{By Dorothy Roberts}}\label{by-dorothy-roberts}}

No book destabilized my fraught notions of racial distinction and
hierarchy --- the belief that each race had different genes, diseases
and natural abilities --- more than this vigorous critique of the
``biopolitics of race.'' Roberts, a professor at the University of
Pennsylvania, shows unequivocally that all people are indeed created
equal, despite political and economic special interests that keep trying
to persuade us otherwise.

New Press, 2011

Ethnicity

\hypertarget{west-indian-immigrants}{%
\subsection{WEST INDIAN IMMIGRANTS}\label{west-indian-immigrants}}

A Black Success Story?

\hypertarget{by-suzanne-model}{%
\subsubsection{By Suzanne Model}\label{by-suzanne-model}}

Some of the same forces have led Americans to believe that the recent
success of black immigrants from the Caribbean proves either that racism
does not exist or that the gap between African-Americans and other
groups in income and wealth is their own fault. But Model's meticulous
study, emphasizing the self-selecting nature of the West Indians who
emigrate to the United States, argues otherwise, showing me, a native of
racially diverse New York City, how such notions --- the foundation of
ethnic racism --- are unsupported by the facts.

Russell Sage Foundation, 2008

Body

\hypertarget{the-condemnation-of-blackness}{%
\subsection{THE CONDEMNATION OF
BLACKNESS}\label{the-condemnation-of-blackness}}

Race, Crime, and the Making of Modern Urban America

\hypertarget{by-khalil-gibran-muhammad}{%
\subsubsection{\texorpdfstring{\textbf{By Khalil Gibran
Muhammad}}{By Khalil Gibran Muhammad}}\label{by-khalil-gibran-muhammad}}

``Black'' and ``criminal'' are as wedded in America as ``star'' and
``spangled.'' Muhammad's book traces these ideas to the late 19th
century, when racist policies led to the disproportionate arrest and
incarceration of blacks, igniting urban whites' fears and bequeathing
tenaciously racist stereotypes.

Harvard University, 2010

Culture

\hypertarget{their-eyes-were-watching-god}{%
\subsection{THEIR EYES WERE WATCHING
GOD}\label{their-eyes-were-watching-god}}

\hypertarget{by-zora-neale-hurston}{%
\subsubsection{\texorpdfstring{\textbf{By Zora Neale
Hurston}}{By Zora Neale Hurston}}\label{by-zora-neale-hurston}}

Of course, the black body exists within a wider black culture --- one
Hurston portrayed with grace and insight in this seminal novel. She
defies racist Americans who would standardize the cultures of white
people or sanitize, eroticize, erase or assimilate those of blacks.

1937

Behavior

\hypertarget{the-negro-artist-and-the-racial-mountain}{%
\subsection{THE NEGRO ARTIST AND THE RACIAL
MOUNTAIN}\label{the-negro-artist-and-the-racial-mountain}}

\hypertarget{by-langston-hughes}{%
\subsubsection{\texorpdfstring{\textbf{By Langston
Hughes}}{By Langston Hughes}}\label{by-langston-hughes}}

``We younger Negro artists who create now intend to express our
individual dark-skinned selves without fear or shame,'' Hughes wrote
nearly 100 years ago. ``We know we are beautiful. And ugly too.'' We are
all \emph{imperfectly} human, and these imperfections are also markers
of human equality.

\href{http://www.hartford-hwp.com/archives/45a/360.html}{The Nation,
June 23, 1926}

Color

\hypertarget{the-bluest-eye}{%
\subsection{THE BLUEST EYE}\label{the-bluest-eye}}

\hypertarget{by-toni-morrison}{%
\subsubsection{\texorpdfstring{\textbf{By Toni
Morrison}}{By Toni Morrison}}\label{by-toni-morrison}}

\hypertarget{the-blacker-the-berry}{%
\subsection{THE BLACKER THE BERRY}\label{the-blacker-the-berry}}

\hypertarget{by-wallace-thurman}{%
\subsubsection{\texorpdfstring{\textbf{By Wallace
Thurman}}{By Wallace Thurman}}\label{by-wallace-thurman}}

Beautiful and hard-working black people come in all shades. If dark
people have less it is not because they are less, a moral eloquently
conveyed in these two classic novels, stirring explorations of colorism.

1970 \textbar{} 1929

Whiteness

\hypertarget{the-autobiography-of-malcolm-x}{%
\subsection{THE AUTOBIOGRAPHY OF MALCOLM
X}\label{the-autobiography-of-malcolm-x}}

\hypertarget{by-malcolm-x-and-alex-haley}{%
\subsubsection{\texorpdfstring{\textbf{By Malcolm X and Alex
Haley}}{By Malcolm X and Alex Haley}}\label{by-malcolm-x-and-alex-haley}}

\hypertarget{dying-of-whiteness}{%
\subsection{DYING OF WHITENESS}\label{dying-of-whiteness}}

How the Politics of Racial Resentment Is Killing America's Heartland

\hypertarget{by-jonathan-m-metzl}{%
\subsubsection{\texorpdfstring{\textbf{By Jonathan M.
Metzl}}{By Jonathan M. Metzl}}\label{by-jonathan-m-metzl}}

Malcolm X began by adoring whiteness, grew to hate white people and,
ultimately, despised the false concept of white superiority --- a killer
of people of color. And not only them: low- and middle-income white
people too, as Metzl's timely book shows, with its look at Trump-era
policies that have unraveled the Affordable Care Act and contributed to
rising gun suicide rates and lowered life expectancies.

1965 \textbar{} Basic Books, 2019

Blackness

\hypertarget{locking-up-our-own}{%
\subsection{LOCKING UP OUR OWN}\label{locking-up-our-own}}

Crime and Punishment in Black America

\hypertarget{by-james-forman-jr}{%
\subsubsection{\texorpdfstring{\textbf{By James Forman
Jr.}}{By James Forman Jr.}}\label{by-james-forman-jr}}

Just as Metzl explains how seemingly pro-white policies are killing
whites, Forman explains how blacks themselves abetted the mass
incarceration of other blacks, beginning in the 1970s. Amid rising crime
rates, black mayors, judges, prosecutors and police chiefs embraced
tough-on-crime policies that they promoted as pro-black with tragic
consequences for black America.

Farrar, Straus \& Giroux, 2017
(\href{https://www.nytimes.com/2017/04/14/books/review/locking-up-our-own-james-forman-jr-colony-in-nation-chris-hayes.html}{Read
the review.})

Class

\hypertarget{black-marxism}{%
\subsection{BLACK MARXISM}\label{black-marxism}}

The Making of the Black Radical Tradition

\hypertarget{by-cedric-j-robinson}{%
\subsubsection{\texorpdfstring{\textbf{By Cedric J.
Robinson}}{By Cedric J. Robinson}}\label{by-cedric-j-robinson}}

Black America has been economically devastated by what Robinson calls
racial capitalism. He chastises white Marxists (and black capitalists)
for failing to acknowledge capitalism's racial character, and for
embracing as sufficient an interpretation of history founded on a
European vision of class struggle.

Zed Press, 1983

Spaces

\hypertarget{waiting-til-the-midnight-hour}{%
\subsection{WAITING 'TIL THE MIDNIGHT
HOUR}\label{waiting-til-the-midnight-hour}}

A Narrative History of Black Power in America

\hypertarget{by-peniel-e-joseph}{%
\subsubsection{\texorpdfstring{\textbf{By Peniel E.
Joseph}}{By Peniel E. Joseph}}\label{by-peniel-e-joseph}}

As racial capitalism deprives black communities of resources,
assimilationists ignore or gentrify these same spaces in the name of
``development'' and ``integration.'' To be antiracist is not only to
promote equity among racial groups, but also among their spaces,
something the black power movement of the 1960s and 1970s understood
well, as Joseph's chronicle makes clear.

Holt, 2006

Gender

\hypertarget{how-we-get-free}{%
\subsection{HOW WE GET FREE}\label{how-we-get-free}}

Black Feminism and the Combahee River Collective

\hypertarget{edited-by-keeanga-yamahtta-taylor}{%
\subsubsection{\texorpdfstring{\textbf{Edited by Keeanga-Yamahtta
Taylor}}{Edited by Keeanga-Yamahtta Taylor}}\label{edited-by-keeanga-yamahtta-taylor}}

\hypertarget{well-read-black-girl}{%
\subsection{WELL-READ BLACK GIRL}\label{well-read-black-girl}}

Finding Our Stories, Discovering Ourselves

\hypertarget{edited-by-glory-edim}{%
\subsubsection{\texorpdfstring{\textbf{Edited by Glory
Edim}}{Edited by Glory Edim}}\label{edited-by-glory-edim}}

I began my career studying, and too often admiring, activists who
demanded black (male) power over black communities, including over black
women, whom they placed on pedestals and under their feet. Black
feminist literature, including these anthologies, helps us recognize
black women ``as human, levelly human,'' as the Combahee River
Collective demanded to be seen in 1977.

Haymarket, 2017 \textbar{} Ballantine, 2018
(\href{https://www.nytimes.com/2018/10/25/books/well-read-black-girl-glory-edim.html}{Read
our profile of Glory Edim.})

Sexuality

\hypertarget{redefining-realness}{%
\subsection{REDEFINING REALNESS}\label{redefining-realness}}

My Path to Womanhood, Identity, Love \& So Much More

\hypertarget{by-janet-mock}{%
\subsubsection{\texorpdfstring{\textbf{by Janet
Mock}}{by Janet Mock}}\label{by-janet-mock}}

\hypertarget{sister-outsider}{%
\subsection{SISTER OUTSIDER}\label{sister-outsider}}

Essays and Speeches

\hypertarget{by-audre-lorde}{%
\subsubsection{\texorpdfstring{\textbf{by Audre
Lorde}}{by Audre Lorde}}\label{by-audre-lorde}}

**** I grew up in a Christian household thinking there was something
abnormal and immoral about queer blacks. My racialized transphobia made
Mock's memoir an agonizing read --- just as my racialized homophobia
made Lorde's essays and speeches a challenge. But pain often precedes
healing.

Atria, 2014 \textbar{} Crossing Press, 1984

By not running from the books that pain us, we can allow them to
transform us. I ran from antiracist books most of my life. But now I
can't stop running \emph{after} them --- scrutinizing myself and my
society, and in the process changing both.

\emph{Ibram X. Kendi, a professor and director of the Antiracist
Research \& Policy Center at American University, is the National Book
Award-winning author of ``Stamped From the Beginning: The Definitive
History of Racist Ideas in America'' and the forthcoming ``How to Be an
Antiracist.''}

\begin{center}\rule{0.5\linewidth}{\linethickness}\end{center}

\emph{Follow New York Times Books on}
\href{https://www.facebook.com/nytbooks/}{\emph{Facebook}}\emph{,}
\href{https://twitter.com/nytimesbooks}{\emph{Twitter}} \emph{and}
\href{https://www.instagram.com/nytbooks/}{\emph{Instagram}}\emph{, sign
up for}
\href{https://www.nytimes.com/newsletters/books-review}{\emph{our
newsletter}} \emph{or}
\href{https://www.nytimes.com/interactive/2017/books/books-calendar.html}{\emph{our
literary calendar}}\emph{. And listen to us on the}
\href{https://www.nytimes.com/column/book-review-podcast}{\emph{Book
Review podcast}}\emph{.}

Advertisement

\protect\hyperlink{after-bottom}{Continue reading the main story}

\hypertarget{site-index}{%
\subsection{Site Index}\label{site-index}}

\hypertarget{site-information-navigation}{%
\subsection{Site Information
Navigation}\label{site-information-navigation}}

\begin{itemize}
\tightlist
\item
  \href{https://help.nytimes.com/hc/en-us/articles/115014792127-Copyright-notice}{©~2020~The
  New York Times Company}
\end{itemize}

\begin{itemize}
\tightlist
\item
  \href{https://www.nytco.com/}{NYTCo}
\item
  \href{https://help.nytimes.com/hc/en-us/articles/115015385887-Contact-Us}{Contact
  Us}
\item
  \href{https://www.nytco.com/careers/}{Work with us}
\item
  \href{https://nytmediakit.com/}{Advertise}
\item
  \href{http://www.tbrandstudio.com/}{T Brand Studio}
\item
  \href{https://www.nytimes.com/privacy/cookie-policy\#how-do-i-manage-trackers}{Your
  Ad Choices}
\item
  \href{https://www.nytimes.com/privacy}{Privacy}
\item
  \href{https://help.nytimes.com/hc/en-us/articles/115014893428-Terms-of-service}{Terms
  of Service}
\item
  \href{https://help.nytimes.com/hc/en-us/articles/115014893968-Terms-of-sale}{Terms
  of Sale}
\item
  \href{https://spiderbites.nytimes.com}{Site Map}
\item
  \href{https://help.nytimes.com/hc/en-us}{Help}
\item
  \href{https://www.nytimes.com/subscription?campaignId=37WXW}{Subscriptions}
\end{itemize}
