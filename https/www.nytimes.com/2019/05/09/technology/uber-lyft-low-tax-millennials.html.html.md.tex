\href{/section/technology}{Technology}\textbar{}They Got Rich Off Uber
and Lyft. Then They Moved to Low-Tax States.

\url{https://nyti.ms/2H8odAU}

\begin{itemize}
\item
\item
\item
\item
\item
\item
\end{itemize}

\includegraphics{https://static01.nyt.com/images/2019/05/12/business/03UBERRICH/00UBERRICH-articleLarge.jpg?quality=75\&auto=webp\&disable=upscale}

Sections

\protect\hyperlink{site-content}{Skip to
content}\protect\hyperlink{site-index}{Skip to site index}

\hypertarget{they-got-rich-off-uber-and-lyft-then-they-moved-to-low-tax-states}{%
\section{They Got Rich Off Uber and Lyft. Then They Moved to Low-Tax
States.}\label{they-got-rich-off-uber-and-lyft-then-they-moved-to-low-tax-states}}

Meet the semiretired millennials who left California for low-tax,
low-stress places like Texas, as their former start-ups stampede toward
the stock market.

Credit...Aart-Jan Venema

Supported by

\protect\hyperlink{after-sponsor}{Continue reading the main story}

By \href{https://www.nytimes.com/by/kate-conger}{Kate Conger}

\begin{itemize}
\item
  May 9, 2019
\item
  \begin{itemize}
  \item
  \item
  \item
  \item
  \item
  \item
  \end{itemize}
\end{itemize}

AUSTIN, Tex. --- Brian McMullen's only plan on a Thursday afternoon in
March was to watch as many college basketball games as possible. Parked
inside his neighborhood bar and grill and eating brunch tacos, he
followed one game on the restaurant's TV screen while another streamed
on his iPhone.

As the games played out, Mr. McMullen talked about his new life in
Austin, Tex., where he had moved last October from San Francisco. Some
of his biggest activities since then: reading the Harry Potter series
for the first time and spending more than 100 hours completing ``Dragon
Quest,'' a role-playing video game. He was also working out a lot, he
said, and teaching himself a coding language to create his own games.

For his medium-term goals, Mr. McMullen said, he and his new wife had
been planning to honeymoon in Japan for a month. But they decided to cut
the trip short to fly to San Francisco to meet friends for the opening
night of ``Avengers: Endgame'' in April, at one point discussing whether
to rent out an entire showing. (They did not.)

``I'm currently taking time off for myself,'' he said.

Mr. McMullen, 33, is part of an exclusive club: the semiretired tech
millennial who left California after getting rich. Like many in this
group, he is a newly minted multimillionaire who became wealthy by
working for high-profile San Francisco start-ups like Uber and Lyft,
which are now
\href{https://www.nytimes.com/2019/04/26/technology/uber-ipo-valuation-price-range.html?rref=collection\%2Fbyline\%2Fkate-conger\&action=click\&contentCollection=undefined\&region=stream\&module=stream_unit\&version=latest\&contentPlacement=1\&pgtype=collection}{about
to go} or have
\href{https://www.nytimes.com/2019/03/29/technology/lyft-stock-price.html}{just
gone public}. Once their wealth was assured, these tech workers quit the
companies and fled California, which has the nation's highest state
income tax, at more than 13 percent, to reside in lower-tax states like
Texas and Florida, where there is no personal state income tax.

``There are a number of places people could go where there's tax
benefits,'' said Mr. McMullen, who joined Uber in 2011 and is tallied in
the company's system as employee No. 16. ``The timing is good in terms
of relocating prior to I.P.O.''

He ticked off Washington and Florida as places where people could have
also saved on taxes. Uber employees who have decamped from San Francisco
to Austin stay in touch through an email list called ``Camp Austin'';
they recently discussed visiting the rodeo, he said.

\href{https://www.nytimes.com/interactive/2019/business/dealbook/ipo-investors.html}{}

\includegraphics{https://static01.nyt.com/images/2019/03/27/business/ipo-payday-still-promo/ipo-payday-still-promo-articleLarge-v2.jpg}

\hypertarget{whos-getting-rich-when-uber-slack-and-other-unicorns-go-public}{%
\subsection{Who's Getting Rich When Uber, Slack and Other `Unicorns' Go
Public}\label{whos-getting-rich-when-uber-slack-and-other-unicorns-go-public}}

Highly valued tech start-ups are going public this year, and their
debuts promise to generate big paydays for employees and investors. Here
are what some of their stakes are worth at their I.P.O.s.

In fleeing California, these millennial millionaires are following a
well-worn tradition. Over the years, many who made a fortune off Silicon
Valley skedaddled to lower-tax locations where they could better protect
their wealth. After the late 1990s dot-com frenzy, Jim Clark, a founder
of Netscape, relocated to Florida. Eduardo Saverin, a Facebook
co-founder, departed the United States altogether: He moved to Singapore
and gave up his American citizenship before the social network's
\href{https://dealbook.nytimes.com/2012/05/17/facebook-raises-16-billion-in-i-p-o/}{2012
initial public offering}.

J.T. Forbus, a tax manager at Bogdan \& Frasco in San Francisco, said he
has been fielding more questions from tech workers about how moving out
of state could help sidestep high taxes, especially as their
\href{https://www.nytimes.com/2019/04/18/technology/pinterest-stock.html}{companies
stampede toward the stock market}. Many tech workers are compensated
with stock, which is generally doled out over four years but can trigger
a hefty tax bill when it ``vests,'' or is earned, and when it is sold.

``It seems to be a question that kind of pops up when an I.P.O. is
happening and someone has substantial shares and could have millions of
dollars coming their way,'' Mr. Forbus said.

The sheltering move is well known by the California Franchise Tax Board,
the state agency responsible for tax collection. In its guide for
Californians who are compensated with equity and plan to move out of
state, it walked people through various potential tax scenarios using
low-tax states like Texas, Florida and Nevada as examples.

To avoid paying California taxes when they eventually sell their shares,
residents truly have to move out of state. California imposes an income
tax on shares vested in the state, but does not tax stock that is sold
after someone moves away.

Some tech employees wrongly assume that simply setting up a P.O. box in
another state will be enough to lower their tax bill, Mr. Forbus said.
Others who choose to split their time between a low-tax state and
Silicon Valley end up keeping detailed calendars and flight logs, in
case they have to prove their whereabouts during an audit. Mr. Forbus
said he has ended up advising some techies to leave California to
decrease their tax bill, such as one couple who struggled to find a home
in their preferred location within their \$2.5 million price range.

The New York Times recently interviewed seven former Uber and Lyft
employees who moved to lower-tax locales. Some declined to speak on the
record, citing concerns that talking frankly about their finances would
hurt their chances with future tech employers, or make them audit
targets. (Uber and Lyft declined to comment.)

\includegraphics{https://static01.nyt.com/images/2019/05/09/business/09uberrich5/09uberrich4-articleLarge.jpg?quality=75\&auto=webp\&disable=upscale}

Many argued that their primary motivation for leaving was a
disillusionment with tech-obsessed San Francisco, and that taxes were
not their main concern. Still, none had chosen to move to a high-tax
jurisdiction like New York or Massachusetts. In a recent
\href{https://inequality.stanford.edu/sites/default/files/millionaire-migration-california-impact-top-tax-rates.pdf}{study
of California tax data}, Stanford University researchers found that high
tax rates alone do not cause millionaires to leave the state, and that
migration increases during stressful lifestyle changes.

Most of the people The Times spoke to were putting their new wealth to
use, buying houses and planning vacations. Several had made vanity
purchases, such as Teslas. One had acquired an artsy dance hall in Texas
as a residence, which included a bathtub in the middle of a bedroom.
Most were taking long sabbaticals from work and experimenting with new
diets, exercise and meditation. A few had launched their own start-ups.

``It's almost a new generation of millennial retirement,'' said Tyler
Mann, 31, who worked at file-hosting service Dropbox and made several
hundred thousand dollars from the company, which went public last year.
He moved to Austin 18 months ago and has since founded his own start-up.

Many tech millennial millionaires said they were relieved to be out of
San Francisco, which has
\href{https://www.nytimes.com/2019/03/07/style/uber-ipo-san-francisco-rich.html}{gotten
increasingly expensive}, crowded and filled with carbon-copy tech bros
who drone on about their start-ups. They talked about how they were
resetting their lives, how stressed they had been in tech and how they
were getting over burnout. They talked about the tech parties they had
attended and complained that the celebrations revolved around work.

``It got monotonous,'' said Nathan Rodriguez, 30, one of Lyft's first 50
employees, who last year traded San Francisco for Austin. ``I got tired
of the keeping-up-with-the-Joneses feeling you have in that kind of
environment.''

Mr. Rodriguez left Lyft in 2017 after working there for four years,
during which the company's valuation shot up more than 38,200 percent.
He then took 10 months off work and went on cross-country road trips. He
said he made less than \$1 million from Lyft and briefly became a
cryptocurrency millionaire before the crypto market crashed in early
2018. He recently joined a start-up in Austin because, he said, he liked
the feeling of having a big impact at a small company, and because he
has hefty medical bills to pay after a bike accident.

Mr. McMullen, a Northern California native, moved to San Francisco eight
years ago from the coastal town of San Luis Obispo, Calif., when a then
tiny start-up called Uber offered him a job in marketing. He had been
working at an Apple Store. His co-workers warned him that start-ups
often tank and urged him to stick with his stable job and benefits.

\href{https://www.nytimes.com/interactive/2019/05/09/business/dealbook/tech-ipos-uber.html}{}

\includegraphics{https://static01.nyt.com/images/2019/05/08/us/tech-ipos-uber-promo-1557349510669/tech-ipos-uber-promo-1557349510669-articleLarge-v3.png}

\hypertarget{uber-is-going-public-how-todays-tech-ipos-differ-from-the-dot-com-boom}{%
\subsection{Uber Is Going Public: How Today's Tech I.P.O.s Differ From
the Dot-Com
Boom}\label{uber-is-going-public-how-todays-tech-ipos-differ-from-the-dot-com-boom}}

Uber, Lyft and others going public this year are more established than
their tech counterparts from the late 90s, but as some see slowing
growth, there are questions about where they go from here.

Mr. McMullen said he didn't listen because he wanted to move to San
Francisco. ``There was a romantic notion of what San Francisco was,'' he
said.

At the time, Uber was only in a few markets and about to launch its
ride-hailing service in New York. The company was valued at \$60 million
by private investors. Its employees were given relatively low salaries
and incentivized with generous stock options.

Mr. McMullen became an Uber community manager, a role that involved
promoting Uber to riders and drivers in San Francisco with promo codes
and other tactics. He later became a brand strategist, planning
marketing campaigns and establishing a voice for the company.

Uber quickly ballooned into a behemoth. Its imminent public offering
could bring a valuation of
\href{https://www.nytimes.com/2019/05/08/technology/uber-ipo-price.html}{around
\$86 billion}, meaning that the company and the stock options it issued
to early employees like Mr. McMullen would likely have increased in
value by about 140,000 percent.

\emph{{[}Update:}
\href{https://www.nytimes.com/2019/05/09/technology/uber-ipo-price.html}{\emph{Uber's
public offering was more muted}}\emph{, valuing the company at around
\$82 billion. \textbar{}}
\href{https://www.nytimes.com/2019/05/10/technology/uber-stock-price-ipo.html}{\emph{Its
trading debut was disappointing}}\emph{, with the opening trade falling
below the I.P.O. price.{]}}

Along the way, Mr. McMullen cashed out some of his Uber shares when the
company let employees sell their stock to private investors. After
working at Uber for so long, he said, he was motivated to leave and go
somewhere he could have a bigger impact.

``I wasn't really feeling like I had the same role in contributing as in
earlier, smaller Uber,'' he said, adding that the growth of the company
paralleled San Francisco's transformation into ``a homogeneous tech
city.''

Image

Nathan Rodriguez, one of Lyft's first 50 employees, last year traded San
Francisco for Austin, Tex.Credit...Cayce Clifford for The New York Times

Although Mr. McMullen has now not worked for seven months and jokes with
friends about being semiretired, he said he plans to work again.

``The idea of retirement as sitting on a sandy beach somewhere, I don't
think is on any millennial's mind,'' he said. Instead, he added, his
generation is focused on seeking fulfillment, searching for the kind of
career that doesn't feel like work. His goal was to ``realign life,'' he
said.

Before leaving San Francisco, Mr. McMullen had drinks with another
former Uber employee --- Alex Priest, 30, who had also become a
millionaire from working at the company. As the two caught up, they
discovered that they had both decided to move to Texas.

``Ninety-five percent of our conversations up to that point would be
talk about the weather for five minutes and then talk about Uber for
three hours,'' Mr. Priest said. ``This was the first conversation we'd
had where we talked about Uber for five seconds and then our lives for
three hours.''

Last May, Mr. McMullen purchased a San Francisco home for \$1.9 million;
he said the property was an investment. In Austin, where his wife has
family, he also bought a home, which Zillow lists as sold for \$620,000.
It is in a rapidly growing neighborhood where small ranch-style homes
are being replaced with multistory condos, packed two per lot.

Last month, Mr. McMullen was back in San Francisco to watch
``\href{https://www.nytimes.com/2019/04/23/movies/avengers-endgame-review.html}{Avengers:
Endgame}.'' He said he saw it three times in three days with different
groups of friends. Each showing fulfilled his expectations, he said.

But being back in San Francisco reminded him of why he had left and made
him excited to return to Texas. ``Maybe we don't want to be there our
entire lives,'' he said of Austin. Still, he said, it felt like a good
start.

Advertisement

\protect\hyperlink{after-bottom}{Continue reading the main story}

\hypertarget{site-index}{%
\subsection{Site Index}\label{site-index}}

\hypertarget{site-information-navigation}{%
\subsection{Site Information
Navigation}\label{site-information-navigation}}

\begin{itemize}
\tightlist
\item
  \href{https://help.nytimes.com/hc/en-us/articles/115014792127-Copyright-notice}{©~2020~The
  New York Times Company}
\end{itemize}

\begin{itemize}
\tightlist
\item
  \href{https://www.nytco.com/}{NYTCo}
\item
  \href{https://help.nytimes.com/hc/en-us/articles/115015385887-Contact-Us}{Contact
  Us}
\item
  \href{https://www.nytco.com/careers/}{Work with us}
\item
  \href{https://nytmediakit.com/}{Advertise}
\item
  \href{http://www.tbrandstudio.com/}{T Brand Studio}
\item
  \href{https://www.nytimes.com/privacy/cookie-policy\#how-do-i-manage-trackers}{Your
  Ad Choices}
\item
  \href{https://www.nytimes.com/privacy}{Privacy}
\item
  \href{https://help.nytimes.com/hc/en-us/articles/115014893428-Terms-of-service}{Terms
  of Service}
\item
  \href{https://help.nytimes.com/hc/en-us/articles/115014893968-Terms-of-sale}{Terms
  of Sale}
\item
  \href{https://spiderbites.nytimes.com}{Site Map}
\item
  \href{https://help.nytimes.com/hc/en-us}{Help}
\item
  \href{https://www.nytimes.com/subscription?campaignId=37WXW}{Subscriptions}
\end{itemize}
