Sections

SEARCH

\protect\hyperlink{site-content}{Skip to
content}\protect\hyperlink{site-index}{Skip to site index}

\href{https://myaccount.nytimes.com/auth/login?response_type=cookie\&client_id=vi}{}

\href{https://www.nytimes.com/section/todayspaper}{Today's Paper}

\href{/section/opinion}{Opinion}\textbar{}Your Car Knows When You Gain
Weight

\url{https://nyti.ms/2YAayIX}

\begin{itemize}
\item
\item
\item
\item
\item
\item
\end{itemize}

Advertisement

\protect\hyperlink{after-top}{Continue reading the main story}

\href{/section/opinion}{Opinion}

Supported by

\protect\hyperlink{after-sponsor}{Continue reading the main story}

\hypertarget{your-car-knows-when-you-gain-weight}{%
\section{Your Car Knows When You Gain
Weight}\label{your-car-knows-when-you-gain-weight}}

Vehicles collect a lot of unusual data. But who owns it?

By Bill Hanvey

Mr. Hanvey is president and chief executive officer of the Auto Care
Association.

\begin{itemize}
\item
  May 20, 2019
\item
  \begin{itemize}
  \item
  \item
  \item
  \item
  \item
  \item
  \end{itemize}
\end{itemize}

\includegraphics{https://static01.nyt.com/images/2019/05/22/opinion/sunday/17Hanvey/3c27e820db0d4c6aa06dd2df1d9c33dc-articleLarge.jpg?quality=75\&auto=webp\&disable=upscale}

Cars produced today are essentially smartphones with wheels. For
drivers, this has meant many new features: automatic braking,
turn-by-turn directions, infotainment. But for all the things we're
getting out of our connected vehicles, carmakers are getting much, much
more: They're constantly collecting data from our vehicles.

Today's cars are equipped with telematics, in the form of an always-on
wireless transmitter that constantly sends vehicle performance and
maintenance data to the manufacturer. Modern cars collect as much as 25
gigabytes of data per hour, the consulting firm
\href{https://www.mckinsey.com/industries/automotive-and-assembly/our-insights/whats-driving-the-connected-car}{McKinsey
estimates}, and it's about much more than performance and maintenance.

Cars not only know how much we weigh but also track how much weight we
gain. They know how fast we drive, where we live, how many children we
have --- even financial information. Connect a phone to a car, and it
knows who we call and who we text.

But who owns and, ultimately, controls that data? And what are carmakers
doing with it?

The issue of ownership is murky. Drivers usually sign away their rights
to data in a small-print clause buried in the ownership or lease
agreement. It's not unlike buying a smartphone. The difference is that
most consumers have no idea vehicles collect data.

We know our smartphones, Nests and Alexas collect data, and we've come
to accept an implicit contract: We trade personal information for
convenience. With cars, we have no such expectation.

What carmakers are doing with the collected data isn't clear. We know
they use it to improve car performance and safety. And we know they have
the ability to sell it to third parties they might choose. Indeed,
Ford's chief executive, Jim Hackett, has spoken in detail about the
company's plans to monetize car data.

Debates around privacy often focus on companies like Facebook. But
today's connected cars --- and tomorrow's autonomous vehicles --- show
how the commercial opportunities in collecting personal data are
limitless. Your location data will allow companies to advertise to you
based on where you live, work or frequently travel. Data gathered from
voice-command technology could also be useful to advertisers.

The data on your driving habits --- how fast you drive, how hard you
brake, whether you always use your seatbelt --- could be valuable to
insurance companies. You may or may not choose to share your data with
these services. But while you can turn off location data on your
cellphone, there's no opt-out feature for your car.

Carmakers use data to alert us when something needs repair or when our
cars need to be taken in for service. What they don't tell us is that by
controlling our data, they can limit where we get that repair or service
done. For almost a century, car and truck owners have been able to take
their vehicles to whichever shop they choose and trust for maintenance
and repair. That may be changing.

\emph{{[}Technology has made our lives easier. But it also means that
your data is no longer your own. We'll examine who is hoarding your
information --- and give you a guide for what you can do about it.}
\href{https://www.nytimes.com/newsletters/privacy-project?action=click\&module=Intentional\&pgtype=Article}{\emph{Sign
up for our limited-run newsletter}}\emph{.{]}}

Because of the increasing complexity of cars and the Internet of Things,
data is critical to repair and service. When carmakers control the data,
they can choose which service centers receive our information. They're
more likely to share our data exclusively with their branded dealerships
than with independent repair shops, which could have the edge in price
and convenience. However, independent repair shops currently make 70
percent of outside warranty repairs throughout the country.

This is a different facet of the privacy conversation. Our anxiety about
data typically focuses on what happens when information is shared with
those we don't want to see it. But what about when information is
withheld from those we do want to see it?

Imagine visiting a medical specialist and learning he can't get access
to the medical history that your doctor maintains, or having a financial
adviser acknowledge that neither of you can see your accounts unless you
pay a fee. It's alarmingly easy to imagine carmakers' charging fees to
independent repair shops that need access to vehicle data to service a
vehicle purchased for tens of thousands of dollars. That fee will lead
to vehicle owners' paying higher repair prices just so that technicians
can obtain the data.

There are more than 180,000 independent repair shops across the country;
most have all the tools needed to work on today's connected and complex
cars, and most of today's highly trained service technicians can perform
anything from basic tuneups to sophisticated electronic diagnostics. But
without access to car data, they're working blindfolded, unable to see
the diagnostic information they need.

The solution is simple. The only person who should control car data is
the car owner (or lessee). He or she should be aware of the data the car
transmits, have control over it and determine who can see it.

The idea that drivers don't control their own data flies in the face of
what consumers want and expect. A 2018 Ipsos survey found that 71
percent of consumers assume vehicle owners already have direct access to
their vehicle data. Not so. Nearly 90 percent of consumers believe
vehicle owners should control who can see their vehicle's data.
Currently they don't.

Digitization of the auto industry is, ultimately, a good thing. Today's
connected cars are paving the way for autonomous vehicles and
vehicle-to-vehicle communications, and eventually
vehicle-to-infrastructure communications making our roads safer. But
unlike Alexa and Nest, consumers are unaware of the degree to which
their own car collects and processes data.

It's clear, because of its value --- as high as \$750 billion by 2030
--- carmakers have no incentive to release control of the data collected
from our vehicles. Policymakers, however, have the opportunity to give
drivers control --- not just so that they can keep their data private
but also so that they can share it with the people they want to see it.
This will let car owners maintain what they've had for a century: the
right to decide who fixes their car.

\emph{Follow}
\href{https://twitter.com/privacyproject}{\emph{@privacyproject}}
\emph{on Twitter and The New York Times Opinion Section on}
\href{https://www.facebook.com/nytopinion}{\emph{Facebook}}
\emph{and}\href{https://www.instagram.com/nytopinion/}{\emph{Instagram}}\emph{.}

\hypertarget{glossary-replacer}{%
\subsection{glossary replacer}\label{glossary-replacer}}

Advertisement

\protect\hyperlink{after-bottom}{Continue reading the main story}

\hypertarget{site-index}{%
\subsection{Site Index}\label{site-index}}

\hypertarget{site-information-navigation}{%
\subsection{Site Information
Navigation}\label{site-information-navigation}}

\begin{itemize}
\tightlist
\item
  \href{https://help.nytimes.com/hc/en-us/articles/115014792127-Copyright-notice}{©~2020~The
  New York Times Company}
\end{itemize}

\begin{itemize}
\tightlist
\item
  \href{https://www.nytco.com/}{NYTCo}
\item
  \href{https://help.nytimes.com/hc/en-us/articles/115015385887-Contact-Us}{Contact
  Us}
\item
  \href{https://www.nytco.com/careers/}{Work with us}
\item
  \href{https://nytmediakit.com/}{Advertise}
\item
  \href{http://www.tbrandstudio.com/}{T Brand Studio}
\item
  \href{https://www.nytimes.com/privacy/cookie-policy\#how-do-i-manage-trackers}{Your
  Ad Choices}
\item
  \href{https://www.nytimes.com/privacy}{Privacy}
\item
  \href{https://help.nytimes.com/hc/en-us/articles/115014893428-Terms-of-service}{Terms
  of Service}
\item
  \href{https://help.nytimes.com/hc/en-us/articles/115014893968-Terms-of-sale}{Terms
  of Sale}
\item
  \href{https://spiderbites.nytimes.com}{Site Map}
\item
  \href{https://help.nytimes.com/hc/en-us}{Help}
\item
  \href{https://www.nytimes.com/subscription?campaignId=37WXW}{Subscriptions}
\end{itemize}
