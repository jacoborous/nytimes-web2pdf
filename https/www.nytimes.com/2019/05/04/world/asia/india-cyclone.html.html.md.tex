Sections

SEARCH

\protect\hyperlink{site-content}{Skip to
content}\protect\hyperlink{site-index}{Skip to site index}

\href{https://www.nytimes.com/section/world/asia}{Asia Pacific}

\href{https://myaccount.nytimes.com/auth/login?response_type=cookie\&client_id=vi}{}

\href{https://www.nytimes.com/section/todayspaper}{Today's Paper}

\href{/section/world/asia}{Asia Pacific}\textbar{}`The Worst Is Over': A
Sigh of Relief in India, Mostly Spared by Cyclone

\url{https://nyti.ms/2H27uOq}

\begin{itemize}
\item
\item
\item
\item
\item
\end{itemize}

Advertisement

\protect\hyperlink{after-top}{Continue reading the main story}

Supported by

\protect\hyperlink{after-sponsor}{Continue reading the main story}

\hypertarget{the-worst-is-over-a-sigh-of-relief-in-india-mostly-spared-by-cyclone}{%
\section{`The Worst Is Over': A Sigh of Relief in India, Mostly Spared
by
Cyclone}\label{the-worst-is-over-a-sigh-of-relief-in-india-mostly-spared-by-cyclone}}

\includegraphics{https://static01.nyt.com/images/2019/05/04/world/04india-cyclone1/merlin_154360377_2c01aa57-9eff-4d10-9c11-c4296e42248b-articleLarge.jpg?quality=75\&auto=webp\&disable=upscale}

By \href{https://www.nytimes.com/by/hari-kumar}{Hari Kumar},
\href{https://www.nytimes.com/by/jeffrey-gettleman}{Jeffrey Gettleman}
and \href{https://www.nytimes.com/by/sameer-yasir}{Sameer Yasir}

\begin{itemize}
\item
  May 4, 2019
\item
  \begin{itemize}
  \item
  \item
  \item
  \item
  \item
  \end{itemize}
\end{itemize}

PURI, India --- The road to Puri was lined with large downed trees.

Electricity poles made of cement and steel lay scattered on the ground,
along with roofing tiles, pieces of billboard and crumpled iron
sheeting. The walls of many buildings had collapsed, leaving countless
people homeless.

But on Saturday as residents of this seaside town emerged from shelters
to assess the property damage unleashed by Cyclone Fani, there was a
sense of relief. The storm, one of the biggest in years, had slammed
into India's eastern coast on Friday, roaring through towns with winds
of 120 miles per hour.

Yet so many precious lives had been spared. Fewer than 20 fatalities
were reported, compared with the thousands killed 20 years ago, when a
similar cyclone swept over the same area.

That is because authorities here had whisked more than a million people
to safety, executing a meticulous evacuation plan that they have been
perfecting ever since that disastrous storm in 1999.

Though many poor people lost everything they owned, as the skies cleared
on Saturday and the sunshine returned, there was a widespread feeling
that this could have been so much deadlier.

``It seems the worst is over,'' said Bishnupada Sethi, the special
relief commissioner in the Indian state of Odisha, which bore the
initial brunt of the storm.

Over the past few days, as the cyclone barreled up the Bay of Bengal
toward India, the authorities in Odisha helped move people to higher
ground and into specially built cyclone shelters. In neighboring
Bangladesh, lashed by the weakened storm on Saturday, another million
people were similarly taken out of harm's way.

\includegraphics{https://static01.nyt.com/images/2019/05/04/world/04india-cyclone2/merlin_154312782_156620a7-8fd4-4193-9c55-5d1c7fafac7e-articleLarge.jpg?quality=75\&auto=webp\&disable=upscale}

It was a huge, meticulously organized lifesaving operation and, it
seems,
\href{https://www.nytimes.com/2019/05/03/world/asia/cyclone-fani-india-evacuations.html}{a
success story for the government's early-warning system}.

Indian officials transmitted millions of text messages, broadcast
warnings over public address systems and sent fleets of buses to scoop
up vulnerable people, bringing them to hundreds of sturdy cyclone
shelters that had been stocked with water and food.

By the time Fani
\href{https://www.nytimes.com/2019/05/02/world/asia/india-cyclone-fani.html}{hit
on Friday morning}, knocking down trees and obliterating shack houses,
just about everybody was safe. Authorities said the relatively few
fatalities were people who had not heeded their warnings.

Many people remained in the shelters, fearful of floods and other
hazards like downed electrical wires.

``They are welcome to stay as long as they want,'' said Satyajit
Mohanty, a police official in Bhubaneswar, an inland city. The airport
there sustained damage but was scheduled to reopen Saturday night.

The cyclone's storm surge, which pushed waves deep inland, swamped
entire villages. Odisha is one of India's poorest states, home to around
46 million people. Mr. Mohanty predicted it would take months for people
to rebuild their homes. He said the state authorities would ``chip in.''

Early indications on Saturday were that Bangladesh might also be spared
a severe toll. Shah Kamal, the country's disaster management secretary,
said only a few deaths had been reported so far, although information
was still being gathered.

Image

Storm-damaged buildings and trees in Puri on
Saturday.Credit...Dibyangshu Sarkar/Agence France-Presse --- Getty
Images

``I have a feeling that Allah favored us this time,'' he said.

More than 1.2 million people in Bangladesh had been moved to shelters
ahead of the storm, Mr. Kamal said, and some could start returning home
as early as Saturday or Sunday. Still, many face a daunting future:
Entire villages were flooded and thousands of acres of crops were
damaged.

In Odisha, the focus was on clearing debris. The sun was out on Saturday
morning, and power saws were buzzing. In many towns, the air smelled
like sawdust. The storm had weakened significantly as it moved inland in
Bangladesh.

``Every hour, we are clearing more roads and more power lines,'' Mr.
Sethi said. ``All agencies are working hard to bring a sense of normalcy
back to the affected areas.''

He said many lessons had been learned in the fearsome cyclone that hit
Odisha in 1999 and killed thousands. Many people were caught flat-footed
in their homes. Some of the dead were found miles from where they had
lived, dragged away by raging cascades.

Puri, then and now, was one of the most vulnerable towns. It is one of
Odisha's resort destinations, right on the beach, and on Friday, it
seems the eye of the storm passed right over it.

Dr. B.B. Dash, head of the government hospital here, said that it had
received 12 dead bodies and that most of the victims died from injuries
caused by flying objects such as an asbestos sheet, a piece of glass or
a rock. Nearly 200 people were injured.

All around Puri, people were inspecting what the storm had done to them.

``See, our house is destroyed, our boat is destroyed --- what can we
do?'' said Pikki Gopi, 21, a fisherman. ``We don't have any place to
work, eat or sleep. We have nothing to do.''

Image

On the seafront in Puri on Saturday.Credit...Dibyangshu Sarkar/Agence
France-Presse --- Getty Images

Mr. Gopi said he had never seen the wind howl like it did on Friday.

``It was a curse of the sea on us,'' he said. ``I do not know why the
sea became angry with us. We respect the sea.''

Nearby, an older couple lay on their backs on the floor of a badly
damaged home, the roof missing in action. Some fishermen on the beach
played cards, and a woman carefully dried some very wet rupee notes in
the sun.

Pinaki Misra, a member of Parliament from Puri, stood impatiently in a
district magistrate's office. The floor was wet and a large crowd had
gathered, including some prominent people such as Mr. Misra who were
pressing for recovery efforts to move faster.

As the district magistrate relayed instructions to his team, Mr. Misra
listed all the problems the people here were facing: The electricity
grid had been knocked out, the water supply was not functioning, roads
remained blocked and the mobile phone network was not working.

``How can life be normal in this area?'' he asked.

But he was quick to add: ``The evacuation work was done very
efficiently. We hope normalcy will be restored soon.``

Shambhu Sethi's father, Baraj, a rice farmer, was one of the relatively
few who died. He stepped outside his home on Friday morning to urinate
and was hit by a collapsing coconut tree.

``He cried but nobody could hear him,'' Mr. Sethi said.

When the family found Baraj 15 minutes later, he was dead.

``What can I do? I am helpless, I couldn't save him,'' Mr. Sethi said.
``It was none of his fault --- that was our fate. This is nature's curse
on my family.''

Advertisement

\protect\hyperlink{after-bottom}{Continue reading the main story}

\hypertarget{site-index}{%
\subsection{Site Index}\label{site-index}}

\hypertarget{site-information-navigation}{%
\subsection{Site Information
Navigation}\label{site-information-navigation}}

\begin{itemize}
\tightlist
\item
  \href{https://help.nytimes.com/hc/en-us/articles/115014792127-Copyright-notice}{©~2020~The
  New York Times Company}
\end{itemize}

\begin{itemize}
\tightlist
\item
  \href{https://www.nytco.com/}{NYTCo}
\item
  \href{https://help.nytimes.com/hc/en-us/articles/115015385887-Contact-Us}{Contact
  Us}
\item
  \href{https://www.nytco.com/careers/}{Work with us}
\item
  \href{https://nytmediakit.com/}{Advertise}
\item
  \href{http://www.tbrandstudio.com/}{T Brand Studio}
\item
  \href{https://www.nytimes.com/privacy/cookie-policy\#how-do-i-manage-trackers}{Your
  Ad Choices}
\item
  \href{https://www.nytimes.com/privacy}{Privacy}
\item
  \href{https://help.nytimes.com/hc/en-us/articles/115014893428-Terms-of-service}{Terms
  of Service}
\item
  \href{https://help.nytimes.com/hc/en-us/articles/115014893968-Terms-of-sale}{Terms
  of Sale}
\item
  \href{https://spiderbites.nytimes.com}{Site Map}
\item
  \href{https://help.nytimes.com/hc/en-us}{Help}
\item
  \href{https://www.nytimes.com/subscription?campaignId=37WXW}{Subscriptions}
\end{itemize}
