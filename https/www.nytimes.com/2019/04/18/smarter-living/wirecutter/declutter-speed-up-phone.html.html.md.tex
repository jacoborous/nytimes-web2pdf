Sections

SEARCH

\protect\hyperlink{site-content}{Skip to
content}\protect\hyperlink{site-index}{Skip to site index}

\href{https://www.nytimes.com/wirecutter/}{Wirecutter}

\href{https://myaccount.nytimes.com/auth/login?response_type=cookie\&client_id=vi}{}

\href{https://www.nytimes.com/section/todayspaper}{Today's Paper}

\href{/wirecutter/}{Wirecutter}\textbar{}How to Declutter and Speed Up
Your Phone

\url{https://nyti.ms/2V9DvNF}

\begin{itemize}
\item
\item
\item
\item
\item
\end{itemize}

Advertisement

\protect\hyperlink{after-top}{Continue reading the main story}

Supported by

\protect\hyperlink{after-sponsor}{Continue reading the main story}

\hypertarget{how-to-declutter-and-speed-up-your-phone}{%
\section{How to Declutter and Speed Up Your
Phone}\label{how-to-declutter-and-speed-up-your-phone}}

If your phone is feeling a little laggy, here are some tips to clear out
old apps and other things that may be slowing it down.

\includegraphics{https://static01.nyt.com/images/2019/04/26/smarter-living/wirecutter/00wc_declutter-slide-YUJU/00wc_declutter-slide-YUJU-articleLarge.jpg?quality=75\&auto=webp\&disable=upscale}

By Thorin Klosowski

Mr. Klosowski is a staff writer at Wirecutter, a product recommendation
site owned by The New York Times Company.

\begin{itemize}
\item
  April 18, 2019
\item
  \begin{itemize}
  \item
  \item
  \item
  \item
  \item
  \end{itemize}
\end{itemize}

\href{https://www.nytimes.com/es/2019/04/25/celular-lento-como-resolver/}{Leer
en español}

If you've never bothered to organize the apps on your phone, clean out
old files or wrangle your notifications into a sensible order, that
disorder can make your phone an overwhelming, slow and buggy device. You
can fix this and give your phone new life. Decluttering takes just a few
minutes.

\hypertarget{delete-apps-you-dont-use}{%
\subsection{Delete apps you don't use}\label{delete-apps-you-dont-use}}

Ever downloaded an app for a single purpose, such as a conference, work
meeting or vacation, and then left that app on your phone to digitally
rot away on the home screen? Be honest.

The fastest, easiest way to declutter your phone is to get rid of apps
you don't need, and both Apple's iPhone and Google's Android provide
simple ways to figure out which apps you don't use.

The easiest way to find those neglected apps is to look at all of your
apps in a list, organized by the ones you use least often. On an iPhone,
head to Settings, General, iPhone Storage. On Android, open the Play
Store, tap the hamburger menu in the top-left corner, tap My Apps \&
Games, Installed, Alphabetical and change it to Last Used. Delete apps
that are listed as Never Used or that you haven't opened in months. I
also prefer to delete rarely used apps for services where I can just use
the website instead.

Once you've cleared out apps you don't need, it's time to organize the
home screen. Everyone's sense of order is different, but having a system
--- any system --- in place is useful to prevent clutter in the future.

Melanie Pinola, managing editor for \href{https://zapier.com/}{Zapier},
has a simple method for organizing folders: ``One thing I learned is to
group apps into folders by verb or action. So, `Write,' `Contact,'
`Read,' etc. This makes it easier to get directly to what you want to do
on your phone and is also gratifying in a way to tie an app you're
opening with a purpose and action item.''

Sometimes organization is fruitless, and if your phone takes too much
time to organize, there's one easy solution: Don't bother.

Instead, get in the practice of launching apps from the search menu. On
an iPhone, pull down on the home screen to open search, type the first
few letters of an app name, and then tap the app when it pops up. On
Android, swipe up from the bottom of the screen to pull up the app
drawer and then start typing. Once you get the hang of launching apps
like this, I recommend limiting your home screen to four or five rows of
apps you use the most and hiding everything else on another page.

\hypertarget{free-up-storage}{%
\subsection{Free up storage}\label{free-up-storage}}

Sometimes I miss 16 gigabyte phones --- the studio apartments of phones
--- which required a certain mindfulness and decision-making to prevent
them from overflowing. In this age of nearly infinite storage, it's easy
to collect more junk, and as you run out of space your phone begins to
feel sluggish. To clear this out and speed up your phone, you need to
peek into some menus you may have never visited.

First up are your messages. Threads filled with GIFs, memes, videos and
photos can take up a ton of space. In iOS, you can change how long your
phone stores messages so it clears out those old threads automatically,
without you having to do it.

\emph{{[}Like what you're reading?}
\href{https://www.nytimes.com/newsletters/smarter-living?module=inline}{\emph{Sign
up here}} \emph{for the Smarter Living newsletter to get stories like
this (and much more!) delivered straight to your inbox every Monday
morning.{]}}

Head to Settings, Messages, and Keep Messages. Once there, set how long
you want to keep messages before they self-destruct. If you want to keep
the text but delete attachments, head instead to Settings, General, then
iPhone Storage, scroll down to Messages and then tap Review large
Attachments. This screen will show you all the big files.

Android's Messages app doesn't have a setting like this, but you can
swipe left or right on a message thread in Messages to archive old
threads. Most third-party apps, including
\href{https://faq.whatsapp.com/en/android/26000068/}{WhatsApp} and
\href{https://www.facebook.com/help/messenger-app/242107552657620/}{Facebook
Messenger}, have some means to clear out old messages.

The biggest storage hogs on your phone are likely photos and videos.
\href{https://www.nytimes.com/2018/01/11/smarter-living/backing-up-your-photos.html}{Back
them up} to an online cloud backup service like
\href{https://www.apple.com/icloud/photos/}{Apple iCloud},
\href{https://photos.google.com/}{Google Photos} or
\href{https://www.amazon.com/Cloud-Drive-Storage/b?ie=UTF8\&node=13234696011}{Amazon's
Prime Photos}. Once you back up the photos, you can delete them from
your phone and access them remotely through the backup service.

If you're still short on space even after taking those steps, the
culprit might be an app that's hoarding data. Podcasts, music services
and video apps are usually the biggest offenders. To see what's taking
up space in iOS, head to Settings, General, then iPhone Storage. On
Android, pull down the notification shade, tap the cog and then select
Storage. This screen displays a list of all the apps on your phone.

For example, the Amazon Prime Video app on my phone takes up two
gigabytes of space, even though I've never downloaded a video. To clear
out app data on an iPhone, tap the app from the Storage screen and then
tap Offload. Once the phone is done deleting everything, tap Reinstall.
On Android, tap the app name and then the Clear Cache button.

\emph{Sign up for the}
\href{https://thewirecutter.us5.list-manage.com/subscribe?u=570aa9140d54361ad5a594320\&id=bb5d08fe40}{\emph{Wirecutter
Weekly Newsletter}} \emph{and get our latest recommendations every
Sunday.}

A version of this article appears at
\href{https://thewirecutter.com/blog/how-to-declutter-your-phone/}{Wirecutter.com}.

Advertisement

\protect\hyperlink{after-bottom}{Continue reading the main story}

\hypertarget{site-index}{%
\subsection{Site Index}\label{site-index}}

\hypertarget{site-information-navigation}{%
\subsection{Site Information
Navigation}\label{site-information-navigation}}

\begin{itemize}
\tightlist
\item
  \href{https://help.nytimes.com/hc/en-us/articles/115014792127-Copyright-notice}{©~2020~The
  New York Times Company}
\end{itemize}

\begin{itemize}
\tightlist
\item
  \href{https://www.nytco.com/}{NYTCo}
\item
  \href{https://help.nytimes.com/hc/en-us/articles/115015385887-Contact-Us}{Contact
  Us}
\item
  \href{https://www.nytco.com/careers/}{Work with us}
\item
  \href{https://nytmediakit.com/}{Advertise}
\item
  \href{http://www.tbrandstudio.com/}{T Brand Studio}
\item
  \href{https://www.nytimes.com/privacy/cookie-policy\#how-do-i-manage-trackers}{Your
  Ad Choices}
\item
  \href{https://www.nytimes.com/privacy}{Privacy}
\item
  \href{https://help.nytimes.com/hc/en-us/articles/115014893428-Terms-of-service}{Terms
  of Service}
\item
  \href{https://help.nytimes.com/hc/en-us/articles/115014893968-Terms-of-sale}{Terms
  of Sale}
\item
  \href{https://spiderbites.nytimes.com}{Site Map}
\item
  \href{https://help.nytimes.com/hc/en-us}{Help}
\item
  \href{https://www.nytimes.com/subscription?campaignId=37WXW}{Subscriptions}
\end{itemize}
