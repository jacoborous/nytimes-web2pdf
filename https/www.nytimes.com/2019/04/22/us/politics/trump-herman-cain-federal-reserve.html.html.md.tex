Sections

SEARCH

\protect\hyperlink{site-content}{Skip to
content}\protect\hyperlink{site-index}{Skip to site index}

\href{https://www.nytimes.com/section/politics}{Politics}

\href{https://myaccount.nytimes.com/auth/login?response_type=cookie\&client_id=vi}{}

\href{https://www.nytimes.com/section/todayspaper}{Today's Paper}

\href{/section/politics}{Politics}\textbar{}As Herman Cain Bows Out of
Fed Contention, Focus Shifts to Stephen Moore

\url{https://nyti.ms/2IBgIVh}

\begin{itemize}
\item
\item
\item
\item
\item
\item
\end{itemize}

Advertisement

\protect\hyperlink{after-top}{Continue reading the main story}

Supported by

\protect\hyperlink{after-sponsor}{Continue reading the main story}

\hypertarget{as-herman-cain-bows-out-of-fed-contention-focus-shifts-to-stephen-moore}{%
\section{As Herman Cain Bows Out of Fed Contention, Focus Shifts to
Stephen
Moore}\label{as-herman-cain-bows-out-of-fed-contention-focus-shifts-to-stephen-moore}}

\includegraphics{https://static01.nyt.com/images/2019/04/22/us/politics/22dc-cain-02/merlin_152455779_b039e958-fb14-4912-aebe-10358f258965-articleLarge.jpg?quality=75\&auto=webp\&disable=upscale}

By \href{https://www.nytimes.com/by/jim-tankersley}{Jim Tankersley} and
\href{https://www.nytimes.com/by/alan-rappeport}{Alan Rappeport}

\begin{itemize}
\item
  April 22, 2019
\item
  \begin{itemize}
  \item
  \item
  \item
  \item
  \item
  \item
  \end{itemize}
\end{itemize}

WASHINGTON --- President Trump announced Monday that
\href{https://www.nytimes.com/2020/07/30/us/politics/herman-cain-dead.html}{Herman
Cain}, one of his two embattled picks for the Federal Reserve Board, had
withdrawn his name from consideration, even as his second candidate came
under new scrutiny over his attitudes toward women.

Mr. Cain, a former pizza chain executive, made his decision as he
battled old accusations of sexual harassment that had halted his 2012
presidential campaign.

His withdrawal bows to political reality in a moment when Mr. Trump has
faced mounting criticism for tapping loyalists to join the historically
independent Fed. And it moved a spotlight to the other man Mr. Trump has
said he wants to put on the Fed, his economic adviser Stephen Moore, who
faced new objections on Monday because of a series of magazine columns
that denigrated women, including his wife at the time.

Mr. Moore has called the writings jokes, but the criticism suggests that
questions of harassment and sexism could prove more consequential for
Mr. Trump's nominees than interest rates and other policy issues. In his
columns, published in the early 2000s by the conservative magazine
National Review, some of which were first reported by CNN, he complained
that women are ``sooo malleable'' because his wife at the time voted for
a Democrat, based on a campaign commercial.

In other pieces, Mr. Moore said that female tennis players ``want equal
pay for inferior work'' and called it a ``travesty'' that women wanted
to play pickup basketball with men. He called for women to be banned
from the N.C.A.A. men's basketball tournament, unless they were as
attractive as the CBS sports journalist Bonnie Bernstein, who, he wrote,
``should wear a halter top.''

``Here's the rule change I propose,'' Mr. Moore
\href{https://web.archive.org/web/20020614192103/http://www.nationalreview.com/moore/moore031902.asp}{wrote
in 2002}. ``No more women refs, no women announcers, no women beer
vendors, no women anything. There is, of course, an exception to this
rule. Women are permitted to participate, if and only if, they look like
Bonnie Bernstein. The fact that Bonnie knows nothing about basketball is
entirely irrelevant.''

He lamented: ``Is there no area in life where men can take vacation from
women? What's next? Women invited to bachelor parties?''

Ms. Bernstein replied on Twitter: ``You want halter tops? Hit the club
scene. You want hoops knowledge? Try actually listening.''

Mr. Moore, who has not been formally nominated for the Fed board, did
not respond to a message seeking comment on the writings on Monday.
Earlier in the day, he said in a text message that Mr. Trump would
follow through on his nomination ``when I get all the paperwork and
financial disclosure done.''

Critics previously questioned Mr. Cain's and Mr. Moore's shifting views
on interest rates, which both men said should be higher under President
Barack Obama, when the nation was struggling to recover from the 2008
financial crisis, but now say should stay low under Mr. Trump, when the
economy is growing. Both men have previously voiced support of a return
to a gold standard, which the United States abandoned decades ago and
few economists now support.

Congressional Republicans say those views are less likely to impede Mr.
Moore's confirmation prospects than concerns over his personal life and
past statements.

White House officials have insisted in recent weeks that Mr. Moore's
nomination was on track, despite controversies over a
\href{https://www.nytimes.com/2019/03/27/us/politics/stephen-moore-taxes.html}{\$75,000
tax lien} filed against him by the Internal Revenue Service and a judge
finding him in contempt of court several years ago for failing to pay
more than \$300,000 in
\href{https://www.nytimes.com/2019/04/02/business/trump-fed-stephen-moore.html}{past-due
child support and alimony}.

\includegraphics{https://static01.nyt.com/images/2019/04/22/us/politics/22dc-cain/22dc-cain-articleLarge-v2.jpg?quality=75\&auto=webp\&disable=upscale}

They had been less certain about Mr. Cain, whose selection stoked
bipartisan concern over his qualifications for the post and allegations
of sexual harassment. At least four Republican senators had said they
would oppose his confirmation, if Mr. Trump were to formally nominate
him ---
\href{https://www.nytimes.com/2019/04/11/business/herman-cain-fed.html}{effectively
killing Mr. Cain's chances} in the Senate, where Republicans have 53
seats.

Mr. Cain
\href{https://www.westernjournal.com/hermancain/real-reasons-withdrew-fed-consideration-direct-source-need/}{said
Monday afternoon on The Western Journal}, a conservative website, that
he had decided to withdraw after prayerful consideration, and after
questioning whether a Fed seat was worth giving up his online commentary
platform and various ``business interests,'' including paid speeches.

``Without getting too specific about how big a pay cut this would be,''
Mr. Cain wrote, ``let's just say I'm pretty confident that if your boss
told you to take a similar pay cut, you'd tell him where to go.''

The decision was an abrupt departure from last week, when Mr. Cain vowed
to stay in the running, telling The Wall Street Journal he was
\href{https://www.wsj.com/articles/herman-cain-says-he-wont-withdraw-from-consideration-for-fed-board-11555529563}{``very
committed''} to being nominated. In an
\href{https://www.wsj.com/articles/the-fed-and-the-professor-standard-11555541719}{op-ed
column in that newspaper}, Mr. Cain criticized what he called ``the
professor standard'' for Fed nominees under presidents dating to Bill
Clinton in the 1990s, which he said had caused the central bank to lose
sight of the importance of keeping the dollar strong and stable to
support economic growth.

In his column on Monday, Mr. Cain said he had begun to question whether
joining the Fed would be, in effect, a step down in influence from his
perch writing conservative commentary. ``I also started wondering if I'd
be giving up too much influence to get a little bit of policy impact,''
he wrote.

The president said Mr. Cain had asked him not to nominate him for the
seat.

``I will respect his wishes,'' Mr. Trump said on Twitter, calling Mr.
Cain ``a great American who truly loves our country.''

With the attention shifting to Mr. Moore on Monday, Democrats and
women's groups called his comments sexist and disqualifying.

``The report about Stephen Moore making sexist comments about women in
sports is disturbing,'' said Representative Carolyn B. Maloney of New
York, the vice chairwoman of the Joint Economic Committee and a critic
of Mr. Moore's nomination, ``and if true would be another reason why he
shouldn't be allowed on the Federal Reserve Board.''

It is unclear whether he, too, might face bipartisan opposition. Some
Republicans said privately earlier this month that Mr. Cain's struggles
could help Mr. Moore, because Republican senators would be unlikely to
vote against both of Mr. Trump's nominees. They were more divided on
Monday, saying Mr. Cain's departure could open Mr. Moore to additional
scrutiny and attacks.

A senior Republican aide on Capitol Hill said conservatives would be
watching how Mr. Moore's writings about women sat with Senators Lisa
Murkowski of Alaska and Susan Collins of Maine, Republicans who are
considered swing votes for his nomination. The aide said Mr. Moore's
chances could also depend on whether his ex-wife, Allison, speaks
publicly about their difficult divorce and its aftermath.

Records of that divorce were unsealed this month in a Virginia
courthouse, and they include allegations of infidelity and Mr. Moore's
failure to pay court-ordered support.

According to filings, in 2010 Mr. Moore started a sexual relationship
with a woman he met through an online dating service. His wife found
bills that showed Mr. Moore pumping gas in the morning near the home of
the other woman and buying an airplane ticket in her name. At their
son's graduation ceremony, Mr. Moore said to his children, in earshot of
Ms. Moore, ``I have two women, and what's really bad is when they fight
over you.''

Before the divorce, Mr. Moore frequently teased his ex-wife in his
National Review columns. In 2001, he wrote that ``she's been acting as
if it's her patriotic duty to single-handedly revive the American
economy with her frenetic pace of consumer spending.'' In 2003, he wrote
that ``Allison consumes but she still doesn't produce.'' In 2004, he
wrote, ``Here's the best news of all: For once, Allison isn't
pregnant.''

Others columns criticized the notion of a ``pay gap'' between male and
female athletes.

``Women tennis pros don't really want equal pay for equal work. They
want equal pay for inferior work,'' Mr. Moore wrote in 2000. ``If there
is an injustice in tennis, it's that women like Martina Hingis and
Monica Seles make millions of dollars a year, even though there are
hundreds of men at the collegiate level (assuming their schools haven't
dropped the sport) who could beat them handily.''

Advertisement

\protect\hyperlink{after-bottom}{Continue reading the main story}

\hypertarget{site-index}{%
\subsection{Site Index}\label{site-index}}

\hypertarget{site-information-navigation}{%
\subsection{Site Information
Navigation}\label{site-information-navigation}}

\begin{itemize}
\tightlist
\item
  \href{https://help.nytimes.com/hc/en-us/articles/115014792127-Copyright-notice}{©~2020~The
  New York Times Company}
\end{itemize}

\begin{itemize}
\tightlist
\item
  \href{https://www.nytco.com/}{NYTCo}
\item
  \href{https://help.nytimes.com/hc/en-us/articles/115015385887-Contact-Us}{Contact
  Us}
\item
  \href{https://www.nytco.com/careers/}{Work with us}
\item
  \href{https://nytmediakit.com/}{Advertise}
\item
  \href{http://www.tbrandstudio.com/}{T Brand Studio}
\item
  \href{https://www.nytimes.com/privacy/cookie-policy\#how-do-i-manage-trackers}{Your
  Ad Choices}
\item
  \href{https://www.nytimes.com/privacy}{Privacy}
\item
  \href{https://help.nytimes.com/hc/en-us/articles/115014893428-Terms-of-service}{Terms
  of Service}
\item
  \href{https://help.nytimes.com/hc/en-us/articles/115014893968-Terms-of-sale}{Terms
  of Sale}
\item
  \href{https://spiderbites.nytimes.com}{Site Map}
\item
  \href{https://help.nytimes.com/hc/en-us}{Help}
\item
  \href{https://www.nytimes.com/subscription?campaignId=37WXW}{Subscriptions}
\end{itemize}
