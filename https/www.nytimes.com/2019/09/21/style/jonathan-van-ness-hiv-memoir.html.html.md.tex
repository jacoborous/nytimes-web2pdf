Sections

SEARCH

\protect\hyperlink{site-content}{Skip to
content}\protect\hyperlink{site-index}{Skip to site index}

\href{https://www.nytimes.com/section/style}{Style}

\href{https://myaccount.nytimes.com/auth/login?response_type=cookie\&client_id=vi}{}

\href{https://www.nytimes.com/section/todayspaper}{Today's Paper}

\href{/section/style}{Style}\textbar{}Jonathan Van Ness of `Queer Eye'
Comes Out

\href{https://nyti.ms/32WMiCV}{https://nyti.ms/32WMiCV}

\begin{itemize}
\item
\item
\item
\item
\item
\item
\end{itemize}

Advertisement

\protect\hyperlink{after-top}{Continue reading the main story}

Supported by

\protect\hyperlink{after-sponsor}{Continue reading the main story}

\hypertarget{jonathan-van-ness-of-queer-eye-comes-out}{%
\section{Jonathan Van Ness of `Queer Eye' Comes
Out}\label{jonathan-van-ness-of-queer-eye-comes-out}}

The reality-show star says he's living with H.I.V., and speaks about
being an addict and a sexual abuse survivor.

\includegraphics{https://static01.nyt.com/images/2019/09/23/fashion/21JonathanVanNess-9/merlin_159579261_79946c8d-cc7b-49ea-a5de-bbf3676480c2-articleLarge.jpg?quality=75\&auto=webp\&disable=upscale}

\href{https://www.nytimes.com/by/alex-hawgood}{\includegraphics{https://static01.nyt.com/images/2019/02/20/multimedia/author-alex-hawgood/author-alex-hawgood-thumbLarge.png}}

By \href{https://www.nytimes.com/by/alex-hawgood}{Alex Hawgood}

\begin{itemize}
\item
  Published Sept. 21, 2019Updated Oct. 11, 2019
\item
  \begin{itemize}
  \item
  \item
  \item
  \item
  \item
  \item
  \end{itemize}
\end{itemize}

Jonathan Van Ness was having a late breakfast at the Empire Diner,
around the corner from his one-bedroom apartment in the Chelsea
neighborhood of Manhattan.

Seated in a window booth, he was serving what he calls his
``16th-century Jesus'' look: Hollywood-starlet tresses, a mustache à la
a Super Mario villain and
\href{https://www.instagram.com/p/B1oDLPLA_Zr/}{fingernails} painted
with cartoon depictions from the 1996 film ``The First Wives Club.''

But Mr. Van Ness was not feeling his normal gorgeous self, the
boisterous ``Yass queen'' merman that fans of
\href{https://www.nytimes.com/2018/02/07/arts/television/queer-eye-netflix-review.html}{``Queer
Eye}'' adore. He was hung over.

And no, it wasn't from partying too much. It was a ``vulnerability
hangover,'' to use a term coined by
\href{https://www.ted.com/talks/brene_brown_on_vulnerability?utm_campaign=tedspread\&utm_medium=referral\&utm_source=tedcomshare}{Brené
Brown}, a TED Talk-famous researcher, to describe feelings of dread
after being forthcoming.

``I've had nightmares every night for the past three months because I'm
scared to be this vulnerable with people,'' Mr. Van Ness said.

For much of the summer, Mr. Van Ness, 32, has been mentally preparing
himself for the release of his piercing memoir,
``\href{https://www.harpercollins.com/9780062906373/over-the-top/}{Over
the Top},'' on Sept. 24, in which a different image of Mr. Van Ness
unspools with remarkable transparency.

\emph{{[}The book is on our}
\href{https://www.nytimes.com/2019/10/11/books/review/jonathan-van-nesss-gorgeous-memoir-is-a-best-seller.html}{\emph{nonfiction
best-seller list}}\emph{.{]}}

Subtitled a ``Raw Journey to Self-Love,'' the book doesn't so much
explode as offer psychological insight into the hirsute gay fairy
godmother in heels or, as he puts it, ``the effervescent, gregarious
majestic center-part-blow-dry cotton-candy figure-skating queen'' that
he portrays on ``Queer Eye.''

``It's hard for me to be as open as I want to be when there are certain
things I haven't shared publicly,'' he said. He cracked his knuckles as
he fidgeted from nerves. ``These are issues that need to be talked
about.''

He ordered another cup of coffee, his fifth of the day, and began
tearing up as he spoke about a particularly painful memory, one of many
that he divulges in his book. When he was much younger, he was abused by
an older boy from church, during what was supposed to be a make-believe
play session.

``For a lot of people who are survivors of sexual assault at a young
age, we have a lot of compounded trauma,'' he said.

Suddenly, a 20-something woman with a ponytail appeared at the table.
``I'm so sorry, I can't take a picture right now,'' he said, discreetly
wiping his eyes.

``Oh, that's fine. I just want to say that I love the show,'' she said.

``Thank you. Namaste. Have a nice day,'' he said, clasping his hands in
prayer.

Mr. Van Ness exhaled and gently took a sip of coffee. ``If you're having
a terrible moment or in the middle of a conversation about something
serious, people don't care,'' he said. ``They want their bubbly J.V.N.
and to get that major selfie.''

\includegraphics{https://static01.nyt.com/images/2019/09/21/fashion/21JonathanVanNess-11/merlin_159579201_1b6e73c8-0b8e-4630-824b-282692a7dabf-articleLarge.jpg?quality=75\&auto=webp\&disable=upscale}

\hypertarget{sex-drugs-and-hair}{%
\subsection{Sex, Drugs and Hair}\label{sex-drugs-and-hair}}

In a sense, the memoir was a way for Mr. Van Ness to tell his story
without interruption. There are certainly moments that may make some
readers pause.

Mr. Van Ness grew up in Quincy, Ill., a small port city along the
Mississippi River, where he was a self-described ``little baby queen''
unafraid to embrace his femininity. It helped to have a mostly
supportive family, including a mother, Mary Winters, he considers a
lifelong best friend.

Ms. Winters's family owns \href{https://quincymediacareers.com}{Quincy
Media}, a media company that operates 16 television stations in
Illinois, Wisconsin and elsewhere, as well as two local newspapers. She
is the company's vice president; Jonathan's father, Jon Van Ness, worked
in sales. (They divorced when he was 5, and his mother remarried four
years later.)

At Quincy Senior High School (which he visits in the latest season of
``Queer Eye''), he leapt over social norms to become the school's first
male cheerleader. Never mind the beer bottles thrown at him during
games.

He wasn't exactly popular, and students spread rumors about his
friendship with a closeted boy from his swim class. Mr. Van Ness felt
humiliated. ``I was too fat, too femme, too loud and too unlovable,'' he
said.

Image

A scene from ``Queer Eye,'' in which Mr. Van Ness is the resident hair
expert.~Credit...Netflix

His lack of self-esteem ran deep. As therapy would later reveal, the
abuse he experienced as a young child planted the seed for other
self-destructive behaviors. In his early teens, he spent hours in AOL
chat rooms (this was the 1990s) and met up with older men for sex. One
man, he recounts in the book, ``turned whiter than Ann Coulter's fan
base'' after learning he was underage.

He found other ways to fill the void, including binge eating junk food
like doughnuts when his stepfather died (he
\href{https://www.instagram.com/p/BhsBoWwhQtt/?utm_source=ig_embed}{gained
70 pounds} in three months).

Eager to leave Quincy, he earned extra credit to skip senior year and
attended the University of Arizona in Tucson. But during his first
semester, he blew a monthly allowance of \$200 from his mother on
cocaine, which he started doing on weekends.

Instead of asking his mother for more money (he was too ashamed and
reckless at the time), he advertised sex for money on Gay.com, a chat
and personals site.

He flunked out of college his first year --- he was 19 --- and sulked
home with his ponytail between his legs.

Unsure what to do with his life, he decided to take the skills honed
from styling the hair of his Barbie dolls to the next level and enrolled
in an 11-month beautician program at
\href{https://aveda.edu/locations/location/minneapolis-mn/}{the Aveda
Institute in Minneapolis}, where his first clients included many Somali
refugees.

After getting his certificate, he moved to Scottsdale, Ariz. (to be near
his dying grandmother) and then to Los Angeles, where he supported
himself as an assistant at a Sally Hershberger salon.

But his addiction to sex and drugs got worse. When he was in his early
20s, a couple he met on Grindr introduced him to smoking
methamphetamine. He went to rehab twice and relapsed both times.

One day, when he was 25, he fainted in a salon while highlighting a
client's hair. The next day he went to Planned Parenthood to diagnose
his flulike symptoms. He tested positive for H.I.V.

``That day was just as devastating as you would think it would be,'' he
writes.

Image

At the Creative Arts Emmy Awards in Los Angeles this
month.~Credit...Nina Prommer/EPA, via Shutterstock

Image

At a Netflix party after the Creative Arts Emmys.Credit...Charley
Gallay/Getty Images for Netflix

Image

At the MTV Video Music Awards in Newark in August.Credit...Evan
Agostini/Invision, via Associated Press

\hypertarget{his-own-makeover}{%
\subsection{His Own Makeover}\label{his-own-makeover}}

He cleaned up his act; he still drinks and smokes marijuana but says he
hasn't done hard drugs in years. And, using money from a family trust,
he started anew in Los Angeles.

Appropriately enough, his foray into entertainment began at the hair
salon. During an appointment with his friend
\href{http://www.eringibson.com}{Erin Gibson}, a comedian who worked for
Funny or Die, the two came up with a parody series called
``\href{https://www.youtube.com/watch?v=N0yi0RQvEi0}{Gay of Thrones},''
in which Mr. Van Ness and a guest comedian offer campy, gay-themed
recaps of ``Game of Thrones.''

The show premiered in 2013 and became a hit. (It has been nominated for
three \href{https://www.emmys.com/shows/gay-thrones}{Creative Arts
Emmys} for short-form variety series). Soon, Mr. Van Ness was offered
roles as a red-carpet commentator and as a host of other web series.

Then, in 2016, his manager called with news that would truly flip his
hair: Netflix was holding auditions for a reboot of ``Queer Eye.'' It
took many weeks, but Mr. Van Ness eventually won the producers over.

Image

Mr. Van Ness with the ``Queer Eye'' cast.Credit...Netflix

In the show's four seasons, the ``Queer Eye'' cast has gone from fringe
gay personalities to mainstream celebrities, with Mr. Van Ness as one of
the series's breakout stars.

In a recent episode set in Kansas City, Mo., he confronts the shame
associated with
\href{https://www.nytimes.com/2018/04/02/fashion/braids-weaves-extensions-and-traction-alopecia.html}{traction
alopecia}, a form of hair loss that predominantly affects black women.
It's a topic rarely discussed on television, and even rarer by someone
who is white.

On Twitter,
\href{https://twitter.com/tressiemcphd/status/1153144560324923392}{Tressie
McMillan Cottom}, an author and professor of sociology at Virginia
Commonwealth University, wrote: ``Jonathan treating this sister with
traction alopecia with love is more care than I can recall a regular
black woman getting on TV ever.'' (When he was shown that tweet, he
burst into tears.)

Mr. Van Ness hopes to bring attention to what he calls ``gorgeous beauty
moments'' like that with his memoir, especially misperceptions about
being H.I.V. positive. He is healthy and now describes himself as an
out-and-proud ``member of the beautiful H.I.V.-positive community.''

``When `Queer Eye' came out, it was really difficult because I was like,
`Do I want to talk about my status?,'' he said. ``And then I was like,
`The Trump administration has done everything they can do to have the
stigmatization of the L.G.B.T. community thrive around me.''' He paused
before adding, ``I do feel the need to talk about this.''

Image

Just as he was about to take a bite of his eggs at the diner, Mr. Van
Ness was interrupted once again. This time it was a boyish young man who
poked his head in the window to profess his admiration.

After another ``namaste,'' which appears to be his shorthand for
``kindly leave,'' Mr. Van Ness resumed his thoughts. ``These are all
difficult subjects to talk about on a makeover show about hair and
makeup,'' he said. ``That doesn't mean `Queer Eye' is less valid, but I
want people to realize you're never too broken to be fixed.''

Advertisement

\protect\hyperlink{after-bottom}{Continue reading the main story}

\hypertarget{site-index}{%
\subsection{Site Index}\label{site-index}}

\hypertarget{site-information-navigation}{%
\subsection{Site Information
Navigation}\label{site-information-navigation}}

\begin{itemize}
\tightlist
\item
  \href{https://help.nytimes.com/hc/en-us/articles/115014792127-Copyright-notice}{©~2020~The
  New York Times Company}
\end{itemize}

\begin{itemize}
\tightlist
\item
  \href{https://www.nytco.com/}{NYTCo}
\item
  \href{https://help.nytimes.com/hc/en-us/articles/115015385887-Contact-Us}{Contact
  Us}
\item
  \href{https://www.nytco.com/careers/}{Work with us}
\item
  \href{https://nytmediakit.com/}{Advertise}
\item
  \href{http://www.tbrandstudio.com/}{T Brand Studio}
\item
  \href{https://www.nytimes.com/privacy/cookie-policy\#how-do-i-manage-trackers}{Your
  Ad Choices}
\item
  \href{https://www.nytimes.com/privacy}{Privacy}
\item
  \href{https://help.nytimes.com/hc/en-us/articles/115014893428-Terms-of-service}{Terms
  of Service}
\item
  \href{https://help.nytimes.com/hc/en-us/articles/115014893968-Terms-of-sale}{Terms
  of Sale}
\item
  \href{https://spiderbites.nytimes.com}{Site Map}
\item
  \href{https://help.nytimes.com/hc/en-us}{Help}
\item
  \href{https://www.nytimes.com/subscription?campaignId=37WXW}{Subscriptions}
\end{itemize}
