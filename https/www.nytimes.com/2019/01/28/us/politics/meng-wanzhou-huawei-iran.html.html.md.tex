Sections

SEARCH

\protect\hyperlink{site-content}{Skip to
content}\protect\hyperlink{site-index}{Skip to site index}

\href{https://www.nytimes.com/section/politics}{Politics}

\href{https://myaccount.nytimes.com/auth/login?response_type=cookie\&client_id=vi}{}

\href{https://www.nytimes.com/section/todayspaper}{Today's Paper}

\href{/section/politics}{Politics}\textbar{}Huawei and Top Executive
Face Criminal Charges in the U.S.

\url{https://nyti.ms/2SbjO6z}

\begin{itemize}
\item
\item
\item
\item
\item
\item
\end{itemize}

Advertisement

\protect\hyperlink{after-top}{Continue reading the main story}

Supported by

\protect\hyperlink{after-sponsor}{Continue reading the main story}

\hypertarget{huawei-and-top-executive-face-criminal-charges-in-the-us}{%
\section{Huawei and Top Executive Face Criminal Charges in the
U.S.}\label{huawei-and-top-executive-face-criminal-charges-in-the-us}}

\includegraphics{https://static01.nyt.com/images/2019/01/29/us/politics/29dc-huawei-2-print/merlin_148541184_c54ef9e9-c27a-4c61-bc9d-a1aad594854b-articleLarge.jpg?quality=75\&auto=webp\&disable=upscale}

By \href{https://www.nytimes.com/by/david-e-sanger}{David E. Sanger},
\href{https://www.nytimes.com/by/katie-benner}{Katie Benner} and
\href{https://www.nytimes.com/by/matthew-goldstein}{Matthew Goldstein}

\begin{itemize}
\item
  Jan. 28, 2019
\item
  \begin{itemize}
  \item
  \item
  \item
  \item
  \item
  \item
  \end{itemize}
\end{itemize}

\href{https://cn.nytimes.com/usa/20190129/meng-wanzhou-huawei-iran/}{阅读简体中文版}\href{https://cn.nytimes.com/usa/20190129/meng-wanzhou-huawei-iran/zh-hant/}{閱讀繁體中文版}

WASHINGTON --- The Justice Department unveiled sweeping charges on
Monday against the Chinese telecom firm Huawei and its chief financial
officer, Meng Wanzhou, outlining a decade-long attempt by the company to
steal trade secrets, obstruct a criminal investigation and evade
economic sanctions on Iran.

The pair of indictments, which were partly unsealed on Monday, come amid
\href{https://www.nytimes.com/2019/01/26/us/politics/huawei-china-us-5g-technology.html}{a
broad and aggressive campaign} by the United States to try to thwart
China's biggest telecom equipment maker. Officials have long suspected
Huawei of working to advance Beijing's global ambitions and undermine
America's interests and have begun taking steps to curb its
international presence.

The charges underscore Washington's determination to prove that Huawei
poses a national security threat and to convince other nations that it
cannot be trusted to build their next generation of wireless networks,
known as 5G. The indictments, based in part on the company's internal
emails, describe a plot to steal testing equipment from T-Mobile
laboratories in Bellevue, Wash. They also cite internal memos, obtained
from Ms. Meng, that prosecutors said link her to an elaborate bank fraud
that helped Huawei profit by evading Iran sanctions.

The acting attorney general, Matthew G. Whitaker, flanked by the heads
of several other cabinet agencies, said the United States would seek to
have Ms. Meng extradited from Canada, where she was
\href{https://www.nytimes.com/2018/12/05/business/huawei-cfo-arrest-canada-extradition.html}{detained
last year} at the request of the United States.

The charges outlined Monday come at a sensitive diplomatic moment, as
top officials from China are expected to arrive in Washington this week
for two days of talks aimed at resolving a monthslong trade war between
the world's two largest economies. Trump administration officials have
insisted that Ms. Meng's detention will not affect the trade talks, but
the timing of the indictment coming so close to in-person discussions is
likely to further strain relations between the two countries.

Ms. Meng is the daughter of Huawei's founder and one of the most
powerful industrialists in the country. Her arrest has outraged the
Chinese government, which has since arrested two Canadians, in apparent
retaliation. The indictment now presents Canada with a politically
charged decision: whether to extradite Ms. Meng to face the fraud
charges, or make a legal or political determination to send her back to
Beijing.

In a statement on Tuesday, China's Foreign Ministry called on the United
States and Canada to let Ms. Meng go.

``For a long time, the U.S. has used state power to smear and attack
certain Chinese companies in an attempt to stifle legitimate business
operations,'' it said. ``Behind that, there is strong political
motivation and manipulation. We strongly urge the U.S. to stop
unreasonable suppression of Chinese companies, including Huawei, and
treat Chinese enterprises fairly and objectively.''

A spokesman for Huawei, Joe Kelly, said it ``is not aware of any
wrongdoing by Ms. Meng, and believes the U.S. courts will ultimately
reach the same conclusion.'' The company also denied violating American
laws.

The indictment unsealed against Ms. Meng is similar to the charges
leveled against the Huawei executive in filings made by federal
prosecutors in connection with the bail hearing in Canada. It claimed
that Huawei defrauded four large banks into clearing transactions with
Iran in violation of international sanctions through a subsidiary called
Skycom. Federal authorities did not identify the banks, but in an
earlier court proceeding in Canada after Ms. Meng's arrest, prosecutors
had identified one of the banks as HSBC.

The most serious new allegation in the indictment, which could have
bearing on the extradition proceeding in Canada, is the contention by
federal prosecutors that Huawei sought to impede the investigation into
the telecom company's attempt to evade economic sanctions on Iran by
destroying or concealing evidence.

The indictment said Huawei moved employees out of the United States so
they could not be called as witnesses before a grand jury in Brooklyn.
And authorities said the company destroyed evidence in order to hinder
the inquiry.

Richard P. Donoghue, the United States attorney for the Eastern District
of New York, said that the telecom firm's actions began in 2007 and
``allowed Iran to evade sanctions imposed by the United States and to
allow Huawei to profit.''

Ms. Meng's lawyer in the United States, Reid Weingarten, a leading
white-collar lawyer at Steptoe \& Johnson in Washington, said that his
client ``should not be a pawn or hostage'' in the relationship between
the United States and China.

``Ms. Meng is an ethical and honorable businesswoman who has never spent
a second of her life plotting to violate any U.S. law, including the
Iranian sanctions,'' Mr. Weingarten said. ``We are confident that
justice will be done.''

The arrest of a top executive for sanctions evasion is unusual. In 2015,
Deutsche Bank was fined \$258 million for violating American sanctions
on Iran and Syria. No executives involved in the scheme were indicted,
though six employees were fired.

Ms. Meng is under house arrest at one of two residences that she owns in
Vancouver. American officials said Monday that they will request her
extradition before a deadline on Wednesday. The next stage of her case
will be decided at the Supreme Court of British Columbia.

``Companies like Huawei pose a dual threat to both our economic and
national security,'' said Christopher A. Wray, the F.B.I. director, who
joined Mr. Whitaker and two other cabinet members, Wilbur Ross, the
commerce secretary, and Kirstjen Nielsen, the homeland security
secretary.

Mr. Wray argued that ``the magnitude of these charges make clear just
how seriously the F.B.I. takes this threat.''

``Today should serve as a warning that we will not tolerate businesses
that violate our laws, obstruct justice or jeopardize national and
economic well-being,'' he added.

Parts of the indictment were redacted and left open the question of
whether the United States had secretly indicted Ms. Meng's father, Ren
Zhengfei, a former People's Liberation Army officer and member of the
Communist Party. A United States government interview with Mr. Ren from
2007 is cited in one of the indictments, to make the case that he misled
investigators, and the name of at least one of those indicted is blacked
out from the publicly filed version of the indictment.

Mr. Whitaker fueled the speculation about an indictment of Mr. Ren when
he told reporters on Monday that the criminal activity ``goes all the
way to the top of the company.''

The Justice Department also accused Huawei of conspiring to steal trade
secrets from a competitor, T-Mobile. The charges relate to
\href{https://www.nytimes.com/2019/01/16/technology/huawei-investigation-trade-secrets.html}{a
criminal investigation that stemmed from a 2014 civil suit} between the
two companies.

In that case, T-Mobile accused Huawei of stealing proprietary robotics
technology that the telecom company used to diagnose quality-control
issues in cellphones. Huawei was found guilty in May 2017. The
indictment cited internal emails from Huawei and its American subsidiary
that appeared to set up a bonus system for employees who could illicitly
obtain the T-Mobile testing system.

``These are very serious actions by a company that appears to be using
corporate espionage not only to enhance their bottom line but to compete
in the world economy,'' Mr. Whitaker said.

The legal drama now shifts to Canada, where the government has warned
that it will not extradite Ms. Meng if it appears that the request is
being made for political reasons. President Trump said after her arrest
that he would consider using her case for leverage in the upcoming trade
negotiations, which fueled speculation that the United States may be
more interested in Ms. Meng's value in winning trade concessions than in
obtaining a conviction.

Canada's ambassador to Beijing
\href{https://www.nytimes.com/2019/01/26/world/americas/canada-ambassador-china-huawei.html}{was
fired over the weekend} by Prime Minister Justin Trudeau for suggesting
that the case against Ms. Meng was political and that Canada might
accede to Chinese demands and return her home.

Mr. Whitaker declined to say Monday whether the White House would
interfere in the criminal case against Ms. Meng. But the array of
officials present at the announcement was clearly intended to
demonstrate a coordinated government effort to go after Huawei.

``Given the seriousness of these charges, and the direct involvement of
cabinet officials in their rollout, today's announcements underscore
that there is a unified full-court press by the administration to hold
China accountable for the theft of proprietary U.S. technology and
violations of U.S. export control and sanctions laws,'' said David
Laufman, the former chief of the Justice Department's
counterintelligence and export control section.

The indictments could further complicate the trade talks that the
administration is holding this week with Beijing. The Trump
administration is seeking significant changes to China's trade
practices, including what it says is a pattern of Beijing pressuring
American companies to hand over valuable technology and outright theft
of intellectual property.

``The Americans are not going to surrender global technological
supremacy without a fight, and the indictment of Huawei is the opening
shot in that struggle,'' said Michael Pillsbury, a China scholar at the
Hudson Institute who advises the Trump administration.

Lawmakers like Senator Mark Warner, Democrat of Virginia, who have long
argued for action to be taken against Chinese technology providers
including Huawei and ZTE, a smaller firm that has faced similar
accusations, called the indictment ``a reminder that we need to take
seriously the risks of doing business with companies like Huawei and
allowing them access to our markets.''

Mr. Warner said that he would continue to press Canada to reconsider
using any Huawei technology as it upgrades its telecommunications
network.

On Tuesday, American intelligence officials are expected to cite 5G
investments by Chinese telecom companies, including Huawei, as a
worldwide threat. And the United States has been
\href{https://www.nytimes.com/2018/12/11/us/politics/trump-china-trade.html?module=inline}{drafting
an executive order}, expected in the coming weeks, that would
effectively ban American companies from using Chinese-origin equipment
in critical telecommunications networks.

Advertisement

\protect\hyperlink{after-bottom}{Continue reading the main story}

\hypertarget{site-index}{%
\subsection{Site Index}\label{site-index}}

\hypertarget{site-information-navigation}{%
\subsection{Site Information
Navigation}\label{site-information-navigation}}

\begin{itemize}
\tightlist
\item
  \href{https://help.nytimes.com/hc/en-us/articles/115014792127-Copyright-notice}{©~2020~The
  New York Times Company}
\end{itemize}

\begin{itemize}
\tightlist
\item
  \href{https://www.nytco.com/}{NYTCo}
\item
  \href{https://help.nytimes.com/hc/en-us/articles/115015385887-Contact-Us}{Contact
  Us}
\item
  \href{https://www.nytco.com/careers/}{Work with us}
\item
  \href{https://nytmediakit.com/}{Advertise}
\item
  \href{http://www.tbrandstudio.com/}{T Brand Studio}
\item
  \href{https://www.nytimes.com/privacy/cookie-policy\#how-do-i-manage-trackers}{Your
  Ad Choices}
\item
  \href{https://www.nytimes.com/privacy}{Privacy}
\item
  \href{https://help.nytimes.com/hc/en-us/articles/115014893428-Terms-of-service}{Terms
  of Service}
\item
  \href{https://help.nytimes.com/hc/en-us/articles/115014893968-Terms-of-sale}{Terms
  of Sale}
\item
  \href{https://spiderbites.nytimes.com}{Site Map}
\item
  \href{https://help.nytimes.com/hc/en-us}{Help}
\item
  \href{https://www.nytimes.com/subscription?campaignId=37WXW}{Subscriptions}
\end{itemize}
