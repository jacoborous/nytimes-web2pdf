Sections

SEARCH

\protect\hyperlink{site-content}{Skip to
content}\protect\hyperlink{site-index}{Skip to site index}

\href{https://www.nytimes.com/section/world/europe}{Europe}

\href{https://myaccount.nytimes.com/auth/login?response_type=cookie\&client_id=vi}{}

\href{https://www.nytimes.com/section/todayspaper}{Today's Paper}

\href{/section/world/europe}{Europe}\textbar{}Poland Arrests 2,
Including Huawei Employee, Accused of Spying for China

\url{https://nyti.ms/2RM7g5q}

\begin{itemize}
\item
\item
\item
\item
\item
\item
\end{itemize}

Advertisement

\protect\hyperlink{after-top}{Continue reading the main story}

Supported by

\protect\hyperlink{after-sponsor}{Continue reading the main story}

\hypertarget{poland-arrests-2-including-huawei-employee-accused-of-spying-for-china}{%
\section{Poland Arrests 2, Including Huawei Employee, Accused of Spying
for
China}\label{poland-arrests-2-including-huawei-employee-accused-of-spying-for-china}}

\includegraphics{https://static01.nyt.com/images/2019/01/12/world/europe/12huawei/12huawei-articleLarge.jpg?quality=75\&auto=webp\&disable=upscale}

By \href{https://www.nytimes.com/by/adam-satariano}{Adam Satariano} and
\href{https://www.nytimes.com/by/joanna-berendt}{Joanna Berendt}

\begin{itemize}
\item
  Jan. 11, 2019
\item
  \begin{itemize}
  \item
  \item
  \item
  \item
  \item
  \item
  \end{itemize}
\end{itemize}

LONDON --- The Polish authorities arrested two people, including a
Chinese employee of the telecommunications giant Huawei, and charged
them with spying for Beijing, officials said on Friday, as the United
States and its allies move to restrict the use of Chinese technology
because of concerns that it is being used for espionage.

The arrest of the Huawei employee is almost certain to escalate tensions
between Western countries and China over the company, which the
authorities in the United States have accused of acting as an arm of the
Chinese government and making equipment designed for spying.

In December, the
\href{https://www.nytimes.com/2018/12/05/business/huawei-cfo-arrest-canada-extradition.html?module=inline}{daughter
of Huawei's founder was arrested in Canada} at the request of the United
States, which said
\href{https://www.nytimes.com/2018/12/07/technology/huawei-meng-wanzhou-fraud.html}{she
had committed fraud} as part of a scheme to violate American sanctions
against companies doing business with Iran. It was unclear whether the
arrests in Poland had been requested by the United States. But a senior
Western diplomat who was briefed on them said the Justice Department had
been working with the Polish government.

Europe is increasingly a battleground in the fight over Huawei, the
world's second-largest smartphone maker and a top supplier of networking
equipment. The company's sales in the region have been growing, but many
countries there now face pressure to reconsider its presence,
particularly as construction begins for the next-generation wireless
networks
\href{https://www.nytimes.com/2018/12/31/technology/personaltech/5g-what-you-need-to-know.html}{known
as 5G}. Germany, Britain, the Czech Republic and Norway are among the
nations that have recently questioned how deeply Huawei should be
involved in developing 5G infrastructure.

Many of the countries adopting this stance are allies of the United
States. Poland, specifically, is
\href{https://www.state.gov/r/pa/ei/bgn/2875.htm}{regarded by the State
Department} as ``one of the United States' strongest partners'' in
continental Europe. And Andrus Ansip, the European Union's vice
president, said last month that countries in the region should be
``worried'' about Huawei and other Chinese companies because of the
cybersecurity risks they pose.

Other countries face the same dilemma. In December, Japan barred Huawei
from obtaining government contracts, and Australia and New Zealand have
taken steps to block the company from being involved in the building of
5G networks in those countries. And the Czech National Cyber and
Information Security Agency recently restated its long-standing warnings
about Huawei and ZTE, another Chinese telecommunications company.

``Picking sides may be unavoidable,'' said Lukasz Olejnik, a research
associate at the Center for Technology and Global Affairs at Oxford
University, which studies the impact of technology on international
relations. ``This is a difficult policy conundrum.''

Huawei has long denied spying for the Chinese government. On Friday, a
spokesman for the company said it had no comment on the arrest in Poland
and insisted that it ``complies with all applicable laws and regulations
in the countries where it operates.''

The second person arrested is an employee of the French
telecommunications company Orange, which confirmed that its office had
been raided and that the man's belongings had been seized.

``We are ready to cooperate with the Internal Security Agency and make
any information it needs available,'' the Orange spokesman, Wojciech
Jabczynski, said.

Hu Xijin, the editor in chief of Global Times, a state-run, nationalist
newspaper in China, took a swipe at Poland on Twitter on Friday,
writing: ``Anything in Poland that is worthy of stealing for Huawei?
Polish national security department flatters itself.''

In another message, he said he had met up with a friend who works at
Huawei. ``He said Huawei is facing great difficulties communicating with
Western public opinion,'' Mr. Hu wrote. ``I said Huawei has been trying
to distance itself from politics, but it has grown too big that politics
is coming to its door. Huawei is innocent.''

The arrested Huawei employee was identified by the authorities only as
Weijing W. He was involved in the company's sales operations in the
country, officials said. According to Polish television, Weijing W.
graduated from Beijing Foreign Studies University with a degree in
Polish studies and once worked at the Chinese Consulate in Gdansk,
Poland. He began working for Huawei in 2011.

The Orange employee was identified as Piotr D., a Polish citizen and
former agent of Poland's internal security service.

Polish law enforcement officers raided the homes and offices of the two
men on Tuesday, officials said. The authorities then had to wait two
days to obtain arrest warrants, typical for Poland. Officials did not
offer details on the alleged crimes, but said the men would be held for
three months while the investigation continued. Both have pleaded not
guilty and have refused to answer questions, the Polish state television
broadcaster, TVP, reported.

The senior Western diplomat briefed on the investigation, who spoke on
the condition of anonymity, would not describe the level of cooperation
between Poland and the United States. The official also declined to
discuss specifics of the case or evidence that may have been shared with
the Polish government.

But the official said the threat posed by Huawei was a high priority for
American officials throughout Europe.

Huawei was founded in 1987 by Ren Zhengfei, a former People's Liberation
Army engineer. The company's equipment is the backbone of mobile
networks around the world, and its smartphones are popular in Europe and
China. Huawei has grown into China's largest maker of telecom equipment,
generating more than \$90 billion in revenue in 2017.

The company has been a leading contender to design 5G networks in Europe
and other parts of the world, but the security concerns of Western
countries have repeatedly hampered its expansion plans.

\href{https://www.nytimes.com/2012/10/09/us/us-panel-calls-huawei-and-zte-national-security-threat.html}{In
a 2012 report}, American lawmakers said Huawei and ZTE were effectively
arms of the Chinese government whose equipment was being used for
espionage. Security firms have reported finding
\href{https://www.nytimes.com/2016/11/16/us/politics/china-phones-software-security.html}{software
installed on Chinese-made phones} that sends users' personal data to
China.

For many years, the United States, where large mobile carriers such as
AT\&T have avoided using Huawei equipment in their networks, has
presented the biggest obstacles to the company.

Last year, United States intelligence agencies told a Senate panel that
Americans should not use Chinese telecom products, and some major
American retailers have stopped selling them. The Federal Communications
Commission is also considering whether to prohibit American telecom
businesses from using equipment from any company deemed a national
security risk,
\href{https://www.nytimes.com/2018/04/17/technology/huawei-trade-war.html}{a
move aimed primarily at Huawei and ZTE} that would effectively shut them
out of creating 5G mobile networks.

Huawei has
\href{https://www.nytimes.com/2018/04/17/technology/china-huawei-washington.html?module=inline}{focused}on
Europe instead, opening several research and development hubs in the
region. More than a quarter of its 2017 revenue came from Europe, the
Middle East and Africa. Only the Chinese market is more important to the
company.

Telecommunication companies in Europe have strong relationships with
Huawei. Deutsche Telekom said in December that it had begun testing a 5G
network in Poland using Huawei equipment. Mr. Jabczynski, the Orange
spokesman, said Huawei was the only company in Poland ready to provide
equipment that met criteria set by state regulations.

Huawei also has around a quarter of the market for smartphones in
Poland, according to the technology research firm Canalys. That makes it
the country's second-largest phone seller after Samsung.

But the challenges facing Huawei in Europe are mounting.

British officials raised alarms about the company's products last year.
The Czech Republic's cybersecurity watchdog has warned against using
Huawei and ZTE products. And just this week, Tor Mikkel Wara, Norway's
justice minister,
\href{https://www.reuters.com/article/us-norway-huawei-tech/norway-considering-whether-to-exclude-huawei-from-building-5g-network-idUSKCN1P31LJ}{told
Reuters} that ``we share the same concerns as the United States and
Britain, and that is espionage on private and state actors in Norway.''

In Poland, at least one of the country's main mobile carriers has been
edging away from the company. The carrier, Play, has long relied on the
company as an equipment supplier. And China Development Bank, a
state-backed lender, has helped finance Play's buying of Huawei
equipment, according to the website of Lasanoz Finance, an advisory firm
that says it worked on the deal. When Play listed its shares on the
Warsaw stock exchange in 2017, it said Huawei had provided a
``significant portion'' of the equipment on its network.

But last July, as concerns about Chinese hardware grew, Play announced
that
\href{https://www.ericsson.com/en/press-releases/2018/7/play-selects-ericsson-to-accelerate-nationwide-mobile-network-expansion-in-poland}{it
had chosen Ericsson as an additional supplier} for certain components.

The arrests in Poland promised to generate more concerns about the
company, said David Balson, a former official in the British
intelligence agency GCHQ who is now the director of intelligence at
Ripjar, a threat detection company in London.

``There is no good outcome whether it's true or not true,'' Mr. Balson
said. ``There is no good way of spinning it.''

Poland and other countries that crack down on Huawei risk retaliation
from China. Shortly after the Canadian authorities arrested Meng
Wanzhou, the daughter of Mr. Ren, the Huawei founder, China detained two
Canadian citizens last month and accused them of undermining China's
national security. China has since
\href{https://www.nytimes.com/2018/12/20/world/asia/china-canadian-detained.html}{detained
other Canadians}.

Ms. Meng has been
\href{https://www.nytimes.com/2018/12/11/technology/huawei-executive-canada-bail-decision.html}{freed
on bail} in Vancouver, British Columbia, pending a decision on whether
she should be extradited to the United States. This week, the public
face of Huawei in Canada, Scott Bradley, left the company.

``This will definitely increase tension in Europe,'' said Christian
Schmidkonz, a professor of Asian-Pacific business studies at the Munich
Business School.

Advertisement

\protect\hyperlink{after-bottom}{Continue reading the main story}

\hypertarget{site-index}{%
\subsection{Site Index}\label{site-index}}

\hypertarget{site-information-navigation}{%
\subsection{Site Information
Navigation}\label{site-information-navigation}}

\begin{itemize}
\tightlist
\item
  \href{https://help.nytimes.com/hc/en-us/articles/115014792127-Copyright-notice}{©~2020~The
  New York Times Company}
\end{itemize}

\begin{itemize}
\tightlist
\item
  \href{https://www.nytco.com/}{NYTCo}
\item
  \href{https://help.nytimes.com/hc/en-us/articles/115015385887-Contact-Us}{Contact
  Us}
\item
  \href{https://www.nytco.com/careers/}{Work with us}
\item
  \href{https://nytmediakit.com/}{Advertise}
\item
  \href{http://www.tbrandstudio.com/}{T Brand Studio}
\item
  \href{https://www.nytimes.com/privacy/cookie-policy\#how-do-i-manage-trackers}{Your
  Ad Choices}
\item
  \href{https://www.nytimes.com/privacy}{Privacy}
\item
  \href{https://help.nytimes.com/hc/en-us/articles/115014893428-Terms-of-service}{Terms
  of Service}
\item
  \href{https://help.nytimes.com/hc/en-us/articles/115014893968-Terms-of-sale}{Terms
  of Sale}
\item
  \href{https://spiderbites.nytimes.com}{Site Map}
\item
  \href{https://help.nytimes.com/hc/en-us}{Help}
\item
  \href{https://www.nytimes.com/subscription?campaignId=37WXW}{Subscriptions}
\end{itemize}
