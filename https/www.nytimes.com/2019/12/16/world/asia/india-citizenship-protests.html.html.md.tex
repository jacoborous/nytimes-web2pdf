Sections

SEARCH

\protect\hyperlink{site-content}{Skip to
content}\protect\hyperlink{site-index}{Skip to site index}

\href{https://www.nytimes.com/section/world/asia}{Asia Pacific}

\href{https://myaccount.nytimes.com/auth/login?response_type=cookie\&client_id=vi}{}

\href{https://www.nytimes.com/section/todayspaper}{Today's Paper}

\href{/section/world/asia}{Asia Pacific}\textbar{}As Protests Rage on
Citizenship Bill, Is India Becoming a Hindu Nation?

\url{https://nyti.ms/2M09eep}

\begin{itemize}
\item
\item
\item
\item
\item
\item
\end{itemize}

Advertisement

\protect\hyperlink{after-top}{Continue reading the main story}

Supported by

\protect\hyperlink{after-sponsor}{Continue reading the main story}

\hypertarget{as-protests-rage-on-citizenship-bill-is-india-becoming-a-hindu-nation}{%
\section{As Protests Rage on Citizenship Bill, Is India Becoming a Hindu
Nation?}\label{as-protests-rage-on-citizenship-bill-is-india-becoming-a-hindu-nation}}

Several people have been killed as unrest spreads to new corners of the
country. Many see the passage of a new law as anti-Muslim.

\includegraphics{https://static01.nyt.com/images/2020/01/16/world/16india-sub/16india-sub-videoSixteenByNine3000.jpg}

\href{https://www.nytimes.com/by/jeffrey-gettleman}{\includegraphics{https://static01.nyt.com/images/2018/10/10/multimedia/author-jeffrey-gettleman/author-jeffrey-gettleman-thumbLarge.png}}\href{https://www.nytimes.com/by/maria-abi-habib}{\includegraphics{https://static01.nyt.com/images/2018/10/08/multimedia/author-maria-abi-habib/author-maria-abi-habib-thumbLarge.png}}

By \href{https://www.nytimes.com/by/jeffrey-gettleman}{Jeffrey
Gettleman} and \href{https://www.nytimes.com/by/maria-abi-habib}{Maria
Abi-Habib}

\begin{itemize}
\item
  Published Dec. 16, 2019Updated Feb. 27, 2020
\item
  \begin{itemize}
  \item
  \item
  \item
  \item
  \item
  \item
  \end{itemize}
\end{itemize}

NEW DELHI --- Prime Minister
\href{https://www.nytimes.com/2019/12/17/world/asia/india-protests-citizenship-muslims.html}{Narendra
Modi's} government has rounded up thousands of Muslims in Kashmir,
revoked the area's autonomy and enforced a citizenship test in
northeastern
\href{https://www.nytimes.com/2019/12/17/world/asia/india-protests-citizenship-muslims.html}{India}
that left nearly two million people potentially stateless, many of them
Muslim.

But it was Mr. Modi's gamble to pass a
\href{https://www.nytimes.com/2019/12/11/world/asia/india-muslims-citizenship-narendra-modi.html}{sweeping
new citizenship law} that favors every South Asian faith other than
Islam that has set off days of widespread protests.

The law, which easily passed both houses of Parliament last week, is the
most overt sign, opponents say, that
\href{https://www.nytimes.com/2019/12/17/world/asia/india-protests-citizenship-muslims.html}{Mr.
Modi} intends to turn
\href{https://www.nytimes.com/2019/12/17/world/asia/india-protests-citizenship-muslims.html}{India}
into a Hindu-centric state that would leave the country's 200 million
Muslims at a calculated disadvantage.

Indian Muslims, who have watched anxiously as Mr. Modi's government has
pursued a
\href{https://www.nytimes.com/2019/12/17/world/asia/india-protests-citizenship-muslims.html}{Hindu}
nationalist program, have finally erupted in anger. Over the past few
days, protests have broken in cities across the country.

\includegraphics{https://static01.nyt.com/images/2019/12/16/world/16india-riots/merlin_166033041_e767e2c0-360f-45f8-83bc-8e0d2dea2667-articleLarge.jpg?quality=75\&auto=webp\&disable=upscale}

Mumbai. Chennai. Varanasi. Guwahati. Hyderabad. Bhopal. Patna.
Pondicherry. The
\href{https://www.nytimes.com/2019/12/16/world/asia/india-citizenship-protests.html?action=click\&module=Top\%20Stories\&pgtype=Homepage}{disturbances
keep spreading}, and on Monday they tied up several areas of the
capital, New Delhi.

\emph{{[}Update:}
\href{http://www.nytimes.com/2020/02/27/world/asia/india-violence-hindu-muslim.html}{\emph{Violence
continues in New Delhi, and the police are criticized}}\emph{.{]}}

Mr. Modi's government has responded by calling out troops, shutting down
the internet and imposing curfews, just as it did when it
\href{https://www.nytimes.com/2019/08/10/world/asia/kashmir-india-pakistan.html}{clamped
down on Kashmir}. In New Delhi, police officers beat unarmed students
with wooden poles, dragging them away and sending scores to the
hospital. In Assam, they shot and killed several young men.

\href{https://www.nytimes.com/2019/12/17/world/asia/india-internet-modi-protests.html}{India's
Muslims} had stayed relatively quiet during the other recent setbacks,
keenly aware of the electoral logic that has pushed them to the margins.
\href{https://www.nytimes.com/2019/12/17/world/asia/india-internet-modi-protests.html}{India}
is about 80 percent Hindu, and 14 percent Muslim, and Mr. Modi and his
party won a crushing election victory in May and handily control the
Parliament.

Image

In the city of Guwahati, one man grieves Sunday for his brother who was
shot by the police last week and later died.Credit...Ahmer Khan for The
New York Times

But
\href{https://www.nytimes.com/2019/12/17/world/asia/india-internet-modi-protests.html}{Indian
Muslims} are feeling increasingly desperate, and so are progressives,
many Indians of other faiths, and those who see a secular government as
fundamental to India's identity and its future.

The world is now weighing in, too. United Nations officials, American
representatives, international advocacy groups and religious
organizations have issued scathing statements saying that the
citizenship law is blatantly discriminatory. Some are even calling
for\href{https://www.uscirf.gov/news-room/press-releases-statements/uscirf-raises-serious-concerns-and-eyes-sanctions}{sanctions}.

Critics are deeply worried that Mr. Modi is trying to wrench India away
from its secular, democratic roots and turn this nation of 1.3 billion
people into a religious state, a homeland for Hindus.

``They want a theocratic state,'' said B.N. Srikrishna, a former judge
on India's Supreme Court. ``This is pushing the country to the brink, to
the brink of chaos.''

``This is how waves of communal violence start in the country,'' he
added.

Image

Police fire tear gas at a crowd of protesters in the city of Guwahati
last week.Credit...Ahmer Khan

Mr. Modi is no stranger to communal violence. The worst bloodshed that
India has seen in recent years exploded on his watch, in 2002, in
Gujarat, when he was the top official in the state and
\href{https://www.nytimes.com/interactive/2014/04/06/world/asia/modi-gujarat-riots-timeline.html\#/\#time287_8514}{clashes
between Hindus and Muslims} killed more than 1,000 people --- most of
them Muslims.

Mr. Modi was widely blamed for not doing enough to stop it. Courts have
cleared him, but many people believe he was at least partly responsible
for the brutality that unfolded.

His grip on power is still firm, even with a weakening economy. The
political opposition, including the once-dominant Indian National
Congress party, has been disorganized and shaky compared with the
juggernaut he and his right-hand man, Amit Shah, the home minister, have
built in their Bharatiya Janata Party.

But this contentious citizenship law, which paves a special path for
non-Muslim migrants in India to become citizens, has galvanized the
opposition. Rival opposition leaders who usually can't agree on anything
are planning protests together. Students from across the country are
rallying to each other's defense. Each episode of harsh police action
captured by mobile phone and beamed around cyberspace catalyzes more
sympathy, more protests --- and more prospects for violence.

Image

The police use force against demonstrators in New Delhi.Credit...Agence
France-Presse --- Getty Images

On Monday, Mr. Modi called for calm,
\href{https://twitter.com/narendramodi}{saying on Twitter} that the law
``does not affect any citizen of India of any religion'' and ``the need
of the hour is for all of us to work together.''

This citizenship law was a much touted campaign promise and a special
wish of his Hindu base. His supporters see a future India as a place
that emphasizes its Hindu heritage as much as possible and eliminates
the special legal protections that exist for Muslims and other
minorities.

Some analysts are quick to point out that the course Mr. Modi's party is
charting could lead to an India dominated by one faith and one
viewpoint, where existential tensions hold back the economy and
hamstring politics --- as embodied by Pakistan, India's struggling,
Islamic archenemy next door.

"We're chasing a failed dream,'' said Yogendra Yadav, a political
commentator.

The new citizenship legislation, called the Citizenship Amendment Act,
expedites Indian citizenship for migrants from some of India's
neighboring countries if they are Hindu, Christian, Buddhist, Sikh,
Parsee or Jain. Only one major religion in South Asia was left off:
Islam.

Indian officials have denied any anti-Muslim bias and said the measure
was intended purely to help persecuted minorities migrating from India's
predominantly Muslim neighbors --- Pakistan, Afghanistan and Bangladesh.

The legislation follows hand in hand with a divisive citizenship test
conducted this summer in one state in northern India and possibly soon
to be expanded nationwide.

Image

Police block the passage of students during a protest in
Lucknow.Credit...Agence France-Presse --- Getty Images

All residents of Assam, along the Bangladesh border, had to produce
documentary proof that they or their ancestors lived in India since
1971. Around two million of Assam's population of 33 million --- a mix
of Hindus and Muslims ---
\href{https://www.nytimes.com/2019/08/31/world/asia/india-muslim-citizen-list.html}{failed
to pass the test} and now risk being rendered stateless. Huge new
prisons are being built to house anyone determined to be an illegal
immigrant.

A widespread belief is that the Indian government will use both these
measures --- the citizenship tests and the new citizenship law --- to
strip away rights from Muslims who have been living in India for
generations. The way this will happen, many Muslim Indians fear, is that
they will be asked to produce old birth certificates or property deeds
necessary to prove citizenship and they will be unable to do so. And
while Hindu residents in the same situation will be given a pass, it
seems, Muslim residents will not have the same legal protections.

The protests started last week in Assam, led by Hindus and joined by
Assamese Muslims and Christians.
\href{https://www.nytimes.com/2019/12/12/world/asia/india-protests-citizenship-bill.html}{Many
Hindus in Assam don't like the new law} either: They fear it could open
the floodgates to poor migrants who will settle in Assam and take their
land. And so, despite the legislation's potential to drive a sectarian
wedge across India, in Assam it backfired, unifying protesters across
religious lines.

Witnesses said that security forces fired live ammunition into a crowd,
killing a 17-year-old Christian student as he walked home. Outraged
demonstrators then burned down train stations and attacked the police.

Struggling to maintain control, the central government sent in the army
and shut down the internet. The angry crowds only grew.

Image

Protesters block a road in Assam.Credit...Ahmer Khan

On Sunday, the protests spread to Delhi. Students of Jamia Millia
Islamia University, a predominantly Muslim institution, gathered
peacefully, witnesses said. But chaos broke out after a separate group
of violent protesters joined the fray and began clashing with the
police, witnesses said.

The police response was swift and indiscriminate, according to
witnesses, and videos widely circulated on social media showed officers
beating students. In one
\href{https://twitter.com/Free_soul_ali/status/1206281963725885440}{video},
a group of female students tries to rescue a young man from the grasp of
the police. A squad of officers in riot gear tears him away and knocks
him down with heavy blows. Even after the women form a protective circle
around him, officers can be seen jabbing the young man with their wooden
poles.

Observers said that while police brutality was common in poorer, more
rural areas of India, it was extremely unusual to see it explode on such
a scale in the capital.

Waqar Azam, a 26-year-old student, was studying in the university
library when students burst in and yelled that the police were coming.
The students locked the doors. Moments later, tear gas canisters crashed
through the windows, filling the library with choking smoke.

``What is happening to Indian Muslims today did not happen overnight,''
Mr. Azam said. ``If we don't protest against it now, we will end up
living like slaves.''

Mr. Modi's supporters have dismissed the protesters as being exclusively
Muslim, or from a die-hard political opposition group. But in Assam,
many protesters said they had voted for the Bharatiya Janata Party and
now regretted it.

On Monday morning, some 5,000 protesters, of many faiths, gathered in
central Guwahati, Assam's capital.

One chant echoed across town: ``Down with Modi!''

Image

Police on guard under a bridge in Guwahati. A curfew was imposed in
response to the growing demonstrations.Credit...Anupam Nath/Associated
Press

Reporting was contributed by Suhasini Raj from Guwahati, India, by Vindu
Goel from Mumbai and by Hari Kumar, Shalini Venugopal, Sameer Yasir and
Karan Deep Singh from New Delhi.

Advertisement

\protect\hyperlink{after-bottom}{Continue reading the main story}

\hypertarget{site-index}{%
\subsection{Site Index}\label{site-index}}

\hypertarget{site-information-navigation}{%
\subsection{Site Information
Navigation}\label{site-information-navigation}}

\begin{itemize}
\tightlist
\item
  \href{https://help.nytimes.com/hc/en-us/articles/115014792127-Copyright-notice}{©~2020~The
  New York Times Company}
\end{itemize}

\begin{itemize}
\tightlist
\item
  \href{https://www.nytco.com/}{NYTCo}
\item
  \href{https://help.nytimes.com/hc/en-us/articles/115015385887-Contact-Us}{Contact
  Us}
\item
  \href{https://www.nytco.com/careers/}{Work with us}
\item
  \href{https://nytmediakit.com/}{Advertise}
\item
  \href{http://www.tbrandstudio.com/}{T Brand Studio}
\item
  \href{https://www.nytimes.com/privacy/cookie-policy\#how-do-i-manage-trackers}{Your
  Ad Choices}
\item
  \href{https://www.nytimes.com/privacy}{Privacy}
\item
  \href{https://help.nytimes.com/hc/en-us/articles/115014893428-Terms-of-service}{Terms
  of Service}
\item
  \href{https://help.nytimes.com/hc/en-us/articles/115014893968-Terms-of-sale}{Terms
  of Sale}
\item
  \href{https://spiderbites.nytimes.com}{Site Map}
\item
  \href{https://help.nytimes.com/hc/en-us}{Help}
\item
  \href{https://www.nytimes.com/subscription?campaignId=37WXW}{Subscriptions}
\end{itemize}
