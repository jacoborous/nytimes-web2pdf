Sections

SEARCH

\protect\hyperlink{site-content}{Skip to
content}\protect\hyperlink{site-index}{Skip to site index}

\href{https://www.nytimes.com/section/business/economy}{Economy}

\href{https://myaccount.nytimes.com/auth/login?response_type=cookie\&client_id=vi}{}

\href{https://www.nytimes.com/section/todayspaper}{Today's Paper}

\href{/section/business/economy}{Economy}\textbar{}Initial China Trade
Deal Defuses Tensions, but U.S. Still Has Concerns

\url{https://nyti.ms/2EgUm7j}

\begin{itemize}
\item
\item
\item
\item
\item
\item
\end{itemize}

\begin{itemize}
\item
  \href{https://www.nytimes.com/2020/08/03/us/elections/biden-vs-trump.html?action=click\&pgtype=Article\&state=default\&region=TOP_BANNER\&context=storylines_menu}{Election
  Updates}
\item
  \href{https://www.nytimes.com/article/biden-vice-president-2020.html?action=click\&pgtype=Article\&state=default\&region=TOP_BANNER\&context=storylines_menu}{Biden's
  V.P. Search}
\item
  \href{https://www.nytimes.com/interactive/2020/07/24/us/politics/trump-biden-campaign-donors.html?action=click\&pgtype=Article\&state=default\&region=TOP_BANNER\&context=storylines_menu}{Map
  of Donations}
\item
  \href{https://www.nytimes.com/interactive/2020/us/elections/delegate-count-primary-results.html?action=click\&pgtype=Article\&state=default\&region=TOP_BANNER\&context=storylines_menu}{Delegate
  Count}
\item
  \href{https://www.nytimes.com/interactive/2019/us/politics/2020-presidential-candidates.html?action=click\&pgtype=Article\&state=default\&region=TOP_BANNER\&context=storylines_menu}{The
  Candidates}
\item
  \href{https://www.nytimes.com/newsletters/politics?action=click\&pgtype=Article\&state=default\&region=TOP_BANNER\&context=storylines_menu}{Politics
  Newsletter}
\end{itemize}

Advertisement

\protect\hyperlink{after-top}{Continue reading the main story}

Supported by

\protect\hyperlink{after-sponsor}{Continue reading the main story}

\hypertarget{initial-china-trade-deal-defuses-tensions-but-us-still-has-concerns}{%
\section{Initial China Trade Deal Defuses Tensions, but U.S. Still Has
Concerns}\label{initial-china-trade-deal-defuses-tensions-but-us-still-has-concerns}}

The agreement includes a commitment by China to buy more agriculture
products and to strengthen laws protecting foreign companies operating
there.

\includegraphics{https://static01.nyt.com/images/2019/12/13/business/13dc-uschina/merlin_142267479_385cbf70-474a-4f59-970a-76b5dc00a4e0-articleLarge.jpg?quality=75\&auto=webp\&disable=upscale}

\href{https://www.nytimes.com/by/alan-rappeport}{\includegraphics{https://static01.nyt.com/images/2018/06/12/multimedia/author-alan-rappeport/author-alan-rappeport-thumbLarge-v2.png}}\href{https://www.nytimes.com/by/ana-swanson}{\includegraphics{https://static01.nyt.com/images/2018/12/10/multimedia/author-ana-swanson/author-ana-swanson-thumbLarge.png}}\href{https://www.nytimes.com/by/keith-bradsher}{\includegraphics{https://static01.nyt.com/images/2018/10/08/multimedia/author-keith-bradsher/author-keith-bradsher-thumbLarge.png}}\href{https://www.nytimes.com/by/chris-buckley}{\includegraphics{https://static01.nyt.com/images/2018/10/08/multimedia/author-chris-buckley/author-chris-buckley-thumbLarge.png}}

By \href{https://www.nytimes.com/by/alan-rappeport}{Alan Rappeport},
\href{https://www.nytimes.com/by/ana-swanson}{Ana Swanson},
\href{https://www.nytimes.com/by/keith-bradsher}{Keith Bradsher} and
\href{https://www.nytimes.com/by/chris-buckley}{Chris Buckley}

\begin{itemize}
\item
  Dec. 13, 2019
\item
  \begin{itemize}
  \item
  \item
  \item
  \item
  \item
  \item
  \end{itemize}
\end{itemize}

WASHINGTON --- When President Trump and China confirmed on Friday that
they had reached an initial trade deal, it helped defuse tensions in a
19-month trade war and avoided another round of punishing tariffs
scheduled for this weekend.

But a
\href{https://www.nytimes.com/2019/12/12/business/economy/trump-china-trade-deal.html}{trade
deal} that took nearly two years to reach and inflicted global economic
damage in the process does little to resolve the United States' biggest
concerns about China's trade practices, including its use of industrial
subsidies and state-owned enterprises to dominate global industries like
steel and solar panels.

If signed, the deal
\href{https://ustr.gov/sites/default/files/US-China-Agreement-Fact-Sheet.pdf}{would
increase} Chinese purchases of American farm and energy products, place
limits on Beijing's ability to weaken its currency and provide greater
protections to American companies operating in China. It would also
reduce some of Mr. Trump's tariffs and forestall new tariffs slated for
Sunday.

Yet its main benefit may be to help Mr. Trump politically --- allowing
him to promote large gains to American farmers devastated by the trade
war, calming anxious investors and
\href{https://www.nytimes.com/2019/12/06/us/politics/trump-economy-impeachment.html}{convincing
voters before the 2020 election} that he has lived up to his promise to
get tough on China.

``It's a phenomenal deal,'' Mr. Trump said on Friday at the White House.
``The China deal covers tremendous manufacturing, farming, a lot of
rules, regulations, a lot of things it covers. It's a Phase 1 deal, but
a lot of big things are covered. And I say affectionately: The farmers
are going to have to go out and buy much larger tractors, because it
means a lot of business, a tremendous amount of business.''

Mr. Trump, who campaigned on rewriting the rules of global trade in
America's favor, has spent the past two years upending diplomatic
processes that have long governed trade policy. He has pressured trading
partners by threatening to scrap existing deals and imposing more
tariffs than any other president in modern history. The United States
now has the highest tariff rate of any advanced nation, higher than even
China, India and Turkey.

Whether that approach has achieved Mr. Trump's goal of putting ``America
First'' is an open question. While many credit his tough tactics with
bringing trading partners to the table, others say it has failed to
produce bigger gains than those achieved through traditional trade
negotiations, and destabilized the global economy in the process.

\hypertarget{latest-updates-2020-election}{%
\section{\texorpdfstring{\href{https://www.nytimes.com/2020/08/03/us/elections/biden-vs-trump.html?action=click\&pgtype=Article\&state=default\&region=MAIN_CONTENT_1\&context=storylines_live_updates}{Latest
Updates: 2020
Election}}{Latest Updates: 2020 Election}}\label{latest-updates-2020-election}}

Updated 2020-08-04T01:23:51.312Z

\begin{itemize}
\tightlist
\item
  \href{https://www.nytimes.com/2020/08/03/us/elections/biden-vs-trump.html?action=click\&pgtype=Article\&state=default\&region=MAIN_CONTENT_1\&context=storylines_live_updates\#link-6494b448}{Trump
  assails mail-in voting anew, citing delays in declaring a winner in a
  New York congressional primary.}
\item
  \href{https://www.nytimes.com/2020/08/03/us/elections/biden-vs-trump.html?action=click\&pgtype=Article\&state=default\&region=MAIN_CONTENT_1\&context=storylines_live_updates\#link-3de249e6}{Obama
  issues his first slate of 2020 endorsements.}
\item
  \href{https://www.nytimes.com/2020/08/03/us/elections/biden-vs-trump.html?action=click\&pgtype=Article\&state=default\&region=MAIN_CONTENT_1\&context=storylines_live_updates\#link-54e34d20}{In
  a big shift, Trump is now encouraging mask-wearing in campaign
  emails.}
\end{itemize}

\href{https://www.nytimes.com/2020/08/03/us/elections/biden-vs-trump.html?action=click\&pgtype=Article\&state=default\&region=MAIN_CONTENT_1\&context=storylines_live_updates}{See
more updates}

``Pardon me if I don't pop champagne, but aside from a cessation of
continued escalation, there is not much worth cheering,'' said Scott
Kennedy, a China expert at the Center for Strategic and International
Studies. ``There is still significant ambiguity about what is in the
deal, but based on what we can surmise, it is unclear if the struggles
of the past two and a half years have been worth it. The costs have been
substantial and far-reaching, the benefits narrow and ephemeral.''

The China deal was Mr. Trump's second trade victory of the week,
after\href{https://www.nytimes.com/2019/12/10/us/politics/usmca-trade-deal.html}{Democrats
agreed to support a revised North American Free Trade Agreement}, moving
it closer to becoming law. Businesses welcomed the update of the 25-year
trade pact, and tech companies in particular have praised the
agreement's strong protections for technology.

The pact broke ground politically, gaining the support of Democratic
lawmakers and unions that have long derided existing trade agreements
and perhaps forging a new bipartisan consensus on trade. But as with the
China deal, the United-States-Mexico-Canada Agreement is more a modest
improvement than a transformative overhaul to the economy.

In a note to clients, Gregory Daco, an economist at Oxford Economics,
called the net economic benefits of U.S.M.C.A. ``negligible,'' but
praised the deal for preventing a potential hit to the economy. The
president had threatened to withdraw from NAFTA entirely if his trade
pact was not advanced.

``The principal commercial benefit of both agreements appears to be the
avoidance of what would have been self-inflicted harm --- tariff
escalation with China and termination of NAFTA,'' said Michael J. Smart,
a managing director at Rock Creek Global Advisors, an advisory firm.

The China deal also averts what would have been an economically damaging
escalation of the trade war before the holiday season and the 2020
campaign. Mr. Trump had planned to slap 15 percent tariffs on shoes,
laptops, toys and other goods on Sunday --- a move that would have
resulted in the United States taxing nearly every Chinese import and
most likely inciting more retaliation from Beijing.

The United States has now collected
\href{https://www.cbp.gov/newsroom/stats/trade}{more than \$39 billion}
from the tariffs placed on \$360 billion worth of Chinese goods, which
Mr. Trump says China pays but economists say falls heavily on American
businesses and consumers.

Robert Lighthizer, Mr. Trump's top trade negotiator and one of the
administration's biggest China hawks, said in a briefing that China had
made substantial commitments to increase purchases of American
agriculture, energy, manufacturing and services products.

China's farm purchases are expected to grow to at least \$40 billion
annually over a period of two years, while total exports of food,
energy, manufactured goods and services to China will increase by a
total of \$200 billion, he said. The deal would increase protections for
American intellectual property and end China's practice of forcing
American companies to transfer technology to Chinese partners. It would
also open Chinese markets for financial services and American exports of
beef, poultry, seafood, infant formula and pet food. Tariffs could go
back in place if China fails to live up to its commitments, he said.

Mr. Lighthizer said he expected the deal to be signed in Washington
during the first week of January and take effect 30 days later. He
acknowledged that future phases of negotiations, which would entail some
of the bigger structural changes the United States has been seeking from
China, could still prove challenging to accomplish. During the
negotiations, the United States repeatedly pressed China to make more
transformative changes to its economy, only to be rebuffed and have the
talks nearly collapse.

``There are still a lot of outstanding issues that you're all aware of
between the United States and China, which are very serious issues,''
Mr. Lighthizer said. ``Our sense is that we're better off doing this in
phases than to sit and make no progress at all.''

``I'm not Pollyanna,'' he added. ``This is going to be a very long term
issue.''

Wang Shouwen, China's vice commerce minister, said at a news conference
in Beijing that the two sides had made ``significant progress'' and that
the United States would remove tariffs ``phase by phase,'' suggesting
that the countries had agreed to roll back more tariffs in the future
when additional agreements are reached.

Mr. Wang said that both sides had agreed to complete legal reviews as
quickly as possible and that an official signing was still being worked
out.

The Chinese government did not echo American assertions that it would
buy \$50 billion of farm goods a year, saying only that purchases would
increase by a ``considerable margin'' to meet China's needs for goods
like soybean and pork.

The trade war has taken a major toll on America's farmers, who have seen
Chinese sales of soybeans, pork and other products dry up, and Mr. Trump
has consistently promised that Beijing will commit to buying \$50
billion worth of farm goods as part of a trade pact. China has resumed
buying some American products, in part because an epidemic is
\href{https://www.nytimes.com/2019/04/22/business/china-pigs-african-swine-fever.html}{ravaging
the country's hogs}, sending
\href{https://www.nytimes.com/2019/10/07/business/china-strategic-pork-reserve.html}{pork
prices soaring}.

``I think they'll hit \$50 billion,'' he said on Friday. ``They've
already stepped it up.''

Evan S. Medeiros, a Georgetown University professor who was senior Asia
director at the National Security Council under President Barack Obama,
said the only thing of substance that had been negotiated ---
agricultural purchases for tariff relief --- was shallow compared with
Mr. Trump's earlier, much grander promises.

The other elements thrown into an announcement --- currency, enforcement
of intellectual property rights and greater access by American companies
to China's financial services sector --- were changes that Beijing had
decided to push through regardless and had been working on already.
Their inclusion was ``window dressing,'' he said.

``The real issues we want addressed are structural barriers,'' none of
which are mentioned in the deal, he said.

Investors greeted Friday's announcement somewhat skeptically. The S\&P
500 barely budged and yields on government bonds fell, suggesting
slightly more pessimism on the outlook for growth and inflation. Prices
of key agricultural commodities such as hogs and soybeans rose. Hog
futures rose more than 1 percent. Soybeans jumped on news of the trade
deal but later pared their gains, settling about 1 percent higher on the
day.

Stephen Moore, the Heritage Foundation economist who advised Mr. Trump's
2016 campaign, said that the agreement should be helpful for Mr. Trump's
re-election prospects because the onslaught of more tariffs next year
would have been a drag on the economy.

``It's not a huge victory for Trump in the sense of the content of the
deal,'' said Mr. Moore, whom the president
\href{https://www.nytimes.com/2019/05/02/business/stephen-moore-fed.html}{considered
this year for a seat on the Federal Reserve board}. ``If this is real,
and the Chinese don't pull back again, it sets Trump up nicely for 2020.
The best thing about this is, if it happens, that it prevents a trade
war escalation.''

Alan Rappeport and Ana Swanson reported from Washington, and Keith
Bradsher and Chris Buckley from Beijing. Annie Karni and Edward Wong
contributed reporting from Washington, and Matt Phillips from New York.

\hypertarget{our-2020-election-guide}{%
\section{Our 2020 Election Guide}\label{our-2020-election-guide}}

Updated Aug. 3, 2020

\begin{itemize}
\item
  \begin{center}\rule{0.5\linewidth}{\linethickness}\end{center}

  \hypertarget{the-latest}{%
  \subsection{The Latest}\label{the-latest}}

  \begin{itemize}
  \tightlist
  \item
    President Trump again assails mail-in voting,
    \href{https://www.nytimes.com/2020/08/03/us/politics/trump-mail-in-voting.html?action=click\&pgtype=Article\&state=default\&region=BELOW_MAIN_CONTENT\&context=storylines_guide}{claiming
    without evidence that the process is plagued by fraud}.
  \end{itemize}
\item
  \begin{center}\rule{0.5\linewidth}{\linethickness}\end{center}

  \hypertarget{bidens-vp-search}{%
  \subsection{Biden's V.P. Search}\label{bidens-vp-search}}

  \begin{itemize}
  \tightlist
  \item
    \href{https://www.nytimes.com/article/biden-vice-president-2020.html?action=click\&pgtype=Article\&state=default\&region=BELOW_MAIN_CONTENT\&context=storylines_guide}{Here
    are 13 women} who have been under consideration to be Joe Biden's
    running mate, and why each might be chosen --- and might not be.
  \end{itemize}
\item
  \begin{center}\rule{0.5\linewidth}{\linethickness}\end{center}

  \hypertarget{keep-up-with-our-coverage}{%
  \subsection{Keep Up With Our
  Coverage}\label{keep-up-with-our-coverage}}

  \begin{itemize}
  \tightlist
  \item
    Get an
    \href{https://www.nytimes.com/newsletters/politics?action=click\&pgtype=Article\&state=default\&region=BELOW_MAIN_CONTENT\&context=storylines_guide}{email}
    recapping the day's news
  \end{itemize}

  \begin{itemize}
  \tightlist
  \item
    Download our mobile app on
    \href{https://apps.apple.com/us/app/nytimes/id284862083?ls=1\&mat_click_id=5c79ae7455014fd1bd66b5610c05b8f2-20191112-16948\&referrer=mat_click_id\%3D5c79ae7455014fd1bd66b5610c05b8f2-20191112-16948\%26link_click_id\%3D722930677036718082}{iOS}
    and
    \href{http://a.localytics.com/android?id=com.nytimes.android\&referrer=utm_source\%3Dother_nyt_mobile_web\%26utm_medium\%3DWeb\%2520page\%26utm_term\%3DGeneral\%2520Mobile\%2520Page\%26utm_campaign\%3DNYT\%2520Mobile\%2520General\%2520Page}{Android}
    and turn on Breaking News and Politics alerts
  \end{itemize}
\end{itemize}

Advertisement

\protect\hyperlink{after-bottom}{Continue reading the main story}

\hypertarget{site-index}{%
\subsection{Site Index}\label{site-index}}

\hypertarget{site-information-navigation}{%
\subsection{Site Information
Navigation}\label{site-information-navigation}}

\begin{itemize}
\tightlist
\item
  \href{https://help.nytimes.com/hc/en-us/articles/115014792127-Copyright-notice}{©~2020~The
  New York Times Company}
\end{itemize}

\begin{itemize}
\tightlist
\item
  \href{https://www.nytco.com/}{NYTCo}
\item
  \href{https://help.nytimes.com/hc/en-us/articles/115015385887-Contact-Us}{Contact
  Us}
\item
  \href{https://www.nytco.com/careers/}{Work with us}
\item
  \href{https://nytmediakit.com/}{Advertise}
\item
  \href{http://www.tbrandstudio.com/}{T Brand Studio}
\item
  \href{https://www.nytimes.com/privacy/cookie-policy\#how-do-i-manage-trackers}{Your
  Ad Choices}
\item
  \href{https://www.nytimes.com/privacy}{Privacy}
\item
  \href{https://help.nytimes.com/hc/en-us/articles/115014893428-Terms-of-service}{Terms
  of Service}
\item
  \href{https://help.nytimes.com/hc/en-us/articles/115014893968-Terms-of-sale}{Terms
  of Sale}
\item
  \href{https://spiderbites.nytimes.com}{Site Map}
\item
  \href{https://help.nytimes.com/hc/en-us}{Help}
\item
  \href{https://www.nytimes.com/subscription?campaignId=37WXW}{Subscriptions}
\end{itemize}
