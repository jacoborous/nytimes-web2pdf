Sections

SEARCH

\protect\hyperlink{site-content}{Skip to
content}\protect\hyperlink{site-index}{Skip to site index}

\href{https://www.nytimes.com/section/business}{Business}

\href{https://myaccount.nytimes.com/auth/login?response_type=cookie\&client_id=vi}{}

\href{https://www.nytimes.com/section/todayspaper}{Today's Paper}

\href{/section/business}{Business}\textbar{}Unions Skeptical Trump's
Trade Deal Will Bring Back Auto Jobs

\url{https://nyti.ms/35edizo}

\begin{itemize}
\item
\item
\item
\item
\item
\end{itemize}

Advertisement

\protect\hyperlink{after-top}{Continue reading the main story}

Supported by

\protect\hyperlink{after-sponsor}{Continue reading the main story}

\hypertarget{unions-skeptical-trumps-trade-deal-will-bring-back-auto-jobs}{%
\section{Unions Skeptical Trump's Trade Deal Will Bring Back Auto
Jobs}\label{unions-skeptical-trumps-trade-deal-will-bring-back-auto-jobs}}

The administration has said its North American trade pact would add
76,000 jobs in the sector. Experts are not so sure.

\includegraphics{https://static01.nyt.com/images/2019/12/11/business/11autojobs1/merlin_164607018_ae5e76fe-72ea-4e11-af66-7a2449d1d35d-articleLarge.jpg?quality=75\&auto=webp\&disable=upscale}

\href{https://www.nytimes.com/by/niraj-chokshi}{\includegraphics{https://static01.nyt.com/images/2018/02/20/multimedia/author-niraj-chokshi/author-niraj-chokshi-thumbLarge.jpg}}

By \href{https://www.nytimes.com/by/niraj-chokshi}{Niraj Chokshi}

\begin{itemize}
\item
  Published Dec. 11, 2019Updated Jan. 29, 2020
\item
  \begin{itemize}
  \item
  \item
  \item
  \item
  \item
  \end{itemize}
\end{itemize}

In his State of the Union address in February, President Trump said the
North American trade agreement his administration had negotiated would
\href{https://www.nytimes.com/2019/02/06/business/nafta-trump-deal.html}{reverse
the flow of auto manufacturing jobs} by ensuring ``that more cars are
proudly stamped with our four beautiful words: Made in the U.S.A.''

But with that deal, the
\href{https://www.nytimes.com/2020/01/29/business/economy/usmca-deal.html}{United
States-Mexico-Canada Agreement}, now
\href{https://www.nytimes.com/2019/12/10/us/politics/usmca-trade-deal.html}{almost
certain to become law}, some analysts and unions are voicing skepticism
that Mr. Trump will be able to make good on that grand promise.

``It's not at all clear that there is going to be a positive effect on
jobs in the auto industry,'' said Mary Lovely, a senior fellow at the
Peterson Institute for International Economics and a professor of
economics at Syracuse University. ``This is the hard lesson of
economics, which is basically there's a lot of factors here.''

The trouble, she and other experts said, is that the very provisions in
Mr. Trump's deal aimed at lifting employment could drive up the cost of
making cars. That, in turn, could reduce demand and jobs. As a result,
the pact's net result on jobs could be a wash.

Those provisions include a requirement that 75 percent of the parts in a
vehicle come from North America, up from the 62.5 percent required by
the North American Free Trade Agreement, which the new trade agreement
is revising.

The new deal would also require that 40 percent to 45 percent of a
vehicle's contents be made by workers earning at least \$16 an hour,
which is much more than the prevailing wage at Mexican auto factories.
(Under
\href{https://www.nytimes.com/2019/10/25/business/gm-contract.html}{labor
agreements General Motors and other automakers recently negotiated with
the United Autoworkers Union}, wages at assembly plants in the United
States can reach \$32 an hour.)

The administration
\href{https://ustr.gov/about-us/policy-offices/press-office/press-releases/2019/april/usmca-estimated-support-76000}{has
argued} that those tightened rules would spur substantial increases in
domestic production and create 76,000 jobs in the United States over
five years. While the deal won the endorsement of the A.F.L.-C.I.O.
after the administration made changes to win the support of Democrats,
other labor unions that represent workers in the auto industry have
questioned the validity of the administration's projections.

The International Association of Machinists and Aerospace Workers
\href{https://www.goiam.org/news/imail/machinists-union-opposes-usmca/}{criticized
the deal} on Tuesday for doing too little to address the outsourcing of
jobs, a sentiment echoed by the U.A.W. on Wednesday.

``While the final text of the agreement has not been made available for
review, we already know that U.S.M.C.A. is highly unlikely to bring
factories back from Mexico, as some have promised,'' the U.A.W.
\href{https://uaw.org/statement-uaw-usmca/}{said in a statement}. The
union expressed the hope that the accord would, at the least, ``stop
some of the bleeding.''

\includegraphics{https://static01.nyt.com/images/2019/12/11/business/11autojobs3/merlin_161953035_c0ccedb6-1e03-4868-a1d6-a5bbcaf2fe44-articleLarge.jpg?quality=75\&auto=webp\&disable=upscale}

Ryan Connelly, a senior analyst at DuckerFrontier, a research firm, said
the revised trade pact was unlikely to have a substantial effect on the
sector, either positive or negative. ``At its best, it will probably
prevent some of the losses that you would expect anyway as the industry
gets more efficient and more automated,'' he said.

In March, \href{https://www.stlouisfed.org/authors/brian-reinbold}{Brian
Reinbold} and \href{https://www.stlouisfed.org/authors/yi-wen}{Yi Wen},
of the Federal Reserve Bank of St. Louis, wrote that while the
U.S.M.C.A. wage requirements might benefit workers in the United States,
the rules could also
\href{https://www.stlouisfed.org/publications/regional-economist/first-quarter-2019/changing-trade-relations-auto-exports}{drive
up the cost of cars} produced in the three countries.

``Based on our previous analysis, U.S.M.C.A. is a solution searching for
a problem in regard to auto trade,'' they wrote. ``It also could make
North American automakers less competitive in a global marketplace.''

That could be important for some automakers that export American-made
cars to China and other countries. BMW has
\href{https://www.nytimes.com/2018/10/26/business/china-jobs-automobiles-trade-war.html}{moved
some production of its popular X3 sport utility vehicle} to China from
its plant in Spartanburg, S.C.

Researchers at the International Monetary Fund
\href{https://www.imf.org/~/media/Files/Publications/WP/2019/WPIEA2019073.ashx}{arrived
at a similar conclusion} to the Fed researchers.

In April, the United States International Trade Commission published a
\href{https://www.usitc.gov/publications/332/pub4889.pdf}{nearly
400-page report} on the potential consequences of the proposed deal,
finding its effects would be mixed for the automotive industry.

Rules governing wages, parts and the sourcing of steel and aluminum
would create about 28,000 automotive jobs, it found. But the cost
increases that would come with those rules would lead to 140,000 fewer
vehicles sold.

Some analysts also said that automakers might decide to sidestep the
pact altogether, opting to pay relatively low tariffs on some vehicles
rather than make them in North America under the new rules.

While there's little clear evidence that the new accord will provide the
jobs boost that Mr. Trump has promised, many industry officials
expressed relief that House Democrats and the Trump administration had
reached an agreement, which lays the groundwork for
\href{https://www.nytimes.com/2020/01/16/us/politics/senate-usmca-approval-trump.html}{congressional
ratification of U.S.M.C.A.} in the coming weeks.

Continuity and certainty are vital to the auto industry because many
companies invested billions of dollars over the last 25 years in plants
across the three countries assuming that tariff-free trade in North
America was so firmly entrenched that no government would consider
getting rid of it. Production is so integrated across the region that
cars and car parts can move across Canada, the United States and Mexico
multiple times during the production process.

Of course, NAFTA's durability was called into question by Mr. Trump, who
called it the
\href{https://www.nytimes.com/2019/12/01/us/politics/trump-trade-deal-usmca.htmlhttps://www.nytimes.com/2019/12/01/us/politics/trump-trade-deal-usmca.html}{``worst
trade deal ever made''} and threatened to unilaterally withdraw the
United States from it.

``Nearly a quarter of all new vehicles sold in the U.S. are assembled in
either Mexico or Canada and we've had decades of this supply chain based
on the old NAFTA,'' said Michelle Krebs, an executive analyst at
Autotrader. ``It's made it really important to continue to have some
kind of agreement that makes it easy to cross borders.''

G.M. expressed a similar sentiment in a statement, calling for ``swift
passage'' of the deal and adding that ``the certainty that comes from
having a trade agreement like U.S.M.C.A. in place allows our company to
invest and source with confidence.''

Advertisement

\protect\hyperlink{after-bottom}{Continue reading the main story}

\hypertarget{site-index}{%
\subsection{Site Index}\label{site-index}}

\hypertarget{site-information-navigation}{%
\subsection{Site Information
Navigation}\label{site-information-navigation}}

\begin{itemize}
\tightlist
\item
  \href{https://help.nytimes.com/hc/en-us/articles/115014792127-Copyright-notice}{©~2020~The
  New York Times Company}
\end{itemize}

\begin{itemize}
\tightlist
\item
  \href{https://www.nytco.com/}{NYTCo}
\item
  \href{https://help.nytimes.com/hc/en-us/articles/115015385887-Contact-Us}{Contact
  Us}
\item
  \href{https://www.nytco.com/careers/}{Work with us}
\item
  \href{https://nytmediakit.com/}{Advertise}
\item
  \href{http://www.tbrandstudio.com/}{T Brand Studio}
\item
  \href{https://www.nytimes.com/privacy/cookie-policy\#how-do-i-manage-trackers}{Your
  Ad Choices}
\item
  \href{https://www.nytimes.com/privacy}{Privacy}
\item
  \href{https://help.nytimes.com/hc/en-us/articles/115014893428-Terms-of-service}{Terms
  of Service}
\item
  \href{https://help.nytimes.com/hc/en-us/articles/115014893968-Terms-of-sale}{Terms
  of Sale}
\item
  \href{https://spiderbites.nytimes.com}{Site Map}
\item
  \href{https://help.nytimes.com/hc/en-us}{Help}
\item
  \href{https://www.nytimes.com/subscription?campaignId=37WXW}{Subscriptions}
\end{itemize}
