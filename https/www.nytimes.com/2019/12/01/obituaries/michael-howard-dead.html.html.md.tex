Sections

SEARCH

\protect\hyperlink{site-content}{Skip to
content}\protect\hyperlink{site-index}{Skip to site index}

\href{https://www.nytimes.com/section/obituaries}{Obituaries}

\href{https://myaccount.nytimes.com/auth/login?response_type=cookie\&client_id=vi}{}

\href{https://www.nytimes.com/section/todayspaper}{Today's Paper}

\href{/section/obituaries}{Obituaries}\textbar{}Michael Howard, Eminent
British Military Historian, Dies at 97

\url{https://nyti.ms/2DACOTq}

\begin{itemize}
\item
\item
\item
\item
\item
\end{itemize}

Advertisement

\protect\hyperlink{after-top}{Continue reading the main story}

Supported by

\protect\hyperlink{after-sponsor}{Continue reading the main story}

\hypertarget{michael-howard-eminent-british-military-historian-dies-at-97}{%
\section{Michael Howard, Eminent British Military Historian, Dies at
97}\label{michael-howard-eminent-british-military-historian-dies-at-97}}

A decorated soldier in World War II, he helped reshape the study of war
and was knighted in 1986 for his academic work.

\includegraphics{https://static01.nyt.com/images/2019/12/02/business/02michael-howard-obit/00michael-howard-articleLarge.jpg?quality=75\&auto=webp\&disable=upscale}

By \href{https://www.nytimes.com/by/alan-cowell}{Alan Cowell}

\begin{itemize}
\item
  Dec. 1, 2019
\item
  \begin{itemize}
  \item
  \item
  \item
  \item
  \item
  \end{itemize}
\end{itemize}

Michael Howard, an eminent military historian and decorated combat
veteran who helped redefine the chronicling of conflict between states
and pioneered a so-called ``English school'' of strategic studies, died
on Saturday in Swindon, in southwest England. He had turned 97 a day
earlier.

His death, in a hospital, was confirmed by the historian Max Hastings, a
friend.

By his own account, Mr. Howard grew up in a world of privilege in the
1920s and 1930s, a member of the upper middle class, used to large homes
populated by servants and nannies to tend the children.

His lineage was distinctive, a blend of Quaker traditions from his
father's family and Judaism from his mother --- a German-born debutante
who had been presented at the royal courts of both Berlin and London
long before the surge of virulent anti-Semitism that characterized
Hitler's Third Reich.

His father, though, raised him as a Christian.

``Christianity,'' Mr. Howard wrote in
\href{https://www.bloomsbury.com/us/captain-professor-9780826491251/}{a
memoir published in 2006}, ``seemed an anchor of certainty and
reassurance in a fast-dissolving world. It still does.''

His upper-crust credentials and connections guaranteed him a place in
the officer class of World War II, initially as a second lieutenant. In
1944, he was awarded the Military Cross --- Britain's third-highest
decoration for gallantry (at that time reserved for officers) --- after
leading a bayonet charge against a German machine gun nest.

In his autobiographical writings, he projected an unusually acute sense
of modesty, attributing much of his successes to what he termed
``arbitrary good luck.''

``The Italian gift of fortuna,'' he wrote, ``was to be bestowed on me in
abundant measure.''

Indeed, an article about him in 2017 in a British magazine, The Oldie,
said Mr. Howard ``deploys every trick in the book to denigrate his own
role and talents, particularly in wartime.''

His military background helped burnish his academic credentials in the
study of war that carried him from King's College London to Oxford,
Stanford and Yale. He was knighted in 1986 and served in one of
Britain's most prestigious academic chairs as the Regius Professor of
Modern History, a position awarded to him by Prime Minister Margaret
Thatcher in 1980.

Among historians, Mr. Howard was credited with changing the profile of
military history from an account of specific battles or campaigns to a
broader assessment of the context of those conflicts.

His most significant works included a study, published in 1961, of the
\href{https://www.goodreads.com/book/show/25863.The_Franco_Prussian_War}{Franco-Prussian
War of 1870-71} that sought to illuminate the societal roots of the
opposing armies. He contributed to a major British study of World War
II. In 1977, he was the co-translator, with the American scholar Peter
Paret, of the 19th-century classic ``On War'' by the German military
philosopher and theorist Carl von Clausewitz.

Image

Mr. Howard's study of the Franco-Prussian War sought to examine the
societal roots of the opposing armies.

One of his major works, ``Strategic Deception in the Second World War,''
covering the activities of British intelligence services, was suppressed
on national security grounds by Mrs. Thatcher in 1979. It was published
a decade later.

Alongside his writings, Mr. Howard played a central role in embedding
the study of war in mainstream British intellectual life. He was a
founder of the prestigious International Institute of Strategic Studies
and promoted
\href{https://www.kcl.ac.uk/sspp/departments/warstudies/index.aspx}{war
studies at King's College London}.

``The history of war, I came to realize, was more than the operational
history of armed forces. It was the study of entire societies,'' Mr.
Howard wrote in his memoir in 2006, titled ``Captain Professor: A Life
in War and Peace.'' ``Only by studying their cultures could one come to
understand what it was they fought about and why they fought in the way
they did.''

``I had to learn not only to think about war in a different way, but
also to think about history itself in a different way,'' he added.

Michael Eliot Howard was born in London on Nov. 29, 1922, the youngest
of three brothers. His father, Geoffrey Eliot Howard, ran a family
company manufacturing pharmaceutical and industrial chemicals. His
mother was Edith Julia Emma Edinger, a socialite and, later, art
collector, whose father, Otto Edinger, emigrated to England in 1875 from
Germany. She had received ``the best possible English upper-class
upbringing'' and had met his father through a shared interest in
mountaineering, Mr. Howard wrote.

He became aware of the consequences of his mother's Jewish ancestry only
when her relatives --- ``a sad procession of refugees'' --- began
arriving in England in the 1930s, fleeing Nazi persecution, he wrote.

``It did not strike home to how closely involved I was with their tragic
circumstances until I visited Auschwitz'' while in his 60s, he said in a
lecture to a Jewish society in Oxford in 2008. Photographs of ``smartly
dressed Jewish ladies'' on their way unwittingly to death camps
``literally made my blood run cold,'' he said, because they made him
realize what would have happened to his own mother if Hitler's armies
had overrun Britain in 1940.

In the lecture, though, he made two potentially contentious points.
First, he argued that Holocaust deniers should not be criminalized but
should be ``argued with, discredited and if necessary mocked.''

``Second, I cannot accept the argument that the Holocaust should not be
historicized: that is, that historians should not try to understand,
explain, and put it into context as we do any other historical event,
because to do so would seem to justify it,'' he said.

Mr. Howard was educated at Wellington College and at Oxford University,
where he studied history.

While still at Wellington, he wrote, he ``acquired for one boy after
another a series of crushes'' --- a precursor to his life as a gay man
at a time when homosexuality was either unlawful or vilified. (Only in
1967 did the British Parliament partially decriminalize homosexuality.
And only in the 21st century did same-sex partnerships and marriages
become legal in most of Britain.)

Mr. Howard maintained a relationship for over half a century with Mark
James, a former research assistant and geographer whom he called his
``indispensable and irreplaceable companion.''

In 1942, he joined the Army as a second lieutenant in the Coldstream
Guards and was sent to join the campaign in Italy. In his memoir,
though, he played down his decoration for heroism. He wrote that his
bayonet charge had been ``simply `a fuite en avant''' --- literally, a
flight forward --- and that ``there was nowhere else to go.''

Compared with others who ended the war without a prestigious medal, he
said: ``I feel an enormous fraud. But I have never tried to give it
back.''

With equal self-deprecation, he wrote of an episode further north in
Italy when an infantryman under his command stepped on a land mine.
Calculating that he could not afford to be captured lest he betray
military secrets, Mr. Howard left the soldier on the battlefield. The
infantryman did not survive.

``Years later I sought out his grave, and sat beside it for a long time,
wondering what else I could have done,'' Mr. Howard wrote in his memoir.
``I still wonder. I only know that I should never have abandoned him as
I did.''

In the postwar years, Mr. Howard returned to Oxford to complete his
studies and, in the early 1950s, helped found the Department of War
Studies at King's College. He also played a key role at think tanks such
as Chatham House and the International Institute of Strategic Studies,
which he founded in 1958.

At that time, he wrote, ``nuclear weapons, and the threat of total
annihilation that they brought with them demanded a total
reconsideration of the nature of war and peace; how to wage the first
and keep the second.''

In 1960, he traveled widely in the United States on a grant from the
Ford Foundation.

``The first revelation was that everything in the U.S.A. was not just
bigger, but usually better than at home.''

A subsequent visit to America produced what he termed a more somber
impression: ``how little the British now mattered'' in the nuclear
face-off between Moscow and Washington. He later taught for six months
as a visiting professor at Stanford.

Back in Britain, Mr. Howard joined an informal group of experts advising
government figures, including Mrs. Thatcher, on defense matters, often
meeting at Chequers, the country retreat of British prime ministers.
``She was not easy company, lacking as she was in any sense of humor and
increasingly impervious to new ideas,'' he recalled.

In 1968, he moved to Oxford as a fellow of All Souls College, an elite
center of postgraduate studies, and later held the Regius chair ---
which he called ``the flagship of the historical profession'' --- at
Oriel College, Oxford.

Information on Mr. Howard's survivors was not immediately available.

After two decades lecturing and writing in Oxford, he moved to Yale in
1989 to take up the Robert A. Lovett Chair of Military and Naval History
until his retirement from the post in 1993.

Arriving in America at the beginning of his tenure, he wrote in his
memoirs, ``I suddenly felt a rush of sheer happiness as, at the age of
67, I sloughed off one outworn life and looked forward to starting
another.''

Advertisement

\protect\hyperlink{after-bottom}{Continue reading the main story}

\hypertarget{site-index}{%
\subsection{Site Index}\label{site-index}}

\hypertarget{site-information-navigation}{%
\subsection{Site Information
Navigation}\label{site-information-navigation}}

\begin{itemize}
\tightlist
\item
  \href{https://help.nytimes.com/hc/en-us/articles/115014792127-Copyright-notice}{©~2020~The
  New York Times Company}
\end{itemize}

\begin{itemize}
\tightlist
\item
  \href{https://www.nytco.com/}{NYTCo}
\item
  \href{https://help.nytimes.com/hc/en-us/articles/115015385887-Contact-Us}{Contact
  Us}
\item
  \href{https://www.nytco.com/careers/}{Work with us}
\item
  \href{https://nytmediakit.com/}{Advertise}
\item
  \href{http://www.tbrandstudio.com/}{T Brand Studio}
\item
  \href{https://www.nytimes.com/privacy/cookie-policy\#how-do-i-manage-trackers}{Your
  Ad Choices}
\item
  \href{https://www.nytimes.com/privacy}{Privacy}
\item
  \href{https://help.nytimes.com/hc/en-us/articles/115014893428-Terms-of-service}{Terms
  of Service}
\item
  \href{https://help.nytimes.com/hc/en-us/articles/115014893968-Terms-of-sale}{Terms
  of Sale}
\item
  \href{https://spiderbites.nytimes.com}{Site Map}
\item
  \href{https://help.nytimes.com/hc/en-us}{Help}
\item
  \href{https://www.nytimes.com/subscription?campaignId=37WXW}{Subscriptions}
\end{itemize}
