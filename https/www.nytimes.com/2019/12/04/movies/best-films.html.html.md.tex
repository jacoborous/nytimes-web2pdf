Sections

SEARCH

\protect\hyperlink{site-content}{Skip to
content}\protect\hyperlink{site-index}{Skip to site index}

\href{https://www.nytimes.com/section/movies}{Movies}

\href{https://myaccount.nytimes.com/auth/login?response_type=cookie\&client_id=vi}{}

\href{https://www.nytimes.com/section/todayspaper}{Today's Paper}

\href{/section/movies}{Movies}\textbar{}Best Movies of 2019

\href{https://nyti.ms/2RiTkiX}{https://nyti.ms/2RiTkiX}

\begin{itemize}
\item
\item
\item
\item
\item
\item
\end{itemize}

Advertisement

\protect\hyperlink{after-top}{Continue reading the main story}

Supported by

\protect\hyperlink{after-sponsor}{Continue reading the main story}

\hypertarget{best-movies-of-2019}{%
\section{Best Movies of 2019}\label{best-movies-of-2019}}

These titles prove that while the streaming arguments rage and the
medium may be a mess, the art form is as healthy as ever.

\includegraphics{https://static01.nyt.com/images/2019/12/08/arts/08movies-year-end-lead/08movies-year-end-articleLarge.jpg?quality=75\&auto=webp\&disable=upscale}

By \href{https://www.nytimes.com/by/a-o--scott}{A.O. Scott} and
\href{https://www.nytimes.com/by/manohla-dargis}{Manohla Dargis}

\begin{itemize}
\item
  Published Dec. 4, 2019Updated Dec. 26, 2019
\item
  \begin{itemize}
  \item
  \item
  \item
  \item
  \item
  \item
  \end{itemize}
\end{itemize}

\href{https://cn.nytimes.com/culture/20191205/best-films/}{阅读简体中文版}\href{https://cn.nytimes.com/culture/20191205/best-films/zh-hant/}{閱讀繁體中文版}

\hypertarget{ao-scotts-list--manohla-dargiss-list}{%
\subsubsection{\texorpdfstring{\protect\hyperlink{link-5eb84891}{A.O.
Scott's List} \textbar{} \protect\hyperlink{link-60ed588b}{Manohla
Dargis's
List}}{A.O. Scott's List \textbar{} Manohla Dargis's List}}\label{ao-scotts-list--manohla-dargiss-list}}

\begin{center}\rule{0.5\linewidth}{\linethickness}\end{center}

A.O. Scott

\hypertarget{films-worth-arguing-about}{%
\subsection{Films Worth Arguing About}\label{films-worth-arguing-about}}

As the movie year winds down, I would like to express my gratitude to
Martin Scorsese. Not only for making
\href{https://www.nytimes.com/2019/09/27/movies/the-irishman-review.html}{``The
Irishman,''} his best movie in a long time and one of the best of 2019
(see below), but also for reminding the world of the value of cinema.

The art form is in one of its periodic identity crises. A big chunk of
our collective attention --- we don't yet know how big, or with what
consequences --- is migrating to streaming platforms whose offerings
include a lot of the stand-alone single-episode narratives that we used
to see mainly in theaters. (Yes, I know: We saw a lot of sequels, too.)
Movie theaters, meanwhile, are dominated by franchise, I.P.-driven
spectacles like the entities in Disney's Marvel Cinematic Universe,
which Scorsese singled out, in an interview in Empire magazine and then
in
\href{https://www.nytimes.com/2019/11/04/opinion/martin-scorsese-marvel.html}{a
New York Times Op-Ed}, as ``not cinema.''

The dust-up that followed his remarks was predictable. Members of the
aggrieved superhero-loving community --- some of whom draw Disney
paychecks --- tut-tutted Scorsese for being old, out of touch, overrated
and, most of all, elitist. Accusing Scorsese (and his defenders) of
elitism was exemplary pseudo-populism, a defense of corporate hegemony
disguised as a celebration of mass taste. To question the apparent
preferences of millions of consumers is to risk being labeled a snob.

In the imaginations of their sore-winner, alpha dog-underdog opponents,
the snobs are simultaneously too dangerous to ignore and too enfeebled
to take seriously. The response is basically, \emph{Shut up! Shut up!
Shut up! Nobody's listening to you anyway!} And the anti-elitist
argument is at bottom a matter of numbers, of quantity trumping quality.
That ``Avengers: Endgame'' and ``Joker'' broke records at the global box
office surely means something, even if the movies themselves don't.

But to paraphrase
\href{https://www.youtube.com/watch?v=k5fJmkv02is}{Justin Timberlake's
character in ``The Social Network''}: a billion dollars isn't cool. You
know what's cool? Movies that offer something more than the sullen
pseudo-politics of ``Joker'' or the elaborate pro-status-quo theatrics
of ``Avengers.'' Movies that, rather than fetishizing self-pity or
sentimentalizing domination, illuminate the cruelty, the comedy and the
grace of the human condition. Movies that treat you as something other
than a passive spectator or an obedient, presold ``fan.'' Movies that
are actually worth arguing about, and thinking about.

Which is more or less what Scorsese meant by ``cinema.'' The word might
make even some of his sympathizers a little uncomfortable. Because it
also exists in other languages, including French, using it might make
you sound like you're putting on airs. (I myself prefer the Italian
pronunciation.) But far from signifying snootiness, the cosmopolitanism
of the term is a sign of the essentially democratic nature of the art
form itself, which is able to leap over barriers of language, custom and
ideology like few others.

(\emph{Looking for the best movies to watch on Netflix, Disney Plus and
other streaming services?}
\href{https://www.nytimes.com/newsletters/watching}{\emph{Subscribe to
our twice-weekly newsletter}}.)

Cinema also migrates across platforms, which is another reason to
embrace the old/new name. In the digital age, ``film'' is a
technological misnomer, attached to the glories of a specific,
no-longer-dominant (though not entirely obsolete) way of making and
projecting pictures. ``Movies'' are, mostly, what we see in theaters (or
cinemas, just to confuse the issue further), while ``moving pictures''
pop up on nearly every surface, distracting us from our distraction.

``Cinema'' is more capacious and also more specific, because it refers
to an aesthetic rather than a technological category. The medium, right
now, is a mess. But the art form is in a state of rude, contentious
health. Looking back on my favorites released in the United States since
January, I'm struck by how many bristle with an argumentative energy
that seems to match the times, even if a lot of the filmmakers cast
their glances back toward earlier modern moments.

Bong Joon Ho's ``Parasite'' and Noah Baumbach's ``Marriage Story''
unfold in a restless present tense, but so does Greta Gerwig's ``Little
Women,'' even though it takes place more than 100 years ago. ``The
Irishman'' and ``Once Upon a Time \ldots{} in Hollywood'' feel like
elegies to an older cinematic ethic, while ``Atlantics'' and ``The Edge
of Democracy'' press into an uncertain future, the terms of which are
prophesied by the blood and rhetoric of Mike Leigh's mighty
``Peterloo.'' The top two entries on my list do all of that and more,
digging so deep into the particular lives of their characters --- a
Macedonian beekeeper and a film student in London --- that they seem to
transcend time altogether.

There's more. There's always more! As long as we trust our eyes and know
where to look.

\includegraphics{https://static01.nyt.com/images/2019/12/08/arts/08movies-year-end-honey/merlin_158307867_4f284605-7257-4aab-827d-39a1b03090ff-articleLarge.jpg?quality=75\&auto=webp\&disable=upscale}

\hypertarget{1-honeyland-tamara-kotevska-and-ljubomir-stefanov}{%
\subsection{\texorpdfstring{1.
\href{https://www.nytimes.com/2019/07/25/movies/honeyland-review.html}{`Honeyland'}
(Tamara Kotevska and Ljubomir
Stefanov)}{1. `Honeyland' (Tamara Kotevska and Ljubomir Stefanov)}}\label{1-honeyland-tamara-kotevska-and-ljubomir-stefanov}}

Conceived as a government-sponsored informational video, this
documentary is nothing less than a found epic, a real-life environmental
allegory and, not least, a stinging comedy about the age-old problem of
inconsiderate neighbors.

\hypertarget{2-the-souvenir-joanna-hogg}{%
\subsection{\texorpdfstring{2.
\href{https://www.nytimes.com/2019/05/16/movies/the-souvenir-review.html}{`The
Souvenir'} (Joanna
Hogg)}{2. `The Souvenir' (Joanna Hogg)}}\label{2-the-souvenir-joanna-hogg}}

Honor Swinton Byrne plays a diffident version of the director's younger
self in an elusive autobiographical film that also functions as a kind
of superhero origin story.

\hypertarget{3-parasite-bong-joon-ho}{%
\subsection{\texorpdfstring{3.
\href{https://www.nytimes.com/2019/10/10/movies/parasite-review.html}{`Parasite'}
(\href{https://www.nytimes.com/2019/10/30/movies/bong-joon-ho-parasite.html}{Bong
Joon
Ho})}{3. `Parasite' (Bong Joon Ho)}}\label{3-parasite-bong-joon-ho}}

I can't think of a film that made me sadder about the state of the world
and more jubilant about the state of movies.

\hypertarget{4-the-irishman-martin-scorsese}{%
\subsection{\texorpdfstring{4.
\href{https://www.nytimes.com/2019/09/27/movies/the-irishman-review.html}{`The
Irishman'} (Martin
Scorsese)}{4. `The Irishman' (Martin Scorsese)}}\label{4-the-irishman-martin-scorsese}}

What is cinema? If you have three and a half hours to spare --- and you
do --- this is a pretty good answer.

\hypertarget{5-marriage-story-noah-baumbach}{%
\subsection{\texorpdfstring{5.
\href{https://www.nytimes.com/2019/11/05/movies/marriage-story-review.html}{`Marriage
Story'}
(\href{https://www.nytimes.com/2019/11/27/movies/marriage-story-noah-baumbach.html}{Noah
Baumbach})}{5. `Marriage Story' (Noah Baumbach)}}\label{5-marriage-story-noah-baumbach}}

The joys and miseries of a creative family in 21st-century New York and
Los Angeles.

\hypertarget{6-little-women-greta-gerwig}{%
\subsection{\texorpdfstring{6.
\href{https://www.nytimes.com/2019/12/23/movies/little-women-review.html}{`Little
Women'} (Greta
Gerwig)}{6. `Little Women' (Greta Gerwig)}}\label{6-little-women-greta-gerwig}}

The joys and miseries of a creative family in 19th-century
Massachusetts.

\hypertarget{7-peterloo-mike-leigh}{%
\subsection{\texorpdfstring{7.
\href{https://www.nytimes.com/2019/04/04/movies/peterloo-review.html}{`Peterloo'}
(Mike Leigh)}{7. `Peterloo' (Mike Leigh)}}\label{7-peterloo-mike-leigh}}

British politics in 1819, full of passion and pageantry, bad faith and
factionalism. It feels like a very short march from then to now.

\hypertarget{8-the-edge-of-democracy-petra-costa}{%
\subsection{\texorpdfstring{8.
\href{https://www.nytimes.com/2019/06/18/movies/edge-of-democracy-review.html}{`The
Edge of Democracy'} (Petra
Costa)}{8. `The Edge of Democracy' (Petra Costa)}}\label{8-the-edge-of-democracy-petra-costa}}

This harrowing documentary, a thoughtful inside look at the events
leading up to the election of Jair Bolsonaro, Brazil's populist
president, is the scariest movie of the year.

\hypertarget{9-once-upon-a-time--in-hollywood-quentin-tarantino}{%
\subsection{\texorpdfstring{9.
\href{https://www.nytimes.com/2019/07/24/movies/once-upon-a-time-in-hollywood-review.html}{`Once
Upon a Time \ldots{} in Hollywood'} (Quentin
Tarantino)}{9. `Once Upon a Time \ldots{} in Hollywood' (Quentin Tarantino)}}\label{9-once-upon-a-time--in-hollywood-quentin-tarantino}}

Another answer to the ``what is cinema?'' question, with special
attention to Brad Pitt's jawline and Margot Robbie's feet.

\hypertarget{10-atlantics-mati-diop}{%
\subsection{\texorpdfstring{10.
\href{https://www.nytimes.com/2019/11/14/movies/atlantics-review.html}{`Atlantics'}
(Mati
Diop)}{10. `Atlantics' (Mati Diop)}}\label{10-atlantics-mati-diop}}

See No. 3. A startlingly original debut feature about the specters that
haunt Dakar, and everywhere else.

And \ldots{}
``\href{https://www.nytimes.com/2019/08/20/movies/american-factory-review.html}{American
Factory,''}
\href{https://www.nytimes.com/2019/03/13/movies/ash-is-purest-white-review.html}{``Ash
Is Purest White''}
\href{https://www.nytimes.com/2019/02/12/movies/birds-of-passage-review.html}{``Birds
of Passage,''}
\href{https://www.nytimes.com/2019/05/22/movies/booksmart-review.html}{``Booksmart,''}
\href{https://www.nytimes.com/2019/06/25/movies/the-chambermaid-review.html}{``The
Chambermaid,''}
\href{https://www.nytimes.com/2019/03/06/movies/an-elephant-sitting-still-review.html}{``An
Elephant Standing Still,''}
``\href{https://www.nytimes.com/2019/11/14/movies/ford-v-ferrari-review.html}{Ford
v Ferrari},''
``\href{https://www.nytimes.com/2019/07/18/movies/i-do-not-care-we-go-down-history-barbarians-review.html}{I
Do Not Care if We Go Down in History as Barbarians},''
``\href{https://www.nytimes.com/2019/03/07/movies/gloria-bell-review.html}{Gloria
Bell},''
``\href{https://www.nytimes.com/2019/04/10/movies/her-smell-review.html}{Her
Smell},''
``\href{https://www.nytimes.com/2019/02/07/movies/high-flying-bird-review.html}{High-Flying
Bird},''
``\href{https://www.nytimes.com/2019/08/01/movies/nightingale-review.html}{The
Nightingale},''
``\href{https://www.nytimes.com/2019/10/03/movies/pain-and-glory-review.html}{Pain
and Glory},'' ``Richard Jewell,''
``\href{https://www.nytimes.com/2019/02/28/movies/transit-review.html}{Transit},''
``\href{https://www.nytimes.com/2019/03/20/movies/us-movie-review.html}{Us}.''

\begin{center}\rule{0.5\linewidth}{\linethickness}\end{center}

Manohla Dargis

\hypertarget{cinema-in-the-age-of-conglomerates}{%
\subsection{Cinema in the Age of
Conglomerates}\label{cinema-in-the-age-of-conglomerates}}

Seen any good or great movies lately? If you are a film critic making a
Top 10 list of the year's best, your annual agony is never that there
are not enough choices --- just the opposite. About 800 new movies will
have opened in New York by the end of the year, which is 11 percent
fewer than were released
\href{https://www.nytimes.com/2014/01/12/movies/flooding-theaters-isnt-good-for-filmmakers-or-filmgoers.html}{a
couple of years ago}. The changes in how movies are now distributed are
having a pronounced impact on theatrical exhibition, which may be a
disaster or a welcome course correction in a glutted market.

The better movies generally open in theaters, just as they have long
done. In the past, a lot of junky titles would have gone straight to
video; these days a lot go straight to streaming, while many others
quickly open and close in theaters before they too flow into streaming
purgatory. Despite this online maw, movies still play in theaters
because, well, people still like the big screen. Some play solely to
qualify for awards or because filmmakers also like the big screen.
Amazon and Netflix open movies in theaters because they see those same
filmmakers and awards as a way to keep, and attract, subscribers.

Cinema has always been a moving target, from the cinematograph era to
the streaming. That's one reason the debate that raged over
\href{https://www.nytimes.com/2019/11/04/opinion/martin-scorsese-marvel.html}{Martin
Scorsese's comments about Marvel movies} not being cinema feels like a
dead end. He is right that nothing is at risk in them, or rather very
little. Even the best ones are absent real risk because they are not
films in the old-fashioned sense: They are delivery systems for an
integrated array of products and experiences (other movies, theme parks,
toys). Their formula is a feature not a bug. The appeal of the familiar
is one way powerful entertainment companies turn ardent viewers into
brand loyalists, reaching fans with a cradle-to-grave consumer strategy.

History will remember this period for Disney's monopolistic muscle; it
will also remember Scorsese's films. It seems unlikely, though, that
history will remember many of the movies Disney now makes. This probably
matters little to the media giant, which has had a busy, record-breaking
year. In March, it finalized its purchase of 21st Century Fox,
effectively destroying a Hollywood pillar. The origins of Fox can be
traced back to around 1904, when William Fox bought a share of a
Brooklyn nickelodeon.
\href{https://www.nytimes.com/2019/03/20/business/media/walt-disney-21st-century-fox-deal.html}{Disney
picked up the empire that rose from that humble beginning for \$71.3
billion} and will absorb it for the express purpose of
\href{https://www.cnbc.com/2019/04/12/disney-wouldnt-have-bought-fox-assets-without-streaming-plans-iger-says.html}{leveraging
Fox assets} to become a global streaming behemoth, just like Netflix.

The
\href{https://variety.com/2019/film/news/fox-history-moments-disney-merger-1203165929/}{end
of Fox} feels like another rattle in the slow death of what many still
call the studio system, which hasn't resembled the factorylike days of
the old MGM for a long time. You can mourn the end of the studios and
revere their legacy --- the art, craft and technique --- but there's no
mourning their racism, sexism, cultivated stupidity and contempt for
art. The old Hollywood studios perfected a way of making films and hired
artists and artisans who succeeded within those confines or transcended
them (or failed or fled). Like the young Scorsese and his friends, the
Cahiers du Cinéma critics who became directors, championed those films.
André Bazin honored ``the genius of the system.''

I tend to think that Hollywood reached its zenith before 1960. Many of
the greatest American films made in the decades since were produced in
spite of terrible studio ideas, more by accident than design, or were
made in the independent realm (at
\href{https://filmmakermagazine.com/107829-end-of-the-road-jim-jarmusch-on-his-johnny-depp-starring-western-death-trip-dead-man/}{times
with European or Asian money}) or while the studios were having a fling
with adventure. One such moment was in the 1970s; another occurred
recently when Miramax shook up the indie world and the studios noticed.
Their interest was fleeting but it's worth recalling that Disney
released Wes Anderson's ``Rushmore,'' Paramount backed Paul Thomas
Anderson's ``There Will Be Blood'' and Warner Bros. put out Richard
Linklater's ``Before Sunset.''

It is also worth remembering that both Spike Lee and Kathryn Bigelow,
two of the filmmakers Scorsese holds up as exemplars of cinema, have
nurtured careers beyond Hollywood and sometimes despite it. After Lee
made his breakout film, ``She's Gotta Have It,'' he worked with the
major studios but he also battled them to protect his vision and
integrity. Bigelow has never directed movies that were financed by a
major studio, though some have released her features. Scorsese's recent
movies have been made, as he recently pointed out, without studio help.
``In the last 10 years,''
\href{https://deadline.com/2019/10/martin-scorsese-netflix-irishman-women-characters-marvel-movies-theme-park-rome-festival-1202765130/}{he
said}, ``my films have been independently financed under difficult
circumstances.''

That is a sobering description of American mainstream movies in the age
of global media conglomerates. Yet, as our yearly lists of favorites
attest, great work always happens.

Image

Antonio Banderas in ``Pain and Glory.''Credit...Manolo Pavón/Sony
Pictures Classics

\hypertarget{1-pain-and-glory-pedro-almoduxf3var}{%
\subsection{\texorpdfstring{1.
`\href{https://www.nytimes.com/2019/10/03/movies/pain-and-glory-review.html}{Pain
and Glory}' (Pedro
Almodóvar)}{1. `Pain and Glory' (Pedro Almodóvar)}}\label{1-pain-and-glory-pedro-almoduxf3var}}

In this wistful, deeply felt masterwork, a filmmaker faces his own
mortality, awakens desire and transforms ragged life into art.

\hypertarget{2-the-irishman-martin-scorsese}{%
\subsection{\texorpdfstring{2.
`\href{https://www.nytimes.com/2019/09/27/movies/the-irishman-review.html}{The
Irishman}' (Martin
Scorsese)}{2. `The Irishman' (Martin Scorsese)}}\label{2-the-irishman-martin-scorsese}}

One of the finest movies of Scorsese's career, this haunting epic about
a murderer for the mob is about tribal loyalty, male violence and a grim
vision of homegrown fascism.

\hypertarget{3-parasite-bong-joon-ho-1}{%
\subsection{\texorpdfstring{3.
`\href{https://www.nytimes.com/2019/10/10/movies/parasite-review.html}{Parasite}'
(Bong Joon
Ho)}{3. `Parasite' (Bong Joon Ho)}}\label{3-parasite-bong-joon-ho-1}}

A perfectly directed movie from one of the greatest filmmakers working
today. If you want to know what cinema is, watch this --- well, just
watch everything on this list.

\hypertarget{4-little-women-greta-gerwig}{%
\subsection{\texorpdfstring{4.
`\href{https://www.nytimes.com/2019/12/23/movies/little-women-review.html}{Little
Women}' (Greta
Gerwig)}{4. `Little Women' (Greta Gerwig)}}\label{4-little-women-greta-gerwig}}

At once faithful and blissfully liberated, this beautiful interpretation
of Louisa May Alcott's novel is the story of a woman finding her voice,
directed by one who already has.

\hypertarget{5-once-upon-a-time--in-hollywood-quentin-tarantino}{%
\subsection{\texorpdfstring{5.
`\href{https://www.nytimes.com/2019/05/22/movies/dicaprio-pitt.html?module=inline}{Once
Upon a Time \ldots{} in Hollywood}' (Quentin
Tarantino)}{5. `Once Upon a Time \ldots{} in Hollywood' (Quentin Tarantino)}}\label{5-once-upon-a-time--in-hollywood-quentin-tarantino}}

Tarantino's nostalgic panegyric to Los Angeles, the internal combustion
engine and old-school masculine cool is a dream of a movie. I could
spend hours watching Margot Robbie's character watch herself in a film
and Brad Pitt's cruise the magically smog-free city in a buttery yellow
1966 Cadillac.

\hypertarget{6-synonyms-nadav-lapid}{%
\subsection{\texorpdfstring{6.
`\href{https://www.nytimes.com/2019/10/24/movies/synonyms-review.html}{Synonyms}'
(Nadav
Lapid)}{6. `Synonyms' (Nadav Lapid)}}\label{6-synonyms-nadav-lapid}}

In this corrosive, funny and sometimes shocking existential cry, a young
ex-soldier flees Israel and tries to shed his country and his identity
by turning himself into a Frenchman.

\hypertarget{7-transit-christian-petzold}{%
\subsection{\texorpdfstring{7.
`\href{https://www.nytimes.com/2019/02/28/movies/transit-review.html}{Transit}'
(Christian
Petzold)}{7. `Transit' (Christian Petzold)}}\label{7-transit-christian-petzold}}

A brilliant allegory that imagines a world in the grip of fascism and
that --- as throngs of desperate people seek asylum --- becomes a
frightening, all-too-real vision of our own world.

\hypertarget{8-american-factory-julia-reichert-and-steven-bognar}{%
\subsection{\texorpdfstring{8.
`\href{https://www.nytimes.com/2019/08/20/movies/american-factory-review.html}{American
Factory}' (Julia Reichert and Steven
Bognar)}{8. `American Factory' (Julia Reichert and Steven Bognar)}}\label{8-american-factory-julia-reichert-and-steven-bognar}}

This powerful documentary tracks what happens when a Chinese company
takes over a shuttered Ohio General Motors factory. Everyone loses but
the billionaire owner.

\hypertarget{9-one-child-nation-nanfu-wang-and-jialing-zhang}{%
\subsection{\texorpdfstring{9.
`\href{https://www.nytimes.com/2019/08/08/movies/one-child-nation-review.html}{One
Child Nation}' (Nanfu Wang and Jialing
Zhang)}{9. `One Child Nation' (Nanfu Wang and Jialing Zhang)}}\label{9-one-child-nation-nanfu-wang-and-jialing-zhang}}

Both site specific yet transcending borders, this devastating
documentary is a damning look at how China's propaganda controls both
minds and bodies.

\hypertarget{10-the-last-black-man-in-san-francisco-joe-talbot}{%
\subsection{\texorpdfstring{10.
`\href{https://www.nytimes.com/2019/06/06/movies/the-last-black-man-in-san-francisco-review.html}{The
Last Black Man in San Francisco}' (Joe
Talbot)}{10. `The Last Black Man in San Francisco' (Joe Talbot)}}\label{10-the-last-black-man-in-san-francisco-joe-talbot}}

A heartfelt, often supremely lovely movie about loss, memory, race and
place that Talbot created with his longtime friend, Jimmie Fails, who
also stars.

And \ldots{}
``\href{https://www.nytimes.com/2019/03/07/movies/3-faces-review.html}{3
Faces}'';
``\href{https://www.nytimes.com/2019/09/19/movies/ad-astra-review-brad-pitt.html}{Ad
Astra}'' (Brad Pitt!);
``\href{https://www.nytimes.com/2019/02/27/movies/apollo-11-review.html}{Apollo
11}'';
``\href{https://www.nytimes.com/2019/11/14/movies/atlantics-review.html}{Atlantics}'';
``\href{https://www.nytimes.com/2019/03/13/movies/ash-is-purest-white-review.html}{Ash
Is Purest White}'';
``\href{https://www.nytimes.com/2019/11/21/movies/a-beautiful-day-in-the-neighborhood-review.html}{A
Beautiful Day in the Neighborhood}'' (sniff, sniff);
``\href{https://www.nytimes.com/2019/03/27/movies/the-brink-review.html}{The
Brink}'';
``\href{https://www.nytimes.com/2019/10/17/movies/the-cave-review.html}{The
Cave}''
``\href{https://www.nytimes.com/2019/05/09/movies/charlie-says-review.html}{Charlie
Says}'' (Mary Harron directed this year's other must-see movie tied to
the Manson murders); ``Clemency'' (Alfre Woodard!); ``The Disappearance
of My Mother'';
``\href{https://www.nytimes.com/2019/10/02/movies/dolemite-is-my-name-review.html}{Dolemite
Is My Name}'' (for the cast, especially Wesley Snipes);
``\href{https://www.nytimes.com/2019/11/14/movies/ford-v-ferrari-review.html}{Ford
v Ferrari}'' (Matt Damon's Tommy Lee Jones voice deserves its own
credit);
``\href{https://www.nytimes.com/2019/08/22/movies/give-me-liberty-review.html}{Give
Me Liberty}'' (my No. 11);
``\href{https://www.nytimes.com/2019/03/07/movies/gloria-bell-review.html}{Gloria
Bell}'' (hail Julianne Moore!);
``\href{https://www.nytimes.com/2019/04/16/movies/hail-satan-review.html}{Hail
Satan?}'' (a great double bill with ``The Brink''); ``Invisible Life'';
``\href{https://www.nytimes.com/2019/07/25/movies/honeyland-review.html}{Honeyland}'';
``\href{https://www.nytimes.com/2019/04/30/movies/knock-down-the-house-review.html}{Knock
Down the House}'';
``\href{https://www.nytimes.com/2019/06/05/movies/late-night-review.html}{Late
Night}'';
``\href{https://www.nytimes.com/2019/06/06/movies/leto-review.html}{Leto}'';
``\href{https://www.nytimes.com/2019/11/05/movies/marriage-story-review.html}{Marriage
Story}''; ``\href{https://midnightfamilyfilm.com/}{Midnight Family}'';
``\href{https://www.nytimes.com/2019/05/09/movies/pasolini-review.html}{Pasolini}'';
``\href{https://www.nytimes.com/2019/04/04/movies/peterloo-review.html}{Peterloo}'';
``Richard Jewell'' (minus the risible Olivia Wilde journalist);
``\href{https://www.nytimes.com/2019/06/11/movies/rolling-thunder-bob-dylan-martin-scorsese-review.html}{Rolling
Thunder Revue}: A Bob Dylan Story by Martin Scorsese'' (No. 12);
``\href{https://www.nytimes.com/2019/05/16/movies/the-souvenir-review.html}{The
Souvenir}''; ``Uncut Gems'';
``\href{https://www.nytimes.com/2019/03/20/movies/us-movie-review.html}{Us}'';
``\href{https://www.nytimes.com/2019/11/21/movies/varda-by-agnes-review.html}{Varda
by Agnès}'' (adieu);
``\href{https://www.nytimes.com/2019/11/14/movies/waves-review.html}{Waves}.''
(No. 13).

Advertisement

\protect\hyperlink{after-bottom}{Continue reading the main story}

\hypertarget{site-index}{%
\subsection{Site Index}\label{site-index}}

\hypertarget{site-information-navigation}{%
\subsection{Site Information
Navigation}\label{site-information-navigation}}

\begin{itemize}
\tightlist
\item
  \href{https://help.nytimes.com/hc/en-us/articles/115014792127-Copyright-notice}{©~2020~The
  New York Times Company}
\end{itemize}

\begin{itemize}
\tightlist
\item
  \href{https://www.nytco.com/}{NYTCo}
\item
  \href{https://help.nytimes.com/hc/en-us/articles/115015385887-Contact-Us}{Contact
  Us}
\item
  \href{https://www.nytco.com/careers/}{Work with us}
\item
  \href{https://nytmediakit.com/}{Advertise}
\item
  \href{http://www.tbrandstudio.com/}{T Brand Studio}
\item
  \href{https://www.nytimes.com/privacy/cookie-policy\#how-do-i-manage-trackers}{Your
  Ad Choices}
\item
  \href{https://www.nytimes.com/privacy}{Privacy}
\item
  \href{https://help.nytimes.com/hc/en-us/articles/115014893428-Terms-of-service}{Terms
  of Service}
\item
  \href{https://help.nytimes.com/hc/en-us/articles/115014893968-Terms-of-sale}{Terms
  of Sale}
\item
  \href{https://spiderbites.nytimes.com}{Site Map}
\item
  \href{https://help.nytimes.com/hc/en-us}{Help}
\item
  \href{https://www.nytimes.com/subscription?campaignId=37WXW}{Subscriptions}
\end{itemize}
