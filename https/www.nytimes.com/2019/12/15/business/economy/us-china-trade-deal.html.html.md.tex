Sections

SEARCH

\protect\hyperlink{site-content}{Skip to
content}\protect\hyperlink{site-index}{Skip to site index}

\href{https://www.nytimes.com/section/business/economy}{Economy}

\href{https://myaccount.nytimes.com/auth/login?response_type=cookie\&client_id=vi}{}

\href{https://www.nytimes.com/section/todayspaper}{Today's Paper}

\href{/section/business/economy}{Economy}\textbar{}Trump Officials
Praise Gains From China Deal, but They Come at a Cost

\url{https://nyti.ms/36EcqUZ}

\begin{itemize}
\item
\item
\item
\item
\item
\end{itemize}

Advertisement

\protect\hyperlink{after-top}{Continue reading the main story}

Supported by

\protect\hyperlink{after-sponsor}{Continue reading the main story}

\hypertarget{trump-officials-praise-gains-from-china-deal-but-they-come-at-a-cost}{%
\section{Trump Officials Praise Gains From China Deal, but They Come at
a
Cost}\label{trump-officials-praise-gains-from-china-deal-but-they-come-at-a-cost}}

The economic losses sustained during a bruising 19-month trade war will
not be easy to make up.

\includegraphics{https://static01.nyt.com/images/2019/12/16/business/15jpdc-chinatrade-print/merlin_162552654_f2e464b8-ba02-4e3b-8e47-1a2c5f800eb1-articleLarge.jpg?quality=75\&auto=webp\&disable=upscale}

\href{https://www.nytimes.com/by/ana-swanson}{\includegraphics{https://static01.nyt.com/images/2018/12/10/multimedia/author-ana-swanson/author-ana-swanson-thumbLarge.png}}\href{https://www.nytimes.com/by/keith-bradsher}{\includegraphics{https://static01.nyt.com/images/2018/10/08/multimedia/author-keith-bradsher/author-keith-bradsher-thumbLarge.png}}

By \href{https://www.nytimes.com/by/ana-swanson}{Ana Swanson} and
\href{https://www.nytimes.com/by/keith-bradsher}{Keith Bradsher}

\begin{itemize}
\item
  Dec. 15, 2019
\item
  \begin{itemize}
  \item
  \item
  \item
  \item
  \item
  \end{itemize}
\end{itemize}

\href{https://cn.nytimes.com/business/20191216/us-china-trade-deal/}{阅读简体中文版}\href{https://cn.nytimes.com/business/20191216/us-china-trade-deal/zh-hant/}{閱讀繁體中文版}

WASHINGTON --- Trump administration officials predicted big gains for
the economy from
\href{https://www.nytimes.com/2019/12/13/business/economy/china-trade-deal.html}{a
newly announced trade deal with China}, but the economic losses
sustained during a bruising 19-month trade war will not be easy to make
up.

In a televised interview on Sunday, President Trump's top trade
negotiator praised the progress that the agreement between the world's
two biggest economies would make on issues like intellectual property,
currency and financial services. He described the deal as ``remarkable''
and predicted that it would roughly double American exports to China by
2021.

Yet the negotiator, Robert Lighthizer, admitted that the limited
agreement, which the administration says is just the first of several
phases, was only a partial victory. He said it would leave many of the
existing tariffs between the countries in place and other bigger changes
to the Chinese economy undone.

``This is a first step in trying to integrate two very different
systems, to the benefit of both of us,'' Mr. Lighthizer, the United
States trade representative, said in an interview on CBS's ``Face the
Nation.'' Anyone who thinks you would change China in one stroke of the
pen ``is foolish,'' he said, adding: ``The president is not foolish. He
is very smart.''

Business groups have welcomed the first-phase trade pact as a sign of
easing tensions in the trade war. On Sunday, Mr. Lighthizer predicted
that Chinese purchases of American products would rise by more than
\$100 billion a year once the agreement, which is expected to be signed
in January, goes into effect.

But the economic benefits of the pact appear to have come at significant
costs --- namely, the tariffs Mr. Trump imposed to force China to accept
an agreement and the uncertainty that
\href{https://www.nytimes.com/2019/11/18/business/trump-trade-war-china.html}{his
unpredictable approach to trade} has created. Those factors have added
new costs for businesses, forced them to undertake expensive changes to
their supply chains and caused them to put off investments and new
hiring.

Once those costs are taken into account, trade experts said, the gains
from the new agreement are less clear.

``It's hard to see this China deal as the vindication of the president's
tactics,'' said Edward Alden, a senior fellow at the Council on Foreign
Relations. ``It's a pretty small deal, coming at a pretty high cost.''

\includegraphics{https://static01.nyt.com/images/2019/12/15/business/15dc-chinatrade2/merlin_165924870_7bf0cbf8-ef4e-49de-afe9-ffc2b76e62a8-articleLarge.jpg?quality=75\&auto=webp\&disable=upscale}

To persuade China to bend to American demands, Mr. Trump imposed more
new tariffs than any other president in modern history. On Friday, Mr.
Trump announced that he would not go forward with an additional tariff
increase planned for Sunday and that he would lower the rate on some of
the tariffs he had already placed on China.

But tariffs on more than \$360 billion of Chinese goods --- the bulk of
products the country exports to the United States --- will stay in place
indefinitely.

The remaining tariffs cover a wide range of product categories in which
American officials contend that the Chinese government has provided huge
subsidies to businesses to become globally competitive. They also
include many goods for which the Trump administration is leery of having
the United States depend on China for national security or economic
security reasons, such as nuclear reactor parts or certain widely used
industrial pumps and motors.

In the interview on Sunday, Mr. Lighthizer described those tariffs as
motivation for China to continue to negotiate with the United States.
But many businesses continue to denounce them as a tax on doing business
with the world's second-largest economy.

Companies that import parts and finished products from China have
already paid nearly \$40 billion in additional taxes since Mr. Trump
imposed his first tariffs,
\href{https://www.cbp.gov/newsroom/stats/trade}{data from United States
Customs and Border Protection} shows. While Mr. Trump insists that China
is paying those tariffs,
\href{https://www.nytimes.com/2019/06/03/business/tariffs-trump-mexico-china.html}{most
economic}
\href{https://bfi.uchicago.edu/working-paper/tariff-passthrough-at-the-border-and-at-the-store-evidence-from-us-trade-policy/}{studies}
have found that the burden of the levies falls more heavily on American
businesses and consumers than Chinese ones.

The deal will need to make up a lot of ground in the area of
agriculture, as well.

Under pressure from the trade war, American farm exports to China have
fallen sharply, as China has put tariffs on American products and
Chinese state purchasers shifted to buying goods from Brazil, Argentina
and other countries. American agricultural exports to China fell from
\$19.6 billion in 2017 to \$9.2 billion in 2018, according to the United
States Agriculture Department, and have remained depressed this year.

Mr. Trump and his advisers have predicted that the deal will result in
China buying \$40 billion to \$50 billion of American farm products per
year. But some analysts have questioned how realistic those estimates
are, given that the highest level of farm products the United States has
ever exported to China was \$26 billion in 2012.

Image

A farm near Colfax, N.D. American farm exports to China have fallen
sharply.Credit...Dan Koeck/Reuters

The uncertainty created by the trade war also appears to have taken a
substantial toll on the American and global economy, particularly by
suppressing business investment.

Mr. Trump and his advisers have pointed to record-low unemployment, a
strong stock market and high consumer confidence as evidence that their
trade war has little downside. But economists say American growth would
be even stronger if not for the trade war.

Mark Zandi, the chief economist at Moody's Analytics, estimated that the
trade war lowered American gross domestic product by a third of a
percentage point in the third quarter, when the American economy
\href{https://www.nytimes.com/2019/10/30/business/economy/us-gdp-growth.html}{grew
by 1.9 percent}.

``The trade war has done significant damage to the economy,'' Mr. Zandi
said. ``You can see the fingerprints of the trade war clearly in the
manufacturing sector.''

The new tariffs that Mr. Trump decided not to move ahead with on Sunday
would have fallen more heavily on American consumers by raising the
price of apparel, smartphones and other finished goods. He also scaled
back tariffs imposed in September on other consumer products.

But earlier tranches of tariffs, which fell more heavily on industrial
components and machinery, will remain in effect. That could ironically
penalize some companies for making goods in the United States, instead
of China.

Robert J. Leo, a lawyer for the American Down and Feather Council, said
that levies would remain in effect on down and feathers from China, but
not on Chinese-made comforters and pillows.

``That means the Chinese manufacturers can manufacture their products
and get them into the country without tariffs,'' where American
manufacturers that import the goods to make products in the United
States will still be charged, Mr. Leo said.

Despite the barriers that remain, Mr. Lighthizer said in the interview
that Friday was ``probably the most momentous day in trade history
ever,'' because in addition to announcing the agreement with China, the
administration submitted its revised United States-Mexico-Canada
Agreement to Congress for a vote.

The two deals ``have been hyped as short-term wins for the U.S.
resulting from hard-nosed negotiations by the Trump administration,''
said Eswar Prasad, a trade professor at Cornell University. ``But the
outcomes of these trade deals hardly compensate for the heightened
uncertainty resulting from the trade tensions unleashed by the Trump
administration on multiple fronts that has hurt business sentiment and
contributed to falling investment.''

The North American deal has gained the support of congressional
Democrats and appeared to be on track for passage in the House of
Representatives as early as this week. But in recent days, Mexico has
raised new concerns about the deal's stronger labor provisions, throwing
up a potential stumbling block to its passage.

Jesús Seade, Mexico's chief negotiator for the pact, flew to Washington
for meetings on Sunday after the United States said it would send as
many as five labor attachés to Mexico to monitor labor conditions under
the deal. Mexico has described the idea as a violation of its
sovereignty.

For its part, the Chinese government appeared over the weekend to be
keeping up its end of the deal struck on Friday, starting with the
cancellation on Sunday of plans to impose further retaliatory tariffs
against the United States.

China's Finance Ministry announced that the country's tariff commission
had rescinded plans to impose tariffs of 5 percent or 10 percent on a
range of American products, notably farm goods like sorghum and seed
corn as well as flavored tea, electric clocks, magnifying glasses and
navigational radars. China
\href{https://www.nytimes.com/2019/08/23/business/china-tariffs-trump.html}{had
previously said} that it would put the tariffs in place if the United
States proceeded with plans to impose further tariffs on Sunday.

The ministry said China would continue to collect 25 percent tariffs on
a wide range of other American goods, in retaliation for the continued
American imposition of 25 percent tariffs on \$250 billion a year worth
of Chinese goods.

Image

A television screen on the floor of the New York Stock Exchange on
Friday shows the press announcement in China regarding the phase one
trade agreement.Credit...Richard Drew/Associated Press

Wang Yi, China's foreign minister, praised the trade deal on Saturday,
and that praise was widely echoed by state media.

Mr. Wang said the Phase 1 pact was based on principles of mutual respect
between China and the United States --- a crucial requirement and
endorsement from Beijing's perspective. He also said the understanding
between the two countries was good news for their economies and for the
global economy.

``It will help to shore up confidence'' in the global economy, Mr. Wang
said, according to state-run Chinese television.

Ana Swanson reported from Washington, and Keith Bradsher from Beijing.
Chris Buckley contributed reporting from Beijing.

Advertisement

\protect\hyperlink{after-bottom}{Continue reading the main story}

\hypertarget{site-index}{%
\subsection{Site Index}\label{site-index}}

\hypertarget{site-information-navigation}{%
\subsection{Site Information
Navigation}\label{site-information-navigation}}

\begin{itemize}
\tightlist
\item
  \href{https://help.nytimes.com/hc/en-us/articles/115014792127-Copyright-notice}{©~2020~The
  New York Times Company}
\end{itemize}

\begin{itemize}
\tightlist
\item
  \href{https://www.nytco.com/}{NYTCo}
\item
  \href{https://help.nytimes.com/hc/en-us/articles/115015385887-Contact-Us}{Contact
  Us}
\item
  \href{https://www.nytco.com/careers/}{Work with us}
\item
  \href{https://nytmediakit.com/}{Advertise}
\item
  \href{http://www.tbrandstudio.com/}{T Brand Studio}
\item
  \href{https://www.nytimes.com/privacy/cookie-policy\#how-do-i-manage-trackers}{Your
  Ad Choices}
\item
  \href{https://www.nytimes.com/privacy}{Privacy}
\item
  \href{https://help.nytimes.com/hc/en-us/articles/115014893428-Terms-of-service}{Terms
  of Service}
\item
  \href{https://help.nytimes.com/hc/en-us/articles/115014893968-Terms-of-sale}{Terms
  of Sale}
\item
  \href{https://spiderbites.nytimes.com}{Site Map}
\item
  \href{https://help.nytimes.com/hc/en-us}{Help}
\item
  \href{https://www.nytimes.com/subscription?campaignId=37WXW}{Subscriptions}
\end{itemize}
