Sections

SEARCH

\protect\hyperlink{site-content}{Skip to
content}\protect\hyperlink{site-index}{Skip to site index}

\href{https://www.nytimes.com/section/movies}{Movies}

\href{https://myaccount.nytimes.com/auth/login?response_type=cookie\&client_id=vi}{}

\href{https://www.nytimes.com/section/todayspaper}{Today's Paper}

\href{/section/movies}{Movies}\textbar{}`Parasite' Review: The Lower
Depths Rise With a Vengeance

\href{https://nyti.ms/2M68mpg}{https://nyti.ms/2M68mpg}

\begin{itemize}
\item
\item
\item
\item
\item
\item
\end{itemize}

Advertisement

\protect\hyperlink{after-top}{Continue reading the main story}

Supported by

\protect\hyperlink{after-sponsor}{Continue reading the main story}

Critic's Pick

\hypertarget{parasite-review-the-lower-depths-rise-with-a-vengeance}{%
\section{`Parasite' Review: The Lower Depths Rise With a
Vengeance}\label{parasite-review-the-lower-depths-rise-with-a-vengeance}}

In Bong Joon Ho's new film, a destitute family occupies a wealthy
household in an elaborate scheme that goes comically --- then horribly
--- wrong.

\includegraphics{https://static01.nyt.com/images/2019/10/21/arts/parasite-anatomy1/parasite-anatomy1-videoSixteenByNineJumbo1600.jpg}

By \href{https://www.nytimes.com/by/manohla-dargis}{Manohla Dargis}

\begin{itemize}
\item
  Published Oct. 10, 2019Updated Feb. 10, 2020
\item
  \begin{itemize}
  \item
  \item
  \item
  \item
  \item
  \item
  \end{itemize}
\end{itemize}

\begin{itemize}
\tightlist
\item
  Parasite\\
  **NYT Critic's Pick Directed by Joon-ho Bong Comedy, Drama, Thriller R
  2h 12m
\end{itemize}

\href{https://www.imdb.com/showtimes/title/tt6751668?ref_=ref_ext_NYT}{Find
Tickets}

When you purchase a ticket for an independently reviewed film through
our site, we earn an affiliate commission.

Midway through the brilliant and deeply unsettling ``Parasite,'' a
destitute man voices empathy for a family that has shown him none.
``They're rich but still nice,'' he says, aglow with good will. His wife
has her doubts. ``They're nice because they're rich,'' she counters.
With their two adult children, they have insinuated themselves into the
lives of their pampered counterparts. It's all going so very well until
their worlds spectacularly collide, erupting with annihilating force.
Comedy turns to tragedy and smiles twist into grimaces as the real world
splatters across the manicured lawn.

The story takes place in South Korea but could easily unfold in Los
Angeles or London. The director Bong Joon Ho
(\href{https://www.nytimes.com/2017/06/27/movies/review-okja-bong-joon-ho.html}{``Okja''})
creates specific spaces and faces --- outer seamlessly meets inner here
--- that are in service to universal ideas about human dignity, class,
life itself. With its open plan and geometric shapes, the modernist home
that becomes the movie's stage (and its house of horrors) looks as
familiar as the cover of a shelter magazine. It's the kind of clean,
bright space that once expressed faith and optimism about the world but
now whispers big-ticket taste and privilege.

\includegraphics{https://static01.nyt.com/images/2019/10/10/arts/10parasitecoverpix/merlin_162276381_35b982ba-822c-4914-98b6-083a213beaec-articleLarge.jpg?quality=75\&auto=webp\&disable=upscale}

``Space and light and order,''
\href{https://www.nytimes.com/topic/person/le-corbusier}{Le Corbusier}
said, are as necessary as ``bread or a place to sleep.'' That's a good
way of telegraphing the larger catastrophe represented by the cramped,
gloomy and altogether disordered basement apartment where Kim Ki-taek
(the great Song Kang Ho) benignly reigns. A sedentary lump (he looks as
if he's taken root), Ki-taek doesn't have a lot obviously going for him.
But he has a home and the affection of his wife and children, and
together they squeeze out a meager living assembling pizza boxes for a
delivery company. They're lousy at it, but that scarcely matters as much
as the petty humiliations that come with even the humblest job.

The Kims' fortunes change after the son, Ki-woo (Choi Woo Shik), lands a
lucrative job as an English-language tutor for the teenage daughter,
Da-hye (Jung Ziso), of the wealthy Park family. The moment that he walks
up the quiet, eerily depopulated street looking for the Park house it's
obvious we're not idling in the lower depths anymore. Ki-woo crosses the
threshold into another world, one of cultivated sensitivities and warmly
polished surfaces that are at once signifiers of bourgeois success and
blunt reproaches to his own family's deprivation. For him, the house
looks like a dream, one that his younger sister and parents soon join by
taking other jobs in the Park home.

Image

Cho Yeo Jeong as the wealthy Mrs. Park in ``Parasite.''Credit...Neon

Take being the operative word. The other Kims don't secure their
positions as art tutor, housekeeper and chauffeur, they seize them,
using lies and charm to get rid of the Parks' other employees ---
including a longtime housekeeper (a terrifically vivid Lee Jung Eun) ---
in a guerrilla incursion executed with fawning smiles. The Parks make it
easy (no background checks). Yet they're not gullible, as Ki-taek
believes, but are instead defined by cultivated helplessness, the
near-infantilization that money affords. In outsourcing their lives, all
the cooking and cleaning and caring for their children, the Parks are as
parasitical as their humorously opportunistic interlopers.

Bong's command of the medium is thrilling. He likes to move the camera,
sometimes just to nudge your attention from where you think it should
be, but always in concert with his restlessly inventive staging. When,
in an early scene, the Kims crowd their superior from the pizza company,
their bodies nearly spilling out of the frame, the image both
underscores the family's closeness and foreshadows their collective
assault on the Parks. Nothing if not a rigorous dialectician, Bong
refuses to sentimentalize the Kims' togetherness or their poverty. But
he does pointedly set it against the relative isolation of the Parks,
who don't often share the same shot much less the same room.

Bong has some ideas in ``Parasite,'' but
\href{https://www.nytimes.com/2019/05/25/movies/cannes-film-festival-winners-parasite.html}{the
movie's greatness} isn't a matter of his apparent ethics or ethos ---
he's on the side of decency --- but of how he delivers truths, often
perversely and without an iota of self-serving cant. (He likes to get
under your skin, not wag his finger.) He accents the rude comedy of the
Kims' struggle with slyness and precision timing, encouraging your
laughter. When the son and daughter can't locate a Wi-Fi signal --- the
family has been tapping a neighbor's --- they find one near the toilet
(an apt tribute to the internet). And when a cloud of fumigation billows
in from outside, an excited Ki-taek insists on keeping the windows open
to take advantage of the free insecticide. They choke, you laugh. You
also squirm.

The lightly comic tone continues after the Kims begin working for the
Parks, despite ripples of unease that develop into riptides. Some of
this disquiet is expressed in the dialogue, including through the Kims'
performative subservience, with its studied courtesies and strategic
hedging. (Bong shares script credit with Han Jin Won.) The poor family
quickly learns what the rich family wants to hear. For their part, Mr.
and Mrs. Park (Lee Sun Kyun and Cho Yeo Jeong) speak the language of
brutal respectability each time they ask for something (a meal, say) or
deploy a metaphor, as when he gripes about people who ``cross the line''
and smell like ``old radishes.''

The turning point comes midway through when the Parks leave on a camping
trip, packing up their Range Rover, outdoor projector included. In their
absence, the Kims bring out the booze, kick back and take over the
house, a break that's cut short when the old housekeeper returns,
bringing a surprise with her. The slapstick becomes more violent, the
stakes more naked, the laughs more terrifying and cruel. By that point,
you are as comfortably settled in as the Kims; the house is so very
pleasant, after all. But the cost of that comfort and those pretty rooms
--- and the eager acquiescence to the unfairness and meanness they
signify --- comes at a terrible price.

\textbf{Parasite}

Rated R for class exploitation and bloody violence. In Korean, with
subtitles. Running time: 2 hours and 12 minutes.

Advertisement

\protect\hyperlink{after-bottom}{Continue reading the main story}

\hypertarget{site-index}{%
\subsection{Site Index}\label{site-index}}

\hypertarget{site-information-navigation}{%
\subsection{Site Information
Navigation}\label{site-information-navigation}}

\begin{itemize}
\tightlist
\item
  \href{https://help.nytimes.com/hc/en-us/articles/115014792127-Copyright-notice}{©~2020~The
  New York Times Company}
\end{itemize}

\begin{itemize}
\tightlist
\item
  \href{https://www.nytco.com/}{NYTCo}
\item
  \href{https://help.nytimes.com/hc/en-us/articles/115015385887-Contact-Us}{Contact
  Us}
\item
  \href{https://www.nytco.com/careers/}{Work with us}
\item
  \href{https://nytmediakit.com/}{Advertise}
\item
  \href{http://www.tbrandstudio.com/}{T Brand Studio}
\item
  \href{https://www.nytimes.com/privacy/cookie-policy\#how-do-i-manage-trackers}{Your
  Ad Choices}
\item
  \href{https://www.nytimes.com/privacy}{Privacy}
\item
  \href{https://help.nytimes.com/hc/en-us/articles/115014893428-Terms-of-service}{Terms
  of Service}
\item
  \href{https://help.nytimes.com/hc/en-us/articles/115014893968-Terms-of-sale}{Terms
  of Sale}
\item
  \href{https://spiderbites.nytimes.com}{Site Map}
\item
  \href{https://help.nytimes.com/hc/en-us}{Help}
\item
  \href{https://www.nytimes.com/subscription?campaignId=37WXW}{Subscriptions}
\end{itemize}
