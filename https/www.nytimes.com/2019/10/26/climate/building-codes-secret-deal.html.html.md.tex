Sections

SEARCH

\protect\hyperlink{site-content}{Skip to
content}\protect\hyperlink{site-index}{Skip to site index}

\href{https://www.nytimes.com/section/climate}{Climate}

\href{https://myaccount.nytimes.com/auth/login?response_type=cookie\&client_id=vi}{}

\href{https://www.nytimes.com/section/todayspaper}{Today's Paper}

\href{/section/climate}{Climate}\textbar{}Secret Deal Helped Housing
Industry Stop Tougher Rules on Climate Change

\url{https://nyti.ms/2NaSl0q}

\begin{itemize}
\item
\item
\item
\item
\item
\end{itemize}

\href{https://www.nytimes.com/section/climate?action=click\&pgtype=Article\&state=default\&region=TOP_BANNER\&context=storylines_menu}{Climate
and Environment}

\begin{itemize}
\tightlist
\item
  \href{https://www.nytimes.com/2020/07/30/climate/sea-level-inland-floods.html?action=click\&pgtype=Article\&state=default\&region=TOP_BANNER\&context=storylines_menu}{Rising
  Seas}
\item
  \href{https://www.nytimes.com/interactive/2020/climate/trump-environment-rollbacks.html?action=click\&pgtype=Article\&state=default\&region=TOP_BANNER\&context=storylines_menu}{Trump's
  Changes}
\item
  \href{https://www.nytimes.com/interactive/2020/04/19/climate/climate-crash-course-1.html?action=click\&pgtype=Article\&state=default\&region=TOP_BANNER\&context=storylines_menu}{Climate
  101}
\item
  \href{https://www.nytimes.com/interactive/2018/08/30/climate/how-much-hotter-is-your-hometown.html?action=click\&pgtype=Article\&state=default\&region=TOP_BANNER\&context=storylines_menu}{Is
  Your Hometown Hotter?}
\item
  \href{https://www.nytimes.com/newsletters/climate-change?action=click\&pgtype=Article\&state=default\&region=TOP_BANNER\&context=storylines_menu}{Newsletter}
\end{itemize}

Advertisement

\protect\hyperlink{after-top}{Continue reading the main story}

Supported by

\protect\hyperlink{after-sponsor}{Continue reading the main story}

\hypertarget{secret-deal-helped-housing-industry-stop-tougher-rules-on-climate-change}{%
\section{Secret Deal Helped Housing Industry Stop Tougher Rules on
Climate
Change}\label{secret-deal-helped-housing-industry-stop-tougher-rules-on-climate-change}}

\includegraphics{https://static01.nyt.com/images/2019/10/25/climate/00CLI-BUILDINGCODES3/merlin_163307559_e8a1a3d9-dc10-461f-90c4-f29ece5a68fb-articleLarge.jpg?quality=75\&auto=webp\&disable=upscale}

\href{https://www.nytimes.com/by/christopher-flavelle}{\includegraphics{https://static01.nyt.com/images/2019/06/28/climate/author-chris-flavelle/author-chris-flavelle-thumbLarge-v3.png}}

By \href{https://www.nytimes.com/by/christopher-flavelle}{Christopher
Flavelle}

\begin{itemize}
\item
  Oct. 26, 2019
\item
  \begin{itemize}
  \item
  \item
  \item
  \item
  \item
  \end{itemize}
\end{itemize}

\emph{Want climate news in your inbox?}
\href{https://www.nytimes.com/newsletters/climate-change}{\emph{Sign up
here
for}}\textbf{\href{https://www.nytimes.com/newsletters/climate-change}{\emph{Climate
Fwd:}}}\emph{, our email newsletter.}

WASHINGTON --- A secret agreement has allowed the nation's homebuilders
to make it much easier to block changes to building codes that would
require new houses to better address climate change, according to
documents reviewed by The New York Times.

The written arrangement, in place for years and not previously
disclosed, guarantees industry representatives four of the 11 voting
seats on two powerful committees that approve building codes that are
widely adopted nationwide. The pact has helped enable the trade group
that controls the seats, the National Association of Home Builders, to
prevent changes that would have made new houses in much of the country
more energy-efficient or more resilient to floods, hurricanes and other
disasters.

The agreement shows that homebuilders accrued ``excessive power over the
development of regulations that governed them,'' said Bill Fay, head of
the Energy Efficient Codes Coalition, which has pushed for more
aggressive standards. Homes accounted for nearly
\href{https://www.eia.gov/tools/faqs/faq.php?id=75\&t=11}{one-fifth} of
all energy-related carbon dioxide emissions nationwide last year.

While four seats is a minority on the two committees, which focus on
residential building codes, the bloc of votes makes it tougher to pass
revisions that the industry opposes. Before the homebuilders gained
seats on the committee that handles energy, for example, the energy
efficiency of those building codes increased 32 percent over six years,
according to a federal analysis. After the industry's influence
expanded, that number was less than 3 percent over the same amount of
time.

``The influence doesn't get any stronger than that,'' said David Cohan,
who managed the Building Codes Energy Program at the Department of
Energy until last year, referring to the agreement. ``It makes it such
an uphill battle.''

Today in Las Vegas, members of the International Code Council, a
Washington-based nonprofit that develops recommended building codes, are
meeting to consider the proposed changes to energy-efficiency building
codes, which get updated every three years. This round, the homebuilders
have opposed changes that include requiring better insulation in attics
and air ducts, as well as a proposal requiring new houses to be equipped
with the circuitry required to install a plug for an electric vehicle
--- potentially making it easier for homeowners to switch to electric
cars in the future.

``Increased adoption of EVs will have a positive effect on overall U.S.
household energy spending and carbon emissions,'' the sponsors of the
proposal wrote. The homebuilders opposed the change, saying the
requirements ``would have a significant impact on affordability,
particularly for entry-level homes and rentals.''

The chairman of the homebuilders association, Greg Ugalde, said it was
appropriate for homebuilders to have a voice on the committees. ``Our
industry is essential in the use of the codes,'' he said in a statement.
``Our members are the ones using them most directly and they know what
works and what doesn't.''

In return for getting seats on the committees, the builders association
agreed to support the adoption of the council's codes by state and local
officials, according to Tim Ballo, a lawyer at the environmental
advocacy group Earthjustice who has seen the agreement.

The homebuilding industry says it opposes proposals that would make
houses more expensive, pricing people out of the market. Supporters of
the changes say they would more than pay for themselves over time with
lower energy bills and reduced likelihood of damage in the event of
floods or other disasters.

\href{https://www.nytimes.com/section/climate?action=click\&pgtype=Article\&state=default\&region=MAIN_CONTENT_1\&context=storylines_keepup}{}

\hypertarget{climate-and-environment-}{%
\subsubsection{Climate and Environment
›}\label{climate-and-environment-}}

\hypertarget{keep-up-on-the-latest-climate-news}{%
\paragraph{Keep Up on the Latest Climate
News}\label{keep-up-on-the-latest-climate-news}}

Updated July 30, 2020

Here's what you need to know about the latest climate change news this
week:

\begin{itemize}
\item
  \begin{itemize}
  \tightlist
  \item
    \href{https://www.nytimes.com/2020/07/30/climate/bangladesh-floods.html?action=click\&pgtype=Article\&state=default\&region=MAIN_CONTENT_1\&context=storylines_keepup}{Floods
    in}\href{https://www.nytimes.com/2020/07/30/climate/bangladesh-floods.html?action=click\&pgtype=Article\&state=default\&region=MAIN_CONTENT_1\&context=storylines_keepup}{Bangladesh}
    are punishing the people least responsible for climate change.
  \item
    As climate change raises sea levels,
    \href{https://www.nytimes.com/2020/07/30/climate/sea-level-inland-floods.html?action=click\&pgtype=Article\&state=default\&region=MAIN_CONTENT_1\&context=storylines_keepup}{storm
    surges and high tides} are likely to push farther inland.
  \item
    The E.P.A. inspector general plans to investigate whether a rollback
    of fuel efficiency standards
    \href{https://www.nytimes.com/2020/07/27/climate/trump-fuel-efficiency-rule.html?action=click\&pgtype=Article\&state=default\&region=MAIN_CONTENT_1\&context=storylines_keepup}{violated
    government rules}.
  \end{itemize}
\end{itemize}

Building codes in the United States are set by state and local
officials, but they are usually based on model codes published by the
International Code Council.

\includegraphics{https://static01.nyt.com/images/2019/10/25/climate/00CLI-BUILDINGCODES2/00CLI-BUILDINGCODES2-articleLarge.jpg?quality=75\&auto=webp\&disable=upscale}

\href{https://www.iccsafe.org/international-code-adoptions/}{Other
countries}, including in Latin America and the Caribbean, also draw from
the codes published by the council.

``The code council has not tried to keep this a secret,'' Whitney Doll,
the council's vice president of communications, said of the agreement.
She said the council had not made any ``official public statements''
about it.

To develop its proposed building codes, the International Code Council
collects proposed changes from building inspectors, engineers and
others, then convenes committees to review them. The proposals
eventually go to a vote by the full membership of more than 8,000 state
and local officials.

However, individual committees act as crucial gatekeepers: If a
committee rejects a change, it is much harder for it to win final
approval, under the code council's rules.

In 2002, the code council signed an agreement with the National
Association of Home Builders --- which represents 140,000 builders,
suppliers and others in the industry --- that gives the association the
right to select four of the 11 voting members of the committee
responsible for the International Residential Code, which governs home
construction. (The other members are chosen by the code council directly
and tend to be a mix of government officials and others who work in the
building industry.)

The confidential agreement's existence has long been a subject of
speculation among people who work on building codes. When contacted by
The Times, the council initially denied having an agreement with
homebuilders. It later acknowledged the agreement and defended it as
appropriate, while declining to provide a copy.

When presented with a summary of the agreement from Mr. Ballo, the
council confirmed that the summary was accurate.

While four seats out of 11 on the residential building-code committee is
a clear minority, it provides considerable power, critics of the
arrangement say. Advocates for any change opposed by the homebuilders
must win the support of six of the committee's seven remaining voting
members.

``It really makes it difficult for the advancement of energy
efficiency,'' said Ron Jones, a former board member at the association
who is critical of its position on codes. ``The homebuilders took them
hostage by saying, `If you don't work with us, we will look elsewhere to
promote other codes.'''

Michael Pfeiffer, senior vice president for technical services at the
International Code Council, said that guaranteeing the seats to the
homebuilders was a way to take advantage of the industry's experience.
``It's all about bringing stakeholders to the table,'' he said.

Mr. Pfeiffer also said it was not unusual for the council's 8,000 voting
members to overturn a decision by a committee. ``You've got the
necessary checks and balances,'' he said.

In messages to its members, the homebuilders association has noted the
committee's effectiveness at stopping proposals it did not support. That
includes issues like mandating tougher foundations in flood-prone areas
or ensuring that roofs were less likely to blow off during a hurricane.

``Only 6 percent of the proposals that NAHB opposed made it through the
committee hearings intact,'' the association
\href{http://nahbnow.com/2015/05/nahb-continues-fight-for-better-codes/}{wrote}
to its members after meetings in 2015. ``NAHB saw favorable votes on 87
percent of the codes proposals, including 92.5 percent of high-priority
proposals,'' it wrote
\href{http://nahbnow.com/2019/09/building-codes-vote-coming-soon-and-nahb-members-can-help/}{again}
about votes last year.

The chairman of the homebuilders association, Mr. Ugalde, said in his
statement that the committee members selected by his organization were
``not a block vote --- they make their own decisions on how to vote.''

Image

A hurricane-rated window in a new house under construction in Breezy
Point in Queens, N.Y.Credit...Gaia Squarci/Bloomberg

In a blog post last year titled
``\href{http://nahbnow.com/2018/02/nahb-advocacy-money-in-your-pocket/}{Money
in Your Pocket},'' the association wrote that its success at keeping
``costly provisions'' out of the 2015 building code alone had saved
homebuilders \$1,000 per housing start. That figure, combined with other
regulatory changes the group claimed credit for, ``demonstrates just how
much value NAHB delivers for members,'' the association wrote.

Advocates for tougher building codes say the effects of decisions like
these will be felt for generations as global warming leads to more
powerful storms and higher risk of damage to property.

``Our definition of cost is what it costs the families when the home is
built poorly,'' said Leslie Chapman-Henderson, president of the Federal
Alliance for Safe Homes, a consumer advocacy group. Those costs, she
said, are more than financial and extend to the emotional and social
burdens of a family being forced out of their home.

The consequences of the deal between the code council and homebuilders
are easiest to measure when it comes to energy efficiency, which came
under the influence of the homebuilders' agreement in 2011.

Until that point, the model building codes had drastically improved the
energy efficiency of new homes with each new three-year edition. The
2009 and 2012 development cycles together reduced homeowners' annual
energy costs by 32 percent, according to an
\href{https://www.energycodes.gov/sites/default/files/documents/NationalResidentialCostEffectiveness.pdf}{analysis
by the Department of Energy}.

Then, after energy-efficiency codes fell under the agreement between the
code council and the homebuilders, that momentum ground to a halt. The
2015 codes, the first to be negotiated after the change, reduced
residential energy use and costs by
\href{https://www.federalregister.gov/documents/2015/06/11/2015-14297/determination-regarding-energy-efficiency-improvements-in-the-2015-international-energy-conservation}{less
than 1 percent}, the Energy Department found. Savings from the 2018
codes were
\href{https://www.energycodes.gov/regulatory/determinations/residential-determination}{less
than 2 percent}.

``The participation of these industry representatives is critical to the
process,'' Ms. Doll, the code council's spokeswoman, said. ``Their input
helps ensure that code change proposals reflect the evolving needs of
the construction industry.''

For more news on climate and the environment,
\href{https://twitter.com/nytclimate}{follow @NYTClimate on Twitter}.

Advertisement

\protect\hyperlink{after-bottom}{Continue reading the main story}

\hypertarget{site-index}{%
\subsection{Site Index}\label{site-index}}

\hypertarget{site-information-navigation}{%
\subsection{Site Information
Navigation}\label{site-information-navigation}}

\begin{itemize}
\tightlist
\item
  \href{https://help.nytimes.com/hc/en-us/articles/115014792127-Copyright-notice}{©~2020~The
  New York Times Company}
\end{itemize}

\begin{itemize}
\tightlist
\item
  \href{https://www.nytco.com/}{NYTCo}
\item
  \href{https://help.nytimes.com/hc/en-us/articles/115015385887-Contact-Us}{Contact
  Us}
\item
  \href{https://www.nytco.com/careers/}{Work with us}
\item
  \href{https://nytmediakit.com/}{Advertise}
\item
  \href{http://www.tbrandstudio.com/}{T Brand Studio}
\item
  \href{https://www.nytimes.com/privacy/cookie-policy\#how-do-i-manage-trackers}{Your
  Ad Choices}
\item
  \href{https://www.nytimes.com/privacy}{Privacy}
\item
  \href{https://help.nytimes.com/hc/en-us/articles/115014893428-Terms-of-service}{Terms
  of Service}
\item
  \href{https://help.nytimes.com/hc/en-us/articles/115014893968-Terms-of-sale}{Terms
  of Sale}
\item
  \href{https://spiderbites.nytimes.com}{Site Map}
\item
  \href{https://help.nytimes.com/hc/en-us}{Help}
\item
  \href{https://www.nytimes.com/subscription?campaignId=37WXW}{Subscriptions}
\end{itemize}
