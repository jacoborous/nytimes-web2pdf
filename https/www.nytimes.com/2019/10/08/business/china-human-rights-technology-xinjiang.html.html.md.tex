Sections

SEARCH

\protect\hyperlink{site-content}{Skip to
content}\protect\hyperlink{site-index}{Skip to site index}

\href{https://www.nytimes.com/section/business}{Business}

\href{https://myaccount.nytimes.com/auth/login?response_type=cookie\&client_id=vi}{}

\href{https://www.nytimes.com/section/todayspaper}{Today's Paper}

\href{/section/business}{Business}\textbar{}By Taking Aim at Chinese
Tech Firms, Trump Signals a Strategy Shift

\url{https://nyti.ms/31ZJLI3}

\begin{itemize}
\item
\item
\item
\item
\item
\end{itemize}

Advertisement

\protect\hyperlink{after-top}{Continue reading the main story}

Supported by

\protect\hyperlink{after-sponsor}{Continue reading the main story}

\hypertarget{by-taking-aim-at-chinese-tech-firms-trump-signals-a-strategy-shift}{%
\section{By Taking Aim at Chinese Tech Firms, Trump Signals a Strategy
Shift}\label{by-taking-aim-at-chinese-tech-firms-trump-signals-a-strategy-shift}}

In blacklisting surveillance companies, the United States is the first
major government to punish China for its crackdown on Muslims.

\includegraphics{https://static01.nyt.com/images/2019/10/08/business/08blacklist1/merlin_155332329_0a5a71f7-b42d-4fb6-8c02-b255730bb254-articleLarge.jpg?quality=75\&auto=webp\&disable=upscale}

\href{https://www.nytimes.com/by/paul-mozur}{\includegraphics{https://static01.nyt.com/images/2018/10/15/multimedia/author-paul-mozur/author-paul-mozur-thumbLarge.png}}\href{https://www.nytimes.com/by/edward-wong}{\includegraphics{https://static01.nyt.com/images/2018/09/24/multimedia/author-edward-wong/author-edward-wong-thumbLarge-v5.png}}

By \href{https://www.nytimes.com/by/paul-mozur}{Paul Mozur} and
\href{https://www.nytimes.com/by/edward-wong}{Edward Wong}

\begin{itemize}
\item
  Oct. 8, 2019
\item
  \begin{itemize}
  \item
  \item
  \item
  \item
  \item
  \end{itemize}
\end{itemize}

SHANGHAI --- The world has
\href{https://www.nytimes.com/2019/09/25/world/asia/china-xinjiang-muslim-camps.html}{largely
sat by} for nearly two years as China
\href{https://www.nytimes.com/2018/09/08/world/asia/china-uighur-muslim-detention-camp.html}{detained
more than one million people}, mostly Muslims and members of minority
ethnic groups, in internment camps to force them to embrace the
Communist Party.

Now, the Trump administration is taking the first public steps by a
major world government toward punishing Beijing. In doing so, it is
opening up a new front in
\href{https://www.nytimes.com/2019/06/26/world/asia/united-states-china-conflict.html}{the
already worsening relationship} between Washington and Beijing:
\href{https://www.nytimes.com/interactive/2019/04/04/world/asia/xinjiang-china-surveillance-prison.html}{human
rights} and the
\href{https://www.nytimes.com/2019/05/22/world/asia/china-surveillance-xinjiang.html}{dystopian
world of digital surveillance}.

Trump administration officials on Monday placed eight Chinese companies
and a number of police departments
\href{https://www.nytimes.com/2019/10/07/us/politics/us-to-blacklist-28-chinese-entities-over-abuses-in-xinjiang.html?rref=collection\%2Fbyline\%2Fpaul-mozur\&action=click\&contentCollection=undefined\&region=stream\&module=stream_unit\&version=latest\&contentPlacement=1\&pgtype=collection}{on
a blacklist that forbids them to buy American-made technology} like
microchips, software and other vital components. The companies are at
the
\href{https://www.nytimes.com/2018/07/08/business/china-surveillance-technology.html}{vanguard
of China's surveillance} and artificial intelligence ambitions, with
many of them selling increasingly sophisticated systems used by
governments to track people.

The White House cited their business in Xinjiang, a region of
northwestern China that is home to a largely Muslim minority group known
as the Uighurs. The United States government says more than one million
ethnic Uighurs and other minorities have been locked in detention camps
there.

On Tuesday, Secretary of State Mike Pompeo announced
\href{https://www.state.gov/u-s-department-of-state-imposes-visa-restrictions-on-chinese-officials-for-repression-in-xinjiang/}{visa
restrictions on Chinese officials} believed to be involved in the
detention or abuse of Muslim ethnic minorities.

The announcements suggest that the Trump administration is increasingly
willing to listen to the advice of American officials focused on the
strategic challenges posed by China and who are concerned about its
human rights abuses, even if President Trump himself never seems to pay
much attention to those.

The restrictions were announced just two days before American and
Chinese officials were set to begin a 13th round of trade talks, most
likely putting a chill over the negotiations.

More broadly, the White House, in its blacklist announcement, signaled a
willingness to take aim at China's technological dreams. China has
plowed billions of dollars into companies developing advanced hardware
and software to catch up with the United States. Some of the companies
added to the list on Monday are among the world's most valuable
artificial intelligence start-ups.

\includegraphics{https://static01.nyt.com/images/2019/10/08/business/08blacklist4/merlin_138777546_0aacc158-d997-4c9d-9283-a54e1bef817a-articleLarge.jpg?quality=75\&auto=webp\&disable=upscale}

Much of that technology --- including facial recognition and computer
vision --- can be used to track people. That includes smartphone
tracking,
\href{https://www.nytimes.com/2017/12/03/business/china-artificial-intelligence.html}{voice-pattern
identification} and systems that track individuals across cities through
powerful cameras. Washington officials have grown increasingly worried
about China's
\href{https://www.nytimes.com/2019/04/24/technology/ecuador-surveillance-cameras-police-government.html}{ambitions
to export its systems} elsewhere, including places known for human
rights abuses.

``This is an important first step in making some of the companies that
have benefited the most from the re-education system in Xinjiang feel
the consequences of their actions,'' said Darren Byler, an
anthropologist at the University of Washington who studies the plight of
the Uighurs.

He said the move signaled that abuse of minority groups in Xinjiang ``is
real and justifies a political and economic response.''

It is also a potentially groundbreaking use of
\href{https://www.nytimes.com/2019/05/20/technology/google-android-huawei.html}{a
powerful tool} that the American government typically uses against
terrorists. The Chinese companies and police departments were placed on
what is called an entity list, which forbids them to buy sensitive
American exports unless Washington grants American companies specific
permission to sell to them.

Use of the entity list over a human rights issue may be a first, said
Julian Ku, a professor of constitutional and international law at
Hofstra University.

``As far as I know, it was the first time Commerce explicitly cited
human rights as a foreign policy interest of the U.S. for purposes of
export controls,'' he said, referring to the Department of Commerce,
which manages the entity lists. ``This is not an implausible reading of
the regulations, but it is new and has potentially very broad
applicability.''

Geng Shuang, a Chinese Foreign Ministry spokesman, said in a briefing on
Tuesday that the White House was using human rights as an excuse to
punish Chinese companies. Many of the companies offer a wide array of
products outside of surveillance, including medical tools that diagnose
tumors, automatic translation services and even social media filters
that slim the waist.

``This goes against the basic principles of international relations, it
interferes in China's internal affairs, and it goes against China's
national security,'' he said. ``There is no human rights issue in
Xinjiang.''

The Chinese Commerce Ministry urged the United States to take the 28
companies and organizations off the entity list as soon as possible.

The immediate effect on the Chinese companies is likely to be minimal,
because many have stockpiled essential supplies, but they could feel
increasing pain if they stay on the blacklist for months or years.

Perhaps more important, it can put a cloud over the companies'
reputations, limiting their sales in the United States or elsewhere and
keeping them from hiring the world's best technology talent.

``The U.S. move today puts up a big roadblock on the road to
internationalization,'' said Matt Sheehan, a fellow at MacroPolo, the
think tank of the Paulson Institute.

``Most global technology companies are setting up labs abroad,
partnering with the best universities around the world and looking to
recruit top talent from everywhere,'' he said. ``That all just got a lot
harder now that they're marked with the scarlet letter of the entity
list.''

The move followed more than a year of internal debate over how to punish
China for its persecution of Muslims in Xinjiang.

Senior officials on the National Security Council and in the State
Department have
\href{https://www.nytimes.com/2019/05/21/us/politics/hikvision-trump.html}{pushed
for the use of the entity list} to target Chinese companies supplying
surveillance technology to the security forces in Xinjiang. They have
also urged Mr. Trump to
\href{https://www.nytimes.com/2018/09/10/world/asia/us-china-sanctions-muslim-camps.html}{approve
sanctions} that would penalize Chinese officials and companies involved
in the abuses.

But top American trade negotiators, including the treasury secretary,
Steven Mnuchin, have cautioned against policies that would upset trade
talks. Mr. Trump has said he wants to reach a trade deal with China.

Until now, other top officials, most notably Mr. Pompeo and Vice
President Mike Pence, have denounced China's policies in Xinjiang but
not enacted punitive measures. This month, American customs officials
\href{https://www.cbp.gov/newsroom/national-media-release/cbp-issues-detention-orders-against-companies-suspected-using-forced}{blocked
products from a Chinese garment maker} in Xinjiang, but they had held
off on stronger action.

The Chinese companies on the list include Hikvision, a major maker of
surveillance cameras, and the well-funded artificial intelligence
start-ups SenseTime and Megvii.

Image

A Hikvision camera in downtown Beijing. American officials worry that
China will export its surveillance systems.Credit...Jason Lee/Reuters

SenseTime said it set ``high ethical standards for A.I. technologies,''
while Megvii said it required ``clients not to weaponize our technology
or solutions or use them for illegal purposes.'' It added that it had
generated no revenue from within Xinjiang in the first half of 2019.

New York Times reporting showed that four of the companies on the list
--- Yitu, Hikvision, Megvii and SenseTime ---~helped build systems
across China that
\href{https://www.nytimes.com/2019/04/14/technology/china-surveillance-artificial-intelligence-racial-profiling.html}{sought
to use facial recognition to automate the detection of Uighurs}.

Government procurement documents, company marketing materials and
official government releases tied all eight companies to various
business operations and sales in Xinjiang. The many local Xinjiang
police bureaus on the list buy commercial American technology like Intel
microchips and Microsoft Windows software, according to procurement
documents.

Mr. Trump's next step could be imposing sanctions on specific Chinese
officials working in Xinjiang. Among the officials discussed is
\href{https://www.nytimes.com/2018/10/13/world/asia/china-muslim-detainment-xinjang-camps.html}{Chen
Quanguo}, a Politburo member who is the party chief in Xinjiang and an
architect of the system of internment camps and surveillance.

The blacklist action is a sign that strategic China hawks have become
even more influential in the administration in recent weeks.

Matthew Pottinger, the senior director for Asia and an architect of
policies aimed at countering China, was
\href{https://www.wsj.com/articles/trump-picks-matt-pottinger-as-deputy-national-security-adviser-11569009233}{promoted
to deputy national security adviser} last month. Earlier, Robert
O'Brien, the administration's top hostage negotiator, replaced John R.
Bolton as national security adviser. Mr. O'Brien has written that China
poses an enormous challenge to the United States.

``This Xinjiang package has been in the works now for months,'' said
\href{https://www.newamerica.org/our-people/samm-sacks/}{Samm Sacks, a
cybersecurity policy fellow} at New America, a think tank. ``So the fact
that it comes out now just ahead of the next round of trade talks sends
a signal from those in the administration who want no deal.''

``So far, Beijing has been quite measured in its response to the U.S.
government,'' she said. ``That probably is over.''

Paul Mozur reported from Shanghai, and Edward Wong from Hong Kong. Lin
Qiqing contributed research.

Advertisement

\protect\hyperlink{after-bottom}{Continue reading the main story}

\hypertarget{site-index}{%
\subsection{Site Index}\label{site-index}}

\hypertarget{site-information-navigation}{%
\subsection{Site Information
Navigation}\label{site-information-navigation}}

\begin{itemize}
\tightlist
\item
  \href{https://help.nytimes.com/hc/en-us/articles/115014792127-Copyright-notice}{©~2020~The
  New York Times Company}
\end{itemize}

\begin{itemize}
\tightlist
\item
  \href{https://www.nytco.com/}{NYTCo}
\item
  \href{https://help.nytimes.com/hc/en-us/articles/115015385887-Contact-Us}{Contact
  Us}
\item
  \href{https://www.nytco.com/careers/}{Work with us}
\item
  \href{https://nytmediakit.com/}{Advertise}
\item
  \href{http://www.tbrandstudio.com/}{T Brand Studio}
\item
  \href{https://www.nytimes.com/privacy/cookie-policy\#how-do-i-manage-trackers}{Your
  Ad Choices}
\item
  \href{https://www.nytimes.com/privacy}{Privacy}
\item
  \href{https://help.nytimes.com/hc/en-us/articles/115014893428-Terms-of-service}{Terms
  of Service}
\item
  \href{https://help.nytimes.com/hc/en-us/articles/115014893968-Terms-of-sale}{Terms
  of Sale}
\item
  \href{https://spiderbites.nytimes.com}{Site Map}
\item
  \href{https://help.nytimes.com/hc/en-us}{Help}
\item
  \href{https://www.nytimes.com/subscription?campaignId=37WXW}{Subscriptions}
\end{itemize}
