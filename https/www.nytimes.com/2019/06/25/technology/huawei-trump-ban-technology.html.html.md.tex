Sections

SEARCH

\protect\hyperlink{site-content}{Skip to
content}\protect\hyperlink{site-index}{Skip to site index}

\href{https://www.nytimes.com/section/technology}{Technology}

\href{https://myaccount.nytimes.com/auth/login?response_type=cookie\&client_id=vi}{}

\href{https://www.nytimes.com/section/todayspaper}{Today's Paper}

\href{/section/technology}{Technology}\textbar{}U.S. Tech Companies
Sidestep a Trump Ban, to Keep Selling to Huawei

\url{https://nyti.ms/2KEDcpo}

\begin{itemize}
\item
\item
\item
\item
\item
\end{itemize}

Advertisement

\protect\hyperlink{after-top}{Continue reading the main story}

Supported by

\protect\hyperlink{after-sponsor}{Continue reading the main story}

\hypertarget{us-tech-companies-sidestep-a-trump-ban-to-keep-selling-to-huawei}{%
\section{U.S. Tech Companies Sidestep a Trump Ban, to Keep Selling to
Huawei}\label{us-tech-companies-sidestep-a-trump-ban-to-keep-selling-to-huawei}}

\includegraphics{https://static01.nyt.com/images/2019/06/25/business/25huaweitrade1/merlin_155583489_d21f3afe-4f47-4588-af91-64d02d0450b6-articleLarge.jpg?quality=75\&auto=webp\&disable=upscale}

By \href{https://www.nytimes.com/by/paul-mozur}{Paul Mozur} and
\href{https://www.nytimes.com/by/cecilia-kang}{Cecilia Kang}

\begin{itemize}
\item
  June 25, 2019
\item
  \begin{itemize}
  \item
  \item
  \item
  \item
  \item
  \end{itemize}
\end{itemize}

\href{https://cn.nytimes.com/business/20190626/huawei-trump-ban-technology/}{阅读简体中文版}\href{https://cn.nytimes.com/business/20190626/huawei-trump-ban-technology/zh-hant/}{閱讀繁體中文版}

SHANGHAI --- United States chip makers are still selling millions of
dollars of products to Huawei despite a Trump administration ban on the
sale of American technology to the Chinese telecommunications giant,
according to four people with knowledge of the sales.

Industry leaders including Intel and Micron have found ways to avoid
labeling goods as American-made, said the people, who spoke on the
condition they not be named because they were not authorized to disclose
the sales.

Goods produced by American companies overseas are not always considered
American-made. The components began to flow to Huawei about three weeks
ago, the people said.

The sales will help Huawei continue to sell products such as smartphones
and servers, and underscore how difficult it is for the Trump
administration to clamp down on companies that it considers a national
security threat, like Huawei. They also hint at the possible unintended
consequences from altering the web of trade relationships that ties
together the world's electronics industry and global commerce.

The Commerce Department's move to block sales to Huawei, by putting it
on a so-called entity list, set off confusion within the Chinese company
and its many American suppliers, the people said. Many executives lacked
deep experience with American trade controls, leading to initial
suspensions in shipments to Huawei until lawyers could puzzle out which
products could be sent. Decisions about what can and cannot be shipped
were also often run by the Commerce Department.

\includegraphics{https://static01.nyt.com/images/2019/06/25/business/25huaweitrade2/merlin_149714850_de19f104-878f-40b3-ba6d-28b0db38a918-articleLarge.jpg?quality=75\&auto=webp\&disable=upscale}

American companies may sell technology supporting current Huawei
products until mid-August. But a ban on components for future Huawei
products is already in place. It's not clear what percentage of the
current sales were for future products. The sales have most likely
already totaled hundreds of millions of dollars, the people estimated.

While the Trump administration has been aware of the sales, officials
are split about how to respond, the people said. Some officials feel
that the sales violate the spirit of the law and undermine government
efforts to pressure Huawei, while others are more supportive because it
lightens the blow of the ban for American corporations. Huawei has said
it buys around \$11 billion in technology from United States companies
each year.

Intel and Micron declined to comment.

``As we have discussed with the U.S. government, it is now clear some
items may be supplied to Huawei consistent with the entity list and
applicable regulations,'' John Neuffer, the president of the
Semiconductor Industry Association, wrote in a
\href{https://www.semiconductors.org/sia-statement-on-the-scope-of-the-addition-of-huawei-to-the-commerce-departments-entity-list/}{statement}
on Friday.

``Each company is impacted differently based on their specific products
and supply chains, and each company must evaluate how best to conduct
its business and remain in compliance.''

In an earnings call Tuesday afternoon, Micron's chief executive, Sanjay
Mehrotra, said the company stopped shipments to Huawei after the
Commerce Department's action last month. But it resumed sales about two
weeks ago after Micron reviewed the entity list rules and ``determined
that we could lawfully resume'' shipping a subset of products, Mr.
Mehrotra said. ``However, there is considerable ongoing uncertainty
around the Huawei situation,'' he added.

A spokesman for the Commerce Department, in response to questions about
the sales to Huawei, referred to a section of the official notice about
the company being added to the entity list, including that the purpose
was to ``prevent activities contrary to the national security or foreign
policy interests of the United States.''

Image

The Idaho-based Micron competes with South Korean companies like Samsung
to supply memory chips that go into Huawei's smartphones.Credit...Kai
Pfaffenbach/Reuters

A senior administration official said that after the Commerce Department
put Huawei on the entity list, the Semiconductor Industry Association
sent a letter to the White House asking for waivers for some companies
to allow them to continue selling components to Huawei. But the
administration did not grant the waivers, he said, and the companies
then found what they assert is a legal basis for continuing their sales.

Administration officials would like to address this issue, he said, but
they do not plan to do so before the G-20 summit in Japan at the end of
this week. Mr. Trump's top priority is to discuss the general trade
dispute with President Xi Jinping of China and get the two sides to
resume trade talks that have dragged on since early 2018, the official
said.

The fate of Huawei, a crown jewel of Chinese innovation and
technological prowess, has become a symbol of the economic and security
standoff between the United States and China. The Trump administration
has warned that Chinese companies like Huawei, which makes telecom
networking equipment, could intercept or secretly divert information to
China. Huawei has denied those charges.

President Xi Jinping of China and President Trump are expected to have
an
\href{https://www.nytimes.com/2019/06/18/us/politics/trump-china-meeting-trade.html?action=click\&module=RelatedCoverage\&pgtype=Article\&region=Footer}{``extended''
talk} this week during the Group of 20 meetings in Japan, a sign that
the two countries are again seeking a compromise after trade discussions
broke down in May. After the talks stalled, the Trump administration
announced new restrictions on Chinese technology companies.

While the Trump administration has pointed to security and legal
concerns to justify its actions, some analysts have worried that Huawei
and other Chinese tech companies were becoming pawns in the trade
negotiations. Along with Huawei, the administration blocked a Chinese
\href{https://www.nytimes.com/2019/06/21/us/politics/us-china-trade-blacklist.html}{supercomputer
maker} from buying American tech, and it is considering adding the
\href{https://www.nytimes.com/2019/05/21/us/politics/hikvision-trump.html?module=inline}{surveillance
technology company Hikvision} to the list.

Kevin Wolf, a former Commerce Department official and partner at the law
firm Akin Gump, has advised several American technology companies that
supply Huawei. He said he told executives that Huawei's addition to the
list did not prevent American suppliers from continuing sales, as long
as the goods and services weren't made in the United States.

Image

The SK Hynix plant in Icheon, South Korea. American companies are
worried about losing market share to foreign rivals.Credit...Pool photo
by Kim Min-hee

A chip, for example, can still be supplied to Huawei if it is
manufactured outside the United States and doesn't contain technology
that can pose national security risks. But there are limits on sales
from American companies. If the chip maker provides services from the
United States for troubleshooting or instruction on how to use the
product, for example, the company would not be able to sell to Huawei
even if the physical chip were made overseas, Mr. Wolf said.

``This is not a loophole or an interpretation because there is no
ambiguity,'' he said. ``It's just esoteric.''

After this article was published online on Tuesday, Garrett Marquis, the
White House National Security Council spokesman, criticized the
companies' workarounds. He said, ``If true, it's disturbing that a
former Senate-confirmed Commerce Department official, who was previously
responsible for enforcement of U.S. export control laws including
through entity list restrictions, may be assisting listed entities to
circumvent those very enforcement mechanisms.''

Mr. Wolf said he does not represent Chinese companies or firms on the
entity list, and he added that Commerce Department officials had
provided him with identical information on the scope of the list in
recent weeks.

In some cases, American companies aren't the only source of important
technology, but they want to avoid losing Huawei's valuable business to
a foreign rival. For instance, the Idaho-based Micron competes with
South Korean companies like Samsung and SK Hynix to supply memory chips
that go into Huawei's smartphones. If Micron is unable to sell to
Huawei, orders could easily be shifted to those rivals.

Beijing has also pressured American companies. This month, the Chinese
government said it would create an
``\href{https://www.nytimes.com/2019/05/31/business/china-list-us-huawei-retaliate.html?action=click\&module=inline\&pgtype=Homepage}{unreliable
entities list}'' to punish companies and individuals it perceived as
damaging Chinese interests. The following week, China's chief economic
planning agency
\href{https://www.nytimes.com/2019/06/08/business/economy/china-huawei-trump.html}{summoned
foreign executives}, including representatives from Microsoft, Dell and
Apple. It warned them that cutting off sales to Chinese companies could
lead to punishment and hinted that the companies should lobby the United
States government to stop the bans. The stakes are high for some of the
American companies, like Apple, which relies on China for many sales and
for much of its production.

Image

A FedEx warehouse in Kernersville, N.C. ``FedEx is a transportation
company, not a law enforcement agency,'' the company said in a complaint
against the government.Credit...Travis Dove for The New York Times

Mr. Wolf said several companies had scrambled to figure out how to
continue sales to Huawei, with some businesses considering a total shift
of manufacturing and services of some products overseas. The escalating
trade battle between the United States and China is ``causing companies
to fundamentally rethink their supply chains,'' he added.

That could mean that American companies shift their know-how, on top of
production, outside the United States, where it would be less easy for
the government to control, said Martin Chorzempa, a research fellow at
the Peterson Institute for International Economics.

``American companies can move some things out of China if that's
problematic for their supply chain, but they can also move the tech
development out of the U.S. if that becomes problematic,'' he said.
``And China remains a large market.''

``Some of the big winners might be other countries,'' Mr. Chorzempa
said.

Some American companies have complained that complying with the tight
restrictions is difficult or impossible, and will take a toll on their
business.

On Monday, FedEx filed a lawsuit against the federal government,
claiming that the Commerce Department's rules placed an ``impossible
burden'' on a company like FedEx to know the origin and technological
makeup of all the shipments it handles.

FedEx's complaint didn't name Huawei specifically. But it said that the
agency's rules that have prohibited exporting American technology to
Chinese companies placed ``an unreasonable burden on FedEx to police the
millions of shipments that transit our network every day.''

``FedEx is a transportation company, not a law enforcement agency,'' the
company said.

A Commerce Department spokesman said it had not yet reviewed FedEx's
complaint but would defend the agency's role in protecting national
security.

Advertisement

\protect\hyperlink{after-bottom}{Continue reading the main story}

\hypertarget{site-index}{%
\subsection{Site Index}\label{site-index}}

\hypertarget{site-information-navigation}{%
\subsection{Site Information
Navigation}\label{site-information-navigation}}

\begin{itemize}
\tightlist
\item
  \href{https://help.nytimes.com/hc/en-us/articles/115014792127-Copyright-notice}{©~2020~The
  New York Times Company}
\end{itemize}

\begin{itemize}
\tightlist
\item
  \href{https://www.nytco.com/}{NYTCo}
\item
  \href{https://help.nytimes.com/hc/en-us/articles/115015385887-Contact-Us}{Contact
  Us}
\item
  \href{https://www.nytco.com/careers/}{Work with us}
\item
  \href{https://nytmediakit.com/}{Advertise}
\item
  \href{http://www.tbrandstudio.com/}{T Brand Studio}
\item
  \href{https://www.nytimes.com/privacy/cookie-policy\#how-do-i-manage-trackers}{Your
  Ad Choices}
\item
  \href{https://www.nytimes.com/privacy}{Privacy}
\item
  \href{https://help.nytimes.com/hc/en-us/articles/115014893428-Terms-of-service}{Terms
  of Service}
\item
  \href{https://help.nytimes.com/hc/en-us/articles/115014893968-Terms-of-sale}{Terms
  of Sale}
\item
  \href{https://spiderbites.nytimes.com}{Site Map}
\item
  \href{https://help.nytimes.com/hc/en-us}{Help}
\item
  \href{https://www.nytimes.com/subscription?campaignId=37WXW}{Subscriptions}
\end{itemize}
