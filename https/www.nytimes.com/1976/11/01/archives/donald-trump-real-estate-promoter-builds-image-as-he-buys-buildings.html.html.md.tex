Sections

SEARCH

\protect\hyperlink{site-content}{Skip to
content}\protect\hyperlink{site-index}{Skip to site index}

\href{https://myaccount.nytimes.com/auth/login?response_type=cookie\&client_id=vi}{}

\href{https://www.nytimes.com/section/todayspaper}{Today's Paper}

Archives\textbar{}Donald Trump, Real Estate Promoter, Builds Image as He
Buys Buildings

\url{https://nyti.ms/1PudYUM}

\begin{itemize}
\item
\item
\item
\item
\item
\end{itemize}

Advertisement

\protect\hyperlink{after-top}{Continue reading the main story}

Supported by

\protect\hyperlink{after-sponsor}{Continue reading the main story}

\hypertarget{donald-trump-real-estate-promoter-builds-image-as-he-buys-buildings}{%
\section{Donald Trump, Real Estate Promoter, Builds Image as He Buys
Buildings}\label{donald-trump-real-estate-promoter-builds-image-as-he-buys-buildings}}

By Judy Klemesrud

\begin{itemize}
\item
  Nov. 1, 1976
\item
  \begin{itemize}
  \item
  \item
  \item
  \item
  \item
  \end{itemize}
\end{itemize}

\includegraphics{https://s1.nyt.com/timesmachine/pages/1/1976/11/01/75642708_360W.png?quality=75\&auto=webp\&disable=upscale}

See the article in its original context from\\
November 1, 1976, Page
41\href{https://store.nytimes.com/collections/new-york-times-page-reprints?utm_source=nytimes\&utm_medium=article-page\&utm_campaign=reprints}{Buy
Reprints}

\href{http://timesmachine.nytimes.com/timesmachine/1976/11/01/75642708.html}{View
on timesmachine}

TimesMachine is an exclusive benefit for home delivery and digital
subscribers.

About the Archive

This is a digitized version of an article from The Times's print
archive, before the start of online publication in 1996. To preserve
these articles as they originally appeared, The Times does not alter,
edit or update them.

Occasionally the digitization process introduces transcription errors or
other problems; we are continuing to work to improve these archived
versions.

He is tall, lean and blond, with dazzling white teeth, and he looks ever
so much like Robert Redford. He rides around town in a chauffeured
silver Cadillac with his initials, DJT, on the plates. He dates slinky
fashion models, belongs to the most elegant clubs and, at only 30 years
of age, estimates that he is worth ``more than \$200 million.''

Flair. It's one of Donald J. Trump's favorite words, and both he, his
friends and his enemies use it when describing his way of life as well
as his business style as New York's No. 1 real estate promoter of the
middle 1970's.

``If a man has flair,'' the energetic, outspoken Mr. Trump said the
other day, ``and is smart and somewhat conservative and has a taste for
what people want, he's bound to be successful in New York.''

Mr. Trump, who is president of the

Brooklyn based Trump Organization, which owns and manages 22,000
apartments, currently has three imaginative Manhattan real‐estate
projects in the works. And much to his delight, his brash, controversial
style has prompted comparisons with his flamboyant idol, the late
William Zeckendorf Sr., who actually developed projects as striking as
those Mr. Trump is proposing.

The proposed projects are.

¶A large Manhattan convention center over the Penn Central
Transportation Company's 34th Street yards. Mr. Trump, who acquired the
development rights from the bankrupt railroad, has drawn up plans for a
\$90 million center, hoping it will replace the stalled convention
center on the Hudson River from 43d to 47th Street.

\textbf{`On Threshold of Coup'}

¶A 1,500‐room Hyatt Regency hotel following the reconstruction of Penn
Central's Commodore Hotel near Grand Central Terminal. Last April, Mr.
Trump received a controversial \$4 million‐a‐year tax abatement from the
city, the first of its kind, for his proposal to rebuild the aging hotel
building.

¶Construction of 14,500 federally subsidized apartments on the Penn
Central's 60th Street yards, to which Mr. Trump has acquired the
development rights. The site is bounded by West 59th and West 72d
Street, West End Avenue and the Hudson River.

``What makes Donald Trump so significant right now,'' said one Manhattan
real estate expert, ``is that there is nobody else who is a private
promoter on a major scale, trying to convince enterpreneurs to develop
major pieces of property.''

Commenting on the Commodore Hotel deal, the expert said he thought Mr.
Trump was ``on the threshold of the greatest real estate coup of the
last miserable three years; if it goes through, you could call him the
William Zeckendorf of Bad Times'.

The other day, Mr. Trump, who says he is publicity shy, allowed a
reporter to accompany him on what he described as a typical work day. It
consisted mainly of visits to his ``jobs,'' the term he uses for housing
projects owned by the Trump Organization, which was founded by his
70‐year‐old father, Fred C. Trump, now the company's chairman.

The day began at 7:45 A.M., when Mr. Trump's chauffeur, Robert Utsey, a
husky, gun‐toting laid‐off New York City policeman who doubles as a
bodyguard, pulled the Cadillac up in front of the Phoenix apartment
building, at 160 East 65th Street.

Mr. Trump, who lives in a threebedroom penthouse apartment done mostly
in beiges and browns and lots of chrome, was waiting in front of the
building. He is 6 feet, 3 inches tall and weighs 190 pounds, and he was
wearing a three‐piece burgundy wool suit, matching patent‐leather shoes,
and a white shirt with the initials ``DJT'' sewn in burgundy thread on
the cuffs.

Speaking occasionally on his car telephone to his secretary and his
banker at Chase Manhattan, Mr. Trump directed his chauffeur to make
stops at the 60th Street yards; the conventioncenter site a federally
subsidized Trump housing project for the aged in East Orange, N.J.,
which he calls ``our philanthropic endeavor'' a middleincome housing
project on Staten Island; the flagship 4,000‐unit Trump Village in
Brooklyn and several other older Trump‐owned projects in Brooklyn that
the company bought in recent years.

``That's one of the reasons for our success---while others were building
over the last three or four years at 10 percent interest, we were
buying, at 5½ percent mortgages,'' Mr. Trump said. ``And the units they
produced in their new buildings were much smaller than the ones we were
buying.''

Although the Trumps have been building in New York City since 1923, the
family has not gotten as much publicity as other real‐estate developers
because they did not enter the Manhattan market until three years ago.

``It was psychology,'' Mr. Trump explained. ``My father knew Brooklyn
very well, and he knew Queens very well. But now, that psychology is
ended.''

\textbf{Employs 1,000 People}

One of the reasons for the current intense push in Manhattan, he said,
is that the Trump Organization, with 15,000 of its 22,000 apartments
situated in New York City (mostly in Broortlyn, Queens and Staten
Island), has a stake in the future of the city.

The organization, which is made up of 60 partnerships and corporations,
also owns apartment buildings in Washington, D. C., Maryland and
Virginia and land in California and Las Vegas, and it employs about
1,000 people.

``New York is either going to get much better or much worse,'' Mr. Trump
predicted, ``and I think it will get much better. I'm not talking about
the South Bronx. I don't know anything about the South Bronx.

``But in Manhattan, I feel a new convention center will be a turning
point for the city it will get rid of all that pornographic garbage in
Times Square. Psychologically, I think if New York City gets a
convention center, it will resurge and rejuvenate.''

As he drove around the city, he exclaimed boyishly, ``Look at that great
building {[}at 56th Street and Madison Avenue{]}. It's available! There
are a lot of good deals around right now.''

What attracts him to the real estate business? ``I love the
architectural creativeness,'' he said. ``For example, the Commodore
Hotel is in one of the most important locations in the city, and its
reconstruction will lead to a rebirth of that area.

``And I like the financial creativeness, too. There's a beauty in
putting together a financial package that really works, whether it be
through tax credits, or a mortgage financing arrangement, or a leaseback
arrangement.''

``Of course, the gamble is an exciting part, too,'' he said, grinning.
``No matter how much you take out of it, you're talking about \$100
million deals, where a 10 percent mistake is \$10 million. But so far,
I've never made a bad deal.''

Donald Trump was in the headlines in 1973, when the Department of
Justice brought suit in Federal Court against the Trump Organization,
charging discrimination against blacks in apartment rentals. Mr. Trump
denied the charges, and later signed an agreement to provide
open‐housing opportunities for minority groups.

``We `never discriminated against blacks,'' Mr. Trump said angrily.
``Five to 10 percent of our units are rented to blacks in the city. But
we won't sign leases with welfare clients unless they have guaranteed
income levels, because otherwise, everyone immediately starts leaving
the building.''

\textbf{`He Has Great Vision'}

Mr. Trump, a glib, nonstop talker, suddenly turned quiet when he stopped
at the Trump Organization's headquarters, at 600 Avenue Z in Brooklyn,
to consult with his father. Face to face, the son seemed affectionately
intimidated by the older man.

``I gave Donald free rein,'' Fred C. Trump said in his office. ``He has
great vision, and everything he touches seems to turn to gold. As long
as he has this great energy in abundance, I'm glad to let him do it.''

``Energy is a word that frequently pops up in discussions about Donald
Trump. Besides being a fast talker, he is a fast walker, a fast eater, a
fast business dealer, and gives the distinct impression of being an
early candidate for a cardac arrest. Some of this energy, he said
proudly could be attributed to the fact that, ``never In my life have I
had a glass of alcohol or a cigarette.''

His father said that Donald was the only one of his five children (three
sons, two daughters) who had shown any interest in the family real
estate business.

Donald, who grew up in the Trumpbuilt family home in Jamaica Estates,
Queens, began learning the business when he was only 12. He continued
helping his father make deals while a student at the Wharton School of
Finance at the University of Pennsylvania, from which he graduated first
in his class in 1968.

``Donald is the smartest person know,'' his father said admiringly.

Fellow real estate executives in this very closely knit industry also
say mostly nice things about Donald Trump, even when given the chance to
speak off the record.

\textbf{`The Jury is Still Out'}

``He's a very adventurous young man, and we're all rooting for him,''
said Samuel J. Lefrak, of the Lefrak Organization. ``He's bold, daring
and swashbuckling. But in my opinion, the jury is still out.''

Harry B. Helmsley of Helmsley‐Spear Inc., said that although he had
never had any dealings with Mr. Trump, he found him to be ``very active
around town: I just hope he can put his deals together.''

Even Preston Robert Tisch, president of Loews Corporation, who is
regarded as Mr. Trump's No. 1 critic in the city, spoke highly of the
young promoter: ``He's a very bright, capable realestate man.''

Real‐estate insiders say Mr. Tisch and Mr. Trump are at odds for two
reasons ---the Commodore Hotel tax abatement deal (Mr. Tisch's company
owns hotels), and the 34th Street convention center site (Mr. Tisch was
long associated with the rival 44th Street conventioncenter site).

Criticism of Mr. Trump came mainly from mortgage bankers and others in
the money end of the real‐estate industry, all of whom requested
anonymity.

``His deals are dramatic, but they haven't come into being,'' said one.
``So far, the chief beneficiary of his creativity has been his public
image.''

Another money man called Mr. Trump ``overrated' and ``totally
obnoxious,'' and said much of his influence had to do with the fact that
he was an early financial supporter of both Governor Carey and Mayor
Beanie and had powerful lawyer (Roy M. Cohn) and powerful public
relations man (Howard Rubinstein).

\textbf{Lunch at `21' Club}

Mr. Trump has been meeting the right people. During lunch at the ``21''
Club, the waiters were bowing and saying, ``Hello Donald,'' and other
lunchers, including Mr. Helmsley and assorted politicians, stopped by to
say hello.

Mr. Trump took exactly one hour for lunch, during which he ate broiled
filet of sole with no butter, drank ginger ale, and chatted with two men
representing the National Jewish Hospital in Denver, which plans to name
him their Man of the Year on Dec. 8 at a dinner in the Waldorf‐Astoria
Hotel.

``I'm not even Jewish, I'm Swedish,'' he said later. ``Most people think
my family is Jewish because we own so many buildings in Brooklyn. But I
guess you don't have to be Jewish to win this award, because they told
me a gentile won it one other year.''

Mr. Trump spent a profitable afternoon, earning a \$140,000 commission
for about 20 minutes work selling part of a housing project for a
friend. A witness to the negotiations said Mr. Trump was a hard‐nosed
broker, refusing to budge from his original terms of \$1.4 million paid
over a four‐year period at 9 percent interest.

\textbf{`Extremely Aggressive'}

The transaction took place at the architectural offices of Poor, Swanke,
Hayden \& Connell, at 400 Park Avenue, where Mr. Trump had gone to visit
Der Scutt, the architect of his proposed \$90million convention center.

``Donald's very demanding,'' the pipepuffing Mr. Scutt said when the
promoter was out of the room. ``He thinks nothing of calling me at 7
A.M. on a Sunday and saying, `I've got an idea. See you in the office in
40 minutes.' And I always go.''

When asked whether he thought Mr. Trump had any shortcomings, the
architect replied: ``He's extremely aggressive when he sells, maybe to
the point of overselling. Like, he'll say the convention center is the
biggest in the world, when it really isn't. He'll exaggerate for the
purpose of making a sale.''

The architect broke into a big smile. ``That Donald,'' he said
admiringly, ``he could sell sand to the Arabs and refrigerators to the
Eskimos.''

Mr. Trump is single, with no plans of getting married in the near
future, although he said he was seeing one woman --- a fashion model ---
fairly regularly. ``If I met the right woman, might get married,'' he
said. ``But right now. I have everything I want or need.''

He said he liked to relax at night by taking a date to such clubs as El
Morocco, Regine's, Le Club or Doubles, or attending Knicks or Rangers
games in Madison Square Garden. (He has season tickets for both teams.)

Mr. Trump ended his ``typical ??? by catching a plane to California,
where he said he planned to wrap up a ``multimillion dollar'' land deal.
He has been spending more and more of his time in the Los Angeles area
lately, staying in a house that he owns, complete with swimming pool and
tennis court, in Beverly Hills.

Is there any danger that Donald Trump will defect to the West Coast?
``Some of the hest deals I've made have been land deals in California,''
he said with a smile. ``I've probably made \$14 million there over the
last two years. But my friends and enemies are all in New York City, so
I'll probably stay here.''

Advertisement

\protect\hyperlink{after-bottom}{Continue reading the main story}

\hypertarget{site-index}{%
\subsection{Site Index}\label{site-index}}

\hypertarget{site-information-navigation}{%
\subsection{Site Information
Navigation}\label{site-information-navigation}}

\begin{itemize}
\tightlist
\item
  \href{https://help.nytimes.com/hc/en-us/articles/115014792127-Copyright-notice}{©~2020~The
  New York Times Company}
\end{itemize}

\begin{itemize}
\tightlist
\item
  \href{https://www.nytco.com/}{NYTCo}
\item
  \href{https://help.nytimes.com/hc/en-us/articles/115015385887-Contact-Us}{Contact
  Us}
\item
  \href{https://www.nytco.com/careers/}{Work with us}
\item
  \href{https://nytmediakit.com/}{Advertise}
\item
  \href{http://www.tbrandstudio.com/}{T Brand Studio}
\item
  \href{https://www.nytimes.com/privacy/cookie-policy\#how-do-i-manage-trackers}{Your
  Ad Choices}
\item
  \href{https://www.nytimes.com/privacy}{Privacy}
\item
  \href{https://help.nytimes.com/hc/en-us/articles/115014893428-Terms-of-service}{Terms
  of Service}
\item
  \href{https://help.nytimes.com/hc/en-us/articles/115014893968-Terms-of-sale}{Terms
  of Sale}
\item
  \href{https://spiderbites.nytimes.com}{Site Map}
\item
  \href{https://help.nytimes.com/hc/en-us}{Help}
\item
  \href{https://www.nytimes.com/subscription?campaignId=37WXW}{Subscriptions}
\end{itemize}
