Sections

SEARCH

\protect\hyperlink{site-content}{Skip to
content}\protect\hyperlink{site-index}{Skip to site index}

\href{https://www.nytimes.com/section/movies}{Movies}

\href{https://myaccount.nytimes.com/auth/login?response_type=cookie\&client_id=vi}{}

\href{https://www.nytimes.com/section/todayspaper}{Today's Paper}

\href{/section/movies}{Movies}\textbar{}FILM REVIEW; Her Mom May Kick,
But a Girl Plays to Win

\begin{itemize}
\item
\item
\item
\item
\item
\end{itemize}

Advertisement

\protect\hyperlink{after-top}{Continue reading the main story}

Supported by

\protect\hyperlink{after-sponsor}{Continue reading the main story}

FILM REVIEW

\hypertarget{film-review-her-mom-may-kick-but-a-girl-plays-to-win}{%
\section{FILM REVIEW; Her Mom May Kick, But a Girl Plays to
Win}\label{film-review-her-mom-may-kick-but-a-girl-plays-to-win}}

\begin{itemize}
\tightlist
\item
  Bend It Like Beckham\\
  Directed by Gurinder Chadha Comedy, Drama, Romance, Sport PG-13 1h 52m
\end{itemize}

By \href{https://www.nytimes.com/by/a-o-scott}{A. O. Scott}

\begin{itemize}
\item
  March 12, 2003
\item
  \begin{itemize}
  \item
  \item
  \item
  \item
  \item
  \end{itemize}
\end{itemize}

''Bend It Like Beckham,'' a genial ethnic sports comedy directed by
Gurinder Chadha, was a big hit in Britain last year, and Fox Searchlight
deserves credit for releasing it in this country with a title that will
be, to much of the American audience, utterly incomprehensible. It
certainly was to me, until a friend who had recently been in London
passed on an explanation that had been given to him in a pub there.
Beckham is David Beckham, star of the Manchester United soccer team (and
the husband of one of the Spice Girls). The observation ''nobody bends
it like Beckham,'' from which the title derives, apparently refers to
his ability to curve the ball past the opposing goalkeeper.

The line is reverently uttered by Jess (Parminder Nagra), the younger
daughter of a family of middle-class Punjabi immigrants (by way of East
Africa) residing in a London suburb. Jess's passion for soccer puts her
at odds with her parents who, while not hidebound traditionalists,
nonetheless think sports are an improper pastime for an almost-grown
teenager with marriage and university to think about.

Her older sister Pinky (Archie Panjabi), whose wedding is approaching,
is puzzled that Jess does not share her boy-crazy, shopping-centered
approach to life. So when Jess, recruited by her new best friend Jules
(Keira Knightley), begins to train with the girls' auxiliary of a local
football club, she precipitates a culture clash that ripples outward
from her own household and becomes more and more complicated until the
big game comes along to sort it all out.

The title, and some of the local dialect, may require a bit of
translation, but the film's warm-and-fuzzy amalgam of multiculturalism
and feminism will look very familiar indeed. It is stuffed to bursting
with affectionate stereotypes and the sticky, somewhat oppressive
Gemütlichkeit that is the hallmark, at least on screen, of immigrant
families, wherever they come from and wherever they reside.

Ms. Chadha's previous movie, ''What's Cooking?'' made this point fairly
explicitly, as it dropped in on the Thanksgiving dinners of a series of
families -\/- Vietnamese, Mexican-American, Jewish and African-American
-\/- living in the same Los Angeles neighborhood and struggling through
a series of domestic crises. If ''Bend It Like Beckham'' had focused on
Pinky rather than Jess, it might have been ''My Big Fat Sikh Wedding.''

Its cheery inoffensiveness, though, is in some ways disappointing. The
South Asian diaspora has inspired some exceptionally clever and
cosmopolitan movies from the likes of Hanif Kureishi and Mira Nair, and
coming after ''Monsoon Wedding'' and ''My Son the Fanatic,'' ''Bend It
Like Beckham'' seems like a step backward. Modern life, for the
characters in Ms. Nair's or Mr. Kureishi's films, is a vortex of
contradiction and confusion, and their stories unfold with an
appropriately swirling, dizzy rhythm. But Ms. Chadha prefers the
schematism of the sitcom, in which humor and pathos are carefully and
predictably rationed, and people have the capacity to change but never
to surprise. The girl-power plot is built, curiously enough, on a
scaffolding of mild misogyny. Both Jess and Jules are held back by their
shrill, overbearing mothers, who insist on outmoded norms of femininity
while the dads, when the chips are down, are patient and supportive.

The cast acquits itself with exemplary good humor. The tiny Ms. Nagra,
her mouth set in a pout of determination, has a charming, disarming
directness that steers the movie through its easily foreseeable
complications. But rather than risk allowing her characters to blossom
into full human oddness, Ms. Chadha saddles them with cute mannerisms
and binds them together with curlicues of plot, all of which are
feverishly tied together by the end.

It is not enough that Jess develop a crush on her coach (Jonathan Rhys
Meyers), which makes her Jules's romantic rival. It must turn out that
all three are having communication problems with their parents, problems
that will be happily solved after the big game, the big wedding and the
big montage that conjoins them. If Ms. Chadha's direction were as
compulsive as the writing (she collaborated on the screenplay with
Guljit Bindra and Paul Mayeda Berges), ''Bend It Like Beckham'' might
have been tighter, funnier and, above all, shorter.

''Bend It Like Beckham'' is rated PG-13. It has some mildly risqué
situations and references.

BEND IT LIKE BECKHAM

Directed by Gurinder Chadha; written by Ms. Chadha, Guljit Bindra and
Paul Mayeda Berges; director of photography, Jong Lin; edited by Justin
Krish; music by Craig Pruess; production designer, Nick Ellis; produced
by Deepak Nayar and Ms. Chadha; released by Fox Searchlight Pictures.
Running time: 112 minutes. This film is rated PG-13.

WITH: Parminder Nagra (Jess Bhamra), Keira Knightley (Jules Paxton),
Jonathan Rhys Meyers (Joe), Anupam Kher (Mr. Bhamra), Archie Panjabi
(Pinky Bhamra), Shaznay Lewis (Mel), Frank Harper (Mike Paxton), Juliet
Stevenson (Paula Paxton) and Shaheen Khan (Mrs. Bhamra).

Advertisement

\protect\hyperlink{after-bottom}{Continue reading the main story}

\hypertarget{site-index}{%
\subsection{Site Index}\label{site-index}}

\hypertarget{site-information-navigation}{%
\subsection{Site Information
Navigation}\label{site-information-navigation}}

\begin{itemize}
\tightlist
\item
  \href{https://help.nytimes.com/hc/en-us/articles/115014792127-Copyright-notice}{©~2020~The
  New York Times Company}
\end{itemize}

\begin{itemize}
\tightlist
\item
  \href{https://www.nytco.com/}{NYTCo}
\item
  \href{https://help.nytimes.com/hc/en-us/articles/115015385887-Contact-Us}{Contact
  Us}
\item
  \href{https://www.nytco.com/careers/}{Work with us}
\item
  \href{https://nytmediakit.com/}{Advertise}
\item
  \href{http://www.tbrandstudio.com/}{T Brand Studio}
\item
  \href{https://www.nytimes.com/privacy/cookie-policy\#how-do-i-manage-trackers}{Your
  Ad Choices}
\item
  \href{https://www.nytimes.com/privacy}{Privacy}
\item
  \href{https://help.nytimes.com/hc/en-us/articles/115014893428-Terms-of-service}{Terms
  of Service}
\item
  \href{https://help.nytimes.com/hc/en-us/articles/115014893968-Terms-of-sale}{Terms
  of Sale}
\item
  \href{https://spiderbites.nytimes.com}{Site Map}
\item
  \href{https://help.nytimes.com/hc/en-us}{Help}
\item
  \href{https://www.nytimes.com/subscription?campaignId=37WXW}{Subscriptions}
\end{itemize}
