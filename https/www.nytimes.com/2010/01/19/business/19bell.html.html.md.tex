Sections

SEARCH

\protect\hyperlink{site-content}{Skip to
content}\protect\hyperlink{site-index}{Skip to site index}

\href{https://www.nytimes.com/section/business}{Business}

\href{https://myaccount.nytimes.com/auth/login?response_type=cookie\&client_id=vi}{}

\href{https://www.nytimes.com/section/todayspaper}{Today's Paper}

\href{/section/business}{Business}\textbar{}Glen W. Bell Jr., Founder of
Taco Bell, Dies at 86

\begin{itemize}
\item
\item
\item
\item
\item
\end{itemize}

Advertisement

\protect\hyperlink{after-top}{Continue reading the main story}

Supported by

\protect\hyperlink{after-sponsor}{Continue reading the main story}

\hypertarget{glen-w-bell-jr-founder-of-taco-bell-dies-at-86}{%
\section{Glen W. Bell Jr., Founder of Taco Bell, Dies at
86}\label{glen-w-bell-jr-founder-of-taco-bell-dies-at-86}}

By \href{https://www.nytimes.com/by/dennis-hevesi}{Dennis Hevesi}

\begin{itemize}
\item
  Jan. 18, 2010
\item
  \begin{itemize}
  \item
  \item
  \item
  \item
  \item
  \end{itemize}
\end{itemize}

Glen W. Bell Jr., whose idea in 1951 to sell crispy-shell tacos from the
window of his hamburger stand became the foundation of Taco Bell, the
restaurant chain that turned Mexican fare into fast food for millions of
Americans, died at his home in Rancho Santa Fe, Calif. He was 86.

His death was announced Sunday on the Taco Bell Web site. No other
details were provided.

Mr. Bell never forgot the first taco buyer at Bell's Hamburgers and Hot
Dogs in San Bernardino, Calif., one of three stands he owned at the
time.

``He was dressed in a suit and as he bit into the taco the juice ran
down his sleeve and dripped on his tie,'' Mr. Bell recalled in ``Taco
Titan: The Glen Bell Story,'' (Bookworld Services, 1999), a biography by
Debra Lee Baldwin. ``I thought, `Uh-oh, we've lost this one.' But he
came back, amazingly enough, and said, `That was good. Gimme
another.'~''

By the time Mr. Bell sold the chain to PepsiCo in 1978, it had grown to
868 restaurants. Today, the company says, more than two billion tacos
and a billion burritos are sold each year at more than 5,600 Taco Bell
restaurants in the United States and around the world.

Drive-in stands dotted San Bernardino when Mr. Bell opened his first one
there in the late 1940s. One competitor, only a few miles away, was the
original stand opened by two brothers with the last name of McDonald.

They all were capitalizing on the emerging Southern California car
culture, offering prompt service and streamlined menus of mostly
standard fare like hamburgers, hot dogs, French fries and milk shakes.

But Mr. Bell, a fan of Mexican food, had a hunch that ground beef,
chopped lettuce, shredded cheese and chili sauce served in the right
wrap could give burgers a run for the money. The problem was which wrap.
Tacos served in Mexican restaurants at the time were made with soft
tortillas.

``If you wanted a dozen, you were in for a wait,'' Mr. Bell said. ``They
stuffed them first, quickly fried them and stuck them together with a
toothpick.''

Image

Glen W. Bell Jr.Credit...Business Wire

The solution: preformed fried shells that would then be stuffed. Mr.
Bell asked a man who made chicken coops to fashion a frying contraption
made of wire.

Tacos became a hit at Bell's, selling for 19 cents each. They were such
a hit that by 1954 Mr. Bell and a partner opened Taco Tia, his first
restaurant selling only Mexican-style food.

Two years and three Taco Tias later, Mr. Bell sold his interest after
his business partner resisted expanding any further. Mr. Bell then
opened another fast-food Mexican restaurant in Pasadena, in 1957, and a
year later took on three partners in a chain called El Taco.

After four El Tacos, Mr. Bell decided he no longer wanted to answer to
any partners. He sold out again. Then, in 1962, with a \$4,000
investment, he opened the first Taco Bell, in Downey, Calif. Over the
next two years, he started eight more Taco Bells, each with a grand
opening featuring live salsa music, searchlights and free sombreros. The
first of its franchises opened in Torrance, Calif., in 1965.

PepsiCo greatly expanded the chain after purchasing it in 1978 for about
\$125 million, then spun it off to Tricon Global Restaurants in 1997.
Tricon changed its name to Yum Brands in 2002.

Glen W. Bell Jr. was born in Lynwood, Calif., on Sept. 3, 1923, one of
six children of Glen and Ruth Johnson Bell. When he was 12, the family
moved to a small farm outside of San Bernardino.

At 16, with the family facing hard times, according to his biography,
Glen Jr. ``goes on the bum'' and ``rides the rails in search of work.''
He joined the Marines in 1943 and served in the Pacific.

Back in San Bernardino after the war, Mr. Bell bought a surplus Army
truck and began hauling adobe bricks at 5 cents each. A miniature golf
course that he leased failed to make a profit. Then, he opened a
hamburger stand in a Hispanic neighborhood.

Mr. Bell married Dorothy Taylor in 1947. They were divorced in 1953. He
is survived by his wife of 54 years, Martha; three sisters, Delores,
Dorothy and Maureen; a daughter, Kathleen; two sons, Gary and Rex; and
four grandchildren.

The trade publication Nation's Restaurant News has credited Mr. Bell
with introducing millions of Americans to Mexican-style food. ``I always
smile,'' Mr. Bell told the magazine in 2008, ``when I hear people say
that they never had a taco until Taco Bell came to town.''

Advertisement

\protect\hyperlink{after-bottom}{Continue reading the main story}

\hypertarget{site-index}{%
\subsection{Site Index}\label{site-index}}

\hypertarget{site-information-navigation}{%
\subsection{Site Information
Navigation}\label{site-information-navigation}}

\begin{itemize}
\tightlist
\item
  \href{https://help.nytimes.com/hc/en-us/articles/115014792127-Copyright-notice}{©~2020~The
  New York Times Company}
\end{itemize}

\begin{itemize}
\tightlist
\item
  \href{https://www.nytco.com/}{NYTCo}
\item
  \href{https://help.nytimes.com/hc/en-us/articles/115015385887-Contact-Us}{Contact
  Us}
\item
  \href{https://www.nytco.com/careers/}{Work with us}
\item
  \href{https://nytmediakit.com/}{Advertise}
\item
  \href{http://www.tbrandstudio.com/}{T Brand Studio}
\item
  \href{https://www.nytimes.com/privacy/cookie-policy\#how-do-i-manage-trackers}{Your
  Ad Choices}
\item
  \href{https://www.nytimes.com/privacy}{Privacy}
\item
  \href{https://help.nytimes.com/hc/en-us/articles/115014893428-Terms-of-service}{Terms
  of Service}
\item
  \href{https://help.nytimes.com/hc/en-us/articles/115014893968-Terms-of-sale}{Terms
  of Sale}
\item
  \href{https://spiderbites.nytimes.com}{Site Map}
\item
  \href{https://help.nytimes.com/hc/en-us}{Help}
\item
  \href{https://www.nytimes.com/subscription?campaignId=37WXW}{Subscriptions}
\end{itemize}
