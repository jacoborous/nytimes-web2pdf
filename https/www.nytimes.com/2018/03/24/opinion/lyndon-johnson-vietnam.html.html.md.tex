Sections

SEARCH

\protect\hyperlink{site-content}{Skip to
content}\protect\hyperlink{site-index}{Skip to site index}

\href{https://www.nytimes.com/section/opinion/sunday}{Sunday Review}

\href{https://myaccount.nytimes.com/auth/login?response_type=cookie\&client_id=vi}{}

\href{https://www.nytimes.com/section/todayspaper}{Today's Paper}

\href{/section/opinion/sunday}{Sunday Review}\textbar{}Why Lyndon
Johnson Dropped Out

\href{https://nyti.ms/2pCXjbv}{https://nyti.ms/2pCXjbv}

\begin{itemize}
\item
\item
\item
\item
\item
\item
\end{itemize}

Advertisement

\protect\hyperlink{after-top}{Continue reading the main story}

Supported by

\protect\hyperlink{after-sponsor}{Continue reading the main story}

\href{/section/opinion}{Opinion}

\href{/column/vietnam-67}{Vietnam '67}

\hypertarget{why-lyndon-johnson-dropped-out}{%
\section{Why Lyndon Johnson Dropped
Out}\label{why-lyndon-johnson-dropped-out}}

By Fredrik Logevall

\begin{itemize}
\item
  March 24, 2018
\item
  \begin{itemize}
  \item
  \item
  \item
  \item
  \item
  \item
  \end{itemize}
\end{itemize}

\includegraphics{https://static01.nyt.com/images/2018/03/25/sunday-review/25Vietnam-Logevall/25Vietnam-Logevall-articleLarge.jpg?quality=75\&auto=webp\&disable=upscale}

A half-century has passed since President Lyndon B. Johnson stunned
Americans by announcing, in a televised address on March 31, 1968, that
he was drastically reducing the bombing of North Vietnam, appealing to
the Hanoi government for negotiations and, most incredible of all,
withdrawing from the presidential election that fall. One imagines the
stupefied reaction in living rooms all across the country: ``Did he just
say what I think he said?''

Johnson did what modern American presidents are never supposed to do:
refrain from seeking re-election. (Since World War II, only Harry Truman
in 1952 has done likewise.) He feared that his health could not
withstand four more years, but what really worried him was the Vietnam
War and the divisions it had created. The war was not just a threat to
his personal legacy; it was a threat to the very foundations of the
liberal political order that he cherished so deeply and that had built
so many middle-class American dreams.

His viewers didn't know it, but Johnson had always suspected this moment
would come. From his earliest days in office, he repeatedly told his
wife, Lady Bird, and aides that he felt trapped on Vietnam, that he
would be crucified for whatever he did, that the conflict in far-off
Southeast Asia would ultimately be his downfall.

Already in May 1964, a year before he committed the country to
large-scale war, Johnson said to his national security adviser, McGeorge
Bundy: ``I don't think it's worth fighting for and I don't think we can
get out. It's just the biggest damned mess.'' A year later, shortly
before the first American ground forces set foot in Vietnam, Johnson
told Senator Richard Russell of Georgia, the chairman of the Armed
Services Committee: ``There ain't no daylight in Vietnam. There's not a
bit.''

Publicly, Johnson projected optimism. But the truth is that he was
always a bleak skeptic on Vietnam --- skeptical that it could be won,
even with American air power and ground troops, especially in view of
the weaknesses of the South Vietnamese military and government, and
skeptical that the outcome truly mattered to American and Western
security.

This attitude was reinforced by the opinions of people he valued. The
Senate Democratic leadership on foreign policy --- J. William Fulbright,
Russell and Mike Mansfield, the majority leader --- privately warned him
in 1964 and '65 against Americanizing the war. Allied leaders abroad did
the same, as did prominent voices in the press.

His own vice president, Hubert H. Humphrey, a savvy politician who
needed no reminder of the risks of ``losing'' a nation to Communism,
insisted, in a memo in mid-February 1965, that the risks of escalation
were far greater.

``If we find ourselves leading from frustration to escalation and end up
short of a war with China but embroiled deeper in fighting in Vietnam
over the next few months,'' Humphrey warned, ``political opposition will
steadily mount,'' because Americans had not been persuaded that a major
war on behalf of an ineffectual Saigon government was justified.

At the same time, no senior military leader in 1965 offered the White
House even a chance of rapid victory in Vietnam. Five years, 500,000
troops, was the general estimate Johnson heard. Where would that put the
president in early 1968, as his campaign for re-election began in
earnest? Right where he found himself as he sat down to deliver his
announcement on March 31: in a protracted war with no end in sight.

So why did he go in? Part of the answer, surely, is that escalation, if
done quietly, gradually and without putting the nation on full war
footing, offered Johnson the path of least immediate resistance (always
a tempting option for a policymaker), especially in domestic political
terms. Given his repeated public affirmations of Vietnam's importance to
American security, it made sense that he would remain steadfast, in the
hope that the new military measures would succeed, lest he face
accusations of backing down, of going soft.

More than that, Johnson made the leap because for him, ``retreat'' from
the struggle was inconceivable. He personalized the war, saw attacks on
the policy as attacks on himself, and failed to see that his landslide
victory in 1964 and the international and domestic context in early 1965
gave him considerable freedom of action --- a point Humphrey cogently
underscored in his February memo.

From Day 1 to the end, Johnson was a hawk on Vietnam, which proves again
that doubting warriors can be committed warriors. He always framed his
options in such a way that standing firm appeared the only reasonable
choice --- it was full retreat, bomb the hell out of China, or stay the
course. Never did he fully explore imaginative ways out of the conflict;
for him, extrication without victory signified humiliation and defeat.

This didn't mean Johnson rejected all talk of negotiations. After
mid-1965 he pressed Under Secretary of State George Ball for new
diplomatic ideas --- although, as Ball later remarked, ``he really meant
merely new channels and procedures.''

When negotiations with North Vietnam at last began in Paris in May 1968,
Johnson took a firm line. He also continued the bombing and indeed
increased it below the 19th Parallel and in Laos. In the 10 months from
March 1 to Dec. 31, 1968, the Pentagon dropped a greater tonnage of
bombs on Indochina than had been expended in the three years prior. This
expanded bombing, Daniel Ellsberg hauntingly concludes in his memoir of
the war, was ``obediently carried out'' by men from Secretary of Defense
Clark Clifford ``on down to flight crews, who believed it served no
national purpose whatever.''

Humphrey won the Democratic nomination in 1968, yet Johnson was
reluctant to throw his full support behind him, privately accusing the
vice president of being cowardly and disloyal whenever he expressed a
desire to move policy even modestly in a dovish direction. Some part of
Johnson preferred to see Richard Nixon as his successor, expecting, with
reason, that the Republican would be more resolute than Humphrey in
pursuing the war.

And so it ended, the saga of Lyndon Johnson's presidency, its denouement
having been anticipated by him at the beginning. He was a man who
dreamed big dreams for the Great Society, who hoped that his prodigious
efforts on civil rights, voting rights, education and Medicare would
earn him a place alongside Abraham Lincoln and Franklin Roosevelt ---
and who anticipated from the start that Vietnam would ruin it all.

It bears all the markings of tragedy, but of a certain kind, more
Shakespearean than Greek, more Macbeth than Agamemnon. Whereas for the
Greek playwrights the universe tends to be deterministic, the hero at
the mercy of forces beyond his control, for Shakespeare the tragedy lies
in the very choices the protagonist makes. His Macbeth is no mere
victim; he contributes to his own demise. The same must be said of
Lyndon Johnson.

For those seeking symbols there is, finally, this: On Jan. 22, 1973,
Johnson died at his Texas ranch, two days after hearing Nixon, in his
second inaugural address, hint at cuts to the Great Society and remind
Americans how far they had come from that bleak time in 1968, when they
faced ``the prospect of seemingly endless war abroad and of destructive
conflict at home.'' The next day, Nixon announced a deal had been
reached in Paris to end the war and ``bring peace with honor.''

Advertisement

\protect\hyperlink{after-bottom}{Continue reading the main story}

\hypertarget{site-index}{%
\subsection{Site Index}\label{site-index}}

\hypertarget{site-information-navigation}{%
\subsection{Site Information
Navigation}\label{site-information-navigation}}

\begin{itemize}
\tightlist
\item
  \href{https://help.nytimes.com/hc/en-us/articles/115014792127-Copyright-notice}{©~2020~The
  New York Times Company}
\end{itemize}

\begin{itemize}
\tightlist
\item
  \href{https://www.nytco.com/}{NYTCo}
\item
  \href{https://help.nytimes.com/hc/en-us/articles/115015385887-Contact-Us}{Contact
  Us}
\item
  \href{https://www.nytco.com/careers/}{Work with us}
\item
  \href{https://nytmediakit.com/}{Advertise}
\item
  \href{http://www.tbrandstudio.com/}{T Brand Studio}
\item
  \href{https://www.nytimes.com/privacy/cookie-policy\#how-do-i-manage-trackers}{Your
  Ad Choices}
\item
  \href{https://www.nytimes.com/privacy}{Privacy}
\item
  \href{https://help.nytimes.com/hc/en-us/articles/115014893428-Terms-of-service}{Terms
  of Service}
\item
  \href{https://help.nytimes.com/hc/en-us/articles/115014893968-Terms-of-sale}{Terms
  of Sale}
\item
  \href{https://spiderbites.nytimes.com}{Site Map}
\item
  \href{https://help.nytimes.com/hc/en-us}{Help}
\item
  \href{https://www.nytimes.com/subscription?campaignId=37WXW}{Subscriptions}
\end{itemize}
