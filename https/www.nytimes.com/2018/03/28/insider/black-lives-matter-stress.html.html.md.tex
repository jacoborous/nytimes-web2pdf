Sections

SEARCH

\protect\hyperlink{site-content}{Skip to
content}\protect\hyperlink{site-index}{Skip to site index}

\href{https://www.nytimes.com/section/reader-center}{Times Insider}

\href{https://myaccount.nytimes.com/auth/login?response_type=cookie\&client_id=vi}{}

\href{https://www.nytimes.com/section/todayspaper}{Today's Paper}

\href{/section/reader-center}{Times Insider}\textbar{}The Quiet
Casualties of the Movement for Black Lives

\href{https://nyti.ms/2uvSSnu}{https://nyti.ms/2uvSSnu}

\begin{itemize}
\item
\item
\item
\item
\item
\item
\end{itemize}

Advertisement

\protect\hyperlink{after-top}{Continue reading the main story}

Supported by

\protect\hyperlink{after-sponsor}{Continue reading the main story}

\hypertarget{the-quiet-casualties-of-the-movement-for-black-lives}{%
\section{The Quiet Casualties of the Movement for Black
Lives}\label{the-quiet-casualties-of-the-movement-for-black-lives}}

\includegraphics{https://static01.nyt.com/images/2018/03/29/pageoneplus/29a2_itt_blm/merlin_135506637_559c8343-167d-4214-9642-a4093100efb0-articleLarge.jpg?quality=75\&auto=webp\&disable=upscale}

By \href{http://www.nytimes.com/by/john-eligon}{John Eligon}

\begin{itemize}
\item
  March 28, 2018
\item
  \begin{itemize}
  \item
  \item
  \item
  \item
  \item
  \item
  \end{itemize}
\end{itemize}

I have spent plenty of time over the past few years talking with Black
Lives Matter activists about their work. The conversations are usually
about things like systems and policies, strategies for winning change
and the path forward. In these moments, just as when we see them taking
to the streets in protest, activists come off as strong and resolute,
unflappable and resilient.

But there is a quieter reality of activism: the mental and emotional
hardship of the work, and the resulting stress and depression that
sometimes make it difficult to even get out of bed.

Though this is not often talked about in the open, it is evident to
anyone paying close attention. Over the past two years at least five
prominent activists have died. Two of them were suicides. One was from a
heart attack at age 27. The other two were homicides, which speaks to
the pressures of activism, too --- the work they do often antagonizes
the police, and so many are wary of turning to the state for protection.

We decided to explore this topic after Muhiyidin Moye, an activist in
Charleston,
\href{https://www.nytimes.com/2018/02/07/us/muhiyidin-moye-dbaha-dead-black-lives-matter.html}{was
fatally shot in New Orleans} last month. What led activists to die young
and how were those deaths affecting people in the movement?

I knew right away that reporting out this story would be challenging.
Activists are often wary of sharing with mainstream news outlets,
feeling that they have been burned in the past and their messages have
been twisted. One activist even expressed concern that my article would
sow divisions within the movement. And talking about mental health is
not easy or comfortable for many people to begin with.

One of the first people to whom I reached out was Ashley Yates. I had
developed a relationship with her since her days as an activist in
Ferguson, Mo., after the police killing of Michael Brown. Ms. Yates
didn't hold back with me: She has already been open on social media
about her struggles within the movement and had a very public falling
out with its leaders.

Ms. Yates had also written about how she was affected by the
hospitalization of Erica Garner, the 27-year-old daughter of Eric
Garner.
\href{https://www.nytimes.com/2017/12/30/nyregion/erica-garner-dead.html}{Ms.
Garner had a heart attack last year and later died}; while she was in a
coma, Ms. Yates shared an image on Instagram of a text message exchange
in which Ms. Yates encouraged Ms. Garner to not be bothered by people
talking negatively about her on social media.

``I have to make clear just how invisible some of the most heinous
violence we experience is,'' Ms. Yates wrote in the Instagram post in
December. ``How we are often left alone on the front lines grown cold
because media and figureheads move on to the next hot story.''

I asked Ms. Yates, who moved to Oakland a couple of years back to work
as an activist full time, if she ever had the urge to just say forget
it, and take her college degree and go into a traditional profession. Of
course she did, she told me, especially when you see someone dropping
dead at 27 of a heart attack.

``It's absolutely scary,'' said Ms. Yates, 32. ``It's enough to make you
want to quit.''

But more than just reflecting on the difficulties of activism and the
trauma that comes with it, Ms. Yates ventured into another area that I
had not thought about: self-care.

As it turns out, taking care of yourself is a big issue in the present
movement, unlike in times past. There are trained ``healers'' in
communities who run workshops and do private counseling for activists.
Ms. Yates started seeing a therapist about a year ago. She also talked
about the things that seem small but can make a big difference for her:
going to the ocean, putting her toes in sand, remembering to eat, taking
time to talk with her friends about things that have nothing to do with
activism. Some of these might seem obvious, but for those immersed in
the work, that's not always the case.

As one reader commented on the story, young activists should temper
their expectations for immediate results ``and put their personal health
first in order to reduce their toxic levels of stress.''

Because, he added, ``stress can kill; emotionally and spiritually, as
well as physically.''

Advertisement

\protect\hyperlink{after-bottom}{Continue reading the main story}

\hypertarget{site-index}{%
\subsection{Site Index}\label{site-index}}

\hypertarget{site-information-navigation}{%
\subsection{Site Information
Navigation}\label{site-information-navigation}}

\begin{itemize}
\tightlist
\item
  \href{https://help.nytimes.com/hc/en-us/articles/115014792127-Copyright-notice}{©~2020~The
  New York Times Company}
\end{itemize}

\begin{itemize}
\tightlist
\item
  \href{https://www.nytco.com/}{NYTCo}
\item
  \href{https://help.nytimes.com/hc/en-us/articles/115015385887-Contact-Us}{Contact
  Us}
\item
  \href{https://www.nytco.com/careers/}{Work with us}
\item
  \href{https://nytmediakit.com/}{Advertise}
\item
  \href{http://www.tbrandstudio.com/}{T Brand Studio}
\item
  \href{https://www.nytimes.com/privacy/cookie-policy\#how-do-i-manage-trackers}{Your
  Ad Choices}
\item
  \href{https://www.nytimes.com/privacy}{Privacy}
\item
  \href{https://help.nytimes.com/hc/en-us/articles/115014893428-Terms-of-service}{Terms
  of Service}
\item
  \href{https://help.nytimes.com/hc/en-us/articles/115014893968-Terms-of-sale}{Terms
  of Sale}
\item
  \href{https://spiderbites.nytimes.com}{Site Map}
\item
  \href{https://help.nytimes.com/hc/en-us}{Help}
\item
  \href{https://www.nytimes.com/subscription?campaignId=37WXW}{Subscriptions}
\end{itemize}
