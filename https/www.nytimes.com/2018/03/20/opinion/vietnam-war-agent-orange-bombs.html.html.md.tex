Sections

SEARCH

\protect\hyperlink{site-content}{Skip to
content}\protect\hyperlink{site-index}{Skip to site index}

\href{https://myaccount.nytimes.com/auth/login?response_type=cookie\&client_id=vi}{}

\href{https://www.nytimes.com/section/todayspaper}{Today's Paper}

\href{/section/opinion}{Opinion}\textbar{}The Vietnam War Is Over. The
Bombs Remain.

\href{https://nyti.ms/2GKy7Hl}{https://nyti.ms/2GKy7Hl}

\begin{itemize}
\item
\item
\item
\item
\item
\item
\end{itemize}

Advertisement

\protect\hyperlink{after-top}{Continue reading the main story}

Supported by

\protect\hyperlink{after-sponsor}{Continue reading the main story}

\href{/section/opinion}{Opinion}

\href{/column/vietnam-67}{Vietnam '67}

\hypertarget{the-vietnam-war-is-over-the-bombs-remain}{%
\section{The Vietnam War Is Over. The Bombs
Remain.}\label{the-vietnam-war-is-over-the-bombs-remain}}

By Ariel Garfinkel

\begin{itemize}
\item
  March 20, 2018
\item
  \begin{itemize}
  \item
  \item
  \item
  \item
  \item
  \item
  \end{itemize}
\end{itemize}

\includegraphics{https://static01.nyt.com/images/2018/03/20/opinion/20Vietnam-garfinkel4/20Vietnam-garfinkel4-articleLarge.jpg?quality=75\&auto=webp\&disable=upscale}

Shading my eyes from the bright sun, I stared into the bomb crater amid
the verdant rice paddies. While it had been nearly 50 years since the
last American planes riddled the countryside near Danang in central
Vietnam, craters still pockmark the land. Some of the deep depressions
remain dry while others, a testament to the ingenuity of the villagers,
serve as watering holes for the oxen that farmers harness to till their
fields.

It was my first week in Vietnam, where I would spend the summer of 2016
conducting research. I was studying the efficacy of international law,
namely whether legal remedies exist for civilian victims of unexploded
ordnance and chemical weapons from the Vietnam War. I had arrived well
versed in the numbers: America dropped three times more ordnance over
Vietnam, Laos and Cambodia than all sides did during World War II.
Estimates are that at least 350,000 tons of live bombs and mines remain
in Vietnam, and that it will take 300 years to clear them from the
Vietnamese landscape at the current rate.

\includegraphics{https://static01.nyt.com/images/2018/03/20/opinion/20Vietnam-garfinkel5/20Vietnam-garfinkel5-articleLarge.jpg?quality=75\&auto=webp\&disable=upscale}

Bombs and other ordnance were dropped on thousands of villages and
hamlets. The most common were cluster bombs, each of which contained
hundreds of baseball-size bomblets; the bombs are designed to explode
near ground level, releasing metal fragments to maim and kill. But many
of the cluster bombs failed to release their contents or, in other
cases, their bomblets failed to detonate.

For the Vietnamese, the war continues. Loss of arms, legs and eyesight
are for the more fortunate ones. Others have lost their family
breadwinners, or their children. Children find baseball-size metal
objects and unwittingly toss the ``toys'' to one another in games of
catch until they explode. Nearly 40,000 Vietnamese have been killed
since the end of the war in 1975, and 67,000 maimed, by land mines,
cluster bombs and other ordnance.

Image

Parts of three American cluster bomblets displayed by a scrap dealer in
Quang Thuan.Credit...Chris Brummitt/Associated Press

That's not the only, or even the worst, legacy of the war that
Vietnamese families still face. Seeking to defoliate entire forests to
expose enemy forces to spotter planes, the Americans dropped 18 million
gallons of chemical herbicide over South Vietnam from 1962 to 1972.
There were several defoliants used, but the best known was Agent Orange.
In 20,000 spraying missions, planes drenched the countryside and an
estimated 3,181 villages.

While entire forests dried up and died typically within weeks of
spraying, it would be years before scientists established that one of
the active ingredients in the defoliants, a group of compounds called
dioxin, is one of the deadliest substances known to humankind. Just 85
grams of dioxin, if evenly distributed, could wipe out a city of eight
million people. But illnesses and deaths from Agent Orange exposure were
only the initial outcomes. Dioxin affects not only people exposed to it,
but also their children, altering DNA. Large numbers of Vietnamese
babies continue to be born with grotesque deformities: misshapen heads,
bulging tumors, underdeveloped brains and nonfunctioning limbs.

The deadly defoliants also rained down on American troops. Researchers
led by Jeanne Stellman of Columbia examined military records of the
flight paths of Agent Orange spraying missions. Comparing those flight
paths to the position of nearby villages and American ground troops
revealed a direct association between exposure and later health
problems.

\href{http://www.columbia.edu/~jms13/articles.html}{These findings},
published in 2003, put an end to the longtime denial by the government
that Agent Orange spraying did not harm American troops. The Department
of Veterans Affairs now assumes, as a blanket policy, that all of the
2.8 million troops who served in Vietnam were exposed to chemical
defoliants, and provides some medical coverage and compensation for
that. But the United States has never acknowledged that it also poisoned
millions of Vietnamese civilians in the same way.

Image

Quynh Thu, 66, with his son Pha Quoc, 21, in A Luoi, in central Vietnam,
in 2009. Mr. Thu was sprayed with Agent Orange during the
war.Credit...Kuni Takahashi/Getty Images

American combat deployments ended in 1973 and all American personnel
were removed from Vietnam by 1975, but the explosive ordnance and
dangerous chemicals remained. Polluted soil and waterways were left
untouched. Innocent children and families would serve as human guinea
pigs to test the long-term results of exposure.

The indiscriminate use of ordnance and chemical weapons against civilian
populations is prohibited under international law, dating back to the
Hague and Geneva Conventions of the late 19th and early 20th centuries.
But for more than a decade, the United States acted in direct
contravention of those agreements, which it had pledged to uphold. Since
that time, numerous additional international treaties and conventions
have come into force that not only prohibit the types of weapons used by
the United States in Vietnam, but also require their cleanup after
hostilities cease.

The United States, however, has done very little to fulfill such
obligations, leaving it largely to the Vietnamese to suffer the results
and to clean up what they can nearly 50 years later. Some have suggested
that because much of the relevant international law requiring cleanup
came into effect after the United States left Vietnam, the country is
absolved of such obligations. But this assertion hangs on a thin thread,
as the unexploded ordnance and defoliants still injure and kill people
today. American responsibility for cleanup is therefore applicable under
international law, not something to be dismissed with a historical wink.

My father's generation served in Vietnam, but the war's continuing
impact is no longer theirs alone to bear. The United States used weapons
against civilians contrary to widely accepted international standards,
and has skirted its responsibilities to clean up what was left behind.
Working to enforce international law, and to assist the Vietnamese in
addressing the deadly mess that remains, is a burden now resting on the
shoulders of a new generation of Americans.

Advertisement

\protect\hyperlink{after-bottom}{Continue reading the main story}

\hypertarget{site-index}{%
\subsection{Site Index}\label{site-index}}

\hypertarget{site-information-navigation}{%
\subsection{Site Information
Navigation}\label{site-information-navigation}}

\begin{itemize}
\tightlist
\item
  \href{https://help.nytimes.com/hc/en-us/articles/115014792127-Copyright-notice}{©~2020~The
  New York Times Company}
\end{itemize}

\begin{itemize}
\tightlist
\item
  \href{https://www.nytco.com/}{NYTCo}
\item
  \href{https://help.nytimes.com/hc/en-us/articles/115015385887-Contact-Us}{Contact
  Us}
\item
  \href{https://www.nytco.com/careers/}{Work with us}
\item
  \href{https://nytmediakit.com/}{Advertise}
\item
  \href{http://www.tbrandstudio.com/}{T Brand Studio}
\item
  \href{https://www.nytimes.com/privacy/cookie-policy\#how-do-i-manage-trackers}{Your
  Ad Choices}
\item
  \href{https://www.nytimes.com/privacy}{Privacy}
\item
  \href{https://help.nytimes.com/hc/en-us/articles/115014893428-Terms-of-service}{Terms
  of Service}
\item
  \href{https://help.nytimes.com/hc/en-us/articles/115014893968-Terms-of-sale}{Terms
  of Sale}
\item
  \href{https://spiderbites.nytimes.com}{Site Map}
\item
  \href{https://help.nytimes.com/hc/en-us}{Help}
\item
  \href{https://www.nytimes.com/subscription?campaignId=37WXW}{Subscriptions}
\end{itemize}
