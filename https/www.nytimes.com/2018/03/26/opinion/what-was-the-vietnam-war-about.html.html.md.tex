Sections

SEARCH

\protect\hyperlink{site-content}{Skip to
content}\protect\hyperlink{site-index}{Skip to site index}

\href{https://myaccount.nytimes.com/auth/login?response_type=cookie\&client_id=vi}{}

\href{https://www.nytimes.com/section/todayspaper}{Today's Paper}

\href{/section/opinion}{Opinion}\textbar{}What Was the Vietnam War
About?

\href{https://nyti.ms/2pCkPFt}{https://nyti.ms/2pCkPFt}

\begin{itemize}
\item
\item
\item
\item
\item
\item
\end{itemize}

Advertisement

\protect\hyperlink{after-top}{Continue reading the main story}

Supported by

\protect\hyperlink{after-sponsor}{Continue reading the main story}

\href{/section/opinion}{Opinion}

\href{/column/vietnam-67}{Vietnam '67}

\hypertarget{what-was-the-vietnam-war-about}{%
\section{What Was the Vietnam War
About?}\label{what-was-the-vietnam-war-about}}

By Christian G. Appy

\begin{itemize}
\item
  March 26, 2018
\item
  \begin{itemize}
  \item
  \item
  \item
  \item
  \item
  \item
  \end{itemize}
\end{itemize}

\includegraphics{https://static01.nyt.com/images/2018/03/26/opinion/26Vietnam-Appy/26Vietnam-Appy-articleLarge.jpg?quality=75\&auto=webp\&disable=upscale}

Was America's war in Vietnam a noble struggle against Communist
aggression, a tragic intervention in a civil conflict, or an imperialist
counterrevolution to crush a movement of national liberation? Those
competing interpretations ignited fiery debates in the 1960s and remain
unresolved today. How we name and define this most controversial of
American wars is not a narrow scholarly exercise, but profoundly shapes
public memory of its meaning and ongoing significance to American
national identity and foreign policy.

During the war years, America's leaders insisted that military force was
necessary to defend a sovereign nation --- South Vietnam --- from
external Communist aggression. As President Lyndon B. Johnson put it in
1965, ``The first reality is that North Vietnam has attacked the
independent nation of South Vietnam. Its object is total conquest.''

Even more disturbing, Johnson quickly added (following a script written
by his predecessors Harry Truman, Dwight Eisenhower and John F.
Kennedy), the Communists in Vietnam were supported and guided by the
Soviet Union and China. Therefore, the war in South Vietnam was not an
isolated, local conflict, irrelevant to American national security, but
rather one that was inseparable from the nation's highest priority ---
the Cold War struggle to contain Communism around the globe. Further
raising the stakes, policymakers warned that if South Vietnam fell to
Communism, neighboring countries would inevitably fall in turn, one
after another, like a row of dominoes.

Three decades later, Robert McNamara, a key architect of the Vietnam War
who served as defense secretary for both Presidents Kennedy and Johnson,
renounced those wartime claims --- the very ones he and others had
invoked to justify the war. In two books, ``In Retrospect'' (1995) and
``Argument Without End'' (2000), McNamara conceded that the United
States had been ``terribly wrong'' to intervene in Vietnam. He
attributed the failure to a lack of knowledge and judgment. If only he
had understood the fervor of Vietnamese nationalism, he wrote, if only
he had known that Hanoi was not the pawn of Beijing or Moscow, if only
he had realized that the domino theory was wrong, he might have
persuaded his presidential bosses to withdraw from Vietnam. Millions of
lives would have been saved. If only.

In fact, however, in the 1960s, when McNamara advocated massive military
escalation in Vietnam, he simply rejected or ignored any evidence that
contradicted Cold War orthodoxy. It's not as if contrary views were
unavailable. In the work of the scholar-journalist Bernard Fall, the
pages of I. F. Stone's Weekly, speeches at university teach-ins and
antiwar rallies and countless other venues, critics pointed out that
after World War II the United States made a clear choice to support the
French effort to re-establish its colonial rule in Indochina, and
eventually assumed the bulk of France's cost for the first Indochina
War. It should have been no surprise, therefore, that Vietnamese
revolutionaries perceived the United States as a neocolonial power when
it committed its own military forces in the next war.

Moreover, critics argued, the primary roots of opposition to the
American-backed government in Saigon were indigenous and deep rooted,
not just in North Vietnam, but throughout the South.

Indeed, from the late 1950s through the mid-1960s the bulk of
Communist-led fighting was carried out by southern guerrillas of the
National Liberation Front, known to its enemies as the Vietcong. Only
after the war was well underway did large units from North Vietnam
arrive on the southern front. Antiwar opponents also challenged the
claim that South Vietnam was an ``independent nation'' established by
the Geneva Accords of 1954. Those agreements called for a
\emph{temporary} partition of Vietnam to be shortly followed by a
nationwide election to choose a single leader for a unified Vietnam.
When it became clear to both Saigon and Washington that the Communist
leader Ho Chi Minh would be the overwhelming victor, the South
Vietnamese government of Ngo Dinh Diem, with American support, decided
to cancel the election.

Thus began a two-decade failed effort to build a permanent country
called ``South Vietnam.'' The government in Saigon was never a malleable
puppet of the United States, but it was nonetheless wholly dependent on
American military and economic support to survive against its enemies,
including many non-Communist parties and factions in the South.

Armed with these criticisms, many opponents of American policy in the
1960s described Vietnam as a civil war --- not like the relatively
clear-cut North-South division of the American Civil War, but a
nationwide struggle of Communist-led forces of the South and North
against the American-backed government in the South. By 1966, this
analysis was even embraced by some mainstream politicians, including
Senator William Fulbright, chairman of the Senate Foreign Relations
Committee, and Senator Eugene McCarthy, who ran as an antiwar
presidential candidate in 1968. Both men called attention to the ``South
Vietnamese civil war'' to emphasize the strength of the southern
insurgency and the failure of the Saigon government to gain the broad
support of its own people.

By 1972, the idea that Vietnam posed a threat to Cold War America was so
discredited, it sometimes sounded as if America's only remaining war aim
was to get back its P.O.W.s (President Richard Nixon bizarrely claimed
that Hanoi was using them as ``negotiating pawns''). Even more
mind-boggling were Nixon's historic 1972 trips to Beijing and Moscow.
Many Americans wondered how Nixon could offer toasts of peace to Mao
Zedong and Leonid Brezhnev while still waging war in Vietnam. As the
journalist Jonathan Schell put it, ``If these great powers were not,
after all, the true foe,'' then the war in Vietnam ``really was a civil
war in a small country, as its opponents had always said, and the United
States had no business taking part in it.''

But alongside the ``civil war'' interpretation, a more radical critique
developed --- the view that America's enemy in Vietnam was engaged in a
long-term war for national liberation and independence, first from the
French and then the United States. According to this position, the war
was best understood not as a Cold War struggle between East and West, or
a Vietnamese civil war, but as an anticolonial struggle, similar to
dozens of others that erupted throughout the Third World in the wake of
World War II. When the French were defeated by Vietnamese
revolutionaries (despite enormous American support), the United States
stepped in directly to wage a counterrevolutionary war against an enemy
determined to achieve full and final independence from foreign control.

This interpretation was shared by many on the antiwar left, including
Daniel Ellsberg, the once-hawkish defense analyst who turned so strongly
against the war that he was willing to sabotage his career by making
public 7,000 pages of classified documents about the history of the
Vietnam War, the so-called Pentagon Papers. Ellsberg made his argument
most succinctly in the 1974 documentary ``Hearts and Minds.''

``The name for a conflict in which you are opposing a revolution is
counterrevolution,'' he said. ``A war in which one side is entirely
financed and equipped and supported by foreigners is not a civil war.''
The question used to be, he added, ``might it be possible that we were
on the wrong side in the Vietnamese war. We weren't on the wrong side;
we are the wrong side.''

In the decades since 1975, all three major interpretations have
persisted. Some writers and historians have embraced President Ronald
Reagan's view that the war was a ``noble cause'' that might have been
won. That position has failed to persuade most specialists in the field,
in large part because it greatly exaggerates the military and political
virtues and success of the United States and the government of South
Vietnam. It also falls short because it depends on counterfactual claims
that victory would have been achieved if only the United States had
extended its support for Diem (instead of greenlighting his overthrow),
or tried a different military strategy, or done a better job winning
hearts and minds. However, the war as it was actually conducted by the
United States and its allies was a disaster by every measure.

In recent decades, a number of historians --- particularly younger
scholars trained in Vietnamese and other languages --- have developed
various versions of the civil war interpretation. Some of them view the
period after the French defeat in 1954 as ``post-colonial,'' a time in
which long-brewing internal conflicts between competing versions of
Vietnamese nationalism came to a head. As the historian Jessica Chapman
of Williams College puts it, ``The Vietnam War was, at its core, a civil
war greatly exacerbated by foreign intervention.'' Others have described
it as a civil war that became ``internationalized.''

While these scholars have greatly enhanced our knowledge of the
complexity and conflict in Vietnamese history, politics and culture,
they don't, in my view, assign enough responsibility to the United
States for causing and expanding the war as a neocolonial power.

Let's try a thought experiment. What if our own Civil War bore some
resemblance to the Vietnamese ``civil war''? For starters, we would have
to imagine that in 1860 a global superpower --- say Britain --- had
strongly promoted Southern secession, provided virtually all of the
funding for the ensuing war and dedicated its vast military to the
battle. We must also imagine that in every Southern state, local,
pro-Union forces took up arms against the Confederacy. Despite enormous
British support, Union forces prevailed. What would Americans call such
a war? Most, I think, would remember it as the Second War of
Independence. Perhaps African-Americans would call it the First War of
Liberation. Only former Confederates and the British might recall it as
a ``civil war.''

I would reverse Chapman's formula and say that the Vietnam War was, at
its core, an American war that exacerbated Vietnamese divisions and
internationalized the conflict. It is true, of course, that many
Vietnamese opposed the Communist path to national liberation, but no
other nationalist party or faction proved capable of gaining enough
support to hold power. Without American intervention, it is hard to
imagine that South Vietnam would have come into being or, if it did,
that it would have endured for long.

Moreover, no other foreign nation deployed millions of troops to South
Vietnam (although the United States did pressure or pay a handful of
other nations, Australia and South Korea most notably, to send smaller
military forces). And no other foreign nation or opponent dropped bombs
(eight million tons!) on South and North Vietnam, Cambodia and Laos. The
introduction of that staggering lethality was the primary driver of a
war that cost three million lives, half of them civilians.

If we continue to excuse American conduct in Vietnam as a
well-intentioned, if tragic, intervention rather than a purposeful
assertion of imperial power, we are less likely to challenge current war
managers who have again mired us in apparently endless wars based on
false or deeply misleading pretexts. Just as in the Vietnam era,
American leaders have ordered troops to distant lands based on boundless
abstractions (``the global war on terror'' instead of the global threat
of ``international Communism''). And once again, their mission is to
prop up governments that demonstrate no capacity to gain the necessary
support of their people. Once again, the United States has waged brutal
counterinsurgencies guaranteed to maim, kill or displace countless
civilians. It has exacerbated international violence and provoked
violent retaliation.

Our leaders, then and now, have insisted that the United States is ``the
greatest force for good in the world'' that wants nothing for itself,
only to defeat ``terror'' and bring peace, stability and
self-determination to other lands. The evidence does not support such a
claim. We need a new, cleareyed vision of our global conduct. A more
critical appraisal of the past is one place to start.

Advertisement

\protect\hyperlink{after-bottom}{Continue reading the main story}

\hypertarget{site-index}{%
\subsection{Site Index}\label{site-index}}

\hypertarget{site-information-navigation}{%
\subsection{Site Information
Navigation}\label{site-information-navigation}}

\begin{itemize}
\tightlist
\item
  \href{https://help.nytimes.com/hc/en-us/articles/115014792127-Copyright-notice}{©~2020~The
  New York Times Company}
\end{itemize}

\begin{itemize}
\tightlist
\item
  \href{https://www.nytco.com/}{NYTCo}
\item
  \href{https://help.nytimes.com/hc/en-us/articles/115015385887-Contact-Us}{Contact
  Us}
\item
  \href{https://www.nytco.com/careers/}{Work with us}
\item
  \href{https://nytmediakit.com/}{Advertise}
\item
  \href{http://www.tbrandstudio.com/}{T Brand Studio}
\item
  \href{https://www.nytimes.com/privacy/cookie-policy\#how-do-i-manage-trackers}{Your
  Ad Choices}
\item
  \href{https://www.nytimes.com/privacy}{Privacy}
\item
  \href{https://help.nytimes.com/hc/en-us/articles/115014893428-Terms-of-service}{Terms
  of Service}
\item
  \href{https://help.nytimes.com/hc/en-us/articles/115014893968-Terms-of-sale}{Terms
  of Sale}
\item
  \href{https://spiderbites.nytimes.com}{Site Map}
\item
  \href{https://help.nytimes.com/hc/en-us}{Help}
\item
  \href{https://www.nytimes.com/subscription?campaignId=37WXW}{Subscriptions}
\end{itemize}
