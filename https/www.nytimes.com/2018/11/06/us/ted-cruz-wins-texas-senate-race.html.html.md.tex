Sections

SEARCH

\protect\hyperlink{site-content}{Skip to
content}\protect\hyperlink{site-index}{Skip to site index}

\href{https://www.nytimes.com/section/us}{U.S.}

\href{https://myaccount.nytimes.com/auth/login?response_type=cookie\&client_id=vi}{}

\href{https://www.nytimes.com/section/todayspaper}{Today's Paper}

\href{/section/us}{U.S.}\textbar{}Ted Cruz Defeats Beto O'Rourke for
Senate in Texas

\url{https://nyti.ms/2REweQv}

\begin{itemize}
\item
\item
\item
\item
\item
\item
\end{itemize}

Advertisement

\protect\hyperlink{after-top}{Continue reading the main story}

Supported by

\protect\hyperlink{after-sponsor}{Continue reading the main story}

\hypertarget{ted-cruz-defeats-beto-orourke-for-senate-in-texas}{%
\section{Ted Cruz Defeats Beto O'Rourke for Senate in
Texas}\label{ted-cruz-defeats-beto-orourke-for-senate-in-texas}}

\includegraphics{https://static01.nyt.com/images/2018/11/07/us/politics/07texas-cruz-sub/07texas-cruz-sub-videoSixteenByNine3000.jpg}

By \href{https://www.nytimes.com/by/manny-fernandez}{Manny Fernandez}

\begin{itemize}
\item
  Nov. 6, 2018
\item
  \begin{itemize}
  \item
  \item
  \item
  \item
  \item
  \item
  \end{itemize}
\end{itemize}

EL PASO, Tex. --- Senator
\href{https://www.nytimes.com/2020/07/13/us/politics/ted-cruz-wearing-no-mask.html}{Ted
Cruz}, Republican of Texas, won re-election on Tuesday in one of the
tightest midterm races in the country, defeating the best-financed and
most popular Democrat to run in Texas in years, Representative Beto
O'Rourke.

Mr. Cruz's narrow victory did more than dash Democratic hopes that the
party could capture a Senate seat in Texas for the first time since
1988. It promised to restore Mr. Cruz's standing as a far-right force in
American politics, after many leaders in his own party questioned
whether he was likable enough to run successfully against a candidate
like Mr. O'Rourke, an El Paso congressman known for his charisma.

``This was an election about hope and about the future, and the people
of Texas rendered a verdict that we want a future with more jobs and
more security and more freedom,'' Mr. Cruz told hundreds of supporters
at a Houston hotel ballroom.

Mr. O'Rourke appeared before his supporters shortly after 10 p.m. Blocks
from the United States border with Mexico, at a minor-league baseball
stadium in his hometown, El Paso, Mr. O'Rourke stepped onto a concert
stage and said he had spoken to Mr. Cruz and congratulated him on his
victory.

\emph{{[}See the}
\href{https://www.nytimes.com/elections/results/texas-senate?action=click\&module=Intentional\&pgtype=Article}{\emph{results
for the Senate race in Texas}} \emph{and}
\href{https://www.nytimes.com/interactive/2018/11/06/us/elections/results-senate-elections.html?action=click\&module=Intentional\&pgtype=Article}{\emph{other
states}}\emph{.{]}}

``I'm as inspired, I'm as hopeful as I've ever been in my life, and
tonight's loss does nothing to diminish the way I feel about Texas or
this country,'' he said, later stepping off the stage as John Lennon's
``Imagine'' played on the loudspeakers.

Republican strategists and insiders said Mr. Cruz's narrow victory did
not mean that Democrats stood to make substantial gains in Texas.
Rather, they believed it had more to do with Mr. Cruz himself, one of
the most divisive political figures in the state, and the anti-Trump
energy of Democrats.

Pivotal factors in the tightness of the election results were the
Republican and independent voters who voted for Mr. O'Rourke but also
cast ballots for top Republicans in other races. Gov. Greg Abbott, a
Republican whose views are in line with Mr. Cruz's but whose style is
far less abrasive, easily won re-election, and a sizable number of
Republicans appeared to have split their votes for Mr. Abbott and for
Mr. O'Rourke.

``It was political nitroglycerin from the minute this campaign
started,'' said Ted Delisi, a Republican political consultant in Austin
who was Senator John Cornyn's chief campaign strategist in 2002. ``Beto
O'Rourke couldn't have run this race against John Cornyn. He couldn't
have run this race against Greg Abbott. This race had to be run against
Ted Cruz, and it had to be run this year. This was the
once-every-20-years opportunity.''

For months on the campaign trail, Mr. Cruz was more often on the ropes
than not, a surprising position for a top Republican incumbent in a
state where Democrats hold no statewide offices.

Mr. Cruz's fund-raising fell far short of his opponent's --- he raised
somewhat more than \$40 million, compared with more than \$70 million
raised by Mr. O'Rourke --- and he sometimes found himself eclipsed on
other fronts as well.

Mr. O'Rourke attracted bigger crowds at some rallies than Mr. Cruz did,
including more than 50,000 people who attended a free concert for Mr.
O'Rourke in Austin starring the country music legend Willie Nelson. Even
Mr. Cruz's rival event --- the rally President Trump attended in Houston
that drew up to 19,000 --- reinforced the belief among his critics that
Republicans were worried.

Party officials acknowledged that the Cruz campaign had underestimated
the threat posed by Mr. O'Rourke early in the race and that it was slow
to gather steam.

Against that backdrop, Mr. Cruz turned up his rhetoric, casting Mr.
O'Rourke as a pro-tax liberal who was anti-police, who favored illegal
immigrants over American citizens and who was ``running to the left of
Bernie Sanders.''

\includegraphics{https://static01.nyt.com/images/2018/11/07/us/politics/07texas-beto1/merlin_146460018_d9f92757-c2fe-4d2c-b719-e23dd79ffb7f-articleLarge.jpg?quality=75\&auto=webp\&disable=upscale}

Throughout the campaign, Mr. Cruz repeatedly had to explain his
relationship with Mr. Trump. After Mr. Cruz's bid for the Republican
presidential nomination failed in 2016, the senator declined at first to
support Mr. Trump, but later eagerly embraced him, even though Mr. Trump
had mocked and demeaned Mr. Cruz's wife and father.

His back-and-forth with the president made Mr. Cruz unappealing to many
Texans, and allowed Mr. O'Rourke to attract some votes from Republicans
--- but ultimately, not enough to change the course of the election. And
it was clear that many voters believed the narrative that Mr. Cruz had
hammered away at --- that Mr. O'Rourke was too liberal for Texas on many
issues, including the border, health care, taxes and gun rights.

Brenda Brauch, 66, a retired grandmother who voted Tuesday in an
affluent section of northwest Austin and considered herself an
independent, said she voted for Mr. Cruz, but described it as a tossup
between him and Mr. O'Rourke. What made the difference to her were the
issues of immigration and border security.

``I don't want open borders,'' she said. ``I think there should be the
right way to come into a country and the wrong way.'' She added of Mr.
O'Rourke, ``I like his enthusiasm, but I thinking he's missing some
points. The border is a real issue for me.''

In his victory speech, Mr. Cruz set aside the aggressive tone he struck
during the campaign and thanked his opponent for a hard-fought race.

``I also want to take a moment to congratulate Beto O'Rourke,'' Mr. Cruz
said. ``He poured his heart into that campaign. He worked tirelessly.''

Some in the crowd booed. ``Listen, listen,'' Mr. Cruz said. ``It's
important. He worked tirelessly. He's a dad, and he took time away from
his kids. And I want to also say, millions across this state were
inspired by his campaign. They didn't prevail, and I am grateful the
people of Texas chose a better path.''

Other Republicans were not as gracious, and they relished their
high-profile victory. James Dickey, the chairman of the Texas Republican
Party, said Mr. O'Rourke's loss echoed the defeat in 2014 of Wendy
Davis, another well-funded Democratic star who lost the governor's race
to Mr. Abbott.

``Given the track record now of Wendy Davis four years ago and
Congressman O'Rourke this time, I hope that Democrat donors from around
the country realize now that they can't buy office in Texas,'' Mr.
Dickey said.

Texas Democrats, meanwhile, saw hope in their relatively narrow margin
of defeat. They also saw progress in Mr. O'Rourke's success in bringing
first-time voters to the polls, winning support from Texans who usually
vote Republican and boosting turnout to a level that likely played a
role in Democratic victories in down-ballot races.

``There is certainly light at the end of the tunnel for Texas
Democrats,'' said Julian Castro, a Democrat who is the former mayor of
San Antonio and former secretary of housing and urban development in the
Obama administration, and who is considering running for president in
2020. ``It's something to build on.''

At times, Mr. O'Rourke's widely watched candidacy seemed to pull a page
from Mr. Cruz's own playbook.

In 2012, Mr. Cruz became a conservative rock star when he bucked the
Texas political establishment and defeated a powerful lieutenant
governor in a Republican primary runoff for the Senate seat. Mr. Cruz
was a Tea Party-backed insurgent whose grass-roots campaign captured
national attention.

In 2018, though, it was Mr. O'Rourke who played the part of grass-roots
insurgent and Mr. Cruz who represented the establishment. It was a role
Mr. Cruz seemed to relish, as he went from rally to rally warning voters
that Texas' longstanding culture and identity were under assault by the
left.

``Don't California our Texas --- you're exactly right,'' Mr. Cruz told
supporters in the East Texas city of Tyler, echoing a phrase a woman in
the audience had used. ``Whenever liberty is threatened,'' he added,
``Texans rise to the occasion.''

Advertisement

\protect\hyperlink{after-bottom}{Continue reading the main story}

\hypertarget{site-index}{%
\subsection{Site Index}\label{site-index}}

\hypertarget{site-information-navigation}{%
\subsection{Site Information
Navigation}\label{site-information-navigation}}

\begin{itemize}
\tightlist
\item
  \href{https://help.nytimes.com/hc/en-us/articles/115014792127-Copyright-notice}{©~2020~The
  New York Times Company}
\end{itemize}

\begin{itemize}
\tightlist
\item
  \href{https://www.nytco.com/}{NYTCo}
\item
  \href{https://help.nytimes.com/hc/en-us/articles/115015385887-Contact-Us}{Contact
  Us}
\item
  \href{https://www.nytco.com/careers/}{Work with us}
\item
  \href{https://nytmediakit.com/}{Advertise}
\item
  \href{http://www.tbrandstudio.com/}{T Brand Studio}
\item
  \href{https://www.nytimes.com/privacy/cookie-policy\#how-do-i-manage-trackers}{Your
  Ad Choices}
\item
  \href{https://www.nytimes.com/privacy}{Privacy}
\item
  \href{https://help.nytimes.com/hc/en-us/articles/115014893428-Terms-of-service}{Terms
  of Service}
\item
  \href{https://help.nytimes.com/hc/en-us/articles/115014893968-Terms-of-sale}{Terms
  of Sale}
\item
  \href{https://spiderbites.nytimes.com}{Site Map}
\item
  \href{https://help.nytimes.com/hc/en-us}{Help}
\item
  \href{https://www.nytimes.com/subscription?campaignId=37WXW}{Subscriptions}
\end{itemize}
