Sections

SEARCH

\protect\hyperlink{site-content}{Skip to
content}\protect\hyperlink{site-index}{Skip to site index}

\href{https://www.nytimes.com/section/politics}{Politics}

\href{https://myaccount.nytimes.com/auth/login?response_type=cookie\&client_id=vi}{}

\href{https://www.nytimes.com/section/todayspaper}{Today's Paper}

\href{/section/politics}{Politics}\textbar{}Manafort Breached Plea Deal
by Repeatedly Lying, Mueller Says

\url{https://nyti.ms/2zra1Pr}

\begin{itemize}
\item
\item
\item
\item
\item
\item
\end{itemize}

Advertisement

\protect\hyperlink{after-top}{Continue reading the main story}

Supported by

\protect\hyperlink{after-sponsor}{Continue reading the main story}

\hypertarget{manafort-breached-plea-deal-by-repeatedly-lying-mueller-says}{%
\section{Manafort Breached Plea Deal by Repeatedly Lying, Mueller
Says}\label{manafort-breached-plea-deal-by-repeatedly-lying-mueller-says}}

\includegraphics{https://static01.nyt.com/images/2018/12/01/us/27dc-manafort1/27dc-manafort1-articleLarge.jpg?quality=75\&auto=webp\&disable=upscale}

By \href{https://www.nytimes.com/by/sharon-lafraniere}{Sharon
LaFraniere}

\begin{itemize}
\item
  Nov. 26, 2018
\item
  \begin{itemize}
  \item
  \item
  \item
  \item
  \item
  \item
  \end{itemize}
\end{itemize}

WASHINGTON --- Paul Manafort, President Trump's former campaign
chairman, repeatedly lied to federal investigators in breach of a plea
agreement he signed two months ago, the special counsel's office said in
\href{https://int.nyt.com/data/documenthelper/491-paul-manafort-plea-agreement-breach/49c0d15e0a829872d8c9/optimized/full.pdf\#page=1}{a
court filing} late on Monday.

Prosecutors working for the special counsel, Robert S. Mueller III, said
Mr. Manafort's ``crimes and lies'' about ``a variety of subject
matters'' relieve them of all promises they made to him in the plea
agreement. But under the terms of the agreement, Mr. Manafort cannot
withdraw his guilty plea.

Defense lawyers disagreed that Mr. Manafort had violated the deal. In
the same filing, they said Mr. Manafort had met repeatedly with the
special counsel's office and ``believes he has provided truthful
information.''

But given the impasse between the two sides, they asked Judge Amy Berman
Jackson of the United States District Court for the District of Columbia
to set a sentencing date for Mr. Manafort, who has been in solitary
confinement in a detention center in Alexandria, Va.

The 11th-hour development in Mr. Manafort's case is a fresh sign of the
special counsel's aggressive approach in investigating Russia's
interference in the 2016 presidential race and whether anyone in the
Trump campaign knew about or assisted Moscow's effort.

Striking a plea deal with Mr. Manafort in September potentially gave
prosecutors access to information that could prove useful to their
investigation. But their filing on Monday, a rare step in a plea deal,
suggested that they thought Mr. Manafort was withholding details that
could be pertinent to the Russia inquiry or other cases.

The question of whether
\href{https://www.nytimes.com/2018/03/28/us/politics/trump-pardon-michael-flynn-paul-manafort-john-dowd.html}{Mr.
Trump might pardon Mr. Manafort} for his crimes has loomed over his case
since he was first indicted a year ago and has lingered as a
possibility. A former lawyer for Mr. Trump broached the prospect of a
pardon with one of Mr. Manafort's lawyers last year, raising questions
about whether he was trying to influence Mr. Manafort's decision about
whether to cooperate with investigators.

The filing Monday suggested that prosecutors do not consider Mr.
Manafort a credible witness. Even if he has provided information that
helps them develop criminal cases, by asserting that he repeatedly lied,
they could hardly call him to testify.

Mr. Manafort had hoped that in agreeing to cooperate with Mr. Mueller's
team, prosecutors would argue that he deserved a lighter punishment. He
is expected to face at least a decade-long prison term for 10 felony
counts including financial fraud and conspiracy to obstruct justice.

Instead, after at least a dozen sessions interrogating him, the special
counsel's prosecutors have not only decided Mr. Manafort does not
deserve leniency, but they also could seek to refile other charges that
they had agreed to dismiss as part of the plea deal.

The prosecutors did not describe what they said Mr. Manafort lied about,
saying they would set forth ``the nature of the defendant's crimes and
lies'' in an upcoming sentencing memo. The sentencing judge does not
have to accept the prosecution's account at face value, and Mr.
Manafort's lawyers are expected to vigorously contest it.

\includegraphics{https://static01.nyt.com/images/2018/08/22/us/politics/22dc-manafortverdict/17dc-manafortverdict-videoSixteenByNine3000.jpg}

A jury in Northern Virginia
\href{https://www.nytimes.com/2018/08/21/us/politics/paul-manafort-trial-verdict.html}{convicted}
Mr. Manafort, 69, of eight counts of financial fraud in August stemming
from his work as a political consultant in Ukraine. The jury deadlocked
on 10 other charges.

Faced with a second trial in the District of Columbia on related charges
in September, he
\href{https://www.nytimes.com/2018/09/14/us/politics/manafort-plea-deal.html}{pleaded
guilty} to two conspiracy counts and agreed to an open-ended arrangement
requiring him to answer ``fully, truthfully, completely and
forthrightly'' questions about ``any and all matters'' of interest to
the government.

It was unclear at that time how much Mr. Manafort had to offer
prosecutors. Although he had arguably deeper ties to pro-Russian figures
than anyone else connected with the Trump campaign, he had consistently
said he had no information against the president. Legal experts
suggested that if he had been able to significantly further Mr.
Mueller's inquiry, he could have negotiated a more favorable deal.

As it is, the plea agreement specifies that if prosecutors decide that
Mr. Manafort has failed to cooperate fully or ``given false, misleading
or incomplete information or testimony,'' they can prosecute him for
crimes to which he did not plead guilty in the District of Columbia.
They could also conceivably pursue the 10 charges on which the Virginia
jury failed to reach a consensus. Mr. Manafort is scheduled to be
sentenced in the Virginia case on Feb. 8.

Mr. Mueller's investigators have charged a number of former aides to Mr.
Trump with lying to them. Three former Trump campaign officials or
advisers have pleaded guilty to misleading federal investigators:
\href{https://www.nytimes.com/2017/12/01/us/politics/michael-flynn-guilty-russia-investigation.html}{Michael
T. Flynn},
\href{https://www.nytimes.com/2018/02/23/us/politics/rick-gates-guilty-plea-mueller-investigation.html}{Rick
Gates} and
\href{https://www.nytimes.com/2018/09/07/us/politics/george-papadopoulos-sentencing-special-counsel-investigation.html}{George
Papadopoulos}, who reported to prison on Monday to serve his 14-day
sentence. A Dutch lawyer,
\href{https://www.nytimes.com/2018/04/03/us/alex-van-der-zwaan-sentencing-russia-investigation-mueller.html}{Alex
van der Zwaan}, who had business dealings with Mr. Manafort, also
pleaded guilty to lying to the special counsel's office.

Most recently, Jerome Corsi, a conservative author and friend of the
former Trump campaign aide Roger J. Stone Jr., said Mr. Mueller's team
is
\href{https://www.nytimes.com/2018/11/23/us/politics/jerome-corsi-plea.html}{pressuring
him to plead guilty} to lying to them about his communications with Mr.
Stone about WikiLeaks. Investigators are looking for links between the
Trump campaign and WikiLeaks, which distributed emails and other
documents that Russian agents stole from Democratic computers before the
2016 election.

Mr. Corsi said on Monday that he refused the plea deal because he did
not deliberately mislead investigators, but merely forgot about an email
chain from two and a half years ago.

In his most recent criticism of the special counsel, Mr. Trump has
suggested that prosecutors are frustrated because they cannot produce
any evidence against his campaign. ``The inner workings of the Mueller
investigation are a total mess,'' he
\href{https://twitter.com/realdonaldtrump/status/1063042585802039296}{wrote
on Twitter} recently.

``They have found no collusion and have gone absolutely nuts. They are
screaming and shouting at people, horribly threatening them to come up
with the answers they want,'' he declared. ``They are a disgrace to our
Nation and don't care how many lives'' they ruin.

Mr. Trump's lawyers
\href{https://www.nytimes.com/2018/11/20/us/politics/trump-mueller-questions-answers.html}{responded
last week} to questions Mr. Mueller had for the president about ties
between his campaign and Russia. Among the questions were inquiries
about what Mr. Trump knew about Russian offers to Mr. Manafort during
the campaign to assist Mr. Trump's presidential run. The president's
lawyers have declined to discuss what he told Mr. Mueller, and it is not
clear whether any of his answers conflicted with what Mr. Manafort told
investigators.

Mr. Manafort's allies have hoped that his sessions with the special
counsel would end soon so he could be sentenced and transferred to a
federal prison, where conditions are comparatively better than in a
local jail. At a recent court hearing in Alexandria, Mr. Manafort came
into the courtroom in a wheelchair, his foot wrapped in a white bandage,
possibly from an attack of gout.

But few of Mr. Manafort's friends predicted that his sentencing would be
hastened by prosecutors declaring him to be a liar. The development
stunned some people close to the White House, as well as legal experts.

``Everybody who lies to Mueller gets called on it --- so he had to know
that Mueller would catch him. So the question is: What was he hiding
that is worse than going to jail for the rest of your life?'' said Joyce
Vance, a professor of law at the University of Alabama law school and
former federal prosecutor. ``There are often rocky dealings with a
cooperator, and Mueller didn't cut bait at the first sign of trouble. It
was likely more than one lie and this would not have been a minor detail
--- it had to be something material and significant and intentional.''

Advertisement

\protect\hyperlink{after-bottom}{Continue reading the main story}

\hypertarget{site-index}{%
\subsection{Site Index}\label{site-index}}

\hypertarget{site-information-navigation}{%
\subsection{Site Information
Navigation}\label{site-information-navigation}}

\begin{itemize}
\tightlist
\item
  \href{https://help.nytimes.com/hc/en-us/articles/115014792127-Copyright-notice}{©~2020~The
  New York Times Company}
\end{itemize}

\begin{itemize}
\tightlist
\item
  \href{https://www.nytco.com/}{NYTCo}
\item
  \href{https://help.nytimes.com/hc/en-us/articles/115015385887-Contact-Us}{Contact
  Us}
\item
  \href{https://www.nytco.com/careers/}{Work with us}
\item
  \href{https://nytmediakit.com/}{Advertise}
\item
  \href{http://www.tbrandstudio.com/}{T Brand Studio}
\item
  \href{https://www.nytimes.com/privacy/cookie-policy\#how-do-i-manage-trackers}{Your
  Ad Choices}
\item
  \href{https://www.nytimes.com/privacy}{Privacy}
\item
  \href{https://help.nytimes.com/hc/en-us/articles/115014893428-Terms-of-service}{Terms
  of Service}
\item
  \href{https://help.nytimes.com/hc/en-us/articles/115014893968-Terms-of-sale}{Terms
  of Sale}
\item
  \href{https://spiderbites.nytimes.com}{Site Map}
\item
  \href{https://help.nytimes.com/hc/en-us}{Help}
\item
  \href{https://www.nytimes.com/subscription?campaignId=37WXW}{Subscriptions}
\end{itemize}
