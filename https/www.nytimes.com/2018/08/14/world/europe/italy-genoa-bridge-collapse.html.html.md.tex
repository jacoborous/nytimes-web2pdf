Sections

SEARCH

\protect\hyperlink{site-content}{Skip to
content}\protect\hyperlink{site-index}{Skip to site index}

\href{https://www.nytimes.com/section/world/europe}{Europe}

\href{https://myaccount.nytimes.com/auth/login?response_type=cookie\&client_id=vi}{}

\href{https://www.nytimes.com/section/todayspaper}{Today's Paper}

\href{/section/world/europe}{Europe}\textbar{}Italy Bridge Collapse
Leaves 37 Dead

\url{https://nyti.ms/2MlZ1u9}

\begin{itemize}
\item
\item
\item
\item
\item
\item
\end{itemize}

Advertisement

\protect\hyperlink{after-top}{Continue reading the main story}

Supported by

\protect\hyperlink{after-sponsor}{Continue reading the main story}

\hypertarget{italy-bridge-collapse-leaves-37-dead}{%
\section{Italy Bridge Collapse Leaves 37
Dead}\label{italy-bridge-collapse-leaves-37-dead}}

\includegraphics{https://static01.nyt.com/images/2018/08/15/world/15Genoa10/15Genoa10-videoSixteenByNineJumbo1600-v3.jpg}

By \href{https://www.nytimes.com/by/gaia-pianigiani}{Gaia Pianigiani},
\href{https://www.nytimes.com/by/elisabetta-povoledo}{Elisabetta
Povoledo} and
\href{https://www.nytimes.com/by/richard-perez-pena}{Richard Pérez-Peña}

\begin{itemize}
\item
  Aug. 14, 2018
\item
  \begin{itemize}
  \item
  \item
  \item
  \item
  \item
  \item
  \end{itemize}
\end{itemize}

\href{https://www.nytimes.com/es/2018/08/14/genova-puente-italia}{Leer
en español}

GENOA, Italy --- A bridge in the heart of Genoa collapsed on Tuesday,
killing 37 people as dozens of vehicles and tons of concrete and steel
plunged onto city streets below in a disaster that prompted an emergency
review of Italy's aging infrastructure.

Adriano Scimpioni said he had just finished his shift at the city's
garbage collection company when ``I heard something like an explosion
and a screech of iron, and in a second we were all covered by a cloud of
dust.''

Mr. Scimpioni went outside, and with his smartphone he started
\href{https://www.facebook.com/adriano.scipioni/videos/1637149753056595/}{broadcasting
the scene} on Facebook.

``I immediately thought of the victims; I held my tears, with
difficulty,'' he said. ``We were impotent, we didn't have the means to
help those who remained underneath.'' He said two of his colleagues had
been killed.

\hypertarget{before-and-after-the-bridge-collapse}{%
\subsection{Before and After the Bridge
Collapse}\label{before-and-after-the-bridge-collapse}}

The cable-stayed Morandi Bridge in Genoa, Italy, collapsed on Tuesday.
This bridge uses very few stays, which were constructed from
pre-stressed concrete instead of steel cables. The collapse occurred at
one of the trestles, a vertical framework of upside-down V's used for
support.

CONCRETE STAY

TRESTLE

CONCRETE STAY

TRESTLE

CONCRETE STAY

TRESTLE

By The New York Times. Before photo by Davide Papalini; After photo by
Reuters.

Long before the collapse,
\href{https://www.nytimes.com/2018/08/15/world/europe/italy-genoa-bridge-collapse.html?action=click\&module=Intentional\&pgtype=Article}{experts
warned that the structure was deteriorating} and possibly dangerous.

The highway bridge fell by as much as 148 feet just before noon, taking
about three dozen cars and three trucks with it, according to Angelo
Borrelli, chief of the Civil Protection Department.

The calamity presented a serious test of Italy's new government, a
coalition of populist parties that was
\href{https://www.nytimes.com/2018/05/23/world/europe/italy-european-union.html}{formed
only months ago}. They rode to power on a wave of anti-establishment
anger, but the coalition is led by people who had little to no
experience governing, and now must demonstrate that they can manage a
crisis and the scrutiny that comes with it.

Giuseppe Conte, who took over as prime minister in June, traveled to the
scene of the disaster. The office of the city's public prosecutor said
it would open a criminal inquiry, examining whether the collapse was
because of negligence, and federal authorities indicated that they, too,
would investigate.

\includegraphics{https://static01.nyt.com/images/2018/08/15/world/15genoa5/15genoa5-articleLarge-v2.jpg?quality=75\&auto=webp\&disable=upscale}

Mr. Conte's deputy prime minister and interior minister, Matteo Salvini,
who is a powerful populist skeptic of the European Union, said that the
bloc's budget rules might have to be lifted so Italy could overhaul its
infrastructure. ``When they tell us that we can't spend our money
because some E.U. constraints don't permit it, I wonder whether these
constraints make any sense,'' he said.

Danilo Toninelli, the transportation minister, said the government had
ordered a comprehensive safety review of Italy's infrastructure ---
including viaducts and bridges built in the 1950s and 1960s during the
country's postwar boom. Much of the infrastructure, he warned, was
showing signs of age. ``This is what Italy needs to do, now,'' he said.

In a phone interview, Antonio Occhiuzzi, director of the Institute for
Construction Technology for Italy's National Research Council, said that
most of the nation's infrastructure ``needs to be carefully
re-examined'' because much ``was at the end'' of its useful life span.
He added, ``In some cases it can be reinforced, in other cases it will
have to be demolished and rebuilt completely.''

Mr. Occhiuzzi said that civil and structural engineering projects in
Italy were complicated by a national patrimony that dates to antiquity.
``Our position, typically, is to try and conserve rather than demolish
and rebuild, as happens in other countries,'' he said.

The collapse of the Genoese bridge was a serious alarm, he said,
``because it is not an isolated case --- in the last three years, a
number of bridges have collapsed in Italy, and they are all around 50
years old.''

Francesco Faccini, a lecturer at the University of Genoa and an expert
on floods and landslides, said ``significant urbanization'' of the
city's flood plains along with, at times, irrational urbanization, led
to channels being narrowed and streams diverted. The result, he said,
was a situation where flooding became more common, especially as the
intensity of downpours has grown in recent years.

In the Sampierdarena neighborhood under the fallen bridge, gas and
electricity were shut off as workers scrambled to check on the security
of buildings. The electricity was later restored.

Giovanna Santacroce, a 25-year-old ice cream vendor who lives a few
hundred yards from the bridge, said she heard what sounded like thunder.
``Then the light went off for a second and I peeked out of the window
and couldn't see the bridge anymore,'' she said. ``On the street people
were crying, debris was everywhere. They ordered us to leave the house,
and we don't know when we'll be able to return.''

Pietro Marzullo, a 31-year-old elder care attendant who was driving in
the area when the disaster occurred, said he helped a woman in shock get
away moments later. ``Small concrete blocks were scattered everywhere,
rolling down the street and around us,'' he said.

Image

More than 240 firefighters were involved in the efforts to remove
survivors from the wreckage.Credit...Stefano Rellandini/Reuters

The national police noted that there was a ``violent cloudburst'' about
the time the section fell, and some witnesses said the bridge was struck
by lightning just before it collapsed, though it was not clear whether
the storm contributed to the failure.

Edoardo Rixi, the deputy transportation minister, said that
\href{https://www.nytimes.com/2018/08/15/world/europe/italy-genoa-bridge-collapse.html}{the
bridge had shown some ``signs of problems'' in the past}. ``A bridge
like that doesn't collapse because of a lightning bolt, or a storm,'' he
said. ``Those who are responsible must be found.''

Mr. Rixi said the entire bridge would be demolished, ``with serious
repercussions on traffic, and problems for citizens and companies.''

Image

Rescuers working among the debris.Credit...Luca Zennaro/ANSA, via
Associated Press

Survivors marveled that the toll was not higher.

``I saw the bridge collapse in front of my eyes; the debris from the
collapse landed 20 meters away from my car,'' Davide Ricci, who was
driving nearby along the bank of the Polcevera River,
\href{http://www.ilsecoloxix.it/p/genova/2018/08/14/ADAu8O9-crollo_morandi_testimonianze.shtml}{told
the Genoese newspaper Il Secolo XIX}. ``It felt as if electricity was
traveling from above downward, as if one of the beams had been struck by
lightning.''

One motorist, Corrado Cusano, told the Italian channel Rai News that he
had just driven across the bridge when ``I turned around and saw the
bridge collapsing.''

``We were stuck in traffic,'' he said. ``There was a long line of cars,
and I only saw the bridge collapsing behind me and then I couldn't see
any cars anymore.''

Image

The office of the city's public prosecutor said it would open a criminal
inquiry, examining whether the collapse was due to
negligence.Credit...Massimo Pinca/Reuters

A stretch of the road, estimated by some witnesses to be more than 300
feet long, fell to earth in a huge dust cloud. Remarkably, some people
driving on the bridge walked away unharmed.

``I am a miracle,'' one of them, Davide Capello,
\href{http://www.lastampa.it/2018/08/14/italia/il-testimone-sono-andato-gi-con-il-ponte-non-so-che-cosa-mi-ha-salvato-aTLxiOyVRug7md4noz1t9I/pagina.html}{told
the Turin newspaper La Stampa}. He said he was driving to the city when
``I heard a noise first and everything collapsed.'' His car fell and
came to rest wedged between massive pieces of debris, yet he was unhurt.

Cars and trucks were abandoned on the road, just short of the new chasm;
their occupants had fled on foot.

Image

People watching the rescue operations.Credit...Andrea Leoni/Agence
France-Presse --- Getty Images

Jagged chunks of debris the size of houses protruded at sharp angles
from a riverbed, and huge pieces of the road, still partly propped up by
support pillars, were left dangling. The scene was strewn with cars and
trucks that had been \href{https://twitter.com/MFerraglioni}{damaged in
the collapse}, while rescue workers struggled in a driving rain to reach
anyone who might be trapped.

Firefighters with rescue baskets were hoisted by cranes onto the piled
wreckage to pull people from the rubble. Others loaded the injured into
paramedic helicopters, or picked their way across the mounds of debris
with dogs trained to sniff and listen for people.

Still
\href{https://twitter.com/emergenzavvf/status/1029361007691526151}{others
used power tools} to cut their way through blocks of concrete and steel
reinforcement to reach victims. In all, more than 240 firefighters took
part in the rescue efforts, officials said.

Image

The segment that fell loomed over multistory structures.Credit...Stefano
Rellandini/Reuters

``I drive on that bridge four times a day and I know that they do
maintenance work on that bridge every single night,'' said Edoardo
Serra, a pharmacist in Genoa who was to return to the city center an
hour after the bridge collapsed. ``They normally closed one direction to
traffic and allowed double circulation on the other.''

The span is known as the Morandi Bridge, for the engineer who designed
it, \href{https://www.archinform.net/arch/1096.htm}{Riccardo Morandi},
who died in 1989. Like many of his signature structures, the bridge,
which opened in 1967, was built of reinforced concrete. Local news media
reported that it had been under repair, though the nature of that work
was not immediately clear.

``It was a bold structure, designed for a volume of traffic that was
much lower than it is today,'' said Diego Zoppi, a Genoese architect.
The bridge, he said, required constant maintenance because it was an
essential traffic hub for the city, connecting Genoa's eastern and
western suburbs with highways to major cities in the north, like Milan.

Image

Vittorio Altamonte, who witnessed the collapse of the bridge on Tuesday,
took these photos of the aftermath.Credit...Vittorio Altamonte

Mr. Morandi designed several major bridges using similar designs and
materials, primarily in Italy but also in Libya, Colombia, Ecuador and
elsewhere.

Giacomo Giampedrone, a councilor for civil protection and infrastructure
in Liguria, the region that includes Genoa, said in a telephone
interview that the bridge and its maintenance were the responsibility of
\href{https://www.autostrade.it/en/home}{Autostrade per l'Italia}, a
highway operator controlled by Atlantia, an Italian holding company
whose main shareholder is the Benetton family.

Autostrade per l'Italia and Atlantia did not respond to emails
requesting comment, and their media offices were closed on Tuesday.
Atlantia shares fell by 5.39 percent on the Milan stock market on
Tuesday.

Autostrade's director for the area of Genoa, Stefano Marigliani, told
the Italian news agency ANSA that the collapse of the bridge ``was
unexpected and sudden with respect to the monitoring that the bridge was
subject to.''

``There was no forewarning,'' he said.

The bridge carried the A10 highway --- a major east-west artery through
the port city that extends west along the coastline to the border with
France --- across the Polcevera River. It was one of two just major
highway bridges that span that waterway, in the western part of the
city, and officials warned that the loss of the crossing could
significantly affect the movement of people and goods.

The impact of the collapse ``will be devastating for the local
economy,'' and the city would be paralyzed for years, said Enrico Musso,
an economics professor at the University of Genoa, and a center-right
politician. A long contested project to build a new highway had only
recently been approved after years of fierce debate, in part because of
concerns about the Morandi Bridge had intensified. ``We knew it was not
eternal,'' Mr. Musso said.

As night fell, the only audible sound was that of rescue workers
drilling through the concrete blocks to look for more vehicles buried
under the rubble, illuminated by large rescue lights.

Advertisement

\protect\hyperlink{after-bottom}{Continue reading the main story}

\hypertarget{site-index}{%
\subsection{Site Index}\label{site-index}}

\hypertarget{site-information-navigation}{%
\subsection{Site Information
Navigation}\label{site-information-navigation}}

\begin{itemize}
\tightlist
\item
  \href{https://help.nytimes.com/hc/en-us/articles/115014792127-Copyright-notice}{©~2020~The
  New York Times Company}
\end{itemize}

\begin{itemize}
\tightlist
\item
  \href{https://www.nytco.com/}{NYTCo}
\item
  \href{https://help.nytimes.com/hc/en-us/articles/115015385887-Contact-Us}{Contact
  Us}
\item
  \href{https://www.nytco.com/careers/}{Work with us}
\item
  \href{https://nytmediakit.com/}{Advertise}
\item
  \href{http://www.tbrandstudio.com/}{T Brand Studio}
\item
  \href{https://www.nytimes.com/privacy/cookie-policy\#how-do-i-manage-trackers}{Your
  Ad Choices}
\item
  \href{https://www.nytimes.com/privacy}{Privacy}
\item
  \href{https://help.nytimes.com/hc/en-us/articles/115014893428-Terms-of-service}{Terms
  of Service}
\item
  \href{https://help.nytimes.com/hc/en-us/articles/115014893968-Terms-of-sale}{Terms
  of Sale}
\item
  \href{https://spiderbites.nytimes.com}{Site Map}
\item
  \href{https://help.nytimes.com/hc/en-us}{Help}
\item
  \href{https://www.nytimes.com/subscription?campaignId=37WXW}{Subscriptions}
\end{itemize}
