Sections

SEARCH

\protect\hyperlink{site-content}{Skip to
content}\protect\hyperlink{site-index}{Skip to site index}

\href{https://www.nytimes.com/section/technology}{Technology}

\href{https://myaccount.nytimes.com/auth/login?response_type=cookie\&client_id=vi}{}

\href{https://www.nytimes.com/section/todayspaper}{Today's Paper}

\href{/section/technology}{Technology}\textbar{}California Passes
Sweeping Law to Protect Online Privacy

\url{https://nyti.ms/2lEdwdX}

\begin{itemize}
\item
\item
\item
\item
\item
\end{itemize}

Advertisement

\protect\hyperlink{after-top}{Continue reading the main story}

Supported by

\protect\hyperlink{after-sponsor}{Continue reading the main story}

\hypertarget{california-passes-sweeping-law-to-protect-online-privacy}{%
\section{California Passes Sweeping Law to Protect Online
Privacy}\label{california-passes-sweeping-law-to-protect-online-privacy}}

\includegraphics{https://static01.nyt.com/images/2018/06/29/business/29CAPRIVACY-dress/29CAPRIVACY-articleLarge.jpg?quality=75\&auto=webp\&disable=upscale}

By \href{https://www.nytimes.com/by/daisuke-wakabayashi}{Daisuke
Wakabayashi}

\begin{itemize}
\item
  June 28, 2018
\item
  \begin{itemize}
  \item
  \item
  \item
  \item
  \item
  \end{itemize}
\end{itemize}

SAN FRANCISCO --- California has passed a digital privacy law granting
consumers more control over and insight into the spread of their
personal information online, creating one of the most significant
regulations overseeing the data-collection practices of technology
companies in the United States.

The bill raced through the State Legislature without opposition on
Thursday and was signed into law by Gov. Jerry Brown, just hours before
a deadline to pull from the November ballot an initiative seeking even
tougher oversight over technology companies.

The new law grants consumers the right to know what information
companies are collecting about them, why they are collecting that data
and with whom they are sharing it. It gives consumers the right to tell
companies to delete their information as well as to not sell or share
their data. Businesses must still give consumers who opt out the same
quality of service.

It also makes it more difficult to share or sell data on children
younger than 16.

The legislation, which goes into effect in January 2020, makes it easier
for consumers to sue companies after a data breach. And it gives the
state's attorney general more authority to fine companies that don't
adhere to the new regulations.

The California law is not as expansive as
\href{https://www.nytimes.com/2018/05/06/technology/gdpr-european-privacy-law.html?action=click\&module=RelatedCoverage\&pgtype=Article\&region=Footer}{Europe's
General Data Protection Regulation, or G.D.P.R.}, a new set of laws
restricting how tech companies collect, store and use personal data.

But Aleecia M. McDonald, an incoming assistant professor at Carnegie
Mellon University who specializes in privacy policy, said California's
privacy measure was one of the most comprehensive in the United States,
since most existing laws --- and there are not many --- do little to
limit what companies can do with consumer information.

``It's a step forward, and it should be appreciated as a step forward
when it's been a long time since there were any steps,'' Ms. McDonald
said.

The legislation is modeled closely on the ballot initiative, which a
\href{https://www.nytimes.com/2018/05/13/business/california-data-privacy-ballot-measure.html}{real
estate developer, Alastair Mactaggart}, spent \$3 million and secured
more than 600,000 signatures to get certified. With the ballot proposal
hanging over legislators' heads, the push for an alternative gained
grudging support.

If the bill had failed to pass before the deadline, the proponents of
the ballot initiative would have taken their case straight to voters in
November, they said.

The state's technology and business lobbies were opposed to the measure
that was passed on Thursday, but they didn't try to derail it because
they thought the ballot initiative was worse.

Even legislators who voted for the bill complained that they had little
choice because a ballot measure would provide less flexibility to make
changes in the future. And some privacy advocates said the bill did not
go as far as the ballot initiative in allowing individuals to sue for
not complying.

\includegraphics{https://static01.nyt.com/images/2018/06/28/business/29CAPRIVACY-2/merlin_138088395_a5158ec2-025f-4f7e-9b8a-19f77805e86c-articleLarge.jpg?quality=75\&auto=webp\&disable=upscale}

Mr. Mactaggart said he wanted a sensible privacy law, whether through a
ballot measure or the legislative process. He said that the Legislature
was the right place to debate such a policy, but that it had been hard
to get legislators to address privacy.

``If we didn't have the initiative process in California, we wouldn't be
here today,'' Mr. Mactaggart said in an interview.

One of the authors of the new law, Assemblyman Ed Chau, a Democrat,
tried last year to pass a bill that would have required internet service
providers to seek permission from customers before accessing, selling or
sharing their browser activity. The bill never made it out of committee
--- an example of the influence of telecommunications and technology
companies in California.

But with the ballot measure looming and a growing awareness of how
technology companies are gobbling up user information --- highlighted by
revelations that the voter profiling firm Cambridge Analytica gained
access to the
\href{https://www.nytimes.com/2018/04/13/technology/facebook-silicon-valley.html?action=click\&module=RelatedCoverage\&pgtype=Article\&region=Footer}{personal
data of millions of Facebook users} --- the legislation went from draft
to law in one week.

``This is a huge step forward to people all across the country dealing
with this very challenging issue,'' State Senator Bob Hertzberg, a
Democrat and a co-author of the bill, said at a news conference after it
was signed.

The ballot initiative, which would have made it easier for private
individuals to sue companies for not adhering to its privacy
requirements, had drawn vocal opposition from industry groups that
worried about the potential liability risk.

The measure included a provision that would have required a 70 percent
majority in both houses of the Legislature to approve any changes after
it became law.

Google, Facebook, Verizon, Comcast and AT\&T each contributed \$200,000
to a committee opposing the proposed ballot measure, and lobbyists had
estimated that businesses would spend \$100 million to campaign against
it before the November election.

Robert Callahan, a vice president of state government affairs for the
Internet Association, an industry group that includes Google, Facebook
and Amazon, said in a statement that the new law contained many
``problematic provisions.'' But the group did not try to obstruct it, he
added, because ``it prevents the even worse ballot initiative from
becoming law in California.''

Mr. Callahan said the group would ``work to correct the inevitable,
negative policy and compliance ramifications this last-minute deal will
create.''

Legislators said they expected to pass ``cleanup bills'' to make any
fixes to the law in the 18 months before it takes effect. Some privacy
advocates are worried that lobbyists for business and technology groups
will use that time to water it down.

Mr. Mactaggart said those concerns are ``overblown.''

``Having gotten this right, it'll be very hard to take it away,'' he
said, noting that the ballot measure had been polling at around 80
percent approval. ``They can't rewrite the law.''

Advertisement

\protect\hyperlink{after-bottom}{Continue reading the main story}

\hypertarget{site-index}{%
\subsection{Site Index}\label{site-index}}

\hypertarget{site-information-navigation}{%
\subsection{Site Information
Navigation}\label{site-information-navigation}}

\begin{itemize}
\tightlist
\item
  \href{https://help.nytimes.com/hc/en-us/articles/115014792127-Copyright-notice}{©~2020~The
  New York Times Company}
\end{itemize}

\begin{itemize}
\tightlist
\item
  \href{https://www.nytco.com/}{NYTCo}
\item
  \href{https://help.nytimes.com/hc/en-us/articles/115015385887-Contact-Us}{Contact
  Us}
\item
  \href{https://www.nytco.com/careers/}{Work with us}
\item
  \href{https://nytmediakit.com/}{Advertise}
\item
  \href{http://www.tbrandstudio.com/}{T Brand Studio}
\item
  \href{https://www.nytimes.com/privacy/cookie-policy\#how-do-i-manage-trackers}{Your
  Ad Choices}
\item
  \href{https://www.nytimes.com/privacy}{Privacy}
\item
  \href{https://help.nytimes.com/hc/en-us/articles/115014893428-Terms-of-service}{Terms
  of Service}
\item
  \href{https://help.nytimes.com/hc/en-us/articles/115014893968-Terms-of-sale}{Terms
  of Sale}
\item
  \href{https://spiderbites.nytimes.com}{Site Map}
\item
  \href{https://help.nytimes.com/hc/en-us}{Help}
\item
  \href{https://www.nytimes.com/subscription?campaignId=37WXW}{Subscriptions}
\end{itemize}
