Sections

SEARCH

\protect\hyperlink{site-content}{Skip to
content}\protect\hyperlink{site-index}{Skip to site index}

\href{https://www.nytimes.com/section/books/review}{Book Review}

\href{https://myaccount.nytimes.com/auth/login?response_type=cookie\&client_id=vi}{}

\href{https://www.nytimes.com/section/todayspaper}{Today's Paper}

\href{/section/books/review}{Book Review}\textbar{}How Did the Nazis
Gain Power in Germany?

\url{https://nyti.ms/2HQCnDU}

\begin{itemize}
\item
\item
\item
\item
\item
\end{itemize}

Advertisement

\protect\hyperlink{after-top}{Continue reading the main story}

Supported by

\protect\hyperlink{after-sponsor}{Continue reading the main story}

Nonfiction

\hypertarget{how-did-the-nazis-gain-power-in-germany}{%
\section{How Did the Nazis Gain Power in
Germany?}\label{how-did-the-nazis-gain-power-in-germany}}

\includegraphics{https://static01.nyt.com/images/2018/06/17/arts/17Snyder1/merlin_138978477_67d1a8cb-b5e7-4c34-ab8c-d6d3efb25159-popup.jpg?quality=75\&auto=webp\&disable=upscale}

Buy Book ▾

\begin{itemize}
\tightlist
\item
  \href{https://www.amazon.com/gp/search?index=books\&tag=NYTBSREV-20\&field-keywords=The+Death+of+Democracy\%3A+Hitler\%E2\%80\%99s+Rise+to+Power+and+the+Downfall+of+the+Weimar+Republic+Benjamin+Carter+Hett}{Amazon}
\item
  \href{https://du-gae-books-dot-nyt-du-prd.appspot.com/buy?title=The+Death+of+Democracy\%3A+Hitler\%E2\%80\%99s+Rise+to+Power+and+the+Downfall+of+the+Weimar+Republic\&author=Benjamin+Carter+Hett}{Apple
  Books}
\item
  \href{https://www.anrdoezrs.net/click-7990613-11819508?url=https\%3A\%2F\%2Fwww.barnesandnoble.com\%2Fw\%2F\%3Fean\%3D9781250162502}{Barnes
  and Noble}
\item
  \href{https://www.anrdoezrs.net/click-7990613-35140?url=https\%3A\%2F\%2Fwww.booksamillion.com\%2Fp\%2FThe\%2BDeath\%2Bof\%2BDemocracy\%253A\%2BHitler\%25E2\%2580\%2599s\%2BRise\%2Bto\%2BPower\%2Band\%2Bthe\%2BDownfall\%2Bof\%2Bthe\%2BWeimar\%2BRepublic\%2FBenjamin\%2BCarter\%2BHett\%2F9781250162502}{Books-A-Million}
\item
  \href{https://bookshop.org/a/3546/9781250162502}{Bookshop}
\item
  \href{https://www.indiebound.org/book/9781250162502?aff=NYT}{Indiebound}
\end{itemize}

When you purchase an independently reviewed book through our site, we
earn an affiliate commission.

By Timothy Snyder

\begin{itemize}
\item
  June 14, 2018
\item
  \begin{itemize}
  \item
  \item
  \item
  \item
  \item
  \end{itemize}
\end{itemize}

\textbf{THE DEATH OF DEMOCRACY}\\
\textbf{Hitler's Rise to Power and the Downfall of the Weimar
Republic}\\
By Benjamin Carter Hett\\
Illustrated. 280 pp. Henry Holt \& Company. \$30.

We ask about the rise of the Nazis from what we think is a great
distance. We take for granted that the Germans of the 1930s were quite
different from ourselves, and that our consideration of their errors
will only confirm our superiority. The opposite is the case. Although
\href{http://urban.hunter.cuny.edu/~hett/}{Benjamin Carter Hett} makes
no comparisons between Germany then and the United States now in ``The
Death of Democracy,'' his extremely fine study of the end of
constitutional rule in Germany, he dissolves those comforting
assumptions. He is not discussing a war in which Germans were enemies or
describing atrocities that we are sure we could never commit. He
presents Hitler's rise as an element of the collapse of a republic
confronting dilemmas of globalization with imperfect instruments and
flawed leaders. With careful prose and fine scholarship, with fine
thumbnail sketches of individuals and concise discussions of
institutions and economics, he brings these events close to us.

The Nazis, in Hett's account, were above all ``a nationalist protest
movement against globalization.'' Even before the Great Depression
brought huge unemployment to Germany, the caprice of the global economy
offered an opportunity to politicians who had simple answers. In their
1920 program, the Nazis proclaimed that ``members of foreign nations
(noncitizens) are to be expelled from Germany.'' Next would come
autarky: Germans would conquer the territory they needed to be
self-sufficient, and then create their own economy in isolation from
that of the rest of the world. As Goebbels put it, ``We want to build a
wall, a protective wall.'' Hitler maintained that the vicissitudes of
globalization were not the result of economic forces but of a Jewish
international conspiracy.

Image

Hett, a professor of history at Hunter College and the Graduate Center
of the City University of New York, sensitively describes a moral crisis
that preceded a moral catastrophe. If Jews were held responsible for
what happened in Germany, then Germans were victims and their actions
always defensive. Political irresponsibility flowed from the unfortunate
example of President
\href{https://www.britannica.com/biography/Paul-von-Hindenburg}{Paul von
Hindenburg}. He was famous as the victor in a battle on the Eastern
Front of World War I, even though the credit was not fully deserved.
Hindenburg could not face the reality of defeat on the Western Front in
1918, and so spread the lie that the German Army had been ``stabbed in
the back'' by Jews and Socialists. This moral weakness of one man
radiated outward. Once Hindenburg won the presidential elections of
1925, Germany was trapped by his oversensitivity about a reputation that
would not withstand scrutiny. He believed that only he could save
Germany, but would not put himself forward to do so, for fear of
damaging his image. Without Hindenburg's founding fiction and odd
posturing, it is unlikely that Hitler would have come to power.

As Hett capably shows, the Nazis were the great artists of victimhood
fiction. Hitler, who had served with German Jews in the war, spread the
idea that Jews had been the enemy within, proposing that the German Army
would have won had some of them been gassed to death. Goebbels had Nazi
storm troopers attack leftists precisely so that he could claim that the
Nazis were victims of Communist violence. Hitler believed in telling
lies so big that their very scale left some residue of credibility. The
Nazi program foresaw that newspapers would serve the ``general good''
rather than reporting, and promised ``legal warfare'' against opponents
who spread information they did not like. They opposed what they called
``the system'' by rejecting its basis in the factual world. Germans were
not rational individuals with interests, the reasoning went, but members
of a tribe that wanted to follow a leader (Führer).

Much of this was familiar from Italian Fascism, but Hitler's attempt to
imitate
\href{http://warfarehistorynetwork.com/daily/benito-mussolini-the-fascist-march-on-rome/}{Mussolini's
March on Rome} failed. When Hitler tried a coup d'état in 1923, he and
the Nazis were easily defeated and he was sentenced to prison, where he
wrote ``Mein Kampf.'' In Hett's account, the electoral rise of the Nazis
in the late 1920s and early 1930s had less to do with his particular
ideas and more to do with an opening on the political spectrum. The
Nazis filled a void between the Catholic electorate of the Center Party
and a working class that voted Socialist or Communist. Their core
constituents, Hett indicates, were Protestants from the countryside or
small towns who felt themselves to be the victims of globalization.

Did the Nazis come to power through democratic elections? In Germany in
the 1930s, as elsewhere, elections continued even as their meaning
changed. The fact that the Nazis used violence to intimidate others
meant that elections were not free in the normal sense. And the system
was rigged in their favor by men in power who had no use for democracy
or for democrats. The Nazis were by no means the handmaidens of German
industry or the German military but, as Hett argues, both businessmen
and officers formed lobbies in the late 1920s that aimed to break the
republic and its bastion, the Social Democrats. They tended to confuse
their particular interests in lower wages and higher military spending
with those of the German nation as a whole. This made it easy to see the
Social Democrats as foreign and hostile.

\includegraphics{https://static01.nyt.com/images/2018/06/17/arts/17Snyder2/merlin_10889113_020b184f-792a-4ffe-bd21-ec7fc461d766-articleLarge.jpg?quality=75\&auto=webp\&disable=upscale}

In a similarly titled book, ``How Democracies Die,'' the political
scientists
\href{https://www.theguardian.com/us-news/commentisfree/2018/jan/21/this-is-how-democracies-die}{Daniel
Ziblatt and Steven Levitsky} have recently argued that the killers of
democracy begin by using the law against itself. Constitutions break
when ill-motivated leaders deliberately expose their vulnerabilities.
Certainly this was the case in Germany in 1930. President Hindenburg was
technically within his rights to dissolve the Reichstag, name a new
chancellor and rule by decree. By turning what was meant to be an
exceptional situation into the rule, however, he transformed the German
government into a feuding clique disconnected from society. Governments
dependent upon the president had no reason to think creatively about
policy, despite the Great Depression. Voters flowed to both extremes, to
the Communists and even more to the Nazis. The Nazis took advantage of
an opportunity created by people who could destroy a republic while
lacking the imagination to see what comes next.

When elections were called in 1932, the purpose was not to confirm
democracy but to bring down the republic. Hindenburg and his advisers
saw the Nazis as a group capable of creating a majority for the right.
The elections were a ``solution'' to a fake crisis that had been, as
Hett puts it, ``manufactured by a political right wing that wanted to
exclude more than half the population from political representation and
refused even the mildest compromise.'' It did not occur to the
president's camp that the Nazis would do as well as they did, or that
their leader would escape their control. And so the feckless schemes of
the conservatives realized the violent dreams of the Nazis. The Nazis
won 37 percent of the vote in July, 33 percent in a November election,
and Hitler became chancellor in January 1933. A few weeks later, he used
the pretext of the arson of the Reichstag to pass an enabling act that
in effect replaced the constitution.

Hindenburg died in 1934 believing that he had saved Germany and his own
reputation. In fact, he had created the conditions for the great horror
of modern times. Hett's book is implicitly addressed to conservatives.
Rather than asking how the left could have acted to stop Hitler, he
closes his book by considering the German conservatives who aided
Hitler's rise, then changed their minds and plotted against him.
Following the recent work of
\href{https://www.cambridge.org/core/journals/central-european-history/article/der-amtssitz-der-opposition-politik-und-staatsumbauplane-im-buro-des-stellvertreters-des-reichskanzlers-in-den-jahren-19331934-by-rainer-orth-cologne-bohlau-verlag-2016-pp-1118-cloth-9000-isbn-9783412505554/C1E87000AC4D05F29CED7E3BE26E512A}{Rainer
Orth}, Hett says that the Night of the Long Knives, the blood purge of
June 1934, was directed mainly against these right-wing opponents.

The conclusions for conservatives of today emerge clearly: Do not break
the rules that hold a republic together, because one day you will need
order. And do not destroy the opponents who respect those rules, because
one day you will miss them.

Advertisement

\protect\hyperlink{after-bottom}{Continue reading the main story}

\hypertarget{site-index}{%
\subsection{Site Index}\label{site-index}}

\hypertarget{site-information-navigation}{%
\subsection{Site Information
Navigation}\label{site-information-navigation}}

\begin{itemize}
\tightlist
\item
  \href{https://help.nytimes.com/hc/en-us/articles/115014792127-Copyright-notice}{©~2020~The
  New York Times Company}
\end{itemize}

\begin{itemize}
\tightlist
\item
  \href{https://www.nytco.com/}{NYTCo}
\item
  \href{https://help.nytimes.com/hc/en-us/articles/115015385887-Contact-Us}{Contact
  Us}
\item
  \href{https://www.nytco.com/careers/}{Work with us}
\item
  \href{https://nytmediakit.com/}{Advertise}
\item
  \href{http://www.tbrandstudio.com/}{T Brand Studio}
\item
  \href{https://www.nytimes.com/privacy/cookie-policy\#how-do-i-manage-trackers}{Your
  Ad Choices}
\item
  \href{https://www.nytimes.com/privacy}{Privacy}
\item
  \href{https://help.nytimes.com/hc/en-us/articles/115014893428-Terms-of-service}{Terms
  of Service}
\item
  \href{https://help.nytimes.com/hc/en-us/articles/115014893968-Terms-of-sale}{Terms
  of Sale}
\item
  \href{https://spiderbites.nytimes.com}{Site Map}
\item
  \href{https://help.nytimes.com/hc/en-us}{Help}
\item
  \href{https://www.nytimes.com/subscription?campaignId=37WXW}{Subscriptions}
\end{itemize}
