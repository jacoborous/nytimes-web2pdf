Sections

SEARCH

\protect\hyperlink{site-content}{Skip to
content}\protect\hyperlink{site-index}{Skip to site index}

\href{https://myaccount.nytimes.com/auth/login?response_type=cookie\&client_id=vi}{}

\href{https://www.nytimes.com/section/todayspaper}{Today's Paper}

When Protest Movements Became Brands

\href{https://nyti.ms/2JQSnIb}{https://nyti.ms/2JQSnIb}

\begin{itemize}
\item
\item
\item
\item
\item
\end{itemize}

Advertisement

\protect\hyperlink{after-top}{Continue reading the main story}

Supported by

\protect\hyperlink{after-sponsor}{Continue reading the main story}

\hypertarget{when-protest-movements-became-brands}{%
\section{When Protest Movements Became
Brands}\label{when-protest-movements-became-brands}}

As socially minded public installations in New York City became popular
in the 1980s, political movements realized their marketing potential.

\includegraphics{https://static01.nyt.com/images/2018/04/03/t-magazine/03tmag-hicart-slide-N71E/03tmag-hicart-slide-N71E-articleLarge.jpg?quality=75\&auto=webp\&disable=upscale}

By Sarah Schulman

\begin{itemize}
\item
  April 16, 2018
\item
  \begin{itemize}
  \item
  \item
  \item
  \item
  \item
  \end{itemize}
\end{itemize}

STARTING IN 1979, when I was 21, I was an out reporter writing for
underground queer newspapers, in which I covered key acts of political
warfare in New York City. This was the beginning of President Reagan's
terrifying and far-reaching unification with the Christian right. As a
particularly despised group of people, queers were fighting police
brutality on one front and racist admission policies in gay and lesbian
bars on the other, homophobia at home and the closet at work. At the
time, there was still no gay rights bill in the city, and sodomy was
illegal in more than half of the states. Simultaneously, there were
people who insisted on being out and who created rich subcultures that
forged new ground in sex, night life, imagination --- and rebellion.

Three years later, in 1982, I witnessed a violent New York City Police
raid on a gay bar, Blues, a place for black men in Times Square. The
cops beat up patrons, breaking mirrors and shattering bottles. Then as
now, police violence against black men largely went unpunished. And when
the gay community came out in anger to demonstrate in favor of Blues's
patrons, the action received no television coverage. The only
documentation was by the Super 8 filmmaker Jim Hubbard. His grainy
footage shows a multiracial, angry gathering of queer men and women
holding small homemade signs; there were no mass-produced T-shirts
emblazoned with slogans, no shiny pins, no professionally printed
posters. This was what resistance looked like then: nothing to buy, and
nothing to sell.

SO HOW DID we get from that to the highly visible, art-directed
expressions of the late 1980s and early '90s, which --- as proven by the
\href{https://www.nytimes.com/2017/12/13/fashion/the-year-in-stuff.html}{pink
pussy hats} and
\href{https://www.nytimes.com/2016/08/23/us/how-blacklivesmatter-came-to-define-a-movement.html}{Black
Lives Matter} shirts of recent years --- still resonate today? Look
first to late-70s conceptual artists, including Jenny Holzer and Barbara
Kruger, who drew attention to cultural contradictions with their stark,
text-based works, originally pasted on buildings and construction sites
in the newly gentrifying SoHo. Public fine art, in combination with the
era's advertising, inspired a generation of graphic designers, including
Avram Finkelstein, whose collective created the famous ``Silence=Death''
logo, later donated to \href{http://www.actupny.org/}{ACT UP} (Aids
Coalition to Unleash Power) in 1987. Provocative protest iconography
soon became the norm: Artists like Tom Kalin, Marlene McCarty, Loring
McAlpin and Robert Vasquez formed Gran Fury, one of a number of art
collectives in ACT UP. In 1991, AIDS activists' four-year campaign to
expand the Centers for Disease Control's definition of AIDS, so that
more infected women could get health benefits, was accompanied by a Gran
Fury poster that read ``Women Don't Get AIDS They Just Die From It.''

Image

Also in '87, the collective General Idea made its first AIDS painting,
updating artist Robert Indiana's 1965 LOVE motif.Credit...Acrylic on
canvas, courtesy of the Estate of General Idea and Esther Schipper,
Berlin. Photo © Andrea Rossetti

In many ways, this AIDS rebellion, and the aesthetic it created, drew
from the 1960s Civil Rights template. Much as the Black Power movement
had established its own look (black berets, leather jackets and a raised
fist), so too did New York's AIDS activists develop a signature
telegenic appearance, with their slogan T-shirts, jeans and Doc Martens.
Artists such as
\href{https://www.nytimes.com/1990/02/17/obituaries/keith-haring-artist-dies-at-31-career-began-in-subway-graffiti.html}{Keith
Haring},
\href{http://www.nytimes.com/1996/01/11/nyregion/felix-gonzalez-torres-38-a-sculptor-of-love-and-loss.html}{Felix
Gonzalez-Torres} and the collective General Idea made highly visible
pieces about loss, inserting AIDS into public spaces, advertisements and
even mass-produced apparel. By 1991, the filmmaker Jerry Tartaglia and
others in an organization called
\href{https://www.visualaids.org/projects/detail/visual-aids-artists-caucus}{Visual
AIDS Artists' Caucus} had conceived of the red AIDS ribbon as a sign of
solidarity in a country that was stigmatizing and isolating infected
people. That year, the 11-year-old actress Daisy Eagan wore one to
accept her Tony Award for her work in
``\href{https://www.nytimes.com/1991/04/26/theater/review-theater-garden-the-secret-of-death-and-birth.html}{The
Secret Garden},'' and the ribbon became an instant piece of iconography
--- of the AIDS movement, of course, but for all future movements as
well, a symbol of the conflation of money and power that was a new kind
of social currency altogether, one in which one should be, or appear to
be, socially aware.

The legacy of those years is today inescapable, as grassroots protest is
our only hope for survival, not something to be marketed and sold, a set
of images and memes. So many disparate communities are under attack
today --- black people, trans people, Muslims, women seeking abortions
--- that it would be foolhardy, not to mention impossible, to try to
homogenize us all under one symbol or slogan. With so many perspectives,
cultures and aesthetics at risk, it's the sheer variety of our
resistance that will express the breadth of our communities. Defying
branding, ultimately, will be the most successful strategy of all.

Advertisement

\protect\hyperlink{after-bottom}{Continue reading the main story}

\hypertarget{site-index}{%
\subsection{Site Index}\label{site-index}}

\hypertarget{site-information-navigation}{%
\subsection{Site Information
Navigation}\label{site-information-navigation}}

\begin{itemize}
\tightlist
\item
  \href{https://help.nytimes.com/hc/en-us/articles/115014792127-Copyright-notice}{©~2020~The
  New York Times Company}
\end{itemize}

\begin{itemize}
\tightlist
\item
  \href{https://www.nytco.com/}{NYTCo}
\item
  \href{https://help.nytimes.com/hc/en-us/articles/115015385887-Contact-Us}{Contact
  Us}
\item
  \href{https://www.nytco.com/careers/}{Work with us}
\item
  \href{https://nytmediakit.com/}{Advertise}
\item
  \href{http://www.tbrandstudio.com/}{T Brand Studio}
\item
  \href{https://www.nytimes.com/privacy/cookie-policy\#how-do-i-manage-trackers}{Your
  Ad Choices}
\item
  \href{https://www.nytimes.com/privacy}{Privacy}
\item
  \href{https://help.nytimes.com/hc/en-us/articles/115014893428-Terms-of-service}{Terms
  of Service}
\item
  \href{https://help.nytimes.com/hc/en-us/articles/115014893968-Terms-of-sale}{Terms
  of Sale}
\item
  \href{https://spiderbites.nytimes.com}{Site Map}
\item
  \href{https://help.nytimes.com/hc/en-us}{Help}
\item
  \href{https://www.nytimes.com/subscription?campaignId=37WXW}{Subscriptions}
\end{itemize}
