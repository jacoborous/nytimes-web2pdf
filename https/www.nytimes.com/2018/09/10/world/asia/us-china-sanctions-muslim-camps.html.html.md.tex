Sections

SEARCH

\protect\hyperlink{site-content}{Skip to
content}\protect\hyperlink{site-index}{Skip to site index}

\href{https://www.nytimes.com/section/world/asia}{Asia Pacific}

\href{https://myaccount.nytimes.com/auth/login?response_type=cookie\&client_id=vi}{}

\href{https://www.nytimes.com/section/todayspaper}{Today's Paper}

\href{/section/world/asia}{Asia Pacific}\textbar{}U.S. Weighs Sanctions
Against Chinese Officials Over Muslim Detention Camps

\url{https://nyti.ms/2MhSrRm}

\begin{itemize}
\item
\item
\item
\item
\item
\item
\end{itemize}

Advertisement

\protect\hyperlink{after-top}{Continue reading the main story}

Supported by

\protect\hyperlink{after-sponsor}{Continue reading the main story}

\hypertarget{us-weighs-sanctions-against-chinese-officials-over-muslim-detention-camps}{%
\section{U.S. Weighs Sanctions Against Chinese Officials Over Muslim
Detention
Camps}\label{us-weighs-sanctions-against-chinese-officials-over-muslim-detention-camps}}

\includegraphics{https://static01.nyt.com/images/2018/09/11/world/11dc-diplo-1-print/merlin_143546448_0a6fae8b-5959-4c04-97f1-39aacaaaf948-articleLarge.jpg?quality=75\&auto=webp\&disable=upscale}

By \href{https://www.nytimes.com/by/edward-wong}{Edward Wong}

\begin{itemize}
\item
  Sept. 10, 2018
\item
  \begin{itemize}
  \item
  \item
  \item
  \item
  \item
  \item
  \end{itemize}
\end{itemize}

\href{https://cn.nytimes.com/usa/20180911/us-china-sanctions-muslim-camps/}{阅读简体中文版}\href{https://cn.nytimes.com/usa/20180911/us-china-sanctions-muslim-camps/zh-hant/}{閱讀繁體中文版}

WASHINGTON --- The Trump administration is considering sanctions against
Chinese senior officials and companies to punish Beijing's detention of
hundreds of thousands of ethnic Uighurs and other minority Muslims in
\href{https://www.nytimes.com/2018/09/08/world/asia/china-uighur-muslim-detention-camp.html}{large
internment camps}, according to current and former American officials.

The economic penalties would be one of the first times the Trump
administration has taken action against China because of human rights
violations. United States officials are also seeking to limit American
sales of surveillance technology that Chinese security agencies and
companies
\href{https://www.buzzfeednews.com/article/meghara/the-police-state-of-the-future-is-already-here\#.nyemLvQxY}{are
using to monitor} Uighurs throughout northwest China.

Discussions to rebuke China for its
\href{https://www.academia.edu/37353916/NEW_Sept_2018_Thoroughly_Reforming_Them_Towards_a_Healthy_Heart_Attitude_-_Chinas_Political_Re-Education_Campaign_in_Xinjiang}{treatment
of its minority Muslims} have been underway for months among officials
at the White House and the Treasury and State Departments. But they
gained urgency two weeks ago,
\href{https://www.cecc.gov/media-center/press-releases/chairs-lead-bipartisan-letter-urging-administration-to-sanction-chinese}{after
members of Congress asked} Secretary of State Mike Pompeo and Treasury
Secretary Steven Mnuchin to impose sanctions on seven Chinese officials.

Until now, President Trump has largely resisted punishing China for its
human rights record, or even accusing it of widespread violations. If
approved, the penalties would fuel an already bitter standoff with
Beijing over trade and pressure on North Korea's nuclear program.

Last month, a United Nations panel
\href{https://www.nytimes.com/2018/08/10/world/asia/china-xinjiang-un-uighurs.html}{confronted
Chinese diplomats} in Geneva over the detentions. The camps for Chinese
Muslims have been the target of growing international criticism and
investigative reports, including by The New York Times.

Human rights advocates and
\href{http://www.jeromecohen.net/jerrys-blog/2018/7/25/what-can-be-done-regarding-xinjiangs-mass-detentions}{legal
scholars} say the
\href{https://www.nytimes.com/2018/08/10/world/asia/china-xinjiang-rahile-dawut.html}{mass
detentions} in the northwest region of Xinjiang are the worst collective
human rights abuse in China in decades. Since taking power in 2012,
President Xi Jinping has steered China on a hard authoritarian course,
which includes increased repression of large ethnic groups in western
China, notably the Uighurs and
\href{https://www.nytimes.com/2018/06/07/world/asia/china-tashi-wangchuk.html}{Tibetans}.

On Sunday, Human Rights Watch
\href{https://www.hrw.org/report/2018/09/09/eradicating-ideological-viruses/chinas-campaign-repression-against-xinjiangs}{released
a detailed report} that concluded that the violations were of a ``scope
and scale not seen in China since the 1966-1976 Cultural Revolution.''
The report, based on interviews with 58 former residents of Xinjiang,
recommended that other nations impose targeted sanctions on Chinese
officials, withhold visas and control exports of technology that could
be used for abuses.

Any new American sanctions would be announced by the Treasury Department
after governmentwide consultations, including with Congress.

Chinese Muslims in the camps are forced to attend daily classes,
denounce aspects of Islam, study mainstream Chinese culture and pledge
loyalty to the Chinese Communist Party. Some detainees who have been
released have
\href{https://www.washingtonpost.com/world/asia_pacific/former-inmates-of-chinas-muslim-re-education-camps-tell-of-brainwashing-torture/2018/05/16/32b330e8-5850-11e8-8b92-45fdd7aaef3c_story.html?utm_term=.c44ce1a08d6a}{described
torture} by security officers.

\includegraphics{https://static01.nyt.com/images/2018/09/11/us/politics/11dc-china3/merlin_135551823_ae4aa658-be46-4a5d-8b39-503c233ae618-articleLarge.jpg?quality=75\&auto=webp\&disable=upscale}

Chinese officials have labeled the process ``transformation through
education'' or ``counter-extremism education.'' But they have not
acknowledged that large groups of Muslims are being detained.

The discussions over the
\href{https://supchina.com/2018/08/22/xinjiang-explainer-chinas-reeducation-camps-for-a-million-muslims/}{mass
detentions in Xinjiang} highlight American efforts on issues that
diverge from the president's priorities. Mr. Trump has rarely made
statements criticizing foreign governments for human rights abuses or
anti-liberal policies, and in fact has
\href{https://www.washingtonpost.com/news/worldviews/wp/2018/03/04/in-a-jokey-speech-trump-praised-chinas-xi-for-ditching-term-limits-saying-maybe-well-give-that-a-shot-some-day/?utm_term=.2a55d1e1905a}{praised
authoritarian leaders, including Mr. Xi}.

The Trump administration has confronted China over economic issues ---
the two countries are in the middle of a
\href{https://www.nytimes.com/2018/09/07/business/trump-china-trade-war-tariffs.html}{prolonged
trade war} --- but has said little about rampant abuses by its security
forces.

``The scale of it --- it's massive,'' Senator Marco Rubio, Republican of
Florida, said of the Muslim detention centers in an interview. ``It
involves not only intimidating people on political speech, but also a
desire to strip people of their identity --- ethnic identity, religious
identity --- on a scale that I'm not sure we've seen in the modern
era.''

Ethnic Uighurs are a Turkic-speaking group that is mostly Sunni Muslim.
With a population of around 11 million, Uighurs are the largest ethnic
group in Xinjiang. Some of the desert oasis towns and villages that they
consider their homeland are being emptied out as security officers
\href{https://www.nchrd.org/2018/07/criminal-arrests-in-xinjiang-account-for-21-of-chinas-total-in-2017/}{force
many Uighurs} into large detention centers for weeks or months.

Gulchehra Hoja, a Uighur-American journalist who works for Radio Free
Asia, which is financed by the United States government,
\href{https://www.cecc.gov/events/hearings/surveillance-suppression-and-mass-detention-xinjiang\%E2\%80\%99s-human-rights-crisis}{said
at a congressional hearing in July} that two dozen of her family members
in Xinjiang were missing, including her brother.

``I hope and pray for my family to be let go and released,'' Ms. Hoja
said. ``But I know even if that happens, they will still live under
constant threat.''

A Chinese law student in Canada, Shawn Zhang, has
\href{http://www.chinafile.com/reporting-opinion/features/what-satellite-images-can-show-us-about-re-education-camps-xinjiang}{compiled
satellite images} that show the scale of some of the detention centers.

In their demand last month, Mr. Rubio and other lawmakers urged
officials at the State and Treasury Departments to impose sanctions on
Chinese companies that have profited from building the camps or the
regionwide surveillance system, which includes
\href{https://www.hrw.org/news/2017/12/13/china-minority-region-collects-dna-millions}{the
collection of biometric and DNA data}. They singled out Hikvision and
Dahua Technology for the surveillance.

Mr. Rubio said the Congressional-Executive Commission on China, of which
he is a chairman, will also ask the Commerce Department to
\href{https://www.cecc.gov/media-center/press-releases/chairs-ask-commerce-secretary-ross-about-sale-of-surveillance-technology}{prevent
American companies} from selling technology to China that could
contribute to the surveillance and tracking.

Image

Congressional lawmakers singled out Chen Quanguo, who became party chief
of Xinjiang in 2016, for sanctions among seven Chinese
officials.Credit...Ng Han Guan/Associated Press

For many years, Chinese officials have talked about the need to suppress
what they call terrorism, separatism and religious extremism in
Xinjiang. In 2009, ethnic violence began soaring in the region. Security
forces carried out mass repression in response, but large-scale
construction of the camps, which now hold as many as one million people,
did not begin until the arrival of Chen Quanguo, who became party chief
of Xinjiang in August 2016, after a stint in the Tibet Autonomous
Region.

The congressional demand, outlined in an Aug. 28 letter, singles out Mr.
Chen among the seven Chinese officials who would be sanctioned.

In Washington, officials grappling with the plight of the Uighurs and
other Chinese Muslims are doing so in the shadow of the mass murders,
rapes and forced displacement of Rohingya Muslims by Burmese military
forces that began in Myanmar in August 2017. More than 700,000 Rohingya
fled to neighboring Bangladesh and live in squalid camps.

Some American officials see the actions of the Chinese government as
another form of the genocide that occurred in Myanmar, according to
people with knowledge of the continuing discussions, who spoke on the
condition of anonymity because they have not been authorized to talk
publicly about the issue.

Sam Brownback, the State Department's ambassador at large for
international religious freedom and former governor of Kansas, supports
taking a hard line against the Chinese government on the issue of
Xinjiang, they said. Mr. Brownback declined to be interviewed.

In April, Laura Stone, an acting deputy assistant secretary for East
Asian and Pacific affairs,
\href{https://www.apnews.com/13ee0b95c05249a8be7c0467bf413b6e}{told
reporters on a visit to Beijing} that the United States could impose
sanctions on Chinese officials involved in the Xinjiang abuses under the
Global Magnitsky Act. The law allows the American government to impose
sanctions on specific foreign officials who are gross violators of human
rights.

That same month, Heather Nauert, the chief spokeswoman for the State
Department,
\href{https://twitter.com/statedeptspox/status/987687844520001537}{called
on China to release all those ``unlawfully detained''} after meeting in
Washington with Ms. Hoja and five other ethnic Uighur journalists who
work in the United States for Radio Free Asia. The journalists shared
details of the mass detentions and of harassment of their own family
members in the region.

The issue of the Uighurs was raised in July at the first international
minister-level forum on global religious freedom, over which Mr. Pompeo
and Vice President Mike Pence presided. Ahead of it, Mr. Pompeo
\href{https://www.usatoday.com/story/opinion/2018/07/24/defend-religious-freedom-global-attacks-mike-pompeo-column/818972002/}{wrote
an op-ed} that listed the Uighurs among several groups suffering
religious persecution. ``These episodes and others like them are
abhorrent,'' he wrote.

In a statement to The Times, the State Department said officials ``are
deeply troubled by the Chinese government's worsening crackdown'' on
Muslims.

``Credible reports indicate that individuals sent by Chinese authorities
to detention centers since April 2017 number at least in the hundreds of
thousands, and possibly millions,'' the statement said.

The Trump administration has used an executive order tied to the
Magnitsky Act once to impose sanctions on a Chinese official. In
December, the White House
\href{https://www.rfa.org/english/news/china/chief-12222017103212.html}{announced
sanctions against Gao Yan}, who was a district police chief in Beijing
when a human-rights activist died in detention.

Advertisement

\protect\hyperlink{after-bottom}{Continue reading the main story}

\hypertarget{site-index}{%
\subsection{Site Index}\label{site-index}}

\hypertarget{site-information-navigation}{%
\subsection{Site Information
Navigation}\label{site-information-navigation}}

\begin{itemize}
\tightlist
\item
  \href{https://help.nytimes.com/hc/en-us/articles/115014792127-Copyright-notice}{©~2020~The
  New York Times Company}
\end{itemize}

\begin{itemize}
\tightlist
\item
  \href{https://www.nytco.com/}{NYTCo}
\item
  \href{https://help.nytimes.com/hc/en-us/articles/115015385887-Contact-Us}{Contact
  Us}
\item
  \href{https://www.nytco.com/careers/}{Work with us}
\item
  \href{https://nytmediakit.com/}{Advertise}
\item
  \href{http://www.tbrandstudio.com/}{T Brand Studio}
\item
  \href{https://www.nytimes.com/privacy/cookie-policy\#how-do-i-manage-trackers}{Your
  Ad Choices}
\item
  \href{https://www.nytimes.com/privacy}{Privacy}
\item
  \href{https://help.nytimes.com/hc/en-us/articles/115014893428-Terms-of-service}{Terms
  of Service}
\item
  \href{https://help.nytimes.com/hc/en-us/articles/115014893968-Terms-of-sale}{Terms
  of Sale}
\item
  \href{https://spiderbites.nytimes.com}{Site Map}
\item
  \href{https://help.nytimes.com/hc/en-us}{Help}
\item
  \href{https://www.nytimes.com/subscription?campaignId=37WXW}{Subscriptions}
\end{itemize}
