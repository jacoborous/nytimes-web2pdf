Sections

SEARCH

\protect\hyperlink{site-content}{Skip to
content}\protect\hyperlink{site-index}{Skip to site index}

\href{https://www.nytimes.com/section/style}{Style}

\href{https://myaccount.nytimes.com/auth/login?response_type=cookie\&client_id=vi}{}

\href{https://www.nytimes.com/section/todayspaper}{Today's Paper}

\href{/section/style}{Style}\textbar{}Vicenza, Italy's Capital of Gold

\href{https://nyti.ms/2zntFMv}{https://nyti.ms/2zntFMv}

\begin{itemize}
\item
\item
\item
\item
\item
\end{itemize}

Advertisement

\protect\hyperlink{after-top}{Continue reading the main story}

Supported by

\protect\hyperlink{after-sponsor}{Continue reading the main story}

\hypertarget{vicenza-italys-capital-of-gold}{%
\section{Vicenza, Italy's Capital of
Gold}\label{vicenza-italys-capital-of-gold}}

\includegraphics{https://static01.nyt.com/images/2018/09/22/style/22vicenza-inyt1/merlin_141511989_42dfd716-db87-4dae-b661-3f089f56d0b9-articleLarge.jpg?quality=75\&auto=webp\&disable=upscale}

By \href{https://www.nytimes.com/by/laura-rysman}{Laura Rysman}

\begin{itemize}
\item
  Sept. 22, 2018
\item
  \begin{itemize}
  \item
  \item
  \item
  \item
  \item
  \end{itemize}
\end{itemize}

VICENZA, Italy --- Vicenza is quaintly medieval at its center, a dense
jumble of old butter-toned dwellings along narrow byways that
occasionally give way to some of the Renaissance's most elegant
architecture, but these structures mask an industrial might that has
made this small city Italy's most productive capital of jewelry.

``We were born to do this kind of thing,'' said Roberto Coin, whose
namesake company is one of Vicenza's most successful brands worldwide.
``We were born to create beauty, we were born to create new ideas. It's
in our DNA. It's what we know how to do.'' Nearly 10 percent of the
100,000-plus population is employed in the jewelry sector, and teenagers
can replace high school with jewelry studies at the Scuola d'Arte e
Mestieri.

The local legacy of jewelry-making predates even the cobbled streets: As
far back as 600 B.C., the Vicentini were crafting clothing fasteners,
called fibula, and other ornaments in bronze. But it was the 14th
century, with its emphasis on craft and guilds (and a 1339 statute
recognizing the goldsmiths' fraglia, or guild), that crowned Vicenza as
a prominent center of the jewelry arts and made its jewelers' guild a
political force among the nobles and merchants --- and of city society
to this day.

\includegraphics{https://static01.nyt.com/images/2018/09/22/style/22vicenza-inyt2/merlin_141511233_9a2655e0-400c-41b4-a045-c48927b330ee-articleLarge.jpg?quality=75\&auto=webp\&disable=upscale}

Vicenza's heart is the Piazza dei Signori, the bustling former Roman
forum whose vast, stone-paved square is home to a centuries-old weekly
market, a legion of aperitivo bars where evening crowds gather in this
wine-loving town, and the storefronts of 10 independent jewelry
businesses.

There were 15 such shops on this piazza already in the 1300s; Soprana,
the house that today has been at its piazza location the longest, was
founded in 1770 by the family of jewelers that had made the famed
precious crown for a statue of the Virgin Mary in the Church of St. Mary
of Monte Berico nearby.

The piazza is dominated by the slightly leaning (but still functioning)
14th-century Bissara clock tower; by two towering columns, topped by
statues of Christ the Redeemer and the winged lion that symbolizes
Venice, the lagoon city about 50 miles east that ruled Vicenza in the
15th century; and by the 16th-century Basilica Palladiana, with its
stately double row of white marble arches by Andrea Palladio, the most
influential architect of the Renaissance and Vicenza's most illustrious
resident.

Since 2014, the Basilica Palladiana has housed the
\href{https://www.nytimes.com/2016/03/14/fashion/jewelry-museum-vicenza-italy.html}{Museo
del Gioiello}, promoted as the only jewelry museum in Italy and one of
just a handful in the world, with a treasure box of an exhibition space
designed by Patricia Urquiola. The museum is just completing what it
says was the largest solo show ever dedicated to the artist and jeweler
Giò Pomodoro, to be followed by an exhibition on crowns and tiaras. The
display includes a rotating selection of jewelry from Vicenza and well
beyond, including the Monte Berico crown; a Lalique 1890 bird brooch
bedecked with a fistful of diamonds; and the Rosa dei Venti choker, set
with panels of brightly colored gemstones, by the contemporary Milanese
jeweler Giampiero Bodino.

Image

The Rosa dei Venti choker by Giampiero Bondino, a modern piece, on
display at the Museo del Gioiello.

Credit...Clara Vannucci for The New York Times

``More than economic value, the museum provides cultural value,'' Alba
Cappellieri, the director, said. ``The museum has enhanced the status of
Vicenza as a jewelry capital, as it was intended to.''

Along with help from the city (which lends the Basilica Palladiana
space) and some industry sponsors, the museum is funded primarily by the
Italian Exhibition Group, which holds Vicenzaoro, the local jewelry
trade show that attracts more exhibitors and attendees than any other in
Italy. The twice-yearly event, scheduled to open Saturday, is held at
the Fiera di Vicenza fairgrounds outside the city center. It drew more
than 56,000 visitors in 2017, with 18,000 of them coming in January. By
comparison, the January event this year attracted 23,000.

``It's not about being the largest fair,'' said Matteo Marzotto, the
exhibition group's vice president. In 1836, his family began Marzotto
Tessuti, now Italy's leading producer of fabric and one of the reasons
Vicenza is also a major supplier of textiles and fashion.

``What we want to be is the most beautiful fair, to offer three days of
business when visitors can experience the Italian lifestyle,'' he said,
pointing at the charms of the Piazza dei Signori, where he was sitting
at El Coq, the city's Michelin-starred restaurant. (Growth, however, is
still a priority, so with exhibitor and visitor numbers increasing,
construction is scheduled to begin in 2019 on a fairgrounds pavilion of
almost 540,000 square feet, a 20 percent expansion.)

Image

The crown of Our Lady of Monte Berico (1900), also at the museum. it is
encrusted with peridot, diamonds, rubies, pearls, sapphires and
amethyst, among other stones.\\
**\\
**

Credit...Clara Vannucci for The New York Times

Deeply linked to the territory's jewelry industry, Vicenzaoro is a
particularly proud showcase for hometown brands such as Pesavento, Fope
and Roberto Coin, although vendors come from around the world to sell.

A city that suffered heavy bombings and deprivation during World War II
(other Italians have taunted the townspeople as mangiagatti, or cat
eaters), Vicenza never lost its connection to the goldsmith's art, and
the economy revived in the 1950s and '60s as it combined its long
jewelry tradition with industrial and technological innovation, helped
along by American investment in the area, including the construction of
a United States military base.

By the 1970s, Vicenza was thriving amid a boom in European and American
jewelry sales; the numbers of artisan ateliers surged, while factories
turned out large quantities of jewelry and particularly of chains ---
thanks to machines invented locally, said Cristina del Mare, a jewelry
historian and one of the Museo del Gioiello's curators. This combination
of skilled craftspeople and technology also established the city as the
workshop for some of the best-known brands, including Gucci, Tiffany \&
Co. and Hermès.

``We're very advanced technologically here, but what makes the
difference is our manual skill,'' said Chiara Carli, who along with
Marino Pesavento founded Pesavento 26 years ago at the Centro Orafa
Vicentina, a complex on the city outskirts that houses 40 companies. The
business creates dramatically Italianate jewelry with an emphasis on
chains, combining the machine-made and 3-D-printed with the
hand-assembled and finished.

Image

Palladio's Teatro Olimpico is a 1585 marvel that recreated an ancient
amphitheater as an indoor playhouse.Credit...Clara Vannucci for The New
York Times

Pesavento is a majority female enterprise, unusual in this mostly male
industry, with 26 women on the 40-person team running its workshops and
offices. But in other aspects the brand is typical of Vicenza's jewelry
companies: It is a family affair, with Ms. Carli's brother and twin
sister working alongside her**.**

``Handcraft is still 80 percent of the work here,'' Ms. Carli said as
she leaned over a woman in a blue smock who was delicately
laser-soldering a silver chain, link by link. But Pesavento also
represents the latest chapter of Vicenza's story: the adjustment since
the 2008 downturn to a weakened Italian economy and difficult global
market.

Pesavento sells jewels of plated silver, not solid gold, and many are
accented with the brand's signature polveri di sogni, a dab of carbon
microparticles that impart the shimmer of black diamonds at a much lower
price. In general today, Vicenza's companies are marketing products that
are less expensive than what they previously offered, but still reflect
Italian style and know-how. ``With the crisis, we were obliged to become
much more business-minded about what we do,'' Ms. Carli said.

``Globalization has killed Italy,'' said Mr. Coin, who says his export
business remains strong despite competition from countries with lower
production costs. ``The bigger got bigger; the smaller got smaller or
disappeared.'' His business falls on the bigger side, while most of
Vicenza's jewelry houses have been small, family-style operations. Mr.
Coin estimates that there were around 5,300 jewelry businesses in the
city when he started in 1977; today, there are 851.

Image

A detail view of a miniature model of the city at the Diocesan Museum.
The original was destroyed by Napoleon's troops in 1797.Credit...Clara
Vannucci for The New York Times

Still, Vicenza has held on to its position better than jewelry-making
outposts in France, Spain and Germany, he noted, thanks to superior
craftsmanship and the standard of Italian style. ``Vicenza must express
the italianità it did in the past,'' he said, a lit cigarette in one
hand as he sipped an espresso at his desk. ``The world expects
expressions of beauty and quality from us.''

It's easy to feel the italianità of the past in Vicenza. Tourists flock
to town to see Palladio's harmoniously symmetrical Renaissance
buildings: the basilica; the Teatro Olimpico, a 1585 marvel that
recreates an ancient amphitheater as an indoor playhouse; and other
Unesco-protected sites.

Yet visitors might easily miss one of the most resonant examples of
architecture: Vicenza in miniature, circa 1577, the year the town
council commissioned Palladio to design a small model of the city. Just
about two feet in diameter and with 300 tiny buildings, the model was
painstakingly created in sterling silver by Vicenza's jewelers,
requiring more than 2,000 hours of handwork. An offering to the Virgin
Mary for the cessation of the plague, it was destroyed by Napoleon's
troops in 1797.

But in 2011 the city had the model recreated, using its appearance in
several Renaissance paintings as a guide. Today, it sits in a spotlit
case at the Diocesan Museum --- a silent, gleaming votive to the
unending gospel of jewelry-making in Vicenza.

Advertisement

\protect\hyperlink{after-bottom}{Continue reading the main story}

\hypertarget{site-index}{%
\subsection{Site Index}\label{site-index}}

\hypertarget{site-information-navigation}{%
\subsection{Site Information
Navigation}\label{site-information-navigation}}

\begin{itemize}
\tightlist
\item
  \href{https://help.nytimes.com/hc/en-us/articles/115014792127-Copyright-notice}{©~2020~The
  New York Times Company}
\end{itemize}

\begin{itemize}
\tightlist
\item
  \href{https://www.nytco.com/}{NYTCo}
\item
  \href{https://help.nytimes.com/hc/en-us/articles/115015385887-Contact-Us}{Contact
  Us}
\item
  \href{https://www.nytco.com/careers/}{Work with us}
\item
  \href{https://nytmediakit.com/}{Advertise}
\item
  \href{http://www.tbrandstudio.com/}{T Brand Studio}
\item
  \href{https://www.nytimes.com/privacy/cookie-policy\#how-do-i-manage-trackers}{Your
  Ad Choices}
\item
  \href{https://www.nytimes.com/privacy}{Privacy}
\item
  \href{https://help.nytimes.com/hc/en-us/articles/115014893428-Terms-of-service}{Terms
  of Service}
\item
  \href{https://help.nytimes.com/hc/en-us/articles/115014893968-Terms-of-sale}{Terms
  of Sale}
\item
  \href{https://spiderbites.nytimes.com}{Site Map}
\item
  \href{https://help.nytimes.com/hc/en-us}{Help}
\item
  \href{https://www.nytimes.com/subscription?campaignId=37WXW}{Subscriptions}
\end{itemize}
