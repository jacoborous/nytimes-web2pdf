Sections

SEARCH

\protect\hyperlink{site-content}{Skip to
content}\protect\hyperlink{site-index}{Skip to site index}

\href{https://www.nytimes.com/section/style}{Style}

\href{https://myaccount.nytimes.com/auth/login?response_type=cookie\&client_id=vi}{}

\href{https://www.nytimes.com/section/todayspaper}{Today's Paper}

\href{/section/style}{Style}\textbar{}Elton John's Farewell Tour
Wardrobe, Explained

\url{https://nyti.ms/2MU6otS}

\begin{itemize}
\item
\item
\item
\item
\item
\item
\end{itemize}

Advertisement

\protect\hyperlink{after-top}{Continue reading the main story}

Supported by

\protect\hyperlink{after-sponsor}{Continue reading the main story}

\hypertarget{elton-johns-farewell-tour-wardrobe-explained}{%
\section{Elton John's Farewell Tour Wardrobe,
Explained}\label{elton-johns-farewell-tour-wardrobe-explained}}

After a 50-year career, the rocket man of pop is retiring from the road.
This is the inside story of his finale style statement, designed by
Gucci, and the secret power of a sequin.

\includegraphics{https://static01.nyt.com/images/2018/09/09/autossell/09ELTONJOHN-1/09ELTONJOHN-1-articleLarge.jpg?quality=75\&auto=webp\&disable=upscale}

\href{https://www.nytimes.com/by/elizabeth-paton}{\includegraphics{https://static01.nyt.com/images/2019/07/16/reader-center/author-elizabeth-paton/author-elizabeth-paton-thumbLarge.png}}

By \href{https://www.nytimes.com/by/elizabeth-paton}{Elizabeth Paton}

\begin{itemize}
\item
  Sept. 7, 2018
\item
  \begin{itemize}
  \item
  \item
  \item
  \item
  \item
  \item
  \end{itemize}
\end{itemize}

NICE, France --- On a late summer afternoon last week, on the sloping
driveway of a palatial 1920s villa high in the hills above the city of
Nice, sat three large white trucks. Lining a path along its lush lawns
were scores of brown cardboard boxes; rolls of Bubble Wrap were unfurled
on the grass.

Inside the mansion's underground gym, neatly laid out by a small army of
Italians largely dressed in black, lay a glittering cornucopia of stage
costumes, twinkling in the Côte d'Azur sunlight that beamed through the
windows.

On one groaning rail were half a dozen 17th-century-style ottoman frock
coats, finished with bright lapels, pearl or gold sequined trims and 3-D
embroidered embellishments, including roaring cats and twisted florals,
a bright white ``E.J.'' emblazoned on the back.

Another rack held a kaleidoscopic lineup of shiny tracksuits. There were
sumptuous racks of kimonos and stacks of powder-hued Swiss cotton
collared shirts, alongside rows of sneakers, pearly slippers and neon
brogues.

Everywhere else there were trays upon trays of glam rock treasure:
mirrored glasses with crystals in every conceivable shape and color;
brooches shaped like thistles, pineapples, starfish and overflowing
bouquets.

\includegraphics{https://static01.nyt.com/images/2018/09/09/autossell/09ELTONJOHN-5/merlin_143073000_b3a74416-c293-4045-9adf-e43123df77c1-articleLarge.jpg?quality=75\&auto=webp\&disable=upscale}

Above it all, perched on a balcony painted pale yellow, sat the British
singer, pianist and composer
\href{https://www.nytimes.com/topic/person/elton-john}{Elton John}, 71,
for whom all that sparkle was meant. After selling
\href{https://www.nytimes.com/2016/01/31/arts/music/elton-john-still-living-in-the-flourish.html?module=Uisil}{300
million records in a career spanning five decades}, Mr. John was deep in
the final stages of preparation for his Farewell Yellow Brick Road, a
three-year, five-continent outing that will be his retirement from
touring.

There will be more than 300 concerts --- the first will take place in
Allentown, Pa., on Sept. 8 --- and a traveling wardrobe of at least two
dozen one-off Gucci outfits, designed by the creative director
\href{https://www.nytimes.com/2018/05/31/style/gucci-cruise-2019.html}{Alessandro
Michele} for his longtime muse and idol. It will be a fitting finale for
a man who always understood the power of a sequin. Give or take 10,000.

``Look, I'm not Mick Jagger, Rod Stewart or David Bowie, tearing from
one end of the stage to the other,'' Mr. John said, clad in an electric
pink T-shirt and glasses, swim shorts and black baseball cap (all by
Gucci). ``I'm always bloody stuck at the piano, aren't I? Clothes have
always had to be part of the show that I put on. They made me memorable.
Though I suppose with hindsight I did go totally and utterly crazy,
especially in the first 30 years of my career.''

Image

Mr. John performing at Twickenham Stoop in 2017.Credit...Ian Gavan/Getty
Images

Raised in a conservative postwar household in the outer London suburbs
in the 1950s, Mr. John termed himself ``a late rebeller''; someone for
whom the joy of dressing up only arrived in his late 20s as he emerged
from the plain chrysalis of Reginald Dwight (his birth name) and
refashioned himself as his own
\href{https://www.nytimes.com/2014/11/30/fashion/elton-john-and-darren-walker-on-race-sexual-identity-and-leaving-the-past-behind.html}{madcap
musical creation}, Elton Hercules John.

It was a gradual process: In the early years, the wacky glasses that
later became the singer's trademark allowed him to cover up the extreme
shyness he felt as a performer.

And for all the frothy outré humor that fell out of the closet with him,
whether the hot pants and fly boots created by the Mr. Freedom founder
Tommy Roberts in the 1970s or the feathery showgirl-inspired costumes
from Bob Mackie during the 1980s, stage outfits also offered a dazzling
armor against the gaze of the outside world.

Image

"I suppose with hindsight I did go totally and utterly crazy, especially
in the first 30 years of my career," Mr. John said of his early fashion
choices.Credit...David Redfern/Redferns, via Getty Images

``Any good costume makes you feel ready to perform. I arrive at a venue,
take a nap, then wake up and pick my outfit. Then I put it on. And
that's the moment when I become Elton John'' Mr. John said, looking a
little tired behind his rose-tinted glasses as he stared out over the
yachts moored in the Mediterranean Sea. ``When I take it all off, that
all disappears again. When I'm offstage I'm not Elton. To my boys, I'm
just Daddy.''

Raising those boys --- Zachary, 7, and Elijah, 5, his young sons with
his husband, David Furnish --- has changed his priorities, he said. Life
on the road no longer held much allure.

``I never thought I would say this. Ten years ago I thought I would die
on stage. Maybe I wanted to then, but I don't now,'' he said. ``I don't
want to keep getting up and flying away from the people I love for
months at a time anymore. At the end of the day, I've been a working
musician in the back of a touring van since I was 17 years old. I've had
a bloody good run. As I sink into the latter stage of my life, I want to
do things differently.''

Offstage, that may well be the case. But for his last hurrah ---
sartorially at least --- old habits seem to die hard, judging by his
tour wardrobe.

According to Jo Hambro, a former creative fashion director of British GQ
and a longtime friend and stylist to Mr. John, the stage outfits are an
``an 18-month passion project.''

There will be three costume changes per show; the first, the singer's
``maestro look,'' which centers on an embroidered tailcoat, will open
the night. Next will come a change into a bright, printed ``rock 'n'
roll'' suit before a final dressing gown over a tracksuit closes the
shows. The Gucci team has made Mr. John multiple options for each look
of the night.

Image

Samples of clothes made by Gucci for traveling.Credit...Julien Mignot
for The New York Times

Mr. Michele, the mastermind behind the much-heralded turnaround at
Gucci, met Mr. John three years in Los Angeles. The actor Jared Leto
introduced them, and they became fast friends.

Mr. Michele had grown up with photographs of Mr. John's flamboyant stage
outfits and album covers pinned onto his mood boards, his records
playing on repeat. The spring-summer 2018 Gucci catwalk show was strewn
with tributes to Mr. John's stage persona: crystal-encrusted swimming
caps, satin shell suits, rhinestone-studded jumpsuits and chunky
platform boots.

In March, Gucci introduced ``Levon,'' a capsule collection inspired by
Mr. John's 1971 single of the same name. It was but a short leap from
there to the tour wardrobe.

``I love Elton, he's a fantastic person, a great artist and true friend.
We have the same attitude, we share the same obsession for collecting
things.'' Mr. Michele wrote in an email. ``There has never been another
pianist pop star dressed quite like him, as a sort of pop divinity, a
great artist who has urged different generations of young people to look
for freedom.''

Image

Credit...Julien Mignot for The New York Times

Image

Credit...Julien Mignot for The New York Times

For Mr. John, Mr. Michele reminded him of another great fashion friend
and creative brother-in-arms: Gianni Versace, who had designed the
costumes for Mr. John's 1992 world tour, and from whom the singer bought
enormous amounts of clothing.

Mr. John applauded the fact that Gucci ``had blown like a hurricane''
through the modern fashion business and ``kicked it up the backside.''
He tutted at designers (Miuccia Prada among them) he felt had~ attempted
to copy the Gucci phenomenon, rather than sticking firmly to their own
vision and path.

``It's Alessandro's mad spirit, the lack of rules, the celebration of
individuality and the fact he's a magpie that makes me love him so
much,'' Mr. John said. ``He's just like Gianni: Nothing is off limits.
He just spoke to me in a way no one has for some time.''

``I had actually sort of given up on vibrant, more eclectic looks,'' he
said.'' I loved Hedi Slimane, but he only makes clothes for skinny,
skinny, skinny people. I like Dries Van Noten too. And I had been
wearing a lot of (Savile Row tailor) Richard James. But now I'll wear
Gucci in my daily life for as long as Alessandro is there. He's like
manna to heaven for me.''

Image

Because each concert is different, "we have the same models in different
colors," Ms. Hambro said.Credit...Julien Mignot for The New York Times

The project began, Ms. Hambro said, with ``this glorious lunch together
at Woodside, Elton's house in Windsor, picking through his mammoth
costume archives, Elton's little boys trying old outfits on, all of them
just behaving like kids in a candy shop.''

Mr. John noted that he had never gotten heavily involved with the
creative process behind the costumes made by any of his collaborators
(prompting Mr. Furnish to let out a quiet chuckle). ``I like to be
surprised. I just give them the perimeters of my persona and let them do
what they want to do,'' Mr. John said. ``I've never been let down yet.''

It's not like he doesn't have other things to think about, after all. He
is currently halfway through
\href{https://www.nytimes.com/2017/01/26/theater/the-devil-wears-prada-aims-for-broadway-as-musical.html?rref=collection\%2Ftimestopic\%2FJohn\%2C\%20Elton\&action=click\&contentCollection=timestopics\&region=stream\&module=stream_unit\&version=latest\&contentPlacement=5\&pgtype=collection}{writing
songs for a Broadway-aimed musical}adaptation of ``The Devil Wears
Prada.'' ``Rocketman,'' a movie about his life, starring Bryce Dallas
Howard and Jamie Bell, of which he is a producer, will be released in
2019, as will a
\href{https://www.nytimes.com/2016/10/13/arts/music/elton-john-autobiography.html?rref=collection\%2Ftimestopic\%2FJohn\%2C\%20Elton\&action=click\&contentCollection=timestopics\&region=stream\&module=stream_unit\&version=latest\&contentPlacement=1\&pgtype=collection}{``no-holds-barred
autobiography,''} written with Alexis Petridis, a journalist.

And there is plenty more that he wants to do, including a book on his
houses and another on his clothes. Including, presumably his final tour
costumes.

``I haven't worn a brand en masse like this since Gianni died in 1997,''
Mr. John said, gesticulating to his head-to-toe Gucci. ``Now I'm
obsessed.''

There would inevitably be some mishaps on tour, he added, using somewhat
saltier language. ``But one thing I'll tell you for absolute certain,''
he said. ``There won't be any wardrobe malfunctions.''

Advertisement

\protect\hyperlink{after-bottom}{Continue reading the main story}

\hypertarget{site-index}{%
\subsection{Site Index}\label{site-index}}

\hypertarget{site-information-navigation}{%
\subsection{Site Information
Navigation}\label{site-information-navigation}}

\begin{itemize}
\tightlist
\item
  \href{https://help.nytimes.com/hc/en-us/articles/115014792127-Copyright-notice}{©~2020~The
  New York Times Company}
\end{itemize}

\begin{itemize}
\tightlist
\item
  \href{https://www.nytco.com/}{NYTCo}
\item
  \href{https://help.nytimes.com/hc/en-us/articles/115015385887-Contact-Us}{Contact
  Us}
\item
  \href{https://www.nytco.com/careers/}{Work with us}
\item
  \href{https://nytmediakit.com/}{Advertise}
\item
  \href{http://www.tbrandstudio.com/}{T Brand Studio}
\item
  \href{https://www.nytimes.com/privacy/cookie-policy\#how-do-i-manage-trackers}{Your
  Ad Choices}
\item
  \href{https://www.nytimes.com/privacy}{Privacy}
\item
  \href{https://help.nytimes.com/hc/en-us/articles/115014893428-Terms-of-service}{Terms
  of Service}
\item
  \href{https://help.nytimes.com/hc/en-us/articles/115014893968-Terms-of-sale}{Terms
  of Sale}
\item
  \href{https://spiderbites.nytimes.com}{Site Map}
\item
  \href{https://help.nytimes.com/hc/en-us}{Help}
\item
  \href{https://www.nytimes.com/subscription?campaignId=37WXW}{Subscriptions}
\end{itemize}
