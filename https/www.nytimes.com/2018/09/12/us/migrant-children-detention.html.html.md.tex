Sections

SEARCH

\protect\hyperlink{site-content}{Skip to
content}\protect\hyperlink{site-index}{Skip to site index}

\href{https://www.nytimes.com/section/us}{U.S.}

\href{https://myaccount.nytimes.com/auth/login?response_type=cookie\&client_id=vi}{}

\href{https://www.nytimes.com/section/todayspaper}{Today's Paper}

\href{/section/us}{U.S.}\textbar{}Detention of Migrant Children Has
Skyrocketed to Highest Levels Ever

\href{https://nyti.ms/2Nat9tC}{https://nyti.ms/2Nat9tC}

\begin{itemize}
\item
\item
\item
\item
\item
\item
\end{itemize}

Advertisement

\protect\hyperlink{after-top}{Continue reading the main story}

Supported by

\protect\hyperlink{after-sponsor}{Continue reading the main story}

\hypertarget{detention-of-migrant-children-has-skyrocketed-to-highest-levels-ever}{%
\section{Detention of Migrant Children Has Skyrocketed to Highest Levels
Ever}\label{detention-of-migrant-children-has-skyrocketed-to-highest-levels-ever}}

\includegraphics{https://static01.nyt.com/images/2018/06/20/reader-center/13MIGRANTKIDS-01/merlin_139819185_e47ce2d3-4c06-48ab-af2d-2729dcdbcd6f-articleLarge.jpg?quality=75\&auto=webp\&disable=upscale}

By \href{https://www.nytimes.com/by/caitlin-dickerson}{Caitlin
Dickerson}

\begin{itemize}
\item
  Sept. 12, 2018
\item
  \begin{itemize}
  \item
  \item
  \item
  \item
  \item
  \item
  \end{itemize}
\end{itemize}

Even though hundreds of children separated from their families after
crossing the border have been released under court order, the overall
number of detained migrant children has exploded to the highest ever
recorded --- a significant counternarrative to the Trump
administration's efforts to reduce the number of undocumented families
coming to the United States.

Population levels at federally contracted shelters for migrant children
have quietly shot up more than fivefold since last summer, according to
data obtained by The New York Times, reaching a total of 12,800 this
month. There were 2,400 such children in custody in May 2017.

The huge increases, which have placed the federal shelter system near
capacity, are due not to an influx of children entering the country, but
a reduction in the number being released to live with families and other
sponsors, the data collected by the Department of Health and Human
Services suggests. Some of those who work in the migrant shelter network
say the bottleneck is straining both the children and the system that
cares for them.

Most of the children crossed the border alone, without their parents.
Many are teenagers from Central America, and they are housed in a system
of more than 100 shelters across the United States, with the highest
concentration near the southwest border.

The new data was reported to members of Congress, who shared it with The
Times. It shows that despite the Trump administration's efforts to
discourage Central American migrants, roughly the same number of
children are crossing the border as in years past. The big difference,
said those familiar with the shelter system, is that red tape and fear
brought on by stricter immigration enforcement have discouraged
relatives and family friends from coming forward to sponsor children.

Shelter capacities have hovered close to 90 percent since at least May,
compared to about 30 percent a year ago. Any new surge in border
crossings, which could happen at any time, could quickly overwhelm the
system, operators say.

``The closer they get to 100 percent, the less ability they will have to
address anything unforeseen,'' said Mark Greenberg, who oversaw the care
of migrant children for the Health and Human Services Department under
President Barack Obama. ``Even if there's not a sudden influx, they will
be running out of capacity soon unless something changes.''

The administration appeared to move to address that on Tuesday, when it
announced that it will triple the size of a temporary ``tent city'' in
Tornillo, Tex., to house up to 3,800 children through the end of the
year. Immigrant advocates and members of Congress reacted to the news
with distress, because conditions are comparatively harsh in such large
overflow facilities, compared with traditional shelters.

Facilities like the one in Tornillo are also more expensive to operate,
according to Representative Rosa DeLauro of Connecticut, the ranking
Democrat on the House Appropriations subcommittee that funds the shelter
program. She said such facilities cost about \$750 per child per day, or
three times the amount of a typical shelter.

``You are flying in the face of child welfare, and we're doing it by
design,'' Ms. DeLauro said. ``You drive up the cost and you prolong the
trauma on these children.''

Federal authorities said they were dealing with high levels of illegal
border crossings and requests for asylum. ``The number of unaccompanied
alien children apprehended are a symptom of the larger issue of a broken
immigration system,'' Evelyn Stauffer, press secretary for the
Department of Health and Human Services, said in a statement. ``That is
why H.H.S. joins the president in calling on Congress to address this
broken system and the pull factors that have led to increasing numbers
at the U.S. border.''

The system for sheltering migrant children came under scrutiny this
summer, when more than 2,500 children who were separated from their
parents were housed in federally contracted shelters under the Trump
administration's zero tolerance border enforcement policy. But those
children were only a fraction of the total number of children who are
currently detained.

Historically, children categorized as ``unaccompanied'' have been placed
with sponsors, such as parents already in the United States, extended
family members or family friends, as soon as the sponsors can be vetted
by federal authorities. But the new data shows that the placement
process has slowed significantly. Monthly releases have plummeted by
about two-thirds since last year.

The delays in vetting sponsors relate, in part, to changes the Trump
administration has made in how the process works. In June, the
authorities announced that potential sponsors and other adult members of
their households would have to submit fingerprints, and that the data
would be shared with immigration authorities.

Traditionally, most sponsors have been undocumented themselves, and
therefore are wary of risking deportation by stepping forward to claim
sponsorship of a child. Even those who are willing to become sponsors
have had to wait months to be fingerprinted and otherwise reviewed.

Federal officials say their vetting procedures are designed to safeguard
the children in their care.

``Children who enter the country illegally are at high risk for
exploitation by traffickers and smugglers,'' Ms. Stauffer said in her
statement.

But the longer children are detained, the more anxious and depressed
they are likely to become, according to Mr. Greenberg, who oversaw the
program under Mr. Obama. When that happens, children may try to harm
themselves or escape, and can become violent with the staff and with one
another, he said.

\href{https://www.nytimes.com/2018/06/24/us/migrant-boy-leaves-texas-shelter.html}{Stories}
of such behavior
\href{https://www.nytimes.com/2018/07/14/us/migrant-children-shelters.html}{have
emerged} through reporting in recent months as the shelter system has
faced intense criticism by members of Congress and the public.

``Being in congregate care for an extended period of time is not a good
thing. It increases the likelihood of things going wrong,'' Mr.
Greenberg said.

The administration funneled children who were separated from their
parents into the shelter system this summer under the earlier policy,
without any apparent collaboration with the officials who oversee the
shelter program.

The separated children injected a new degree of chaos into the
facilities, according to several shelter operators, who spoke
anonymously because they are barred by the government from speaking to
the news media. The children were younger and more traumatized than
those the shelters were used to dealing with, and they arrived without a
plan for when they could be released or to whom.

But the system had already been overwhelmed for months, operators said,
as children continued to flow in while fewer were being discharged.

The shelter system has overflowed before. In 2014, when unaccompanied
children flooded across the border in unprecedented numbers, a lack of
shelter space led to a backup of children at the border in what
authorities referred to at the time as a humanitarian crisis.

Since then, new facilities have been constructed or arranged by contract
--- and they are now nearing capacity.

Advertisement

\protect\hyperlink{after-bottom}{Continue reading the main story}

\hypertarget{site-index}{%
\subsection{Site Index}\label{site-index}}

\hypertarget{site-information-navigation}{%
\subsection{Site Information
Navigation}\label{site-information-navigation}}

\begin{itemize}
\tightlist
\item
  \href{https://help.nytimes.com/hc/en-us/articles/115014792127-Copyright-notice}{©~2020~The
  New York Times Company}
\end{itemize}

\begin{itemize}
\tightlist
\item
  \href{https://www.nytco.com/}{NYTCo}
\item
  \href{https://help.nytimes.com/hc/en-us/articles/115015385887-Contact-Us}{Contact
  Us}
\item
  \href{https://www.nytco.com/careers/}{Work with us}
\item
  \href{https://nytmediakit.com/}{Advertise}
\item
  \href{http://www.tbrandstudio.com/}{T Brand Studio}
\item
  \href{https://www.nytimes.com/privacy/cookie-policy\#how-do-i-manage-trackers}{Your
  Ad Choices}
\item
  \href{https://www.nytimes.com/privacy}{Privacy}
\item
  \href{https://help.nytimes.com/hc/en-us/articles/115014893428-Terms-of-service}{Terms
  of Service}
\item
  \href{https://help.nytimes.com/hc/en-us/articles/115014893968-Terms-of-sale}{Terms
  of Sale}
\item
  \href{https://spiderbites.nytimes.com}{Site Map}
\item
  \href{https://help.nytimes.com/hc/en-us}{Help}
\item
  \href{https://www.nytimes.com/subscription?campaignId=37WXW}{Subscriptions}
\end{itemize}
