Sections

SEARCH

\protect\hyperlink{site-content}{Skip to
content}\protect\hyperlink{site-index}{Skip to site index}

\href{https://www.nytimes.com/section/world/europe}{Europe}

\href{https://myaccount.nytimes.com/auth/login?response_type=cookie\&client_id=vi}{}

\href{https://www.nytimes.com/section/todayspaper}{Today's Paper}

\href{/section/world/europe}{Europe}\textbar{}Hungary Keeps Visit by
Putin Low-Key as It Seeks to Repair Relations With West

\url{https://nyti.ms/1CFWM7r}

\begin{itemize}
\item
\item
\item
\item
\item
\end{itemize}

Advertisement

\protect\hyperlink{after-top}{Continue reading the main story}

Supported by

\protect\hyperlink{after-sponsor}{Continue reading the main story}

\hypertarget{hungary-keeps-visit-by-putin-low-key-as-it-seeks-to-repair-relations-with-west}{%
\section{Hungary Keeps Visit by Putin Low-Key as It Seeks to Repair
Relations With
West}\label{hungary-keeps-visit-by-putin-low-key-as-it-seeks-to-repair-relations-with-west}}

\includegraphics{https://static01.nyt.com/images/2015/02/18/world/18HUNGARY/18HUNGARY-articleLarge.jpg?quality=75\&auto=webp\&disable=upscale}

By \href{https://www.nytimes.com/by/rick-lyman}{Rick Lyman} and Helene
Bienvenu

\begin{itemize}
\item
  Feb. 17, 2015
\item
  \begin{itemize}
  \item
  \item
  \item
  \item
  \item
  \end{itemize}
\end{itemize}

WARSAW --- When Vladimir V. Putin played host at the Kremlin early last
year to Hungary's prime minister,
\href{http://topics.nytimes.com/top/reference/timestopics/people/o/viktor_orban/index.html?8qa}{Viktor
Orban}, the notion of a reciprocal visit that would bring the Russian
president onto European Union soil seemed perfectly in keeping with a
whole series of stick-in-the-eye moves toward Europe by the two leaders
in recent years.

But by Tuesday, when Mr. Putin and his entourage finally touched down in
Budapest, the meeting seemed to demonstrate less a fresh diplomatic
conquest than a demonstration of the Russian leader's shrunken
diplomatic reach.

The combination of a festering Ukraine crisis, Russia's growing economic
woes and Mr. Orban's desire to repair relations with the West made this
latest stop in Mr. Putin's increasingly active itinerary --- he traveled
to Egypt earlier this month, and stopped in India, Turkey and Uzbekistan
in December --- into a low-key, blink-and-you'll-miss-it event.

``I think it is not accidental that the Hungarian government did not
want to over-promote this meeting right now,'' said Peter Kreko,
director of the Political Capital Institute, a Budapest research group.
``Orban realized quite late that while it was completely O.K. to do
business with Russia before the Ukraine crisis, it is not the same
since.''

It was a far cry from Mr. Putin's triumphant reception in Serbia in
October, when the state news media celebrated his arrival and leaders
threw a military parade, complete with aerial acrobatics involving
Serbian and Russian jets, and presented him with the Order of the
Republic, Serbia's highest honor.

In Budapest, Mr. Putin arrived Tuesday with an entourage heavy with
energy officials and then kept largely out of sight.

He laid flowers on the graves of Russian soldiers killed in World War
II, drawing some criticism because the same section of the cemetery
contains a monument to Russian troops killed in 1956 while quashing an
anti-Communist uprising. He had private meetings with Mr. Orban and with
President Janos Ader of Hungary. Five bilateral agreements were signed.

But Mr. Putin's only major public event was an evening news conference
with Mr. Orban.

Mr. Putin confined the bulk of his remarks to economic deals between the
two nations, and mentioned Ukraine and its fragile cease-fire only
briefly, reacting mildly to a question about the United States' possibly
sending defensive weapons to Ukraine.

``I am deeply convinced that no matter what weapons you provide to
Ukraine, it is always bad to supply arms to an armed conflict,'' he
said.

For his part, Mr. Orban refrained from his past praise of ``illiberal
democracy'' and made no mention of his implacable opposition to
``economic immigration.'' Instead, he blandly praised the
\href{http://www.nytimes.com/reuters/2015/02/12/world/europe/12reuters-ukraine-crisis-minsk-agreement-factbox.html}{cease-fire
agreement} made in Minsk, Belarus, and advocated finding a way to repair
Russia's broken relationship with the European Union.

``We are convinced that the isolation of Russia from Europe is not
feasible,'' he said.

The cooler tenor of Mr. Putin's visit to Budapest ``will send a clear
message to Russian politicians that this is not Serbia,'' Mr. Kreko
said.

The visit, arranged last month, had become more important to Mr. Putin,
who wanted to demonstrate that at least one European Union leader was
still open to a bilateral visit, said Zoltan Sz. Biro, a historian who
served as foreign policy adviser in a previous Socialist government.

``Putin requested the visit,'' Mr. Biro said. ``Orban is embarrassed by
it but couldn't refuse.''

Since Mr. Orban drew sharp criticism last year from Western leaders for
his pro-Russian rhetoric and growing authoritarian moves, which
opponents compared to Mr. Putin's governing style, his government has
softened its stance --- outwardly, at least --- stressing that it
continued to vote for the European Union's sanctions against Russia.

``There is a definite sign of correction,'' Mr. Kreko said. ``You can
see it in the rhetoric.''

Last year, Mr. Orban drew flak from Western leaders for calling the
sanctions against Russia counterproductive and for voicing concerns
about the autonomy of ethnic Hungarians in Ukraine in language similar
to that used by Mr. Putin to justify his support for ethnic Russian
rebels.

Now, he pointedly preceded the Putin visit with a trip to Kiev, where he
met with President Petro O. Poroshenko of Ukraine and reiterated his
support for Ukraine's territorial integrity.

Chancellor Angela Merkel of Germany visited Budapest this month in a
high-profile event.

The idea of warmer relations with Russia is a thorny one for many
Hungarians, especially older ones who remember the long years under
Communist rule. But financially struggling Hungarians are also eager for
any economic lifelines that might provide investment and jobs.

About 2,000 anti-Putin protesters gathered Monday evening outside
Budapest's eastern train station and marched through the streets to the
western station.

The crowd waved Hungarian, Ukrainian and European Union flags and
carried signs with slogans like ``Russians Go Home!'' while loudspeakers
played ``Back in the U.S.S.R.'' by the Beatles.

Andras Nemeth, 31, an economist who joined the line of protesters, said
he saw some disturbing similarities between Mr. Putin and Mr. Orban.

``They like to be at the center of power,'' he said, ``and use their
charisma to keep control of their population and convey easy, populist
messages.''

Advertisement

\protect\hyperlink{after-bottom}{Continue reading the main story}

\hypertarget{site-index}{%
\subsection{Site Index}\label{site-index}}

\hypertarget{site-information-navigation}{%
\subsection{Site Information
Navigation}\label{site-information-navigation}}

\begin{itemize}
\tightlist
\item
  \href{https://help.nytimes.com/hc/en-us/articles/115014792127-Copyright-notice}{©~2020~The
  New York Times Company}
\end{itemize}

\begin{itemize}
\tightlist
\item
  \href{https://www.nytco.com/}{NYTCo}
\item
  \href{https://help.nytimes.com/hc/en-us/articles/115015385887-Contact-Us}{Contact
  Us}
\item
  \href{https://www.nytco.com/careers/}{Work with us}
\item
  \href{https://nytmediakit.com/}{Advertise}
\item
  \href{http://www.tbrandstudio.com/}{T Brand Studio}
\item
  \href{https://www.nytimes.com/privacy/cookie-policy\#how-do-i-manage-trackers}{Your
  Ad Choices}
\item
  \href{https://www.nytimes.com/privacy}{Privacy}
\item
  \href{https://help.nytimes.com/hc/en-us/articles/115014893428-Terms-of-service}{Terms
  of Service}
\item
  \href{https://help.nytimes.com/hc/en-us/articles/115014893968-Terms-of-sale}{Terms
  of Sale}
\item
  \href{https://spiderbites.nytimes.com}{Site Map}
\item
  \href{https://help.nytimes.com/hc/en-us}{Help}
\item
  \href{https://www.nytimes.com/subscription?campaignId=37WXW}{Subscriptions}
\end{itemize}
