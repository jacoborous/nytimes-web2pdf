Sections

SEARCH

\protect\hyperlink{site-content}{Skip to
content}\protect\hyperlink{site-index}{Skip to site index}

\href{https://www.nytimes.com/section/world/asia}{Asia Pacific}

\href{https://myaccount.nytimes.com/auth/login?response_type=cookie\&client_id=vi}{}

\href{https://www.nytimes.com/section/todayspaper}{Today's Paper}

\href{/section/world/asia}{Asia Pacific}\textbar{}The Trans-Pacific
Partnership and a President's Legacy

\url{https://nyti.ms/1HHNqMO}

\begin{itemize}
\item
\item
\item
\item
\item
\item
\end{itemize}

Advertisement

\protect\hyperlink{after-top}{Continue reading the main story}

Supported by

\protect\hyperlink{after-sponsor}{Continue reading the main story}

News Analysis

\hypertarget{the-trans-pacific-partnership-and-a-presidents-legacy}{%
\section{The Trans-Pacific Partnership and a President's
Legacy}\label{the-trans-pacific-partnership-and-a-presidents-legacy}}

\includegraphics{https://static01.nyt.com/images/2015/06/16/world/15Prexy-web/15Prexy-web-articleLarge.jpg?quality=75\&auto=webp\&disable=upscale}

By \href{http://www.nytimes.com/by/peter-baker}{Peter Baker}

\begin{itemize}
\item
  June 14, 2015
\item
  \begin{itemize}
  \item
  \item
  \item
  \item
  \item
  \item
  \end{itemize}
\end{itemize}

WASHINGTON --- For more than six years, the short walk from the Oval
Office downstairs to the Situation Room has all too often meant bad news
or grim choices. Whether it was war in the Middle East, Russian
aggression in Ukraine or the hunt for terrorists around the globe,
President Obama's foreign policy has felt consumed by guns and drones.

So the 12-nation trade deal Mr. Obama has been negotiating in Asia took
on special meaning for a president eager to change the world. It was a
way to leave behind a positive legacy abroad, one that could be
measured, he hoped, by the number of lives improved rather than by the
number of bodies left behind. And if the Pacific really is the future,
Mr. Obama wanted to position the United States to lead the way.

As it turned out, the biggest challenge to securing that legacy has been
at home, and not overseas, as Mr. Obama's fellow Democrats last week
\href{http://www.nytimes.com/2015/06/13/us/politics/obamas-trade-bills-face-tough-battle-against-house-democrats.html}{shot
down legislation} crucial to finalizing the trade agreement on the
grounds that it would hurt rather than help America. Unless he can
convince scores of Democrats to change their votes in the coming days,
the centerpiece of his much-touted re-engagement with Asia will slip
away along with one of the last chances he has to leave his imprint on
the world before leaving office.

``If the president cannot get'' trade promotion authority ``through
Congress, it is a disaster for his Asia policy,'' said Michael J. Green,
a former Asia adviser to President George W. Bush and now at Georgetown
University and the Center for Strategic and International Studies. ``The
administration will be dismissed as lame duck at a time when China is
flexing its muscles.''

Moreover, Mr. Green and other analysts said, a failure to follow through
on the trade deal would lead to Japan, Vietnam and other putative
partners reversing course on economic reforms or tariff concessions
required to join the multilateral trade zone with the United States,
known as the Trans-Pacific Partnership, or T.P.P. And momentum may shift
to economic institutions and agreements that do not include the United
States, including the new
\href{http://www.nytimes.com/2015/04/03/world/asia/china-asian-infrastructure-investment-bank.html}{Asian
Infrastructure Investment Bank} that China is creating over American
resistance.

``Domestically we tend to view trade through a political prism by way of
winners and losers,'' said Jon Huntsman, a Republican former governor of
Utah who served as Mr. Obama's ambassador to China before mounting a
campaign to challenge his re-election in 2012. ``In Asia, it's seen as
directly tied to our leadership and commitment to the region. A failed
T.P.P. would create an influence vacuum that others, primarily China,
would fill.''

The trade agreement, about a decade in the making, would stitch together
the United States with 11 other nations along the Pacific Rim, including
Canada, Mexico, Japan, Vietnam, Malaysia and Australia, creating a
free-trade zone for about 40 percent of the world's economy. It would
lower tariffs, while setting rules for resolving trade disputes, setting
patents and protecting intellectual property. China is not part of the
group.

Mr. Obama, congressional Republicans and business groups argue that it
will unlock foreign markets to American goods and level the playing
field by forcing Asian competitors to improve labor and environmental
standards. But House Democrats, labor unions and environmental groups
argue that it will benefit big corporations, further bleed American
manufacturing jobs and fail to adequately enforce the workplace
standards it promises.

The administration tried making a foreign policy argument over the last
few weeks, maintaining that if the United States does not seal the trade
pact, it will be leaving the field to China, which has been exerting its
clout in recent years, whether by investing in energy supplies in Africa
and the Middle East or by asserting claims over disputed waters and
islands.

But some on the left argue that the administration is using China to
scare lawmakers and exaggerating the competition. ``I just don't buy
this China boogeyman stuff,'' said Jared Bernstein, a senior fellow at
the Center on Budget and Policy Priorities and a former economic adviser
to Vice President Joseph R. Biden Jr. ``If anything, I see China as
being more inward looking, devoting less resources to mercantile-type
trade and more to internal investment, consumption and developing human
capital.''

Supporters of the trade pact hope to hold a new vote this week on the
part of the trade package rejected by the House on Friday but will need
to secure 90 more votes. Representative Paul D. Ryan, the Wisconsin
Republican who has worked with the White House to secure trade
negotiating authority, said on Sunday that if Mr. Obama wanted to avoid
being a ``very lame-duck president,'' he would have to win over members
of his own party.

``I think that this can be salvaged because I think people are going to
realize just how big the consequences are for American leadership,'' Mr.
Ryan said on ``Fox News Sunday.''

But there was little indication over the weekend that many minds had
been changed on the political left. ``We need to regroup and come up
with a trade policy which demands that corporate America start investing
in this country rather than in countries all over the world,'' Senator
Bernie Sanders of Vermont, a candidate for the Democratic presidential
nomination, said on ``Face the Nation'' on CBS.

The Pacific trade pact was meant to be one of three major foreign policy
achievements Mr. Obama wanted to secure before leaving office; all three
are on the line this summer. He faces a June 30 deadline to seal an
agreement with Iran to scale back its nuclear program and he hopes to
follow through on his reconciliation with Cuba by formally restoring
diplomatic relations.

Those three initiatives take on even more significance for Mr. Obama as
he confronts the reality that he is likely to turn over the White House
to his successor without having defeated the Islamic State in Syria and
Iraq or resolved the conflict with Russia over Ukraine. He still plans
to leave office as the president who ended the American war in
Afghanistan, but the security situation there remains fluid.

Amid all those wartime issues, the Asia initiative was to be the
long-term investment that would pay off years later. The White House has
deployed more military forces to the region and made it a focus. But the
trade pact was to be the most tangible element of the policy.

The trade talks have reached a decisive turning point. Negotiators have
drafted the bulk of the agreement but other countries are holding back
to see if Mr. Obama wins the authority he needs from Congress before
completing the pact. With Asia on his mind, Mr. Obama recently hosted
Prime Minister Shinzo Abe of Japan and he now plans to welcome President
Xi Jinping of China to the White House in coming weeks, a meeting that
will be colored by his failure to achieve his trade goal should he not
turn House Democrats around.

``There is no Asia pivot without an economic component, and that
component is tied up in T.P.P.,'' said Walter Lohman, director of the
Asian studies program at the Heritage Foundation. The challenge for Mr.
Obama, he added, is that no matter how important the trade pact may be
to his foreign policy, Congress will consider it through an economic
lens. ``Geopolitics gets it very few votes.''

Peter A. Petri, a professor of international finance at Brandeis
University, said he just returned from a trip to Asia. ``Many people
there are dismayed by our political impasse,'' he said. ``They simply
don't understand or want an America that is defensive and distant.'' He
said that was actually true of China as well, despite the
competitiveness between the two countries.

``The pivot makes no sense without American economic partnerships,'' he
added, ``and taking economics out of the relationship would leave only
zero-sum strategic competition.''

Advertisement

\protect\hyperlink{after-bottom}{Continue reading the main story}

\hypertarget{site-index}{%
\subsection{Site Index}\label{site-index}}

\hypertarget{site-information-navigation}{%
\subsection{Site Information
Navigation}\label{site-information-navigation}}

\begin{itemize}
\tightlist
\item
  \href{https://help.nytimes.com/hc/en-us/articles/115014792127-Copyright-notice}{©~2020~The
  New York Times Company}
\end{itemize}

\begin{itemize}
\tightlist
\item
  \href{https://www.nytco.com/}{NYTCo}
\item
  \href{https://help.nytimes.com/hc/en-us/articles/115015385887-Contact-Us}{Contact
  Us}
\item
  \href{https://www.nytco.com/careers/}{Work with us}
\item
  \href{https://nytmediakit.com/}{Advertise}
\item
  \href{http://www.tbrandstudio.com/}{T Brand Studio}
\item
  \href{https://www.nytimes.com/privacy/cookie-policy\#how-do-i-manage-trackers}{Your
  Ad Choices}
\item
  \href{https://www.nytimes.com/privacy}{Privacy}
\item
  \href{https://help.nytimes.com/hc/en-us/articles/115014893428-Terms-of-service}{Terms
  of Service}
\item
  \href{https://help.nytimes.com/hc/en-us/articles/115014893968-Terms-of-sale}{Terms
  of Sale}
\item
  \href{https://spiderbites.nytimes.com}{Site Map}
\item
  \href{https://help.nytimes.com/hc/en-us}{Help}
\item
  \href{https://www.nytimes.com/subscription?campaignId=37WXW}{Subscriptions}
\end{itemize}
