Sections

SEARCH

\protect\hyperlink{site-content}{Skip to
content}\protect\hyperlink{site-index}{Skip to site index}

\href{https://www.nytimes.com/section/us}{U.S.}

\href{https://myaccount.nytimes.com/auth/login?response_type=cookie\&client_id=vi}{}

\href{https://www.nytimes.com/section/todayspaper}{Today's Paper}

\href{/section/us}{U.S.}\textbar{}Dzhokhar Tsarnaev Given Death Penalty
in Boston Marathon Bombing

\url{https://nyti.ms/1d3IpCz}

\begin{itemize}
\item
\item
\item
\item
\item
\item
\end{itemize}

Advertisement

\protect\hyperlink{after-top}{Continue reading the main story}

Supported by

\protect\hyperlink{after-sponsor}{Continue reading the main story}

\hypertarget{dzhokhar-tsarnaev-given-death-penalty-in-boston-marathon-bombing}{%
\section{Dzhokhar Tsarnaev Given Death Penalty in Boston Marathon
Bombing}\label{dzhokhar-tsarnaev-given-death-penalty-in-boston-marathon-bombing}}

By \href{http://www.nytimes.com/by/katharine-q-seelye}{Katharine Q.
Seelye}

\begin{itemize}
\item
  May 15, 2015
\item
  \begin{itemize}
  \item
  \item
  \item
  \item
  \item
  \item
  \end{itemize}
\end{itemize}

BOSTON --- Two years after bombs in two backpacks transformed the Boston
Marathon from a sunny rite of spring to a smoky battlefield with bodies
dismembered, a federal jury on Friday condemned Dzhokhar Tsarnaev to
death for his role in the 2013 attack.

In a sweeping rejection of the defense case, the jury found that death
was the appropriate punishment for six of 17 capital counts --- all six
related to Mr. Tsarnaev's planting of a pressure-cooker bomb on Boylston
Street, which his lawyers never disputed. Mr. Tsarnaev, 21, stood
stone-faced in court, his hands folded in front of him, as the verdict
was read, his lawyers standing grimly at his side.

Immediate reaction was mostly subdued.

``Happy is not the word I would use,'' said Karen Brassard, who suffered
grievous leg injuries in the bombing. ``There's nothing happy about
having to take somebody's life. I'm satisfied, I'm grateful that they
came to that conclusion, because for me I think it was the just
conclusion.''

Image

Dzhokar TsarnaevCredit...Federal Bureau of Investigation

She said she understood that all-but-certain appeals meant the case
could drag out over years if not decades. ``But right now,'' she said,
``it feels like we can take a breath and kind of actually breathe
again.''

The bombings two years ago turned one of this city's most cherished
athletic events into a grim tragedy --- the worst terrorist attack on
American soil since Sept. 11, 2001. Three people were killed, and 17
people lost at least one leg. More than 240 others sustained serious
injuries.

Last month, after deliberating for 11 hours, the jury found Mr. Tsarnaev
guilty of all 30 charges against him in connection with the bombings and
the death a few days later of a fourth person, an M.I.T. police officer.
The same jury spent 14 hours over three days deliberating the sentence.

With its decision, the jury rejected virtually every argument that the
defense put forth, including the centerpiece of its case --- that Mr.
Tsarnaev's older brother, Tamerlan, had held a malevolent sway over him
and led him into committing the crimes.

\includegraphics{https://static01.nyt.com/images/2015/05/15/multimedia/tsarnaev-sentence/tsarnaev-sentence-videoSixteenByNine3000.jpg}

According to verdict forms that the jurors completed, only three of the
12 jurors believed that Dzhokhar Tsarnaev had acted under his brother's
influence.

Beyond that, the jury put little stock in any part of the defense. Only
two jurors believed that Mr. Tsarnaev had expressed sorrow and remorse
for his actions, a stinging rebuke to the assertion by Sister Helen
Prejean, a Roman Catholic nun and renowned death penalty opponent, that
he was ``genuinely sorry'' for what he had done.

When the jury entered the courtroom at 3:10 p.m. Friday, the forewoman
passed an envelope to Judge George A. O'Toole Jr. of United States
District Court, who had presided over the case. Jurors remained standing
while the clerk read aloud the 24-page verdict form, which took 20
minutes. It was not clear until the end that the sentence was death,
though all signs along the way pointed in that direction.

Not a sound was heard in the packed courtroom throughout the
proceedings. Those in attendance --- survivors, victims' families, the
public, the news media --- had been sternly warned that any outburst
would amount to contempt of court.

\includegraphics{https://static01.nyt.com/images/2015/05/16/us/16MARATHONWEB3/16MARATHONWEB3-articleLarge.jpg?quality=75\&auto=webp\&disable=upscale}

The Tsarnaev verdict goes against the grain in Massachusetts, which has
no death penalty for state crimes. Throughout the trial,
\href{http://www.nytimes.com/2015/03/24/us/most-boston-residents-prefer-life-term-over-death-penalty-in-marathon-case-poll-shows.html}{polls}
also showed that residents overwhelmingly favored life in prison for Mr.
Tsarnaev.

Many respondents said that life in prison for one so young would be a
fate worse than death, and some worried that execution would make him a
martyr.

But all the jurors in his case had to be ``death qualified'' --- willing
to impose the death penalty to serve. In that sense, the jury was not
representative of the state.

Mayor Martin J. Walsh said in a statement that the sentencing brought
``a small amount of closure to the survivors, families and all impacted
by the violent and tragic events.'' His statement avoided explicit
praise of the verdict.

Some legal experts said that the jury's 14 hours of deliberations seemed
relatively quick in a case this complex. Eric M. Freedman, a death
penalty specialist at Hofstra University Law School, said that the
relative speed of the verdict could provide the defense with two
possible grounds for appeal: ``the failure to grant a change of venue,
despite the overwhelming evidence the defense presented about community
attitudes in Boston,'' he said, and ``the failure to instruct the jury
that if a single juror refused to vote for death, the result would be a
life sentence.''

``Unfortunately for all concerned,'' Mr. Freedman said, ``this is only
the first step on a long road.''

But other lawyers said that 14 hours was not all that fast and doubted
that it provided grounds for appeal.

``I've seen juries return verdicts in 25 minutes if the evidence is
strong,'' said Michael Kendall, a former federal prosecutor in Boston.
``But rarely do you have a case like this --- a crime of such enormity
to start with, plus a mountain of evidence and a defendant who is so
unsympathetic.''

He said he thought the jury had been struck by Mr. Tsarnaev's
callousness. ``After he blows up this child on purpose,'' he said of
8-year-old Martin Richard, the youngest of the victims, ``he's out at
the convenience store buying milk, then he smokes a little dope and
plans on blowing up New York.''

Image

Judy Clarke, center, and Tim Watkins, left, lawyers for Dzhokhar
Tsarnaev, had no comment leaving court in Boston on
Friday.Credit...Brian Snyder/Reuters

Among those in the courtroom were Bill and Denise Richard, the parents
of Martin and of a daughter, Jane, who was 7 when she lost a leg in the
attack. Despite their losses, the Richard family had called for Mr.
Tsarnaev to receive life in prison. They said they feared that appeals
would drag out a death sentence for years, making it hard for them to
move forward with their lives.

The jury, which was not sequestered, had been told to shield itself from
news accounts of the trial, and it is not known whether word of the
Richard family's decision had filtered through to any of the jurors.

Many of the jurors looked emotionally depleted after the sentence was
read, with some near tears. They had been involved in the case since
January, when jury selection began, and had heard testimony over 10
weeks, much of it gruesome and horrific as survivors described losing
their limbs and their loved ones.

Judge O'Toole did not set a date for formally sentencing Mr. Tsarnaev.
But at that point, some of the survivors will have a chance to tell the
court --- and Mr. Tsarnaev --- how the bombings had affected their
lives.

Image

Carlos Arredondo, outside the federal courthouse in Boston on Friday,
became known as the cowboy-hat-wearing bystander who helped rescue Jeff
Bauman, who lost his legs in the attack.Credit...Sean Proctor for The
New York Times

It was the first time a federal jury had sentenced a terrorist to death
in the post-Sept. 11 era, according to Kevin McNally, director of the
Federal Death Penalty Resource Counsel Project, which coordinates the
defense in capital punishment cases.

Attorney General Loretta E. Lynch called the death sentence a ``fitting
punishment.''

In Russia, when informed of the verdict by a reporter, Mr. Tsarnaev's
father, Anzor, simply exhaled and hung up. He then turned off his
cellphone.

Prosecutors had portrayed Mr. Tsarnaev, who immigrated to Cambridge,
Mass., from the Russian Caucasus with his family in 2002, as a
coldblooded, unrepentant jihadist who sought to kill innocent Americans
in retaliation for the deaths of innocent Muslims in American-led wars
in Iraq and Afghanistan.

``After all of the carnage and fear and terror that he has caused, the
right decision is clear,'' a federal prosecutor, Steven Mellin, said in
his closing argument. ``The only sentence that will do justice in this
case is a sentence of death.''

With death sentences, an appeal is all but inevitable, and the process
generally takes years if not decades to play out. Of the 80 federal
defendants sentenced to death since 1988, only three, including Timothy
J. McVeigh, the Oklahoma City bomber, have been executed. Some of the
sentences were vacated or the defendants died or committed suicide.

Most cases are still tied up in appeal.

Advertisement

\protect\hyperlink{after-bottom}{Continue reading the main story}

\hypertarget{site-index}{%
\subsection{Site Index}\label{site-index}}

\hypertarget{site-information-navigation}{%
\subsection{Site Information
Navigation}\label{site-information-navigation}}

\begin{itemize}
\tightlist
\item
  \href{https://help.nytimes.com/hc/en-us/articles/115014792127-Copyright-notice}{©~2020~The
  New York Times Company}
\end{itemize}

\begin{itemize}
\tightlist
\item
  \href{https://www.nytco.com/}{NYTCo}
\item
  \href{https://help.nytimes.com/hc/en-us/articles/115015385887-Contact-Us}{Contact
  Us}
\item
  \href{https://www.nytco.com/careers/}{Work with us}
\item
  \href{https://nytmediakit.com/}{Advertise}
\item
  \href{http://www.tbrandstudio.com/}{T Brand Studio}
\item
  \href{https://www.nytimes.com/privacy/cookie-policy\#how-do-i-manage-trackers}{Your
  Ad Choices}
\item
  \href{https://www.nytimes.com/privacy}{Privacy}
\item
  \href{https://help.nytimes.com/hc/en-us/articles/115014893428-Terms-of-service}{Terms
  of Service}
\item
  \href{https://help.nytimes.com/hc/en-us/articles/115014893968-Terms-of-sale}{Terms
  of Sale}
\item
  \href{https://spiderbites.nytimes.com}{Site Map}
\item
  \href{https://help.nytimes.com/hc/en-us}{Help}
\item
  \href{https://www.nytimes.com/subscription?campaignId=37WXW}{Subscriptions}
\end{itemize}
