 **NYTimes.com no longer supports Internet Explorer 9 or earlier. Please
upgrade your browser.
\href{http://www.nytimes.com/content/help/site/ie9-support.html}{LEARN
MORE »}

**Sections

**Home

**Search

\hypertarget{the-new-york-times}{%
\subsection{\texorpdfstring{\href{http://www.nytimes.com/}{The New York
Times}}{The New York Times}}\label{the-new-york-times}}

\hypertarget{-obituaries-}{%
\subsubsection{\texorpdfstring{
\href{https://www.nytimes.com/section/obituaries}{Obituaries}
}{ Obituaries }}\label{-obituaries-}}

 \href{https://www.nytimes.com/section/obituaries}{Obituaries}
\textbar{}Remarkable People We Overlooked in Our Obituaries

**Close search

\hypertarget{site-search-navigation}{%
\subsection{Site Search Navigation}\label{site-search-navigation}}

Search NYTimes.com

**Clear this text input

Go

\url{https://nyti.ms/2G6cPnh}

\hypertarget{site-navigation}{%
\subsection{Site Navigation}\label{site-navigation}}

\hypertarget{site-mobile-navigation}{%
\subsection{Site Mobile Navigation}\label{site-mobile-navigation}}

\hypertarget{remarkable-people-we-overlooked-in-our-obituaries}{%
\section{Remarkable People We Overlooked in Our
Obituaries}\label{remarkable-people-we-overlooked-in-our-obituaries}}

The poet Sylvia Plath and the novelist Charlotte Brontë. Ida B. Wells,
the anti-lynching activist. These extraordinary people --- and so many
others --- did not have obituaries in The New York Times. Until now.

Madhubala in "Mahal", produced by The Bombay Talkies Studios in 1949.
Alice Guy Blaché shooting a scene in 1906 of her film "La Fée aux
Choux", distributed by Gaumont Pathé Archives. Annemarie Schwarzenbach
in "Une Suisse Rebelle, Annemarie Schwarzenbach 1908-1942", produced by
Troubadour Films.

\hypertarget{overlooked}{%
\section{Overlooked}\label{overlooked}}

Since 1851, obituaries in The New York Times have been dominated by
white men. Now, we're adding the stories of other remarkable people.

\protect\hyperlink{latest}{}

View the Latest

\section{}

By AMISHA PADNANI and JESSICA BENNETT MARCH 8, 2018

Obituary writing is more about life than death: the last word, a
testament to a human contribution.

Yet who gets remembered --- and how --- inherently involves judgment. To
look back at the obituary archives can, therefore, be a stark lesson in
how society valued various achievements and achievers.

Since 1851, The New York Times has published thousands of obituaries: of
heads of state, opera singers, the
\href{https://www.nytimes.com/2005/11/23/us/ruth-m-siems-inventor-of-stuffing-dies-at-74.html}{inventor}
of Stove Top stuffing and the
\href{https://www.nytimes.com/2008/11/25/business/25james.html}{namer}
of the Slinky. The vast majority chronicled the lives of men, mostly
white ones.

\href{https://www.nytimes.com/2018/03/08/obituaries/overlooked-charlotte-bronte.html}{Charlotte
Brontë} wrote ``Jane Eyre''; \href{https://www.nytimes.com}{Emily Warren
Roebling} oversaw construction of the Brooklyn Bridge when her husband
fell ill; \href{https://www.nytimes.com}{Madhubala} transfixed
Bollywood; \href{https://www.nytimes.com}{Ida B. Wells} campaigned
against lynching. Yet all of their deaths went unremarked in our pages,
until now.

Below you'll find obituaries for these and others who left indelible
marks but were nonetheless overlooked. We'll be adding to this
collection each week, as Overlooked becomes a regular feature in the
\href{https://www.nytimes.com/section/obituaries}{obituaries section},
and expanding our lens beyond women.

\emph{You can use
\href{https://www.nytimes.com/interactive/2018/obituaries/formacist-overlooked.html}{this
form} to nominate candidates for future ``Overlooked'' obits.
\href{https://www.nytimes.com/2018/03/08/obituaries/overlooked-from-the-death-desk-why-most-obits-are-still-of-white-men.html}{Read
an essay} from our obituaries editor about how he approaches subjects
and learn more about
\href{https://www.nytimes.com/2018/03/08/insider/overlooked-obituary.html}{how
the project came to be}.}

\href{https://www.nytimes.com/interactive/2018/obituaries/overlooked.html}{Read
more}

\begin{center}\rule{0.5\linewidth}{\linethickness}\end{center}

\includegraphics{https://static01.nyt.com/images/2018/03/11/obituaries/00overlooked-images-slide-VJ69/00overlooked-images-slide-VJ69-master180-v2.jpg}

1862-1931

\hypertarget{ida-b-wells}{%
\section{Ida B. Wells}\label{ida-b-wells}}

Took on racism in the Deep South with powerful reporting on lynchings.

By CAITLIN DICKERSON MARCH 8, 2018

It was not all that unusual when, in 1892, a mob dragged Thomas Moss out
of a Memphis jail in his pajamas and shot him to death over a feud that
began with a game of marbles. But his lynching changed history because
of its effect on one of the nation's most influential journalists, who
was also the godmother of his first child: Ida B. Wells.

``It is with no pleasure that I have dipped my hands in the corruption
here exposed,'' Wells wrote in 1892 in the introduction to ``Southern
Horrors,'' one of her seminal works about lynching, ``Somebody must show
that the Afro-American race is more sinned against than sinning, and it
seems to have fallen upon me to do so.''

\href{https://www.nytimes.com/interactive/2018/obituaries/overlooked-ida-b-wells.html}{Read
more}

\begin{center}\rule{0.5\linewidth}{\linethickness}\end{center}

\includegraphics{https://static01.nyt.com/images/2018/03/11/obituaries/00overlooked-images-slide-L2IM/00overlooked-images-slide-L2IM-master180-v2.jpg}

c. 1875-1907

\hypertarget{qiu-jin}{%
\section{Qiu Jin}\label{qiu-jin}}

A feminist poet and revolutionary who became a martyr known as China's
`Joan of Arc.'

By AMY QIN MARCH 8, 2018

With her passion for wine, swords and bomb making, Qiu Jin was unlike
most women born in late 19th-century China. As a girl, she wrote poetry
and studied Chinese martial heroines like Hua Mulan (yes,
\href{https://www.imdb.com/title/tt0120762/}{\emph{that} Mulan})
fantasizing about one day seeing her own name in the history books.

But her ambitions ran up against China's deeply rooted patriarchal
society, which held that a woman's place remained in the home.
Undeterred, Qiu rose to become an early and fierce advocate for the
liberation of Chinese women, defying prevailing Confucian gender and
class norms by unbinding her feet, cross-dressing and leaving her young
family to pursue an education abroad.

\href{https://www.nytimes.com/interactive/2018/obituaries/overlooked-qiu-jin.html}{Read
more}

\begin{center}\rule{0.5\linewidth}{\linethickness}\end{center}

\includegraphics{https://static01.nyt.com/images/2018/03/11/obituaries/00overlooked-images-slide-MTG6/00overlooked-images-slide-MTG6-master180-v3.jpg}

1852-1886

\hypertarget{mary-ewing-outerbridge}{%
\section{Mary Ewing Outerbridge}\label{mary-ewing-outerbridge}}

Established what may have been America's first tennis court in the
1870s.

By AMISHA PADNANI MARCH 8, 2018

Mary Ewing Outerbridge didn't have an easy time bringing tennis to
America in 1874.

First she had to get past customs agents. And they were suspicious. What
was this large net? Clearly it wasn't for fishing, they said. And what
were these stringed things with long handles?

\href{https://www.nytimes.com/interactive/2018/obituaries/overlooked-mary-ewing-outerbridge.html}{Read
more}

\begin{center}\rule{0.5\linewidth}{\linethickness}\end{center}

\includegraphics{https://static01.nyt.com/images/2018/03/11/obituaries/00overlooked-images-slide-NG2G/00overlooked-images-slide-NG2G-master180-v3.jpg}

1923-1971

\hypertarget{diane-arbus}{%
\section{Diane Arbus}\label{diane-arbus}}

A photographer whose portraits have compelled or repelled generations of
viewers.

By JAMES ESTRIN MARCH 8, 2018

Diane Arbus was a daughter of privilege who spent much of her adult life
documenting those on the periphery of society. Since she killed herself
in 1971, her unblinking portraits have made her a seminal figure in
modern-day photography and an influence on three generations of
photographers, though she is perhaps just as famous for her
unconventional lifestyle and her suicide.

\href{https://www.nytimes.com/interactive/2018/obituaries/overlooked-diane-arbus.html}{Read
more}

\begin{center}\rule{0.5\linewidth}{\linethickness}\end{center}

\includegraphics{https://static01.nyt.com/images/2018/03/11/obituaries/00overlooked-images-slide-OM05/00overlooked-images-slide-OM05-master180-v2.jpg}

1945-1992

\hypertarget{marsha-p-johnson}{%
\section{Marsha P. Johnson}\label{marsha-p-johnson}}

A transgender pioneer and activist who was a fixture of Greenwich
Village street life.

By SEWELL CHAN MARCH 8, 2018

Marsha P. Johnson was an activist, a prostitute, a drag performer and,
for nearly three decades, a fixture of street life in Greenwich Village.
She was a central figure in a gay liberation movement energized by the
1969 police raid on the
\href{https://www.nytimes.com/2016/06/25/nyregion/stonewall-inn-named-national-monument-a-first-for-gay-rights-movement.html}{Stonewall
Inn}. She was a
\href{https://onlineonly.christies.com/s/andy-warhol-christies-andys-eye-candy/ladies-gentlemen-marsha-johnson-98/6817}{model}
for
\href{http://www.nytimes.com/1987/02/23/obituaries/andy-warhol-pop-artist-dies.html}{Andy
Warhol}. She battled severe mental illness. She was usually destitute
and, for much of her life, effectively homeless.

When she died at 46,
\href{https://www.nytimes.com/2017/10/05/movies/the-death-and-life-of-marsha-p-johnson-review.html}{under
murky circumstances}, in summer 1992, Johnson was mourned by her many
friends, but her death did not attract much notice in the mainstream
press.

\href{https://www.nytimes.com/interactive/2018/obituaries/overlooked-marsha-p-johnson.html}{Read
more}

\begin{center}\rule{0.5\linewidth}{\linethickness}\end{center}

\includegraphics{https://static01.nyt.com/images/2018/03/11/obituaries/00overlooked-images-slide-DUH5/00overlooked-images-slide-DUH5-master180-v2.jpg}

1932-1963

\hypertarget{sylvia-plath}{%
\section{Sylvia Plath}\label{sylvia-plath}}

A postwar poet unafraid to confront her own despair.

By ANEMONA HARTOCOLLIS MARCH 8, 2018

She made sure to spare the children, leaving milk and bread for the two
toddlers to find when they woke up. She stuffed the cracks of the doors
and windows with cloths and tea towels. Then she turned on the gas.

On the morning of Feb. 11, 1963, a Monday, a nurse found the poet Sylvia
Plath in her flat on Fitzroy Road in London, an address where W.B. Yeats
had once lived. She was ``lying on the floor of the kitchen with her
head resting on the oven,'' according to a local paper, the St. Pancras
Chronicle.

\href{https://www.nytimes.com/interactive/2018/obituaries/overlooked-sylvia-plath.html}{Read
more}

\begin{center}\rule{0.5\linewidth}{\linethickness}\end{center}

\includegraphics{https://static01.nyt.com/images/2018/03/11/obituaries/00overlooked-images-slide-YBVZ/00overlooked-images-slide-YBVZ-master180-v2.jpg}

1920-1951

\hypertarget{henrietta-lacks}{%
\section{Henrietta Lacks}\label{henrietta-lacks}}

Cancer cells were taken from her body without permission. They led to a
medical revolution.

By ADEEL HASSAN MARCH 8, 2018

She never traveled farther than Baltimore from her family home in
southern Virginia, but her cells have traveled around the earth and far
above it, too.

She was buried in an unmarked grave, but the trillions of those cells
--- generated from a tiny patch taken from her body --- are labeled in
university labs and biotechnology companies across the world, where they
continue to spawn and to play the critical role in a 67-year parade of
medical advances.

\href{https://www.nytimes.com/interactive/2018/obituaries/overlooked-henrietta-lacks.html}{Read
more}

\begin{center}\rule{0.5\linewidth}{\linethickness}\end{center}

\includegraphics{https://static01.nyt.com/images/2018/03/11/obituaries/00overlooked-images-slide-UZH4/00overlooked-images-slide-UZH4-master180-v2.jpg}

1933-1969

\hypertarget{madhubala}{%
\section{Madhubala}\label{madhubala}}

A Bollywood legend whose tragic life mirrored Marilyn Monroe's.

By AISHA KHAN MARCH 8, 2018

It was probably the first ghost story in Indian cinema. A bewildered
young man in a mansion chasing glimpses of an ethereal, veiled beauty.
The movie, ``Mahal,'' was a huge success, making the lead actress,
Madhubala, who was barely 16, a superstar overnight.

Nearly seven decades later, strains of the film's signature song,
``\href{https://www.youtube.com/watch?v=03DXW_rV54U}{Aayega aane wala}''
(He will come), are instantly recognizable to most Indians, evoking the
suspenseful tale of lost love and reincarnation.

\href{https://www.nytimes.com/interactive/2018/obituaries/overlooked-madhubala.html}{Read
more}

\begin{center}\rule{0.5\linewidth}{\linethickness}\end{center}

\includegraphics{https://static01.nyt.com/images/2018/03/11/obituaries/00overlooked-images-slide-7YGG/00overlooked-images-slide-7YGG-master180-v3.jpg}

1843-1903

\hypertarget{emily-warren-roebling}{%
\section{Emily Warren Roebling}\label{emily-warren-roebling}}

Oversaw the construction of the Brooklyn Bridge after her engineer
husband fell ill.

By JESSICA BENNETT MARCH 8, 2018

It was not customary for a woman to accompany a man to a construction
site in the late 19th century. Petticoats tended to get in the way of
physical work.

But when Washington A. Roebling, the chief engineer of the Brooklyn
Bridge, fell ill, it was his wife, Emily Warren Roebling, who stepped in
--- managing, liaising and politicking between city officials, workers,
and her husband's bedside to see the world's first steel-wire suspension
bridge to completion. She would become the first person to cross the
bridge, too --- carrying a rooster with her, as the story has it, for
good luck.

\href{https://www.nytimes.com/interactive/2018/obituaries/overlooked-emily-warren-roebling.html}{Read
more}

\begin{center}\rule{0.5\linewidth}{\linethickness}\end{center}

\includegraphics{https://static01.nyt.com/images/2018/03/11/obituaries/00overlooked-images-slide-FTPD/00overlooked-images-slide-FTPD-master180-v2.jpg}

1891-1964

\hypertarget{nella-larsen}{%
\section{Nella Larsen}\label{nella-larsen}}

A Harlem Renaissance-era writer whose heritage informed her modernist
take on the topic of race.

By BONNIE WERTHEIM MARCH 8, 2018

When Nella Larsen died, in 1964, she left little behind: a ground-floor
apartment, two published novels, some short stories, a few letters. She
was childless, divorced and estranged from her half sister, who, in some
accounts, upon learning she was to inherit \$35,000 of Larsen's savings,
\href{https://books.google.com/books?id=yIdqlAFm_-cC\&pg=PA23\&lpg=PA23\&dq\#v=onepage\&q\&f=false}{denied
knowing the writer existed}.

It was a fitting end for a woman whose entire life had been a story of
swift erasure.

\href{https://www.nytimes.com/interactive/2018/obituaries/overlooked-nella-larsen.html}{Read
more}

\begin{center}\rule{0.5\linewidth}{\linethickness}\end{center}

\includegraphics{https://static01.nyt.com/images/2018/03/19/obituaries/19lovelace-obit/00overlooked-images-slide-FRJU-master180-v2.jpg}

1815-1852

\hypertarget{ada-lovelace}{%
\section{Ada Lovelace}\label{ada-lovelace}}

A gifted mathematician who is now recognized as the first computer
programmer.

By CLAIRE CAIN MILLER MARCH 8, 2018

A century before the dawn of the computer age, Ada Lovelace imagined the
modern-day, general-purpose computer. It could be programmed to follow
instructions, she wrote in 1843. It could not just calculate but also
create, as it ``weaves algebraic patterns just as the
\href{https://en.wikipedia.org/wiki/Jacquard_loom}{Jacquard loom} weaves
flowers and leaves.''

The computer she was writing about, the British inventor Charles
Babbage's
\href{http://www.nytimes.com/interactive/2011/11/07/science/before-its-time-machine.html}{Analytical
Engine}, was never built. But her writings about computing have earned
Lovelace --- who died of uterine cancer in 1852 at 36 --- recognition as
the first computer programmer.

\href{https://www.nytimes.com/interactive/2018/obituaries/overlooked-ada-lovelace.html}{Read
more}

\begin{center}\rule{0.5\linewidth}{\linethickness}\end{center}

\includegraphics{https://static01.nyt.com/images/2018/03/11/obituaries/00overlooked-ALL-photos-slide-E73O/00overlooked-ALL-photos-slide-E73O-master180-v2.jpg}

1878-1955

\hypertarget{margaret-abbott}{%
\section{Margaret Abbott}\label{margaret-abbott}}

The first American woman to win an Olympic championship.

By MARGALIT FOX MARCH 8, 2018

The first American woman to win an Olympic championship died without
ever knowing what she had achieved.

That woman, Margaret Abbott, won the ladies' golf competition, as the
event was genteelly known, at the 1900 Games in Paris. She received a
gilded porcelain bowl, a smattering of coverage in the newspapers and
then nothing.

\href{https://www.nytimes.com/interactive/2018/obituaries/overlooked-margaret-abbott.html}{Read
more}

\begin{center}\rule{0.5\linewidth}{\linethickness}\end{center}

\hypertarget{more-in-overlooked}{%
\subsubsection{\texorpdfstring{\href{https://www.nytimes.com/spotlight/overlooked}{More
in 'Overlooked'}}{More in 'Overlooked'}}\label{more-in-overlooked}}

\begin{itemize}
\item ~
  \hypertarget{behind-overlooked}{%
  \subsubsection{Behind Overlooked}\label{behind-overlooked}}
\item
  How
  \href{https://www.nytimes.com/2018/03/08/insider/overlooked-obituary.html}{one
  editor's personal thoughts} on diversity led to Overlooked. Also, a
  look at
  \href{https://www.nytimes.com/2018/03/14/insider/women-obituaries.html}{our
  efforts to collect data} and an explanation of
  \href{https://www.nytimes.com/2018/03/08/obituaries/overlooked-from-the-death-desk-why-most-obits-are-still-of-white-men.html}{how
  obituaries are chosen}. Learn more about The New York Times Gender
  Initiative
  \href{https://www.nytimes.com/2017/12/13/reader-center/jessica-bennett-our-new-gender-editor-answers-your-questions.html}{here}.
\end{itemize}

\begin{itemize}
\tightlist
\item
  Editors
\item
  Amisha Padnani
\item
  Jessica Bennett
\item
  Kaly Soto
\item
  Kathleen A. Flynn
\item
  Destinée-Charisse Royal
\item
  Maya Salam
\item
  Ed Shanahan
\item
  Susanna Timmons
\item
  Amy Virshup
\end{itemize}

\begin{itemize}
\tightlist
\item
  Photo Editors
\item
  Sandra Stevenson
\item
  Beth Flynn
\item
  Nakyung Han
\item
  Amanda Boe
\end{itemize}

\begin{itemize}
\tightlist
\item
  Art Direction
\item
  Antonio De Luca
\item
  Agnes Lee
\end{itemize}

\begin{itemize}
\tightlist
\item
  Research
\item
  Albert Sun
\item
  Doris Burke
\item
  Jeff Roth
\end{itemize}

\begin{itemize}
\tightlist
\item
  Online Production
\item
  Umi Syam
\item
  Meghan Louttit
\end{itemize}

\begin{itemize}
\tightlist
\item
  Print Production
\item
  Fred Bierman
\item
  Andrew Sondern
\end{itemize}

\begin{itemize}
\tightlist
\item
  Special Thanks
\item
  William McDonald
\item
  William O'Donnell
\item
  James Nieves
\item
  Susan Wessling
\end{itemize}

\begin{itemize}
\tightlist
\item
  Video footage: Madhubala in ``Mahal,'' produced by Bombay Talkies
  studio in 1949. Alice Guy Blaché shooting a scene for a film in 1906
  distributed by Gaumont Pathé Archives. Annemarie Schwarzenbach in
  ``Une Suisse Rebelle, Annemarie Schwarzenbach 1908-1942,'' produced by
  Troubadour Films. Madhubala's obitutary is in this collection.
  Obituaries are planned for Schwarzenbach and Blaché.
\end{itemize}

\hypertarget{more-on-nytimescom}{%
\subsection{More on NYTimes.com}\label{more-on-nytimescom}}

Advertisement

\hypertarget{site-information-navigation}{%
\subsection{Site Information
Navigation}\label{site-information-navigation}}

\begin{itemize}
\tightlist
\item
  \href{https://help.nytimes.com/hc/en-us/articles/115014792127-Copyright-notice}{©
  2020 The New York Times Company}
\item
  \href{https://www.nytimes.com}{Home}
\item
  \href{https://www.nytimes.com/search/}{Search}
\item
  Accessibility concerns? Email us at
  \href{mailto:accessibility@nytimes.com}{\nolinkurl{accessibility@nytimes.com}}.
  We would love to hear from you.
\item
  \href{https://help.nytimes.com/hc/en-us/articles/115015385887-Contact-Us}{Contact
  Us}
\item
  \href{https://www.nytco.com/careers/}{Work with us}
\item
  \href{https://nytmediakit.com/}{Advertise}
\item
  \href{https://help.nytimes.com/hc/en-us/articles/115014892108-Privacy-policy\#pp}{Your
  Ad Choices}
\item
  \href{https://help.nytimes.com/hc/en-us/articles/115014892108-Privacy-policy}{Privacy}
\item
  \href{https://help.nytimes.com/hc/en-us/articles/115014893428-Terms-of-service}{Terms
  of Service}
\item
  \href{https://help.nytimes.com/hc/en-us/articles/115014893968-Terms-of-sale}{Terms
  of Sale}
\end{itemize}

\hypertarget{site-information-navigation-1}{%
\subsection{Site Information
Navigation}\label{site-information-navigation-1}}

\begin{itemize}
\tightlist
\item
  \href{https://spiderbites.nytimes.com}{Site Map}
\item
  \href{https://help.nytimes.com/hc/en-us}{Help}
\item
  \href{https://help.nytimes.com/hc/en-us/articles/115015385887-Contact-Us?redir=myacc}{Site
  Feedback}
\item
  \href{https://www.nytimes.com/subscription?campaignId=37WXW}{Subscriptions}
\end{itemize}
