\hypertarget{the-asian-american-fashion-designers-who-shaped-the-industry}{%
\section{The Asian-American Fashion Designers Who Shaped the
Industry}\label{the-asian-american-fashion-designers-who-shaped-the-industry}}

April 13, 2020

\begin{itemize}
\item
\item
\item
\item
\end{itemize}

Kimora Lee Simmons, Phillip Lim and many more have collectively advanced
issues of representation.

\href{https://www.nytimes.com/interactive/2020/04/13/t-magazine/culture-issue-2020.html}{We
Are Family}

\hypertarget{chapter-4-the-new-guard}{%
\subparagraph{Chapter 4: The New Guard}\label{chapter-4-the-new-guard}}

\hypertarget{previous}{%
\subparagraph{Previous}\label{previous}}

\hypertarget{next}{%
\subparagraph{Next}\label{next}}

\hypertarget{the-asian-american-fashion-designers-who-shaped-the-industry-1}{%
\section{The Asian-American Fashion Designers Who Shaped the
Industry}\label{the-asian-american-fashion-designers-who-shaped-the-industry-1}}

\hypertarget{the-designers}{%
\subsection{The Designers}\label{the-designers}}

Why there are so many Asian-Americans in fashion, and how they changed
the industry.

By \href{https://www.nytimes.com/by/thessaly-la-force}{Thessaly La
Force}

April 13, 2020

SHARE

JOSEPH ALTUZARRA

CHRIS LEBAR13

EUNICE LEEUNIS

JENNY CHENGGauntlett Cheng

BIBHU MOHAPATRA

MAKIE YAHAGI

JADE LAICreatures of Comfort

DAO-YI CHOWPublic School

YEOHLEE TENG

PHILLIP LIM3.1 Phillip Lim

KIMORA LEE SIMMONSBaby Phat

RICHARD CHAIClub Monaco

KEVIN KIMIISE

TOMMY TONDeveaux

THAKOON PANICHGUL

KIM SHUI

RUI ZHOU

PRABAL GURUNG

SANDY LIANG

LAURA KIMOscar de la Renta and Monse

MARY PING

SNOW XUE GAO

PETER SOM

JASON WU

JI OH

DYLAN CAOCommission

JIN KAYCommission

HUY LUONGCommission

DEREK LAM

Clockwise from top left: \textbf{JOSEPH ALTUZARRA}; \textbf{CHRIS LEBA},
R13; \textbf{EUNICE LEE}, UNIS; \textbf{JENNY CHENG}, Gauntlett Cheng;
\textbf{BIBHU MOHAPATRA}; \textbf{MAKIE YAHAGI}; \textbf{JADE LAI},
Creatures of Comfort; \textbf{DAO-YI CHOW}, Public School;
\textbf{YEOHLEE TENG}; \textbf{PHILLIP LIM}, 3.1 Phillip Lim;
\textbf{KIMORA LEE SIMMONS}, Baby Phat; \textbf{RICHARD CHAI}, Club
Monaco; \textbf{KEVIN KIM}, IISE; \textbf{TOMMY TON}, Deveaux;
\textbf{THAKOON PANICHGUL}; \textbf{KIM SHUI}; \textbf{RUI ZHOU};
\textbf{PRABAL GURUNG}; \textbf{SANDY LIANG}; \textbf{LAURA KIM}, Oscar
de la Renta and Monse; \textbf{MARY PING}; \textbf{SNOW XUE GAO};
\textbf{PETER SOM}; \textbf{JASON WU}; \textbf{JI OH}; \textbf{DYLAN
CAO}, \textbf{JIN KAY} and \textbf{HUY LUONG}, Commission; and
\textbf{DEREK LAM}. Photographed at the Morgan Library in New York City
on Feb. 17, 2020. Renee Cox

IF THERE CAN be said to be a single group of people who have most
effectively counteracted the
\href{https://www.nytimes.com/2019/08/26/t-magazine/asian-american-comedians.html}{monolithic
idea} of what it means to be Asian-American --- a label that encompasses
people with ancestry from countries in East Asia, the Pacific and South
Asia --- it might be the several dozen Asian-American designers who
began to shape the fashion industry in New York beginning in the early
'80s. Their collective presence corrects at least some of the tokenism
that has often defined the fashion world, where diversity can often be a
one-dimensional gesture.

The very first Asian-American fashion designers were pioneers such as
\href{https://www.nytimes.com/2017/05/11/t-magazine/fashion/anna-sui-profile-style.html}{Anna
Sui}, \href{https://www.viviennetam.com/}{Vivienne Tam},
\href{https://www.verawang.com/}{Vera Wang} and
\href{https://babyphat.com/}{Kimora Lee Simmons} --- women who launched
their labels in a market dominated by Calvin Klein, Bill Blass, Ralph
Lauren,
\href{https://www.nytimes.com/2018/04/09/t-magazine/michael-kors-80s-photos-inspiration.html}{Michael
Kors}, \href{https://www.nytimes.com/topic/person/donna-karan}{Donna
Karan} and
\href{https://www.nytimes.com/2020/02/10/t-magazine/marc-jacobs.html}{Marc
Jacobs}. Each had their own aesthetic: Sui with her peasant blouses and
flouncy skirts; Tam with her updated takes on the traditional cheongsam;
Simmons with her body-conscious velour tracksuits; and Wang, who built a
new bridal vernacular with her clean, minimalist gowns. By the aughts, a
new vanguard of Asian-American designers (here, we use the term to
include both American citizens of Asian descent and Asians who work in
America) were launching their own labels, including
\href{https://www.nytimes.com/2020/01/24/t-magazine/phillip-lim-shrimp-recipe.html}{Phillip
Lim}, \href{https://www.instagram.com/richardchai/?hl=en}{Richard Chai},
\href{https://www.alexanderwang.com/us-en/}{Alexander Wang},
\href{https://www.petersom.com/}{Peter Som},
\href{https://www.bibhu.com/}{Bibhu Mohapatra},
\href{https://www.dereklam.com/us}{Derek Lam} and others. Their clothes
shared no unifying aesthetic aside from being contemporary --- a rebuke
to the Orientalist belief that there is one particular Asian mode of
expression. (Lim recalled early on being asked by reporters why he
didn't use dragons and red silk brocade in his designs.) Their arrival
also added depth to the notion of who, exactly, could be an American
fashion designer while also announcing a more organic age of inclusion
--- these designers, for example, made a point to cast Asian and other
nonwhite models for their shows and campaigns. As they gained acclaim,
their presence was bolstered by the success of more conceptual designers
such as \href{https://www.instagram.com/lilmingling/?hl=en}{Jade Lai},
\href{https://yeohlee.com/}{Yeohlee Teng} and
\href{https://tmagazine.blogs.nytimes.com/2014/05/20/perfect-pairing-collaboration-sculptural-jewelry-mary-ping-piece-a-conviction/}{Mary
Ping}. After the 2008 recession, many independent designers were unable
to sustain themselves alongside luxury conglomerates. Still, those who
have continued ---
\href{https://www.nytimes.com/2019/09/08/t-magazine/jason-wu-newspaper-dress.html}{Jason
Wu},
\href{https://www.nytimes.com/2017/10/02/t-magazine/joseph-altuzarra-paris-fashion-week.html}{Joseph
Altuzarra},
\href{https://www.nytimes.com/2019/11/12/t-magazine/prabal-gurung-book-photos.html}{Prabal
Gurung}, to name a few --- have ushered in the next generation. Jin Kay,
Dylan Cao and Huy Luong of the label
\href{https://www.nytimes.com/2018/12/26/t-magazine/best-new-fashion-brands.html}{Commission},
which was created in 2018 and inspired by the stylish Vietnamese and
Korean women of the late '80s and '90s, spent time at Lim's and Gurung's
labels as well as at bigger European brands like
\href{https://www.nytimes.com/2018/10/15/t-magazine/alessandro-michele-gucci-interview.html}{Gucci}.
At around the same time, \href{https://peterdo.net/}{Peter Do}, who used
to work with
\href{https://www.nytimes.com/2018/03/01/t-magazine/fashion/celine-phoebe-philo-era-stories-sofia-coppola-stella-tennant.html}{Phoebe
Philo at Celine}, launched his own label in New York. There's also now
'90s-grunge inspired designers such as Jenny Cheng, of
\href{https://www.nytimes.com/2017/02/17/t-magazine/fashion/new-york-fashion-week-top-10-moments.html}{Gauntlett
Cheng}, as well as
\href{https://www.nytimes.com/2019/02/12/fashion/sandy-liang-fleece.html}{Sandy
Liang}, whose label features raver-like cargo pants for the club kids of
the Lower East Side, among many more.

The T List \textbar{}

Sign up here

The rise of these designers can be explained, in part, by the
professionalization of fashion --- the
\href{https://www.newschool.edu/parsons/fashion-school/}{School of
Fashion at Parsons} and the
\href{https://www.nytimes.com/topic/organization/fashion-institute-of-technology}{Fashion
Institute of Technology} in Manhattan are incubators for talent (in
2010, The New York Times reported that roughly 70 percent of Parsons's
international students came from Asia; 23 percent of F.I.T.'s students
were Asian or Asian-American). Their achievements have also helped Asian
parents accept that fashion design can be a replicable pathway to
success. And if many come from families with backgrounds in the garment
industry, much like the Jewish-American designers who preceded them (the
grandparents of Lam, who is of Chinese descent, owned a large
bridal-gown factory in San Francisco; Lim's mother, who emigrated from
China, worked as a seamstress in Los Angeles), many also found their way
into fashion through other arenas: Lim cites the music videos he used to
watch on MTV as a teenager growing up in Southern California as being
more of a direct influence on him than anything else. ``I wanted to be
part of \emph{that} tribe,'' he said.

BUT NONE OF this explicitly answers why so many Asian-Americans have
gravitated toward fashion. Asian-American representation in the broader
glamour industry --- not only on the red carpets and in fashion shows
but also
\href{https://www.nytimes.com/2018/11/06/t-magazine/asian-american-actors-representation.html}{in
film and television} --- has been slow to come. There are still too few
actors and models of Asian descent to accurately reflect the diversity
of this country, one where Asian-Americans are approximately 6 percent
of the population, according to the 2018 census. Whatever lingering
resistance exists to \emph{seeing} Asian-Americans speaks to a
persistent invisibility --- in part because Asian-Americans don't neatly
fit into the black-white racial dynamics of America, and also because
our own history in this country is easily diminished or, worse,
\href{https://www.nytimes.com/2019/11/04/t-magazine/japanese-american-novel.html}{often
left unacknowledged}. The success of these designers is very hard won,
as the funders, buyers and editors in fashion are still essentially the
same as they were decades ago, and because building a brand takes years
of dedication along with significant investment.

Perhaps one reason there are so many Asian-American designers is that
fashion values the concept of presentation --- design is a way to
connect to the cultural values of craftsmanship and use of luxury
materials so historically prevalent in East and South Asian countries.
Perhaps, too, the stereotypes surrounding assimilation (especially the
model minority) have made it easier for gatekeepers to accept and take
risks on them (today, 48 of the more than 500 members of the Council of
Fashion Designers of America identify as Asian; by comparison, only 19
identify as black and 31 as Latinx). More commercially, the popularity
of these American designers translates abroad, as brands such as Lim's
or Alexander Wang's proved to be viable in not just the European markets
but in China, South Korea and Japan as well, where consumer spending
power has increased in the last two decades.

Finally, there's also a less tangible factor, which is that the last
half century of these designers' work (as well as that of Japanese
designers such as
\href{https://www.nytimes.com/2018/09/03/t-magazine/rei-kawakubo-comme-des-garcons-menswear.html}{Rei
Kawakubo}, \href{https://theshopyohjiyamamoto.com/}{Yohji Yamamoto},
\href{https://www.nytimes.com/2016/10/17/t-magazine/junya-watanabe-interview.html}{Junya
Watanabe} and \href{https://www.isseymiyake.com/en/}{Issey Miyake}) has
aligned designers of Asian descent with the avant-garde. Asian
invisibility may still persist, but these designers have become an
indelible part of our collective consciousness when it comes to what we
wear and how we choose to wear it. And that can only grow over time. Yet
when you ask Gurung why there are so many Asian-American designers, he
replies, ``Are there, \emph{really}?'' There is always, he was pointedly
saying, room for more.

Thessaly La Force is a features director at T Magazine. Renee Cox is a
Jamaican-American artist. Her work focuses on feminist theory and black
womanhood. Set design: Todd Knopke. Hair and makeup: Laura de Leon at
Joe Management using Chanel Les Beiges. Hair and makeup for Prabal
Gurung: Van Truong at MAC. Hair and makeup assistants: Robert Reyes and
Anna Kurihara.

\href{https://www.nytimes.com/2020/04/13/t-magazine/mary-ping-newspaper-bag.html}{}

\hypertarget{watch-mary-ping-create-an-it-bag-out-of-newspaperapril-13-2020}{%
\paragraph{Watch Mary Ping Create an `It' Bag Out of NewspaperApril 13,
2020}\label{watch-mary-ping-create-an-it-bag-out-of-newspaperapril-13-2020}}

\includegraphics{https://static01.nyt.com/images/2020/04/13/t-magazine/13tmag-mary-ping-promo/13tmag-mary-ping-promo-mediumThreeByTwo210.jpg}
\href{https://www.nytimes.com/2018/11/06/t-magazine/asian-american-actors-representation.html}{}

\hypertarget{why-do-asian-americans-remain-largely-unseen-in-film-and-televisionnov-6-2018}{%
\paragraph{Why Do Asian-Americans Remain Largely Unseen in Film and
Television?Nov. 6,
2018}\label{why-do-asian-americans-remain-largely-unseen-in-film-and-televisionnov-6-2018}}

\includegraphics{https://static01.nyt.com/images/2018/11/06/t-magazine/06tmag-asianamerican-hp/06tmag-asianamerican-slide-KR8V-mediumThreeByTwo210.jpg}
\href{https://www.nytimes.com/2019/09/08/t-magazine/jason-wu-newspaper-dress.html}{}

\hypertarget{watch-jason-wu-make-a-dress-out-of-newspapersept-8-2019}{%
\paragraph{Watch Jason Wu Make a Dress Out of NewspaperSept. 8,
2019}\label{watch-jason-wu-make-a-dress-out-of-newspapersept-8-2019}}

\includegraphics{https://static01.nyt.com/images/2019/09/10/t-magazine/8tmag-wu-01/8tmag-wu-01-mediumThreeByTwo210.jpg}
\href{https://www.nytimes.com/2019/08/26/t-magazine/asian-american-comedians.html}{}

\hypertarget{the-comedians-challenging-stereotypes-about-asian-american-masculinityaug-26-2019}{%
\paragraph{The Comedians Challenging Stereotypes About Asian-American
MasculinityAug. 26,
2019}\label{the-comedians-challenging-stereotypes-about-asian-american-masculinityaug-26-2019}}

\includegraphics{https://static01.nyt.com/images/2019/08/26/t-magazine/26tmag-comics-slide-GMUP/26tmag-comics-slide-GMUP-mediumThreeByTwo210-v2.jpg}
\href{https://www.nytimes.com/2019/11/08/t-magazine/opening-ceremony-hats.html}{}

\hypertarget{watch-opening-ceremonys-founders-create-hats-inspired-by-their-mothersnov-8-2019}{%
\paragraph{Watch Opening Ceremony's Founders Create Hats Inspired by
Their MothersNov. 8,
2019}\label{watch-opening-ceremonys-founders-create-hats-inspired-by-their-mothersnov-8-2019}}

\includegraphics{https://static01.nyt.com/images/2019/11/06/t-magazine/06tmag-oc-mts-02/06tmag-oc-mts-02-mediumThreeByTwo210.jpg}

\hypertarget{we-are-family-1}{%
\subsubsection{We Are Family}\label{we-are-family-1}}

\hypertarget{chapter-1-heirs-and-alumni}{%
\paragraph{Chapter 1: Heirs and
Alumni}\label{chapter-1-heirs-and-alumni}}

\href{/interactive/2020/04/13/t-magazine/black-art-galleries.html}{}

\hypertarget{the-artists}{%
\subparagraph{The Artists}\label{the-artists}}

\href{/interactive/2020/04/13/t-magazine/italian-fashion-design-houses.html}{}

\hypertarget{the-dynasties}{%
\subparagraph{The Dynasties}\label{the-dynasties}}

\href{/interactive/2020/04/13/t-magazine/gordon-parks.html}{}

\hypertarget{the-directors}{%
\subparagraph{The Directors}\label{the-directors}}

\href{/interactive/2020/04/13/t-magazine/enrique-olvera-chef.html}{}

\hypertarget{the-disciples}{%
\subparagraph{The Disciples}\label{the-disciples}}

\href{/interactive/2020/04/13/t-magazine/royal-academy-antwerp.html}{}

\hypertarget{the-graduates}{%
\subparagraph{The Graduates}\label{the-graduates}}

\hypertarget{chapter-2-reunions-and-reconsiderations}{%
\paragraph{Chapter 2: Reunions and
Reconsiderations}\label{chapter-2-reunions-and-reconsiderations}}

\href{/interactive/2020/04/13/t-magazine/ninth-street-greenwich-village-neighbors.html}{}

\hypertarget{the-neighbors}{%
\subparagraph{The Neighbors}\label{the-neighbors}}

\href{/interactive/2020/04/13/t-magazine/omen-restaurant-nyc.html}{}

\hypertarget{the-regulars}{%
\subparagraph{The Regulars}\label{the-regulars}}

\href{/interactive/2020/04/13/t-magazine/hair-musical-broadway.html}{}

\hypertarget{hair-1967}{%
\subparagraph{Hair (1967)}\label{hair-1967}}

\href{/interactive/2020/04/13/t-magazine/sweeney-todd-revival.html}{}

\hypertarget{sweeney-todd-2005-revival}{%
\subparagraph{Sweeney Todd (2005
Revival)}\label{sweeney-todd-2005-revival}}

\href{/interactive/2020/04/13/t-magazine/daughters-of-the-dust.html}{}

\hypertarget{daughters-of-the-dust-1991}{%
\subparagraph{Daughters of the Dust
(1991)}\label{daughters-of-the-dust-1991}}

\hypertarget{chapter-3-legends-pioneers-and-survivors}{%
\paragraph{Chapter 3: Legends Pioneers and
Survivors}\label{chapter-3-legends-pioneers-and-survivors}}

\href{/interactive/2020/04/13/t-magazine/butch-stud-lesbian.html}{}

\hypertarget{the-renegades}{%
\subparagraph{The Renegades}\label{the-renegades}}

\href{/interactive/2020/04/13/t-magazine/act-up-aids.html}{}

\hypertarget{the-activists}{%
\subparagraph{The Activists}\label{the-activists}}

\href{/interactive/2020/04/13/t-magazine/artist-recluse.html}{}

\hypertarget{the-shadows}{%
\subparagraph{The Shadows}\label{the-shadows}}

\href{/interactive/2020/04/13/t-magazine/black-actresses-bassett-berry-blige-henson-whitfield-elise.html}{}

\hypertarget{the-veterans}{%
\subparagraph{The Veterans}\label{the-veterans}}

\hypertarget{chapter-4-the-new-guard-1}{%
\paragraph{Chapter 4: The New Guard}\label{chapter-4-the-new-guard-1}}

\href{/interactive/2020/04/13/t-magazine/asian-american-fashion-designers.html}{}

\hypertarget{the-designers-1}{%
\subparagraph{The Designers}\label{the-designers-1}}

\href{13tmag-beauties.html}{}

\hypertarget{the-beauties}{%
\subparagraph{The Beauties}\label{the-beauties}}

\href{/interactive/2020/04/13/t-magazine/nyc-downtown-nightlife-party-scene.html}{}

\hypertarget{the-scenemakers}{%
\subparagraph{The Scenemakers}\label{the-scenemakers}}

\href{/interactive/2020/04/13/t-magazine/maria-cornejo-olivier-rousteing-telfar-clemens-alessandro-michele.html\#olivier-rousteing-and-co}{}

\hypertarget{olivier-rousteing-and-co}{%
\subparagraph{Olivier Rousteing and
Co.}\label{olivier-rousteing-and-co}}

\href{/interactive/2020/04/13/t-magazine/maria-cornejo-olivier-rousteing-telfar-clemens-alessandro-michele.html\#maria-cornejo-and-co}{}

\hypertarget{maria-cornejo-and-co}{%
\subparagraph{Maria Cornejo and Co.}\label{maria-cornejo-and-co}}

\href{/interactive/2020/04/13/t-magazine/maria-cornejo-olivier-rousteing-telfar-clemens-alessandro-michele.html\#telfar-clemens-and-co}{}

\hypertarget{telfar-clemens-and-co}{%
\subparagraph{Telfar Clemens and Co.}\label{telfar-clemens-and-co}}

\href{/interactive/2020/04/13/t-magazine/maria-cornejo-olivier-rousteing-telfar-clemens-alessandro-michele.html\#alessandro-michele-and-co}{}

\hypertarget{alessandro-michele-and-co}{%
\subparagraph{Alessandro Michele and
Co.}\label{alessandro-michele-and-co}}

\href{/interactive/2020/04/13/t-magazine/foreign-correspondents.html}{}

\hypertarget{the-journalists}{%
\subparagraph{The Journalists}\label{the-journalists}}

\begin{itemize}
\item
\item
\item
\item
\end{itemize}

Advertisement

\protect\hyperlink{after-bottom}{Continue reading the main story}

\hypertarget{site-index}{%
\subsection{Site Index}\label{site-index}}

\hypertarget{site-information-navigation}{%
\subsection{Site Information
Navigation}\label{site-information-navigation}}

\begin{itemize}
\tightlist
\item
  \href{https://help.nytimes.com/hc/en-us/articles/115014792127-Copyright-notice}{©~2020~The
  New York Times Company}
\end{itemize}

\begin{itemize}
\tightlist
\item
  \href{https://www.nytco.com/}{NYTCo}
\item
  \href{https://help.nytimes.com/hc/en-us/articles/115015385887-Contact-Us}{Contact
  Us}
\item
  \href{https://www.nytco.com/careers/}{Work with us}
\item
  \href{https://nytmediakit.com/}{Advertise}
\item
  \href{http://www.tbrandstudio.com/}{T Brand Studio}
\item
  \href{https://www.nytimes.com/privacy/cookie-policy\#how-do-i-manage-trackers}{Your
  Ad Choices}
\item
  \href{https://www.nytimes.com/privacy}{Privacy}
\item
  \href{https://help.nytimes.com/hc/en-us/articles/115014893428-Terms-of-service}{Terms
  of Service}
\item
  \href{https://help.nytimes.com/hc/en-us/articles/115014893968-Terms-of-sale}{Terms
  of Sale}
\item
  \href{https://spiderbites.nytimes.com}{Site Map}
\item
  \href{https://help.nytimes.com/hc/en-us}{Help}
\item
  \href{https://www.nytimes.com/subscription?campaignId=37WXW}{Subscriptions}
\end{itemize}
