\hypertarget{how-daughters-of-the-dust-sent-ripples-through-the-film-world}{%
\section{How `Daughters of the Dust' Sent Ripples Through the Film
World}\label{how-daughters-of-the-dust-sent-ripples-through-the-film-world}}

April 13, 2020

\begin{itemize}
\item
\item
\item
\item
\end{itemize}

With her lyrical work, made in 1991, Julie Dash and her collaborators
recentered the black female gaze.

\href{https://www.nytimes.com/interactive/2020/04/13/t-magazine/culture-issue-2020.html}{We
Are Family}

\hypertarget{chapter-2-reunions-and-reconsiderations}{%
\subparagraph{Chapter 2: Reunions and
Reconsiderations}\label{chapter-2-reunions-and-reconsiderations}}

\hypertarget{previous}{%
\subparagraph{Previous}\label{previous}}

\hypertarget{next}{%
\subparagraph{Next}\label{next}}

\textbf{The forerunners}

\hypertarget{how-daughters-of-the-dust-sent-ripples-through-the-film-world-1}{%
\section{How `Daughters of the Dust' Sent Ripples Through the Film
World}\label{how-daughters-of-the-dust-sent-ripples-through-the-film-world-1}}

Every now and then, a piece of American performance is so memorable that
it both redefines its medium and reframes the culture at large. Here, an
appraisal of one such enduring and heavily referenced work --- a seminal
1991 film that centered the black female experience and gaze ---
alongside a gathering of the stars who not only made it but were made by
it, too.

By \href{https://www.nytimes.com/by/a-o--scott}{A.O. Scott}

April 13, 2020

SHARE

ONE OF THE sharpest memories of my first year living in New York is
standing in a line stretching from the
\href{https://www.nytimes.com/topic/organization/film-forum}{Film Forum}
box office along West Houston Street all the way to Sixth Avenue. This
was 1992, before movie tickets were bought on the internet, and the
occasion was the opening run of
``\href{https://www.nytimes.com/watching/titles/daughters-of-the-dust}{Daughters
of the Dust},''
\href{https://www.nytimes.com/2016/11/20/movies/julie-dash-daughters-of-the-dust.html}{Julie
Dash}'s first feature.

The film, which takes place on a single day in 1902, on an island off
the coast of South Carolina, was a watershed --- part of the larger
blossoming of black independent cinema whose most famous exponents were
\href{https://www.nytimes.com/topic/person/spike-lee}{Spike Lee} and
\href{https://www.nytimes.com/2019/05/14/style/john-singleton-black-film-movies.html}{John
Singleton}, and a unique, unprecedented work of art in its own right.
Dash and her collaborators (some of whom are pictured here) offered a
new approach to historical narrative, a new kind of feminist filmmaking,
a new way of thinking about the American past and a new visual
aesthetic. You left the theater trailing memories of surf and slanting
light, of women in pale cotton frocks trading gossip and wisdom on a
wide beach, of time seeming to stand still even as it moved relentlessly
forward.

The T List \textbar{}

Sign up here

In most previous Hollywood depictions of the Old South, Dash said in a
recent interview, ``we were used to seeing tropes. Someone always had to
pull out a harmonica.'' She was determined to avoid clichés about
Southern rural life, and to sidestep the usual conventions of
high-minded period entertainment. ``We shared intimate stories,'' she
tells me, ``profound moments that connect us to the history of the
diaspora --- without explaining, without going all National Geographic.
A meditative type of thing, structured the way an African griot would
narrate a family's history over days and days.'' The story, set in the
early 20th-century past, reaches even further back, concerning itself
with the structure of memory and the transmission of ancient communal
knowledge.

If ``Daughters'' isn't a standard costume drama, the costumes ---
especially those long-sleeved white cotton dresses worn by the women who
are its central characters --- are integral to its look and meaning. The
world of Dataw Island, a
\href{https://www.nytimes.com/2019/04/15/travel/south-carolina-gullah-geechee-low-country.html}{Gullah-Geechee
Sea Island} inhabited by people formerly enslaved on indigo plantations
and their children and grandchildren, is recreated with both painstaking
fidelity to detail and thrilling imaginative freedom. The multistranded
story emerges from the daily patterns of work, play and ritual, all of
which are observed with documentary precision and lyrical intensity:
Instead of images of whip-scarred backs to convey the cruelty of
enslavement, Dash showed hands dyed blue by the harsh indigo plant.

Courtesy of Cohen Film Collection

The movie itself, shot on location and using natural light, arose from
the fusion of deep scholarship and an almost mystical sense of place.
``I had put together these pamphlets --- sort of little journals --- for
everyone on the history of the Sea Islands,'' Dash says. ``The sun was
coming up, we'd be getting ready to shoot at 5 a.m., knowing this was
hallowed ground, this was our Ellis Island. The waves would be rolling
in, and sometimes people would be weeping.''

IT MAKES SENSE that a film devoted to the power and resilience of
collective experience was born in such inspired collaboration, drawing
on the talent and discipline of more than a few extraordinary artists.
\href{https://www.nytimes.com/2019/08/14/t-magazine/arthur-jafa-in-bloom.html}{Arthur
Jafa}, the director of photography, would go on to a career as a
groundbreaking cinematographer and video artist, working with Spike Lee,
\href{https://www.nytimes.com/interactive/2017/11/29/t-magazine/jay-z-dean-baquet-interview.html}{Jay-Z},
\href{https://www.nytimes.com/2014/06/03/t-magazine/beyonce-the-woman-on-top-of-the-world.html}{Beyoncé}
and
\href{https://www.nytimes.com/2018/10/15/t-magazine/solange-interview.html}{Solange
Knowles}. Alva Rogers, who played Eula Peazant (one of the film's
several matriarchal figures), is a theater artist. Cheryl Lynn Bruce,
whose Viola Peazant is one of the principal daughters, has long been a
mainstay of the Chicago stage. Her husband,
\href{https://www.nytimes.com/2016/10/17/t-magazine/kerry-james-marshall-artist.html}{Kerry
James Marshall}, the production designer, is now one of the most
important living American painters. His career retrospective,
``Mastry,'' which I saw at the
\href{https://www.nytimes.com/topic/organization/metropolitan-museum-of-art}{Metropolitan
Museum of Art} in 2017, was a life changer, a cultural event not unlike
the first screenings of ``Daughters of the Dust'' all those years ago.

The movie itself has sent ripples of influence through the culture. It's
an explicit point of reference in Beyoncé's ``Lemonade,'' with its
elliptical approach to African-American memory and its dreamy images of
women on the beach, and a benchmark for younger black filmmakers such as
\href{https://www.nytimes.com/2020/02/06/magazine/dee-rees-black-female-director.html}{Dee
Rees} and
\href{https://www.nytimes.com/interactive/2018/10/04/magazine/barry-jenkins-james-baldwin-if-beale-street-could-talk.html}{Barry
Jenkins}. Dash, along with many of her peers, was trying to fulfill an
imperative famously enunciated by
\href{https://www.nytimes.com/topic/person/toni-morrison}{Toni
Morrison}: to make work that would not solicit or rely upon the white
gaze. Not that white audiences and artists weren't paying attention. The
films of
\href{https://www.nytimes.com/2015/08/17/t-magazine/alice-alba-rohrwacher.html}{Sofia
Coppola}, especially
``\href{https://www.nytimes.com/watching/titles/the-virgin-suicides}{The
Virgin Suicides},''
``\href{https://www.nytimes.com/watching/titles/marie-antoinette}{Marie
Antoinette}'' and
``\href{https://www.nytimes.com/watching/titles/the-beguiled}{The
Beguiled},'' unfold in languid, feminine spaces conceptually (though not
culturally) adjacent to the live-oak and palmetto-shaded groves of Dataw
Island.

The other thing that distinguishes ``Daughters of the Dust'' is its
perspective. It is not about explaining black history to white people,
or making an appeal for recognition. It's about opening up a space of
memory and feeling within a larger history that had been misunderstood,
marginalized and erased outright by the dominant culture. Much is to be
learned from watching it --- about kinship and spiritual life in the Sea
Islands, about how the enslaved tried to protect their families and
shared memories from annihilation --- but the spirit of the lesson is
the warm rigor of a family member rather than the cold didacticism of a
professor. The film's beauty is its argument; its radical politics
reside in its aesthetic boldness.

Courtesy of Cohen Film Collection

Not that anyone needs to take my word for it. I am well aware, as I was
when I lined up on Houston Street almost three decades ago, of being in
possession of a white gaze, and that when I say that ``Daughters of the
Dust'' is one of the movies of my life, I may be summoning the dreaded
specter of cultural appropriation. So let me be as clear as I can be.
What I am trying to articulate here is not the appreciation or approval
of a critic, which this film hardly needs. This is nothing more, and
nothing less, than an attempt to give voice to 28 years of awe.

A.O. Scott is a critic at large at The New York Times and the author of
``Better Living Through Criticism.'' David Chow is a food, lifestyle and
travel photographer. Andres Gonzalez's most recent book, ``American
Origami,'' received the 2019 Light Work Photobook Award. He is a
visiting lecturer at the San Francisco Art Institute. Portraits taken in
Los Angeles, Chicago, New York and Atlanta on March 6, 11, 13 and 16,
2020. Hair and makeup for Cheryl Lynn Bruce and Kerry James Marshall:
Sydney Zenon at Mastermind Management Group. Hair and makeup for Alva
Rogers: Aichatou Kamate at the Teknique Agency. Hair and makeup for
Julie Dash: Brittany Hervey.

\href{https://www.nytimes.com/2019/08/14/t-magazine/arthur-jafa-in-bloom.html}{}

\hypertarget{arthur-jafa-in-bloomaug-14-2019}{%
\paragraph{Arthur Jafa in BloomAug. 14,
2019}\label{arthur-jafa-in-bloomaug-14-2019}}

\includegraphics{https://static01.nyt.com/images/2019/08/18/t-magazine/18tmag-jafa-slide-YAWD/18tmag-jafa-slide-YAWD-mediumThreeByTwo210-v2.jpg}
\href{https://www.nytimes.com/2016/10/17/t-magazine/kerry-james-marshall-artist.html}{}

\hypertarget{kerry-james-marshall-is-shifting-the-color-of-art-historyoct-17-2016}{%
\paragraph{Kerry James Marshall Is Shifting the Color of Art HistoryOct.
17,
2016}\label{kerry-james-marshall-is-shifting-the-color-of-art-historyoct-17-2016}}

\includegraphics{https://static01.nyt.com/images/2016/10/17/t-magazine/17tmag-kjmpromo/17tmag-kjmpromo-mediumThreeByTwo210.jpg}

\hypertarget{we-are-family-1}{%
\subsubsection{We Are Family}\label{we-are-family-1}}

\hypertarget{chapter-1-heirs-and-alumni}{%
\paragraph{Chapter 1: Heirs and
Alumni}\label{chapter-1-heirs-and-alumni}}

\href{/interactive/2020/04/13/t-magazine/black-art-galleries.html}{}

\hypertarget{the-artists}{%
\subparagraph{The Artists}\label{the-artists}}

\href{/interactive/2020/04/13/t-magazine/italian-fashion-design-houses.html}{}

\hypertarget{the-dynasties}{%
\subparagraph{The Dynasties}\label{the-dynasties}}

\href{/interactive/2020/04/13/t-magazine/gordon-parks.html}{}

\hypertarget{the-directors}{%
\subparagraph{The Directors}\label{the-directors}}

\href{/interactive/2020/04/13/t-magazine/enrique-olvera-chef.html}{}

\hypertarget{the-disciples}{%
\subparagraph{The Disciples}\label{the-disciples}}

\href{/interactive/2020/04/13/t-magazine/royal-academy-antwerp.html}{}

\hypertarget{the-graduates}{%
\subparagraph{The Graduates}\label{the-graduates}}

\hypertarget{chapter-2-reunions-and-reconsiderations-1}{%
\paragraph{Chapter 2: Reunions and
Reconsiderations}\label{chapter-2-reunions-and-reconsiderations-1}}

\href{/interactive/2020/04/13/t-magazine/ninth-street-greenwich-village-neighbors.html}{}

\hypertarget{the-neighbors}{%
\subparagraph{The Neighbors}\label{the-neighbors}}

\href{/interactive/2020/04/13/t-magazine/omen-restaurant-nyc.html}{}

\hypertarget{the-regulars}{%
\subparagraph{The Regulars}\label{the-regulars}}

\href{/interactive/2020/04/13/t-magazine/hair-musical-broadway.html}{}

\hypertarget{hair-1967}{%
\subparagraph{Hair (1967)}\label{hair-1967}}

\href{/interactive/2020/04/13/t-magazine/sweeney-todd-revival.html}{}

\hypertarget{sweeney-todd-2005-revival}{%
\subparagraph{Sweeney Todd (2005
Revival)}\label{sweeney-todd-2005-revival}}

\href{/interactive/2020/04/13/t-magazine/daughters-of-the-dust.html}{}

\hypertarget{daughters-of-the-dust-1991}{%
\subparagraph{Daughters of the Dust
(1991)}\label{daughters-of-the-dust-1991}}

\hypertarget{chapter-3-legends-pioneers-and-survivors}{%
\paragraph{Chapter 3: Legends Pioneers and
Survivors}\label{chapter-3-legends-pioneers-and-survivors}}

\href{/interactive/2020/04/13/t-magazine/butch-stud-lesbian.html}{}

\hypertarget{the-renegades}{%
\subparagraph{The Renegades}\label{the-renegades}}

\href{/interactive/2020/04/13/t-magazine/act-up-aids.html}{}

\hypertarget{the-activists}{%
\subparagraph{The Activists}\label{the-activists}}

\href{/interactive/2020/04/13/t-magazine/artist-recluse.html}{}

\hypertarget{the-shadows}{%
\subparagraph{The Shadows}\label{the-shadows}}

\href{/interactive/2020/04/13/t-magazine/black-actresses-bassett-berry-blige-henson-whitfield-elise.html}{}

\hypertarget{the-veterans}{%
\subparagraph{The Veterans}\label{the-veterans}}

\hypertarget{chapter-4-the-new-guard}{%
\paragraph{Chapter 4: The New Guard}\label{chapter-4-the-new-guard}}

\href{/interactive/2020/04/13/t-magazine/asian-american-fashion-designers.html}{}

\hypertarget{the-designers}{%
\subparagraph{The Designers}\label{the-designers}}

\href{13tmag-beauties.html}{}

\hypertarget{the-beauties}{%
\subparagraph{The Beauties}\label{the-beauties}}

\href{/interactive/2020/04/13/t-magazine/nyc-downtown-nightlife-party-scene.html}{}

\hypertarget{the-scenemakers}{%
\subparagraph{The Scenemakers}\label{the-scenemakers}}

\href{/interactive/2020/04/13/t-magazine/maria-cornejo-olivier-rousteing-telfar-clemens-alessandro-michele.html\#olivier-rousteing-and-co}{}

\hypertarget{olivier-rousteing-and-co}{%
\subparagraph{Olivier Rousteing and
Co.}\label{olivier-rousteing-and-co}}

\href{/interactive/2020/04/13/t-magazine/maria-cornejo-olivier-rousteing-telfar-clemens-alessandro-michele.html\#maria-cornejo-and-co}{}

\hypertarget{maria-cornejo-and-co}{%
\subparagraph{Maria Cornejo and Co.}\label{maria-cornejo-and-co}}

\href{/interactive/2020/04/13/t-magazine/maria-cornejo-olivier-rousteing-telfar-clemens-alessandro-michele.html\#telfar-clemens-and-co}{}

\hypertarget{telfar-clemens-and-co}{%
\subparagraph{Telfar Clemens and Co.}\label{telfar-clemens-and-co}}

\href{/interactive/2020/04/13/t-magazine/maria-cornejo-olivier-rousteing-telfar-clemens-alessandro-michele.html\#alessandro-michele-and-co}{}

\hypertarget{alessandro-michele-and-co}{%
\subparagraph{Alessandro Michele and
Co.}\label{alessandro-michele-and-co}}

\href{/interactive/2020/04/13/t-magazine/foreign-correspondents.html}{}

\hypertarget{the-journalists}{%
\subparagraph{The Journalists}\label{the-journalists}}

\begin{itemize}
\item
\item
\item
\item
\end{itemize}

Advertisement

\protect\hyperlink{after-bottom}{Continue reading the main story}

\hypertarget{site-index}{%
\subsection{Site Index}\label{site-index}}

\hypertarget{site-information-navigation}{%
\subsection{Site Information
Navigation}\label{site-information-navigation}}

\begin{itemize}
\tightlist
\item
  \href{https://help.nytimes.com/hc/en-us/articles/115014792127-Copyright-notice}{©~2020~The
  New York Times Company}
\end{itemize}

\begin{itemize}
\tightlist
\item
  \href{https://www.nytco.com/}{NYTCo}
\item
  \href{https://help.nytimes.com/hc/en-us/articles/115015385887-Contact-Us}{Contact
  Us}
\item
  \href{https://www.nytco.com/careers/}{Work with us}
\item
  \href{https://nytmediakit.com/}{Advertise}
\item
  \href{http://www.tbrandstudio.com/}{T Brand Studio}
\item
  \href{https://www.nytimes.com/privacy/cookie-policy\#how-do-i-manage-trackers}{Your
  Ad Choices}
\item
  \href{https://www.nytimes.com/privacy}{Privacy}
\item
  \href{https://help.nytimes.com/hc/en-us/articles/115014893428-Terms-of-service}{Terms
  of Service}
\item
  \href{https://help.nytimes.com/hc/en-us/articles/115014893968-Terms-of-sale}{Terms
  of Sale}
\item
  \href{https://spiderbites.nytimes.com}{Site Map}
\item
  \href{https://help.nytimes.com/hc/en-us}{Help}
\item
  \href{https://www.nytimes.com/subscription?campaignId=37WXW}{Subscriptions}
\end{itemize}
