Sections

SEARCH

\protect\hyperlink{site-content}{Skip to
content}\protect\hyperlink{site-index}{Skip to site index}

\hypertarget{comments}{%
\subsection{\texorpdfstring{\protect\hyperlink{commentsContainer}{Comments}}{Comments}}\label{comments}}

\href{}{They Survived Hurricane Dorian. Their Community Will
Not.}\href{}{Skip to Comments}

The comments section is closed. To submit a letter to the editor for
publication, write to
\href{mailto:letters@nytimes.com}{\nolinkurl{letters@nytimes.com}}.

\hypertarget{they-survived-hurricane-dorian-their-community-will-not}{%
\section{They Survived Hurricane Dorian. Their Community Will
Not.}\label{they-survived-hurricane-dorian-their-community-will-not}}

By \href{https://www.nytimes.com/by/kk-rebecca-lai}{K.K. Rebecca Lai},
\href{https://www.nytimes.com/by/derek-watkins}{Derek Watkins}, Niko
Koppel and \href{https://www.nytimes.com/by/anjali-singhvi}{Anjali
Singhvi}Sept. 25, 2019

\begin{itemize}
\item
\item
\item
\item
\item
  \emph{+}
\end{itemize}

"\textgreater{}

This is what's left of Sand Banks, a small community on Great Abaco
Island in the Bahamas that was obliterated by Hurricane Dorian.

Few houses here were strong enough to withstand the storm. Now the
Haitian immigrants who lived in this shantytown will find it especially
hard to rebuild.

This three-dimensional model shows what remained of the community after
the Category 5 hurricane battered the Bahamas for two days in early
September.

"\textgreater{}

The tightly packed houses, mostly built with plywood and two-by-fours,
were easily shredded. Offrane Francis's home, with its cement
foundation, was one of the few left partially standing.

Mr. Francis had dug one of two working wells in the community. A handful
of neighbors and friends gathered there to wash clothes one evening.

Knox Kronenberg via Austin Helping Abaco

\includegraphics{https://static01.nyt.com/newsgraphics/2019/09/11/hurricane-dorian-abaco/3f8765a5dcebd265a939e6cd3ff9ca5621ccaa89/before.jpg}

Most families had already evacuated to Nassau, the capital, about 100
miles away, to find shelter and aid. Some chose to remain in Sand Banks
because they were unfamiliar with the capital. Others were hoping to
start picking up the pieces.

Knox Kronenberg via Austin Helping Abaco

The people who lived in Sand Banks had little and lost nearly
everything. This photo from July offers a glimpse of the community
before the storm.

Knox Kronenberg via Austin Helping Abaco

Most of the 250 or so residents had come from Haiti and found work in
construction or landscaping in resorts and homes in the area, about 20
miles from Marsh Harbour.

\includegraphics{https://static01.nyt.com/newsgraphics/2019/09/11/hurricane-dorian-abaco/3f8765a5dcebd265a939e6cd3ff9ca5621ccaa89/map-abaco-Artboard_1.png}

Sand Banks

Marsh

Harbour

GREAT

ABACO

ISLAND

10 miles

Families typically made less than \$400 a week. Sand Banks had no
running water, and electricity was available only in a few houses.

 It was one of several Haitian settlements on Great Abaco Island. The
hurricane also leveled two others, The Mudd and Pigeon Peas, in Marsh
Harbour. Across the island, dozens were killed.

Here in Sand Banks, everyone survived, with many residents riding out
the storm together in a nearby church. But now they are stuck in
uncertainty.

"\textgreater{}

Ironel Joseph, 47, lived with his family in a small blue house, No. 12.
While three of his children played nearby, he worked to salvage what he
could.

This was Mr. Joseph's second house in Sand Banks. The first was
destroyed in a fire that wiped out half the community in 2014.

The government did not allow rebuilding after the fire --- part of a
long struggle with the Haitian communities over what officials described
as unsafe conditions in the shantytowns. Threats of eviction and
demolition were common, and residents feared more of the same in the
storm's aftermath.

Some weighed whether to stay in Sand Banks or follow their neighbors to
Nassau, or even return to Haiti.

``Going back to Haiti is on our mind, but we don't want to go back,''
said Joseph Fredet, 47, a gardener. ``It's harder there.'' He, his wife
and his daughter were still sleeping in the church.

"\textgreater{}

Jeannot Pierre, 43, and his wife, Wilda, headed to the airport after the
storm to try to get off the island but could not.

Mr. Pierre said that he felt Haitians were often discriminated against
in the Bahamas. ``Some of them just want to try to rush you out of the
country,'' he said.

Mr. Pierre had no clothes and no money left. He said he wanted to leave
the island because he couldn't handle being surrounded by others talking
endlessly about what to do next.

Haitians have long been a vital part of Abaco's workforce and could play
an important role in rebuilding across the island. Mr. Francis said he
had already found work in recovery and reconstruction efforts.

"\textgreater{}

But no one can stay in Sand Banks in the long run. Confirming residents'
fears, the government announced last week that rebuilding in Abaco's
shantytowns was prohibited.

About this article

Footage of Sand Banks was collected on Sept. 7 and 8, four days after
Hurricane Dorian passed. The 3D model was created with a process called
photogrammetry, which used hundreds of aerial photographs to reconstruct
the community.

Derek Watkins, K.K. Rebecca Lai and Niko Koppel reported from Sand
Banks, the Bahamas, and Anjali Singhvi from New York. The model was
produced by Justin Blinder, Or Fleisher, Miles Peyton and Guilherme
Rambelli.

Write a comment

\begin{itemize}
\item
\item
\item
\item
\end{itemize}

Advertisement

\protect\hyperlink{after-bottom}{Continue reading the main story}

\hypertarget{site-index}{%
\subsection{Site Index}\label{site-index}}

\hypertarget{site-information-navigation}{%
\subsection{Site Information
Navigation}\label{site-information-navigation}}

\begin{itemize}
\tightlist
\item
  \href{https://help.nytimes.com/hc/en-us/articles/115014792127-Copyright-notice}{©~2020~The
  New York Times Company}
\end{itemize}

\begin{itemize}
\tightlist
\item
  \href{https://www.nytco.com/}{NYTCo}
\item
  \href{https://help.nytimes.com/hc/en-us/articles/115015385887-Contact-Us}{Contact
  Us}
\item
  \href{https://www.nytco.com/careers/}{Work with us}
\item
  \href{https://nytmediakit.com/}{Advertise}
\item
  \href{http://www.tbrandstudio.com/}{T Brand Studio}
\item
  \href{https://www.nytimes.com/privacy/cookie-policy\#how-do-i-manage-trackers}{Your
  Ad Choices}
\item
  \href{https://www.nytimes.com/privacy}{Privacy}
\item
  \href{https://help.nytimes.com/hc/en-us/articles/115014893428-Terms-of-service}{Terms
  of Service}
\item
  \href{https://help.nytimes.com/hc/en-us/articles/115014893968-Terms-of-sale}{Terms
  of Sale}
\item
  \href{https://spiderbites.nytimes.com}{Site Map}
\item
  \href{https://help.nytimes.com/hc/en-us}{Help}
\item
  \href{https://www.nytimes.com/subscription?campaignId=37WXW}{Subscriptions}
\end{itemize}
