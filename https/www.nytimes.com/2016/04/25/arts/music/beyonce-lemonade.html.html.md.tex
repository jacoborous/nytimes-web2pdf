Sections

SEARCH

\protect\hyperlink{site-content}{Skip to
content}\protect\hyperlink{site-index}{Skip to site index}

\href{https://www.nytimes.com/section/arts/music}{Music}

\href{https://myaccount.nytimes.com/auth/login?response_type=cookie\&client_id=vi}{}

\href{https://www.nytimes.com/section/todayspaper}{Today's Paper}

\href{/section/arts/music}{Music}\textbar{}Review: Beyoncé Makes
`Lemonade' Out of Marital Strife

\url{https://nyti.ms/1SFurK7}

\begin{itemize}
\item
\item
\item
\item
\item
\end{itemize}

Advertisement

\protect\hyperlink{after-top}{Continue reading the main story}

Supported by

\protect\hyperlink{after-sponsor}{Continue reading the main story}

\hypertarget{review-beyoncuxe9-makes-lemonade-out-of-marital-strife}{%
\section{Review: Beyoncé Makes `Lemonade' Out of Marital
Strife}\label{review-beyoncuxe9-makes-lemonade-out-of-marital-strife}}

\includegraphics{https://static01.nyt.com/images/2016/04/24/arts/25BEYONCE-hp/25BEYONCE-hp-articleLarge-v2.jpg?quality=75\&auto=webp\&disable=upscale}

By \href{http://www.nytimes.com/by/jon-pareles}{Jon Pareles}

\begin{itemize}
\item
  April 24, 2016
\item
  \begin{itemize}
  \item
  \item
  \item
  \item
  \item
  \end{itemize}
\end{itemize}

Marital strife smolders, explodes and uneasily subsides on ``Lemonade''
(Parkwood Entertainment), the album Beyoncé flash-released on Saturday
night. ``You can taste the dishonesty/It's all over your breath'' are
the first words she sings in ``Pray You Catch Me,'' and that's just the
beginning of an album that probes betrayal, jealousy, revenge and rage
before dutifully willing itself toward reconciliation at the end. Many
of the accusations are aimed specifically and recognizably at her
husband, Shawn Carter, the rapper Jay Z. ``Tonight I regret the night I
put that ring on,'' she talk-sings in ``Sorry,'' a twitchy, flippant
song that's by no means an apology. It's a combative, unglossy track on
an album full of them.

``Lemonade'' is the kind of album that a star like Beyoncé (as well as,
lately, Rihanna) can release in the streaming era because she's already
guaranteed attention for her every utterance. The album is not beholden
to radio formats or presold by a single; fans are likely to explore the
whole album, streaming every track and hearing how far afield --- a
brass band, stomping blues-rock, ultraslow avant-R\&B --- Beyoncé is
willing to go. As she did with her
\href{http://www.nytimes.com/2013/12/14/arts/music/beyonces-new-album-is-steamy-and-sleek.html?_r=0}{2013
album, ``Beyoncé,''} she has also paired the music with full-length
video that expands and deepens its impact.

On their own, the songs can be taken as one star's personal, domestic
dramas, waiting to be mined by the tabloids. But with the video, they
testify to situations and emotions countless women endure. It's not a
divorce announcement; the singer, songwriter and director is credited as
Beyoncé Knowles Carter.

Beyoncé released ``Lemonade'' online at 10 p.m. on April 23, immediately
after the HBO showing of the hourlong ``visual album'' version. It's a
quick-cutting music video that intersperses the songs, and broadens
them, with compelling poetry from the Somali-British writer
\href{http://www.newyorker.com/culture/cultural-comment/the-writing-life-of-a-young-prolific-poet-warsan-shire}{Warsan
Shire}, poems that often extend women's physicality toward the
archetypal. As Beyoncé recites them, Ms. Shire's words radically reframe
the songs, so they are no longer one woman's struggles but tribulations
shared through generations of mothers and daughters. The video is filled
with images of female solidarity and of family, Southern and African
roots, women of all ages and roles and eras. Often, Beyoncé is joined by
African-American women in white clothes enacting shared work, gatherings
of women or eerie communal rituals. Beyoncé, in multiple hairstyles and
fashions, is shown both alluring and unglamorous: hard-faced, unhappy,
sweaty, harshly lit. For the last few songs she often appears in a
puffy-sleeved antebellum-style dress remade with fabric patterns derived
from African textiles, a rich twist.

The album title comes from a family gathering that's shown in the video
and heard on a track: the 90th birthday of Hattie White, Jay Z's
grandmother, who says, ``I was served lemons but I made lemonade.''

\includegraphics{https://static01.nyt.com/images/2016/04/25/arts/25beyoncelp-web1/25beyoncelp-web1-articleLarge.jpg?quality=75\&auto=webp\&disable=upscale}

``Lemonade'' is not necessarily the album listeners might have expected
after ``Formation,'' the
\href{http://www.nytimes.com/2016/02/07/arts/music/beyonce-formation-super-bowl-video.html}{song
Beyoncé performed at the Super Bowl} with dancers in Black Panther-style
outfits and in a video clip using images of New Orleans, of
African-Americans in a plantation mansion and of Beyoncé atop a police
car, sinking under a flood. It's the last song on ``Lemonade,'' almost a
postscript; it's not in the extended video.

One other song on ``Lemonade'' mixes preaching and a prison song (both
collected by John and Alan Lomax), a Kendrick Lamar rap and 1960s
fuzz-tone psychedelia (sampling the collectors' item Puerto Rican band
\href{http://www.nowagainrecords.com/kaleidoscope/}{Kaleidoscope}) to
call for ``Freedom'': ``I break chains all by myself/Won't let my
freedom rot in hell,'' Beyoncé vows.

But most of ``Lemonade'' arrives like a follow-through to ``Jealous'' on
the 2013 ``Beyoncé,'' a song that moans, ``I hate you for your lies.''
``Jealous'' is offset on ``Beyoncé'' by songs about ecstatic lust, a
topic largely absent on ``Lemonade.'' In most of the new songs, Beyoncé
has been taken for granted or pushed aside. It's a situation that, she
finds, is both ``a wicked way to treat the girl that loves you'' and
also flabbergasting given that she is, after all, Beyoncé. Beyoncé!:
``The baddest woman in the game,'' as she sings in ``Hold Up.''
Fact-check: She is.

Her reactions swing from sorrow to rage to determined loyalty, and she
reaches beyond the electronic-R\&B of ``Beyoncé'' to embrace new
influences and collaborators: the Yeah Yeah Yeahs, Father John Misty,
Vampire Weekend's Ezra Koenig, Animal Collective and Led Zeppelin.
``Don't Hurt Yourself,'' a collaboration with Jack White, is a
funk-bottomed blues-rocker that has Beyoncé fighting back, declaring,
``You ain't trying hard enough/You ain't loving hard enough,'' working
up to a scream. ``Pray You Catch Me'' is one of two collaborations with
the British songwriter James Blake: slow-motion ballads of suspicion and
longing. During ``Forward,'' the other Blake collaboration, the video
has its most moving sequence: family members stoically holding
photographs of men who were killed by police. It's followed by a scene
of a New Orleans Mardi Gras Indian in full feathered and beaded costume,
shaking a tambourine in posh dining rooms as if to exorcise them.

Yet eventually, she makes peace with trying to hold on. ``Love
Drought,'' with whispery vocals amid pillowy synthesizers, points out
that ``10 times out of nine I know you're lying,'' but strives to
reconnect. ``Sandcastles,'' a slow piano hymn that eventually gathers a
choir, recalls a dish-smashing fight but turns a double negative into a
positive: ``I know I promised that I couldn't stay, baby/Every promise
don't work out that way.'' By the time Beyoncé reaches ``All Night,'' a
gospelly ballad roughened with electric guitar, she resolves to ``Give
you some time to prove I can trust you again.''

Will it work out? No one knows. But in the meantime she sings
wholeheartedly, encapsulates deep dilemmas in terse singalong lines and
touches on ideas and emotions that so many people feel. She is a star
whose world is vastly different from that of her listeners. But in
matters of the heart, with their complications and paradoxes, Beyoncé
joins all of us.

Advertisement

\protect\hyperlink{after-bottom}{Continue reading the main story}

\hypertarget{site-index}{%
\subsection{Site Index}\label{site-index}}

\hypertarget{site-information-navigation}{%
\subsection{Site Information
Navigation}\label{site-information-navigation}}

\begin{itemize}
\tightlist
\item
  \href{https://help.nytimes.com/hc/en-us/articles/115014792127-Copyright-notice}{©~2020~The
  New York Times Company}
\end{itemize}

\begin{itemize}
\tightlist
\item
  \href{https://www.nytco.com/}{NYTCo}
\item
  \href{https://help.nytimes.com/hc/en-us/articles/115015385887-Contact-Us}{Contact
  Us}
\item
  \href{https://www.nytco.com/careers/}{Work with us}
\item
  \href{https://nytmediakit.com/}{Advertise}
\item
  \href{http://www.tbrandstudio.com/}{T Brand Studio}
\item
  \href{https://www.nytimes.com/privacy/cookie-policy\#how-do-i-manage-trackers}{Your
  Ad Choices}
\item
  \href{https://www.nytimes.com/privacy}{Privacy}
\item
  \href{https://help.nytimes.com/hc/en-us/articles/115014893428-Terms-of-service}{Terms
  of Service}
\item
  \href{https://help.nytimes.com/hc/en-us/articles/115014893968-Terms-of-sale}{Terms
  of Sale}
\item
  \href{https://spiderbites.nytimes.com}{Site Map}
\item
  \href{https://help.nytimes.com/hc/en-us}{Help}
\item
  \href{https://www.nytimes.com/subscription?campaignId=37WXW}{Subscriptions}
\end{itemize}
