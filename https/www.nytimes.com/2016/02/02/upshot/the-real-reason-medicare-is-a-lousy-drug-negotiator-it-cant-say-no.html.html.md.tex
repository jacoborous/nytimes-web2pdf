Sections

SEARCH

\protect\hyperlink{site-content}{Skip to
content}\protect\hyperlink{site-index}{Skip to site index}

\href{https://myaccount.nytimes.com/auth/login?response_type=cookie\&client_id=vi}{}

\href{https://www.nytimes.com/section/todayspaper}{Today's Paper}

\href{/section/upshot}{The Upshot}\textbar{}The Real Reason Medicare Is
a Lousy Drug Negotiator: It Can't Say No

\url{https://nyti.ms/1Kn25jR}

\begin{itemize}
\item
\item
\item
\item
\item
\item
\end{itemize}

Advertisement

\protect\hyperlink{after-top}{Continue reading the main story}

Supported by

\protect\hyperlink{after-sponsor}{Continue reading the main story}

Upshot

Public Health

\hypertarget{the-real-reason-medicare-is-a-lousy-drug-negotiator-it-cant-say-no}{%
\section{The Real Reason Medicare Is a Lousy Drug Negotiator: It Can't
Say
No}\label{the-real-reason-medicare-is-a-lousy-drug-negotiator-it-cant-say-no}}

\includegraphics{https://static01.nyt.com/images/2016/02/01/upshot/02up-drugs/02up-drugs-articleLarge-v6.jpg?quality=75\&auto=webp\&disable=upscale}

By \href{http://www.nytimes.com/by/margot-sanger-katz}{Margot
Sanger-Katz}

\begin{itemize}
\item
  Feb. 2, 2016
\item
  \begin{itemize}
  \item
  \item
  \item
  \item
  \item
  \item
  \end{itemize}
\end{itemize}

A good negotiator needs to be able to walk away.

That is a rule that, surely, Donald Trump
\href{http://www.nytimes.com/2015/09/20/upshot/donald-trump-and-the-art-of-the-public-sector-deal.html}{knows}.
And yet in suggesting that Medicare could find big discounts by letting
the government negotiate directly over drug prices, he seems to have
forgotten it.

Mr. Trump has joined
\href{https://www.hillaryclinton.com/briefing/factsheets/2015/09/21/hillary-clinton-plan-for-lowering-prescription-drug-costs/}{Hillary
Clinton} and
\href{https://berniesanders.com/issues/medicare-for-all/}{Bernie
Sanders} in calling for a federal government program to negotiate for
Medicare's drug prices. The current system has private insurance
companies each negotiating separate deals on behalf of large groups of
Medicare patients. Right now, the program is O.K. at negotiating, saving
as much as 30 percent off the list price of drugs, according to
government reports. But Medicare still pays much, much more than
government health systems in other countries.

The idea of government directly negotiating with drug makers has been a
liberal favorite ever since Medicare began paying for drugs 10 years
ago. You can see the appeal. The thinking goes like this: Medicare's
drug plans cover about 37 million people. Maybe if it bargained on
behalf of all those beneficiaries as one, instead of dividing them into
a series of smaller groups, it could get better deals. Other countries,
like Britain, where the government purchases drugs for everyone in bulk,
pay much, much less for drugs than the United States. In those
countries, private companies don't do the negotiating; the government
does. And they don't split the big market.

``We don't do it,'' Mr. Trump said at a Farmington, N.H., campaign
event,
\href{http://www.statnews.com/2016/01/26/trump-negotiate-drug-prices/}{according
to The Associated Press.} ``Why? Because of drug companies.''

But if you talk to experts who study the pharmaceutical market in the
United States, they aren't optimistic that, by itself, letting the
government play drug negotiator would take a big bite out of
prescription drug spending.

For one, the companies that run the Medicare drug plans aren't really
that small, because they also provide drug benefits to companies and
individuals with health insurance. The largest pharmacy benefit manager,
Express Scripts, covers more than 85 million Americans. Other big
pharmacy benefit managers include CVS/Caremark and OptumRx.

But size isn't the only issue. The real problem is that Medicare can
very rarely say ``No way'' to a drug company. Medicare beneficiaries
wanted the program to cover most drugs that older people would want to
use. So Congress put in place rules that strengthen the hand of the drug
companies in negotiations.

Medicare is required to cover
\href{https://www.cms.gov/Medicare/Prescription-Drug-Coverage/PrescriptionDrugCovContra/downloads/chapter6.pdf\#page=23}{almost
every cancer treatment that is approved} by the Food and Drug
Administration, for example, one of six categories where the drug plans
can almost never say no to the drug companies. If a drug maker comes out
with a new cancer medicine with a sky-high price, there's not much a
Medicare plan can do to talk it down.

``If you say, `We need to get lower prices,' and they just say, `No,'
what are you going to do?'' said Walid Gellad, an associate professor of
medicine and the co-director of the Center for Pharmaceutical Policy and
Prescribing at the University of Pittsburgh.

The Congressional Budget Office has examined several proposals to allow
the government to negotiate on drug prices, and it has repeatedly said
that the savings would be
``\href{https://www.cbo.gov/sites/default/files/108th-congress-2003-2004/reports/03-03-wyden.pdf}{negligible}''
without other major policy changes. Medicare's actuary has reached
\href{https://www.cms.gov/Newsroom/MediaReleaseDatabase/Press-releases/2007-Press-releases-items/2007-01-11.html}{similar
conclusions}.

``To negotiate prices any further, the government would need to impose
access or coverage restrictions on medicines,'' said Doug Elmendorf,
testifying before Congress in 2009. Elmendorf was the director of the
budget office then; he is now the dean of Harvard's John F. Kennedy
School of Government.

The government does have one program that can say ``no'' to drug
companies, and it gets much better deals than Medicare. The Department
of Veterans Affairs negotiates hard with drugmakers. But it is also
bound by fewer rules than Medicare, and one result is that it covers far
fewer drugs.

\href{http://papers.ssrn.com/sol3/papers.cfm?abstract_id=1809665}{A 2011
analysis} by Austin Frakt, an economist and Upshot contributor, and two
co-authors found that if Medicare limited drugs the way the V.A. does,
it could save about \$510 in drug spending for every beneficiary every
year. But those beneficiaries would lose access to many drugs they were
previously taking. In fact, many older patients who get their health
insurance from the V.A. also sign up for Medicare drug plans to cover
medicines that the V.A. won't.

Other countries act more like the V.A. than Medicare. In Britain, drug
makers that won't negotiate won't be able to sell any drugs at all.
That's real leverage, but it also means that patients in England don't
have access to all the drugs that older people in the United States
might want to take.

The trade-offs between price and generosity are real and wrenching. None
of the candidates currently talking about allowing Medicare to negotiate
for drugs have endorsed allowing Medicare to say no more often.
\href{https://berniesanders.com/issues/medicare-for-all/}{Bernie
Sanders's plan} strongly suggests that
\href{http://www.nytimes.com/2016/01/20/upshot/for-now-bernie-sanderss-health-plan-is-more-of-a-tax-plan.html}{all
drugs would be covered with no co-payments at all}. Such a plan might
make the government a more generous insurer than it is now, but could
also result in even higher drug prices.

Advertisement

\protect\hyperlink{after-bottom}{Continue reading the main story}

\hypertarget{site-index}{%
\subsection{Site Index}\label{site-index}}

\hypertarget{site-information-navigation}{%
\subsection{Site Information
Navigation}\label{site-information-navigation}}

\begin{itemize}
\tightlist
\item
  \href{https://help.nytimes.com/hc/en-us/articles/115014792127-Copyright-notice}{©~2020~The
  New York Times Company}
\end{itemize}

\begin{itemize}
\tightlist
\item
  \href{https://www.nytco.com/}{NYTCo}
\item
  \href{https://help.nytimes.com/hc/en-us/articles/115015385887-Contact-Us}{Contact
  Us}
\item
  \href{https://www.nytco.com/careers/}{Work with us}
\item
  \href{https://nytmediakit.com/}{Advertise}
\item
  \href{http://www.tbrandstudio.com/}{T Brand Studio}
\item
  \href{https://www.nytimes.com/privacy/cookie-policy\#how-do-i-manage-trackers}{Your
  Ad Choices}
\item
  \href{https://www.nytimes.com/privacy}{Privacy}
\item
  \href{https://help.nytimes.com/hc/en-us/articles/115014893428-Terms-of-service}{Terms
  of Service}
\item
  \href{https://help.nytimes.com/hc/en-us/articles/115014893968-Terms-of-sale}{Terms
  of Sale}
\item
  \href{https://spiderbites.nytimes.com}{Site Map}
\item
  \href{https://help.nytimes.com/hc/en-us}{Help}
\item
  \href{https://www.nytimes.com/subscription?campaignId=37WXW}{Subscriptions}
\end{itemize}
