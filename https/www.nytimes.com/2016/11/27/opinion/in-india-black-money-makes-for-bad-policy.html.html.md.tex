Sections

SEARCH

\protect\hyperlink{site-content}{Skip to
content}\protect\hyperlink{site-index}{Skip to site index}

\href{https://myaccount.nytimes.com/auth/login?response_type=cookie\&client_id=vi}{}

\href{https://www.nytimes.com/section/todayspaper}{Today's Paper}

\href{/section/opinion}{Opinion}\textbar{}In India, Black Money Makes
for Bad Policy

\url{https://nyti.ms/2g81tUw}

\begin{itemize}
\item
\item
\item
\item
\item
\end{itemize}

Advertisement

\protect\hyperlink{after-top}{Continue reading the main story}

Supported by

\protect\hyperlink{after-sponsor}{Continue reading the main story}

\href{/section/opinion}{Opinion}

Op-Ed Contributor

\hypertarget{in-india-black-money-makes-for-bad-policy}{%
\section{In India, Black Money Makes for Bad
Policy}\label{in-india-black-money-makes-for-bad-policy}}

By Kaushik Basu

\begin{itemize}
\item
  Nov. 27, 2016
\item
  \begin{itemize}
  \item
  \item
  \item
  \item
  \item
  \end{itemize}
\end{itemize}

\includegraphics{https://static01.nyt.com/images/2016/11/28/opinion/28basu/28basu-articleLarge.jpg?quality=75\&auto=webp\&disable=upscale}

NEW DELHI --- On Nov. 8, the Indian government announced an immediate
ban on two major bills that account for the vast majority of all
currency in circulation. Indians would have until the end of the year to
change those notes for other bills, including newly minted ones.

On Wednesday, the government released via a smartphone app called
``Narendra Modi,'' named after the prime minister, the results of a
survey purporting to show 90 percent support for its so-called
demonetization policy.

\href{http://scroll.in/article/822358/the-daily-fix-modis-cynical-app-survey-betrays-the-governments-insensitivity-on-demonetisation}{The
poll was rightly criticized.}In the two weeks after the measure was
announced, millions of Indians stricken with small panic
\href{http://www.nytimes.com/2016/11/09/business/india-bans-largest-currency-bills-for-now-n-bid-to-cut-corruption.html}{rushed
out to banks}; A.T.M.s and tellers soon ran dry.
\href{http://www.bloombergquint.com/business/2016/11/09/the-beginning-of-the-end-of-the-parallel-economy-in-india}{Some
98 percent of all transactions in India}, measured by volume, are
conducted in cash.

Demonetization was ostensibly implemented to combat corruption,
terrorism financing and
\href{http://www.narendramodi.in/text-of-prime-minister-s-address-to-the-nation-533024}{inflation}.
But it was poorly designed, with scant attention paid to the laws of the
market, and it is likely to fail. So far its effects have been
disastrous for the middle- and lower-middle classes,
\href{http://scroll.in/article/822101/how-four-families-have-survived-two-weeks-of-demonetisation}{as
well as the poor}. And the worst may be yet to come.

India has a large amount of what is known as ``black money,'' meaning
cash or any other form of wealth that has evaded taxation. According to
a 2010 World Bank estimate, the most reliable available, the shadow
economy in India makes up one-fifth of the country's G.D.P. (A 2013
study by McKinsey, the consulting firm, puts the figure at
\href{http://webcache.googleusercontent.com/search?q=cache:jQBKN2vqEa0J:www.mckinsey.com/\%7E/media/mckinsey/dotcom/client_service/financial\%2520services/latest\%2520thinking/payments/mop16_forging_a_path_to_payments_digitization.ashx+\&cd=1\&hl=en\&ct=clnk\&gl}{more
than one-quarter}.)

Black money tends to exacerbate inequality because the biggest evasions
occur at the top of the income spectrum. It also deprives the government
of money to spend on infrastructure and public services like health care
and education. According to the World Bank's most recent estimate, from
2012,
\href{https://issuu.com/world.bank.publications/docs/9781464806834?e=0/35179276}{India's
tax-to-G.D.P. ratio is about 11 percent}, compared with about 14 percent
for Brazil, about 26 percent for South Africa and about 35 percent for
Denmark.

The government's wish to tackle these problems is laudable, but
demonetization is a ham-fisted move that will put only a temporary dent
in corruption, if even that, and is likely to rock the entire economy.

Many Indians have been scrambling to change their old notes, causing
snaking queues in front of banks and desperation among the poor, many of
whom have no bank account and live from cash earnings.

Anyone seeking to convert more than 250,000 rupees (about \$3,650) must
explain why they hold so much cash, or failing that, must pay a penalty.
The requirement has already spawned a new black market to service people
wishing to offload: Large amounts of illicit cash are broken into
smaller blocks and deposited by teams of illegal couriers.

Demonetization is mostly hurting people who aren't its intended targets.
Because sellers of certain durables, such as jewelry and property, often
insist on cash payments, many individuals who have no illegal money
build up cash reserves over time. Relatively poor women stash away cash
beyond their husbands' reach, as savings for the children or the
household.

Small hoarders often fear being questioned about the source of their
money --- they are accustomed to being harassed by tax collectors, among
others --- and may choose instead to forgo some of their savings.

People have also been skimping in response to the new policy, causing
demand for certain basic goods to fall, which has hurt farmers and small
producers and could eventually lead them to scale back on their
activities.

And even more pain is around the corner. With so much money in
circulation suddenly ceasing to be legal tender, India's economic growth
is bound to nose-dive. Another risk is that the Indian rupee could
depreciate as a result of people and investors moving to more robust
currencies.

The government's demonetization dragnet will no doubt catch some illicit
cash. Some people will turn in their black money and pay a penalty;
others will
\href{http://www.ndtv.com/india-news/sacks-full-of-burnt-500-and-1-000-rupee-notes-in-uttar-pradesh-1623440}{destroy
part of their illegal stashes}in order not to draw attention to their
businesses. But the overall benefits will be small and fleeting.

One reason is that the bulk of black money in India isn't money at all:
It's held in gold and silver, real estate and overseas bank accounts.
Another is that even if demonetization can flush out the black money
that is held in cash, with no improvement in catching and punishing tax
evaders, people with ill-gotten gains will simply start saving in the
new bills currently being issued.

When the government
\href{http://finmin.nic.in/press_room/2016/press_cancellation_high_denomination_notes.pdf}{announced
demonetization}, it also justified the measure as a way to curb
terrorism financing that relies on counterfeit rupee notes, as well as
to dampen inflation.

Both these justifications are flawed. Catching fake notes already in
circulation neither helps trap the terrorists who minted them nor
prevents more such money from being injected into the economy. It simply
inconveniences the people who use it as legal tender, the vast majority
of whom had no hand in its creation.

There also is no evidence that black money actually is more inflationary
than white money; nor in theory should it be. Black money is just money
held by people instead of the government. It's an excessive money supply
that tends to create inflation; whether that money is white or black
makes little difference.

Demonetization may have been well-intentioned, but it was a major
mistake. The government should reverse it. It could at least declare
that 500 rupee notes, which many poorer people frequently use, are legal
again.

And if the government really does want to limit the amount of black
money in circulation, it would do better to move India toward becoming a
more cashless society. About 53 percent of adult Indians
\href{http://datatopics.worldbank.org/financialinclusion/country/india}{have
a bank account}, but many signed up at the government's initiative and
so quite a few of the accounts are dormant. On the other hand, more than
one billion people in India have a cellphone, and this could be tapped
to encourage more active banking, in the form of mobile banking.

India's push to issue a unique I.D. number to all Indians based on their
biometric information is a major step in the right direction.
\href{https://portal.uidai.gov.in/uidwebportal/dashboard.do}{More than
one billion people} have already been registered, according to the
government, potentially enabling them to use an app to collect pensions,
for example.

Tackling corruption also goes beyond currency, cash or even banking. It
requires changing institutions and mind-sets, and carefully crafting
policies that acknowledge the complexity of economic and social life.
The government could start by increasing penalties for tax evasion and
amending India's outdated anti-graft laws.

In a country like India, where the illegal economy is so intimately
intertwined with the mainstream economy, one inept government
intervention against shadow activities can do a lot of harm to the vast
majority, who are just trying to make a legitimate living.

Advertisement

\protect\hyperlink{after-bottom}{Continue reading the main story}

\hypertarget{site-index}{%
\subsection{Site Index}\label{site-index}}

\hypertarget{site-information-navigation}{%
\subsection{Site Information
Navigation}\label{site-information-navigation}}

\begin{itemize}
\tightlist
\item
  \href{https://help.nytimes.com/hc/en-us/articles/115014792127-Copyright-notice}{©~2020~The
  New York Times Company}
\end{itemize}

\begin{itemize}
\tightlist
\item
  \href{https://www.nytco.com/}{NYTCo}
\item
  \href{https://help.nytimes.com/hc/en-us/articles/115015385887-Contact-Us}{Contact
  Us}
\item
  \href{https://www.nytco.com/careers/}{Work with us}
\item
  \href{https://nytmediakit.com/}{Advertise}
\item
  \href{http://www.tbrandstudio.com/}{T Brand Studio}
\item
  \href{https://www.nytimes.com/privacy/cookie-policy\#how-do-i-manage-trackers}{Your
  Ad Choices}
\item
  \href{https://www.nytimes.com/privacy}{Privacy}
\item
  \href{https://help.nytimes.com/hc/en-us/articles/115014893428-Terms-of-service}{Terms
  of Service}
\item
  \href{https://help.nytimes.com/hc/en-us/articles/115014893968-Terms-of-sale}{Terms
  of Sale}
\item
  \href{https://spiderbites.nytimes.com}{Site Map}
\item
  \href{https://help.nytimes.com/hc/en-us}{Help}
\item
  \href{https://www.nytimes.com/subscription?campaignId=37WXW}{Subscriptions}
\end{itemize}
