Sections

SEARCH

\protect\hyperlink{site-content}{Skip to
content}\protect\hyperlink{site-index}{Skip to site index}

\href{https://www.nytimes.com/section/politics}{Politics}

\href{https://myaccount.nytimes.com/auth/login?response_type=cookie\&client_id=vi}{}

\href{https://www.nytimes.com/section/todayspaper}{Today's Paper}

\href{/section/politics}{Politics}\textbar{}Republicans Divided Between
Romney and Giuliani for Secretary of State

\url{https://nyti.ms/2glH1wn}

\begin{itemize}
\item
\item
\item
\item
\item
\item
\end{itemize}

Advertisement

\protect\hyperlink{after-top}{Continue reading the main story}

Supported by

\protect\hyperlink{after-sponsor}{Continue reading the main story}

\hypertarget{republicans-divided-between-romney-and-giuliani-for-secretary-of-state}{%
\section{Republicans Divided Between Romney and Giuliani for Secretary
of
State}\label{republicans-divided-between-romney-and-giuliani-for-secretary-of-state}}

\includegraphics{https://static01.nyt.com/images/2016/11/25/us/25state1/25state1-articleLarge.jpg?quality=75\&auto=webp\&disable=upscale}

By \href{http://www.nytimes.com/by/jeremy-w-peters}{Jeremy W. Peters}
and \href{http://www.nytimes.com/by/maggie-haberman}{Maggie Haberman}

\begin{itemize}
\item
  Nov. 24, 2016
\item
  \begin{itemize}
  \item
  \item
  \item
  \item
  \item
  \item
  \end{itemize}
\end{itemize}

WASHINGTON --- Rival factions of Republicans are locked in an
increasingly caustic and public battle to influence President-elect
Donald J. Trump's choice for secretary of state, leaving a prominent
hole in an otherwise quickly formed national security team that is
unlikely to be filled until next week at the earliest.

The debate inside Mr. Trump's wide circle of formal and informal
advisers --- pitting supporters of one leading contender, Mitt Romney,
against those of another, Rudolph W. Giuliani --- has led to the kind of
dramatic airing of differences that characterized Mr. Trump's
unconventional and often squabbling campaign team.

And it traces the outlines of the enduring split in the Republican Party
between establishment figures who scoffed at Mr. Trump's chances of
victory and the grass-roots insurgents who backed him as a disrupter of
the Washington power structure.

The most publicly vocal faction has been the group opposed to Mr.
Romney, which has questioned whether he would be loyal after his searing
criticism of Mr. Trump during the campaign. But Mr. Trump himself has
told aides that he believes Mr. Romney ``looks the part'' and would make
a fine secretary of state, a senior Trump official said on Thursday. Mr.
Trump, who is always difficult to read and is capable of changing his
mind at any minute, has also praised Mr. Giuliani in recent
conversations with acquaintances.

Even Thanksgiving did not provide a reprieve from the extraordinary
public efforts to cast doubt on Mr. Romney. Mr. Trump's campaign
manager, Kellyanne Conway,
\href{https://twitter.com/KellyannePolls/status/801787470337740801}{said
on Twitter} that she had received ``a deluge'' of concern from people
warning against picking Mr. Romney, the 2012 Republican presidential
nominee and former Massachusetts governor.

Those raising concerns about Mr. Giuliani, the former New York City
mayor and an early and loyal supporter of Mr. Trump, have said they fear
that his tangle of foreign business ties could lead to a damaging
confirmation battle. They also worry that Mr. Giuliani lacks the vigor
for the globe-trotting job.

Both Mr. Romney and Mr. Giuliani have made their interest in the role
known to Mr. Trump. But while Mr. Giuliani has been very public about
his intentions --- angering Mr. Trump at times with his statements ---
Mr. Romney has been more reserved.

\includegraphics{https://static01.nyt.com/images/2016/11/25/us/25state2/25state2-articleLarge.jpg?quality=75\&auto=webp\&disable=upscale}

The split over the two men has opened the door for another candidate
altogether. One potential pick Mr. Trump and his team have entertained
is Gen. John F. Kelly of the Marines, a former head of the United States
Southern Command. Others are David H. Petraeus, the retired general and
former C.I.A. director, and Senator Bob Corker of Tennessee, according
to two people involved in the process.

Asked about Mr. Trump's deliberations, a spokesman, Jason Miller, said
in an email Thursday, ``The president-elect is meeting with a number of
well-qualified potential selections for this important position who
share his America First foreign policy --- some of whom have been made
public and others who have not --- and the president-elect will make
public his decision when he has finalized it.''

Mr. Romney would represent a departure from the hard-liners Mr. Trump
\href{http://www.nytimes.com/2016/11/19/us/politics/donald-trump-administration.html}{has
already picked for his national security team}. But aides like Stephen
K. Bannon, Mr. Trump's chief strategist, have expressed doubts about Mr.
Romney's loyalty given his
\href{http://www.nytimes.com/2016/03/04/us/politics/mitt-romney-speech.html}{denunciation}
of Mr. Trump as a ``phony'' and a ``fraud.'' Mr. Bannon and others have
told colleagues they fear that a State Department under Mr. Romney could
turn into something of a rogue agency.

Asked to explain her Twitter post about Mr. Romney, Ms. Conway said that
while she trusted Mr. Trump's judgment, she found it notable that the
most outrage directed at Mr. Trump from the party's grass-roots ``is not
against something he said, but something he may do.'' In
\href{https://twitter.com/KellyannePolls/status/801796408521138176}{another
post}, she said that being ``loyal'' was an important characteristic for
a secretary of state.

Others hoping to catch Mr. Trump's ear have taken their message to a
place they know he is likely to absorb it: cable news. Joe Scarborough,
the MSNBC host, who has spoken with Mr. Trump about his concerns that
Mr. Giuliani would not be confirmed by the Senate, has taken to making
those arguments on a daily basis on his morning show, which he knows Mr.
Trump watches.

Others, like Newt Gingrich, the former Republican House speaker, and
Mike Huckabee, the former governor of Arkansas, have gone on television
to try to dissuade Mr. Trump from picking Mr. Romney. Mr. Huckabee,
\href{http://latimesblogs.latimes.com/showtracker/2008/01/excerpts-from-p.html}{who
said} during the 2008 presidential campaign that Mr. Romney reminded
voters of ``the guy who laid them off,'' told Fox News on Wednesday that
picking Mr. Romney would be
``\href{http://insider.foxnews.com/2016/11/23/huckabee-choosing-romney-secy-state-would-be-real-insult-trump-voters}{a
real insult}'' to Mr. Trump's supporters. Mr. Giuliani is a favorite of
the Republican voters who turned out in large numbers to lift Mr. Trump
to victory.

Sean Hannity, a Fox News host whose opinion Mr. Trump often privately
solicits, has also been deeply critical of Mr. Romney on his show.

Shortly after the election, Mr. Giuliani told associates that he
believed the job was his. He had communicated to Mr. Trump's top
advisers that it was the only post he was interested in, according to
the people briefed on the discussions.

Image

Gen. John F. Kelly, the former head of the United States Southern
Command, has emerged as an alternative to Mr. Romney and Mr.
Giuliani.Credit...Mandel Ngan/Agence France-Presse --- Getty Images

But he began to run afoul of Mr. Trump when
\href{http://www.wsj.com/articles/rudy-giuliani-john-bolton-are-leading-candidates-for-next-secretary-of-state-1479156004}{he
told a Wall Street Journal forum} that he would probably be a better
candidate than John R. Bolton, who served as one of George W. Bush's
ambassadors to the United Nations.

And when reports surfaced about Mr. Giuliani's
\href{http://www.nytimes.com/2016/11/16/us/politics/donald-trump-cabinet-rudy-giuliani.html?_r=0}{foreign
business entanglements} and highly compensated speechmaking, Mr. Trump
grew even warier. His firm, Giuliani Partners, has had contracts with
the government of Qatar, and Mr. Giuliani has given paid speeches to a
shadowy Iranian opposition group that until 2012 was on the State
Department's list of foreign terrorist organizations.

As a backup plan, some of Mr. Trump's aides encouraged him to meet with
Mr. Romney. Though some in Mr. Trump's inner circle, like Reince
Priebus, his choice for chief of staff, thought that such a meeting
would anger the president-elect's supporters, Mr. Trump went ahead. In
the meantime, he started sounding out Mr. Giuliani on a different post,
director of national intelligence. Mr. Trump's advisers have discussed
the role for Mr. Giuliani, but there has been no indication he wants it.

What many people believed would be a perfunctory meeting with Mr. Romney
last weekend at Mr. Trump's golf club in Bedminster, N.J., turned into
something more substantial.

Mr. Trump liked Mr. Romney quite a bit, and was intrigued by the
possibility of such a camera-ready option to represent the country
around the globe, advisers to Mr. Trump said. The following day, Mr.
Giuliani met with Mr. Trump and urged him to make a decision in one
direction or the other.

Mr. Romney, who was mocked in 2012 when he described Russia as the
greatest geopolitical foe of the United States, has seen his stock in
the Republican Party rise since his loss to President Obama, although he
is still viewed skeptically by the party's grass-roots. His allies
believe that his position on Russia has been vindicated, but it is
starkly at odds with Mr. Trump's stated desire for a better relationship
with the Russian president, Vladimir V. Putin.

Privately, Mr. Giuliani has expressed his frustration at going from
front-runner for secretary of state to a contender who has to convince
Mr. Trump of his strengths. He is particularly irritated over the focus
on his business ties.

The option of a third person like General Kelly has gained currency in
recent days inside the transition team. A respected leader, General
Kelly served as the senior military assistant to former Defense
Secretary Leon E. Panetta. He led the Southern Command, responsible for
all United States military activities in South and Central America, for
four years under Mr. Obama. And his appointment would fit Mr. Trump's
\href{http://www.nytimes.com/2016/11/21/us/politics/donald-trump-national-security-military.html}{inclination
toward} putting people with combat experience in senior foreign policy
roles.

Advertisement

\protect\hyperlink{after-bottom}{Continue reading the main story}

\hypertarget{site-index}{%
\subsection{Site Index}\label{site-index}}

\hypertarget{site-information-navigation}{%
\subsection{Site Information
Navigation}\label{site-information-navigation}}

\begin{itemize}
\tightlist
\item
  \href{https://help.nytimes.com/hc/en-us/articles/115014792127-Copyright-notice}{©~2020~The
  New York Times Company}
\end{itemize}

\begin{itemize}
\tightlist
\item
  \href{https://www.nytco.com/}{NYTCo}
\item
  \href{https://help.nytimes.com/hc/en-us/articles/115015385887-Contact-Us}{Contact
  Us}
\item
  \href{https://www.nytco.com/careers/}{Work with us}
\item
  \href{https://nytmediakit.com/}{Advertise}
\item
  \href{http://www.tbrandstudio.com/}{T Brand Studio}
\item
  \href{https://www.nytimes.com/privacy/cookie-policy\#how-do-i-manage-trackers}{Your
  Ad Choices}
\item
  \href{https://www.nytimes.com/privacy}{Privacy}
\item
  \href{https://help.nytimes.com/hc/en-us/articles/115014893428-Terms-of-service}{Terms
  of Service}
\item
  \href{https://help.nytimes.com/hc/en-us/articles/115014893968-Terms-of-sale}{Terms
  of Sale}
\item
  \href{https://spiderbites.nytimes.com}{Site Map}
\item
  \href{https://help.nytimes.com/hc/en-us}{Help}
\item
  \href{https://www.nytimes.com/subscription?campaignId=37WXW}{Subscriptions}
\end{itemize}
