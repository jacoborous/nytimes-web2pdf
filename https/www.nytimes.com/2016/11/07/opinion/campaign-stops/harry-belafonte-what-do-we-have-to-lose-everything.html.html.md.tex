Sections

SEARCH

\protect\hyperlink{site-content}{Skip to
content}\protect\hyperlink{site-index}{Skip to site index}

\href{https://myaccount.nytimes.com/auth/login?response_type=cookie\&client_id=vi}{}

\href{https://www.nytimes.com/section/todayspaper}{Today's Paper}

\href{/section/opinion}{Opinion}\textbar{}Harry Belafonte: What Do We
Have to Lose? Everything

\url{https://nyti.ms/2ewyAgk}

\begin{itemize}
\item
\item
\item
\item
\item
\item
\end{itemize}

Advertisement

\protect\hyperlink{after-top}{Continue reading the main story}

Supported by

\protect\hyperlink{after-sponsor}{Continue reading the main story}

\href{/section/opinion}{Opinion}

Op-Ed Contributor

\hypertarget{harry-belafonte-what-do-we-have-to-lose-everything}{%
\section{Harry Belafonte: What Do We Have to Lose?
Everything}\label{harry-belafonte-what-do-we-have-to-lose-everything}}

By Harry Belafonte

\begin{itemize}
\item
  Nov. 7, 2016
\item
  \begin{itemize}
  \item
  \item
  \item
  \item
  \item
  \item
  \end{itemize}
\end{itemize}

\includegraphics{https://static01.nyt.com/images/2016/11/07/opinion/07belafonteWeb/07belafonteWeb-articleLarge.jpg?quality=75\&auto=webp\&disable=upscale}

\emph{``O, yes,}

\emph{I say it plain,}

\emph{America never was America to me,}

\emph{And yet I swear this oath ---}

\emph{America will be!''}

--- Langston Hughes, ``Let America Be America Again''

What old men know is that everything can change. Langston Hughes wrote
these lines when I was 8 years old, in the very different America of
1935.

It was an America where the life of a black person didn't count for
much. Where women were still second-class citizens, where Jews and other
ethnic whites were looked on with suspicion, and immigrants were kept
out almost completely unless they came from certain approved countries
in Northern Europe. Where gay people dared not speak the name of their
love, and where ``passing'' --- as white, as a WASP, as heterosexual, as
something, anything else that fit in with what America was supposed to
be --- was a commonplace, with all of the self-abasement and the shame
that entailed.

It was an America still ruled, at its base, by violence. Where
lynchings, and especially the threat of lynchings, were used to keep
minorities away from the ballot box and in their place. Where companies
amassed arsenals of weapons for goons to use against their own employees
and recruited the police and National Guardsmen to help them if these
private corporate armies proved insufficient. Where destitute veterans
of World War I were driven from the streets of Washington with tear gas
and bayonets, after they went to our nation's capital to ask for the
money they were owed.

Much of that was how America had always been. We changed it, many of us,
through some of the proudest struggles of our history. It wasn't easy,
and sometimes it wasn't pretty, but we did it, together. We won voting
rights for all. We ended Jim Crow, and we pushed open the Golden Door
again to welcome immigrants. We achieved full rights for women, and
fought to let people of all genders and sexual orientations stand in the
light. And if we have not yet created the America that Langston Hughes
swore will be --- ``The land that never has been yet'' --- if there is
still much to be done, at least we have advanced our standards of
humanity, hope and decency to places where many people never thought we
could reach.

What old men know, too, is that all that is gained can be lost. Lost
just as the liberation that the Civil War and Emancipation brought was
squandered after Reconstruction, by a white America grown morally weary,
or bent on revenge. Lost as the gains of our labor unions have been for
decades now, pushed back until so many of us stand alone in the
workplace, before unfettered corporate power. Lost as the vote is being
lost by legislative chicanery. Lost as so many powerful interests would
have us lose the benefits of the social welfare state, privatize Social
Security, and annihilate Obamacare altogether.

If he wins this Tuesday, Donald J. Trump would be, at 70, the oldest
president ever elected. But there is much about Mr. Trump that is always
young, and not in a good way. There is something permanently feckless
and immature in the man. It can be seen in how he mangles virtually the
same words that Langston Hughes used.

When Hughes writes, in the first two lines of his poem, ``Let America be
America again/ Let it be the dream it used to be,'' he acknowledges that
America is primarily a dream, a hope, an aspiration, that may never be
fully attainable, but that spurs us to be better, to be larger. He
follows this with the repeated counterpoint, ``America never was America
to me,'' and through the rest of this remarkable poem he alternates
between the oppressed and the wronged of America, and the great dreams
that they have for their country, that can never be extinguished.

Mr. Trump, who is not a poet, either in his late-night tweets or on the
speaker's stump, sees American greatness as some heavy, dead thing that
we must reacquire. Like a bar of gold, perhaps, or a bank vault, or one
of the lifeless, anonymous buildings he loves to put up. It is a
simplistic notion, reducing all the complexity of the American
experience to a vague greatness, and his prescription for the future is
just as undefined, a promise that we will return to ``winning'' without
ever spelling out what we will win --- save for the exclusion of
``others,'' the reduction of women to sexual tally points, the
re-closeting of so many of us.

With his simple, mean, boy's heart, Mr. Trump wants us to follow him
blind into a restoration that is not possible and could not be endured
if it were. Many of his followers acknowledge that (``He may get us all
killed'') but want to have someone in the White House who will really
``blow things up.''

What old men know is that things blown up --- customs, folkways, social
compacts, human bodies --- cannot so easily be put right. What Langston
Hughes so yearned for when he asked that America be America again was
the realization of an age-old people's struggle, not the vaporous
fantasies of a petty tyrant. Mr. Trump asks us what we have to lose, and
we must answer, only the dream, only everything.

Advertisement

\protect\hyperlink{after-bottom}{Continue reading the main story}

\hypertarget{site-index}{%
\subsection{Site Index}\label{site-index}}

\hypertarget{site-information-navigation}{%
\subsection{Site Information
Navigation}\label{site-information-navigation}}

\begin{itemize}
\tightlist
\item
  \href{https://help.nytimes.com/hc/en-us/articles/115014792127-Copyright-notice}{©~2020~The
  New York Times Company}
\end{itemize}

\begin{itemize}
\tightlist
\item
  \href{https://www.nytco.com/}{NYTCo}
\item
  \href{https://help.nytimes.com/hc/en-us/articles/115015385887-Contact-Us}{Contact
  Us}
\item
  \href{https://www.nytco.com/careers/}{Work with us}
\item
  \href{https://nytmediakit.com/}{Advertise}
\item
  \href{http://www.tbrandstudio.com/}{T Brand Studio}
\item
  \href{https://www.nytimes.com/privacy/cookie-policy\#how-do-i-manage-trackers}{Your
  Ad Choices}
\item
  \href{https://www.nytimes.com/privacy}{Privacy}
\item
  \href{https://help.nytimes.com/hc/en-us/articles/115014893428-Terms-of-service}{Terms
  of Service}
\item
  \href{https://help.nytimes.com/hc/en-us/articles/115014893968-Terms-of-sale}{Terms
  of Sale}
\item
  \href{https://spiderbites.nytimes.com}{Site Map}
\item
  \href{https://help.nytimes.com/hc/en-us}{Help}
\item
  \href{https://www.nytimes.com/subscription?campaignId=37WXW}{Subscriptions}
\end{itemize}
