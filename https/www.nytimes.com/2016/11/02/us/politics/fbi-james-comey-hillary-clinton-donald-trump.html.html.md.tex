Sections

SEARCH

\protect\hyperlink{site-content}{Skip to
content}\protect\hyperlink{site-index}{Skip to site index}

\href{https://www.nytimes.com/section/politics}{Politics}

\href{https://myaccount.nytimes.com/auth/login?response_type=cookie\&client_id=vi}{}

\href{https://www.nytimes.com/section/todayspaper}{Today's Paper}

\href{/section/politics}{Politics}\textbar{}F.B.I.'s Email Disclosure
Broke a Pattern Followed Even This Summer

\url{https://nyti.ms/2e0mnW2}

\begin{itemize}
\item
\item
\item
\item
\item
\item
\end{itemize}

Advertisement

\protect\hyperlink{after-top}{Continue reading the main story}

Supported by

\protect\hyperlink{after-sponsor}{Continue reading the main story}

\hypertarget{fbis-email-disclosure-broke-a-pattern-followed-even-this-summer}{%
\section{F.B.I.'s Email Disclosure Broke a Pattern Followed Even This
Summer}\label{fbis-email-disclosure-broke-a-pattern-followed-even-this-summer}}

\includegraphics{https://static01.nyt.com/images/2016/11/02/us/02fbi-1/29fbi-articleLarge.jpg?quality=75\&auto=webp\&disable=upscale}

By \href{http://www.nytimes.com/by/matt-apuzzo}{Matt Apuzzo},
\href{http://www.nytimes.com/by/michael-s-schmidt}{Michael S. Schmidt},
\href{https://www.nytimes.com/by/adam-goldman}{Adam Goldman} and
\href{http://www.nytimes.com/by/william-k-rashbaum}{William K. Rashbaum}

\begin{itemize}
\item
  Nov. 1, 2016
\item
  \begin{itemize}
  \item
  \item
  \item
  \item
  \item
  \item
  \end{itemize}
\end{itemize}

WASHINGTON --- The F.B.I. and Justice Department faced a hard decision
in two investigations this past summer that had the potential to rock
the presidential election. The first case involved Donald J. Trump's
campaign chairman, Paul Manafort, and secretive business dealings in
Ukraine. The second focused on Hillary Clinton's relationships with
donors to her family foundation.

At the urging of the Justice Department, the F.B.I. agreed not to issue
subpoenas or take other steps that would make the cases public so close
to the election, according to federal law enforcement officials.

Against this backdrop, the decision of the F.B.I. director, James B.
Comey, to send a letter to Congress last week about
\href{http://www.nytimes.com/2016/10/29/us/politics/fbi-hillary-clinton-email.html}{a
renewed inquiry} concerning Mrs. Clinton's emails is not just
\href{http://www.nytimes.com/2016/10/30/us/politics/comey-clinton-email-justice.html}{a
departure from longstanding policy}; it has plunged the F.B.I. and the
Justice Department directly into the election, precisely what Justice
officials were trying to avoid.

Mr. Comey's letter, which he sent over the objections of the Justice
Department, stirred outrage across party lines. It unleashed a torrent
of news that laid bare the government's
\href{http://www.wsj.com/articles/laptop-may-include-thousands-of-emails-linked-to-hillary-clintons-private-server-1477854957}{internal
deliberations} and exposed the infighting and occasional mistrust
between rank-and-file F.B.I. agents and senior department officials.

Since Mr. Comey's revelation, the F.B.I. has
\href{http://www.nytimes.com/2016/11/01/us/politics/hillary-clinton-huma-abedin-emails-fbi.html}{hurried
to analyze} a cache of emails belonging to one of Mrs. Clinton's aides,
Huma Abedin. It is increasingly unlikely that the review will be
complete by Election Day, F.B.I. officials said, although they said
there was a chance they could offer updates before Nov. 8.

\href{https://www.nytimes.com/interactive/2016/11/02/us/elections/James-Comey-options-classified-information-Hillary-Clinton-elections.html}{}

\includegraphics{https://static01.nyt.com/images/2016/11/02/us/elections/James-Comey-options-classified-information-Hillary-Clinton-elections-1478129558108/James-Comey-options-classified-information-Hillary-Clinton-elections-1478129558108-largeHorizontalJumbo.png}

\hypertarget{these-are-the-bad-and-worse-options-james-comey-faced}{%
\subsection{These Are the Bad (and Worse) Options James Comey
Faced}\label{these-are-the-bad-and-worse-options-james-comey-faced}}

When federal officials concluded their investigation into Hillary
Clinton's use of a private email server as secretary of state, the
F.B.I. director, James B. Comey, had a decision to make on how to
announce that news. The choices he made in July set the F.B.I. on the
path toward the predicament it faces today.

The mood at the F.B.I. is dark, and nobody is willing to predict what
the coming days will bring, particularly if agents and analysts do not
complete their review of Ms. Abedin's emails by Election Day. Officials
said it would take something extraordinary to change the conclusion that
nobody should be charged. But the absence of information has allowed
festering speculation that the emails must be significant.

Daniel C. Richman, an adviser to Mr. Comey and a Columbia University law
professor, argued that despite the backlash, Mr. Comey's decision to
inform Congress preserved the F.B.I.'s independence, which will
ultimately benefit the next president.

``Those arguing that the director should have remained silent until the
new emails could be reviewed --- even if that process lasted, or was
delayed, until after the election --- give too little thought to the
governing that needs to happen after November,'' Mr. Richman said. ``If
the F.B.I. director doesn't have the credibility to keep Congress from
interfering in the bureau's work and to assure Congress that a matter
has been or is being looked into, the new administration will pay a high
price.''

Former senior law enforcement officials in both parties, though, say Mr.
Comey's decision to break with Justice Department guidelines caused
these problems. Had he handled the case the way the F.B.I. handled its
investigations into the Clinton Foundation and Mr. Manafort over the
summer, the argument goes, he would have endured criticism from
Republicans in the future but would have preserved a larger principle
that has guided cases involving both parties.

In the Ukraine case, agents in Washington are investigating the
relationship between foreigners and Mr. Manafort, who was Mr. Trump's
campaign chairman from June until August. For a decade beginning in
2005, Mr. Manafort advised Ukrainian politicians, including Viktor F.
Yanukovych, who served as president from 2010 to 2014, when he
\href{http://www.nytimes.com/2014/02/23/world/europe/with-presidents-departure-ukraine-looks-toward-a-murky-future.html}{fled
the country} amid protests.

\includegraphics{https://static01.nyt.com/images/2016/11/02/us/02fbi-web3/02fbi-web3-articleLarge.jpg?quality=75\&auto=webp\&disable=upscale}

The cases involving Mr. Manafort and Mrs. Clinton were described by
federal law enforcement officials who spoke on the condition of
anonymity because they were not authorized to discuss open
investigations.

In an email, Mr. Manafort denied the existence of an F.B.I.
investigation into his business dealings and said that reports of one
were ``an outrageous smear being driven by'' Democrats who are trying to
distract attention from Ms. Abedin's emails. ``There is nothing of my
business activities to investigate,'' he said.

In August,
\href{http://www.nytimes.com/2016/08/15/us/politics/paul-manafort-ukraine-donald-trump.html}{The
New York Times reported} that anti-corruption investigators in Ukraine
had found handwritten ledgers showing \$12.7 million in undisclosed
payments to Mr. Manafort. The investigators, from Ukraine's newly formed
National Anti-Corruption Bureau, assert that the payments were part of
an illegal off-the-books system whose beneficiaries also included
elected officials.

The Times reported that other prosecutors in Ukraine were examining a
group of offshore shell companies that members of Mr. Yanukovych's inner
circle had used to fund lavish lifestyles, including a presidential
residence with a private zoo, golf course and tennis court. Those
prosecutors are looking at many transactions that involved Mr. Manafort,
including an \$18 million deal that sold Ukrainian cable television
assets to a partnership put together by Mr. Manafort and a Russian
oligarch, Oleg Deripaska, a close ally of President Vladimir V. Putin of
Russia.

Four days after The Times published the story about Mr. Manafort's
business dealings,
\href{http://www.nytimes.com/2016/08/20/us/politics/paul-manafort-resigns-donald-trump.html}{he
resigned} as Mr. Trump's campaign chairman. The F.B.I.'s criminal
investigation continues, though agents have followed the Justice
Department's guidance and have not taken overt steps in the case.

Image

Paul Manafort, then Donald J. Trump's campaign chairman, in July. The
F.B.I. investigated whether he was involved in secretive business
dealings in Ukraine.Credit...Sam Hodgson for The New York Times

In August, around the same time the decision was made to keep the
Manafort investigation at a low simmer, the F.B.I. grappled with whether
to issue subpoenas in the Clinton Foundation case, which, like the
Manafort matter, was in its preliminary stages. The investigation, based
in New York, had not developed much evidence and was based mostly on
information that had surfaced in news stories and the book ``Clinton
Cash,'' according to several law enforcement officials briefed on the
case.

The book asserted that foreign entities gave money to former President
Bill Clinton and the Clinton Foundation, and in return received favors
from the State Department when Mrs. Clinton was secretary of state. Mrs.
Clinton has adamantly denied those claims.

In meetings, the Justice Department and senior F.B.I. officials agreed
that making the Clinton Foundation investigation public could influence
the presidential race and suggest they were favoring Mr. Trump. But
waiting, they acknowledged, could open them up to criticism from
Republicans, who were demanding an investigation.

They agreed to keep the case open but wait until after the election to
determine their next steps. The move infuriated some agents, who thought
that the F.B.I.'s leaders were reining them in because of politics.

Mr. Comey's allies say he could not have simply followed this script
last week when he learned that agents had discovered new emails on a
laptop belonging to
\href{http://www.nytimes.com/2016/10/31/us/politics/anthony-weiner-democratic-reaction.html}{Ms.
Abedin's estranged husband}, Anthony D. Weiner. News of the search would
surely leak, he concluded, and it would appear that he had withheld the
information from Congress. He also thought that he alone was a trusted
voice on the Clinton case because of Attorney General Loretta E. Lynch's
\href{http://www.nytimes.com/2016/07/01/us/politics/meeting-between-bill-clinton-and-loretta-lynch-provokes-political-furor.html}{highly
criticized meeting} with Mr. Clinton in the original investigation's
final days, officials said.

But F.B.I. agents, many of whom are strongly supportive of Mr. Comey and
his approachable leadership style, have struggled to defend his
decision. They acknowledge that he was in a bind, but say the backlash
against the F.B.I. is unlike any in recent history.

Advertisement

\protect\hyperlink{after-bottom}{Continue reading the main story}

\hypertarget{site-index}{%
\subsection{Site Index}\label{site-index}}

\hypertarget{site-information-navigation}{%
\subsection{Site Information
Navigation}\label{site-information-navigation}}

\begin{itemize}
\tightlist
\item
  \href{https://help.nytimes.com/hc/en-us/articles/115014792127-Copyright-notice}{©~2020~The
  New York Times Company}
\end{itemize}

\begin{itemize}
\tightlist
\item
  \href{https://www.nytco.com/}{NYTCo}
\item
  \href{https://help.nytimes.com/hc/en-us/articles/115015385887-Contact-Us}{Contact
  Us}
\item
  \href{https://www.nytco.com/careers/}{Work with us}
\item
  \href{https://nytmediakit.com/}{Advertise}
\item
  \href{http://www.tbrandstudio.com/}{T Brand Studio}
\item
  \href{https://www.nytimes.com/privacy/cookie-policy\#how-do-i-manage-trackers}{Your
  Ad Choices}
\item
  \href{https://www.nytimes.com/privacy}{Privacy}
\item
  \href{https://help.nytimes.com/hc/en-us/articles/115014893428-Terms-of-service}{Terms
  of Service}
\item
  \href{https://help.nytimes.com/hc/en-us/articles/115014893968-Terms-of-sale}{Terms
  of Sale}
\item
  \href{https://spiderbites.nytimes.com}{Site Map}
\item
  \href{https://help.nytimes.com/hc/en-us}{Help}
\item
  \href{https://www.nytimes.com/subscription?campaignId=37WXW}{Subscriptions}
\end{itemize}
