Sections

SEARCH

\protect\hyperlink{site-content}{Skip to
content}\protect\hyperlink{site-index}{Skip to site index}

\href{https://www.nytimes.com/section/business}{Business}

\href{https://myaccount.nytimes.com/auth/login?response_type=cookie\&client_id=vi}{}

\href{https://www.nytimes.com/section/todayspaper}{Today's Paper}

\href{/section/business}{Business}\textbar{}Workers Pay More for Health
Care as Companies Shift Burden, Survey Finds

\url{https://nyti.ms/2cEIv6G}

\begin{itemize}
\item
\item
\item
\item
\item
\end{itemize}

Advertisement

\protect\hyperlink{after-top}{Continue reading the main story}

Supported by

\protect\hyperlink{after-sponsor}{Continue reading the main story}

\hypertarget{workers-pay-more-for-health-care-as-companies-shift-burden-survey-finds}{%
\section{Workers Pay More for Health Care as Companies Shift Burden,
Survey
Finds}\label{workers-pay-more-for-health-care-as-companies-shift-burden-survey-finds}}

By \href{http://www.nytimes.com/by/reed-abelson}{Reed Abelson}

\begin{itemize}
\item
  Sept. 14, 2016
\item
  \begin{itemize}
  \item
  \item
  \item
  \item
  \item
  \end{itemize}
\end{itemize}

State health insurance exchanges created under the new health care law
are in turmoil. By contrast, the employer market --- where the majority
of Americans still get their coverage --- seems like a bastion of
stability.

An \href{http://ehbs.kff.org/}{analysis} by the Kaiser Family Foundation
released on Wednesday shows that the share of employers offering
coverage remained steady this year, and that the cost of premiums for
health plans remained largely unchanged.

``We see historic moderation in premiums and health spending and
costs,'' said Drew Altman, the chief executive of the Kaiser foundation,
a nonprofit in Menlo Park, Calif., that closely tracks the health
insurance markets.

But underneath some of those figures, some important changes are
underway. The biggest shift is that workers continue to pay an
ever-greater share of their medical bills, a trend for several years
now. That is why Mr. Altman said that despite the overall moderation in
costs, ``it doesn't feel that way to average people.''

Kaiser's annual survey of employer health benefits represents a yearly
snapshot of the coverage companies offer their workers, and highlights
from the survey are being published online in Health Affairs, an
academic journal. About 150 million people are covered by an employer, a
much larger group than the 11 million or so who buy coverage on the
exchanges created under the federal health care law. On Tuesday, the
\href{http://www.nytimes.com/2016/09/14/business/economy/us-census-household-income-poverty-wealth-2015.html?hp\&action=click\&pgtype=Homepage\&clickSource=story-heading\&module=first-column-region\&region=top-news\&WT.nav=top-news\&_r=0}{Census
Bureau reported} that the percentage of uninsured Americans fell last
year, to 9.1 percent, in part because of the strength of the employer
market.

The latest survey helps shed some light on the quickly evolving
insurance industry. Here are a few highlights.

\hypertarget{slow-rise-in-premiums}{%
\subsection{Slow Rise in Premiums}\label{slow-rise-in-premiums}}

Annual family premiums rose an average of 3 percent, about in line with
the average increase in workers' wages, to \$18,142 a year, according to
Kaiser, which surveyed more than 1,900 employers.

This continued a significant slowdown in price increases. In the period
since 2011, the premiums for plans covering a family rose 20 percent,
compared with 31 percent from 2006 to 2011, and a 63 percent increase
from 2001 to 2006.

How long will this last? Mr. Altman says he is doubtful that the
reprieve from sharply rising costs is permanent and that he thinks rates
will go up again, especially if the economy heats up. ``I don't think
we're living in a new world,'' he said.

Exactly why the increases have been so modest in recent years is up for
debate. The Obama administration argues that some measures in the
federal health care law have helped slow the rise in health care costs.
Some other experts tend to point to the lingering effects of a sluggish
economy as well as the push by employers to shift more costs onto
workers.

\hypertarget{increasing-deductibles}{%
\subsection{Increasing Deductibles}\label{increasing-deductibles}}

While employer-sponsored plans typically have much lower deductibles
than the most popular plans found on the exchanges, more employees have
deductibles, and those deductibles are increasing.

Over all, employees have deductibles that are about 50 percent higher
than they were five years ago. Four out of five covered employees pay a
deductible, which averages about \$1,500 each, Kaiser found. Employees
who get insurance through a smaller company have deductibles that now
average \$2,100.

Workers are also paying a greater share of the premiums, contributing
\$5,277 annually toward a family plan, nearly a third of the total cost.

The move by employers and insurers to push more health costs onto
workers is significant, said Mr. Altman, who describes it as a stealth
move to ``skimpier insurance.'' Proponents of higher cost sharing say
that asking people to pay more of their medical bills causes them to be
more discriminating about what health care they use.

\hypertarget{networks-shrinking}{%
\subsection{Networks Shrinking}\label{networks-shrinking}}

Remember the days of being able to go to any doctor or hospital of your
choice? Many employers still choose plans that let workers visit a
doctor out of network, although employees are paying increasingly more
for the privilege.

But more companies are opting for less choice for their employees. This
year, slightly fewer than half of workers are enrolled in so-called
preferred provider organization plans, or P.P.O.s, compared with 58
percent in 2014. While you pay more when you go outside the plan's
network, you are still covered if you go to a doctor or hospital that
does not belong.

Employers started turning to these plans in the 1990s, when resistance
to health maintenance organizations, or H.M.O.s, grew. Employers and
insurers tend to favor more restrictive plans because they usually
contain costs better.

The tide may be reversing somewhat.
\href{http://www.nytimes.com/2016/02/29/business/trying-to-revive-hmos-but-without-those-scarlet-letters.html}{The
H.M.O. has been rethought} in new forms that are without some of the
drawbacks of the old-style health maintenance organizations, like
requiring people to get permission to go to a specialist.

As a result, some employers are dropping P.P.O.s in favor of the more
restrictive plans, Mr. Altman said. ``We're beginning to see that
reverse,'' he said.

The trend toward narrower networks is also seen in plans offered on the
exchanges, where the McKinsey Center for U.S. Health System Reform
recently estimated that two-thirds were H.M.O.s offering a sharply
limited choice of doctors and hospitals this year.

\hypertarget{employers-staying-put}{%
\subsection{Employers Staying Put}\label{employers-staying-put}}

The recent Kaiser survey also emphasizes the
\href{http://www.nytimes.com/2016/04/05/business/employers-keep-health-insurance-despite-affordable-care-act.html}{endurance
of the employer market}, despite earlier predictions that the health
care law would usher in its demise. Most companies are still offering
health benefits to their employees, with the percentage virtually
unchanged from last year at 56 percent.

**``**We're just not seeing a big impact on employers,'' Mr. Altman
said.

There is also little proof that companies are cutting workers' hours to
avoid the law's requirement that they offer full-time workers health
insurance. Employers with at least 50 full-time workers are required to
offer benefits or pay a penalty. More than 90 percent of companies with
at least 50 workers are offering coverage.

Advertisement

\protect\hyperlink{after-bottom}{Continue reading the main story}

\hypertarget{site-index}{%
\subsection{Site Index}\label{site-index}}

\hypertarget{site-information-navigation}{%
\subsection{Site Information
Navigation}\label{site-information-navigation}}

\begin{itemize}
\tightlist
\item
  \href{https://help.nytimes.com/hc/en-us/articles/115014792127-Copyright-notice}{©~2020~The
  New York Times Company}
\end{itemize}

\begin{itemize}
\tightlist
\item
  \href{https://www.nytco.com/}{NYTCo}
\item
  \href{https://help.nytimes.com/hc/en-us/articles/115015385887-Contact-Us}{Contact
  Us}
\item
  \href{https://www.nytco.com/careers/}{Work with us}
\item
  \href{https://nytmediakit.com/}{Advertise}
\item
  \href{http://www.tbrandstudio.com/}{T Brand Studio}
\item
  \href{https://www.nytimes.com/privacy/cookie-policy\#how-do-i-manage-trackers}{Your
  Ad Choices}
\item
  \href{https://www.nytimes.com/privacy}{Privacy}
\item
  \href{https://help.nytimes.com/hc/en-us/articles/115014893428-Terms-of-service}{Terms
  of Service}
\item
  \href{https://help.nytimes.com/hc/en-us/articles/115014893968-Terms-of-sale}{Terms
  of Sale}
\item
  \href{https://spiderbites.nytimes.com}{Site Map}
\item
  \href{https://help.nytimes.com/hc/en-us}{Help}
\item
  \href{https://www.nytimes.com/subscription?campaignId=37WXW}{Subscriptions}
\end{itemize}
