Sections

SEARCH

\protect\hyperlink{site-content}{Skip to
content}\protect\hyperlink{site-index}{Skip to site index}

\href{https://www.nytimes.com/section/politics}{Politics}

\href{https://myaccount.nytimes.com/auth/login?response_type=cookie\&client_id=vi}{}

\href{https://www.nytimes.com/section/todayspaper}{Today's Paper}

\href{/section/politics}{Politics}\textbar{}Michael Flynn Is Harsh Judge
of C.I.A.'s Role

\url{https://nyti.ms/2gAR7xS}

\begin{itemize}
\item
\item
\item
\item
\item
\end{itemize}

Advertisement

\protect\hyperlink{after-top}{Continue reading the main story}

Supported by

\protect\hyperlink{after-sponsor}{Continue reading the main story}

\hypertarget{michael-flynn-is-harsh-judge-of-cias-role}{%
\section{Michael Flynn Is Harsh Judge of C.I.A.'s
Role}\label{michael-flynn-is-harsh-judge-of-cias-role}}

\includegraphics{https://static01.nyt.com/images/2016/12/13/us/13flynn/13flynn-articleInline.jpg?quality=75\&auto=webp\&disable=upscale}

By \href{http://www.nytimes.com/by/matthew-rosenberg}{Matthew Rosenberg}

\begin{itemize}
\item
  Dec. 12, 2016
\item
  \begin{itemize}
  \item
  \item
  \item
  \item
  \item
  \end{itemize}
\end{itemize}

WASHINGTON --- Long before Lt. Gen. Michael T. Flynn became Donald J.
Trump's choice for
\href{https://www.nytimes.com/2016/11/18/us/politics/michael-flynn-national-security-adviser-donald-trump.html?_r=0}{national
security adviser}, he believed that the Central Intelligence Agency had
become a political tool of the Obama administration --- a view now
echoed by the president-elect in his mocking dismissals of C.I.A.
assessments that Russia sought to
\href{https://www.nytimes.com/2016/12/11/us/politics/cia-judgment-intelligence-russia-hacking-evidence.html}{tip
the election} in Mr. Trump's favor.

``They've lost sight of who they actually work for,'' Mr. Flynn said in
an interview with The New York Times in October 2015. ``They work for
the American people. They don't work for the president of the United
States.'' He added, speaking of the agency's leadership: ``Frankly, it's
become a very political organization.''

Mr. Flynn's assessment that the C.I.A. is a political arm of the Obama
administration is not widely shared by Republicans or Democrats in
Washington. But it has appeared to have been internalized by the one
person who matters most right now: Mr. Trump.

In the past few days, Mr. Trump has sought to portray reports of the
agency's assessments that Russia actively tried to interfere in the
election as a desperate attempt by sore losers to taint his presidency
before it begins. His denigration of C.I.A. officials as ``the same
people that said Saddam Hussein had weapons of mass destruction'' has
opened up an extraordinary rift between the president-elect and the
nation's intelligence community that is unlikely to be bridged anytime
soon.

Although it is unclear how much Mr. Flynn, 57, is responsible for Mr.
Trump's response to the C.I.A. assessment, during the presidential
campaign he had substantial influence on the president-elect. He brought
to the campaign views on Muslims and national security that tended to
hew far closer to the right-wing fringes than the mainstream of the
Republican Party.

Mr. Flynn also appears to have helped set the tone for Mr. Trump's testy
relationship with the intelligence community. In August, when the Trump
campaign received its first intelligence briefing, Mr. Flynn was so
combative with the briefers that another person in the room had to urge
him to settle down, according to a person familiar with the episode who
was told about it in confidence.

\href{https://www.nytimes.com/interactive/2016/us/politics/donald-trump-administration.html}{}

\includegraphics{https://static01.nyt.com/images/2016/11/11/us/politics/donald-trump-administration-1478905372015/donald-trump-administration-1478905372015-square640.jpg}

\hypertarget{donald-trumps-cabinet-is-complete-heres-the-full-list}{%
\subsection{Donald Trump's Cabinet Is Complete. Here's the Full
List.}\label{donald-trumps-cabinet-is-complete-heres-the-full-list}}

A list of appointees and nominees for top posts in the new
administration.

On any number of issues --- from the Obama administration's failure to
foresee the rise of the Islamic State to Mr. Flynn's ouster as chief of
the Defense Intelligence Agency, the intelligence arm of the Defense
Department --- he has made it clear in recent years that he sees the
political hand of the C.I.A. at work.

As director of the D.I.A. from 2012 to 2014, he pushed hard for his
agency, long treated as second-rate by the C.I.A., to be given greater
access to the trove of documents and other materials seized during the
raid that killed Osama bin Laden in Abbottabad, Pakistan, in May 2011.
The C.I.A. controlled the material, and Mr. Flynn became convinced that
the agency was refusing to share or declassify much of it because of
fears that it would undermine the administration's narrative about Al
Qaeda's waning strength at the time Bin Laden was killed.

``It's all political with'' the C.I.A. leadership, Mr. Flynn said in the
2015 interview, which focused on the rise of the Islamic State and
American national security.

``If they put out what we knew, then the president could have not said,
in a national election, `Al Qaeda's on the run and we've killed Bin
Laden,''' he said, referring to Mr. Obama's 2012 re-election campaign.
``Even today, he talks about Bin Laden as though that was a stroke of
genius.''

Mr. Flynn also questioned the decision to kill Bin Laden. ``Killing Bin
Laden was the wrong thing to do,'' he said. ``They could have captured
him.''

In killing Bin Laden, he said, ``we created a new version of Allah.''

``What we should have done is shown him to be a decrepit old guy, put
him in a freaking cage, in a cell, and put him on trial,'' Mr. Flynn
added. ``Make it a big messy trial, make it global.''

Mr. Flynn has also said that the C.I.A., at the urging of the White
House, was playing down warnings from the D.I.A. about the resurgence of
Al Qaeda in Iraq, which would later become the Islamic State. ``I'm
telling you, the C.I.A. has a lot to reflect on because of this,'' he
said.

A number of current and former officials dispute Mr. Flynn's account,
saying concerns about the resurgence of Islamist militants in the midst
of Syria's civil war were widespread in the intelligence community.

Mr. Flynn, who was fired from the D.I.A. after serving only two years of
a three-year appointment, has described his dismissal as an act of
political retribution by the C.I.A. and Obama administration officials
who did not want to hear what he was saying.

Other officials, including some with direct knowledge of the decision to
dismiss Mr. Flynn, said he was forced out for a more straightforward
reason: He was not a good manager, and his efforts to reform the agency
left it in chaos.

It was not Mr. Flynn's first run-in with the civilian intelligence
community. The ill will stretches back years, current and former
officials said, and it transformed into open hostility when Mr. Flynn
was running military intelligence under Gen. Stanley A. McChrystal in
Afghanistan.

In January 2010, after less than a year on the job, Mr. Flynn released a
paper,
``\href{https://www.cnas.org/publications/reports/fixing-intel-a-blueprint-for-making-intelligence-relevant-in-afghanistan}{Fixing
Intel},'' that was highly critical of American intelligence work in
Afghanistan. It bluntly stated that ``the U.S. intelligence community is
only marginally relevant to the overall strategy,'' and said that it had
only itself to blame because it had failed to understand Afghanistan's
cultural complexities.

The paper was widely praised in defense circles as insightful. But at
the C.I.A., officials were furious at what they saw as a direct attack
on the aptitude and professionalism of the roughly 1,000 agency
personnel who were serving in Afghanistan at the time.

They were also incensed at the timing of the paper, which became public
five days after a suicide attack that killed seven C.I.A. officers at a
base in eastern Afghanistan. Mr. Flynn's searing critique was seen at
the agency as the height of insensitivity.

Mr. Flynn has been unapologetic about his views of not only the C.I.A.
but other national security agencies, including the D.I.A. under his
leadership.

``They've really been lying to the American public,'' he said in the
interview, referring to the Obama administration and much of the
national security and intelligence establishment. ``The Department of
Defense and those of us that have allowed this sort of a happy talk ---
`We're moving in the right direction, things are working.' It's not. The
Taliban are going to come back into power, or ISIS is going to come back
into power.''

Advertisement

\protect\hyperlink{after-bottom}{Continue reading the main story}

\hypertarget{site-index}{%
\subsection{Site Index}\label{site-index}}

\hypertarget{site-information-navigation}{%
\subsection{Site Information
Navigation}\label{site-information-navigation}}

\begin{itemize}
\tightlist
\item
  \href{https://help.nytimes.com/hc/en-us/articles/115014792127-Copyright-notice}{©~2020~The
  New York Times Company}
\end{itemize}

\begin{itemize}
\tightlist
\item
  \href{https://www.nytco.com/}{NYTCo}
\item
  \href{https://help.nytimes.com/hc/en-us/articles/115015385887-Contact-Us}{Contact
  Us}
\item
  \href{https://www.nytco.com/careers/}{Work with us}
\item
  \href{https://nytmediakit.com/}{Advertise}
\item
  \href{http://www.tbrandstudio.com/}{T Brand Studio}
\item
  \href{https://www.nytimes.com/privacy/cookie-policy\#how-do-i-manage-trackers}{Your
  Ad Choices}
\item
  \href{https://www.nytimes.com/privacy}{Privacy}
\item
  \href{https://help.nytimes.com/hc/en-us/articles/115014893428-Terms-of-service}{Terms
  of Service}
\item
  \href{https://help.nytimes.com/hc/en-us/articles/115014893968-Terms-of-sale}{Terms
  of Sale}
\item
  \href{https://spiderbites.nytimes.com}{Site Map}
\item
  \href{https://help.nytimes.com/hc/en-us}{Help}
\item
  \href{https://www.nytimes.com/subscription?campaignId=37WXW}{Subscriptions}
\end{itemize}
