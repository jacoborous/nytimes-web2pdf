Sections

SEARCH

\protect\hyperlink{site-content}{Skip to
content}\protect\hyperlink{site-index}{Skip to site index}

\href{https://www.nytimes.com/section/business/economy}{Economy}

\href{https://myaccount.nytimes.com/auth/login?response_type=cookie\&client_id=vi}{}

\href{https://www.nytimes.com/section/todayspaper}{Today's Paper}

\href{/section/business/economy}{Economy}\textbar{}Fed Raises Key
Interest Rate, Citing Strengthening Economy

\url{https://nyti.ms/2hFTGPt}

\begin{itemize}
\item
\item
\item
\item
\item
\end{itemize}

Advertisement

\protect\hyperlink{after-top}{Continue reading the main story}

Supported by

\protect\hyperlink{after-sponsor}{Continue reading the main story}

\hypertarget{fed-raises-key-interest-rate-citing-strengthening-economy}{%
\section{Fed Raises Key Interest Rate, Citing Strengthening
Economy}\label{fed-raises-key-interest-rate-citing-strengthening-economy}}

\includegraphics{https://static01.nyt.com/images/2016/12/15/business/15FED/15FED-videoSixteenByNineJumbo1600.jpg}

By \href{http://www.nytimes.com/by/binyamin-appelbaum}{Binyamin
Appelbaum}

\begin{itemize}
\item
  Dec. 14, 2016
\item
  \begin{itemize}
  \item
  \item
  \item
  \item
  \item
  \end{itemize}
\end{itemize}

WASHINGTON --- The Federal Reserve raised its benchmark interest rate
Wednesday for just the second time since the financial crisis of 2008,
saying the American economy is expanding at a healthy pace and setting
itself up as a counterweight to President-elect Donald J. Trump's push
for considerably faster growth.

The Fed cited the steady growth of employment and other economic
measures, and signaled that it expects to raise rates more quickly next
year to prevent the economy from growing too quickly.

``My colleagues and I are recognizing the considerable progress the
economy has made,'' Janet L. Yellen, the Fed's chairwoman, said at a
news conference after the announcement. ``We expect the economy will
continue to perform well.''

The widely expected decision moves the Fed's benchmark rate to a range
of 0.5 percent to 0.75 percent, still very low by historical standards.
Low rates support economic growth by encouraging borrowing and
risk-taking.

The American economy has expanded by about 2 percent a year over the
last six years, and the unemployment rate has fallen to 4.6 percent. The
Fed's assessment that the economy is growing at a healthy pace --- not
too hot, not too cold --- is starkly at odds with Mr. Trump, who has
promised 4 percent growth and has described job creation as ``terrible''
and economic growth as anemic.

Already on Wednesday, one Republican member of the House Financial
Services Committee, Representative Roger Williams of Texas, criticized
the Fed's move.

``Today's decision by the Fed to raise the interest rate is entirely
premature and will be burdensome to a nation already struggling to pull
itself out of this slow-growth Obama economy,'' Mr. Williams said in a
statement. ``By making rates even higher, the Fed is effectively making
our hardships even harder.''

Mr. Williams did not object when the Fed raised rates last December.

In
\href{https://www.federalreserve.gov/monetarypolicy/files/monetary20161214a1.pdf}{announcing
the decision} after a two-day meeting of the Fed's policy-making
committee, the central bank gave little indication that Mr. Trump's
election had altered its economic outlook. The Fed said it still
expected a slow economic expansion and a steady march toward higher
rates. In separate forecasts also published Wednesday, Fed officials
predicted three rate increases in 2017.

For the first time in recent years, however, there is a real possibility
of significant changes in fiscal policy. Republicans will control the
White House and both chambers of Congress, and Mr. Trump has promised to
increase economic growth and job creation through tax cuts and
infrastructure spending.

Those measures could spur faster growth after a presidential campaign in
which Mr. Trump regularly disparaged the economy's performance under
President Obama. But the Fed reiterated Wednesday that the economy is
already expanding at roughly the maximum sustainable pace.

Fed officials also see evidence that the labor market is tightening.
Several Fed districts reported labor shortages in the central bank's
\href{https://www.federalreserve.gov/monetarypolicy/beigebook/beigebook201611.htm}{most
recent compilation} of economic reports. In the Philadelphia district,
construction workers are hard to find. Atlanta reported a shortage of
nurses; Kansas City, truck drivers; Dallas, tech workers.

Faster growth, in the Fed's judgment, would probably lead to higher
inflation. As a result, if Republicans succeed in invigorating growth,
the Fed
\href{http://www.nytimes.com/2016/12/13/business/economy/federal-reserve-interest-rates.html}{is
likely to raise rates more quickly}. The greater the stimulus, the
faster interest rates are likely to rise.

``Your expectation should depend very little on what you think that the
F.O.M.C. is thinking and very much on your view of Trump policies and
their macro effects,'' said Jon Faust, a professor of economics at Johns
Hopkins University and a former adviser to Ms. Yellen, referring to the
Federal Open Market Committee. ``Don't focus on the Fed. As James
Carville regularly reminded the other Clinton on the campaign trail:
It's the economy, stupid.''

Ms. Yellen emphasized that the Fed was not prejudging the likely course
of events. She declined several times to comment on the merits of Mr.
Trump's plans or to predict their consequences for the economy.

``We're operating under a cloud of uncertainty at the moment,'' Ms.
Yellen said.

Fed officials predicted that they would raise the Fed's benchmark rate a
little more quickly in the coming years, reaching 2.1 percent by the end
of 2018. In September,
\href{https://www.federalreserve.gov/monetarypolicy/files/fomcprojtabl20160921.pdf}{they
had predicted} that it would reach 1.9 percent by the end of 2018. The
new projections, however, reflect a significantly slower pace of
increase than last December, when they expected the rate to reach
\href{https://www.federalreserve.gov/monetarypolicy/files/fomcprojtabl20151216.pdf}{3.3
percent by 2018}.

The combination of steady growth and faster rate increases indicates
that some Fed officials expect the central bank to end up offsetting a
modest increase in fiscal stimulus. But Ms. Yellen said most Fed
officials were reserving judgment.

``Changes in fiscal policy or other economic policies could affect the
economic outlook,'' she said. ``Of course, it is far too early to know
how those changes will unfold.''

\href{https://www.nytimes.com/interactive/2015/12/16/upshot/fed-interest-rates-rube-goldberg-machine.html}{}

\includegraphics{https://static01.nyt.com/images/2015/12/16/upshot/fed-interest-rates-rube-goldberg-machine-1450212792328/fed-interest-rates-rube-goldberg-machine-1450212792328-videoLarge.jpg}

\hypertarget{what-happens-when-the-fed-raises-rates-in-one-rube-goldberg-machine}{%
\subsection{What Happens When the Fed Raises Rates, in One Rube Goldberg
Machine}\label{what-happens-when-the-fed-raises-rates-in-one-rube-goldberg-machine}}

Exactly seven years ago, the Federal Reserve cut interest rates to
almost zero in order to nurse the ailing economy back to health.
Recently it changed direction. This is how it works.

The tensions between monetary and fiscal policy will develop slowly.
Legislation takes time to write, and any economic impact would generally
be felt in coming years. Political pressures, however, may build more
quickly.

Mr. Trump has made clear in the past that he likes low interest rates
--- and some of his plans, like infrastructure investment, will be much
easier to fund if rates remain low.

``The Fed is in a tricky place,'' said Michael Feroli, chief United
States economist at JPMorgan Chase. ``They're trying not to prejudge how
Congress and the administration duke it out, but once they see that, I
think they will respond.''

There is also uncertainty about the Fed's leadership. Ms. Yellen's term
as chairwoman ends in February 2018, and Mr. Trump has said he would
prefer a Republican.

Ms. Yellen could remain on the board, a possibility she said Wednesday
she had not ruled out. But the Fed, under different leadership, might
well choose a different path forward. Some conservative economists,
notably John Taylor of Stanford University, argue that the bank should
already have raised rates above 1 percent.

The economy, for now, keeps plodding along. Steady job growth has
reduced the unemployment rate to a level the Fed considers healthy. A
little unemployment is natural as people change jobs and businesses
close. Ms. Yellen and other Fed officials have said they see some signs
of stronger wage growth. Inflation, too, has picked up a little in
recent months, although both wages and inflation continue to rise more
slowly than the Fed would like to see.

Ms. Yellen described the rate increase as ``a vote of confidence in the
economy.''

The decision was made by a unanimous vote of the 10 members of the
Federal Open Market Committee, the first time in recent months the Fed
has acted by consensus.

Some economists argue that the Fed should wait until inflation
strengthens before raising rates, to test whether a stronger economy
would persuade some people sidelined during the downturn to start
looking for jobs. That would expand the labor force. Unemployment
remains particularly high among minorities.

That view, however, has found little support among Fed officials, who
worry that interest rates will have to be raised more quickly if they
wait too long, increasing the chances of pushing the economy into
recession.

``Apparently, Fed officials think the economy is growing too quickly,''
said Ady Barkan, the director of Fed Up, a coalition of liberal groups
that has pressed the Fed to continue its stimulus campaign. ``I doubt
you can find many other Americans who share that opinion. And it's a
strange conclusion to draw in the wake of an election that was so
heavily impacted by voters' economic discontent.''

Advertisement

\protect\hyperlink{after-bottom}{Continue reading the main story}

\hypertarget{site-index}{%
\subsection{Site Index}\label{site-index}}

\hypertarget{site-information-navigation}{%
\subsection{Site Information
Navigation}\label{site-information-navigation}}

\begin{itemize}
\tightlist
\item
  \href{https://help.nytimes.com/hc/en-us/articles/115014792127-Copyright-notice}{©~2020~The
  New York Times Company}
\end{itemize}

\begin{itemize}
\tightlist
\item
  \href{https://www.nytco.com/}{NYTCo}
\item
  \href{https://help.nytimes.com/hc/en-us/articles/115015385887-Contact-Us}{Contact
  Us}
\item
  \href{https://www.nytco.com/careers/}{Work with us}
\item
  \href{https://nytmediakit.com/}{Advertise}
\item
  \href{http://www.tbrandstudio.com/}{T Brand Studio}
\item
  \href{https://www.nytimes.com/privacy/cookie-policy\#how-do-i-manage-trackers}{Your
  Ad Choices}
\item
  \href{https://www.nytimes.com/privacy}{Privacy}
\item
  \href{https://help.nytimes.com/hc/en-us/articles/115014893428-Terms-of-service}{Terms
  of Service}
\item
  \href{https://help.nytimes.com/hc/en-us/articles/115014893968-Terms-of-sale}{Terms
  of Sale}
\item
  \href{https://spiderbites.nytimes.com}{Site Map}
\item
  \href{https://help.nytimes.com/hc/en-us}{Help}
\item
  \href{https://www.nytimes.com/subscription?campaignId=37WXW}{Subscriptions}
\end{itemize}
