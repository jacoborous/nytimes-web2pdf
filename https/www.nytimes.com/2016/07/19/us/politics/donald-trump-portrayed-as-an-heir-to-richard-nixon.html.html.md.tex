Sections

SEARCH

\protect\hyperlink{site-content}{Skip to
content}\protect\hyperlink{site-index}{Skip to site index}

\href{https://www.nytimes.com/section/politics}{Politics}

\href{https://myaccount.nytimes.com/auth/login?response_type=cookie\&client_id=vi}{}

\href{https://www.nytimes.com/section/todayspaper}{Today's Paper}

\href{/section/politics}{Politics}\textbar{}It's Donald Trump's
Convention. But the Inspiration? Nixon.

\url{https://nyti.ms/2a4GXjZ}

\begin{itemize}
\item
\item
\item
\item
\item
\end{itemize}

Advertisement

\protect\hyperlink{after-top}{Continue reading the main story}

Supported by

\protect\hyperlink{after-sponsor}{Continue reading the main story}

News Analysis

\hypertarget{its-donald-trumps-convention-but-the-inspiration-nixon}{%
\section{It's Donald Trump's Convention. But the Inspiration?
Nixon.}\label{its-donald-trumps-convention-but-the-inspiration-nixon}}

\includegraphics{https://static01.nyt.com/images/2016/07/19/us/19assess-jp1/19assess-jp1-videoSixteenByNineJumbo1600-v2.jpg}

By \href{http://www.nytimes.com/by/michael-barbaro}{Michael Barbaro} and
\href{http://www.nytimes.com/by/alexander-burns}{Alexander Burns}

\begin{itemize}
\item
  July 18, 2016
\item
  \begin{itemize}
  \item
  \item
  \item
  \item
  \item
  \end{itemize}
\end{itemize}

CLEVELAND --- Let Trump be Trump, his aides have always insisted. And
let his convention serve as an unapologetic tribute to his singular,
erratic, untamed persona.

``I want,'' the candidate has often said, ``to be myself.''

But on the opening night of the
\href{http://www.nytimes.com/2016/07/20/us/politics/republican-national-convention.html}{Republican
National Convention} here on Monday, Donald J. Trump was conspicuously
trying to conjure somebody else: Richard M. Nixon.

In an evening of severe speeches evoking the tone and themes of Nixon's
successful 1968 campaign, Mr. Trump's allies and aides proudly portrayed
him as the heir to the disgraced former president's law-and-order
message, his mastery of political self-reinvention and his rebukes of
overreaching liberal government.

It was a remarkable embrace --- open and unhesitating --- of Nixon's
polarizing campaign tactics, and of his overt appeals to Americans
frightened by a chaotic stew of war, mass protests and racial unrest.

And it demonstrated that, wisely or mistakenly, Mr. Trump sees the path
to victory this fall as the exploitation of the country's anxieties
about race, its fears of terrorism and its mood of disaffection,
especially among white, working-class Americans.

In a startling disclosure on the first day of the convention, Mr.
Trump's campaign chairman, Paul Manafort, declared that the candidate
was using, as the template for his own prime-time speech accepting the
Republican nomination, Nixon's convention address 48 years ago in Miami
Beach. ``If you go back and read,'' Mr. Manafort said at a Bloomberg
News breakfast, ``that speech is pretty much on line with a lot of the
issues that are going on today.''

\includegraphics{https://static01.nyt.com/images/2016/07/18/us/1968-chicago-protest/1968-chicago-protest-videoSixteenByNine3000.jpg}

Mr. Trump himself, in an interview, drew explicit comparisons between
his candidacy and Nixon's, and between the current political climate and
that of the United States in 1968.

``I think what Nixon understood is that when the world is falling apart,
people want a strong leader whose highest priority is protecting America
first,'' Mr. Trump said recently. ``The '60s were bad, really bad. And
it's really bad now. Americans feel like it's chaos again.''

The inaugural night of the convention deliberately evoked social
cataclysm and physical danger.

``The vast majority of Americans today do not feel safe,'' Rudolph W.
Giuliani, the former mayor of New York, told the audience. ``They fear
for their children. They fear for themselves.'' Retired Lt. Gen. Michael
Flynn warned, ``Our way of life is in jeopardy.'' And Representative
Michael McCaul of Texas, the chairman of the House Homeland Security
Committee, called on Americans to ``take back our country and make
America safe again.''

In emulating Nixon, Mr. Trump has chosen an unusual and tarnished figure
as a source of inspiration.

Nixon sought the presidency under starkly different circumstances, as a
conventional politician in a country that was nearly 90 percent white.
It will be difficult for Mr. Trump to recreate the Nixon candidacy as a
political insurgent in a year when 30 percent of voters are likely to be
racial minorities.

And by clinging to a dark chapter in Republican history, Mr. Trump has
pained the party's leaders. House Speaker Paul D. Ryan on Monday sternly
advised Mr. Trump to avoid the politics of race and identity. ``The fear
I have is the right is now starting to practice it,'' Mr. Ryan said.
``We have to get rid of that. That is going to divide this country.''

Yet for advisers to Mr. Trump, who is seen by most voters as a divisive
and untrustworthy figure, the Nixon campaign seems to offer a plausible
political blueprint.

\href{https://www.nytimes.com/interactive/2016/07/18/us/elections/gop-conventions-speakers.html}{}

\includegraphics{https://static01.nyt.com/images/2016/07/19/us/19livepromo/19livepromo-videoLarge.jpg}

\hypertarget{republican-convention-day-1-analysis}{%
\subsection{Republican Convention Day 1:
Analysis}\label{republican-convention-day-1-analysis}}

Times journalists provided live analysis of the first night of the
Republican National Convention as Donald J. Trump aims to unify the
party.

Viewed for much of the 1960s as a devious and even nasty politician,
Nixon reintroduced himself to the country during the 1968 campaign as
earnest and sympathetic, and offered himself as a bulwark against forces
tearing at the seams of society.

Mr. Manafort said in an interview that the Nixon campaign showed it was
possible to ``get people to see you in a different light'' as a ``strong
leader but also a human one.''

Comparisons between the toxic political brew of 1968 --- racial discord,
rising crime, street demonstrations, white anxiety --- and the strains
in United States society today are frequently exaggerated and
oversimplified. The backdrop of Nixon's election was a nation absorbing
the seismic upheavals of the Vietnam War, the Voting Rights Act, the
assassinations of the Rev. Dr. Martin Luther King Jr. and Robert F.
Kennedy, and widespread rioting in America's cities. ``We see Americans
hating each other, fighting each other, killing each other at home,''
Nixon said in his acceptance speech.

Today, the collision of a campaign and social turmoil appears, for now,
nowhere near as combustible. But the Trump campaign has worked to seize
on the parallels that do exist: the killings of police officers in Texas
and Louisiana; growing street demonstrations by supporters of Black
Lives Matter; and scenes of mass bloodshed overseas.

In Mr. Trump's telling, American neighborhoods are besieged by crime,
with entire cities rocked by disorder. Branding himself as ``the
law-and-order candidate,'' Mr. Trump in recent weeks has exhorted voters
to stand with the police, much as Nixon encouraged ``the non-shouters,
the non-demonstrators'' to stand with him.

What started, a year ago, as an occasional borrowing of Nixonian phrases
--- like describing Trump voters as a ``silent majority'' --- has turned
into a homage. ``It's literally plagiarism,'' said Kevin Mattson, a
professor of history at Ohio University who has written a book about
Nixon. ``I was taken aback by that.''

Mr. Trump has regularly embellished the facts to cast the 2016 campaign
in an ominous light, with imagery more suited to the Nixon era than to a
modern presidential race. He has described crime as having gone
``through the roof,'' even though the rate of violent crime has dropped
by half since 1990.

\href{https://www.nytimes.com/slideshow/2016/07/18/us/politics/the-first-day-of-the-republican-national-convention.html}{}

\hypertarget{the-first-day-of-the-republican-national-convention}{%
\subsection{The First Day of the Republican National
Convention}\label{the-first-day-of-the-republican-national-convention}}

14 Photos

View Slide Show ›

\includegraphics{https://static01.nyt.com/images/2016/07/18/us/politics/20160718convention-mobile-slide-D8GW/20160718convention-mobile-slide-D8GW-articleLarge-v2.jpg?quality=75\&auto=webp\&disable=upscale}

Stephen Crowley/The New York Times

Jeff Greenfield, a political columnist who was a campaign aide to Robert
F. Kennedy in 1968, called it ludicrous to liken the current political
atmosphere to that election. But Mr. Trump, he said, was tapping into a
sense of wariness that for many voters has begun to override the more
benign realities of 2016.

``The country seems to be in a pretty unhappy mood, so even if law and
order may not be the direct answer, there's a sense of conflation,'' Mr.
Greenfield said. ``The loss of jobs, the sense of cultural unease or
upheaval, the sense that things are falling apart in some way.''

But Mr. Trump is at risk of misreading Nixon, whose calls to crack down
on crime were intertwined with themes of national unity and can-do
optimism.

Though his convention speech dwelled on gloomy themes, Nixon did
something else, too: He spoke poignantly about his dreams as a boy
growing up poor in California, and he appealed to highly educated
suburban voters who have eyed Mr. Trump with deep suspicion.

Edward F. Cox, Nixon's son-in-law and the chairman of New York's
Republican Party, said the Nixon-Agnew campaign of 1968 had been
caricatured as an angry and divisive affair, with the gentler notes of
the acceptance speech largely forgotten.

Mr. Cox, who attended Nixon's speech and was in Cleveland this week to
support Mr. Trump, said criminal justice would be a powerful theme in
2016. But he urged the presumptive Republican nominee to incorporate
stronger notes of uplift, too.

``You need to have that optimism in this speech,'' Mr. Cox said.
``History doesn't repeat itself, but it's certainly rhyming, and it's a
softer rhyme here.''

Advertisement

\protect\hyperlink{after-bottom}{Continue reading the main story}

\hypertarget{site-index}{%
\subsection{Site Index}\label{site-index}}

\hypertarget{site-information-navigation}{%
\subsection{Site Information
Navigation}\label{site-information-navigation}}

\begin{itemize}
\tightlist
\item
  \href{https://help.nytimes.com/hc/en-us/articles/115014792127-Copyright-notice}{©~2020~The
  New York Times Company}
\end{itemize}

\begin{itemize}
\tightlist
\item
  \href{https://www.nytco.com/}{NYTCo}
\item
  \href{https://help.nytimes.com/hc/en-us/articles/115015385887-Contact-Us}{Contact
  Us}
\item
  \href{https://www.nytco.com/careers/}{Work with us}
\item
  \href{https://nytmediakit.com/}{Advertise}
\item
  \href{http://www.tbrandstudio.com/}{T Brand Studio}
\item
  \href{https://www.nytimes.com/privacy/cookie-policy\#how-do-i-manage-trackers}{Your
  Ad Choices}
\item
  \href{https://www.nytimes.com/privacy}{Privacy}
\item
  \href{https://help.nytimes.com/hc/en-us/articles/115014893428-Terms-of-service}{Terms
  of Service}
\item
  \href{https://help.nytimes.com/hc/en-us/articles/115014893968-Terms-of-sale}{Terms
  of Sale}
\item
  \href{https://spiderbites.nytimes.com}{Site Map}
\item
  \href{https://help.nytimes.com/hc/en-us}{Help}
\item
  \href{https://www.nytimes.com/subscription?campaignId=37WXW}{Subscriptions}
\end{itemize}
