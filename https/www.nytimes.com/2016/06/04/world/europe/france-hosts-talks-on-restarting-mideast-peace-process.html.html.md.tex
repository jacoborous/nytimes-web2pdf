Sections

SEARCH

\protect\hyperlink{site-content}{Skip to
content}\protect\hyperlink{site-index}{Skip to site index}

\href{https://www.nytimes.com/section/world/europe}{Europe}

\href{https://myaccount.nytimes.com/auth/login?response_type=cookie\&client_id=vi}{}

\href{https://www.nytimes.com/section/todayspaper}{Today's Paper}

\href{/section/world/europe}{Europe}\textbar{}Trying to Fill Void,
France Holds Talks on Mideast Peace Process

\url{https://nyti.ms/22DrWc6}

\begin{itemize}
\item
\item
\item
\item
\item
\end{itemize}

Advertisement

\protect\hyperlink{after-top}{Continue reading the main story}

Supported by

\protect\hyperlink{after-sponsor}{Continue reading the main story}

\hypertarget{trying-to-fill-void-france-holds-talks-on-mideast-peace-process}{%
\section{Trying to Fill Void, France Holds Talks on Mideast Peace
Process}\label{trying-to-fill-void-france-holds-talks-on-mideast-peace-process}}

\includegraphics{https://static01.nyt.com/images/2016/06/04/world/MIDEAST/MIDEAST-articleLarge.jpg?quality=75\&auto=webp\&disable=upscale}

By \href{https://www.nytimes.com/by/aurelien-breeden}{Aurelien Breeden}

\begin{itemize}
\item
  June 3, 2016
\item
  \begin{itemize}
  \item
  \item
  \item
  \item
  \item
  \end{itemize}
\end{itemize}

PARIS --- With the Obama administration having effectively given up on
negotiating a deal between the Israelis and the Palestinians, France
tried its hand on Friday at making Middle East peace but ended the day
with little to show for its efforts.

Even before President François Hollande convened diplomats from 29
countries for the session --- including Secretary of State John Kerry
but no representatives of either the Israelis or the Palestinians ---
France had backed off its initial hope that it could produce progress
where United States-led efforts had not.

Still, the meeting underscored how the tense relationship between
President Obama and Prime Minister Benjamin Netanyahu of Israel, weak
leadership among the Palestinians and the array of other conflicts in
the region have combined to create a diplomatic void at a time when
Europe's support for Israel has shown some cracks.

``The French are taking advantage of the vacuum left by the United
States,'' Frédérique Schillo, a French historian who specializes in
Israel and international relations, said in a telephone interview from
Jerusalem.

``They have duly noted the failure of Kerry's mission,'' Ms. Schillo
said, referring to nine months of talks spearheaded by Mr. Kerry that
\href{http://nyti.ms/1izHQ2L}{collapsed in 2014}.

``And they have also noted a certain disengagement of the Americans in
the Middle East,'' she added, including in Israel, but also on other
issues like the conflict in Syria, where the United States backed out of
conducting airstrikes in 2013,
\href{http://www.nytimes.com/2016/02/23/world/europe/laurent-fabius-obama-syria-war.html?_r=0}{leaving
the French bitter}.

Since then, however, the United States has been by far the most active
country in the fight against President Bashar al-Assad of Syria.

Mr. Hollande acknowledged the complexity of the peace talks in his
opening remarks on Friday.

``We are no longer in the situation of 1993, with the
\href{http://www.nytimes.com/2015/10/01/world/middleeast/palestinians-mahmoud-abbas-oslo-peace-accords.html}{Oslo
accords}, or of 2002, with the
\href{http://www.nytimes.com/2002/03/27/world/mideast-turmoil-arabs-beirut-arab-officials-vow-move-saudi-peace-plan.html}{Arab
peace initiative}. We aren't in the situation of 2007, with the big
international conference in
\href{http://www.nytimes.com/2007/11/25/world/middleeast/25annapolis.html}{Annapolis},''
Mr. Hollande said. ``We are in 2016, with the war in Syria, with the war
in Iraq, with terrorism and fundamentalism.''

In the two years since Mr. Kerry's efforts to negotiate a deal between
the Israelis and the Palestinians broke down, tensions between the two
parties have simmered and flared repeatedly, prompting France to propose
a new process.

But in the months since revealing its initiative in January, France had
already tempered its ambitions and its policy positions amid a change in
its foreign policy leadership. The French foreign minister at the time
of the initial proposal,
\href{http://www.nytimes.com/2016/02/11/world/europe/laurent-fabius-france-resignation.html}{Laurent
Fabius}, had said in January that France would
\href{http://www.nytimes.com/2016/01/30/world/middleeast/france-plans-mideast-peace-effort-and-recognition-of-palestine-if-it-fails.html}{unilaterally
recognize Palestine} as an independent state if the effort failed. Mr.
Fabius has since been replaced by Jean-Marc Ayrault and is no longer
pushing for unilateral recognition of an independent Palestinian state.

The meeting on Friday, which lasted only about three hours, amounted to
little more than an extended photo opportunity, a way for France to show
that it was still committed to a peace process in the Middle East.

In a statement issued after the conference, the participants said that
they had ``reaffirmed'' their commitment to a two-state solution, and
expressed alarm about the situation on the ground, ``in particular
continued acts of violence and ongoing settlement activity.''

``The participants underscored that the status quo is not sustainable,''
the statement said. The statement called for ``fully ending the Israeli
occupation that began in 1967,'' language that differed from that
typically used by the United States in its diplomacy around the
conflict.

But the conference produced few concrete measures to be taken in the
near future. Instead, French officials said, the meeting was the first
step toward fostering a positive environment for the Israelis and the
Palestinians to return to the negotiating table. The French said they
would coordinate discussions and help organize another international
conference by the end of the year, this time with the Israelis and the
Palestinians.

``The goal isn't to force the parties to negotiate,'' Mr. Ayrault said
at a news conference after the meeting. ``But we are not doomed to do
nothing, doomed to stay sit idly by as observers, simply expressing
regrets.''

Israel and the Palestinians
\href{http://www.nytimes.com/2016/05/19/world/middleeast/french-plan-for-middle-east-peace-talks-hits-a-familiar-snag.html}{have
expressed} strong disagreements about the French initiative. The
Palestinians, who have spoken of the need to ``internationalize'' the
Israeli-Palestinian conflict, have welcomed it.

Saeb Erekat, a senior Palestine Liberation Organization official and the
Palestinians' chief negotiator, said in
\href{http://www.haaretz.com/opinion/1.722924}{an op-ed published} in
the Israeli newspaper Haaretz on Thursday that the French initiative was
the ``the flicker of hope Palestine has been waiting for.''

``We are confident that it will provide a clear framework with defined
parameters for the resumption of negotiations,'' Mr. Erekat wrote.

Israel, however, is stridently opposed to France's initiative.

Reacting to the meeting held in Paris on Friday, the Israeli Foreign
Ministry said that the conference ``constituted a missed opportunity.''

``History will record that the conference in Paris only hardened the
Palestinian position and distanced the chances for peace,'' the
statement said.

Speaking on the eve of the Paris meeting, Dore Gold, the director
general of the Israeli Foreign Ministry, even compared the French effort
to the Sykes-Picot agreement, a secret colonialist pact signed by
Britain and France 100 years ago to divide up the territory of the
Ottoman Empire.

Describing the modern Middle East as being ``in an advanced stage of
meltdown,'' Mr. Gold said, ``Initiatives of this sort failed then and
will fail today.''

The Israeli leadership says it prefers a regional track whereby the
moderate Arab states would provide the infrastructure for a resumption
of direct Israeli-Palestinian negotiations.

But it is unclear to what extent Arab leaders will be willing cooperate
openly with Israel's right-wing government, and after years of futile,
intermittent negotiations with Israel, the Palestinians say they have
lost hope in bilateral talks.

After the meeting in Paris on Friday, when asked about Mr. Gold's
comparison of the French initiative to the Sykes-Picot agreement,
Federica Mogherini, the European Union's foreign policy chief, said that
without a ``regional and international framework'' the two parties would
not ``spontaneously'' sit down at the negotiating table.

``It is not about imposing, it is not about dictating, it is not even
about indicating the steps or the content,'' Ms. Mogherini said. ``It is
about creating the space, the possibility, the framework for the parties
to re-engage seriously, credibly.''

``We still refer to the Middle East process, but the reality of fact is
that at this moment there is no peace process at all,'' she said.

Advertisement

\protect\hyperlink{after-bottom}{Continue reading the main story}

\hypertarget{site-index}{%
\subsection{Site Index}\label{site-index}}

\hypertarget{site-information-navigation}{%
\subsection{Site Information
Navigation}\label{site-information-navigation}}

\begin{itemize}
\tightlist
\item
  \href{https://help.nytimes.com/hc/en-us/articles/115014792127-Copyright-notice}{©~2020~The
  New York Times Company}
\end{itemize}

\begin{itemize}
\tightlist
\item
  \href{https://www.nytco.com/}{NYTCo}
\item
  \href{https://help.nytimes.com/hc/en-us/articles/115015385887-Contact-Us}{Contact
  Us}
\item
  \href{https://www.nytco.com/careers/}{Work with us}
\item
  \href{https://nytmediakit.com/}{Advertise}
\item
  \href{http://www.tbrandstudio.com/}{T Brand Studio}
\item
  \href{https://www.nytimes.com/privacy/cookie-policy\#how-do-i-manage-trackers}{Your
  Ad Choices}
\item
  \href{https://www.nytimes.com/privacy}{Privacy}
\item
  \href{https://help.nytimes.com/hc/en-us/articles/115014893428-Terms-of-service}{Terms
  of Service}
\item
  \href{https://help.nytimes.com/hc/en-us/articles/115014893968-Terms-of-sale}{Terms
  of Sale}
\item
  \href{https://spiderbites.nytimes.com}{Site Map}
\item
  \href{https://help.nytimes.com/hc/en-us}{Help}
\item
  \href{https://www.nytimes.com/subscription?campaignId=37WXW}{Subscriptions}
\end{itemize}
