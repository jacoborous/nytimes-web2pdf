Sections

SEARCH

\protect\hyperlink{site-content}{Skip to
content}\protect\hyperlink{site-index}{Skip to site index}

\href{https://www.nytimes.com/section/technology}{Technology}

\href{https://myaccount.nytimes.com/auth/login?response_type=cookie\&client_id=vi}{}

\href{https://www.nytimes.com/section/todayspaper}{Today's Paper}

\href{/section/technology}{Technology}\textbar{}U.S. Subpoenas Huawei
Over Its Dealings in Iran and North Korea

\url{https://nyti.ms/1RQdtl5}

\begin{itemize}
\item
\item
\item
\item
\item
\end{itemize}

Advertisement

\protect\hyperlink{after-top}{Continue reading the main story}

Supported by

\protect\hyperlink{after-sponsor}{Continue reading the main story}

\hypertarget{us-subpoenas-huawei-over-its-dealings-in-iran-and-north-korea}{%
\section{U.S. Subpoenas Huawei Over Its Dealings in Iran and North
Korea}\label{us-subpoenas-huawei-over-its-dealings-in-iran-and-north-korea}}

\includegraphics{https://static01.nyt.com/images/2016/06/03/business/03huawei/03huawei-articleLarge.jpg?quality=75\&auto=webp\&disable=upscale}

By \href{https://www.nytimes.com/by/paul-mozur}{Paul Mozur}

\begin{itemize}
\item
  June 2, 2016
\item
  \begin{itemize}
  \item
  \item
  \item
  \item
  \item
  \end{itemize}
\end{itemize}

HONG KONG --- Huawei Technologies has become China's most successful
international technology company, in part by tapping markets as varied
as Britain, India and Kenya.

But it also moved into
\href{https://www.facebook.com/pages/Huawei-Syria-Office/200124263335445}{markets
like Syria}, where American officials have imposed limits on sales of
technology that could be used to commit human rights abuses, and into
Iran, where sanctions have only recently been eased. And its presence in
such countries is now coming under greater scrutiny.

The United States Commerce Department is demanding that the company,
based in the south China city of Shenzhen, turn over all information
regarding the export or re-export of American technology to Cuba, Iran,
North Korea, Sudan and Syria, according to a subpoena sent to Huawei and
viewed by The New York Times. The subpoena is part of an investigation
into whether Huawei broke United States export controls.

Sent to Huawei's American headquarters in the Dallas suburb of Plano,
the subpoena called for Huawei to turn over information related to
shipments to those countries over the past five years. It also sought
evidence of shipments to the countries indirectly through front or shell
companies. The subpoena directed company officials to testify last month
in Irving, Tex., or to provide information before then; it was not clear
whether the meeting took place.

Huawei has not been accused of wrongdoing. In a statement, the company
said it was committed to complying with laws and regulations where it
operated. The document, which was issued by the Commerce Department
office that investigates export violations, is an administrative
subpoena, meaning it does not indicate a criminal investigation.

Still, the scrutiny over Huawei's dealings with those countries is
emblematic of growing discord between the United States and China over
control of global communications technology. It also illustrates how
technology companies from both countries have been pulled into the
high-stakes geopolitical contest over cybersecurity and the global
management of the internet.

If the investigation finds that Huawei was acting counter to United
States national security or foreign policy interests, it could limit the
company's access to crucial American-made components and other tech
products. Given Huawei's size and reach, that could affect the
development of cellular networks and other large-scale technology
infrastructure projects across the world.

``We do not comment with regard to ongoing investigations,'' a Commerce
Department spokesman said.

The subpoena was issued after the United States briefly blocked in March
sales of American technology to Huawei's smaller Chinese rival, ZTE,
over similar concerns. As part of their move against ZTE, American
officials released internal ZTE documents that showed the Chinese
company used a rival's business efforts in those countries as a model.
While the rival was not named in the documents, its description
\href{http://www.nytimes.com/2016/03/19/technology/zte-document-raises-questions-about-huawei-and-sanctions.html}{matched
Huawei}.

With the new investigation into Huawei, the United States is going after
a much larger target. In 2014, Huawei reported revenue of about \$60
billion, about four times that of ZTE. Depending on the measure, it
ranks with Ericsson of Sweden as the world's largest supplier of the
base stations and other equipment that make mobile telecommunications
systems run.

Though the subpoena did not indicate whether any actions would be taken
against Huawei, any major United States step to block the sales of
American tech equipment to Huawei would have major implications for
telecom networks across the world. Many of Huawei's products use
American components or work with American technology.

Huawei has long benefited from access to easy credit from China's
state-run lenders as it has expanded into areas where China seeks
influence. But the company has drawn skepticism in the United States,
where officials have put an effective block on selling its telecom
infrastructure equipment. China has used the move as a justification to
push back against the market dominance of American companies like Cisco,
IBM and Qualcomm in China.

Disclosures by the former American intelligence contractor Edward J.
Snowden revealed that as the United States publicly raised concerns
about the security of Huawei products, the United States National
Security Agency was busy working to tunnel its own backdoor access into
Huawei equipment and to snoop on Huawei's communications to look for
links to the Chinese military.

Huawei has not shied from agreements that could draw criticism. In
September, it signed a deal with Syria's Communications and Technology
Ministry to help the country develop its communications networks.

Huawei's business in Iran has fallen under American criticism in the
past. In 2011, Huawei said in a statement that it would voluntarily
restrict the growth of its business in Iran. A year later, six American
lawmakers wrote a letter to the State Department, calling for an
investigation into whether Huawei was violating sanctions on Iran.
Recently, the Congressional Research Service released a report that said
that companies like Huawei appeared to have fulfilled pledges not to
sell technology for blocking telecommunications in 2014.

Advertisement

\protect\hyperlink{after-bottom}{Continue reading the main story}

\hypertarget{site-index}{%
\subsection{Site Index}\label{site-index}}

\hypertarget{site-information-navigation}{%
\subsection{Site Information
Navigation}\label{site-information-navigation}}

\begin{itemize}
\tightlist
\item
  \href{https://help.nytimes.com/hc/en-us/articles/115014792127-Copyright-notice}{©~2020~The
  New York Times Company}
\end{itemize}

\begin{itemize}
\tightlist
\item
  \href{https://www.nytco.com/}{NYTCo}
\item
  \href{https://help.nytimes.com/hc/en-us/articles/115015385887-Contact-Us}{Contact
  Us}
\item
  \href{https://www.nytco.com/careers/}{Work with us}
\item
  \href{https://nytmediakit.com/}{Advertise}
\item
  \href{http://www.tbrandstudio.com/}{T Brand Studio}
\item
  \href{https://www.nytimes.com/privacy/cookie-policy\#how-do-i-manage-trackers}{Your
  Ad Choices}
\item
  \href{https://www.nytimes.com/privacy}{Privacy}
\item
  \href{https://help.nytimes.com/hc/en-us/articles/115014893428-Terms-of-service}{Terms
  of Service}
\item
  \href{https://help.nytimes.com/hc/en-us/articles/115014893968-Terms-of-sale}{Terms
  of Sale}
\item
  \href{https://spiderbites.nytimes.com}{Site Map}
\item
  \href{https://help.nytimes.com/hc/en-us}{Help}
\item
  \href{https://www.nytimes.com/subscription?campaignId=37WXW}{Subscriptions}
\end{itemize}
