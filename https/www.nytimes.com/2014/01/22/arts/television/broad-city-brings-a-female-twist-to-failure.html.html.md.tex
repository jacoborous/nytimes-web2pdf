Sections

SEARCH

\protect\hyperlink{site-content}{Skip to
content}\protect\hyperlink{site-index}{Skip to site index}

\href{https://www.nytimes.com/section/arts/television}{Television}

\href{https://myaccount.nytimes.com/auth/login?response_type=cookie\&client_id=vi}{}

\href{https://www.nytimes.com/section/todayspaper}{Today's Paper}

\href{/section/arts/television}{Television}\textbar{}They're Young,
Female and, Darn It, Need to Work

\href{https://nyti.ms/1eoOrcu}{https://nyti.ms/1eoOrcu}

\begin{itemize}
\item
\item
\item
\item
\item
\end{itemize}

Advertisement

\protect\hyperlink{after-top}{Continue reading the main story}

Supported by

\protect\hyperlink{after-sponsor}{Continue reading the main story}

Television Review

\hypertarget{theyre-young-female-and-darn-it-need-to-work}{%
\section{They're Young, Female and, Darn It, Need to
Work}\label{theyre-young-female-and-darn-it-need-to-work}}

\includegraphics{https://static01.nyt.com/images/2014/01/22/arts/22BROADCITY/22BROADCITY-articleLarge.jpg?quality=75\&auto=webp\&disable=upscale}

By \href{http://www.nytimes.com/by/alessandra-stanley}{Alessandra
Stanley}

\begin{itemize}
\item
  Jan. 21, 2014
\item
  \begin{itemize}
  \item
  \item
  \item
  \item
  \item
  \end{itemize}
\end{itemize}

For so long, women tried to succeed as much as men.

Now, they have to fail as badly.

``\href{http://www.comedycentral.com/shows/broad-city}{Broad City},'' a
new series starting on Wednesday on Comedy Central, is funny, and, like
so many other shows on that channel, brazen about skewering the
millennial generation as hapless 20-somethings with no ambition, talent
or self-respect. These slackers happen to be women, as the show's
creators, Abbi Jacobson and Ilana Glazer, play comic versions of
themselves.

Amy Poehler is an executive producer, but the series got its start on
the Internet and is more linear, unpolished and narrowly comedic than
``Girls'' on HBO --- Abbi and Ilana are so feckless that they make Lena
Dunham's Hannah seem like Warren Buffett.

``Broad City'' can't even be compared to
\href{http://www.comedycentral.com/shows/workaholics}{``Workaholics,''}
a show about loser male friends that precedes it on Comedy Central: The
new series is not nearly as sweet.

It looks as if it could almost be a rough early draft for ``2 Broke
Girls,'' on CBS: sort of a crude, transgressive and poorly typed
treatment that one producer kind of likes and that the network rejects
out of hand.

\includegraphics{https://static01.nyt.com/images/2014/01/21/multimedia/broadcity/broadcity-videoSixteenByNine1050.jpg}

Even with today's permissive standards, no network show would open with
two friends video chatting quite this way: Abbi is staring curiously at
an enormous vibrator, labeled with a note that says ``Tuesday 7 a.m.,''
while Ilana is astride a man (Hannibal Buress). Technically, they are
having sex, but distractedly at best. What Ilana is really into is
coaxing Abbi to go with her to a pop-up Lil Wayne concert.

They cannot really afford concert tickets, especially because such an
outing requires marijuana and drinks, and the episode revolves around
their idiotic attempts to raise the money.

Abbi, the meeker of the two, works as a cleaner at a high-end spinning
studio, Soulstice (a parody of SoulCycle). She fantasizes about someday
leading a spin class, but mostly unclogs toilets.

Ilana, bolder and more flamboyant, doesn't even pretend to work at her
job, at an e-commerce company. She peruses Craigslist for a job that can
get them quick cash and posts their availability this way: ``We're just
2 Jewesses tryin' to make a buck.''

The show is as puerile and scatological as any male-centered series on
Comedy Central, but oddly enough, it's the self-degradation that gives
it feminist cachet. As Sarah Silverman proved with her series on Comedy
Central,
``\href{http://www.nytimes.com/2008/10/23/arts/television/23silv.html}{The
Sarah Silverman Program},'' which ran from 2007 to 2010, female
characters are increasingly entitled to be as indolent, selfish and
incompetent as male ones.

\includegraphics{https://static01.nyt.com/images/2014/01/22/arts/sub22BROADCITYjp/sub22BROADCITYjp-articleLarge.jpg?quality=75\&auto=webp\&disable=upscale}

That was not imaginable in earlier eras of television. The most
successful sitcom heroines were plucky strivers, and the comedy lay in
the pratfalls they took to achieve their goals --- Lucy trying to break
into show business on ``I Love Lucy,'' Mary Richards trying to mix love
and career on ``The Mary Tyler Moore Show,'' or Murphy Brown trying to
get her way on all things.

The most notable exception was Roseanne Barr, who created a new template
with her slovenly, lazy and anything but perky persona on ``Roseanne,''
in 1988. And even Roseanne eventually had to adapt to sitcom
expectations, getting a job and taking loving care of her family.

Even now, most women on network comedies are plucky strivers, including
the Brooklyn diner waitresses of
\href{http://www.cbs.com/shows/2_broke_girls/}{``2 Broke Girls.''} Max
(Kat Dennings) may be the cynical, wisecracking layabout of the duo, but
for all her supposed do-nothing negativity, she has a talent for baking
and ambition to make something of it.

Whitney Cummings is the co-creator of that show. The character Ms.
Cummings played on another series she created, ``Whitney,'' on NBC, had
a lot more in common with Abbi and Ilana than with the heroines of ``2
Broke Girls,'' and while ``Whitney'' was, in many ways, darker and
funnier than that show, it was canceled after two seasons. Women can be
raunchy on sitcoms, but they cannot be uninspiring.

Those rules don't apply on cable, where shows have a mandate to flout
network conventions. It's not just Ms. Dunham on ``Girls.'' Julia
Louis-Dreyfus is a hilariously lazy, self-centered phony on ``Veep.''
And now Abbi and Ilana are the ``Laverne \& Shirley'' of their times:
They don't chant and skip to work, they smoke pot and shirk.

Advertisement

\protect\hyperlink{after-bottom}{Continue reading the main story}

\hypertarget{site-index}{%
\subsection{Site Index}\label{site-index}}

\hypertarget{site-information-navigation}{%
\subsection{Site Information
Navigation}\label{site-information-navigation}}

\begin{itemize}
\tightlist
\item
  \href{https://help.nytimes.com/hc/en-us/articles/115014792127-Copyright-notice}{©~2020~The
  New York Times Company}
\end{itemize}

\begin{itemize}
\tightlist
\item
  \href{https://www.nytco.com/}{NYTCo}
\item
  \href{https://help.nytimes.com/hc/en-us/articles/115015385887-Contact-Us}{Contact
  Us}
\item
  \href{https://www.nytco.com/careers/}{Work with us}
\item
  \href{https://nytmediakit.com/}{Advertise}
\item
  \href{http://www.tbrandstudio.com/}{T Brand Studio}
\item
  \href{https://www.nytimes.com/privacy/cookie-policy\#how-do-i-manage-trackers}{Your
  Ad Choices}
\item
  \href{https://www.nytimes.com/privacy}{Privacy}
\item
  \href{https://help.nytimes.com/hc/en-us/articles/115014893428-Terms-of-service}{Terms
  of Service}
\item
  \href{https://help.nytimes.com/hc/en-us/articles/115014893968-Terms-of-sale}{Terms
  of Sale}
\item
  \href{https://spiderbites.nytimes.com}{Site Map}
\item
  \href{https://help.nytimes.com/hc/en-us}{Help}
\item
  \href{https://www.nytimes.com/subscription?campaignId=37WXW}{Subscriptions}
\end{itemize}
