Sections

SEARCH

\protect\hyperlink{site-content}{Skip to
content}\protect\hyperlink{site-index}{Skip to site index}

\href{https://www.nytimes.com/section/us}{U.S.}

\href{https://myaccount.nytimes.com/auth/login?response_type=cookie\&client_id=vi}{}

\href{https://www.nytimes.com/section/todayspaper}{Today's Paper}

\href{/section/us}{U.S.}\textbar{}Witnesses Told Grand Jury That Michael
Brown Charged at Darren Wilson, Prosecutor Says

\url{https://nyti.ms/1C8JhA0}

\begin{itemize}
\item
\item
\item
\item
\item
\end{itemize}

Advertisement

\protect\hyperlink{after-top}{Continue reading the main story}

Supported by

\protect\hyperlink{after-sponsor}{Continue reading the main story}

\hypertarget{witnesses-told-grand-jury-that-michael-brown-charged-at-darren-wilson-prosecutor-says}{%
\section{Witnesses Told Grand Jury That Michael Brown Charged at Darren
Wilson, Prosecutor
Says}\label{witnesses-told-grand-jury-that-michael-brown-charged-at-darren-wilson-prosecutor-says}}

\includegraphics{https://static01.nyt.com/images/2014/11/24/multimedia/ferguson-jury-decision/ferguson-jury-decision-videoSixteenByNine1050.jpg}

By \href{http://www.nytimes.com/by/erik-eckholm}{Erik Eckholm}

\begin{itemize}
\item
  Nov. 24, 2014
\item
  \begin{itemize}
  \item
  \item
  \item
  \item
  \item
  \end{itemize}
\end{itemize}

The most credible eyewitnesses to the shooting death of Michael Brown in
Ferguson, Mo., said he had charged toward Police Officer Darren Wilson
just before the final, fatal shots, the St. Louis County prosecutor said
Monday night as he sought to explain why a grand jury had not found
probable cause to indict the officer.

The accounts of several other witnesses from the Ferguson neighborhood
where Mr. Brown, 18 and unarmed, met his death on Aug. 9 --- including
those who said Mr. Brown was trying to surrender --- changed over time
or were inconsistent with physical evidence, the prosecutor, Robert P.
McCulloch, said in a news conference.

``The duty of the grand jury is to separate fact and fiction,'' he said
in a statement watched by a tense nation. ``No probable cause exists to
file any charges against Darren Wilson.''

Mr. McCulloch praised the grand jurors, who met on 25 days over a
three-month period and heard 60 witnesses, for pouring ``their hearts
and souls into this process'' and said that only by hearing all the
evidence, as they had, could one fairly judge the case.

The task facing the St. Louis County grand jury was not to determine
whether Officer Darren Wilson was guilty of a crime, but whether there
was evidence to justify bringing charges, which could have ranged from
negligent manslaughter to intentional murder.

The fact that at least nine members of the 12-member panel could not
agree to indict the officer indicates that they accepted the narrative
of self-defense put forth by Officer Wilson in his voluntary, four hours
of testimony before the grand jury. Mr. McCulloch, in his summary of the
months of testimony, said it was supported by the most reliable
eyewitness accounts --- from African-Americans in the vicinity of the
shooting --- as well as physical evidence and the consistent results of
three autopsies.

At issue, under the Missouri law governing use of deadly force by law
enforcement as well as general rules for self-defense, was if Officer
Wilson ``reasonably believed'' that he or others were in serious danger.

According to
\href{http://graphics8.nytimes.com/newsgraphics/2014/11/24/ferguson-assets/grand-jury-testimony.pdf}{transcripts}
released Monday night, Officer Wilson testified that after he
encountered Mr. Brown and a friend walking in the street, he realized
the pair might be those being sought for stealing cigarillos from a
convenience store minutes earlier.

According to witnesses and blood and other evidence found inside the
car, Officer Wilson first fired two shots while he struggled with Mr.
Brown through the window of his patrol vehicle, a Chevrolet Tahoe,
grazing Mr. Brown's hand.

\href{https://www.nytimes.com/interactive/2014/08/13/us/ferguson-missouri-town-under-siege-after-police-shooting.html}{}

\includegraphics{https://static01.nyt.com/images/2014/08/13/us/ferguson-missouri-town-under-siege-after-police-shooting-1415998664223/ferguson-missouri-town-under-siege-after-police-shooting-1415998664223-videoLarge-v2.png}

\hypertarget{what-happened-in-ferguson}{%
\subsection{What Happened in
Ferguson?}\label{what-happened-in-ferguson}}

Here's what you need to know about events in Ferguson, Mo.

Mr. Brown started to run away, with Officer Wilson in chase, then
stopped and turned. According to the prosecutor's summary, the officer
fired five shots as Mr. Brown charged him, then another five shots as he
made what one witness called a ``full charge.''

Only 90 seconds passed between Officer Wilson's first encounter with the
youths and the arrival of a backup police car, just after the shooting
stopped, the prosecutor said.

Probable cause is not a stiff standard. It does not require that most of
the evidence be incriminating, let alone be proof ``beyond a reasonable
doubt,'' as required in a criminal trial. Instead, grand juries are
ordinarily instructed to issue an indictment when there is ``some
evidence'' of guilt, legal experts said.

To Mr. Brown's parents and their supporters, the case for bringing at
least some charge in this case seemed open and shut. But the jurors also
had to consider whether Officer Wilson acted within the limits of the
lethal-force law, raising the threshold for an indictment.

Independent legal experts said it was impossible to analyze the grand
jury decision without studying the transcripts of the testimony as well
as the police reports, autopsies and forensic evidence that might shed
light on what Mr. Brown was doing in his final seconds: whether he was
menacing the officer or, as a friend who was with him said, trying to
surrender.

Some people claiming to be eyewitnesses said Mr. Brown was shot in the
back, Mr. McCulloch said, but later changed their stories when autopsies
found no injuries entering his back. But others, African-Americans who
did not speak out publicly, he said, consistently said that the youth
had menaced the officer.

\href{https://www.nytimes.com/interactive/2014/11/09/us/10ferguson-michael-brown-shooting-grand-jury-darren-wilson.html}{}

\includegraphics{https://static01.nyt.com/images/2014/08/12/us/JP-STLOUIS3/JP-STLOUIS3-videoLarge-v2.jpg}

\hypertarget{tracking-the-events-in-the-wake-of-michael-browns-shooting}{%
\subsection{Tracking the Events in the Wake of Michael Brown's
Shooting}\label{tracking-the-events-in-the-wake-of-michael-browns-shooting}}

Updates on the events in Ferguson, Mo., following the shooting of
Michael Brown, an unarmed teenager, by a police officer on Aug. 9.

Mr. McCulloch, had promised that he would allay any suspicions about the
fairness of the proceedings by releasing, with names redacted,
transcripts of testimony and other evidence heard by the panel.

The release of grand jury information, secret by law, is rare, and Mr.
McCulloch originally said he would first seek a judge's permission. But
on Monday, his office said it had determined that it had a right to
release most of the transcripts and it did so Monday night.

The grand jury, which included three African-Americans, deliberated for
two days. By law, the final vote on whether to bring an indictment is
secret and the jurors are legally prohibited from discussing their
deliberations.

The United States Department of Justice is conducting a separate
investigation of whether Officer Wilson, who is white, intentionally
acted to deprive Mr. Brown, an African-American, of his civil rights.
But the bar for such cases is a high one, and officials have privately
said they are unlikely to bring federal charges. The Justice Department
is also conducting a broader investigation into the practices of the
Ferguson Police Department.

Advertisement

\protect\hyperlink{after-bottom}{Continue reading the main story}

\hypertarget{site-index}{%
\subsection{Site Index}\label{site-index}}

\hypertarget{site-information-navigation}{%
\subsection{Site Information
Navigation}\label{site-information-navigation}}

\begin{itemize}
\tightlist
\item
  \href{https://help.nytimes.com/hc/en-us/articles/115014792127-Copyright-notice}{©~2020~The
  New York Times Company}
\end{itemize}

\begin{itemize}
\tightlist
\item
  \href{https://www.nytco.com/}{NYTCo}
\item
  \href{https://help.nytimes.com/hc/en-us/articles/115015385887-Contact-Us}{Contact
  Us}
\item
  \href{https://www.nytco.com/careers/}{Work with us}
\item
  \href{https://nytmediakit.com/}{Advertise}
\item
  \href{http://www.tbrandstudio.com/}{T Brand Studio}
\item
  \href{https://www.nytimes.com/privacy/cookie-policy\#how-do-i-manage-trackers}{Your
  Ad Choices}
\item
  \href{https://www.nytimes.com/privacy}{Privacy}
\item
  \href{https://help.nytimes.com/hc/en-us/articles/115014893428-Terms-of-service}{Terms
  of Service}
\item
  \href{https://help.nytimes.com/hc/en-us/articles/115014893968-Terms-of-sale}{Terms
  of Sale}
\item
  \href{https://spiderbites.nytimes.com}{Site Map}
\item
  \href{https://help.nytimes.com/hc/en-us}{Help}
\item
  \href{https://www.nytimes.com/subscription?campaignId=37WXW}{Subscriptions}
\end{itemize}
