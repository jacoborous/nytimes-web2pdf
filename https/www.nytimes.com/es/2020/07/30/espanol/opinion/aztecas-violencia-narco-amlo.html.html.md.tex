Sections

SEARCH

\protect\hyperlink{site-content}{Skip to
content}\protect\hyperlink{site-index}{Skip to site index}

\href{https://www.nytimes.com/es/section/opinion}{Opinión}

\href{https://myaccount.nytimes.com/auth/login?response_type=cookie\&client_id=vi}{}

\href{https://www.nytimes.com/section/todayspaper}{Today's Paper}

\href{/es/section/opinion}{Opinión}\textbar{}La tierra en préstamo: una
gramática de la violencia en México

\url{https://nyti.ms/2D2TDJS}

\begin{itemize}
\item
\item
\item
\item
\item
\item
\end{itemize}

Advertisement

\protect\hyperlink{after-top}{Continue reading the main story}

\href{/es/section/opinion}{Opinión}

Supported by

\protect\hyperlink{after-sponsor}{Continue reading the main story}

Comentario

\hypertarget{la-tierra-en-pruxe9stamo-una-gramuxe1tica-de-la-violencia-en-muxe9xico}{%
\section{La tierra en préstamo: una gramática de la violencia en
México}\label{la-tierra-en-pruxe9stamo-una-gramuxe1tica-de-la-violencia-en-muxe9xico}}

El hallazgo de un inmenso altar fúnebre azteca permite reflexionar sobre
las urgencias actuales sin fantasías atávicas pero con un nítido sentido
de la historia y los desafíos del presente.

\includegraphics{https://static01.nyt.com/images/2020/07/29/opinion/29villoro-sub/29villoro-sub-articleLarge.jpg?quality=75\&auto=webp\&disable=upscale}

Por Juan Villoro

Es novelista y cronista.

\begin{itemize}
\item
  30 de julio de 2020
\item
  \begin{itemize}
  \item
  \item
  \item
  \item
  \item
  \item
  \end{itemize}
\end{itemize}

\href{https://www.nytimes.com/newsletters/el-times}{Regístrate para
recibir nuestro boletín} con lo mejor de The New York Times.

\begin{center}\rule{0.5\linewidth}{\linethickness}\end{center}

CIUDAD DE MÉXICO --- México vive la peor violencia desde la Revolución
(1910-1920); sin embargo, en su primer informe de gobierno, el
presidente Andrés Manuel López Obrador dedicó 40 segundos al tema.

El crimen organizado ocupa el territorio y diversifica su economía. A la
piratería, el secuestro, la trata y el narcotráfico, añade el robo de
combustible, los narcocréditos, la agricultura de exportación, la
minería, el control del agua, el cobro de derecho de suelo e incluso
prácticas clientelistas como el
\href{https://www.nytimes.com/es/2020/07/07/espanol/opinion/coronavirus-carteles-drogas-mexico.html}{reparto
de alimentos y medicinas}.

La soberanía nacional es relativa, según confirmó el periodista de El
País Jacobo García en su alucinante viaje por la región de Michoacán,
donde se cultiva el 70 por ciento de la producción mundial de aguacate:
``La carretera de la muerte no es la que recorre Los Andes o la ladera
de los Anapurna, sino los 36 kilómetros que unen Jalisco y Michoacán a
través de Jilotitlán'', escribió en septiembre de 2019, luego de
recorrer parajes que le recordaron escenas de guerra en Siria, Irak y
Afganistán.

El país se desgaja sin una política de seguridad que haga frente a la
situación. López Obrador cortó con las fallidas estrategias anteriores,
medida imprescindible, pero los
\href{https://www.gob.mx/sesnsp/acciones-y-programas/victimas-nueva-metodologia?state=published}{asesinatos
aumentan}. ¿Hay salida? La respuesta equivale a un vacío:
\href{https://elpais.com/internacional/2019/10/19/actualidad/1571506100_833972.html}{40
segundos} de informe presidencial.

Mientras esto sucede, los arqueólogos hallan restos del imperio azteca
que remiten a una violencia remota. ¿Podemos vernos reflejados en esos
saldos del origen como si nos asomáramos al Espejo Humeante de
Tezcatlipoca, Señor de la Fatalidad, donde el ser humano debía escrutar
su condición inescapable?

\hypertarget{una-torre-de-cruxe1neos}{%
\subsection{Una torre de cráneos}\label{una-torre-de-cruxe1neos}}

La Ciudad de México tiene otra ciudad bajo la tierra. Por códices
prehispánicos y crónicas de frailes y conquistadores, los arqueólogos
saben de la existencia de sitios que no han sido explorados.

La céntrica calle de República de Guatemala se extiende sobre la antigua
ruta sagrada de la muerte. Ahí se encontraba el juego de pelota azteca,
donde el perdedor era ofrendado a los dioses, y en 2006 ahí fue hallada
la efigie de Tlaltecuhtli, deidad dual, masculina y femenina, que devora
las inmundicias y da a luz nueva vida.

En 2015 se
\href{https://historia.nationalgeographic.com.es/a/descubren-gran-tzompantli-tenochtitlan_9609/1}{descubrió}
el vestigio más importante en la relación de los antiguos mexicanos con
la muerte: el \emph{tzompantli}, inmenso altar de cráneos. En el número
24 de Guatemala la remodelación de una casa confirmó que excavar en esa
parte de la ciudad es una arqueología accidental. En este caso, se
encontró la base de una torre de cráneos consagrada a Huitzilopochtli,
dios del sol y la guerra. Durante la conquista, Andrés de Tapia, soldado
de Hernán Cortés, creyó distinguir ahí 136.000 cráneos y el fraile Diego
Durán, 80.000, cifras seguramente exageradas por el temor reverencial
que provocaba esa empalizada fúnebre. En \emph{Muerte a filo de
obsidiana}, Eduardo Matos Moctezuma, quien condujo la exploración del
Templo Mayor, define al \emph{tzompantli} como ``la manifestación más
evidente del control político-religioso'' que la jerarquía de sacerdotes
y militares ejercía sobre su propio pueblo.

A partir de octubre de 2016, el arqueólogo Raúl Barrera se hizo cargo de
los trabajos en Guatemala 24. El sitio aún no ha sido abierto al
público, pero pude visitarlo el 16 de noviembre de 2017, dos meses
después del terremoto que derribó numerosos edificios en la ciudad. La
casona colonial resistió de milagro los embates telúricos y la
excavación en el sótano. Vigas de madera, dispuestas en equis, apuntalan
los muros. A unos metros, el Museo del Templo Mayor muestra una
representación en piedra del \emph{tzompantli}. Esa asombrosa geometría
de la muerte no deja de ser abstracta. El enjambre de cráneos que sale
del lodo en Guatemala 24 no suplanta un hecho; lo constata: miles de
cuencas vacías escrutan la nada desde hace quinientos años.

De acuerdo con Barrera, la mayoría de los sacrificados eran cautivos de
guerra y se llegaron a incluir cráneos de españoles. La principal
revelación de campo ha sido que, aunque el
\href{https://www.abc.es/historia/abci-tzompantli-bestial-ofrenda-azteca-dioses-derribaron-conquistadores-espanoles-201809240905_noticia.html}{75
por ciento} de los restos pertenecen a hombres, el 25 por ciento es de
mujeres y niños. En la economía sacrificial de los aztecas, destinada a
pacificar dioses veleidosos, había que ofrendar prisioneros, pero
también prescindir de lo más querido. La vida no se despreciaba;
aumentaba de valor al entregarse de ese modo.

La torre de cráneos semidescarnados de casi cinco metros de diámetro
realzaba el poder político-religioso en Tenochtitlan. Una ciudad de
alrededor 250.000 habitantes confluía en ese escenario. Ante ese trato
con la muerte, conviene recordar lo que Georges Dumézil escribió a
propósito de las ``rarezas'' del pasado: interpretar los ``hechos
religiosos arcaicos'' en su justa dimensión implica prescindir de ``las
ideas bárbaras y engañosas que las escuelas imaginan''. En \emph{La
muerte entre los mexicas}, Matos Moctezuma entiende así el
\emph{tzompantli}: ``Los dioses, a veces beligerantes, a veces
benévolos, deberán ser ofrendados por diversos medios para que jueguen
un papel que tienen encomendado dentro de la estructura universal. Entre
lo más preciado que el hombre posee está el hombre mismo, de allí que el
sacrificio de su vida conlleve, en buena medida, la continuación del
movimiento por medio del cual hay vida''.

Hace unos días le pregunté a Matos Moctezuma sobre la cantidad de
cráneos que esperan hallar en el \emph{tzompantli}: ``Un cálculo
preliminar podría dar unos 2000 cráneos, pero Raúl Barrera cree que
podrían ser 5000, y hay que recordar que se iban quitando algunos y
colocando otros nuevos'', comenta.

Toda estadística fúnebre es excesiva: cada hueso constata un fin
irreparable. Los cráneos ensartados en el \emph{tzompantli} integran un
ábaco de ofrendas a los dioses. Aunque no es fácil contemplarlo,
responde a un significado; el sacrificio era una plegaria: alimentaba al
sol para que no dejara de brillar.

En Guatemala 24 la tierra conserva la humedad de la laguna que fue
sepultada para edificar la Ciudad de México. Ahí, los siglos enrarecen
el aire y los cráneos enrarecen el presente. El mundo del sacrificio
azteca nos resulta ajeno, pero hay claves para entenderlo. En
comparación, el México contemporáneo es más absurdo. ¿Cómo explicar un
país de fosas clandestinas (más de 3000 en los últimos 14 años) donde se
muere sin otra causa que el despojo?

\hypertarget{150-disparos-en-tres-minutos}{%
\subsection{150 disparos en tres
minutos}\label{150-disparos-en-tres-minutos}}

El 26 de junio, a las 6:35 de la mañana, un camión bloqueó Paseo de la
Reforma, emblemática avenida de la Ciudad de México, y 28 sicarios
balacearon el coche de Omar García Harfuch, secretario de Seguridad
Ciudadana. Viajaba con dos escoltas que
\href{https://www.animalpolitico.com/2020/06/minutos-segundos-atentado-contra-garcia-harfuch/}{fallecieron},
al igual que una vendedora que pasaba por la zona. García Harfuch
sobrevivió gracias al blindaje nivel 5 plus del vehículo. Tres horas
después del atentado
\href{https://twitter.com/OHarfuch/status/1276523720022962177}{escribió
en Twitter}: ``Esta mañana fuimos cobardemente atacados por el CJNG
(Cártel Jalisco Nueva Generación){[}\ldots{}{]}, tengo tres impactos de
bala y varias esquirlas''.

Los atacantes fueron repelidos por cuatro guardaespaldas que iban en
otro coche, que quedó fuera del cerco de fuego, y por patrullas que
llegaron un minuto después.
\href{https://www.animalpolitico.com/2020/06/dos-detenidos-mas-atentado-harfuch/}{Veintiún}
sospechosos han sido arrestados. El atentado fue un notable fracaso,
pero lo que llama la atención no es la impericia de quienes dispararon
más de 150 balazos sin dar con su objetivo, sino su espectacularidad, el
despliegue teatral de la osadía. La prioridad no era asesinar, sino
demostrar que eso es posible en el corazón de la capital mexicana.

En sintonía con esta estrategia, el 17 de julio
\href{https://www.animalpolitico.com/2020/07/grupo-elite-cartel-jalisco-videos-sedena/}{circuló
un video} en el que el Cártel Jalisco Nueva Generación despliega sus
tropas. La cámara recorre una larguísima fila de vehículos pintados de
camuflaje. En cada portezuela, una calavera y las siglas CJNG acreditan
al ``grupo de élite''. Un ejército encapuchado alza el puño y grita:
``¡Pura gente del Mencho!'', en alusión a su líder, Nemesio Oseguera
Cervantes.

Días después, el ``Doble R'', miembro prominente del cártel,
\href{https://www.sinembargo.mx/23-07-2020/3829062}{aclaró}: ``Nuestra
guerra no es contra el pueblo ni es contra el gobierno''. Según esta
versión, el desfile estaba destinado a amedrentar al ``Marro'', José
Antonio Yépez Ortiz, líder del competidor Cártel de Santa Rosa de Lima.

El narcotráfico ejerce un poderío visible al tiempo que el gobierno se
repliega. La Guardia Nacional creada por el gobierno de Andrés Manuel
López Obrador estaba destinada a reunir y coordinar grupos judiciales
dispersos; sin embargo, desde su creación ha debido atender otras
tareas. La más importante es la contención de migrantes a Estados
Unidos. Donald Trump desistió de su amenaza de aumentar los aranceles a
las exportaciones mexicanas a cambio de que se controlara el tráfico de
indocumentados. De este modo canjeó un tema económico por una exigencia
migratoria, convirtiendo a la Guardia Nacional en extensión de la Border
Patrol. Su promesa de que México pagaría por construir un muro en la
frontera encontró una forma perversa de volverse cierta: el ejército
mexicano debe actuar como una pared cuyo espesor va de Centroamérica al
río Bravo.

La distracción de las fuerzas federales en tareas migratorias, a las que
se añade el control de puertos y aduanas, y las restricciones de la
pandemia (circunstancia aprovechada por el narco y de la que Ioan Grillo
\href{https://www.nytimes.com/es/2020/07/07/espanol/opinion/coronavirus-carteles-drogas-mexico.html}{escribió
en estas páginas}), dificulta el combate al crimen organizado.

¿Hay una estrategia clara al respecto? López Obrador ha hecho llamados
morales a los capos, pidiendo que piensen en sus madres y aconsejando
repartir ``abrazos, no balazos''. Ante la violencia ha usado expresiones
de repudio infantil:
\href{https://www.youtube.com/watch?v=5QzshrEKzzY}{``¡fuchi,
guacala!''}. Mientras tanto, los asesinatos aumentan: la BBC informó que
en 2019, primer año del actual gobierno, se cometieron 34,582 homicidios
dolosos, un
\href{https://www.bbc.com/mundo/noticias-america-latina-51186916}{2.5
por ciento} más que en 2018, hasta entonces el año más cruento en
nuestra historia reciente.

De manera encomiable, López Obrador se propuso acabar con la política de
``guerra contra las drogas'' que el presidente panista Felipe Calderón
calcó de la gestión de Richard Nixon y del Plan Colombia. Ordenó que el
ejército saliera de sus cuarteles en diciembre de 2006, a dos semanas de
haber asumido la presidencia, cuando la oposición cuestionaba el
resultado electoral. No pidió que el Congreso respaldara la medida ni la
propuso en su campaña. Esa iniciativa fue, por decir lo menos,
precipitada. Seis años después había
\href{https://www.infobae.com/america/mexico/2019/10/12/la-guerra-de-felipe-calderon-contra-el-narco-el-inicio-de-una-espiral-de-violencia-sin-fin/}{más
de 100.000 muertos} y más de 30.000 desaparecidos. Calderón insistió en
que el incremento de la violencia se debía a que los cárteles combatían
entre sí por nuevas plazas; se refirió a los narcos como ``los
malosos'', seres extraños infiltrados en el país, sin comprender que
pertenecían al tejido social y que la solución no podía ser
exclusivamente militar. Al combatir fuego con fuego solo hubo un
resultado: todo mexicano podía ser un ``daño colateral''.

Calderón apeló a la ocupación del territorio y la presencia física del
ejército. En \emph{Topología de la violencia}, el filósofo Byung-Chul
Han
\href{https://books.google.com.mx/books?id=8AOIDwAAQBAJ\&pg=PT7\&lpg=PT7\&dq=El+gobierno+se+vale+de+la+simbolog\%C3\%ADa+de+la+sangre.+La+violencia+directa+opera+como+insignia+de+poder.+En+este+caso,+la+violencia+no+se+oculta.\&source=bl\&ots=1cVn6vn2vs\&sig=ACfU3U3f-3zydv-yjYqwRF6UUJE4avD9dA\&hl=es-419\&sa=X\&ved=2ahUKEwjCyMaf5vPqAhVN-qwKHf5QB4UQ6AEwAHoECAoQAQ\#v=onepage\&q\&f=false}{identifica}
esta estrategia con la dominación arcaica: ``El gobierno se vale de la
simbología de la sangre. La violencia directa opera como insignia de
poder. En este caso, la violencia no se oculta. Se hace visible y se
manifiesta. No tiene ningún tipo de pudor. No se muda ni se muestra
medio desnuda, sino elocuente y sustancial''. En esa lógica, ``la
violencia infligida a otro aumenta la capacidad de supervivencia''.

Una fotografía alcanzó el nivel de ícono en la guerra de Calderón. En
2009, la Marina ultimó al capo Arturo Beltrán Leyva y
\href{https://www.elmundo.es/america/2009/12/18/mexico/1261156208.html}{fotografió}
su cuerpo desnudo, tapizado de billetes ensangrentados. Un presunto acto
de justicia se convirtió en gesto de venganza.

El Estado moderno sustituye la visibilidad de la violencia por formas
más opacas de control. Calderón apostó, como ahora lo hace el Cártel
Jalisco Nueva Generación, por exhibir la fuerza para amedrentar al
adversario. Durante su mandato, los periódicos publicaron
``ejecutómetros''. El marcador rojo no favoreció al presidente panista.
Calderón ignoraba la fuerza de su oponente y su grado de infiltración en
los más diversos mandos del gobierno. En una batalla ante un enemigo
difuso, sin nociones de frente y retaguardia, llevaba todas las de
perder.

El hartazgo ante la sangre hizo que en las elecciones de 2012 el PAN
quedara tercer puesto. El país prefirió el regreso del PRI, partido
paleontológico que había gobernado de 1929 a 2000, y que solo se
modernizó en la medida en que su candidato, Enrique Peña Nieto, lucía
mejor en televisión que en la realidad.

A partir de 2012 cambió el discurso oficial. Si Calderón colocó el
militarismo al centro de su gobierno y se puso un uniforme que le
quedaba tan grande como los destinos del país, Peña Nieto consideró que
la violencia era un problema de percepción que mejoraría al no hablar de
él (con el mismo sentido de la evasión, propuso ``pasar página'' al caso
Ayotzinapa, desapareciendo de la memoria a los desaparecidos de la
realidad).

López Obrador no pertenece a la cleptocracia que gobernó el país durante
casi un siglo en beneficio propio ni está dispuesto a poner en práctica
las conductas represivas del PRI y el PAN. Su gobierno, avalado por
\href{https://computos2018.ine.mx/\#/presidencia/nacional/1/1/1/1}{30
millones de votos}, representa un \emph{corte de sentido} respecto a
políticas anteriores. Sin embargo, eso no basta para que tenga éxito. Su
alianza con los evangelistas y los empresarios más poderosos del país,
su desdén por la
\href{https://www.nytimes.com/es/2020/07/06/espanol/opinion/clase-media-mexico.html}{clase
media}, su apuesta por combustibles fósiles, su imparable caudillismo y
su injurioso trato a ambientalistas, feministas, científicos, pueblos
originarios, periodistas y víctimas de la violencia trazan el retrato de
un populista conservador con ocasionales arrebatos izquierdistas.

\hypertarget{suxf3focles-en-sinaloa}{%
\subsection{Sófocles en Sinaloa}\label{suxf3focles-en-sinaloa}}

El 17 de octubre de 2019, Ovidio Guzmán, hijo del célebre ``Chapo'',
\href{https://www.nytimes.com/2019/10/18/world/americas/mexico-cartel-chapo-son-guzman.html}{fue
detenido} por policías antinarcóticos en Culiacán. El Cártel de Sinaloa
reaccionó con una protesta que dejó 68 vehículos militares con impactos
de bala, ocho muertos, 16 heridos, 19 bloqueos y un motín en la cárcel
que liberó a 45 reos. Los 32 grados de temperatura de la capital
sinaloense aumentaron con el fuego. En ese clima incendiario, un narco
que negociaba la liberación del ``Chapito''
\href{https://elpais.com/internacional/2019/10/18/mexico/1571422172_123900.html}{se
dirigió} a los militares con afrentosa superioridad: ``Se te está
hablando bien, suéltalo y vete tranquilo, y no se te va a hacer nada, si
no te va a cargar la verga''.

En esas apremiantes circunstancias, López Obrador ordenó la liberación
de Ovidio Guzmán: ``No puede valer más la captura de un delincuente que
las vidas de las personas'',
\href{https://www.gob.mx/presidencia/prensa/no-puede-valer-mas-la-captura-de-un-delincuente-que-las-vidas-de-las-personas-afirma-presidente-en-oaxaca}{explicó}.
La declaración contrasta
\href{https://heraldodemexico.com.mx/pais/amlo-salva-vidas-inocentes-calderon-decia-que-se-perderian-valdria-la-pena/}{con
la de Calderón} para justificar su estrategia: ``Costará vidas humanas
inocentes, pero vale la pena seguir adelante''.

Aunque se evitó un mal mayor, no hubo consenso en un país fracturado.
López Obrador apeló a un sentido humanitario de la justicia; sin
embargo, para muchos, mostró debilidad. La revista Proceso tituló así su
portada: ``Ustedes mandan''.

El tema es más complejo de lo que parece. En su espléndido artículo
``\href{https://perrocronico.com/la-tentacion-de-la-guerra/}{La
tentación de la guerra}'', Oswaldo Zavala, profesor en la Universidad de
la Ciudad de Nueva York y autor de \emph{Los cárteles no existen},
señala que el operativo de Culiacán fue ordenado por el Grupo de
Análisis de Información del Narcotráfico en posible coordinación con la
DEA y el gobierno de Sinaloa, del PRI, siguiendo la lógica de
intervención estadounidense pactada por Calderón desde 2008 en la
Iniciativa Mérida. Un mes antes de la captura, el 16 de septiembre,
Uttam Dhillon, entonces director interino de la DEA, estuvo en Culiacán
según reportes periodísticos. Zavala no descarta que los muchos errores
del operativo hayan sido provocados voluntariamente para exhibir al
gobierno. Carlos Demetrio Gaytán, subsecretario de la Defensa con
Calderón,
\href{https://aristeguinoticias.com/3010/mexico/cuestiona-general-decisiones-estrategicas-del-ejecutivo-que-no-han-convencido-a-todos/}{ha
dicho} que los militares se siente ``agraviados'' y ``ofendidos'' por
una política que los excluye. Las dudas que despierta la fallida captura
llevan a una pregunta: ¿a quién le interesa reactivar la ``guerra contra
las drogas''? ``La ocupación militar y el fenómeno de la
paramilitarización en México han sido un vehículo para el despojo y
apropiación ilegal de tierras que permiten el avance de megaproyectos de
extracción de recursos naturales como gas, petróleo y minería'',
responde Zavala. Ese núcleo complejo ayuda a entender que López Obrador
se haya apartado de un lance que le era ajeno en muchos sentidos.

Hace más de 2000 años, Sófocles contrastó en \emph{Antígona} el derecho
humanitario con las obligaciones ante el Estado. López Obrador evitó una
matanza y liberó a un enemigo poderoso. La opinión pública, versión
moderna del coro griego, juzgó que se trataba de un gesto de rendición,
del mismo modo en que, en marzo de 2020, condenó que el
\href{https://www.eluniversal.com.mx/nacion/politica/amlo-saluda-de-mano-mama-de-el-chapo-en-sinaloa}{presidente
saludara de mano} a la madre del ``Chapo'' y, en enero, se negara a
recibir a víctimas de la violencia encabezadas por el poeta Javier
Sicilia.

El presidente fue amonestado por el coro, pero Atenas lo respalda: su
aceptación en abril fue de
\href{https://politica.expansion.mx/mexico/2020/05/04/la-aprobacion-de-amlo-registra-un-alza-de-ocho-puntos-segun-encuesta}{68
por ciento}, según una encuesta de El Financiero.

\hypertarget{el-mensaje-de-los-huesos}{%
\subsection{El mensaje de los huesos}\label{el-mensaje-de-los-huesos}}

En 2010, Felipe Calderón ordenó que las osamentas de los héroes de la
independencia fueran exhumadas para recorrer el país en un cortejo
fúnebre. Los remanentes de los próceres integraron un \emph{tzompantli}
portátil, acorde con la hipervisibilidad del poder que el presidente
panista buscaba en su ``guerra contra el narcotráfico''.

López Obrador rompió en forma meritoria con esa conducta. Sin embargo,
mientras el narco avanza de manera ostensible, como lo hicieron las
huestes de Calderón, no parece haber una estrategia global que se le
oponga. El presidente habla todas las mañanas, pero tiene el talento
distractor de hablar siempre de ``otra cosa''. Su gramática ante la
violencia aún está por conjugarse.

Victor Hugo envió una carta a Benito Juárez pidiendo que perdonara la
vida del usurpador Maximiliano: ``Que el violador de sus principios sea
salvado por un principio. Que tenga esta dicha y esta vergüenza''. La
máxima afrenta al adversario consiste en no ser como él. El combate a la
violencia pasa por no ejercerla inútilmente.

La fuerza ética de ese planteamiento es evidente, pero no basta para
pacificar un país. En su torrente retórico, López Obrador no ha
formulado planes específicos para recuperar el tejido social. Ante las
más variadas interrogantes responde que actuará ``con honestidad'',
principio muy rara vez observado por sus predecesores, que, sin embargo,
no garantiza el control del territorio.

El mundo náhuatl rindió religiosa pleitesía a la muerte. Al mismo
tiempo, de manera rebelde, repudió esa sumisión en su poesía, cargada de
angustia y tristeza ante la fugacidad de todas las cosas. Un poema
anónimo pregunta: ``¿Es nuestra casa la tierra?'' y otro responde:
``Vivimos en tierra prestada''.

En 2021 se cumplirán 500 años de la caída de Tenochtitlan. En lo que
llega esa fecha, los arqueólogos liberan cráneos en el
\emph{tzompantli}.

Mientras tanto, México se convierte en una inmensa necrópolis, sembrada
de cráneos contemporáneos. Cada reliquia exige una razón. ¿Tiene sentido
la sangre derramada?

En el año más violento de nuestra historia reciente, resuena la voz del
poeta náhuatl: la vida es incierta en la tierra que nos fue prestada.

Juan Villoro es escritor y periodista. Su libro más reciente es \emph{El
vértigo horizontal. Una ciudad llamada México}.

Advertisement

\protect\hyperlink{after-bottom}{Continue reading the main story}

\hypertarget{site-index}{%
\subsection{Site Index}\label{site-index}}

\hypertarget{site-information-navigation}{%
\subsection{Site Information
Navigation}\label{site-information-navigation}}

\begin{itemize}
\tightlist
\item
  \href{https://help.nytimes.com/hc/en-us/articles/115014792127-Copyright-notice}{©~2020~The
  New York Times Company}
\end{itemize}

\begin{itemize}
\tightlist
\item
  \href{https://www.nytco.com/}{NYTCo}
\item
  \href{https://help.nytimes.com/hc/en-us/articles/115015385887-Contact-Us}{Contact
  Us}
\item
  \href{https://www.nytco.com/careers/}{Work with us}
\item
  \href{https://nytmediakit.com/}{Advertise}
\item
  \href{http://www.tbrandstudio.com/}{T Brand Studio}
\item
  \href{https://www.nytimes.com/privacy/cookie-policy\#how-do-i-manage-trackers}{Your
  Ad Choices}
\item
  \href{https://www.nytimes.com/privacy}{Privacy}
\item
  \href{https://help.nytimes.com/hc/en-us/articles/115014893428-Terms-of-service}{Terms
  of Service}
\item
  \href{https://help.nytimes.com/hc/en-us/articles/115014893968-Terms-of-sale}{Terms
  of Sale}
\item
  \href{https://spiderbites.nytimes.com}{Site Map}
\item
  \href{https://help.nytimes.com/hc/en-us}{Help}
\item
  \href{https://www.nytimes.com/subscription?campaignId=37WXW}{Subscriptions}
\end{itemize}
