Sections

SEARCH

\protect\hyperlink{site-content}{Skip to
content}\protect\hyperlink{site-index}{Skip to site index}

\href{https://www.nytimes.com/es/section/mundo}{Mundo}

\href{https://myaccount.nytimes.com/auth/login?response_type=cookie\&client_id=vi}{}

\href{https://www.nytimes.com/section/todayspaper}{Today's Paper}

\href{/es/section/mundo}{Mundo}\textbar{}China silencia a quienes
critican el brote del mortal virus

\url{https://nyti.ms/2tMyrTP}

\begin{itemize}
\item
\item
\item
\item
\item
\end{itemize}

\href{https://www.nytimes.com/es/spotlight/coronavirus?action=click\&pgtype=Article\&state=default\&region=TOP_BANNER\&context=storylines_menu}{El
brote de coronavirus}

\begin{itemize}
\tightlist
\item
  \href{https://www.nytimes.com/es/interactive/2020/espanol/mundo/coronavirus-en-estados-unidos.html?action=click\&pgtype=Article\&state=default\&region=TOP_BANNER\&context=storylines_menu}{Mapa
  y casos en EE. UU.}
\item
  \href{https://www.nytimes.com/es/2020/07/23/espanol/america-latina/bolivia-cloro-coronavirus-ivermectina.html?action=click\&pgtype=Article\&state=default\&region=TOP_BANNER\&context=storylines_menu}{Dióxido
  de cloro, ivermectina y más: ¿funcionan?}
\item
  \href{https://www.nytimes.com/es/interactive/2020/science/coronavirus-tratamientos-curas.html?action=click\&pgtype=Article\&state=default\&region=TOP_BANNER\&context=storylines_menu}{Fármacos
  y tratamientos}
\item
  \href{https://www.nytimes.com/es/2020/07/28/espanol/ciencia-y-tecnologia/anticuerpos-coronavirus-inmunidad.html?action=click\&pgtype=Article\&state=default\&region=TOP_BANNER\&context=storylines_menu}{Anticuerpos
  e inmunidad}
\item
  \href{https://www.nytimes.com/es/2020/04/29/espanol/estilos-de-vida/oximetro-para-que-sirve.html?action=click\&pgtype=Article\&state=default\&region=TOP_BANNER\&context=storylines_menu}{Oxímetros}
\end{itemize}

Advertisement

\protect\hyperlink{after-top}{Continue reading the main story}

Supported by

\protect\hyperlink{after-sponsor}{Continue reading the main story}

\hypertarget{china-silencia-a-quienes-critican-el-brote-del-mortal-virus}{%
\section{China silencia a quienes critican el brote del mortal
virus}\label{china-silencia-a-quienes-critican-el-brote-del-mortal-virus}}

Pekín ha respondido más rápido al coronavirus que al SARS, pero aún
acalla y castiga a quienes se desvían de la línea oficial.

\includegraphics{https://static01.nyt.com/images/2020/01/22/business/22newworld01-ES/merlin_167504928_7d74d205-3a0f-4098-9a0b-005f2ebb3004-articleLarge.jpg?quality=75\&auto=webp\&disable=upscale}

\href{https://www.nytimes.com/by/li-yuan}{\includegraphics{https://static01.nyt.com/images/2019/12/04/reader-center/author-li-yuan/author-li-yuan-thumbLarge.png}}

Por \href{https://www.nytimes.com/by/li-yuan}{Li Yuan}

\begin{itemize}
\item
  24 de enero de 2020
\item
  \begin{itemize}
  \item
  \item
  \item
  \item
  \item
  \end{itemize}
\end{itemize}

\href{https://www.nytimes.com/2020/01/22/health/virus-corona.html}{Read
in English}

El desastre del síndrome respiratorio agudo grave (SRAG, o SARS en
inglés) se suponía que arrastraría a China hacia una nueva era de
responsabilidad y transparencia. La enfermedad mortal repercutió por
todo el mundo hace 17 años gracias
\href{https://www.nytimes.com/2003/04/13/weekinreview/china-discovers-secrecy-is-expensive.html}{a
la complicidad del gobierno chino}que ocultó su propagación. Cuando la
magnitud del problema se hizo evidente, intelectuales, periodistas y
otros críticos de China hicieron esfuerzos para avergonzar a Pekín hasta
el punto de que tuvo que hablar francamente sobre ese problema.

``El SRAG ha sido el 11 de septiembre de nuestro país'', afirmó Xu
Zhiyuan\href{https://www.nytimes.com/2003/05/13/world/rude-awakening.html}{en
una entrevista}con The New York Times en 2003. En ese entonces, Zhiyuan
era un joven columnista y crítico acérrimo del manejo gubernamental del
SRAG. ``Nos ha obligado a prestarle atención al verdadero significado de
la globalización''.

Hoy, China enfrenta la
propagación\href{https://www.nytimes.com/2020/01/21/world/asia/china-coronavirus-wuhan.html?action=click\&module=Top\%20Stories\&pgtype=Homepage}{de
otra misteriosa enfermedad}, un coronavirus, que hasta el momento ha
matado a 17 personas e infectado a más de 570. Y si bien la respuesta de
Pekín ha mejorado en varios sentidos, ha sufrido retrocesos en otros.
Están censurando la crítica. Están deteniendo a personas por difundir lo
que califican como ``rumores''. Están suprimiendo la información que
consideran alarmante.

Aunque los censores gubernamentales están limpiando con ahínco el
internet chino, la comunidad en línea del país está registrando su
descontento y preocupación por la manera como Pekín ha manejado el nuevo
virus, que desde diciembre se ha propagado desde la ciudad de Wuhan a
otros
países,\href{https://www.nytimes.com/2020/01/21/health/cdc-coronavirus.html}{incluyendo
Estados Unidos}.

``Pensé que el SRAG obligaría a que China repensara su modelo de
gobernabilidad'', escribió Xu ---quien en la actualidad es el moderador
de un programa de entrevistas de video--- el 21 de enero en las redes
sociales, donde también publicó una captura de pantalla de su cita del
2003 en el Times. ``Fui demasiado ingenuo''.

\href{https://www.nytimes.com/interactive/2020/01/21/world/asia/china-coronavirus-maps.html}{}

\includegraphics{https://static01.nyt.com/images/2020/01/31/us/china-wuhan-coronavirus-promo-1579641872730/china-wuhan-coronavirus-promo-1579641872730-articleLarge-v21.jpg}

\hypertarget{wuhan-coronavirus-map-tracking-the-spread-of-the-outbreak}{%
\subsection{Wuhan Coronavirus Map: Tracking the Spread of the
Outbreak}\label{wuhan-coronavirus-map-tracking-the-spread-of-the-outbreak}}

The virus has sickened tens of thousands of people in China and a number
of other countries.

China ha mejorado de muchas maneras desde la epidemia del SRAG. Su
economía ha crecido ocho veces. Ha construido más rascacielos, metros y
líneas de alta velocidad que cualquier otro país. Sus empresas de
tecnología están a la par de los gigantes de Silicon Valley. Su
burocracia, más receptiva, hace que más personas reciban atención
médica, servicios sociales e incluso mejoras en su calidad de vida con,
por ejemplo, acceso a nuevos parques.

En lo relacionado al manejo de enfermedades, el sistema de salud pública
ha mejorado considerablemente. Wuhan, el epicentro del brote, también
aloja
\href{https://twitter.com/ShanghaiEye/status/1093160842638446592}{uno de
los laboratorios}de investigación de enfermedades epidémicas más
avanzados del mundo.

Estas mejoras han venido con un costo. El gobierno ha endurecido su
control sobre
\href{https://www.nytimes.com/topic/destination/internet-censorship-in-china}{internet},
los
\href{https://www.nytimes.com/2019/07/12/world/asia/china-journalists-crackdown.html}{medios}
y la
\href{https://www.nytimes.com/2017/07/25/magazine/the-lonely-crusade-of-chinas-human-rights-lawyers.html}{sociedad
civil}. Tiene más dinero y mayor habilidad para controlar el flujo de
información por todo el país.

Como consecuencia, muchos de los
\href{https://www.nytimes.com/2004/01/07/international/asia/7-at-chinese-paper-held-after-articles-on-new-sars-case.html}{medios
de comunicación,} grupos de defensa,
\href{https://www.nytimes.com/2004/07/21/world/china-releases-the-sars-whistle-blower.html}{activistas}
y otras personas que obligaron al gobierno a rendir cuentas en el 2003,
han sido silenciados o marginados.

\includegraphics{https://static01.nyt.com/images/2020/01/22/business/22newworld02-ES/merlin_10781530_adf140ad-43a3-4a13-8917-7207fa4b7d20-articleLarge.jpg?quality=75\&auto=webp\&disable=upscale}

``El sistema no es exitoso en tanto ha destruido a personas con
integridad, instituciones con credibilidad y a una sociedad capaz de
narrar sus propias historias'' dijo Xu en redes sociales. ``Lo que queda
es un poder arrogante, un montón de información desordenada y muchos
individuos frágiles, aislados y molestos''.

Incluso mientras el nuevo virus se propagaba en Wuhan, el gobierno se
dio a la tarea de guardar las apariencias.

El primer caso fue reportado el 8 de diciembre. Mientras la enfermedad
se propagaba, funcionarios de Wuhan insistían en que era tratable y que
estaba controlada. La policía interrogó a ocho personas que publicaron
información sobre el virus en las redes sociales, afirmando que habían
difundido ``rumores''.

El 18 de enero, dos días antes de que Wuhan le informara al planeta
sobre la gravedad del brote, la ciudad organizó
\href{https://m.weibo.cn/status/4462935805605012?}{un banquete
comunitario} al que asistieron más de 40.000 familias, para lograr así
que la localidad pudiera competir por el récord mundial de más platos
servidos en un evento. El día en que Wuhan dio la noticia al mundo,
también anunció que estaba repartiendo 200.000 entradas gratis a los
residentes para las actividades festivas durante las vacaciones del Año
Nuevo lunar, las cuales comienzan el 25 de enero.

El gobierno central respaldó a los funcionarios de Wuhan. Wang Guangfa,
un importante experto gubernamental en enfermedades respiratorias, le
dijo el 10 de enero al canal del estado Televisión Central de China, que
la neumonía de Wuhan estaba ``bajo control'' y que era principalmente
una ``condición leve''. Once días más tarde, Wang
\href{https://www.weibo.com/2656274875/IqCSrAXel?type=comment\#_rnd1579700865617}{le
confirmó} a los medios chinos que al parecer se había contagiado del
virus durante una inspección en Wuhan.

Reconocer una epidemia puede tomar tiempo. China no es el primer
gobierno al que una enfermedad lo toma por sorpresa.

Sin embargo, las decisiones tomadas por los funcionarios del gobierno
tuvieron un impacto en un importante eje comercial y de transporte.
Wuhan es una ciudad de 11 millones de habitantes, entre ellos casi 1
millón de estudiantes universitarios de todo el país. Para cuando se
reveló la gravedad del brote, ya había iniciado la temporada de 40 días
de viajes del Año Nuevo lunar, en el que la población china toma un
estimado combinado de
\href{http://www.xinhuanet.com/english/2020-01/10/c_138694239.htm}{3000
millones} de viajes.

La población podría haber tomado diferentes decisiones si los sitios
webs y los titulares hubieran descrito esta creciente preocupación.
Pero, en vez de eso, viajaron. El 21 de enero, los cinco casos
confirmados en Pekín fueron de personas que en enero viajaron a Wuhan
por negocios, estudios o placer.

Image

El 22 de enero unos paramédicos transportaron a un hombre que se creía
era el primer paciente de coronavirus de Wuhan en Hong Kong.Credit...Lam
Yik Fei para The New York Times

Hasta hace una semana, algunas personas en China lo llamaban el ``virus
patriótico''. Aparecieron casos en Hong Kong, Tailandia, Vietnam, Japón
y en otros sitios en Asia. Ninguna otra ciudad china excepto Wuhan
reportó casos de infección. No fue hasta que los medios de noticias de
Hong Kong reportaron durante el fin de semana que el virus había sido
detectado en otras ciudades, que los funcionarios de muchos otros
lugares dieron la cara.

Algunos críticos ven paralelismos con el SRAG. En 2003, el periódico
cantonés Southern Metropolis Daily fue el primero en reportar el brote
de SRAG. Un médico militar, Jiang Yanyong, contó lo que sabía. En ese
momento fue que los funcionarios empezaron a actuar.

``La manera como este virus entró a la opinión pública es la misma que
la del SRAG hace 17 años'', dijo Rose Luqiu, una profesora de periodismo
que cubrió todo lo relacionado al SRAG como reportera de Phoenix
Television en Hong Kong.

Muchas de esas voces valientes del 2003 ya no están. Como casi todos los
medios de comunicación chinos que estaban activos en la década de 1990 y
2000, el Southern Metropolis Daily ha perdido su libertad de buscar y
difundir información con el fin de que los gobiernos locales, o incluso
Pekín, tengan que rendir cuentas. Solo un puñado de medios de noticias
de China continental están cubriendo la crisis actual de manera crítica,
y solo usando un tono analítico.

En 2003, la televisora Phoenix Television envió a Luqiu, entonces una
reportera estrella, de Irak a Pekín para que informara sobre el SRAG.
Siguió durante una semana al nuevo alcalde de Pekín, Wang Qishan, para
cubrir el modo en que el gobierno lidiaba con la crisis. Wang después se
convirtió en vicepresidente de China.

Ese tipo de apertura ahora sería inimaginable. La semana pasada, cuando
un grupo de periodistas de Hong Kong visitaron el hospital de Wuhan que
más pacientes de coronavirus acogió, la policía
\href{http://news.tvb.com/local/5e1d7dc434b0315b57158f6a/\%E6\%9C\%AC\%E6\%B8\%AF\%E5\%A4\%9A\%E5\%90\%8D\%E8\%A8\%98\%E8\%80\%85\%E6\%AD\%A6\%E6\%BC\%A2\%E9\%86\%AB\%E9\%99\%A2\%E6\%8E\%A1\%E8\%A8\%AA\%E6\%99\%82\%E8\%A2\%AB\%E6\%89\%A3\%E6\%9F\%A5-\%E8\%A6\%81\%E6\%B1\%82\%E5\%88\%AA\%E9\%99\%A4\%E6\%8B\%8D\%E6\%94\%9D\%E7\%B4\%A0\%E6\%9D\%90}{los
detuvo} durante horas. Se les pidió que borrasen sus grabaciones
televisivas y que entregaran sus teléfonos y cámaras para ser
inspeccionados.

El martes, Luqiu escribió un artículo para qq.com, el sitio de noticias
que pertenece al gigante de internet Tencent, sobre las medidas que el
gobierno hongkongnés ha tomado para enfrentar el virus. El artículo fue
eliminado 10 horas más tarde.

El doctor Jiang, el médico militar que en 2003 se convirtió en un
denunciante, se encuentra bajo arresto domiciliario intermitente y tiene
\href{https://www.nytimes.com/2007/07/13/world/asia/13doctor.html}{prohibido}
visitar Estados Unidos, donde se le va a entregar un premio de derechos
humanos. También se le muestra como mal ejemplo. En una pregunta de
opción múltiple en un examen de una escuela preparatoria en 2017 se
cuestiona su decisión. La respuesta correcta es la B: Jiang estuvo
equivocado porque dañó los intereses de la nación, la sociedad y la
comunidad y debería estar sujeto a castigo legal.

Image

Un trabajador rocía desinfectante en la estación de tren de Hankou en
Wuhan el 22 de enero.Credit...Agence France-Presse --- Getty Images

La difusión de informaciones en China ha mejorado de muchas maneras
desde el SRAG. Esta vez, el gobierno admitió el problema mucho más
rápido. Los funcionarios de Pekín han mostrado una determinación por ser
más transparentes. Un importante comité del partido afirmó el 21 de
enero que no toleraría ninguna iniciativa que intentara esconder las
infecciones.

``Todo aquel que de manera deliberada obstruya u oculte información
debido a sus propios intereses, quedará eternamente clavado al pilar de
la vergüenza histórica'', sentenció el comité en una publicación de
WeChat. La publicación fue posteriormente eliminada.

Pero cuando el gobierno es la única fuente de información, los consejos
sabios y las pistas valiosas pueden perderse. Un departamento policial
en la provincia oriental de Shandong publicó el 22 de enero en la red
social Weibo (similar a Twitter), que había detenido a cuatro residentes
que difundieron rumores de que había un presunto paciente de coronavirus
en el distrito. En ese ambiente hostil, otros no se atreven a hablar.

``Las autoridades están enviando la señal de que solo las agencias
gubernamentales pueden hablar sobre la epidemia'', escribió en su blog
personal Yu Ping, antiguo periodista del Southern Metropolis Daily.
``Todos los demás simplemente deben callarse''.

``No es divulgación pública'', añadió Yu. ``Es un crudo monopolio de la
información''.

Li Yuan escribe la columna Nuevo Nuevo Mundo, que se centra en la
intersección de la tecnología, los negocios y la política en China y en
toda Asia. \href{https://twitter.com/liyuan6}{@liyuan6}

\begin{center}\rule{0.5\linewidth}{\linethickness}\end{center}

Advertisement

\protect\hyperlink{after-bottom}{Continue reading the main story}

\hypertarget{site-index}{%
\subsection{Site Index}\label{site-index}}

\hypertarget{site-information-navigation}{%
\subsection{Site Information
Navigation}\label{site-information-navigation}}

\begin{itemize}
\tightlist
\item
  \href{https://help.nytimes.com/hc/en-us/articles/115014792127-Copyright-notice}{©~2020~The
  New York Times Company}
\end{itemize}

\begin{itemize}
\tightlist
\item
  \href{https://www.nytco.com/}{NYTCo}
\item
  \href{https://help.nytimes.com/hc/en-us/articles/115015385887-Contact-Us}{Contact
  Us}
\item
  \href{https://www.nytco.com/careers/}{Work with us}
\item
  \href{https://nytmediakit.com/}{Advertise}
\item
  \href{http://www.tbrandstudio.com/}{T Brand Studio}
\item
  \href{https://www.nytimes.com/privacy/cookie-policy\#how-do-i-manage-trackers}{Your
  Ad Choices}
\item
  \href{https://www.nytimes.com/privacy}{Privacy}
\item
  \href{https://help.nytimes.com/hc/en-us/articles/115014893428-Terms-of-service}{Terms
  of Service}
\item
  \href{https://help.nytimes.com/hc/en-us/articles/115014893968-Terms-of-sale}{Terms
  of Sale}
\item
  \href{https://spiderbites.nytimes.com}{Site Map}
\item
  \href{https://help.nytimes.com/hc/en-us}{Help}
\item
  \href{https://www.nytimes.com/subscription?campaignId=37WXW}{Subscriptions}
\end{itemize}
