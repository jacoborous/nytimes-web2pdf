Sections

SEARCH

\protect\hyperlink{site-content}{Skip to
content}\protect\hyperlink{site-index}{Skip to site index}

\href{https://www.nytimes.com/es/section/mundo}{Mundo}

\href{https://myaccount.nytimes.com/auth/login?response_type=cookie\&client_id=vi}{}

\href{https://www.nytimes.com/section/todayspaper}{Today's Paper}

\href{/es/section/mundo}{Mundo}\textbar{}Trump sostiene una Biblia para
las cámaras: los expertos opinan del gesto

\url{https://nyti.ms/2Xy3ChP}

\begin{itemize}
\item
\item
\item
\item
\item
\end{itemize}

Advertisement

\protect\hyperlink{after-top}{Continue reading the main story}

Supported by

\protect\hyperlink{after-sponsor}{Continue reading the main story}

Análisis

\hypertarget{trump-sostiene-una-biblia-para-las-cuxe1maras-los-expertos-opinan-del-gesto}{%
\section{Trump sostiene una Biblia para las cámaras: los expertos opinan
del
gesto}\label{trump-sostiene-una-biblia-para-las-cuxe1maras-los-expertos-opinan-del-gesto}}

Los seguidores incondicionales del presidente apreciaron el mensaje.
Otros lo interpretaron como algo más alarmante.

\includegraphics{https://static01.nyt.com/images/2020/06/02/us/politics/04TRUMP-BIBLE-ES/merlin_173085330_9e87b8cc-bf17-451b-bfeb-6c00f96fcc1a-articleLarge.jpg?quality=75\&auto=webp\&disable=upscale}

\href{https://www.nytimes.com/by/matt-flegenheimer}{\includegraphics{https://static01.nyt.com/images/2018/10/02/multimedia/author-matt-flegenheimer/author-matt-flegenheimer-thumbLarge.png}}

Por \href{https://www.nytimes.com/by/matt-flegenheimer}{Matt
Flegenheimer}

\begin{itemize}
\item
  4 de junio de 2020
\item
  \begin{itemize}
  \item
  \item
  \item
  \item
  \item
  \end{itemize}
\end{itemize}

\href{https://www.nytimes.com/2020/06/02/us/politics/trump-holds-bible-photo.html}{Read
in English}

\href{https://www.nytimes.com/newsletters/el-times}{Regístrate para
recibir nuestro boletín} por correo con lo mejor de The New York Times.

\begin{center}\rule{0.5\linewidth}{\linethickness}\end{center}

Si el líder de cualquier otra nación, en cualquier otra capital en
llamas, le ordenara a la policía y a las agencias de seguridad ``dominar
las calles'' en contra de los manifestantes, y luego atravesara a pie un
parque, donde agentes gubernamentales retiran de su camino a los
manifestantes por la fuerza, para que llegue hasta una iglesia y
sostenga en alto una Biblia frente a las cámaras como una especie de
trofeo de campeonato\ldots{} ¿Cuál habría sido la opinión de Estados
Unidos?

``Si viéramos esto en otro país, estaríamos muy preocupados y
debatiríamos sobre las consecuencias que tiene el comportamiento de esos
Estados en la política exterior'', dijo Kori Schake, quien ha trabajado
en el Pentágono y como asesora política del partido Republicano.

Según algunos opositores y académicos, ha llegado el momento de tener
esa conversación.

Desde los primeros días de este gobierno y debido a la tendencia del
mandatario a saltarse las normas, los críticos inquietos, los académicos
y los expertos en política exterior han vigilado las señales de la vena
antidemocrática del presidente Donald Trump. Y el ejercicio no siempre
ha requerido una investigación exhaustiva.

Sin embargo, da la impresión de que la respuesta de la Casa Blanca
frente a los actuales signos de trauma nacional ha tomado otro rumbo, lo
que ha producido el tipo de escenas que algunos pesimistas habían
profetizado desde hacía algún tiempo y que han añadido a las crecientes
crisis sociales y de salud pública una preocupación sobre el estado de
la democracia estadounidense.

El secretario de Defensa de Trump, Mark T. Esper,
\href{https://www.nytimes.com/2020/06/01/us/politics/police-military-gear.html}{les
dijo a los gobernadores} que ``dominen el espacio de batalla'' en contra
de los manifestantes. Un helicóptero Black Hawk sobrevoló el barrio
chino de la ciudad a tan baja altura que rompió las ramas de los árboles
y los letreros de los costados de los edificios, una
\href{https://www.nytimes.com/2020/06/02/us/politics/military-national-guard-trump-protests.html}{maniobra
de demostración de fuerza} que a menudo se usa en las zonas de combate
para ahuyentar a los insurgentes.

Y presidiendo todo estaba el hombre que amenazó con enviar al ejército
estadounidense a los estados donde los gobernadores no pudieran
restaurar la calma, y que etiquetó de ``organizadores'' del terror a los
manifestantes que usaron la violencia para llamar la atención sobre la
brutalidad policial en contra de las personas negras.

Aunque, hasta ahora, el episodio se ha procesado siguiendo las líneas
ideológicas tradicionales, las reacciones también se han mezclado con
pasiones más urgentes que coinciden con estos tiempos.

Muchos de los admiradores de Trump han apoyado sus promesas de detener
el caos, alabando la imagen religiosa que utilizó, de una manera
bastante evidente, como oportunidad para tomarse una foto.

``No cabe duda de que todos los creyentes con los que he platicado
aprecian lo que hizo el presidente y el mensaje que estaba enviando'',
dijo Robert Jeffress, el pastor de First Baptist Dallas y un
incondicional simpatizante evangélico de Trump. ``Creo que será uno de
esos momentos históricos de su presidencia, en especial cuando se
compare con las noches de violencia que ocurrieron en todo el país''.

\includegraphics{https://static01.nyt.com/images/2020/06/02/us/politics/04TRUMP-BIBLE-ES-02/merlin_173090928_e945f848-0738-4d73-90e5-286991ee47c9-articleLarge.jpg?quality=75\&auto=webp\&disable=upscale}

Mientras tanto, algunos demócratas están utilizando un término al que
han recurrido en algunas otras ocasiones a lo largo de estos tres años y
medio, pero tal vez nunca con tanta frecuencia y convicción.

``Las palabras de un dictador'', dijo la senadora de California Kamala
Harris.

``Se comporta como un dictador'', tuiteó el senador de Massachusetts Ed
Markey.

``Solo nos queda cerrar los ojos y esperar que de alguna manera no
llegará tan lejos, aunque acaba de ordenar que el gobierno federal les
dispare a manifestantes inocentes'', señaló en una entrevista el
representante Ruben Gallego, de Arizona. ``Debemos aceptar el hecho de
que, si se le da la oportunidad, este presidente intentará ser un
dictador''.

Gallego, un veterano de la guerra de Irak, predijo que los líderes del
ejército pronto iban a tener que tomar una decisión: ``Tendrán que
decirle que no al presidente y negarse a seguir órdenes ilegales''.

El 2 de junio, el almirante Mike Mullen, otrora jefe del Estado Mayor
Conjunto, pareció hacer eco de esta ansiedad en
\href{https://www.theatlantic.com/ideas/archive/2020/06/american-cities-are-not-battlespaces/612553/}{un
artículo para The Atlantic}. Aunque sostiene que confía en que los
oficiales uniformados obedecerán las órdenes legales, escribió que tenía
menos fe ``en la congruencia de las órdenes que van a recibir de este
comandante en jefe''.

Expertos en sistemas democráticos han sido cuidadosos en distinguir
ciertos rasgos y datos conspicuos ---los instintos de quiebra de límites
de Trump, su bravuconería inveterada, su afición por algunas frases
asociadas con caudillos--- de los desafíos más legítimos a las
instituciones e ideales del país.

Ellos señalan que los eventos recientes son ampliamente consistentes con
el espíritu del mandato de Trump hasta la fecha, gran parte de lo cual
han encontrado problemático: tenemos un presidente que ya había
despedido a un director del FBI que encabezaba una investigación sobre
su campaña; que instó a una potencia extranjera a investigar a un rival
político; que eliminó a los inspectores generales encargados de
supervisar su gobierno; que dirigió una cruzada pública contra su propio
Departamento de Justicia para que retirase los cargos contra su primer
asesor de seguridad nacional, que ya se había declarado culpable.

Yascha Mounk, un profesor titular de la Universidad Johns Hopkins que ha
escrito extensamente sobre las amenazas para la democracia liberal, dijo
que era más fácil comprender a Trump como ``un populista autoritario''.
Según Mounk, en la concepción de autoridad de Trump, ``esto quiere decir
que él y solo él representa de verdad al pueblo. Y cualquiera que no
piense igual que ellos, cualquiera que lo critique, en virtud de ese
hecho es un enemigo del pueblo''.

Proyectar el poderío militar como si fuese un poder político personal es
un ejemplo de eso, sugirió Mounk.

``Cuando hizo su juramento al entrar al cargo, no creo que Donald Trump
haya pensado: `Quiero ser un dictador'. No creo que hoy quiera ser un
dictador'', agregó. ``Pero no creo que sea descabellado preocuparse de
que, en caso de ser reelegido, el sistema democrático de Estados Unidos
esté en grave peligro''.

Invocar la religión como lo hizo Trump, en el contexto de la fuerza para
hacer cumplir la ley, fue visto por varios académicos como un rasgo
especialmente notable.

Katherine Stewart, autora que se ha enfocado en la derecha cristiana,
dijo que la visita del lunes a la iglesia le recordó a líderes políticos
como el primer ministro de Hungría, Viktor Orbán, y el presidente de
Turquía, Recep Tayyip Erdogan.

``Trump no cita nada de la Biblia. En realidad, solo la usa como un
símbolo puro de identidad partidista'', dijo Stewart. Y agregó: ``El
autoritarismo a menudo viene velado de religión''.

Schake, la directora de estudios extranjeros y política de defensa en el
American Enterprise Institute, sonó un poco más optimista. Las
advertencias sobre una recaída en el autoritarismo no fueron tan
alarmistas, explicó, ``pero aún no comparto esa inquietud''.

``Sigo siendo optimista al pensar que el Congreso, incluidos los
republicanos del Congreso, verá que le hemos dado demasiada libertad al
jefe del ejecutivo de este país'', dijo.

Hasta la fecha, hay pocos indicios de esto; y pareciera poco incentivo
político para que los líderes del partido condenen una figura que sigue
siendo muy popular con su base (y cuya conducta violenta era famosa
desde antes de su elección).

Esta semana, la mayoría de los legisladores republicanos se ha rehusado
a criticar a Trump, aunque un puñado aceptó en público que no está de
acuerdo con su comportamiento.

El martes, el senador Ben Sasse de Nebraska se declaró ``en contra de
desalojar una protesta pacífica para aprovechar la oportunidad de
tomarse una fotografía que usa la palabra de Dios como una herramienta
política''. El gobernador de Massachusetts, Charlie Baker, quien a
menudo ha estado dispuesto a criticar a Trump,
\href{https://edition.cnn.com/2020/06/01/politics/charlie-baker-donald-trump-governor-call/index.html}{lamentó}
las ``palabras incendiarias'' del presidente. Y el senador Tim Scott de
Carolina del Sur, el republicano negro más prominente de la capital,
criticó la decisión de dispersar violentamente a los manifestantes de la
zona para tomar la fotografía presidencial.

Hasta ahora, Trump parece simplemente insumiso. Pasó buena parte de la
mañana del martes tuiteando sobre el desorden de Nueva York: les ordenó
a los líderes locales que llamaran a la ``GUARDIA NACIONAL'' e insistió
en que una ``MAYORÍA SILENCIOSA'' seguía estando de su lado.

Y calificó las acciones en Washington como un éxito digno de emular.

``DC no tuvo problemas anoche'', escribió el presidente. ``Muchas
detenciones. Gran trabajo hecho por todos. Fuerza abrumadora.
Dominación''.

Matt Flegenheimer es un reportero que cubre política estadounidense.
Empezó en 2011 en el Times en la sección Metro en 2011, donde cubría
transporte, la alcaldía de Nueva York y las campañas electorales.
\href{https://twitter.com/mattfleg}{@mattfleg}

\begin{center}\rule{0.5\linewidth}{\linethickness}\end{center}

Advertisement

\protect\hyperlink{after-bottom}{Continue reading the main story}

\hypertarget{site-index}{%
\subsection{Site Index}\label{site-index}}

\hypertarget{site-information-navigation}{%
\subsection{Site Information
Navigation}\label{site-information-navigation}}

\begin{itemize}
\tightlist
\item
  \href{https://help.nytimes.com/hc/en-us/articles/115014792127-Copyright-notice}{©~2020~The
  New York Times Company}
\end{itemize}

\begin{itemize}
\tightlist
\item
  \href{https://www.nytco.com/}{NYTCo}
\item
  \href{https://help.nytimes.com/hc/en-us/articles/115015385887-Contact-Us}{Contact
  Us}
\item
  \href{https://www.nytco.com/careers/}{Work with us}
\item
  \href{https://nytmediakit.com/}{Advertise}
\item
  \href{http://www.tbrandstudio.com/}{T Brand Studio}
\item
  \href{https://www.nytimes.com/privacy/cookie-policy\#how-do-i-manage-trackers}{Your
  Ad Choices}
\item
  \href{https://www.nytimes.com/privacy}{Privacy}
\item
  \href{https://help.nytimes.com/hc/en-us/articles/115014893428-Terms-of-service}{Terms
  of Service}
\item
  \href{https://help.nytimes.com/hc/en-us/articles/115014893968-Terms-of-sale}{Terms
  of Sale}
\item
  \href{https://spiderbites.nytimes.com}{Site Map}
\item
  \href{https://help.nytimes.com/hc/en-us}{Help}
\item
  \href{https://www.nytimes.com/subscription?campaignId=37WXW}{Subscriptions}
\end{itemize}
