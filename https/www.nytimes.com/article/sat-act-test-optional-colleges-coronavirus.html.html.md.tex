Sections

SEARCH

\protect\hyperlink{site-content}{Skip to
content}\protect\hyperlink{site-index}{Skip to site index}

\href{https://www.nytimes.com/section/us}{U.S.}

\href{https://myaccount.nytimes.com/auth/login?response_type=cookie\&client_id=vi}{}

\href{https://www.nytimes.com/section/todayspaper}{Today's Paper}

\href{/section/us}{U.S.}\textbar{}More Colleges Are Waiving SAT and ACT
Requirements

\url{https://nyti.ms/2VYpPDE}

\begin{itemize}
\item
\item
\item
\item
\item
\end{itemize}

\href{https://www.nytimes.com/news-event/coronavirus?action=click\&pgtype=Article\&state=default\&region=TOP_BANNER\&context=storylines_menu}{The
Coronavirus Outbreak}

\begin{itemize}
\tightlist
\item
  live\href{https://www.nytimes.com/2020/08/02/world/coronavirus-updates.html?action=click\&pgtype=Article\&state=default\&region=TOP_BANNER\&context=storylines_menu}{Latest
  Updates}
\item
  \href{https://www.nytimes.com/interactive/2020/us/coronavirus-us-cases.html?action=click\&pgtype=Article\&state=default\&region=TOP_BANNER\&context=storylines_menu}{Maps
  and Cases}
\item
  \href{https://www.nytimes.com/interactive/2020/science/coronavirus-vaccine-tracker.html?action=click\&pgtype=Article\&state=default\&region=TOP_BANNER\&context=storylines_menu}{Vaccine
  Tracker}
\item
  \href{https://www.nytimes.com/interactive/2020/07/29/us/schools-reopening-coronavirus.html?action=click\&pgtype=Article\&state=default\&region=TOP_BANNER\&context=storylines_menu}{What
  School May Look Like}
\item
  \href{https://www.nytimes.com/live/2020/07/31/business/stock-market-today-coronavirus?action=click\&pgtype=Article\&state=default\&region=TOP_BANNER\&context=storylines_menu}{Economy}
\end{itemize}

Advertisement

\protect\hyperlink{after-top}{Continue reading the main story}

Supported by

\protect\hyperlink{after-sponsor}{Continue reading the main story}

\hypertarget{more-colleges-are-waiving-sat-and-act-requirements}{%
\section{More Colleges Are Waiving SAT and ACT
Requirements}\label{more-colleges-are-waiving-sat-and-act-requirements}}

With high school disrupted, a growing number of schools --- including
Harvard and Cornell --- are waiving standardized testing requirements
for 2021 applicants.

\includegraphics{https://static01.nyt.com/images/2020/04/14/us/15xp-virus-satact1/14xp-virus-satact1-articleLarge.jpg?quality=75\&auto=webp\&disable=upscale}

\href{https://www.nytimes.com/by/neil-vigdor}{\includegraphics{https://static01.nyt.com/images/2019/07/25/reader-center/author-neil-vigdor/author-neil-vigdor-thumbLarge.png}}\href{https://www.nytimes.com/by/johnny-diaz}{\includegraphics{https://static01.nyt.com/images/2019/11/05/reader-center/author-johnny-diaz/author-johnny-diaz-thumbLarge.png}}

By \href{https://www.nytimes.com/by/neil-vigdor}{Neil Vigdor} and
\href{https://www.nytimes.com/by/johnny-diaz}{Johnny Diaz}

\begin{itemize}
\item
  May 21, 2020
\item
  \begin{itemize}
  \item
  \item
  \item
  \item
  \item
  \end{itemize}
\end{itemize}

They are a rite of passage as well as anxiety-inducing letters for
millions of students: the
\href{https://www.nytimes.com/2020/06/02/us/at-home-sat-coronavirus.html}{SAT}
and ACT.

But with many high schools closed or teaching remotely for the rest of
the academic year, a growing number of colleges and universities are
waiving standardized test requirements amid the coronavirus pandemic.

Some went further than that, with leaders of the University of
California system voting on May 21 to
\href{https://www.nytimes.com/2020/05/21/us/university-california-sat-act.html}{phase
out the SAT and ACT as an admissions requirement} over the next four
years.

The test will be optional through 2024, when the system's 10 schools
could develop its own admissions test. The system had already eliminated
the standardized test requirement because of the outbreak of the virus.

Proponents of the change said the university system would still attract
top students and that too much weight had been given to the tests.

``No one has ever asked me, `What was your SAT?''' said Lark Park, a
regent for the University of California system.

At least two Ivy League schools adjusted their policies for applicants.
On April 22, Cornell University suspended its testing requirements,
saying in a
\href{https://admissions.cornell.edu/news/cornell-university-suspends-actsat-testing-requirement-2021-applicants}{statement}
that ``due to this extraordinary circumstance,'' students could submit
applications without ACT or SAT exam results starting in August 2021.

\hypertarget{latest-updates-global-coronavirus-outbreak}{%
\section{\texorpdfstring{\href{https://www.nytimes.com/2020/08/01/world/coronavirus-covid-19.html?action=click\&pgtype=Article\&state=default\&region=MAIN_CONTENT_1\&context=storylines_live_updates}{Latest
Updates: Global Coronavirus
Outbreak}}{Latest Updates: Global Coronavirus Outbreak}}\label{latest-updates-global-coronavirus-outbreak}}

Updated 2020-08-02T17:52:35.962Z

\begin{itemize}
\tightlist
\item
  \href{https://www.nytimes.com/2020/08/01/world/coronavirus-covid-19.html?action=click\&pgtype=Article\&state=default\&region=MAIN_CONTENT_1\&context=storylines_live_updates\#link-34047410}{The
  U.S. reels as July cases more than double the total of any other
  month.}
\item
  \href{https://www.nytimes.com/2020/08/01/world/coronavirus-covid-19.html?action=click\&pgtype=Article\&state=default\&region=MAIN_CONTENT_1\&context=storylines_live_updates\#link-780ec966}{Top
  U.S. officials work to break an impasse over the federal jobless
  benefit.}
\item
  \href{https://www.nytimes.com/2020/08/01/world/coronavirus-covid-19.html?action=click\&pgtype=Article\&state=default\&region=MAIN_CONTENT_1\&context=storylines_live_updates\#link-2bc8948}{Its
  outbreak untamed, Melbourne goes into even greater lockdown.}
\end{itemize}

\href{https://www.nytimes.com/2020/08/01/world/coronavirus-covid-19.html?action=click\&pgtype=Article\&state=default\&region=MAIN_CONTENT_1\&context=storylines_live_updates}{See
more updates}

More live coverage:
\href{https://www.nytimes.com/live/2020/07/31/business/stock-market-today-coronavirus?action=click\&pgtype=Article\&state=default\&region=MAIN_CONTENT_1\&context=storylines_live_updates}{Markets}

``We can't pre-define in absolute, comprehensive terms what economic or
personal disruptions will look like,'' the statement added. ``We don't
plan to require any students to justify their reasons for not submitting
test results.''

And Harvard College recently told applying high school juniors that they
would not face penalties if they were unable to submit SAT subject test
results and Advanced Placement test scores.

``We know that there are fewer opportunities to take the SAT or ACT
given the cancellations to date,'' Harvard
\href{https://college.harvard.edu/about/news-announcements/special-message-high-school-juniors-applying-harvard}{said
in a statement.} ``You will not be disadvantaged in any way if you do
not submit subject tests.''

Two other Ivy League schools, Princeton and the University of
Pennsylvania, do not require students take SAT subject tests for
admission. In a recent
\href{https://admission.princeton.edu/how-apply/standardized-testing/statement-applicants-princetons-class-2025}{message}to
potential applicants, Princeton's dean of admission, Karen Richardson
said, ``while our policy has long been that SAT subject tests are
recommended but not required, now seems the appropriate time to
reiterate that applicants who do not submit subject tests will not be
disadvantaged in our process.'' She added, **``**SAT or ACT test scores
are only one part of our holistic review.''

So far, Dartmouth College, Yale and Brown University still require them,
as do Stanford and highly selective colleges on the West Coast. But many
of the schools that compete with those big names are moving ahead to
make the tests optional.

More than two dozen institutions --- including highly selective liberal
arts colleges like
\href{https://communications.williams.edu/news-releases/4_6_2020_test_optional/}{Williams}
and
\href{https://www.amherst.edu/news/press-releases/node/768482}{Amherst},
both in Massachusetts --- announced this spring that the tests would be
optional for applicants seeking to enroll in 2021.

The easing of test requirements comes as education reform groups have
criticized the SAT and ACT, which they contend give wealthier students
an advantage because their families can afford expensive prep exams and
coaches. The nonprofit National Center for Fair and Open Testing keeps a
growing list of hundreds of higher education institutions already moving
away from requiring the tests.

Williams College, which U.S. News \& World Report ranked as the
\href{https://www.usnews.com/best-colleges/williams-college-2229}{top
national liberal arts college in its annual guide} this year, took the
step last week and said it was waiving the requirement for both
first-year and transfer applicants.

``This is an unprecedented moment for students and colleges alike and it
calls for a change to the usual way of doing things,'' Liz Creighton,
the dean of admission and financial aid at Williams, said in a statement
on April 6.

The same day that Williams made the exams optional, so did its rival,
Amherst.

``If future test dates are not available in students' local areas or if
students are worried about how to test in a socially distant manner, we
do not want them to feel pressure to put themselves in situations that
are not in their best interest,'' Matt McGann, the dean of admission and
financial aid at Amherst, said in a statement. ``And we wanted to
provide clarity and ease anxiety as soon as we could.''

\href{https://www.nytimes.com/news-event/coronavirus?action=click\&pgtype=Article\&state=default\&region=MAIN_CONTENT_3\&context=storylines_faq}{}

\hypertarget{the-coronavirus-outbreak-}{%
\subsubsection{The Coronavirus Outbreak
›}\label{the-coronavirus-outbreak-}}

\hypertarget{frequently-asked-questions}{%
\paragraph{Frequently Asked
Questions}\label{frequently-asked-questions}}

Updated July 27, 2020

\begin{itemize}
\item ~
  \hypertarget{should-i-refinance-my-mortgage}{%
  \paragraph{Should I refinance my
  mortgage?}\label{should-i-refinance-my-mortgage}}

  \begin{itemize}
  \tightlist
  \item
    \href{https://www.nytimes.com/article/coronavirus-money-unemployment.html?action=click\&pgtype=Article\&state=default\&region=MAIN_CONTENT_3\&context=storylines_faq}{It
    could be a good idea,} because mortgage rates have
    \href{https://www.nytimes.com/2020/07/16/business/mortgage-rates-below-3-percent.html?action=click\&pgtype=Article\&state=default\&region=MAIN_CONTENT_3\&context=storylines_faq}{never
    been lower.} Refinancing requests have pushed mortgage applications
    to some of the highest levels since 2008, so be prepared to get in
    line. But defaults are also up, so if you're thinking about buying a
    home, be aware that some lenders have tightened their standards.
  \end{itemize}
\item ~
  \hypertarget{what-is-school-going-to-look-like-in-september}{%
  \paragraph{What is school going to look like in
  September?}\label{what-is-school-going-to-look-like-in-september}}

  \begin{itemize}
  \tightlist
  \item
    It is unlikely that many schools will return to a normal schedule
    this fall, requiring the grind of
    \href{https://www.nytimes.com/2020/06/05/us/coronavirus-education-lost-learning.html?action=click\&pgtype=Article\&state=default\&region=MAIN_CONTENT_3\&context=storylines_faq}{online
    learning},
    \href{https://www.nytimes.com/2020/05/29/us/coronavirus-child-care-centers.html?action=click\&pgtype=Article\&state=default\&region=MAIN_CONTENT_3\&context=storylines_faq}{makeshift
    child care} and
    \href{https://www.nytimes.com/2020/06/03/business/economy/coronavirus-working-women.html?action=click\&pgtype=Article\&state=default\&region=MAIN_CONTENT_3\&context=storylines_faq}{stunted
    workdays} to continue. California's two largest public school
    districts --- Los Angeles and San Diego --- said on July 13, that
    \href{https://www.nytimes.com/2020/07/13/us/lausd-san-diego-school-reopening.html?action=click\&pgtype=Article\&state=default\&region=MAIN_CONTENT_3\&context=storylines_faq}{instruction
    will be remote-only in the fall}, citing concerns that surging
    coronavirus infections in their areas pose too dire a risk for
    students and teachers. Together, the two districts enroll some
    825,000 students. They are the largest in the country so far to
    abandon plans for even a partial physical return to classrooms when
    they reopen in August. For other districts, the solution won't be an
    all-or-nothing approach.
    \href{https://bioethics.jhu.edu/research-and-outreach/projects/eschool-initiative/school-policy-tracker/}{Many
    systems}, including the nation's largest, New York City, are
    devising
    \href{https://www.nytimes.com/2020/06/26/us/coronavirus-schools-reopen-fall.html?action=click\&pgtype=Article\&state=default\&region=MAIN_CONTENT_3\&context=storylines_faq}{hybrid
    plans} that involve spending some days in classrooms and other days
    online. There's no national policy on this yet, so check with your
    municipal school system regularly to see what is happening in your
    community.
  \end{itemize}
\item ~
  \hypertarget{is-the-coronavirus-airborne}{%
  \paragraph{Is the coronavirus
  airborne?}\label{is-the-coronavirus-airborne}}

  \begin{itemize}
  \tightlist
  \item
    The coronavirus
    \href{https://www.nytimes.com/2020/07/04/health/239-experts-with-one-big-claim-the-coronavirus-is-airborne.html?action=click\&pgtype=Article\&state=default\&region=MAIN_CONTENT_3\&context=storylines_faq}{can
    stay aloft for hours in tiny droplets in stagnant air}, infecting
    people as they inhale, mounting scientific evidence suggests. This
    risk is highest in crowded indoor spaces with poor ventilation, and
    may help explain super-spreading events reported in meatpacking
    plants, churches and restaurants.
    \href{https://www.nytimes.com/2020/07/06/health/coronavirus-airborne-aerosols.html?action=click\&pgtype=Article\&state=default\&region=MAIN_CONTENT_3\&context=storylines_faq}{It's
    unclear how often the virus is spread} via these tiny droplets, or
    aerosols, compared with larger droplets that are expelled when a
    sick person coughs or sneezes, or transmitted through contact with
    contaminated surfaces, said Linsey Marr, an aerosol expert at
    Virginia Tech. Aerosols are released even when a person without
    symptoms exhales, talks or sings, according to Dr. Marr and more
    than 200 other experts, who
    \href{https://academic.oup.com/cid/article/doi/10.1093/cid/ciaa939/5867798}{have
    outlined the evidence in an open letter to the World Health
    Organization}.
  \end{itemize}
\item ~
  \hypertarget{what-are-the-symptoms-of-coronavirus}{%
  \paragraph{What are the symptoms of
  coronavirus?}\label{what-are-the-symptoms-of-coronavirus}}

  \begin{itemize}
  \tightlist
  \item
    Common symptoms
    \href{https://www.nytimes.com/article/symptoms-coronavirus.html?action=click\&pgtype=Article\&state=default\&region=MAIN_CONTENT_3\&context=storylines_faq}{include
    fever, a dry cough, fatigue and difficulty breathing or shortness of
    breath.} Some of these symptoms overlap with those of the flu,
    making detection difficult, but runny noses and stuffy sinuses are
    less common.
    \href{https://www.nytimes.com/2020/04/27/health/coronavirus-symptoms-cdc.html?action=click\&pgtype=Article\&state=default\&region=MAIN_CONTENT_3\&context=storylines_faq}{The
    C.D.C. has also} added chills, muscle pain, sore throat, headache
    and a new loss of the sense of taste or smell as symptoms to look
    out for. Most people fall ill five to seven days after exposure, but
    symptoms may appear in as few as two days or as many as 14 days.
  \end{itemize}
\item ~
  \hypertarget{does-asymptomatic-transmission-of-covid-19-happen}{%
  \paragraph{Does asymptomatic transmission of Covid-19
  happen?}\label{does-asymptomatic-transmission-of-covid-19-happen}}

  \begin{itemize}
  \tightlist
  \item
    So far, the evidence seems to show it does. A widely cited
    \href{https://www.nature.com/articles/s41591-020-0869-5}{paper}
    published in April suggests that people are most infectious about
    two days before the onset of coronavirus symptoms and estimated that
    44 percent of new infections were a result of transmission from
    people who were not yet showing symptoms. Recently, a top expert at
    the World Health Organization stated that transmission of the
    coronavirus by people who did not have symptoms was ``very rare,''
    \href{https://www.nytimes.com/2020/06/09/world/coronavirus-updates.html?action=click\&pgtype=Article\&state=default\&region=MAIN_CONTENT_3\&context=storylines_faq\#link-1f302e21}{but
    she later walked back that statement.}
  \end{itemize}
\end{itemize}

The College Board, which administers the SAT, said on May 12 that it
supports colleges ``that are rightfully emphasizing flexibility for the
admissions process for next year'' and noted that the organization's
``responsibility is to ensure that every student has the option and the
opportunity to take the SAT.''

``As many colleges are considering temporary shifts to test-flexible and
test-optional admissions policies due to the coronavirus, it is
important that they help students understand how these changes will
impact admission, access to specialized programs, and scholarships,''
the College Board said.

A spokesman for the ACT said it ``respects the right of every college to
determine its own admission policies, particularly in the midst of a
crisis such as Covid-19 where flexibility and managing disruption is
paramount,'' while noting that ``despite the immediate effects of
Covid19 on admissions, it is clear that ACT scores add meaningful
insight and significant value above and beyond other factors used in the
college admission process.''

``We are committed to supporting students' needs for obtaining an ACT
score during these trying times by working to provide them with multiple
options to take the test,'' the spokesman said.

In the Boston area,
\href{https://admissions.tufts.edu/blogs/inside-admissions/post/tufts-introduces-sat-act-test-optional-admissions-policy/}{Tufts},
\href{https://news.northeastern.edu/2020/04/07/northeastern-adopts-test-optional-policy-for-students-applying-for-the-2021-22-academic-year-due-to-uncertainty-caused-by-covid-19/}{Northeastern}
and
\href{https://www.bu.edu/articles/2020/bu-standardized-tests-optional-admission/}{Boston
University} have all adopted an optional-testing policy.

Both
\href{https://stories.vassar.edu/2020/vassar-waives-standardized-tests-for-admission.html}{Vassar}
and
\href{https://www.pomona.edu/news/2020/04/02-pomona-college-adopts-test-optional-policy-students-applying-entry-fall-2021}{Pomona}
Colleges have waived standardized tests in their admission requirement.
\href{https://www.davidson.edu/news/2020/03/30/davidson-college-goes-test-optional-three-year-pilot}{Davidson
College} in North Carolina,
\href{https://www.haverford.edu/admission/blog/haverford-adopts-test-optional-policy}{Haverford
College} in Pennsylvania and
\href{https://news.rhodes.edu/stories/rhodes-college-adopts-test-optional-admissions-policy}{Rhodes
College} in Tennessee will move to optional testing for three years as
part of a pilot program and then re-evaluate their testing requirements.

In Washington State, where the outbreak struck early, the
\href{https://admit.washington.edu/apply/freshman/how-to-apply/test-scores/}{University
of Washington} took similar steps. The
\href{https://around.uoregon.edu/content/uo-adopts-sat-and-act-test-optional-admissions-policy}{University
of Oregon},
\href{https://admissions.oregonstate.edu/test-optional-admissions}{Oregon
State University} and
\href{https://www.scrippscollege.edu/admission/admission-announcements}{Scripps
College} in Southern California all recently announced that they would
no longer require standardized tests.

\href{https://admissions.tcu.edu/frogblog/posts/2020/temporarily-test-optional.php}{Texas
Christian University} and
\href{https://new.trinity.edu/news/trinity-authorizes-test-optional-admissions-students}{Trinity
University} in San Antonio made testing optional for next year's
applicants, with Trinity adopting the policy for a three-year period.

Both
\href{https://admission.tulane.edu/apply/instructions/standardized-tests}{Tulane
University} in New Orleans and
\href{https://case.edu/admission/features/admission-financial-aid/message-dean}{Case
Western Reserve University} in Cleveland have waived the testing
requirement for next year's applicants.

Aimee Ortiz contributed reporting.

Advertisement

\protect\hyperlink{after-bottom}{Continue reading the main story}

\hypertarget{site-index}{%
\subsection{Site Index}\label{site-index}}

\hypertarget{site-information-navigation}{%
\subsection{Site Information
Navigation}\label{site-information-navigation}}

\begin{itemize}
\tightlist
\item
  \href{https://help.nytimes.com/hc/en-us/articles/115014792127-Copyright-notice}{©~2020~The
  New York Times Company}
\end{itemize}

\begin{itemize}
\tightlist
\item
  \href{https://www.nytco.com/}{NYTCo}
\item
  \href{https://help.nytimes.com/hc/en-us/articles/115015385887-Contact-Us}{Contact
  Us}
\item
  \href{https://www.nytco.com/careers/}{Work with us}
\item
  \href{https://nytmediakit.com/}{Advertise}
\item
  \href{http://www.tbrandstudio.com/}{T Brand Studio}
\item
  \href{https://www.nytimes.com/privacy/cookie-policy\#how-do-i-manage-trackers}{Your
  Ad Choices}
\item
  \href{https://www.nytimes.com/privacy}{Privacy}
\item
  \href{https://help.nytimes.com/hc/en-us/articles/115014893428-Terms-of-service}{Terms
  of Service}
\item
  \href{https://help.nytimes.com/hc/en-us/articles/115014893968-Terms-of-sale}{Terms
  of Sale}
\item
  \href{https://spiderbites.nytimes.com}{Site Map}
\item
  \href{https://help.nytimes.com/hc/en-us}{Help}
\item
  \href{https://www.nytimes.com/subscription?campaignId=37WXW}{Subscriptions}
\end{itemize}
