Sections

SEARCH

\protect\hyperlink{site-content}{Skip to
content}\protect\hyperlink{site-index}{Skip to site index}

\href{https://www.nytimes.com/section/nyregion}{New York}

\href{https://myaccount.nytimes.com/auth/login?response_type=cookie\&client_id=vi}{}

\href{https://www.nytimes.com/section/todayspaper}{Today's Paper}

\href{/section/nyregion}{New York}\textbar{}Pain of Israel's Withdrawal
From Gaza Strip Is Felt by American Jews, Too

\begin{itemize}
\item
\item
\item
\item
\item
\end{itemize}

Advertisement

\protect\hyperlink{after-top}{Continue reading the main story}

Supported by

\protect\hyperlink{after-sponsor}{Continue reading the main story}

\hypertarget{pain-of-israels-withdrawal-from-gaza-strip-is-felt-by-american-jews-too}{%
\section{Pain of Israel's Withdrawal From Gaza Strip Is Felt by American
Jews,
Too}\label{pain-of-israels-withdrawal-from-gaza-strip-is-felt-by-american-jews-too}}

By \href{https://www.nytimes.com/by/joseph-berger}{Joseph Berger} and
Robin Shulman

\begin{itemize}
\item
  Aug. 14, 2005
\item
  \begin{itemize}
  \item
  \item
  \item
  \item
  \item
  \end{itemize}
\end{itemize}

\textbf{Correction Appended}

Starting this week, American Jews are likely to see wrenching scenes of
Jewish soldiers expelling defiant Jewish settlers from their homes and
farms in the Gaza Strip, as Israel begins its pullout there. The
experience for many Americans will be almost as painful and perplexing
as it will be for Israelis, because the two societies are so interwoven,
with the Gaza settlers' ranks made up of many transplanted New Yorkers
and other Americans.

Still, like most Israelis, Jews in New York and across the nation
largely support the government of Ariel Sharon in its plan to pull the
9,000 settlers out of Gaza.

Practically every secular American Jewish group has lined up behind
disengagement, as have major Reform and Conservative Jewish
organizations, though their support has been in the muted form of op-ed
articles and newspaper ads, rather than demonstrations.

Even the more left-wing Jewish-American groups that have long favored
disengagement say they have seen no need to pound the drums for it,
since the Sharon government, the Bush administration and American and
Israeli public opinion are overwhelmingly behind it. "There's really not
a lot of convincing to do," said Lewis E. Roth, assistant director of
the Washington-based Americans for Peace Now.

As in Israel, however, there is fierce opposition to the move, and much
of it is centered among the Orthodox, particularly the same ardently
Zionistic adherents of modern Orthodoxy and members of Lubavitch Hasidic
synagogues who make up much of the settler movement in the dominant Gush
Katif string of settlements in Gaza.

"Nine thousand people moved into the Gush Katif area at the behest of a
Labor government and other governments," said Rabbi Pesach Lerner,
executive vice president of the National Council of Young Israel, a
Manhattan-based group made up of 150 Orthodox congregations. "For years
they lived under fire and they wake up one day and are told, 'You're
history.' It's never happened before in Jewish history that Jewish
people exile Jewish people."

(The Israeli government did evacuate resistant settlers once before -\/-
when Israel completed the return of the Sinai peninsula to Egypt in
1982.)

Still, some Orthodox rabbis, often thought of as fiery activists in
American Jewish politics, have taken pains to admonish Jews not to
encourage Israeli soldiers to disobey army orders and urged them to
cease comparing the evictions to the Nazi deportations.

"Both the left and the right must guard their language," wrote Rabbi Avi
Weiss of the Bronx, who opposes disengagement, in an op-ed article in
The Forward, the 108-year-old Jewish weekly. "The settlers are not
'occupiers' and Prime Minister Sharon is not a 'fascist.' While a word
is a word and a deed is a deed, words lead to deeds."

At midnight last night, Jews protesting the Gaza withdrawal and fasting
for Tishah b'Ab, the traditional day of mourning for the destruction of
the First and Second Temples, were to gather at the Israeli Consulate,
on the East Side of Manhattan, for a vigil. Today, there is to be a
rally at the consulate and a march to the United Nations. The marchers
are to be joined there by a caravan of Lubavitch Hasidim from Brighton
Beach, Brooklyn. An even larger rally is scheduled for Tuesday across
from the United Nations.

On Friday, in the heavily Orthodox Midwood neighborhood of Brooklyn,
streamers and ribbons in orange -\/- the color of support for the
resistant Gaza settlers -\/- dangled from car antennas and side mirrors.

Sharon Rudolph, a 46-year-old receptionist shopping for fruits and
vegetables, wore an orange shirt that said, "Let My People Stay." She
had just returned from a visit to Gaza. "We can't be giving it back -\/-
it doesn't belong to the Arabs," she said.

Mitchell Orlian, a professor of Bible at Yeshiva University, made sure
to buy lettuce, parsley and dill imported from Gush Katif, as he always
does. "If I were there, I would let myself be dragged out," he said.

There has not been an equivalent outpouring of vocal support from those
who agree with the pullout, beyond op-ed articles or sermons like one
given just over a week ago by Rabbi Donald Goor of Temple Judea, a
Reform synagogue in Tarzana, Calif., who said, "those who are
questioning the withdrawal are not only questioning this government but
questioning government itself."

One leader of a major national Jewish group, who spoke on the condition
of anonymity because of the sharp differences among his members, said
that while his group supported disengagement, "there's no great
enthusiasm" for it. He said there were worries that Israel's unilateral
gesture would "be rewarded with more violence," and that a seaport and
an airport in Gaza that may someday be reopened would become gateways
for weapons to be used against Israelis.

A problem for more than a few liberal Jews is that the disengagement is
being pushed by someone who was regarded as their archenemy in the
theater of Israeli politics: Mr. Sharon, who promoted and built
settlements as a government minister. Mr. Sharon has in recent years
decided that it no longer makes sense to have Israeli soldiers protect
so few settlers in an area that has more than 1.3 million Palestinians.

"The honest-to-God liberals who would be enthusiastic are so suspicious
of Sharon that it would take a lot of wooing to get them out in the
streets, and nobody's wooing them," said J.J. Goldberg, editor of The
Forward.

There is ambivalence within the multicolored world of Orthodoxy as well.
Avi Shafran, a spokesman for Agudath Israel of America, a leading
traditionalist group, said that while most of its members were unhappy
with the withdrawal, they recognized that the move was being made by an
elected Israeli government.

Many settlers were influenced by rabbinical and philosophical champions
of a greater Israel that embraces Gaza and the West Bank, leaders who
contended that giving up territory was a sin. Most Hasidim, according to
Mr. Goldberg, have never accepted this thesis.

Still, opposition to disengagement is a minority view. An annual survey
by the American Jewish Committee found support for unilateral
disengagement running 65 to 28, with the remainder unsure. David A.
Harris, the organization's executive director, said "our business is not
to second-guess decisions of war and peace made by democratically
elected Israeli governments."

Rabbi Eric Yoffie, president of the Union for Reform Judaism and a
supporter of the pullout, has criticized the Conference of Presidents of
Major American Jewish Organizations, of which he is a member, for
failing to lobby more effectively for disengagement. Malcolm I.
Hoenlein, the conference's executive vice chairman, denied that the
support has been lukewarm.

But he acknowledged that many members are upset not just about the
removal of settlers but also about the uprooting of 30 synagogues, six
yeshivas, and several cemeteries, and they fear that withdrawal from
Gaza will shift terrorism to the West Bank. "People feel pain," he said.

Correction: August 22, 2005, Monday An article on Aug. 14 about the
reaction of American Jews to Israel's withdrawal from Gaza referred
incompletely to comments by Avi Shafran, a spokesman for the Orthodox
group Agudath Israel of America. Rabbi Shafran said he did not mean to
suggest that his group's members have come to accept the withdrawal,
only that they opposed active resistance to it because the decision to
withdraw was made by an elected Israeli government.

Advertisement

\protect\hyperlink{after-bottom}{Continue reading the main story}

\hypertarget{site-index}{%
\subsection{Site Index}\label{site-index}}

\hypertarget{site-information-navigation}{%
\subsection{Site Information
Navigation}\label{site-information-navigation}}

\begin{itemize}
\tightlist
\item
  \href{https://help.nytimes.com/hc/en-us/articles/115014792127-Copyright-notice}{©~2020~The
  New York Times Company}
\end{itemize}

\begin{itemize}
\tightlist
\item
  \href{https://www.nytco.com/}{NYTCo}
\item
  \href{https://help.nytimes.com/hc/en-us/articles/115015385887-Contact-Us}{Contact
  Us}
\item
  \href{https://www.nytco.com/careers/}{Work with us}
\item
  \href{https://nytmediakit.com/}{Advertise}
\item
  \href{http://www.tbrandstudio.com/}{T Brand Studio}
\item
  \href{https://www.nytimes.com/privacy/cookie-policy\#how-do-i-manage-trackers}{Your
  Ad Choices}
\item
  \href{https://www.nytimes.com/privacy}{Privacy}
\item
  \href{https://help.nytimes.com/hc/en-us/articles/115014893428-Terms-of-service}{Terms
  of Service}
\item
  \href{https://help.nytimes.com/hc/en-us/articles/115014893968-Terms-of-sale}{Terms
  of Sale}
\item
  \href{https://spiderbites.nytimes.com}{Site Map}
\item
  \href{https://help.nytimes.com/hc/en-us}{Help}
\item
  \href{https://www.nytimes.com/subscription?campaignId=37WXW}{Subscriptions}
\end{itemize}
