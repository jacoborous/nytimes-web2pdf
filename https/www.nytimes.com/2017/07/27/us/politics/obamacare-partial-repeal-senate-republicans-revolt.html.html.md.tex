Sections

SEARCH

\protect\hyperlink{site-content}{Skip to
content}\protect\hyperlink{site-index}{Skip to site index}

\href{https://www.nytimes.com/section/politics}{Politics}

\href{https://myaccount.nytimes.com/auth/login?response_type=cookie\&client_id=vi}{}

\href{https://www.nytimes.com/section/todayspaper}{Today's Paper}

\href{/section/politics}{Politics}\textbar{}Senate Rejects Slimmed-Down
Obamacare Repeal as McCain Votes No

\url{https://nyti.ms/2u3w6OB}

\begin{itemize}
\item
\item
\item
\item
\item
\item
\end{itemize}

Advertisement

\protect\hyperlink{after-top}{Continue reading the main story}

Supported by

\protect\hyperlink{after-sponsor}{Continue reading the main story}

\hypertarget{senate-rejects-slimmed-down-obamacare-repeal-as-mccain-votes-no}{%
\section{Senate Rejects Slimmed-Down Obamacare Repeal as McCain Votes
No}\label{senate-rejects-slimmed-down-obamacare-repeal-as-mccain-votes-no}}

\includegraphics{https://static01.nyt.com/images/2017/07/28/us/politics/McCain-for-video-clip/McCain-for-video-clip-videoSixteenByNineJumbo1600.jpg}

By \href{https://www.nytimes.com/by/robert-pear}{Robert Pear} and
\href{http://www.nytimes.com/by/thomas-kaplan}{Thomas Kaplan}

\begin{itemize}
\item
  July 27, 2017
\item
  \begin{itemize}
  \item
  \item
  \item
  \item
  \item
  \item
  \end{itemize}
\end{itemize}

WASHINGTON --- The Senate in the early hours of Friday morning rejected
a new, scaled-down Republican plan to repeal parts of the Affordable
Care Act, derailing the Republicans' seven-year campaign to dismantle
President Barack Obama's signature health care law and dealing a huge
political setback to President Trump.

Senator John McCain of Arizona, who just this week returned to the
Senate after receiving a diagnosis of brain cancer,
\href{https://www.nytimes.com/2017/07/28/us/politics/john-mccain-vote-trump-obamacare.html}{cast
the decisive vote} to defeat the proposal, joining two other
Republicans, Susan Collins of Maine and Lisa Murkowski of Alaska, in
opposing it.

The 49-to-51 vote was also a humiliating setback for the Senate majority
leader, Mitch McConnell of Kentucky, who has nurtured his reputation as
a master tactician and spent the last three months trying to devise a
repeal bill that could win support from members of his caucus.

As the clock ticked toward the final vote, which took place around 1:30
a.m., suspense built on the Senate floor. Mr. McCain was engaged in a
lengthy, animated conversation with Vice President Mike Pence, who had
come to the Capitol prepared to cast the tiebreaking vote for the
measure. A few minutes later, when Mr. McCain ambled over to the
Democratic side of the chamber, he was embraced by Senator Dianne
Feinstein, Democrat of California. A little later Senator Amy Klobuchar,
Democrat of Minnesota, put her arm around Mr. McCain.

The roll had yet to be called, but the body language suggested that the
Trump administration had failed in its effort to flip the Arizona
senator whom President Trump hailed on Tuesday as an ``American hero.''

Many senators announced their votes in booming voices. Mr. McCain
quietly signaled his vote with a thumbs-down gesture. He later offered
an explanation on Twitter:

After the tally was final, Mr. Trump tweeted:

The truncated Republican plan that ultimately fell was far less than
what Republicans once envisioned. Republican leaders, unable to overcome
complaints from both moderate and conservative members of their caucus,
said the skeletal plan was just a vehicle to permit negotiations with
the House, which passed a much more ambitious repeal bill in early May.

The ``skinny repeal'' bill, as it became known at the Capitol this week,
would still have had broad effects on health care. The bill would have
\href{https://www.cbo.gov/system/files/115th-congress-2017-2018/costestimate/s.a.667.pdf}{increased
the number of people who are uninsured by 15 million} next year compared
with current law, according to the nonpartisan Congressional Budget
Office. Premiums for people buying insurance on their own would have
increased roughly 20 percent, the budget office said.

\includegraphics{https://static01.nyt.com/images/2017/07/28/us/28dc-health_web11/28dc-health_web11-articleInline-v4.jpg?quality=75\&auto=webp\&disable=upscale}

Unlike previous setbacks, Friday morning's health care defeat had the
ring of finality. After the result was announced, the Senate quickly
moved on to routine business. Mr. McConnell canceled a session scheduled
for Friday and announced that the Senate would take up the nomination of
a federal circuit judge on Monday afternoon.

With so many senators in both parties railing against the fast-track
procedures that Republican leaders used, a return to health care seemed
certain to go through the committees, where bipartisanship and
deliberation are more likely.

``We are not celebrating,'' said the Senate Democratic leader, Chuck
Schumer of New York. ``We are relieved that millions and millions of
people who would have been so drastically hurt by the three proposals
put forward will at least retain their health care, be able to deal with
pre-existing conditions.''

Mr. McConnell said he was proud of his vote to start unwinding the
Affordable Care Act. ``What we tried to accomplish for the American
people was the right thing for the country,'' Mr. McConnell said. ``And
our only regret tonight, our only regret, is that we didn't achieve what
we had hoped to accomplish.''

Image

Vice President Mike Pence arriving at the Capitol late Thursday. He was
not able to save the measure with a tiebreaking vote.Credit...Zach
Gibson/Getty Images

The new, eight-page Senate bill, called the Health Care Freedom Act, was
unveiled just hours before the vote. It would have ended the requirement
that most people have health coverage, known as the individual mandate.
But it would not have put in place other incentives for people to obtain
coverage --- a situation that insurers say would leave them with a pool
of sicker, costlier customers. It would also have ended the requirement
that large employers offer coverage to their workers.

The ``skinny repeal'' would have delayed a tax on medical devices. It
would also have cut off federal funds for Planned Parenthood for one
year and increased federal grants to community health centers. And it
would have increased the limit on contributions to tax-favored health
savings accounts.

In addition, the bill would have made it much easier for states to waive
federal requirements that health insurance plans provide consumers with
a minimum set of benefits like maternity care and prescription drugs. It
would have eliminated funds provided by the Affordable Care Act for a
wide range of prevention and public health programs.

Before rolling out the new legislation, Senate leaders had to deal with
a rebellion from Republican senators who demanded ironclad assurances
that the legislation would never become law.

Mr. McCain and Senators Lindsey Graham of South Carolina and Ron Johnson
of Wisconsin insisted that House leaders promise that the bill would not
be enacted.

``I'm not going to vote for a bill that is terrible policy and horrible
politics just because we have to get something done,'' Mr. Graham said
at a news conference, calling the stripped-down bill a ``disaster'' and
a ``fraud'' as a replacement for the health law.

Mr. Graham eventually voted for the measure after receiving an assurance
from the House speaker, Paul D. Ryan, that the two chambers would
negotiate their differences if the Senate passed the legislation.

``If moving forward requires a conference committee, that is something
the House is willing to do,'' Mr. Ryan said in a statement. ``The
reality, however, is that repealing and replacing Obamacare still
ultimately requires the Senate to produce 51 votes for an actual plan.''

Image

From left, Senators Lindsey Graham of South Carolina, Ron Johnson of
Wisconsin and John McCain of Arizona, all Republicans, spoke about the
``skinny repeal'' bill on Thursday.Credit...Gabriella Demczuk for The
New York Times

But Mr. Ryan left open the possibility that if a compromise measure had
failed in the Senate, the House could still pass the stripped-down
Senate health bill. That helped push Mr. McCain to ``no.''

Republican senators found themselves in the strange position of hoping
their bill would never be approved by the House.

``It may very well be a good vehicle to get us into conference, but you
got to make sure that it's not so good that the House simply passes it
rather than going to conference,'' said Senator Michael Rounds,
Republican of South Dakota. Mr. Rounds, who built a successful insurance
business in his home state, said he was concerned that ``the markets may
collapse'' if the Senate bill ever took effect.

Two influential House conservatives made clear that they did not want to
simply pass the Senate bill. Representative Mark Walker, Republican of
North Carolina and the chairman of the conservative Republican Study
Committee, said he favored a conference, calling the bill ``ugly to the
bone.''

\href{https://www.nytimes.com/interactive/2017/07/27/us/politics/document-Read-the-Senate-Skinny-Repeal-Bill.html}{}

\includegraphics{https://static01.nyt.com/images/2017/07/27/us/politics/image-Read-the-Senate-Skinny-Repeal-Bill/image-Read-the-Senate-Skinny-Repeal-Bill-thumbLarge.gif}

\hypertarget{read-the-senate-skinny-repeal-bill}{%
\subsection{Read the Senate `Skinny Repeal'
Bill}\label{read-the-senate-skinny-repeal-bill}}

Republicans on Thursday released a narrow measure to roll back parts of
the Affordable Care Act.

And Representative Mark Meadows, Republican of North Carolina and the
chairman of the hard-line Freedom Caucus, said that for many
conservatives, it would be a ``nonstarter'' to send President Trump a
bill that has ``gotten so skinny that it doesn't resemble a repeal.''

But senators had at least some reason to be nervous. The House majority
leader, Kevin McCarthy of California, notified House members that
``pending Senate action on health care,'' the House schedule could
change, and that ``all members should remain flexible in their travel
plans over the next few days.'' That did not sound like a man preparing
for protracted House-Senate negotiations.

Representative Chris Collins, Republican of New York and a key ally of
Mr. Trump, said the stripped-down bill would be ``better than nothing''
if it became apparent that the Senate did not have the votes for a more
ambitious bill.

``It becomes a binary choice,'' he said. ``If it's this or nothing, who
wants to go home and say I did nothing?''

Image

Mr. Graham, right, called the Senate bill a ``disaster,'' and Mr.
Johnson, left, said it ``doesn't even come close to honoring our promise
of repealing Obamacare.''Credit...Gabriella Demczuk for The New York
Times

``No one can guarantee anything,'' he added, sending a message to
senators wanting assurances.

Even some senators who voted for the measure Friday conceded that its
enactment could have been disastrous. It would have repealed the mandate
that most Americans have insurance, without another mechanism to push
Americans to maintain insurance coverage. Under those circumstances,
healthy people could wait to buy insurance until they are sick. The
insurance markets would become dominated by the chronically ill, and
premiums would soar, insurers warned.

America's Health Insurance Plans, the Blue Cross Blue Shield Association
and the American Medical Association all expressed similar concerns.

``We would oppose an approach that eliminates the individual coverage
requirement, does not offer alternative continuous coverage solutions,
and does not include measures to immediately stabilize the individual
market,'' said America's Health Insurance Plans, a trade group for the
industry.

On the other side, the Trump administration twisted arms. Mr. Trump
directed Interior Secretary Ryan Zinke to call Ms. Murkowski, the Alaska
senator, to remind her of issues affecting her state that are controlled
by the Interior Department, according to people familiar with the call,
who requested anonymity because they were not authorized to speak to the
press.

Ms. Murkowski confirmed to reporters that she had received a call from
Mr. Zinke, but she declined to describe the details. However, people
familiar with the call described her reaction to it as ``furious.''

Advertisement

\protect\hyperlink{after-bottom}{Continue reading the main story}

\hypertarget{site-index}{%
\subsection{Site Index}\label{site-index}}

\hypertarget{site-information-navigation}{%
\subsection{Site Information
Navigation}\label{site-information-navigation}}

\begin{itemize}
\tightlist
\item
  \href{https://help.nytimes.com/hc/en-us/articles/115014792127-Copyright-notice}{©~2020~The
  New York Times Company}
\end{itemize}

\begin{itemize}
\tightlist
\item
  \href{https://www.nytco.com/}{NYTCo}
\item
  \href{https://help.nytimes.com/hc/en-us/articles/115015385887-Contact-Us}{Contact
  Us}
\item
  \href{https://www.nytco.com/careers/}{Work with us}
\item
  \href{https://nytmediakit.com/}{Advertise}
\item
  \href{http://www.tbrandstudio.com/}{T Brand Studio}
\item
  \href{https://www.nytimes.com/privacy/cookie-policy\#how-do-i-manage-trackers}{Your
  Ad Choices}
\item
  \href{https://www.nytimes.com/privacy}{Privacy}
\item
  \href{https://help.nytimes.com/hc/en-us/articles/115014893428-Terms-of-service}{Terms
  of Service}
\item
  \href{https://help.nytimes.com/hc/en-us/articles/115014893968-Terms-of-sale}{Terms
  of Sale}
\item
  \href{https://spiderbites.nytimes.com}{Site Map}
\item
  \href{https://help.nytimes.com/hc/en-us}{Help}
\item
  \href{https://www.nytimes.com/subscription?campaignId=37WXW}{Subscriptions}
\end{itemize}
