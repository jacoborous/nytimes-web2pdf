Sections

SEARCH

\protect\hyperlink{site-content}{Skip to
content}\protect\hyperlink{site-index}{Skip to site index}

\href{https://www.nytimes.com/section/world/asia}{Asia Pacific}

\href{https://myaccount.nytimes.com/auth/login?response_type=cookie\&client_id=vi}{}

\href{https://www.nytimes.com/section/todayspaper}{Today's Paper}

\href{/section/world/asia}{Asia Pacific}\textbar{}Samsung Heir Faces
Arrest on Charges of Bribing South Korea's President

\url{https://nyti.ms/2iBzBJm}

\begin{itemize}
\item
\item
\item
\item
\item
\end{itemize}

Advertisement

\protect\hyperlink{after-top}{Continue reading the main story}

Supported by

\protect\hyperlink{after-sponsor}{Continue reading the main story}

\hypertarget{samsung-heir-faces-arrest-on-charges-of-bribing-south-koreas-president}{%
\section{Samsung Heir Faces Arrest on Charges of Bribing South Korea's
President}\label{samsung-heir-faces-arrest-on-charges-of-bribing-south-koreas-president}}

\includegraphics{https://static01.nyt.com/images/2017/01/17/world/17KOREA-1/17KOREA-1-articleLarge.jpg?quality=75\&auto=webp\&disable=upscale}

By \href{http://www.nytimes.com/by/choe-sang-hun}{Choe Sang-Hun}

\begin{itemize}
\item
  Jan. 15, 2017
\item
  \begin{itemize}
  \item
  \item
  \item
  \item
  \item
  \end{itemize}
\end{itemize}

SEOUL, South Korea --- The sprawling investigation into President Park
Geun-hye of South Korea took a dramatic turn on Monday with word that
prosecutors were seeking the arrest of the de facto head of Samsung, one
of the world's largest conglomerates, on charges that he bribed the
president and her secretive confidante.

A prosecutor's call for the arrest of Jay Y. Lee, the vice chairman of
Samsung and only son of the company's incapacitated chairman, Lee
Kun-hee, brings new scrutiny to the deep ties between top government
officials and the handful of
\href{https://www.nytimes.com/2017/01/02/world/asia/south-korea-park-geun-hye-samsung.html}{corporations
that dominate} South Korea's economy.

Mr. Lee is accused of instructing Samsung subsidiaries to make payments
totaling 43 billion won (\$36 million) to the family of Ms. Park's
confidante, Choi Soon-sil, and to two foundations that Ms. Choi
controlled, in exchange for help from Ms. Park in facilitating a
father-to-son transfer of ownership control of Samsung.

The special prosecutor, Park Young-soo, said that the money represented
bribes from Samsung, and that Mr. Lee embezzled some of the money from
his companies. Mr. Park said he had asked a Seoul court to issue an
arrest warrant for Mr. Lee; it usually takes a few days for a court to
decide whether to grant such a warrant.

``We have enough evidence to establish President Park and Choi Soon-sil
as co-conspirators sharing profits'' in the bribery scheme, Lee
Kyu-chul, a spokesman of the special prosecutor, said during a news
briefing on Monday.

If Mr. Lee is arrested, it will be a milestone in South Korea's efforts
to fight corruption involving the country's powerful family-controlled
conglomerates, known as chaebol. It could also disrupt Mr. Lee's efforts
to inherit management control of Samsung, whose tentacles in consumer
electronics, shipbuilding and a range of other industries reach
throughout South Korea's economy.

Mr. Lee's arrest on a bribery charge would further corner Ms. Park. She
is already being tried
\href{https://www.nytimes.com/2017/01/03/world/asia/south-korea-president-impeachment-trial.html}{in
the Constitutional Court}, after the National Assembly
\href{https://www.nytimes.com/2016/12/09/world/asia/south-korea-president-park-geun-hye-impeached.html}{voted
on Dec. 9 to impeach her}. Prosecutors have argued in that trial that
Ms. Park and Ms. Choi colluded to collect millions from Samsung and
other big businesses, either through coercion or through trades of
political favors for bribes. Ms. Park and Ms. Choi have denied
wrongdoing.

Although Samsung has often been investigated on corruption allegations,
neither Mr. Lee nor his father has spent time in jail. The elder Mr. Lee
was convicted of bribery in 1996, and of tax evasion and breach of trust
in 2009, but in each case, he was not arrested, and his prison terms
were suspended. Each time, his criminal record was later erased in
presidential pardons, and he soon returned to Samsung's leadership.

The South Korean government and the judiciary have been similarly
lenient toward other business tycoons convicted of white-collar crimes,
deepening public mistrust of the justice system and raising doubts among
foreign investors who want tighter corporate governance in South Korea.

The chairman of Samsung has been incapacitated since a
\href{https://www.nytimes.com/2014/05/12/business/international/samsungs-chairman-has-surgery-after-heart-attack.html}{heart
attack in 2014}. His son has been running the conglomerate, which has
annual revenue of 270 trillion won (\$229 billion). Its crown jewel,
Samsung Electronics, accounts for one-fifth of all South Korean exports.

Corruption scandals in the chaebol often stem from maneuvers to transfer
wealth control from one generation of the dominant family to the next,
and the trouble at Samsung is no exception.

In 2009, the elder Mr. Lee was convicted of evading taxes on 4.5
trillion won (\$3.8 billion) that he secretly inherited from his father,
\href{http://www.nytimes.com/1987/11/20/obituaries/lee-byung-chull-77-industrialist-of-korea.html}{Lee
Byung-chull}, the founder of Samsung. The funds were kept hidden in the
bank and securities accounts of Samsung executives. He was also
convicted of involvement in helping his son buy stocks of a Samsung
subsidiary at an illegally low price.

In the current scandal, Samsung was accused of making payments to Ms.
Choi in exchange for a decision by the government-controlled National
Pension Service to support a contentious 2015 merger of two Samsung
affiliates.
\href{https://www.nytimes.com/2016/12/31/world/asia/south-korea-samsung-merger-moon-hyung-pyo.html}{Moon
Hyung-pyo}, the chairman of the pension fund, was indicted on Monday on
charges that he illegally pressed the fund to back that merger when he
was South Korea's health and welfare minister.

The special prosecutor said that Mr. Moon acted on behalf of Ms. Park.

The national pension fund's support was crucial for the merger, which
analysts said helped Mr. Lee inherit control of Samsung from his father.
Elliott Management, an American activist hedge fund, and other investors
in Samsung had campaigned to block the merger, saying that it wronged
minority shareholders by grossly undervaluing the shares of one of the
two Samsung companies, Samsung C\&T.

Samsung issued a statement on Monday denying that it paid bribes or made
``improper requests related to the merger of Samsung affiliates or the
leadership transition.''

Allegations that Ms. Park helped Ms. Choi extort millions in bribes from
Samsung and other companies are at the heart of the corruption scandal
that led to the impeachment vote. Ms. Park's powers have been suspended
while the Constitutional Court decides whether to end her presidency.

Ms. Choi was indicted in November on charges of coercing 53 big
businesses, including Samsung, to contribute \$69 million to her two
foundations. Prosecutors identified
\href{https://www.nytimes.com/2016/11/20/world/asia/park-geun-hye-south-korea-extortion-accomplice-prosecutors.html}{Ms.
Park as an accomplice} but stopped short of filing any charges against
the businesses, all of which insisted that they were under government
pressure to donate the money.

Samsung made the largest payments to Ms. Choi's foundations, totaling
\$17 million. Unlike the other corporate contributors, it went beyond
the donations, taking steps that included signing an \$18 million
contract with a sports management company that Ms. Choi ran in Germany.
That money financed a program for training Korean equestrians that
mainly
\href{https://www.nytimes.com/2017/01/02/world/asia/south-korea-scandal-choi-soon-sil-daughter.html}{benefited
Ms. Choi's daughter}. Samsung also contributed \$1.3 million to a winter
sports program for young athletes that was run by Ms. Choi and a
relative.

Mr. Lee testified at a National Assembly hearing last month that he was
not involved in the decision by Samsung to make the payments. He also
said that they were not voluntary donations, suggesting that the company
was a victim of extortion, not a participant in bribery.

But the special prosecutor's team accused Mr. Lee on Monday of perjuring
himself during the parliamentary hearing. The prosecutors said they had
evidence that Mr. Lee personally ordered the payments after receiving a
request from Ms. Park.

Some conservative South Koreans, as well as some pro-business newspaper
editorials, have warned that an arrest of Mr. Lee could hurt the
national economy by disrupting Samsung's management. Lee Kyu-chul, the
special prosecutor's spokesman, responded, ``We determined that
establishing justice is far more important.''

Despite the case's high profile, its impact on Samsung's day-to-day
operations will probably be subdued at first, according to analysts.
Samsung Electronics, which makes microchips, screens and smartphones, is
run by professional managers who are used to operating on their own
without Mr. Lee's direct involvement.

Still, Mark Newman, an analyst with Sanford C. Bernstein, a research and
brokerage firm, said that Mr. Lee's legal issues could ultimately slow
down larger strategic decisions at Samsung.

``Big things, like restructuring or a new big multibillion-dollar
investment, things like that might be tricky for him to get to,'' he
said.

He likened the company's clout to that of large, organized special
interests in Washington.

``In the U.S., there are a lot of powerful lobbies, like the gun lobby,
but in Korea the equivalent is the Samsung lobby or the Hyundai lobby,''
Mr. Newman said. ``They're very powerful, but they're being found out.''

Advertisement

\protect\hyperlink{after-bottom}{Continue reading the main story}

\hypertarget{site-index}{%
\subsection{Site Index}\label{site-index}}

\hypertarget{site-information-navigation}{%
\subsection{Site Information
Navigation}\label{site-information-navigation}}

\begin{itemize}
\tightlist
\item
  \href{https://help.nytimes.com/hc/en-us/articles/115014792127-Copyright-notice}{©~2020~The
  New York Times Company}
\end{itemize}

\begin{itemize}
\tightlist
\item
  \href{https://www.nytco.com/}{NYTCo}
\item
  \href{https://help.nytimes.com/hc/en-us/articles/115015385887-Contact-Us}{Contact
  Us}
\item
  \href{https://www.nytco.com/careers/}{Work with us}
\item
  \href{https://nytmediakit.com/}{Advertise}
\item
  \href{http://www.tbrandstudio.com/}{T Brand Studio}
\item
  \href{https://www.nytimes.com/privacy/cookie-policy\#how-do-i-manage-trackers}{Your
  Ad Choices}
\item
  \href{https://www.nytimes.com/privacy}{Privacy}
\item
  \href{https://help.nytimes.com/hc/en-us/articles/115014893428-Terms-of-service}{Terms
  of Service}
\item
  \href{https://help.nytimes.com/hc/en-us/articles/115014893968-Terms-of-sale}{Terms
  of Sale}
\item
  \href{https://spiderbites.nytimes.com}{Site Map}
\item
  \href{https://help.nytimes.com/hc/en-us}{Help}
\item
  \href{https://www.nytimes.com/subscription?campaignId=37WXW}{Subscriptions}
\end{itemize}
