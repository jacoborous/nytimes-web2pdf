Sections

SEARCH

\protect\hyperlink{site-content}{Skip to
content}\protect\hyperlink{site-index}{Skip to site index}

\href{https://www.nytimes.com/section/world/canada}{Canada}

\href{https://myaccount.nytimes.com/auth/login?response_type=cookie\&client_id=vi}{}

\href{https://www.nytimes.com/section/todayspaper}{Today's Paper}

\href{/section/world/canada}{Canada}\textbar{}For Justin Trudeau,
Canada's Leader, Revival of Keystone XL Upsets a Balancing Act

\url{https://nyti.ms/2k5eRLG}

\begin{itemize}
\item
\item
\item
\item
\item
\item
\end{itemize}

Advertisement

\protect\hyperlink{after-top}{Continue reading the main story}

Supported by

\protect\hyperlink{after-sponsor}{Continue reading the main story}

\hypertarget{for-justin-trudeau-canadas-leader-revival-of-keystone-xl-upsets-a-balancing-act}{%
\section{For Justin Trudeau, Canada's Leader, Revival of Keystone XL
Upsets a Balancing
Act}\label{for-justin-trudeau-canadas-leader-revival-of-keystone-xl-upsets-a-balancing-act}}

\includegraphics{https://static01.nyt.com/images/2017/01/26/world/26canada/26canada-articleInline.jpg?quality=75\&auto=webp\&disable=upscale}

By \href{http://www.nytimes.com/by/ian-austen}{Ian Austen} and
\href{http://www.nytimes.com/by/clifford-krauss}{Clifford Krauss}

\begin{itemize}
\item
  Jan. 25, 2017
\item
  \begin{itemize}
  \item
  \item
  \item
  \item
  \item
  \item
  \end{itemize}
\end{itemize}

OTTAWA --- If there is such a thing as unwelcome good news, President
Trump may have handed some to Prime Minister Justin Trudeau of Canada on
Tuesday when he
\href{https://www.nytimes.com/2017/01/24/us/politics/keystone-dakota-pipeline-trump.html}{revived
a cross-border oil pipeline project}.

The move is likely to complicate a delicate balancing act Mr. Trudeau
has been trying to keep up: He has long maintained that Canada needs to
develop its energy industry, but he also stands for aggressively cutting
the country's carbon emissions.

The pipeline, known as Keystone XL, is meant to help carry crude oil
from the oil sands of Alberta to refineries on the Gulf of Mexico. The
Obama administration
\href{https://www.nytimes.com/2015/11/07/us/obama-expected-to-reject-construction-of-keystone-xl-oil-pipeline.html}{blocked
the project} for environmental reasons --- and in doing so, spared Mr.
Trudeau from having to reckon with the consequences of building it.

Now, Mr. Trump has put Keystone XL back on the table, and with it the
prospect of significantly increased oil production and carbon emissions,
upending Canada's climate calculus.

Mr. Trudeau has been trying to avoid taking the path of his father,
Pierre Elliott Trudeau, whose
\href{http://www.thecanadianencyclopedia.ca/en/article/national-energy-program/}{energy
policies as prime minister} in the 1980s deeply soiled the family name
among Albertans. The elder Mr. Trudeau imposed extra taxes and royalties
along with price controls that angered the province's oil industry. The
younger Mr. Trudeau spoke up swiftly on Tuesday, welcoming Mr. Trump's
action. And the Canadian leader's public support for the energy industry
has helped him sell his national
\href{https://www.nytimes.com/2016/12/08/world/canada/canada-carbon-pricing-environment-energy.html}{carbon-pricing
program} in the province --- a program that polls strongly in areas away
from the oil patch, especially Ontario and British Columbia.

But now people on all sides are angry at him. Along with a decision in
late November to approve
\href{https://www.nytimes.com/2016/11/29/world/canada/canada-trudeau-kinder-morgan-pipeline.html}{two
other pipeline projects}, Mr. Trudeau's welcome of the Keystone XL
revival has alienated many environmentalists in Canada, and has also
strained his relations with some indigenous communities affected by the
pipelines. And hours after Mr. Trump's move was announced, Mr. Trudeau
found himself in a
\href{http://www.theglobeandmail.com/news/politics/trudeau-challenged-on-oil-sands-at-rowdy-calgary-town-hall/article33755072/}{heated
public exchange} in Calgary, Alberta, with a supporter of the oil sands.

``The federal government now has to be able to demonstrate that its
climate plan is still relevant with increased oil-sands production,''
said Simon Dyer, the regional director for Alberta at the Pembina
Institute, an environmental group that accepts Mr. Trudeau's argument
that the oil sands are vital to the economy. ``When you start to stack
up the potential for increased production from increased pipeline
capacity, it strains credibility that oil-sands emissions targets will
be met.''

Dennis McConaghy, a former executive at TransCanada, the company that
proposed Keystone XL, said he welcomed both Mr. Trump's move and Mr.
Trudeau's endorsement of the project. But he agreed with Mr. Dyer that
building the pipeline might make it impossible for the government to
meet its emissions targets.

``The country needs to find a balance between a credible carbon policy
and seizing this economic opportunity,'' said Mr. McConaghy, who
recently published \href{https://www.dundurn.com/books/Dysfunction}{a
book} on Keystone XL. ``But on carbon, they have outlined commitments
which will be very challenging to meet, especially given increased oil
sands production.''

Mr. Trudeau's cross-country tour of town-hall-style meetings this month
has shown how tricky the balancing act can be.

When
\href{https://www.nytimes.com/2017/01/18/world/canada/justin-trudeau-canada-town-hall-tour.html}{an
attendee at a banquet hall in Peterborough, Ontario}, challenged Mr.
Trudeau on his earlier approval of the pipeline, he dived into his usual
talk about Canada transitioning away from fossil fuels but still needing
a vibrant energy industry. But then he veered slightly off script.

``You can't make a choice between what's good for the environment and
what's good for the economy,'' Mr. Trudeau told the crowd. ``We can't
shut down the oil sands tomorrow. We need to phase them out. We need to
manage the transition off of our dependence on fossil fuels.''

That suggestion that Mr. Trudeau wanted to ``phase out'' the oil sands
quickly attracted angry attention from conservative politicians and
talk-radio hosts in Alberta, and the context --- that he was defending
the oil industry --- was largely lost.

Mr. Trudeau's road show arrived on Tuesday in Calgary, a stronghold of
the opposition Conservative Party where his reception was never likely
to be warm. His remark in Ontario was soon thrown back at him. A man
wearing an ``I Love Oil Sands'' T-shirt and a ``Make America Great
Again'' cap accused Mr. Trudeau of offering different messages about the
energy industry in different parts of the country.

``You are either a liar or you're confused,'' the man told him. ``And
I'm beginning to think it's both.''

The usually imperturbableMr. Trudeau raised his voice as he replied:
``The responsibility of any Canadian prime minister is to get our
resources to market and, yes, that includes our oil-sands fossil
fuels.'' Video of the exchange suggested that the crowd was divided
about Mr. Trudeau's position.

``There is a belief in the current government on the need to act on
climate change,'' said Tim Gray, the executive director of Environmental
Defense Canada. ``And they very much feel the way to do that was to get
Alberta on side.''

Reviving the pipeline project will not help, he said. ``The old-school
model is still dominating the discussion in the country, and people have
latched on to this idea that pipelines need to be built,'' Mr. Gray
said. ``But the world is passing us by.''

Keystone XL was originally planned to enter service around 2012. Some
analysts say the market has changed since then, to the point that there
may no longer be a compelling business case for building it.

Oil production is quickly rebounding in the United States, threatening
again to flood the domestic and international oil markets. Fuel prices
at American pumps are relatively cheap without the pipeline.

If Keystone XL is built --- and many hurdles remain --- the pipeline
would deliver up to 830,000 barrels of crude oil a day from Canada and
North Dakota to Steele City, Neb., where it would enter an existing
piping network bound for the gulf refineries. With the American market
already oversupplied, most of the additional oil from Canada would
probably have to be re-exported in refined form to foreign markets, or
would replace American crude that would then be exported.

The project would create few permanent jobs, but it would generate
thousands of short-term jobs in construction, which is why several
powerful unions support the project.

``Keystone XL is just not a strategic artery anymore,'' said Tom Kloza,
the global head of energy analysis at Oil Price Information Service.
``It's not crucial to the circulation of North American crude oil. It's
become a focus of political partisanship, like the potted palm in the
Jack Lemmon movie `Mister Roberts.'''

In the end, a completed Keystone XL pipeline may prove to be little more
than a favor to Canada, where several American oil companies have made
huge investments in the oil sands fields.

``In today's world, the Canadians need the pipeline to get their oil to
the world market,'' said Philip K. Verleger Jr., an energy economist.
``But it makes no difference to the United States.''

Advertisement

\protect\hyperlink{after-bottom}{Continue reading the main story}

\hypertarget{site-index}{%
\subsection{Site Index}\label{site-index}}

\hypertarget{site-information-navigation}{%
\subsection{Site Information
Navigation}\label{site-information-navigation}}

\begin{itemize}
\tightlist
\item
  \href{https://help.nytimes.com/hc/en-us/articles/115014792127-Copyright-notice}{©~2020~The
  New York Times Company}
\end{itemize}

\begin{itemize}
\tightlist
\item
  \href{https://www.nytco.com/}{NYTCo}
\item
  \href{https://help.nytimes.com/hc/en-us/articles/115015385887-Contact-Us}{Contact
  Us}
\item
  \href{https://www.nytco.com/careers/}{Work with us}
\item
  \href{https://nytmediakit.com/}{Advertise}
\item
  \href{http://www.tbrandstudio.com/}{T Brand Studio}
\item
  \href{https://www.nytimes.com/privacy/cookie-policy\#how-do-i-manage-trackers}{Your
  Ad Choices}
\item
  \href{https://www.nytimes.com/privacy}{Privacy}
\item
  \href{https://help.nytimes.com/hc/en-us/articles/115014893428-Terms-of-service}{Terms
  of Service}
\item
  \href{https://help.nytimes.com/hc/en-us/articles/115014893968-Terms-of-sale}{Terms
  of Sale}
\item
  \href{https://spiderbites.nytimes.com}{Site Map}
\item
  \href{https://help.nytimes.com/hc/en-us}{Help}
\item
  \href{https://www.nytimes.com/subscription?campaignId=37WXW}{Subscriptions}
\end{itemize}
