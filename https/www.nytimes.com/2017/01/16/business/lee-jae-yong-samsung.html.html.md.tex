Sections

SEARCH

\protect\hyperlink{site-content}{Skip to
content}\protect\hyperlink{site-index}{Skip to site index}

\href{https://www.nytimes.com/section/business}{Business}

\href{https://myaccount.nytimes.com/auth/login?response_type=cookie\&client_id=vi}{}

\href{https://www.nytimes.com/section/todayspaper}{Today's Paper}

\href{/section/business}{Business}\textbar{}Political Crisis Engulfs
Samsung, a Firm Tied to South Korea's Success

\url{https://nyti.ms/2jQDUhO}

\begin{itemize}
\item
\item
\item
\item
\item
\end{itemize}

Advertisement

\protect\hyperlink{after-top}{Continue reading the main story}

Supported by

\protect\hyperlink{after-sponsor}{Continue reading the main story}

\hypertarget{political-crisis-engulfs-samsung-a-firm-tied-to-south-koreas-success}{%
\section{Political Crisis Engulfs Samsung, a Firm Tied to South Korea's
Success}\label{political-crisis-engulfs-samsung-a-firm-tied-to-south-koreas-success}}

\includegraphics{https://static01.nyt.com/images/2017/01/16/business/17SAMSUNG/17SAMSUNG-videoSixteenByNine3000.jpg}

By \href{https://www.nytimes.com/by/paul-mozur}{Paul Mozur}

\begin{itemize}
\item
  Jan. 16, 2017
\item
  \begin{itemize}
  \item
  \item
  \item
  \item
  \item
  \end{itemize}
\end{itemize}

HONG KONG --- Samsung, the giant maker of products as varied as cargo
ships and smartphones, is one of the driving forces of the South Korean
economy. Now, the fate of its leadership could be tied to the sprawling
investigation of a corruption scandal involving the nation's president.

Prosecutors on Monday
\href{https://www.nytimes.com/2017/01/15/world/asia/south-korea-samsung-arrest-jay-lee-park-geun-hye.html}{called
for the arrest} of the company's heir-apparent, Lee Jae-yong.
Prosecutors contend that Mr. Lee bribed President Park Geun-hye and one
of her confidants in exchange for political favors.

The charges were leveled at a tough time for the company, which was
struggling in some of its core businesses while trying to break from its
past and forge a new management culture. Samsung's Galaxy Note 7,
intended to showcase its ability to innovate and challenge Apple for
global smartphone supremacy, instead showed a propensity to burst into
flames,
\href{https://www.nytimes.com/2016/09/03/business/samsung-galaxy-note-battery.html}{and
was recalled}. Samsung has been without a chairman for two years, after
the patriarch had a heart attack.

The crisis is particularly difficult not only for the company, but for
South Korea's relatively young democracy. One of the country's most
important companies might be wounded, and the nation's leadership has
been thrown into disarray.

Suave and mellow, Mr. Lee has often been portrayed as the right leader
for a vast, stodgy business empire needing to refocus.

The bribery crisis has dominated the news for days in South Korea, where
Samsung and other huge ``chaebols,'' or family-controlled industrial
empires, are embedded in politics and in the national identity. Samsung
alone accounts for about 20 percent of South Korea's exports.

The country's largest chaebols have faced legal battles and convictions
in recent years, but these episodes often end with the industrialists
receiving pardons.

Mr. Lee might not be arrested at all. That will be decided this week.

Prosecutors say Samsung made payments in exchange for a decision by the
government-controlled National Pension Service to support a contentious
2015 merger of two Samsung affiliates. Analysts say the merger helped
Mr. Lee inherit control of Samsung from his father.

In a statement on Monday, Samsung denied any bribery or making
``improper requests related to the merger of Samsung affiliates or the
leadership transition.'' But the move by prosecutors casts Mr. Lee in a
different light, as a leader similar to his father, who was convicted of
white-collar crimes but pardoned twice. It also could raise questions
about Mr. Lee's pledge to make Samsung more transparent and more
responsive to shareholders.

South Korea's president, Ms. Park, has
\href{https://www.nytimes.com/2017/01/15/world/asia/south-korea-samsung-arrest-jay-lee-park-geun-hye.html}{denied
wrongdoing}.

Mr. Lee, who goes by Jay Y. Lee in the West, is part of the third
generation of the Lee family to lead the conglomerate. While his father
was widely seen as old-fashioned, Mr. Lee has been painted by Samsung as
equally comfortable in Silicon Valley and in the boardroom, as a leader
who could open up the clannish business and reinvigorate Samsung's
flagging reputation for innovation in consumer technology. Samsung is
the world's biggest maker of cellphones.

``Beyond the arrest itself, this is going to be a big blow to the
narrative they've been building,'' said Geoffrey Cain, the author of a
coming book on Samsung. ``It's hard to convince shareholders and
partners they are a hip Silicon Valley-style company when these charges
show them to be a company run like a feudal dynasty.''

Samsung is in effect a collection of widely disparate companies, each
with an autonomous, professional management team. Samsung Electronics,
the most important of those companies, is broken into high-functioning
business units. Because of that, some observers say that the company
could continue to run smoothly without Mr. Lee.

Important matters that require Mr. Lee's imprimatur could be delayed
amid the distractions, said Mark Newman, an analyst at Sanford C.
Bernstein and Company. ``I think that's the thing you need to worry
about: big decisions,'' he said.

If executives lower in the hierarchy are implicated, day-to-day
operations could become complicated.
\href{http://english.yonhapnews.co.kr/news/2017/01/16/0200000000AEN20170116001453315.html}{According
to South Korean news reports}, other top company leaders have been
questioned.

Mr. Lee's role at Samsung is not necessarily imperiled. South Korea has
a long history of pardoning the leaders of its largest family-run
companies. Because of Samsung's size, past efforts to penalize the
company or its executives have been met with worries about the effect on
South Korea's economy.

But Mr. Lee's problems could influence the company's strategic decisions
at a time of major challenges, such as how to bounce back from the
debacle over its Galaxy Note 7. A spate of fires last year forced
Samsung to recall and then cease production of the phone, tarnishing the
company's name and making it the object of airline safety announcements
and jokes on American late-night talk shows.

Analysts highlighted two executives who run large parts of the company's
operations who they said could take on more responsibility. One is Kwon
Oh-hyun, who helped Samsung become a leader in memory and semiconductor
operations. The other, Choi Gee-sung, was crucial in building the
company's television unit, which eclipsed Sony's a decade ago.

Mr. Lee, 48, is vice chairman of Samsung, but he has been widely
considered the company's de facto leader since his father, Lee Kun-hee,
had a heart attack in 2014. The elder Mr. Lee led Samsung from its roots
as a components supplier to a maker of consumer products, establishing a
global brand and shedding a reputation for cheap goods. In one example
of his drive, two decades ago he gathered 150,000 Samsung phones that
failed quality tests and set them on fire.

Under the younger Mr. Lee, Samsung has focused again on components, a
sector that has benefited from companies wanting to add features to
their phones. Late last year, the company struck an \$8 billion deal to
buy Harman International to make smart components for cars.

But Mr. Lee has also pledged to make changes to ensure that Samsung's
corporate governance meets international standards. As a
family-controlled group with dense links among its units, the
corporation has a structure that has drawn criticism from some
investors.

Samsung has also pressed for cultural changes to foster innovation and a
start-up-style environment, and Mr. Lee has been portrayed as a part of
that shift. At Samsung, he is credited with negotiating an important
deal with Steve Jobs to help make the Apple iPod.

``Start-up Samsung,'' an initiative announced last year, sought to relax
what many described as a strict corporate culture. Announced with a
coordinated pledge by top executives, the push was
\href{https://www.techinasia.com/samsung-wants-to-be-a-startup}{derided
by some} in the technology news media as evidence of Samsung's top-down
structure.

Despite such efforts, Samsung has not followed up with a next act in the
wake of the Galaxy Note 7 debacle, and prominent projects like creating
an operating system have largely been considered failures.

Advertisement

\protect\hyperlink{after-bottom}{Continue reading the main story}

\hypertarget{site-index}{%
\subsection{Site Index}\label{site-index}}

\hypertarget{site-information-navigation}{%
\subsection{Site Information
Navigation}\label{site-information-navigation}}

\begin{itemize}
\tightlist
\item
  \href{https://help.nytimes.com/hc/en-us/articles/115014792127-Copyright-notice}{©~2020~The
  New York Times Company}
\end{itemize}

\begin{itemize}
\tightlist
\item
  \href{https://www.nytco.com/}{NYTCo}
\item
  \href{https://help.nytimes.com/hc/en-us/articles/115015385887-Contact-Us}{Contact
  Us}
\item
  \href{https://www.nytco.com/careers/}{Work with us}
\item
  \href{https://nytmediakit.com/}{Advertise}
\item
  \href{http://www.tbrandstudio.com/}{T Brand Studio}
\item
  \href{https://www.nytimes.com/privacy/cookie-policy\#how-do-i-manage-trackers}{Your
  Ad Choices}
\item
  \href{https://www.nytimes.com/privacy}{Privacy}
\item
  \href{https://help.nytimes.com/hc/en-us/articles/115014893428-Terms-of-service}{Terms
  of Service}
\item
  \href{https://help.nytimes.com/hc/en-us/articles/115014893968-Terms-of-sale}{Terms
  of Sale}
\item
  \href{https://spiderbites.nytimes.com}{Site Map}
\item
  \href{https://help.nytimes.com/hc/en-us}{Help}
\item
  \href{https://www.nytimes.com/subscription?campaignId=37WXW}{Subscriptions}
\end{itemize}
