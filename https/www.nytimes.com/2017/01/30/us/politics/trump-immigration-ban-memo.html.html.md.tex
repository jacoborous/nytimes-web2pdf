Sections

SEARCH

\protect\hyperlink{site-content}{Skip to
content}\protect\hyperlink{site-index}{Skip to site index}

\href{https://www.nytimes.com/section/politics}{Politics}

\href{https://myaccount.nytimes.com/auth/login?response_type=cookie\&client_id=vi}{}

\href{https://www.nytimes.com/section/todayspaper}{Today's Paper}

\href{/section/politics}{Politics}\textbar{}Trump Fires Acting Attorney
General Who Defied Him

\url{https://nyti.ms/2jOiH8q}

\begin{itemize}
\item
\item
\item
\item
\item
\item
\end{itemize}

Advertisement

\protect\hyperlink{after-top}{Continue reading the main story}

Supported by

\protect\hyperlink{after-sponsor}{Continue reading the main story}

\hypertarget{trump-fires-acting-attorney-general-who-defied-him}{%
\section{Trump Fires Acting Attorney General Who Defied
Him}\label{trump-fires-acting-attorney-general-who-defied-him}}

\includegraphics{https://static01.nyt.com/images/2017/01/31/us/31yates/31yates-articleLarge-v2.jpg?quality=75\&auto=webp\&disable=upscale}

By \href{http://www.nytimes.com/by/michael-d-shear}{Michael D. Shear},
\href{http://www.nytimes.com/by/mark-landler}{Mark Landler},
\href{http://www.nytimes.com/by/matt-apuzzo}{Matt Apuzzo} and
\href{http://www.nytimes.com/by/eric-lichtblau}{Eric Lichtblau}

\begin{itemize}
\item
  Jan. 30, 2017
\item
  \begin{itemize}
  \item
  \item
  \item
  \item
  \item
  \item
  \end{itemize}
\end{itemize}

WASHINGTON --- President Trump fired his acting attorney general on
Monday night, removing her as the nation's top law enforcement officer
after she defiantly refused to defend his executive order closing the
nation's borders to refugees and people from predominantly Muslim
countries.

In an escalating crisis for his 10-day-old administration, the president
declared in a statement that Sally Q. Yates, who had served as deputy
attorney general under President Barack Obama, had betrayed the
administration by announcing that Justice Department lawyers would not
defend Mr. Trump's order against legal challenges.

The president replaced Ms. Yates with Dana J. Boente, the United States
attorney for the Eastern District of Virginia, saying that he would
serve as attorney general until Congress acts to confirm Senator Jeff
Sessions of Alabama. In his first act in his new role, Mr. Boente
announced that he was rescinding Ms. Yates's order.

Monday's events have transformed the confirmation of Mr. Sessions into a
referendum on Mr. Trump's immigration order. Action in the Senate could
come as early as Tuesday.

Ms. Yates's order was a remarkable rebuke by a government official to a
sitting president, and it recalled the so-called Saturday Night Massacre
in 1973, when President Richard M. Nixon fired his attorney general and
deputy attorney general for refusing to dismiss the special prosecutor
in the Watergate case.

\includegraphics{https://static01.nyt.com/images/2017/01/31/us/31dissen-boente/31dissen-boente-articleLarge-v2.jpg?quality=75\&auto=webp\&disable=upscale}

Mr. Boente was sworn in at 9 p.m., according to White House officials,
who did not provide details about who performed the ceremony. In a
statement, Mr. Boente pledged to ``defend and enforce the laws of our
country.''

At 9:15 p.m., Ms. Yates received a hand-delivered letter at the Justice
Department that informed her that she was fired. Signed by John
DeStefano, one of Mr. Trump's White House aides, the letter informed Ms.
Yates that ``the president has removed you from the office of Deputy
Attorney General of the United States.''

Two minutes later, the White House officials lashed out at Ms. Yates in
a statement issued by Sean Spicer, the White House press secretary.

``Ms. Yates is an Obama administration appointee who is weak on borders
and very weak on illegal immigration,'' the statement said.

The firing of Ms. Yates came at the end of a turbulent three days that
began on Friday with Mr. Trump's
\href{https://www.nytimes.com/2017/01/27/us/politics/refugee-muslim-executive-order-trump.html?_r=0}{signing
of his executive order}. The action stranded travelers around the world,
led to protests around the country and created alarm inside the
bureaucracy.

Image

Sean Spicer, the White House press secretary, defended Mr. Trump's visa
ban during a briefing on Monday.Credit...Stephen Crowley/The New York
Times

Ms. Yates, like other senior government officials, was caught by
surprise by the executive order and agonized over the weekend about how
to respond, two Justice Department officials involved in the weekend
deliberations said. Ms. Yates considered resigning but she told
colleagues she did not want to leave it to her successor to face the
same dilemma.

By Monday afternoon, Ms. Yates added to a deepening sense of anxiety in
the nation's capital by publicly confronting the president with a
stinging challenge to his authority, laying bare a deep divide at the
Justice Department, within the diplomatic corps and elsewhere in the
government over the wisdom of his order.

``At present, I am not convinced that the defense of the executive order
is consistent with these responsibilities, nor am I convinced that the
executive order is lawful,'' Ms. Yates wrote in a letter to Justice
Department lawyers.

Mr. Trump's senior aides huddled together in the West Wing to determine
what to do.

They decided quickly that her insubordination could not stand, according
to an administration official familiar with the deliberations. Among the
chief concerns was whether Mr. Sessions could be confirmed quickly by
the Senate.

After Reince Priebus, the White House chief of staff, received
reassurances from Senator Mitch McConnell of Kentucky, the Republican
leader, that the confirmation was on track, aides took their
recommendation to Mr. Trump in the White House residence.

The president decided quickly: She has to go, he told them.

The official statement from Mr. Spicer accused Ms. Yates of failing to
fulfill her duty to defend a ``legal order designed to protect the
citizens of the United States'' that had been approved by the Justice
Department's Office of Legal Counsel.

``It is time to get serious about protecting our country,'' Mr. Spicer
said in the statement. He accused Democrats of holding up the
confirmation of Mr. Sessions for political reasons. ``Calling for
tougher vetting for individuals traveling from seven dangerous places is
not extreme. It is reasonable and necessary to protect our country.''

Former Justice Department officials said the president's action would
send a deep shudder through an agency that was already on edge as
officials anticipated an ideological overhaul once Mr. Sessions takes
over. One former senior official said that department lawyers would be
unnerved by the firing.

\includegraphics{https://static01.nyt.com/images/2017/01/31/us/politics/yates-cong-reax/yates-cong-reax-videoSixteenByNine3000.jpg}

Democrats, meanwhile, hailed Ms. Yates as a principled defender of what
she thought was right. Senator Chuck Schumer of New York, the Democratic
leader, said in a statement that the ``attorney general should be loyal
and pledge fidelity to the law, not the White House. The fact that this
administration doesn't understand that is chilling.''

Mr. Boente has told the White House that he is willing to sign off on
Mr. Trump's executive order on refugees and immigration, according to
Joshua Stueve, a spokesman for the United States attorney's office in
Alexandria, Va., where Mr. Boente has served as the top prosecutor since
2015.

\href{https://www.nytimes.com/interactive/2017/business/trump-immigration-ban-company-reaction.html}{}

\includegraphics{https://static01.nyt.com/images/2017/01/31/business/bizreact/bizreact-articleLarge.jpg}

\hypertarget{starbucks-exxon-apple-companies-challenging-or-silent-on-trumps-immigration-ban}{%
\subsection{Starbucks, Exxon, Apple: Companies Challenging (or Silent
on) Trump's Immigration
Ban}\label{starbucks-exxon-apple-companies-challenging-or-silent-on-trumps-immigration-ban}}

The reaction from major American companies to President Trump's order
has ranged from silence to outrage.

Mr. Boente, who has been a prosecutor with the Justice Department for 31
years, had no hesitation about accepting the acting attorney general's
job given his ``seniority and loyalty'' to the department, Mr. Stueve
said in a telephone interview on Monday night.

As acting attorney general, Ms. Yates was the only person at the Justice
Department authorized to sign applications for foreign surveillance
warrants. Administrations of both parties have interpreted surveillance
laws as requiring foreign surveillance warrants be signed only by
Senate-confirmed Justice Department officials. Mr. Boente was
Senate-confirmed as United States attorney and, though the situation is
unprecedented, the White House said he was authorized to sign the
warrants.

Ms. Yates's decision had effectively overruled a finding by the Justice
Department's Office of Legal Counsel, which had already approved the
executive order ``with respect to form and legality.''

Ms. Yates said her determination in deciding not to defend the order was
broader, however, and included questions not only about the order's
lawfulness, but also whether it was a ``wise or just'' policy. She also
alluded to unspecified statements the White House had made before
signing the order, which she factored into her review.

Mr. Trump initially responded to the letter with a post on Twitter at
7:45 p.m., complaining that the Senate's delay in confirming his cabinet
nominees had resulted in leaving Ms. Yates in place.

The 1973 ``Saturday Night Massacre'' led to a constitutional crisis that
ended when Robert H. Bork, the solicitor general, acceded to Mr. Nixon's
order and fired
\href{http://www.nytimes.com/2004/05/30/nyregion/archibald-cox-92-is-dead-helped-prosecute-watergate.html}{Archibald
Cox}, the special prosecutor.

Ms. Yates, a career prosecutor, is different because she is a holdover
from the Obama administration. She agreed to Mr. Trump's request to stay
on as acting attorney general until Mr. Sessions is confirmed to be
attorney general.

Advertisement

\protect\hyperlink{after-bottom}{Continue reading the main story}

\hypertarget{site-index}{%
\subsection{Site Index}\label{site-index}}

\hypertarget{site-information-navigation}{%
\subsection{Site Information
Navigation}\label{site-information-navigation}}

\begin{itemize}
\tightlist
\item
  \href{https://help.nytimes.com/hc/en-us/articles/115014792127-Copyright-notice}{©~2020~The
  New York Times Company}
\end{itemize}

\begin{itemize}
\tightlist
\item
  \href{https://www.nytco.com/}{NYTCo}
\item
  \href{https://help.nytimes.com/hc/en-us/articles/115015385887-Contact-Us}{Contact
  Us}
\item
  \href{https://www.nytco.com/careers/}{Work with us}
\item
  \href{https://nytmediakit.com/}{Advertise}
\item
  \href{http://www.tbrandstudio.com/}{T Brand Studio}
\item
  \href{https://www.nytimes.com/privacy/cookie-policy\#how-do-i-manage-trackers}{Your
  Ad Choices}
\item
  \href{https://www.nytimes.com/privacy}{Privacy}
\item
  \href{https://help.nytimes.com/hc/en-us/articles/115014893428-Terms-of-service}{Terms
  of Service}
\item
  \href{https://help.nytimes.com/hc/en-us/articles/115014893968-Terms-of-sale}{Terms
  of Sale}
\item
  \href{https://spiderbites.nytimes.com}{Site Map}
\item
  \href{https://help.nytimes.com/hc/en-us}{Help}
\item
  \href{https://www.nytimes.com/subscription?campaignId=37WXW}{Subscriptions}
\end{itemize}
