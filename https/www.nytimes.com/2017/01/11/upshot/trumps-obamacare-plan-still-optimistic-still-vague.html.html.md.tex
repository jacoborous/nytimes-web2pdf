Sections

SEARCH

\protect\hyperlink{site-content}{Skip to
content}\protect\hyperlink{site-index}{Skip to site index}

\href{https://myaccount.nytimes.com/auth/login?response_type=cookie\&client_id=vi}{}

\href{https://www.nytimes.com/section/todayspaper}{Today's Paper}

\href{/section/upshot}{The Upshot}\textbar{}Why Trump's Obamacare
Promise Will Be So Hard to Keep

\url{https://nyti.ms/2jx1fF3}

\begin{itemize}
\item
\item
\item
\item
\item
\item
\end{itemize}

Advertisement

\protect\hyperlink{after-top}{Continue reading the main story}

Supported by

\protect\hyperlink{after-sponsor}{Continue reading the main story}

Upshot

Public Health

\hypertarget{why-trumps-obamacare-promise-will-be-so-hard-to-keep}{%
\section{Why Trump's Obamacare Promise Will Be So Hard to
Keep}\label{why-trumps-obamacare-promise-will-be-so-hard-to-keep}}

By \href{http://www.nytimes.com/by/margot-sanger-katz}{Margot
Sanger-Katz}

\begin{itemize}
\item
  Jan. 11, 2017
\item
  \begin{itemize}
  \item
  \item
  \item
  \item
  \item
  \item
  \end{itemize}
\end{itemize}

As a candidate back in July 2015, Donald J. Trump promised that he would
repeal Obamacare and replace it with ``something terrific.''

The Senate voted, 51 to 48, on Thursday morning for a measure setting
Congress on the path toward repealing President Obama's health care law,
and Mr. Trump is now a few days from taking office. The public, however,
knows little more about his proposal than it did in 2015.

In comments
\href{https://www.nytimes.com/2017/01/10/us/repeal-affordable-care-act-donald-trump.html}{to
The New York Times} on Tuesday and in his news
\href{https://www.nytimes.com/2017/01/11/us/politics/trump-press-conference-transcript.html}{conference
on Wednesday}, Mr. Trump described when a Republican health reform bill
would be released --- ``very quickly.'' But he has yet to give details
about the policies it would contain.

``We're going to have a health care that is far less expensive and far
better,'' he said Wednesday.

\includegraphics{https://static01.nyt.com/images/2017/01/12/us/12up-healthplan1/12up-healthplan1-articleLarge.jpg?quality=75\&auto=webp\&disable=upscale}

Mr. Trump accurately describes problems with the current health care
system for Americans under 65: ``You have deductibles that are so high,
that after people go broke paying their premiums which are going through
the roof, the health care can't even be used by them because their
deductibles bills are so high.''

\href{http://www.cbsnews.com/news/face-the-nation-transcript-january-8-2017-mcconnell-priebus-booker-morell-woolsey/}{Mitch
McConnell}, the Senate majority leader, and House Speaker Paul Ryan have
also spoken forcefully in recent days about how health care is too
expensive.

Premiums for health insurance plans in the United States are high. And
\href{https://www.nytimes.com/2016/09/15/business/health-insurance-analysis-kaiser.html}{increasing
deductibles} can make needed coverage a financial stretch
\href{https://www.nytimes.com/2016/01/06/upshot/lost-jobs-houses-savings-even-insured-often-face-crushing-medical-debt.html}{even
for the insured}.
\href{http://kff.org/health-costs/poll-finding/kaiser-health-tracking-poll-health-care-priorities-for-2017/}{Recent
polling} from the nonpartisan Kaiser Family Foundation suggests that the
public agrees with Mr. Trump's assessment: High out-of-pocket spending
on health care is Americans' No. 1 health care concern. (Mr. Trump has
promised that he will not make major changes to Medicare, the program
for Americans 65 and older.)

But solving those problems is not as easy as identifying them. The real
reason health care premiums and deductibles are so high is that medical
care is very expensive in the United States ---
\href{http://stats.oecd.org/Index.aspx?DataSetCode=SHA}{far more costly}
than it is anywhere else in the world. The United States pays very high
prices to doctors and hospitals and drug and device makers, and
Americans use a lot of that expensive medical care. That means that the
country spent far more on health care than its peers even when tens of
millions of Americans lacked health coverage.

Obamacare has been successful in getting health insurance to people who
lacked it before. About 20 million more Americans had insurance last
year than before the law passed, according to
\href{https://aspe.hhs.gov/sites/default/files/pdf/187551/ACA2010-2016.pdf}{an
Obama administration estimate}. But the health law, largely focused on
health insurance regulation, did not drive down the cost of medical
treatments. Health care, and health insurance, continues to be
expensive.

That means that a Republican health reform plan that is both cheaper and
better than Obamacare will be hard to deliver.

Republicans in Congress and right-leaning think tanks have put together
a number of possible Obamacare replacement plans. They are all slightly
different, and it is unclear which one Mr. Trump and congressional
leaders will choose. But none of them solve both sides of the ``less
expensive and far better'' equation.

Most of the G.O.P. plans manage to be less expensive for the federal
government --- by offering stingier federal payments in helping people
buy insurance and allowing the coverage people buy to be skimpier. But
those proposals will tend to increase, not decrease, the amount many
Americans spend on their health care.

\href{https://www.nytimes.com/2015/08/20/upshot/walker-and-rubio-health-proposals-are-less-concerned-about-poor.html}{Lower-income
people will end up paying a larger share} of their income to buy
coverage than they do under Obamacare. Deductibles and other forms of
out-of-pocket spending, capped under Obamacare, will tend to rise in
many plans. Millions to tens of millions fewer Americans will have
coverage under such plans, according to independent estimates.

Republicans also want to pare back the minimum package of benefits that
plans must cover, which will drive up costs substantially for some
patients, while reducing them for others. If the bill eliminates
requirements to pay for
\href{http://www.theatlantic.com/politics/archive/2013/11/why-is-maternity-care-such-an-issue-for-obamacare-opponents/281396/}{maternity
care} or prescription drugs, for example, that could lower the sticker
price of a health plan, but will make health care much more expensive
for anyone who has a baby or takes medication.

There will be people who will be better off under a Republican plan.
Higher-income, healthy people who buy their own insurance have been the
most disadvantaged group under Obamacare, and their fortunes would
improve. The Republican plans, with their skimpier benefits, and more
generous tax assistance for the wealthy, would offer them a better deal.

The G.O.P. plans tend to be worse for people who need insurance and are
poor or have major health problems. (Some sick people may be far worse
off. Mr. Trump has promised that people with
\href{https://www.nytimes.com/2016/11/15/upshot/why-keeping-only-the-popular-parts-of-obamacare-wont-work.html}{pre-existing
health conditions will be covered} under his plan, but not all the
Republican plans offer them the kind of coverage that they can get under
Obamacare.)

The Affordable Care Act is a case study in these trade-offs. Most of the
things its creators did to try to make health insurance ``far better,''
like requiring minimum benefits or banning lifetime coverage limits,
also made it more expensive. The things they did to make insurance
``less expensive,'' like encouraging higher deductibles or requiring all
Americans to buy health insurance or pay a fine, are top anti-Obamacare
talking points.

The consensus Democratic approach to making Obamacare ``far better'' has
been to make it more expensive for the federal government, but less
expensive for individuals. Proposals circulated by President Obama and
Hillary Clinton would involve more federal spending on subsidies to help
make insurance more affordable for more people, but at the expense of
higher taxes.

So far, the Republican plans have tended to engage in the same
trade-offs, but tilt in the opposite direction, emphasizing government
savings over program generosity.

The recent statements from Mr. Trump suggest that the coming replacement
plan --- promised in the next few weeks --- will be able to achieve both
goals simultaneously. Mr. Trump has provided almost no detail about what
will be in it.

Advertisement

\protect\hyperlink{after-bottom}{Continue reading the main story}

\hypertarget{site-index}{%
\subsection{Site Index}\label{site-index}}

\hypertarget{site-information-navigation}{%
\subsection{Site Information
Navigation}\label{site-information-navigation}}

\begin{itemize}
\tightlist
\item
  \href{https://help.nytimes.com/hc/en-us/articles/115014792127-Copyright-notice}{©~2020~The
  New York Times Company}
\end{itemize}

\begin{itemize}
\tightlist
\item
  \href{https://www.nytco.com/}{NYTCo}
\item
  \href{https://help.nytimes.com/hc/en-us/articles/115015385887-Contact-Us}{Contact
  Us}
\item
  \href{https://www.nytco.com/careers/}{Work with us}
\item
  \href{https://nytmediakit.com/}{Advertise}
\item
  \href{http://www.tbrandstudio.com/}{T Brand Studio}
\item
  \href{https://www.nytimes.com/privacy/cookie-policy\#how-do-i-manage-trackers}{Your
  Ad Choices}
\item
  \href{https://www.nytimes.com/privacy}{Privacy}
\item
  \href{https://help.nytimes.com/hc/en-us/articles/115014893428-Terms-of-service}{Terms
  of Service}
\item
  \href{https://help.nytimes.com/hc/en-us/articles/115014893968-Terms-of-sale}{Terms
  of Sale}
\item
  \href{https://spiderbites.nytimes.com}{Site Map}
\item
  \href{https://help.nytimes.com/hc/en-us}{Help}
\item
  \href{https://www.nytimes.com/subscription?campaignId=37WXW}{Subscriptions}
\end{itemize}
