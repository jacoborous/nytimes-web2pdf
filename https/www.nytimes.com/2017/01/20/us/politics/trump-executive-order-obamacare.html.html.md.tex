Sections

SEARCH

\protect\hyperlink{site-content}{Skip to
content}\protect\hyperlink{site-index}{Skip to site index}

\href{https://www.nytimes.com/section/politics}{Politics}

\href{https://myaccount.nytimes.com/auth/login?response_type=cookie\&client_id=vi}{}

\href{https://www.nytimes.com/section/todayspaper}{Today's Paper}

\href{/section/politics}{Politics}\textbar{}Trump Issues Executive Order
Scaling Back Parts of Obamacare

\url{https://nyti.ms/2kamuwO}

\begin{itemize}
\item
\item
\item
\item
\item
\end{itemize}

Advertisement

\protect\hyperlink{after-top}{Continue reading the main story}

Supported by

\protect\hyperlink{after-sponsor}{Continue reading the main story}

\hypertarget{trump-issues-executive-order-scaling-back-parts-of-obamacare}{%
\section{Trump Issues Executive Order Scaling Back Parts of
Obamacare}\label{trump-issues-executive-order-scaling-back-parts-of-obamacare}}

\includegraphics{https://static01.nyt.com/images/2017/01/21/us/21orders1/21orders1-articleInline.jpg?quality=75\&auto=webp\&disable=upscale}

By \href{https://www.nytimes.com/by/julie-hirschfeld-davis}{Julie
Hirschfeld Davis} and
\href{https://www.nytimes.com/by/robert-pear}{Robert Pear}

\begin{itemize}
\item
  Jan. 20, 2017
\item
  \begin{itemize}
  \item
  \item
  \item
  \item
  \item
  \end{itemize}
\end{itemize}

WASHINGTON --- In his first executive order, President Trump on Friday
directed government agencies to scale back as many aspects of the
Affordable Care Act as possible, moving within hours of being sworn in
to fulfill his pledge to eviscerate Barack Obama's signature health care
law.

The one-page order, which Mr. Trump signed in a hastily arranged Oval
Office ceremony shortly before departing for the inaugural balls, gave
no specifics about which aspects of the law it was targeting. But its
broad language gave federal agencies wide latitude to change, delay or
waive provisions of the law that they deemed overly costly for insurers,
drug makers, doctors, patients or states, suggesting that it could have
wide-ranging impact, and essentially allowing the dismantling of the law
to begin even before Congress moves to repeal it.

\href{https://www.nytimes.com/interactive/2016/12/03/us/politics/why-it-will-be-hard-to-repeal-obamacare.html}{}

\includegraphics{https://static01.nyt.com/images/2016/12/03/us/politics/why-it-will-be-hard-to-repeal-obamacare-1480740532639/why-it-will-be-hard-to-repeal-obamacare-1480740532639-thumbLarge.jpg}

\hypertarget{how-republicans-can-repeal-obamacare-piece-by-piece}{%
\subsection{How Republicans Can Repeal Obamacare Piece by
Piece}\label{how-republicans-can-repeal-obamacare-piece-by-piece}}

Peeling away pieces of the law could lead to market chaos.

The order states what Mr. Trump made clear during his campaign: that it
is his administration's policy to seek the ``prompt repeal'' of the law,
which has come to be known as Obamacare. But he and Republicans on
Capitol Hill have not yet devised a replacement, making such action
unlikely in the immediate term.

``In the meantime,'' the order said, ``pending such repeal, it is
imperative for the executive branch to ensure that the law is being
efficiently implemented, take all actions consistent with law to
minimize the unwarranted economic and regulatory burdens of the act, and
prepare to afford the states more flexibility and control to create a
more free and open health care market.''

The order has symbolic as well as substantive significance, allowing Mr.
Trump to claim he acted immediately to do away with a health care law he
has repeatedly called disastrous, even while it remains in place and he
navigates the politically perilous process of repealing and replacing
it.

\href{https://www.nytimes.com/interactive/2017/01/20/us/politics/donald-trump-inauguration-speech-transcript.html}{}

\includegraphics{https://static01.nyt.com/images/2017/01/20/us/politics/donald-trump-inauguration-speech-transcript-1484930410968/donald-trump-inauguration-speech-transcript-1484930410968-square640-v3.jpg}

\hypertarget{donald-trumps-inaugural-speech-annotated}{%
\subsection{Donald Trump's Inaugural Speech,
Annotated}\label{donald-trumps-inaugural-speech-annotated}}

New York Times reporters analyze the 45th president's comments.

Using the phrase ``to the maximum extent permitted by law,'' the order
directs federal agencies to move decisively to implement changes,
including granting flexibility that insurers and states had long
implored the Obama administration to provide.

It also instructs them to work to create a system that allows the sale
of health insurance across state lines, which Republicans have long
proposed as the centerpiece of an alternative to the law.

``This action demonstrates that President Trump is committed to fixing
the damage caused by Obamacare as soon as possible,'' said Senator John
Barrasso, Republican of Wyoming.

The order does not direct the Department of Health and Human Services to
ease any particular aspect of the 2010 law, but it could result in a
substantial weakening of one of its central features: the so-called
``individual mandate'' that requires most Americans to have health
insurance or pay a tax penalty.

While the Obama administration allowed ``hardship exemptions'' to the
mandate, the Trump administration could conceivably interpret the
requirement in a more lenient way, so that more people would not be
penalized.

Likewise, federal officials could be more receptive to state requests
for waivers under Medicaid, the federal-state program that covers more
than 70 million low-income people. A number of Republican governors and
state legislators would like to charge higher premiums or co-payments
than are now allowed. Some states want to provide a less generous, less
expensive package of benefits, or require some able-bodied adults to
engage in work activities as a condition of receiving Medicaid.

\includegraphics{https://static01.nyt.com/images/2017/01/21/us/21MEDIA-01p/21inauguration-trump-highlights-videoSixteenByNineJumbo1600-v2.jpg}

Still, while Mr. Trump's directive allows officials to take steps that
increase costs for consumers, that result is not inevitable. Indeed, the
order says officials should try to reduce costs and burdens on
consumers.

Over the last six years, insurance company executives have bitterly
complained that federal insurance regulations were extremely
prescriptive and onerous. By relaxing some of those rules, the Trump
administration could make the individual insurance market more
attractive to insurers. And insurers might then be more willing to stay
in or return to the public marketplaces established under the Affordable
Care Act.

In the last year, a number of insurers have dropped out of those
markets, leaving consumers with fewer health plans to choose from.

House Republican leaders recently asked governors for recommendations on
health policy, and governors from both parties said the federal
government should scale back its regulation of health insurance.

Gov. Bill Haslam of Tennessee, a Republican, said this month that
federal officials should ``reconsider the premise that health insurance
public policy should be directed from Washington.'' He said that federal
rules for setting insurance rates and defining ``essential health
benefits'' should be more flexible.

Advertisement

\protect\hyperlink{after-bottom}{Continue reading the main story}

\hypertarget{site-index}{%
\subsection{Site Index}\label{site-index}}

\hypertarget{site-information-navigation}{%
\subsection{Site Information
Navigation}\label{site-information-navigation}}

\begin{itemize}
\tightlist
\item
  \href{https://help.nytimes.com/hc/en-us/articles/115014792127-Copyright-notice}{©~2020~The
  New York Times Company}
\end{itemize}

\begin{itemize}
\tightlist
\item
  \href{https://www.nytco.com/}{NYTCo}
\item
  \href{https://help.nytimes.com/hc/en-us/articles/115015385887-Contact-Us}{Contact
  Us}
\item
  \href{https://www.nytco.com/careers/}{Work with us}
\item
  \href{https://nytmediakit.com/}{Advertise}
\item
  \href{http://www.tbrandstudio.com/}{T Brand Studio}
\item
  \href{https://www.nytimes.com/privacy/cookie-policy\#how-do-i-manage-trackers}{Your
  Ad Choices}
\item
  \href{https://www.nytimes.com/privacy}{Privacy}
\item
  \href{https://help.nytimes.com/hc/en-us/articles/115014893428-Terms-of-service}{Terms
  of Service}
\item
  \href{https://help.nytimes.com/hc/en-us/articles/115014893968-Terms-of-sale}{Terms
  of Sale}
\item
  \href{https://spiderbites.nytimes.com}{Site Map}
\item
  \href{https://help.nytimes.com/hc/en-us}{Help}
\item
  \href{https://www.nytimes.com/subscription?campaignId=37WXW}{Subscriptions}
\end{itemize}
