Sections

SEARCH

\protect\hyperlink{site-content}{Skip to
content}\protect\hyperlink{site-index}{Skip to site index}

\href{https://www.nytimes.com/section/politics}{Politics}

\href{https://myaccount.nytimes.com/auth/login?response_type=cookie\&client_id=vi}{}

\href{https://www.nytimes.com/section/todayspaper}{Today's Paper}

\href{/section/politics}{Politics}\textbar{}Obama Pardons James
Cartwright, General Who Lied to F.B.I. in Leak Case

\url{https://nyti.ms/2k2iW3H}

\begin{itemize}
\item
\item
\item
\item
\item
\end{itemize}

Advertisement

\protect\hyperlink{after-top}{Continue reading the main story}

Supported by

\protect\hyperlink{after-sponsor}{Continue reading the main story}

\hypertarget{obama-pardons-james-cartwright-general-who-lied-to-fbi-in-leak-case}{%
\section{Obama Pardons James Cartwright, General Who Lied to F.B.I. in
Leak
Case}\label{obama-pardons-james-cartwright-general-who-lied-to-fbi-in-leak-case}}

\includegraphics{https://static01.nyt.com/images/2017/01/18/us/18cartwright/11cartwright-articleLarge.jpg?quality=75\&auto=webp\&disable=upscale}

By \href{http://www.nytimes.com/by/charlie-savage}{Charlie Savage}

\begin{itemize}
\item
  Jan. 17, 2017
\item
  \begin{itemize}
  \item
  \item
  \item
  \item
  \item
  \end{itemize}
\end{itemize}

WASHINGTON --- President Obama on Tuesday pardoned James E. Cartwright,
a retired Marine Corps general and former vice chairman of the Joint
Chiefs of Staff who pleaded guilty to lying to the F.B.I. about his
discussions with reporters about Iran's nuclear program, saving him from
a possible prison sentence.

General Cartwright, who was a key member of Mr. Obama's national
security team in his first term and earned a reputation as the
president's favorite general, pleaded guilty late last year to
misleading investigators looking into the leaking of classified
information about cyberattacks against Iran.

He was
\href{https://www.nytimes.com/2017/01/10/us/politics/leak-iran-james-cartwright.html}{due
to be sentenced} this month. His defense team had asked for a year of
probation and 600 hours of community service, but prosecutors had asked
the judge overseeing his case to send him to prison for two years.

Now, the retired general will be spared such punishment.

Both General Cartwright and his lawyer, Gregory Craig, a former White
House counsel to Mr. Obama, thanked the president in statements. ``The
president's decision is wise and just, and it achieves the right
result,'' Mr. Craig said. ``It allows General Cartwright to continue his
life's work --- to serve, protect and defend the nation he loves. It
allows the nation to continue to benefit from his vast experience and
knowledge.''

General Cartwright left government in 2011. The leak investigation that
ensnared him began in June 2012, when David E. Sanger, a reporter for
The New York Times, published a book, ``Confront and Conceal,'' and
\href{http://www.nytimes.com/2012/06/01/world/middleeast/obama-ordered-wave-of-cyberattacks-against-iran.html}{an
article in The Times} that described Operation Olympic Games, an
American-Israeli covert effort to sabotage Iranian nuclear centrifuges
with a computer virus. F.B.I. agents came to believe that General
Cartwright had also been a source for a February 2012 Newsweek
\href{http://www.newsweek.com/obamas-dangerous-game-iran-65711}{article}
that discussed cyberattacks against Iran.

But when F.B.I. agents interviewed the retired general about the book
and articles, he initially lied about his discussions with the
journalists, according to a government sentencing memo.

The memo said the agents showed the general emails that contradicted his
account, and he passed out and was hospitalized. Several days later,
when the interview resumed, he changed his account of the discussions.

General Cartwright's defense team has argued that he spoke with the
reporters in order to shape stories they had already reported and to try
to prevent publication of more damaging information. Prosecutors had
cast doubt on that theory, arguing that he did not articulate this
approach when they interviewed him.

But in a background briefing with reporters, a senior White House
official said Mr. Obama had made his decision in part because of General
Cartwright's description of his motive, as well as because of a letter
by Mr. Sanger saying that he had already learned about the program
before speaking to the general and that the conversation with the
general informed his thinking about which information to withhold.

In a statement, Mr. Sanger said he was ``happy to see that President
Obama has taken this step,'' reiterating that he had had ``many sources,
from around the world'' and that General Cartwright had ``showed concern
that information damaging to U.S. interests not be made public.''

Mr. Sanger added: ``The Times has frequently said that stories like this
one are critical to helping Americans understand how decisions on vital
national security matters are made. Leak investigations have the effect
of making people less willing to talk, and the result is often a loss
for our democracy.''

In November, General Cartwright agreed to a plea deal to bring an end to
the four-year investigation in which he pleaded guilty to misleading the
F.B.I., but not to the unauthorized disclosure of information. At his
plea hearing, he said that it was ``wrong'' to have misled the F.B.I.
and that he accepted full responsibility.

``I knew I was not the source of the story, and I didn't want to be
blamed for the leak,'' he said then. ``My only goal in talking to the
reporters was to protect American interests and lives; I love my country
and continue to this day to do everything I can to defend it.''

Advertisement

\protect\hyperlink{after-bottom}{Continue reading the main story}

\hypertarget{site-index}{%
\subsection{Site Index}\label{site-index}}

\hypertarget{site-information-navigation}{%
\subsection{Site Information
Navigation}\label{site-information-navigation}}

\begin{itemize}
\tightlist
\item
  \href{https://help.nytimes.com/hc/en-us/articles/115014792127-Copyright-notice}{©~2020~The
  New York Times Company}
\end{itemize}

\begin{itemize}
\tightlist
\item
  \href{https://www.nytco.com/}{NYTCo}
\item
  \href{https://help.nytimes.com/hc/en-us/articles/115015385887-Contact-Us}{Contact
  Us}
\item
  \href{https://www.nytco.com/careers/}{Work with us}
\item
  \href{https://nytmediakit.com/}{Advertise}
\item
  \href{http://www.tbrandstudio.com/}{T Brand Studio}
\item
  \href{https://www.nytimes.com/privacy/cookie-policy\#how-do-i-manage-trackers}{Your
  Ad Choices}
\item
  \href{https://www.nytimes.com/privacy}{Privacy}
\item
  \href{https://help.nytimes.com/hc/en-us/articles/115014893428-Terms-of-service}{Terms
  of Service}
\item
  \href{https://help.nytimes.com/hc/en-us/articles/115014893968-Terms-of-sale}{Terms
  of Sale}
\item
  \href{https://spiderbites.nytimes.com}{Site Map}
\item
  \href{https://help.nytimes.com/hc/en-us}{Help}
\item
  \href{https://www.nytimes.com/subscription?campaignId=37WXW}{Subscriptions}
\end{itemize}
