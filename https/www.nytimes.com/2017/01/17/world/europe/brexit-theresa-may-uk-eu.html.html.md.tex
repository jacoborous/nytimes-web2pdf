Sections

SEARCH

\protect\hyperlink{site-content}{Skip to
content}\protect\hyperlink{site-index}{Skip to site index}

\href{https://www.nytimes.com/section/world/europe}{Europe}

\href{https://myaccount.nytimes.com/auth/login?response_type=cookie\&client_id=vi}{}

\href{https://www.nytimes.com/section/todayspaper}{Today's Paper}

\href{/section/world/europe}{Europe}\textbar{}In `Brexit' Speech,
Theresa May Outlines Clean Break for U.K.

\url{https://nyti.ms/2iGTXB0}

\begin{itemize}
\item
\item
\item
\item
\item
\item
\end{itemize}

Advertisement

\protect\hyperlink{after-top}{Continue reading the main story}

Supported by

\protect\hyperlink{after-sponsor}{Continue reading the main story}

\hypertarget{in-brexit-speech-theresa-may-outlines-clean-break-for-uk}{%
\section{In `Brexit' Speech, Theresa May Outlines Clean Break for
U.K.}\label{in-brexit-speech-theresa-may-outlines-clean-break-for-uk}}

\includegraphics{https://static01.nyt.com/images/2017/01/18/world/europe/18Brexit3/18Brexit3-videoSixteenByNine3000.jpg}

By \href{http://www.nytimes.com/by/stephen-castle}{Stephen Castle} and
\href{http://www.nytimes.com/by/steven-erlanger}{Steven Erlanger}

\begin{itemize}
\item
  Jan. 17, 2017
\item
  \begin{itemize}
  \item
  \item
  \item
  \item
  \item
  \item
  \end{itemize}
\end{itemize}

LONDON --- ``Get on with it.''

With those words late in a
\href{http://www.nytimes.com/2017/01/17/world/europe/brexit-speech-quotes.html}{major
speech} on Tuesday, Prime Minister Theresa May charted Britain's course
toward a clean break with the European Union and expressed her fondest
hope: that the time for ``division and discord'' is over.

Her much-anticipated
\href{http://www.telegraph.co.uk/news/2017/01/17/theresa-mays-brexit-speech-full/}{speech}
outlined what promised to be a hugely complex, drawn-out negotiation,
and it defined the broad objectives, but not the details, of British
withdrawal. ``The United Kingdom is leaving the European Union, and my
job is to get the right deal for Britain as we do,'' she said.

With the address, Mrs. May began the jockeying that will lead to a break
after more than four decades of tight integration, and define Britain's
relations with its neighbors for decades to come.

She confirmed that Britain is determined to regain control of migration
from the European Union and rejected the supremacy of the European Court
of Justice. That stance is anathema to the European Union, which has
made the free movement of people --- as well as goods, capital and
services --- a bedrock principle and which relies on the court to
arbitrate.

``Let me be clear,'' Mrs. May said, acknowledging the differences.
``What I am proposing cannot mean remaining in the single market.''

She said that she hoped to complete a final deal with the European Union
by March 2019 and that it would be voted on by both houses of
Parliament. She was not clear about what would happen if Parliament
rejected the deal, though some speculated that a rejection would result
in the sort of chaotic, ``cliff edge'' breakup that she and Britain's
bankers and business leaders hoped to avoid.

Mrs. May struck a diplomatic note, including an appeal for a new
partnership with Continental Europe, but not at all costs.

``We seek a new and equal partnership --- between an independent,
self-governing, global Britain and our friends and allies in the E.U.,''
Mrs. May said. ``Not partial membership of the European Union, associate
membership of the European Union, or anything that leaves us half in,
half out.''

And she appealed to Britons, especially to those in Scotland, Wales and
Northern Ireland, to unite behind the government and stop refighting the
referendum that backed leaving the bloc, which she had opposed.

The reaction among her opponents in the ``remain'' camp was predictably
harsh and seemed to herald a long and bruising process.

``Theresa May has confirmed Britain is heading for a hard Brexit,'' said
Tim Farron, the leader of the centrist Liberal Democrats. ``She claimed
people voted to leave the single market. They didn't. She has made the
choice to do massive damage to the British economy.''

The Labour leader, Jeremy Corbyn, accused the Tories of turning Britain
into ``a bargain-basement tax haven,'' with their
\href{https://www.welt.de/english-news/article161182946/Philip-Hammond-issues-threat-to-EU-partners.html}{recent
threat} to slash corporate taxes if a good deal cannot be reached with
the European Union.

The speech, which provided some degree of substance, gained a warmer
reception in the markets, with the pound seeming to stabilize after
several jittery days. It rose as much as 1 percent against the dollar
during her speech, while stocks on the London exchange fell.

Supporters of a withdrawal have been encouraged as well by reports that
other countries in the bloc
\href{https://www.theguardian.com/business/2017/jan/13/eu-negotiator-wants-special-deal-over-access-to-city-post-brexit}{have
recognized that they might suffer} if there was a complete rupture and
they were denied access to London's financial services sector. But
British
\href{https://www.nytimes.com/2017/01/15/world/europe/brexit-firms-business-relocate.html}{businesses
remained nervous}.

Carolyn Fairbairn, director general of the Confederation of British
Industry, a business lobbying group, welcomed the greater clarity
provided by Mrs. May but worried that ``ruling out membership of the
single market has reduced options for maintaining a barrier-free trading
relationship between the U.K. and the E.U.''

Kallum Pickering, senior Britain economist at Berenberg Bank in London,
was more blunt, writing in an analysis that ``as we do not expect the
E.U. to compromise its principles, the U.K. is set to face significant
economic consequences from Brexit.''

Few analysts expect the negotiations to go as smoothly or as quickly as
Mrs. May seemed to say in her speech. In recognition of the troubles
that may lie ahead, Mark Boleat, the policy chairman for the City of
London Corporation, the heart of Britain's financial services industry,
urged Mrs. May to swiftly secure a transition deal that would provide
the certainty that businesses crave.

Charles Brasted, a partner at Hogan Lovells, an international law firm,
cautioned that the deal Mrs. May wanted was likely to be seen by the
European Union as ``precisely the cherry picking that they have warned
against.'' He added: ``The objectives are now clear. The path towards
them is uncharted.''

\includegraphics{https://static01.nyt.com/images/2017/01/18/world/18BREXIT-2/18BREXIT-2-articleLarge.jpg?quality=75\&auto=webp\&disable=upscale}

But he warned that ``every one of the aspirations expressed by the U.K.
government today will demand exceptional political skill to negotiate
and will be complex to implement legally and commercially.''

In Europe, Donald Tusk, the president of the European Council,
\href{https://twitter.com/eucopresident/status/821364567150362624?s=03}{said
on Twitter}: ``Sad process, surrealistic times but at least more
realistic announcement on \#Brexit.'' Germany's foreign minister,
Frank-Walter Steinmeier, welcomed Mrs. May's ``desire for a positive and
constructive partnership, a friendship with a strong E.U.,'' which
Germany would reciprocate.

Mrs. May's speech, delivered in the grand surroundings of Lancaster
House in London, was the most closely watched statement on European
policy since January 2013, when the prime minister at the time, David
Cameron, promised to hold a referendum on European Union membership.

The prospect that Britain would remain part of the single market has
been fading since Mrs. May said in October that she would demand
complete control of migration from the European Union and release from
the European Court of Justice.

The extent to which Mrs. May would be willing to compromise to maintain
some access to the single market and to the customs union for goods was
less clear. Membership in the customs union limits the ability of member
countries to strike individual free-trade deals with non-European
nations. So she said she wanted a deal that would allow Britain to trade
freely with the world, but still have as much tariff-free trade as
possible with European Union countries.

Ideally,
\href{https://www.nytimes.com/2016/11/29/world/europe/uk-brexit-european-union.html}{Britain
would like to have its cake and eat it}, in the memorable phrase of the
foreign secretary, Boris Johnson. In other words, Britain would reject
what it disliked about the bloc, like freedom of movement, but keep
trade unencumbered as it tried to get the best possible trading deal
consistent with its other objectives.

While European nations are expected to be stingy with market access, Mr.
Pickering says he believes they will eventually bend.

In the final deal, he wrote, he still expects Britain and the European
Union to agree to a deal in which ``the U.K. maintains a good level of
access to the E.U.'s goods markets and limited access to the
less-developed services markets.''

``Crucially, we expect the U.K. to lose its E.U. financial services
passport,'' Mr. Pickering wrote, referring to a system that allowed
banks based in Britain to offer financial services throughout the bloc.
``This follows from the U.K. raising some modest barriers to migration
from the E.U.''

Many European Union countries have backed taking a hard line against
Britain to send a message to other member states that might consider
leaving. Anticipating that, Mrs. May said that Britain wanted a
successful European Union and a friendly partnership, but that ``no deal
for Britain is better than a bad deal for Britain.''

Advertisement

\protect\hyperlink{after-bottom}{Continue reading the main story}

\hypertarget{site-index}{%
\subsection{Site Index}\label{site-index}}

\hypertarget{site-information-navigation}{%
\subsection{Site Information
Navigation}\label{site-information-navigation}}

\begin{itemize}
\tightlist
\item
  \href{https://help.nytimes.com/hc/en-us/articles/115014792127-Copyright-notice}{©~2020~The
  New York Times Company}
\end{itemize}

\begin{itemize}
\tightlist
\item
  \href{https://www.nytco.com/}{NYTCo}
\item
  \href{https://help.nytimes.com/hc/en-us/articles/115015385887-Contact-Us}{Contact
  Us}
\item
  \href{https://www.nytco.com/careers/}{Work with us}
\item
  \href{https://nytmediakit.com/}{Advertise}
\item
  \href{http://www.tbrandstudio.com/}{T Brand Studio}
\item
  \href{https://www.nytimes.com/privacy/cookie-policy\#how-do-i-manage-trackers}{Your
  Ad Choices}
\item
  \href{https://www.nytimes.com/privacy}{Privacy}
\item
  \href{https://help.nytimes.com/hc/en-us/articles/115014893428-Terms-of-service}{Terms
  of Service}
\item
  \href{https://help.nytimes.com/hc/en-us/articles/115014893968-Terms-of-sale}{Terms
  of Sale}
\item
  \href{https://spiderbites.nytimes.com}{Site Map}
\item
  \href{https://help.nytimes.com/hc/en-us}{Help}
\item
  \href{https://www.nytimes.com/subscription?campaignId=37WXW}{Subscriptions}
\end{itemize}
