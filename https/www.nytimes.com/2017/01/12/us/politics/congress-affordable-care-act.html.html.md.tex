Sections

SEARCH

\protect\hyperlink{site-content}{Skip to
content}\protect\hyperlink{site-index}{Skip to site index}

\href{https://www.nytimes.com/section/politics}{Politics}

\href{https://myaccount.nytimes.com/auth/login?response_type=cookie\&client_id=vi}{}

\href{https://www.nytimes.com/section/todayspaper}{Today's Paper}

\href{/section/politics}{Politics}\textbar{}House Expected to Follow
Senate's Lead on Rush to Repeal Health Law

\url{https://nyti.ms/2iq2yrK}

\begin{itemize}
\item
\item
\item
\item
\item
\end{itemize}

Advertisement

\protect\hyperlink{after-top}{Continue reading the main story}

Supported by

\protect\hyperlink{after-sponsor}{Continue reading the main story}

\hypertarget{house-expected-to-follow-senates-lead-on-rush-to-repeal-health-law}{%
\section{House Expected to Follow Senate's Lead on Rush to Repeal Health
Law}\label{house-expected-to-follow-senates-lead-on-rush-to-repeal-health-law}}

\includegraphics{https://static01.nyt.com/images/2017/01/13/us/13cong1/13cong1-articleLarge.jpg?quality=75\&auto=webp\&disable=upscale}

By \href{http://www.nytimes.com/by/thomas-kaplan}{Thomas Kaplan},
\href{https://www.nytimes.com/by/robert-pear}{Robert Pear} and
\href{https://www.nytimes.com/by/emmarie-huetteman}{Emmarie Huetteman}

\begin{itemize}
\item
  Jan. 12, 2017
\item
  \begin{itemize}
  \item
  \item
  \item
  \item
  \item
  \end{itemize}
\end{itemize}

WASHINGTON --- The House is expected to give final approval on Friday to
a measure that would allow Republicans to speedily gut the Affordable
Care Act with no threat of a Senate filibuster, a move that would thrust
the question of what health law would come next front and center even
before President-elect Donald J. Trump takes office.

The House vote would come after the Senate narrowly approved the same
measure, a budget blueprint, early Thursday morning. Americans woke up
Thursday to the realization that a Republican Congress was serious about
repealing President Obama's signature domestic achievement --- a move
that could leave 20 million Americans unsure of their health coverage
and millions more wondering if protections offered by the Affordable
Care Act could soon be taken away.

``This is a critical step forward, the first step toward bringing relief
from this failed law,'' Senator Mitch McConnell of Kentucky, the
majority leader, said.

Democrats said the rush to repeal was the height of legislative
irresponsibility and would endanger the health of millions.

``For the life of me, I can't understand the need to take health care
away from people, and why in the world anybody would even contemplate
doing that without something to replace it,'' said Representative Louise
M. Slaughter of New York. ``Just snatching it out from under them and
it's gone. I think that there's going to be a mighty rumble in this
country, an outburst of anger and fear.''

What comes next may be the most pressing problem facing Republicans, who
may find that dismantling the health law is far easier than replacing it
with one that can unite their fractious members --- and win over some
Democrats.

After a marathon session, the Senate voted 51 to 48 to approve a budget
measure that would clear the way for the health care law to be repealed
with a simple Senate majority. As the House approached its vote, some
Republicans remained reluctant to act without a clear strategy to
replace the health law.

``We'd like to see a little more flesh on the bone before we sign on the
dotted line,'' said Representative Andy Harris of Maryland, an
anesthesiologist and a member of the conservative House Freedom Caucus.

Republicans skeptical of moving forward risked looking hostile to the
repeal effort. And there was a prevailing sense of the importance of
following through on a campaign promise upon which so many House
Republicans had staked their political reputations.

``This is an issue that really and truly, in some ways, put two-thirds
of our conference here,'' said Representative Doug Collins of Georgia, a
member of the party's leadership team.

``Everybody wants to get it right,'' he said.

Republican leaders sought to reassure members that the House budget vote
--- procedurally important as it is --- is only the first step in an
exhaustive process to repeal and replace the Affordable Care Act. Four
committees in the House and Senate would then be tasked with drafting
the legislation that would gut the existing health law.

\includegraphics{https://static01.nyt.com/images/2017/01/13/us/13cong2/13cong2-articleLarge.jpg?quality=75\&auto=webp\&disable=upscale}

The concerns fostered a remarkable alignment between some centrist
Republicans and their counterparts in the House Freedom Caucus, the
hard-right group that is disposed to disagree with its own party's
leaders.

Speaker Paul D. Ryan of Wisconsin worked to soothe concerns even as he
expressed the urgent need to get rid of a law that Republicans hate.
Republicans would embark on ``a thoughtful, step-by-step process,'' he
said, even though the law is ``collapsing while we speak.''

Mr. Ryan also said he was working with Mr. Trump and Vice
President-elect Mike Pence. Mr. Trump called this week for a near
simultaneous repeal and replacement of the Affordable Care Act.

``We are in complete sync,'' Mr. Ryan said.

But Republicans face a significant challenge in passing the necessary
legislation to replace the health care law. They can repeal major parts
of the existing law without facing a filibuster, but they would not be
able put in place a full replacement in the same measure, because arcane
budget rules limit what can be included in such a filibuster-proof bill.

Instead, they would almost certainly need to pass another bill or
multiple bills with 60 Senate votes, and that would require at least
some Democratic cooperation.

Senator Joe Manchin III, Democrat of West Virginia, a prime target for
Republican wooing, asserted on Thursday that Mr. Trump did not want to
``repeal'' Obamacare but ``repair'' it. He cited Mr. Trump's stated
support for popular provisions like requiring insurers to provide
coverage for people with pre-existing medical conditions.

``We're in a repair mode,'' Mr. Manchin said. ``They need 60 votes to
repair. I'm actually happy to work with them.''

Republicans in Congress have offered many replacement ideas, but it is
not clear whether their most conservative members will ever be able to
agree on legislation acceptable to the party's moderates.

A manifesto issued by House Republicans in June outlined a consensus
proposal, produced by the chairmen of four House committees, including
Representative Tom Price of Georgia, chosen by Mr. Trump to be secretary
of health and human services.

Mr. Trump and congressional leaders said they were counting on Mr. Price
to help them write a replacement for the Affordable Care Act, most
likely drawing from a bill that he introduced in July 2009 and has
reintroduced several times since.

``I think there's an acknowledgment both by the administration coming in
and people around here that his imprint needs to be on this,'' Senator
Bob Corker, Republican of Tennessee, said.

Senate Republicans do not have a detailed plan. But Senator Lamar
Alexander of Tennessee, chairman of the health committee, laid out a
road map on the Senate floor this week that pointed to a measure
potentially more expansive than House plans.

\href{https://www.nytimes.com/interactive/2016/12/03/us/politics/why-it-will-be-hard-to-repeal-obamacare.html}{}

\includegraphics{https://static01.nyt.com/images/2016/12/03/us/politics/why-it-will-be-hard-to-repeal-obamacare-1480740532639/why-it-will-be-hard-to-repeal-obamacare-1480740532639-thumbLarge.jpg}

\hypertarget{how-republicans-can-repeal-obamacare-piece-by-piece}{%
\subsection{How Republicans Can Repeal Obamacare Piece by
Piece}\label{how-republicans-can-repeal-obamacare-piece-by-piece}}

Peeling away pieces of the law could lead to market chaos.

The major Republican proposals have not been analyzed by the
Congressional Budget Office, so no independent or authoritative
estimates exist of their costs or the number of people who might gain or
lose coverage.

On several points, the major Republican proposals agree.

They would eliminate the requirements that most Americans carry health
insurance and that larger employers offer it to employees.

They would offer tax credits for health insurance and new tax incentives
for health savings accounts; provide subsidies for state high-risk
pools, to help people who could not otherwise obtain insurance; and make
it easier for insurance companies to sell policies across state lines.

They would also provide some protection for people with pre-existing
conditions who have maintained ``continuous coverage.'' They could not
be dropped by an insurer and could move from one plan to another, but a
person with a pre-existing condition seeking insurance after a lapse of
coverage could in some cases be charged higher rates. The protections
would be weaker than those in the Affordable Care Act.

Republicans also do not agree on how to pay for a replacement plan. In
the House document, Republicans proposed limiting the value of tax-free
health benefits that employers could provide to employees.

Under the current law, employees do not have to pay federal income tax
on contributions that employers make to their health insurance. House
Republicans said this open-ended subsidy had encouraged people to select
more expensive coverage, driving up premiums.

But business groups, labor unions and some conservative lawmakers
vehemently oppose that change, saying it amounts to a new tax on
benefits and on working families. Senate Republicans have also not
expressed support for the idea.

Many House Republicans, including Mr. Price, would provide tax credits
to help people buy insurance. But the amount of assistance would
increase with age and would not be tied to income, as it is under the
existing health care law.

The subsidies would probably be smaller than under the Affordable Care
Act. But insurance would be less expensive, Republicans say, because the
government would impose fewer requirements.

Mr. Alexander said he would ``allow individuals to use their Obamacare
subsidies to purchase state-approved insurance outside the Obamacare
exchanges.'' Under the health care law, such assistance can be used only
in the insurance exchanges.

Many Republicans say states should have much more power to define
``essential health benefits.''

On Medicaid, the federal insurance program for low-income people, House
Republicans would roll back the Affordable Care Act's expansion and give
each state a fixed amount of money for each beneficiary --- or a lump
sum of federal money for all of a state's Medicaid program.

But more than half the states, including some with Republican governors,
have expanded Medicaid eligibility under Mr. Obama's law, with large
sums of federal money, and pragmatic Republicans are reluctant to snatch
away the federal money that has allowed big increases in Medicaid
enrollment.

Advertisement

\protect\hyperlink{after-bottom}{Continue reading the main story}

\hypertarget{site-index}{%
\subsection{Site Index}\label{site-index}}

\hypertarget{site-information-navigation}{%
\subsection{Site Information
Navigation}\label{site-information-navigation}}

\begin{itemize}
\tightlist
\item
  \href{https://help.nytimes.com/hc/en-us/articles/115014792127-Copyright-notice}{©~2020~The
  New York Times Company}
\end{itemize}

\begin{itemize}
\tightlist
\item
  \href{https://www.nytco.com/}{NYTCo}
\item
  \href{https://help.nytimes.com/hc/en-us/articles/115015385887-Contact-Us}{Contact
  Us}
\item
  \href{https://www.nytco.com/careers/}{Work with us}
\item
  \href{https://nytmediakit.com/}{Advertise}
\item
  \href{http://www.tbrandstudio.com/}{T Brand Studio}
\item
  \href{https://www.nytimes.com/privacy/cookie-policy\#how-do-i-manage-trackers}{Your
  Ad Choices}
\item
  \href{https://www.nytimes.com/privacy}{Privacy}
\item
  \href{https://help.nytimes.com/hc/en-us/articles/115014893428-Terms-of-service}{Terms
  of Service}
\item
  \href{https://help.nytimes.com/hc/en-us/articles/115014893968-Terms-of-sale}{Terms
  of Sale}
\item
  \href{https://spiderbites.nytimes.com}{Site Map}
\item
  \href{https://help.nytimes.com/hc/en-us}{Help}
\item
  \href{https://www.nytimes.com/subscription?campaignId=37WXW}{Subscriptions}
\end{itemize}
