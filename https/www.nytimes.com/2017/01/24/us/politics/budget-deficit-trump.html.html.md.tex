Sections

SEARCH

\protect\hyperlink{site-content}{Skip to
content}\protect\hyperlink{site-index}{Skip to site index}

\href{https://www.nytimes.com/section/politics}{Politics}

\href{https://myaccount.nytimes.com/auth/login?response_type=cookie\&client_id=vi}{}

\href{https://www.nytimes.com/section/todayspaper}{Today's Paper}

\href{/section/politics}{Politics}\textbar{}Federal Debt Projected to
Grow by Nearly \$10 Trillion Over Next Decade

\url{https://nyti.ms/2koT3XP}

\begin{itemize}
\item
\item
\item
\item
\item
\end{itemize}

Advertisement

\protect\hyperlink{after-top}{Continue reading the main story}

Supported by

\protect\hyperlink{after-sponsor}{Continue reading the main story}

\hypertarget{federal-debt-projected-to-grow-by-nearly-10-trillion-over-next-decade}{%
\section{Federal Debt Projected to Grow by Nearly \$10 Trillion Over
Next
Decade}\label{federal-debt-projected-to-grow-by-nearly-10-trillion-over-next-decade}}

\includegraphics{https://static01.nyt.com/images/2017/01/25/us/25deficit2/25deficit2-articleInline-v2.jpg?quality=75\&auto=webp\&disable=upscale}

By \href{https://www.nytimes.com/by/alan-rappeport}{Alan Rappeport}

\begin{itemize}
\item
  Jan. 24, 2017
\item
  \begin{itemize}
  \item
  \item
  \item
  \item
  \item
  \end{itemize}
\end{itemize}

WASHINGTON --- After seven years of fitful declines, the federal budget
deficit is projected to swell again, adding nearly \$10 trillion to the
federal debt over the next 10 years, according to
\href{https://www.cbo.gov/publication/52370}{projections from the
nonpartisan Congressional Budget Office}. The numbers reveal the strain
that government debt could have on the economy as President Trump
presses to slash taxes and ramp up spending.

The deficit figures released Tuesday will be a major challenge to House
Republicans, who were swept to power in 2010 on fears of a bloated
deficit and who made controlling red ink a major part of their agenda
under former President Barack Obama.

Statutory caps imposed in 2011 on domestic and military spending have
helped temper the deficit. But those controls are likely to be swamped
by health care and Social Security spending that will rise with an aging
population.

Now, congressional leaders will have to choose between their fealty to
the cause of fiscal prudence and the demands of the new president, who
wants
\href{https://www.nytimes.com/2016/11/10/nyregion/what-trump-clinton-and-voters-agreed-on-better-infrastructure.html}{\$1
trillion in infrastructure work} over 10 years, a surge in military
spending and large tax cuts for individuals and corporations.

At a confirmation hearing on Tuesday, senators from both parties
peppered Representative Mick Mulvaney of South Carolina, Mr. Trump's
choice to be the White House budget director, with questions about how
Mr. Trump intended to keep his promise to protect Social Security and
Medicare while addressing the budget shortfall.

Mr. Mulvaney said that it would be his role to deliver hard truths to
Mr. Trump. One of those truths, he suggested, could be the need to raise
the eligibility age for Social Security, a proposal that is sure to be
contentious.

``I haven't been quiet and shy since I've been here,'' Mr. Mulvaney
said. ``I have to imagine the president knew what he was getting when he
asked me to fill this role. I'd like to think it's why he hired me.''

He added, ``I believe, as a matter of principle, that the debt is a
problem that must be addressed sooner rather than later.''

The deficit is expected to shrink this fiscal year and next before
increasing in 2019 and beyond. Deficits would cumulatively total \$9.4
trillion from 2018 to 2027, the budget office projects. By 2023, the
deficit would reach \$1 trillion, and in 2027, a projected \$1.4
trillion deficit would be equal to 5 percent of the economy.

Most economists believe that deficits are helpful when economies are in
recession, but some say that when they are near full employment, as the
United States economy is now, deficits should be kept below 3 percent of
the economy to avoid a drag on investment --- or worse, a financial
crisis.

The Congressional Budget Office's budget and economic outlook said that
the share of debt held by the public was expected to reach 89 percent of
gross domestic product in 2027. That level could increase the risk of a
financial crisis and raise the possibility that investors will become
skittish about financing the government's borrowing, some economists
say, although many countries have far higher debt levels.

After the release of the report, the
\href{http://rsc.walker.house.gov/}{Republican Study Committee}, the
main organization for House conservatives, signaled that it would not
ignore that rising red ink to accommodate Mr. Trump's spending
ambitions.

\includegraphics{https://static01.nyt.com/images/2017/01/25/us/25deficit/25deficit-articleInline.jpg?quality=75\&auto=webp\&disable=upscale}

``Without changes to the federal budget, we are on a path to fiscal
crisis with spending, deficits and debt continuing to balloon out of
control,'' said Representative Mark Walker of North Carolina, the
chairman of the group.

Besides the deficit, tepid economic growth is also a concern. Over the
next 10 years, real economic output is projected to grow at an annual
rate of 1.9 percent.

Mr. Trump has promised that his combination of tax cuts and investment
in infrastructure will push growth above 4 percent, and Mr. Mulvaney
argued on Tuesday that spurring growth was the most effective way of
reducing the debt without imposing painful cuts to social safety-net
programs.

Some senators such as Angus King, independent of Maine, warned against a
return to much-debated theories that tax cuts would generate enough
growth to pay for themselves and reduce the deficit.

Democrats are likely to oppose large tax cuts, but they will press Mr.
Trump to make good on his promise to spend big on infrastructure. Senate
Democrats on Tuesday unveiled a \$1 trillion plan to rebuild the
nation's roads, bridges, rails, transit systems, airports, sewer systems
and power grid. ``We will not cut middle-class programs like education
and health care to pay for it,'' said Senator Chuck Schumer of New York,
the Democratic leader.

When it was pointed out that he was proposing the kind of deficit
spending that Republicans were sure to balk at, Mr. Schumer was
dismissive.

``We Democrats believe that this should be a measure to get the economy
going,'' he said.

Republicans are pressing their own spending priorities, urging passage
of legislation that would lift statutory spending caps on national
defense, imposed by the Budget Control Act of 2011, while leaving in
place the caps on domestic spending.

In one of the testier exchanges of Mr. Mulvaney's two confirmation
hearings on Tuesday, Senator John McCain, Republican of Arizona,
actually pressed the nominee to accept more military spending,
castigating him for struggling to remember his House votes to cut
military budgets.

``Boy, I'll tell you, I would remember if I voted to cut our defenses
the way you did, Congressman,'' Mr. McCain said. ``Maybe you don't take
it with the seriousness that it deserves.''

Like many of Mr. Trump's picks, Mr. Mulvaney has taken positions that
contradict Mr. Trump, and it remained unclear how that dynamic would
play out in the White House if he is confirmed.

Senator Bernie Sanders of Vermont, the former Democratic presidential
candidate, said that Mr. Mulvaney's views on the deficit, Social
Security, Medicare and Medicaid were out of sync with Mr. Trump's
campaign promises to protect the programs. He said he feared that Mr.
Mulvaney would pull the president to the right.

``It does not make sense to me to have a key adviser to the president
having views directly in opposition to what the president campaigned
on,'' Mr. Sanders said.

Despite the concern about a growing deficit, the tenor of Mr. Mulvaney's
nomination hearings did not reflect the sense of anxiety that was
palpable when Mr. Obama was assembling his economic team during the
financial crisis eight years ago. The economy, now growing steadily,
appears to be giving policy makers time.

The Congressional Budget Office report said the economy was on ``solid
ground,'' with increasing output and job growth on the horizon, a sharp
contrast to the ``carnage'' detailed by Mr. Trump in his inaugural
address.

Advertisement

\protect\hyperlink{after-bottom}{Continue reading the main story}

\hypertarget{site-index}{%
\subsection{Site Index}\label{site-index}}

\hypertarget{site-information-navigation}{%
\subsection{Site Information
Navigation}\label{site-information-navigation}}

\begin{itemize}
\tightlist
\item
  \href{https://help.nytimes.com/hc/en-us/articles/115014792127-Copyright-notice}{©~2020~The
  New York Times Company}
\end{itemize}

\begin{itemize}
\tightlist
\item
  \href{https://www.nytco.com/}{NYTCo}
\item
  \href{https://help.nytimes.com/hc/en-us/articles/115015385887-Contact-Us}{Contact
  Us}
\item
  \href{https://www.nytco.com/careers/}{Work with us}
\item
  \href{https://nytmediakit.com/}{Advertise}
\item
  \href{http://www.tbrandstudio.com/}{T Brand Studio}
\item
  \href{https://www.nytimes.com/privacy/cookie-policy\#how-do-i-manage-trackers}{Your
  Ad Choices}
\item
  \href{https://www.nytimes.com/privacy}{Privacy}
\item
  \href{https://help.nytimes.com/hc/en-us/articles/115014893428-Terms-of-service}{Terms
  of Service}
\item
  \href{https://help.nytimes.com/hc/en-us/articles/115014893968-Terms-of-sale}{Terms
  of Sale}
\item
  \href{https://spiderbites.nytimes.com}{Site Map}
\item
  \href{https://help.nytimes.com/hc/en-us}{Help}
\item
  \href{https://www.nytimes.com/subscription?campaignId=37WXW}{Subscriptions}
\end{itemize}
