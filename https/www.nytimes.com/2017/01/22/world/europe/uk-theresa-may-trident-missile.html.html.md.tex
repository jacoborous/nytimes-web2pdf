Sections

SEARCH

\protect\hyperlink{site-content}{Skip to
content}\protect\hyperlink{site-index}{Skip to site index}

\href{https://www.nytimes.com/section/world/europe}{Europe}

\href{https://myaccount.nytimes.com/auth/login?response_type=cookie\&client_id=vi}{}

\href{https://www.nytimes.com/section/todayspaper}{Today's Paper}

\href{/section/world/europe}{Europe}\textbar{}Theresa May Is Grilled
Over U.K. Missile Test Failure

\url{https://nyti.ms/2kfnvUb}

\begin{itemize}
\item
\item
\item
\item
\item
\end{itemize}

Advertisement

\protect\hyperlink{after-top}{Continue reading the main story}

Supported by

\protect\hyperlink{after-sponsor}{Continue reading the main story}

\hypertarget{theresa-may-is-grilled-over-uk-missile-test-failure}{%
\section{Theresa May Is Grilled Over U.K. Missile Test
Failure}\label{theresa-may-is-grilled-over-uk-missile-test-failure}}

\includegraphics{https://static01.nyt.com/images/2017/01/23/world/23BRITAIN/23BRITAIN-articleInline.jpg?quality=75\&auto=webp\&disable=upscale}

By \href{http://www.nytimes.com/by/steven-erlanger}{Steven Erlanger}

\begin{itemize}
\item
  Jan. 22, 2017
\item
  \begin{itemize}
  \item
  \item
  \item
  \item
  \item
  \end{itemize}
\end{itemize}

LONDON --- Prime Minister Theresa May of Britain refused to comment on
Sunday about the reported failure of an unarmed British Trident missile
that was test-fired from a submarine off the coast of Florida in June.

Mrs. May said in a
\href{http://www.bbc.co.uk/news/uk-38708823}{television interview with
the BBC} that she had ``absolute faith in our Trident missiles.'' But
she would not say whether she had known about the failure or whether, as
\href{http://www.thetimes.co.uk/edition/news/no-10-covered-up-trident-missile-fiasco-hch3shsrn}{The
Sunday Times of London reported}, it had been covered up by Downing
Street under her predecessor, David Cameron, shortly before the
referendum on
\href{https://www.nytimes.com/news-event/britain-brexit-european-union}{Britain's
exit} from the European Union.

Mrs. May did not mention any missile failure in her first major speech
to Parliament on July 18, when she
\href{https://www.nytimes.com/2016/07/19/world/europe/theresa-may-britain-nuclear-weapons.html}{persuaded
Parliament} to spend up to 40 billion pounds, or about \$53 billion
then, on four new submarines to keep Britain's nuclear deterrent up to
date.

``There are tests that take place all the time, regularly, for our
nuclear deterrence,'' she said on Sunday. ``What we were talking about
in that debate that took place was about the future.''

The Sunday Times reported that the Trident II D5 missile, which was
designed to carry a nuclear warhead but was unarmed for the test, had
veered off course after being fired from HMS Vengeance, one of Britain's
four aging nuclear-armed submarines.

The British Navy had not performed such a test for four years because of
the expense of the missile, but had carried out tests in 2000, 2005,
2009 and 2012, all of which had been successful and publicized by the
Ministry of Defense. The current test took place after the submarine had
been refitted with new missile launch equipment and upgraded computer
systems.

Replacing Trident has been controversial because of the cost and because
the current leader of the Labour Party,
\href{https://www.nytimes.com/2016/01/18/world/europe/british-labour-leader-offers-compromise-on-trident-program.html}{Jeremy
Corbyn, long an antinuclear campaigner}, is opposed to retaining
Britain's nuclear deterrent, while his party's official position has
been to retain and renew it.

``It's a pretty catastrophic error when a missile goes in the wrong
direction, and while it wasn't armed, goodness knows what the
consequences of that could have been,'' Mr. Corbyn said on Sunday.

Speaking to Sky News, he said, ``We understand the prime minister chose
not to inform Parliament, and instead it came out through the media.''
He repeated his belief that Britain should commit to nuclear
disarmament.

Kevan Jones, a Labour member of Parliament and a former defense
minister, called for an inquiry into the failed missile test. ``The
U.K.'s independent nuclear deterrent is a vital cornerstone for the
nation's defense,'' he said. Parliament is likely to ask Defense
Secretary Michael Fallon to answer questions about the report.

Separately, Mrs. May confirmed that she would meet with President Trump
in Washington on Friday in the first visit of a foreign leader to the
new president, a traditional prize that Britain has been seeking avidly.
She said she would emphasize to Mr. Trump the importance of the NATO
military alliance, calling it a ``bulwark'' of the West, and would say
that Britain favors the progress and cohesion of the European Union,
even though the country plans to leave the bloc.

Trade will be an important topic, she said, with Britain wanting new
free-trade agreements with key countries, including the United States,
after it leaves the European Union. Mr. Trump, a supporter of
``Brexit,'' as Britain's departure from the bloc is known, has said that
he is open to early talks on such a deal with Britain. Legally, no new
deal can be made until Britain formally leaves the European Union, which
is unlikely for at least two years.

Mr. Trump's slogan of ``America First'' and his protectionist comments
may mean that a trade deal will be difficult to negotiate despite the
good will expressed by both sides.

Mrs. May was asked about Mr. Trump's attitudes toward women. ``I've
already said that some of the comments that Donald Trump has made in
relation to women are unacceptable, some of those he himself has
apologized for,'' she said.

When she meets Mr. Trump, she said, ``I think the biggest statement that
will be made about the role of women is the fact that I will be there as
a female prime minister.''

Advertisement

\protect\hyperlink{after-bottom}{Continue reading the main story}

\hypertarget{site-index}{%
\subsection{Site Index}\label{site-index}}

\hypertarget{site-information-navigation}{%
\subsection{Site Information
Navigation}\label{site-information-navigation}}

\begin{itemize}
\tightlist
\item
  \href{https://help.nytimes.com/hc/en-us/articles/115014792127-Copyright-notice}{©~2020~The
  New York Times Company}
\end{itemize}

\begin{itemize}
\tightlist
\item
  \href{https://www.nytco.com/}{NYTCo}
\item
  \href{https://help.nytimes.com/hc/en-us/articles/115015385887-Contact-Us}{Contact
  Us}
\item
  \href{https://www.nytco.com/careers/}{Work with us}
\item
  \href{https://nytmediakit.com/}{Advertise}
\item
  \href{http://www.tbrandstudio.com/}{T Brand Studio}
\item
  \href{https://www.nytimes.com/privacy/cookie-policy\#how-do-i-manage-trackers}{Your
  Ad Choices}
\item
  \href{https://www.nytimes.com/privacy}{Privacy}
\item
  \href{https://help.nytimes.com/hc/en-us/articles/115014893428-Terms-of-service}{Terms
  of Service}
\item
  \href{https://help.nytimes.com/hc/en-us/articles/115014893968-Terms-of-sale}{Terms
  of Sale}
\item
  \href{https://spiderbites.nytimes.com}{Site Map}
\item
  \href{https://help.nytimes.com/hc/en-us}{Help}
\item
  \href{https://www.nytimes.com/subscription?campaignId=37WXW}{Subscriptions}
\end{itemize}
