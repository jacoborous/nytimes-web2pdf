Sections

SEARCH

\protect\hyperlink{site-content}{Skip to
content}\protect\hyperlink{site-index}{Skip to site index}

\href{https://www.nytimes.com/section/world/asia}{Asia Pacific}

\href{https://myaccount.nytimes.com/auth/login?response_type=cookie\&client_id=vi}{}

\href{https://www.nytimes.com/section/todayspaper}{Today's Paper}

\href{/section/world/asia}{Asia Pacific}\textbar{}Jim Mattis Says U.S.
Is `Shoulder to Shoulder' With Japan

\url{https://nyti.ms/2k9BZIj}

\begin{itemize}
\item
\item
\item
\item
\item
\end{itemize}

Advertisement

\protect\hyperlink{after-top}{Continue reading the main story}

Supported by

\protect\hyperlink{after-sponsor}{Continue reading the main story}

\hypertarget{jim-mattis-says-us-is-shoulder-to-shoulder-with-japan}{%
\section{Jim Mattis Says U.S. Is `Shoulder to Shoulder' With
Japan}\label{jim-mattis-says-us-is-shoulder-to-shoulder-with-japan}}

\includegraphics{https://static01.nyt.com/images/2017/02/04/world/04military1/04military1-articleInline.jpg?quality=75\&auto=webp\&disable=upscale}

By \href{http://www.nytimes.com/by/michael-r-gordon}{Michael R. Gordon}
and \href{http://www.nytimes.com/by/motoko-rich}{Motoko Rich}

\begin{itemize}
\item
  Feb. 3, 2017
\item
  \begin{itemize}
  \item
  \item
  \item
  \item
  \item
  \end{itemize}
\end{itemize}

TOKYO --- Defense Secretary Jim Mattis assured Japan's prime minister on
Friday that the United States would stand by its mutual defense treaty
with the country, despite statements by President Trump during last
year's campaign that suggested he might pull back from American security
commitments in Asia.

``I want there to be no misunderstanding during the transition in
Washington that we stand firmly, 100 percent, shoulder to shoulder with
you and the Japanese people,'' Mr. Mattis said at the start of a meeting
with Prime Minister Shinzo Abe.

Mr. Trump, when he was running for president, had complained that the
defense treaty was one-sided, even suggesting that the United States
should threaten to withdraw from it unless Japan did more to compensate
Washington for helping to defend its territory.

``You always have to be prepared to walk,'' Mr. Trump said in August.
``I don't think it's going to be necessary. It could be, though.''

With North Korea
\href{https://www.nytimes.com/2017/01/01/world/asia/north-korea-intercontinental-ballistic-missile-test-kim-jong-un.html}{threatening
to test an intercontinental ballistic missile} and China claiming
sovereignty over nearly all of the South China Sea, the Trump
administration is now emphasizing its
``\href{https://www.nytimes.com/reuters/2017/01/28/world/asia/28reuters-usa-trump-japan-whitehouse.html}{ironclad}''
commitment to the security of Japan and South Korea.

Still, Mr. Trump's conflicting signals and his transactional approach to
diplomacy have made allies nervous. So even as the president shakes up
Washington, Mr. Mattis has
\href{https://www.nytimes.com/2017/02/02/world/asia/james-mattis-us-korea-thaad.html}{scored
points in Asia} by emphasizing that the United States' relationship with
its allies has not fundamentally changed.

In Japan, that included assurances about Article 5 of the treaty, which
commits the United States to helping defend territory that Japan
administers should it be attacked.

``Due to some of the provocations out of North Korea and other
challenges that we jointly face, I want to make certain that Article 5
of our mutual defense treaty is understood to be as real to us today as
it was a year ago, five years ago, and as it will be a year and 10 years
from now,'' Mr. Mattis told Mr. Abe.

\includegraphics{https://static01.nyt.com/images/2017/02/04/world/04military2/04military2-articleInline.jpg?quality=75\&auto=webp\&disable=upscale}

Japan hosts about 50,000 American troops on bases throughout the
country, and it relies on the American nuclear umbrella. Given that the
alliance has lasted since the end of World War II, analysts said it was
somewhat surprising that it would be necessary to reassure Japan that it
remained solid.

``It's sad that we've even come to this point,'' said Jeffrey Hornung, a
fellow in the security and foreign affairs program at Sasakawa Peace
Foundation USA, a research institute based in Washington. ``But in the
context that we're in, that reassurance is very important.''

A particular concern for Mr. Abe, who is scheduled to meet with Mr.
Trump in Washington next week, is confirmation that the United States
would defend Japan in any confrontation with China over a set of
disputed islands in the East China Sea, known in Japan as the Senkaku
and in China as the Diaoyu.

Former President Barack Obama declared during a visit to Tokyo three
years ago that the security treaty
\href{https://www.nytimes.com/2014/04/25/world/asia/obama-asia.html}{obligated
the United States to do so}, the first president to state that
commitment publicly. Mr. Mattis did not explicitly mention the islands
publicly Friday, but American officials said later that he had
reiterated the United States ``longstanding'' position on the islands in
his private meeting with Mr. Abe.

Mr. Mattis made a point of saying on Friday that he had served on the
Japanese island of Okinawa as a young Marine. Mr. Abe, for his part,
said it was important that Japan was one of the first places Mr. Mattis
had visited as defense secretary.

Earlier Friday, Mr. Mattis wound up a two-day visit to Seoul, South
Korea, where he sought to reassure officials that the American
commitment to defend their country, particularly in the face of North
Korea's accelerating nuclear threat, was also unchanged.

``Any attack on the United States, or our allies, will be defeated,''
Mr. Mattis said in a joint appearance with the South Korean defense
minister, Han Min-koo. ``And any use of nuclear weapons would be met
with a response that would be effective and overwhelming,'' he added,
using a formulation that has been employed by previous Pentagon chiefs.

During Mr. Mattis's talks with South Korean officials, the two sides
affirmed that they would move ahead with plans developed under the Obama
administration to deploy a new antimissile system to counter North
Korea's medium-range missiles. China has objected to the deployment of
the system, known as Thaad, which stands for Terminal High Altitude Area
Defense.

``Due to North Korea's threatening rhetoric and destabilizing behavior,
we are taking defensive steps like deploying the highly effective Thaad
antimissile unit,'' Mr. Mattis said on Friday.

Mr. Mattis did not specify when the Thaad system would be deployed.
American and South Korean officials have said that it would be before
the end of the year, though American officials would like to see the
timetable accelerated.

Advertisement

\protect\hyperlink{after-bottom}{Continue reading the main story}

\hypertarget{site-index}{%
\subsection{Site Index}\label{site-index}}

\hypertarget{site-information-navigation}{%
\subsection{Site Information
Navigation}\label{site-information-navigation}}

\begin{itemize}
\tightlist
\item
  \href{https://help.nytimes.com/hc/en-us/articles/115014792127-Copyright-notice}{©~2020~The
  New York Times Company}
\end{itemize}

\begin{itemize}
\tightlist
\item
  \href{https://www.nytco.com/}{NYTCo}
\item
  \href{https://help.nytimes.com/hc/en-us/articles/115015385887-Contact-Us}{Contact
  Us}
\item
  \href{https://www.nytco.com/careers/}{Work with us}
\item
  \href{https://nytmediakit.com/}{Advertise}
\item
  \href{http://www.tbrandstudio.com/}{T Brand Studio}
\item
  \href{https://www.nytimes.com/privacy/cookie-policy\#how-do-i-manage-trackers}{Your
  Ad Choices}
\item
  \href{https://www.nytimes.com/privacy}{Privacy}
\item
  \href{https://help.nytimes.com/hc/en-us/articles/115014893428-Terms-of-service}{Terms
  of Service}
\item
  \href{https://help.nytimes.com/hc/en-us/articles/115014893968-Terms-of-sale}{Terms
  of Sale}
\item
  \href{https://spiderbites.nytimes.com}{Site Map}
\item
  \href{https://help.nytimes.com/hc/en-us}{Help}
\item
  \href{https://www.nytimes.com/subscription?campaignId=37WXW}{Subscriptions}
\end{itemize}
