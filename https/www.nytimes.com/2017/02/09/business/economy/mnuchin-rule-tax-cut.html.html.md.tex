Sections

SEARCH

\protect\hyperlink{site-content}{Skip to
content}\protect\hyperlink{site-index}{Skip to site index}

\href{https://www.nytimes.com/section/business/economy}{Economy}

\href{https://myaccount.nytimes.com/auth/login?response_type=cookie\&client_id=vi}{}

\href{https://www.nytimes.com/section/todayspaper}{Today's Paper}

\href{/section/business/economy}{Economy}\textbar{}Treasury Nominee Vows
No Tax Cut for Rich. Math Says the Opposite.

\url{https://nyti.ms/2kT9cca}

\begin{itemize}
\item
\item
\item
\item
\item
\end{itemize}

Advertisement

\protect\hyperlink{after-top}{Continue reading the main story}

Supported by

\protect\hyperlink{after-sponsor}{Continue reading the main story}

\hypertarget{treasury-nominee-vows-no-tax-cut-for-rich-math-says-the-opposite}{%
\section{Treasury Nominee Vows No Tax Cut for Rich. Math Says the
Opposite.}\label{treasury-nominee-vows-no-tax-cut-for-rich-math-says-the-opposite}}

\includegraphics{https://static01.nyt.com/images/2017/02/07/business/cnbc-mnuchin2/cnbc-mnuchin2-videoSixteenByNineJumbo1600.png}

By \href{http://www.nytimes.com/by/patricia-cohen}{Patricia Cohen}

\begin{itemize}
\item
  Feb. 9, 2017
\item
  \begin{itemize}
  \item
  \item
  \item
  \item
  \item
  \end{itemize}
\end{itemize}

The newly christened ``Mnuchin rule'' --- the assurance given by the
Treasury nominee Steven T. Mnuchin that ``there would be no absolute tax
cut for the upper class'' --- seems as if it was made to be broken.

Mr. Mnuchin initially made the statement during an
\href{http://www.cnbc.com/2016/11/30/exclusive-steve-mnuchin-no-absolute-tax-cut-for-the-upper-class.html}{interview
on CNBC in November}, after President Trump chose him for the cabinet.
At Mr. Mnuchin's confirmation hearing, Senator Ron Wyden, an Oregon
Democrat, rebranded the comment as a ``rule,'' transforming a throwaway
line into a formal pledge.

Whether it will be kept may become clearer in two or three weeks --- the
timing Mr. Trump
\href{https://www.nytimes.com/video/business/dealbook/100000004922130/trump-on-phenomenal-tax-plan.html}{mentioned
Thursday} for delivering a ``phenomenal'' tax plan.

Although Mr. Mnuchin said any rate reductions at the top would be offset
by the closing of fat loopholes, his guarantee appears impossible to
fulfill either under the tax overhaul that the House Republicans are
pushing or similar, sketchier proposals that Mr. Trump has offered.

Redesigning the tax code with an eye fixed on lower rates has been a
Republican mission for decades, and one that Mr. Trump adopted. That
prospect, combined with a promised regulatory retreat, has pumped up the
stock market and fueled optimism among business leaders.

At the same time, the president has raised expectations among his
working-class supporters that ``the rich will pay their fair share,''
and that ``special-interest loopholes that have been so good for Wall
Street investors, and for people like me, but unfair to American
workers''
\href{http://www.cnbc.com/2016/08/08/restrained-trump-goes-at-clinton-for-tax-plan.html}{will
be eliminated}. Mr. Mnuchin, soon to be one of the administration's top
economic policy officials, promised ``a big tax cut for the middle
class.''

Yet analyses of the
\href{https://www.nytimes.com/2016/11/13/business/economy/trump-and-congress-both-want-tax-cuts-the-question-is-which-ones.html}{president's
and the House Republicans' plans} consistently conclude that the wealthy
will receive the largest tax cuts by far.

Start with the House blueprint, which at the moment is the closest thing
to a working draft that exists. The nonpartisan Tax Policy Center, a
joint project of the Urban Institute and Brookings Institution, found
``high-income taxpayers
\href{http://www.taxpolicycenter.org/publications/analysis-house-gop-tax-plan}{would
receive the biggest cuts}, both in dollar terms and as a percentage of
income.''

How big? ``Three-quarters of the tax cuts would benefit the top 1
percent of taxpayers,'' if the plan were put into effect this year, it
said. The highest-income households --- the top 0.1 percent --- would
get ``an average tax cut of about \$1.3 million, 16.9 percent of
after-tax income.''

Those in the middle fifth of incomes would get a tax cut of almost
\$260, or 0.5 percent, while the poorest would get about \$50.

That split would worsen down the road, the Tax Policy Center says: ``In
2025 the top 1 percent of households would receive nearly 100 percent of
the total tax reduction.''

Those wary of any potential liberal bias could turn to the
conservative-leaning Tax Foundation.
\href{https://taxfoundation.org/details-and-analysis-2016-house-republican-tax-reform-plan/}{Its
analysis} found a smaller gap between the wealthy and everyone else, but
a gap nonetheless. The foundation concluded that four out of five
taxpayers would see only a 0.2 to 0.5 percent increase in after-tax
income, while those in the top 1 percent of the income scale would save
at least 10 times as much, or 5.3 percent. That's nearly \$40,000 extra
for those at the top, compared to \$67 for those smack dab in the middle
of the income scale.

``The Mnuchin rule is already being broken as Republicans look to strip
away hundreds of billions of dollars in Affordable Care Act tax credits
for working Americans to pay for a giant tax break for the wealthy,''
Senator Wyden said. ``Bottom line is it's unfair to cut benefits that
the middle class depends on, all so the wealthy pay a lower rate.''

Mr. Mnuchin did not respond to a request for comment.

Republicans argue their plan makes everyone a winner --- that lower
taxes will unleash an enormous swell of economic growth, raising wages,
incomes and tax revenue all around.

The historical record does not offer much support for the claim that
slashing taxes for the most affluent creates growth. Yet even assuming
the rosiest of forecasts, the top 1 percent, according to the Tax
Foundation, would still receive close to a \$100,000 tax cut --- 32
times as much as a middle-income family.

Mr. Mnuchin has offered his own formula for adhering to the standard he
laid down, explaining that ``any reductions we have in upper-income
taxes would be offset by less deductions.''

That would require some otherworldly mathematical magic, however.

Consider the list of proposals that would reduce taxes on the rich:

\begin{quote}
■ Cut the top income tax rate to 33 percent, from 39.6 percent.

■ Cut taxes on capital gains,
\href{https://www.taxpolicycenter.org/sites/default/files/alfresco/publication-pdfs/904606-Tax-Reform-and-the-Tax-Treatment-of-Capital-Gains.PDF}{70
percent of which flow to the top 1 percent}.

■ Eliminate the estate tax, which applies to a tiny number of people,
couples that have estates bigger than \$10.8 million.

■ Eliminate the 3.8 percent surtax on high earners' investment income
that has been used to subsidize health care for poorer Americans.

■ End the
\href{http://topics.nytimes.com/top/reference/timestopics/subjects/a/alternative_minimum_tax/index.html?inline=nyt-classifier}{alternative
minimum tax}, which currently limits deductions for high earners.

■ Lower taxes on cash flow and income that passes from small businesses
to their owners, which also primarily benefits wealthier Americans.
\end{quote}

Now, what deductions could be eliminated that would offset all those
cuts at the top? There aren't many, said
\href{http://www.aei.org/scholar/alan-d-viard/}{Alan Viard}, an
economist at the conservative American Enterprise Institute. If
Republicans insist on lowering taxes on top wages, capital gains,
estates and cash-flow and pass-through income as advertised, ``there's
not a lot of latitude to limit itemized deductions further,'' Mr. Viard
said.

Any plan to curb itemized deductions would be partly offset by Mr.
Trump's plan to increase the standard deduction. Curtailing mortgage
deductions for the most expensive homes is probably a good idea, Mr.
Viard said, but that isn't going to do much to raise revenue from those
at the top of the income pyramid, and the deduction is already roughly
limited to the interest paid on \$1 million in mortgage debt.

Such alternative ideas, however, assume that the Mnuchin rule will have
a meaningful impact on what the White House will propose or Congress
will debate. Not everyone is convinced that it will. As Mr. Viard said,
``I don't know how much interest there is in fulfilling that statement
by Mnuchin, however it's interpreted.''

Advertisement

\protect\hyperlink{after-bottom}{Continue reading the main story}

\hypertarget{site-index}{%
\subsection{Site Index}\label{site-index}}

\hypertarget{site-information-navigation}{%
\subsection{Site Information
Navigation}\label{site-information-navigation}}

\begin{itemize}
\tightlist
\item
  \href{https://help.nytimes.com/hc/en-us/articles/115014792127-Copyright-notice}{©~2020~The
  New York Times Company}
\end{itemize}

\begin{itemize}
\tightlist
\item
  \href{https://www.nytco.com/}{NYTCo}
\item
  \href{https://help.nytimes.com/hc/en-us/articles/115015385887-Contact-Us}{Contact
  Us}
\item
  \href{https://www.nytco.com/careers/}{Work with us}
\item
  \href{https://nytmediakit.com/}{Advertise}
\item
  \href{http://www.tbrandstudio.com/}{T Brand Studio}
\item
  \href{https://www.nytimes.com/privacy/cookie-policy\#how-do-i-manage-trackers}{Your
  Ad Choices}
\item
  \href{https://www.nytimes.com/privacy}{Privacy}
\item
  \href{https://help.nytimes.com/hc/en-us/articles/115014893428-Terms-of-service}{Terms
  of Service}
\item
  \href{https://help.nytimes.com/hc/en-us/articles/115014893968-Terms-of-sale}{Terms
  of Sale}
\item
  \href{https://spiderbites.nytimes.com}{Site Map}
\item
  \href{https://help.nytimes.com/hc/en-us}{Help}
\item
  \href{https://www.nytimes.com/subscription?campaignId=37WXW}{Subscriptions}
\end{itemize}
