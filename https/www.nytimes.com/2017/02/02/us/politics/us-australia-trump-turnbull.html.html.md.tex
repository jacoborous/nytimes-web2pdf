Sections

SEARCH

\protect\hyperlink{site-content}{Skip to
content}\protect\hyperlink{site-index}{Skip to site index}

\href{https://www.nytimes.com/section/politics}{Politics}

\href{https://myaccount.nytimes.com/auth/login?response_type=cookie\&client_id=vi}{}

\href{https://www.nytimes.com/section/todayspaper}{Today's Paper}

\href{/section/politics}{Politics}\textbar{}U.S.-Australia Rift Is
Possible After Trump Ends Call With Prime Minister

\url{https://nyti.ms/2jXXdWX}

\begin{itemize}
\item
\item
\item
\item
\item
\end{itemize}

Advertisement

\protect\hyperlink{after-top}{Continue reading the main story}

Supported by

\protect\hyperlink{after-sponsor}{Continue reading the main story}

\hypertarget{us-australia-rift-is-possible-after-trump-ends-call-with-prime-minister}{%
\section{U.S.-Australia Rift Is Possible After Trump Ends Call With
Prime
Minister}\label{us-australia-rift-is-possible-after-trump-ends-call-with-prime-minister}}

\includegraphics{https://static01.nyt.com/images/2017/02/03/us/02australia-web1/02australia-web1-articleInline.jpg?quality=75\&auto=webp\&disable=upscale}

By \href{https://www.nytimes.com/by/glenn-thrush}{Glenn Thrush} and
Michelle Innis

\begin{itemize}
\item
  Feb. 2, 2017
\item
  \begin{itemize}
  \item
  \item
  \item
  \item
  \item
  \end{itemize}
\end{itemize}

WASHINGTON --- A phone call between President Trump and the Australian
prime minister is threatening to develop into a diplomatic rift between
two stalwart allies after the two men exchanged harsh words over refugee
policy, and Mr. Trump abruptly ended the call.

The phone call last Saturday between Mr. Trump and Prime Minister
Malcolm Turnbull turned contentious after the Australian leader pressed
the president to honor an agreement to accept 1,250 refugees from an
Australian detention center.

Late Wednesday night, Mr. Trump reiterated his anger over the agreement
on Twitter. He called the agreement a ``dumb deal'' and blamed the Obama
administration for accepting it but then said that he would ``study''
it. The tweet was posted after The Washington Post reported details of
the phone call.

\begin{quote}
Do you believe it? The Obama Administration agreed to take thousands of
illegal immigrants from Australia. Why? I will study this dumb deal!

--- Donald J. Trump (@realDonaldTrump)
\href{https://twitter.com/realDonaldTrump/status/827002559122567168?ref_src=twsrc\%5Etfw}{February
2, 2017}
\end{quote}

The leaders of the two allies did not seem to agree on the outcome of
the conversation. Mr. Trump's tweet suggested the agreement could be at
risk while Mr. Turnbull said that, despite the bluntness of the
discussion, the United States had
\href{https://www.nytimes.com/2017/01/30/world/australia/trump-us-refugee-manus-nauru.html}{committed
to upholding the arrangement}.

The flare-up --- and conflicting characterizations of the call from Mr.
Trump and Mr. Turnbull --- threatened to do lasting damage to relations
between the two countries and could drive Canberra closer to China,
which has a robust trading relationship with Australia and is competing
with Washington to become the dominant force in the Asia-Pacific region.

A senior Trump administration official said the president told Mr.
Turnbull on Saturday that the refugees could include the ``next Boston
bombers.'' He also said he was ``going to get killed'' politically by
the deal, given that the day before he signed an
\href{https://www.nytimes.com/2017/01/27/us/politics/trump-syrian-refugees.html}{executive
order} to stem the refugee flow into the United States and refuse visas
for all citizens from seven Muslim countries.

The Trump administration official said the call was shorter than
planned, and ended abruptly after Mr. Turnbull told the president it was
necessary for the refugees to be accepted.

The details of the call were confirmed by a senior administration
official with direct knowledge of the exchange who spoke on condition of
anonymity because he was not authorized to publicly discuss the
diplomatic talks.

Mr. Turnbull, speaking Thursday at a press briefing in Australia's
southern state of Victoria, refused to comment at length on the
telephone call, or say whether it had ended sooner than expected. But he
did acknowledge that it had been candid.

``I've seen that report,'' Mr. Turnbull said of the
\href{https://www.washingtonpost.com/world/national-security/no-gday-mate-on-call-with-australian-pm-trump-badgers-and-brags/2017/02/01/88a3bfb0-e8bf-11e6-80c2-30e57e57e05d_story.html?utm_term=.a6b6fc6bea02}{Washington
Post account}, ``and I'm not going to comment on the conversation, other
than to say that in the course of the conversation, as you know and as
was confirmed by the president's official spokesman in the White House,
the president assured me that he would continue with, honor the
agreement we entered into with the Obama administration with respect to
refugee resettlement.''

\includegraphics{https://static01.nyt.com/images/2017/02/03/world/03manus-diary/03manus-diary-videoSixteenByNine3000-v2.jpg}

Pressed about Mr. Trump's tone, and whether the president ended the call
by hanging up, Mr. Turnbull refused to comment. ``It's better that these
things, these conversations are conducted candidly, frankly,
privately,'' he said.

Mr. Turnbull again stated that Australia's relationship with the United
States remained robust, but if the deal to resettle the refugees falls
through, Canberra will be left with a seemingly intractable political
problem at home.

The Australian government has a policy that bars any refugees who
attempted to arrive by boat from ever setting foot in the country. The
majority of the refugees being held on the Pacific islands of Nauru and
Manus are from Iran and Iraq. Both are Muslim-majority nations that are
among the seven countries --- including Libya,
\href{http://topics.nytimes.com/top/news/international/countriesandterritories/somalia/index.html?inline=nyt-geo}{Somalia},
Sudan, Syria and Yemen --- whose citizens are barred from entering the
United States for at least 90 days under an executive order signed by
Mr. Trump last week.

``I can assure you the relationship is very strong,'' Mr. Turnbull said.
``The fact that we received the assurance that we did, the fact that it
was confirmed, the very extensive engagement we have with the new
administration underlines the closeness of the alliance.''

``But as Australians know me very well --- I stand up for Australia in
every forum --- public or private.''

Bill Shorten, the leader of Australia's opposition Labor Party, said
there were two versions of the conversation between Mr. Turnbull and Mr.
Trump over the refugee deal, and Mr. Turnbull should be ``straight with
the Australian people.''

Mr. Turnbull ``made it clear he had a constructive discussion'' over the
refugee deal, Mr. Shorten said. ``But now it appears another, different
version of the same conversation has emerged.''

Kim Beazley, a former Australian ambassador to the United States who
served in Washington during much of the Obama administration, said the
impact of the flare-up would be ``minimal'' if the refugee deal remained
in force. But he added, ``If the tonality is true you wouldn't want to
have too many conversations like that.''

It was not the only awkward call last week between Mr. Trump and a world
leader. Earlier, on Friday, Mr. Trump joked to President Enrique Peña
Nieto of Mexico that he would deploy troops to Mexico if the Mexican
government failed to control ``bad hombres down there.''

On Wednesday night, the senior Trump administration official said the
president's comments to Mr. Peña Nieto were made in jest and the
comments reflected Mr. Trump's standing offer to help Mexico battle drug
gangs and control border crossings. The official said the conversation
between the two presidents was friendly, and Mr. Peña Nieto did not
appear to be offended.

The Mexican government issued a statement denying the A.P. report and
said it did not ``correspond to reality.''

Advertisement

\protect\hyperlink{after-bottom}{Continue reading the main story}

\hypertarget{site-index}{%
\subsection{Site Index}\label{site-index}}

\hypertarget{site-information-navigation}{%
\subsection{Site Information
Navigation}\label{site-information-navigation}}

\begin{itemize}
\tightlist
\item
  \href{https://help.nytimes.com/hc/en-us/articles/115014792127-Copyright-notice}{©~2020~The
  New York Times Company}
\end{itemize}

\begin{itemize}
\tightlist
\item
  \href{https://www.nytco.com/}{NYTCo}
\item
  \href{https://help.nytimes.com/hc/en-us/articles/115015385887-Contact-Us}{Contact
  Us}
\item
  \href{https://www.nytco.com/careers/}{Work with us}
\item
  \href{https://nytmediakit.com/}{Advertise}
\item
  \href{http://www.tbrandstudio.com/}{T Brand Studio}
\item
  \href{https://www.nytimes.com/privacy/cookie-policy\#how-do-i-manage-trackers}{Your
  Ad Choices}
\item
  \href{https://www.nytimes.com/privacy}{Privacy}
\item
  \href{https://help.nytimes.com/hc/en-us/articles/115014893428-Terms-of-service}{Terms
  of Service}
\item
  \href{https://help.nytimes.com/hc/en-us/articles/115014893968-Terms-of-sale}{Terms
  of Sale}
\item
  \href{https://spiderbites.nytimes.com}{Site Map}
\item
  \href{https://help.nytimes.com/hc/en-us}{Help}
\item
  \href{https://www.nytimes.com/subscription?campaignId=37WXW}{Subscriptions}
\end{itemize}
