Sections

SEARCH

\protect\hyperlink{site-content}{Skip to
content}\protect\hyperlink{site-index}{Skip to site index}

\href{https://www.nytimes.com/section/world/asia}{Asia Pacific}

\href{https://myaccount.nytimes.com/auth/login?response_type=cookie\&client_id=vi}{}

\href{https://www.nytimes.com/section/todayspaper}{Today's Paper}

\href{/section/world/asia}{Asia Pacific}\textbar{}Samsung's Leader Is
Indicted on Bribery Charges

\url{https://nyti.ms/2m31MmL}

\begin{itemize}
\item
\item
\item
\item
\item
\end{itemize}

Advertisement

\protect\hyperlink{after-top}{Continue reading the main story}

Supported by

\protect\hyperlink{after-sponsor}{Continue reading the main story}

\hypertarget{samsungs-leader-is-indicted-on-bribery-charges}{%
\section{Samsung's Leader Is Indicted on Bribery
Charges}\label{samsungs-leader-is-indicted-on-bribery-charges}}

\includegraphics{https://static01.nyt.com/images/2017/03/01/world/01Indite/01Indite-articleLarge-v2.jpg?quality=75\&auto=webp\&disable=upscale}

By \href{http://www.nytimes.com/by/choe-sang-hun}{Choe Sang-Hun}

\begin{itemize}
\item
  Feb. 28, 2017
\item
  \begin{itemize}
  \item
  \item
  \item
  \item
  \item
  \end{itemize}
\end{itemize}

SEOUL, South Korea --- The head of Samsung, one of the world's largest
conglomerates, was indicted on bribery and embezzlement charges on
Tuesday, becoming one of the most prominent business tycoons ever to
face trial in South Korea.

The indictment of Lee Jae-yong, the company's de facto leader, came at
the end of a special prosecutor's 90-day investigation of a corruption
scandal that has already led to the
\href{https://www.nytimes.com/2016/12/08/world/asia/south-korea-park-geun-hye-accusations-impeachment.html}{impeachment
of President Park Geun-hye}. When huge crowds took to the streets in
recent months to demand that she leave office, they also called for the
toppling of Mr. Lee and other corporate titans.

Mr. Lee was
\href{https://www.nytimes.com/2017/02/16/world/asia/korea-samsung-lee-jae-yong.html}{arrested}
on Feb. 17, a dramatic development in South Korea's struggle to end
collusive ties between the government and the family-controlled
conglomerates, or chaebol, that dominate the economy.

Four other senior executives of Samsung were also indicted Tuesday, but
not arrested, on the same corruption charges as Mr. Lee, and three of
the four resigned. Those indictments had been expected and were not seen
as indications of a threat to the Lee family's control of the business.

South Koreans have grown weary of endemic corruption and the country's
traditional leniency toward tycoons accused of white-collar crimes. For
decades, presidents have entered office vowing to end such favoritism,
but they all eventually backtracked. Anticorruption advocates say Mr.
Lee's indictment and trial will be a test of whether the system can
finally make a dent in those cozy relationships.

Samsung, by far the largest of the chaebol, has long been a symbol of
power and wealth in a nation that has transformed itself from an
agrarian economy to one of the world's technological powerhouses.
Samsung's market capitalization accounts for one-fourth of the value of
all listed companies in South Korea, and its main unit, Samsung
Electronics, alone ships 20 percent of the country's total exports.

Mr. Lee was accused of giving or promising \$38 million in bribes to
\href{https://www.nytimes.com/2016/11/01/world/asia/south-korea-park-geun-hye-choi-soon-sil.html}{Choi
Soon-sil}, a secretive confidante of Ms. Park. In return, the prosecutor
said in his indictment, Mr. Lee received political favors, most notably
government support for a merger of two Samsung affiliates in 2015 that
helped him inherit corporate control from his incapacitated father, Lee
Kun-hee.

Mr. Lee was also accused of committing perjury when he insisted during a
parliamentary hearing that he had never bribed Ms. Choi or Ms. Park. He
still claims that the ``donations'' Samsung paid out to Ms. Choi were
coerced, suggesting that the company was extorted.

Samsung has said it will try to clear Mr. Lee's name at trial. It did
not immediately comment on his indictment on Tuesday. In the South
Korean system, once a suspect is formally arrested, indictment
automatically follows, unless evidence emerges that proves the person's
innocence. Those cases are extremely rare.

Mr. Lee, 48, a vice chairman of Samsung, has been running the company
since his father had a
\href{https://www.nytimes.com/2014/05/12/business/international/samsungs-chairman-has-surgery-after-heart-attack.html}{heart
attack in 2014}. His indictment comes after a challenging period for the
company, which issued a global recall of its Galaxy Note 7 smartphones,
the most ambitious product launched under his leadership, because they
\href{https://www.nytimes.com/2016/09/03/business/samsung-galaxy-note-battery.html}{were
prone to catching fire}.

The elder Mr. Lee was convicted of bribery and tax evasion twice but
never spent a day in jail. Each time, he was pardoned by the president
and returned to the company. At least six of the nation's top 10 chaebol
--- which generate revenue equivalent to more than 80 percent of gross
domestic product --- are led by men once convicted of white-collar
crimes.

Ms. Park was identified as a criminal accomplice in November, when state
prosecutors indicted Ms. Choi on charges of extorting tens of millions
of dollars from Samsung and other chaebol by leveraging her connections
with the president. But she was protected from indictment while in
office.

On Tuesday, the special prosecutor, Park Young-soo, added a bribery
charge to the case against Ms. Choi, who is already on trial. He said
Ms. Park could also face bribery and extortion charges once she leaves
office. Ms. Park denies any wrongdoing, saying the money from Samsung
was part of ``donations'' that businesses provided to two foundations
that prosecutors said were controlled by Ms. Choi.

Ms. Park's presidential powers have been suspended since
\href{https://www.nytimes.com/2016/12/09/world/asia/south-korea-president-park-geun-hye-impeached.html}{the
National Assembly impeached her} in December. The Constitutional Court
is expected to rule in the coming weeks on whether Ms. Park should be
formally ousted or reinstated and allowed to finish her five-year term,
which ends next February.

In the current scandal, Samsung was accused of making payments to Ms.
Choi in exchange for a crucial vote by the government-controlled
National Pension Service to support the 2015 merger of two Samsung
affiliates. The special prosecutor says Ms. Park ordered the pension
fund to support the merger on Mr. Lee's behalf.

The merger caused a loss of at least \$123 million for the national
pension fund, which held large stakes in the two affiliates, but it
increased the stock value of the Lee family by at least \$758 million,
the prosecutor said.

The four executives under Mr. Lee who were also indicted belonged to
Samsung Electronics or to the conglomerate's powerful, secretive
Corporate Strategy Office. Critics say the office worked mainly to
tighten the Lee family's imperial control of the conglomerate and
enforce the father-to-son transfer of leadership. On Tuesday, Samsung
said it was disbanding the office as part of its effort to make its
corporate governance more transparent.

Advertisement

\protect\hyperlink{after-bottom}{Continue reading the main story}

\hypertarget{site-index}{%
\subsection{Site Index}\label{site-index}}

\hypertarget{site-information-navigation}{%
\subsection{Site Information
Navigation}\label{site-information-navigation}}

\begin{itemize}
\tightlist
\item
  \href{https://help.nytimes.com/hc/en-us/articles/115014792127-Copyright-notice}{©~2020~The
  New York Times Company}
\end{itemize}

\begin{itemize}
\tightlist
\item
  \href{https://www.nytco.com/}{NYTCo}
\item
  \href{https://help.nytimes.com/hc/en-us/articles/115015385887-Contact-Us}{Contact
  Us}
\item
  \href{https://www.nytco.com/careers/}{Work with us}
\item
  \href{https://nytmediakit.com/}{Advertise}
\item
  \href{http://www.tbrandstudio.com/}{T Brand Studio}
\item
  \href{https://www.nytimes.com/privacy/cookie-policy\#how-do-i-manage-trackers}{Your
  Ad Choices}
\item
  \href{https://www.nytimes.com/privacy}{Privacy}
\item
  \href{https://help.nytimes.com/hc/en-us/articles/115014893428-Terms-of-service}{Terms
  of Service}
\item
  \href{https://help.nytimes.com/hc/en-us/articles/115014893968-Terms-of-sale}{Terms
  of Sale}
\item
  \href{https://spiderbites.nytimes.com}{Site Map}
\item
  \href{https://help.nytimes.com/hc/en-us}{Help}
\item
  \href{https://www.nytimes.com/subscription?campaignId=37WXW}{Subscriptions}
\end{itemize}
