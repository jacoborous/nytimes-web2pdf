Sections

SEARCH

\protect\hyperlink{site-content}{Skip to
content}\protect\hyperlink{site-index}{Skip to site index}

\href{https://www.nytimes.com/section/politics}{Politics}

\href{https://myaccount.nytimes.com/auth/login?response_type=cookie\&client_id=vi}{}

\href{https://www.nytimes.com/section/todayspaper}{Today's Paper}

\href{/section/politics}{Politics}\textbar{}Elliott Abrams,
Neoconservative Who Rejected Trump, May Serve Him

\url{https://nyti.ms/2jVYn4w}

\begin{itemize}
\item
\item
\item
\item
\item
\end{itemize}

Advertisement

\protect\hyperlink{after-top}{Continue reading the main story}

Supported by

\protect\hyperlink{after-sponsor}{Continue reading the main story}

\hypertarget{elliott-abrams-neoconservative-who-rejected-trump-may-serve-him}{%
\section{Elliott Abrams, Neoconservative Who Rejected Trump, May Serve
Him}\label{elliott-abrams-neoconservative-who-rejected-trump-may-serve-him}}

\includegraphics{https://static01.nyt.com/images/2017/02/07/us/07abrams/07abrams-articleInline.jpg?quality=75\&auto=webp\&disable=upscale}

By \href{http://www.nytimes.com/by/gardiner-harris}{Gardiner Harris} and
\href{http://www.nytimes.com/by/david-e-sanger}{David E. Sanger}

\begin{itemize}
\item
  Feb. 6, 2017
\item
  \begin{itemize}
  \item
  \item
  \item
  \item
  \item
  \end{itemize}
\end{itemize}

WASHINGTON --- Elliott Abrams, a neoconservative who has long argued for
an activist foreign policy that spreads American values around the
world, was advising Republicans just last spring to ``keep your
distance'' from Donald J. Trump and offering advice about what the party
should do after the ``Trump collapse.''

On Tuesday, Mr. Abrams is set to visit President Trump in the White
House to determine whether there is a job for him in the new
administration, as the State Department's No. 2 official.

As part of the vetting process to see whether Mr. Abrams will serve as
Secretary of State Rex W. Tillerson's deputy, his writings have been
scrutinized in a White House suspicious of anyone who was not a Trump
loyalist from the beginning.

But the advantage of picking Mr. Abrams is clear: He knows the inner
workings of the department, he served under Ronald Reagan and George W.
Bush, and, like Mr. Trump, he is often a critic of the Washington
foreign policy establishment. Of course, he is also a member of it.

It appears that selecting Mr. Abrams is not a done deal, with much
probably riding on his conversation with a president whose campaign he
urged others not to join. But some are already predicting that he will
emerge as part of the Trump team.

``I think he's pretty close to being named,'' James Jay Carafano, a
Heritage Foundation fellow who advised the Trump transition team on the
State Department, said in an interview.

Mr. Abrams, 69, is described politely in foreign policy circles as a
``controversial'' figure, but that deeply understates the case.

He is remembered best for the days when he was an assistant secretary of
state during the Reagan administration, and his conviction in 1991 on
two misdemeanor counts of withholding information from Congress during
the Iran-contra affair. He was later pardoned by President George Bush,
and that moment has largely receded from memory --- although if he is
nominated, there is little question that Democrats will bring it up
again.

Still, his selection would calm many at the State Department who worry
that Mr. Tillerson, who has never served in government, is about to
discover how running a large government bureaucracy full of dissenting
opinions differs from running Exxon Mobil, where he was chief executive.
Mr. Abrams knows the building well and, with a genial style and sharp
views, knows how to navigate the national security bureaucracy.

His last stint in government was as deputy national security adviser
during the George W. Bush administration. He often collaborated but
sometimes feuded with Secretary of State Condoleezza Rice over whether
the Middle East peace process was, in fact, meaningful.

``The peace process was like Tinkerbell in that if we all just believe
in it firmly enough, it really would survive,'' he wrote in a book,
``Tested by Zion,'' about the Bush administration's efforts.

In ordinary times, nominating Mr. Abrams would set off tremendous
opposition, especially from the left. But Mr. Trump's many promises
during the campaign to upend decades of bipartisan foreign policy
doctrine, as well as a series of phone calls and meetings in recent
weeks that left allies feeling insulted, led many experts to applaud him
as a man who could put the Trump administration on a more predictable
path.

``For the most part, I think he would be welcomed in the State
Department,'' said Dennis Ross, the senior Middle East adviser under
President Barack Obama. ``He's seen as serious, responsible and
knowledgeable.''

Mr. Abrams's nomination would be the beginning of an important process
to fill out the State Department's top ranks. The Trump administration,
as is common, asked nearly all of the Obama administration's political
appointees to leave their posts, including some seen as vital to the
day-to-day management of far-flung operations.

But the speed of their departures, before they could hand off their
duties to successors, was jarring. So was the arrival last week of Matt
Mowers, a young former political aide to Gov. Chris Christie of New
Jersey, as a senior White House adviser to the State Department, which
added to growing unease that the White House would be pulling the
strings at the department.

While Mr. Abrams did not sign either of two letters from former
officials who said they would never work for Mr. Trump, he once told the
Voice of America that Mr. Trump had ``no appreciation whatsoever of the
importance of allies and alliances.''

He wrote
\href{http://www.weeklystandard.com/when-you-cant-stand-your-candidate/article/2002283}{an
article} in The Weekly Standard titled ``When You Can't Stand Your
Candidate'' --- ostensibly a look back at the 1972 election --- urging
that the Republican convention not be ``a coronation wherein Trump and
Trumpism are unchallenged.''

But Mr. Abrams's years of direct experience in government are seen as an
important asset for Mr. Tillerson. And his thick skin from previous
battles could help if Mr. Tillerson seeks a wholesale revamping of the
department.

Advertisement

\protect\hyperlink{after-bottom}{Continue reading the main story}

\hypertarget{site-index}{%
\subsection{Site Index}\label{site-index}}

\hypertarget{site-information-navigation}{%
\subsection{Site Information
Navigation}\label{site-information-navigation}}

\begin{itemize}
\tightlist
\item
  \href{https://help.nytimes.com/hc/en-us/articles/115014792127-Copyright-notice}{©~2020~The
  New York Times Company}
\end{itemize}

\begin{itemize}
\tightlist
\item
  \href{https://www.nytco.com/}{NYTCo}
\item
  \href{https://help.nytimes.com/hc/en-us/articles/115015385887-Contact-Us}{Contact
  Us}
\item
  \href{https://www.nytco.com/careers/}{Work with us}
\item
  \href{https://nytmediakit.com/}{Advertise}
\item
  \href{http://www.tbrandstudio.com/}{T Brand Studio}
\item
  \href{https://www.nytimes.com/privacy/cookie-policy\#how-do-i-manage-trackers}{Your
  Ad Choices}
\item
  \href{https://www.nytimes.com/privacy}{Privacy}
\item
  \href{https://help.nytimes.com/hc/en-us/articles/115014893428-Terms-of-service}{Terms
  of Service}
\item
  \href{https://help.nytimes.com/hc/en-us/articles/115014893968-Terms-of-sale}{Terms
  of Sale}
\item
  \href{https://spiderbites.nytimes.com}{Site Map}
\item
  \href{https://help.nytimes.com/hc/en-us}{Help}
\item
  \href{https://www.nytimes.com/subscription?campaignId=37WXW}{Subscriptions}
\end{itemize}
