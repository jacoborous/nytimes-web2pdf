Sections

SEARCH

\protect\hyperlink{site-content}{Skip to
content}\protect\hyperlink{site-index}{Skip to site index}

\href{https://www.nytimes.com/section/world/asia}{Asia Pacific}

\href{https://myaccount.nytimes.com/auth/login?response_type=cookie\&client_id=vi}{}

\href{https://www.nytimes.com/section/todayspaper}{Today's Paper}

\href{/section/world/asia}{Asia Pacific}\textbar{}Chinese Censors Have
New Target: Celebrity News

\url{https://nyti.ms/2s4dNLJ}

\begin{itemize}
\item
\item
\item
\item
\item
\end{itemize}

Advertisement

\protect\hyperlink{after-top}{Continue reading the main story}

Supported by

\protect\hyperlink{after-sponsor}{Continue reading the main story}

\hypertarget{chinese-censors-have-new-target-celebrity-news}{%
\section{Chinese Censors Have New Target: Celebrity
News}\label{chinese-censors-have-new-target-celebrity-news}}

!{[}An internet cafe in Beijing. Entertainment news had typically been
thought of as a safe zone for reporting in China.

Credit...Roman Pilipey/European Pressphoto
Agency{]}(\url{https://static01.nyt.com/images/2017/06/10/world/10wechat-1/10wechat-1-articleLarge.jpg?quality=75\&auto=webp\&disable=upscale})

By Amy Qin

\begin{itemize}
\item
  June 9, 2017
\item
  \begin{itemize}
  \item
  \item
  \item
  \item
  \item
  \end{itemize}
\end{itemize}

\href{https://cn.nytimes.com/china/20170612/china-celebrity-news-wechat/}{阅读简体中文版}

BEIJING --- Whether read openly and voraciously or behind closed doors,
celebrity gossip plays an integral role in the entertainment world,
connecting stars and the big businesses that back them to an audience
eager for the juiciest of details.

But to some officials in China, the bloggers that report those tidbits
play another role: a threat to public order.

A large number of Chinese ``celebrity news'' blogs have disappeared in
recent days after coming under the scrutiny of China's cyberspace
regulators. Their absence comes amid a broader tightening of online and
media controls ahead of a once-in-every-five-years meeting of
\href{https://www.nytimes.com/2016/10/05/world/asia/china-president-xi-jinping-successor.html}{top
Communist Party leaders} this year, at which party officials will
consider major decisions about who will lead the country in the coming
years.

At a meeting on Wednesday with representatives from China's leading
internet companies, officials from the Beijing bureau of the Cyberspace
Administration of China, the country's top online regulator, called on
the companies to ``actively promote socialist core values'' and create a
``healthy, uplifting environment for mainstream opinion'' by combating
vulgar and sensationalist coverage of celebrity scandals and lifestyles.

Since that meeting,
\href{http://news.xinhuanet.com/politics/2017-06/07/c_1121104271.htm}{reported}
by the state broadcaster China Central Television, major Chinese
internet companies like Tencent and Baidu and the news aggregation
platform Jinri Toutiao have shut down more than 80 popular
entertainment-related public accounts, according to state news outlets.
Many were on Tencent's WeChat social-media service, which is widely used
in China and is increasingly a source of news and information.

Many of the closed blogs and accounts were making a tidy profit from
advertising revenue, and some recently turned to venture capital
investors as a route to growth. Zhuo Wei, known as China's No. 1
paparazzo, had more than seven million followers for his coverage of
celebrities like the singer Faye Wong and the Chinese actress Bai Baihe.
He could not immediately be reached for comment.

At least one of the closed accounts was affiliated with a global brand.
The entertainment-related WeChat account of the fashion magazine
Harper's Bazaar was shut down, although its account on Weibo, another
social-media service, and its general WeChat account appeared to have
survived. A spokesman for Harper's Bazaar could not immediately be
reached for comment.

The closings prompted a swift outcry from China's media circles. While
the local news industry has long been subjected to strict government
censorship on politics and other topics deemed to be delicate,
entertainment news has typically been viewed as safe.

``In China, there were only two areas before that we could say had news
freedom: One was entertainment, and the other was sports,'' said Gao
Ming, host of the podcast Radio HiLight and a former editor at
AsiaContent.com, an online entertainment news company. ``But now I think
the government is trying to send a message that all the news needs to be
within its control.''

Using a Chinese phrase that means to make an example of someone or
something, Mr. Gao added, ``They're killing the chicken to scare the
monkeys.''

Of particular concern for many fans and industry watchers was the
shutdown of popular entertainment-related accounts that were not in the
business of trading celebrity gossip. One of these was the WeChat public
account of the popular movie-review blog Dushe Dianying, which still had
more than 4.8 million followers on Weibo. In July, the Chinese news
media reported that Dushe Dianying was valued at about \$44 million,
after completing an early round of financing.

``Generally, China's leaders have been obsessed with the containment of
negative coverage, and under Xi Jinping we've seen a rather dramatic
decline in serious coverage by China's media,'' David Bandurski, editor
of the China Media Project at the University of Hong Kong, said in
emailed comments, referring to the Chinese president. ``What we're now
seeing is a war on the nonserious.''

``It's no longer enough for media content to avoid the negative,'' he
added. ``It must be adequately positive.''

The shutdowns come after a
\href{https://www.nytimes.com/2017/05/31/business/china-cybersecurity-law.html}{new
cybersecurity law} and new
\href{http://news.xinhuanet.com/2017-05/22/c_1121016109.htm}{regulations}
issued by the Cyberspace Administration of China concerning the
provision of news information through social media platforms like WeChat
came into effect last week. According to the regulations, all online
publishers --- including websites, apps, blogs and social media accounts
--- must obtain permits from the authorities in order to publish news or
news commentaries.

``These regulations could have quite a dramatic impact on the ecosystem
of public accounts on WeChat, and on the broader online space,'' Mr.
Bandurski said. ``This is something we must watch closely.''

Some of the affected accounts have already begun to regroup. By Thursday
morning, Go Ying Studio, known for celebrity photographs and gossip, had
created a new account and posted an official apology. ``For us, this was
a very profound lesson,'' the statement read. ``We wholeheartedly accept
the criticism and help given to us by different corners of society, and
from now on, we will make sure to strengthen our own thinking and moral
education.''

But the future of many of the accounts remains uncertain --- a cause for
concern among the bloggers' millions of fans.

``To me, it has become a habit to read some of these accounts every
evening to relax,'' lamented one reader, Luna, in a comment on a
Chinese-language article about the deleted accounts. ``People need these
kinds of positive entertainment news. I am so sad.''

Advertisement

\protect\hyperlink{after-bottom}{Continue reading the main story}

\hypertarget{site-index}{%
\subsection{Site Index}\label{site-index}}

\hypertarget{site-information-navigation}{%
\subsection{Site Information
Navigation}\label{site-information-navigation}}

\begin{itemize}
\tightlist
\item
  \href{https://help.nytimes.com/hc/en-us/articles/115014792127-Copyright-notice}{©~2020~The
  New York Times Company}
\end{itemize}

\begin{itemize}
\tightlist
\item
  \href{https://www.nytco.com/}{NYTCo}
\item
  \href{https://help.nytimes.com/hc/en-us/articles/115015385887-Contact-Us}{Contact
  Us}
\item
  \href{https://www.nytco.com/careers/}{Work with us}
\item
  \href{https://nytmediakit.com/}{Advertise}
\item
  \href{http://www.tbrandstudio.com/}{T Brand Studio}
\item
  \href{https://www.nytimes.com/privacy/cookie-policy\#how-do-i-manage-trackers}{Your
  Ad Choices}
\item
  \href{https://www.nytimes.com/privacy}{Privacy}
\item
  \href{https://help.nytimes.com/hc/en-us/articles/115014893428-Terms-of-service}{Terms
  of Service}
\item
  \href{https://help.nytimes.com/hc/en-us/articles/115014893968-Terms-of-sale}{Terms
  of Sale}
\item
  \href{https://spiderbites.nytimes.com}{Site Map}
\item
  \href{https://help.nytimes.com/hc/en-us}{Help}
\item
  \href{https://www.nytimes.com/subscription?campaignId=37WXW}{Subscriptions}
\end{itemize}
