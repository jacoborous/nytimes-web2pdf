Sections

SEARCH

\protect\hyperlink{site-content}{Skip to
content}\protect\hyperlink{site-index}{Skip to site index}

\href{https://www.nytimes.com/section/world/middleeast}{Middle East}

\href{https://myaccount.nytimes.com/auth/login?response_type=cookie\&client_id=vi}{}

\href{https://www.nytimes.com/section/todayspaper}{Today's Paper}

\href{/section/world/middleeast}{Middle East}\textbar{}Arab TV Series
Dramatizes Life Under ISIS

\url{https://nyti.ms/2rkvW43}

\begin{itemize}
\item
\item
\item
\item
\item
\end{itemize}

Advertisement

\protect\hyperlink{after-top}{Continue reading the main story}

Supported by

\protect\hyperlink{after-sponsor}{Continue reading the main story}

\hypertarget{arab-tv-series-dramatizes-life-under-isis}{%
\section{Arab TV Series Dramatizes Life Under
ISIS}\label{arab-tv-series-dramatizes-life-under-isis}}

\includegraphics{https://static01.nyt.com/images/2017/05/15/world/middleeast/isistv/isistv-articleLarge.jpg?quality=75\&auto=webp\&disable=upscale}

By \href{http://www.nytimes.com/by/ben-hubbard}{Ben Hubbard}

\begin{itemize}
\item
  May 16, 2017
\item
  \begin{itemize}
  \item
  \item
  \item
  \item
  \item
  \end{itemize}
\end{itemize}

A mother travels to Syria to find her son after he ran away to join the
Islamic State. A Christian renounces her faith and plots to blow up a
church. A black-clad matron tells teenage girls to rest up before they
are raped by extremist fighters.

These frighteningly familiar stories of life under the Islamic State in
Iraq and Syria will this month become plotlines on prime-time television
across the Arab world.

A sprawling, 30-part dramatic series is scheduled to make its debut on
\href{http://www.mbc.net/en/corporate/channels/mbc1.html}{MBC 1}, the
Arab world's most watched satellite channel, during the holy month of
Ramadan, said Ali Jaber, the director of television for the MBC Network.

The network shared with The New York Times video, available below, from
three episodes of the program, ``Black Crows.''

It paints a picture of the Islamic State, also known as ISIS, as a
brutal criminal organization run by corrupt and hypocritical leaders.
But recruits are depicted as victims, and women who challenge the
militants' control are heroes.

In one episode from the series, a Yazidi slave is sent to clean an
Islamic State fighter's room, where his bored wife asks if the woman is
hungry or would like to watch a movie. The captive woman is outraged.

\includegraphics{https://static01.nyt.com/images/2017/05/13/world/13blackcrows-3/13blackcrows-3-videoSixteenByNineJumbo1600.jpg}

The stories of women dominate the series, the producers said, because
they offered rich dramatic material. A majority of the channel's viewers
are women.

In another episode, Islamic State commanders indoctrinate children into
their ranks.

\includegraphics{https://static01.nyt.com/images/2017/05/13/world/13blackcrows-2/13blackcrows-2-videoSixteenByNineJumbo1600.jpg}

Like the Islamic State's recruits, the cast comes from across the Arab
world, and the program's plotlines reflect well-known headlines about
the group's atrocities.

Ramadan, which will begin around May 27, is a month on the Islamic
calendar during which Muslims fast from dawn to dusk. It is also peak
television season in the Arab world, where families gather after
breaking their fast to binge-watch shows late into the night.

In television terms, ``it's like the Super Bowl for 30 days straight,''
said Mazen Hayek, a spokesman for MBC.

Typical programming includes romances, comedies and historical dramas,
some of which reflect current events. Though the new MBC production has
the trappings of a drama, and some of the costumes and makeup can be
cartoonish, the series, set behind the jihadists' front lines, is not
light viewing.

\includegraphics{https://static01.nyt.com/images/2017/05/15/world/isistv2/isistv2-articleLarge.jpg?quality=75\&auto=webp\&disable=upscale}

Another story line involves a journalist whose fiancé became an Islamic
State suicide bomber. She goes undercover to report on the group, and
pledges to abandon her Christian faith and blow up a church.

The actress, Samar Allam, said in a phone interview that being on set
and getting into character depressed her, but that she hoped the program
would make people think in a way that news reports about the Islamic
State's violence could not.

``ISIS is a danger to all of humanity,'' she said, using an acronym for
the group. The show allowed her to ``show my hatred and my condemnation
of this group, to express it in a concrete way.''

Marwa Mohamed, a Saudi actress, plays a woman who kills her husband for
cheating on her and flees to join the Islamic State with her two sons.
After one is sexually abused and the other is killed, she struggles to
escape.

``It is important to wake people up and show them that Islam is not
that,'' Ms. Mohamed said.

She said she hoped that despite the dark subject matter, viewers would
tune in for the human stories. ``It's not all terrorism and war,'' she
said. ``There are lots of dramatic stories in it as well.''

The series echoes news coverage of the Islamic State, with explosions
that leave bodies scattered and gunmen waving black flags, but it
dramatizes the lives of people forced to live under the group.

\includegraphics{https://static01.nyt.com/images/2017/05/13/world/middleeast/13blackcrows-1/13blackcrows-1-videoSixteenByNineJumbo1600.jpg}

Mr. Jaber of MBC, who is known in the Arab world for being a judge on
the reality competition ``Arabs Got Talent,'' said the series sought to
harness the influence of popular television to undermine the narrative
that the Islamic State uses to entice recruits.

``We believe that this is an epidemic, this is a disease that we have to
muster the courage to address and fight,'' he said.

Still, producing a program about the Islamic State in the region where
the group has done the most damage carries risks.

A comedy show on MBC that mocked the group led to death threats against
its star. And Mr. Jaber said some sponsors might hesitate to advertise
their products on such a violent series about a terrorist group.

Image

The series depicts the Islamic State as a brutal criminal organization
run by corrupt and hypocritical leaders.Credit...MBC 1

``It will bring eyeballs, it will bring buzz and ratings and reputation,
but no money,'' he said.

Since the Islamic State stormed through Syria and Iraq, shocking the
world with its
\href{https://www.nytimes.com/2014/08/20/world/middleeast/isis-james-foley-syria-execution.html}{choreographed
beheadings} and
\href{https://www.nytimes.com/2015/02/04/world/middleeast/isis-said-to-burn-captive-jordanian-pilot-to-death-in-new-video.html}{elaborate
executions}, governments have struggled to defeat the group and counter
its
\href{https://www.nytimes.com/2014/08/31/world/middleeast/isis-displaying-a-deft-command-of-varied-media.html?_r=0}{potent,
made-for-television propaganda}.

Some diplomats have told Mr. Jaber that they like the idea of using
television to challenge the jihadists' message. In March, he was invited
to discuss the show with Western and Middle Eastern diplomats at a
\href{https://www.nytimes.com/2017/03/22/world/middleeast/rex-tillerson-isis.html}{summit
meeting} in Washington hosted by Secretary of State
\href{http://topics.nytimes.com/top/reference/timestopics/people/t/rex_w_tillerson/index.html?inline=nyt-per}{Rex
W. Tillerson}.

Other Ramadan series, dramas and comedies have referred to the Islamic
State, but ``Black Crows'' appears to be the first to be set entirely in
the militants' world, said Rebecca Joubin, an associate professor of
Arab studies at Davidson College who studies the region's television
programs.

The most popular Ramadan shows are often escapist love stories, full of
beautiful people wearing nice clothes. ``A lot of people are like: `I
don't want to watch this stuff. I see it on the news every day,' '' Ms.
Joubin said.

Still, the series' producers expect it to be widely viewed during
Ramadan, when MBC 1 traditionally sees a spike in viewership.

The show will be broadcast in Arabic as ``Al Gharabeeb Al Soud,'' and
the network hopes to produce an English-language version for wider
distribution.

Advertisement

\protect\hyperlink{after-bottom}{Continue reading the main story}

\hypertarget{site-index}{%
\subsection{Site Index}\label{site-index}}

\hypertarget{site-information-navigation}{%
\subsection{Site Information
Navigation}\label{site-information-navigation}}

\begin{itemize}
\tightlist
\item
  \href{https://help.nytimes.com/hc/en-us/articles/115014792127-Copyright-notice}{©~2020~The
  New York Times Company}
\end{itemize}

\begin{itemize}
\tightlist
\item
  \href{https://www.nytco.com/}{NYTCo}
\item
  \href{https://help.nytimes.com/hc/en-us/articles/115015385887-Contact-Us}{Contact
  Us}
\item
  \href{https://www.nytco.com/careers/}{Work with us}
\item
  \href{https://nytmediakit.com/}{Advertise}
\item
  \href{http://www.tbrandstudio.com/}{T Brand Studio}
\item
  \href{https://www.nytimes.com/privacy/cookie-policy\#how-do-i-manage-trackers}{Your
  Ad Choices}
\item
  \href{https://www.nytimes.com/privacy}{Privacy}
\item
  \href{https://help.nytimes.com/hc/en-us/articles/115014893428-Terms-of-service}{Terms
  of Service}
\item
  \href{https://help.nytimes.com/hc/en-us/articles/115014893968-Terms-of-sale}{Terms
  of Sale}
\item
  \href{https://spiderbites.nytimes.com}{Site Map}
\item
  \href{https://help.nytimes.com/hc/en-us}{Help}
\item
  \href{https://www.nytimes.com/subscription?campaignId=37WXW}{Subscriptions}
\end{itemize}
