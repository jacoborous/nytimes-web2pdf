Sections

SEARCH

\protect\hyperlink{site-content}{Skip to
content}\protect\hyperlink{site-index}{Skip to site index}

\href{https://myaccount.nytimes.com/auth/login?response_type=cookie\&client_id=vi}{}

\href{https://www.nytimes.com/section/todayspaper}{Today's Paper}

\href{/section/opinion}{Opinion}\textbar{}What If the Russian Revolution
Had Never Happened?

\href{https://nyti.ms/2hKpZv1}{https://nyti.ms/2hKpZv1}

\begin{itemize}
\item
\item
\item
\item
\item
\item
\end{itemize}

Advertisement

\protect\hyperlink{after-top}{Continue reading the main story}

Supported by

\protect\hyperlink{after-sponsor}{Continue reading the main story}

\href{/section/opinion}{Opinion}

\href{/column/red-century}{Red Century}

\hypertarget{what-if-the-russian-revolution-had-never-happened}{%
\section{What If the Russian Revolution Had Never
Happened?}\label{what-if-the-russian-revolution-had-never-happened}}

By Simon Sebag Montefiore

\begin{itemize}
\item
  Nov. 6, 2017
\item
  \begin{itemize}
  \item
  \item
  \item
  \item
  \item
  \item
  \end{itemize}
\end{itemize}

\includegraphics{https://static01.nyt.com/images/2017/11/06/opinion/06redcenturyWeb/06redcenturyWeb-articleLarge.jpg?quality=75\&auto=webp\&disable=upscale}

The October Revolution, organized by Vladimir Lenin exactly a century
ago, is still relevant today in ways that would have seemed unimaginable
when Soviet Communism collapsed.

Marxist-Leninism (albeit in the unique capitalist-Maoist form) still
propels China, the world's surging hyperpower, even as that same
ideology ruins Cuba and Venezuela. Meanwhile, North Korea, a dystopian
Leninist monarchy with nuclear weapons, terrifies the world. Even more
surprisingly, Communism is experiencing a resurrection in democratic
Britain: Jeremy Corbyn, that quasi-Leninist comfortingly disguised as
cuddly grey-beard, is the most extreme politician ever to lead one of
Britain's two main parties, and he is inching toward power.

But Lenin's tactics, too, are resurgent. He was a sophisticated genius
of merciless zero-sum gain, expressed by his phrase
\href{https://en.wikipedia.org/wiki/Who,_whom\%3F}{``Kto kovo?''} ---
literally, ``Who, whom?'' asking the question who controls whom and,
more important, who kills whom. President Trump is some ways the
personification of a new Bolshevism of the right where the ends justify
the means and acceptable tactics include lies and smears, and the
exploitation of what Lenin called useful idiots. It's no coincidence
that President Trump's chief campaign strategist, Steve Bannon,
\href{https://www.thedailybeast.com/steve-bannon-trumps-top-guy-told-me-he-was-a-leninist}{once
boasted ``I am a Leninist.''}

One hundred years later, as its events continue to reverberate and
inspire, October 1917 looms epic, mythic, mesmerizing. Its effects were
so enormous that it seems impossible that it might not have happened the
way it did.

And yet it nearly didn't.

There was nothing inevitable about the Bolshevik revolution. By 1917,
the Romanov monarchy was decaying quickly, but its emperors may have
saved themselves had they not missed repeated chances to reform. The
other absolute monarchies of Europe --- the Ottomans, the Habsburgs ---
fell because they were defeated in World War I. Would the Romanovs have
fallen, too, if they had survived just one more year to share in the
victory of November 1918?

By 1913, the czar's secret police had dispersed and vanquished the
opposition. Just before the fall of the czar, Lenin reflected to his
wife that revolution ``won't happen in our lifetime.'' Ultimately, it
was a spontaneous, disorganized popular uprising and a crisis of
military loyalty that forced Nicholas's abdication. When that moment
arrived, Lenin was in Zurich,
\href{https://www.jacobinmag.com/2016/10/trotsky-new-york-socialist-party-debs-revolution/}{Trotsky
in New York} and Stalin in Siberia.

Lenin initially thought it was ``a hoax.'' He was lucky that Germany
inserted him like a bacillus (via
\href{https://www.nytimes.com/2017/06/19/opinion/was-lenin-a-german-agent.html}{the
so-called sealed train}) to take Russia out of the war. Back in
Petrograd, Lenin, aided by fellow-radicals Trotsky and Stalin, had to
overpower erring Bolshevik comrades, who proposed cooperation with the
provisional government, and force them to agree to his plan for a coup.
The government should have found and killed him but it failed to do so.
He succeeded.

Even the ``storming'' of the Winter Palace --- restaged in
\href{https://www.youtube.com/watch?v=fLihunxEzwE}{a 1920 propaganda
spectacular} as a people's triumph --- was no storming at all. Lenin
rages as it took days to seize the main buildings of the government,
while the palace itself was taken by climbing through unlocked windows,
undefended except for adolescent cadets --- followed by a bacchanalia,
with drunk Bolsheviks slurping the czar's Château d'Yquem 1847 out of
the gutters.

October might have heralded a short-lived interim, like so many other
failed revolutions of that era. Any coordinated attack by White armies,
the other side in the Russian civil war, or any intervention by Western
forces would have swept the Bolsheviks away. It all depended on Lenin.
He was very nearly overthrown in a coup by rebellious coalition partners
but he made his own luck, though, by a combination of ideological
passion, ruthless pragmatism, unchecked bloodletting and the will to
establish a dictatorship. And sometimes, he just got plain lucky: On
Aug. 30, 1918, he was shot while addressing a crowd of workers at a
factory in Moscow. He survived by inches.

Had any of these events foiled Lenin, our own times would be radically
different. Without Lenin there would have been no Hitler. Hitler owed
much of his rise to the support of conservative elites who feared a
Bolshevik revolution on German soil and who believed that he alone could
defeat Marxism. And the rest of his radical program was likewise
justified by the threat of Leninist revolution. His anti-Semitism, his
anti-Slavic plan for Lebensraum and above all the invasion of the Soviet
Union in 1941 were supported by the elites and the people because of the
fear of what the Nazis called ``Judeo-Bolshevism.''

Without the Russian Revolution of 1917, Hitler would likely have ended
up painting postcards in one of the same flophouses where he started. No
Lenin, no Hitler --- and the 20th century becomes unimaginable. Indeed,
the very geography of our imagination becomes unimaginable.

The East would look as different as the West. Mao, who received huge
amounts of Soviet aid in the 1940s, would not have conquered China,
which might still be ruled by the family of
\href{https://www.nytimes.com/topic/person/chiang-kaishek}{Chiang
Kai-shek}. The inspirations that illuminated the mountains of Cuba and
the jungles of Vietnam would never have been. Kim Jong-un, pantomimic
pastiche of Stalin, would not exist. There would have been no Cold War.
The tournaments of power would likely have been just as vicious --- just
differently vicious.

The Russian Revolution mobilized a popular passion across the world
based on Marxism-Leninism, fueled by
\href{https://www.ft.com/content/21eccc8e-7d35-11e7-ab01-a13271d1ee9c}{messianic
zeal}. It was, perhaps, after the three Abrahamic religions, the
greatest millenarian rapture of human history.

That virtuous idealism justified any monstrosity. The Bolsheviks admired
the cleansing purges of Robespierre's Reign of Terror: ``A revolution
without firing squads is meaningless,'' Lenin said. The Bolsheviks
created the first professional revolutionaries, the first total police
state, the first modern mass-mobilization on behalf of class war against
counterrevolution. Bolshevism was a mind-set, an idiosyncratic culture
with an intolerant paranoid wordview obsessed with abstruse Marxist
ideology. Their zeal justified the mass killings of all enemies, real
and potential, not just by Lenin or Stalin but also Mao, Pol Pot in
Cambodia, Mengistu Haile Mariam in Ethiopia. It also gave birth to slave
labor camps, economic catastrophe and untold psychological damage.
(These events are now so long ago that the horrors have been blurred and
history forgotten; a glamorous glow of power and idealism lingers to
intoxicate young voters disenchanted with the bland dithering of liberal
capitalism.)

And then there is Russia, the successor to the Soviet Union. President
Vladimir Putin's power is enforced by his fellow former K.G.B. officers,
the heirs of Lenin and Stalin's secret police. Mr. Putin and his regime
have adopted the Leninist tactics of ``konspiratsia'' and
``dezinformatsiya*,''* which have turned out to be ideally suited to
today's technologies. Americans may have invented the internet, but they
saw it (decadently) as a means of making money or (naïvely) as a magical
click to freedom. The Russians, bred on Leninist cynicism, harnessed it
to undermine American democracy.

Mr. Putin mourned the fall of Soviet Union as
\href{http://www.nbcnews.com/id/7632057/ns/world_news/t/putin-soviet-collapse-genuine-tragedy/\#.WfzGyBNSxmM}{``the
greatest geopolitical catastrophe''} of the 20th century, yet he regards
Lenin as an agent of chaos between two epochs of national grandeur ---
the Romanovs before Nicholas II (Peter the Great and Alexander III are
favorites) and Soviet Union's superpower glory under Stalin.

Mr. Putin presents himself as a czar --- and like any czar, he fears
revolution above all else. That is why it is victory against Germany in
1945, not the Bolshevik Revolution of 1917 that is the founding myth of
Putinist Russia. Hence the irony that while the West has been discussing
the revolution at length, Russia is largely
\href{https://www.theatlantic.com/international/archive/2017/04/russia-putin-revolution-lenin-nicholas-1917/521571/}{pretending
it never happened}. Lenin's marble mausoleum in Red Square must echo
with his laughter because that's just the sort of serpentine political
calculation he would have appreciated.

Advertisement

\protect\hyperlink{after-bottom}{Continue reading the main story}

\hypertarget{site-index}{%
\subsection{Site Index}\label{site-index}}

\hypertarget{site-information-navigation}{%
\subsection{Site Information
Navigation}\label{site-information-navigation}}

\begin{itemize}
\tightlist
\item
  \href{https://help.nytimes.com/hc/en-us/articles/115014792127-Copyright-notice}{©~2020~The
  New York Times Company}
\end{itemize}

\begin{itemize}
\tightlist
\item
  \href{https://www.nytco.com/}{NYTCo}
\item
  \href{https://help.nytimes.com/hc/en-us/articles/115015385887-Contact-Us}{Contact
  Us}
\item
  \href{https://www.nytco.com/careers/}{Work with us}
\item
  \href{https://nytmediakit.com/}{Advertise}
\item
  \href{http://www.tbrandstudio.com/}{T Brand Studio}
\item
  \href{https://www.nytimes.com/privacy/cookie-policy\#how-do-i-manage-trackers}{Your
  Ad Choices}
\item
  \href{https://www.nytimes.com/privacy}{Privacy}
\item
  \href{https://help.nytimes.com/hc/en-us/articles/115014893428-Terms-of-service}{Terms
  of Service}
\item
  \href{https://help.nytimes.com/hc/en-us/articles/115014893968-Terms-of-sale}{Terms
  of Sale}
\item
  \href{https://spiderbites.nytimes.com}{Site Map}
\item
  \href{https://help.nytimes.com/hc/en-us}{Help}
\item
  \href{https://www.nytimes.com/subscription?campaignId=37WXW}{Subscriptions}
\end{itemize}
