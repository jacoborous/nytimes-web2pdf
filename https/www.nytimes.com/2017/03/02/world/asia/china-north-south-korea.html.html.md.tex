Sections

SEARCH

\protect\hyperlink{site-content}{Skip to
content}\protect\hyperlink{site-index}{Skip to site index}

\href{https://www.nytimes.com/section/world/asia}{Asia Pacific}

\href{https://myaccount.nytimes.com/auth/login?response_type=cookie\&client_id=vi}{}

\href{https://www.nytimes.com/section/todayspaper}{Today's Paper}

\href{/section/world/asia}{Asia Pacific}\textbar{}North and South Korea
Give China a Double Headache

\url{https://nyti.ms/2mwWQaO}

\begin{itemize}
\item
\item
\item
\item
\item
\end{itemize}

Advertisement

\protect\hyperlink{after-top}{Continue reading the main story}

Supported by

\protect\hyperlink{after-sponsor}{Continue reading the main story}

\hypertarget{north-and-south-korea-give-china-a-double-headache}{%
\section{North and South Korea Give China a Double
Headache}\label{north-and-south-korea-give-china-a-double-headache}}

\includegraphics{https://static01.nyt.com/images/2017/03/03/world/03china-1/03china-1-articleInline.jpg?quality=75\&auto=webp\&disable=upscale}

By \href{http://www.nytimes.com/by/jane-perlez}{Jane Perlez} and
\href{http://www.nytimes.com/by/choe-sang-hun}{Choe Sang-Hun}

\begin{itemize}
\item
  March 2, 2017
\item
  \begin{itemize}
  \item
  \item
  \item
  \item
  \item
  \end{itemize}
\end{itemize}

\href{http://cn.nytimes.com/asia-pacific/20170303/china-north-south-korea/}{阅读简体中文版}

BEIJING --- The Chinese government is ratcheting up pressure on South
Korea over its plans to deploy an American missile defense system, with
the state-controlled news media urging the public to boycott South
Korean retail products and threatening diplomatic and even military
repercussions.

China's latest pronouncements follow months of not-so-subtle punitive
measures that have already taken a toll on the South Korean economy,
including an unofficial ban on Korean television shows and pop stars.
The campaign risks a backlash in South Korea even as Beijing's relations
with North Korea have also grown strained --- a sign of how recent
advances in the North's nuclear program have put China in a bind and are
upsetting the regional security balance.

On Thursday, South Korea and the United States began talks in Seoul to
finalize details of the deployment of the so-called Terminal
High-Altitude Area Defense System, or Thaad, according to the South's
Foreign Ministry. Both countries say the system's purpose is to defend
the South against North Korea's
\href{https://www.nytimes.com/2017/02/13/world/asia/north-korea-missile-launch-success.html}{growing
missile and nuclear threat}, but China has objected strongly to the
system, which it sees as an American attempt to encircle it.

No date has been set for the system's deployment, but the Pentagon said
on Wednesday that Defense Secretary Jim Mattis wanted it in place ``as
soon as feasible.'' Military experts said the United States could use
C-17 transport aircraft to quickly move the system's truck-mounted
launchers, interceptors, radar, fire control units and support equipment
to South Korea.

China responded with anger when
\href{https://www.nytimes.com/2016/07/08/world/asia/south-korea-and-us-agree-to-deploy-missile-defense-system.html}{South
Korea agreed in July} to accept the Thaad system, and it has made its
displeasure known as plans have moved toward the final stages in recent
days.

An outspoken Chinese general, Luo Yuan, now retired, recommended a tough
series of responses in an article on Thursday, going so far as to
suggest a military strike against the missile system. ``We could conduct
a surgical hard-kill operation that would destroy the target, paralyzing
it and making it unable to hit back,'' General Luo wrote in the Global
Times, a state-run newspaper that often features strident, nationalist
views.

\includegraphics{https://static01.nyt.com/images/2017/03/03/world/03china-4/03china-4-articleInline.jpg?quality=75\&auto=webp\&disable=upscale}

``Since the United States, Japan and South Korea choose not to respect
China's major security concerns, China does not need to be a gentleman
on everything,'' the general wrote. ``We must not undermine our own
security interests while respecting the security interests of others.''

People's Daily, the Communist Party newspaper that is often considered
the official voice of the leadership, said in its international edition
this week that China should consider a ``de facto'' severance of
diplomatic ties with South Korea.

It said in a commentary that China should take ``political and military
measures'' against South Korea and that it should consider coordinating
with Russia in dealing with what it called the ``U.S.-Japan-South Korea
antimissile network.'' The paper was referring in part to statements by
Japan that it might consider using Thaad as a defense against North
Korea.

China has said that the Thaad system would threaten its nuclear
deterrent capacity. It said the system's powerful radar would make it
much easier for the United States to detect Chinese missiles and would
give the American military much more time to intercept them.

Chinese state news outlets have also suggested a consumer boycott of
South Korean products. Much of China's anger has been borne by Lotte, a
South Korean conglomerate that provided the government with land for the
Thaad deployment in a deal that was finalized this week. Lotte has
stores and shopping malls across China, and modest groups of mostly
older Chinese held protests at the company's outlets in several cities
on Thursday.

On Wednesday, the Lotte website serving Chinese shoppers was hacked, the
company said. On Thursday, another hacking attack shut down its
duty-free shop's website for several hours. Lotte also said that some
construction had been stopped by the Chinese authorities on the grounds
that it had failed a fire inspection.

Image

An undated photograph of a Thaad test.Credit...U.S. Department of
Defense, via Reuters

In recent months, popular South Korean stars have been denied visas to
perform in China, and South Korean TV shows have been blocked from
Chinese video streaming websites. Many in South Korea say they believe
those actions are in retaliation for the Thaad issue, though China has
denied any link.

One of the musicians denied a visa was
\href{https://www.nytimes.com/2017/01/23/world/asia/sumi-jo-soprano-maria-callas.html}{Sumi
Jo,} a coloratura soprano who has toured China almost every year for the
past decade. Her brother, Jay Jo, said that she had been unable this
year to get the government-approved invitation letter required for an
entry visa.

``As soon as the opportunities reopen, she will resume her concerts in
China,'' Mr. Jo said. ``But right now, we have no idea when that will
happen.''

Trade experts said Beijing might be reluctant to take more extreme
economic measures. China is South Korea's largest trading partner by
far, but South Korea is also China's fourth-largest, and Beijing would
probably be reluctant to damage those ties during the current economic
slowdown.

South Korean politicians have said that Washington wants the Thaad
system deployed by mid-May, when many expect presidential elections to
be held in the South. President Park Geun-hye was impeached by South
Korea's legislature in December over a corruption scandal, and she
awaits a ruling by the country's Constitutional Court on whether she
will be permanently removed from office. The court's decision is
expected in the coming weeks, and if it rules against her, a new
president will be elected 60 days later.

South Korea's progressive opposition is seen as having a strong chance
of winning the presidency should that election be held. Opposition
politicians have expressed skepticism about the Thaad system, and some
have charged that the United States wants to rush the deployment to
ensure that it is completed before a new president takes office.

Image

President Xi Jinping of China with President Park Geun-hye of South
Korea in Seoul in 2014. Relations between the two countries have soured
recently.Credit...Pool photo by Kim Hong-Ji

Members of the largest opposition party, the Democratic Party, have
visited China twice since August. In January, in an unusual development,
a delegation from the party met with the Chinese foreign minister, Wang
Yi.

China had hoped it could persuade the South's next president to refuse
to agree to Thaad, said Cheng Xiaohe, an associate professor of
international relations at Renmin University in Beijing. ``Now China is
afraid Thaad will be deployed before the new president of South Korea is
in office,'' he said.

Even as China's fury toward the South is on full display, it is also at
odds with the North. A North Korean diplomat, Ri Kil-song, arrived in
Beijing on Tuesday for five days of talks, an apparent effort by
Pyongyang to reach out to China, its economic and political benefactor.

Mr. Ri and Mr. Wang, the Chinese foreign minister, made soothing public
statements on Wednesday about the ``traditional friendship'' between
their two countries. Behind the scenes, though, things are unlikely to
have been so smooth.

Last month,
\href{https://www.nytimes.com/2017/02/18/world/asia/north-korea-china-coal-imports-suspended.html}{China
suspended its imports of North Korean coal} for the rest of the year, a
surprise move that appeared to be a response to
\href{https://www.nytimes.com/2017/03/01/world/asia/malaysia-kim-jong-nam-embassy-immunity.html}{the
brazen killing in Malaysia of Kim Jong-nam}, the estranged half brother
of the North Korean leader, Kim Jong-un. South Korea has accused the
North of carrying out the attack.

The killing may have been taken as an affront by Beijing because the
victim had lived in Macau, a Chinese special administrative region. Kim
Jong-nam had expressed admiration for China's market economy, and some
analysts have speculated that China saw him as a potential replacement
for his erratic half brother.

``One thing after another is happening,'' Mr. Cheng, the Renmin
University professor, said of China's simultaneous troubles with the
Koreas. ``Not good things --- all bad things.''

Advertisement

\protect\hyperlink{after-bottom}{Continue reading the main story}

\hypertarget{site-index}{%
\subsection{Site Index}\label{site-index}}

\hypertarget{site-information-navigation}{%
\subsection{Site Information
Navigation}\label{site-information-navigation}}

\begin{itemize}
\tightlist
\item
  \href{https://help.nytimes.com/hc/en-us/articles/115014792127-Copyright-notice}{©~2020~The
  New York Times Company}
\end{itemize}

\begin{itemize}
\tightlist
\item
  \href{https://www.nytco.com/}{NYTCo}
\item
  \href{https://help.nytimes.com/hc/en-us/articles/115015385887-Contact-Us}{Contact
  Us}
\item
  \href{https://www.nytco.com/careers/}{Work with us}
\item
  \href{https://nytmediakit.com/}{Advertise}
\item
  \href{http://www.tbrandstudio.com/}{T Brand Studio}
\item
  \href{https://www.nytimes.com/privacy/cookie-policy\#how-do-i-manage-trackers}{Your
  Ad Choices}
\item
  \href{https://www.nytimes.com/privacy}{Privacy}
\item
  \href{https://help.nytimes.com/hc/en-us/articles/115014893428-Terms-of-service}{Terms
  of Service}
\item
  \href{https://help.nytimes.com/hc/en-us/articles/115014893968-Terms-of-sale}{Terms
  of Sale}
\item
  \href{https://spiderbites.nytimes.com}{Site Map}
\item
  \href{https://help.nytimes.com/hc/en-us}{Help}
\item
  \href{https://www.nytimes.com/subscription?campaignId=37WXW}{Subscriptions}
\end{itemize}
