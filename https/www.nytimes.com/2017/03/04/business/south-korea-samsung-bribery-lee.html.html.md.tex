Sections

SEARCH

\protect\hyperlink{site-content}{Skip to
content}\protect\hyperlink{site-index}{Skip to site index}

\href{https://www.nytimes.com/section/business}{Business}

\href{https://myaccount.nytimes.com/auth/login?response_type=cookie\&client_id=vi}{}

\href{https://www.nytimes.com/section/todayspaper}{Today's Paper}

\href{/section/business}{Business}\textbar{}Samsung Bribery Scandal
Threatens South Korea Success Story

\url{https://nyti.ms/2lHfurJ}

\begin{itemize}
\item
\item
\item
\item
\item
\end{itemize}

Advertisement

\protect\hyperlink{after-top}{Continue reading the main story}

Supported by

\protect\hyperlink{after-sponsor}{Continue reading the main story}

\hypertarget{samsung-bribery-scandal-threatens-south-korea-success-story}{%
\section{Samsung Bribery Scandal Threatens South Korea Success
Story}\label{samsung-bribery-scandal-threatens-south-korea-success-story}}

\includegraphics{https://static01.nyt.com/images/2017/03/05/business/05Samsung1/05Samsung1-articleInline.jpg?quality=75\&auto=webp\&disable=upscale}

By \href{http://www.nytimes.com/by/choe-sang-hun}{Choe Sang-Hun} and
\href{https://www.nytimes.com/by/paul-mozur}{Paul Mozur}

\begin{itemize}
\item
  March 4, 2017
\item
  \begin{itemize}
  \item
  \item
  \item
  \item
  \item
  \end{itemize}
\end{itemize}

\href{https://www.nytimes.com/es/2017/03/09/el-escandalo-de-samsung-amenaza-la-narrativa-de-exito-de-corea-del-sur/}{Leer
en español}

SEOUL, South Korea --- Jay Y. Lee, heir to one of the world's biggest
corporate empires, followed in the footsteps of his prominent father. He
took charge of key businesses. He hobnobbed with his country's
president. He brought in new ideas.

Then,
\href{http://www.nytimes.com/2008/04/17/business/worldbusiness/17iht-samsung.4.12107507.html}{like
his father}, he was
\href{https://www.nytimes.com/2017/02/28/world/asia/lee-jae-yong-samsung.html}{charged}
with breaking the law.

Mr. Lee, the de facto leader of South Korea's Samsung Group, was
indicted on a charge of bribery this past week, accused of taking part
in a political scandal that has rocked his home country. The image of a
pillar of industry,
\href{https://www.yahoo.com/news/arrested-samsung-heir-appears-handcuffed-questioning-083230348.html}{in
handcuffs}, escorted by the police from jail to meet with prosecutors
sent a message that shocked even a jaded public: South Korea's postwar
economic order is under threat.

Samsung has experienced this before. Since the 1960s, when the company
was caught smuggling artificial sweetener, its leaders have been in and
out of trouble with minor consequences.
\href{http://www.nytimes.com/1996/08/27/world/death-sentence-for-ex-president-chun-a-landmark-for-korea.html}{Twice},
Mr. Lee's father was
\href{http://www.nytimes.com/2009/12/30/business/global/30samsung.html}{saved
from prison} by a South Korean president worried that anything that hurt
Samsung could also hurt an economic machine that lifted millions of
people from the ashes of war.

The thought of Mr. Lee in jail has raised a tantalizing prospect for
many South Koreans: This time could be different.

South Korea's political turmoil could usher in a new group of leaders
less inclined to treat its business titans with kid gloves. The public
is
\href{https://www.nytimes.com/2016/07/05/business/dealbook/south-korea-targets-executives-pressed-by-an-angry-public.html}{increasingly
fed up} with white-collar crime. Further, South Korea's unique blend of
business, politics and top-down hierarchical management is looking
increasingly untenable in a modern age of innovation, public
dissatisfaction with the old order and cutthroat competition from China
and the rest of the world.

Mr. Lee and key aides are accused of using bribes to cement the family's
control of the Samsung empire --- an accusation that, if proven, would
add to growing public perception that the country's business elite are
in it only for themselves. The ensuing scandal threatens to topple South
Korea's government and lead to tougher moves against big business here.

Prosecutors also indicted a group of executives tasked with cementing
Mr. Lee's grip over the sprawling empire --- a group that critics say
epitomizes where the country's business culture went wrong.

South Koreans are now coming to grips with a tale of corporate power and
family intrigue. It includes a mysteriously ill patriarch, a
\href{https://www.nytimes.com/2016/11/01/world/asia/south-korea-park-geun-hye-choi-soon-sil.html}{Rasputin-like}
confidante of the country's president, a shadowy Samsung office that has
disappeared and reappeared before, and an alleged bribe in the form of a
horse.

But it also includes the growing realization that, to compete, Samsung
and South Korea as a whole will have to take a hard look at the way they
do business.

``We should not miss this opportunity to cut the corrupt ties between
politics and businesses,'' said
\href{https://www.nytimes.com/2016/12/09/world/asia/south-korea-who-could-replace-park.html}{Moon
Jae-in}, a lawmaker and opposition leader who is the favorite to become
South Korea's next president. ``Only when Samsung repents its collusion
with politics and its anti-market activities, like seeking political
favors, can it become stronger.''

\hypertarget{rise-of-an-empire}{%
\subsection{Rise of an Empire}\label{rise-of-an-empire}}

Like its home country, Samsung is a giant in transition. The
conglomerate is one of the world's top sellers of televisions and
smartphones, a key supplier for the innards of Apple's iPhones, and the
maker of products as varied as cargo ships and credit cards. Its many
companies posted estimated combined sales of \$262 billion. By itself,
it accounts for one-fifth of its home country's exports.

All of that is under pressure. The infamous
\href{https://www.nytimes.com/2016/10/12/business/international/samsung-galaxy-note7-terminated.html}{failure}
of its
\href{https://www.nytimes.com/2017/01/23/business/samsung-galaxy-note7-fires.html}{fire-prone
Galaxy Note 7 smartphone} tarnished its name. Chinese
\href{https://www.nytimes.com/2016/10/19/business/samsung-galaxy-note7-china-test.html}{rivals}
are making cheaper --- and increasingly sophisticated --- phones,
televisions and appliances. Its shipbuilding business is
\href{http://english.yonhapnews.co.kr/news/2017/01/12/0200000000AEN20170112002500320.html}{shedding
jobs}. China's government is investing lavishly to build out rival
makers of microchips and memory. The scandal will delay long-term
efforts to address these problems.

Mr. Lee still may emerge unscathed. He argues that he didn't commit
bribery and that, instead, he is the victim of extortion. Samsung said
in a statement that it neither paid bribes nor sought favors, and said
the truth would come to light in court.

But Mr. Lee's troubles --- he was arrested, unprecedented for the leader
of South Korea's largest company, before he was indicted --- were a jolt
to those who assumed Samsung could once again turn to its political ties
for help.

Samsung, which means ``three stars'' in Korean, has long prospered from
politics. Founded by a grandfather of Mr. Lee as a small fish and
produce trader in 1938, it branched out into new businesses after the
Korean War, including textiles, sugar and alcohol --- and later, in
1969, electronics.

The expansion was, in many respects, funded by the government. Its
military dictator,
\href{http://www.nytimes.com/1995/11/24/world/ruthless-ex-dictator-getting-credit-for-south-korea-s-rise.html}{Park
Chung-hee}, wanted to transform South Korea into a country that made
what it needed and exported the rest. At the same time, Mr. Park and his
successors faced pressure from the country's struggling populace to get
tough on businesses that were seen as profiting from the government's
efforts to kick-start the economy.

The two sides struck a compromise. Mr. Park allowed Mr. Lee's
grandfather and other business leaders to keep their wealth. In return,
he urged them to invest in the country's economic development and to
back his push to make South Korea an export powerhouse. To help, he
plied them with cheap bank loans, beneficial ``buy Korea'' policies and
other inducements.

This business-government partnership helped wealthy business leaders and
their families squeeze out smaller businesses and international
competitors. Those groups came to be known as
\href{https://www.nytimes.com/2017/02/17/business/south-korea-chaebol-samsung.html}{chaebol},
or ``rich clan'' in Korean, and they dominate South Korea's economic
life to this day. The nation's top 10 chaebol generate combined annual
revenue equivalent to 80 percent of South Korea's total economic output,
according to analyst estimates.

``To South Koreans, chaebol have two faces,'' said Kim Sang-jo, an
economist at Hansung University in Seoul and an authority on Samsung.
``On one hand, chaebol symbolize corrupt ties between business and
politics, so people call for reforming chaebol. On the other hand, the
economy is so heavily dependent on chaebol that people fear that shaking
them too much might make them collapse.''

``South Korean people,'' he added, ``have never had a chance to learn
how a modern globalized company should be run.''

Jay Y. Lee's father, Lee Kun-hee, became Samsung's chairman after his
father, Lee Byung-chull,
\href{http://www.nytimes.com/1987/11/20/obituaries/lee-byung-chull-77-industrialist-of-korea.html}{died
in 1987}. He faced an immediate challenge: South Korea's economy had
become one of the most successful in Asia, but abroad its products were
seen as cheap and unreliable.

In a move that is now part of Samsung's internal lore, Lee Kun-hee told
his (mostly male) executives to ``change everything except your wife and
kids'' and worked to improve Samsung's quality. He brought in foreign
experts. When one batch of phones turned out to be defective, in 1995,
he gathered thousands of them into a pile at one of Samsung's factories
and set them on fire. The effort worked. Today Samsung is one of a very
few electronics makers to command premium prices for its high-end
televisions, smartphones and kitchen appliances.

But Samsung also found itself repeatedly entangled in some of South
Korea's biggest corruption scandals. Lee Kun-hee was twice convicted,
first of bribery and then of tax evasion. Each time, he received a
suspended prison term and was later pardoned by the president.

In 2014, Lee Kun-hee
\href{https://www.nytimes.com/2014/05/12/business/international/samsungs-chairman-has-surgery-after-heart-attack.html}{had
a heart attack} and disappeared from public view. Samsung has said he is
incapacitated, but it hasn't released details. Still, thanks to a
powerful group of executives and advisers, the Lee family's grip on
Samsung remained unchallenged.

\includegraphics{https://static01.nyt.com/images/2017/03/05/business/05SAMSUNG2/05SAMSUNG2-articleInline.jpg?quality=75\&auto=webp\&disable=upscale}

\hypertarget{a-new-generation}{%
\subsection{A New Generation}\label{a-new-generation}}

When Lee Kun-hee's son, Jay Y. Lee, now 48, took effective reins of the
family empire, he didn't do it alone.

The Lee family's ownership in the various Samsung companies isn't clear
because many are not publicly traded, but it is widely believed to hold
only small minority stakes. Instead, it controls the companies through
loyal executives, as well as through interlocking contracts and
shareholdings between the companies and other shadowy links.

At Samsung, some of the most loyal executives staffed the corporate
strategy office. The office linked Samsung's chairman with a host of
professional executives who run dozens of individual Samsung
subsidiaries, which together now employ a half-million people globally.
But the office had a less publicized role: It worked to enable the Lee
family to pull off the dynastic transfer of control over Samsung's
management.

Mr. Lee, polite and casual where his father could be remote and distant,
by title a Samsung vice chairman, had the appearance of a with-it
executive who wanted to bring Samsung into the modern era. Though
Samsung's electronics arm made cutting-edge hardware, it was largely
missing out on the boom in mobile apps and online services. He also
believed that Samsung's strict corporate culture was holding back
innovation.

In a few ways, under Mr. Lee, Samsung began to loosen up. He began a
campaign to discourage managers from using harsh language with
underlings, a common occurrence in South Korean offices. He also pledged
to cut meetings and office hours, encouraged workers to challenge their
bosses, and barred employees from using hierarchical titles in
addressing one another --- a norm in the South Korean corporate world.
The initiative, heavily touted by the company, was called Start-Up
Samsung.

But critics still saw a top-down approach. For example, the company
announced Start-Up Samsung with a coordinated pledge from top
executives.

Some current and former employees, who spoke on condition of anonymity
for fear of losing their jobs, said pressure from the top has only grown
worse under Mr. Lee. Samsung's phone business --- which was losing
market share
\href{https://www.nytimes.com/2015/04/02/technology/personaltech/with-galaxy-s6-and-s6-edge-samsung-tries-to-regain-its-footing.html}{in
China and elsewhere} to cheap and increasingly sophisticated Chinese
products made by Huawei and OnePlus --- had made gains against Apple at
the top end of the smartphone market. Hoping to capitalize, Samsung
executives rushed to get its most powerful phone yet, the Galaxy Note 7,
on the market before Apple introduced its new iPhone 7.

The result was disaster. First, a few Note 7 phones
\href{https://www.nytimes.com/2016/09/10/technology/samsung-galaxy-consumer-product-safety.html}{caught
fire}. Then, after
\href{https://www.nytimes.com/2016/09/03/business/samsung-galaxy-note-battery.html}{an
embarrassing and costly recall}, some
\href{https://www.nytimes.com/2016/10/11/business/samsung-galaxy-note-fires.html}{new
versions caught on fire} too. In an unusual move in the gadget business,
Samsung pulled the phone from the market and canceled it. Kim Sang-jo,
the Hansung University economist, and other outside experts blamed Mr.
Lee and the corporate strategy office for the Note 7 debacle, saying
they pushed big goals without listening to lower-level managers.

At the same time, the corporate strategy office was coming under fire
for another move, a deal that would strengthen the family's hold on
Samsung --- but one that would also engulf it in scandal.

\hypertarget{turmoil-at-the-top}{%
\subsection{Turmoil at the Top}\label{turmoil-at-the-top}}

Under that office, Samsung pushed a merger of two of its units, Cheil
Industries and Samsung C\&T. Many outside shareholders
\href{https://www.nytimes.com/2015/06/05/business/dealbook/an-activist-investor-takes-aim-at-bid-for-samsung.html}{opposed
the move}, which would cement Mr. Lee's control over the whole
conglomerate.

The trick was getting government approval --- and to win it, prosecutors
say, Mr. Lee and the corporate strategy office broke the law.

According to prosecutors, Mr. Lee met with South Korea's president, Park
Geun-hye, three times in an effort to cement the deal and smooth Mr.
Lee's rise to power. In return, they say, Ms. Park, who has been
impeached, asked Mr. Lee to support two foundations controlled by her
secret confidante, a woman named
\href{https://www.nytimes.com/2016/10/28/world/asia/south-korea-choi-soon-sil.html}{Choi
Soon-sil}. According to prosecutors, Mr. Lee and members of Samsung's
corporate strategy office gave foundations and businesses linked to Ms.
Choi \$38 million. Samsung's contributions to Ms. Choi included an
\href{https://www.washingtonpost.com/world/samsung-scion-to-be-indicted-on-bribery-charges/2017/02/28/b5988fca-863c-4ca7-8b7d-9e105549d9a9_story.html?utm_term=.7eb096d19257}{\$900,000
horse} for her
\href{https://www.nytimes.com/2017/01/02/world/asia/south-korea-scandal-choi-soon-sil-daughter.html}{equestrian
daughter}.

Image

Lee Kun-hee, the former Samsung Group chairman and the father of Jay Y.
Lee, the company's de facto chairman, leaves after his trial for tax
evasion at a Seoul court in 2008. He was handed a suspended three-year
jail term.Credit...Jo Yong-hak/Reuters

The country's national pension fund, a major shareholder of the two
Samsung companies,
\href{https://www.nytimes.com/2016/12/31/world/asia/south-korea-samsung-merger-moon-hyung-pyo.html}{approved
the deal}. According to prosecutors, the completed merger increased the
stock value of the Lee family by at least \$758 million. The country's
national pension fund lost at least \$123 million on the deal.

In a statement, Samsung said shareholders of both companies approved the
deal before any donations were made. But it also said it introduced
measures to manage donations, including greater public disclosure and
more reviews by executives and directors.

Ms. Choi's relationship with Ms. Park last year ignited a nationwide
scandal. Ms. Choi, who has no formal government title, was reported to
have been involved with writing Ms. Park's speeches, choosing top
government officials and even selecting her outfits. As the scandal
spread, the alleged bribes came to the attention of prosecutors.

For many South Koreans, the scandal resonates. Ms. Park, who could soon
be
\href{https://www.nytimes.com/2016/12/22/world/asia/south-korea-president-park-impeachment.html}{removed
from office}, is the daughter of the dictator who gave Mr. Lee's family
the support to broaden Samsung into the giant it is today. It also
underscores the perception that today's chaebol leaders are hurting more
than they are helping the country.

Others have said that the company culture has grown more strict since
Mr. Lee's father was incapacitated and pressures on the company to
perform well under Mr. Lee grew.

``They forgot their fathers' and grandfathers' entrepreneurship,'' said
Mr. Moon, the opposition leader, referring to Mr. Lee and other
third-generation chaebol leaders, ``and chose to make easy money.''

Stung by the scandal, Samsung closed the corporate strategy office.
Skeptics point out that Samsung has done that before. Mr. Lee's father
disbanded the office when he ran into legal trouble, only to
re-establish it under its current name once those troubles blew over.
Its functions may simply migrate to another part of the empire, they
say.

``The mission right now is to save Jay Y. Lee,'' said Chang Sea-jin, a
professor at the National University of Singapore. ``It's like `Saving
Private Ryan.'''

Samsung said in a statement that its companies would be managed
independently by its chief executives and directors.

But there is reason to believe this time could be different.

Already Mr. Lee has been punished more than his father, who never spent
any time in jail. The case also implicates South Korea's president,
while approval ratings for her pro-business party have plummeted,
meaning political protection from the top is less likely. And public
anger about collusion between companies and the government has never
been greater, with protests breaking out this year.

It isn't clear whether somebody else may emerge to lead Samsung. Lee
Boo-jin, Mr. Lee's sister and the country's richest woman, runs the
Samsung affiliate Hotel Shilla, which also operates duty-free shops.
With her brother in jail, some analysts speculated that she might try to
increase her profile within the vast corporate empire. But most analysts
say Samsung's complicated hierarchy is already dominated by executives
loyal to Mr. Lee.

While analysts warn it isn't likely anything will immediately change,
Samsung is undoubtedly facing a new level of pressure and scrutiny.

``The thing about Samsung is, it's a giant wound-up ball of yarn of
cross-holdings,'' said Geoffrey Cain, the author of a coming book on
Samsung. ``The setup is so complicated that sometimes I wonder if a
group of smart people could find a way to tug at the right connection
and find a way to loosen it up, to unravel it a bit, just to see if they
could pull it away from the company.''

He added, ``That's never happened in Korea before.''

Advertisement

\protect\hyperlink{after-bottom}{Continue reading the main story}

\hypertarget{site-index}{%
\subsection{Site Index}\label{site-index}}

\hypertarget{site-information-navigation}{%
\subsection{Site Information
Navigation}\label{site-information-navigation}}

\begin{itemize}
\tightlist
\item
  \href{https://help.nytimes.com/hc/en-us/articles/115014792127-Copyright-notice}{©~2020~The
  New York Times Company}
\end{itemize}

\begin{itemize}
\tightlist
\item
  \href{https://www.nytco.com/}{NYTCo}
\item
  \href{https://help.nytimes.com/hc/en-us/articles/115015385887-Contact-Us}{Contact
  Us}
\item
  \href{https://www.nytco.com/careers/}{Work with us}
\item
  \href{https://nytmediakit.com/}{Advertise}
\item
  \href{http://www.tbrandstudio.com/}{T Brand Studio}
\item
  \href{https://www.nytimes.com/privacy/cookie-policy\#how-do-i-manage-trackers}{Your
  Ad Choices}
\item
  \href{https://www.nytimes.com/privacy}{Privacy}
\item
  \href{https://help.nytimes.com/hc/en-us/articles/115014893428-Terms-of-service}{Terms
  of Service}
\item
  \href{https://help.nytimes.com/hc/en-us/articles/115014893968-Terms-of-sale}{Terms
  of Sale}
\item
  \href{https://spiderbites.nytimes.com}{Site Map}
\item
  \href{https://help.nytimes.com/hc/en-us}{Help}
\item
  \href{https://www.nytimes.com/subscription?campaignId=37WXW}{Subscriptions}
\end{itemize}
