Sections

SEARCH

\protect\hyperlink{site-content}{Skip to
content}\protect\hyperlink{site-index}{Skip to site index}

\href{https://www.nytimes.com/section/world/asia}{Asia Pacific}

\href{https://myaccount.nytimes.com/auth/login?response_type=cookie\&client_id=vi}{}

\href{https://www.nytimes.com/section/todayspaper}{Today's Paper}

\href{/section/world/asia}{Asia Pacific}\textbar{}Malaysian Inquiry in
Kim Jong-nam Killing Hampered as Suspects Hide in Embassy

\url{https://nyti.ms/2lyGHxe}

\begin{itemize}
\item
\item
\item
\item
\item
\end{itemize}

Advertisement

\protect\hyperlink{after-top}{Continue reading the main story}

Supported by

\protect\hyperlink{after-sponsor}{Continue reading the main story}

\hypertarget{malaysian-inquiry-in-kim-jong-nam-killing-hampered-as-suspects-hide-in-embassy}{%
\section{Malaysian Inquiry in Kim Jong-nam Killing Hampered as Suspects
Hide in
Embassy}\label{malaysian-inquiry-in-kim-jong-nam-killing-hampered-as-suspects-hide-in-embassy}}

\includegraphics{https://static01.nyt.com/images/2017/03/02/world/02kim-1/02kim-1-articleLarge.jpg?quality=75\&auto=webp\&disable=upscale}

By \href{https://www.nytimes.com/by/richard-c-paddock}{Richard C.
Paddock}

\begin{itemize}
\item
  March 1, 2017
\item
  \begin{itemize}
  \item
  \item
  \item
  \item
  \item
  \end{itemize}
\end{itemize}

KUALA LUMPUR, Malaysia --- For years, North Korea has enjoyed the
freedom for its citizens to visit, work and live in Malaysia, a rare
privilege for a nation considered an outlaw by most of the world.

Now that freedom is in danger, with the North Korean Embassy in a suburb
of Kuala Lumpur, the Malaysian capital, at the center of a murder
investigation that is upending the cozy diplomatic relationship between
the two countries.

Two North Korean men accused of participating in
\href{https://www.nytimes.com/2017/02/22/world/asia/kim-jong-nam-assassination-korea-malaysia.html}{the
Feb. 13 assassination of Kim Jong-nam}, the half brother of North
Korea's leader, Kim Jong-un, have taken refuge in the embassy and are
refusing to cooperate with the police. Their stance is presenting the
Malaysian authorities with a daunting challenge as they try to crack a
case with major international ramifications.

One of the two men, Hyon Kwang-song, is a high-ranking embassy employee
who claims diplomatic immunity and, as a result, is untouchable by the
police. The other, Kim Uk-il, an employee of the state-owned airline,
Air Koryo, is safe from arrest as long as he remains on the embassy
grounds.

South Korean intelligence officials
\href{https://www.nytimes.com/2017/02/27/world/asia/north-korea-kim-jong-nam-state-security.html}{said
on Monday} that Mr. Hyon worked for North Korea's Ministry of State
Security, the country's secret police.

``The embassy is considered the sovereign territory of the country
concerned, so the authorities cannot enter without permission,'' said
Sivananthan Nithyanantham, a Malaysian lawyer who has served as counsel
at the International Criminal Court in The Hague. ``To do so would be
akin to entering foreign soil without consent and would be a serious
breach of diplomatic protocol.''

Image

Kim Uk-il, an employee of North Korean state airline, Air Koryo, is safe
from arrest as long as he remains inside the embassy
grounds.Credit...Royal Malaysian Police, via European Pressphoto Agency

The Vienna Convention of 1961 gives diplomats and embassies a special
protected status intended to safeguard the conduct of international
affairs. But over the years, there have been several high-profile cases
of diplomats and citizens who have sought to use these protections to
avoid prosecution for serious, nondiplomatic crimes.

During a tense standoff with the United States, Gen. Manuel Antonio
Noriega, the former Panamanian dictator, took refuge in the de facto
Vatican embassy in Panama City in 1989 to avoid capture by United States
troops who had come to seize him. He was
\href{http://www.nytimes.com/1990/01/05/world/noriega-s-surrender-chronology-vatican-issues-ultimatum-general-takes-walk.html?pagewanted=all.}{forced
to leave} after 10 days when the Vatican declined to give him asylum.

And Dominique Strauss-Kahn, the former head of the International
Monetary Fund, unsuccessfully sought to invoke diplomatic immunity to
avoid a lawsuit alleging that he sexually assaulted a hotel maid in
Manhattan in 2011. But his claim of immunity
\href{http://www.nytimes.com/2012/05/02/nyregion/strauss-kahns-claim-of-diplomatic-immunity-is-rejected.html}{was
rejected} by a New York State judge because Mr. Strauss-Kahn had already
left that post before the suit was filed.

One of the best-known diplomatic asylum seekers is Julian Assange, the
WikiLeaks founder who has been
\href{https://www.nytimes.com/2017/02/17/world/europe/julian-assange-ecuador-embassy.html}{holed
up at the Ecuadorean Embassy} in London for five years to avoid
extradition to Sweden on accusations of rape. Although he is not a
diplomat, Ecuador has granted him asylum and allowed him to stay at the
embassy.

Like Mr. Assange, Kim Uk-il, the North Korean airline employee, is
vulnerable to arrest should he ever leave the embassy grounds --- or, in
his case, if relations sour to the point that Malaysia and North Korea
cut diplomatic ties and the embassy closes.

Intelligence services, including the C.I.A., routinely assign agents to
work in foreign embassies in the guise of diplomats, largely because of
the protections of diplomatic immunity. Governments sometimes expel
these agents when espionage is uncovered. But it is rare for someone
working under diplomatic cover to be linked to a murder and for a
government to seek an arrest.

The police say
\href{https://www.nytimes.com/2017/02/15/world/asia/kim-jong-nam-assassination-north-korea.html}{Kim
Jong-nam} was assassinated by two women who rubbed VX nerve agent on his
face. Siti Aisyah, 25, of Indonesia, and Doan Thi Huong, 28, of Vietnam,
\href{https://www.nytimes.com/2017/02/28/world/asia/north-korea-kim-jong-nam-death.html}{were
charged on Wednesday with his murder}. They have said they thought they
were participating in a harmless prank.

\includegraphics{https://static01.nyt.com/images/2017/03/02/world/02kim-2/02kim-2-articleInline.jpg?quality=75\&auto=webp\&disable=upscale}

South Korea has blamed the North Korean government for Mr. Kim's
assassination, and the Malaysian police have identified eight North
Korean men, including Mr. Hyon, an embassy second secretary, and Kim
Uk-il as participants in the plot.

North Korea said on Wednesday that the conclusion that Mr. Kim had been
killed by VX nerve agent was ``the height of absurdity'' because such a
poison is so powerful that it would have killed more than just one
person.

The poisoning in the middle of Kuala Lumpur's busy international airport
has prompted some Malaysians to call for an examination of their
country's role in helping North Korea connect with the outside world ---
and to question whether the North should be allowed to have an embassy
in Malaysia.

Dennis Ignatius, a former Malaysian ambassador to several Western
Hemisphere countries, called Malaysian officials ``naïve and gullible''
in dealing with North Korea and questioned why the rogue state had ever
been allowed to open its embassy in the first place.

He urged the government --- sometimes known by the same name as its
geographic location, Putrajaya --- to downgrade relations with North
Korea. He suggested expelling North Korea's ambassador, revoking the
visas of North Koreans working in Malaysia and closing Malaysia's
embassy in Pyongyang, the North Korean capital. Malaysia has already
recalled its ambassador there for consultations.

``The real question is why Putrajaya has allowed North Korea to turn
Malaysia into one of its most important bases of operation in the region
from which to carry out clandestine activities, circumvent U.N.
sanctions and engage in all sorts of illicit enterprises to earn hard
currency for the regime,'' he wrote in
\href{https://dennisignatius.com/2017/02/27/assassination-puts-the-spotlight-on-malaysia/}{a
blistering blog post} this week.

Under the Vienna Convention, countries can declare a foreign diplomat
``persona non grata.'' Malaysia is said to be considering that
designation for Mr. Hyon and his superior, Ambassador Kang Chol, who
issued a strongly worded statement last week accusing Malaysia of
colluding with South Korea in the Kim case.

Image

Kang Chol, North Korea's ambassador to Malaysia, at the embassy in Kuala
Lumpur last week. Mr. Kang issued a strongly worded statement accusing
Malaysia of colluding with South Korea in the case.Credit...Manan
Vatsyayana/Agence France-Presse --- Getty Images

Both Malaysia and North Korea have signed the Vienna agreement, which
allows a country to waive immunity for its own diplomats.

This happens only rarely. Malaysia waived immunity in the case of its
military attaché, Muhammad Rizalman bin Ismail, who was arrested in New
Zealand in 2014 on suspicion of sexually assaulting a young woman.

He claimed diplomatic immunity and left New Zealand to avoid
prosecution. But given the nature of the charges, Malaysia revoked his
immunity and returned him to face trial in New Zealand, where he pleaded
guilty.

In another unusual case, in 1997, President Eduard A. Shevardnadze of
Georgia revoked the immunity of
\href{http://www.nytimes.com/1997/10/09/us/georgian-diplomat-pleads-guilty-in-death-of-teen-age-girl.html}{Gueorgui
Makharadze}, a high-ranking diplomat at the embassy in Washington, who
was then tried and convicted in the drunken-driving death of a
16-year-old girl in Maryland.

In 2011, American officials argued that Raymond A. Davis, a C.I.A.
contractor who killed two Pakistanis on a crowded street in Lahore, was
entitled to diplomatic immunity, a claim rejected by the Pakistani
government. He was eventually freed and left the country after the
victims' families were promised
\href{http://www.nytimes.com/2011/03/17/world/asia/17pakistan.html}{millions
of dollars in ``blood money.''}

In Kuala Lumpur, the Malaysian government, which declined to discuss the
case, could face a protracted standoff with North Korea over the two
suspects in the embassy.

About 1,000 North Koreans live and work in Malaysia, where their
companies have rare access to global markets and the international
banking system. For their part, Malaysians can visit North Korea without
a visa, but few have reason to go. With such an imbalanced relationship,
Malaysia may have little to lose by severing ties with North Korea if it
continues to deny the police access to the suspects.

``This is certainly one of the world's most secretive and ostracized
countries, and probably for good reasons,'' said Oh Ei Sun, a former
secretary to Prime Minister Najib Razak and an adjunct senior fellow at
the S. Rajaratnam School of International Studies in Singapore. ``We
should really think twice about letting them come in freely.''

Advertisement

\protect\hyperlink{after-bottom}{Continue reading the main story}

\hypertarget{site-index}{%
\subsection{Site Index}\label{site-index}}

\hypertarget{site-information-navigation}{%
\subsection{Site Information
Navigation}\label{site-information-navigation}}

\begin{itemize}
\tightlist
\item
  \href{https://help.nytimes.com/hc/en-us/articles/115014792127-Copyright-notice}{©~2020~The
  New York Times Company}
\end{itemize}

\begin{itemize}
\tightlist
\item
  \href{https://www.nytco.com/}{NYTCo}
\item
  \href{https://help.nytimes.com/hc/en-us/articles/115015385887-Contact-Us}{Contact
  Us}
\item
  \href{https://www.nytco.com/careers/}{Work with us}
\item
  \href{https://nytmediakit.com/}{Advertise}
\item
  \href{http://www.tbrandstudio.com/}{T Brand Studio}
\item
  \href{https://www.nytimes.com/privacy/cookie-policy\#how-do-i-manage-trackers}{Your
  Ad Choices}
\item
  \href{https://www.nytimes.com/privacy}{Privacy}
\item
  \href{https://help.nytimes.com/hc/en-us/articles/115014893428-Terms-of-service}{Terms
  of Service}
\item
  \href{https://help.nytimes.com/hc/en-us/articles/115014893968-Terms-of-sale}{Terms
  of Sale}
\item
  \href{https://spiderbites.nytimes.com}{Site Map}
\item
  \href{https://help.nytimes.com/hc/en-us}{Help}
\item
  \href{https://www.nytimes.com/subscription?campaignId=37WXW}{Subscriptions}
\end{itemize}
