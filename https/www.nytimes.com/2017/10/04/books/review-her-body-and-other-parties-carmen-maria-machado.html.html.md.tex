Sections

SEARCH

\protect\hyperlink{site-content}{Skip to
content}\protect\hyperlink{site-index}{Skip to site index}

\href{https://www.nytimes.com/section/books}{Books}

\href{https://myaccount.nytimes.com/auth/login?response_type=cookie\&client_id=vi}{}

\href{https://www.nytimes.com/section/todayspaper}{Today's Paper}

\href{/section/books}{Books}\textbar{}Fairy Tales About the Fears Within

\begin{itemize}
\item
\item
\item
\item
\item
\end{itemize}

Advertisement

\protect\hyperlink{after-top}{Continue reading the main story}

Supported by

\protect\hyperlink{after-sponsor}{Continue reading the main story}

\href{/column/books-of-the-times}{Books of The Times}

\hypertarget{fairy-tales-about-the-fears-within}{%
\section{Fairy Tales About the Fears
Within}\label{fairy-tales-about-the-fears-within}}

By \href{https://www.nytimes.com/by/parul-sehgal}{Parul Sehgal}

\begin{itemize}
\item
  Oct. 4, 2017
\item
  \begin{itemize}
  \item
  \item
  \item
  \item
  \item
  \end{itemize}
\end{itemize}

\includegraphics{https://static01.nyt.com/images/2017/10/06/books/06bookmachado1/06bookmachado1-articleInline.jpg?quality=75\&auto=webp\&disable=upscale}

She has been decapitated twice, had her right arm sawed off once and
been smeared with paint too many times to count. No public monument has
faced such steady abuse as the statue of Hans Christian Andersen's
Little Mermaid perched on a large rock in a Copenhagen harbor. Among her
most faithful assailants have been feminist groups protesting her as ``a
symbol of hostility to women.'' In 2006, they attached a dildo to her
hand, in honor of International Women's Day.

There might be no better illustration of the lasting, unsettling power
of fairy tales. Despite efforts to sanitize them or give them a feminist
slant, a whiff of something disreputable lingers, something slightly
kinky. ``Children know something they can't tell,'' Djuna Barnes wrote
in ``Nightwood.'' ``They like Red Riding Hood and the wolf in bed!''

``Her Body and Other Parties,'' by Carmen Maria Machado, is a love
letter to an obstinate genre that won't be gentrified. It's a wild
thing, this book, covered in sequins and scales, blazing with the
influence of fabulists from Angela Carter to Kelly Link and Helen
Oyeyemi, and borrowing from science fiction, queer theory and horror.

Published just this week, ``Her Body and Other Parties'' was released in
the wake of its success: It's been named a finalist for the National
Book Award and for the
\href{https://www.kirkusreviews.com/prize/2017/finalists/}{Kirkus
Prize}, and its publisher, Graywolf, has already gone back for a third
printing. Not since Karen Russell's ``St. Lucy's Home for Girls Raised
by Wolves,'' in 2006, has a debut collection of short stories from a
relatively unknown author garnered such attention, or deserved it more.

The eight fables in Machado's book all depict women on the verge. A wife
struggles to keep her husband from untying the mysterious ribbon she
wears around her neck. The victim of a violent assault discovers she can
hear the thoughts of the actors in porn films. Two women make a baby
together --- or do they? The book's novella-length centerpiece,
``Especially Heinous,'' rewrites almost 300 episodes of ``Law \& Order:
Special Victims Unit,'' arguably the dominant fairy tale of our time,
with its ritualistic opening riff, the women in distress, the tidy
resolutions.

Machado is fluent in the vocabulary of fairy tales --- her stories are
full of foxes, foundlings, nooses and gowns --- but she remixes it to
her own ends. Her fiction is both matter-of-factly and gorgeously queer.
She writes about loving and living with women and men with such heat and
specificity that it feels revelatory.

Image

Carmen Maria MachadoCredit...Tom Storm

Everything returns to the body in these pieces: the ``one-beer-deep
feeling'' of holding a baby; sex so good you feel ``like a bottle
breaking against a brick wall.''

But if Machado is strong on pleasure, she's better on despair, on our
rage at our bodies --- for their ugliness and unruliness, their excess
and inadequacy and, worst of all, their temerity to abandon us
altogether.

The strongest and scariest story here, ``The Husband Stitch'' --- look
up the term if you dare --- follows a character through marriage,
childbirth and childrearing. Running parallel are the gossip and ghost
stories she hears about unlucky brides, unlucky pregnant women, women
who got in the wrong car, who trusted the wrong doctor, married the
wrong man, married the right man. ``Stories have this way of running
together like raindrops in a pond,'' the narrator thinks. ``Each is
borne from the clouds separate, but once they have come together, there
is no way to tell them apart.''

In the old myths, women were fenced in by forests, towers, spells. In
Machado's work, cautionary tales are all that's required. Fear keeps
women in line. Their own minds act in the place of moats.

Fairy tales were meant to inoculate us against dread, or so the theory
goes; to offer children controlled exposure to frightening things --- to
jealousy, to adult sexuality. Terror in doses. Even Angela Carter, who
claimed with characteristic relish that her work ``cuts like a steel
blade at the base of a man's penis,'' wrote some joyous endings. In her
telling, Red Riding Hood and the wolf make the loveliest couple.

Machado offers a more complicated solace. She doesn't contain our
terror, she stokes it and teaches us about it.

We see what her characters cannot --- that some of the scariest monsters
come from within. And learning to identify what to fear, and to fear the
right things, can be a kind of power.

``Life is too short to be afraid of nothing,'' Machado writes. ``And I
will show you.''

Advertisement

\protect\hyperlink{after-bottom}{Continue reading the main story}

\hypertarget{site-index}{%
\subsection{Site Index}\label{site-index}}

\hypertarget{site-information-navigation}{%
\subsection{Site Information
Navigation}\label{site-information-navigation}}

\begin{itemize}
\tightlist
\item
  \href{https://help.nytimes.com/hc/en-us/articles/115014792127-Copyright-notice}{©~2020~The
  New York Times Company}
\end{itemize}

\begin{itemize}
\tightlist
\item
  \href{https://www.nytco.com/}{NYTCo}
\item
  \href{https://help.nytimes.com/hc/en-us/articles/115015385887-Contact-Us}{Contact
  Us}
\item
  \href{https://www.nytco.com/careers/}{Work with us}
\item
  \href{https://nytmediakit.com/}{Advertise}
\item
  \href{http://www.tbrandstudio.com/}{T Brand Studio}
\item
  \href{https://www.nytimes.com/privacy/cookie-policy\#how-do-i-manage-trackers}{Your
  Ad Choices}
\item
  \href{https://www.nytimes.com/privacy}{Privacy}
\item
  \href{https://help.nytimes.com/hc/en-us/articles/115014893428-Terms-of-service}{Terms
  of Service}
\item
  \href{https://help.nytimes.com/hc/en-us/articles/115014893968-Terms-of-sale}{Terms
  of Sale}
\item
  \href{https://spiderbites.nytimes.com}{Site Map}
\item
  \href{https://help.nytimes.com/hc/en-us}{Help}
\item
  \href{https://www.nytimes.com/subscription?campaignId=37WXW}{Subscriptions}
\end{itemize}
