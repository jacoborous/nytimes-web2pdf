Sections

SEARCH

\protect\hyperlink{site-content}{Skip to
content}\protect\hyperlink{site-index}{Skip to site index}

new video loaded: Why the Iowa Caucuses Are So Important

transcript

Back

bars

0:00/9:11

-9:11

transcript

\hypertarget{why-the-iowa-caucuses-are-so-important}{%
\subsection{Why the Iowa Caucuses Are So
Important}\label{why-the-iowa-caucuses-are-so-important}}

\hypertarget{protests-in-the-1960s-a-mimeograph-machine-and-a-long-shot-candidate-all-contributed-to-iowas-unlikely-role-in-the-presidential-election-process}{%
\paragraph{Protests in the 1960s, a mimeograph machine and a long-shot
candidate all contributed to Iowa's unlikely role in the presidential
election
process.}\label{protests-in-the-1960s-a-mimeograph-machine-and-a-long-shot-candidate-all-contributed-to-iowas-unlikely-role-in-the-presidential-election-process}}

\begin{itemize}
\tightlist
\item
  This was Iowa caucus night back in the mid-1970s. And these are
  members of the national media covering the voting. It was so unusual
  to see national media in Iowa back then that people actually paid to
  watch them. ``The Democratic Party charged \$15 a head for people to
  watch the media watch the people.'' See, in previous years, Iowa's
  caucuses just hadn't attracted national attention. ``There are 3,000
  frozen media members in downtown Des Moines \ldots'' Just over a
  decade later, Iowa is the place to be. ``\ldots{} It's Iowa caucus
  night. Let's party.'' {[}shouting{]} The caucuses are now a key part
  of the presidential election cycle. ``Bush, 57.'' They're the first
  chance to see what kind of support candidates have among voters. So
  how did we get here, from caucuses that only Iowans seem to care about
  to the national spectacle we see today? Turns out, a lot of it was
  accidental. For most of Iowa's history, its caucuses were dominated by
  political insiders. There was little room for input from rank-and-file
  members. An historian writing in the 1940s put it like this: ``The
  larger number of party voters were deprived of a voice.'' But the old
  ways start coming to an end in 1968. The country's in turmoil, and so
  is the Democratic Party, mostly over the Vietnam War and civil rights.
  Basically, the party establishment wants to handle things one way, and
  many rank-and-file members have other ideas. All this comes to a head
  as the Democrats hold their national convention. Protesters gather
  outside. So do police. Inside, the mood is also tense. All this
  division leads the Democratic Party to rethink the nomination rules to
  include the voices of all party members in the process. This is how we
  come to the moment when Iowa becomes key to electing a president,
  basically by accident. First up, how Iowa became first to hold a
  presidential contest. It starts with new rules to give everyday
  members more of a say. So by 1972, winning Iowa now involves four
  stages. Iowans choose their top candidates, first at the precinct
  level. These are the caucuses at the heart of this story. But
  technically, there's further voting at the county, congressional
  district and state levels. The new rules make things a lot more
  inclusive, but this creates new delays. Committees need to be formed,
  and everyone needs to have up-to-date party materials. The problem is,
  the state party only has an old mimeograph machine to make copies of
  all this. It's really slow. So because of an old machine and a bunch
  of new logistics, the party decides it needs at least a month between
  each step to do it all. The national convention is set for early July,
  so you'd think that the state-level convention would happen about a
  month before, in June. Except, the party can't find a venue that's
  available to hold everyone. That little detail helps push everything
  earlier in a chain reaction. See what's going on here? The precinct
  caucuses now have to happen early in the year. The party chooses a
  date that makes Iowa's the first presidential contest. The New
  Hampshire primary has been the first kickoff contest since the 1950s,
  but Iowa Democrats aren't necessarily looking for national attention.
  They just think it'll be fun to be first. Still, attention is what
  they get. The story begins with George McGovern. ``People didn't know
  much about the Iowa caucuses. As a matter of fact, there wasn't a
  great deal of interest in them.'' He's the long-shot candidate. He's
  been at the bottom of national polls. ``He often walked the campaign
  trail alone, little known by the voters.'' Most people think this guy,
  Edmund Muskie, is going to be the big winner in Iowa. ``That challenge
  is great, but we can meet it.'' Then comes caucus night. As the people
  vote, state party officials gather at their headquarters. Richard
  Bender is one of them. ``And we had about 10 or 12 press people show
  up. These press people included one guy, Johnny Apple.'' Johnny Apple,
  a 37-year-old political correspondent for The New York Times. Iowa's
  Democrats aren't ready to publicize the results right away. They
  hadn't expected much demand. According to Bender, only Johnny Apple
  asked for them that night. ``I happen to be fascinated with such
  things, so I made it my business, beforehand, to understand it.''
  Bender sets up a phone tree to gather results from across the state.
  He adds them up himself with a calculator. And the next day, Apple's
  article helps swing the national spotlight onto the caucuses. He's got
  quite the story to tell. Muskie's won, but just barely. Not the
  runaway win people were expecting. And McGovern comes in a strong
  second. No one expected that, either. The reformed caucus rules helped
  a long-shot candidate rise to the top. And because this is happening
  so early in the election now, and because Apple's article gives the
  results national coverage, something else happens. ``That got picked
  up by some of the national news shows.'' ``The Democratic front-runner
  has been damaged in Iowa.'' ``And wow, all of a sudden, we were being
  paid attention to.'' McGovern eventually wins the Democratic
  nomination. ``I accept your nomination with a full and grateful
  heart.'' He loses the presidential election, but some haven't
  forgotten what those early caucuses did for McGovern, including
  Georgia's former governor, Jimmy Carter. Three years later \ldots{}
  ``There was a major headline on the editorial page of the Atlanta
  Constitution that said, `Jimmy Carter's running for what?'
  {[}laughter{]} And the `What' was about this big. {[}applause{]} I'm
  running for president.'' \ldots{} Carter heads to Iowa before any
  other Democratic candidate. He's got no national profile. ``He didn't
  have hordes of press following him around. It was a very lonely
  campaign.'' Washington pundits call his candidacy laughable. ``I
  remember when we couldn't find a microphone.'' ``Jimmy Who?'' becomes
  a catchphrase. Carter's own campaign film plays it up. ``Jimmy who?''
  ``I don't know who he is.'' But as long as Iowans come to know him and
  like him, Carter bets that the media will start paying attention, just
  like with McGovern four years earlier. Carter campaigns as locally as
  possible. One day, he learns that he's been invited on a local TV
  show. ``And I said, that is great. I can't believe it. I said, `What
  are we going to do?' He said, `Do you have any favorite recipes?' And
  I said, `What do you mean, recipes?' He said, `Well, this is a cooking
  show.' Well, they put a white apron on me and a chef's hat. That was
  my only access to TV when I first began to campaign in Iowa.'' His
  opponents are in Iowa, too, but they spend far less time there. Carter
  wins. ``Surprisingly top of the class after his win in a somewhat
  obscure race in Iowa against the others.'' ``You can't tell until we
  go to the other 49 states, but it's encouraging for us.'' A year later
  \ldots{} ``I, Jimmy Carter, do solemnly swear ---'' \ldots{} he
  becomes the 39th president. Now we need to head to 1980 because we
  haven't talked about the Republicans yet. Here's the state's
  Republican chairman that year. He's asked why Iowa's caucuses have
  become so important. ``I think because Jimmy Carter got his start in
  Iowa in 1976.'' The Republicans in Iowa are keen to copy the
  Democrat's success, and one candidate in particular gets inspired by
  Carter's underdog win: George H.W. Bush. He's running against Ronald
  Reagan, Bob Dole and others, and he's near the bottom of the pack.
  ``Your name isn't really a household word, but Ronald Reagan can ---''
  But Bush goes big in Iowa. He gets a surprise win. It's a far cry from
  just months before. ``I was an asterisk in those days. And my feelings
  got hurt. And now, I'm no longer an asterisk.'' Bush is now the third
  underdog to get a boost from the caucuses. The next morning on CBS, he
  distills the essence of this new Iowa effect. ``We will have forward,
  `Big Mo' on our side, as they say in athletics.'' `` `Big Mo?' ''
  ``Yeah. Mo --- momentum.'' Bush loses to Reagan, but becomes vice
  president. And the desire to capture the ``Big Mo'' from Iowa has only
  grown, thanks in large part to Iowa's embrace of being first, and the
  media storm that descends every four years. That's despite the fact
  that most candidates who win \ldots{} ``This is a job interview.''
  \ldots{} don't become president. Plus, many point out that the state's
  overwhelmingly white population doesn't reflect the country's
  diversity. ``I actually think that we can find places that represent
  that balance of urban and rural better.'' But the race to get the
  ``Big Mo'' out of Iowa persists because it's the first chance to upend
  expectations, and put political fates in the voters' hands.
\end{itemize}

\hypertarget{why-the-iowa-caucuses-are-so-important-1}{%
\section{Why the Iowa Caucuses Are So
Important}\label{why-the-iowa-caucuses-are-so-important-1}}

By David Botti and Sarah Kerr•January 28, 2020

\hypertarget{protests-in-the-1960s-a-mimeograph-machine-and-a-long-shot-candidate-all-contributed-to-iowas-unlikely-role-in-the-presidential-election-process-1}{%
\subsection{Protests in the 1960s, a mimeograph machine and a long-shot
candidate all contributed to Iowa's unlikely role in the presidential
election
process.}\label{protests-in-the-1960s-a-mimeograph-machine-and-a-long-shot-candidate-all-contributed-to-iowas-unlikely-role-in-the-presidential-election-process-1}}

\begin{itemize}
\item
\item
\item
\item
\end{itemize}

\begin{itemize}
\item
  \includegraphics{https://static01.nyt.com/images/2020/02/03/us/politics/28hwgh-iowa-1-print/28hwgh-iowa-1-square320.jpg}

  NOW PLAYING

  \hypertarget{why-the-iowa-caucuses-are-so-important-2}{%
  \subsubsection{Why the Iowa Caucuses Are So
  Important}\label{why-the-iowa-caucuses-are-so-important-2}}
\item
  \href{https://www.nytimes.com/video/us/politics/100000006639814/presidential-candidates-campaigns.html?action=click\&module=video-series-bar\&region=header\&pgtype=Article\&playlistId=video/how-we-got-here}{}

  \includegraphics{https://static01.nyt.com/images/2019/11/05/us/politics/vid-campaigns-1/vid-campaigns-1-square320-v2.jpg}

  9:27

  \hypertarget{presidential-candidates-crave-the-spotlight-200-years-ago-that-was-taboo}{%
  \subsubsection{Presidential Candidates Crave the Spotlight. 200 Years
  Ago That Was
  Taboo.}\label{presidential-candidates-crave-the-spotlight-200-years-ago-that-was-taboo}}
\item
  \href{https://www.nytimes.com/video/us/100000006542254/climate-change-lawns.html?action=click\&module=video-series-bar\&region=header\&pgtype=Article\&playlistId=video/how-we-got-here}{}

  \includegraphics{https://static01.nyt.com/images/2019/08/10/autossell/09vid-lawns-1/09vid-lawns-1-square320.jpg}

  7:11

  \hypertarget{the-great-american-lawn-how-the-dream-was-manufactured}{%
  \subsubsection{The Great American Lawn: How the Dream Was
  Manufactured}\label{the-great-american-lawn-how-the-dream-was-manufactured}}
\item
  \href{https://www.nytimes.com/video/us/100000006456208/mexico-border-wall-immigration.html?action=click\&module=video-series-bar\&region=header\&pgtype=Article\&playlistId=video/how-we-got-here}{}

  \includegraphics{https://static01.nyt.com/images/2019/04/11/autossell/vidxx-hwgh-border-3/vidxx-hwgh-border-3-square320-v3.jpg}

  7:13

  \hypertarget{when-entering-the-us-was-as-easy-as-crossing-a-street}{%
  \subsubsection{When Entering the U.S. Was as Easy as Crossing a
  Street}\label{when-entering-the-us-was-as-easy-as-crossing-a-street}}
\item
  \href{https://www.nytimes.com/video/us/politics/100000006046278/supreme-court-confirmation-hearing-nominee-senate-history.html?action=click\&module=video-series-bar\&region=header\&pgtype=Article\&playlistId=video/how-we-got-here}{}

  \includegraphics{https://static01.nyt.com/images/2018/09/07/us/politics/kav-vid-sub/kav-vid-sub-square320.jpg}

  4:45

  \hypertarget{how-supreme-court-confirmations-became-partisan-spectacles}{%
  \subsubsection{How Supreme Court Confirmations Became Partisan
  Spectacles}\label{how-supreme-court-confirmations-became-partisan-spectacles}}
\end{itemize}

Recent episodes in How We Got Here

A video series that traces the origins of today's issues.

A video series that traces the origins of today's issues.

\href{/video}{}

\href{/video/latest-video}{Latest Video}

\href{/video/hk-protest}{Hong Kong Protests}

\href{/video/2020-Elections}{2020 Elections}

\href{/video/Most-Viewed}{Most-Viewed}

\href{/video/investigations}{Visual Investigations}

\href{/video/on-the-ground}{The Dispatch}

\href{/video/diaryofasong}{Diary of a Song}

\href{/video/how-we-got-here}{How We Got Here}

\href{/video/magazine}{Magazine}

\href{/video/t-magazine}{T Magazine}

\href{/video/op-docs}{Op-Docs}

\href{/video/opinion}{Opinion}

Advertisement

\protect\hyperlink{after-bottom}{Continue reading the main story}

\hypertarget{site-index}{%
\subsection{Site Index}\label{site-index}}

\hypertarget{site-information-navigation}{%
\subsection{Site Information
Navigation}\label{site-information-navigation}}

\begin{itemize}
\tightlist
\item
  \href{https://help.nytimes.com/hc/en-us/articles/115014792127-Copyright-notice}{©~2020~The
  New York Times Company}
\end{itemize}

\begin{itemize}
\tightlist
\item
  \href{https://www.nytco.com/}{NYTCo}
\item
  \href{https://help.nytimes.com/hc/en-us/articles/115015385887-Contact-Us}{Contact
  Us}
\item
  \href{https://www.nytco.com/careers/}{Work with us}
\item
  \href{https://nytmediakit.com/}{Advertise}
\item
  \href{http://www.tbrandstudio.com/}{T Brand Studio}
\item
  \href{https://www.nytimes.com/privacy/cookie-policy\#how-do-i-manage-trackers}{Your
  Ad Choices}
\item
  \href{https://www.nytimes.com/privacy}{Privacy}
\item
  \href{https://help.nytimes.com/hc/en-us/articles/115014893428-Terms-of-service}{Terms
  of Service}
\item
  \href{https://help.nytimes.com/hc/en-us/articles/115014893968-Terms-of-sale}{Terms
  of Sale}
\item
  \href{https://spiderbites.nytimes.com}{Site Map}
\item
  \href{https://help.nytimes.com/hc/en-us}{Help}
\item
  \href{https://www.nytimes.com/subscription?campaignId=37WXW}{Subscriptions}
\end{itemize}
