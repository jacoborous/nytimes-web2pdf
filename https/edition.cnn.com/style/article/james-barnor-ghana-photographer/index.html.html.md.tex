\href{/}{}\href{/style}{}

\href{/style/specials/africa-avant-garde}{Africa Avant
Garde}\href{/style/specials/africa-avant-garde}{View All}

\href{/style/arts}{arts}

\hypertarget{from-accra-to-london-how-photographer-james-barnor-captured-decades-of-style}{%
\section{From Accra to London, how photographer James Barnor captured
decades of
style}\label{from-accra-to-london-how-photographer-james-barnor-captured-decades-of-style}}

Updated 19th June 2020

\includegraphics{https://dynaimage.cdn.cnn.com/cnn/e_blur:500,q_auto:low,w_50,c_fill,g_auto,h_50,ar_1:1/http\%3A\%2F\%2Fcdn.cnn.com\%2Fcnnnext\%2Fdam\%2Fassets\%2F200618170758-07-james-barnor.jpg}

Credit: James Barnor/courtesy October Gallery

\includegraphics{https://dynaimage.cdn.cnn.com/cnn/e_blur:500,q_auto:low,w_50,c_fill,g_auto,h_50,ar_1:1/http\%3A\%2F\%2Fcdn.cnn.com\%2Fcnnnext\%2Fdam\%2Fassets\%2F200618170758-07-james-barnor-super-tease.jpg}

From Accra to London, how photographer James Barnor captured decades of
style

Written by Emma Firth, CNN

When he was 17 James Barnor took his first picture, using a small camera
a craft teacher gifted him. His subject was a "clever and lovely" girl
who he knew from school.

"Growing up in Ghana, I was surrounded by people who wanted to have
their picture taken," Barnor said during a phone interview. "I don't
regret not taking pictures of landscapes. I started out as an apprentice
portraiture photographer; people came to be photographed or (I would) go
to weddings and school groups."

During his career spanning six decades, the Accra-born photographer has
remained unwavering in his mission statement: People are more
interesting than places. Now, having just turned 91, Barnor is one of
Ghana's most well-known photographers, though it's only in this century
that his work has been celebrated in exhibitions across Europe and in
the US -\/- after curator Nana Oforiatta-Ayim organized his first solo
exhibition in 2007, held at the Black Cultural Archives in London.

\includegraphics{https://dynaimage.cdn.cnn.com/cnn/e_blur:500,q_auto:low,w_50,c_fit/http\%3A\%2F\%2Fcdn.cnn.com\%2Fcnnnext\%2Fdam\%2Fassets\%2F200618173350-17-james-barnor.jpg}

AGIP with Graphic Designer, 1974 Credit: James Barnor

Since then, Barnor's photos have been acquired for The Victoria and
Albert Museum and Tate's permanent collections, and last year the Nubuke
Foundation in Accra held a retrospective of his work (the first time the
foundation has hosted a retrospective of a homegrown photographer). And
the Serpentine Galleries in London were scheduled to host a
retrospective to coincide with Barnor's birthday this month (due to
Covid-19 restrictions, this is now postponed until 2021).

\href{/style/article/vogue-challenge-cover/index.html}{}

\includegraphics{https://dynaimage.cdn.cnn.com/cnn/e_blur:500,q_auto:low,w_50,c_fill,g_auto,h_50,ar_1:1/http\%3A\%2F\%2Fcdn.cnn.com\%2Fcnnnext\%2Fdam\%2Fassets\%2F200615123223-02-vogue-challenge-kalekye-mumo.jpg}

People are putting themselves on the cover of Vogue to promote diversity

It was as the first photojournalist appointed to the Daily Graphic, a
Ghanaian state-owned daily newspaper in Accra, that Barnor honed his
craft and developed a documentary eye. He took pictures of the city's
people and events -\/- everything car accidents and football matches to
locals at the market.

"I'd just take my camera and my bicycle and go to the market," he said.
"I'd spend about 20 minutes there and I'd come back to my darkroom, with
pictures all telling different stories. I did that a lot. You capture
the market woman the way she actually is. It takes some art, some
patience, and technique to photograph people in that setting."

\includegraphics{https://dynaimage.cdn.cnn.com/cnn/e_blur:500,q_auto:low,w_50,c_fit/http\%3A\%2F\%2Fcdn.cnn.com\%2Fcnnnext\%2Fdam\%2Fassets\%2F200618170923-08-james-barnor.jpg}

Untitled \#4, Sick-Hagemeyer shop assistant, Accra, 1971 Credit: James
Barnor/courtesy October Gallery

Alongside his photojournalism, Barnor took portraits of local residents
at his own makeshift studio, which he christened Ever Young, in the
Jamestown district of Accra, mostly between 1949 and 1959. While this
was predominately a commercial proposition, taking on paid portraiture
commissions and passport headshots, he developed a natural exchange with
many of his subjects.

Barnor's studio was like a "community center," he remembers, and he made
"people feel at home," by talking to and getting to know them. "Young
men would come by to have a chat and have their photograph taken," he
said. "Most people had confidence in me already. Everybody knew me in
Ghana as a successful photographer -\/- they knew they would be
satisfied."

\href{/style/article/a-ti-de-oye-diran-photography/index.html}{}

\includegraphics{https://dynaimage.cdn.cnn.com/cnn/e_blur:500,q_auto:low,w_50,c_fill,g_auto,h_50,ar_1:1/http\%3A\%2F\%2Fcdn.cnn.com\%2Fcnnnext\%2Fdam\%2Fassets\%2F200528110101-13-africa-avant-garde.jpg}

Vintage family pictures inspired this Nigerian photographer's latest
work

Barnor says he believes the photographs he took during this time showed
a different, stylish view of his home country -\/- one that belied
assumptions. "When I had my studio in Ghana people thought we
(Ghanaians) didn't dress up," he said. "But all my sitters, my friends,
were fashion conscious -\/- women would often request full-length photos
with shoes, a handbag and their accessories."

In 1959 Barnor moved to the UK after receiving a scholarship to study at
the Medway College of Art in Rochester, Kent. After graduation in 1961,
he moved to South London, reveling in the capital city's "colors,
adverts, uniforms and music."

\begin{itemize}
\item
  \includegraphics{https://dynaimage.cdn.cnn.com/cnn/e_blur:500,q_auto:low,w_50,c_fit,h_28,ar_16:9/http\%3A\%2F\%2Fcdn.cnn.com\%2Fcnnnext\%2Fdam\%2Fassets\%2F200618170200-04-james-barnor.jpg}
\item
  \includegraphics{https://dynaimage.cdn.cnn.com/cnn/e_blur:500,q_auto:low,w_50,c_fit,h_28,ar_16:9/http\%3A\%2F\%2Fcdn.cnn.com\%2Fcnnnext\%2Fdam\%2Fassets\%2F200618172951-14-james-barnor.jpg}
\item
  \includegraphics{https://dynaimage.cdn.cnn.com/cnn/e_blur:500,q_auto:low,w_50,c_fit,h_28,ar_16:9/http\%3A\%2F\%2Fcdn.cnn.com\%2Fcnnnext\%2Fdam\%2Fassets\%2F200618170923-08-james-barnor.jpg}
\item
  \includegraphics{https://dynaimage.cdn.cnn.com/cnn/e_blur:500,q_auto:low,w_50,c_fit,h_28,ar_16:9/http\%3A\%2F\%2Fcdn.cnn.com\%2Fcnnnext\%2Fdam\%2Fassets\%2F200618173138-15-james-barnor.jpg}
\item
  \includegraphics{https://dynaimage.cdn.cnn.com/cnn/e_blur:500,q_auto:low,w_50,c_fit,h_28,ar_16:9/http\%3A\%2F\%2Fcdn.cnn.com\%2Fcnnnext\%2Fdam\%2Fassets\%2F200618173245-16-james-barnor.jpg}
\item
  \includegraphics{https://dynaimage.cdn.cnn.com/cnn/e_blur:500,q_auto:low,w_50,c_fit,h_28,ar_16:9/http\%3A\%2F\%2Fcdn.cnn.com\%2Fcnnnext\%2Fdam\%2Fassets\%2F200618173530-18-james-barnor.jpg}
\item
  \includegraphics{https://dynaimage.cdn.cnn.com/cnn/e_blur:500,q_auto:low,w_50,c_fit,h_28,ar_16:9/http\%3A\%2F\%2Fcdn.cnn.com\%2Fcnnnext\%2Fdam\%2Fassets\%2F200618172159-11-james-barnor.jpg}
\item
  \includegraphics{https://dynaimage.cdn.cnn.com/cnn/e_blur:500,q_auto:low,w_50,c_fit,h_28,ar_16:9/http\%3A\%2F\%2Fcdn.cnn.com\%2Fcnnnext\%2Fdam\%2Fassets\%2F200618173350-17-james-barnor.jpg}
\item
  \includegraphics{https://dynaimage.cdn.cnn.com/cnn/e_blur:500,q_auto:low,w_50,c_fit,h_28,ar_16:9/http\%3A\%2F\%2Fcdn.cnn.com\%2Fcnnnext\%2Fdam\%2Fassets\%2F200618172302-12-james-barnor.jpg}
\item
  \includegraphics{https://dynaimage.cdn.cnn.com/cnn/e_blur:500,q_auto:low,w_50,c_fit,h_28,ar_16:9/http\%3A\%2F\%2Fcdn.cnn.com\%2Fcnnnext\%2Fdam\%2Fassets\%2F200618172818-13-james-barnor.jpg}
\item
  \includegraphics{https://dynaimage.cdn.cnn.com/cnn/e_blur:500,q_auto:low,w_50,c_fit,h_28,ar_16:9/http\%3A\%2F\%2Fcdn.cnn.com\%2Fcnnnext\%2Fdam\%2Fassets\%2F200618172023-10-james-barnor.jpg}
\item
  \includegraphics{https://dynaimage.cdn.cnn.com/cnn/e_blur:500,q_auto:low,w_50,c_fit,h_28,ar_16:9/http\%3A\%2F\%2Fcdn.cnn.com\%2Fcnnnext\%2Fdam\%2Fassets\%2F200618170313-05-james-barnor.jpg}
\item
  \includegraphics{https://dynaimage.cdn.cnn.com/cnn/e_blur:500,q_auto:low,w_50,c_fit,h_28,ar_16:9/http\%3A\%2F\%2Fcdn.cnn.com\%2Fcnnnext\%2Fdam\%2Fassets\%2F200618171056-09-james-barnor.jpg}
\item
  \includegraphics{https://dynaimage.cdn.cnn.com/cnn/e_blur:500,q_auto:low,w_50,c_fit,h_28,ar_16:9/http\%3A\%2F\%2Fcdn.cnn.com\%2Fcnnnext\%2Fdam\%2Fassets\%2F200618165702-02-james-barnor.jpg}
\item
  \includegraphics{https://dynaimage.cdn.cnn.com/cnn/e_blur:500,q_auto:low,w_50,c_fit,h_28,ar_16:9/http\%3A\%2F\%2Fcdn.cnn.com\%2Fcnnnext\%2Fdam\%2Fassets\%2F200618165915-03-james-barnor.jpg}
\end{itemize}

1/15

"Drum cover girl Marie Hallowi \#1, Rochester, Kent" (1966) Credit:
James Barnor/courtesy October Gallery

But it wasn't all smooth sailing. "If you wanted a room or something and
you were Black (people would say) 'Oh -\/- it's gone,'" Barnor said.
"This is what it was like. I was lucky though -\/- I call myself Lucky
Jim. Before I left Ghana, I befriended somebody who worked at the
Colonial Office (a government department that oversaw the administration
of Britain's territories). He helped me find somewhere to live,
introducing me to a Jamaican man in Peckham."

Once in London, Barnor mostly produced work for anti-Apartheid journal
Drum magazine, for whom he had already worked making fashion stories
while he was still living in Ghana. At the time, Barnor recalls, the
magazine stood out for featuring Black models in its pages and on its
covers.

\href{/style/article/kyle-weeks-photography-africa/index.html}{}

\includegraphics{https://dynaimage.cdn.cnn.com/cnn/e_blur:500,q_auto:low,w_50,c_fill,g_auto,h_50,ar_1:1/http\%3A\%2F\%2Fcdn.cnn.com\%2Fcnnnext\%2Fdam\%2Fassets\%2F200227174248-kyle-weeks-5.jpg}

Photographer Kyle Weeks uses his medium to show that manhood in Africa
is not a singular image

"There weren't magazines or newspapers showing Black models -\/- Drum
started it," he said. "Any time I saw a Drum cover in London, side by
side with international magazines, I felt really satisfied. I knew I was
recording something. I knew I had to take care of my negatives."

As part of his work with Drum, Barnor captured the BBC's first Black
broadcaster, Mike Eghan, on the steps of the famous Eros statue in
Piccadilly Circus, central London. Another photo shows street-scouted
cover girl, 19-year-old Erlin Ibreck -\/- whom he met waiting for a bus
-\/- feeding pigeons in Trafalgar Square.

\includegraphics{https://dynaimage.cdn.cnn.com/cnn/e_blur:500,q_auto:low,w_50,c_fit/http\%3A\%2F\%2Fcdn.cnn.com\%2Fcnnnext\%2Fdam\%2Fassets\%2F200618170534-06-james-barner.jpg}

Erlin Ibreck at Trafalgar Square, 1966 Credit: James Barnor/courtesy
October Gallery

Like his time capturing individual style in his studio portraiture in
Accra, in the UK Barnor amassed a large archive of street photos,
chronicling how Black people were dressing in bright western fashions
distinctive of the era. But unlike his monochrome photos from earlier
days, he was now working in color, a transition that has helped define
his signature style. "My learning, and everything, is around how the
Black body appears in color," he said. "With all these lovely color
dresses."

\includegraphics{https://dynaimage.cdn.cnn.com/cnn/e_blur:500,q_auto:low,w_50,c_fit/http\%3A\%2F\%2Fcdn.cnn.com\%2Fcnnnext\%2Fdam\%2Fassets\%2F200618170313-05-james-barnor.jpg}

Drum Cover Girl Marie Hallowi at Charing Cross Station, London, 1966
Credit: James Barnor/courtesy October Gallery

As well as paid fashion editorial work during the 1960s, Barnor took
photos of his friends and sought out subjects from the Ghanaian and
wider African communities in South London. The resulting archive -\/-
spanning about ten years form 1959-1969 -\/- is a goldmine of street
style reportage, editorial shoots and portraiture of the African
diaspora in the UK.

And though standalone landscapes were not of interest to Barnor, he
liked to choose locations and backgrounds that spoke to time and place.

"I like vantage points or places which connect to where you are," Barnor
said. "A secluded background, unless it's helping in some way
artistically, is not so good. I have a picture of two friends, I posed
them in front of a telephone kiosk. When you see the image, you know:
This is London."

\includegraphics{https://dynaimage.cdn.cnn.com/cnn/e_blur:500,q_auto:low,w_50,c_fit/http\%3A\%2F\%2Fcdn.cnn.com\%2Fcnnnext\%2Fdam\%2Fassets\%2F200618165330-01-james-barnor.jpg}

A young James Barnor Credit: James Barnor

\href{/style/article/mous-lamrabat-photography/index.html}{}

\includegraphics{https://dynaimage.cdn.cnn.com/cnn/e_blur:500,q_auto:low,w_50,c_fill,g_auto,h_50,ar_1:1/http\%3A\%2F\%2Fcdn.cnn.com\%2Fcnnnext\%2Fdam\%2Fassets\%2F200420101240-04-mous-lamrabat.jpg}

Provocative photographer Mous Lamrabat subverts North African
stereotypes

Barnor eventually returned to Ghana, in 1969, and spent over two decades
there before returning to London, where he now resides. During that time
he opened the country's first color processing laboratory, with the
hopes of sharing his knowledge and experience. "You want to bring people
up, you want to advise them in whatever way you can,"

But skills, he says, are not the only thing that's important, and to
aspiring photographers, Barnor's advice is simple: Tell a story that
means something to you. "You can google all the technical stuff. It's
the ideas that you have that're important. The community development,
the self-involvement. Go and learn and be knowledgeable and take the
camera. The story is the picture."

Search

\begin{itemize}
\tightlist
\item
  \href{/us}{US}

  \begin{itemize}
  \tightlist
  \item
    \href{/specials/us/crime-and-justice}{Crime + Justice}
  \item
    \href{/specials/us/energy-and-environment}{Energy + Environment}
  \item
    \href{/specials/us/extreme-weather}{Extreme Weather}
  \item
    \href{/specials/space-science}{Space + Science}
  \end{itemize}
\item
  \href{/world}{World}

  \begin{itemize}
  \tightlist
  \item
    \href{/africa}{Africa}
  \item
    \href{/americas}{Americas}
  \item
    \href{/asia}{Asia}
  \item
    \href{/australia}{Australia}
  \item
    \href{/china}{China}
  \item
    \href{/europe}{Europe}
  \item
    \href{/india}{India}
  \item
    \href{/middle-east}{Middle East}
  \item
    \href{/uk}{United Kingdom}
  \end{itemize}
\item
  \href{/politics}{Politics}

  \begin{itemize}
  \tightlist
  \item
    \href{/specials/politics/president-donald-trump-45}{45}
  \item
    \href{/specials/politics/congress-capitol-hill}{Congress}
  \item
    \href{/specials/politics/supreme-court-nine}{SCOTUS}
  \item
    \href{/specials/politics/fact-check-politics}{Facts First}
  \item
    \href{/specials/politics/2020-election-coverage}{2020}
  \item
    \href{/election/2020/candidates}{Candidates}
  \end{itemize}
\item
  \href{/business}{Business}

  \begin{itemize}
  \tightlist
  \item
    \href{https://money.cnn.com/data/markets/}{Markets}
  \item
    \href{/business/tech}{Tech}
  \item
    \href{/business/media}{Media}
  \item
    \href{/business/success}{Success}
  \item
    \href{/business/perspectives}{Perspectives}
  \item
    \href{/business/videos}{Videos}
  \end{itemize}
\item
  \href{/opinions}{Opinion}

  \begin{itemize}
  \tightlist
  \item
    \href{/specials/opinion/opinion-politics}{Political Op-Eds}
  \item
    \href{/specials/opinion/opinion-social-issues}{Social Commentary}
  \end{itemize}
\item
  \href{/health}{Health}

  \begin{itemize}
  \tightlist
  \item
    \href{/specials/health/food-diet}{Food}
  \item
    \href{/specials/health/fitness-excercise}{Fitness}
  \item
    \href{/specials/health/wellness}{Wellness}
  \item
    \href{/specials/health/parenting}{Parenting}
  \item
    \href{/specials/health/vital-signs}{Vital Signs}
  \end{itemize}
\item
  \href{/entertainment}{Entertainment}

  \begin{itemize}
  \tightlist
  \item
    \href{/entertainment/celebrities}{Stars}
  \item
    \href{/entertainment/movies}{Screen}
  \item
    \href{/entertainment/tv-shows}{Binge}
  \item
    \href{/entertainment/culture}{Culture}
  \item
    \href{/business/media}{Media}
  \end{itemize}
\item
  \href{/business/tech}{Tech}

  \begin{itemize}
  \tightlist
  \item
    \href{/specials/tech/innovate}{Innovate}
  \item
    \href{/specials/tech/gadget}{Gadget}
  \item
    \href{/specials/tech/mission-ahead}{Mission: Ahead}
  \item
    \href{/specials/tech/upstarts}{Upstarts}
  \item
    \href{/specials/tech/work-transformed}{Work Transformed}
  \item
    \href{/specials/tech/innovative-cities}{Innovative Cities}
  \end{itemize}
\item
  \href{/style}{Style}

  \begin{itemize}
  \tightlist
  \item
    \href{/style/arts}{Arts}
  \item
    \href{/style/design}{Design}
  \item
    \href{/style/fashion}{Fashion}
  \item
    \href{/style/architecture}{Architecture}
  \item
    \href{/style/luxury}{Luxury}
  \item
    \href{/style/beauty}{Beauty}
  \item
    \href{/style/videos}{Video}
  \end{itemize}
\item
  \href{/travel}{Travel}

  \begin{itemize}
  \tightlist
  \item
    \href{/travel/destinations}{Destinations}
  \item
    \href{/travel/food-and-drink}{Food \& Drink}
  \item
    \href{/travel/news}{News}
  \item
    \href{/travel/stay}{Stay}
  \item
    \href{/travel/videos}{Videos}
  \end{itemize}
\item
  \href{http://bleacherreport.com}{Sports}

  \begin{itemize}
  \tightlist
  \item
    \href{http://bleacherreport.com/nfl}{Pro Football}
  \item
    \href{http://bleacherreport.com/college-football}{College Football}
  \item
    \href{http://bleacherreport.com/nba}{Basketball}
  \item
    \href{http://bleacherreport.com/mlb}{Baseball}
  \item
    \href{http://bleacherreport.com/world-football}{Soccer}
  \item
    \href{/specials/sport/winter-olympics-2018}{Olympics}
  \end{itemize}
\item
  \href{/videos}{Videos}

  \begin{itemize}
  \tightlist
  \item
    \href{//cnn.it/go2}{Live TV}
  \item
    \href{/specials/digital-studios}{Digital Studios}
  \item
    \href{/specials/videos/digital-shorts}{CNN Films}
  \item
    \href{/specials/videos/hln}{HLN}
  \item
    \href{/tv/schedule/cnn}{TV Schedule}
  \item
    \href{/specials/tv/all-shows}{TV Shows A-Z}
  \item
    \href{/vr}{CNNVR}
  \end{itemize}
\item
  \href{//coupons.cnn.com}{Coupons}

  \begin{itemize}
  \tightlist
  \item
    \href{/cnn-underscored/}{CNN Underscored}
  \item
    \href{/specials/cnn-underscored/explore/}{Explore}
  \item
    \href{/specials/cnn-underscored/wellness/}{Wellness}
  \item
    \href{/specials/cnn-underscored/gadgets/}{Gadgets}
  \item
    \href{/specials/cnn-underscored/lifestyle/}{Lifestyle}
  \item
    \href{//store.cnn.com/?utm_source=cnn.com\&utm_medium=referral\&utm_campaign=navbar}{CNN
    Store}
  \end{itemize}
\item
  \href{/more}{More}

  \begin{itemize}
  \tightlist
  \item
    \href{/specials/photos}{Photos}
  \item
    \href{/specials/cnn-longform}{Longform}
  \item
    \href{/specials/cnn-investigates}{Investigations}
  \item
    \href{/specials/profiles}{CNN Profiles}
  \item
    \href{/specials/more/cnn-leadership}{CNN Leadership}
  \item
    \href{/email/subscription}{CNN Newsletters}
  \item
    \href{https://www.turnerjobs.com/search-jobs?orgIds=1174\&ac=19299}{Work
    for CNN}
  \end{itemize}
\end{itemize}

\begin{center}\rule{0.5\linewidth}{\linethickness}\end{center}

Follow CNN

\begin{itemize}
\item
\item
\item
\end{itemize}

\begin{center}\rule{0.5\linewidth}{\linethickness}\end{center}

\begin{itemize}
\tightlist
\item
  \href{/terms}{Terms of Use}
\item
  \href{/privacy}{Privacy Policy}
\item
  \href{/accessibility}{Accessibility \& CC}
\item
  \protect\hyperlink{}{AdChoices}
\item
  \href{/about}{About Us}
\item
  \href{/tour}{CNN Studio Tours}
\item
  \href{/msa}{Modern Slavery Act Statement}
\item
  \href{https://commercial.cnn.com}{Advertise with us}
\item
  \href{//store.cnn.com}{CNN Store}
\item
  \href{/newsletters}{Newsletters}
\item
  \href{/transcripts}{Transcripts}
\item
  \href{/collection}{License Footage}
\item
  \href{http://cnnnewsource.com}{CNN Newsource}
\item
  \href{https://www.cnn.com/sitemap.html}{Sitemap}
\end{itemize}

© 2020 Cable News Network.\href{//www.turner.com}{Turner Broadcasting
System, Inc.}All Rights Reserved.CNN Sans ™ \& © 2016 Cable News
Network.
