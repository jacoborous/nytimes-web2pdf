\href{/}{}\href{/travel}{}

\href{/travel/destinations}{Destinations}\href{/travel/food-and-drink}{Food
\&
Drink}\href{/travel/news}{News}\href{/travel/stay}{Stay}\href{/travel/videos}{Video}

Search

Menu

\hypertarget{the-uncle-roger-controversy-why-people-are-outraged-by-a-video-about-cooking-rice}{%
\section{The Uncle Roger controversy: Why people are outraged by a video
about cooking
rice}\label{the-uncle-roger-controversy-why-people-are-outraged-by-a-video-about-cooking-rice}}

Analysis by Jessie Yeung, CNN • Updated 30th July 2020

FacebookTwitterEmail

Up next

(CNN) --- On July 8, Malaysian comedian Nigel Ng uploaded
\href{https://www.youtube.com/watch?v=53me-ICi_f8}{to YouTube}
\href{https://www.youtube.com/watch?v=53me-ICi_f8}{a video} titled
"DISGUSTED by this Egg Fried Rice Video," under his comedic persona
"Uncle Roger."

In the video, Ng slammed **** BBC Food presenter Hersha Patel's
unconventional way of cooking Chinese-style egg-fried rice, which
included draining the rice through a strainer after boiling.

"What she doing? Oh my god. You're killing me, woman. Drain the -\/-
she's draining rice with colander! How can you drain rice with colander?
This is not pasta!" he exclaimed.

Shortly afterward, he groaned, "You're ruining the rice," as Patel used
tap water to wash it of starch.

What Ng intended to be a comedic video sparked a firestorm of dismay and
disbelief as it ricocheted around the internet, gaining more than 7
million views on YouTube and nearly 40 million on Twitter.

Many viewers, including Asian-American celebrities such as writer Jenny
Yang, derided Patel's methods for departing from how Chinese egg-fried
rice is traditionally made. Patel hadn't washed the rice before boiling
it. She had added too much water. She should have used day-old rice. The
scrambled egg was overcooked instead of runny.

"THIS RICE COOKING IS A HATE CRIME," Yang
\href{https://twitter.com/jennyyangtv/status/1286131803246350336}{joked
on Twitter.}

Ng, who is based in London, tried to defuse the situation by filming a
short clip with Patel announcing they are planning a collaboration.
"While this guy's blown up like nobody's business, I've been trolled,"
Patel said
\href{https://twitter.com/hershapatel1/status/1286794759982522370}{in
the video}, claiming she had been simply presenting the BBC's recipe and
that "I know how to cook rice."

The BBC has not publicly commented on Ng's or Patel's remarks.

Rice is a staple ingredient in Asia, and has been adopted by cuisines
globally since it was first domesticated in China more than 9,400 years
ago, according to
\href{https://phys.org/news/2017-05-domesticated-rice-dated-years-china.html}{Chinese
researchers}. There are countless ways to prepare rice -\/- you can
steam it, fry it, simmer it slowly in broth like Italian risotto or
scorch it to develop a crispy crust like Iranian tahdig.

But the issue at hand goes beyond a difference in opinion on the varying
methods of cooking rice.

The controversy over the BBC Food clip, and the reaction it provoked
within certain Asian communities, speaks to a broader, long-standing
debate about the intersection of food, ethnicity and culture -\/- the
fundamental question of who is allowed to cook what food.

\hypertarget{appropriating-and-whitewashing-food}{%
\subsubsection{Appropriating and whitewashing
food}\label{appropriating-and-whitewashing-food}}

Countless White chefs in recent years have been accused of cultural
appropriation by creating food from other ethnic groups using methods
and phrases that are deemed "unauthentic," disrespectful, and sometimes
outright racist.

Last year, for instance, an Asian food critic
\href{https://www.cnn.com/travel/article/gordon-ramsey-asian-restaurant-cultural-appropriation-intl-scli/index.html}{accused
celebrity chef Gordon Ramsay} of tokenism, after he launched a
restaurant described in promotional material as "an authentic Asian
eating house."

The restaurant didn't differentiate between wildly different and unique
types of Asian cuisines, lumping them all together as generically Asian.
And at the time of the opening, it did not appear to have any Asian
chefs.

Related content

\href{/travel/article/gordon-ramsey-asian-restaurant-cultural-appropriation-intl-scli/index.html}{Gordon
Ramsay's new 'authentic Asian' restaurant kicks off cultural
appropriation dispute}

"Japanese? Chinese? It's all Asian, who cares," wrote the critic, Angela
Hui, in a scathing
\href{https://london.eater.com/2019/4/11/18306082/gordon-ramsay-cultural-appropriation-lucky-cat-london-preview}{Instagram
story.}

CNN reached out to Ramsay's restaurant group for comment after the
initial controversy.

Tokenism is when racial, ethnic, or cultural diversity is emphasized
only on a symbolic level, without much substantial effort to understand
that culture -\/- in Ramsay's case, labeling a restaurant "Asian"
without taking the time to differentiate between these individual
nuanced cuisines.

Food is not just sustenance, it carries history and heritage, which is
why many people are deeply offended when these traditional methods of
cooking **** are cast aside.

Sometimes chefs don't just change up cooking methods, they blatantly
insult the cuisine and culture of origin.

One notorious example is the Chinese-inspired restaurant Lucky Lee's in
New York. When it opened in 2019, the White owner said it would serve
"clean" food that wouldn't make people feel "bloated and icky"
afterwards -\/- the implication being that regular Chinese food was
somehow unhealthy. That sparked uproar and the restaurant closed
\href{https://ny.eater.com/2019/12/6/20999639/lucky-lees-greenwich-village-clean-chinese-closed}{eight
months later}.

And then there are chefs who fail to acknowledge a dish's ethnic origins
at all -\/- the equivalent of whitewashing food.

The New York Times food columnist Alison Roman, also a White woman,
gained internet fame for her recipe for a
\href{https://cooking.nytimes.com/recipes/1019772-spiced-chickpea-stew-with-coconut-and-turmeric}{"Spiced
Chickpea Stew with Coconut and Turmeric"} -\/- which sounds an awful lot
like an Indian or Jamaican curry. But in an interview with
\href{https://jezebel.com/alison-roman-is-more-than-thestew-1838861751}{Jezebel},
she said: "I'm like y'all, this is not a curry ... I've never made a
curry." Roman's refusal to call it a curry and her denial of its ethnic
background prompted critic Roxana Hadadi to call it
\href{https://www.pajiba.com/celebrities_are_better_than_you/alison-roman-and-the-exhausting-prevalence-of-ethnic-erasure-in-popular-food-culture.php}{"colonialism
as cuisine."}

In response to the backlash, NYT eventually added a line in Roman's
recipe on their website, saying it "evokes stews found in South India
and parts of the Caribbean."

Related content

\href{/2020/01/18/asia/chinese-restaurant-syndrome-msg-intl-hnk-scli/index.html}{MSG
in Chinese food isn't unhealthy -\/- you're just racist, activists say}

But some people have pushed back against the idea of cultural
appropriation.

Gatekeeping food prevents innovation,
\href{https://www.foodrepublic.com/2017/06/01/cultural-appropriation-food/}{some
say}: **** for instance, fusion foods are born from chefs experimenting
with different cuisines. Many also point out that food is meant to be
shared, and its power is often directly tied to the communal eating
experience.

Setting boundaries around food -\/- for example, saying only Chinese
people can cook Chinese food, or Chinese food can only be cooked a
certain way, as those reacting to Ng's video posit -\/- seems like the
antithesis of this sharing spirit in our globalized world.

But sharing is different from appropriating without respect, especially
when the chefs who do it profit from portraying those foods.

\hypertarget{a-reckoning-in-food-media}{%
\subsubsection{A reckoning in food
media}\label{a-reckoning-in-food-media}}

The Uncle Roger video is the latest in a string of incidents that have
drawn attention to issues surrounding food and culture. This summer has
seen the reckoning on race and racism, embodied by the Black Lives
Matter movement, spread from the streets to newsrooms and companies.

Within food media, Bon Appetit -\/- owned by Conde Nast -\/- is the
best-known example. Current staffers, including assistant food editor
Sohla El-Waylly, accused the company of underpaying and exploiting
employees of color, and viewers called out the brand for numerous
instances of food appropriation.

For instance, irate viewers pointed to the time Bon Appetit had a White
chef demonstrate how to cook
\href{https://www.bonappetit.com/story/how-you-should-eating-pho?mbid=social_facebook}{Vietnamese
pho}, with the title "PSA: This Is How You Should Be Eating Pho." There
was also the time they "reinvented" the Filipino dessert
\href{https://www.bonappetit.com/recipe/ode-to-halo-halo}{Halo-halo} by
stuffing it with gummy bears and popcorn, spurring scorn from readers.

Each time, the brand would issue an apology and a promise to do better
-\/- but it has been happening for years.

After this summer's explosive allegations, the company
\href{https://www.bonappetit.com/story/recipe-audit?utm_brand=ba\&mbid=social_twitter\&utm_social-type=owned\&utm_source=twitter\&utm_medium=social}{released
a statement in June,} acknowledging that "BA's recipes for Vietnamese
pho, mumbo sauce, flaky bread, and White-guy kimchi all erased these
recipes' origins or, worse, lampooned them."

Related content

\href{/2020/06/10/media/bon-appetit-apologizes/index.html}{Bon Appetit
vows to resolve pay inequities and prioritize people of color for editor
in chief search}

"In all these cases and more, BA has been called out for appropriation,
for decontextualizing recipes from non-White cultures, and for knighting
'experts' without considering if that person should, in fact, claim
mastery of a cuisine that isn't theirs," wrote Joey Hernandez, BA's
research director, in the statement.

The Bon Appetit debacle also prompted other questions about biases
within established institutions. Who chooses what dishes get more
coverage? Why do publications continue to use language that frames
"ethnic" food as occasionally bizarre and often incomprehensible -\/-
for example, Bloomberg calling tofu a
\href{https://www.bloomberg.com/news/articles/2020-06-11/the-pandemic-opens-the-door-to-tofu-makers-who-race-to-meet-deman?sref=783qcuCY}{"white,
chewy and bland"} food people are "learning to love?" Bloomberg
eventually removed these phrases from their article after international
backlash.

And why are "ethnic" chefs -\/- a euphemism for non-Whites -\/- often
paid less? Bon Appetit fans were further outraged when Somali chef Hawa
Hassan \href{https://i.imgur.com/QUCsdbx.png}{revealed last month} that
she was only paid \$400 per video, and El-Waylly
\href{https://twitter.com/sarahmanavis/status/1270069650060083207?ref_src=twsrc\%5Etfw\%7Ctwcamp\%5Etweetembed\%7Ctwterm\%5E1270069650060083207\%7Ctwgr\%5E\&ref_url=https\%3A\%2F\%2Fwww.insider.com\%2Fbon-appetit-sohla-el-waylly-people-rallying-behind-adam-rapoport-2020-6}{blasted
Bon Appetit} for only paying her \$50,000 to "assist mostly white
editors with significantly less experience than me."

These themes sound abstract at times -\/- but they're linked to and help
perpetuate broader **** real-life inequalities such as workplace
discrimination, pay inequity, power imbalances and prevailing Whiteness
in the food world.

Ng and Patel may not have intended for their respective videos, and
upcoming collaboration, to raise these questions.

But viewers' frustrations are inherently tied to the idea that there is
an authentic way to cook fried rice, and that Patel's errors are made
worse by the fact she is a non-Chinese chef presenting herself as an
authority on the dish.

"FOR ANYONE WHO IS TRYING TO SAY THERE ARE MULTIPLE WAYS OF COOKING
RICE, WELL OF COURSE THERE ARE. AND I LOVE THEM ALL,"
\href{https://twitter.com/jennyyangtv/status/1286351923407396864}{tweeted
Yang,} the Asian American writer. "BUT THIS IS *NOT* HOW YOU MAKE
DELICIOUS FRIED RICE, THE DISH OF MY PEOPLES, THE SUBJECT OF THIS
VIDEO."

\href{//www.cnn.com/interactive/travel/best-beaches}{}

A year of the world'sBest BeachesThere's a perfect beach for every week
of the year. Join us on a 12-month journey to see them all

Go to the best beaches

\includegraphics{https://dynaimage.cdn.cnn.com/cnn/e_blur:500,q_auto:low,w_50,c_fill,g_auto,h_28,ar_16:9/http\%3A\%2F\%2Fcdn.cnn.com\%2Fcnnnext\%2Fdam\%2Fassets\%2F171220172042-best-beaches-promo.jpg}

Search

\begin{itemize}
\tightlist
\item
  \href{/us}{US}

  \begin{itemize}
  \tightlist
  \item
    \href{/specials/us/crime-and-justice}{Crime + Justice}
  \item
    \href{/specials/us/energy-and-environment}{Energy + Environment}
  \item
    \href{/specials/us/extreme-weather}{Extreme Weather}
  \item
    \href{/specials/space-science}{Space + Science}
  \end{itemize}
\item
  \href{/world}{World}

  \begin{itemize}
  \tightlist
  \item
    \href{/africa}{Africa}
  \item
    \href{/americas}{Americas}
  \item
    \href{/asia}{Asia}
  \item
    \href{/australia}{Australia}
  \item
    \href{/china}{China}
  \item
    \href{/europe}{Europe}
  \item
    \href{/india}{India}
  \item
    \href{/middle-east}{Middle East}
  \item
    \href{/uk}{United Kingdom}
  \end{itemize}
\item
  \href{/politics}{Politics}

  \begin{itemize}
  \tightlist
  \item
    \href{/specials/politics/president-donald-trump-45}{45}
  \item
    \href{/specials/politics/congress-capitol-hill}{Congress}
  \item
    \href{/specials/politics/supreme-court-nine}{SCOTUS}
  \item
    \href{/specials/politics/fact-check-politics}{Facts First}
  \item
    \href{/specials/politics/2020-election-coverage}{2020}
  \item
    \href{/election/2020/candidates}{Candidates}
  \end{itemize}
\item
  \href{/business}{Business}

  \begin{itemize}
  \tightlist
  \item
    \href{https://money.cnn.com/data/markets/}{Markets}
  \item
    \href{/business/tech}{Tech}
  \item
    \href{/business/media}{Media}
  \item
    \href{/business/success}{Success}
  \item
    \href{/business/perspectives}{Perspectives}
  \item
    \href{/business/videos}{Videos}
  \end{itemize}
\item
  \href{/opinions}{Opinion}

  \begin{itemize}
  \tightlist
  \item
    \href{/specials/opinion/opinion-politics}{Political Op-Eds}
  \item
    \href{/specials/opinion/opinion-social-issues}{Social Commentary}
  \end{itemize}
\item
  \href{/health}{Health}

  \begin{itemize}
  \tightlist
  \item
    \href{/specials/health/food-diet}{Food}
  \item
    \href{/specials/health/fitness-excercise}{Fitness}
  \item
    \href{/specials/health/wellness}{Wellness}
  \item
    \href{/specials/health/parenting}{Parenting}
  \item
    \href{/specials/health/vital-signs}{Vital Signs}
  \end{itemize}
\item
  \href{/entertainment}{Entertainment}

  \begin{itemize}
  \tightlist
  \item
    \href{/entertainment/celebrities}{Stars}
  \item
    \href{/entertainment/movies}{Screen}
  \item
    \href{/entertainment/tv-shows}{Binge}
  \item
    \href{/entertainment/culture}{Culture}
  \item
    \href{/business/media}{Media}
  \end{itemize}
\item
  \href{/business/tech}{Tech}

  \begin{itemize}
  \tightlist
  \item
    \href{/specials/tech/innovate}{Innovate}
  \item
    \href{/specials/tech/gadget}{Gadget}
  \item
    \href{/specials/tech/mission-ahead}{Mission: Ahead}
  \item
    \href{/specials/tech/upstarts}{Upstarts}
  \item
    \href{/specials/tech/work-transformed}{Work Transformed}
  \item
    \href{/specials/tech/innovative-cities}{Innovative Cities}
  \end{itemize}
\item
  \href{/style}{Style}

  \begin{itemize}
  \tightlist
  \item
    \href{/style/arts}{Arts}
  \item
    \href{/style/design}{Design}
  \item
    \href{/style/fashion}{Fashion}
  \item
    \href{/style/architecture}{Architecture}
  \item
    \href{/style/luxury}{Luxury}
  \item
    \href{/style/beauty}{Beauty}
  \item
    \href{/style/videos}{Video}
  \end{itemize}
\item
  \href{/travel}{Travel}

  \begin{itemize}
  \tightlist
  \item
    \href{/travel/destinations}{Destinations}
  \item
    \href{/travel/food-and-drink}{Food \& Drink}
  \item
    \href{/travel/news}{News}
  \item
    \href{/travel/stay}{Stay}
  \item
    \href{/travel/videos}{Videos}
  \end{itemize}
\item
  \href{http://bleacherreport.com}{Sports}

  \begin{itemize}
  \tightlist
  \item
    \href{http://bleacherreport.com/nfl}{Pro Football}
  \item
    \href{http://bleacherreport.com/college-football}{College Football}
  \item
    \href{http://bleacherreport.com/nba}{Basketball}
  \item
    \href{http://bleacherreport.com/mlb}{Baseball}
  \item
    \href{http://bleacherreport.com/world-football}{Soccer}
  \item
    \href{/specials/sport/winter-olympics-2018}{Olympics}
  \end{itemize}
\item
  \href{/videos}{Videos}

  \begin{itemize}
  \tightlist
  \item
    \href{//cnn.it/go2}{Live TV}
  \item
    \href{/specials/digital-studios}{Digital Studios}
  \item
    \href{/specials/videos/digital-shorts}{CNN Films}
  \item
    \href{/specials/videos/hln}{HLN}
  \item
    \href{/tv/schedule/cnn}{TV Schedule}
  \item
    \href{/specials/tv/all-shows}{TV Shows A-Z}
  \item
    \href{/vr}{CNNVR}
  \end{itemize}
\item
  \href{//coupons.cnn.com}{Coupons}

  \begin{itemize}
  \tightlist
  \item
    \href{/cnn-underscored/}{CNN Underscored}
  \item
    \href{/specials/cnn-underscored/explore/}{Explore}
  \item
    \href{/specials/cnn-underscored/wellness/}{Wellness}
  \item
    \href{/specials/cnn-underscored/gadgets/}{Gadgets}
  \item
    \href{/specials/cnn-underscored/lifestyle/}{Lifestyle}
  \item
    \href{//store.cnn.com/?utm_source=cnn.com\&utm_medium=referral\&utm_campaign=navbar}{CNN
    Store}
  \end{itemize}
\item
  \href{/more}{More}

  \begin{itemize}
  \tightlist
  \item
    \href{/specials/photos}{Photos}
  \item
    \href{/specials/cnn-longform}{Longform}
  \item
    \href{/specials/cnn-investigates}{Investigations}
  \item
    \href{/specials/profiles}{CNN Profiles}
  \item
    \href{/specials/more/cnn-leadership}{CNN Leadership}
  \item
    \href{/email/subscription}{CNN Newsletters}
  \item
    \href{https://www.turnerjobs.com/search-jobs?orgIds=1174\&ac=19299}{Work
    for CNN}
  \end{itemize}
\end{itemize}

\begin{center}\rule{0.5\linewidth}{\linethickness}\end{center}

Follow CNN

\begin{itemize}
\item
\item
\item
\end{itemize}

\begin{center}\rule{0.5\linewidth}{\linethickness}\end{center}

\begin{itemize}
\tightlist
\item
  \href{/terms}{Terms of Use}
\item
  \href{/privacy}{Privacy Policy}
\item
  \href{/accessibility}{Accessibility \& CC}
\item
  \protect\hyperlink{}{AdChoices}
\item
  \href{/about}{About Us}
\item
  \href{/tour}{CNN Studio Tours}
\item
  \href{/msa}{Modern Slavery Act Statement}
\item
  \href{https://commercial.cnn.com}{Advertise with us}
\item
  \href{//store.cnn.com}{CNN Store}
\item
  \href{/newsletters}{Newsletters}
\item
  \href{/transcripts}{Transcripts}
\item
  \href{/collection}{License Footage}
\item
  \href{http://cnnnewsource.com}{CNN Newsource}
\item
  \href{https://www.cnn.com/sitemap.html}{Sitemap}
\end{itemize}

© 2020 Cable News Network.\href{//www.turner.com}{Turner Broadcasting
System, Inc.}All Rights Reserved.CNN Sans ™ \& © 2016 Cable News
Network.
