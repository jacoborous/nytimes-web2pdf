Sections

SEARCH

\protect\hyperlink{site-content}{Skip to
content}\protect\hyperlink{site-index}{Skip to site index}

\href{https://www.nytimes3xbfgragh.onion/section/world}{World}

\href{https://myaccount.nytimes3xbfgragh.onion/auth/login?response_type=cookie\&client_id=vi}{}

\href{https://www.nytimes3xbfgragh.onion/section/todayspaper}{Today's
Paper}

\href{/section/world}{World}\textbar{}Final Passage of Bill To Normalize
U.S. Ties Is Approved, 83 to 15 : Senate Vote Introduces New Era in
China Trade

\begin{itemize}
\item
\item
\item
\item
\item
\end{itemize}

Advertisement

\protect\hyperlink{after-top}{Continue reading the main story}

Supported by

\protect\hyperlink{after-sponsor}{Continue reading the main story}

\hypertarget{final-passage-of-bill-to-normalize-us-ties-is-approved-83-to-15--senate-vote-introduces-new-era-in-china-trade}{%
\section{Final Passage of Bill To Normalize U.S. Ties Is Approved, 83 to
15 : Senate Vote Introduces New Era in China
Trade}\label{final-passage-of-bill-to-normalize-us-ties-is-approved-83-to-15--senate-vote-introduces-new-era-in-china-trade}}

By \href{https://www.nytimes3xbfgragh.onion/by/brian-knowlton}{Brian
Knowlton, International Herald Tribune}

\begin{itemize}
\item
  Sept. 20, 2000
\item
  \begin{itemize}
  \item
  \item
  \item
  \item
  \item
  \end{itemize}
\end{itemize}

\textbf{WASHINGTON---} The Senate, by a resounding 83-to-15 vote, gave
final approval to legislation Tuesday to normalize trade relations with
China permanently, a move expected to unleash a period of fierce
competition and rocky change in the global economic system, but also to
increase stability in the world's most populous country and nudge it
toward greater political openness.

Passage of the bill for permanent normal trade relations clears the way
for China to achieve long-coveted membership in the World Trade
Organization.

That in turn will pry far more open its doors to trade in areas ranging
from agriculture to telecommunications and help transform both its
economic and political relations with the outside world.

Once WTO membership becomes official, probably by early next year, China
will significantly lower its tariffs while opening markets to products
and investment from the United States, Europe and other countries.

President Bill Clinton has been among the most outspoken advocates of
the bill and is expected to sign it into law within days.

Advisers to Mr. Clinton say that its passage is a powerful vindication
of his policy of engagement with China and that it will usher in an era
of pragmatic cooperation. They called the legislation a hallmark
achievement of his presidency.

Mr. Clinton praised Congress for passing the legislation, and moving to
work with China in the interests of both countries. "We will find, I
believe, that America has more influence in China with an outstretched
hand than with a clenched fist," he said.

"The more China opens its markets to our products," he said, "the wider
it opens its doors to economic freedom and the more fully it will
liberate the potential of its people."

Once China joins the WTO, he said, "our high-tech companies will help to
speed th e information revolution there, outside competition will speed
the demise of China's huge state industries and spur the enterprise of
private sector involvement."

This, he said, would "diminish the role of government in people's daily
lives," and strengthen, not weaken, Chinese advocates of human rights,
higher labor standards, a clean environment and the rule of law.

David Lampton, director of China studies at the School of Advanced
International Studies of Johns Hopkins University, called the
legislation "probably the single most important decision the United
States has made on China policy" since the dramatic cooling of relations
after the bloody crushing of pro-democracy protesters in Beijing in June
1989.

Mr. Lampton predicted "far-reaching consequences for American business
in China" as well as "dramatic social and economic effects in China."

U.S. companies expect to benefit from billions of dollars in new
business and an end to years of uncertainty in which they had put off
major decisions about investing in China. The business relationship has
grown rapidly but remains lopsided, partly because of Chinese market
restrictions and partly because of the vast discrepancy in wealth
between the countries.

China is expected to export a record dollars 90 billion in clothing,
textiles, computers and other goods to the United States this year; U.S.
companies will sell about dollars 15 billion in goods to China. China,
including Hong Kong, is the world's fourth-largest exporter.

China has regarded the U.S. debate with discomfort and some irritation.
Officials in Beijing, flustered when the Clinton administration told
congressional doubters that all the compromises in the deal were coming
from China, say life will steadily improve for all Chinese as trade with
the United States surges.

But a Foreign Ministry spokesman, Sun Yuxi, said that passage "not only
serves the interests of China, it is also in the interests of the United
States."

Mr. Lampton said he expected a rocky transition of 10 to 20 years as
China adjusts its economic system, adapts its bureaucracy and confronts
fears of destabilizing change.

"Until China has a stable financial system and a legal system and a
reliable judicial system, there'll be massive problems in China trying
to accommodate the demands of the West," he said. "In the short run, we
can expect lots of friction."

But for the longer term, Mr. Lampton, a former president of the National
Committee on United States-China Relations, called the changes "a big
positive."

U.S. proponents promise a lift in economic activity that will have a
visible impact on ordinary Americans; normalized trade relations, they
say, will provide an anchor of stability for China that will draw it
inevitably away from its Communist roots and move it gradually toward
greater wealth and democracy.

Yet by opening the way for China to join the WTO, the Senate vote will
substantially affect the face of the organization, which referees world
trading practices from its Geneva headquarters.

At the WTO's conference late last year in Seattle, developing countries,
more united than in the past, presented themselves as an undeniable
counterweight to wealthier nations, pressing differing views on trade,
labor practices and the environment. China seems sure to side with them
on many issues.

But China will also become more competitive with other developing
countries, particularly in its pursuit of the U.S. market for clothing
and textile products.

In addition, some skeptics of Chinese entry into the trade organization
say Beijing has little record as a reliable trade partner on which to
base new faith of a sudden conversion.

Labor unions, human rights groups and many conservative politicians had
opposed permanent normal trade relations, saying the United States would
lose the ability to conduct the yearly reviews of Chinese behavior that
had provided a forum to denounce Beijing for a range of complaints over
its use of prison labor, religious persecution, weapons proliferation
and more.

But the yearly extension of most-favored trade status for China had
become near-automatic, owing to support from U.S. businesses and farm
groups, some congressional Republicans and the White House.

The legislation represents the most significant change in U.S. trade law
since the North American Free Trade Agreement in 1993 lowered trade
barriers among the United States, Canada and Mexico.

Permanent normal trade relations is supported by the major-party U.S.
presidential candidates, Vice President Al Gore and Governor George W.
Bush.

As a member of the WTO, China is to reduce tariffs on U.S.-made goods to
9 percent by 2005 from 25 percent in 1997. Tariffs on high-tech products
will be eliminated, while those on automobiles will fall to 25 percent
from at least 80 percent. The American Farm Bureau Federation predicts
that with Chinese agriculture tariffs halved, U.S. agriculture exports
could increase by dollars 2 billion a year.

In the meantime, China is expected to make major investments in
infrastructure, with U.S. companies expected to benefit substantially.

Approval of the normal trade relations bill will "remove one major
shadow over American businesses in China," said Dong Tao, senior
regional economist with Credit Suisse First Boston in Hong Kong, Agence
France-Presse reported from Shanghai.

Whether the growing U.S. trade deficit with China will be reduced
remains unclear.

Senate passage had been expected since the House of Representatives
approved the bill in May by a vote of 237 to 197.

U.S. companies have been busily preparing themselves to benefit from the
greater access to market sectors from financial services to automobiles
and luxury goods when China joins the WTO. General Motors expects to
generate dollars 2 billion in exports to support China's young but
growing vehicle market over the next five years.

There will be "unprecedented opportunities," John Schachter of the
Business Roundtable said.

Advertisement

\protect\hyperlink{after-bottom}{Continue reading the main story}

\hypertarget{site-index}{%
\subsection{Site Index}\label{site-index}}

\hypertarget{site-information-navigation}{%
\subsection{Site Information
Navigation}\label{site-information-navigation}}

\begin{itemize}
\tightlist
\item
  \href{https://help.nytimes3xbfgragh.onion/hc/en-us/articles/115014792127-Copyright-notice}{©~2020~The
  New York Times Company}
\end{itemize}

\begin{itemize}
\tightlist
\item
  \href{https://www.nytco.com/}{NYTCo}
\item
  \href{https://help.nytimes3xbfgragh.onion/hc/en-us/articles/115015385887-Contact-Us}{Contact
  Us}
\item
  \href{https://www.nytco.com/careers/}{Work with us}
\item
  \href{https://nytmediakit.com/}{Advertise}
\item
  \href{http://www.tbrandstudio.com/}{T Brand Studio}
\item
  \href{https://www.nytimes3xbfgragh.onion/privacy/cookie-policy\#how-do-i-manage-trackers}{Your
  Ad Choices}
\item
  \href{https://www.nytimes3xbfgragh.onion/privacy}{Privacy}
\item
  \href{https://help.nytimes3xbfgragh.onion/hc/en-us/articles/115014893428-Terms-of-service}{Terms
  of Service}
\item
  \href{https://help.nytimes3xbfgragh.onion/hc/en-us/articles/115014893968-Terms-of-sale}{Terms
  of Sale}
\item
  \href{https://spiderbites.nytimes3xbfgragh.onion}{Site Map}
\item
  \href{https://help.nytimes3xbfgragh.onion/hc/en-us}{Help}
\item
  \href{https://www.nytimes3xbfgragh.onion/subscription?campaignId=37WXW}{Subscriptions}
\end{itemize}
