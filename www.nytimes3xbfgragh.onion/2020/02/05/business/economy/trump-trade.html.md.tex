Sections

SEARCH

\protect\hyperlink{site-content}{Skip to
content}\protect\hyperlink{site-index}{Skip to site index}

\href{https://www.nytimes3xbfgragh.onion/section/business/economy}{Economy}

\href{https://myaccount.nytimes3xbfgragh.onion/auth/login?response_type=cookie\&client_id=vi}{}

\href{https://www.nytimes3xbfgragh.onion/section/todayspaper}{Today's
Paper}

\href{/section/business/economy}{Economy}\textbar{}U.S. Trade Deficit
Shrinks, but Not Because Factories Are Returning

\url{https://nyti.ms/36ZrvAc}

\begin{itemize}
\item
\item
\item
\item
\item
\end{itemize}

Advertisement

\protect\hyperlink{after-top}{Continue reading the main story}

Supported by

\protect\hyperlink{after-sponsor}{Continue reading the main story}

\hypertarget{us-trade-deficit-shrinks-but-not-because-factories-are-returning}{%
\section{U.S. Trade Deficit Shrinks, but Not Because Factories Are
Returning}\label{us-trade-deficit-shrinks-but-not-because-factories-are-returning}}

New trade data for 2019 reflects a cooling economy and a year of
aggressive trade clashes, particularly with China.

\href{https://www.nytimes3xbfgragh.onion/by/ana-swanson}{\includegraphics{https://static01.graylady3jvrrxbe.onion/images/2018/12/10/multimedia/author-ana-swanson/author-ana-swanson-thumbLarge.png}}

By \href{https://www.nytimes3xbfgragh.onion/by/ana-swanson}{Ana Swanson}

\begin{itemize}
\item
  Feb. 5, 2020
\item
  \begin{itemize}
  \item
  \item
  \item
  \item
  \item
  \end{itemize}
\end{itemize}

WASHINGTON --- The overall United States trade deficit
\href{https://www.census.gov/foreign-trade/Press-Release/current_press_release/ft900.pdf}{shrank
last year} for the first time in six years as the American economy
cooled, domestic oil production soared and President Trump waged an
aggressive global trade war to
\href{https://www.nytimes3xbfgragh.onion/2020/01/15/business/economy/china-trade-deal.html}{rewrite
America's trading terms.}

The trade deficit for both goods and services fell to **** \$616.8
billion in 2019, down \$10.9 billion from the previous year, according
to
\href{https://www.census.gov/foreign-trade/Press-Release/current_press_release/index.html}{data
released} by the Commerce Department on Wednesday.

Both imports and exports fell as
\href{https://www.nytimes3xbfgragh.onion/2019/09/03/business/economy/manufacturing-economy-slowdown.html}{American
factory activity slowed} and businesses and consumers felt the impact of
tariffs imposed on China, the European Union, Canada, Mexico and other
nations. Total American exports dropped \$1.5 billion to roughly \$2.5
trillion, while imports fell \$12.5 billion to \$3.1 trillion.

Soaring domestic oil production was a major factor in the shrinking
trade deficit, cutting into imports of foreign crude oil by \$30.3
billion last year. Exports of civilian aircraft also fell \$12.6 billion
last year, reflecting the fallout from the deadly crashes of
\href{https://www.nytimes3xbfgragh.onion/2020/01/29/business/boeing-737-max-costs.html}{Boeing's
737 Max airplane}.

But the most dramatic changes in global trade flows occurred with China,
the target of Mr. Trump's biggest economic offensive.

The trade deficit in goods with China shrank \$73.9 billion to \$345.6
billion in 2019. It was the first drop on an annual basis since 2016, as
both the United States and China placed tariffs on hundreds of billions
of dollars of each others' products.

In particular, American imports from China fell sharply in the final two
months of the year, as companies worked to avoid tariffs that Mr. Trump
has placed on \$360 billion worth of Chinese goods and the potential
that he could tax nearly everything from China.

Mr. Trump and his advisers have pointed to trends in trade flows as
evidence that his trade policies are helping to revive factories and
construction sites around the nation.

``This is a blue collar boom,'' Mr. Trump said in the State of the Union
address on Tuesday evening.

But most economists have been skeptical, saying that the country's
factory activity
\href{https://fred.stlouisfed.org/series/INDPRO}{weakened last year},
and that the trade flows largely reflect a cooling American and global
economy.

Economists say the hefty tariffs Mr. Trump has placed on China have
encouraged American consumers to purchase goods from other countries and
have
\href{https://www.nytimes3xbfgragh.onion/2020/01/10/upshot/economy-in-a-nutshell-manufacturing-in-recession-services-booming.html}{not
led to an American manufacturing renaissance}.

``Tariffs to date have clearly had a significant impact on imports from
China,'' said Brad Setser, a senior fellow at the Council on Foreign
Relations. ``They equally clearly have not led to a stronger U.S.
manufacturing sector.''

Rather than bringing manufacturing back to the United States, the clash
with China has caused American companies and consumers to shift
purchases to other countries, like Mexico, Vietnam and South Korea, said
Mary E. Lovely, a senior fellow at the Peterson Institute for
International Economics.

Data released Wednesday morning showed the trade deficit in goods with
Mexico increased \$21.1 billion last year to a record \$101.8 billion,
as the United States brought in more goods from its southern neighbor.
The trade deficit in goods with Canada grew by \$8 billion, while the
gap with Taiwan increased by \$7.8 billion.

``You're going to see this rearrangement of the deck chairs,'' Ms.
Lovely said.

The trade deficit in goods with the European Union also expanded to a
record \$177.9 billion in 2019,
\href{https://www.nytimes3xbfgragh.onion/2020/01/16/business/economy/trump-EU-trade-fight.html}{presaging
Mr. Trump's next conflict}. In recent weeks, Mr. Trump has said that his
attention was shifting to Europe now that he has signed trade deals with
China, Japan, Canada and Mexico.

Mr. Trump has criticized Europe for selling more to the United States
than it buys and has
\href{https://www.nytimes3xbfgragh.onion/2019/06/18/business/ecb-mario-draghi-stimulus.html}{accused
its central bank} of pushing down the value of the euro to make it
easier for European companies to compete against American rivals. His
administration is already imposing tariffs on Europe over airplane
subsidies, and is threatening further levies in response to its digital
taxes and on its cars.

Many economists have predicted that Mr. Trump's trade deal with China
would give businesses more certainty about trading conditions and cause
imports from China to rebound, at least in part, in the coming months.

But the spread of a deadly coronavirus has thrown those predictions into
question. China has shuttered factories, canceled flights and placed
entire cities on lockdown to stop the spread of the virus, weighing
heavily on trade. And China
\href{https://www.nytimes3xbfgragh.onion/2020/02/03/business/economy/coronavirus-china-trade-economy.html}{may
delay} some of its planned purchases of American goods as a result.

Mr. Trump has long pointed to the United States trade deficit --- the
gap between what America exports and what it imports --- as proof that
America is at a competitive disadvantage because of unfair practices by
China and other countries.

In the president's view, American businesses would be making more at
home and consumers would be buying more domestic goods if countries like
China weren't subsidizing their industries and manipulating their
currencies to make their products cheaper.

Some analysts agree with that perspective. Michael Stumo, the chief
executive of the Coalition for a Prosperous America, which has supported
Mr. Trump's trade moves, said the shrinking trade deficit showed that
American consumers were shifting to buying more American-made products,
and that Mr. Trump should make his China tariffs permanent.

``Rebuilding U.S. manufacturing is the single most important step
Washington can take to increase prosperity for America's middle class,''
Mr. Stumo said.

But most economists argue that the trade deficit is a poor metric for
measuring the health of the economy or America's trading relationships.
While a falling trade deficit can be a sign of a growing economy, the
measure can fall for a variety of other reasons, many of them unrelated
to trade and not all of them positive.

Mr. Setser said that a falling trade deficit can sometimes be a sign of
the kind of manufacturing boom that the Trump administration has been
trying to engineer. In that case, American factory production would be
rising, displacing foreign products from the American market and causing
imports to fall and exports to rise.

But that is not the situation the United States finds itself in, he
said. Instead, factory activity has been weak, and both American imports
and exports have contracted, he said.

In addition, tariffs and trade uncertainty appear to have cut into
business investment,
\href{https://www.nytimes3xbfgragh.onion/2020/01/30/business/economy/gdp-numbers.html}{slowing
economic growth}. When petroleum products are excluded, the United
States trade deficit in goods
\href{https://www.census.gov/foreign-trade/Press-Release/current_press_release/exh9.pdf}{actually
rose} compared with the year before.

Speaking at an event at George Washington University on Tuesday, Janet
L. Yellen, the former Federal Reserve chair, said that the bilateral
trade deficit between the United States and China was ``not the proper
focus.''

Ms. Yellen said that Mr. Trump and some of his advisers see the trade
gap ``as a symptom of relationships being unfair.'' But for many
economists, a country's overall trade deficit with the rest of the world
just means that country is spending more than the output it can produce
itself, she said.

``Most economists think that a country's savings and investment are
decisions that aren't affected by trade policy,'' she said.

Economists point to another major reason focusing on the trade deficit
can be misleading: The gap with China is exaggerated because of how the
data is calculated. United States trade data counts the entire value of
a good as coming from the country it was assembled in.

China is still a global center for assembling products like smartphones
and laptops, but many of the components and the technology that goes
into these goods are made elsewhere.

Take a smartphone, for example. A touch screen might be made in Taiwan,
or a microprocessor in South Korea. The chips may come from American
companies like Qualcomm or Texas Instruments, and the product may have
been developed and marketed in the United States.

All of those companies and their employees will receive a share of the
final profits. But if all of those components are assembled in China
before being shipped to the United States, trade statistics will record
the entire value of the phone as being generated in China.

Economists say this method of measurement may exaggerate the trade
deficit with China, perhaps by as much as one-third.

Some analysts do see a victory for the United States in the falling
trade deficit with China: those in Washington who see China as an
increasing national security threat.

China's profits from what it sells to the United States and other
nations helps fund its efforts to expand its influence around the globe,
like its Belt and Road infrastructure building project, activities that
do not benefit the United States, said Derek Scissors, a resident
scholar at the American Enterprise Institute.

``I'd rather put the hard currency in the hands of the South Koreans,
the Vietnamese. And normal economists just don't think that way,'' he
said.

``Economists will say, `Oh great, the president has had success on a
meaningless indicator he made up for political reasons,''' Mr. Scissors
added. ``I agree with that. But I want to trade more with my friends''
and less with dictators, he said.

Jeanna Smialek contributed reporting from Washington.

Advertisement

\protect\hyperlink{after-bottom}{Continue reading the main story}

\hypertarget{site-index}{%
\subsection{Site Index}\label{site-index}}

\hypertarget{site-information-navigation}{%
\subsection{Site Information
Navigation}\label{site-information-navigation}}

\begin{itemize}
\tightlist
\item
  \href{https://help.nytimes3xbfgragh.onion/hc/en-us/articles/115014792127-Copyright-notice}{©~2020~The
  New York Times Company}
\end{itemize}

\begin{itemize}
\tightlist
\item
  \href{https://www.nytco.com/}{NYTCo}
\item
  \href{https://help.nytimes3xbfgragh.onion/hc/en-us/articles/115015385887-Contact-Us}{Contact
  Us}
\item
  \href{https://www.nytco.com/careers/}{Work with us}
\item
  \href{https://nytmediakit.com/}{Advertise}
\item
  \href{http://www.tbrandstudio.com/}{T Brand Studio}
\item
  \href{https://www.nytimes3xbfgragh.onion/privacy/cookie-policy\#how-do-i-manage-trackers}{Your
  Ad Choices}
\item
  \href{https://www.nytimes3xbfgragh.onion/privacy}{Privacy}
\item
  \href{https://help.nytimes3xbfgragh.onion/hc/en-us/articles/115014893428-Terms-of-service}{Terms
  of Service}
\item
  \href{https://help.nytimes3xbfgragh.onion/hc/en-us/articles/115014893968-Terms-of-sale}{Terms
  of Sale}
\item
  \href{https://spiderbites.nytimes3xbfgragh.onion}{Site Map}
\item
  \href{https://help.nytimes3xbfgragh.onion/hc/en-us}{Help}
\item
  \href{https://www.nytimes3xbfgragh.onion/subscription?campaignId=37WXW}{Subscriptions}
\end{itemize}
