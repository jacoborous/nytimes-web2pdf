Sections

SEARCH

\protect\hyperlink{site-content}{Skip to
content}\protect\hyperlink{site-index}{Skip to site index}

\href{https://myaccount.nytimes3xbfgragh.onion/auth/login?response_type=cookie\&client_id=vi}{}

\href{https://www.nytimes3xbfgragh.onion/section/todayspaper}{Today's
Paper}

The Sensuous Pleasures of Handmade Marshmallows

\url{https://nyti.ms/31v2Ouc}

\begin{itemize}
\item
\item
\item
\item
\item
\end{itemize}

Advertisement

\protect\hyperlink{after-top}{Continue reading the main story}

Supported by

\protect\hyperlink{after-sponsor}{Continue reading the main story}

\href{/column/magazine-eat}{Eat}

\hypertarget{the-sensuous-pleasures-of-handmade-marshmallows}{%
\section{The Sensuous Pleasures of Handmade
Marshmallows}\label{the-sensuous-pleasures-of-handmade-marshmallows}}

\includegraphics{https://static01.graylady3jvrrxbe.onion/images/2020/02/09/magazine/09mag-eat/09mag-eat-articleLarge.jpg?quality=75\&auto=webp\&disable=upscale}

By Dorie Greenspan

\begin{itemize}
\item
  Feb. 5, 2020
\item
  \begin{itemize}
  \item
  \item
  \item
  \item
  \item
  \end{itemize}
\end{itemize}

I can't remember the name of the restaurant, or when I had this meal,
which dates it to at least before smartphones and Instagram. I'm pretty
sure the chef was French and the meal was dinner. I'm positive that the
restaurant was very fancy --- fancy enough that the servers wore white
gloves. And fanciful enough that the last of the courses was bonbons of
many colors, wheeled out on a trolley that could have held a royal
wedding cake. There must have been chocolates and lollipops and squares
of \emph{pâte de fruits} (gummy bears' ritzy relatives), but I had eyes
only for the tall, glass, footed apothecary jars with lids shaped like
the Kremlin's pointy domes.

The server set the jars' tops aside, reached in with silver tongs and
lifted a long, chubby white band of powdery marshmallow, snipping off
squares with shiny oversize scissors. As far as I can recall, no one at
the table clapped, but had they, it would have been for glee: It was all
so lovely, a grown-up moment intended to bring back the sweetness of
childhood.

For my French friends around the table, the marshmallows invoked a
nostalgia for a time when candies were made by hand and buying one from
a small shop was a treat. For me, there was no nostalgia, no memory, no
reference; the marshmallows were a revelation: pale, soft, even fragile,
delicately sweet and almost effervescent --- they melted on my tongue.
They were unlike anything I'd known that shared the same name. I wanted
to call them something different.

I had just that thought a few years later when I tasted the marshmallows
Christine Moore created in Pasadena for her Little Flower Candy Company.
Moore started her business by making caramels at home, and she was happy
to make just caramels --- cutting them by hand, wrapping each piece in
wax paper, tying packages with ribbons and labeling them with the logo
she'd drawn at the kitchen table. ``I like making the same thing and
doing it well,'' she told me. But then in 2001, a friend asked her to
bring candies to a fund-raiser in Los Angeles, and while she wanted to
say yes, she knew she couldn't: ``It was impossible for me to afford the
ingredients, and so I made marshmallows, a table full of them.'' Moore's
marshmallows had the same effect on the L.A. chefs who were there as
those lanyards of marshmallows had on me. Soon Moore was making caramels
and marshmallows, still her only offerings.

After tasting Moore's marshmallows and following her recipe many times,
I'm no less surprised by them now. Knowing how they're made only adds to
their allure: Marshmallows are a lesson in the transformative power of
heat and air.

\includegraphics{https://static01.graylady3jvrrxbe.onion/images/2020/02/09/magazine/09mag-eat-03/09mag-eat-03-articleLarge.jpg?quality=75\&auto=webp\&disable=upscale}

While the ancient Egyptians, who are thought to have been the first to
make a mallow sweet, drawing sap from the plant and mixing it with
honey, surely had to stir the stiff mass by hand, I advise a heavy-duty
mixer. Even then, you'll be beating for almost 15 minutes, during which
time you'll be mesmerized by what's happening. The mixture won't look
appealing when you pour boiling sugar syrup over gelatin. But as you
beat, it goes from murky to opaque, from beige to pure white, from thin,
to thicker, to billowing as it mounts, spinning around the wires of the
beater and more than tripling in volume. It peaks and swirls and looks
like meringue, even though there are no eggs. It's beautiful! (It's also
unmanageable until it has had time to set.)

The instant you cut the first marshmallow --- I follow Moore's lead and
cut cubes with crisp right-angled corners --- you know that you've made
something singular. The marshmallow is light, of course --- Moore says
``pillowy.'' It's sturdy enough to stack, sandwich in s'mores or melt in
hot chocolate, but it's tender, with skin as soft as a petal. Press the
candy gently, and it will languidly return to form. Handmade
marshmallows are sensuous, and so much of the delight is in their
texture, which is soft and lithe, almost like custard. They're sweet ---
they're meant to be --- but oddly, they're not \emph{all that} sweet.

I'm a purist: I flavor my marshmallows with vanilla. But the
possibilities for divergence are many. A drop of citrus or herb oil will
give you a different taste and aroma. A drop of natural colorant will
give you any shade you love. And a package of the candies will most
likely spark memories in French friends, probably be a discovery for
Americans and almost certainly turn up on Instagram no matter who gets
them. And yes, I take a picture of them every time, not for fear of
forgetting, but for the pleasure of remembering.

Recipe:
\href{https://cooking.nytimes3xbfgragh.onion/recipes/1020847-vanilla-marshmallows}{Vanilla
marshmallows}

Advertisement

\protect\hyperlink{after-bottom}{Continue reading the main story}

\hypertarget{site-index}{%
\subsection{Site Index}\label{site-index}}

\hypertarget{site-information-navigation}{%
\subsection{Site Information
Navigation}\label{site-information-navigation}}

\begin{itemize}
\tightlist
\item
  \href{https://help.nytimes3xbfgragh.onion/hc/en-us/articles/115014792127-Copyright-notice}{©~2020~The
  New York Times Company}
\end{itemize}

\begin{itemize}
\tightlist
\item
  \href{https://www.nytco.com/}{NYTCo}
\item
  \href{https://help.nytimes3xbfgragh.onion/hc/en-us/articles/115015385887-Contact-Us}{Contact
  Us}
\item
  \href{https://www.nytco.com/careers/}{Work with us}
\item
  \href{https://nytmediakit.com/}{Advertise}
\item
  \href{http://www.tbrandstudio.com/}{T Brand Studio}
\item
  \href{https://www.nytimes3xbfgragh.onion/privacy/cookie-policy\#how-do-i-manage-trackers}{Your
  Ad Choices}
\item
  \href{https://www.nytimes3xbfgragh.onion/privacy}{Privacy}
\item
  \href{https://help.nytimes3xbfgragh.onion/hc/en-us/articles/115014893428-Terms-of-service}{Terms
  of Service}
\item
  \href{https://help.nytimes3xbfgragh.onion/hc/en-us/articles/115014893968-Terms-of-sale}{Terms
  of Sale}
\item
  \href{https://spiderbites.nytimes3xbfgragh.onion}{Site Map}
\item
  \href{https://help.nytimes3xbfgragh.onion/hc/en-us}{Help}
\item
  \href{https://www.nytimes3xbfgragh.onion/subscription?campaignId=37WXW}{Subscriptions}
\end{itemize}
