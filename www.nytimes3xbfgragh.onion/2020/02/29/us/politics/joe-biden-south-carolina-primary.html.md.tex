Sections

SEARCH

\protect\hyperlink{site-content}{Skip to
content}\protect\hyperlink{site-index}{Skip to site index}

\href{https://www.nytimes3xbfgragh.onion/section/politics}{Politics}

\href{https://myaccount.nytimes3xbfgragh.onion/auth/login?response_type=cookie\&client_id=vi}{}

\href{https://www.nytimes3xbfgragh.onion/section/todayspaper}{Today's
Paper}

\href{/section/politics}{Politics}\textbar{}Winning South Carolina,
Biden Makes Case Against Sanders: `Win Big or Lose'

\url{https://nyti.ms/2I9L2Ed}

\begin{itemize}
\item
\item
\item
\item
\item
\item
\end{itemize}

\begin{itemize}
\item
  \href{https://www.nytimes3xbfgragh.onion/live/2020/09/07/us/trump-vs-biden?action=click\&pgtype=Article\&state=default\&region=TOP_BANNER\&context=storylines_menu}{Election
  Updates}
\item
  \href{https://www.nytimes3xbfgragh.onion/interactive/2020/us/elections/election-states-biden-trump.html?action=click\&pgtype=Article\&state=default\&region=TOP_BANNER\&context=storylines_menu}{Paths
  to 270}
\item
  \href{https://www.nytimes3xbfgragh.onion/interactive/2020/08/31/us/politics/vote-by-mail-deadlines.html?action=click\&pgtype=Article\&state=default\&region=TOP_BANNER\&context=storylines_menu}{Voting
  by Mail}
\item
  \href{https://www.nytimes3xbfgragh.onion/interactive/2019/us/elections/2020-presidential-election-calendar.html?action=click\&pgtype=Article\&state=default\&region=TOP_BANNER\&context=storylines_menu}{Key
  Dates}
\item
  \href{https://www.nytimes3xbfgragh.onion/newsletters/politics?action=click\&pgtype=Article\&state=default\&region=TOP_BANNER\&context=storylines_menu}{Politics
  Newsletter}
\end{itemize}

Advertisement

\protect\hyperlink{after-top}{Continue reading the main story}

Supported by

\protect\hyperlink{after-sponsor}{Continue reading the main story}

\hypertarget{winning-south-carolina-biden-makes-case-against-sanders-win-big-or-lose}{%
\section{Winning South Carolina, Biden Makes Case Against Sanders: `Win
Big or
Lose'}\label{winning-south-carolina-biden-makes-case-against-sanders-win-big-or-lose}}

Joseph R. Biden Jr. drew on his decades-long relationships and leveraged
his close bond with black voters to wrap up a state long considered his
stronghold.

\includegraphics{https://static01.graylady3jvrrxbe.onion/images/2020/03/01/us/politics/29SUBJPsc-ledeall-print/29sc-ledeall-top-videoSixteenByNine3000.jpg}

\href{https://www.nytimes3xbfgragh.onion/by/jonathan-martin}{\includegraphics{https://static01.graylady3jvrrxbe.onion/images/2018/11/06/multimedia/author-jonathan-martin/author-jonathan-martin-thumbLarge.png}}\href{https://www.nytimes3xbfgragh.onion/by/alexander-burns}{\includegraphics{https://static01.graylady3jvrrxbe.onion/images/2018/09/25/multimedia/author-alexander-burns/author-alexander-burns-thumbLarge-v2.png}}

By \href{https://www.nytimes3xbfgragh.onion/by/jonathan-martin}{Jonathan
Martin} and
\href{https://www.nytimes3xbfgragh.onion/by/alexander-burns}{Alexander
Burns}

\begin{itemize}
\item
  Feb. 29, 2020
\item
  \begin{itemize}
  \item
  \item
  \item
  \item
  \item
  \item
  \end{itemize}
\end{itemize}

COLUMBIA, S.C. --- Joseph R. Biden Jr. scored a decisive victory in
\href{https://www.nytimes3xbfgragh.onion/live/2020/south-carolina-primary-02-29}{the
South Carolina primary} on Saturday, reviving his listing campaign and
establishing himself as the leading contender to slow Senator Bernie
Sanders as the turbulent Democratic race turns to a slew of
coast-to-coast contests on Tuesday.

Propelled by an outpouring of support from South Carolina's
African-American voters, Mr. Biden easily overcame a late effort by Mr.
Sanders to stage an upset. The victory in a state long seen as his
firewall will vault Mr. Biden into Super Tuesday, where polls open in
just over 48 hours, as the clear alternative to Mr. Sanders for
establishment-aligned Democrats.

Mr. Biden, the former vice president, captured just under 50 percent of
the vote, well ahead of Mr. Sanders, who had 20 percent. Tom Steyer, the
California billionaire, was a distant third, followed by Pete Buttigieg
and Senator Elizabeth Warren of Massachusetts. The victory enabled Mr.
Biden to significantly narrow Mr. Sanders's pledged delegate lead, but
he did not appear poised to overtake him.

Mr. Biden, in an exuberant victory speech on Saturday night, looked
ahead to a long, ideological struggle and made repeated arguments
against Mr. Sanders, though not by name.

He said voters faced a momentous choice in the coming days. Democrats,
Mr. Biden argued, wanted results rather than revolution, improvements to
the Affordable Care Act rather than a disruptive transformation of the
health care system, and a candidate who would ``take on the N.R.A. and
gun manufacturers and not protect them.''

``If Democrats want a nominee who's a Democrat, a lifelong Democrat, a
proud Democrat, an Obama-Biden Democrat, join us,'' Mr. Biden said,
adding, ``We have the option of winning big or losing big. That's the
choice.''

As much as the results here offered new life to Mr. Biden, the one-time
front-runner, they dealt a perhaps fatal blow to two moderates, Mr.
Buttigieg, the former mayor of South Bend, Ind., and Senator Amy
Klobuchar. Both had been hoping to overtake Mr. Biden as the candidate
of the party's center, but again proved unable to win nonwhite voters;
Mr. Buttigieg received only 2 percent of support from black voters,
according to early exit polls.

Perhaps even more consequentially, Mr. Biden's triumph here also
increased pressure on Michael R. Bloomberg to best Mr. Biden in the 15
states and territories voting Tuesday --- or consider exiting the race.

Ms. Warren, a progressive rival to Mr. Sanders, also showed no strong
appeal to African-American voters in the Republican-leaning state. But
unlike the moderate candidates, Ms. Warren was unlikely to face similar
pressure to make way for Mr. Biden, and some party leaders hope she will
stay in the race and complicate Mr. Sanders's efforts to consolidate the
left.

Mr. Biden also overcame a challenge from Mr. Steyer, a former hedge fund
investor who poured millions of dollars into courting black voters, and
in some cases putting influential state lawmakers on his campaign
payroll. But Mr. Steyer fell far short of the breakthrough his campaign
believed was possible, and several hours after the polls closed he
\href{https://www.nytimes3xbfgragh.onion/2020/02/29/us/politics/tom-steyer-drops-out.html}{dropped
out of the race}.

For Mr. Biden, 77, the victory here was a moment to savor.

Low on cash and without a victory in the first three contests, Mr. Biden
desperately needed South Carolina, a state for which he has long had a
personal affection, to resurrect his third and perhaps final quest for
the presidency.

\includegraphics{https://static01.graylady3jvrrxbe.onion/images/2020/02/29/us/politics/29sc-ledeall-crowd/merlin_169806657_1966e1a6-8c2a-400d-b896-cea09d76f0a6-articleLarge.jpg?quality=75\&auto=webp\&disable=upscale}

Facing a humiliating fifth-place finish in New Hampshire earlier this
month, Mr. Biden flew out of the New England cold before the polls had
even closed there and effectively staked his campaign on South Carolina,
telling supporters in Columbia that evening that he was counting on the
state's more racially diverse set of voters to offset his dismal showing
in the first two states, both heavily white.

Then, after finishing a distant second to Mr. Sanders in Nevada, he came
directly to South Carolina. He campaigned almost exclusively here while
other Democrats fanned out across the much larger map of states that
vote Tuesday.

In the debate this week, Mr. Biden promised to win South Carolina and
projected confidence that he would prevail with African-Americans. He
did both, claiming black voters with 64 percent, far better than Mr.
Sanders's 15 percent, exit polls showed.

The results here represented at least an interruption of what had loomed
as a march to the nomination by Mr. Sanders. South Carolina was the
first state where Mr. Sanders did not finish at the top, and his distant
second to Mr. Biden came even after he had made a late effort to score a
win.

Though Mr. Biden had led in every poll of South Carolina, Mr. Sanders,
after winning in a landslide in Nevada, decided to try to deliver a
finishing blow against Mr. Biden. Mr. Sanders increased his television
advertising in the state and intensified his campaign schedule, with the
goal of denying Mr. Biden the chance to reignite his candidacy and
perhaps wrapping up the nomination fight by the middle of March.

Addressing supporters in Virginia, Mr. Sanders, 78, acknowledged Mr.
Biden's success in South Carolina and advised his audience to prepare
for the ups and downs of a long campaign. ``That will not be the only
defeat,'' Mr. Sanders said of South Carolina. ``There are a lot of
states in this country, nobody wins them all.''

But ticking off his victories so far, Mr. Sanders also pointed in a
confident tone toward Tuesday's primaries as the next frontier.

Ms. Warren, at a rally in Houston, also looked ahead to those contests.
``I'll be the first to say that the first four contests haven't gone
exactly as I'd hoped,'' she told supporters. ``But Super Tuesday is
three days away and we're looking forward to gaining as many delegates
to the convention as we can.''

Having carried South Carolina as a kind of favorite-son candidate, Mr.
Biden is counting on that result to ripple throughout the region and
help him recover some of the support from black voters elsewhere that he
lost in recent months, largely to Mr. Bloomberg. He needs voters to
shift back in his direction quickly if he is to edge ahead of Mr.
Sanders in enough states to deliver a strong showing on Super Tuesday.

But absent an overwhelming wave of new support for Mr. Biden, the
best-case scenario for his campaign may still be a daunting one: a
monthslong battle against a tireless opponent with superior financial
and organizational resources at his disposal, and a formidable well of
support from the Democratic Party's left wing.

Mr. Sanders has had a weekslong head start in a number of key Super
Tuesday states where early voting has long been underway, including
California, which on its own could give Mr. Sanders a sizable lead in
the national delegate count.

Mr. Biden is also likely to face pressure and scrutiny from his fellow
Democrats of a kind he has not received in weeks, in a test of whether a
candidate who has spent most of the race grasping for his political
footing can achieve sustained momentum for the first time since voting
began.

Even though --- or perhaps because --- he has been favored all along to
win the South Carolina primary, Mr. Biden drew virtually no attacks from
other candidates in the run-up to Saturday's vote.

In South Carolina, Mr. Biden wielded two powerful assets: longstanding
relationships and a direct connection to Mr. Obama, who is beloved by
black voters.

Mr. Biden was also aided immensely by his close bond with Representative
James E. Clyburn, the highest-ranking African-American in Congress and
the most influential Democrat in South Carolina.

After months of remaining neutral, Mr. Clyburn offered Mr. Biden a
full-throated endorsement on Wednesday before a bank of television
cameras and photographers. On Saturday, nearly 50 percent of South
Carolina voters said Mr. Clyburn's support was an important factor in
their decision, according to exit polls.

Even more crucial to Mr. Biden was his service under the nation's first
black president, a relationship that earned him a reservoir of good will
in a state where 56 percent of the Democratic electorate on Saturday was
African-American, according to exit polls.

``He was Obama's vice president and he stuck by him,'' said Luther
Johnson, a Columbia resident who came to see Mr. Biden at a black-owned
barbershop on Friday.

Image

Moderate candidates are seeking to slow the rise of Senator Bernie
Sanders before Super Tuesday this week.Credit...Erin Schaff/The New York
Times

Mr. Biden was noticeably more at ease as he wound his way through South
Carolina's churches, barbershops and barbecue joints than he had been in
Iowa and New Hampshire. As he likes to remind people here, he has
vacationed in the state's Lowcountry for decades and, as a young senator
mourning the death of his first wife, forged a close friendship with
Ernest F. Hollings, South Carolina's long-serving senator.

But Mr. Biden did not last long enough in his first two presidential
campaigns to make it to South Carolina --- Saturday marked his first win
there, and his first primary victory anywhere, in his three White House
bids.

South Carolina is the second consecutive diverse nominating state in
which the share of white voters casting ballots was higher this year
than it had been in the 2016 primaries --- the latest evidence that
President Trump has nudged some Republicans and independents into the
Democratic column.

More people in South Carolina voted in this primary than in any other in
its history, breaking the record set in the 2008 nominating contest.

Mr. Biden's back-against-the-wall victory was in keeping with South
Carolina's tradition of turning around presidential campaigns. George W.
Bush in 2000, Mr. Obama in 2008 and Hillary Clinton in 2016 all revived
their candidacies in the state after losing decisively in New Hampshire.

Mr. Biden's win on Saturday is no guarantee he will be catapulted to the
nomination in the same fashion. Even as voters were going to the polls
on Saturday, Mr. Clyburn offered a blistering assessment of Mr. Biden's
organization.

``We will have to sit down and get serious about how we retool this
campaign,'' the lawmaker said on CNN, adding: ``I'm not going to sit
back idly and watch people mishandle this campaign.''

Mr. Biden's operations
\href{https://www.nytimes3xbfgragh.onion/2020/02/26/us/politics/joe-biden-california-super-tuesday.html?searchResultPosition=1}{have
been sorely lacking in Super Tuesday states}, local Democrats say. Mr.
Sanders is poised to rack up hundreds of delegates that day, including
in large states like California, and Mr. Bloomberg and Ms. Warren are
also in contention to claim delegates.

Mr. Buttigieg is also hoping to be competitive, but on a conference call
on Saturday his campaign aides declined to say how many delegates he
would need to win to remain viable.

``We really believe this is about limiting Senator Sanders's lead and
making sure that it is possible for an opposing candidate to close the
gap in the remaining states that become more friendly,'' said Michael
Halle, a senior adviser to Mr. Buttigieg.

And in a sign that Mr. Bloomberg had no intention of yielding to a
potential Biden comeback, his campaign announced on Saturday that it had
purchased a lengthy block of airtime on multiple national networks for a
three-minute commercial on Sunday, styled as an address by Mr. Bloomberg
to the American people about the looming threat of the coronavirus.

Mr. Bloomberg's campaign manager, Kevin Sheekey, responded to the South
Carolina primary with a statement stressing that the candidate had not
yet appeared on any primary ballot, and nodding toward the national
scope of the primary on Tuesday.

``Mike is the only candidate to campaign in all fourteen Super Tuesday
states over the last two months and we look forward to Tuesday,'' Mr.
Sheekey said.

\hypertarget{our-2020-election-guide}{%
\section{Our 2020 Election Guide}\label{our-2020-election-guide}}

Updated ~Sept. 7, 2020

\begin{center}\rule{0.5\linewidth}{\linethickness}\end{center}

\begin{itemize}
\item ~
  \hypertarget{the-latest}{%
  \subsection{The Latest}\label{the-latest}}

  \begin{itemize}
  \item
    The unofficial Labor Day kickoff to the fall presidential campaign
    centered on Pennsylvania and Wisconsin,
    \href{https://www.nytimes3xbfgragh.onion/2020/09/07/us/politics/wisconsin-biden-harris-trump-pence.html?action=click\&pgtype=Article\&state=default\&region=BELOW_MAIN_CONTENT\&context=storylines_guide}{two
    pivotal states for both President Trump and Joseph R. Biden Jr}.
  \end{itemize}
\item ~
  \hypertarget{how-to-win-270}{%
  \subsection{How to Win 270}\label{how-to-win-270}}

  \begin{itemize}
  \item
    Joe Biden and Donald Trump need 270 electoral votes to reach the
    White House. Try building
    \href{https://www.nytimes3xbfgragh.onion/interactive/2020/us/elections/election-states-biden-trump.html?action=click\&pgtype=Article\&state=default\&region=BELOW_MAIN_CONTENT\&context=storylines_guide}{your
    own coalition of battleground states}~to see potential outcomes.
  \end{itemize}
\item ~
  \hypertarget{voting-by-mail}{%
  \subsection{Voting by Mail}\label{voting-by-mail}}

  \begin{itemize}
  \item
    Will you have enough time to vote by mail in your state? Yes, but
    it's risky to procrastinate.
    \href{https://www.nytimes3xbfgragh.onion/interactive/2020/08/31/us/politics/vote-by-mail-deadlines.html?action=click\&pgtype=Article\&state=default\&region=BELOW_MAIN_CONTENT\&context=storylines_guide}{Check
    your state's deadline.}
  \item
    \href{https://www.nytimes3xbfgragh.onion/interactive/2020/us/elections/joe-biden.html?action=click\&pgtype=Article\&state=default\&region=BELOW_MAIN_CONTENT\&context=storylines_guide}{}

    \hypertarget{joe-biden}{%
    \section{Joe Biden}\label{joe-biden}}

    \hypertarget{democrat}{%
    \subsection{Democrat}\label{democrat}}

    \href{https://www.nytimes3xbfgragh.onion/interactive/2020/us/elections/donald-trump.html?action=click\&pgtype=Article\&state=default\&region=BELOW_MAIN_CONTENT\&context=storylines_guide}{}

    \hypertarget{donald-trump}{%
    \section{Donald Trump}\label{donald-trump}}

    \hypertarget{republican}{%
    \subsection{Republican}\label{republican}}
  \end{itemize}
\item
  \hypertarget{keep-up-with-our-coverage}{%
  \subsection{Keep Up With Our
  Coverage}\label{keep-up-with-our-coverage}}

  \begin{itemize}
  \item
    Get an
    \href{https://www.nytimes3xbfgragh.onion/newsletters/politics?action=click\&pgtype=Article\&state=default\&region=BELOW_MAIN_CONTENT\&context=storylines_guide}{email}~recapping
    the day's news
  \item
    Download our mobile app on
    \href{https://apps.apple.com/us/app/nytimes/id284862083?ls=1\&mat_click_id=5c79ae7455014fd1bd66b5610c05b8f2-20191112-16948\&referrer=mat_click_id\%3D5c79ae7455014fd1bd66b5610c05b8f2-20191112-16948\%26link_click_id\%3D722930677036718082}{iOS}~and
    \href{http://a.localytics.com/android?id=com.nytimes.android\&referrer=utm_source\%3Dother_nyt_mobile_web\%26utm_medium\%3DWeb\%2520page\%26utm_term\%3DGeneral\%2520Mobile\%2520Page\%26utm_campaign\%3DNYT\%2520Mobile\%2520General\%2520Page}{Android}~and
    turn on Breaking News and Politics alerts
  \end{itemize}
\end{itemize}

Advertisement

\protect\hyperlink{after-bottom}{Continue reading the main story}

\hypertarget{site-index}{%
\subsection{Site Index}\label{site-index}}

\hypertarget{site-information-navigation}{%
\subsection{Site Information
Navigation}\label{site-information-navigation}}

\begin{itemize}
\tightlist
\item
  \href{https://help.nytimes3xbfgragh.onion/hc/en-us/articles/115014792127-Copyright-notice}{©~2020~The
  New York Times Company}
\end{itemize}

\begin{itemize}
\tightlist
\item
  \href{https://www.nytco.com/}{NYTCo}
\item
  \href{https://help.nytimes3xbfgragh.onion/hc/en-us/articles/115015385887-Contact-Us}{Contact
  Us}
\item
  \href{https://www.nytco.com/careers/}{Work with us}
\item
  \href{https://nytmediakit.com/}{Advertise}
\item
  \href{http://www.tbrandstudio.com/}{T Brand Studio}
\item
  \href{https://www.nytimes3xbfgragh.onion/privacy/cookie-policy\#how-do-i-manage-trackers}{Your
  Ad Choices}
\item
  \href{https://www.nytimes3xbfgragh.onion/privacy}{Privacy}
\item
  \href{https://help.nytimes3xbfgragh.onion/hc/en-us/articles/115014893428-Terms-of-service}{Terms
  of Service}
\item
  \href{https://help.nytimes3xbfgragh.onion/hc/en-us/articles/115014893968-Terms-of-sale}{Terms
  of Sale}
\item
  \href{https://spiderbites.nytimes3xbfgragh.onion}{Site Map}
\item
  \href{https://help.nytimes3xbfgragh.onion/hc/en-us}{Help}
\item
  \href{https://www.nytimes3xbfgragh.onion/subscription?campaignId=37WXW}{Subscriptions}
\end{itemize}
