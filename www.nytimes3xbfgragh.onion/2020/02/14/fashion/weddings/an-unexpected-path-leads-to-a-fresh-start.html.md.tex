Sections

SEARCH

\protect\hyperlink{site-content}{Skip to
content}\protect\hyperlink{site-index}{Skip to site index}

\href{https://www.nytimes3xbfgragh.onion/section/fashion/weddings}{Love}

\href{https://myaccount.nytimes3xbfgragh.onion/auth/login?response_type=cookie\&client_id=vi}{}

\href{https://www.nytimes3xbfgragh.onion/section/todayspaper}{Today's
Paper}

\href{/section/fashion/weddings}{Love}\textbar{}An Unexpected Path Leads
to a Fresh Start

\url{https://nyti.ms/39xcUO1}

\begin{itemize}
\item
\item
\item
\item
\item
\end{itemize}

Advertisement

\protect\hyperlink{after-top}{Continue reading the main story}

Supported by

\protect\hyperlink{after-sponsor}{Continue reading the main story}

VOWS

\hypertarget{an-unexpected-path-leads-to-a-fresh-start}{%
\section{An Unexpected Path Leads to a Fresh
Start}\label{an-unexpected-path-leads-to-a-fresh-start}}

An elopement in Moab, Utah, provided a wedding site as big as all
outdoors for Hailey Moore and Kristopher Hansen.

\includegraphics{https://static01.graylady3jvrrxbe.onion/images/2020/02/16/fashion/14VOWS-MoabElop1/merlin_168348378_b89cc2ed-243c-4498-aacf-2c5c09918a17-articleLarge.jpg?quality=75\&auto=webp\&disable=upscale}

By Lois Smith Brady

\begin{itemize}
\item
  Feb. 14, 2020
\item
  \begin{itemize}
  \item
  \item
  \item
  \item
  \item
  \end{itemize}
\end{itemize}

At 6:30 a.m. on Jan. 31, Hailey Moore and Dr. Kristopher Hansen stood in
the dark at the Mesa Arch trailhead inside Canyonlands National Park
near Moab, Utah. The half-mile to the arch was snow-covered and wound
past prickly bushes, boulders, frozen puddles, cactuses and
dead-but-still-standing trees.

Ms. Moore, 40, and Dr. Hansen, 38, were eloping, with no guests. They
resembled strange birds that had flown far from their natural habitat.
He wore blue suit pants, a gray vest, a perfectly pressed white shirt
and carried a blue jacket on a hanger while she wore a lace wedding gown
with a long train. The bride would later say the hike was like
participating in a ``trash the dress'' photo shoot.

Their only companions were Jess and Austin Drawhorn, a married team of
photographers who are based in Boulder, Colo., and specialize in what
they call adventure elopements. The Drawhorns, who were dressed like
backcountry hikers, wore headlamps and handed the bride and groom metal
cleats to pull over their shoes. ``The most important thing we can do
today is not fall off a cliff,'' Ms. Drawhorn said.

Dr. Hansen replied, ``I'm terribly afraid of heights so you don't have
to worry.''

\includegraphics{https://static01.graylady3jvrrxbe.onion/images/2020/02/16/fashion/14VOWS-MoabElop2/merlin_168348258_2dac00a6-b7d6-4b6d-8803-5899801469e6-articleLarge.jpg?quality=75\&auto=webp\&disable=upscale}

The couple, who told their families they were eloping, met Jan. 29, 2019
in the Java Hutt Cafe inside Carilion Roanoke Memorial Hospital in
Roanoke, Va. Dr. Hansen, an oncologist and hematologist, was there for a
medical conference. Ms. Moore, who was working as a hospice nurse at the
time, was wearing scrubs and getting ready to meet the family of a
patient.

They were standing next to each other in line. ``I thought she was
absolutely gorgeous,'' he said. ``I thought I did not stand a chance,
but I made casual chitchat with the hopes of pulling something out of
it. She's a very strong woman and came across as such. That was both
scary and attractive.''

He did manage to give her his business card but figured he'd never hear
from her again. Then, that night, she texted him: ``Hey, it's Hailey.''

He remembers thinking: ``This is the confident woman from the coffee
shop.'' He added, ``I was doing cartwheels and trying to keep myself
calm and collected.'' Dr. Hansen comes across as thoughtful,
approachable and funny, a formal physician who likes to wear extremely
informal socks. Describing his sock collection, he said, ``I have
flamingoes. I've got piñata socks on right now. I have palm tree ones.
It's a talking point with work. It's something to talk about in the room
to break the ice, a less serious conversation.''

He and Ms. Moore soon met for dinner at Table 50, a restaurant in
Roanoke where she lived at the time. (He lived in nearby Blacksburg.)
They talked about medicine and their previous relationships. Both had
been married before and divorced, neither happily.

Image

Jess and Austin Drawhorn, who are married photographers, planned the
adventure elopement for the couple.~Credit...Barton Glasser for The New
York Times

``I was pretty hellbent on being single for life and I'm pretty sure
Kris was, too,'' Ms. Moore said. ``When we met, right out of the gate,
it was like, `Just so you know, I think you're cool but there's no way
I'd ever get married again.''' She was especially guarded because unlike
him, she has children from her previous marriage.

``We both had huge walls around ourselves, a force field if you will,''
he said.

So, they proceeded gingerly. Soon after they met, she became a travel
nurse, working temporarily at various hospitals around Virginia. ``In
the very beginning, he used to FaceTime me, and play the guitar for me
when I was away on assignments,'' she said. ``That just melted me.''

He sometimes drove to visit her on his days off. ``She was in Richmond,
she was in Culpeper,'' he said. ``She would work the night shift and I
would wait until she got off in the morning and then we'd go to the
Waffle House.''

They'd talk for hours. ``He's humble,'' she said. ``I've been a nurse
for 11 years and I always said I'd never date a doctor. They're full of
themselves, you know. I told him, `You're totally different. You're
still nice and normal.'''

Two months after they met, they took a trip to Florida. ``We had such a
good time,'' she said. ``I was like, `We're going to be really close
buddies and travel.' Then, you find yourself thinking about them all
day. You can't \emph{not} think about them.''

Image

The couple began married life 2,000 feet above a gooseneck in the
Colorado River, at Dead Horse Point in Moab, Utah.~Credit...Barton
Glasser for The New York Times

In Florida, on a whim, they got matching tattoos of flying birds. ``We
got along so well, the other got through the force field much quicker
than either of us anticipated,'' he said.

Early on, they had serious discussions about the future while still
promising each other: You can walk away whenever you want.

``A lot of conversations revolved around, `Listen, this is what I have
planned for my future and if that's not in line with your future, let's
be realistic and not inflict harm,''' he said.

It turned out their long-term goals and interests were aligned in many
ways. Both want to travel as much as possible, partly because they are
aware that good health is precious and temporary and even long lives are
short. Both are readers. She has an eclectic collection of old hardcover
books and she's always on the hunt for more. ``Goodwill is my hot
spot,'' she said. Crucially, Dr. Hansen told her he wants children and
she said she was open to that possibility.

\emph{{[}}\href{https://www.nytimes3xbfgragh.onion/newsletters/love-letter?module=inline}{\emph{Sign
up for Love Letter and always get the latest in Modern Love, weddings,
and relationships in the news by email.}}\emph{{]}}

By summer, they were much less resistant to the idea of marriage.
``Things changed for me,'' she said. ``It's just a fluttering feeling
you get in your chest when you've been looking at the same person for
six months. It wasn't really chest pain but a fluttering.''

In August, they went scuba diving in Aruba. He's an experienced diver
while she gets nervous underwater. ``We did two amazing dives,'' she
said. ``One was a shipwreck and one was a huge coral reef. Every time, I
looked to my right or my left and there he was watching me. He never
left my side once. It's the same in life. I won't notice he's there but
he's right there.''

Image

The bride said the hike was like participating in a ``trash the dress''
photo shoot.Credit...Barton Glasser for The New York Times

By the end of that trip, she was hoping he'd ask her to marry him. ``I
thought, I really love this man,'' she said. ``That's it. He broke
through all my walls.'' She added that the word love does not adequately
describe their connection. ``What we feel is so much bigger than that
small word.''

On Nov. 26, while they were looking at the ornately decorated Christmas
trees inside Hotel Roanoke, he knelt down and said, ``Do you want to do
this?''

They recently moved to Grand Junction, Colo., where he begins work this
month at St. Mary's Medical Center and she is still figuring out her
next career move. They decided to elope partly because, as Ms. Moore
said, their whole relationship has had the ``spontaneous, over-the-top
and intimate'' spirit of an elopement.

On the morning of the wedding, in the bobbing light of the Drawhorns'
headlamps, the couple slipped, laughed and held on to each other as they
made their way to Mesa Arch. The bride's long train snagged on a branch,
then a rock. ``Oh, this thing!'' she yelped at one point. ``Where's the
scissors?''

At the arch, the groom put on his blue jacket and changed into nicer
dress shoes he'd carried in a backpack. He was wearing special wedding
socks, bright blue ones decorated with otters swimming in pairs. (At the
online store where he found them, they were called Significant Otter
socks.)

Mesa Arch is a broad, low-lying, sandstone arch atop a cliff overlooking
the desert floor. Just peering over the edge gives you the feeling of
dropping on a roller coaster. There are stone towers and pinnacles
stationed like sentinels throughout the desert; rock formations that
resemble half-melted sand castles; cracked and weather-beaten buttes;
and in the far distance, snow-covered mountains.

Image

On their first day of marriage, the couple watched the sun rise across
the canyons at 7:25 a.m., which was an hour after their hike to the
point had begun.Credit...Barton Glasser for The New York Times

The couple wanted to watch the sunrise at the arch before saying their
vows. At first, a narrow strip of bright orange, like eyeliner, appeared
on the horizon. Then, as the sun rose, it lit up the entire arch to the
point where it glowed almost as red as embers.

``Ready baby?'' the groom said.

For the actual ceremony, the couple and the Drawhorns drove about 20
minutes to Dead Horse Point State Park where the view is even more
incredible. Dead Horse Point overlooks the Colorado River at a spot
where the river makes a series of big, winding, giant slalom-like turns
through canyon walls. There were no tourists, no wind, nothing but rocks
and earth.

The bride and groom passed by the lookout where there's a safety railing
and instead chose to say their vows, which they wrote separately, on an
unprotected ledge. (Imagine a rocky diving platform with a drop-off of
2,000 feet.) ``It's not too late to back out!'' the bride exclaimed
before reading hers.

Ms. Drawhorn, who became an ordained minister online through Open
Ministry, doubled as the officiant. She read a passage cobbled together
from the writings of Carl Sagan, the science writer and cosmologist,
that included the line: ``From within one of the billions of species to
walk upon this tiny speck in the universe, we find two imperfect people
who want to share their short lives.''

At the end of the ceremony, the groom looked at the bride and said,
``Gotcha!''

Then, they scrambled up a nearby snow-covered boulder that was even
closer to the edge. The bride seemed unfazed while the groom was
covering his eyes with his hands and refusing to look down. ``A thousand
years go by and I'll be the one guy standing here when this thing
shifts,'' he said. ``I'll get a trailhead named after me.''

\begin{center}\rule{0.5\linewidth}{\linethickness}\end{center}

\hypertarget{on-this-day}{%
\subsubsection{On This Day}\label{on-this-day}}

\textbf{When} Jan. 31, 2020

\textbf{Where} Mesa Arch trailhead, Canyonlands National Park near Moab,
Utah

\textbf{The Reception} The couple did not have a reception, but they did
celebrate afterward in their own way. ``We went to Starbucks
immediately,'' Dr. Hansen said. ``It was so early and I didn't get my
coffee.''

\textbf{Adventure Elopements} The Drawhorns are part of a trend in the
West of young couples specializing in photographing elopements that
sometimes require rock climbing, kayaking or canyoneering skills to get
to the location of the wedding. Some of the photographers live in vans
or Airstreams, and travel from elopement to elopement. Many post on
Instagram: @thedrawhorns, @thehearnes, @thefoxes, @cedarandpines,
@vowofthewild are few examples.

\textbf{Why Elope?} It's inexpensive and it's just the two of you. The
Drawhorns eloped in Rocky Mountain National Park in Colorado and had a
backyard reception later at a total cost of \$2,000. While elopements in
the past may have involved couples running away, usually to Las Vegas
and sometimes for some scandalous reason, that's changed. Couples want a
fun, intense experience. ``Now, couples are opting into it,'' Mr.
Drawhorn said. ``They're choosing it.''

\emph{Continue following our fashion and lifestyle coverage on Facebook
(}\href{https://www.facebookcorewwwi.onion/nytimesstyles}{\emph{Styles}}
\emph{and}
\href{https://www.facebookcorewwwi.onion/modernlove}{\emph{Modern
Love}}\emph{), Twitter
(}\href{https://twitter.com/nytstyles}{\emph{Styles}}\emph{,}
\href{https://twitter.com/nytfashion}{\emph{Fashion}} \emph{and}
\href{https://twitter.com/nytimesvows}{\emph{Weddings}}\emph{) and}
\href{https://instagram.com/nytimesfashion}{\emph{Instagram}}\emph{.}

Advertisement

\protect\hyperlink{after-bottom}{Continue reading the main story}

\hypertarget{site-index}{%
\subsection{Site Index}\label{site-index}}

\hypertarget{site-information-navigation}{%
\subsection{Site Information
Navigation}\label{site-information-navigation}}

\begin{itemize}
\tightlist
\item
  \href{https://help.nytimes3xbfgragh.onion/hc/en-us/articles/115014792127-Copyright-notice}{©~2020~The
  New York Times Company}
\end{itemize}

\begin{itemize}
\tightlist
\item
  \href{https://www.nytco.com/}{NYTCo}
\item
  \href{https://help.nytimes3xbfgragh.onion/hc/en-us/articles/115015385887-Contact-Us}{Contact
  Us}
\item
  \href{https://www.nytco.com/careers/}{Work with us}
\item
  \href{https://nytmediakit.com/}{Advertise}
\item
  \href{http://www.tbrandstudio.com/}{T Brand Studio}
\item
  \href{https://www.nytimes3xbfgragh.onion/privacy/cookie-policy\#how-do-i-manage-trackers}{Your
  Ad Choices}
\item
  \href{https://www.nytimes3xbfgragh.onion/privacy}{Privacy}
\item
  \href{https://help.nytimes3xbfgragh.onion/hc/en-us/articles/115014893428-Terms-of-service}{Terms
  of Service}
\item
  \href{https://help.nytimes3xbfgragh.onion/hc/en-us/articles/115014893968-Terms-of-sale}{Terms
  of Sale}
\item
  \href{https://spiderbites.nytimes3xbfgragh.onion}{Site Map}
\item
  \href{https://help.nytimes3xbfgragh.onion/hc/en-us}{Help}
\item
  \href{https://www.nytimes3xbfgragh.onion/subscription?campaignId=37WXW}{Subscriptions}
\end{itemize}
