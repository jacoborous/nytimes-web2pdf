Sections

SEARCH

\protect\hyperlink{site-content}{Skip to
content}\protect\hyperlink{site-index}{Skip to site index}

\href{https://www.nytimes3xbfgragh.onion/section/sports/soccer}{Soccer}

\href{https://myaccount.nytimes3xbfgragh.onion/auth/login?response_type=cookie\&client_id=vi}{}

\href{https://www.nytimes3xbfgragh.onion/section/todayspaper}{Today's
Paper}

\href{/section/sports/soccer}{Soccer}\textbar{}Dortmund's Erling Haaland
Adds to His Totals, and to His Legend

\url{https://nyti.ms/38Dia2Q}

\begin{itemize}
\item
\item
\item
\item
\item
\end{itemize}

Advertisement

\protect\hyperlink{after-top}{Continue reading the main story}

Supported by

\protect\hyperlink{after-sponsor}{Continue reading the main story}

\hypertarget{dortmunds-erling-haaland-adds-to-his-totals-and-to-his-legend}{%
\section{Dortmund's Erling Haaland Adds to His Totals, and to His
Legend}\label{dortmunds-erling-haaland-adds-to-his-totals-and-to-his-legend}}

Borussia Dortmund's teenage striker scored twice in a win over Paris
St.-Germain in the Champions League, once again making a tough job look
simple.

\includegraphics{https://static01.graylady3jvrrxbe.onion/images/2020/02/18/sports/18onsoccer1/18onsoccer1-articleLarge.jpg?quality=75\&auto=webp\&disable=upscale}

\href{https://www.nytimes3xbfgragh.onion/by/rory-smith}{\includegraphics{https://static01.graylady3jvrrxbe.onion/images/2019/08/23/sports/Rory-Smith-better/Rory-Smith-thumbLarge.png}}

By \href{https://www.nytimes3xbfgragh.onion/by/rory-smith}{Rory Smith}

\begin{itemize}
\item
  Feb. 18, 2020
\item
  \begin{itemize}
  \item
  \item
  \item
  \item
  \item
  \end{itemize}
\end{itemize}

DORTMUND, Germany --- Borussia Dortmund's players hung back a little,
idling halfway between the center circle and the goal. With Paris
St.-Germain jerseys slung over their shoulders, the spoils of battle,
they clapped each other on the back, they exchanged high fives, they
ruffled each others' hair. Most of all, though, they waited for him to
have his moment.

A few yards ahead, Erling Haaland stood in front of the Yellow Wall, the
soaring South stand of Signal Iduna Park. Just after Christmas, he had
finally decided to join Borussia Dortmund --- picking the club from
\href{https://www.nytimes3xbfgragh.onion/2020/01/04/sports/soccer/how-to-buy-say-youll-sell.html}{a
long, slavering queue of would-be suitors,} ranging from Manchester
United to RB Leipzig --- in part because of the prospect of playing in
front of what is, arguably, the most iconic terrace in European soccer.

The support of the Yellow Wall forms a considerable part of Dortmund's
sales pitch. Haaland had, by all accounts, long hoped to experience it.
Now here he was, 51 days later, basking in its adulation. He applauded
it, a little. He raised his arms above his head in celebration. He
offered a thumbs-up. Mostly, he just stared.

\begin{quote}
🎶 🏆 EUROPAPOKAL, EUROPAPOKAL, EUROPAPOKAL, EU-RO-PA-POKAL 🏆🎶
\href{https://twitter.com/hashtag/BVBPSG?src=hash\&ref_src=twsrc\%5Etfw}{\#BVBPSG}
\href{https://t.co/nbbAnjS4m5}{pic.twitter.com/nbbAnjS4m5}

--- Borussia Dortmund (@BVB)
\href{https://twitter.com/BVB/status/1229887057977380864?ref_src=twsrc\%5Etfw}{February
18, 2020}
\end{quote}

In return, the thousands of fans in front of him, the bricks of the
Yellow Wall, showered him with love. In deference to what he had just
done, his teammates waited. They let him take the acclaim. Only when
Axel Witsel could wait no longer, when he just had to start dancing, did
they start to join Haaland. Until then, it was all about him. He has
that ability: the capacity to make everyone else a bystander.

Haaland's start to life at Dortmund has, frankly, been a little
unrealistic. So too, for that matter, has been his first season in the
Champions League. There is clearly a glitch in the system somewhere, a
fault in the algorithm. This is, after all, the most exclusive
tournament in world soccer. It is the highest level of the game.

It is an aspiration, a dream, the ultimate test. Players spend years
hoping to have a chance to play in the competition; many of the finest
of their generation will end their careers without ever having made
quite the impression on it that they might had hoped. Haaland --- still
only 19, still a touch raw, still learning --- is making it all look
suspiciously easy.

He scored a hat-trick in his first game in the Champions League, back in
September, back when he was still playing for Red Bull Salzburg. He
scored in his next four games in the competition, too; only Liverpool,
in his sixth Champions League match, stopped his run.

Then he moved to Dortmund. He was a substitute in his first game for his
new club. He came on in the 56th minute. Twenty-three minutes later, he
had scored a hat-trick. He scored two more in his next game. The
following week, he scored twice in his first start. He currently has
eight goals in five appearances in the Bundesliga.

\includegraphics{https://static01.graylady3jvrrxbe.onion/images/2020/02/18/sports/18onsoccer2/merlin_169118607_cbbe6b29-29eb-4ccd-84cb-84c9a6a92a2e-articleLarge.jpg?quality=75\&auto=webp\&disable=upscale}

The last 16 of the Champions League was supposed to be a step up, of
course, a challenge of another magnitude. P.S.G. is, after all, one of
the most expensive teams ever built. It is a team rated --- perhaps a
little self-servingly --- as the favorite to win the tournament by
Jürgen Klopp, the manager of the reigning champion, Liverpool. Asking a
teenager to carry the fight to a defense of Thiago Silva, Marquinhos and
Presnel Kimpembe felt like a bold call from Lucien Favre, Dortmund's
coach.

Those who know Haaland, though, those who have tracked his career from
its beginnings in Norway, say that he possesses a rare calm, a sort of
beatific single-mindedness. He is fazed by little, or nothing. He
betrays not a flicker of nervousness. He is not the sort to worry that
he does not belong.

Strikers considerably more experienced than him might, perhaps, have
grown a little frustrated Tuesday in a first half that was a little more
cagey than most expected. Dortmund has won countless admirers in recent
years for its sense of adventure, its risk-taking, its dynamism. It has
long had a fatal flaw, though: a tendency toward self-immolation, an
ability to scupper itself at any given moment.

It is a trait shared by P.S.G., at least in the Champions League. The
French champion's attack is fearsome --- so good that it could afford to
leave Edinson Cavani and Mauro Icardi on the bench at Dortmund --- but
it is not quite good enough to mask what is more a psychological
vulnerability than anything else: invariably, P.S.G. seems to melt in
the white heat of the competition it has been built to win.

This was, then, supposed to be one of those wild games that occur ever
more frequently in this tournament, all breakneck counterattacking and
slapstick defending. That was what everyone wanted to see --- nine goals
here, split the difference, head back to Paris in three weeks to do it
all again --- with two notable exceptions: Favre and Thomas Tuchel,
P.S.G.'s coach.

It is significant that the coaches of two of Europe's most expansive
teams decided, when the stakes were high, that caution had to come
first. Both played with a back five, and two holding midfielders. Both,
uncharacteristically, chose to mask their weaknesses, rather than
accentuate their strengths.

Image

Haaland picked Dortmund from a long list of suitors in January. It would
be hard to find a Dortmund fan unhappy with the
signing.Credit...Wolfgang Rattay/Reuters

For more than an hour, the stalemate held: P.S.G. had the possession,
Dortmund tried to strike on the break. Haaland, starved of
opportunities, stuck to his job: he chased and harried; he took up his
pressing positions; he sniffed around for chances. He did not lose
heart. He did not lose focus. His moment was coming. Or rather, his
moments.

The first strike, to open the scoring, was pure predator: that rare
ability that strikers have to somehow turn up at the right place and the
right time, stretching out a leg, lifting the ball over Keylor Navas.
But it was the second, after Neymar had tied the score, that shook the
Yellow Wall. Darting on to a pass from Giovanni Reyna, the 17-year-old
American thrown on as a substitute, taking a touch, and then sending a
left-footed shot screeching past Navas.

That was Haaland's 10th goal in seven Champions League games this
season. It was his 11th in his time at Dortmund. Raw data, though, is an
unsatisfactory metric to communicate what made it so special: it was the
power of the shot, how early he took it, the certainty with which he did
so. Haaland has only just arrived --- in Dortmund, in the Champions
League --- but he already knows he belongs. His teammates sensed it,
too. This, now, is his moment. There is nothing else to do but let him
enjoy it.

Advertisement

\protect\hyperlink{after-bottom}{Continue reading the main story}

\hypertarget{site-index}{%
\subsection{Site Index}\label{site-index}}

\hypertarget{site-information-navigation}{%
\subsection{Site Information
Navigation}\label{site-information-navigation}}

\begin{itemize}
\tightlist
\item
  \href{https://help.nytimes3xbfgragh.onion/hc/en-us/articles/115014792127-Copyright-notice}{©~2020~The
  New York Times Company}
\end{itemize}

\begin{itemize}
\tightlist
\item
  \href{https://www.nytco.com/}{NYTCo}
\item
  \href{https://help.nytimes3xbfgragh.onion/hc/en-us/articles/115015385887-Contact-Us}{Contact
  Us}
\item
  \href{https://www.nytco.com/careers/}{Work with us}
\item
  \href{https://nytmediakit.com/}{Advertise}
\item
  \href{http://www.tbrandstudio.com/}{T Brand Studio}
\item
  \href{https://www.nytimes3xbfgragh.onion/privacy/cookie-policy\#how-do-i-manage-trackers}{Your
  Ad Choices}
\item
  \href{https://www.nytimes3xbfgragh.onion/privacy}{Privacy}
\item
  \href{https://help.nytimes3xbfgragh.onion/hc/en-us/articles/115014893428-Terms-of-service}{Terms
  of Service}
\item
  \href{https://help.nytimes3xbfgragh.onion/hc/en-us/articles/115014893968-Terms-of-sale}{Terms
  of Sale}
\item
  \href{https://spiderbites.nytimes3xbfgragh.onion}{Site Map}
\item
  \href{https://help.nytimes3xbfgragh.onion/hc/en-us}{Help}
\item
  \href{https://www.nytimes3xbfgragh.onion/subscription?campaignId=37WXW}{Subscriptions}
\end{itemize}
