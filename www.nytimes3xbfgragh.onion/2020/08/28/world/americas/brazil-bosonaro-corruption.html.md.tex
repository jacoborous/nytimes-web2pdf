Sections

SEARCH

\protect\hyperlink{site-content}{Skip to
content}\protect\hyperlink{site-index}{Skip to site index}

\href{https://www.nytimes3xbfgragh.onion/section/world/americas}{Americas}

\href{https://myaccount.nytimes3xbfgragh.onion/auth/login?response_type=cookie\&client_id=vi}{}

\href{https://www.nytimes3xbfgragh.onion/section/todayspaper}{Today's
Paper}

\href{/section/world/americas}{Americas}\textbar{}`A Family Business:'
Graft Investigation Threatens Brazil's Bolsonaro

\url{https://nyti.ms/3lvfta6}

\begin{itemize}
\item
\item
\item
\item
\item
\end{itemize}

Advertisement

\protect\hyperlink{after-top}{Continue reading the main story}

Supported by

\protect\hyperlink{after-sponsor}{Continue reading the main story}

\hypertarget{a-family-business-graft-investigation-threatens-brazils-bolsonaro}{%
\section{`A Family Business:' Graft Investigation Threatens Brazil's
Bolsonaro}\label{a-family-business-graft-investigation-threatens-brazils-bolsonaro}}

Brazilians are asking a question that could threaten President Jair
Bolsonaro's political future: Why did his wife and son receive payments
from a man under investigation for corruption?

\includegraphics{https://static01.graylady3jvrrxbe.onion/images/2020/08/26/world/00brazil-bolsonaro/merlin_175119246_422ebd52-fc5b-4ad7-8e44-7cfa480af79e-articleLarge.jpg?quality=75\&auto=webp\&disable=upscale}

By \href{https://www.nytimes3xbfgragh.onion/by/ernesto-londono}{Ernesto
Londoño}, Manuela Andreoni and Letícia Casado

\begin{itemize}
\item
  Aug. 28, 2020
\item
  \begin{itemize}
  \item
  \item
  \item
  \item
  \item
  \end{itemize}
\end{itemize}

RIO DE JANEIRO --- Brazil's president, Jair Bolsonaro, was visiting a
cathedral in the capital in recent days when a reporter threw out a
question: President, why did your wife receive \$16,000 from a former
aide under investigation for corruption?

The response was aggressive, even for a president known for venting his
anger at journalists and critics.

``What I'd like to do,'' Mr. Bolsonaro told the reporter, ``is smash
your mouth in.''

In his two years in office, as Mr. Bolsonaro and
\href{https://www.nytimes3xbfgragh.onion/2019/02/01/world/americas/brazil-flavio-bolsonaro.html}{his
inner circle}, including his sons, have become engulfed in a growing
number of criminal and legislative investigations, he has
\href{https://www.nytimes3xbfgragh.onion/2020/05/29/world/americas/brazil-bolsonaro-supreme-court.html}{lashed
out} at reporters, investigators and
\href{https://www.nytimes3xbfgragh.onion/2020/04/24/world/americas/brazil-bolsonaro-moro.html}{even
members of his own cabinet} who have dared go against him.

But the case involving the former aide and family confidant --- which
revolves around the potential theft of public sector wages --- has
particularly rattled Mr. Bolsonaro's nerves by putting his wife and his
oldest son at the center of a corruption investigation that has
developed into one of his biggest personal and political liabilities.

\includegraphics{https://static01.graylady3jvrrxbe.onion/images/2020/08/26/world/00brazil-bolsonaro-arrest/merlin_173663091_30316bad-8875-4680-b7f5-a8b4283af6e5-articleLarge.jpg?quality=75\&auto=webp\&disable=upscale}

The expanding set of inquiries into the president and his family are
testing the independence and strength of the justice system in one of
the world's largest democracies, with the largest economy in the
southern hemisphere. Just a few years ago, Brazil's judiciary earned
global accolades for taking down powerful officials and business titans
in an
\href{https://www.nytimes3xbfgragh.onion/2016/06/11/world/americas/brazil-corruption-dilma-rousseff-operation-car-wash.html}{anti-corruption
crusade} that upended the political establishment.

Now Mr. Bolsonaro,
\href{https://www.nytimes3xbfgragh.onion/2018/10/29/world/americas/jair-bolsonaro-brazil-profile.html}{whose
astonishing rise} from the fringes of far-right politics to the
presidency was largely propelled by a promise to root out graft and
crime, stands accused of undermining the rule of law, as the scandals
inch ever closer to the presidential palace.

Experts say the evidence that has come to light so far in the case of
the former aide, Fabrício Queiroz, suggests the Bolsonaro family partook
in a scheme known as \emph{rachadinha}, which is common in the lower
rungs of politics in Brazil. It involves siphoning off taxpayer money by
keeping ghost employees on payroll or hiring people who agree to kick
back a share of their salary to the boss.

``The suspicion is that this was a family business that lasted many
years and moved a lot of money,'' Bruno Brandão, the executive director
of Transparency International in Brazil, said of the graft scheme
involving the former aide. ``These suppositions are very serious,
corroborated by solid evidence, in an investigation that is based on
highly irregular financial transactions.''

In court filings and leaks to the press, the authorities have outlined
their suspicion that starting in 2007, Mr. Queiroz helped the
president's oldest son, Flávio Bolsonaro, steal public funds by
pocketing part of the wages of people on his payroll when he was a state
representative. Flávio Bolsonaro was elected to the Senate in 2018.

Image

Flávio Bolsonaro, the president's son, is at the center of a widening
scandal that also involves the president's wife.Credit...Eraldo
Peres/Associated Press

Between 2011 and 2016, Mr. Queiroz funneled thousands of dollars to the
president's wife, Michelle Bolsonaro, in transactions neither of them
can explain. Prosecutors also believe deposits made to the president's
son might be connected to the scheme.

Drawing on a vast dossier of financial records, investigators are trying
to determine whether the irregular cash flow at a chocolate shop Flávio
Bolsonaro bought in 2015, and a series of real estate purchases he made
in cash, amount to money laundering.

Separately, a Brazilian newspaper uncovered that one of Mr. Queiroz's
daughters, Nathália Queiroz,
\href{https://www1.folha.uol.com.br/poder/2018/12/ex-secretaria-parlamentar-de-jair-bolsonaro-atuava-como-personal-trainer-no-rio.shtml?utm_source=whatsapp\&utm_medium=social\&utm_campaign=compwa}{was
on the payroll}of the president's former congressional office in
Brasília between 2016 and 2018, even though she was working as a
personal trainer in Rio de Janeiro at the time.

Bank records obtained by prosecutors show that Ms. Queiroz made monthly
payments to her father that totaled tens of thousands of dollars between
2017 and 2018.

The president's office declined to comment on the case on behalf of Mr.
Bolsonaro and his wife. Michelle Bolsonaro met her husband in 2006 while
she was working as a secretary in Congress. After the two began dating,
she joined his legislative staff, a move that tripled her
\href{https://www1.folha.uol.com.br/poder/2017/12/1941623-bolsonaro-empregou-e-promoveu-a-mulher-em-gabinete-na-camara.shtml}{salary.}

Paulo Emílio Catta Preta, a lawyer representing Mr. Queiroz, said the
transactions involving the Bolsonaro family ``have absolutely nothing to
do with alleged misappropriation of funds.'' Flávio Bolsonaro's attorney
did not respond to a request for interviews.

In a recent interview, Vice President Hamilton Mourão defended the
administration's record on corruption, noting that it has not been
embroiled in the type of multimillion dollar kickbacks schemes uncovered
during previous governments. He deplored the leaking to the press of so
much information about the Queiroz investigation, arguing that there is
an effort underway to ``fabricate a narrative for public opinion.''

Image

Mr. Bolsonaro and his inner circle have become engulfed in a growing
number of criminal and legislative investigations.Credit...Adriano
Machado/Reuters

The investigation began taking shape shortly after Mr. Bolsonaro's
decisive electoral victory in October 2018. He beat a leftist party
whose enormous popularity crumbled as its leaders were charged in
kickback schemes involving large government contracts and transnational
business deals.

Shortly after the election, prosecutors in Rio de Janeiro noted that Mr.
Queiroz's bank activity in 2016 and 2017, while he was on Flávio
Bolsonaro's payroll, was incompatible with his reported earnings.

Since then, other legislative and criminal investigations have put the
Bolsonaro family on the defensive.

Another of the president's sons, Carlos Bolsonaro, is being investigated
on
\href{https://www1.folha.uol.com.br/poder/2020/07/quebra-de-sigilo-da-rachadinha-atinge-ex-assessores-de-carlos-bolsonaro-agora-sem-foro.shtml}{similar
charges of diverting public funds} during his time as a City Council
member in Rio de Janeiro, and in connection with a case about
disinformation campaigns waged online. A third son,
\href{https://politica.estadao.com.br/blogs/fausto-macedo/relembre-as-investigacoes-que-envolvem-flavio-carlos-e-eduardo-bolsonaro/}{Eduardo
Bolsonaro}, is also involved in the disinformation case.

As criminal and legislative investigations entangled people close to the
president, his government led or backed efforts that have weakened the
hand of anticorruption prosecutors. They included making it harder for
investigators to obtain bank records to build criminal cases. A new law
subjects prosecutors to punishments including fines and criminal charges
for misconduct.

Those actions contributed to the dramatic exit of Mr. Bolsonaro's most
popular cabinet member,
\href{https://www.nytimes3xbfgragh.onion/2020/04/24/world/americas/brazil-bolsonaro-moro.html?searchResultPosition=1}{Sergio
Moro}, who in April accused the president of seeking to replace the head
of the federal police in order to shield friends and relatives from
criminal investigations.

The Supreme Court is investigating whether the president's conduct
amounted to obstruction of justice.

The events surrounding the exit of Mr. Moro, a former federal judge who
had become
\href{https://www.nytimes3xbfgragh.onion/2017/08/25/world/americas/judge-sergio-moro-brazil-anti-corruption.html}{the
most emblematic figure} in the anticorruption crusade that began in
2014, are widely seen as a de facto abandonment of the president's
promise to fight corruption. A bill with sweeping anticorruption reforms
Mr. Moro championed has been abandoned.

Image

The former Justice Minister Sergio Moro resigned in April, accusing the
president of trying to protect friends and relatives from
scrutiny.Credit...Eraldo Peres/Associated Press

``Our perception is that white collar criminals are celebrating,'' said
Melina Flores, a federal prosecutor who worked on high-profile
corruption cases in Brasília, the capital.

Investigators are struggling to make headway. The fight against
corruption, which once sparked mass protests, has lost resonance as
Brazil faces the world's second-highest number of deaths from the
coronavirus, behind only the United States, and the economic meltdown
that followed.

The shift in national focus has allowed the restoration of an unspoken
system in which powerful judges and politicians protect each other's
interests, said Carlos Fernando dos Santos Lima, a former prosecutor who
worked on politically explosive investigations.

``It's a return to the old political practice of being shielded by
judicial maneuvers,'' he said. ``In Brazil we have a republic of
untouchables and a republic for the rest of the population.''

Against that backdrop, prosecutors in the case have found ways to keep
the investigation in the public eye --- even as Mr. Queiroz sought to
remain out of sight and the Bolsonaro family downplayed his
significance.

In June, investigators armed with an arrest warrant for Mr. Queiroz
found him in a São Paulo residence that belongs to one of Mr.
Bolsonaro's lawyers, Frederick Wassef.

The arrest, which dominated front pages and news broadcasts for days,
was followed by leaks to the press that Mr. Queiroz had wired Michelle
Bolsonaro far more money than investigators had initially disclosed.
That called into question the president's account that a single payment
disclosed in 2018 was made to repay a debt.

After Mr. Bolsonaro lashed out at the reporter with O Globo newspaper on
Sunday, thousands of Brazilians who are critical of the president turned
to social media to echo his question: ``President, why did your wife
receive \$16,000 from Fabricio Queiroz?''

The stakes are high for the first lady. Unlike her husband and Flávio
Bolsonaro, she is not an elected official, which deprives her of the
protections from prosecution that they enjoy.

Just how politically damaging the case will be for Mr. Bolsonaro in the
long run is unclear, analysts say. Despite his cavalier handling of the
coronavirus pandemic, which has contributed to the death of more than
118,000 Brazilians, the president has broadened his support base
slightly by giving emergency aid to millions of Brazilians.

``The majority of Brazilians are thinking a lot more about survival than
political matters,'' Mauro Paulino, the director of the Datafolha
polling firm, said. ``When survival is your primary concern, corruption
becomes a secondary issue.''

Advertisement

\protect\hyperlink{after-bottom}{Continue reading the main story}

\hypertarget{site-index}{%
\subsection{Site Index}\label{site-index}}

\hypertarget{site-information-navigation}{%
\subsection{Site Information
Navigation}\label{site-information-navigation}}

\begin{itemize}
\tightlist
\item
  \href{https://help.nytimes3xbfgragh.onion/hc/en-us/articles/115014792127-Copyright-notice}{©~2020~The
  New York Times Company}
\end{itemize}

\begin{itemize}
\tightlist
\item
  \href{https://www.nytco.com/}{NYTCo}
\item
  \href{https://help.nytimes3xbfgragh.onion/hc/en-us/articles/115015385887-Contact-Us}{Contact
  Us}
\item
  \href{https://www.nytco.com/careers/}{Work with us}
\item
  \href{https://nytmediakit.com/}{Advertise}
\item
  \href{http://www.tbrandstudio.com/}{T Brand Studio}
\item
  \href{https://www.nytimes3xbfgragh.onion/privacy/cookie-policy\#how-do-i-manage-trackers}{Your
  Ad Choices}
\item
  \href{https://www.nytimes3xbfgragh.onion/privacy}{Privacy}
\item
  \href{https://help.nytimes3xbfgragh.onion/hc/en-us/articles/115014893428-Terms-of-service}{Terms
  of Service}
\item
  \href{https://help.nytimes3xbfgragh.onion/hc/en-us/articles/115014893968-Terms-of-sale}{Terms
  of Sale}
\item
  \href{https://spiderbites.nytimes3xbfgragh.onion}{Site Map}
\item
  \href{https://help.nytimes3xbfgragh.onion/hc/en-us}{Help}
\item
  \href{https://www.nytimes3xbfgragh.onion/subscription?campaignId=37WXW}{Subscriptions}
\end{itemize}
