Sections

SEARCH

\protect\hyperlink{site-content}{Skip to
content}\protect\hyperlink{site-index}{Skip to site index}

\href{https://myaccount.nytimes3xbfgragh.onion/auth/login?response_type=cookie\&client_id=vi}{}

\href{https://www.nytimes3xbfgragh.onion/section/todayspaper}{Today's
Paper}

Today's Chefs Are Honoring a Vital Tradition: Feeding the Revolution

\url{https://nyti.ms/3b6WXjM}

\begin{itemize}
\item
\item
\item
\item
\item
\end{itemize}

\hypertarget{race-and-america}{%
\subsubsection{\texorpdfstring{\href{https://www.nytimes3xbfgragh.onion/news-event/george-floyd-protests-minneapolis-new-york-los-angeles?name=styln-george-floyd\&region=TOP_BANNER\&block=storyline_menu_recirc\&action=click\&pgtype=Article\&impression_id=635b5c60-f287-11ea-96c9-f92c07a4cad0\&variant=undefined}{Race
and America}}{Race and America}}\label{race-and-america}}

\begin{itemize}
\tightlist
\item
  \href{https://www.nytimes3xbfgragh.onion/2020/09/04/nyregion/rochester-police-daniel-prude.html?name=styln-george-floyd\&region=TOP_BANNER\&block=storyline_menu_recirc\&action=click\&pgtype=Article\&impression_id=635b5c61-f287-11ea-96c9-f92c07a4cad0\&variant=undefined}{What
  Happened in Rochester, N.Y.}
\item
  \href{https://www.nytimes3xbfgragh.onion/2020/09/01/us/politics/trump-fact-check-protests.html?name=styln-george-floyd\&region=TOP_BANNER\&block=storyline_menu_recirc\&action=click\&pgtype=Article\&impression_id=635b5c62-f287-11ea-96c9-f92c07a4cad0\&variant=undefined}{Trump
  Fact Check}
\item
  \href{https://www.nytimes3xbfgragh.onion/2020/08/30/us/portland-shooting-explained.html?name=styln-george-floyd\&region=TOP_BANNER\&block=storyline_menu_recirc\&action=click\&pgtype=Article\&impression_id=635b8370-f287-11ea-96c9-f92c07a4cad0\&variant=undefined}{Portland
  Shooting}
\item
  \href{https://www.nytimes3xbfgragh.onion/2020/08/30/us/breonna-taylor-police-killing.html?name=styln-george-floyd\&region=TOP_BANNER\&block=storyline_menu_recirc\&action=click\&pgtype=Article\&impression_id=635b8371-f287-11ea-96c9-f92c07a4cad0\&variant=undefined}{Breonna
  Taylor's Life and Death}
\end{itemize}

Advertisement

\protect\hyperlink{after-top}{Continue reading the main story}

Supported by

\protect\hyperlink{after-sponsor}{Continue reading the main story}

\includegraphics{https://static01.graylady3jvrrxbe.onion/images/2020/08/30/t-magazine/30tmag-hunger/30tmag-hunger-articleLarge.jpg?quality=75\&auto=webp\&disable=upscale}

\hypertarget{todays-chefs-are-honoring-a-vital-tradition-feeding-the-revolution}{%
\section{Today's Chefs Are Honoring a Vital Tradition: Feeding the
Revolution}\label{todays-chefs-are-honoring-a-vital-tradition-feeding-the-revolution}}

As Americans have taken to the streets demanding racial justice,
restaurants and nonprofits have provided meals for them, building upon a
long legacy of food as resistance.

By Ligaya Mishan

Aug. 28, 2020

THE KITCHEN IS an arsenal. Bring out the pots and pans and bang them
like drums, clash lids, whack spoons. That the weapons are so humble is
the point: everyday essentials, available to all. Orchestras equipped
like this once paraded through medieval Europe, in the traditions of the
French charivari and its English equivalent, ``rough music,'' descending
on the homes of those believed to have betrayed social mores --- what
the English folk historian Violet Alford called ``the beginning of
popular justice.'' This took on a more political tone in the mid-19th
century, with housewives in Paris bashing pans outside their landlords'
windows, demanding relief from rent. Today, such disruptive noisemaking
has been borrowed as a form of collective action around the world, often
under the name \emph{cacerolazo} (from the Spanish for ``casserole''),
popularized by a 1971 demonstration against food shortages in Chile in
which more than 5,000 women took to the streets rattling pots, the
emptiness of which gave testament to their cause.

Food has always been central to resistance, because its lack is the most
fundamental of inequities. What kind of society lets its own people
starve, whether by negligence or knowing exploitation? When global food
prices spiked in 2008 and again in 2010, due in part to rampant,
unregulated speculation in agricultural commodity futures as well as
droughts and desertification of arable lands brought on by climate
change, more than a hundred million people were pushed into poverty,
leading to civil unrest from Senegal to Uzbekistan, Nepal to Peru.
Against this backdrop, the Arab Spring started, arguably, with an apple
--- two baskets of them, to be exact, confiscated by a market inspector
in December 2010 from Mohamed Bouazizi, a 26-year-old
fruit-and-vegetable vendor in the small town of Sidi Bouzid, Tunisia.
This was just the latest offense in a longstanding pattern of
corruption, but this time the young man said: \emph{No more}. He set
himself on fire in front of a government building and died in a hospital
three weeks later, inspiring protests across the country that within a
month had ended the rule of the president of 23 years.

Image

Chilean women in the streets with pots and pans in the early
1970s.Credit...via Dsausa.Org

Image

Mahatma Gandhi on the 1930 salt march in Dandi, Gujurat, India, a
protest against the monopoly of the British colonial government's salt
production.Credit...Rühe/UIllstein Bild/Getty Images

No food is too small to tip the balance. Consider the machinations
surrounding an ingredient as basic and imperative as salt, which the
Indian leader Mahatma Gandhi described as ``perhaps the greatest
necessity of life'' after air and water. In the first century B.C.,
Emperor Zhao of the Western Han dynasty in China convened a debate
between bureaucrats who favored a state monopoly on salt, enabling them
to inflate the price, and Confucian scholars who questioned the
government's elevation of gain over righteousness. ``Never should
material profit appear as a motive of government,'' the scholars argued,
to which the worldly emperor replied with a sniff: ``You put all your
faith in the past and turn your backs upon the present.'' For centuries,
this dominion proved so lucrative --- at one point during the Tang
dynasty (A.D. 618-907), cash income from salt yielded over half of all
government revenue, and during the Yuan (1279-1368), up to 80 percent
--- that it survived the fall of multiple dynasties and the rise of the
Communist Party and was not abolished until 2014, the longest-lasting
monopoly in history. China was not alone: The much-despised gabelle, a
tax first imposed on salt in France in the 13th century and at times
soaring to 10 times the cost of producing salt in the first place,
became a rallying cry for the French Revolution and likely contributed
to the deaths of dozens of tax collectors at the guillotine; and in
1930, Gandhi defied the British Raj's monopoly on what he called ``the
only condiment of the poor'' by stooping to grasp a fistful of mud and
salt on the tidal flats of a coastal village in the western state of
Gujarat.

In a revolutionary context, food is at once literal and symbolic, a
totem of power and its usurpation. For Black Americans in the late
1950s, Coca-Cola was at once ubiquitous and forbidden: At segregated
soda fountains and lunch counters, it was accessible only to white
people until the civil rights activist Carol Parks Hahn and 30 fellow
students took seats in
\href{https://www.npr.org/templates/story/story.php?storyId=6355095}{Dockum
Drug Store} in Wichita, Kan., in 1958 and ordered the soft drink. They
were denied but remained, sitting in shifts, returning day after day,
asking for a Coke --- that icon of Americana, that badge of belonging
--- until the owner capitulated and agreed to serve everyone.

Image

A clipping from the the front page of the Aug. 7, 1958, issue of The
Enlightener, an African-American paper in Wichita, Kan., showing the
sit-in at Dockum Drug Store.Credit...Courtesy of the Wichita-Sedgewick
County Historical Museum

Again, it's the mundanity of the food, its theoretical availability to
all, that reveals how exclusion and oppression are built into the
everyday, tacitly permitted by those who benefit from the system. Marie
Antoinette, the 18th-century Austrian archduchess later executed as the
queen of France, never said, ``Let them eat cake'' --- the French
philosopher Jean-Jacques Rousseau attributed a similar phrase to
another, anonymous royal in his ``Confessions,'' written before she even
married the dauphin --- but we want to believe she did because it so
perfectly captures our sense of a world in which the haves blithely or
willfully disregard the constraints on the have-nots. The slogan ``Eat
the rich,'' itself falsely credited to Rousseau and newly popular as a
battle cry among agitators for change, gleefully turns this notion
around: When we run out of sustenance, when our cupboards are bare, the
Marie Antoinettes of the world will be our feast, the frivolous thing we
will be forced to consume.

TO EAT THE RICH is of course mere rhetoric, a fantasy of vengeance. The
terrible irony is that for those in extremis, one of the most radical
forms of protest is to shun food entirely --- to visit violence on
oneself, turning it inward, internalizing the crime of the oppressor so
that its corrosive impact is made manifest to the world. The history of
hunger strikes is long, going back to the age-old Indian custom of
dharna (historically, sitting at the threshold of a debtor and fasting
until the debt was cleared, and today a more general term for a sit-in)
and the Celtic troscad, which predated Christianity's arrival in Ireland
in the early fifth century. This was not mere ritual: Troscad was a
legally sanctioned means of extracting justice from someone of higher
rank and a rare tool of the poor ``against the mighty,'' as the late
19th- and early 20th-century Irish nationalist Laurence Ginnell wrote.
Once all other avenues of redress had been attempted and exhausted, you
would wait publicly at the doorstep of the wrongdoer and refuse to eat
until reparations were made. The act of self-starvation so disrupted the
social order, some thought it took on a supernatural aura, with the
intimation that the damage done to the victim's body would redound upon
the offending party, exacting a spiritual price. (There were legal
consequences to fear, too, including, in some circumstances, a doubling
of the amount of reparations required.)

If the potency of troscad rested in part on the bonds and expectations
of a small community, where refusing hospitality to a guest at your door
was a mark of dishonor, the modern hunger strike has had to rely on a
broader sense of outrage. Sometimes this is achieved by exposing the
callousness of the oppressor, as in the case of imprisoned suffragists
in early 20th-century England, who were subject to brutal force-feedings
that broke teeth and caused internal injuries, drawing widespread public
condemnation. In 1981, 10 members of the Irish Republican Army were
allowed to starve to death over months in a paramilitary-style prison in
Northern Ireland, their troscad --- and request to be recognized as
political prisoners instead of common criminals --- unanswered; some in
the British press greeted their deaths as a victory (``I will shed no
tears,'' one newspaper editor wrote), but the world spoke out against
such indifference, and across Ireland, the dead were mourned and
celebrated as martyrs --- the leader of the protest, Bobby Sands, had
been elected to Parliament while on strike, and upward of 70,000 people
attended his funeral --- pressuring the British government to improve
prison conditions.

Image

A 1981 poster of imprisoned Irish Republican Army members to raise
awareness of the 1980-81 Irish hunger strike, distributed by the Irish
Prisoners of War Committee, New York City.Credit...Stuart
Lutz/Gado/Getty Images

To have moral force, the hunger strike had to be a last resort. For the
Irish nationalists, as outlined in a statement released on the day of
the strike, it was a ``demonstration of our selflessness'' --- as
opposed to the selfishness of criminals out for personal gain --- ``and
the justness of our cause.'' The student activists who occupied
Tiananmen Square in Beijing in the spring of 1989 explicitly framed
their decision to stop eating as a sacrifice on behalf of their country:
``Although our bones are still forming, although we are too young for
death, we are ready to leave you. We must go; we are answering the call
of Chinese history.'' In keeping with the exalted language, the hunger
strike was orchestrated as spectacle, with more than 3,000 students
eventually joining the fast and some even rejecting water, accelerating
their decline in Tiananmen's midday sun. Hundreds of thousands of
supporters crowded the square, and doctors and desperate parents
hovered, ratcheting up the anxiety against a backdrop of throbbing
ambulance sirens as strikers lost consciousness and were hauled off to
the hospital. It wasn't simply the students' youth but their privilege
as part of the educated class that made their willingness to risk
everything so persuasive; by starving themselves, they earned
credibility and galvanized the country --- until the government declared
martial law and troops opened fire on the protesters. In the aftermath,
thousands were detained, and, decades later, all references to the
massacre continue to be censored within China.

Gandhi, who endured 17 fasts in his resistance to British imperialism,
cautioned that, even when successful, a hunger strike could be merely
coercive rather than persuasive: Your opponents might make concessions
but not actually believe they'd done anything wrong. The result is a
temporary fix, a slapped-on bandage, rather than lasting change.

PERHAPS THE MOST direct use of food as a weapon is its co-opting as
ammunition. Eggs once launched as projectiles at maudlin actors in
18th-century England are now wielded against politicians and masters of
the universe, along with milkshakes, custard pies and, in Greece,
yogurt. This isn't flippant but strategic, for eggs, even if rotten and
foul, don't wound like stones or grenades, instead delivering a dose of
humiliation that falls just shy of a proper felony (and its legal
consequences). Mockery likewise calls attention to unfair structures of
power: ``Every joke is a tiny revolution,'' the British writer George
Orwell wrote in 1945. ``Whatever destroys dignity, and brings down the
mighty from their seats, preferably with a bump, is funny.'' And
throwing the ingredients of what would otherwise make a meal both
revokes their promise of nourishment and subverts the notion of sharing
food as an act of hospitality and community, laying bare the lie behind
our supposed commitment as a society to take care of one another.

But food can also be used to mend a broken social contract --- to
reaffirm our bonds despite the failures of the system. In this, it may
be the stealthiest of weapons. To help fund the 1956 boycott of city
buses in Montgomery, Ala., the civil rights activist
\href{https://www.nytimes3xbfgragh.onion/2019/07/31/obituaries/georgia-gilmore-overlooked.html}{Georgia
Gilmore} raised thousands of dollars --- enough to keep the boycott
going for 381 days --- by recruiting women to cook and sell meals and
desserts (fried chicken sandwiches, poundcake, sweet-potato pie) as part
of an underground network she named the Club From Nowhere, kept secret
so that the women involved wouldn't get fired by their white employers.
In a modern correlative, the writer Klancy Miller's forthcoming
magazine, For the Culture, dedicated to Black women in the world of food
and drink, was funded in part by North Carolina bakers, who donated
proceeds from roasted-sweet-potato scones and peach doughnuts sold
online this past June and July. The Florida-born chef Kia Damon took
Gilmore as a model for her own Supper Club From Nowhere, which began in
2017 as a dinner series showcasing her interpretations of ancestral
recipes, working to rectify the neglect and erasure of Black American
heritage.

There is a step beyond this: acknowledging that cooking is inevitably
political, from the dishes we choose to make to where our ingredients
come from, but also, at the most basic level, in who gets to eat.
Nourishment is a prerequisite to equity. In June, Damon began
fund-raising to build a community kitchen and co-op grocery in Downtown
Brooklyn with a mission to distribute fresh produce and pantry goods to
people living in food deserts, which the United States Department of
Agriculture has historically defined as areas where residents have
little access to affordable, healthy food because of low income and
limited transportation options. (Some prefer the term ``food apartheid''
to underscore that these deserts don't exist by chance --- that the
market forces that give rise to them are part of a larger system of
imbalance.)

Image

Two young boys~eating~during a~free-breakfast-for-children~program
sponsored by the Black Panther Party, New York City, 1969.Credit... Bev
Grant/Getty Images

``Hunger is one of the means of oppression,'' the editors of the Black
Panther Party's official newspaper wrote in 1969, when its members began
providing free breakfast to schoolchildren in Oakland, Calif. ``How can
our children learn anything when most of their stomachs are empty?''
Soon the program expanded to party chapters from Milwaukee and New
Orleans to Brownsville, Brooklyn. At the time, J. Edgar Hoover, the
director of the F.B.I., called this act of charity ``potentially the
greatest threat to efforts by authorities to neutralize the BPP and
destroy what it stands for,'' without clarifying which he feared more:
the boost it would give to the Panthers' image or the prospect of Black
youth gaining greater access to nutrition and education. Despite this
animus, the federal government eventually followed the Panthers' lead;
its own school breakfast program, which had launched in limited form for
around 80,000 students in 1966, was made permanent and national in 1975,
and now feeds more than 14 million children each day.

The issue of food insecurity in the United States is more urgent than
ever: Since March, the Covid-19 pandemic has cost tens of millions of
Americans their jobs, including nearly 40 percent of those with
household incomes under \$40,000 a year. Food banks have seen up to a
200 percent increase in requests for help, with lines of cars and waits
that stretch for miles and hours. In the absence of a functional
government safety net, mutual-aid networks have sprung up --- ad hoc,
mobilized by volunteers and buoyed by small donations, eschewing
hierarchy in favor of collective decisions --- to deliver meals and food
supplies to health workers and families in need. Instead of accepting a
failed system, you build a new one.

Image

Donated pizza is passed out in Zuccotti Park for members and supporters
of the Occupy Wall Street movement, Oct. 1, 2011.Credit...Mario
Tama/Getty Images

SINCE THE DEATH of
\href{https://www.nytimes3xbfgragh.onion/2020/05/31/us/george-floyd-investigation.html}{George
Floyd} at the hands of the Minneapolis police in May, protesters have
filled the streets across the country, marching for hours. How to feed
them? A number of restaurants along the routes have opened their doors,
their whereabouts posted on crowdsourced online maps, doling out
provisions and in some cases turning their dining rooms into rudimentary
clinics for those caught in tear gas or hit by rubber bullets. Community
organizations hand out water and snacks. Often, the food is
astonishingly plentiful, as if transformed from a few loaves and fishes
--- as when out-of-town donors from as far away as Greece arranged to
have pastrami sandwiches and pizzas sent by the dozen to Zuccotti Park
during the Occupy Wall Street protests in 2011, or when newlyweds
dropped off slices of wedding cake for the environmentalist group
Extinction Rebellion in London's Trafalgar Square in 2019.

Image

Fuel the People provided over 1,000 meals for Juneteenth protesters in
Washington, D.C.Credit...Courtesy of Fuel the People

But Gaïana Joseph and Allegra Massaro --- who with their brothers
Roodharvens Joseph and Lorenzo Massaro founded the nonprofit Fuel the
People, which fed over 10,000 protesters in the first few weeks of
demonstrations in New York and Washington, D.C. --- see another
opportunity: ``to redistribute wealth back into the community,'' Joseph
said. Cash donations, solicited on Instagram and contributed online, are
used to buy food from Black- and immigrant-owned restaurants, businesses
that have been struggling during the pandemic; when the restaurants pack
meals for protesters, Joseph and Massaro make sure that their logos are
affixed so that the protesters in turn might seek out the establishments
as future diners. They hope, too, to bring a wider audience to cuisines
outside the American mainstream by offering the likes of Ethiopian
sambusas and Haitian patties. ``People are getting very good food,''
Massaro said with a laugh. Their volunteers --- equipped with
megaphones, hand sanitizer, tear-gas repellent and the Legal Aid Society
hotline number in case of conflict with the police --- ride bikes, some
hauling wagons, and pick up trash and recyclables along the way. At each
protest, they have hot food ready at the march's end, kept warm in
insulated bags and served in a kind of mass communal dinner. Often,
everyone sits down --- ``they've done this work,'' Joseph said, and
they've earned this rest, this chance to talk, look around and share,
for a moment, the possibility of change.

There is anger in resistance, but also jubilation. ``Giving food to
other people is an act of love,'' Joseph said, ``an act of compassion.''
And even when the government set curfews, trying to quell the protests,
people still found a way to make themselves heard. They stood at their
windows, banging their pots and pans, clamoring for a better world.

Advertisement

\protect\hyperlink{after-bottom}{Continue reading the main story}

\hypertarget{site-index}{%
\subsection{Site Index}\label{site-index}}

\hypertarget{site-information-navigation}{%
\subsection{Site Information
Navigation}\label{site-information-navigation}}

\begin{itemize}
\tightlist
\item
  \href{https://help.nytimes3xbfgragh.onion/hc/en-us/articles/115014792127-Copyright-notice}{©~2020~The
  New York Times Company}
\end{itemize}

\begin{itemize}
\tightlist
\item
  \href{https://www.nytco.com/}{NYTCo}
\item
  \href{https://help.nytimes3xbfgragh.onion/hc/en-us/articles/115015385887-Contact-Us}{Contact
  Us}
\item
  \href{https://www.nytco.com/careers/}{Work with us}
\item
  \href{https://nytmediakit.com/}{Advertise}
\item
  \href{http://www.tbrandstudio.com/}{T Brand Studio}
\item
  \href{https://www.nytimes3xbfgragh.onion/privacy/cookie-policy\#how-do-i-manage-trackers}{Your
  Ad Choices}
\item
  \href{https://www.nytimes3xbfgragh.onion/privacy}{Privacy}
\item
  \href{https://help.nytimes3xbfgragh.onion/hc/en-us/articles/115014893428-Terms-of-service}{Terms
  of Service}
\item
  \href{https://help.nytimes3xbfgragh.onion/hc/en-us/articles/115014893968-Terms-of-sale}{Terms
  of Sale}
\item
  \href{https://spiderbites.nytimes3xbfgragh.onion}{Site Map}
\item
  \href{https://help.nytimes3xbfgragh.onion/hc/en-us}{Help}
\item
  \href{https://www.nytimes3xbfgragh.onion/subscription?campaignId=37WXW}{Subscriptions}
\end{itemize}
