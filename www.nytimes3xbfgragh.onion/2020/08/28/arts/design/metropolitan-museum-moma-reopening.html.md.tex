Sections

SEARCH

\protect\hyperlink{site-content}{Skip to
content}\protect\hyperlink{site-index}{Skip to site index}

\href{https://www.nytimes3xbfgragh.onion/section/arts/design}{Art \&
Design}

\href{https://myaccount.nytimes3xbfgragh.onion/auth/login?response_type=cookie\&client_id=vi}{}

\href{https://www.nytimes3xbfgragh.onion/section/todayspaper}{Today's
Paper}

\href{/section/arts/design}{Art \& Design}\textbar{}A Visitor's Guide to
MoMA and the Met

\url{https://nyti.ms/31BXDKX}

\begin{itemize}
\item
\item
\item
\item
\item
\end{itemize}

\hypertarget{the-coronavirus-outbreak}{%
\subsubsection{\texorpdfstring{\href{https://www.nytimes3xbfgragh.onion/news-event/coronavirus?name=styln-coronavirus-national\&region=TOP_BANNER\&block=storyline_menu_recirc\&action=click\&pgtype=Article\&impression_id=0c412b20-f27e-11ea-b9bc-3185cbd6b8a2\&variant=undefined}{The
Coronavirus
Outbreak}}{The Coronavirus Outbreak}}\label{the-coronavirus-outbreak}}

\begin{itemize}
\tightlist
\item
  live\href{https://www.nytimes3xbfgragh.onion/2020/09/08/world/covid-19-coronavirus.html?name=styln-coronavirus-national\&region=TOP_BANNER\&block=storyline_menu_recirc\&action=click\&pgtype=Article\&impression_id=0c415230-f27e-11ea-b9bc-3185cbd6b8a2\&variant=undefined}{Latest
  Updates}
\item
  \href{https://www.nytimes3xbfgragh.onion/interactive/2020/us/coronavirus-us-cases.html?name=styln-coronavirus-national\&region=TOP_BANNER\&block=storyline_menu_recirc\&action=click\&pgtype=Article\&impression_id=0c415231-f27e-11ea-b9bc-3185cbd6b8a2\&variant=undefined}{Maps
  and Cases}
\item
  \href{https://www.nytimes3xbfgragh.onion/interactive/2020/science/coronavirus-vaccine-tracker.html?name=styln-coronavirus-national\&region=TOP_BANNER\&block=storyline_menu_recirc\&action=click\&pgtype=Article\&impression_id=0c415232-f27e-11ea-b9bc-3185cbd6b8a2\&variant=undefined}{Vaccine
  Tracker}
\item
  \href{https://www.nytimes3xbfgragh.onion/2020/09/02/your-money/eviction-moratorium-covid.html?name=styln-coronavirus-national\&region=TOP_BANNER\&block=storyline_menu_recirc\&action=click\&pgtype=Article\&impression_id=0c415233-f27e-11ea-b9bc-3185cbd6b8a2\&variant=undefined}{Eviction
  Moratorium}
\item
  \href{https://www.nytimes3xbfgragh.onion/interactive/2020/09/02/magazine/food-insecurity-hunger-us.html?name=styln-coronavirus-national\&region=TOP_BANNER\&block=storyline_menu_recirc\&action=click\&pgtype=Article\&impression_id=0c415234-f27e-11ea-b9bc-3185cbd6b8a2\&variant=undefined}{American
  Hunger}
\end{itemize}

Advertisement

\protect\hyperlink{after-top}{Continue reading the main story}

Supported by

\protect\hyperlink{after-sponsor}{Continue reading the main story}

\hypertarget{a-visitors-guide-to-moma-and-the-met}{%
\section{A Visitor's Guide to MoMA and the
Met}\label{a-visitors-guide-to-moma-and-the-met}}

What you need to know before you head back to the museums, from safety
precautions to the exhibitions still on view.

\includegraphics{https://static01.graylady3jvrrxbe.onion/images/2020/08/29/arts/29met-sidebar1/merlin_176253825_d75e1ebf-f8a6-4939-802b-3a97174c9eac-articleLarge.jpg?quality=75\&auto=webp\&disable=upscale}

Aug. 28, 2020

\begin{itemize}
\item
\item
\item
\item
\item
\end{itemize}

Before you head back to the Museum of Modern Art and
\href{https://www.nytimes3xbfgragh.onion/2020/08/25/arts/design/met-museum-reopening.html}{the
Metropolitan Museum of Art}, there are some things you need to know:
capacity will be limited to 25 percent, temperature checks and face
masks will be mandatory, and the advance purchase of tickets is
required. Each museum has specific guidelines, so you'll want to visit
their websites: \href{https://www.moma.org/}{moma.org} and
\href{https://www.metmuseum.org/}{metmuseum.org}. Our critics have
reviewed some of the new shows ---
``\href{https://www.nytimes3xbfgragh.onion/2020/08/27/arts/design/moma-reopens-felix-feneon.html}{Félix
Fénéon}'' at MoMA, and
``\href{https://www.nytimes3xbfgragh.onion/2020/08/27/arts/design/met-museum-reopens-anniversary.html}{Making
the Met,}''
``\href{https://www.nytimes3xbfgragh.onion/2020/08/27/arts/design/jacob-lawrence-metropolitan-museum.html}{Jacob
Lawrence}'' and
\href{https://www.nytimes3xbfgragh.onion/2020/08/27/arts/design/met-roof-hector-zamora-wall.html}{Héctor
Zamora's rooftop sculpture} at the Met --- but they also weighed in
previously about several exhibitions that are still on view. Below,
you'll find an overview of those shows, as well as a partial listing of
some of the museums opening in the coming days. \emph{NICOLE HERRINGTON}

\hypertarget{moma}{%
\subsection{MoMA}\label{moma}}

\emph{The museum is now open. MoMA PS1 remains closed, though it plans
to reopen Sept. 17 with the exhibition ``Marking Time: Art in the Age of
Incarceration.''}

\includegraphics{https://static01.graylady3jvrrxbe.onion/images/2020/02/28/arts/27judd-review-cover/27judd-review-cover-articleLarge.jpg?quality=75\&auto=webp\&disable=upscale}

\textbf{`JUDD'} (through Jan. 9) This retrospective of some 70 works by
the American artist Donald Judd is his first in New York in more than 30
years. It ranges from formally spare early abstract sculptures to the
high-color work done before his death in 1994. The show is a beautiful
thing: carefully winnowed, persuasively installed, just the right size.
Judd once said that for art to matter, ``it needs only to be
interesting.'' (Holland Cotter)

\textbf{`DOROTHEA LANGE: WORDS \& PICTURES'} (through Sept. 19) As this
revelatory, heartening exhibition shows, Lange was an artist who made
remarkable pictures throughout a career that covered more than four
decades. The photos she took in 1942 of interned Japanese-Americans
display state-administered cruelty with stone-cold clarity. Her
prescient photographs of environmental degradation portray the human
cost of building a dam. Her empathetic portraits of African-American
field hands shine a light on a system of peonage that predated and
outlasted the 1930s. (Arthur Lubow)

\textbf{THE COLLECTIONS} MoMA recently celebrated its latest expansion
with these inaugural shows, drawing from its collection. \textbf{``Sur
Moderno: Journeys of Abstraction --- the Patricia Phelps de Cisneros
Gift''} (through Sept. 12) presents a selection of South American
postwar art so substantial that it could reorient the museum's focus.
For \textbf{``The Shape of Shape,''} the latest iteration of the
museum's Artist's Choice series, the painter Amy Sillman filled a large
gallery with an astounding array of carefully juxtaposed works from
across the collection (through Oct. 4). \textbf{``Taking a Thread for a
Walk''} (through Jan. 10) looks at the role of weaving in modern art
beyond textiles. And \textbf{``Private Lives Public Spaces''} (through
Feb. 21), a video installation in the galleries just outside the main
movie auditoriums, comprises 47 hours of neglected footage from the
museum's collection. (Roberta Smith)

\hypertarget{the-met}{%
\subsection{The Met}\label{the-met}}

\emph{The museum is open to members now and reopens to the public on
Saturday, but the Cloisters remains closed until Sept. 12. (The Met
Breuer is now officially closed.)}

Image

A detail of ``Kneeling Dignitary,'' Middle Niger civilization, Mali,
12th-14th century, in~``Sahel: Art and Empires on the Shores of the
Sahara.''~Credit...Karsten Moran for The New York Times

\textbf{`SAHEL: ART AND EMPIRES ON THE SHORES OF THE SAHARA'} (through
Oct. 26) Sahel was the name once given by traders crossing the oceanic
Sahara to the welcoming grasslands that marked the desert's southern
rim, terrain that is now Mali, Mauritania, Niger and Senegal. To early
travelers, art from the region must have looked like a rich but
bewildering hybrid. It still does, which may be one reason it stands, in
the West, somewhat outside an accepted ``African'' canon. This fabulous
\href{https://www.metmuseum.org/exhibitions/listings/2020/sahel-art-empire-sahara}{exhibition}
goes for the richness. One look tells you that variety within variety,
difference talking to difference, is the story here. New ideas spring up
from local soil and arrive from afar. Ethnicities and ideologies collide
and embrace. Cultural influences get swapped, dropped and recouped in a
multitrack sequencing that is the very definition of history. (Holland
Cotter)

\textbf{`THE GREAT HALL COMMISSION: KENT MONKMAN, MISTIKOSIWAK (WOODEN
BOAT PEOPLE)'} (through September) These two monumental paintings offer
narratives inspired by a Euro-American tradition of history painting but
are entirely present-tense and polemical in theme. Kent Monkman, a
Canadian artist of mixed Cree and Irish heritage, makes the colonial
violence done to North America's first peoples his central subject but,
crucially, flips the cliché of Native American victimhood on its head.
Here, Indigenous peoples are immigrant-welcoming rescuers, led by the
heroic figure of Monkman's alter ego, the gender-fluid tribal leader
Miss Chief Eagle Testickle, avatar of the global future that will see
humankind moving beyond the wars of identity --- racial, sexual,
political --- in which it is now fatefully immersed. (Holland Cotter)

\textbf{`IN PURSUIT OF FASHION: THE SANDY SCHREIER COLLECTION'} (through
Sept. 27) Featuring 80 pieces of clothing and accessories,
\href{https://www.metmuseum.org/exhibitions/listings/2019/in-pursuit-of-fashion-the-sandy-schreier-collection}{this
exhibition} is, more than anything else, the reflection of one woman's
love affair with fashion. Schreier's collection, and the part of it on
view, contains all the major names, but what defines it more than
anything else is her own appreciation for pretty things. Hidden away
between the Balenciagas and the Chanels, the Diors and the Adrians, are
treasures by little-known or even unknown designers that are a delight
to discover. It is these less famous names whose impact lingers, in part
because they are so unexpected. (Vanessa Friedman)

\textbf{`ARTE DEL MAR: ARTISTIC EXCHANGE IN THE CARIBBEAN'} (through
Jan. 10). This exhibition of art from the West Indies concentrates on
the ritual objects --- thrones, vessels and mysterious bird-shaped
stones --- of the Taíno people, who inhabited the islands now called
Hispaniola, Puerto Rico, Cuba and Turks and Caicos. On these islands,
and on the Caribbean-facing coasts of Central America, styles mingled
and migrated, and art had both religious and diplomatic functions; one
extravagant gold pendant here, in the shape of a bird with splayed wings
and zigzagging necklaces, traveled from Panama all the way to the
Antilles. (Jason Farago)

\hypertarget{other-upcoming-openings}{%
\subsection{Other Upcoming Openings}\label{other-upcoming-openings}}

Image

Some of the New Yorkers featured in the New-York Historical Society's
outdoor exhibition, ``Hope Wanted: New York City Under Quarantine.''
From left, Michelene Wilkerson, Melanie Wilkerson, Leticia Lucero, Kay
Hickman (the project's photographer) and Dara
Wishingrad.Credit...Vincent Tullo for The New York Times

\textbf{Now open:} Museum of the City of New York; the American Folk Art
Museum; and the Whitney Museum of American Art, which is open to members
through Aug. 31 (it opens to the public Sept. 3).

\textbf{Opening in September:} The American Museum of Natural History
(to members Sept. 2 and the general public Sept. 9); the Bronx Museum of
the Arts (Sept. 9); the Museum of Jewish Heritage (to members Sept. 9-10
and the general public Sept. 13); the New-York Historical Society (Sept.
11, though its outdoor exhibition,
``\href{https://www.nytimes3xbfgragh.onion/2020/08/13/arts/design/New-York-Historical-covid-reopen.html}{Hope
Wanted: New York City Under Quarantine},'' is now open); the New Museum
(Sept. 15); and the Guggenheim Museum (to members and patrons Sept.
30-Oct. 2 and the public on Oct. 3).

Advertisement

\protect\hyperlink{after-bottom}{Continue reading the main story}

\hypertarget{site-index}{%
\subsection{Site Index}\label{site-index}}

\hypertarget{site-information-navigation}{%
\subsection{Site Information
Navigation}\label{site-information-navigation}}

\begin{itemize}
\tightlist
\item
  \href{https://help.nytimes3xbfgragh.onion/hc/en-us/articles/115014792127-Copyright-notice}{©~2020~The
  New York Times Company}
\end{itemize}

\begin{itemize}
\tightlist
\item
  \href{https://www.nytco.com/}{NYTCo}
\item
  \href{https://help.nytimes3xbfgragh.onion/hc/en-us/articles/115015385887-Contact-Us}{Contact
  Us}
\item
  \href{https://www.nytco.com/careers/}{Work with us}
\item
  \href{https://nytmediakit.com/}{Advertise}
\item
  \href{http://www.tbrandstudio.com/}{T Brand Studio}
\item
  \href{https://www.nytimes3xbfgragh.onion/privacy/cookie-policy\#how-do-i-manage-trackers}{Your
  Ad Choices}
\item
  \href{https://www.nytimes3xbfgragh.onion/privacy}{Privacy}
\item
  \href{https://help.nytimes3xbfgragh.onion/hc/en-us/articles/115014893428-Terms-of-service}{Terms
  of Service}
\item
  \href{https://help.nytimes3xbfgragh.onion/hc/en-us/articles/115014893968-Terms-of-sale}{Terms
  of Sale}
\item
  \href{https://spiderbites.nytimes3xbfgragh.onion}{Site Map}
\item
  \href{https://help.nytimes3xbfgragh.onion/hc/en-us}{Help}
\item
  \href{https://www.nytimes3xbfgragh.onion/subscription?campaignId=37WXW}{Subscriptions}
\end{itemize}
