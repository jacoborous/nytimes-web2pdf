Sections

SEARCH

\protect\hyperlink{site-content}{Skip to
content}\protect\hyperlink{site-index}{Skip to site index}

\href{https://www.nytimes3xbfgragh.onion/section/style}{Style}

\href{https://myaccount.nytimes3xbfgragh.onion/auth/login?response_type=cookie\&client_id=vi}{}

\href{https://www.nytimes3xbfgragh.onion/section/todayspaper}{Today's
Paper}

\href{/section/style}{Style}\textbar{}Will They See Me as a Son, a
Daughter or a Stranger?

\url{https://nyti.ms/3jmmicu}

\begin{itemize}
\item
\item
\item
\item
\item
\end{itemize}

Advertisement

\protect\hyperlink{after-top}{Continue reading the main story}

Supported by

\protect\hyperlink{after-sponsor}{Continue reading the main story}

Modern Love

\hypertarget{will-they-see-me-as-a-son-a-daughter-or-a-stranger}{%
\section{Will They See Me as a Son, a Daughter or a
Stranger?}\label{will-they-see-me-as-a-son-a-daughter-or-a-stranger}}

Thirteen years ago, my parents tried to wish away my gender transition.
Now, Alzheimer's threatens to erase their memory of me altogether.

\includegraphics{https://static01.graylady3jvrrxbe.onion/images/2020/08/30/fashion/30MODERN-TRANS/30MODERN-TRANS-articleLarge.jpg?quality=75\&auto=webp\&disable=upscale}

By Les Tyler Johnson

\begin{itemize}
\item
  Aug. 28, 2020
\item
  \begin{itemize}
  \item
  \item
  \item
  \item
  \item
  \end{itemize}
\end{itemize}

Unlike my mother's usual emails (with the entire message crammed into
the subject line), this one had no text, only an attached photo. I
clicked on the file and reeled at the picture of her, emaciated, with a
deep gash above her bruised eye.

I called immediately. ``Mom, what happened?''

``I fell. At the flat place by the puzzles.'' Then she said, ``I'm
going,'' and hung up.

Between our terse conversations and images like this, I hardly recognize
the once talkative woman who tended to my needs and listened to my
desires as a child.

In all fairness, as her transgender son, I realize there have been times
when she has found me unrecognizable, too.

Thirteen years ago, when I was 34, I injected myself with testosterone
for the first time and began a physical transition from female to male.
Very quickly, my voice lowered, muscles grew and hair appeared all over
my body, except, unfortunately, where my hairline started to recede.

For the first time, I felt right, whole. The image that greeted me in
the mirror reflected what I had been feeling inside. What didn't
transition so smoothly was my relationship with my parents.

Early on, they wrote me a letter encouraging me to reconsider my
decision to become a man. The message, while cruel, was likely founded
in misinformation, fear and concern for me, but that was no comfort in
the midst of a life-changing journey, which now would not include their
support. Over time, our relationship healed --- somewhat. Mostly, we
just didn't speak about it.

Then, in 2015, my father emailed me, acknowledging his difficulty
recalling words and confirming that a recent MRI indicated early signs
of Alzheimer's. Three years later, my mother learned that she had it,
too.

A visit with them at their house last summer confirmed they were both
entrenched in a mire of dementia from which they will never escape. My
father and I sat in armchairs while my mother looked out the living room
window and said, ``I need to call the tree guy about those white
rectangles,'' wondering aloud why the wind hadn't knocked them from the
branches.

I followed her gaze but couldn't see the rectangles.

``Stand here,'' she said. ``Look.''

It took me a minute to see what she saw: light from the skylights above
us bouncing off the window glass and forming rectangular reflections,
which appeared to her to be outside.

Difficulties with spatial relationships and depth perception are common
among people with Alzheimer's. My parents have the other typical
symptoms also: memory loss, confusion about where they are or what time
it is, inability to hold a conversation, moodiness, paranoia. And, in my
father's case, increasing difficulty with walking and eating.

\emph{{[}}\href{https://www.nytimes3xbfgragh.onion/newsletters/love-letter}{\emph{Sign
up for Love Letter, our weekly email about Modern Love, weddings and
relationships.}}\emph{{]}}

These changes are upsetting to us in different ways. My parents were
trained as journalists and worked as writers. Now they are frustrated by
their inability to use language in ways that used to feel so natural. My
father rarely speaks, and my mother can only form sentences on good
days. Even then, she calls keys ``stuff that goes with my car.''

This, and the slow erosion of their independence, starting with the
cancellation of their drivers' licenses and the opening of their home to
round-the-clock care, has them feeling defeated. As a child of parents
with a terminal illness, I am obviously upset, too. Unexpected, however,
is the sense that my identities as a man, husband and father --- all
predicated on my gender transition --- seem to be falling away, too, as
their dementia progresses and they forget who I am.

My gender shift was just the start of many changes: a year after I began
my transition, a mutual friend introduced me, via email, to a bold woman
with a big personality and irreverent sense of humor. We didn't meet in
person until two months later when I picked her up at the airport on a
Friday night. We got married that Sunday.

In the years since, we adopted two boys from foster care when each was
9: our first in 2012 and his younger brother three years later. Then, in
a manner of months after the November 2016 presidential election, the
four of us left our home, jobs and friends in the United States and
emigrated to Canada.

People are often curious about the speed with which these life changes
occurred. Wasn't it risky to marry someone after knowing them for only a
few days? Didn't you miss too much of your children's lives, adopting
them when they were older? Weren't you stressed deciding in March to
move to a new country in May?

I usually respond that these changes, although major, pale in comparison
to changing one's gender. Once you've done that, nothing else seems as
risky, fast or profound.

Yet, the pace of these changes is matched only by the speed of my
parents' decline.

The nature of Alzheimer's is that affected individuals often can recall
memories from their distant past more easily than recent events. On good
days, my mother can describe how in the late 1960s she ``found'' my
father while working as a reporter at a small newspaper in Kansas,
where, on one of the top floors, there was a ``single desk and one chair
to share.'' When my father was brought in as a guest reporter on loan
from The Kansas City Star, that week's top story was: ``Love Connection
in the Newsroom.''

They both remember the birth of their first child too: a daughter. They
named me Lesley, after the nurse whose sense of urgency saved my life.
The story my mother still likes to tell is that the umbilical cord was
wrapped around my neck four times. As she says, ``Lesley, the nurse,
told the doctor, `Get off the golf course and deliver this baby!'''

Their memories of my new life in Canada, however --- along with the
adoption of grandchildren they adore, my marriage and, most of all, my
gender transition --- are receding, if not already gone.

``Honey, what is the name of that man? Do you remember?'' My mother
pointed at the massive, plaster head atop a wooden platform in their
living room. This sculpture of their friend Ernest was one of my
father's more successful forays into three-dimensional art. It used to
occupy a prominent space in the hallway as one of the first things
people saw when they entered my childhood home.

Hearing a familiar voice, my father turned to look in her direction.

``Never mind,'' my mother said, moving toward the shelf. ``I bet I kept
a photograph of him. That's something I would do.''

She grabbed the head, tipped it to its side, reached up into the neck
and pulled out a photograph. It was a photo of the sculpture itself, not
of Ernest.

She looked at the sculpture and compared it to the picture. ``Wow, you
have such amazing talent,'' she said to my father. ``Your sculpture
looks just like the man in the photo!''

If my parents are unable to recognize the difference between a
photograph of a white plaster sculpture and one of an actual person, how
much longer will they be able to recognize me as their child, much less
their son?

Last summer, while I was sitting with my parents at their kitchen table,
my mother finished her lunch, stood up and announced, ``I'm going
outside to pull some weeds.''

I waited indoors with my father as he finished his meal, until tears
began flowing down his face.

``Dad, what's wrong?'' I said.

``I, I just have to \ldots{} get used \ldots{} she's gone forever.''

He apparently believed that because she had left his sight, she was
dead.

I used to ask myself: ``Will I lose my new identities once I can't be
identified?'' But after witnessing my father sobbing at the table, I
have learned the answer. My mother just left the room. The fact that he
didn't remember didn't mean she no longer existed.

Likewise, the fact that my parents will lose their memory of my
transition won't mean I will cease to be a man, a devoted husband, a
loving dad and my parents' son.

My parents were instrumental in creating the framework that supports who
I have become. Then, over the last 40 years, they witnessed me assuming
the responsibility of building upon that foundation. Although they
haven't always agreed with my choices, they came to accept me for who I
am.

As Alzheimer's eats away at their sense of who I was and who I have
become, I hope it will be my essence that they retain. That's what
matters most, for any of us.

\href{https://www.lestjohnson.com/}{Les Tyler Johnson} is a teacher and
writer who lives in Halifax, Nova Scotia.

Modern Love can be reached at
\href{mailto:modernlove@NYTimes.com}{\nolinkurl{modernlove@NYTimes.com}}.

Want more from Modern Love? Watch the
\href{https://www.nytimes3xbfgragh.onion/2019/09/12/style/modern-love-tv-show-trailer.html}{TV
series}; sign up for the
\href{https://www.nytimes3xbfgragh.onion/newsletters/love-letter}{newsletter};
or listen to the
\href{https://www.nytimes3xbfgragh.onion/column/modern-love-podcast}{podcast}
on
\href{https://itunes.apple.com/us/podcast/modern-love/id1065559535?mt=2\&version=meter+at+0\&module=meter-Links\&pgtype=article\&contentId=\&mediaId=\&referrer=\&priority=true\&action=click\&contentCollection=meter-links-click}{iTunes},
\href{https://open.spotify.com/show/03Er7mSPq9IEewOgbPD3vO}{Spotify} or
\href{https://play.google.com/music/listen?u=0\#/ps/Iktqjbkz7bychbnofblw32dik64}{Google
Play}. We also have swag at
\href{https://store.nytimes3xbfgragh.onion/collections/modern-love}{the
NYT Store} and a book,
``\href{https://www.penguinrandomhouse.com/books/623036/modern-love-revised-and-updated-by-edited-by-daniel-jones-with-contributions-by-andrew-rannells-ayelet-waldman-amy-krouse-rosenthal-veronica-chambers-and-more/}{Modern
Love: True Stories of Love, Loss, and Redemption}.''

Advertisement

\protect\hyperlink{after-bottom}{Continue reading the main story}

\hypertarget{site-index}{%
\subsection{Site Index}\label{site-index}}

\hypertarget{site-information-navigation}{%
\subsection{Site Information
Navigation}\label{site-information-navigation}}

\begin{itemize}
\tightlist
\item
  \href{https://help.nytimes3xbfgragh.onion/hc/en-us/articles/115014792127-Copyright-notice}{©~2020~The
  New York Times Company}
\end{itemize}

\begin{itemize}
\tightlist
\item
  \href{https://www.nytco.com/}{NYTCo}
\item
  \href{https://help.nytimes3xbfgragh.onion/hc/en-us/articles/115015385887-Contact-Us}{Contact
  Us}
\item
  \href{https://www.nytco.com/careers/}{Work with us}
\item
  \href{https://nytmediakit.com/}{Advertise}
\item
  \href{http://www.tbrandstudio.com/}{T Brand Studio}
\item
  \href{https://www.nytimes3xbfgragh.onion/privacy/cookie-policy\#how-do-i-manage-trackers}{Your
  Ad Choices}
\item
  \href{https://www.nytimes3xbfgragh.onion/privacy}{Privacy}
\item
  \href{https://help.nytimes3xbfgragh.onion/hc/en-us/articles/115014893428-Terms-of-service}{Terms
  of Service}
\item
  \href{https://help.nytimes3xbfgragh.onion/hc/en-us/articles/115014893968-Terms-of-sale}{Terms
  of Sale}
\item
  \href{https://spiderbites.nytimes3xbfgragh.onion}{Site Map}
\item
  \href{https://help.nytimes3xbfgragh.onion/hc/en-us}{Help}
\item
  \href{https://www.nytimes3xbfgragh.onion/subscription?campaignId=37WXW}{Subscriptions}
\end{itemize}
