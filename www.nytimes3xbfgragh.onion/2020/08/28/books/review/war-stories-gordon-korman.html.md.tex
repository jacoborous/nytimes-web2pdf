Sections

SEARCH

\protect\hyperlink{site-content}{Skip to
content}\protect\hyperlink{site-index}{Skip to site index}

\href{https://www.nytimes3xbfgragh.onion/section/books/review}{Book
Review}

\href{https://myaccount.nytimes3xbfgragh.onion/auth/login?response_type=cookie\&client_id=vi}{}

\href{https://www.nytimes3xbfgragh.onion/section/todayspaper}{Today's
Paper}

\href{/section/books/review}{Book Review}\textbar{}Children in a World
Gone Mad: 5 Kids' Books Set During World War II

\url{https://nyti.ms/31AW9jQ}

\begin{itemize}
\item
\item
\item
\item
\item
\end{itemize}

Advertisement

\protect\hyperlink{after-top}{Continue reading the main story}

Supported by

\protect\hyperlink{after-sponsor}{Continue reading the main story}

\href{/column/childrens-books}{Children's Books}

\hypertarget{children-in-a-world-gone-mad-5-kids-books-set-during-world-war-ii}{%
\section{Children in a World Gone Mad: 5 Kids' Books Set During World
War
II}\label{children-in-a-world-gone-mad-5-kids-books-set-during-world-war-ii}}

\includegraphics{https://static01.graylady3jvrrxbe.onion/images/2020/08/30/books/review/30Wein/30Wein-articleLarge.jpg?quality=75\&auto=webp\&disable=upscale}

By Elizabeth Wein

\begin{itemize}
\item
  Aug. 28, 2020
\item
  \begin{itemize}
  \item
  \item
  \item
  \item
  \item
  \end{itemize}
\end{itemize}

Their World War II settings are what make these five titles obvious
companions. But their tributes to family are what strongly connect them.
All five commemorate the generational and lateral bonds that sustain
children across the globe, and that are all too easily broken.

Dip into these very different books and you'll find resentment and
reconciliation between siblings, heartbreak as extended families are
obliterated, strength as old men open their homes to grandchildren and
joy as a new generation is brought into the world. You'll be transported
to New York, London, France, Cuba, Poland, Russia, Turkestan; you'll
weep at the shocking randomness by which lives are suddenly torn apart.
But you'll come away convinced that in times of crisis those who love us
--- or who do their best to love us --- are the key to our survival.

\textbf{THE SUMMER WE FOUND THE BABY}\\
By Amy Hest\\
181 pp. Candlewick. \$16.99.\\
(Ages 10 and up)

Amy Hest gently and expertly explores these themes in ``The Summer We
Found the Baby*.''* The story takes place on the American home front in
an undisclosed year and is related by three instantly engaging
narrators: 11-year-old Julie and 6-year-old Martha Sweet, summering in
Belle Beach, Long Island, and their neighbor 12-year-old Bruno Ben-Eli,
son of the local grocer. The baby of the title turns up on the doorstep
of the new children's library on the day of a gala celebrating its
opening, and the rest of the summer is remembered in breezy flashbacks
as the three narrators chase one another around trying to do what's best
for the mysterious and patient infant in the basket. The characters are
exuberant and full of personality, but the whiff of tragedy in the
background is genuine: The Sweets' mother is dead and Bruno's adored
older brother Ben is serving his country in the Pacific. The reality of
combat is never far away, as wounded soldiers recuperate in the nearby
military hospital and the whole town participates in all-too-frequent
oceanside memorial services whenever another local boy is killed.
Through it all, the novel wears its wartime and historical mantle with a
summer lightness of spirit; loss never quenches its persistent
undercurrent of hope.

\textbf{RIP TO THE RESCUE}\\
By Miriam Halahmy\\
208 pp. Holiday House. \$16.99.\\
(Ages 8 to 12)

A leap across the Atlantic takes us to another home front in Miriam
Halahmy's ``Rip to the Rescue,'' as we follow gutsy Jack Castle through
the bomb-strewn streets of London at the height of the Blitz. At 13,
Jack lies about his age to become an air raid bicycle messenger,
providing a vital service as phone lines are severed and city streets
are blocked by wreckage. ``Rip'' is Jack's dog, based on Britain's
original search-and-rescue dog, a mixed-breed terrier who saved more
than 100 lives during World War II thanks to his ability to identify
survivors trapped under rubble.

Jack's fictional role as Blitz messenger boy and Rip's owner creates the
framework for a nail-biting adventure, but this is also a tale of
friendship and family, in which no one is completely whole. Jack's
friend Paula desperately tries to hide her Jewish heritage while she
stockpiles supplies in anticipation of a Nazi invasion. Jack strives to
prove his worth to his depressed amputee father, and to face up to
schoolmates who mock his one deaf ear. ``Useless'' and ``Deaf Nellie''
by day, fearless Blitz messenger boy by night --- wearing the uniform he
secretly keeps at his granddad's --- Jack is a true superhero. The
metaphor goes unspoken but will surely appeal to young readers.

\textbf{WAR STORIES}\\
By Gordon Korman\\
240 pp. Scholastic. \$17.99.\\
(Ages 8 to 12)

Gordon Korman brings the gritty horrors of combat to life in ``War
Stories.'' The exhaustively destructive liberation of France in 1944 is
seen through the eyes of the 17-year-old U.S. Army G.I. Jacob Firestone,
known as ``High School'' to the rest of his squad because he's so young.
Jacob shares ``war stories'' through a structure I think of as
``generational flashback,'' allowing the contemporary reader to relate
to Jacob's great-grandson, Trevor, who at 12 is obsessed via video games
with the military side of the European theater. Trevor, his peace-loving
dad, Daniel (a history teacher whose store of facts comes in handy for
background information), and the irrepressibly confident old soldier
Jacob (now 93) make a pilgrimage to France on the 75th anniversary of
V-E (Victory in Europe) Day. The shadowy menace of La Vérité, a group
bent on revenge for a mysterious error of judgment committed by Jacob 75
years earlier, adds modern contrast and tension. Its intimidation of
Trevor's family includes slashed tires and a fake bomb in their hotel
room. The concept of a personal vendetta replicating war, in the form of
modern terrorism, brings the past uncomfortably close both for Trevor
and for the reader.

\textbf{LETTERS FROM CUBA}\\
By Ruth Behar\\
272 pp. Nancy Paulsen Books. \$17.99.\\
(Ages 10 and up)

Dispossession and flight are the lenses through which wartime is viewed
in Ruth Behar's ``Letters From Cuba,'' based on the little-known saga of
Jews who emigrated from Poland to Cuba to escape persecution in the
decade prior to World War II. The novel is framed as a series of
letters, mostly written in 1938 from the 11-year-old émigré Esther to
her little sister Malka as Esther and her father struggle to earn enough
money to bring the rest of the family to Cuba from increasingly
dangerous and hostile Poland. Practical, hardworking Esther sets up a
dressmaking business as she learns to speak and read Spanish; she makes
friends with a cosmopolitan and diverse group of Cuban citizens,
including Chinese immigrants and the granddaughter of a former African
slave, and participates in their various religious celebrations without
ever losing her Jewish faith. This is a quiet story of determination,
and an openly loving tribute to the author's grandmother, who made the
real journey that inspired Esther's fictional one.

\textbf{CHANCE}\\
\textbf{Escape From the Holocaust}\\
By Uri Shulevitz\\
336 pp. Farrar, Straus \& Giroux. \$19.99.\\
(Ages 8 to 14)

Rounding out this eclectic selection of books is one more journey ---
this one autobiographical. The Caldecott Medalist Uri Shulevitz's
``Chance: Escape From the Holocaust'' is a harrowing, engaging and
utterly honest account of the author's childhood, forged in the crucible
of war and affected by it long after it is over. Uri is 4 when German
bombs begin falling on Warsaw. Following a serendipitous conversation
with a Jewish refugee, his father drags him and his mother out of Poland
and into Russia. There they languish for a year and a half in a
detention camp in the far north Archangel region near the White Sea
until after the German invasion of the Soviet Union. They spend the
remainder of the war in Turkestan, foreign pariahs on the verge of
starvation.

Blind luck plays an enormous role in Uri's survival, and the one
consistency in his terrifying, nomadic young life is the urge to draw.
This memoir is lavishly illustrated with the author's own drawings ---
some of them modern renderings that complement the text, others
astonishing childhood originals. Told without bitterness, in a
relatable, straightforward voice, it reminds us that creativity and
survival go hand in hand. For a child in a world gone mad, encouragement
to embrace that creative urge is sometimes the greatest love a parent
can bestow.

Advertisement

\protect\hyperlink{after-bottom}{Continue reading the main story}

\hypertarget{site-index}{%
\subsection{Site Index}\label{site-index}}

\hypertarget{site-information-navigation}{%
\subsection{Site Information
Navigation}\label{site-information-navigation}}

\begin{itemize}
\tightlist
\item
  \href{https://help.nytimes3xbfgragh.onion/hc/en-us/articles/115014792127-Copyright-notice}{©~2020~The
  New York Times Company}
\end{itemize}

\begin{itemize}
\tightlist
\item
  \href{https://www.nytco.com/}{NYTCo}
\item
  \href{https://help.nytimes3xbfgragh.onion/hc/en-us/articles/115015385887-Contact-Us}{Contact
  Us}
\item
  \href{https://www.nytco.com/careers/}{Work with us}
\item
  \href{https://nytmediakit.com/}{Advertise}
\item
  \href{http://www.tbrandstudio.com/}{T Brand Studio}
\item
  \href{https://www.nytimes3xbfgragh.onion/privacy/cookie-policy\#how-do-i-manage-trackers}{Your
  Ad Choices}
\item
  \href{https://www.nytimes3xbfgragh.onion/privacy}{Privacy}
\item
  \href{https://help.nytimes3xbfgragh.onion/hc/en-us/articles/115014893428-Terms-of-service}{Terms
  of Service}
\item
  \href{https://help.nytimes3xbfgragh.onion/hc/en-us/articles/115014893968-Terms-of-sale}{Terms
  of Sale}
\item
  \href{https://spiderbites.nytimes3xbfgragh.onion}{Site Map}
\item
  \href{https://help.nytimes3xbfgragh.onion/hc/en-us}{Help}
\item
  \href{https://www.nytimes3xbfgragh.onion/subscription?campaignId=37WXW}{Subscriptions}
\end{itemize}
