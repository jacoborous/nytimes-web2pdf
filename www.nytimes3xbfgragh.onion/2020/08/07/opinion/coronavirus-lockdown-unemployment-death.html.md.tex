Sections

SEARCH

\protect\hyperlink{site-content}{Skip to
content}\protect\hyperlink{site-index}{Skip to site index}

\href{https://myaccount.nytimes3xbfgragh.onion/auth/login?response_type=cookie\&client_id=vi}{}

\href{https://www.nytimes3xbfgragh.onion/section/todayspaper}{Today's
Paper}

\href{/section/opinion}{Opinion}\textbar{}Here's How to Crush the Virus
Until Vaccines Arrive

\url{https://nyti.ms/2C93aPd}

\begin{itemize}
\item
\item
\item
\item
\item
\item
\end{itemize}

Advertisement

\protect\hyperlink{after-top}{Continue reading the main story}

\href{/section/opinion}{Opinion}

Supported by

\protect\hyperlink{after-sponsor}{Continue reading the main story}

\hypertarget{heres-how-to-crush-the-virus-until-vaccines-arrive}{%
\section{Here's How to Crush the Virus Until Vaccines
Arrive}\label{heres-how-to-crush-the-virus-until-vaccines-arrive}}

To save lives, and save the economy, we need another lockdown.

By Michael T. Osterholm and Neel Kashkari

Dr. Osterholm is director of the Center for Infectious Disease Research
and Policy at the University of Minnesota. Mr. Kashkari is president of
the Federal Reserve Bank of Minneapolis.

\begin{itemize}
\item
  Aug. 7, 2020
\item
  \begin{itemize}
  \item
  \item
  \item
  \item
  \item
  \item
  \end{itemize}
\end{itemize}

\includegraphics{https://static01.graylady3jvrrxbe.onion/images/2020/08/10/opinion/10Osterholm1/07Osterholm1-articleLarge.jpg?quality=75\&auto=webp\&disable=upscale}

In just weeks we could almost stop the viral fire that has swept across
this country over the past six months and continues to rage out of
control. It will require sacrifice but save many thousands of lives.

We believe the choice is clear. We can continue to allow the coronavirus
to spread rapidly throughout the country or we can commit to a more
restrictive lockdown, state by state, for up to six weeks to crush the
spread of the virus to less than one new case per 100,000 people per
day.

That's the point at which we will be able to limit the increase in new
cases through aggressive public health measures, just as other countries
have done. But we're a long way from there right now.

The imperative for this is clear because as a nation what we have done
so far hasn't worked. Some 160,000 people have died, and in recent days,
roughly a thousand have died a day. An estimated
\href{https://www.nytimes3xbfgragh.onion/2020/08/06/business/economy/unemployment-claims.html}{30
million} Americans are collecting unemployment.

On Jan. 30, when the World Health Organization declared Covid-19 a
public health emergency, there were 9,439 reported cases worldwide, most
in China, and only
\href{https://www.cnn.com/asia/live-news/coronavirus-outbreak-01-30-20-intl-hnk/h_4263dd94af73bd404b425b133637a0e7}{six}
reported cases in the United States.

On July 30, six months later, there were 17 million cases reported
worldwide, including 676,000 deaths. The United States had four million
reported cases and 155,000 deaths. More than a third of all U.S. cases
occurred during July alone.

And the next six months could make what we have experienced so far seem
like just a warm-up to a greater catastrophe. With many schools and
colleges starting, stores and businesses reopening, and the beginning of
the indoor heating season, new case numbers will grow quickly.

Why did the United States' Covid-19 containment response fail,
particularly compared with the successful results of so many nations in
Asia, Europe and even our neighbor Canada?

Simply, we gave up on our lockdown efforts to control virus transmission
well before the virus was under control. Many other countries didn't let
up until the number of cases was greatly reduced, even in places that
had extensive outbreaks in March and April. Once the number of new cases
in those areas was driven to less than one per 100,000 people per day as
a result of their lockdowns, limiting the increase of new cases was
possible with a combination of testing, contact tracing, case isolation
and extensive monitoring of positive tests.

The United States recorded its lowest seven-day average since March 31
on May 28, when it was 21,000 cases, or 6.4 new cases per 100,000 people
per day. This rate was seven to 10 times higher than the rates in
countries that successfully contained their new infections. While many
countries are now experiencing modest flare-ups of the virus, their case
loads are in the hundreds or low thousands of infections per day, not
tens of thousands, and small enough that public health officials can
largely control the spread.

In contrast, the United States reopened too quickly and is now
experiencing around 50,000 or more new cases per day.

While cases are falling in the hard-hit areas of Arizona, California,
Florida and Texas because of the imposition of some physical-distancing
measures, they are rapidly increasing in a few of Midwestern states. In
Minnesota, we just documented the most new cases in a one-week period
since the pandemic began.

At this level of national cases --- 17 new cases per 100,000 people per
day --- we simply don't have the public health tools to bring the
pandemic under control. Our testing capacity is overwhelmed in many
areas, resulting in delays that make contact tracing and other measures
to control the virus virtually impossible.

Don't confuse short-term case reductions in some states as permanent. We
made that mistake before. Some have claimed that the widespread use of
masks is enough to control the pandemic, but let us face reality: Gov.
Gavin Newsom of California issued a public masking mandate on June 18, a
day when 3,700 cases were reported in the state. On July 25, the
seven-day daily case average was 10,231. We support the wearing of masks
by all Americans, but masking mandates and soft limitations on indoor
crowds in places such as bars and restaurants are not enough to control
this pandemic.

To successfully drive down our case rate to less than one per 100,000
people per day, we should mandate sheltering in place for everyone but
the truly essential workers. By that, we mean people must stay at home
and leave only for essential reasons: food shopping and visits to
doctors and pharmacies while wearing masks and washing hands frequently.
According to the Economic Policy Institute, 39 percent of workers in the
United States are in essential categories. The problem with the
March-to-May lockdown was that it was not uniformly stringent across the
country. For example, Minnesota deemed 78 percent of its workers
essential. To be effective, the lockdown has to be as comprehensive and
strict as possible.

If we aren't willing to take this action, millions more cases with many
more deaths are likely before a vaccine might be available. In addition,
the economic recovery will be much slower, with far more business
failures and high unemployment for the next year or two. The path of the
virus will determine the path of the economy. There won't be a robust
economic recovery until we get control of the virus.

If we do this aggressively, the testing and tracing capacity we've built
will support reopening the economy as other countries have done, allow
children to go back to school and citizens to vote in person in
November. All of this will lead to a stronger, faster economic recovery,
moving people from unemployment to work.

We know that a stringent lockdown can have serious health consequences
for patients who can't get access to routine care. But over the past six
months, medical professionals have learned how to protect patients and
staffs from spreading the coronavirus; therefore we should be able to
maintain access to regular medical care during any new lockdown.

This pandemic is deeply unfair. Millions of low-wage, front-line service
workers have lost their jobs or been put in harm's way, while most
higher-wage, white-collar workers have been spared. But it is even more
unfair than that; those of us who've kept our jobs are actually saving
more money because we aren't going out to restaurants or movies, or on
vacations. Unlike in prior recessions, remarkably, the personal savings
rate has soared to 20 percent from around 8 percent in January.

Because we are saving more, we have the resources to support those who
have been laid off. Typically when the government runs deficits, it must
rely on foreign investors to buy the debt because Americans aren't
generating enough savings to fund it. But we can finance the added
deficits for Covid-19 relief from our own domestic savings. Those
savings end up funding investment in the economy. That's why traditional
concerns about racking up too much government debt do not apply in this
situation. It is much safer for a country to fund its deficits
domestically than from abroad.

Congress should be aggressive in supporting people who've lost jobs
because of Covid-19. It's not only the right thing to do but also vital
for our economic recovery. If people can't pay their bills, it will
ripple through the economy and make the downturn much worse, with many
more bankruptcies, and the national recovery much slower.

There is no trade-off between health and the economy. Both require
aggressively getting control of the virus. History will judge us harshly
if we miss this life- and economy-saving opportunity to get it right
this time.

\href{https://www.cidrap.umn.edu/about-us/cidrap-staff/michael-t-osterholm-phd-mph}{Michael
T. Osterholm} is a professor and director of the Center for Infectious
Disease Research and Policy at the University of Minnesota.
\href{https://www.federalreservehistory.org/people/neel_kashkari}{Neel
Kashkari} is president of the Federal Reserve Bank of Minneapolis.

\emph{The Times is committed to publishing}
\href{https://www.nytimes3xbfgragh.onion/2019/01/31/opinion/letters/letters-to-editor-new-york-times-women.html}{\emph{a
diversity of letters}} \emph{to the editor. We'd like to hear what you
think about this or any of our articles. Here are some}
\href{https://help.nytimes3xbfgragh.onion/hc/en-us/articles/115014925288-How-to-submit-a-letter-to-the-editor}{\emph{tips}}\emph{.
And here's our email:}
\href{mailto:letters@NYTimes.com}{\emph{letters@NYTimes.com}}\emph{.}

\emph{Follow The New York Times Opinion section on}
\href{https://www.facebookcorewwwi.onion/nytopinion}{\emph{Facebook}}\emph{,}
\href{http://twitter.com/NYTOpinion}{\emph{Twitter (@NYTopinion)}}
\emph{and}
\href{https://www.instagram.com/nytopinion/}{\emph{Instagram}}\emph{.}

Advertisement

\protect\hyperlink{after-bottom}{Continue reading the main story}

\hypertarget{site-index}{%
\subsection{Site Index}\label{site-index}}

\hypertarget{site-information-navigation}{%
\subsection{Site Information
Navigation}\label{site-information-navigation}}

\begin{itemize}
\tightlist
\item
  \href{https://help.nytimes3xbfgragh.onion/hc/en-us/articles/115014792127-Copyright-notice}{©~2020~The
  New York Times Company}
\end{itemize}

\begin{itemize}
\tightlist
\item
  \href{https://www.nytco.com/}{NYTCo}
\item
  \href{https://help.nytimes3xbfgragh.onion/hc/en-us/articles/115015385887-Contact-Us}{Contact
  Us}
\item
  \href{https://www.nytco.com/careers/}{Work with us}
\item
  \href{https://nytmediakit.com/}{Advertise}
\item
  \href{http://www.tbrandstudio.com/}{T Brand Studio}
\item
  \href{https://www.nytimes3xbfgragh.onion/privacy/cookie-policy\#how-do-i-manage-trackers}{Your
  Ad Choices}
\item
  \href{https://www.nytimes3xbfgragh.onion/privacy}{Privacy}
\item
  \href{https://help.nytimes3xbfgragh.onion/hc/en-us/articles/115014893428-Terms-of-service}{Terms
  of Service}
\item
  \href{https://help.nytimes3xbfgragh.onion/hc/en-us/articles/115014893968-Terms-of-sale}{Terms
  of Sale}
\item
  \href{https://spiderbites.nytimes3xbfgragh.onion}{Site Map}
\item
  \href{https://help.nytimes3xbfgragh.onion/hc/en-us}{Help}
\item
  \href{https://www.nytimes3xbfgragh.onion/subscription?campaignId=37WXW}{Subscriptions}
\end{itemize}
