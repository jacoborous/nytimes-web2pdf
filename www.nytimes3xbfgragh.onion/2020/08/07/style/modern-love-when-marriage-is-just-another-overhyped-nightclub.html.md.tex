Sections

SEARCH

\protect\hyperlink{site-content}{Skip to
content}\protect\hyperlink{site-index}{Skip to site index}

\href{https://www.nytimes3xbfgragh.onion/section/style}{Style}

\href{https://myaccount.nytimes3xbfgragh.onion/auth/login?response_type=cookie\&client_id=vi}{}

\href{https://www.nytimes3xbfgragh.onion/section/todayspaper}{Today's
Paper}

\href{/section/style}{Style}\textbar{}When Marriage Is Just Another
Overhyped Nightclub

\url{https://nyti.ms/3aaanej}

\begin{itemize}
\item
\item
\item
\item
\item
\end{itemize}

Advertisement

\protect\hyperlink{after-top}{Continue reading the main story}

Supported by

\protect\hyperlink{after-sponsor}{Continue reading the main story}

Modern Love

\hypertarget{when-marriage-is-just-another-overhyped-nightclub}{%
\section{When Marriage Is Just Another Overhyped
Nightclub}\label{when-marriage-is-just-another-overhyped-nightclub}}

Being single in your 30s can feel like waiting to enter a popular club,
only to get in and think: What's the big deal?

\includegraphics{https://static01.graylady3jvrrxbe.onion/images/2020/08/09/fashion/09MODERN-MARRIAGECLUB/09MODERN-MARRIAGECLUB-articleLarge.jpg?quality=75\&auto=webp\&disable=upscale}

By Katerina Tsasis

\begin{itemize}
\item
  Published Aug. 7, 2020Updated Aug. 8, 2020
\item
  \begin{itemize}
  \item
  \item
  \item
  \item
  \item
  \end{itemize}
\end{itemize}

\href{https://www.nytimes3xbfgragh.onion/es/2020/08/16/espanol/estilos-de-vida/soltera-amor.html}{Leer
en español}

People treat you differently when you are steadily single. Not everyone,
not all the time, not always overtly, not necessarily unkindly. They ask
why no one has snatched you up, offer to set you up on blind dates, seat
you at the singles table at formal events. They extend last-minute
invitations to dinner parties when someone else has bailed.

They make you feel as if you are not the norm, despite the fact that
U.S. census data tells us singlehood is, in fact, increasingly the norm.

As a child, I belonged to an immigrant community that viewed marriage
and motherhood as a woman's primary goal in life. The stories around me
were full of weddings as happy endings: ``Friends,'' ``Sex and the
City,'' ``Full House.'' Every romcom. Every sitcom. ``Pride and
Prejudice,'' ``Little Women,'' every fairy tale. Brangelina, Kim and
Kanye, the outsize interest Americans take in British royal weddings.

I did the typical things: went to college, worked, made friends, went
out, met men in bars, at school, at the office. Meeting people was easy;
forging relationships was hard. It was the early 2000s in Los Angeles, a
place where it seemed everyone wanted to keep their options open. I
frequently found myself in relationship purgatory --- seeing someone but
not really dating, dating but not in a relationship, or in a
relationship but not one with a future.

\emph{{[}}\href{https://www.nytimes3xbfgragh.onion/newsletters/love-letter}{\emph{Sign
up for Love Letter, our weekly email about Modern Love, weddings and
relationships.}}\emph{{]}}

It was around this time that my younger sister finished college and
announced her engagement. I was about to move overseas to attend an
M.B.A. program. Commentary from the auntie-types in my life became more
pointed. ``Don't wait too long!'' they teased, joking but not really.
From their point of view, I was spending time on the wrong priorities.
At 26, I needed to get down to business.

``Are you still planning to go?'' my mother asked.

Here's another thing that happens when you're single: Your time and
plans are perceived as less fixed and less valid than for people who are
married.

You're the one expected to make long schleps to see loved ones for the
holidays or to stay later at work when your colleagues need to pick up
their children. With my sister's wedding on the horizon, there was an
unspoken expectation that I wouldn't miss any of the lead-up to the
happy event.

I went to Europe anyway.

When I traveled home for my sister's wedding, the customs agent was
confused by my rumpled messenger bag with its two changes of clothes.

``That's all you've got?'' he said.

I have never felt less encumbered before or since, packing so light it
felt like I was floating, eager to get back to my adventures.

Over that next year I learned new subjects, traveled to a dozen
countries, practiced speaking other languages, watched an opera staged
on the steps of a castle, hiked Mount Kilimanjaro, drove the terrifying
roundabout at the Arc de Triomphe.

It was also a year in which I experienced aggressive advances from male
classmates, ``locker room talk'' peppered into casual conversation, and
a steady stream of low-key sexism and mansplaining. The idea of dating
had never felt more daunting or less appealing.

When I returned to California, I found many of my friends had settled
into serious relationships that were leading toward marriage. At this
point, I had stopped believing one needed a partner to be fulfilled in
life, but I still thought I must be lacking in some fundamental way ---
not good enough, attractive enough, nice enough, or something enough ---
in comparison.

Friends, relatives, acquaintances and even strangers will obligingly
point out what you, as a single person, seem to lack. A friend of mine
went to see a doctor regarding a mental health question and his
prescription was that she needed a boyfriend. Well-meaning relatives
urged her to go to church to find a man, even though she's agnostic.

I have been told I'm too picky, not getting any younger, should put
myself out there more, have to fight for love, and should look for a guy
who's more attractive and less attractive, more nerdy and less nerdy,
more assertive and less assertive.

Men I have barely known or haven't known at all have told me I should
wear more makeup, change my attitude, do more situps, dress differently,
smile more. I've heard it on a first date, walking down the street
minding my business, and in the middle of a conversation about a totally
different subject.

It's a strange thing to continue to look for the ``right'' person while
bristling against the expectation to do just that. I kept meeting
people: happy hours, meetup groups, online dating. I tried new things:
Salsa dancing! Scooter rides! Spelunking! I spent time on friendships,
hobbies, adventures.

Mixed in with the fun stuff were sad and lonely moments, bad
relationships and painful breakups, but I no longer believed that I was
lacking, despite the cues I continued to receive from friends, family,
society. Life felt good, fulfilling and full. I didn't have to wait for
someone else to create my happily ever after.

By my mid-30s I had moved to Austin, Texas, and my parents fretted about
me long-distance. Their lives hadn't been easy, and they had only had
each other to lean on. My father worried I wouldn't have anyone to take
care of me. What if I got sick? What if I needed help?

My mother, bewildered at my inability to find someone, said, ``It's not
like she has three heads!''

I dated more. Coffee dates that fizzled out faster than foam on a
cappuccino. A happy hour date where I drank too much on an empty stomach
and bought a round for the bar. A dinner date with someone who kept
excusing himself to answer his phone. A relationship with someone who
wasn't ready to commit. A relationship with someone who pined for an ex.

And then, a relationship that worked.

There wasn't any magic about it, no soul awakening, no personal
reckoning, no neat and tidy reason as to why it worked where the others
hadn't. I met a man who is a lovely human being. We found shared
interests and chemistry. We treated each other with kindness and
respect. I'm pretty sure if I had met him years before, or years later,
the outcome would have been the same: We got married.

I'm the same person, living in the same place, doing the same job, with
the same friends and the same hobbies. There was nothing worse about me
before. There is nothing better about me now. And yet, people who
treated my singlehood with curiosity, pity or disregard are now warmer
and more welcoming. It's as if I have joined the club.

I am asked fewer questions about my personal life. My spouse and I are
invited on outings with other couples. It's accepted without question or
complaint when I decide to stay home for the holidays instead of
traveling to visit extended family. Unwanted advances are cut short by
the words ``I'm married'' when a ``No, thank you'' wasn't adequate
before.

What does another person's legal declaration really say about you? Does
it confer validation? Does it make you seem more normal? Does it draw
new boundaries around you? Does it make you seem safer?

I love my partner and enjoy sharing our day-to-day lives, but marriage
--- this thing young girls are taught to venerate --- hasn't transformed
my life. It's more like weaving new strands into an existing tapestry
than ditching a drab pattern for a more colorful one.

When I lived in Los Angeles, I used to go out with friends and queue for
hours to get into some new, exclusive club, only to finally get in and
discover there wasn't much going on inside. The social pressure
regarding marriage feels like that, an emphasis on getting through the
doorway without enough care for what lies beyond.

Our experiences vary. I can only describe mine. We punish and reward
people for how well they conform to our ideals without even realizing
it. We punish ourselves when the things we're told to want keep us from
appreciating and enjoying the things we have.

Someone may read this and find my thoughts obvious, trite, outdated.
Someone may read this and think I have missed out in life. I'm writing
it anyway, for the times I thought: ``Maybe I'm imagining things'' and
``Maybe they're right'' and ``Maybe there is something wrong with my
life.''

What I have to say to my friends who feel pressure from family or
society as they navigate dating, relationships or a single life, and who
have been told they are somehow less than whole because they're on their
own: You are not. A full and meaningful life belongs to us all, no
wedding required.

\href{https://www.kt.marketing/}{Katerina Tsasis} is a marketing
strategist and writer in Austin, Tex.

Modern Love can be reached at
\href{mailto:modernlove@NYTimes.com}{\nolinkurl{modernlove@NYTimes.com}}.

Want more from Modern Love? Watch the
\href{https://www.nytimes3xbfgragh.onion/2019/09/12/style/modern-love-tv-show-trailer.html}{TV
series}; sign up for the
\href{https://www.nytimes3xbfgragh.onion/newsletters/love-letter}{newsletter};
or listen to the
\href{https://www.nytimes3xbfgragh.onion/column/modern-love-podcast}{podcast}
on
\href{https://itunes.apple.com/us/podcast/modern-love/id1065559535?mt=2\&version=meter+at+0\&module=meter-Links\&pgtype=article\&contentId=\&mediaId=\&referrer=\&priority=true\&action=click\&contentCollection=meter-links-click}{iTunes},
\href{https://open.spotify.com/show/03Er7mSPq9IEewOgbPD3vO}{Spotify} or
\href{https://play.google.com/music/listen?u=0\#/ps/Iktqjbkz7bychbnofblw32dik64}{Google
Play}. We also have swag at
\href{https://store.nytimes3xbfgragh.onion/collections/modern-love}{the
NYT Store} and a book,
``\href{https://www.penguinrandomhouse.com/books/623036/modern-love-revised-and-updated-by-edited-by-daniel-jones-with-contributions-by-andrew-rannells-ayelet-waldman-amy-krouse-rosenthal-veronica-chambers-and-more/}{Modern
Love: True Stories of Love, Loss, and Redemption}.''

Advertisement

\protect\hyperlink{after-bottom}{Continue reading the main story}

\hypertarget{site-index}{%
\subsection{Site Index}\label{site-index}}

\hypertarget{site-information-navigation}{%
\subsection{Site Information
Navigation}\label{site-information-navigation}}

\begin{itemize}
\tightlist
\item
  \href{https://help.nytimes3xbfgragh.onion/hc/en-us/articles/115014792127-Copyright-notice}{©~2020~The
  New York Times Company}
\end{itemize}

\begin{itemize}
\tightlist
\item
  \href{https://www.nytco.com/}{NYTCo}
\item
  \href{https://help.nytimes3xbfgragh.onion/hc/en-us/articles/115015385887-Contact-Us}{Contact
  Us}
\item
  \href{https://www.nytco.com/careers/}{Work with us}
\item
  \href{https://nytmediakit.com/}{Advertise}
\item
  \href{http://www.tbrandstudio.com/}{T Brand Studio}
\item
  \href{https://www.nytimes3xbfgragh.onion/privacy/cookie-policy\#how-do-i-manage-trackers}{Your
  Ad Choices}
\item
  \href{https://www.nytimes3xbfgragh.onion/privacy}{Privacy}
\item
  \href{https://help.nytimes3xbfgragh.onion/hc/en-us/articles/115014893428-Terms-of-service}{Terms
  of Service}
\item
  \href{https://help.nytimes3xbfgragh.onion/hc/en-us/articles/115014893968-Terms-of-sale}{Terms
  of Sale}
\item
  \href{https://spiderbites.nytimes3xbfgragh.onion}{Site Map}
\item
  \href{https://help.nytimes3xbfgragh.onion/hc/en-us}{Help}
\item
  \href{https://www.nytimes3xbfgragh.onion/subscription?campaignId=37WXW}{Subscriptions}
\end{itemize}
