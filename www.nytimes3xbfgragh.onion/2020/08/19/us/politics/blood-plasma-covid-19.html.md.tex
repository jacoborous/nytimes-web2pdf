Sections

SEARCH

\protect\hyperlink{site-content}{Skip to
content}\protect\hyperlink{site-index}{Skip to site index}

\href{https://www.nytimes3xbfgragh.onion/section/politics}{Politics}

\href{https://myaccount.nytimes3xbfgragh.onion/auth/login?response_type=cookie\&client_id=vi}{}

\href{https://www.nytimes3xbfgragh.onion/section/todayspaper}{Today's
Paper}

\href{/section/politics}{Politics}\textbar{}F.D.A.'s Emergency Approval
of Blood Plasma Is Now on Hold

\url{https://nyti.ms/2Q7q2BT}

\begin{itemize}
\item
\item
\item
\item
\item
\end{itemize}

\hypertarget{the-coronavirus-outbreak}{%
\subsubsection{\texorpdfstring{\href{https://www.nytimes3xbfgragh.onion/news-event/coronavirus?name=styln-coronavirus-national\&region=TOP_BANNER\&block=storyline_menu_recirc\&action=click\&pgtype=Article\&impression_id=45479000-f52e-11ea-a92a-d1075f6219dd\&variant=undefined}{The
Coronavirus
Outbreak}}{The Coronavirus Outbreak}}\label{the-coronavirus-outbreak}}

\begin{itemize}
\tightlist
\item
  live\href{https://www.nytimes3xbfgragh.onion/2020/09/12/world/covid-19-coronavirus.html?name=styln-coronavirus-national\&region=TOP_BANNER\&block=storyline_menu_recirc\&action=click\&pgtype=Article\&impression_id=45479001-f52e-11ea-a92a-d1075f6219dd\&variant=undefined}{Latest
  Updates}
\item
  \href{https://www.nytimes3xbfgragh.onion/interactive/2020/us/coronavirus-us-cases.html?name=styln-coronavirus-national\&region=TOP_BANNER\&block=storyline_menu_recirc\&action=click\&pgtype=Article\&impression_id=45479002-f52e-11ea-a92a-d1075f6219dd\&variant=undefined}{Maps
  and Cases}
\item
  \href{https://www.nytimes3xbfgragh.onion/interactive/2020/science/coronavirus-vaccine-tracker.html?name=styln-coronavirus-national\&region=TOP_BANNER\&block=storyline_menu_recirc\&action=click\&pgtype=Article\&impression_id=4547b710-f52e-11ea-a92a-d1075f6219dd\&variant=undefined}{Vaccine
  Tracker}
\item
  \href{https://www.nytimes3xbfgragh.onion/2020/09/10/us/politics/fda-coronavirus-vaccine.html?name=styln-coronavirus-national\&region=TOP_BANNER\&block=storyline_menu_recirc\&action=click\&pgtype=Article\&impression_id=4547b711-f52e-11ea-a92a-d1075f6219dd\&variant=undefined}{F.D.A.
  Regulators' Self-Defense}
\item
  \href{https://www.nytimes3xbfgragh.onion/2020/09/09/upshot/coronavirus-surprise-test-fees.html?name=styln-coronavirus-national\&region=TOP_BANNER\&block=storyline_menu_recirc\&action=click\&pgtype=Article\&impression_id=4547b712-f52e-11ea-a92a-d1075f6219dd\&variant=undefined}{Surprise
  Test Fees}
\end{itemize}

Advertisement

\protect\hyperlink{after-top}{Continue reading the main story}

Supported by

\protect\hyperlink{after-sponsor}{Continue reading the main story}

\hypertarget{fdas-emergency-approval-of-blood-plasma-is-now-on-hold}{%
\section{F.D.A.'s Emergency Approval of Blood Plasma Is Now on
Hold}\label{fdas-emergency-approval-of-blood-plasma-is-now-on-hold}}

Government health leaders including Dr. Francis S. Collins and Dr.
Anthony S. Fauci urged caution last week, citing weak data from the
country's largest plasma study.

\includegraphics{https://static01.graylady3jvrrxbe.onion/images/2020/08/13/business/00virus-plasma-1/merlin_174329514_3b40dd0e-a16d-4006-a811-4510614384a9-articleLarge.jpg?quality=75\&auto=webp\&disable=upscale}

\href{https://www.nytimes3xbfgragh.onion/by/noah-weiland}{\includegraphics{https://static01.graylady3jvrrxbe.onion/images/2019/07/23/reader-center/author-noah-weiland/author-noah-weiland-thumbLarge.png}}\href{https://www.nytimes3xbfgragh.onion/by/sharon-lafraniere}{\includegraphics{https://static01.graylady3jvrrxbe.onion/images/2018/07/12/multimedia/author-sharon-lafraniere/author-sharon-lafraniere-thumbLarge.png}}\href{https://www.nytimes3xbfgragh.onion/by/sheri-fink}{\includegraphics{https://static01.graylady3jvrrxbe.onion/images/2018/08/24/multimedia/author-sheri-fink/author-sheri-fink-thumbLarge.png}}

By \href{https://www.nytimes3xbfgragh.onion/by/noah-weiland}{Noah
Weiland},
\href{https://www.nytimes3xbfgragh.onion/by/sharon-lafraniere}{Sharon
LaFraniere} and
\href{https://www.nytimes3xbfgragh.onion/by/sheri-fink}{Sheri Fink}

\begin{itemize}
\item
  Published Aug. 19, 2020Updated Aug. 28, 2020
\item
  \begin{itemize}
  \item
  \item
  \item
  \item
  \item
  \end{itemize}
\end{itemize}

WASHINGTON --- Last week, just as the Food and Drug Administration was
preparing to issue an emergency authorization for
\href{https://www.nytimes3xbfgragh.onion/2020/08/28/health/blood-plasma-fda.html}{blood
plasma} as a Covid-19 treatment, a group of top federal health officials
including Dr. Francis S. Collins and Dr. Anthony S. Fauci intervened,
arguing that emerging data on the treatment was too weak, according to
two senior administration officials.

The authorization is on hold for now as more data is reviewed, according
to H. Clifford Lane, the clinical director at the National Institute of
Allergy and Infectious Diseases. An emergency approval could still be
issued in the near future, he said.

Donated by people who have survived the disease, antibody-rich plasma is
considered safe. President Trump has hailed it as a ``beautiful
ingredient'' in the veins of people who have survived Covid-19.

But
\href{https://www.nytimes3xbfgragh.onion/2020/08/04/health/trump-plasma.html}{clinical
trials have not proved} whether plasma can help people fighting the
coronavirus.

Several top health officials --- led by Dr. Collins, the director of the
National Institutes of Health; Dr. Fauci, the government's top
infectious disease expert; and Dr. Lane ---~urged their colleagues last
week to hold off, citing
\href{https://www.medrxiv.org/content/10.1101/2020.08.12.20169359v1}{recent
data} from the country's largest plasma study, run by the Mayo Clinic.
They thought the study's data to date was not strong enough to warrant
an emergency approval.

``The three of us are pretty aligned on the importance of robust data
through randomized control trials, and that a pandemic does not change
that,'' Dr. Lane said in an interview on Tuesday.

The drafted emergency authorization leaned on the history of plasma's
use in other disease outbreaks and on animal research and a spate of
plasma studies, including the \href{https://www.uscovidplasma.org/}{Mayo
Clinic's program}, which has given infusions to more than 66,000
Covid-19 patients thanks to financing from the federal government.

An
\href{https://www.nytimes3xbfgragh.onion/2020/08/28/health/blood-plasma-fda.html}{F.D.A.}
spokeswoman declined to comment, but Mr. Trump reacted angrily,
suggesting the decision was politically motivated.

\hypertarget{latest-updates-the-coronavirus-outbreak}{%
\section{\texorpdfstring{\href{https://www.nytimes3xbfgragh.onion/2020/09/11/world/covid-19-coronavirus.html?action=click\&pgtype=Article\&state=default\&region=MAIN_CONTENT_1\&context=storylines_live_updates}{Latest
Updates: The Coronavirus
Outbreak}}{Latest Updates: The Coronavirus Outbreak}}\label{latest-updates-the-coronavirus-outbreak}}

Updated 2020-09-12T12:04:20.515Z

\begin{itemize}
\tightlist
\item
  \href{https://www.nytimes3xbfgragh.onion/2020/09/11/world/covid-19-coronavirus.html?action=click\&pgtype=Article\&state=default\&region=MAIN_CONTENT_1\&context=storylines_live_updates\#link-dfb8a16}{Fauci
  cautions the virus could disrupt life in the U.S. until `maybe even
  towards the end of 2021.'}
\item
  \href{https://www.nytimes3xbfgragh.onion/2020/09/11/world/covid-19-coronavirus.html?action=click\&pgtype=Article\&state=default\&region=MAIN_CONTENT_1\&context=storylines_live_updates\#link-7104d154}{From
  Asia to Africa, China promotes its vaccine candidates to win friends.}
\item
  \href{https://www.nytimes3xbfgragh.onion/2020/09/11/world/covid-19-coronavirus.html?action=click\&pgtype=Article\&state=default\&region=MAIN_CONTENT_1\&context=storylines_live_updates\#link-393ad215}{The
  other way the virus will kill: hunger.}
\end{itemize}

\href{https://www.nytimes3xbfgragh.onion/2020/09/11/world/covid-19-coronavirus.html?action=click\&pgtype=Article\&state=default\&region=MAIN_CONTENT_1\&context=storylines_live_updates}{See
more updates}

More live coverage:
\href{https://www.nytimes3xbfgragh.onion/live/2020/09/11/business/stock-market-today-coronavirus?action=click\&pgtype=Article\&state=default\&region=MAIN_CONTENT_1\&context=storylines_live_updates}{Markets}

``You have a lot of people over there who don't want to rush things
because they want to do it after November 3,'' he said referring to
Election Day, as he proclaimed without evidence that convalescent plasma
helped ``way over 50 percent'' of Covid-19 patients infused with it.

Plasma, the pale yellow liquid leftover after blood is stripped of its
red and white cells, has been the subject of
\href{https://www.nytimes3xbfgragh.onion/2020/08/04/health/trump-plasma.html}{months
of intense enthusiasm} from scientists, celebrities and Mr. Trump, part
of the administration's push for coronavirus treatments as a stopgap
while pharmaceutical companies race to complete dozens of clinical
trials for coronavirus vaccines.

Emergency authorizations, which do not require the same level of
evidence as a full F.D.A. approval would, have been a fraught subject
for the government during the pandemic. The agency gave one to the
malaria drugs hydroxychloroquine and chloroquine only to
\href{https://www.nytimes3xbfgragh.onion/2020/06/15/health/fda-hydroxychloroquine-malaria.html}{rescind
it months later} after the drugs were found to be ineffective against
the coronavirus, and potentially harmful. An emergency authorization for
blood plasma would most likely ease the clerical burdens on hospitals in
conducting infusions.

Senior health officials have privately expressed concern about the rapid
growth of the Mayo program and the perceived rush to declare plasma
effective without the affirmation of results from randomized trials,
which scientists have long relied on as the gold standard of evidence.
Skyrocketing enrollment in the program has prompted a debate among
researchers about what kind of empirical certainty is needed in treating
patients in a public health emergency.

An emergency approval now would ``change the way people view trials,''
said Dr. Mila B. Ortigoza, an infectious disease specialist at N.Y.U.
Langone Health who started a trial with colleagues at Montefiore Medical
Center.

``We want to make sure that when we say it works, we are confident, with
indisputable evidence,'' she said. ``We're dealing with patients' lives
here.''

Unlike the malaria drugs, plasma, which has been used
\href{https://www.ncbi.nlm.nih.gov/pmc/articles/PMC4781783/}{since the
1890s} to treat infectious diseases, has earned the attention of a
highly credentialed community of microbiologists and immunologists eager
to prove its usefulness. The Mayo Clinic has already published analysis
on tens of thousands of patients in its expanded access program showing
that plasma is safe.

The most recent
\href{https://www.medrxiv.org/content/10.1101/2020.08.12.20169359v1}{batch
of data}from the program included more than 35,000 Covid-19 patients,
many of them in intensive care and on ventilators, and suggested that
plasma administered within three days of a diagnosis reduced mortality
rates. When calculated a month after the infusions, the death rate of
patients who received plasma within three days of diagnosis was lower
(21.6 percent) than it was for those who received plasma later (26.7
percent).

But the study did not have a control group of patients given a placebo
to compare with those given plasma, making it difficult for scientists
to assess whether the treatment really worked. And given the limited
supply of plasma, it is not clear how realistic treating patients within
three days of diagnosis would be.

The program's enrollment has surged to more than 30 times as high as
initially expected, complicating the ability of scientists to recruit
sick patients to randomized trials.

It ``ballooned to a degree that, you know, is becoming unmanageable,''
Dr. Lane said.

Statisticians at the F.D.A. are now examining the Mayo data to better
understand what factors other than the treatment might have influenced
patient responses, such as higher-quality care in the hospital, Dr. Lane
said.

A research team from Houston Methodist hospitals also
\href{https://ajp.amjpathol.org/article/S0002-9440(20)30370-9/fulltext}{published
preliminary results} from a plasma trial last week. Their study of
hospitalized Covid-19 patients in the American Journal of Pathology
reported that a group of 136 patients who received the treatment were
more likely to be alive four weeks later compared with 251 patients who
did not receive it. That study found a statistically significant benefit
only when patients were treated within three days of admission and when
the plasma contained a high concentration of antibodies.

The Houston study was not randomized, meaning that all of the patients
enrolled received the treatment and none received a placebo. (The
researchers later compared their outcomes to records from other Covid-19
patients who were not in the study but were matched to be similar to
them.)

\includegraphics{https://static01.graylady3jvrrxbe.onion/images/2020/08/13/business/00virus-plasma-2/merlin_174329469_b5f5c73d-5e40-4498-b38e-94542d04f84f-articleLarge.jpg?quality=75\&auto=webp\&disable=upscale}

A surge in cases in Texas this summer quickly brought the hospital
system to its enrollment cap, and doctors there have not been able to
provide the experimental treatment since mid-July. If the F.D.A. gave an
emergency authorization, doctors at the hospital could possibly begin
administering it again, said Dr. Eric Salazar, the study's principal
investigator.

But an emergency authorization could have the unintended effect of
\href{https://www.nytimes3xbfgragh.onion/2020/08/04/health/trump-plasma.html}{making
it harder for rigorous clinical trials} to definitively show whether
plasma works. Scientists have struggled to recruit patients for
randomized trials, as many patients and their doctors --- knowing they
could get the treatment under the Mayo program --- have been unwilling
to risk receiving a placebo.

Last month, one such
\href{https://www.medrxiv.org/content/10.1101/2020.07.01.20139857v1}{trial
in the Netherlands} was stopped when researchers realized that patients
given plasma showed no difference in mortality, length of hospital stay
or disease severity compared with those given a placebo. Most of the
patients had already developed their own antibodies by the time they
entered the study, the researchers noted.

At least 10 randomized trials in the United States have collectively
enrolled only a few hundred people. They have also been stymied by the
waning of the virus outbreak in many cities, complicating the ability of
researchers to recruit sick people. Dr. Collins has encouraged a
strategy of pooling the results from randomized trials, an idea that has
met resistance from some researchers.

Dr. R. Scott Wright, who is helping oversee the Mayo Clinic's plasma
program, was an early proponent of conducting randomized trials. But he
said in a recent interview that the mechanics of setting up large
studies were complicated by early shortages of plasma, coordination via
videoconference calls and the difficulty of predicting where the virus
would spread next.

If the F.D.A. does grant the emergency authorization, it could make it
even harder to get answers, said Dr. Ortigoza of N.Y.U.

``We will keep going, because we're in desperate need of a randomized
placebo-controlled trial for convalescent plasma,'' she said. ``This is
something our country and the world really needs right now.''

Noah Weiland and Sharon LaFraniere reported from Washington, and Sheri
Fink from Houston. Katie Thomas contributed reporting from Chicago.

Advertisement

\protect\hyperlink{after-bottom}{Continue reading the main story}

\hypertarget{site-index}{%
\subsection{Site Index}\label{site-index}}

\hypertarget{site-information-navigation}{%
\subsection{Site Information
Navigation}\label{site-information-navigation}}

\begin{itemize}
\tightlist
\item
  \href{https://help.nytimes3xbfgragh.onion/hc/en-us/articles/115014792127-Copyright-notice}{©~2020~The
  New York Times Company}
\end{itemize}

\begin{itemize}
\tightlist
\item
  \href{https://www.nytco.com/}{NYTCo}
\item
  \href{https://help.nytimes3xbfgragh.onion/hc/en-us/articles/115015385887-Contact-Us}{Contact
  Us}
\item
  \href{https://www.nytco.com/careers/}{Work with us}
\item
  \href{https://nytmediakit.com/}{Advertise}
\item
  \href{http://www.tbrandstudio.com/}{T Brand Studio}
\item
  \href{https://www.nytimes3xbfgragh.onion/privacy/cookie-policy\#how-do-i-manage-trackers}{Your
  Ad Choices}
\item
  \href{https://www.nytimes3xbfgragh.onion/privacy}{Privacy}
\item
  \href{https://help.nytimes3xbfgragh.onion/hc/en-us/articles/115014893428-Terms-of-service}{Terms
  of Service}
\item
  \href{https://help.nytimes3xbfgragh.onion/hc/en-us/articles/115014893968-Terms-of-sale}{Terms
  of Sale}
\item
  \href{https://spiderbites.nytimes3xbfgragh.onion}{Site Map}
\item
  \href{https://help.nytimes3xbfgragh.onion/hc/en-us}{Help}
\item
  \href{https://www.nytimes3xbfgragh.onion/subscription?campaignId=37WXW}{Subscriptions}
\end{itemize}
