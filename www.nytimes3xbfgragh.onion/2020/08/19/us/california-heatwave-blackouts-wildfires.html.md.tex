Sections

SEARCH

\protect\hyperlink{site-content}{Skip to
content}\protect\hyperlink{site-index}{Skip to site index}

\href{https://www.nytimes3xbfgragh.onion/section/us}{U.S.}

\href{https://myaccount.nytimes3xbfgragh.onion/auth/login?response_type=cookie\&client_id=vi}{}

\href{https://www.nytimes3xbfgragh.onion/section/todayspaper}{Today's
Paper}

\href{/section/us}{U.S.}\textbar{}Fires, Blackouts, a Heat Wave and a
Pandemic: California's `Horrible' Month

\url{https://nyti.ms/34hqWEe}

\begin{itemize}
\item
\item
\item
\item
\item
\item
\end{itemize}

\hypertarget{wildfires-in-the-west}{%
\subsubsection{\texorpdfstring{\href{https://www.nytimes3xbfgragh.onion/spotlight/california-wildfires?name=styln-california-wildfires\&region=TOP_BANNER\&block=storyline_menu_recirc\&action=click\&pgtype=Article\&impression_id=8a6d9060-f2b5-11ea-9df0-4b0f68b7e02f\&variant=undefined}{Wildfires
in the West}}{Wildfires in the West}}\label{wildfires-in-the-west}}

\begin{itemize}
\tightlist
\item
  live\href{https://www.nytimes3xbfgragh.onion/2020/09/09/us/wildfires-live-updates.html?name=styln-california-wildfires\&region=TOP_BANNER\&block=storyline_menu_recirc\&action=click\&pgtype=Article\&impression_id=8a6d9061-f2b5-11ea-9df0-4b0f68b7e02f\&variant=undefined}{Fires
  Updates}
\item
  \href{https://www.nytimes3xbfgragh.onion/2020/09/07/us/ca-wildfires-heatwave.html?name=styln-california-wildfires\&region=TOP_BANNER\&block=storyline_menu_recirc\&action=click\&pgtype=Article\&impression_id=8a6d9062-f2b5-11ea-9df0-4b0f68b7e02f\&variant=undefined}{Fires
  Span California}
\item
  \href{https://www.nytimes3xbfgragh.onion/2020/08/26/us/california-wildfires-lake-berryessa.html?name=styln-california-wildfires\&region=TOP_BANNER\&block=storyline_menu_recirc\&action=click\&pgtype=Article\&impression_id=8a6db770-f2b5-11ea-9df0-4b0f68b7e02f\&variant=undefined}{The
  Trauma of 2020}
\item
  \href{https://www.nytimes3xbfgragh.onion/article/why-does-california-have-wildfires.html?name=styln-california-wildfires\&region=TOP_BANNER\&block=storyline_menu_recirc\&action=click\&pgtype=Article\&impression_id=8a6db771-f2b5-11ea-9df0-4b0f68b7e02f\&variant=undefined}{California's
  Disastrous Season}
\item
  \href{https://www.nytimes3xbfgragh.onion/2020/09/08/us/california-wildfire-heat-wave.html?name=styln-california-wildfires\&region=TOP_BANNER\&block=storyline_menu_recirc\&action=click\&pgtype=Article\&impression_id=8a6db772-f2b5-11ea-9df0-4b0f68b7e02f\&variant=undefined}{Newsletter}
\end{itemize}

Advertisement

\protect\hyperlink{after-top}{Continue reading the main story}

Supported by

\protect\hyperlink{after-sponsor}{Continue reading the main story}

\hypertarget{fires-blackouts-a-heat-wave-and-a-pandemic-californias-horrible-month}{%
\section{Fires, Blackouts, a Heat Wave and a Pandemic: California's
`Horrible'
Month}\label{fires-blackouts-a-heat-wave-and-a-pandemic-californias-horrible-month}}

The nation's most-populated state is facing multiple crises, including
23 major wildfires raging while the daily death toll from the
coronavirus is above 100.

\includegraphics{https://static01.graylady3jvrrxbe.onion/images/2020/08/19/us/19wildfires-1/merlin_175905363_40013fe4-4d2f-4df1-9495-3734fa4e4e46-articleLarge.jpg?quality=75\&auto=webp\&disable=upscale}

\href{https://www.nytimes3xbfgragh.onion/by/thomas-fuller}{\includegraphics{https://static01.graylady3jvrrxbe.onion/images/2018/06/12/multimedia/author-thomas-fuller/author-thomas-fuller-thumbLarge.png}}

By \href{https://www.nytimes3xbfgragh.onion/by/thomas-fuller}{Thomas
Fuller}

\begin{itemize}
\item
  Aug. 19, 2020
\item
  \begin{itemize}
  \item
  \item
  \item
  \item
  \item
  \item
  \end{itemize}
\end{itemize}

VACAVILLE, Calif. --- How many things can go wrong at once?

On Wednesday millions of California residents were smothered by
smoke-filled skies as
\href{https://www.nytimes3xbfgragh.onion/2020/08/20/us/northern-california-wildfires.html}{dozens
of wildfires raged out of control}. They braced for triple-digit
temperatures, the sixth day of a punishing heat wave that included
\href{https://www.nytimes3xbfgragh.onion/2020/08/17/climate/death-valley-hottest-temperature-on-earth.html}{a
recent reading of 130 degrees in Death Valley}. They braced for possible
power outages because the state's grid is overloaded, the latest sign of
an energy crisis. And they continued to fight a virus that is killing
130 Californians a day.

Even for a state accustomed to disaster, August has been a terrible
month.

Across the state there were 23 major fires reported on Wednesday and
more than 300 smaller ones.

In the San Francisco Bay Area alone there were 15 wildfires, most of
them burning out of control and feeding off the grasses and shrubs
desiccated by the extreme heat. Thousands of residents were ordered
evacuated in the wine country of Napa County and from the hills above
Silicon Valley in Santa Cruz and San Mateo Counties.

In Southern California, fires were reported in Ventura and Riverside
Counties --- and sweeping through one of the world's biggest collections
of Joshua trees, burning a 43,000-acre stretch of the Mojave National
Preserve. Images of the fire showed the iconic trees shooting flames
into the air like blowtorches.

\includegraphics{https://static01.graylady3jvrrxbe.onion/images/2020/08/19/us/19wildfires-2/merlin_175832655_18c7dfaa-22bb-4b66-b5a7-d533d6a6d0ff-articleLarge.jpg?quality=75\&auto=webp\&disable=upscale}

The evening breezes that many Californians rely on to chase the heat
from their homes had vanished. And for those with air-conditioning, the
power outages were a constant threat to that remedy.

But closer to the fires, residents had more urgent concerns.

Edie Kansas left her home outside Vacaville, northeast of San Francisco,
at 1 a.m. on Wednesday as a wall of fire traveling down hillsides
threatened the cattle ranch that has been in her family since the 1860s.
When wildfires struck in past years, inmate fire crews from nearby
prisons quickly arrived to help protect homes. But this year, partly
because of the coronavirus, the number of inmate crews has been slashed.
Some prisoners are under quarantine and others were released early to
mitigate the spread of the virus in prisons.

The fires, the power outages and the threat of the coronavirus have
conspired to make 2020 the worst year Ms. Kansas can remember.

``This year,'' Ms. Kansas said. ``It's just so horrible.''

On Wednesday, a helicopter pilot taking part in firefighting operations
in Fresno County died in a crash while attempting to drop water,
according to a Cal Fire spokesman.

The wildfires threatening Vacaville are known together as the L.N.U.
Lightning Complex, and have destroyed more than 50 homes and are
threatening nearly 2,000 more buildings, the authorities said.

West of Vacaville on Wednesday afternoon, houses along Pleasants Valley
Road were consumed by flames, ash was flying through the air and smoke
poured from vast rows of fire plodding down forested hills.

Image

 Some health officials are concerned that smoke pollution could make
people more susceptible to respiratory infections like
Covid-19.Credit...Jim Wilson/The New York Times

In just 12 hours, from Tuesday evening to Wednesday morning, the area's
fires, which have injured four people, grew more than 14,000 acres. They
now cover more than 46,000 acres in Napa, Sonoma and Solano Counties ---
larger than the size of Washington, D.C. --- and are completely
uncontained.

California has had 6,754 fires this year, Gov. Gavin Newsom said on
Wednesday, compared with just more than 4,000 at the same time last
year.

But Mr. Newsom, who declared a state of emergency on Tuesday to access
out-of-state resources, emphasized that California was painfully
familiar with the challenges of a busy wildfire season, and that
officials have been bracing for months. ``This is what the state does,''
he said.

Mr. Newsom thanked other governors for sending additional resources,
including Gov. Doug Ducey of Arizona and Gov. Greg Abbott of Texas.
``We're putting everything we have on these fires,'' he said.

Mr. Newsom also mobilized the California National Guard to assist with
relief efforts.

The cause of the fires is still under investigation but many appear to
have been started by an unusually large number of lightning strikes over
the weekend. Chief Jeremy Rahn, a Cal Fire spokesman, said California
had experienced ``a historic lightning siege'' over the past 72 hours
that resulted in about 11,000 lightning strikes, igniting more than 367
new wildfires.

Image

The cause of the fires is still under investigation but many appear to
have been started by an unusually large number of lightning strikes over
the weekend.Credit...Max Whittaker for The New York Times

Even before the season began, Mark Ghilarducci, the director of the
state's office of emergency services, said the pandemic was bringing
``an almost oppressive level of complexity'' to fire planning, from
evacuation plans to reductions in manpower, notably among inmate fire
crews. Cal Fire said it usually had about 190 inmate fire crews but this
year had only 90 deployed or ready to deploy. Inmates currently make up
about 1,300 of the 6,900 firefighters deployed across the state.

While it is too early to say whether climate change influenced this heat
wave, warming linked to human-caused emissions of greenhouse gases has
generally contributed to the state's worsening fires. Climate change has
also expanded the fire season, once largely confined from August to
November, to nearly year-round.

``And if that's not bad enough,'' Mr. Ghilarducci said, ``now we have to
deal with a worldwide pandemic. In a fire season. With the power off.
What else do you want from us?''

New fire precautions were announced in July by Mr. Newsom. Among them:
protocols to beef up fire crews and to prevent the virus from spreading
in evacuation centers. The new evacuation rules include health
screenings upon entry to a shelter, extra cleaning, prepackaged meals,
cordoning off evacuees with coronavirus symptoms, and the repurposing of
college dorms, Airbnb houses, campgrounds and hotels into evacuation
shelters.

``We have to think differently,'' Mr. Ghilarducci said. ``We know
sticking everybody into a big room at a fairground isn't going to work
this year.''

Image

Inmate firefighters on a break from battling the River Fire in
Salinas.Credit...Noah Berger/Associated Press

\href{https://www.latimes.com/california/story/2020-08-05/la-me-apple-fire-evacuations-coronavirus}{In
Riverside},
\href{https://www.theunion.com/news/nevada-county-evacuations-continue-as-firefighters-battle-jones-fire/}{Nevada}
and
\href{https://www.sfchronicle.com/california-wildfires/article/Alameda-County-vegetation-fire-prompts-evacuation-15488467.php}{Contra
Costa Counties}, dozens of evacuated families are being sent first to
emergency hotel lodging rather than to the high school gyms that usually
serve as evacuation centers.

In the coastal town of Pescadero, south of San Francisco, authorities
used the high school as an evacuation center on Wednesday. Normally,
cots would be set up for people to spend the night. But no one is
allowed inside now, so aid workers have been setting up displaced
residents at nearby hotels.

Rita Mancera, the executive director of Puente, a social services
organization helping evacuees, said people have been bringing their
pigs, turkeys, goats, cows and horses to the school parking lot.

Masked volunteers were handing out water, food and hand sanitizer.
People waiting at the school have to sit outside or in their cars.
Dealing with the evacuees during a pandemic was ``kind of
overwhelming,'' Ms. Mancera said. ``We're asking people to be social
distanced.''

Power cuts have added an extra layer of complexity to the multiple
crises in the state.

Mr. Newsom blamed a lack of planning in an angry letter to the energy
agencies on Monday.

``Collectively, energy regulators failed to anticipate this event and to
take necessary actions to ensure reliable power to Californians,'' Mr.
Newsom said, adding, ``This cannot stand.''

Image

The wildfires threatening Vacaville are known together as the L.N.U.
Lightning Complex and cover an area larger than Washington,
D.C.Credit...Jim Wilson/The New York Times

The state's electrical grid is deep in transition from a
fossil-fuel-driven system to one increasingly reliant on renewable
energy. Dozens of workhorse power plants have been shuttered. Some had
grown old, inefficient and environmentally hazardous to the air and
marine life. Others proved uneconomical as the state pushed carbon-free
sources like solar and wind.

With the threat of even more destructive and aggressive fires in the
fall, when faster winds propel them across the parched landscape, some
health officials are concerned that smoke pollution could make people
more susceptible to respiratory infections like Covid-19.

The fires in California are already spreading smoke across a wide
region, with the National Weather Service's Bay Area office warning that
air quality in the area will be ``very poor for the foreseeable
future.''

In many parts of the Bay Area, the air quality index, a measure of the
level of air pollution, was higher than 200 on Wednesday.

That number is high compared with other cities known for poor air
quality like New Delhi, which had an index of 154, and Beijing, where
that number has hovered around 150 this week. The air quality index
scale goes up to 500, but anything above 100 is considered unhealthy,
and above 200 is ``very unhealthy,'' according to the Environmental
Protection Agency.

\href{https://www.nytimes3xbfgragh.onion/2020/04/07/climate/air-pollution-coronavirus-covid.html}{Studies
have also shown} that in areas with poor air quality, people are more
likely to die if they contract the coronavirus. And coughing, difficulty
breathing and headaches are symptoms that both the virus and wildfire
smoke exposure can cause, making it more difficult to know which may be
the source.

Solano County, which includes Vacaville and has about 450,000 residents,
has been averaging about 76 new coronavirus cases a day over the last
two weeks, according to a New York Times database.

Image

A burned house in Vacaville, where smoke poured from vast rows of fire
plodding down forested hills.Credit...Max Whittaker for The New York
Times

For some Vacaville residents, losing power made the situation even more
treacherous. As a wildfire approached his home, Philip Galbraith did not
receive any type of alert when his power shut off on Tuesday night. He
assumed it was part of intentional blackouts meant to lower power usage.

Then a neighbor began ``desperately banging'' on his door, alerting him
to the evacuation.

At 2:45 a.m. he fled.

``I got out of the house, in pretty much what I had on,'' he said. ``I
got my son and we left.''

A two-hour drive southwest, in Pescadero, Lynne Bowman gestured to the
trailer where she slept.

``This is where I live now,'' Ms. Bowman said. She, her husband and her
daughter evacuated their house on Tuesday in 45 minutes, bringing
clothes, jewelry and their two dogs, Viggo and Hedy.

Just days earlier, Ms. Bowman was celebrating her daughter's wedding, a
20-person socially distanced affair. Now, she is contemplating the
confluence of catastrophic events in the area.

``Yeah, pandemic, fire,'' she said. ``I mean, it is apocalyptic in many
ways.''

Reporting was contributed by Kellen Browning from Davenport, Calif.,
Ivan Penn from Burbank, Calif., Jill Cowan from Los Angeles, Shawn
Hubler from Sacramento, Henry Fountain from Albuquerque, and Nicholas
Bogel-Burroughs, Lucy Tompkins and Derrick Bryson Taylor from New York.

Advertisement

\protect\hyperlink{after-bottom}{Continue reading the main story}

\hypertarget{site-index}{%
\subsection{Site Index}\label{site-index}}

\hypertarget{site-information-navigation}{%
\subsection{Site Information
Navigation}\label{site-information-navigation}}

\begin{itemize}
\tightlist
\item
  \href{https://help.nytimes3xbfgragh.onion/hc/en-us/articles/115014792127-Copyright-notice}{©~2020~The
  New York Times Company}
\end{itemize}

\begin{itemize}
\tightlist
\item
  \href{https://www.nytco.com/}{NYTCo}
\item
  \href{https://help.nytimes3xbfgragh.onion/hc/en-us/articles/115015385887-Contact-Us}{Contact
  Us}
\item
  \href{https://www.nytco.com/careers/}{Work with us}
\item
  \href{https://nytmediakit.com/}{Advertise}
\item
  \href{http://www.tbrandstudio.com/}{T Brand Studio}
\item
  \href{https://www.nytimes3xbfgragh.onion/privacy/cookie-policy\#how-do-i-manage-trackers}{Your
  Ad Choices}
\item
  \href{https://www.nytimes3xbfgragh.onion/privacy}{Privacy}
\item
  \href{https://help.nytimes3xbfgragh.onion/hc/en-us/articles/115014893428-Terms-of-service}{Terms
  of Service}
\item
  \href{https://help.nytimes3xbfgragh.onion/hc/en-us/articles/115014893968-Terms-of-sale}{Terms
  of Sale}
\item
  \href{https://spiderbites.nytimes3xbfgragh.onion}{Site Map}
\item
  \href{https://help.nytimes3xbfgragh.onion/hc/en-us}{Help}
\item
  \href{https://www.nytimes3xbfgragh.onion/subscription?campaignId=37WXW}{Subscriptions}
\end{itemize}
